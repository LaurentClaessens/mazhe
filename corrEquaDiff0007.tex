% This is part of the Exercices et corrigés de mathématique générale.
% Copyright (C) 2009
%   Laurent Claessens
% See the file fdl-1.3.txt for copying conditions.
\begin{corrige}{EquaDiff0007}

Le changement de fonction inconnue $y(x)=\frac{ t(x) }{ x }$ induit le changement suivant pour la dérivée :
\begin{equation}
	y'(x)=\frac{1}{ x }\left( t'-\frac{ t }{ x } \right),
\end{equation}
que nous substituons dans l'équation proposée\footnote{Le verbe \og proposer\fg{} est évidement un euphémisme.}. Nous trouvons
\begin{equation}
	\begin{aligned}[]
		\frac{ t }{ x }(1-t)-x(1+t)\frac{1}{ x }\left( t'-\frac{ t }{ x } \right)&=0\\
		\frac{ t-t^2 }{ x }-\frac{ (1+t)(xt'-t) }{ x }&=0\\
		-xt'(1+t)+2t&=0\\
		\frac{ (1+t) }{ 2t }t'&=\frac{1}{ x }.
	\end{aligned}
\end{equation}
où nous avons fait une simplification par $x$. La solution que nous allons obtenir n'est donc pas censée être valable en $x=0$. Ici, et dans tout le chapitre sur les équations différentielles, nous ne discutons pas le domaine de validité des solutions.

En passant le $dx$ de l'autre côté, et en intégrant des deux côtés,
\begin{equation}
	\int\frac{ (1+t) }{ 2t }dt=\int \frac{ dx }{ x }.
\end{equation}
Nous avons donc $\ln(x)+C=\frac{ \ln(t) }{ 2 }+\frac{ t }{2}$. Ici, nous allons faire quelque chose de subtil. La constante $C$ peut valoir n'importe quoi, donc il n'y a pas de mal à changer de constante et écrire $C=\ln(K)$ où $K$ est une nouvelle constante. Alors, dans le membre de gauche nous avons $\ln(x)+\ln(K)=\ln(Kx)$. Cette manipulation a fait \og rentrer\fg{} la constante dans le logarithme.

Nous faisons de même avec le dénominateur $2$ du membre de droite : nous le faisons passer à gauche, et nous le rentrons dans le logarithme sous forme d'un carré, nous avons donc
\begin{equation}
	\ln(Kx^2)= \ln(t)+t,
\end{equation}
donc, en prenant l'exponentielle des deux côtés,
\begin{equation}
	\begin{aligned}[]
		Kx^2= e^{\ln(t)} e^{t}&=xye^{xy}\\
		Kx=y e^{xy}.
	\end{aligned}
\end{equation}
Encore une fois, nous ne pouvons pas écrire la solution sous forme explicite $y=y(x)$.

\end{corrige}
