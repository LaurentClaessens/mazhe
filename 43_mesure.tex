% This is part of Mes notes de mathématique
% Copyright (c) 2011-2015
%   Laurent Claessens, Carlotta Donadello
% See the file fdl-1.3.txt for copying conditions.

%+++++++++++++++++++++++++++++++++++++++++++++++++++++++++++++++++++++++++++++++++++++++++++++++++++++++++++++++++++++++++++
\section{Mesure de Lebesgue sur \texorpdfstring{$ \eR$}{R}}
%+++++++++++++++++++++++++++++++++++++++++++++++++++++++++++++++++++++++++++++++++++++++++++++++++++++++++++++++++++++++++++
\label{SecZTFooXlkwk}

Nous notons \( \mS\) l'ensemble des intervalles\footnote{Définition \ref{DefEYAooMYYTz}.} de \( \eR\).

\begin{proposition}
    L'ensemble réunions finies d'éléments de \( \mS\) est une algèbre de parties de \( \eR\).
\end{proposition}
Cette algèbre de parties de $\eR$ est notée \( \tribA_{\mS}\).

\begin{proof}

    Nous devons vérifier la définition \ref{DefTCUoogGDud}. Les ensembles \( \eR\) et \( \emptyset\) sont des intervalles et font donc partie de \( \tribA_{\mS}\).
    
    Si \( A\in\tribA_{\mS}\) se décompose en union d'intervalles de la forme \( (a_k,b_k)\) avec \( k=1,\ldots, n\) (ici nous mettons des parenthèses au lieu de crochets parce qu'a priori nous ne savons pas). Alors
    \begin{equation}
        A^c=\bigcup_{k=0}^{k}(b_k,a_{k+1})
    \end{equation}
    où nous avons posé \( b_0=-\infty\) et \( a_{n+1}=+\infty\). Ici encore les parenthèses sont soit fermées soit ouvertes en fonction de ce qu'étaient celles dans la décomposition de \( A\). Quoi qu'il en soit, cette décomposition de \( A^c\) montre que \( A^c\in\tribA_{\mS}\).

    Enfin si \( A,B\in\tribA_{\mS}\) alors \( A\cup B\in\tribA_{\mS}\).
\end{proof}

\begin{lemma}
    Tout élément de \( \tribA_{\mS}\) admet une décomposition minimale unique en réunion finie d'intervalles. Cette décomposition est formée d'intervalles deux à deux disjoints.
\end{lemma}

\begin{proof}
    Nous allons montrer que si \( A\in\tribA_{\mS}\), alors la décomposition minimale consiste en les composantes connexes de \( A\). Pour cela nous rappelons que la proposition \ref{PropInterssiConn} dit qu'une partie de \( \eR\) est connexe si et seulement si elle est un intervalle. D'abord cela nous dit immédiatement que les composantes connexes de \( A\) forment une décomposition de \( A\) en intervalles. Nous devons prouver qu'elle est minimale.

    Soit \( \{ C_k \}_{k=1,\ldots, n}\) les composantes connexes de \( A\). Aucun connexe de \( \eR\) contenu dans \( A\) ne peut intersecter plus d'un des \( C_k\), et par conséquent nous ne pouvons pas décomposer \( A\) en moins de \( n\) intervalles. 
    
    Pour l'unicité, soit \( \{ I_k \}_{k=1,\ldots, n}\) un ensemble de \( n\) intervalles tels que \( \bigcup_{k=1}^nI_k=A\). Chacun des \( I_k\) intersecte un et un seul des \( C_k\). En effet si \( x\in I_k\cap C_i\) et \( y\in I_k\cap C_j\), alors \( \mathopen[ x , y \mathclose]\subset I_k\) parce que \( I_k\) est un intervalle. Mais \( C_i\) étant le plus grand connexe contenant \( x\), \( \mathopen[ x , y \mathclose]\subset C_i\) et de la même façon, \( \mathopen[ x , y \mathclose]\subset C_j\). Par conséquent \( C_i\) et \( C_j\) sont tous deux la composante connexe de \( x\) et \( y\). Nous en déduisons que \( C_i=C_j\), c'est à dire \( i=j\).

    Par ailleurs nous avons \( I_k\cap I_l=\emptyset\) dès que \( k\neq l\) parce que sinon l'ensemble \( I_k\cap I_l\) serait connexe et la décomposition des \( \{ I_k \}_{k=1,\ldots, n} \) ne serait pas minimale : en remplaçant \( I_k\) et \( I_l\) par \( I_k\cup I_l\) on aurait eu une décomposition contenant moins d'éléments. Donc à renumérotation près nous pouvons supposer que \( I_k\) intersecte \( C_l\) si et seulement si \( k=l\).

    Dans ce cas nous devons avoir \( I_k=C_k\), sinon les éléments de \( C_k\setminus I_k\) ne seraient pas dans \( \bigcup_{i=1}^nI_i\).
\end{proof}

\begin{definition}[longueur d'intervalle\cite{MesureLebesgueLi}]
    Si \( I\) est un intervalle d'extrémités \( a\) et \( b\) avec \( -\infty\leq a\leq b\leq +\infty\) alors nous définissons la \defe{longueur}{longueur!d'un intervalle}\index{intervalle!longueur} de \( I\) par
    \begin{equation}
        \ell(I)=\begin{cases}
            b-a    &   \text{si \( -\infty<a\leq b< +\infty\)}\\
            \infty    &    \text{si \( a\) ou \( b\) est infini}
        \end{cases}
    \end{equation}
    Si \( A\in\tribA_{\mS}\) et si sa décomposition minimale est \( A=\bigcup_{k=1}^nI_k\), alors on définit
    \begin{equation}
        \ell(A)=\sum_{k=1}^n\ell(I_k).
    \end{equation}
\end{definition}

Le lemme suivant nous indique que nous pouvons calculer la longueur d'un élément de \( \tribA_{\mS}\) sans savoir la décomposition minimale, pourvu que l'on connaisse une décomposition disjointe.
\begin{lemma}[\cite{MesureLebesgueLi}]\label{LemIUQooEzHun}
    Si
    \begin{equation}
        B=\bigcup_{r=1}^pJ_r
    \end{equation}
    est une décomposition de \( B\in\tribA_{\mS}\) en intervalles deux à deux disjoints alors
    \begin{equation}
        \ell(B)=\sum_{r=1}^p\ell(J_r).
    \end{equation}
\end{lemma}

\begin{proof}
    Nous prouvons dans un premier temps le résultat dans le cas où \( B=I\) est un intervalle. Soit \( I\) un intervalle et une décomposition en intervalles disjoints \( I=\bigcup_{r=1}^pJ_r\). Nous montrons qu'alors \( \ell(I)=\sum_{r=1}^p\ell(J_r)\). Nous verrons ensuite comment passer au cas où \( B\) est un élément générique de \( \tribA_{\mS}\).
    \begin{subproof}
    \item[Si \( B=I\) est un intervalle infini]

        Si \( I\) est infini alors un des \( J_r\) soit l'être et donc \( \sum_{r=1}^p\ell(J_r)=\infty=\ell(I)\).
    \item[Si \( B=I\) est un intervalle ininfini]

    Pour chaque \( r=1,\ldots, p\) nous notons \( a_r\) et \( b_r\) les extrémités de \( J_r\). Vu que les \( J_r\) sont connexes et disjoints, si \( a_k\leq a_l\) alors \( b_k\leq a_l\), sinon l'ensemble (non vide) \( \mathopen] a_l , b_k \mathclose[ \) serait dans l'intersection \( I_k\cap I_l\) qui, elle, est vide. Plus généralement, si \( x\in J_k\) et \( y\in J_l\) avec \( x<y\) alors pour tout \( x'\in J_k\) et tout \( y'\in J_l\) nous avons \( x'<y'\). Vu qu'il y a un nombre fini d'ensembles \( J_r\), nous pouvons les classer dans l'ordre croissant :
        \begin{equation}
            a_1\leq b_1\leq a_2\leq b_2\leq \ldots\leq b_{p-1}\leq a_p\leq b_p.
        \end{equation}
        Vu que les \( J_r\) sont disjoints et que leur union est connexe nous avons en réalité
        \begin{equation}
            a=a_1\leq b_1=a_2\leq b_2=a_3\leq\ldots\leq b_{p-1}= a_p\leq b_p,
        \end{equation}
        donc une somme télescopique donne
        \begin{equation}
            \ell(I)=b-a=\sum_{r=1}^p(b_r-a_r)=\sum_{r=1}^p\ell(J_r).
        \end{equation}

    \item[Si \( B\) n'est pas un intervalle]
        Soit \( \{ I_k \}_{k=1,\ldots, n}\) la décomposition minimale de \( B\). Alors
        \begin{equation}
            \spadesuit=\ell(B)=\sum_{k=1}^n\ell(I_k)=\sum_{k=1}^n\ell\big( \bigcup_{r=1}^p(I_k\cap J_r) \big).
        \end{equation}
        Mais \( I_k\) est un intervalle et s'écrit comme union disjointe \( I_k=\bigcup_{r=1}^p(I_k\cap J_r)\), donc par la première partie
        \begin{equation}
            \spadesuit=\sum_{k=1}^n\sum_{r=1}^p\ell(I_k\cap J_r)=\sum_{r=1}^p\sum_{k=1}^n\ell(I_k\cap J_r).
        \end{equation}
        Ici \( J_r\) est un intervalle qui se décompose en \( J_r=\bigcup_{k=1}^n(I_k\cap J_r)\), donc nous pouvons encore utiliser la première partie :
        \begin{equation}
            \spadesuit=\sum_{r=1}^p\ell(J_r),
        \end{equation}
        ce qu'il fallait.
    \end{subproof}
\end{proof}

\begin{lemma}   \label{LemPIOooRLkbo}
    Si \( A,B\in\tribA_{\mS}\) avec \( A\subset B\) alors \( \ell(A)\leq \ell(B)\).
\end{lemma}

\begin{proof}
    Nous avons évidemment \( B=A\cup B\setminus A\). Notons que \( B\setminus A\in\tribA_{\mS}\) par le lemme \ref{LemBFKootqXKl}. Si \( \{ I_k \}\) est une décomposition disjointe de \( A\) et \( \{ J_i \}\) une de \( B\setminus A\) alors \( \{ I_k \}\cup\{ J_i \}\) est une décomposition disjointe de \( A\cup B\setminus A\) et le lemme \ref{LemIUQooEzHun} nous dit que
    \begin{equation}
        \ell(B)=\ell(A\cup B\setminus A)=\ell(A)+\ell(B\setminus A).
    \end{equation}
    Par conséquent \( \ell(B)\geq \ell(A)\).
\end{proof}

\begin{lemma}   \label{LemUMVooZJgMu}
    Si \( I\) est un intervalle et si il se décompose en
    \begin{equation}
        I=\bigcup_{n\in \eN}I_n
    \end{equation}
    où les \( I_n\) sont des intervalles disjoints, alors
    \begin{equation}
        \ell(I)=\sum_{n=1}^{\infty}\ell(I_n).
    \end{equation}
\end{lemma}

\begin{proof}
    Nous allons encore diviser la preuve en deux parties suivant que \( I\) soit de longueur finie ou pas.   
    \begin{subproof}

        \item[Si \( I\) est de longueur finie]
        
            Soient \( a\) et \( b\) les extrémités de \( I\) : \( -\infty<a\leq b< +\infty\). Pour tout \( N\geq 1\) nous avons
            \begin{equation}
                \sum_{n=1}^N\ell(I_n)=\ell\big( \bigcup_{n=1}^nI_n \big)\leq \ell(I).
            \end{equation}
            La première égalité est le lemme dans le cas d'une union finie \ref{LemIUQooEzHun}. L'inégalité est le lemme \ref{LemPIOooRLkbo}. Cela étant vrai pour tout $N$, à la limite \( N\to\infty\) nous conservons l'inégalité :
            \begin{equation}
                \sum_{n=1}^{\infty}\ell(I_n)\leq \ell(I).
            \end{equation}
            Nous devons encore voir l'inégalité inverse. Pour cela nous supposons que \( a<b\). Sinon \( \ell(I)=0\) et tous les \( I_n\) doivent être vide sauf un qui contiendra seulement \( \{ a \}\) (si \( I\) le contient).

            Soit \( \epsilon>0\) avec \( \epsilon<b-a\) et l'intervalle
            \begin{equation}
                \mathopen[ a+\frac{ \epsilon }{ 4 } , b-\frac{ \epsilon }{ 4 } \mathclose]=\mathopen[ a' , b' \mathclose]\subset I.
            \end{equation}
            Si les \( a_n\) et le \( b_n\) sont le extrémités des \( I_n\) alors
            \begin{equation}
                \mathopen[ a' , b' \mathclose]\subset I=\bigcup_{n\geq 1}I_n\subset\bigcup_{n\geq 1}\mathopen] a_n-\frac{ \epsilon }{ 2^{n+2} } , b_n+\frac{ \epsilon }{ 2^{n+2} } \mathclose[=\bigcup_{n\geq 1}\mathopen] a'_n , b'_n \mathclose[
            \end{equation}
            où nous avons posé \( a'_n=a_n-\epsilon/2^{n+2}\) et \( b'_n=b_n+\epsilon/2^{n+2}\). Nous avons donc recouvert le compact\footnote{Lemme \ref{LemOACGWxV}.} \( \mathopen[ a' , b' \mathclose]\) par des ouverts. Nous pouvons donc en extraire un sous-recouvrement fini (c'est la définition de la compacité), c'est à dire une partie finie \( F\) de \( \eN\) telle que 
            \begin{equation}
                \mathopen[ a' , b' \mathclose]\subset \bigcup_{n\in F}\mathopen] a'_n , b'_n \mathclose[.
            \end{equation}
            Le lemme \ref{LemPIOooRLkbo} nous dit alors que
            \begin{equation}
                \heartsuit=b'-a'\leq \ell\big( \bigcup_{n\in F}\mathopen] a'_n , b'_n \mathclose[ \big)\leq \sum_{n\in F}(b'_n-a'_n).
            \end{equation}
            La seconde inégalité se prouve en recopiant\footnote{Nous ne pouvons pas invoquer directement le lemme \ref{LemZQUooMdCpq} parce que nous n'avons pas encore prouvé que \( \ell\) était une mesure sur \( (\eR,\tribA_{\mS})\).} la preuve de \ref{LemZQUooMdCpq}. Nous continuons le calcul :
            \begin{equation}
                \heartsuit\leq\sum_{n\in F}(b_n-a_n)+\sum_{n\in F}\frac{ \epsilon }{ 2^{n+1} }\leq \sum_{n\in F}(b_n-a_n)+\frac{ \epsilon }{2}.
            \end{equation}
            Mais \( b'-a'=(b-a)-\frac{ \epsilon }{2}\), donc
            \begin{equation}
                b-a-\frac{ \epsilon }{2}\leq \sum_{n\in F}(b_n-a_n)+\frac{ \epsilon }{2}.
            \end{equation}
            D'où nous déduisons que
            \begin{equation}
                \ell(I)=b-a\leq \sum_{n\in F}(b_n-a_n)+\epsilon\leq \sum_{n\in \eN}(b_n-a_n)+\epsilon=\sum_{n\in \eN}\ell(I_n)+\epsilon.
            \end{equation}
            Cela étant valable pour tout \( \epsilon\) nous déduisons que
            \begin{equation}
                \ell(I)\leq\sum_{n\in \eN}\ell(I_n).
            \end{equation}

        \item[Si \( I\) est de longueur infinie]

        Étant donné que \( I\) est un intervalle de longueur infinie, il contient au moins un ensemble du type \( \mathopen] -\infty , a \mathclose]\) ou \( \mathopen[ a , +\infty [\); donc  pour tout \( M>0\), il existe \( N\geq 1\) tel que
            \begin{equation}
                \ell\big( I\cap\mathopen[ -N , N \mathclose] \big)\geq M.
            \end{equation}
            Mais \( I\cap\mathopen[ -N , N \mathclose]\) est un intervalle et 
            \begin{equation}
                I\cap\mathopen[ -N , N \mathclose]=\bigcup_{n\in \eN}I_n\cap\mathopen[ -N , N \mathclose]
            \end{equation}
            qui est une union disjointe. Par conséquent,
            \begin{equation}
                M\leq \ell\big( I\cap\mathopen[ -N , N \mathclose] \big)=\sum_n\ell\big( I_n\cap\mathopen[ -N , N \mathclose] \big)\leq\sum_n\ell(I_n).
            \end{equation}
            Cela étant vrai pour tout \( M>0\), nous concluons que
            \begin{equation}
                \sum_{n\in \eN}\ell(I_n)=\infty.
            \end{equation}
    \end{subproof}
\end{proof}

\begin{remark}
    Pour la preuve de \ref{LemUMVooZJgMu} nous ne pouvons pas classer les \( I_n\) en ordre croissant comme nous l'avons fait dans la preuve de \ref{LemIUQooEzHun}. En effet si \( I=\mathopen[ 0 , 1 \mathclose]\) et que nous recouvrons \( \mathopen[ 0 , \frac{ 1 }{2} [\) et \( \mathopen] \frac{ 1 }{2} , 1 \mathclose]\) par une infinité d'intervalles chacun, nous ne pouvons plus les classer par ordre croissant.
\end{remark}

\begin{proposition}[\cite{MesureLebesgueLi}]     \label{PropULFoodgXrR}
    La fonction \( \ell\) ainsi définie est une mesure \( \sigma\)-finie sur l'algèbre de parties \( \tribA_{\mS}\).
\end{proposition}

\begin{proof}
    Le fait que \( \ell\) soit \( \sigma\)-finie provient par exemple du fait que \( \ell\big( \mathopen] -n , n \mathclose[ \big)=2n\) tandis que \( \bigcup_n\mathopen] -n , n \mathclose[=\eR\).

        Nous devons à présent prouver que \( \ell\) est additive. Soient \( (A_i)_{i\in \eN}\) des éléments disjoints de \( \tribA_{\mS}\), avec leurs décomposition minimales
            \begin{equation}
                A_i=\bigcup_{k=1}^nI^{(i)}_k.
            \end{equation}
            Pour chaque \( i\in \eN\), le lemme \ref{LemUMVooZJgMu} nous indique que
            \begin{equation}
                \ell(A_i)=\sum_{k\in \eN}\ell(I^{(i)}_k).
            \end{equation}
            L'ensemble \( \eN\times \eN\) est dénombrable et nous pouvons considérer la décomposition 
            \begin{equation}
                \bigcup_{i\in \eN}A_i=\bigcup_{(i,k)\in \eN\times \eN}I^{(i)}_k.
            \end{equation}
            Cette décomposition n'est pas spécialement minimale\footnote{\( A_1\) pourrait contenir \( \mathopen[ 0 , 1 \mathclose]\) et \( A_2\) contenir \( \mathopen] 1 , 2 \mathclose]\).} mais elle est disjointe.
            Le lemme \ref{LemUMVooZJgMu} donne
            \begin{equation}
                \ell(\bigcup_i A_i)=\sum_{(i,k)\in \eN\times \eN}\ell(I_k^{(i)})=\sum_{i\in \eN}\left( \sum_{k\in \eN}\ell(I^{(i)}_k)\right)=\sum_{i\in \eN}\ell(A_i).
            \end{equation}
            La décomposition de la somme sur \( \eN^2\) en deux sommes sur \( \eN\) est faite en vertu de la proposition \ref{PropVQCooYiWTs}.
            
\end{proof}

%--------------------------------------------------------------------------------------------------------------------------- 
\subsection{Mesure et tribu de Lebesgue}
%---------------------------------------------------------------------------------------------------------------------------

\begin{theorem} \label{ThoDESooEyDOe}
    Il existe une unique mesure \( \lambda\) sur \( \big( \eR,\Borelien(\eR) \big)\) telle que
    \begin{equation}
        \lambda\big( \mathopen] a , b \mathclose[ \big)=b-a
    \end{equation}
    pour tout \( a\leq b\) dans \( \eR\).
\end{theorem}

\begin{proof}
    
    L'existence provient du théorème de prolongement de Hahn \ref{ThoLCQoojiFfZ} : la mesure \( \ell\) sur \( (\tribA_{\mS})\) se prolonge à \( \sigma(\tribA_{\mS})=\Borelien(\eR)\).

    Nous ne pouvons pas prouver l'unicité en invoquant la partie unicité de Hahn (c'est tentant parce que \( \ell\) est \( \sigma\)-finie) parce que dans ce théorème nous ne fixons la valeur de \( \lambda\) que sur une toute petite partie de \( \tribA_{\mS}\). Nous allons cependant voir que cette petite partie suffit à garantir l'unicité.

    La classe 
    \begin{equation}
        \tribD=\{ \mathopen] a , b \mathclose[\tq -\infty<a\leq b< +\infty \}
    \end{equation}
    est stable par intersection finie et engendre la tribu borélienne. En effet \( \tribD\) contient toutes les boules et donc une base dénombrable de la topologie de \( \eR\) (proposition \ref{PropNBSooraAFr}). Donc tous les ouverts de \( \eR\) sont dans \( \sigma(\tribD)\) et \( \sigma(\tribD)=\Borelien(\eR)\). Nous pouvons donc dire grâce au théorème \ref{ThoJDYlsXu} qu'il y a unicité de la mesure sur \( \Borelien(\eR)\) lorsque les valeurs sur \( \tribD\) sont fixées.
\end{proof}

\begin{definition}      \label{DefooYZSQooSOcyYN}
    La mesure de l'espace mesuré \( \big( \eR,\Borelien(\eR),\lambda \big)\) donné par le théorème \ref{ThoDESooEyDOe} est la \defe{mesure de Lebesgue}{mesure!de Lebesgue} sur \( \big( \eR,\Borelien(\eR) \big)\).

    Nous définissons aussi la \defe{tribu de Lebesgue}{tribu!de Lebesgue} par la proposition \ref{PropIIHooAIbfj} : \( \big( \eR,\Lebesgue(\eR),\lambda \big)\) est l'espace mesuré complété de \( \big( \eR,\Borelien(\eR), \lambda \big)\).
\end{definition}


\begin{remark}
    Il n'est pas évident que la tribu de Lebesgue soit plus grande que celle des boréliens, ni que la tribu des parties soit plus grande que celle de Lebesgue. Nous mentionnons cependant les faits suivants.
    %TODO : donner des exemples
    \begin{enumerate}
        \item
            Il existe des ensembles mesurables non-boréliens, et cela ne nécessite pas l'axiome du choix. Un argument classique de cardinalité est donné dans \cite{SFYoobgQUp}. La construction la plus explicite que j'aie trouvée est dans \cite{XSHoosgoQa}, mais ça a l'air de demander des connaissances précises sur les ordinaux.
        \item
            Vu que l'ensemble de Cantor \( C\) est mesurable de mesure nulle, tout sous-ensemble de Cantor est mesurable de mesure nulle parce que la tribu de Lebesgue est complète par définition. Le cardinal de \( \partP(C)\) est strictement supérieur à la puissance du continu, alors que le cardinal de l'ensemble des boréliens est au plus égal à la puissance du continu. Donc il existe des non boréliens contenus dans Cantor; de tels non boréliens sont alors mesurables au sens de Lebesgue.

            Pour savoir des choses sur la cardinalité de l'ensemble des parties, on peut aller voir dans \cite{KZIoofzFLV}.
        \item
            Si nous admettons l'axiome du choix alors il existe des ensembles non mesurables au sens de Lebesgue. Nous en verrons un dans l'exemple \ref{EXooCZCFooRPgKjj}.
    \end{enumerate}
\end{remark}

\begin{example}[Un ouvert contenant tous les rationnel et de mesure arbitrairement petite]
    Il est possible de construire un ouvert de $\eR$ contenant \( \eQ\) et de mesure de Lebesgue plus petite que \( \epsilon\). Pour cela si \( (q_i)\) est une énumération des rationnels, il suffit de prendre
    \begin{equation}
        \mO=\bigcup_{n=1}^{\infty}B(q_n,\frac{ \epsilon }{ 2^{n+1} }).
    \end{equation}
    Cela est un ouvert comme union d'ouverts, ça contient tous les rationnels, et sa mesure se majore. En effet le théorème \ref{ThoDESooEyDOe} donne \( \lambda\big( B(q_n,\frac{\epsilon }{ 2^n }) \big)=\frac{ \epsilon }{ 2^n }\). Vu que ces boules ne sont a priori pas disjointes, le lemme \ref{LemPMprYuC} donne 
    \begin{equation}
        \lambda(\mO) \leq \sum_{n=1}^{\infty}\frac{ \epsilon }{ 2^n }=\epsilon
    \end{equation}
    par \eqref{EqPZOWooMdSRvY} avec \( q=\frac{ 1 }{2}\).

    Par complémentarité, nous pouvons construire un ensemble fermé de mesure non nulle et ne contenant aucun rationnel. Et même un fermé dans \( \mathopen[ 0 , 1 \mathclose]\), de mesure \( 1-\epsilon\) ne contenant aucun rationnel. 
    
    Cela peut surprendre parce qu'il existe des tonnes de suites d'irrationnels qui convergent vers des rationnels\footnote{Si \( q\in \eQ\) et \( r\in \eR\setminus \eQ\) alors la suite \( (q+r/10^k)_k\) est une suite d'irrationnels convergente vers le rationnel \( q\)}, et il semble difficile de créer un ensemble contenant beaucoup d'irrationnels tout en préservant la propriété de fermeture vis à vis des suites convergentes.
\end{example}

%--------------------------------------------------------------------------------------------------------------------------- 
\subsection{Propriétés de la mesure de Lebesgue}
%---------------------------------------------------------------------------------------------------------------------------

\begin{proposition}
    Tout ensemble dénombrable de \( \eR\) est mesurable de mesure nulle.
\end{proposition}

\begin{proof}
    Un point de \( \eR\) est un intervalle de mesure nulle. Si \( D\) est dénombrable, il est union disjointes et dénombrable de points. Le lemme \ref{LemUMVooZJgMu} nous dit alors que sa mesure est \( \lambda(D)=\sum_{i=1}^{\infty}\lambda(\{ a_i \})=0\).
\end{proof}

\begin{remark}
    Il existe cependant des ensembles non dénombrables et tout de même de mesure nulle. Par exemple l'ensemble de Cantor (voir la proposition \ref{PropBEWooXZdKN}).
\end{remark}


\begin{proposition}     \label{PropooOACLooLMIUuY}
    La mesure de Lebesgue est invariante par translation, c'est à dire que si \( A\) est mesurable alors \( \lambda(A)=\lambda(A+\alpha)\) pour tout réel \( \alpha\).
\end{proposition}

\begin{proof}
    Nous commençons par les intervalles ouverts :
    \begin{equation}
    \lambda\big( \mathopen] a , b \mathclose[+\alpha \big)=\lambda\big( \mathopen] a+\alpha , b+\alpha \mathclose[ \big)=(b+\alpha)-(a+\alpha)=b-a=\lambda\big( \mathopen] a , b \mathclose[ \big).
    \end{equation}
    D'après ce qui est dit dans l'exemple \ref{ExDMPoohtNAj}, la mesure de Lebesgue sur les boréliens est invariante par translation.

    Si \( A\) est mesurable alors il existe un borélien \( B\) et un ensemble négligeable \( N\) tels que \( A=B\cup N\) par la caractérisation \ref{EqFJIoorxZNU} de la complétion. Alors \( A+\alpha=B+\alpha\cup N+\alpha\) et \( N+\alpha\) est encore un ensemble négligeable. Donc \( \lambda(A+\alpha)=\alpha(B+\alpha)=\lambda(B)\).
\end{proof}

Le mesure \( \ell\) définie sur l'algèbre de parties \( \tribA_{\mS}\) (voir proposition \ref{PropULFoodgXrR}). La proposition \ref{PropIUOoobjfIB} nous donne donc une mesure extérieure par
\begin{equation}    \label{EqJGXoogdKqb}
    \lambda^*(X)=\inf\{ \sum_n\ell(A_n);A_n\in\tribA_{\mS},X\subset\bigcup_nA_n \}.
\end{equation}

La proposition suivante montre que cette mesure extérieure peut être exprimée seulement avec des intervalles ouverts.
\begin{proposition} \label{PropTNOooDcfwn}
    Nous avons
    \begin{equation}
        \lambda^*(X)=\inf\{ \sum_{n\geq 1}\ell(I_n);\text{\( I_n\) sont des intervalles ouverts et }X\subset\bigcup_nI_n \}.
    \end{equation}
\end{proposition}

\begin{proof}
    Nous savons que dans la définition \eqref{EqJGXoogdKqb}, chacun des \( A_n\) est une réunion disjointe d'intervalles (pas spécialement ouverts) deux à deux disjoints; donc
    \begin{equation}
        \lambda^*(X)=\inf\{ \sum_n\ell(I_n);I_n\in\mS,X\subset\bigcup_nI_n \}.
    \end{equation}
    Soit \( \epsilon>0\). Si \( A\subset\bigcup_nI_n\), pour chaque \( n\geq 1\) nous considérons un intervalle ouvert \( J_n\) tel que \( I_n\subset J_n\) et \( \ell(I_n)+\frac{ \epsilon }{ 2^n }\leq \ell(J_n)\). Faisant cela pour chacun des découpages de \( X\) en intervalles nous trouvons
    \begin{equation}
        \lambda^*(X)\leq \inf\{ \sum_n\ell(J_n)\text{ \( J_n\) est ouvert et }X\subset\bigcup_nJ_n \}+\epsilon.
    \end{equation}
    Étant donné que \( \epsilon\) est arbitraire nous avons l'égalité.
\end{proof}

\begin{proposition}[\cite{MesureLebesgueLi}]    \label{PropMXIoojpKvd}
    Si \( X\subset \eR\) est tel que \( \lambda^*(X)<\infty\) alors
    \begin{enumerate}
        \item   \label{ItemGJUoozrDILi}
            Pour tout \( \epsilon>0\) il existe un ouvert \( \Omega_{\epsilon}\) tel que
            \begin{subequations}
                \begin{numcases}{}
                    X\subset\Omega_{\epsilon}\\
                    \lambda(\Omega_{\epsilon})\leq \lambda^*(X)+\epsilon.
                \end{numcases}
            \end{subequations}
        \item   \label{ItemGJUoozrDILii}
            Il existe une intersection dénombrable d'ouverts \( G\) telle que
            \begin{subequations}
                \begin{numcases}{}
                    X\subset G\\
                    \lambda(G)=\lambda^*(X).
                \end{numcases}
            \end{subequations}
    \end{enumerate}
\end{proposition}

\begin{proof}
    Pour \ref{ItemGJUoozrDILi}, la proposition \ref{PropTNOooDcfwn} nous a déjà dit que
    \begin{equation}
        \lambda^*(X)=\inf\{ \sum_n\ell(I_n)\text{ \( I_n\) est un intervalle ouvert}, X\subset\bigcup_nI_n \},
    \end{equation}
    donc si \( \epsilon>0\), il existe des intervalles ouverts \( I_n\) tels que 
    \begin{subequations}
        \begin{numcases}{}
            X\subset\bigcup_nI_n\\
            \sum_n\ell(I_n)\leq \lambda^*(X)+\epsilon.
        \end{numcases}
    \end{subequations}
    Si nous posons \( \Omega_{\epsilon}=\bigcup_nI_n\), alors nous avons bien
    \begin{subequations}
        \begin{numcases}{}
            X\subset\Omega_{\epsilon}\\
            \lambda(\Omega_{\epsilon})\leq\sum_n\ell(I_n)\leq \lambda^*(X)+\epsilon.
        \end{numcases}
    \end{subequations}

    En ce qui concerne \ref{ItemGJUoozrDILii}, pour chaque \( k\geq 1\) nous considérons l'ensemble \( \Omega_{1/k}\) obtenu comme précédemment avec \( \epsilon=1/k\) et nous posons \( G=\bigcap_{k\geq 1}\Omega_{1/k}\). Cela est une intersection dénombrable d'ouverts vérifiant \( X\subset G\) (parce que \( X\subset \Omega_{1/k}\) pour tout \( k\)) et donc \( \lambda^*(X)\leq\lambda^*(G)=\lambda(G)\). De plus pour tout \( k\) nous avons 
    \begin{equation}
        \lambda(G)\leq(\Omega_{1/k})\leq \lambda^*(X)+\frac{1}{ k }
    \end{equation}
    pour tout \( k\). En faisant \( k\to \infty\) nous avons
    \begin{equation}
        \lambda(G)\leq \lambda^*(X).
    \end{equation}
    Au final
    \begin{equation}
        \lambda(G)\leq \lambda^*(X)\leq \lambda(G),
    \end{equation}
    d'où l'égalité.
\end{proof}

\begin{corollary}
    Une partir \( N\subset \eR\) est négligeable\footnote{Définition \ref{DefAVDoomkuXi}.} si et seulement si \( \lambda^*(N)=0\).
\end{corollary}

\begin{proof}
    Nous savons que si \( N\) est négligeable il existe un borélien \( Y\) tel que \( N\subset Y\) avec \( \lambda(Y)=0\). Par conséquent\footnote{Au péril d'être lourd nous rappelons que \( \lambda^*\) est défini sur toutes les parties de \( \eR\).}
    \begin{equation}
        \lambda^*(N)\leq \lambda^*(Y)=\lambda(Y)=0.
    \end{equation}
    
    Pour l'implication inverse nous supposons que \( \lambda^*(N)=0\) et nous prenons l'ensemble \( G\) définit par la proposition \ref{PropMXIoojpKvd}\ref{ItemGJUoozrDILii} : c'est un borélien contenant \( N\) et tel que \( \lambda(G)=\lambda^*(N)=0\). L'ensemble \( N\) est donc négligeable.
\end{proof}

\begin{theorem}[Régularité extérieure de la mesure de Lebesgue] \label{ThoHFXooONFRN}
    Pour tout mesurable \( A\subset \eR\) nous avons
    \begin{equation}
        \lambda(A)=\inf\{ \lambda(\Omega);\text{\( \Omega\) ouvert contenant \( A\)} \}.
    \end{equation}
\end{theorem}
\index{régularité!extérieure de la mesure de Lebesgue}

\begin{proof}
    Nous commençons par le cas où \( B\) est un borélien.
    \begin{subproof}

        \item[Si \( B\) borélien, \( \lambda(B)<\infty\)]
        
        Soit \( \epsilon>0\); par la proposition \ref{PropMXIoojpKvd}\ref{ItemGJUoozrDILi} il existe un ouvert \( \Omega_{\epsilon}\) contenant \( B\) tel que \( \lambda(\Omega_{\epsilon})\leq \lambda^*(B)+\epsilon\). Vu qu'ici \( B\) est borélien, \( \lambda^*(B)=\lambda(B)\) et nous concluons que pour tout \( \epsilon\) il existe un ouvert \( \Omega_{\epsilon}\) tel que
        \begin{subequations}
            \begin{numcases}{}
                B\subset\Omega_{\epsilon}\\
                \lambda(\Omega_{\epsilon})\leq \lambda(B)+\epsilon,
            \end{numcases}
        \end{subequations}
        et donc
        \begin{equation}
            \lambda(B)=\inf\{ \lambda(\Omega);\text{ \( \Omega\) ouvert contenant \( B\) } \}.
        \end{equation}
        
        \item[Si \( B\) borélien, \( \lambda(B)=+\infty\)]

            Dans ce cas l'infimum est pris uniquement sur des ouverts \( \Omega\) tels que \( \lambda(\Omega)=\infty\).

        \item[Si \( A\) est mesurable non borélien]
    
    Nous passons maintenant au cas où \( A \) est mesurable sans être borélien. Il s'écrit donc \( A=B\cup N\) avec \( B\) borélien et \( N\) négligeable par la proposition \ref{thoCRMootPojn}, et par définition \( \lambda(A)=\lambda(B)\). Si \( Y\) est un borélien tel que \( N\subset Y\) et \( \lambda(Y)=0\) alors
    \begin{subequations}
        \begin{align}
            \lambda(A)=\lambda(B)&=\inf\{ \lambda(\Omega)\tq \text{ \( \Omega\) ouvert}, B\subset\Omega \}\label{subeqMTHoopkSKOi}\\
            &\leq\inf\{ \lambda(\Omega)\tq \text{ \( \Omega\) ouvert}, B\cup N\subset\Omega \}  \label{subeqMTHoopkSKOii}\\
            &\leq\inf_{\Omega',Y'}\{ \lambda(\Omega'\cup Y')\tq \text{ \( \Omega'\), \( Y'\) ouverts}, B\subset\Omega', Y\subset Y' \}\label{subeqMTHoopkSKOiii}\\
            &\leq\inf_{\Omega',Y'}\{ \lambda(\Omega')+\lambda(Y')\tq \text{ \( \Omega'\), \( Y'\) ouverts},  B\subset\Omega',Y\subset Y' \}\label{subeqMTHoopkSKOiv}\\
            &\leq\inf_{\Omega'}\{ \lambda(\Omega')\tq \text{ \( \Omega'\) ouvert},  B\subset\Omega \}\label{subeqMTHoopkSKOv}\\
            &=\lambda(B).
        \end{align}
    \end{subequations}
    Justifications :
    \begin{itemize}
        \item \eqref{subeqMTHoopkSKOi} Le cas borélien déjà fait.
        \item \eqref{subeqMTHoopkSKOii} Les ouverts \( \Omega\) tels que \( B\cup N\subset \Omega\) vérifient a fortiori \( B\subset \Omega\); nous avons donc agrandit l'ensemble sur lequel l'infimum est pris.
        \item \eqref{subeqMTHoopkSKOiii} Parmi les ouverts \( \Omega\) qui recouvrent \( B\cup N\), il y a ceux de la forme \( \Omega'\cup Y'\) où \( \Omega'\) recouvre \( B\) et \( Y'\) est un ouvert contenant \( Y\). Donc nous avons rétréci l'ensemble sur lequel l'infimum est pris et par conséquent agrandit l'infimum.
        \item \eqref{subeqMTHoopkSKOiv} Mesure d'une union majorée par la somme des mesures.
        \item \eqref{subeqMTHoopkSKOv} Vu que \( Y\) est borélien, \( \lambda(Y)=\inf_{\text{\( Y'\) ouvert}}\{ \lambda(Y')\tq Y\subset Y' \}=0\). Donc pour tout \( \Omega'\) et tout \( \epsilon>0\), nous pouvons trouver un \( Y'\) vérifiant les conditions tel que \( \lambda(\Omega')+\lambda(Y')\leq \lambda(\Omega')+\epsilon\).
    \end{itemize}
    Toutes les inégalités sont des égalités en en particulier \eqref{subeqMTHoopkSKOii} donne
    \begin{equation}
        \lambda(A)=\inf\{ \lambda(\Omega)\tq \text{ \( \Omega\) ouvert}, B\cup N\subset\Omega \},
    \end{equation}
    ce qu'il fallait.
    \end{subproof}
    
\end{proof}

\begin{proposition}[\cite{MesureLebesgueLi}]    \label{PropEZNoofLkVb}
    Si \( A\) est mesurable dans \( \eR\) et si \( \epsilon>0\) alors il existe un ouvert \( \Omega_{\epsilon}\) et un fermé \( F_{\epsilon}\) tels que
    \begin{subequations}    \label{subeqHNEooaNqDu}
        \begin{numcases}{}
            F_{\epsilon}\subset A\subset \Omega_{\epsilon}\\
            \lambda(\Omega_{\epsilon}\setminus F_{\epsilon})\leq \epsilon.
        \end{numcases}
    \end{subequations}
\end{proposition}

\begin{proof}
    Nous commençons par le cas d'un borélien \( B\).
    \begin{subproof}
        \item[Première étape]
        
            Montrons qu'il existe un ouvert \( U_{\epsilon}\) tel que
            \begin{subequations}
                \begin{numcases}{}
                    B\subset U_{\epsilon}\\
                    \lambda(U_{\epsilon}\setminus B)\leq \frac{ \epsilon }{2}.
                \end{numcases}
            \end{subequations}
            Si \( \lambda(B)<\infty\) alors le théorème \ref{ThoHFXooONFRN} nous donne un ouvert \( U_{\epsilon}\) tel que \( B\subset U_{\epsilon}\) et \( \lambda(U_{\epsilon})\leq \lambda(B)+\frac{ \epsilon }{2}\). Nous avons alors
            \begin{equation}
                \lambda(\Omega_{\epsilon}\setminus B)=\lambda(\Omega_{\epsilon})-\lambda(B)\leq \frac{ \epsilon }{2}.
            \end{equation}
            Si par contre \( \lambda(B)=\infty\), nous posons \( B_n=B\cap\mathopen[ -n , n \mathclose]\) et \( \epsilon_n=\epsilon/2^{n+1}\). Pour chaque \( n\) nous avons un ouvert \( \Omega_n\) tel que
            \begin{subequations}
                \begin{numcases}{}
                    B_n\subset \Omega_n\\
                    \lambda(\Omega_n\setminus B_n)\leq \frac{ \epsilon }{ 2^{n+1} }
                \end{numcases}
            \end{subequations}
            Par conséquent en posant \( \Omega=\bigcup_{n\geq 1}\Omega_n\) nous avons\footnote{Nous utilisons la petite relation ensembliste \( \big( \bigcup_nA_n \big)\setminus\big( \bigcup_nB_n \big)\subset \bigcup_n(A_n\setminus B_n)\).}
            \begin{subequations}
                \begin{numcases}{}
                    B\subset \Omega\\
                    \lambda(\Omega\setminus B)\leq \lambda\big( \bigcup_n(\Omega_n\setminus B_n) \big)\leq \sum_{n\geq 1}\lambda(\Omega_n\setminus B_n)=\frac{ \epsilon }{2}.
                \end{numcases}
            \end{subequations}
            La première étape est terminée.

        \item[Deuxième étape]

            Nous prouvons à présent qu'il existe un ouvert \( \Omega_{\epsilon}\) et un fermé \( F_{\epsilon}\) tels que
            \begin{subequations}
                \begin{numcases}{}
                    F_{\epsilon}\subset B\subset \Omega_{\epsilon}\\
                    \lambda(\Omega_{\epsilon}\setminus B)\leq \frac{ \epsilon }{2}\\
                    \lambda(B\setminus F_{\epsilon})\leq \frac{ \epsilon }{2}.
                \end{numcases}
            \end{subequations}
            L'ouvert \( \Omega_{\epsilon}\), nous l'avons déjà de l'étape précédente. Pour le fermé, nous appliquons la première étape au borélien \( B^c\); ce qui nous trouvons est un ouvert \( G_{\epsilon}\) tel que
            \begin{subequations}
                \begin{numcases}{}
                    B^c\subset G_{\epsilon}\\
                    \lambda(G_{\epsilon}\setminus B^c)\leq \frac{ \epsilon }{2}.
                \end{numcases}
            \end{subequations}
            En posant \( F_{\epsilon}=G_{\epsilon}^c\) nous avons un fermé tel que \( F_{\epsilon}\subset B\) et
            \begin{equation}
                \lambda(B\setminus F_{\epsilon})=\lambda(F_{\epsilon}^c\setminus B^c)=\lambda(G_{\epsilon}\setminus B^c)\leq \frac{ \epsilon }{2}.
            \end{equation}
            
        \item[Dernière étape]

            Les ensembles \( F_{\epsilon}\) et \( \Omega_{\epsilon}\) trouvés à la deuxième étape donnent bien les relations \eqref{subeqHNEooaNqDu}. En effet \( \Omega_{\epsilon}\setminus F_{\epsilon}=(\Omega_{\epsilon}\setminus B)\cup(B\setminus F_{\epsilon})\), donc
            \begin{equation}
                \lambda(\Omega_{\epsilon}\setminus F_{\epsilon})\leq \lambda(\Omega_{\epsilon}\setminus B)+\lambda(B\setminus F_{\epsilon})=\epsilon.
            \end{equation}
    \end{subproof}
    Nous passons au cas où \( A=B\cup N\) est mesurable. Nous commençons par prendre les \( \Omega_{\epsilon}\) et \( F_{\epsilon}\) qui correspondent à \( B\) :
    \begin{subequations}
        \begin{numcases}{}
            F_{\epsilon}\subset B\subset \Omega_{\epsilon}\\
            \lambda(\Omega_{\epsilon}\setminus F_{\epsilon})\leq \epsilon.
        \end{numcases}
    \end{subequations}
    Soit \( Y\) un borélien tel que \( N\subset Y\) et \( \lambda(Y)\) puis un ouvert \( Y'\) tel que \( \lambda(Y')\leq \epsilon\) et \( Y\subset Y'\). L'existence d'un tel \( Y'\) est assurée par la proposition \ref{ThoHFXooONFRN} appliquée à \( Y\). Nous vérifions que les ensembles \( F_{\epsilon}\) et \( \Omega_{\epsilon}\cup Y'\) fonctionnent. En effet \( \Omega_{\epsilon}\cup Y'\setminus F_{\epsilon}\subset (\Omega_{\epsilon}\setminus F_{\epsilon})\cup Y'\), donc
    \begin{subequations}
        \begin{numcases}{}
            F_{\epsilon}\subset B\cup N\subset \Omega_{\epsilon}\cup Y'\\
            \lambda\big( (\Omega_{\epsilon}\setminus F_{\epsilon}) \big)\leq \lambda(\Omega_{\epsilon}\setminus F_{\epsilon})+\lambda(Y')\leq 2\epsilon.
        \end{numcases}
    \end{subequations}
    Donc en réalité il faut choisir \( \Omega_{\epsilon/2}\), \( F_{\epsilon/2}\) et \( \lambda(Y')\leq \epsilon/2\).
\end{proof}

\begin{theorem}[Régularité intérieure de la mesure de Lebesgue]
    Si \( A\) est mesurable dans \( \eR\) alors
    \begin{equation}
        \lambda(A)=\sup\{ \lambda(K); \text{\( K\) compact contenu dans \( A\)} \}.
    \end{equation}
\end{theorem}
\index{régularité!intérieure de la mesure de Lebesgue}

\begin{proof}
    Par la proposition \ref{PropEZNoofLkVb} nous avons
    \begin{equation}    \label{EqTPEooUHTbH}
        \lambda(A)=\sup_{\text{\( F\) fermé dans \( A\)}}\lambda(F).
    \end{equation}
    Pour un tel \( F\) nous posons \( K_n=F\cap\mathopen[ -n , n \mathclose]\) qui est compact\footnote{parce que fermé et borné, théorème de Borel-Lebesgue \ref{ThoXTEooxFmdI}.} et contenu dans \( B\). De plus le lemme \ref{LemAZGByEs}\ref{ItemJWUooRXNPcii} nous dit que
    \begin{equation}
        \lambda(F)=\lim_{n\to \infty} \lambda(K_n)
    \end{equation}
    Donc tous les \( \lambda(F)\) peuvent être arbitrairement approchés par un \( \lambda(K)\) avec \( K\) compact dans \( A\), et le supremum \eqref{EqTPEooUHTbH} n'est pas affecté en nous restreignant à prendre des compacts contenus dans \( B\) :
    \begin{equation}    
        \lambda(A)=\sup_{\text{\( F\) fermé dans \( A\)}}\lambda(F)=\sup_{\text{\( K\) compact dans \( A\)}}\lambda(K).
    \end{equation}
\end{proof}

%--------------------------------------------------------------------------------------------------------------------------- 
\subsection{Ensemble de Cantor}
%---------------------------------------------------------------------------------------------------------------------------

Nous considérons la fonction donnant l'écriture décimale des nombres définie en \eqref{EqXXXooOTsCK}.

\begin{definition}[Ensemble de Cantor]  \label{DefIYDooVIDJs}
    Soit \( K_0=\mathopen[ 0 , 1 [\) et les ensembles \( K_n\) définis par la récurrence
        \begin{equation}
            K_{n+1}=\big( \frac{1}{ 3 }K_n \big)\cup\big( \frac{1}{ 3 }(K_n+2) \big).
        \end{equation}
        L'ensemble
        \begin{equation}
            K=\bigcup_{n\geq 0}K_n
        \end{equation}
        est l'\defe{ensemble triadique de Cantor}{Cantor!ensemble}\index{ensemble!de Cantor}.
\end{definition}
Les principales propriétés de l'ensemble de Cantor sont qu'il est non dénombrable (proposition \ref{PropTPPooDySbm}) et borélien de mesure nulle (proposition \ref{PropBEWooXZdKN}).

\begin{normaltext}
    L'idée de base pour prouver que l'ensemble \( K\) est non dénombrable est que ses éléments sont les nombres qui s'écrivent en base \( 3\) sans utiliser le chiffre \( 1\). En prenant un nombre sans \( 1\) écrit en base \( 3\), en changeant tous les \( 2\) en \( 1\) et en lisant le résultat en base \( 2\), nous obtenons tous les nombres possibles en base \( 2\) et donc une quantité non dénombrable. L'idée est donc simple et astucieuse. La mise en musique est un peu plus délicate parce qu'il faut faire attention aux queues de suites; c'est pour cela que nous avons construit l'ensemble de Cantor en partant de \( \mathopen[ 0 , 1 [\) et non de \( \mathopen[ 0 , 1 \mathclose]\).
\end{normaltext}

Le lemme suivant dit précisément ce que nous entendons en disant que les éléments de l'ensemble de Cantor sont les nombres qui s'écrivent en base \( 3\) sans utiliser le chiffre \( 1\).
\begin{lemma}   \label{LemAZGoosKzEm}
    Soit \( n\in \eN\) et \( x\in \eD_3\); nous avons \( \varphi_3(x)\in K_n\in\) si et seulement si \( x_1,\ldots, x_n\in\{ 0,2 \}\).
\end{lemma}

\begin{proof}
    Nous procédons par récurrence en commençant avec \( n=1\). Si \( x_1=1\) alors
    \begin{equation}
        \varphi_3(x)=\frac{1}{ 3 }+\sum_{k=2}^{\infty}\frac{ x_k }{ 3^k }\in\mathopen[ \frac{1}{ 3 } , \frac{ 2 }{ 3 } [.
    \end{equation}
    Notons que \( \varphi_3(x)=\frac{ 2 }{ 3 }\) est impossible parce que ça demanderait une queue de suite de \( 2\). Par conséquent \( \varphi_3(x)=\mathopen[ 0 , 1 [\setminus\mathopen[ \frac{1}{ 3 } , \frac{ 2 }{ 3 } [=K_1\).

        Nous passons à la récurrence. 
        
        \begin{subproof}
        \item[Sens direct]
        
        Nous supposons que \( x_1,\ldots, x_{n+1}\in\{ 0,2 \}\) et nous montrons que \( \varphi_3(x)\in K_{n+1}\). La chose surprenante est que nous n'allons pas considérer deux cas suivant que \( x_{n+1}\) vaut \( 0\) ou \( 1\); nous allons considérer deux cas suivant\footnote{Pour comprendre pourquoi, faire un dessin de comment \( K_n\) se transforme en \( K_{n+1}\) et remarquer dans \( K_2\), les deux premiers segments ne sont pas une division du premier segment de \( K_1\), mais bien une copie des \emph{deux} segments de \( K_1\).} que \( x_1\) vaut \( 0\) ou \( 1\). Écrivons encore \( \varphi_3(x)\) :
    \begin{equation}
        \varphi_3(x)=\sum_{k=1}^{n+1}\frac{ x_k }{ 3^k }+\sum_{k=n+2}^{\infty}\frac{ x_k }{ 3^k }.
    \end{equation}
    \begin{subproof}
        \item[Si \( x_1=0\)]
            Alors nous avons
            \begin{equation}
                3\varphi_3(x)=\sum_{k=2}^{\infty}\frac{ x_k }{ 3^{k-1} }=\sum_{k=1}^{\infty}\frac{ x_{k+1} }{ 3^k }=\varphi_3(x_2,\ldots, x_n,x_{n+1},\ldots)
            \end{equation}
            Vu que par hypothèse \( x_2,\ldots, x_{n+1}\) sont dans \( \{ 0,2 \}\) nous avons \( 3\varphi_3(x)\in K_n\) par hypothèse de récurrence. Cela implique que \( \varphi_3(x)\in K_{n+1}\).
        \item[Si \( x_1=2\)]
            Alors
            \begin{equation}
                \varphi_3(x)=\frac{ 2 }{ 3 }+\sum_{k=2}^{\infty}\frac{ x_k }{ 3^k },
            \end{equation}
            et
            \begin{equation}
                3\varphi_3(x)-2=\sum_{k=1}^{\infty}\frac{ x_{k+1} }{ 3^k }=\varphi(x_2,\ldots, x_{n+1},\ldots),
            \end{equation}
            et donc là nous avons \( 3\varphi_3(x)-2\in K_n\), ce qui implique encore \( \varphi_3(x)\in K_{n+1}\).
    \end{subproof}
    
        \item[Sens réciproque]
        
            Nous devons maintenant prouver que \( \varphi_3(x)\in K_{n+1}\) implique \( x_1,\ldots, x_{n+1}\in\{ 0,2 \}\). Par le même calcul que précédemment nous avons soit
            \begin{equation}
                3\varphi_3(x)=\varphi_3(x_2,\ldots, x_{n+1},\ldots),
            \end{equation}
            si \( x_1=0\), soit
            \begin{equation}
                3\varphi_3(x)-2=\varphi_3(x_2,\ldots, x_{n+1},\ldots),
            \end{equation}
            si \( x_1=2\). Dans les deux cas, si \( x_l=1\) pour un certain \( 2\leq l\leq n+1\), alors l'hypothèse de récurrence donne que ces éléments ne sont pas dans \( K_n\) et donc \( \varphi_3(x)\) pas dans \( K_{n+1}\).

        \end{subproof}
\end{proof}

\begin{corollary}   \label{CorSEDooJmeXt}
    En posant \( \eE=\{ x\in\eD_3\tq x_i\neq 1\forall i \}\) nous avons \( K=\varphi_3(\eE)\). Et plus précisément, \( \varphi_3\colon \eE\to K\) est une bijection.
\end{corollary}

\begin{proof}
    Nous divisons la preuve en trois étapes.
    \begin{subproof}
    \item[Image contenue dans \( K\)]
        Si \( x\in \eE\) et \( n\in \eN\) nous avons \( x_1,\ldots, x_n\in\{ 0,2 \}\) et donc \( \varphi_3(x)\in K_n\) par la proposition \ref{LemAZGoosKzEm}. Donc
        \begin{equation}
            \varphi_3(x)\in\bigcup_{n\geq 1}K_n=K.
        \end{equation}
    \item[Injective]
        L'application \( \varphi_3\colon \eE\to K\) est injective parce qu'elle est déjà injective depuis \( \eD_3\).
    \item[Surjective]
        Soit \( p\in K\subset\mathopen[ 0 , 1 [\). Vu que \( \varphi_3\colon \eD_3\to \mathopen[ 0 , 1 [\) est surjective (théorème \ref{ThoRXBootpUpd}), il existe \( x\in \eD_3\) tel que \( \varphi_3(x)=p\). Pour tout \( n\) nous avons \( \varphi_3(x)\in K_n\) et donc \( x_1,\ldots, x_n\in\{ 0,2 \}\) et donc au final \( x\in \eE\).
    \end{subproof}
\end{proof}

\begin{proposition}\label{PropTPPooDySbm}
    L'ensemble de Cantor est non dénombrable.
\end{proposition}

\begin{proof}

    Nous avons prouvé à la proposition \ref{PropNNHooYTVFw} que l'ensemble \( \eD_2\) n'était pas dénombrable. Nous allons à présent prouver que l'application
    \begin{equation}
        \begin{aligned}
            \psi\colon \eD_2&\to K \\
            c&\mapsto \varphi_3(  \text{\( c\) en remplaçant les \( 1\) par des \( 2\)}  ) 
        \end{aligned}
    \end{equation}
    est une bijection. Le fait que \( \psi\) soit injective est une conséquence du fait que ce soit la composition de deux applications injectives (le remplacement et \( \varphi_3\)). Il faut par contre montrer que l'image est égale à \( K\), en notant qu'il n'est pas évident a priori que l'image soit contenue dans \( K\).

    L'opération qui consiste à remplacer les \( 1\) par des \( 2\) est une bijection \( \eD_2\to \eE\). Le corollaire \ref{CorSEDooJmeXt} nous dit aussi que \( \varphi_3\colon \eE\to K\) est une bijection. En tant que composée de bijections, \( \psi\) est une bijection.

    Étant en bijection avec \( \eD_2\) qui n'est pas dénombrable par la proposition \ref{PropNNHooYTVFw}, l'ensemble de Cantor n'est pas dénombrable.
\end{proof}

\begin{proposition}[Ensemble de Cantor]    \label{PropBEWooXZdKN}
    L'ensemble de Cantor\footnote{Définition \ref{DefIYDooVIDJs}} est borélien, non dénombrable et de mesure nulle.
\end{proposition}

\begin{proof}
    Nous reprenons les notations de la définition \ref{DefIYDooVIDJs}. Le fait que l'ensemble de Cantor soit non dénombrable a été prouvé dans la proposition \ref{PropTPPooDySbm}.

    L'ensemble de Cantor étant une intersection dénombrable de boréliens, il est borélien par le lemme \ref{LemBWNlKfA}. Vu que \( K_n\subset\mathopen[ 0 , 1 [\) nous avons \( \frac{1}{ 3 }K_n\leq \frac{1}{ 3 }\) et \( \frac{1}{ 3 }(K_n+2)\geq \frac{ 2 }{ 3 }\), donc \( K_n\) est une union disjointe de \( 2^n\) intervalles de mesure \( 2/3^n\). Nous avons donc
        \begin{equation}
            \lambda(K_n)=\left( \frac{ 2 }{ 3 } \right)^n.
        \end{equation}
        L'ensemble de Cantor étant contenu dans chacun des \( K_n\), sa mesure est plus petite que la mesure de chacun des \( K_n\) (lemme \ref{LemPMprYuC}) et donc \( \lambda(K)\leq \left( \frac{ 2 }{ 3 } \right)^n\) pour tout \( n\); ergo \( \lambda(K)=0\).
\end{proof}

%--------------------------------------------------------------------------------------------------------------------------- 
\subsection{Ensemble de Vitali (non mesurable)}
%---------------------------------------------------------------------------------------------------------------------------

\begin{example}[Un ensemble non mesurable au sens de Lebesgue]      \label{EXooCZCFooRPgKjj}
    Nous considérons\cite{ooIARBooPdOgAQ} 
    l'ensemble quotient \( \eR/\eQ\); chaque classe intersecte l'intervalle \( \mathopen[ 0 , 1 \mathclose]\). Grâce à l'axiome du choix (voir \ref{NORooLMBYooYjUoju}) nous pouvons construire un ensemble \( V\) contenant un représentant dans \( \mathopen[ 0 , 1 \mathclose]\) de chaque classe. Un tel ensemble est un \defe{ensemble de Vitali}{Vitali (ensemble)}. Nous allons prouver que \( V\) n'est pas mesurable.

    Supposons que \( V\) soit mesurable.Alors tous les ensemble de la forme \( V+q\) (\( q\in \eQ\)) sont mesurables et ont même mesure par la proposition \ref{PropooOACLooLMIUuY}. Nous posons
    \begin{equation}
        A=\bigcup_{\substack{q\in\eQ\\-1\leq q\leq 1}}(V+q)\subset\mathopen[ -1 , 2 \mathclose].
    \end{equation}
    Cela est une union disjointe d'ensembles mesurables. Donc
    \begin{equation}
        \lambda(A)=\sum_{\substack{q\in\eQ\\-1\leq q\leq 1}}\lambda(V+q).
    \end{equation}
    Vu que \( A\subset\mathopen[ -1 , 2 \mathclose]\) nous avons \( \lambda(A)\leq 3\) et donc tous les termes de la somme doivent être nuls. Nous avons donc \( \lambda(A)=0\).

    Prouvons toutefois que \( \mathopen[ 0 , 1 \mathclose]\subset A\), ce qui serait une contradiction. Soit \( x\in\mathopen[ 0 , 1 \mathclose]\); il est dans une des classes de \( \eR/\eQ\) et donc il existe \( v\in V\) tel que \( x-v\in \eQ\). De plus \( x,v\in \mathopen[ 0 , 1 \mathclose]\), donc
    \begin{equation}
        -1\leq x-v\leq 1.
    \end{equation}
    Cela fait que \( x\in V+(x-v)\subset A\). Nous avons donc \( x\in A\) et donc \( \mathopen[ 0 , 1 \mathclose]\subset A\). En conséquence de quoi nous aurions \( \lambda(A)\geq 1\).
\end{example}

%+++++++++++++++++++++++++++++++++++++++++++++++++++++++++++++++++++++++++++++++++++++++++++++++++++++++++++++++++++++++++++ 
\section{Tribu et mesure de Lebesgue sur \texorpdfstring{$ \eR^d$}{Rd}}
%+++++++++++++++++++++++++++++++++++++++++++++++++++++++++++++++++++++++++++++++++++++++++++++++++++++++++++++++++++++++++++

Quelques liens internes :
\begin{itemize}
    \item Le produit de tribus est donné par la définition \ref{DefTribProfGfYTuR},     % Cette référence doit être vers le haut.
    \item le produit d'espaces mesurés est donné par la définition \ref{DefUMlBCAO}.     % Cette référence doit être vers le haut.
\end{itemize}

La mesure de Lebesgue sur \( \eR^d\), notée \( \lambda_N\) est d'abord la mesure définie sur
\begin{equation}
 \big( \eR^d,\Borelien(\eR)\otimes\ldots\otimes\Borelien(\eR) \big)
\end{equation}
comme le produit \( \lambda\otimes\ldots\otimes \lambda\). Ensuite nous nous souvenons du corollaire \ref{CorWOOOooHcoEEF} qui donne \( \lambda_N\) comme une mesure sur
\begin{equation}
 \big( \eR^N,\Borelien(\eR^N) \big).
\end{equation}
Et enfin \( \lambda_N\) est la complétion de cette mesure.

\begin{proposition}[\cite{OYRmzAa}]     \label{PropSKXGooRFHQst}
    Tout ouvert de \( \eR^n\) est une union dénombrable et disjointe de cubes semi-ouverts.
\end{proposition}

\begin{proof}
    Nous allons même montrer que ces cubes peuvent être choisis sur un quadrillage.

    Soit \( G\) un ouvert de \( \eR^n\). Soit \( \{ Q_i^{1} \}_{i\in \eN}\) un découpage de \( \eR^n\) en cubes semi-ouverts de côté \( 1\) et dont les sommets sont en les coordonnées entières. Ils sont de la forme
    \begin{equation}
        \prod_{i=1}^n\mathopen[ n_i , n_i+1 \mathclose[
    \end{equation}
    où les \( n_i\) sont des entiers. Ce sont des cubes disjoints. Nous considérons ensuite pour chaque \( k>1\) le découpage \( \{ Q_i^{(k)} \}_{i\in\eN}\) de \( \eR^n\) en cubes de côtés \( 2^{-k}\) qui consiste à découper en \( 2\) les côtés des cubes du découpage \( Q^{(k-1)}\). Ces cubes forment encore un découpage dénombrable de \( \eR^n\) en des cubes disjoints. Ils sont de la forme
    \begin{equation}
        \prod_{i=1}^n\mathopen[ \frac{ n_i }{ 2^k } , \frac{ n_i+1 }{ 2^k } \mathclose[
    \end{equation}
    où les \( n_i\) sont encore entiers. Ensuite nous considérons \( \mE\) l'union de tous les \( Q_i^{(k)}\) contenus dans \( G\).

    Montrons que \( \mE=G\). D'abord \( \mE\subset G\) parce que \( \mE\) est une union d'ensembles contenus dans \( G\). Ensuite si \( x\in G\), il existe une boule de rayon \( r\) autour de \( x\) contenue dans \( G\); alors un des ensembles \( Q_i^{(k)}\) avec \( 2^{-j}<\frac{ r }{2}\) est contenue dans \( B(x,r)\) et donc dans \( \mE\).

    Bien entendu l'union qui donne \( \mE\) n'est pas satisfaisante par ce que les \( Q_i^{(k+1)}\) sont contenus dans les \( Q_i^{(k)}\); les intersections sont donc loin d'être vides.

    Nous faisons ceci : 
    \begin{subequations}
        \begin{align}
            R^{(0)}&=\{ Q_i^{(1)} \text{contenu dans \( G\)} \}\\
            R^{(k+1)}&=\{ Q_i^{(k+1)}\text{contenus dans \( G\) et pas dans \( R^{(k)}\)} \}.
        \end{align}
    \end{subequations}
    En fin de compte l'union de tous les ensembles contenus dans les \( R^{(k)}\) forment encore \( \eR^n\), mais sont d'intersection vide.
\end{proof}

Les cubes dont il est question dans cette preuve, de côtés \( 2^{-k}\) sont souvent appelés des cubes \defe{dyadiques}{dyadique}.

\begin{corollary}[\cite{OYRmzAa}]     \label{CorTHDQooWMSbJe}
    Tout ouvert de \( \eR^n\) est une union dénombrable de cubes presque disjoints\footnote{«presque» au sens où les intersections éventuelles sont de mesure de Lebesgue nulle.}.
\end{corollary}

\begin{proof}
    Il suffit de prendre les cubes de la proposition \ref{PropSKXGooRFHQst} et de les fermer. Ce que l'on ajoute est de mesure nulle.
\end{proof}

\begin{remark}
    La proposition \ref{PropSKXGooRFHQst} est une propriété seulement de la topologie de \( \eR^n\) alors que le corollaire fait intervenir la mesure de Lebesgue parce qu'il faut bien dire que les intersections sont de mesure (de Lebesgue) nulle.
\end{remark}

%--------------------------------------------------------------------------------------------------------------------------- 
\subsection{Ensembles négligeables}
%---------------------------------------------------------------------------------------------------------------------------

\begin{lemma}[\cite{VSMEooLwNLHd}]      \label{LemWHKJooGPuxEN}
    L'image d'une partie négligeables de \( \eR^N\) par une application Lipschitz est négligeable.
\end{lemma}

\begin{proof}
    Soit \( N\) une partie négligeable de \( \eR^N\) et une application Lipschitz \( f\colon N\to \eR^N\). Soit \( Q\subset \eR^N\) un cube borné de côté \( r\). Pour tout \( x,x'\in N\cap Q\) nous avons
    \begin{equation}
        \| f(x)-f(x') \|\leq C\| x-x' \|\leq Cr.
    \end{equation}
    Donc \( f(N\cap Q)\) est dans une boule de rayon \( Cr\). Mais comme toutes les normes sont équivalentes\footnote{Proposition \ref{PropLJEJooMOWPNi}} sur \( \eR^N\) nous pouvons tout aussi bien prendre la norme \( \| . \|_1\) au lieu de la norme \( \| . \|_2\) (qui est touours la norme prise implicitement lorsqu'on parle de \( \eR^n\)), de telle sorte que les boules soient des cubes. Quoi qu'il en soit, \( f(N\cap Q)\) est contenu dans un cube de coté \( 2Cr\) et au niveau de la mesure extérieure,
    \begin{equation}
        m^*\big( f(N\cap Q) \big)\leq (2Cr)^N=(2C)^Nr^N,
    \end{equation}
    ou encore
    \begin{equation}
        m\big(f(N\cap Q)\big)\leq (2C)^Nm(Q)
    \end{equation}
    parce que \( r^N\) est la mesure du cube \( Q\).

    Soit maintenant \( \epsilon>0\); vu que \( N\) est négligeable, il existe un ouvert \( U\) contenant \( N\) et tel que \( m(U)<\epsilon\). Ce \( U\) est une union presque disjointe de cubes dyadiques \( (Q_n)\) par le corollaire \ref{CorTHDQooWMSbJe}. Nous avons alors
    \begin{subequations}
        \begin{align}
            m^*\big( f(N) \big)&=m^*\big( f(\bigcup_nN\cap Q_n) \big)\\
            &=m^*\big( \bigcup_nf(N\cap Q_n) \big)\\
            &\leq \sum_nm^*(f(N\cap Q_n))\\
            &\leq \sum_n(2C)^Nm(Q_n)\\
            &=(2C)^Nm(U)\\
            &<(2C)^d\epsilon.
        \end{align}
    \end{subequations}
    Au final, \( m^*\big( f(N) \big)\leq (2C)^N\epsilon\).  L'ensemble \( N\) est donc négligeable parce que le lemme \ref{LemXOUNooUbtpxm} le dit : \( m^*(N)=0\).
\end{proof} 

\begin{corollary}
    Un sous-espace vectoriel strict de \( \eR^N\) est négligeable.
\end{corollary}

\begin{proof}
    Un sous-espace vectoriel strict de \( \eR^N\) de dimension \( k<N\) est l'image de
    \begin{equation}
        A=\{ t_1e_1+\ldots +t_ke_k\tq t_i\in \eR \}
    \end{equation}
    par une application linéaire. Ce \( A\) est un pavé de mesure de Lebesgue nulle. Donc l'image est négligeable par le lemme \ref{LemWHKJooGPuxEN}.
\end{proof}

%--------------------------------------------------------------------------------------------------------------------------- 
\subsection{Parties et fonctions mesurables}
%---------------------------------------------------------------------------------------------------------------------------

Pour rappel, la notion d'application de classe \( C^1\) est donnée par la définition \ref{DefJYBZooPTsfZx}.

\begin{proposition}     \label{PropRDRNooFnZSKt}
    Soient \( U\) et \( V\) des ouverts de \( \eR^N\) et \( \phi\colon U\to V\) un \( C^1\)-difféomorphisme. Si \( E\subset U\) est mesurable, alors \( \phi(E)\) est mesurable\footnote{Ici «mesurable» parle de mesurabilité au sens de la tribu de Lebesgue, c'est à dire pas seulement les boréliens.}.
\end{proposition}

\begin{proof}
    Si \( E\) est mesurable, il existe un borélien \( B\) et un ensemble négligeable \( N\) tels que \( E=B\cup N\). Vu que \( \phi\) est un homéomorphisme, l'application \( \phi^{-1}\) est borélienne parce que continue (théorème \ref{ThoJDOKooKaaiJh}). Nous avons 
    \begin{equation}
        \phi(B)=(\phi^{-1})^{-1}(B),
    \end{equation}
    c'est à dire que \( \phi(B)\) est l'image inverse de \( B\) par \( \phi^{-1}\). L'ensemble \( \phi(B)\) est donc borélien.

    Il reste à voir que \( \phi(N)\) est négligeable. Soit \( Q\subset U\) une cube compact. L'application \( d\phi\colon Q\to \aL(\eR^N,\eR^N)\) est continue et donc bornée (par la remarque \ref{RemATQVooDnZBbs}) sur le compact \( Q\). Par les accroissements finis (théorème \ref{ThoNAKKght}), l'application \( \phi\) est donc Lipschitz sur \( Q\). La partie \( \phi(N\cap Q)\) est alors négligeable par le lemme \ref{LemWHKJooGPuxEN}. Pour conclure,
    \begin{equation}
        \phi(N)=\bigcup_i\phi(N\cap Q_i)
    \end{equation}
    où les \( Q_i\) sont tous des cubes compacts. Donc \( \phi(N)\) est une union dénombrable d'ensembles négligeables; ergo négligeable lui-même par le lemme \ref{LemVKNooOCOQw}.
\end{proof}

\begin{proposition}
    Soient \( U\) et \( V\) des ouverts de \( \eR^N\) et \( \phi\colon U\to V\) un \( C^1\)-difféomorphisme. Si \( f\colon V\to \eC\) est mesurable, alors \(f\circ \phi\colon U\to \eC\) l'est.
\end{proposition}

\begin{proof}
    Soit \( A\) une partie mesurable de \( \eC\). Il nous faut prouver que
    \begin{equation}
        (f\circ\phi)^{-1}(A)=\phi^{-1}\big( f^{-1}(A) \big)
    \end{equation}
    soit mesurable. Par hypothèse , \( f^{-1}(A)\) est mesurable. Vu que \( \phi\) est un \( C^1\)-difféomorphisme, elle et son inverse sont mesurables par la proposition \ref{PropRDRNooFnZSKt}. Donc l'image du mesurable \( f^{-1}(A)\) par \( \phi^{-1}\) est encore mesurable. 
\end{proof}

%--------------------------------------------------------------------------------------------------------------------------- 
\subsection{Propriétés d'unicité}
%---------------------------------------------------------------------------------------------------------------------------

\begin{corollary}       \label{CorMPDAooDJRrom}
    La mesure \( \lambda_N\) est l'unique mesure sur \(   (\eR^N,  \Borelien(\eR^N) )   \) à satisfaire 
    \begin{equation}
        \mu\big( \prod_{i=1}^N\mathopen[ a_i , b_i \mathclose] \big)=\prod_{i=1}^n| a_i-b_i |
    \end{equation}
\end{corollary}

\begin{proof}
    Par définition de la mesure produit, \( \lambda_N\) est l'unique mesure sur \(   (\eR^N,  \Borelien(\eR)\otimes\ldots\otimes\Borelien(\eR) )   \) à satisfaire la condition. La proposition \ref{CorWOOOooHcoEEF} conclut.
\end{proof}

Vu que les compacts de \( \eR^n\) sont les fermés bornés (théorème \ref{ThoXTEooxFmdI}), et que tout borné est dans un tel produit d'intervalle, la mesure de Lebesgue est une mesure de Borel (définition \ref{DefFMTEooMjbWKK}\ref{ItemTTPTooStDcpw}).

\begin{theorem}[\cite{PMTIooJjAmWR}]
    La mesure de Lebesgue est invariante par translation. Autrement dit si \( A\) est mesurable dans \( \eR^n\) et si \( a\in \eR^n\) alors \( A+a\) est mesurable et
    \begin{equation}
        \lambda_N(A+a)=\lambda_N(A).
    \end{equation}
\end{theorem}


\begin{proof}
    Nous supposons que \( A\) est borélien; sinon il l'est à ensemble négligeable près. Nous notons \( t_a\) la translation et nous nommons \( \mu\) la mesure donnée par
    \begin{equation}
        \mu(A)=\lambda_N(A+a).
    \end{equation}
    Vu que
    \begin{equation}
        \mu\big( \prod_{n=1}^N\mathopen[ r_n , s_n \mathclose[ \big)=\lambda_N\big( \prod_i\mathopen[ r_n+a_n , s_n+a_n [ \big)=\prod_i| s_n-r_n |.
    \end{equation}
    Vu qu'il y a unicité de la mesure vérifiant cette propriété (corollaire \ref{CorMPDAooDJRrom}), nous avons \( \mu=\lambda_N\).
\end{proof}

Pour la suite nous notons \( Q_0\) le cube unité de \( \eR^N\) : \( Q_0=\big( \mathopen[ 0 , 1 \mathclose[ \big)^N\).

\begin{theorem}[\cite{PMTIooJjAmWR}]        \label{ThoCABFooHbUzWc}
    Soit \( \mu\) une mesure positive sur \( \eR^N\) telle que
    \begin{enumerate}
        \item
            \( \mu\) soit invariante par translation (des boréliens),
        \item
            \( \mu(Q_0)=1\).
    \end{enumerate}
    Alors \( \mu=\lambda_N\).
\end{theorem}
    
\begin{proof}
    Pour simplifier l'écriture nous faisons \( N=2\). Notre but est de prouver que \( \mu(  \mathopen[ 0 , r \mathclose[\times \mathopen[ 0 , r' \mathclose[ )=rr'\) pour tout \( r,r'\in \eR\).

    \begin{subproof}
    \item[Longueur =\( 1/J\)]
        Soient \( J,K\) des entiers. Nous pouvons diviser le cube \( Q_0\) en rectangles de cotés \( 1/J\) et \( A/K\) :
        \begin{equation}
            Q_0=\bigcup_{\substack{1\leq j\leq J\\1\leq k\leq K}}\mathopen[ \frac{ j-1 }{ J } , \frac{ j }{ J } \mathclose[\times \mathopen[ \frac{ k-1 }{ K } , \frac{ k }{ K } \mathclose[
        \end{equation}
        où l'union est disjointe. En ce qui concerne la mesure nous commençons par utiliser la sous-additivité :
        \begin{equation}
            \mu(Q_0)=\sum_{\substack{1\leq j\leq J\\1\leq k\leq K}}\mu\left(  \mathopen[ \frac{ j-1 }{ J } , \frac{ j }{ J } \mathclose[\times \mathopen[ \frac{ k-1 }{ K } , \frac{ k }{ K } \mathclose[      \right).
        \end{equation}
        Nous utilisons ensuite, sur chacun des termes séparément l'invariance par translation selon les vecteurs \( (\frac{ j-1 }{ J },0)\) et \( ( 0,\frac{ k-1 }{ K } )\) :
        \begin{equation}
            1=\mu(Q_0)=\sum_{\substack{1\leq j\leq J\\1\leq k\leq K}}\mu\left(  \mathopen[ 0,\frac{1}{ J } \mathclose[\times \mathopen[0,\frac{1}{ K }\mathclose[      \right)=JK\mu\mu\left(  \mathopen[ 0,\frac{1}{ J } \mathclose[\times \mathopen[0,\frac{1}{ K }\mathclose[      \right),
        \end{equation}
        et donc
        \begin{equation}
            \mu\left(  \mathopen[ 0,\frac{1}{ J } \mathclose[\times \mathopen[0,\frac{1}{ K }\mathclose[      \right)=\frac{1}{ J }\times \frac{1}{ K }.
        \end{equation}
    \item[Longueur \( L/K\)]

        Soient \( L,M\) des entiers et calculons :
        \begin{subequations}
            \begin{align}
                \mu\left( \mathopen[ \frac{ 0 }{ J } , \frac{ L }{ J } \mathclose[\times \mathopen[ \frac{ 0 }{ K } , \frac{ M }{ K } \mathclose[ \right)&=\sum_{\substack{0\leq l\leq L-1\\0\leq m\leq M-1}}\mu\left(   \mathopen[    \frac{ l }{ J },\frac{ l+1 }{ J }  \mathclose[\times \mathopen[ \frac{ m }{ K } , \frac{ m+1 }{ K } \mathclose[      \right)\\
                    &=LM\mu\left(  \mathopen[ \frac{ 0 }{ J } , \frac{ 1 }{ J } \mathclose[\times \mathopen[ \frac{ 0 }{ K } , \frac{ 1 }{ K } \mathclose[  \right)\\
                        &=LM\times \frac{1}{ J }\times \frac{1}{ K }.
            \end{align}
        \end{subequations}
        Nous avons donc, pour tout \( J,K,L,M\) : 
        \begin{equation}
            \mu\left( \mathopen[ 0 , \frac{ L }{ J } \mathclose[\times \mathopen[ 0, \frac{ M }{ K } \mathclose[ \right)=\frac{ L }{ J }\times \frac{ M }{ K },
        \end{equation}
        c'est à dire que pour tout \( r,s\in \eQ^+\) nous avons
        \begin{equation}
            \mu\big(   \mathopen[ 0 , r \mathclose[\times \mathopen[ 0 , s \mathclose[ \big)=rs.
        \end{equation}
    \item[Longueur réelle]
        Nous passons au cas de longueur réelle. Soit \( a>0\) et une suite croissante de rationnels \( r_n\to a\). Une telle suite existe par la proposition \ref{PropSLCUooUFgiSR}. Nous avons \( \mathopen[ 0 , a \mathclose[=\bigcup_{n\geq 1}\mathopen[ 0 , r_n \mathclose[\) où l'union n'est pas disjointe mais croissante, ce qui permet d'utiliser le lemme \ref{LemAZGByEs}\ref{ItemJWUooRXNPci} pour écrire
        \begin{equation}
            \mu\big( \mathopen[ 0 , a \mathclose[ \big)=\mu\left( \bigcup_{n\geq 1}\mathopen[ 0 , r_n \mathclose[ \right)=\lim_{n\to \infty} \mu\big( \mathopen[ 0 , r_n \mathclose[ \big)=\lim_{n\to \infty} r_n=a.
        \end{equation}
    \end{subproof}

    Enfin, si \( a,a'\in \eR\), l'invariance par translation donne
    \begin{equation}
        \mu\big( \mathopen[ a , a' \mathclose[ \big)=\mu\big( \mathopen[ 0 , a'-a \mathclose[ \big)=a'-a.
    \end{equation}
    Par unicité de la mesure ayant cette propriété, nous avons \( \mu=\lambda_N\).
\end{proof}

\begin{corollary}       \label{CorKGMRooHWOQGP}
    Si \( \mu\) est une mesure positive sur \( \eR^N\) invariante par translation et telle que \( \mu(Q_0)=C<\infty\) alors \( \mu=C\lambda_N\).
\end{corollary}

\begin{proof}
    Si \( C>0\) nous considérons la mesure \( \frac{1}{ C }\mu\) qui vérifie \( (\frac{1}{ C }\mu)(Q_0)=1\). En conséquence du théorème \ref{ThoCABFooHbUzWc}, \( \frac{1}{ C }\mu=\lambda_N\) et \( \mu=C\lambda_N\).

    Si au contraire \( C=0\) alors nous pouvons paver \( \eR^N\) avec des cubes \( Q_i\) de côté \( 1\) qui ont tous mesure \( 0\). Par conséquent, \( \eR^N=\bigcup_{i=1}^{\infty}Q_i\), donc \( \mu(\eR^N)=\sum_i\mu(Q_i)=0\). Par conséquent \( \mu=0\) parce que toute partie de \( \eR^N\) a une mesure au maximum égale à celle de \( \eR^N\).
\end{proof}

%--------------------------------------------------------------------------------------------------------------------------- 
\subsection{Régularité}
%---------------------------------------------------------------------------------------------------------------------------

Les différentes notions de régularité pour une mesure sont données dans la définition \ref{DefFMTEooMjbWKK}. Ce sont essentiellement des questions de compatibilité entre la mesure et la topologie.
\begin{proposition}
    La mesure de Lebesgue est une mesure de Radon sur tout ouvert de \( \eR^N\).
\end{proposition}

\begin{proof}
    Soit \( V\) un ouvert de \( \eR^N\). C'est localement compact et dénombrable à l'infini. Il suffit de prouver que \( \lambda_N\) est de Borel sur \( V\) pour que le théorème \ref{PropNCASooBnbFrc} conclue à la régularité de la mesure de Lebesgue.

    Soit \( K\) un compact de \( V\). Par la proposition \ref{PropGBZUooRKaOxy} c'est également un compact de \( \eR^N\). Par conséquent \( K\) est dans un pavé fermé de \( \eR^N\) du type
    \begin{equation}
        K\subset \prod_{n=1}^N\mathopen[ a_n , b_n \mathclose]
    \end{equation}
    et donc en passant par le corollaire \ref{CorMPDAooDJRrom},
    \begin{equation}
        \lambda_N(K)\leq \prod_{i=1}^N(b_n-a_n)<\infty.
    \end{equation}
    Nous avons démontré que \( \lambda_N\) reste fini sur tout compact de \( V\).
\end{proof}
