% This is part of Analyse Starter CTU
% Copyright (c) 2014
%   Laurent Claessens,Carlotta Donadello
% See the file fdl-1.3.txt for copying conditions.

\begin{exercice}\label{exoanalyseCTU-0022}

\emph{On pourra répondre en proposant une représentation graphique possible des fonctions.}
\begin{enumerate}
\item On sait que la fonction $f_{1}$ est strictement croissante sur chacun des intervalles $[a,b]$ et $]b,c]$. Peut-on affirmer qu'elle est strictement croissante sur l'intervalle  $[a,c]$ ?
\item Déterminer une fonction $f_{2}$ et un intervalle $I$ tels que la fonction $f_{2}$ ne soit pas bijective de $I$ dans $f_{2}(I)$.  
\item Déterminer une fonction $f_{3}$ et un intervalle $J$ tels que la fonction $f_{3}$  soit  bijective de $J$ dans $f_{3}(J)$, mais non monotone.  
\end{enumerate}


\corrref{analyseCTU-0022}
\end{exercice}
