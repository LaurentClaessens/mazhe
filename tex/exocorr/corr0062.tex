% This is part of Exercices et corrigés de CdI-1
% Copyright (c) 2011,2017
%   Laurent Claessens
% See the file fdl-1.3.txt for copying conditions.

\begin{corrige}{0062}

\begin{enumerate}

\item
La série
\begin{equation*}
	\sum_{i=1}^\infty \frac{1}{n^2}
\end{equation*}
converge absolument.

\item
On connaît la somme de la série géométrique de raison $q$, on en
  déduit que
  \begin{equation*}
    \sum_{i=0}^\infty \frac1{2^i} = \frac1{1-\frac12} = 2
  \end{equation*}
  comme demandé.

\item
La série harmonique
  \begin{equation*}
    \sum_{i=0}^\infty \frac1i
  \end{equation*}
  est l'exemple standard de cette situation.

\item
Un exemple simple est
  \begin{equation*}
    \sum_{i=0}^\infty {(-1)^i}
  \end{equation*}
  dont la suite des sommes partielles est $(1, 0, 1, 0, 1, \ldots)$.

\item
La série harmonique alternée
  \begin{equation*}
    \sum_{i=0}^\infty {(-1)}^i\frac1i
  \end{equation*}
  est un bel exemple.

\item
Ceci est impossible : si la suite des sommes partielles converge, elle doit être de Cauchy, et en particulier la différence entre deux de ses termes successifs, c'est-à-dire le terme général, doit pouvoir être rendue aussi petite que voulu (voir la proposition \ref{PROPooYDFUooTGnYQg}).  
\end{enumerate}


\end{corrige}
