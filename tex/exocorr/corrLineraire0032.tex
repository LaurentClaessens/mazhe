% This is part of the Exercices et corrigés de mathématique générale.
% Copyright (C) 2009
%   Laurent Claessens
% See the file fdl-1.3.txt for copying conditions.
\begin{corrige}{Lineraire0032}

	\begin{enumerate}

		\item
	Nous avons
	\begin{equation}
		A=\begin{pmatrix}
			1	&	2	\\ 
			0	&	3	
		\end{pmatrix},
	\end{equation}
	et donc
	\begin{equation}
		A-\lambda\mtu=\begin{pmatrix}
			1-\lambda	&	2	\\ 
			0	&	4-\lambda	
		\end{pmatrix},
	\end{equation}
	dont le déterminant est $(1-\lambda)(3-\lambda)$. Les seules valeurs propres possibles sont donc
	\begin{equation}
		\begin{aligned}[]
			\lambda_1&=1\\
			\lambda_2&=3.
		\end{aligned}
	\end{equation}
	Il faut maintenant regarder, pour chacune de ces valeurs si il y a vraiment des vecteurs qui correspondent. Pour $\lambda_1=1$, nous devons résoudre
	\begin{equation}
		\begin{pmatrix}
			0	&	2	\\ 
			0	&	2	
		\end{pmatrix}\begin{pmatrix}
			x	\\ 
			y	
		\end{pmatrix}=\begin{pmatrix}
			0	\\ 
			0	
		\end{pmatrix}.
	\end{equation}
	La réponse est que tous les vecteurs de la forme $\begin{pmatrix}
		x	\\ 
		0	
	\end{pmatrix}$ sont des vecteurs propres de la matrice $A$ pour la valeur propre $1$. Pour $\lambda_2$, nous faisons le calcul
	\begin{equation}
		(A-3\mtu)\begin{pmatrix}
			x	\\ 
			y	
		\end{pmatrix}=\begin{pmatrix}
			-2	&	2	\\ 
			0	&	0	
		\end{pmatrix}\begin{pmatrix}
			x	\\ 
			y	
		\end{pmatrix}=\begin{pmatrix}
			0	\\ 
			0	
		\end{pmatrix},
	\end{equation}
	dont les solutions sont données par $x=y$, c'est-à-dire par tous les vecteurs multiples de $\begin{pmatrix}
		1	\\ 
		1	
	\end{pmatrix}$. Vérifions, juste pour le fun que la matrice $A$ multiplie bien le vecteur $\begin{pmatrix}
		2	\\ 
		2	
	\end{pmatrix}$ par $3$. En effet,
	\begin{equation}
		\begin{pmatrix}
			1	&	2	\\ 
			0	&	3	
		\end{pmatrix}\begin{pmatrix}
			2	\\ 
			2	
		\end{pmatrix}=\begin{pmatrix}
			2+4	\\ 
			0+6	
		\end{pmatrix}=\begin{pmatrix}
			6	\\ 
			6	
		\end{pmatrix}.
	\end{equation}
	Bien joué !
\item
	Passons à la matrice
	\begin{equation}
		B=\begin{pmatrix}
			0	&	1	&	0	\\
			1	&	0	&	0	\\
			0	&	0	&	1
		\end{pmatrix}.
	\end{equation}
	Le déterminant de $B-\lambda\mtu$ est donné par
	\begin{equation}
		\begin{vmatrix}
			-\lambda	&	1	&	0	\\
			1	&	-\lambda	&	0	\\
			0	&	0	&	1-\lambda
		\end{vmatrix}=(1-\lambda)(\lambda^2-1).
	\end{equation}
	Donc les valeurs propres sont données par $\lambda=\pm 1$.

	Commençons par $\lambda=1$. Nous devons résoudre
	\begin{equation}
		\begin{pmatrix}
			-1	&	1	&	0	\\
			1	&	-1	&	0	\\
			0	&	0	&	0
		\end{pmatrix}\begin{pmatrix}
			x	\\ 
			y	\\ 
			z	
		\end{pmatrix}=\begin{pmatrix}
			0	\\ 
			0	\\ 
			0	
		\end{pmatrix}.
	\end{equation}
	Donc $-x+y=0$ et $x-y=0$. La seule contrainte est $x=y$. Il y a deux vecteurs linéairement indépendants qui vérifient cette contrainte :
	\begin{equation}
		\begin{aligned}[]
			\begin{pmatrix}
				1	\\ 
				1	\\ 
				1	
			\end{pmatrix},&\text{et}&\begin{pmatrix}
				1	\\ 
				1	\\ 
				0	
			\end{pmatrix}.
		\end{aligned}
	\end{equation}
	Pour $\lambda=-1$, on doit résoudre
	\begin{equation}
		\begin{pmatrix}
			1	&	1	&	0	\\
			1	&	1	&	0	\\
			0	&	0	&	2
		\end{pmatrix}\begin{pmatrix}
			x	\\ 
			y	\\ 
			z	
		\end{pmatrix}=\begin{pmatrix}
			0	\\ 
			0	\\ 
			0	
		\end{pmatrix}.
	\end{equation}
	La résolution donne $x+y=0$ et $z=0$. L'espace des vecteurs qui vérifient cette condition est engendré par le vecteur $\begin{pmatrix}
		1	\\ 
		-1	\\ 
		0	
	\end{pmatrix}$.

	Les matrices $A$ et $B$ sont diagonalisables parce qu'elles possèdent une base de vecteurs propres.
\item
	Passons à la matrice
	\begin{equation}
		E=\begin{pmatrix}
			1	&	3	&	5	\\
			0	&	1	&	0	\\
			2	&	0	&	10
		\end{pmatrix}.
	\end{equation}
	Nous avons 
	\begin{equation}
		\det(E-\lambda\mtu)=(1-\lambda)^2(10-\lambda)-10(1-\lambda)=-\lambda(\lambda-1)(\lambda-11).
	\end{equation}
	Il faudra donc passer en revue les trois valeurs possibles $\lambda=0$, $\lambda=1$ et $\lambda=11$. Pour $\lambda=0$, nous avons à résoudre
	\begin{equation}
		\begin{pmatrix}
			1	&	3	&	5	\\
			0	&	1	&	0	\\
			2	&	0	&	10
		\end{pmatrix}\begin{pmatrix}
			x	\\ 
			y	\\ 
			z	
		\end{pmatrix}=\begin{pmatrix}
			0	\\ 
			0	\\ 
			0	
		\end{pmatrix},
	\end{equation}
	ce qui donne $x+5z=0$ et $y=0$, donc on a l'espace engendré par le vecteur $\begin{pmatrix}
		1	\\ 
		0	\\ 
		-1/5	
	\end{pmatrix}$. Pour la valeur propre $\lambda=1$, on trouve le vecteur $\begin{pmatrix}
		-9/2	\\ 
		-5/3	\\ 
		1	
	\end{pmatrix}$, et enfin pour $\lambda=11$, il faut résoudre le système
	\begin{equation}
		\begin{cases}
			x+3y+5z=11x\\
			y=11y\\
			2x+10z=11z
		\end{cases},
	\end{equation}
	ce qui donne immédiatement $y=0$ et puis $2x-z=0$. L'espace propre correspondant est engendré par le vecteur $\begin{pmatrix}
		1/2	\\ 
		0	\\ 
		1	
	\end{pmatrix}$. Vérification :
	\begin{equation}
		\begin{pmatrix}
			1	&	3	&	5	\\
			0	&	1	&	0	\\
			2	&	0	&	10
		\end{pmatrix}\begin{pmatrix}
			1/2	\\ 
			0	\\ 
			1	
		\end{pmatrix}=\begin{pmatrix}
			11/2	\\ 
			0	\\ 
			11	
		\end{pmatrix}.
	\end{equation}

\end{enumerate}

\end{corrige}
