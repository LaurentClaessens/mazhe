% This is part of the Exercices et corrigés de CdI-2.
% Copyright (C) 2008, 2009
%   Laurent Claessens
% See the file fdl-1.3.txt for copying conditions.


\begin{corrige}{_I-3-11}

\begin{enumerate}
\item 

Nous allons étudier l'existence de l'intégrale proposée en utilisant le corollaire \ref{CorAlphaLCasInteabf}. Pour cela, remarquons d'abord que $t^{x-1}$ peut avoir un problème en $t=0$, tandis que $(1-t)^{y-1}$ peut avoir un problème en $t=1$. Pour cette raison, nous étudions séparément l'intégrale sur $[0,\frac{ 1 }{2}]$ et sur $[\frac{ 1 }{2},1]$.

Pour étudier l'existence de 
\begin{equation}
	I=\int_0^{\frac{ 1 }{2}}t^{x-1}(1-t)^{y-1},
\end{equation}
nous calculons
\begin{equation}
	\lim_{t\to0}t^{\alpha+x-1}(1-t)^{y-1}=\lim_{t\to 0}t^{\alpha+x-1}=L.
\end{equation}
Pour avoir $L<\infty$, nous avons besoin de $\alpha<1-x$. Si $x>0$, il suffit de prendre $\alpha<1$, et nous avons $L=0$, ce qui prouve l'existence de l'intégrale. Mais si $x\leq 0$, en prenant $\alpha$ entre $1$ et $1-x$, nous avons $L=\infty$ et $\alpha>1$, donc pas d'existence de l'intégrale. Pour la même raison, nous trouvons que l'intégrale n'existe que si $y>0$. En conclusion, $\beta(x,y)$ existe si et seulement si $x>0$ et $y>0$. 

Notez que ce résultat pouvait être deviné très simplement en comptant les degrés : si $x\leq 0$, autour de $0$, le facteur $(1-t)^{y-1}$ reste borne (pour toute valeur de $y$), tandis que $t^{x-1}$ croit plus vite que $\frac{1}{ t }$, ce qui donne la divergence. Dès que $x>0$, la fonction $t^{x-1}$ ne croit plus aussi vite que $\frac{1}{ t }$ en zéro, et il y a convergence. Idem pour $y$.

\item
Le changement de variable $1-t=u$ dans l'intégrale
\begin{equation}
	\beta(y,x)=\int_0^1t^{y-1}(1-t)^{x-1}dt=-\int_1^0u^{x-1}(1-u)^{y-1}du=\beta(x,y).
\end{equation}

\item
Pour commencer, le changement de variable $t=u/(1+1)$ et $dt=du/(1+u)^2$ fait le travail :
\begin{equation}
	\begin{aligned}[]
	\beta(x,y)=\int_0^1t^{x-1}(1-t)^{y-1}dt&=\int_0^{\infty}\left( \frac{ u }{ 1+u } \right)^{x-1}\left( \frac{1}{ 1+u } \right)^{y-1}\frac{ du }{ (1+u)^2 }\\
						&=\int_0^{\infty}\frac{ u^{x-1} }{ (1+u)^{x+y} }du.
	\end{aligned}
\end{equation}
L'intégrale que nous devons calculer maintenant n'est autre que $\beta(3,2)$. En utilisant la formule
\begin{equation}
	\Gamma(x)\Gamma(y)=\beta(x,y)\Gamma(x+y)
\end{equation}
de la page $I.80$  nous trouvons $\beta(3,2)=\Gamma(3)\Gamma(2)/\Gamma(5)$. Nous utilisons maintenant le lien entre $\Gamma$ et la factorielle :
\begin{equation}
	\beta(3,2)=\frac{ \Gamma(3)\Gamma(2) }{ \Gamma(5) }=\frac{ 2!1! }{ 4! }=\frac{1}{ 12 }.
\end{equation}

\end{enumerate}
\end{corrige}
