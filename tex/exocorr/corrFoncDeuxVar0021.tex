% This is part of the Exercices et corrigés de mathématique générale.
% Copyright (C) 2010
%   Laurent Claessens
% See the file fdl-1.3.txt for copying conditions.

\begin{corrige}{FoncDeuxVar0021}

	(IGNE 1121, 9.3) La fonction $f$ proposée est formée des trois fonctions
	\begin{equation}
		\begin{aligned}[]
			f_1(x,y)&=y\sin(x)\\
			f_2(x,y)&= e^{x}y^2\\
			f_3(x,y)&=x^2y.
		\end{aligned}
	\end{equation}
	Il s'agit maintenant de calculer toutes les dérivées partielles au point $(0,1)$ et puis de les mettre en matrice. Nous avons les dérivées partielles
	\begin{equation}
		\begin{aligned}[]
			\frac{ \partial f_1 }{ \partial x }&=y\cos(x)	& \frac{ \partial f_1 }{ \partial y }&=\sin(x)\\
			\frac{ \partial f_2 }{ \partial x }&=e^xy^2	& \frac{ \partial f_2 }{ \partial y }&=2ye^x\\
			\frac{ \partial f_3 }{ \partial x }&=2xy	& \frac{ \partial f_3 }{ \partial y }&=x^2.
		\end{aligned}
	\end{equation}
	En évaluant tout ça au point $(0,1)$ la matrice jacobienne devient
	\begin{equation}
		J_f(0,1)=\begin{pmatrix}
			1	&	0	\\
			1	&	2	\\
			0	&	0	
		\end{pmatrix}.
	\end{equation}
	Lorsqu'on calcule $f(x+a,x+b)$ où l'on suppose que $a$ et $b$ sont petits, on a la notation compacte
	\begin{equation}
		f(x+a,y+b)=f(x,y)+J(x,y)\cdot\begin{pmatrix}
			a	\\ 
			b	
		\end{pmatrix}.
	\end{equation}
	Ici nous avons évidement $x=0$, $y=1$, $a=0.01$ et $b=0.02$ et donc
	\begin{equation}
		f(0+0.01,1+0.02)\simeq f(0,1)+(0.01,0.05,0)=(0.01,1.05,0).
	\end{equation}

\end{corrige}
