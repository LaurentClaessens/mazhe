% This is part of Exercices et corrigés de CdI-1
% Copyright (c) 2011,2014
%   Laurent Claessens
% See the file fdl-1.3.txt for copying conditions.

\begin{corrige}{TP20090003}




\paragraph{Équations paramétriques}
Soit $C$ le cercle d'équation $(x-R)^2 + z^2 = r^2$, avec $r < R$. Il
s'écrit sous forme paramétrique (avec $\varphi \in \eR$)~:
\begin{equation*}
  \begin{cases}
    x = R + r\cos \varphi\\
    y = 0\\
    z = r \sin \varphi
  \end{cases}
\end{equation*}
Considérons la rotation de $\eR^3$ d'angle $\theta$ autour de l'axe
$0z$ (qui est bien disjoint et coplanaire à $C$) : c'est une
transformation linéaire dont la matrice est
\begin{equation*}
  R_\theta =
  \begin{pmatrix}
    \cos \theta & -\sin \theta & 0\\
    \sin \theta & \cos \theta & 0\\
    0 & 0 & 1
  \end{pmatrix}.
\end{equation*}
Le tore $T$ peut alors s'obtenir par rotation des points du cercle $C$ sous
l'action de $R_\theta$ ($\theta \in \eR$)~:
\begin{equation*}
%   \begin{pmatrix}
%     x\\y\\z
%   \end{pmatrix}
%   =
  R_\theta%
  \begin{pmatrix}
    R + r\cos \varphi\\
    0\\
    r \sin \varphi
  \end{pmatrix}
  =
  \begin{pmatrix}
    \cos \theta & -\sin \theta & 0\\
    \sin \theta & \cos \theta & 0\\
    0 & 0 & 1
  \end{pmatrix}
  \begin{pmatrix}
    R + r\cos \varphi\\
    0\\
    r \sin \varphi
  \end{pmatrix}
  =
  \begin{pmatrix}
   (R + r\cos \varphi) \cos \theta\\
   (R + r\cos \varphi) \sin \theta\\
   r \sin \varphi
  \end{pmatrix}
\end{equation*}
d'où les équations paramétriques
\begin{equation}\label{torus_parameq}
  T \equiv
  \begin{cases}
    x = (R + r\cos \varphi) \cos \theta\\
    y = (R + r\cos \varphi) \sin \theta\\
    z = r \sin \varphi
  \end{cases} \qquad \text{où }\varphi,\theta \in \eR
\end{equation}
Ces équations ne fournissent évidemment pas une bijection entre
l'ensemble des paramètres $\varphi,\theta$ et les points du tore. C'est
un revêtement.

\begin{quote}
  Mentionnons au passage l'existence d'une équation sous forme
  implicite~: le système ci-dessus est équivalent à
  \begin{equation*}
    \begin{cases}
      \sqrt{x^2+y^2} - R = r \cos \varphi\\
      z = r \sin \varphi
    \end{cases}
  \end{equation*}
  et donc équivalent à $(\sqrt{x^2+y^2} - R)^2 + z^2 = r^2$. Le tore a
  donc pour équation, en éliminant la racine,
  \begin{equation*}
    (x^2 + y^2 + z^2 + R^2 - r^2)^2 = 4 R^2 (x^2+y^2).
  \end{equation*}
\end{quote}

\paragraph{Cartes et atlas}
Il y a bien entendu une infinité d'atlas faisant du tore une variété,
et un seul suffit à notre bonheur. Nous allons néanmoins donner
plusieurs possibilités, pour montrer plusieurs approches du
problème.

\subparagraph{Version 1} Étant donné le point $p_0 = (x_0,y_0,z_0)$,
nous allons définir une carte $(U,F_{p_0})$, où $U$ sera un ouvert
(relativement au tore) bien choisi autour de $p_0$, et $F_{p_0}$ une
restriction de la paramétrisation ci-dessus.

\begin{quote}
  L'idée est que la formule qui donne la paramétrisation du tore,
  ci-dessus, est un bon candidat ``carte'', mais trois problèmes
  s'enchainent alors~:

  D'abord, ce n'est pas une bijection. Pour pallier à ce problème,
  nous allons restreindre l'espace des paramètres $\varphi,\theta$
  jusqu'à ce qu'on puisse ``inverser'' la formule.

  Deuxième problème, c'est qu'il est difficile d'inverser
  explicitement des formules contenant des sinus et cosinus. Nous
  allons donc appliquer le théorème de la fonction réciproque pour
  nous ``simplifier la vie'' (tout est question de point de vue).

  Troisième problème~: le théorème de la fonction réciproque ne
  fonctionne que d'un ouvert de $\eR^n$ dans un ouvert de $\eR^n$,
  avec le même ``$n$''. Le tore ``vit'' dans $\eR^3$ mais n'en est pas un
  ouvert, nous allons donc l'épaissir (en laissant varier le petit
  rayon $r$) pour obtenir des ``coordonnées toriques'' valables dans
  un voisinage du tore. Ces coordonnées toriques seront inversibles,
  et nous permettrons de nous en sortir.
\end{quote}

Soit $F$ l'application ``coordonnées toriques'' définie par
\begin{equation}
	\begin{aligned}
		F\colon \eR_0^+\times\eR\times\eR&\to \eR^3 \\
		(\rho,\varphi,\theta)&\mapsto \begin{pmatrix}
			(R+r\cos\varphi)\cos\theta	\\ 
			(R+\rho\cos\varphi)\sin\theta	\\ 
			\rho\sin\varphi	
		\end{pmatrix}
	\end{aligned}
\end{equation}
(de sorte que le tore $T$ est obtenu en fixant $\rho = r$ et en
laissant varier $\varphi,\theta$).  Le déterminant de la matrice
jacobienne de $F$, donné par $J = -\rho (R+\rho \cos \varphi)$, est
non nul sur le tore $T$ puisqu'alors $\rho = r > 0$ et $R - \rho
\cos\varphi \geq R - \rho = R - r > 0$.

Soit $p_0 = (x_0,y_0,z_0)$ un point du tore, et soit $(\varphi_0,
\theta_0)$ un couple de réels tels que $F(r, \varphi_0, \theta_0) =
p_0$. Alors la différentielle de $F$ en $(r, \varphi_0, \theta_0)$ est un
isomorphisme linéaire (car le déterminant de la jacobienne est
non-nul), et le théorème de la fonction réciproque nous fournit alors
un ouvert $\tilde W \subset \eR^+_0 \times \eR \times \eR$ autour de
$(r, \varphi_0, \theta_0)$ et un ouvert $\tilde U \subset \eR^3$ autour
de $p_0$ tels que $F_{| \tilde W} : \tilde W \to \tilde U$ est un
difféomorphisme. Notons $G : \tilde U \to \tilde W$ sa réciproque.

Soit $F_{p_0}$ la restriction de $F_{| \tilde W}$ à $\rho = r$, c'est-à-dire
\begin{equation*}
  F_{p_0} : W \subset \eR^2 \to \eR^3 : (\varphi,\theta) \mapsto F(r,\varphi,\theta)
\end{equation*}
où $W = \{ (\varphi,\theta) \in \eR^2 \tq (r,\varphi,\theta) \in \tilde
W\}$, et soit $U = F(W) = \tilde U \cap T$. Alors $(U, F_{p_0})$ est
une carte autour de $p_0$~:
\begin{enumerate}
\item $F_{p_0}$ est continue, car c'est une restriction de $F$ (qui est
  continue).
\item $F_{p_0}$ est une injection de $W$ dans $U$, car c'est une
  restriction de $F_{| \tilde W}$ (qui est un difféomorphisme,
  donc en particulier injective).
\item $F_{p_0}$ est une surjection, par définition de $U$.
\item Par les deux derniers points, $F_{p_0}$ est une bijection ; de
  plus sa réciproque est continue car c'est la restriction de $G$ à
  $U$ (et $G$ est différentiable donc en particulier continue).
\item $F_{p_0}$ est de classe $C^1$ (comme application d'un ouvert de
  $\eR^2$ dans $\eR^3$) car est composée d'applications
  différentiables (sinus, cosinus, produits d'applications
  différentiables,\dots).
\item Le rang de la différentielle de $F_{p_0}$ est maximal, car sa
  matrice jacobienne (en $(\varphi_0,\theta_0)$) est la matrice jacobienne de $F$
  (en $(r,\varphi_0,\theta_0)$) dont on a enlevé la première colonne
  (correspondant aux dérivées par rapport à $\rho$). Or cette dernière
  était de rang trois, il reste donc une matrice de rang deux.
\end{enumerate}

Ceci prouve qu'on a bien une carte autour de $p_0$, et comme on l'a
fait en un point arbitraire du tore, on a bien une variété de
dimension $2$ dans $\eR^3$.

\subparagraph{Version 2} L'atlas ci-dessus est composé de cartes
données par le théorème de la fonction implicite. On peut bien sûr
choisir nous même la carte qu'on veut autour d'un point $p_0$ fixé sur
le tore.

Notons $f$ la paramétrisation du tore~:
\begin{equation}
	\begin{aligned}
		f\colon \eR\times\eR&\to \eR^3 \\
		(\varphi,\theta)&\mapsto \begin{pmatrix}
			(R+r\cos\varphi)\cos\theta	\\ 
			(R+r\cos\varphi)\sin\theta	\\ 
			r\sin\varphi	
		\end{pmatrix}
	\end{aligned}
\end{equation}
et donnons nous $(\varphi_0,\theta_0)$ tels que $p_0 =
f(\varphi_0,\theta_0)$.

On définit
\begin{equation*}
  W = \mathopen]\varphi_0 - \pi; \varphi_0 + \pi\mathclose[
  \times \mathopen]\theta_0 - \pi; \theta_0 + \pi\mathclose[
\end{equation*}
et on pose $U = f(W)$.

Montrons que $f_W : W \to U$ est une bijection~: c'est évidemment
une surjection par définition de $U$, et il reste à montrer qu'elle
est injective. Supposons avoir $(u,v)$ et $(x,y)$ dans $W$ tels que
$f(u,v) = f(x,y)$. C'est-à-dire~:
\begin{equation}
	\begin{aligned}[]
  (R + r\cos u) \cos v 	&= (R + r\cos x) \cos y\\
  (R + r\cos u) \sin v 	&= (R + r\cos x) \sin y \\
  r \sin u		& = r \sin x\\
	\end{aligned}
\end{equation}
En sommant les carrés des deux premières équations on obtient
\begin{equation*}
  (R + r\cos u)^2 = (R + r\cos x)^2
\end{equation*}
ce qui montre que $\cos u = \cos x$ (car $r < R \Rightarrow R + r\cos
x > 0$). D'autre part, on sait que $\sin u = \sin x$. Deux angles dans
un intervalle de longueur $2 \pi$ n'ont même sinus et même cosinus que
s'ils sont égaux, ce qui prouve que $u = x$. Avec cette information,
les deux premières équations donnent alors $v$ et $y$ ont également
même sinus et même cosinus, donc $v = y$. On en déduit $(u,v) =
(x,y)$, et l'injectivité.

Les autres points à montrer pour avoir une carte (continûment
différentiable, inverse continue, de rang maximal) s'obtiennent de la
même manière que dans la première version.

\subparagraph{Version 2 bis}Plutôt que de construire une carte autour
d'un point donné du tore, on peut définir un atlas de quelques cartes
bien choisies recouvrant la totalité du tore. C'est par exemple le cas
si on définit les ouverts ($f$ désigne la paramétrisation, comme ci-dessus)
\begin{equation*}
  \begin{split}
    W_1 &= \mathopen]0; 2 \pi\mathclose[ \times \mathopen]0; 2
    \pi\mathclose[ \qquad U_1 = f(W_1)\\
    W_2 &= \mathopen]-\pi; \pi\mathclose[ \times \mathopen]-\pi;
    \pi\mathclose[ \qquad U_2 = f(W_2)\\
    W_3 &= \mathopen]-\frac\pi2; \frac{3\pi}2\mathclose[ \times \mathopen]-\frac\pi2;
    \frac{3\pi}2\mathclose[ \qquad U_3 = f(W_3)\\
  \end{split}
\end{equation*}

Dans cette situation, $U_1$, $U_2$ et $U_3$ recouvrent bien le
tore. L'application de carte est dans chaque cas la restriction de $f$
à $W_i$, c'est-à-dire $f_i : W_i \to
U_i$. Pour vérifier que chaque $(U_i,f_i)$ définit une carte, on peut
procéder comme ci-dessus.

\subparagraph{Version 3}
Une autre manière de construire des cartes se base sur l'idée qu'un
tore dont on enlève un cercle bien choisi est un cylindre. Et un
cylindre est homéomorphe à un anneau~: si on se donne le cylindre
\begin{equation*}
  C \equiv \{ (x,y,z) \tq x^2 + y^2 = 1, 1 < z < 2 \}
\end{equation*}
et l'anneau
\begin{equation*}
  A \equiv \{ (x,y) \tq 1 < x^2 + y^2 < 4 \}
\end{equation*}
on a l'homéomorphisme~:
\begin{equation*}
  C \to A : (x,y,z) \mapsto (x z, y z)
\end{equation*}
Il n'est alors pas très dur de construire des cartes basées sur ce
principe.

\subparagraph{Version 4 [Idée de Patrick Weber]}
Toujours sur l'idée que construire une carte revient à enlever des
points au tore et obtenir une surface qu'on sait déplier, on peut se
baser sur la paramétrisation 
\begin{equation}
	\begin{aligned}
		f\colon \eR\times\eR&\to \eR^3 \\
		(\varphi,\theta)&\mapsto \begin{pmatrix}
			(R+r\cos\varphi)\cos\theta	\\ 
			(R+r\cos\varphi)\sin\theta	\\ 
			r\sin\varphi	
		\end{pmatrix}
	\end{aligned}
\end{equation}
et enlever le cercle $\varphi=\theta$. Posons
\begin{equation*}
  \begin{split}
    W_1 = \{ (\varphi,\theta) \tq 0 < \varphi < 2\pi \text{et}
    \varphi
    < \theta < \varphi + 2\pi \}\\
    W_2 = \{ (\varphi,\theta) \tq 0 < \varphi < 2\pi \text{et}
    \varphi - \pi < \theta < \varphi + \pi \}
  \end{split}
\end{equation*}
et $U_i = f(W_i)$. Alors pour $i = 1, 2$, $f$ est un homéomorphisme entre $W_i$ et
$U_i$, et on obtient un atlas à deux cartes.

\paragraph{Plan tangent}
\subparagraph{Version 1}
Ayant l'équation du tore sous forme implicite, on peut l'utiliser pour
déterminer le plan tangent au tore en un point $p_0 = (x_0,y_0,z_0)$~:
il est donné par l'équation
\begin{equation}
	\begin{aligned}[]
  &4 (x_02 + y_02 + z_02 - R2 - r2) x_0 (x - x_0)\\ 
  &+ 4 (x_02 +y_02 + z_02 - R2 - r2) y_0 (y - y_0)\\
  &+ 4 (x_02 + y_02 + z_02 + R2 - r2) z_0 (z - z_0) = 0.
	\end{aligned}
\end{equation}
Ceci n'est valable que si la forme implicite est ``la bonne'' (il faut
que le gradient soit non-nul en chaque point du tore), ce qui est vrai
mais est à démontrer.

\subparagraph{Version 2}
On sait que le plan tangent sera le plan perpendiculaire au ``petit
rayon'' du tore. Ce dernier est le vecteur qui relie (en notant encore
$F$ les ``coordonnées toriques'' définites ci-dessus) le point
$F(0,\varphi_0,\theta_0)$ et le point $F(r,\varphi_0,\theta_0)$. Ce
vecteur a pour composantes
\begin{equation*}
  F(r,\varphi_0,\theta_0) - F(0,\varphi_0,\theta_0) = 
  \begin{pmatrix}
    r \cos \varphi \cos \theta,r \cos \varphi \sin \theta,r \sin
    \varphi
  \end{pmatrix}
\end{equation*}
et donc le plan tangent a pour équation
\begin{equation*}
  \cos \varphi \cos \theta (x - (R + r \cos \varphi)\sin\theta) + \cos
  \varphi \sin \theta (y - (R + r \cos \varphi)\cos\theta) + \sin
  \varphi (z - r \sin \varphi) = 0
\end{equation*}

\paragraph{Volume du tore}

\subparagraph{Version 1}
Utilisons les ``coordonnées toriques'' définies ci-dessus pour écrire
le changement de coordonnées
\begin{equation*}
  \begin{cases}
    x = (R + \rho\cos \varphi) \cos \theta\\
    y = (R + \rho\cos \varphi) \sin \theta\\
    z = \rho \sin \varphi%
  \end{cases}
\end{equation*}
pour $0 < \rho < r$, $0 < \varphi < 2\pi$ et $0 < \theta < 2\pi$. Ceci
paramétrise l'intérieur du tore (à un ensemble de mesure nulle près),
et la valeur absolue du jacobien de cette transformation est $\rho (R
+ \rho \cos \varphi)$. Le volume du tore est donc l'intégrale
\begin{equation*}
  \begin{split}
    &\int_0^{2\pi} \int_0^{2\pi} \int_0^r \rho (R + \rho \cos \varphi) d\rho d\varphi  d\theta\\
    = &\int_0^{2\pi} \int_0^{2\pi} \left(\frac12 R r^2 +
      \frac 13 r^3 \cos \varphi\right)  d\varphi  d\theta\\
    = &\int_0^{2\pi}  \pi R r^2  d\theta\\
    = &2 \pi^2 R r^2
  \end{split}
\end{equation*}
avec la remarque que c'est exactement le volume d'un cylindre de
hauteur $2\pi R$ et de rayon $r$, correspondant au tore qu'on aurait
tranché comme un saucisson pour le redresser.


\subparagraph{Version 2}
Une autre approche consiste à coup le tore ``horizontalement''. La
mesure d'une tranche horizontale est l'aire d'un anneau dont le grand
rayon est $R + \sqrt{r^2 - z^2}$ et le rayon intérieur est $R - \sqrt{r^2 - z^2}$
annulaires, et à intégrer pour $z$ entre $-r$ et $r$. C'est-à-dire
\begin{equation}
	\begin{aligned}[]
  		\int_{-r}^r \left(\pi (R+\sqrt{r^2-z^2})^2 - \pi (R-\sqrt{r^2-z^2})^2\right)  d z 	&= \pi \int_{-r}^r 4 R \sqrt{r^2 - z^2}  d z\\
													&= 4 \pi R \int_{-r}^r \sqrt{r^2-z^2}  d z
	\end{aligned}
\end{equation}
où l'intégrale qui reste à calculer est exactement l'aire d'un demi-disque de rayon $r$, c'est-à-dire $\frac{ 1 }{2} \pi r^2$. On en déduit que le
volume du tore est $2 \pi^2 r^2 R$.


% La matrice jacobienne de $F_{\varphi_0,\theta_0}$ est
% \begin{equation*}
%   \begin{pmatrix}
%     - r \cos(\theta) \sin(\varphi) & - r \sin(\theta) \sin(\varphi) & r \cos
%     (\varphi)\\
%     - (R+r\cos(\varphi)) \sin(\theta) & (R + r \cos (\varphi)) \cos(\theta)
%     & 0
%   \end{pmatrix}
% \end{equation*}
% et le déterminant ``des deux premières colonnes'' vaut $-
% (R+r\cos(\varphi)) r \sin(\varphi)$, dans lequel seul le facteur
% $\sin(\varphi)$ peut s'annuler (car $0 \neq r < R$). Lorsque $\sin(\varphi)$
% s'annule, la matrice jacobienne devient (puisqu'alors $\cos(\varphi) = \pm 1$)
% \begin{equation*}
%   \begin{pmatrix}
%     0 & 0 & \pm r\\
%     - (R \pm r ) \sin(\theta) & (R \pm r) \cos(\theta) & 0
%   \end{pmatrix}
% \end{equation*}
% et, comme $\sin(\theta)$ et $\cos(\theta)$ ne sont
% pas nuls en même temps, les deux lignes de cette matrice forment des
% vecteurs indépendants, ce qui prouve que la matrice est de rang maximal pour tout $\theta$.


\end{corrige}
