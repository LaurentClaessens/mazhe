\begin{exercice}\label{exoMatlab0025}

Le professeur Gyros a observé, pendant 12 jours, la croissance d'une population de bactéries dans une assiette de frites. Malheureusement il a égaré une partie des résultats et voudrait les reconstituer à partir du tableau partiel suivant (on ne se préoccupera pas des unités dans lesquelles est exprimée la taille de la population).
\[ \begin{array}{|c|cccccc|}
\hline
\text{Jour } (j) & 1 & 3 & 6 & 7 & 10 & 12 \\
\hline
\text{Population } (P(j)) & 12 & 16 & 30 & 35 & 63 & 90 \\
\hline
\end{array} \]
En supposant que la croissance de population obéisse à peu près à une loi du type $P(j)=Ce^{\lambda j}$, trouvez les coefficients $C$ et $\lambda$ et utilisez ces estimations pour compléter le tableau ci-dessus.

\emph{Indication : L'approximation demandée revient à une approximation par un polynôme du premier degré pour $\ln(P(j))$.}

\corrref{Matlab0025}
\end{exercice}
