% This is part of Outils mathématiques
% Copyright (c) 2011
%   Laurent Claessens
% See the file fdl-1.3.txt for copying conditions.

\begin{corrige}{OutilsMath-0118}

    \begin{enumerate}
        \item
            Lorsqu'on écrit \( \Omega=\mathopen[ 0 , 1 \mathclose]\times \mathopen[ 1 , 2 \mathclose]\), cela signifie que \( x\) va de \( 0\) à \( 1\) et que \( y\) va de \( 1\) à \( 2\). L'intégrale à calculer est donc
            \begin{equation}
                \int_0^1dx\int_1^2dy (x^2y+\sqrt{x}y^2)=\frac{ 37 }{18}.
            \end{equation}
        \item

            L'intervalle de \( x\) est
            \begin{equation}
                x\colon 0\to 1.
            \end{equation}
            Pour chaque \( x\), la variable \( y\) est limitée par \( x^2\), donc
            \begin{equation}
                y\colon 0\to x^2.
            \end{equation}
            D'où l'intégrale
            \begin{equation}
                \begin{aligned}[]
                    \int_0^1dx\int_0^{x^2}dy(x^2+x)\sqrt{y}&=\int_0^1dx(x^2+x)\left[ \frac{ 2 }{ 3 }y^{3/2} \right]_{y=0}^{y=x^2}\\
                    &=\frac{ 2 }{ 3 }\int_0^1(x^2+x)x^3\\
                    &=\frac{ 11 }{ 45 }.
                \end{aligned}
            \end{equation}

            Attention : il ne peut pas rester de \( x\) ou de \( y\) dans la réponse finale ! Ici l'ordre d'intégration est important.
            
    \end{enumerate}
    

\end{corrige}
