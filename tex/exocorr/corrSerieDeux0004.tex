% This is part of Exercices et corrections de MAT1151
% Copyright (C) 2010
%   Laurent Claessens
% See the file LICENCE.txt for copying conditions.

\begin{corrige}{SerieDeux0004}

	\begin{enumerate}

		\item
			La différentielle $df(x)=12x$ est une homothétie de rapport $12$ et sa norme est donc $12$ en vertu de ce que nous avons dit à l'exercice \ref{exoSerieDeux0002}\ref{ItemExoDeuxDeuxa}.
		\item Il est préférable de lire d'abord la résolution du point \ref{ItemCDeuxquatre}. Au vu de ce qui y est dit, il faut calculer $\| \alpha^t(w) \|$ lorsque $\alpha$ est la rotation et $w=(1,2,0)$. Si $\alpha$ est une rotation, c'est une matrice \wikipedia{fr}{Matrice_orthogonale}{orthogonale} et sa transposée est égale à son inverse. En particulier, $\alpha^t$ est encore un opérateur de rotation autour de l'axe $(1,1,1)$. Cette rotation ne change évidement\footnote{Évidemment ?} pas la norme de $w$, et donc la norme de la différentielle est égale à
			\begin{equation}
				\| (1,2,0) \|=\sqrt{5}.
			\end{equation}
			
		\item\label{ItemCDeuxquatre}
			Oublions un instant l'application $\alpha$ et voyons la norme de l'application $x\mapsto\langle x, w\rangle $. Nous devons calculer
			\begin{equation}
				\sup_{| x |=1}\{ | \langle x,w\rangle | \}.
			\end{equation}
			Le produit scalaire est maximum lorsque $x$ est parallèle à $w$. Dans ce cas, elle vaut $| w |$ parce que $\| x \|=1$.

			Que se passe-t-il lorsque l'on met maintenant l'application $\alpha$ ? Il n'est évidement plus du tout garanti que $\alpha(x)$ sera parallèle à $w$ lorsque $x$ est parallèle à $w$. Nous pouvons par contre écrire ceci :
			\begin{equation}
				\langle \alpha(x), w\rangle =\langle x, \alpha^t(w)\rangle 
			\end{equation}
			en utilisant la transposée. Nous tombons donc dans le même cas que si $\alpha$ n'était pas là, mais avec le vecteur $\alpha^t(w)$ au lieu de $w$. La norme recherchée est donc
			\begin{equation}
				| \alpha^t(w) |.
			\end{equation}

		\item
			Nous devons calculer le supremum de la norme du vecteur (dans $\eR^3$) donné par l'équation \eqref{EqActionDiffddeuxtrois} lorsque $\| (v_x,v_y) \|=1$, c'est-à-dire
			\begin{equation}
				\| df_{(1,1)} \|=\sup_{x^2+y^2=1}\{ x^2+y^2+4(x+y)^2 \}.
			\end{equation}
			Pour maximiser la fonction $x^2+y^2+4(x+y)^2$ sur le cercle, nous passons en coordonnées polaires : $x=\cos(\alpha)$ et $y=\sin(\alpha)$. Nous trouvons
			\begin{equation}
				5+8\sin(\alpha)\cos(\alpha)=5+4\sin(2\alpha).
			\end{equation}
			Cette fonction prend son maximum lorsque $\sin(2\alpha)=1$ et donc la norme vaut
			\begin{equation}
				\| df_{(1,1)} \|=9.
			\end{equation}

	\end{enumerate}

\end{corrige}
