% This is part of Exercices et corrigés de CdI-1
% Copyright (c) 2011
%   Laurent Claessens
% See the file fdl-1.3.txt for copying conditions.

\begin{corrige}{EqsDiff0008}

\begin{enumerate}

\item
$y''-6y'+9y=\frac{  e^{3x} }{ x^2 }$. Le polynôme caractéristique de l'homogène a une racine double $x=3$, donc un système fondamental est donné par
\begin{equation}
	\begin{aligned}[]
		y_1(x)&= e^{3x}\\
		y_2(x)&= xe^{3x}.
	\end{aligned}
\end{equation}
Le \href{http://fr.wikipedia.org/wiki/Wronskien}{wronskien} est
\begin{equation}
	W=\det\begin{pmatrix}
	 e^{3x}x &	x e^{3x}	\\ 
	3 e^{3x}	&	 e^{3x}+3x e^{3x}
\end{pmatrix}= e^{6x},
\end{equation}
et les deux autres qu'il faut calculer sont
\begin{equation}
	\begin{aligned}[]
		W_1&=\det\begin{pmatrix}
	0	&	x e^{3x}	\\ 
	\frac{  e^{3x} }{ x^2 }	&	 e^{3x}(1+3x)	
\end{pmatrix}=-\frac{  e^{6x} }{ x },\\
		W_2&=\det\begin{pmatrix}
	 e^{3x}	&	0	\\ 
	3 e^{3x}	&	\frac{  e^{3x} }{ x^2 }
\end{pmatrix}=\frac{  e^{6x} }{ x^2 }.
	\end{aligned}
\end{equation}
En utilisant la méthode usuelle, nous trouvons	
\begin{equation}
	\begin{aligned}[]
		c'_1&=\frac{ W_1 }{ W }=-\frac{1}{ x }\\
		c'_2&=\frac{ W_2 }{ W }=\frac{1}{ x^2 },
	\end{aligned}
\end{equation}
ce qui donne $c_1(x)=-\ln(x)$ et $c_2(x)=-1/x$. Une solution particulière est donc donnée par
\begin{equation}
	y_P(x)=-\ln(x) e^{3x}- e^{3x}.
\end{equation}
Notons que le deuxième morceau est même une solution de l'équation homogène, nous pouvons donc l'oublier dans la solution particulière. En fin de compte, la solution générale est
\begin{equation}
	y(x)=A e^{3x}+Bx e^{3x}-\ln(x) e^{3x}.
\end{equation}
Il n'y a pas de solutions au problème de Cauchy parce que $y(0)$ n'est pas définie. Remarquez que le $v(x)$ est $ e^{3x}/x^2$, donc $0$ est en dehors du domaine de continuité de $v(x)$, et donc les théorèmes ne s'appliquent pas. Il n'est donc pas choquant qu'il n'y ait pas de solutions.

\item
$y''-t= e^{-t}\sin( e^{-t})+\cos( e^{-t})$. L'équation homogène a pour système fondamental $y_1(x)=e^x$ et $y_2(x)= e^{-x}$. Les déterminants à calculer sont
\begin{equation}
	\begin{aligned}[]
		W&=\det\begin{pmatrix}
	e^x	&	 e^{-x}	\\ 
	e^x	&	- e^{-x}	
\end{pmatrix}=-1-1=-2,\\
		W_1&=\det\begin{pmatrix}
	0	&	 e^{-x}	\\ 
	 e^{-x}\sin( e^{-x})+\cos( e^{-x})	&	- e^{-x}	
\end{pmatrix}=- e^{-2x}\sin( e^{-x})+ e^{-x}\cos( e^{-x}),\\
		W_2&=\det\begin{pmatrix}
	e^x	&	0	\\ 
	e^x	&	e^{-x}\sin( e^{-x})+\cos( e^{-x})	
\end{pmatrix}=\sin( e^{-x})+e^x\cos( e^{-x}).
	\end{aligned}
\end{equation}
De là, nous déduisons
\begin{equation}
	\begin{aligned}[]
		c'_1&=\frac{  e^{-2x} }{ 2 }\big( \sin( e^{-x})+ e^{x}\cos( e^{-x}) \big)\\
		c'_2&=-\frac{ 1 }{2}\big( \sin( e^{-x})+ e^{x}\cos( e^{-x}) \big).
	\end{aligned}
\end{equation}
La seconde s'intègre assez facilement et donne $c_2(x)=e^x\cos( e^{-x})/2$, tandis que pour trouver $c_1$, il faut changer de variable et poser $u= e^{-x}$ pour trouver
\begin{equation}
	c_1=-\frac{ 1 }{2}\int\big( u\sin(u)+\cos(u) \big)=-\frac{ 1 }{2}\int\big( u\sin(u) \big)'=-\frac{ 1 }{2} e^{-x}\sin( e^{-x}).
\end{equation}

\item
$y''+y=1/\sin(x)$. Le polynôme caractéristique admet les deux solutions complexes $\pm i$, et donc nous avons le système fondamental réel
\begin{equation}
	\begin{aligned}[]
		y_1=\cos(x)\\
		y_2=\sin(x).
	\end{aligned}
\end{equation}
Les calculs sont assez simples :
\begin{equation}
	\begin{aligned}[]
		W&=\det\begin{pmatrix}
	\cos(x)	&	\sin(x)	\\ 
	-\sin(x)	&	\cos(x)	
\end{pmatrix}=1\\
		W_1&=\det\begin{pmatrix}
	0	&	\sin(x)	\\ 
	\frac{1}{ \sin(x) }	&	\cos(x)	
\end{pmatrix}=-1\\
		W_2&=\det\begin{pmatrix}
	\cos(x)	&	0	\\ 
	-\sin(x)	&	\frac{1}{ \sin(x) }
\end{pmatrix}=\tan(x).
	\end{aligned}
\end{equation}
Après une séance d'intégration, nous trouvons la solution de l'équation différentielle sous la forme
\begin{equation}
	y(x)=-x\cos(x)+\ln\big( \sin(x) \big)\sin(x)+A\cos(x)+B\sin(x).
\end{equation}

\item
$y'''+y= e^{x/2}$. Le polynôme caractéristique admet deux solutions complexes conjuguées :
\begin{equation}
	x=\frac{ 1\pm\sqrt{2}i }{ 2 },
\end{equation}
en plus de la solution réelle $x=-1$. Un système fondamental de solutions réelle est donné par
\begin{equation}
	\begin{aligned}[]
		y_1&= e^{-x}\\
		y_2&= e^{x/2}\cos(\omega x)\\
		y_2&= e^{x/2}\sin(\omega x)\\
	\end{aligned}
\end{equation}
où nous avons posé $\omega=\sqrt{3}/2$. Le Wronskien est
\begin{equation}
	\begin{aligned}[]
	&W=\\
	&\begin{vmatrix}
   e^{-x} 	&	 e^{x/2}\cos(\omega x)	&	 e^{x/2}\sin(\omega x)\\ 
  - e^{-x}	&	 -\frac{ 1 }{2}\big[ \sqrt{3} e^{x/2}\sin(\omega x)- e^{x/2}\cos(\omega x)\big]	& -\frac{ 1 }{2}\big[\sqrt{3} e^{x/2}\sin(\omega x)+ e^{x/2}\cos(\omega x)\big]\\ 
  e^{-x}	&	 -\frac{ 1 }{2}\big[ \sqrt{3} e^{x/2}\sin(\omega x)+ e^{x/2}\cos(\omega x)\big]	& 	 -\frac{ 1 }{2}\big[ \sqrt{3} e^{x/2}\sin(\omega x)- e^{x/2}\cos(\omega x)\big]	  
\end{vmatrix}\\
		&=3\omega.
	\end{aligned}
\end{equation}
Les autres calculs ne sont pas plus faciles et donnent
\begin{equation}
	\begin{aligned}[]
		W_1&=\omega e^{3x/2}\\
		W_2&=-\frac{ 1 }{2}\big( 3\sin(\omega x)+\sqrt{3}\cos(\omega x) \big)\\
		W_3&=-\frac{ 1 }{2}\big( 3\sin(\omega x)+\sqrt{3}\cos(\omega x) \big).
	\end{aligned}
\end{equation}
Nous trouvons ensuite facilement que
\begin{equation}
	c_1(x)=\frac{ 2 }{ 9 } e^{3x/2}.
\end{equation}
Nous trouvons aussi
\begin{equation}
	\begin{aligned}[]
		c'_2(x)&=-\frac{ 3\sin(\omega x)+\sqrt{3}\cos(\omega x) }{ 3\sqrt{3} }\\
		c'_3(x)&=-\frac{ \sqrt{3}\sin(\omega x)-3\cos(\omega x) }{ 3\sqrt{3} }\\
		c_2(x)&=-\frac{1}{ 3\sqrt{3} }\big( 2\sin(\omega x)-2\sqrt{3}\cos(\omega x) \big)\\
		c_3(x)&=-\frac{1}{ 3\sqrt{3} }\big( 2\sqrt{3}\sin(\omega x)+2\cos(\omega x) \big).
	\end{aligned}
\end{equation}
\end{enumerate}

\end{corrige}
