% This is part of Outils mathématiques
% Copyright (c) 2011
%   Laurent Claessens
% See the file fdl-1.3.txt for copying conditions.

\begin{corrige}{OutilsMath-0111}

    Seules les deux faces $x=1$ et $x=3$ sont perpendiculaires au champ de vecteurs donné. Les quatre autres faces sont parallèles et donc de flux nul.

    La première face prend la paramétrisation
    \begin{equation}
        \phi_1(y,z)=\begin{pmatrix}
            1    \\ 
            y    \\ 
            z    
        \end{pmatrix}
    \end{equation}
    avec $(u,z)\in\mathopen[ 1 , 3 \mathclose]\times \mathopen[ 1 , 3 \mathclose]$. Le vecteur normal est $e_x$, mais comme nous voulons suivre la convention de prendre des vecteurs normaux \emph{extérieurs}, nous prenons $-e_x$.
    Nous avons aussi $F\big( \phi_1(y,z) \big)=e_x$, donc
    \begin{equation}
        \Phi_1=\int_1^3dy\int_1^3 e_x\cdot (-e_x)dz=-4.
    \end{equation}
    
    La seconde face est donnée par
    \begin{equation}
        \phi_2(y,z)=\begin{pmatrix}
            3    \\ 
            y    \\ 
            z    
        \end{pmatrix}.
    \end{equation}
    Le vecteur normal est $e_x$, et il pointe vers l'extérieur. Nous avons $F\big( \phi_2(y,z) \big)=\frac{1}{ 3 }e_x$, et donc le second flux est
    \begin{equation}
        Phi_2=\int_1^3dy\int_1^3\frac{1}{ 3 }e_x\cdot (e_x)=\frac{ 4 }{ 3 }.
    \end{equation}
    Le flux total au travers du cube est donné par la somme :
    \begin{equation}
        \Phi=-4+\frac{ 4 }{ 3 }=-\frac{ 8 }{ 3 }.
    \end{equation}

\end{corrige}
