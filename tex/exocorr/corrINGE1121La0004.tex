% This is part of the Exercices et corrigés de mathématique générale.
% Copyright (C) 2009-2010
%   Laurent Claessens
% See the file fdl-1.3.txt for copying conditions.


\begin{corrige}{INGE1121La0004}

	Dans les deux cas, il faut utiliser le fait que $(AB)^{-1}=B^{-1} A^{-1}$ et $(AB)^t=B^tA^t$.
	\begin{enumerate}

		\item
			Calculons le produit $(AB)^t(AB)$ :
			\begin{equation}
				(AB)^t(AB)=B^t\underbrace{A^tA}_{1}B=B^tB=1
			\end{equation}
			parce que $AA^t=BB^t=1$ du fait que $A$ et $B$ soient orthogonales.
		\item
			La matrice $TB^{-1}$ est $T(AT)^{-1}=TT^{-1}A^{-1}=A^{-1}$. Mais on sait que si $A$ est orthogonale, alors $A^{-1}$ est également orthogonale.
	\end{enumerate}

\end{corrige}
