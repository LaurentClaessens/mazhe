% This is part of the Exercices et corrigés de mathématique générale.
% Copyright (C) 2009-2010
%   Laurent Claessens
% See the file fdl-1.3.txt for copying conditions.


\begin{exercice}\label{exoINGE1121La0017}

	(INGE 1121, 6.3) Soit $p$ la forme quadratique associée à la matrice
	\begin{equation}
		A=\begin{pmatrix}
			 2	&	0	&	-2	&	0	\\
			 0	&	5	&	0	&	-1	\\
			 -2	&	0	&	2	&	0	\\ 
			 0	&	-1	&	0	&	5	 
		 \end{pmatrix}.
	\end{equation}
	\begin{enumerate}

		\item
			Calculer le range de $A$.
		\item
			Sachant que $(1,1,-1,1)$ est vecteur propre de $A$, diagonaliser $A$ au moyen d'une matrice orthogonale.

		\item
			Écrire la forme quadratique $X^tAX$ comme somme pondérée de carrés en fonction des $x_i$.
		\item
			Déterminer le genre de la forme quadratique $X\mapsto X^tAX$.
		\item
			Écrire $p$ comme somme pondérée de carrés en fonction de $x_1$, $x_2$, $x_3$ et $x_4$.

	\end{enumerate}

\corrref{INGE1121La0017}
\end{exercice}
