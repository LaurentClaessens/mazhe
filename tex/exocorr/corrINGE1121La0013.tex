% This is part of Exercices et corrigés de mathématique générale.
% Copyright (C) 2009-2010
%   Laurent Claessens
% See the file fdl-1.3.txt for copying conditions.


\begin{corrige}{INGE1121La0013}

	Nous pourrions évidement calculer les valeurs propres de la matrice et en regarder les signes. Hélas, la matrice étant de taille $3$, le polynôme caractéristique sera de degré trois, et il n'est pas certain que nous serons capables de \wikipedia{fr}{Méthode_de_Cardan}{le résoudre}.

	Heureusement, en ce qui concerne la recherche du genre d'une application quadratique, les mineurs principaux font autant l'affaire que les valeurs propres. Ici les mineurs principaux sont
	\begin{equation}
		\begin{aligned}[]
			m_1&=\det\begin{pmatrix}
				3	&	-1	&	-1	\\
				-1	&	3	&	1	\\
				-1	&	1	&	3
			\end{pmatrix},
			&m_2&=\det\begin{pmatrix}
				3	&	-1	\\ 
				-1	&	3	
			\end{pmatrix},
			&m_3=\det(3).
		\end{aligned}
	\end{equation}
	Les calculs vite faits bien faits donnent $m_1=20$, $m_2=10$ et $m_3=3$, de telle sorte qu'ils sont tous strictement positifs. Le genre est donc \emph{défini positif}.

\end{corrige}
