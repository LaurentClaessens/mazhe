\begin{corrige}{GeomAnal-0044}

    Nous considérons l'espace vectoriel normé
    \begin{equation}
        \big( \eM_n(\eR),\| . \| \big).
    \end{equation}
    Nous acceptons que cet espace est \defe{complet}{complet}, c'est à dire que toute suite de Cauchy dans cet espace converge vers un élément de cet espace.

    \begin{enumerate}
        \item
            Soit la suite des sommes partielles dans \( \eM_n(\eR)\)
            \begin{equation}
                x_n=\sum_{k=0}^nB^k.
            \end{equation}
            Nous allons prouver que cette suite est de Cauchy : \( x_n-x_m=\sum_{k=n}^{m}B^k\), nous avons
            \begin{equation}
                \| x_n-x_m \|\leq\sum_{k=n}^m\| B^k \|\leq \sum_{k=n}^m\| B \|^k.
            \end{equation}
            Étant donné que \( \| B \|\) est un réel plus petit que \( 1\), nous sommes dans le cas d'une série de puissance réelles et nous savons que pour tout \( \epsilon>0\), il existe \( N\in\eN\) tel que \( n,m\geq N\) implique
            \begin{equation}
                \sum_{k=n}^m\| B \|^k<\epsilon.
            \end{equation}
            Nous savons donc que la suite \( (x_n)\) est de Cauchy, et nous acceptions que cela prouve qu'elle converge vers un élément de \( \eM_n(\eR)\).

            Prouvons à présent que \( \sum_{k=0}^{\infty}B^k\) est l'inverse de \( \mtu-B\). Pour cela, remarquons que
            \begin{subequations}
                \begin{align}
                    (\mtu-B)\sum_{k=0}^NB^k&=\sum_{k=0}^NB^k-\sum_{k=1}^{N+1}B^k\\
                    &=B^0-B^{N+1}\\
                    &=\mtu-B^{N+1}.
                \end{align}
            \end{subequations}
            Par conséquent
            \begin{equation}
                \begin{aligned}[]
                    (\mtu-B)\sum_{k=0}^{\infty}B^k&=\lim_{N\to\infty}(\mtu-B)\sum_{k=0}^NB^k\\
                    &=\lim_{N\to\infty}\mtu-B^{N+1}.
                \end{aligned}
            \end{equation}
            Mais nous avons \( \lim_{N\to\infty}B^{N+1}=0\) parce que \( \| B^{N+1} \|\leq \| B \|^{N+1}\to 0\).

        \item
            Si \( \lambda\) est une valeur propre de \( A\), alors la matrice \( A-\lambda\mtu\) n'est pas inversible. Évidement si \( \lambda=0\) nous avons \( \lambda<\| A \|\). Nous supposons que \( \lambda\neq 0\). La matrice
            \begin{equation}
                \mtu-\frac{ A }{ \lambda }
            \end{equation}
            n'est pas non plus inversible, ce qui implique que
            \begin{equation}
                \left\| \frac{ A }{ \lambda } \right\|\geq 1,
            \end{equation}
            et par conséquent \( \| A \|\geq \lambda\).
    \end{enumerate}

\end{corrige}
