% This is part of Exercices et corrigés de CdI-1
% Copyright (c) 2011
%   Laurent Claessens
% See the file fdl-1.3.txt for copying conditions.

\begin{corrige}{00035}

Une propriété fondamentale du supremum est la suivante: pour tout $\epsilon>0$, 
\begin{equation}
	\begin{aligned}[]
		\exists a\in A&\tq a\geq \sup(A)-\epsilon/2\\
		\exists b\in B&\tq b\geq \sup(B)-\epsilon/2.
	\end{aligned}
\end{equation}
Pour tout $\epsilon$, nous avons donc, pour $a$ et $b$ bien choisis dans $A$ et $B$,
\begin{equation}
	\sup(A)+\sup(B)-\epsilon\leq a+b\leq\sup(A+B).
\end{equation}
Par conséquent\footnote{Il est important que vous méditiez sur le fait que si $x+\epsilon\geq y$ pour tout $\epsilon>0$, alors $x\geq y$.}, nous avons
\begin{equation}		\label{EqAplusBsupAsupBIneqPrem}	
	\sup(A)+\sup(B)\leq\sup(A+B).
\end{equation}
Afin d'obtenir l'inégalité dans l'autre sens, regardons ce que l'on peut dire de $\sup(A+B)-\epsilon$. Nous écrivons la propriété fondamentale du supremum :
\begin{equation}
	\exists x\in A+B\tq \sup(A+B)-\epsilon<x,
\end{equation}
en d'autres termes (puisque $x\in A+B$), il existe $a\in A$ et $b\in B$ tels que
\begin{equation}		\label{EqAplusBsupAsupB}
	\sup(A+B)-\epsilon\leq a+b\leq\sup(A)+\sup(B).
\end{equation}
La dernière équation est juste le fait que $a\leq\sup(A)$ et $b\leq\sup(B)$. Les inégalités \eqref{EqAplusBsupAsupB} étant vraies pour tout $\epsilon$, nous avons
\begin{equation}
	\sup(A+B)\leq\sup(A)+\sup(B).
\end{equation}
Ceci combiné à l'inégalité \eqref{EqAplusBsupAsupBIneqPrem} donne l'égalité attendue.

\end{corrige}
