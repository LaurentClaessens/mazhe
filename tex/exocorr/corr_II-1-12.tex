\begin{corrige}{_II-1-12}

	Récrivons l'équation du facteur intégrant \eqref{EqDuFacteurIntegrant} :
	\begin{equation}		\label{EqFactIntGenII112}
		M(\partial_yP-\partial_tQ)=Q(\partial_tM)-P(\partial_yM).
	\end{equation}

	\begin{enumerate}

		\item
		      Nous allons prouver que l'équation du facteur intégrant accepte une solution lorsque $M$ est homogène de degré $-(n+1)$, c'est-à-dire quand $M$ satisfait l'équation
		      \begin{equation}
			      t\partial_tM+y\partial_yM=-(n+1)M.
		      \end{equation}
		      Nous remplaçons dans \eqref{EqFactIntGenII112} toute les dérivées par rapport à $y$ par des dérivées par rapport à $t$ en utilisant les formules d'Euler :
		      \begin{equation}
			      \begin{aligned}[]
				      \partial_yP & \to \frac{1}{ y }(nP-t\partial_tP)                 \\
				      \partial_yM & \to \frac{1}{ y }\big( -(n+1)M-t\partial_tM \big).
			      \end{aligned}
		      \end{equation}
		      En remettant les termes ensembles, nous trouvons
		      \begin{equation}
			      \frac{ MP }{ y }+\frac{ t }{ y }\partial_t(MP)+\partial_t(MQ).
		      \end{equation}
		      Les deux premiers termes du membre de gauche peuvent être mit en une seule dérivée par rapport à $t$ en remarquant que
		      \begin{equation}
			      \frac{ MP }{ y }+\frac{ t }{ y }\partial_t(MP)=\partial_t\left( \frac{ tMP }{ y } \right).
		      \end{equation}
		      Ce que nous trouvons alors est $\partial_t\left( \frac{ tMP }{ y }+MQ \right)=0$, et donc
		      \begin{equation}
			      M(Q+tP/y)=C.
		      \end{equation}
		      Attention : la constante est une constante \emph{par rapport à $t$}, elle peut dépendre de $y$. Ce que nous avons est
		      \begin{equation}
			      M=\frac{ C(y) }{ tP+Qy }
		      \end{equation}
		      où $C(y)$ est une fonction de $y$. Cela est la forme que doit avoir $M$ pour satisfaire à l'équation du facteur intégrant. Pour que $M$ soit, de plus homogène de degré $-(m+1)$, il faut en plus que
		      \begin{equation}
			      M(\lambda t,\lambda y)=\lambda^{-m-1}M(t,y),
		      \end{equation}
		      ce qui va imposer des restrictions sur $C(y)$. Le but de l'exercice est de prouver que cette restriction peut être satisfaire par un choix convenable de $C(y)$. En utilisant l'homogénéité de $P$ et $Q$, nous trouvons
		      \begin{equation}
			      M(\lambda t,\lambda y)=\frac{ M(\lambda y) }{ \lambda^{n+1}(tP+yQ) }.
		      \end{equation}
		      Donc, nous en déduisons que pour tout $y$ et pour tout $\lambda$, $C(y)=C(\lambda y)$. Une fonction constante satisfait à cette exigence. Donc nous avons un facteur intégrant sous la forme
		      \begin{equation}
			      M(t,y)=\frac{1}{ tP+yQ }.
		      \end{equation}

		\item
		      Nous reprenons l'équation $P+Qy'=0$, et nous posons $v=ty$. L'équation du facteur intégrant devient
		      \begin{equation}
			      M\Big( \partial_y \big( yp(ty) \big)-\partial_t \big( tq(ty) \big)  \Big)=Q(\partial_t M)-P(\partial_y M).
		      \end{equation}
		      Nous utilisons la règle de Leibniz et la règle de la dérivation de fonctions composées pour effectuer les dérivations. Le membre de gauche devient
		      \begin{equation}
			      M\big(p-q+v\partial_v(p-q)\big),
		      \end{equation}
		      tandis que pour calculer le membre de droite, nous nous souvenons de l'hypothèse $M=M(ty)=M(v)$, et nous pouvons donc écrire $\partial_y M=(\partial_vM)t$, ce qui mène à l'équation
		      \begin{equation}
			      M\big(p-q+v\partial_v(p-q)\big)=-v\frac{ \partial M }{ \partial v }(p-q).
		      \end{equation}
		      Si $f=p-q$, nous avons donc l'équation
		      \begin{equation}
			      \frac{ M }{ \partial_vM }=\frac{ \partial_vf }{ f },
		      \end{equation}
		      dont les solutions sont $M=k/f$.


		\item Dans le premier cas, si $P$ et $Q$ sont homogènes de même degré, alors l'équation $P+qy'=0$ se récrit $y'=-P/Q$, qui est homogène de degré zéro, et nous pouvons suivre la méthode donnée au point \ref{SubSecEqDiffHomo}.

		      Dans le second cas, nous avons l'équation
		      \begin{equation}
			      yp(v)dt+tq(v)dy=0,
		      \end{equation}
		      et l'hypothèse nous permet de dire que $ydt=dv-tdy$, ce qui met l'équation sous la forme
		      \begin{equation}
			      (dv-tdy)p+tqdy=0,
		      \end{equation}
		      dans laquelle il n'y a plus de $dt$. Cette équation peut être mise sous la forme
		      \begin{equation}
			      \frac{ dy }{ y }=\frac{ p }{ v(p-q) }\frac{ dv }{ v },
		      \end{equation}
		      qui se résout par quadrature.

	\end{enumerate}

\end{corrige}
% This is part of the Exercices et corrigés de CdI-2.
% Copyright (C) 2008, 2009
%   Laurent Claessens
% See the file fdl-1.3.txt for copying conditions.

