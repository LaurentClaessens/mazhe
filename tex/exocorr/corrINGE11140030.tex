% This is part of Un soupçon de physique, sans être agressif pour autant
% Copyright (C) 2006-2009, 2019
%   Laurent Claessens
% See the file fdl-1.3.txt for copying conditions.


\begin{corrige}{INGE11140030}

	\begin{enumerate}

		\item
			Ici, il faut multiplier et diviser par la binôme conjugué et d'appliquer le produit remarquable pour faire disparaitre les racines du numérateur.
			\begin{equation}
				\begin{aligned}[]
					\sqrt{x^2+1}-\sqrt{x^2-1}	&=		\frac{ \left(\sqrt{x^2+1}-\sqrt{x^2-1}\right) \left(\sqrt{x^2+1}+\sqrt{x^2-1}\right)  }{  \left(\sqrt{x^2+1}+\sqrt{x^2-1}\right)  }\\
					&=\frac{ x^2+1-(x^2-1) }{  \left(\sqrt{x^2+1}+\sqrt{x^2-1}\right)  }\\
					&=\frac{ 2 }{  \left(\sqrt{x^2+1}+\sqrt{x^2-1}\right)  }.
				\end{aligned}
			\end{equation}
			À partir de cette dernière expression, il est clair que
			\begin{equation}
				\lim_{x\to \pm\infty} =0.
			\end{equation}

		\item
			Ici, l'astuce est de multiplier et diviser par le binôme conjugué de ce qui se trouve au numérateur~:
			\begin{equation}
				\frac{ \sqrt{x+4}-2 }{ x }= \frac{ \left( \sqrt{x+4}-2 \right)\left( \sqrt{x+4}+2 \right)}{ x\left( \sqrt{x+4}+2 \right) }.
			\end{equation}
			L'avantage de cette façon de faire est qu'on peut maintenant utiliser le produit remarquable $(A+B)(A-B)=A^2-B^2$ au numérateur pour faire disparaitre la racine. Après simplifications, nous nous retrouvons  avec
			\begin{equation}
				\frac{1}{ \sqrt{x+4}+2 },
			\end{equation}
			dont la limite pour $x\to 0$ est $\frac{1}{ 4 }$.

	\end{enumerate}

\end{corrige}
