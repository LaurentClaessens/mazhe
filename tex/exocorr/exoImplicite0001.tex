% This is part of Exercices et corrigés de CdI-1
% Copyright (c) 2011, 2020
%   Laurent Claessens
% See the file fdl-1.3.txt for copying conditions.

\begin{exercice}\label{exoImplicite0001}

Soit $F: \eR^2 \times \eR_0^+  \rightarrow  \eR : (x,y,z) \mapsto  z +
\ln(z) - xy$
\begin{enumerate}
\item
Prouver qu'il existe une fonction $Z(x,y)$ dans un voisinage de $(1,1)$
telle que 
\begin{equation}
    F(x,y,Z(x,y))=0
\end{equation}
Prouver l'unicité d'une telle fonction.
\item
Calculer $(\partial_x Z)(x,y)$, $(\partial_y Z)(x,y)$ et $(\partial^2_{xy}
Z)(x,y)$ en fonction de $x$, $y $, $Z(x,y)$.
\end{enumerate}

\corrref{Implicite0001}
\end{exercice}
