\begin{corrige}{EspVectoNorme0006}

  \begin{description}
    \item[Première méthode]  Soit $(x_n, y_n)_{n\in\eN}$ une suite dans $A$ convergente dans $\eR^2$  et appellons $(\bar x,\bar y)$ sa limite. Comme le produit est une application bilinéaire continue on a que $\lim_{n} x_ny_n= \bar x \bar y$, donc $\bar x \bar y=1$. En outre, comme toutes les normes de $\eR^2$ sont équivalentes, nous pouvons écrire que la convergence de $(x_n, y_n)_{n\in\eN}$ signifie que pour tout $\varepsilon>0$ fixé il existe un $\bar n$ tel que si $n\geq \bar n$ alors  
\begin{equation}
  \|(x_n, y_n)-(\bar x, \bar y)\|_\infty\leq \varepsilon,
\end{equation}
cela implique la convergence des suites $(x_n)_n$ et $(y_n)_n$ vers $\bar x$ et $\bar y$ respectivement, parce que pour tout $n\geq \bar n$
\begin{equation}
 |x_n-\bar x|\leq \max\{|x_n-\bar x|,|y_n-\bar y| \}= \|(x_n, y_n)-(\bar x, \bar y)\|_\infty\leq \varepsilon,
\end{equation} 
le même calcul étant valable pour $\|y_n-\bar y\|$. Le point $(\bar x,\bar y)$ est donc dans dans la fermeture de la region $x>0$, $y>0$, qui est $[0,\infty[\times [0,\infty[$. De plus la condition $\bar x \bar y=1$ nous dit que ni $\bar x$ ni $\bar y$ est égal à zéro, donc $\bar x>0$ et $\bar y>0$. Bref, le point  $(\bar x, \bar y)$ est dans $A$. Nous disons alors que $A$ est fermé parce qu'il contient tous ses points d'accumulation.  
 \item[Deuxième méthode] L'ensemble $A$ est la branche d'hyperbole $y=1/x$ située dans la partie $x>0$ du plan. La méthode usuelle pour prouver qu'un ensemble est fermé est de prouver que son complémentaire est ouvert, et la méthode usuelle pour prouver qu'un ensemble est ouvert est de montrer qu'autour de chaque point de l'ensemble, il existe une boule contenue dans l'ensemble.

	Les points du complémentaire de $A$ sont les points $(a,b)$ de $\eR^2$ qui satisfont une des trois propriétés suivantes
	\begin{subequations}
		\begin{align}
			a\leq 0\\
			b\leq 0\\
			ab\neq 1		\label{subEqxyneq106}.
		\end{align}
	\end{subequations}

	Si $ab\neq 1$, il existe une boule autour de $(a,b)$ dans laquelle tout $(x,y)$ satisfait $xy\neq 1$. En effet, nous pouvons considérer la fonction $f(x,y)=xy-1$. Si cette fonction est strictement positive en un point, alors elle reste strictement positive dans une boule (proposition \ref{Propfaposfxposcont}). Le problème est déjà réglé pour tous les points $(x,y)$ qui satisfont à la propriété \eqref{subEqxyneq106}. 

	Si un point $(a,b)$ est hors de $A$ parce que $a\leq 0$, alors soit $a=0$ (dans ce cas, $ab\neq 1$ et nous retombons dans le cas précédent), soit $a<0$. Dans ce dernier cas, il existe une boule centrée en $(a,b)$ qui reste dans le demi-plan $x<0$.

	Le cas $b<0$ se traite de la même manière. Nous avons donc montré qu'autour de chaque point hors de $A$, il existe une boule qui n'intersecte pas $A$. Cela signifie que le complémentaire de $A$ est ouvert, et donc que $A$ est fermé.

  \end{description}
\end{corrige}
