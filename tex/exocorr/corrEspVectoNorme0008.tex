\begin{corrige}{EspVectoNorme0008}

	Nous considérons la fonction $g\colon \eR^2\to \eR$ donnée par $g(x,y)=f(x)-y$. Cela est une fonction continue parce que c'est une différence de fonctions continues. Par définition,
	\begin{equation}
		A=\{ (x,y)\in\eR^2\tq g(x,y)>0 \}.
	\end{equation}
	Par la proposition \ref{Propfaposfxposcont}, autour de chaque point $(x,y)$ tel que $g(x,y)>0$ (c'est-à-dire autour de chaque point de $A$), il existe une boule sur laquelle $g$ reste strictement positive. L'ensemble $A$ est donc ouvert.

	Pour prouver que l'ensemble $B$ est fermé, prouver que le complémentaire est ouvert, c'est-à-dire que les points tels que $y-f(x)<0$ forment un ouvert. Cela revient au même que ce que nous avons fait pour $A$.

\end{corrige}
