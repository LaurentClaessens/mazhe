% This is part of Exercices de mathématique pour SVT
% Copyright (c) 2010-2011,2015
%   Laurent Claessens et Carlotta Donadello
% See the file fdl-1.3.txt for copying conditions.

\begin{corrige}{TD3-0009}

	C'est la combinaison des propositions \ref{Propufulimite} et \ref{Propsuiteborncv} qui va nous permettre de nous en sortir.

	Lorsque $u_{n+1}$ est donné directement en fonction de $u_n$, c'est-à-dire lorsque $u_{n+1}=f(u_n)$, les «candidats limites» sont les nombres $u$ tels que
	\begin{equation}
		u=f(u).
	\end{equation}
	Dans le cas qui nous intéresse ici, les candidats seront donnés par l'équation
	\begin{equation}		\label{Equauupbp}
		u=\frac{ au }{ (1+u)^b }.
	\end{equation}
	La première solution à cette équation est $u=0$. Afin de trouver les solutions différentes de zéro, nous pouvons simplifier l'équation \eqref{Equauupbp} par $u$:
	\begin{equation}
		1=\frac{ a }{ (1+u)^b },
	\end{equation}
	et donc $(1+u)^b=a$, c'est-à-dire
	\begin{equation}
		u=a^{1/b}-1.
	\end{equation}
	Dans tous les cas nous savons donc que la limite de la suite (si elle existe !) sera donnée par zéro ou par $a^{1/b}-1$.

	\begin{enumerate}
		\item
			Nous allons prouver que la suite est décroissante et bornée vers le bas par zéro. Cela prouvera qu'elle est convergente. Après, il faudra déterminer si la limite est zéro ou bien $a^{1/b}-1$.

			Montrons que $u_n>0$ pour tout $n$. Nous faisons cela par récurrence. D'abord l'énoncé dit que $u_0$ est positif. Ensuite, si $u_n$ est positif, alors
			\begin{equation}
				u_{n+1}=\frac{ 1 }{2}\frac{ u_n }{ (1+u_n)^b }
			\end{equation}
			est également positif.

			Pour prouver la décroissance, nous calculons
			\begin{equation}
				\frac{ u_{n+1} }{ u_n }=\frac{ a }{ (1+u_n)^b }=\frac{ 1 }{2}\frac{ 1 }{ (1+u_n)^b }.
			\end{equation}
			Maintenant, le truc est de remarquer que si $u_n>0$, alors, $1+u_n>1$ et donc $(1+u_n)^b>1$. Cela fait que la fraction est plus petite que $1$ et donc que
			\begin{equation}
				\frac{ u_{n+1} }{ u_n }<1.
			\end{equation}
			Cela signifie que $u_n>u_{n+1}$ et donc que la suite est décroissante.

            Nous savons maintenant que la suite est convergente. Les deux candidats à être limite sont $0$ et $a^{1/b}-1=(\frac{ 1 }{2})^{1/b}-1<0$. Étant donné que nous avons prouvé que la suite reste positive, le second candidat limite n'est pas possible (il est strictement négatif parce que $(\frac{1}{ 2 })^{1/b}$ est toujours négatif). La limite est donc zéro.

		\item
			Si $a=2$, $b=1$ et $u_0=2$, nous avons comme candidats les nombres $0$ et $2^{1}-1=1$. Notre but est donc de montrer que la suite est convergente et d'écarter la possibilité que la limite soit zéro.

			La stratégie sera donc de montrer que la suite est toujours plus grande ou égale à $1$, et de montrer qu'elle est décroissante. Comme ça elle sera convergente (décroissante et bornée vers le bas), et la limite ne pourra pas être $0$.

			Supposons que $u_n>1$. Alors $u_{n+1}=\frac{ 2u_n }{ 1+u_n }>1$. En effet, $2u_n-(1+u_n)=u_n-1>0$, ce qui signifie que $2u_n>(1+u_n)$ et donc que $\frac{ 2u_n }{ 1+u_n }>1$. La suite est donc bornée vers le bas par $1$. Elle est également décroissante parce que
			\begin{equation}
				\frac{ u_{n+1} }{ u_n }=\frac{ 2 }{ 1+u_n }<1
			\end{equation}
			si $u_n>1$.

			La suite est donc décroissante, bornée vers le bas par $1$ (donc convergente) et ses candidats limites sont $0$ et $1$. La seule limite possible est donc $1$.

		\item
			La suite est donnée par
			\begin{equation}
				u_{n+1}=\frac{ 2u_n }{ 1+u_n }
			\end{equation}
			et $u_0=\frac{ 1 }{2}$. Les candidats limites sont $u=0$ et $u=a^{1/b}-1=2-1=1$. Étant donné que la suite part de $\frac{ 1 }{2}$, nous devrions montrer que la suite est croissante et bornée par le haut. 

			Pour la croissance, supposons que $u\leq 1$ et calculons
			\begin{equation}
				\frac{ u_{n+1} }{ u_n }=\frac{ 2 }{ 1+u_n }\geq \frac{ 2 }{ 1+1 }=1.
			\end{equation}
			Donc tant que la suite est plus petite que $1$, elle est croissante.

            Pour prouver que la suite est bornée par $1$, nous allons prouver que si $\frac{ 1 }{2}\leq u_n\leq 1$, alors $u_{n+1}\leq 1$. En effet demander \( u_{n+1}\leq 1\) est équivalent à demander
            \begin{equation}
                2u_n\leq 1+u_n,
            \end{equation}
            parce que \( 1+u_n>0\). C'est-à-dire \( u_n\leq 1\). Donc par récurrence tous les termes de la suite sont plus petits ou égaux à \( 1\). Or nous venons de dire que tant que la suite est plus petite que \( 1\), elle est croissante. La suite est donc toujours croissante.
    
			La suite étant croissante et bornée par $1$,elle est convergente. Mais comme les deux seuls candidats sont $0$ et $1$, vu qu'on part de $\frac{ 1 }{2}$, la seule limite possible est $1$.
		\item
			Les candidats limites sont $u=0$ et $u=2^{1/(1/2)}-1=2^2-1=3$. La suite part de $2$. Donc si nous pouvions montrer que la suite est croissante et bornée, ce serait gagné.

			La suite est donnée par
			\begin{equation}
				u_{n+1}=\frac{ 2u_n }{ \sqrt{1+u_n} }
			\end{equation}
            Nous montrons que tant que \( u_n\leq 3\), la suite est croissante. Il existe plusieurs possibilités.

            \begin{enumerate}
                    \item
                Nous avons la majoration
                \begin{equation}        \label{Eqmaktzzzncn}
                    \frac{ u_{n+1} }{ u_n }=\frac{ 2 }{ \sqrt{1+u_n} }\geq \frac{ 2 }{ \sqrt{1+3} }=1.
                \end{equation}
                La suite est donc croissante tant qu'elle est en dessous de \( 3\). Il peut encore arriver que la suite dépasse \( 3\). Nous montrons à présent qu'il n'en est rien. Nous allons montrer que si \( u_n\) est plus petit que \( 3\), alors \( u_{n+1}\) est encore plus petit que \( 3\).

                Pour cela nous considérons la fonction
                \begin{equation}
                    f(x)=\frac{ 2x }{ \sqrt{1+x} }.
                \end{equation}
                C'est une fonction croissante pour les \( x\) positifs parce que sa dérivée vaut
                \begin{equation}
                    \frac{ 2 }{ \sqrt{1+x} }-\frac{ x }{ (1+x)^{3/2} }=\frac{ 2(x+1)-x }{ (x+1)^{3/2} }=\frac{ x+2 }{ (x+1)^{3/2} }\geq 0.
                \end{equation}
                Donc la plus grande valeur de \( f(x)\) pour \( x\in\mathopen[ 0 , 3 \mathclose]\) est \( f(3)=3\). En particulier, la plus grande valeur que pourrait prendre
                \begin{equation}
                    u_{n+1}=\frac{ 2u_n }{ \sqrt{1+u_n} }=f(u_n)
                \end{equation}
                lorsque \( u_n\) est plus petit que $3$ est \( 3\). La suite est donc bornée par \( 3\).
            
                \item
                    Une autre façon de prouver le fait que la suite soit bornée par \( 3\) est de procéder par l'absurde en supposant que la suite passe par une valeur plus grande que \( 3\). Tout d'abord nous pouvons perfectionner la majoration \eqref{Eqmaktzzzncn} en écrivant 
                    \begin{equation}
                        \frac{ u_{n+1} }{ u_n }=\frac{ 2 }{ (1+u_n)^{1/2} }.
                    \end{equation}
                    Si \( u_n>3\), la suite est décroissante. Supposons que \( u_{n}<3\), \( u_{n+1}>3\) et calculons \( u_{n+2}\). Nous nous attendons à avoir \( u_{n+2}<u_{n+1}\) parce que \( u_{n+1}>3\). Calculons :
                    \begin{equation}
                        \begin{aligned}[]
                            n_{n+2}&=\frac{ 2u_{n+1} }{ (1+u_{n+1})^{1/2} }\\
                            &=u_{n+1}\underbrace{\frac{ 2(1+u_n)^{1/4} }{ \big( (1+u_n)^{1/2}+2u_n \big)^{1/2} }}_{A}.
                        \end{aligned}
                    \end{equation}
                    Nous allons montrer que si \( u_n<3\), alors \( A>1\), ce qui signifierais que \( u_{n+2}>u_{n+1}\) et contredirait le fait que la suite est croissante tant que \( u_n>3\). Demander \( A>1\) revient à demander
                    \begin{equation}
                        2(1+u_n)^{1/4}<\big( (1+u_n)^{1/2}+2u_n \big)^{1/2}.
                    \end{equation}
                    En élevant deux fois au carré nous trouvons l'inéquation
                    \begin{equation}        \label{EqzztzzznqunA}
                        -4u_n^2+9u_n+9>0.
                    \end{equation}
                    Le polynôme du membre de gauche n'est positif que entre \( -3/4\) et \( 3\). Donc si \( u_n>3\), l'inégalité \eqref{EqzztzzznqunA} n'est jamais satisfaite. Par conséquent \( A<1\) et \( u_{n+2}<u_{n+1}\), ce qui contredit le fait que \( u_n>3\).

                        
                \end{enumerate}
            
	\end{enumerate}
	
\end{corrige}
