\begin{corrige}{CalculDifferentiel0003}

	La première chose à faire est de savoir la forme générale d'une application linéaire de $\eR^n$ dans $\eR$. Tout vecteur $x\in\eR^n$ s'écrit sous la forme $x=\sum_{i=1}^nx_ie_i$ où $\{ e_i\}$ est la base canonique de $\eR^n$. En utilisant la linéarité de $L$,
	\begin{equation}
		L(x)=\sum_i L(x_ie_i)=\sum_ix_iL(e_i)=\sum_i a_ix_i
	\end{equation}
	où nous avons nommé $a_i=L(e_i)$. Maintenant $L$ est continue parce que c'est une somme de fonctions continues (dont les projections $x\mapsto x_i$). En ce qui concerne les dérivées partielles, nous calculons en utilisant $L(x)=\sum_ia_ix_i$ et en utilisant la linéarité de la dérivée :
	\begin{equation}
		\begin{aligned}[]
			\frac{ \partial L }{ \partial x_k }(x)&=\sum_ia_i\frac{ \partial x_i }{ \partial x_k }\\
			&=\sum_ia_i\delta_{ik}\\
			&=a_k
		\end{aligned}
	\end{equation}
	parce que la dérivée de $x_i$ par rapport à $x_k$ vaut $1$ si $i=k$ et vaut zéro si $i\neq k$.

	Pour montrer qu'elle est différentiable, nous allons procéder en deux étapes. D'abord nous allons trouver un candidat pour la différentielle, et ensuite nous allons voir que ce candidat fonctionne. Le candidat est donné par les formules du lemme \ref{LemdfaSurLesPartielles} :
	\begin{equation}
		dL(x).u=\sum_i\frac{ \partial L }{ \partial x_i }(x)u_i=\sum_ia_iu_i=L(u).
	\end{equation}
	Cela nous dit que \emph{si $L$ est différentiable}, alors sa différentielle est $L$ elle-même. Testons donc cela dans le critère \eqref{EqCritereDefDiff} :
	\begin{equation}
		\lim_{h\to 0} \frac{ \| L(a+h)-L(a)-T(h) \| }{ \| h \| }
	\end{equation}
	où $T$ est notre candidat différentielle, c'est-à-dire $T(h)=L(h)$. En utilisant la linéarité de $L$, nous voyons que $L(a+h)=L(a)+L(h)$ et donc le numérateur est identiquement nul. Cela prouve que $L$ est différentiable et que la différentielle est bien celle que nous avons devinée.

\end{corrige}
