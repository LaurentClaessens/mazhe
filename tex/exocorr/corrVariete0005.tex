% This is part of Exercices et corrigés de CdI-1
% Copyright (c) 2011,2017
%   Laurent Claessens
% See the file fdl-1.3.txt for copying conditions.

\begin{corrige}{Variete0005}

	Attention : exercice difficile\footnote{Si vous trouvez une méthode plus simple, merci de me le faire savoir} !

	Le lagrangien du problème est
	\begin{equation}
		L(x,y,z,\lambda)=x^2+y^2+z^2+\lambda(x^2+y^2+xy+x+y+z-1-z^2),
	\end{equation}
	et les équations à résoudre sont
	\begin{equation}
		\left\{
		\begin{array}{ll}
			2x+2\lambda x+\lambda y+\lambda=0\\
			2y+2\lambda y+\lambda x+\lambda =0\\
			2z+\lambda-2\lambda z=0\\
			x^2+y^2+xy+x+y+z-1-z^2=0.
		\end{array}
		\right.
	\end{equation}
	Il est possible de résoudre les trois premières équations pour $x$, $y$ et $z$ :
	\begin{equation}
		\begin{aligned}[]
			x&=-\frac{ \lambda }{ 2\lambda+2 },&y&=-\frac{ \lambda }{ 3\lambda+2 },&z&=\frac{ \lambda }{ 2\lambda-2 }.
		\end{aligned}
	\end{equation}
	En remettant dans la dernière équation, cela donne une équation épouvantable pour $\lambda$:
	\begin{equation}
		33\lambda^4-22\lambda^3-44\lambda^2-8\lambda+16=0.
	\end{equation}
	En réfléchissant beaucoup, on peut prouver que les solutions sont sont comprises entre $1/2$ et $3/2$. Cela fait que $| x |<\frac{ 3 }{ 7 }$, $| y |<\frac{ 3 }{ 7 }$ et $| z |<\frac{ 3 }{ 2 }$, et donc que
	\begin{equation}
		f(x,y,z)<\frac{ 513 }{ 196 }\simeq 2.61.
	\end{equation}
	
		

\end{corrige}
