% This is part of Exercices et corrigés de CdI-1
% Copyright (c) 2011
%   Laurent Claessens
% See the file fdl-1.3.txt for copying conditions.

\begin{corrige}{OutilsMath-0050}

    L'exercice revient à maximiser la fonction $d$ donnée. Pour ce faire, on dérive:
    \begin{equation}
        d'(\alpha)=\frac{ 2v_0^2 }{ g }\big( \cos^2(\alpha)-\sin^2(\alpha) \big).
    \end{equation}
    Note : ne pas croire que la parenthèse fait $1$. Nous avons $d'(\alpha)=0$ si
    \begin{equation}
        \cos(\alpha)=\pm\sin(\alpha).
    \end{equation}
    Si nous ne considérons que les angles $\alpha\in\mathopen[ 0 , \frac{ \pi }{ 2 } \mathclose]$, la seule solution est $\alpha=\frac{ \pi }{ 4 }$.  Afin d'être certain que cet angle donne un maximum (et non un minimum ou un point d'inflexion), il faut faire le tableau de signe de $d'$ pour $\alpha$ entre $0$ et $\frac{ \pi }{2}$.

    Étant donné que pour $\alpha=0$, nous avons $\cos(\alpha)>\sin(\alpha)$ et que pour $\alpha=\pi/2$ nous avons $\sin(\alpha)>\cos(\alpha)$, nous voyons que $\alpha=\frac{ \pi }{ 4 }$ est bien un maximum de $d$.

    La distance atteinte est donc donnée par
    \begin{equation}
        d\big( \frac{ \pi }{ 4 } \big)=\frac{ 2v_0^2 }{ g }\cos\frac{ \pi }{ 4 }\sin\frac{ \pi }{ 4 }=\frac{ v_0^2 }{ g }.
    \end{equation}

\end{corrige}
