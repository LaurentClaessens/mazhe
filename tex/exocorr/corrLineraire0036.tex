% This is part of the Exercices et corrigés de mathématique générale.
% Copyright (C) 2009
%   Laurent Claessens
% See the file fdl-1.3.txt for copying conditions.
\begin{corrige}{Lineraire0036}

	Nous avons
	\begin{equation}
		\det\begin{pmatrix}
			1-\lambda	&	0	\\ 
			0	&	i-\lambda	
		\end{pmatrix}=(1-\lambda)(i-\lambda),
	\end{equation}
	et donc les deux valeurs propres sont $1$ et $i$. Pour $\lambda=1$, on résou
	\begin{equation}
		\begin{pmatrix}
			0	&	0	\\ 
			0	&	i-1	
		\end{pmatrix}\begin{pmatrix}
			x	\\ 
			y	
		\end{pmatrix}=\begin{pmatrix}
			0	\\ 
			0	
		\end{pmatrix},
	\end{equation}
	donc $x$ est libre et $(i-1)y=0$, c'est-à-dire que nous avons le vecteur
	\begin{equation}
		\begin{pmatrix}
			1	\\ 
			0	
		\end{pmatrix}.
	\end{equation}
	Pour la valeur propre $\lambda=i$, par contre, nous résolvons
	\begin{equation}
		\begin{pmatrix}
			1-i	&	0	\\ 
			0	&	0	
		\end{pmatrix}\begin{pmatrix}
			x	\\ 
			y	
		\end{pmatrix},
	\end{equation}
	donc $y$ est libre et $(1-u)x=0$, c'est-à-dire $x=0$. Le vecteur propre associé est donc
	\begin{equation}
		\begin{pmatrix}
			0	\\ 
			1	
		\end{pmatrix}.
	\end{equation}
	Il y a deux vecteurs linéairement indépendants, donc c'est une base et la matrice est diagonalisable.

	Faisons maintenant la matrice
	\begin{equation}
		B=\begin{pmatrix}
			1	&	1	\\ 
			0	&	i	
		\end{pmatrix}.
	\end{equation}
	Ses valeurs propres sont encore $\lambda_1=1$ et $\lambda_2=i$. Pour $\lambda_1=1$, il faut résoudre $Av=v$, c'est-à-dire
	\begin{equation}
		\begin{pmatrix}
			1	&	1	\\ 
			0	&	i	
		\end{pmatrix}\begin{pmatrix}
			x	\\ 
			y	
		\end{pmatrix}=\begin{pmatrix}
			x	\\ 
			y	
		\end{pmatrix},
	\end{equation}
	et la solution est $x+y=x$ et $iy=y$. Cela donne $y=0$ et $x$ libre, c'est-à-dire qu'une base de cet espace propre est $\begin{pmatrix}
		1	\\ 
		0	
	\end{pmatrix}$. Pour $\lambda_2=i$, on doit résoudre $Av=iv$, c'est-à-dire $x+y=ix$ et $iy=it$. Pas de contraintes sur $y$, mais $x=y/(1-i)$. Nous avons donc le vecteur propre
	\begin{equation}
		\begin{pmatrix}
			1/(1-i)	\\ 
			1	
		\end{pmatrix}=\begin{pmatrix}
			i+1	\\ 
			2	
		\end{pmatrix}.
	\end{equation}

\end{corrige}
