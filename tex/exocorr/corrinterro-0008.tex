% This is part of Exercices de mathématique pour SVT
% Copyright (C) 2010
%   Laurent Claessens et Carlotta Donadello
% See the file fdl-1.3.txt for copying conditions.

\begin{corrige}{interro-0008}

	\begin{enumerate}
		\item
			Nous avons
			\begin{equation}
				(f\circ g)(x)=f\big( g(x) \big)=g(x)^2-2g(x)=\frac{1}{ (1+x)^2 }-\frac{ 2 }{ 1+x }.
			\end{equation}
			Le domaine de définition est donné par $x\neq 1$ à cause du dénominateur.
		\item
			Nous avons
			\begin{equation}
				(g\circ f)(x)=g\big( f(x) \big)=\frac{1}{ 1+f(x) }=\frac{1}{ x^2-2x+1 }.
			\end{equation}
			Le domaine de définition est donné par $\eR$ moins les $x$ tels que $x^2-2x+1=0$. Étant donné que $x^2-2x+1=(x-1)^2$, le domaine est $x\neq 1$.
	\end{enumerate}

\end{corrige}
