\begin{corrige}{_II-1-14}

\begin{enumerate}

\item $y''+\frac{ 3y' }{ t+1 }+\frac{ y }{ (t+1)^2 }=0$.
Après multiplication par $(t+1)^2$, nous avons l'équation sous forme usuelle :
\begin{equation}	\label{EqEulerSSFormeII114}
	y''(t+1)^2+3y'(t+1)+y=0
\end{equation}
Nous posons $\tau=\ln(t+1)$. Ici, afin d'éviter des confusions entre les dérivations par rapport à $t$ et celles par rapport à $\tau$, il est bon de changer également de nom pour la fonction inconnue :
\begin{equation}
	z(\tau)=y\big( t(\tau) \big),
\end{equation}
ou bien
\begin{equation}
	z\big( \tau(t) \big)=y(t).
\end{equation}
Maintenant nous calculons $y'(t)$ et $y''(t)$. Pour la première dérivée, nous avons
\begin{equation}
	\begin{aligned}[]
	y'(t)	&=\frac{ d }{ dt }\big( y(t) \big)\\
		&=\frac{ d }{ dt }\left( z\big( \tau(t) \big) \right)\\
		&=z'\big( \tau(t) \big)\frac{ d\tau }{ dt }(t)\\
		&=z'\big( \tau(t) \big)\frac{ 1 }{ t+1 },
	\end{aligned}
\end{equation}
et pour la seconde, nous utilisons Leibnitz :
\begin{equation}
	\begin{aligned}[]
		y''(t)	&=\frac{ d }{ dt }\left( \frac{ z'\big( \tau(t) \big) }{ t+1 } \right)\\
			&=\frac{1}{ (t+1)^2 }\Big[  z''\big( \tau(t) \big)-z'\big( \tau(t) \big) \Big].
	\end{aligned}
\end{equation}
En remplaçant cela et $y(t)=z\big( \tau(t) \big)$, l'équation de départ devient alors
\begin{equation}
	z''+2z'+z=0.
\end{equation}
Pour résoudre cette équation, nous calculons les racines du polynôme caractéristique $r^2+2r+1=0$, qui n'a qu'une racine double : $r=-1$. Le système fondamental de solutions est donc $z_1(\tau)= e^{-\tau}$ et $z_2(\tau)= \tau e^{-\tau}$. Nous trouvons donc
\begin{equation}
	z(\tau)=A e^{-\tau}+B\tau e^{-\tau}.
\end{equation}
Nous pouvons maintenant effectuer le changement de variable inverse en utilisant $y(t)=z\big( \tau(t) \big)$ :
\begin{equation}
	y(t)=(A+B\tau(t)) e^{-\tau(t)},
\end{equation}
et donc $y(t)=\big( B\ln(t+1)+A \big)(t+1)^{-1}$.


\item
$y''+\frac{ 3y' }{ t+1 }+\frac{ y }{ (t+1)^2 }=1$.
Cette équation ne diffère de la précédente que par le second membre. Nous repartons donc de l'équation \eqref{EqEulerSSFormeII114} dans laquelle il faut modifier le second membre :
\begin{equation}
	y''(t+1)^2+3y'(t+1)+y=(t+1)^2.
\end{equation}
Nous connaissons déjà la solution générale de l'équation homogène associée. Il reste à trouver une particulière. Avec la variable $\tau$, le second membre est $ e^{2\tau}$. L'équation dont il nous faut une solution particulière est
\begin{equation}
	z''+2y'+y= e^{2\tau}.
\end{equation}
On essaye $z_P(\tau)=a e^{2\tau}$. C'est très vite vu que $a=\frac{1}{ 9 }$ est ce qu'il faut.

\item
$y''=\frac{ a+b-1 }{ t }y'-\frac{ ab }{ t^2 }y$.
Nous posons $\tau=\ln(t)$, et nous tombons sur l'équation
\begin{equation}
	z''-(a+b)z'+abz=0.
\end{equation}
L'équation caractéristique a comme solutions $r_1=a$ et $r_2=b$. Si $a\neq b$, alors ce sont deux solutions distinctes.

\begin{enumerate}
\item Si $a\neq b$. Dans ce cas, nous avons $y(t)=z\big( \tau(t) \big)=A e^{a\tau(t)}+B e^{b\tau(t)}$, et donc
\begin{equation}
	y(t)=At^a+Bt^b.
\end{equation}
\item Si $a=b$, alors $z(\tau)=A e^{a\tau}+B e^{b\tau}$, et donc
\begin{equation}
	y(t)=\big( A+B\ln(t) \big)t^a.
\end{equation}
\end{enumerate}

\end{enumerate}

\end{corrige}
% This is part of the Exercices et corrigés de CdI-2.
% Copyright (C) 2008, 2009
%   Laurent Claessens
% See the file fdl-1.3.txt for copying conditions.


