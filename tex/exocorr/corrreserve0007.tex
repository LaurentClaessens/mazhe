% This is part of Mes notes de mathématique
% Copyright (c) 2012
%   Laurent Claessens
% See the file fdl-1.3.txt for copying conditions.


\begin{corrige}{reserve0007}

    Les dérivées partielles sont données par
    \begin{verbatim}
    ----------------------------------------------------------------------
    | Sage Version 4.8, Release Date: 2012-01-20                         |
    | Type notebook() for the GUI, and license() for information.        |
    ----------------------------------------------------------------------
    sage: f(x,y,z)=exp(2*x)*sin(y)
    sage: f.diff(x)
    (x, y, z) |--> 2*e^(2*x)*sin(y)
    sage: f.diff(y)
    (x, y, z) |--> e^(2*x)*cos(y)
    sage: f.diff(z)
    (x, y, z) |--> 0
    \end{verbatim}
    Si nous voulons les calculer au point \( \big( \ln\sqrt{3},\pi/2,4 \big)\) nous faisons :
    \begin{verbatim}
    sage: f.diff(x)(x=ln(sqrt(3)),y=pi/2,z=0) 
    2*e^(2*log(sqrt(3)))
    sage: f.diff(x)(x=ln(sqrt(3)),y=pi/2,z=0).simplify_full()
    6
    sage: f.diff(y)(x=ln(sqrt(3)),y=pi/2,z=0).simplify_full()
    0
    sage: f.diff(z)(x=ln(sqrt(3)),y=pi/2,z=0).simplify_full()
    0
    \end{verbatim}
    Notez l'appel à la méthode \info{simplify\_full()} pour simplifier les expressions. Les réponses sont donc
    \begin{subequations}
        \begin{align}
            \frac{ \partial f }{ \partial x }\big( \ln\sqrt{3},\pi/2,4 \big)=6\\
            \frac{ \partial f }{ \partial y }\big( \ln\sqrt{3},\pi/2,4 \big)=0\\
            \frac{ \partial f }{ \partial z }\big( \ln\sqrt{3},\pi/2,4 \big)=0.
        \end{align}
    \end{subequations}
    
    Le vecteur \( \nabla f\) n'est rien d'autre que le vecteur formé par les dérivées partielles, c'est-à-dire
    \begin{equation}
        \nabla f=\begin{pmatrix}
            2 e^{2x}\sin(y)    \\ 
            e^{2x}\cos(y)    \\ 
            0    
        \end{pmatrix}.
    \end{equation}
    
    Étant donné que \( F\) dérive du potentiel \( f\), l'intégrale se réduit à la différence des valeurs de \( f\) aux bornes :
    \begin{equation}
        \int_{\gamma}F=f(\ln\sqrt{3},\pi/2,4)-f(0,0,0)=3-0=3.
    \end{equation}

\end{corrige}
