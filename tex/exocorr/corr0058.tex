% This is part of Exercices et corrigés de CdI-1
% Copyright (c) 2011
%   Laurent Claessens
% See the file fdl-1.3.txt for copying conditions.

\begin{corrige}{0058}

Les dérivées partielles sont
\begin{equation}
	\begin{aligned}[]
		(\partial_xf)(a,b)&=8a,&&\text{et}&(\partial_yf)(a,b)=2b.
	\end{aligned}
\end{equation}
Nous voyons tout de suite que la fonction monte donc plus vite en suivant la direction $x$ (coefficient $8$) qu'en suivant la direction $y$ (coefficient $2$).

Le plan tangent à $f$ en $\big( a,b,f(a,b) \big)$ est donné par 
\begin{equation}
	\begin{aligned}[]
		T_{(a,b)}(x,y)	&=f(a,b)+\frac{ \partial f }{ \partial x }(a,b)\cdot\big( (x,y)-(a,b) \big)_x+\frac{ \partial f }{ \partial y }(a,b)\cdot\big( (x,y)-(a,b) \big)_y\\
				&=4a^2+b^2+8a(x-a)+2b(y-b)\\
				&=8ax+2by-4a^2-b^2.
	\end{aligned}
\end{equation}
Petite vérification : $T_{(a,b)}(a,b)=f(a,b)$.

La différentielle $df_{(a,b)}$ existe parce que $f$ est polynomiale. En utilisant les dérivées partielles,
\begin{equation}
	df_{(a,b)}(u_1,u_2)=8au_1+2bu_2.
\end{equation}
En termes de matrices,
\begin{equation}
	df_{(a,b)}=\begin{pmatrix}
	8a	\\ 
	2b	
\end{pmatrix}.
\end{equation}
La différentielle est l'application linéaire dont la matrice est formée des « coefficients angulaires » du plan tangent.

On demande de calculer $\frac{ \partial f }{ \partial u }(a,b)$ avec $u=(\frac{ 1 }{2},\frac{ \sqrt{3} }{2})$ et $(a,b)=(1,2)$. Étant donné que $f$ est différentiable, nous avons
\begin{equation}
	\begin{aligned}[]
	\frac{ \partial f }{ \partial u }(a,b)=df_{(a,b)}(u)&=\frac{ \partial f }{ \partial x }(a,b)u_1+\frac{ \partial f }{ \partial y }(a,b)u_2\\
			&=8au_1+2bu_2\\
			&=4+2\sqrt{3}.
	\end{aligned}
\end{equation}
Le gradient est encore une variation sur le même thème :
\begin{equation}
	\nabla f(a,b)=\begin{pmatrix} 
	\frac{ \partial f }{ \partial x }(a,b)	&	\frac{ \partial f }{ \partial y }(a,b)	
\end{pmatrix}=\begin{pmatrix} 
	8a	&	2b	
\end{pmatrix},
\end{equation}
c'est le vecteur qu'il faut suivre en partant de $(a,b)$ pour voir la fonction $f$ monter le plus vite possible. Nous avons, juste en remplaçant, que
\begin{equation}
	\nabla f(1,1)=(8,2).
\end{equation}

Afin de comprendre le lien entre la tangente à une courbe de niveau et le plan tangent, il faut lire la page 94 du cours, ou alors les notes idéologiques de la sous-section \ref{ssecConceptPlanTag}. 

Le vecteur tangent à la courbe de niveau $4x^2+y^2=5$ en $(1,1)$ est un vecteur qui est perpendiculaire à $\nabla f(1,1)=(8,2)$. Le vecteur $1,-4$ par exemple fait l'affaire. Nous cherchons donc la droite qui passe par $(1,1)$ et dont le vecteur directeur est $(1,-4)$. L'équation de cette droite est
\begin{equation}
	y=-4x+5.
\end{equation}


\end{corrige}
