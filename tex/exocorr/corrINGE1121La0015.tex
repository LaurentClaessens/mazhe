% This is part of the Exercices et corrigés de mathématique générale.
% Copyright (C) 2009-2010
%   Laurent Claessens
% See the file fdl-1.3.txt for copying conditions.


\begin{corrige}{INGE1121La0015}

	Une façon économique de procéder est de calculer les mineurs principaux de la matrice associée. Nous avons
	\begin{equation}
		m_1=\det
		\begin{pmatrix}
			9	&	-3	&	6\sqrt{2}	\\
			-3	&	17	&	2\sqrt{2}	\\
			6\sqrt{2}	&	2\sqrt{2}	&	10
		\end{pmatrix}=\det
		\begin{pmatrix}
			9	&	-3	&	6\sqrt{2}	\\
			0	&	48	&	12\sqrt{2}	\\
			0	&	2	&	\sqrt{2}/2
		\end{pmatrix}=0,
	\end{equation}
	puis
	\begin{equation}
		m_2=\det\begin{pmatrix}
			9	&	-3	\\ 
			-3	&	17	
		\end{pmatrix}=9\cdot 17-9=144,
	\end{equation}
	et enfin
	\begin{equation}
		m_3=\det(9)=9.
	\end{equation}
	Deux strictement positifs et un nul, cela fait \emph{semi définie positive}.

\end{corrige}
