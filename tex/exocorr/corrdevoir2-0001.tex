\begin{corrige}{devoir2-0001}
  
  \begin{enumerate}

  \item La proposition 3.1 nous dit que toute application linéaire de $\eR^2$ dans $\eR^3$ est associée à une matrice avec 3 lignes et 2 colonnes. En fait, une fois qu'on a fixé des bases sur les espaces  $\eR^2$ et $\eR^3$, on trouve une unique matrice $A$ telle que $T(x)=Ax$, pour tout $x$ dans $\eR^2$.

 Notre exemple sera donc 
 \begin{equation}
   T(x,y)= \begin{pmatrix}
     1 &  2\\
     3 & 4 \\
     5 & 6
   \end{pmatrix}
   \begin{pmatrix}
     x\\
     y
   \end{pmatrix}.
 \end{equation}
Si on n'aime pas la notation matricielle, la même application linéaire s'écrit $T(x,y)= (x+2y, 3x+4y, 5x+6y)$.

 
 Une application affine est la somme d'une application linéaire et d'une application constante. Notre exemple sera alors 
 \begin{equation}
   T(x,y,z)= 
   \begin{pmatrix}
     1&2&3
   \end{pmatrix}
   \begin{pmatrix}
     x\\
     y\\
     z
   \end{pmatrix}+ 4.
 \end{equation}
  \item Une application de $\eR$ dans $\eR$ qui n'est ni linéaire ni affine est la fonction $\cos (x)$. On voit tout de suite que $\cos(0)=1$, la fonction n'est donc pas linéaire, en plus la fonction $g(x)=\cos(x)-1$ n'est pas linéaire non plus, parce que $2g(\pi/2)= 0$ alors que $g(\pi)= \cos(\pi)-1=-2$. On pouvait aussi dire que $\cos(x)$ n'est ni linéaire ni affine parce que sa dérivée n'est pas une fonction constante. Bien sûr il y a énormément d'autres exemples possibles, notamment : la fonction sinus, l'exponentielle, le logarithme, toute les fonctions polynomiales de degré supérieur à 1 $\ldots$.
  \end{enumerate}
\end{corrige}
