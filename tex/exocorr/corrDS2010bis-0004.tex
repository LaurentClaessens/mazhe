% This is part of Exercices de mathématique pour SVT
% Copyright (C) 2010
%   Laurent Claessens et Carlotta Donadello
% See the file fdl-1.3.txt for copying conditions.

\begin{corrige}{DS2010bis-0004}


	\begin{enumerate}
		\item
			Il faut prouver par récurrence. D'abord la formule proposée est vraie pour $n=0$ parce que $0(0+1)/2=0$. Ensuite, il faut vérifier que la formule est correcte pour $k+1$ en supposant qu'elle soit correcte pour tous indices $n$ entre $0$ et $k$. En posant $n=k+1$ dans la formule à démontrer, nous avons
			\begin{equation}
				u_{k+1}=u_{k}+k+1=\frac{k(k+1)}{2}+k+1=\frac{(k+1)(k+2)}{2},
			\end{equation}
			ce qui est bien la formule avec $n=k+1$. 
		\item
			La limite est donc $\lim_{n\to \infty} u_n=\infty$ parce que $u_n$ est la somme de tous les nombres entiers positifs entre $0$ et $n$, donc c'est une suite monotone croissante qui n'est pas bornée.
	\end{enumerate}

\end{corrige}
