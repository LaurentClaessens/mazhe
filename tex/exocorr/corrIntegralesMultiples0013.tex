\begin{corrige}{IntegralesMultiples0013}

On peut décrire ce solide en coordonnées sphériques. Considérons d'abord la section de $T$ contenue dans le plan $y$-$z$, pour $y>0$. Il s'agit du cercle $D$ de rayon $a$ centré au point $(0,a,0)$. Les points de $D$ correspondent aux points $(\rho, \theta, \phi)$ tels que 
\begin{equation}
  \begin{array}{l}
    \theta= \frac{\pi}{2},\\
    \phi\in\left[0,\pi\right]\\
    \rho\in\left[0,2a\sin(\phi)\right].
  \end{array}
\end{equation}
On obtient tous les autres points dans $T$ en faisant tourner $D$ autour de l'axe vertical : cela correspond à faire varier $\theta$ dans l'intervalle $[0,2\pi]$. 

L'intégrale à calculer est alors
\begin{equation}
  \int_{0}^{2\pi}\int_{0}^{\pi}\int_{0}^{2a\sin(\phi)} \rho^2\sin(\phi)\, d\rho\, d\phi \, d\theta = \frac{16a^3\pi}{3} \int_{0}^{\pi} \sin^4(\phi)\, d\phi=2a^3\pi^2. 
\end{equation}



\end{corrige}
