% This is part of Exercices et corrigés de CdI-1
% Copyright (c) 2011-2012,2014, 2019
%   Laurent Claessens
% See the file fdl-1.3.txt for copying conditions.

%\usepackage{graphicx}
%\newcommand{\e}{\'{e}}
%\newcommand{\ee}{\`{e}}
%\newcommand{\ac}{\`{a} }
%\newcommand{\f}{\frac}
%\newcommand{\arcth}{{\rm arctanh}}
%\newcommand{\arcsh}{{\rm arcsinh}}
%\newcommand{\arcch}{{\rm arccosh}}
%\newcommand{\csec}{{\rm cosec}}
%\newcommand{\cotan}{{\rm cotg}}
%\newcommand{\cis}{(\cos+i\sin)( }
%\newcommand{\Rn}{\rm {I\!\!\, R}} 

%+++++++++++++++++++++++++++++++++++++++++++++++++++++++++++++++++++++++++++++++++++++++++++++++++++++++++++++++++++++++++++
					\section{Nombres complexes}
%+++++++++++++++++++++++++++++++++++++++++++++++++++++++++++++++++++++++++++++++++++++++++++++++++++++++++++++++++++++++++++

\[ \begin{tabular}{| c || c || | c | | c | | c | | c | c | }
\hline  no &         ${ 1}                                                         $    & no     &        ${ 2}  $                           	         & no         &      ${ 3}  $            \\    \hline \hline
           a   &      $4+5i                      $  				            & a  &   $4\cis{\frac{5\pi}{6}})$     		                  &a&  $\pm(\frac{\sqrt 2}{2}+ \frac{\sqrt 2}{2}i)$  \\ \hline 
           b  &      $11i                                        $   			   & b  &  $\frac{-1+i\sqrt 3}{4}=\frac{ 1 }{2} e^{\frac{ 2i\pi }{ 3 }}      $          		          &b & $3\cis \frac{\pi}{6}+\frac{2k\pi}{6})$ \\ \hline
           c   &      $\frac{31}{21}-\frac{7}{20}i                      $   		   & c  & $\cis\frac{\pi}{6})	 $   				   &c&  $\cis \frac{\pi}{6}+\frac{2k\pi}{6})$  \\ \hline 
           d   &      $-1+7i                                  $   			   & d &   $\sqrt 2\cis\frac{-3\pi}{4})$ 			        &d&  $ 2\cis -\frac{\pi}{6}+\frac{2k\pi}{6}) $\\ \hline 
           e   &      $58          $   						   & e &  $4   	\cis\frac{\pi}{6})	 $			&e& $\cis \frac{2k\pi}{3})$ \\ \hline
           f   &      $\frac{23}{30}-\frac{2}{15}i   $    			            & f  & $16      $       		 
& &  \\ \hline
           g   &      $\frac{12+5i}{13}   			 $   			   & g  & $8i$                  					 & &  \\ \hline
           h   &    $\frac{5-7i}{74}  			 $ 			    	   & h  & $\frac{-1-\sqrt3 i}{2}= e^{\frac{ 2i\pi }{ 3 }}$                                         & & \\ \hline
           i    &      $      	\frac{6+2\sqrt 7 i}{64}         $  			   & i  & $ \frac{1}{8}$                &          &                           \\  \hline
           j    &      $      	1				         $    		  & j  & $ \sqrt 2 \cis \frac{-\pi}{4} )$                &          &                           \\  \hline
          k   &      $      	1-12i				         $    		  &  &                 &          &                           \\  \hline           
\end{tabular} \]


\noindent{ Exercice $4$}\\
$(a)$ { Montrer que si $(x+iy)$ est une racine carrée de $(a+ib)$ o\`{u} $x, y, a, b \in \eR^n$, alors $x$ et $y$ sont solutions des équations }
\[ \begin{array}{c} x^2-y^2=a\\ 
             			2xy=b \end{array}\]
\\


\noindent Pour le voir il suffit d'écrire $(x+iy)^2 = a+ib$ et d'égaliser partie réelle à partie réelle, partie imaginaire à partie imaginaire.\\
$(b)$\hspace{0,3cm} $\pm(\frac{\sqrt 2 }{2}(3+i))$

\[ \begin{tabular}{| c || c || | c | | c | |}
\hline  no &         ${ 5}                                                         $ 		   & no     &                            	           \\    \hline \hline
           a   &  $    \frac{1\pm i}{2}                    $  				            & f 	 	&   		                    \\ \hline 
           b  &      $ \frac{3i \pm \sqrt23}{4}                                       $   	   & g  &            		          \\ \hline
           c   &      $\frac{2i}{1-i}$,$-2\frac{1+i}{1-i}                      $   		   & h  & $-2+2^{1/5} \cis\frac{\pi}{15}+\frac{2k\pi}{5})	 $   				    \\ \hline 
           d   &      $                             $   			  	   & i &   		    \\ \hline 
         e   &      $\pm\sqrt{1+\sqrt2}\cis \frac{\pi}{4}) $ ,    						   &   &     	 	\\
              &      $\pm\sqrt{\sqrt2-1}\cis \frac{-\pi}{4})   $   						   &   &                \\ \hline
          \end{tabular} \]
          
\noindent{ Exercice $6$. 
Par application de la formule donnant $z^n$, exprimez $\cos2x$, $\cos3x$ et $\cos4x$ en fonction de puissances de $\cos x$.}
          \\
          
          
\noindent Poser $z=\cos x + i\sin x$. On sait alors que \[z^n=\underbrace{(\cos x+i\sin x)^n}_*=\cos nx + i \sin nx.\] Il faut donc développer $*$ par la formule du binôme et égaler les deux membres. Par exemple:\[\begin{array}{cccrcc}
               						z^2= & (\cos x+i\sin x)^2 & = & \cos^2 x-\sin^2 x+2i\cos x\sin x &   & \\
							          & 	\Longrightarrow    &    &          \cos^2 x-\sin^2 x       	&= &	\cos2x \\
							           & 	\Longrightarrow    &      &          \cos2x    	&= &	 2\cos^2x-1\end{array}\]

\noindent On trouvera alors: \[ \begin{array} {rl}
					\cos3x&=4\cos^3x--3\cos x\\
					\cos4x&=8\cos^4x-8\cos^2x+1\end{array}\]


\noindent{ Exercice $7$}\\
$a=2$, $b=-\frac{\sqrt2}{4}$.


\section{Graphes de quelques fonctions qu'il est bon de connaitre}

% TODO: refaire les dessins manquants.

Les exponentielles sont à la figure \ref{LabelFigDessinExp}.
\newcommand{\CaptionFigDessinExp}{Des exponentielles}
\input{auto/pictures_tex/Fig_DessinExp.pstricks}

%Les exponentielles sont à la figure \ref{LabelFigDessinExp}, 
%\newcommand{\CaptionFigDessinExp}{Des exponentielles}
%\input{auto/pictures_tex/Fig_DessinExp.pstricks}

%Les logarithmes sont à la figure \ref{LabelFigDessinLog}
%\newcommand{\CaptionFigDessinLog}{Des logarithmes. Aucune des deux ne monte très vite, et plus la base augmente, moins ça monte vite.}
%\input{auto/pictures_tex/Fig_DessinLog.pstricks}

%À la figure \ref{LabelFigDessinAbs} se trouvent mes valeurs absolues
%\newcommand{\CaptionFigDessinAbs}{Des graphiques de valeurs absolues}
%\input{auto/pictures_tex/Fig_DessinAbs.pstricks}

%À la figure \ref{LabelFigDessinHyperbolique} se trouvent mes fonctions hyperboliques
%\newcommand{\CaptionFigDessinHyperbolique}{Des graphiques de fonctions hyperboliques}
%\input{auto/pictures_tex/Fig_DessinHyperbolique.pstricks}

%Le graphe de $\sin(x)/x$ est sur la figure \ref{LabelFigSinxx}
%\newcommand{\CaptionFigSinxx}{La fonction $x\mapsto\frac{ \sin(x) }{ x }$}
%\input{auto/pictures_tex/Fig_Sinxx.pstricks}

%Le graphe de $\cos(x)/x$ est à la figure \ref{LabelFigCosxx}.
%\newcommand{\CaptionFigCosxx}{La fonction $x\mapsto\frac{ \cos(x) }{ x }$}
%\input{auto/pictures_tex/Fig_Cosxx.pstricks}

%D'autres valeurs absolues à la figure \ref{LabelFigAbsx}
%\newcommand{\CaptionFigAbsx}{La fonction valeur absolue et quelques autres.}
%\input{auto/pictures_tex/Fig_Absx.pstricks}

\section{Intégration}

Petite note stratégique pour l'intégration par partie. Une partie de l'art est de choisir correctement que $u$ et quel $dv$ on prend. Il y a une petite règle qui permet de choisir assez souvent : l'ordre de priorité du choix pour $u$ est
\let\OldTheEnumi\theenumi
\renewcommand{\theenumi}{\arabic{enumi}}
\begin{enumerate}
\item $\ln x$
\item $x^n$
\item $ e^{kx}$,
\end{enumerate}
\let\theenumi\OldTheEnumi
c'est-à-dire que si il y a un logarithme, il faut le choisir comme $u$; si il n'y a pas de logarithme, mais une puissance de $x$, alors il faut choisir la puissance de $x$; et si il n'y a ni logarithme ni puissance, alors on choisit l'exponentielle si il y en a.

%--------------------------------------------------------------------------------------------------------------------------- 
\subsection{Un exemple d'intégrale pas simple}
%---------------------------------------------------------------------------------------------------------------------------



Dans la méthode de l'intégration de fraction rationnelles, l'apothéose est de devoir intégrer 
\begin{equation}
	K_s=\int \frac{ dt }{ (1+t^2)^s }.
\end{equation}
Afin de prouver la formule de récurrence, nous commençons par écrire le numérateur sous la forme $(1+t^2-t^2)dt$ :
\begin{equation}
	K_s=\int \frac{ 1 }{ (1+t^2)^{s-1} }-\int \frac{ t^2 }{ (1+t^2)^s }.
\end{equation}
Le premier terme vaut $K_{s-1}$, tandis que nous intégrons le second par partie en posant
\begin{equation}
	\begin{aligned}[]
		u&=t		&	dv&=\frac{ t }{ (1+t^2)^s }\\
		du&=dt		&	v=&\frac{1}{ 2(s-1) }\cdot \frac{1}{ (1+t^2)^{s-1} }.
	\end{aligned}
\end{equation}
Nous tombons sur
\begin{equation}
	K_{s}=K_{s-1}-\left( \frac{ t }{ 2(1-s)(1+t^2)^{s-1} } - \frac{1}{ 2(1-s) }\int \frac{ dt }{ (1+t)^{s-1} } \right)
\end{equation}
Le dernier terme donne encore un multiple de $K_{s-1}$.

\[ \begin{tabular}{| c || c || | c | | c | | c | | c | c | }
\hline  no &         ${ 11}                                                         $    & no     &        ${ 12}  $                           	         &          &       \\    \hline \hline
           $1$   &      $\frac{x^3}{3}+3x+\ln(x)                        $  & $1$  &   $\frac{\sin^3(x^2+1)}{6}$     		&$10$&  $\frac{2}{3}(2+t)^{3/2}$  \\ \hline 
           $2$   &      $ \frac{x^3}{3}                                              $   & $2$  &  $(t^2+6)^{3/2}       $          		&$11$&  $\frac{2}{3}(1+e^x)^{3/2}$ \\ \hline
           $3$   &      $3\frac{x^5}{5} + 2x^3+3x                         $   & $3$  & $\sin(4+y^3)	 $      &$12$&  $\frac{1}{9}(1+e^{3x})^3$  \\ \hline
           $4$   &      $\frac{4x^{7/2}}{7}                                       $   &  $4$ & $\ln| x+2|$    					&$13$&  $ 2\ln  | 2+\sqrt x | $\\ \hline 
           $5$   &      $\frac{y^5}{5}+2y-\frac{y^{-3}}{3}                   $   &  $5$ &  $-\ln| \cos x|$       		&$14$		&$-\frac{1}{ 2 }\ln\big( 1-\ln(x)^2 \big)$ \\ \hline
           $6$   &      $\frac{x^2}{2}+\frac{x^3}{3}-\frac{4x^{5/2}}{5}	 $    & $6$  & $\frac{-1}{14}\ln| 2 - 7 x^2|       $        &$15$& $\ln  |x+ \sin x | $ \\ \hline
           $7$   &      $\frac{3x^2-6x)^4}{24}   			 $    & $7$  & $ \ln |1 + \ln x|  $                  &$16$& $\frac{\sin^2x}{2}$  \\ \hline
                     &                        					      & $8$  & $-\frac{1}{\sin( x )}$                         &$17$ & $2e^{\sqrt x} + \frac{2}{3}x^{3/2} $\\ \hline
                     &      $      					         $     & $9$  & $ \frac{1}{16}\ln  | b+4x^4 | $                &          &                           \\  \hline
                 
             
\end{tabular} \]


\[ \begin{tabular}{| c || c || | c | | c |c | | c | }
\hline  no &         ${ 13}                                          $    & no    &  ${ 14}  $                                                              & no & ${ 17}  $       \\    \hline \hline
           $1$   &      $      \frac{5^{2x}}{2\ln 5}                  $        & $1$  &  $ -\frac{1}{3} \arctan\frac{x-1}{3}$                               & $1$ & $2\sin \sqrt x$			  \\ \hline 
           $2$   &      $        \frac{e^{\sin 2t}}{2}                  $       & $2$  &  $\frac{1}{2\sqrt 5 }\ln | \frac{x-\sqrt 5 }{x+\sqrt 5} |  $ & $2$ & $\tan (x) - x$         		\\ \hline
           $3$   &      $2e^{\sqrt x }                              $        & $3$  &  $\frac{1}{5}\ln | \frac{x }{x- 5} |$                                     & $3$ &  $\frac{2\sin(\frac{3x}{2})}{3(\cos(\frac{3x}{2})-\sin(\frac{3x}{2}))}$	\\ \hline  
           $4$   &      $ e^{\tan y}                                   $      &  $4$ &  $\frac{1}{2}\arctan \frac{x}{2}$    	                                & $4$ &	 $\frac{1}{\pi}\sin(\pi x)$					\\ \hline 
           $5$   &      $\frac{e^{4x}-e^{-4x}}{4}-2x               $        &  $5$ &  $\frac{1}{6}\ln|\frac{-1+3x}{1+3x} |$       		    &         &		 \\ \hline
                   &              							   & $6$  &  $\frac{4}{\sqrt 26}\arctan\frac{5x+3}{\sqrt 26} $ &         &       \\ \hline
                   &       			                                      & $7$  &  $\frac{4}{13}\ln|\frac{-2+x}{3+5x} | $                           &          &      \\ \hline
                        
\end{tabular} \]

\[ \begin{tabular}{| c || c || | c | | c | | }
\hline       no &         ${ 15}                                          $                                   &  no    &  ${ 16}  $                                                                                              \\    \hline \hline
             $1$   &      $      \frac{1}{3}\arcsin \left(\frac{x-1}{3}\right)                  $          & $1$  &  $ -\frac{x^4}{4} \arctan x - \frac{x^3}{12}+\frac{x}{4}-\frac{1}{4}\arctan x$    \\ \hline 
            $2$   &      $      \arcch(x+1)              $                                                     & $2$  &  $\frac{x^4\ln| x|}{4}-\frac{x^4}{16} $          						\\ \hline
                      &   and           $   2\arcsh(\sqrt{\frac{x}{2}}) $         			        & $3$  &  $\frac{e^x}{1+x}$       									\\ \hline 
           $4$   &        $  7^{-1/2}\arcsh\left( \frac{-4+7x}{\sqrt 47} \right)               $                               &  $4$ &  $-\frac{1}{4}(-1+2y^2)\cos 2y + \frac{y}{2}$    					\\ \hline 
         $4$   &      $ \arcch(2x+3)     $       							&  $5$ &  $\frac{1}{27}e^{3x}(2-6x+9x^2)$       				 \\ \hline
           $5$&       $\frac{2\sqrt{-2+x} \ln |\sqrt{-2 + x} +\sqrt{x}| }{\sqrt{2-x}}   $               & $6$  &  $\frac{\cos^2 x}{2}-\frac{1}{8}\cos 4x $        \\ \hline
            $6$   &      $-2\sqrt{3}\arcsin\left(\frac{-2-3x}{\sqrt{19}}\right)$                & $7$  &  $\frac{1}{4}(x^2-2\cos x -2x\sin x) $                \\ \hline
                $7$      &       			     $\frac{6}{\sqrt{7}}\arcsh\left(\frac{-4+7x}{\sqrt 45}\right)$                                                                      & $8$  &  Pas exprimable comme fonction élémentaire             \\ \hline
                   &       			                                                                         & $9$  &   $-x\cos x +\sin x$           \\ \hline
                    &       			                                                                         & $10$ &  $(x-1)e^x$            \\ \hline
                    &       			                                                                         & $11$ &  $2x\cos x + (x^2-2)\sin x$            \\ \hline
  \end{tabular} \]

 
\[ \begin{tabular}{| c || c ||  }
\hline       no &         ${ 18}                                          $                                                                                                               \\    \hline \hline
             $1$   & $\frac{1}{12}(4x^3+6\arctan x +3\ln(\frac{x-1}{x+1})      $            \\ \hline 
            $2$   &  $ x+\frac{4}{\sqrt 3}\arctan(\frac{1+2x}{\sqrt 3})+ \frac{2}{3}\ln(x-1)- \frac{1}{3}\ln(1+x+x^2) +\frac{4}{3}\ln(x^3-1)$                 \\ \hline

           $3$   &      $  - \frac{1}{\sqrt 3}\arctan(\frac{1+2x}{\sqrt 3})     +\ln(x)- \frac{1}{2}\ln(1+x+x^2)       $ 						\\ \hline 
           $4$   &      $   \frac{1}{4\sqrt 2}(-2\arctan(1-\sqrt2x)+2\arctan(1+\sqrt2x)+\ln(\frac{1+\sqrt2x+x^2}{-1+\sqrt2x-x^2})             $                                                 				\\ \hline 
              $5$   &      $   -3\arctan x+\frac{1}{2}\ln(1+x^2)$   				 \\ \hline
               $6$  &      $-\frac{2}{x}-\sqrt2\arctan(\frac{x}{\sqrt2})+\frac{1}{2}\ln(2+x^2)$                       \\ \hline
              $7$ &      $\frac{1}{3\sqrt5}\arctan(\frac{-2+3x}{\sqrt5}) -\frac{1}{6}\ln(3-4x+3x^2) $                        \\ \hline
               $8$ &     $-2x+\frac{x^2}{2}+4\ln(3+2x)$  			                                                                                \\ \hline
              $9$     &     $\frac{1}{2}\ln x -\frac{1}{2}\ln(2+x)$  			                                                                       \\ \hline
               $10$  &  A voir...     			                                                               \\ \hline
              $11$  &  $\frac{3}{2}\arctan(2x)+\frac{1}{4}\ln(1+4x^2)$			                                                               \\ \hline   
               $12$ &  $2\sqrt2\arctan(\frac{-4+x}{\sqrt2})+\frac{1}{2}\ln(18-8x+x^2)$			                                                               \\ \hline   
              $13$  &  $-\frac{1}{22}(11+3\sqrt11)\ln(3+\sqrt{11}+x)+-\frac{1}{22}(11-3\sqrt11)\ln(-3+\sqrt{11}-x)$			                                                               \\ \hline   
              $14$  &  $\frac{9}{5\sqrt{14}}\arctan(\frac{-1+5x}{\sqrt{14}})-\frac{1}{10}\ln(3-2x+5x^2)$			                                                               \\ \hline   
              $15$  &  $\frac{x^2}{2}+\arctan x-\frac{1}{2}\ln(1+x^2)$			                                                               \\ \hline   
              $16$  &  $4x-\frac{15}{2}\arctan(\frac{x}{2})$			                                                               \\ \hline   
              $17$  & A voir...		                                                               \\ \hline   
              $18$  &  $-\ln x +3\ln(-1+3x)-\ln(1+3x)$			                                                               \\ \hline   
              $19$  &  $\ln(-1+ x) +3\ln(x)-2\ln(2+x)$			                                                               \\ \hline   
              $20$  & A voir...	                                                               \\ \hline   
              $21$  &  $\frac{1}{4\sqrt2}(-2\arctan(1-\sqrt2 x)+2\arctan(1+\sqrt2x)+\ln(\frac{-1+\sqrt2x-x^2}{1+\sqrt2x+x^2})$			                                                               \\ \hline   
\end{tabular} \]


\[ \begin{tabular}{| c || c ||  }
\hline       no &         ${ 18}                              $       mal classés                                                                                                        \\    \hline \hline
             $1$   & $-\frac{\cos x}{2}-\frac{1}{10}\cos(5x)      $            \\ \hline 
            $2$   &  $ \frac{x}{2}-\frac{1}{4}\sin(2x)$                 \\ \hline

           $3$   &      $  \arctan(\ln(x)) $ 						\\ \hline 
           $4$   &      $   \frac{1}{2}\ln[\frac{-1+\ln x}{1+\ln x}]            $                                                 				\\ \hline 
           $5$	  &      $x$                                                       \\ \hline   
\end{tabular} \]





\[ \begin{tabular}{| c || c ||  }
\hline       no &         ${ 19}                                          $                                                                                                               \\    \hline \hline
             $1$   & $-x+\frac{1}{3}\tan(3x)$            \\ \hline 
            $2$   &  $ \ln(\cos x)+\frac{1}{2}\sec^2(x)$                 \\ \hline
           $3$   &   A voir...						\\ \hline 
           $4$   &      $   \frac{1}{ab}\arctan(\frac{b\tan y}{a})             $     \\ \hline 
              $5$   &      $  \frac{x}{5}+\frac{4}{15}\ln(3\cos x-\sin x) + \frac{1}{3}\ln(\sin x)    $   				 \\ \hline
               $6$  &      $-\frac{2}{\sqrt{-a^2+b^2}}\arcth(\frac{(a-b)\tan(\frac{x}{2})}{\sqrt{-a^2+b^2})}$                \\ \hline
              $7$ &     A voir...                       \\ \hline
               $8$ &     $\ln(-7 + \cos(2 x))$  			                                                                                \\ \hline
              $9$     &     $\frac{\tan[x]}{2}$  			                                                                       \\ \hline
               $10$  &  $x - \frac{1}{3}\tan(\frac{3x}{2})$  			                                                               \\ \hline
              $11$  &  $\frac{3\sin(x)}{4}+\frac{1}{12}\sin(3x)$			                                                               \\ \hline   
               $12$ &  $\frac{1}{2440}[1225\sin x +245\sin 3x+49\sin 5x + 5\sin 7x]$			                                                               \\ \hline   
              $13$  &  $\frac{1}{192}[-60+60x -\sin(6-6x) -9\sin(4-4x)+45\sin(2-2x)]$			                                                               \\ \hline   
              $14$  &  $\frac{1}{384}[24x -3\sin(4x) -3\sin(8x)+\sin(12x)]$			                                                               \\ \hline   
              $15$  &  $\frac{1}{\sqrt{-b^2+ac}}\arctan[\frac{b+c\tan x}{\sqrt{-b^2+ac}}]$ + discussion			                                                               \\ \hline   
              $16$  &  $5\csec^2(\frac{x}{5})-\frac{5}{4}\csec^4( \frac{x}{5})+5\ln(\sin \frac{x}{5})$			                                                               \\ \hline   
              $17$  & $\frac{1}{35}[6+\cos(2x)]\sec^2(x)\tan^5(x)$		                                                               \\ \hline   
              $18$  &  A voir...                                                               \\ \hline   
              $19$  &  $25[\cos(x)+\frac{1}{3}\cos(3x)-\ln(\cos(\frac{x}{2}))+\ln(\sin(\frac{x}{2}))]$			                                                               \\ \hline   
              $20$  & $-\ln[\cos(\frac{x}{2} )]+\ln[\cos(\frac{x}{2})+\sin(\frac{x}{2})]$                                                          \\ \hline   
              $21$  &  $-\ln[\cos(\frac{x}{2} )]+\frac{1}{2} \ln[\sin(\frac{x}{2})]-\frac{1}{4} \sec^2(\frac{x}{2})]$			                                                               \\ \hline   
              $22$  &  $-\frac{2}{3}\arctan[3\cotg(\frac{x}{2})]$			                                                               \\ \hline   
              $23$  &  $\frac{x}{5}-\frac{8}{15}\arcth[\frac{1}{3}\tan(\frac{x}{2})]$			                                                               \\ \hline   
              $24$  &  $-\arctan[\cos(x)]$			                                                               \\ \hline   
              $25$  &  $(2+\sec^2[\frac{x}{3}])\tan(\frac{x}{3})$			                                                               \\ \hline   

\end{tabular} \]


\[ \begin{tabular}{| c || c ||  }
\hline       no &         ${ 20}                              $                                                                                        \\    \hline \hline
             $1$   & $2\arctan(\sqrt x)   $            \\ \hline 
            $2$   &  $ \ln\frac{\sqrt z +1}{\sqrt z -1}$                 \\ \hline

           $3$   &      $\frac{3}{2}\ln(-3+\sqrt x)+\frac{1}{2}\ln(1+\sqrt x)$ 						\\ \hline 
           $4$   &      $  \frac{1}{2}   \ln\frac{\sqrt{1+ x} -2}{\sqrt{1+x} +2}       $                                                 				\\ \hline 
           $5$	  &      $-3\ln(-1+x^{1/3})+\ln x$                                                       \\ \hline   
           $6$	  &      $\frac{2}{3}\sqrt{3+2x}-\frac{2}{3}\sqrt{\frac{5}{3}}\arcth[\sqrt{\frac{3}{5}}\sqrt{3+2x}]$                                                       \\ \hline   

\end{tabular} \]


La suite viendra.
