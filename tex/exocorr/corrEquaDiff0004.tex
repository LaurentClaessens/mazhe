% This is part of the Exercices et corrigés de mathématique générale.
% Copyright (C) 2009
%   Laurent Claessens
% See the file fdl-1.3.txt for copying conditions.
\begin{corrige}{EquaDiff0004}

\begin{enumerate}

\item
Nous substituons $y(x)=\cos(x)$ et $y'(x)=-\sin(x)$ dans l'équation, et nous trouvons
\begin{equation}
	-\sin(x)+\cos(x)f(x)=\sin(x),
\end{equation}
d'où nous déduisons $f(x)=2\tan(x)$.

\item
Nous avons donc l'équation différentielle
\begin{equation}
	y'+2y\tan(x)=\sin(x)
\end{equation}
à résoudre. C'est une équation linéaire, donc nous commençons par l'homogène associée : $y'_H+2y_H\tan(x)=0$ qui se transforme en
\begin{equation}
	\frac{ y'_H }{ y_H }=-2\tan(x),
\end{equation}
et dont la solution est
\begin{equation}
	y_H(x)=K e^{-2\ln(1/\cos(x))}=K\cos^2(x).
\end{equation}
Nous trouvons la solution générale de l'équation non homogène par la méthode de \emph{variations des constantes}. Nous posons dons
\begin{equation}
	\begin{aligned}[]
		y(x)&=K(x)\cos^2(x),
		y'(x)&=K'\cos^2(x)-2K\cos(x)\sin(x)
	\end{aligned}
\end{equation}
et nous injectons cette fonction dans l'équation de départ. Après simplifications, il ne reste qu'une équation pour $K$:
\begin{equation}
	K'(x)=\frac{ \sin(x) }{ \cos^2(x) },
\end{equation}
dont la solution est $K(x)=\frac{1}{ \cos(x) }+C$, ce qui donne la solution générale de l'équation de départ:
\begin{equation}
	y(x)=\left( \frac{1}{ \cos(x) }+C \right)\cos^2(x)=\cos(x)\big( 1+C\cos(x) \big).
\end{equation}

\item
Un tout petit peu de calcul montre que 
\begin{equation}
	y'(\pi/4)=-\frac{1}{ \sqrt{2} }-C,
\end{equation}
qui s'annule lorsque $C=-\frac{1}{ \sqrt{2} }$. La solution demandée est donc
\begin{equation}
	y(x)=\cos(x)\big( 1+\frac{ \cos(x) }{ \sqrt{2} } \big).
\end{equation}

\end{enumerate}

\end{corrige}
