% This is part of the Exercices et corrigés de mathématique générale.
% Copyright (C) 2010
%   Laurent Claessens
% See the file fdl-1.3.txt for copying conditions.

\begin{corrige}{FoncDeuxVar0022}

	Pour les fonctions à valeurs réelles, la jacobienne n'est autre que le vecteur gradient, c'est-à-dire les différentes dérivées partielles mises les unes à côté des autres.
	\begin{enumerate}

		\item
			Ici nous avons
			\begin{equation}
				\begin{aligned}[]
					\frac{ \partial f }{ \partial x }&=y\sin(xy)+xy^2\cos(xy)\\
					\frac{ \partial f }{ \partial y }&=x^2y\cos(xy)+x\sin(xy),
				\end{aligned}
			\end{equation}
			et la matrice jacobienne est la matrice ligne 
			\begin{equation}
				\begin{pmatrix} 
					y\sin(xy)+xy^2\cos(xy)	&	x^2y\cos(xy)+x\sin(xy).	
				\end{pmatrix}
			\end{equation}
		\item
		\item

	\end{enumerate}
	

\end{corrige}
