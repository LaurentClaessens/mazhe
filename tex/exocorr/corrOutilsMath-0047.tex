% This is part of Exercices et corrigés de CdI-1
% Copyright (c) 2011
%   Laurent Claessens
% See the file fdl-1.3.txt for copying conditions.

\begin{corrige}{OutilsMath-0047}

    Le vecteur $(x,y,z)$ sera perpendiculaire à $(1,3,3)$ si (et seulement si) $x+2y+z=0$. On peut par exemple prendre
    \begin{equation}
        w=\begin{pmatrix}
            -2    \\ 
            1    \\ 
            0    
        \end{pmatrix}.
    \end{equation}
    En ce qui concerne un vecteur de norme $1$, il suffit de prendre $w/\| w \|$, c'est-à-dire
    \begin{equation}
        \frac{1}{ \sqrt{5} }\begin{pmatrix}
            -2    \\ 
            1    \\ 
            0    
        \end{pmatrix}.
    \end{equation}
    Il y a évidemment de nombreuses autres possibilités.

\end{corrige}
