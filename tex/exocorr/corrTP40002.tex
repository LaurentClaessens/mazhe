% This is part of the Exercices et corrigés de mathématique générale.
% Copyright (C) 2009
%   Laurent Claessens
% See the file fdl-1.3.txt for copying conditions.
\begin{corrige}{TP40002}

	L'équation homogène est
	\begin{equation}
		y'_H+2y_H\tan(x)=0,
	\end{equation}
	qui donne 
	\begin{equation}
		\frac{ y'_H }{ y_H }=-2\tan(x).
	\end{equation}
	En intégrant des deux côtés,
	\begin{equation}
		y_H(x)=K\cos^2(x).
	\end{equation}
	Afin de trouver la solution générale de l'équation non homogène, nous posons
	\begin{equation}
		y(x)=K(x)\cos^2(x),
	\end{equation}
	dont la dérivée est
	\begin{equation}
		y'=K'\cos^2(x)+-2K\sin(x)\cos(x).
	\end{equation}
	En remettant le tout dans l'équation de départ, nous trouvons l'équation suivante pour $K$ :
	\begin{equation}
		K'(x)\frac{1}{ \cos(x) }.
	\end{equation}
	Cette intégrale se règle en posant $u=\sin(x)$, donc $\cos(x)=\sqrt{1-u^2}$, donc
	\begin{equation}
		\int\frac{1}{ \sqrt{1-u^2} }\frac{ du }{ \cos(x) }=\int\frac{ du }{ \sqrt{1-u^2}\sqrt{1-u^2} }.
	\end{equation}
	Nous avons donc
	\begin{equation}
		K(x)=\frac{1}{ 2 }\ln\left( \frac{ \sin(x)+1 }{ \sin(x)-1 } \right)+C.
	\end{equation}
	Au final, la solution générale de l'équation différentielle est
	\begin{equation}
		y(x)=\frac{\cos^2(x)}{ 2 }\ln\left|  \frac{ \sin(x)+1 }{ \sin(x)-1 } \right|+C\cos^2(x).
	\end{equation}

	I s'agit maintenant de trouver la solution qui vaut zéro lorsque $x=\pi$. Pour cela, nous calculons $y(\pi)$. Ce sont les valeurs absolues du logarithme qui nous sauvent la vie, et on trouve qu'avec $C=0$, le tout vaut zéro.

\end{corrige}
