\begin{corrige}{GeomAnal-0043}

    Les normes \( 1\) et \( \infty\) sont relativement simples :
    \begin{equation}
        \| A \|_1=\max\{ 5,5,5 \}=5
    \end{equation}
    et
    \begin{equation}
        \| A \|_{\infty}=\max\{ 4,7,4 \}.
    \end{equation}
    
    Pour la norme \( \| A \|_2\), nous devons calculer le rayon spectral de \( A^tA\). La première chose à faire est de calculer les valeurs propres. Ce sont des calculs à faire à l'ordinateur.
    \begin{verbatim}
   ----------------------------------------------------------------------
| Sage Version 4.7.1, Release Date: 2011-08-11                       |
| Type notebook() for the GUI, and license() for information.        |
----------------------------------------------------------------------
sage: A=matrix(3,3,[2,1,1,2,3,2,1,1,2])
sage: M=A.transpose()*A
sage: M
[ 9  9  8]
[ 9 11  9]
[ 8  9  9]
sage: M.characteristic_polynomial()
x^3 - 29*x^2 + 53*x - 25
sage: solve(  x^3 - 29*x^2 + 53*x - 25,x)
[x == -3*sqrt(19) + 14, x == 3*sqrt(19) + 14, x == 1] 
    \end{verbatim}
Les valeurs propres de cette matrice \( M=A^tA\) sont 
\begin{equation}
    \{ \lambda_i \}=\{ 1,14+3\sqrt{19},14-3\sqrt{19} \}.
\end{equation}
La plus grande valeur propre est donc \( 14+3\sqrt{19}\), et la norme \( 2\) de la matrice est
\begin{equation}
    \| A \|_2=\sqrt{\rho(A^tA)}=\sqrt{14+3\sqrt{19}}\simeq 5.203
\end{equation}
\end{corrige}
