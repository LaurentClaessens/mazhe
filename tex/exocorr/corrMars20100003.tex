% This is part of the Exercices et corrigés de mathématique générale.
% Copyright (C) 2009-2010
%   Laurent Claessens
% See the file fdl-1.3.txt for copying conditions.


\begin{corrige}{Mars20100003}

	Si on n'a aucune idée de la solution, le mieux est d'essayer sur un petit exemple pour se donner une idée. Prenons par exemple cette petite matrice dont la somme des éléments de chaque ligne est $4$:
	\begin{equation}
		\begin{pmatrix}
			1	&	3	\\ 
			2	&	2	
		\end{pmatrix}.
	\end{equation}
	Cherchons les vecteurs propres pour la valeur propre $4$. Le système à résoudre est
	\begin{equation}
		\begin{pmatrix}
			1	&	3	\\ 
			2	&	2	
		\end{pmatrix}\begin{pmatrix}
			x	\\ 
			y	
		\end{pmatrix}=4\begin{pmatrix}
			x	\\ 
			y	
		\end{pmatrix},
	\end{equation}
	dont les solutions sont $x=y$, c'est à dire les multiples de $\begin{pmatrix}
		1	\\ 
		1	
	\end{pmatrix}$.
	Qu'est-ce qu'on parie que dans le cas de l'exercice la réponse sera que le vecteur propre est
	\begin{equation}
		\begin{pmatrix}
			1	\\ 
			1	\\ 
			1	\\ 
			1	\\ 
			1	
		\end{pmatrix}\, ?
	\end{equation}
	Regardons donc le produit
	\begin{equation}
		\begin{pmatrix}
			a_{11}	&	a_{12}	&	a_{13}	&	a_{14}	&	a_{15}\\	
			a_{21}	&	a_{22}	&	a_{23}	&	a_{24}	&	a_{25}\\	
			a_{31}	&	a_{22}	&	a_{33}	&	a_{34}	&	a_{35}\\	
			a_{41}	&	a_{42}	&	a_{43}	&	a_{44}	&	a_{45}\\	
			a_{51}	&	a_{52}	&	a_{53}	&	a_{54}	&	a_{55}
		\end{pmatrix}
		\begin{pmatrix}
			1	\\ 
			1	\\ 
			1	\\ 
			1	\\ 
			1	
		\end{pmatrix}
		=
		\begin{pmatrix}
			a_{11}+a_{12}+a_{13}+a_{14}+a_{15}	\\ 
			a_{21}+a_{22}+a_{23}+a_{24}+a_{25}	\\ 
			a_{31}+a_{32}+a_{33}+a_{34}+a_{35}	\\ 
			a_{41}+a_{42}+a_{43}+a_{44}+a_{45}	\\ 
			a_{51}+a_{52}+a_{53}+a_{54}+a_{55}
		\end{pmatrix}.
	\end{equation}
	Par hypothèse, le vecteur du membre de droite vaut
	\begin{equation}
		\begin{pmatrix}
			\alpha	\\ 
			\alpha	\\ 
			\alpha	\\ 
			\alpha	\\ 
			\alpha	
		\end{pmatrix}=
		\alpha\begin{pmatrix}
			1	\\ 
			1	\\ 
			1	\\ 
			1	\\ 
			1	
		\end{pmatrix},
	\end{equation}
	donc on a bien un vecteur propre de valeur propre $\alpha$.

\end{corrige}
