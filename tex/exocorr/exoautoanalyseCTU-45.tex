 % This is part of Analyse Starter CTU
% Copyright (c) 2014
%   Laurent Claessens,Carlotta Donadello
% See the file fdl-1.3.txt for copying conditions.

\begin{exercice}\label{exoautoanalyseCTU-45}



On considère la fonction $f$ de la variable réelle $x$, définie par $f(x)=\arcsin \left(\dfrac{2x}{1+x^2}\right)-2\arctan (x)$.
\begin{enumerate}
\item Déterminer l'ensemble de définition de $f$.
\item Étudier la parité de $f$.
\item Calculer $f(0)$, $f(1)$, $f(\sqrt{3})$ et \parbox{2 cm}{$\displaystyle\lim_{x\to+\infty}f(x)$}.
\item Justifier que $f$ est dérivable sur chacun des intervalles $]-\infty\,;\,-1[$, $]-1\,;\,1[$ et $]1\,;\,+\infty[$, puis calculer l'expression de  $f'(x)$ .
\item En déduire l'expression de $f(x)$.
\item Établir le tableau de variations de $f$.
\item Tracer la courbe représentative de $f$.
\end{enumerate}






\corrref{exoautoanalyseCTU-45}
\end{exercice}
