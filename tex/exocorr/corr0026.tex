% This is part of Exercices et corrigés de CdI-1
% Copyright (c) 2011
%   Laurent Claessens
% See the file fdl-1.3.txt for copying conditions.

\begin{corrige}{0026}

\begin{enumerate}
\item La suite $k\mapsto \cos(2k\pi)$ n'est en réalité autre que la suite constante $x_k=1$. Elle converge donc vers $1$.
\item Parmi les sous-suites de la suite $k\mapsto\cos(\frac{ \pi }{ 3 }k)$, se trouvent les sous-suites constantes $x_k=\cos(\frac{ \pi }{ 3 }+2k\pi)$ et $y_k=\cos(\frac{ 4\pi }{ 3 }+2k\pi)<0$. La suite ne peut donc pas converger.
\item Nous étudions la limite de la fonction
\begin{equation}
	f(x)=x\big( a^{1/x}-1 \big)
\end{equation}
lorsque $x\to\infty$. Nous tombons sur une indétermination $\infty\times 0$ qui se lève en utilisant la règle de L'Hôpital. Attention : la règle de l'Hospital ne peut pas être utilisée telle quelle : il faut utiliser $fg=f/(1/g)$ afin de retrouver un cas de type $\frac{ 0 }{ 0 }$. Le calcul est donc le suivant :
\begin{equation}
	\begin{aligned}[]
		\lim_{x\to\infty} x\big( a^{1/x}-1 \big)	&=\lim_{x\to\infty}\frac{ (a^{1/x}-1) }{ 1/x }\\
				&=\lim_{x\to\infty}\frac{ \frac{ -a^{1/x}\ln(a) }{ x^2 } }{ -1/x^2 }\\
				&=\lim_{x\to\infty}a^{1/x}\ln(a)\\
				&=\ln(a).
	\end{aligned}
\end{equation}

\end{enumerate}


\end{corrige}
