% This is part of Un soupçon de physique, sans être agressif pour autant
% Copyright (C) 2006-2009
%   Laurent Claessens
% See the file fdl-1.3.txt for copying conditions.


\begin{corrige}{INGE11140033}

	\begin{enumerate}

		\item
			Cette suite est la suite type qui tend vers zéro. En effet, prenons n'importe quel $\epsilon>0$, et trouvons à partir de quel moment dans la suite, $s_n<\epsilon$. Nous cherchons dons les $n$ pour lesquels $\frac{1}{ n }<\epsilon$. Cette inéquation est facile à résoudre~: la réponse est
			\begin{equation}
				n>\frac{1}{ \epsilon }.
			\end{equation}
			Prenons par exemple $\epsilon=0.1$. Effectivement, dès que $n>10$, nous avons $s_n=\frac{1}{ n }<0.1$.

		\item
			Cette suite ressemble à une fonction, donc nous allons appliquer à peu près les mêmes recettes. D'abord nous mettons $n$ en évidence au numérateur et au dénominateur et nous simplifions~:
			\begin{equation}
				s_n=\frac{ 2 }{ 3-\frac{ 2 }{ n } }-\frac{ 1/n }{ 3-\frac{ 2 }{ n } }.
			\end{equation}
			À ce moment nous appliquons le théorème qui dit que $\lim(a_n-b_n)=\lim(a_n)-\lim(b_n)$ (quand les limites existent). Ici,
			\begin{equation}
				\begin{aligned}[]
					a_n&=\frac{ 2 }{ 3-\frac{ 2 }{ n } } & b_n&=\frac{ 1/n }{ 3-\frac{ 2 }{ n } }.
				\end{aligned}
			\end{equation}
			En décomposant encore une fois chacun des deux termes en utilisant le théorème qui dit que $\lim(a_n/b_n)=\frac{ \lim a_n }{ \lim b_n }$, et en utilisant le fait que $\frac{1}{ n }\to 0$, nous avons que le limite du tout vaut $\frac{ 2 }{ 3 }$, comme nous nous y attendions depuis le début.

		\item
			Nous pouvons utiliser la technique de l'étau en écrivant que pour tout $n$,
			\begin{equation}
				-\frac{1}{ n }\leq \frac{ (-1)^n }{ n }\leq \frac{1}{ n }.
			\end{equation}
			Les deux suites extrêmes tendent vers $0$, donc celle du milieu tend vers zéro.

	\end{enumerate}

\end{corrige}
