% This is part of Exercices et corrigés de CdI-1
% Copyright (c) 2011,2012
%   Laurent Claessens
% See the file fdl-1.3.txt for copying conditions.

\begin{corrige}{0047}

La continuité en $(0,0)$ est évidente parce que $f(r,\theta)$ est majorée par $r$ dans la boule $B(0,r)$. Calculons la dérivée directionnelle de $f$ dans la direction $u$ au point $(0,0)$. Ce que nous trouvons est
\begin{equation}
	\frac{ \partial f }{ \partial u }(0,0)=\lim_{t\to 0}\frac{ f(tu_1,tu_2) }{ t }=
\begin{cases}
	u_1	&	\text{si }u_1u_2>0\\
	u_2	&	 \text{si }u_1u_2\leq 0
\end{cases}
\end{equation}
Voyons si cette application est linéaire par rapport à $u$ (une application définie par morceau, c'est toujours un peu mal parti pour être linéaire). Afin d'alléger la notation, nous notons $A$ cette application.

Nous avons
\begin{equation}
	A\big( (1,2)+(-3,4) \big)=A(-2,6)=6,
\end{equation}
tandis que
\begin{equation}
	A(1,2)+A(-3,4)=5\neq 6.
\end{equation}
L'application $A$ n'est donc pas linéaire, ce qui fait que $f$ n'est pas différentiable en $(0,0)$.

\end{corrige}
