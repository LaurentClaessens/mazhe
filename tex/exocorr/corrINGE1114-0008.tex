% This is part of Un soupçon de physique, sans être agressif pour autant
% Copyright (C) 2006-2011
%   Laurent Claessens
% See the file fdl-1.3.txt for copying conditions.


\begin{corrige}{INGE1114-0008}

	\begin{enumerate}

		\item
			$\sup E=1$ et $1\in E$, donc $\max E=1$. Par contre, $\inf E=0$ et $0\notin E$, donc il n'y a pas de minimum.
		\item
			Ici, $0$ est dans l'ensemble, donc la différence avec le précédent est que $\min E=0$. 
		\item
		\item
		\item
			Pour rappel, $0\in\eQ$ et $\sqrt{2}\notin\eQ$. Il n'y a donc pas de minimum (parce que $0$ n'est pas dans l'ensemble), mais $\inf E=0$ et $\max E=\sup E=\sqrt{2}$.
		\item
			Dans la suite des $\frac{1}{ n }+(-1)^n$, un terme sur deux s'approche de $1$ et l'autre terme sur deux s'approche de $-1$ (dessiner sur un graphe les cinq ou six premiers termes pour s'en persuader). Nous avons donc $\inf E=-1$ et $\max E=\sup E=\frac{1}{ 2 }+1=\frac{ 3 }{ 2 }$. Il n'y a pas de minimum.

	\end{enumerate}

\end{corrige}
