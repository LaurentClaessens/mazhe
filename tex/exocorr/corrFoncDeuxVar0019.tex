% This is part of the Exercices et corrigés de mathématique générale.
% Copyright (C) 2010
%   Laurent Claessens
% See the file fdl-1.3.txt for copying conditions.

\begin{corrige}{FoncDeuxVar0019}

	Pour prouver la différentiabilité, nous allons prouver que les dérivées partielles $\partial_xf$ et $\partial_yf$ sont continues. La fonction sera alors différentiable par le théorème \ref{ThoProuverDiffable}.

	\begin{enumerate}

		\item
			Calculons
			\begin{equation}
				\begin{aligned}[]
					\frac{ \partial f }{ \partial x }&=\frac{ 24x^2y^3(4x^2+y^2)-64x^4y^3 }{ (4x^2+y^3)^2 }\\
					&=\frac{ 8x^2y^3(4x^2+3y^2) }{ (4x^2+y^2) }.
				\end{aligned}
			\end{equation}
			Cette fonction est continue partout en dehors du point $(0,0)$ (parce que c'est une fraction de polynômes). Au point $(0,0)$, elle est également continue parce qu'elle a une limite qui vaut zéro. 
			
			Attention : ce qui fait la continuité de la \emph{dérivée} n'est pas le fait que la limite de $\partial_xf$ soit zéro, mais simplement le fait que cette limite existe. Le fait qu'ici la dérivée vaille numériquement la même chose que la fonction elle-même est une coïncidence.

			En ce qui concerne l'autre dérivée partielle nous avons la même chose.

			Les dérivées partielles étant continues, la fonction est différentiable.
		\item	Pas différentiable.
		\item
			La dérivée partielle dans la direction $x$ est
			\begin{equation}
				\frac{ \partial f }{ \partial x }=\ln\sqrt{x^2+y^2}+\frac{ 2x^2 }{ \sqrt{x^2+y^2} }.
			\end{equation}
			La limite du premier terme a l'air problématique, et le second terme tendant simplement vers zéro, il n'y a aucune chance qu'il compense. Pour en être sûr, prenons le chemin $(t,t)$ et calculons
			\begin{equation}
				\lim_{t\to 0}\ln(| t |)+\frac{ 2 }{ \sqrt{2} }t.
			\end{equation}
			Cette limite n'existe pas.

			La fonction proposée n'est pas différentiable en $(0,0)$. Pas besoin de calculer la dérivée partielle par rapport à $y$.

	\end{enumerate}
\end{corrige}

