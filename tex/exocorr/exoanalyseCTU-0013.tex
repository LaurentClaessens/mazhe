% This is part of Analyse Starter CTU
% Copyright (c) 2014
%   Laurent Claessens,Carlotta Donadello
% See the file fdl-1.3.txt for copying conditions.

\begin{exercice}\label{exoanalyseCTU-0013}


\begin{enumerate}
\item 
Compléter le tableau suivant. 
  \begin{equation}
    \begin{array}{|c|c|c|}\hline 
      \text{Primitive } \int f(x)\, dx & \quad\text{Fonction } f(x)\quad &\quad \text{Dérivée } f'(x)\quad \\\hline 
      \vspace{1mm}&&\\
      \quad\ldots\quad & \quad e^x \quad&\quad \ldots \quad\\
      \vspace{1mm}&&\\\hline
      \vspace{1mm}&&\\
      \quad\ldots & \frac{1}{x} & \ldots\\
      \vspace{1mm}&&\\\hline 
      \vspace{1mm}&&\\
      \quad\ldots & \cos(x) & \ldots \\
      \vspace{1mm}&&\\\hline 
      \vspace{1mm}&&\\
      \quad\ldots & \quad x^\alpha\quad & \ldots \\
      \vspace{1mm}&&\\\hline 
    \end{array}
  \end{equation}
  \item Calculer les intégrales suivantes
    \begin{enumerate}
    \item $\displaystyle \int_0^1 x^3+x^{1/3} \, dx$ ;
    \item $\displaystyle \int_\pi^{3\pi/2} 5\sin(x) \, dx$ ;
    \item $\displaystyle \int_1^{2} e^x + \frac{1}{x} \, dx$. 
    \end{enumerate}
\item Calculer l'intégrale suivante par la méthode du changement de variable
 \[ \int_1^2 \frac{3x^2}{x^3 +17} \, dx.\]
\end{enumerate}


\corrref{analyseCTU-0013}
\end{exercice}
