% This is part of Exercices et corrigés de CdI-1
% Copyright (c) 2011
%   Laurent Claessens
% See the file fdl-1.3.txt for copying conditions.

\begin{corrige}{OutilsMath-0032}

    Nous devons trouver une fonction $V(x,y,z)$ telle que $\nabla V(r)=k\frac{ r }{ \| r \|^3 }$, c'est-à-dire telle que
    \begin{equation}
        \nabla V\begin{pmatrix}
            x    \\ 
            y    \\ 
            z    
        \end{pmatrix}=\frac{ k }{ (x^2+y^2+z^2)^{3/2} }\begin{pmatrix}
            x    \\ 
            y    \\ 
            z    
        \end{pmatrix}.
    \end{equation}
    Cette équation revient aux trois équations suivantes :
    \begin{equation}
        \begin{aligned}[]
            \frac{ \partial V }{ \partial x }(x,y,z)=\frac{ kx }{ (x^2+y^2+z^2)^{3/2} }\\
            \frac{ \partial V }{ \partial y }(x,y,z)=\frac{ ky }{ (x^2+y^2+z^2)^{3/2} }\\
            \frac{ \partial V }{ \partial z }(x,y,z)=\frac{ kz }{ (x^2+y^2+z^2)^{3/2} }.
        \end{aligned}
    \end{equation}
    Il est facile de voir que la fonction
    \begin{equation}
        V(x,y,z)=\frac{ -k }{ \sqrt{x^2+y^2+z^2} }
    \end{equation}
    fait l'affaire. Nous pouvons aussi l'écrire
    \begin{equation}
        V(r)=-k\frac{1}{ \| r \| }.
    \end{equation}
\end{corrige}
