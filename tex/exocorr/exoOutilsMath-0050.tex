% This is part of Exercices et corrigés de CdI-1
% Copyright (c) 2011
%   Laurent Claessens
% See the file fdl-1.3.txt for copying conditions.

\begin{exercice}\label{exoOutilsMath-0050}

    Au lancer du poids, si on lance le poids avec une vitesse initiale $v_0>0$ et un angle $\alpha$, alors on atteint une distance
    \begin{equation}
        d(\alpha)=\frac{ 2v_0^2 }{ g }\cos(\alpha)\sin(\alpha).
    \end{equation}

    \begin{enumerate}
        \item
            Avec quel angle faut-il lancer le poids afin d'obtenir le meilleur résultat possible ?
        \item
            À quelle distance arrive le poids dans ces conditions ?
    \end{enumerate}
    Pour rappel, $g$ est une constante positive. Pour des raisons physiques évidentes (faire un dessin de la situation), la réponse doit être un angle $\alpha\in\mathopen[ 0 , \pi/1 \mathclose]$. Les autres solutions éventuelles doivent être rejetées.

\corrref{OutilsMath-0050}
\end{exercice}
