% This is part of the Exercices et corrigés de mathématique générale.
% Copyright (C) 2009, 2018
%   Laurent Claessens
% See the file fdl-1.3.txt for copying conditions.
\begin{corrige}{Janvier013}



Les intersections se trouvent en résolvant $e^{2x} + 1 = 2 a e^x$ ce
qui se fait en posant par exemple $t = e^x$, et donc il faut résoudre
(pour $t > 0$) l'équation
\begin{equation*}
  t^2  - 2 a t + 1 = 0
\end{equation*}
dont les solutions existent si $| a |\geq 1$ (car le discriminant
est $4(a^2 -1)$) et sont alors données par
\begin{equation*}
  t = a \pm \sqrt{a^2-1}.
\end{equation*}
Comme $\sqrt{a^2 - 1} <  |a|$ dès que $|a| > 1$, on en déduit
que $t < 0$ si $a \leq -1$, et $t > 0$ si $a \geq 1$, ce qui montre
qu'il n'y a d'intersection que pour $a \geq 1$.

De plus, le calcul du discriminant et la discussion précédente
montrent que l'intersection est unique si $a = 1$, et cette
intersection a lieu en $x = \ln(1) = 0$.

Il reste à voir que la dérivée de $e^{2x} + 1$ (en $x = 0$) est égale
à la dérivée de $2 a e^x$ (en $x = 0$). Or la première dérivée est $2
e^{2x}$, et la deuxième est $2 a e^x$, ce qui donne $2$ dans les deux
cas pour $a = 1$ et $x = 0$.


\end{corrige}
