\begin{corrige}{SC_serie3-0001}

Un polynôme est donné par un vecteur qui contient les coefficients. Ainsi, le polynôme $x^2-2x+3$ sera représenté par $p =[1,-2,3]$. Attention : dans la tête de Matlab, ce $p$ reste un vecteur. Lui demander de faire \verb+plot(p)+ ne va pas du tout lui faire tracer le graphe du polynôme.

En ce qui concerne le tracé, il se fait que Matlab place automatiquement les nombres complexes dans le plan complexe. Ainsi, si $z$ est un nombre complexe, \verb+plot(z)+ affichera le point du plan qui correspond à $z$.

\lstinputlisting{tex/matlab/SC_exo_3-1.m}

\end{corrige}
