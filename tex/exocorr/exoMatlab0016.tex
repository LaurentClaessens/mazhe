\begin{exercice}\label{exoMatlab0016}

	En relativité, on démontre que si la longueur d'un objet est $l_0$, alors un observateur en mouvement à la vitesse $v$ mesurera une longueur donnée par
	\begin{equation}
		l(v)=l_0\sqrt{1-\frac{ v^2 }{ c^2 }}
	\end{equation}
	où $c=\unit{3\cdot 10^8}{\meter\per\second}$ est une constante physique.

	Tracez le graphe de la longueur $l(v)$ observée en fonction de $v$ dans le cas d'un objet de taille $l_0=\unit{1.3}{\meter}$, pour $v$ allant de $0$ à $\unit{3\cdot 10^8}{\meter\per\second}$.

\corrref{Matlab0016}
\end{exercice}
