% This is part of Exercices et corrigés de CdI-1
% Copyright (c) 2011
%   Laurent Claessens
% See the file fdl-1.3.txt for copying conditions.

\begin{exercice}\label{exoOutilsMath-0095}
    
    On considère la fonction 
    \begin{equation}
        f\colon (x,y,z)\in\eR^2\to x^2yz+2y^2\sin(xyz).
    \end{equation}
    \begin{enumerate}
        \item
            Calculer les dérivées partielles premières de $f$.
        \item
            Quelle est la différentielle de $f$ au point $(1,1,\pi)$ ?
        \item
            Donner une approximation de $f(1+10^{-2},1-10^{-3},\pi+10^{-4})$.
        \item
            Soit $\overrightarrow{F}(x,y,z)=\nabla f(x,y,z)$ le gradient de $f$ au point $(x,y,z)$. Que vaut la circulation de $\overrightarrow{F}$ le long d'une courbe d'extrémités $(1,2,\pi)$ et $(1,1,\frac{ \pi }{ 2 })$.
    \end{enumerate}

\corrref{OutilsMath-0095}
\end{exercice}
