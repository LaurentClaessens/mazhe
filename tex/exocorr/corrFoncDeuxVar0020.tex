% This is part of the Exercices et corrigés de mathématique générale.
% Copyright (C) 2010
%   Laurent Claessens
% See the file fdl-1.3.txt for copying conditions.

\begin{corrige}{FoncDeuxVar0020}

	En notant $u$ et $v$ les variables de $f$ nous avons
	\begin{equation}
		\frac{ \partial F }{ \partial x }(0,0)=\frac{ \partial f }{ \partial u }\big( g(0,0),0 \big)\frac{ \partial g }{ \partial x }(0,0)
		+\frac{ \partial f }{ \partial v }\big( g(0,0),0 \big)\frac{ \partial y }{ \partial x }.
	\end{equation}
	Le second terme disparaît parce que $\partial_xy=0$. En remplaçant chaque morceau par sa valeur, nous trouvons
	\begin{equation}
		\frac{ \partial f }{ \partial u }(1,0)\frac{ \partial g }{ \partial x }(0,0)=-2\cdot(-1)=2.
	\end{equation}
	Ici, il faut remarquer que les notations de l'énoncé sont faites pour induire en erreur : lorsqu'il est écrit $\frac{ \partial f }{ \partial x }$, on veut dire «la dérivée de $f$ par rapport à sa première variable». Ce n'est donc pas le même $x$ que les autres.

	En ce qui concerne la dérivée de $F$ par rapport à $y$, il faut faire le même raisonnement en tenant compte du fait que $\partial_yy=1$. Le résultat est
	\begin{equation}
		\frac{ \partial F }{ \partial y }(0,0)=2.
	\end{equation}
	

\end{corrige}
