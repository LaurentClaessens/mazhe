% This is part of the Exercices et corrigés de CdI-2.
% Copyright (C) 2008, 2009
%   Laurent Claessens
% See the file fdl-1.3.txt for copying conditions.


\begin{corrige}{_I-1-14}

Par le critère des séries de puissances (poser $y=(x-3)/(2x+1)$), nous avons la convergence uniforme de $f(x)$ pour $\left| \frac{ x-3 }{ 2x+1 } \right|<1$, donc pour $x\in[1,3]$. La série des dérivées se somme en utilisant la même technique que pour l'exercice \ref{exo_I-1-13} :
\begin{equation}
	\begin{aligned}[]
		g(x)&=\sum_{n=1}^{\infty}\left( \frac{ x-3 }{ 2x+1 } \right)^{n-1}\frac{ 2x+1-2(x-3) }{ (2x+1)^2 }\\
			&= \frac{ 7 }{ (2x+1)^2 }\sum_{n=1}^{\infty}\left( \frac{ x-3 }{ 2x+1 } \right)^{n-1}\\
			&= \frac{ 7 }{ (2x+1)^2 }\sum_{n=0}^{\infty}\left( \frac{ x-3 }{ 2x+1 } \right)^{n}\\
			&=\frac{ 7 }{ (2x+1)^2 }\frac{1}{ 1-\frac{ x-3 }{ 2x+1 } }\\
			&=\frac{ 7 }{ (x+4)(2x+1) }.
	\end{aligned}
\end{equation}
Étant donné que la série $\sum_{n=0}^{\infty}\left( \frac{ x-3 }{ 2x+1 } \right)^n$ converge uniformément sur $[1,3]$, sur cet intervalle, $g(x)$ est effectivement la dérivée de $f(x)$.

\end{corrige}
