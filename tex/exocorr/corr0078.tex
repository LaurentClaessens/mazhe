% This is part of Exercices et corrigés de CdI-1
% Copyright (c) 2011,2014
%   Laurent Claessens
% See the file fdl-1.3.txt for copying conditions.

\begin{corrige}{0078}



Soit $A = \mathopen[2,3\mathclose[\setminus\{e\}$ et vérifions que
$\Int A = \mathopen]2,3\mathclose[\setminus\{e\}$~:
\begin{enumerate}
\item Vérifions d'abord que $2$ n'est pas un point intérieur à $A$. En
  effet, si on considère la boule $B(2,\epsilon)$ autour de $2$, le
  point $2 - \epsilon/2$ est bien dans cette boule mais $2 -
  \epsilon/2 < 2$ (car $\epsilon > 0$), donc $2 - \epsilon/2 \notin
  A$.

\item De plus, $\mathopen]2,3\mathclose[\setminus\{e\}$ est la réunion
  des ouverts $\mathopen]2,e\mathclose[$ et
  $\mathopen]e,3\mathclose[$, c'est donc un ouvert, qui est contenu
  dans $A$.
\end{enumerate}

On en déduit que $\mathopen]2,3\mathclose[\setminus\{e\}$ est le plus
grand ouvert contenu dans $A$, c'est donc bien l'intérieur de $A$.

Similairement, pour l'adhérence, on veut montrer que $\Adh A =
[2,3]$~:
\begin{enumerate}
\item Notons d'abord que pour $n$ assez grand, les quantités $e +
  \sfrac1n$ et $3 - \sfrac1n$ sont dans $A$. Les suites
  correspondantes, $x_n = e + \sfrac1n$ et $y_n = 3- \sfrac1n$,
  tendent vers $e$ et $3$ respectivement. On en déduit que $e$ et $3$
  doivent être dans l'adhérence de $A$.

\item D'autre part, l'ensemble $[2,3]$ étant fermé, c'est forcément le
  plus petit fermé contenant $A$.
\end{enumerate}

Le fait que $A$ n'est pas ouvert ni fermé résulte des définitions car
$A$ n'est ni égal à son adhérence, ni égal à son intérieur. Le fait
que $A$ n'est pas connexe par arcs peut se justifier comme suit~: $A$
est inclus dans la réunion $\mathopen]-\infty,e\mathclose[ \cup
\mathopen ]e,+\infty\mathclose[$. Ces deux ouverts
$\mathopen]-\infty,e\mathclose[$ et $\mathopen ]e,+\infty\mathclose[$
sont disjoints, et aucun d'eux ne contient entièrement $A$. Donc $A$
n'est pas connexe, et en particulier n'est pas connexe par arcs.


\end{corrige}
