% This is part of Exercices et corrigés de CdI-1
% Copyright (c) 2011, 2019
%   Laurent Claessens
% See the file fdl-1.3.txt for copying conditions.

\begin{corrige}{0021}

\begin{enumerate}
\item \marginpar{Dessin, un jour peut-être\ldots}
\item Nous avons
\begin{equation}
	x_{n+2}-x_n=\frac{ x_n-x_{n+1} }{ 2 }-x_{n+1}=-\frac{1}{ 2 }(x_{n+1}-x_n).
\end{equation}

\item 
Nous avons
\begin{equation}
	x_{n+1}-x_1=x_{n+1}-x_n+x_n-x_{n-1}+x_{n-1}-\ldots +x_3-x_2+x_2-x_1
\end{equation}
que l'on regroupe deux à deux pour obtenir le résultat.

\item
Si nous posons $q=-1/2$, nous avons, en utilisant le résultat précédent, que
\begin{equation}
	x_{n+1}=\left( \sum_{k=0}^{n-1}q^k \right)(b-a)+a
\end{equation}
où nous avons utilisé le fait que $\sum_{k=1}^nq^{k-1}=\sum_{k=0}^{n-1}q^k$. Maintenant, il faut savoir prendre la limite, c'est-à-dire sommer cette série géométrique. Cela se fait en lisant \href{http://fr.wikipedia.org/wiki/Série_géométrique}{wikipedia} : la somme d'une série géométrique de raison $q$ est $1/(1-q)$. De là, la conclusion vient aisément.

\end{enumerate}

{\bf Correction alternative}

\begin{enumerate}
\item \marginpar{Toujours pas de dessin :-(}

\item
 Nous voulons montrer pour tout $n \geq 1$ la proposition
  \begin{equation*}
    P(n) \equiv x_{n+1} - x_n = {\left(-\frac12\right)}^{n-1} (b-a).
  \end{equation*}

  La proposition $P(1)$ est vraie, on trouve $b-a = b-a$. Supposons
  $P(i)$ vraie pour $i \geq 1$ fixé. Nous avons alors successivement
  \begin{equation*}
    \begin{split}
      x_{i+2} - x_{i+1} &= \frac12 (x_{i} + x_{i+1}) - x_{i+1}\\
      &= -\frac12 \left(x_{i+1} - x_i\right)\\
      &= -\frac12 {\left(-\frac12\right)}^{i-1} (b-a) \quad\text{car } P(i+1) \text{ est supposée vraie}\\
      &= {\left(-\frac12\right)}^{i} (b-a)
    \end{split}
  \end{equation*}
  ce qui est exactement la relation $P(i+1)$. Par le principe de
  récurrence, $P(n)$ est vraie pour tout naturel $n \geq 1$, ce qu'on
  voulait démontrer.
\item On veut une égalité sur $x_{n+1} - x_1$, cest-à-dire la
  différence entre deux termes éloignés dans la suite. L'astuce
  usuelle dans ce cas est de faire apparaitre les différences des
  termes successifs :
  \begin{equation*}
    \begin{split}
      x_{n+1} - x_1 &= x_{n+1} + (- x_n + x_n) + (- x_{n-1} + x_{n-1})
      + \ldots\\
      & \qquad + (- x_3 + x_3) + (- x_2 + x_2) - x_1\\
      &= (x_{n+1} - x_n) + (x_n - x_{n-1}) + (x_{n-1} + \ldots\\
      & \qquad - x_3) + (x_3 - x_2) + (x_2 - x_1)\\
      &= \sum_{i=1}^n (x_{i+1} - x_i)\\
      &= \sum_{i=1}^n \left[{\left(-\frac12\right)}^{i-1} (b-a)\right]\\
      &= \sum_{i=1}^n \left[{\left(-\frac12\right)}^{i-1}\right] (b-a)
    \end{split}
  \end{equation*}
  ce qui prouve l'égalité souhaitée.
\item On se rappelle de la formule de la \emph{série géométrique}
  \begin{equation*}
    \sum_{k=0}^i q^k = \frac{1 - q^{i+1}}{1 - q}
  \end{equation*}
  ce qui, appliquée au cas qui nous intéresse, donne (avec $q =
  \sfrac{-1}2$)
  \begin{equation*}
    \begin{split}
      x_{n+1} &= x_1 + \sum_{i=1}^n
      \left[{\left(-\frac12\right)}^{i-1}\right] (b-a)
      \qquad\text{(regardez bien les indices !)}\\
      &= x_1 + \frac{1 - (-\frac12)^n}{1 - (-\frac12)} (b-a)\\
      &= x_1 + \frac{1 - (-\frac12)^n}{\frac32} (b-a)
    \end{split}
  \end{equation*}
  et, étant donné que $\abs{-\frac12} < 1$, en passant à la limite on
  a
  \begin{equation*}
    \limite n \infty (-\frac12)^n = 0
  \end{equation*}
  ce qui montre que
  \begin{equation*}
    \limite n\infty (x_{n+1}) = x_1 + \frac{1 - 0}{\frac32} (b-a) = a +
    \frac23 (b-a) = \frac23 b + \frac13 a
  \end{equation*}
  qui est le résultat attendu.
\end{enumerate}


\end{corrige}
