% This is part of Analyse Starter CTU
% Copyright (c) 2014
%   Laurent Claessens,Carlotta Donadello
% See the file fdl-1.3.txt for copying conditions.

\begin{corrige}{starterST-0009}

\begin{enumerate}
\item \begin{enumerate}
\item C'est un cas particulier d'utilisation des formules, qui devraient \^etre bien connues, pour les valeurs de cosinus et de sinus d'une somme ou d'une différence d'angles. 
\begin{Aretenir}
  \begin{equation}
    \cos(\alpha \pm \beta) = \cos(\alpha)\cos(\beta)\mp\sin(\alpha)\sin(\beta) ; 
  \end{equation}
\begin{equation}
    \sin(\alpha \pm \beta) = \sin(\alpha)\cos(\beta)\pm\cos(\alpha)\sin(\beta).  
  \end{equation}
\end{Aretenir}
Nous obtenons 
\begin{Aretenir}
  \begin{equation}
    \cos(2x) = \cos^2(x)-\sin^2(x) ; 
  \end{equation}
\begin{equation}
    \sin(2x) = 2\sin(x)\cos(x).  
  \end{equation}
\end{Aretenir}
\item On sait que $\displaystyle \tan(x) = \frac{\sin(x)}{\cos(x)}$, par conséquent, les formules du point précédent nous donnent 
\begin{equation*}
  \tan(2x) = \frac{\sin(2x)}{\cos(2x)} =\frac{2\sin(x)\cos(x)}{\cos^2(x)-\sin^2(x)} = \frac{2\cos^2(x)\tan(x)}{\cos^2(x)\left(1-\tan^2(x)\right)} = \frac{2\tan(x)}{1-\tan^2(x)}.
\end{equation*}
\item L'angle $\dfrac{\pi}{8}$ est la moitié de l'angle $\dfrac{\pi}{4}$, dont on connaît les valeurs de sinus et cosinus :  $\displaystyle \sin\left(\frac{\pi}{4}\right) = \cos\left(\frac{\pi}{4}\right) = \frac{\sqrt{2}}{2}$. Pour trouver  $\displaystyle \cos \left(\frac{\pi}{8}\right)$ et $\displaystyle  \sin \left(\frac{\pi}{8}\right)$ nous écrivons alors un système de deux équations 
\begin{equation*}
  \begin{cases}
   \frac{\sqrt{2}}{2} =  \cos^2 \left(\frac{\pi}{8}\right) - \sin^2 \left(\frac{\pi}{8}\right) , \\
   \frac{\sqrt{2}}{2} = 2\cos \left(\frac{\pi}{8}\right)\sin \left(\frac{\pi}{8}\right).
  \end{cases}
\end{equation*}
Ensuite on utilise la relation $\cos^2(x) + \sin^2(x) =1$, pour tout $x\in \eR$, ce qui nous donne, dans la première équation $\displaystyle \cos \left(\frac{\pi}{8}\right) = \left(\frac{1}{2} + \frac{\sqrt{2}}{4}\right)^{1/2}$ et ensuite  $\displaystyle \sin \left(\frac{\pi}{8}\right) = \left(\frac{1}{2} -\frac{\sqrt{2}}{4}\right)^{1/2}$. Il est possible (mais demande un peu plus de travail) de ne pas utiliser du tout la relation $\cos^2(x) + \sin^2(x) =1$ car le système nous donne déjà suffisamment d'information. 
\end{enumerate}
\item \begin{enumerate}
\item Ici aussi, il suffit d'écrire les angles $\displaystyle \frac{\pi}{12}$, $\displaystyle \frac{7\pi}{12}$ et $\displaystyle \frac{5\pi}{12}$ comme sommes ou différences d'angles dont on connaît les valeurs de $\cos$, $\sin$ et $\tan$. Les formules vues dans les premiers points de cet exercice nous permettrons de conclure. Une possibilité est la suivante $\displaystyle \frac{\pi}{12} =\frac{\pi}{3} - \frac{\pi}{4} $,  $\displaystyle \frac{5\pi}{12} =\frac{\pi}{4} + \frac{\pi}{6} $, $\displaystyle \frac{7\pi}{12} =\frac{\pi}{4} + \frac{\pi}{3} $. 
\item 
  \begin{enumerate}
  \item Dans la première équation il faut juste chercher un angle $\varphi$ tel que $\cos(\varphi) = \frac{1}{2}$ et $\sin(\varphi) = \frac{\sqrt{3}}{2} $. La réponse est donc $\varphi = \frac{\pi}{6}$
  \item Dans cette deuxième équation il faut que $\cos(\varphi) = - \sin(\varphi) = \frac{1}{A}$. On trouve alors $A = \sqrt{2}$ et $\varphi = -\frac{\pi}{4}$.
  \end{enumerate}
\end{enumerate}
\end{enumerate}

\end{corrige}
