\begin{corrige}{CalculDifferentiel0010}

	Nous effectuons le changement de variable suivant :
	\begin{equation}
		\begin{aligned}[]
			x(u,v)&=u&u(x,y)&=x\\
			y(u,v)&=v+u^2&v(x,y)&=y-x^2.
		\end{aligned}
	\end{equation}
	Nous allons avoir besoin des dérivées de ce changement de variable :
	\begin{equation}
		\begin{aligned}[]
			\frac{ \partial u }{ \partial x }&=1&\frac{ \partial v }{ \partial x }&=-2x\\
			\frac{ \partial u }{ \partial y }&=0&\frac{ \partial v }{ \partial y }&=1.
		\end{aligned}
	\end{equation}
	Maintenant nous considérons la fonction $\tilde f$ comme étant la fonction $f$ «vue dans les variables $u$ et $v$», c'est-à-dire
	\begin{equation}
		\tilde f(u,v)=f\big( x(u,v),y(u,v) \big).
	\end{equation}
	Nous pouvons aussi voir $f$ comme donnée en termes de $\tilde f$ par la formule inverse :
	\begin{equation}
		f(x,y)=\tilde f\big( u(x,y),v(x,y) \big).
	\end{equation}
	Nous calculons les dérivées de $f$ en termes de celles de $\tilde f$ en utilisant la formule de dérivation de fonctions composées :
	\begin{equation}
		\begin{aligned}[]
			\frac{ \partial f }{ \partial x }(x,y)&=\frac{ \partial \tilde f }{ \partial u }(u,v)\underbrace{\frac{ \partial u }{ \partial x }(x,y)}_{=1}+\frac{ \partial \tilde f }{ \partial v }(u,v)\underbrace{\frac{ \partial v }{ \partial x }(x,y)}_{=-2x}\\
			&=\partial_u\tilde f(u,v)-2x\partial_v\tilde f(u,v).
		\end{aligned}
	\end{equation}
	De la même façon,
	\begin{equation}
		\begin{aligned}[]
			\frac{ \partial f }{ \partial y }(x,y)&=\frac{ \partial \tilde f }{ \partial u }(u,v)\frac{ \partial u }{ \partial y }(x,y)+\frac{ \partial \tilde f }{ \partial v }(u,v)\frac{ \partial v }{ \partial y }(x,y)\\
			&=\partial_v\tilde f(u,v).
		\end{aligned}
	\end{equation}
	Il est sous-entendu que lorsque nous écrivons $\partial_v\tilde f(u,v)$, nous entendons
	\begin{equation}
		\frac{ \partial \tilde f }{ \partial v }\big( u(x,y),v(x,y) \big),
	\end{equation}
	en tant que fonction de $x$ et $y$.

	L'équation \eqref{eCDuzEqares} devient donc $\partial_u\tilde f-2x\partial_v\tilde f+2x\partial_v\tilde f=0$, c'est-à-dire simplement
	\begin{equation}
		\frac{ \partial \tilde f }{ \partial u }(u,v)=0.
	\end{equation}
	La résolution de cette équation est qu'il doit exister une fonction $\psi\colon \eR\to \eR$ telle que
	\begin{equation}
		\tilde f(u,v)=c+\psi(v).
	\end{equation}
	La fonction $f$ donnée en terme des variables $x$ et $y$ devient
	\begin{equation}
		f(x,y)=\tilde f\big( u(x,y),v(x,y) \big)=c+\psi\big( v(x,y) \big)=c+\psi(y-x^2).
	\end{equation}
	Notez que l'on peut écrire plus simplement $f(x,y)=\psi(y-x^2)$ parce que la fonction $\psi$ étant arbitraire, on peut la redéfinir pour inclure la constante.

\end{corrige}
