% This is part of Exercices et corrigés de CdI-1
% Copyright (c) 2011
%   Laurent Claessens
% See the file fdl-1.3.txt for copying conditions.

\begin{corrige}{OutilsMath-0019}

	En remplaçant $x$ par $r\cos(\theta)$ et $y$ par $r\sin(\theta)$, nous trouvons immédiatement l'équation
	\begin{equation}
		r^2=\frac{1}{ \sin(\theta)\cos(\theta) }.
	\end{equation}
	Il faut trouver le domaine de variation de $\theta$. Étant donné que $r^2$ doit être positif, seuls les $\theta$ tels que $\sin(\theta)\cos(\theta)>0$ sont éligibles. Nous avons donc le domaine de variation
	\begin{equation}
		\theta\in\mathopen] 0 , \frac{ \pi }{2} \mathclose[\cup\mathopen] \pi , \frac{ 3\pi }{2} \mathclose[.
	\end{equation}
	En confirmation, le graphique de la figure \ref{LabelFigExoUnSurxPolaire} montre bien que la courbe n'est présente que dans les premiers et troisièmes quadrants.
	\newcommand{\CaptionFigExoUnSurxPolaire}{La fonction $y=1/x$.}
	\input{auto/pictures_tex/Fig_ExoUnSurxPolaire.pstricks}


\end{corrige}
