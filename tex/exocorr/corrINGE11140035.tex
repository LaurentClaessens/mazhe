% This is part of Un soupçon de physique, sans être agressif pour autant
% Copyright (C) 2006-2009
%   Laurent Claessens
% See the file fdl-1.3.txt for copying conditions.


\begin{corrige}{INGE11140035}

	Pour prouver que la suite est croissante, nous calculons $\frac{ s_n }{ s_{n-1} }$, et nous prouvons que cette fraction est plus grande que $1$ pour tout $n$. Nous avons
	\begin{equation}
		\frac{ s_n }{ s_{n-1} }=\frac{ \sqrt{3 s_{n-1}} }{ s_{n-1 }}=\frac{ \sqrt{3} }{ \sqrt{s_{n-1}} }.
	\end{equation}
	Pour que cela soit plus grand que $1$, il faut d'abord s'assurer que $s_{n-1}<3$. Cela se fait par récurrence. En effet, si $s_k<3$, alors
	\begin{equation}
		s_{k+1}=\sqrt{3s_k}<\sqrt{3\cdot 3}=3.
	\end{equation}
	Nous avons donc bien que tous les éléments de la suite sont plus petits que $3$, et donc que la suite est croissante et bornée. 

	La borne se trouve en utilisant l'astuce suivante. Nous savons que pour tout $n$, 
	\begin{equation}			\label{EqLimPresNsqrttrois}
		s_{n+1}=\sqrt{3s_n}.
	\end{equation}
	Si nous notons $l=\lim s_n$, et si nous prenons la limite des deux côtés de \eqref{EqLimPresNsqrttrois}, nous trouvons
	\begin{equation}
		l=\sqrt{3l},
	\end{equation}
	dont la solution est $l=3$.

\end{corrige}
