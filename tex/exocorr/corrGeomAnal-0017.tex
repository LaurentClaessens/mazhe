\begin{corrige}{GeomAnal-0017}

    \begin{enumerate}
        \item
            Étant donné que \( \partial_xf(x,y)=3x+y\), nous savons que \( f(x,y)\) doit être de la forme
            \begin{equation}        \label{EqGAzzusui}
                f(x,y)=\frac{ 3 }{ 2 }x^2+yx+a(y)
            \end{equation}
            où \( a(y)\) est une fonction seulement de \( y\). Aucune fonction de ce type n'a \( 4sin(x,y)\) comme dérivée par rapport à \( y\). En effet, la dérivée de \eqref{EqGAzzusui} par rapport à \( y\) est
            \begin{equation}    \label{EqGAzzuuii}
                3x+y+a'(y),
            \end{equation}
            et il n'y a aucun choix de fonction \( a(y)\) tel que \eqref{EqGAzzuuii} se réduise à \( 4\sin(xy)\).

        \item
            Vu que toutes les dérivées partielles de \( h\) sont continues sur \( \eR^3\setminus\{ 0,1,0 \}\), la fonction \( h\) y est différentiable (proposition \ref{Diff_totale})
    \end{enumerate}

% TODO : À terminer

\end{corrige}
