% This is part of Exercices et corrigés de CdI-1
% Copyright (c) 2011
%   Laurent Claessens
% See the file fdl-1.3.txt for copying conditions.

\begin{corrige}{OutilsMath-0096}

    \begin{enumerate}
        \item
            Le rotationnel s'écrit
            \begin{equation}
                \nabla\times F=\begin{vmatrix}
                    e_x    &   e_y    &   e_z    \\
                    \partial_x    &   \partial_y    &   \partial_z    \\
                    \frac{ y }{ x^2+y^2 }    &   \frac{ -x }{ x^2+y^2 }    &   0
                \end{vmatrix}=\left( \frac{ \partial F_y }{ \partial x }-\frac{ \partial F_x }{ \partial y } \right)e_z.
            \end{equation}
            En ce qui concerne les dérivées partielles, nous avons
            \begin{equation}
                \begin{aligned}[]
                    \frac{ \partial F_y }{ \partial x }&=\frac{ x^2-y^2 }{ (x^2+y^2)^2 }\\
                    \frac{ \partial F_x }{ \partial y }&=\frac{ x^2-y^2 }{ (x^2+y^2)^2 },
                \end{aligned}
            \end{equation}
            et par conséquent $\nabla\times F=0$.
        \item
            Nous avons
            \begin{equation}
                F\big( \sigma(t) \big)=F\big( \cos(t),\sin(t),0 \big)=\begin{pmatrix}
                    \sin(t)    \\ 
                    -\cos(t)    \\ 
                    0    
                \end{pmatrix},
            \end{equation}
            et
            \begin{equation}
                \sigma'(t)=\begin{pmatrix}
                    -\sin(t)    \\ 
                    \cos(t)    \\ 
                    0    
                \end{pmatrix}.
            \end{equation}
            Il faut donc intégrer
            \begin{equation}
                \int_{\sigma}F=\int_0^{2\pi}(-1)dt=-2\pi.
            \end{equation}
            
        \item
            Le fait que l'intégrale de $F$ le long d'un chemin fermé ne soit pas nulle implique que le champ ne dérive pas d'un potentiel. En effet, dès qu'un champ dérive d'un potentiel, son intégrale sur un chemin fermé vaut zéro.
    \end{enumerate}

\end{corrige}
