\begin{corrige}{controlecontinu0002}
  \begin{enumerate}
  \item Le domaine de définition de la fonction 
    \begin{equation}
      f(x,y)=\frac{\sqrt{y-3}}{\ln (x+y)}.
    \end{equation}
est la partie de $\eR$ déterminée par le conditions 
\[
y\geq 3, \qquad, x+y > 0, \quad \ln(x+y)\neq 0.
\]
cette dernière condition peut s'écrire aussi $x+y \neq 1$.
  \item La courbe de niveau $-1$ n'existe pas, celle de niveau $0$ est donnée par le point $(0,3)$, la courbe de niveau $1$ est un cercle de rayon $1$ centrée au point $(0,3)$.  
  \end{enumerate}


\end{corrige}
