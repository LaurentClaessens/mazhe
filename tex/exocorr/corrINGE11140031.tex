% This is part of Un soupçon de physique, sans être agressif pour autant
% Copyright (C) 2006-2009-2011
%   Laurent Claessens
% See the file fdl-1.3.txt for copying conditions.

\begin{corrige}{INGE11140031}

	\begin{enumerate}


		\item
			Cet exercice, comme beaucoup qui contiennent un cosinus ou un sinus en facteur, se traite avec la règle de l'étau. En effet, étant donné que $\sin(\frac{1}{ x })$ est coincé entre $-1$ et $1$, nous avons toujours les inéquations
			\begin{equation}
				-x\leq x\sin(\frac{1}{ x })\leq x.
			\end{equation}
			Or, nous savons que $-x$ et $x$ tendent tous les deux vers zéro lorsque $x$ tend vers zéro. Nous concluons que ce qui se trouve au milieu tend également vers zéro lorsque $x\to 0$.

		\item
			Ce genre d'exercices avec un $x$ dans l'exposant d'une fraction doit faire penser à la limite qui définit le nombre $e$. Commençons par inverser la fraction en mettant un signe dans l'exposant, et séparons en deux facteurs~:
			\begin{equation}
				\left( \frac{ x }{ x+1 } \right)^{x+2}=\left( \frac{ x+1 }{ x } \right)^{-x-2}=\left( \frac{ x+1 }{ x } \right)^{-x} \left( \frac{ x+1 }{ x } \right)^{-2}
			\end{equation}
			Le second facteur tend vers $1$ lorsque $x$ tend vers l'infini. Pour le second,
			\begin{equation}
				\left( \frac{ x+1 }{ x } \right)^{-x}=\left( 1+\frac{1}{ x } \right)^{-x}=\left( \left( 1+\frac{1}{ x } \right)^x \right)^{-1}.
			\end{equation}
			La limite du tout fait donc $e^{-1}$.

		\item
			La subtilité de cet exercice est de faire apparaître la limite $\lim_{x\to 0} \sin(x)/x=1$ que l'on connaît. Pour ce faire, on commence par multiplier et diviser par $x$:
			\begin{equation}
				\frac{ \sin(\alpha x) }{ \sin(\beta x) }=\frac{ x }{ x } \frac{ \sin(\alpha x) }{ \sin(\beta x) }
			\end{equation}
			À ce niveau, nous avons les combinaisons $\sin(\alpha x)/x$ et $x/\sin(\beta x)$ qui apparaissent. Ce n'est pas encore tout à fait ce que l'on veut. Nous multiplions et divisons par $\alpha\beta$ pour mettre tout comme il faut :
			\begin{equation}
				\frac{ \sin(\alpha x) }{ \sin(\beta x) }=\frac{ x }{ x } \frac{ \alpha\beta }{ \alpha\beta }\frac{ \sin(\alpha x) }{ \sin(\beta x) }
			\end{equation}
			La limite de cette expression se décompose en trois morceaux. D'abord la constante $\alpha/\beta$, et ensuite les deux limites
			\begin{subequations}
				\begin{align}
					\lim_{x\to 0} \frac{ \beta x }{ \sin(\beta x) }\to 1\\
					\lim_{x\to 0} \frac{\sin(\alpha x)}{ \alpha x }\to 1.
				\end{align}
			\end{subequations}
			La réponse est donc que
			\begin{equation}
				\lim_{x\to 0} \frac{ \sin(\alpha x) }{ \sin(\beta x) }=\frac{ \alpha }{ \beta }.
			\end{equation}

		\item
            Pour cet exercice et les autres avec les exponentielles, voir la sections \ref{SecCalcLimFHtQNu}.
            
		\item
			Pour cet exercice, on peut mettre le plus haut degré de $x$ en évidence en sortant le $x^2$ de la racine. Ce dernier sort sous la forme de la valeur absolue : $\sqrt{x^2}=| x |$. De plus, lors de la simplification par $x$, la valeur absolue se transforme en $\pm x$ suivant que $x$ est positif ou négatif. Nous avons
			\begin{equation}
				\frac{ \sqrt{x^2+1}-x }{ x-2 }=\frac{ | x |\sqrt{1+\frac{1}{ x }}-x }{ x(1-\frac{ 2 }{ x }) }=\frac{ \pm\sqrt{1+\frac{1}{ x^2 }}-1 }{ 1-\frac{ 2 }{ x } }.
			\end{equation}
			Lorsqu'on fait la limite $x\to \pm\infty$, le dénominateur tend de toutes façons vers un. Le numérateur tend, lui, vers zéro si le signe est positif et vers $-2$ si le signe est négatif. Donc
			\begin{subequations}
				\begin{align}
					\lim_{x\to \infty} &=0\\
					\lim_{x\to -\infty} &=-2
				\end{align}
			\end{subequations}

		\item
			Il s'agit d'utiliser l'astuce du binôme conjugué tant pour le numérateur que pour le dénominateur. Il faut donc multiplier et diviser par $\sqrt{x+2j}+2$ et par $\sqrt{x+7}+3$ en même temps. En utilisant les produits remarquables, et en simplifiant par $(x-2)$, on trouve
			\begin{equation}
				\frac{ \sqrt{x+7}+3 }{ \sqrt{x+2}+2 },
			\end{equation}
			dont la limite vaut 
			\begin{equation}
				\lim_{x\to 2} =\frac{ 3 }{ 2 }.
			\end{equation}

		\item
			Si on développe le $(x+h)^3$, nous trouvons au numérateur
			\begin{equation}
				x^3+3hx^2+3xh^2+h^3-x^3.
			\end{equation}
			Les $x^3$ se simplifient et il ne reste plus que des termes qui contiennent $h$. En écrivant la fraction complète, les $h$ se simplifient avec le dénominateur et nous tombons sur
			\begin{equation}
				\lim_{h\to 0} 2x^2+x^2=3x^2
			\end{equation}
			qui ne dépend plus de $h$.
			
	\end{enumerate}

\end{corrige}
