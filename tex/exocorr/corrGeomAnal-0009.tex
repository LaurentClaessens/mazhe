\begin{corrige}{GeomAnal-0009}

	Nous commençons par prouver que $\overline{ A\times B }\subset\bar A\times \bar B$. Un élément $(a,b)$ dans $\overline{ A\times B }$ peut toujours être vu comme la limite d'une suite dans $A\times B$. Soit donc $(a_n,b_n)$ une suite telle que $\lim(a_n,b_n)=(a,b)$ avec $a_n\in A$ et $b_n\in B$. Par le lemme \ref{LemCvVxWcvVW}, nous avons une convergence «composante par composante» : $a_n\to a$ et $b_n\to b$. Mais la suite $(a_n)$ est contenue dans $A$, donc sa limite, $a$, est dans $\bar A$. De la même manière, $b\in\bar B$. Par conséquent $(a,b)\in \bar A\times\bar B$.

	Il faut maintenant prouver l'inclusion inverse. Soit $(a,b)\in\bar A\times \bar B$. Nous avons des suites $(a_n)$ et $(b_n)$ dans $A$ et $B$ respectivement qui convergent vers $a$ et $b$. En utilisant à nouveau le lemme \ref{LemCvVxWcvVW} (mais cette fois dans le sens inverse), nous avons
	\begin{equation}
		\lim\underbrace{(a_n,b_n)}_{\in A\times B}=(a,b)\in\overline{ A\times B }.
	\end{equation}
	La dernière appartenance est simplement le fait qu'une suite convergente contenue dans $A\times B$ converge dans $\overline{ A\times B }$.

\end{corrige}
