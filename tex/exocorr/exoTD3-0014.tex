% This is part of Exercices de mathématique pour SVT
% Copyright (c) 2010
%   Laurent Claessens et Carlotta Donadello
% See the file fdl-1.3.txt for copying conditions.

\begin{exercice}\label{exoTD3-0014}

	Modèle avec prédation et prélèvement.

	Soit $(u_n)_{n\in\eN}$ définie par
	\begin{equation}
		\begin{cases}
			u_{n+1}=a\frac{ u_n^2 }{ u_n^2+b^2 }-Eu_n	&	\forall n\in\eN\\
			u_0=x,
		\end{cases}
	\end{equation}
	où $a>0$, $b$ et $E>0$ sont des réels.
	\begin{enumerate}
		\item
			Chercher les limites possibles d'une telle suite.
		\item
			Si $E=0$ et $a^2>4b^2$, en utilisant la question précédente et le fait que la fonction $t\mapsto\frac{ at^2 }{ t^2+b^2 }$ est croissante, montrer que pour $0<x<\frac{ a-\sqrt{a^2-4b^2} }{ 2 }$, alors
			\begin{equation}
				0<a\frac{ u_n }{ u_n^2+b^2 }<1
			\end{equation}
			pour tout $n\in\eN$.
		\item
			Si $E=0$ et $a^2>4b^2$, montrer que si $0<x<\frac{ a-\sqrt{a^2-4b^2} }{2}$ alors la suite $(u_n)$ converge vers 0.
		\item
			Si $a>2$ et $b=1$, montrer que si $E>(a-2)/2$, alors la suite est décroissante et si $x>0$, elle converge vers 0.
		\item
			Quelle est la morale de cette dernière question ?
	\end{enumerate}

\corrref{TD3-0014}
\end{exercice}
