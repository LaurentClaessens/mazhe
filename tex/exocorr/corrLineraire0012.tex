% This is part of the Exercices et corrigés de mathématique générale.
% Copyright (C) 2009
%   Laurent Claessens
% See the file fdl-1.3.txt for copying conditions.
\begin{corrige}{Lineraire0012}

	Si nous posons 
	\begin{equation}
		X=\begin{pmatrix}
			 a	&	b	&	c	&	d	\\ 
			  e	&	f	&	g	&	h	\\
			   i	&	j	&	k	&	l	\\ 
			    m	&	n	&	o	&	p		 
			     \end{pmatrix},
	\end{equation}
	nous trouvons
	\begin{equation}
		\begin{aligned}[]
			AX&=\begin{pmatrix}
				 e	&	f	&	g	&	h	\\
				 i	&	j	&	k	&	l	\\
				 m	&	n	&	o	&	p	\\ 
				  0	&	0	&	0	&	0	 
				   \end{pmatrix},\\
			XA&=\begin{pmatrix}
				 0	&	a	&	b	&	c	\\
				 0	&	e	&	f	&	g	\\
				 0	&	i	&	j	&	k	\\ 
				 0	&	m	&	n	&	o	 
				  \end{pmatrix}.
		\end{aligned}
	\end{equation}
	En égalisant les deux, nous trouvons
	\begin{equation}
		j=e=i=m=n=o=0,
	\end{equation}
	ainsi que $f=a$, $b=g$, $h=c$, $j=e$, $k=f$, $l=g$, $n=i$, $o=j$, $p=k=f$. Étant donné que $e=i=0$, nous avons directement $j=n=0$. Remarquez que certaines variables arrivent dans plusieurs équations. Après avoir bien posé toutes les égalités, le résultat est
	\begin{equation}
		\begin{pmatrix}
			 a	&	b	&	c	&	d	\\
			 0	&	a	&	b	&	c	\\
			 0	&	0	&	a	&	b	\\ 
			 0	&	0	&	0	&	a	 
			  \end{pmatrix}.
	\end{equation}

\end{corrige}
