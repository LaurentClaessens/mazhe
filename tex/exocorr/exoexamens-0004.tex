% This is part of Mes notes de mathématique
% Copyright (c) 2012
%   Laurent Claessens
% See the file fdl-1.3.txt for copying conditions.


\begin{exercice}\label{exoexamens-0004}

    On considère le champ de vecteurs défini sur \( \eR^2\setminus\{ (0,0) \}\)
    \begin{equation}
        F(x,y)=\begin{pmatrix}
            \frac{ y }{ x^2+y^2 }    \\ 
            -\frac{ x }{ x^2+y^2 }    
        \end{pmatrix}.
    \end{equation}
    \begin{enumerate}
        \item       \label{ItemhSHjVc}
            Calculer le rotationnel de \( F\).
        \item
            On considère le chemin
            \begin{equation}
                \sigma\colon t\in\mathopen[ 0 , 2\pi \mathclose]\to \sigma(t)=\begin{pmatrix}
                    \cos t    \\ 
                    \sin t    
                \end{pmatrix}.
            \end{equation}
            Calculer la circulation de \( F\) le long de \( \sigma\).
        \item       \label{ItemFVMsyC}
            En déduire que \( F\) ne dérive pas d'un potentiel scalaire sur \( \eR^2\setminus\{ (0,0) \}\).
        \item
            Quelle conclusion peut-on tirer des questions \ref{ItemhSHjVc} et \ref{ItemFVMsyC} ?
    \end{enumerate}
    

\corrref{examens-0004}
\end{exercice}
