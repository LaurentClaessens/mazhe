% This is part of Analyse Starter CTU
% Copyright (c) 2014
%   Laurent Claessens,Carlotta Donadello
% See the file fdl-1.3.txt for copying conditions.

\begin{corrige}{TD5-00002devoir}
  \begin{itemize}
  \item $I_{1}=\displaystyle \int_{1}^{\pi} \frac{4x^3}{x^4+5}\, dx = \left[\ln(x^4+5)\right]_1^\pi = \ln\left(\frac{\pi^2+5}{6}\right)$.
  \item On utilise le changement de variable $u = \sin(x)$, qui comporte $du = \cos(x)\,dx$,  $u(-\pi/2) = -1$ et  $u(0) = 0$. Nous avons alors que  $I_{4}=\displaystyle \int_{-1}^{0} u^2\, du = \left[\frac{u^3}{3}\right]_{-1}^0= \frac{1}{3}$.
  \item On utilise le changement de variable $x = a\sin(u)$, qui comporte $dx = a\cos(u)\,du$ et, en utilisant la fonction $u(x) = \arcsin(x/a)$, $u(a) = \pi/2$  et  $u(0) = 0$. On obtient
\[
 I_{7}=\int_{0}^{\pi/2}\sqrt{a^2-a^2\sin^2(u)} a\cos(u)\, du = a^2\int_{0}^{\pi/2} |\cos(u)|\cos(u)\, du.
\]
Cette dernière intégrale est égale à $\displaystyle a^2\int_{0}^{\pi/2} \cos^2(u)\, du$, car toutes les valeurs prises par la fonction cosinus lorsque $x$ varie entre $0$ et $\pi/2$ sont positives. En intégrant par parties on trouve que 
 \[
\int_{0}^{\pi/2} \cos^2(u)\, du =\int_{0}^{\pi/2} \sin^2(u)\, du,
\]
d'où on peut écrire 
 \[
\int_{0}^{\pi/2} \cos^2(u)\, du =\int_{0}^{\pi/2} \frac{\sin^2(u) +\cos^2(u)}{2}\, du = \frac{\pi}{4}.
\]
La velauer de l'intégrale $ I_{7}$ est $a^2\frac{\pi}{4}$.
    \item $\displaystyle I_{11}=\int_{-3}^0\frac{1}{ x^2-3x+2 }dx$. Il est facile de vérifier que $\frac{1}{ x^2-3x+2 } =- \frac{ 1}{ x-1 }+\frac{ 1}{ x-2 }$. En écrivant la fonction à intégrer comme la somme de deux termes nous avons alors 
\[
I_{11}=-\int_{-3}^0\frac{ 1}{ x-1 }\,dx+\int_{-3}^0\frac{ 1}{ x-2 }\,dx = \left[-\ln(|x-1|)+\ln(|x-2|)\right]_{-3}^0 = \ln\left(\frac{8}{5}\right).
\]
    \item Par parties : $\displaystyle I_{12}=\int_1^2\ln^2(x)dx =\left[x\ln^2(x)\right]_{1}^2 - \int_1^22\ln(x)dx = \left[x\ln^2(x)-2x\ln(x) + 2x\right]_{1}^2 = 2\left(\ln^2(2)-2\ln(2) + 1\right) $.

  \end{itemize}
\end{corrige}
