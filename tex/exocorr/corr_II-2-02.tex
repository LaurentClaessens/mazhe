% This is part of the Exercices et corrigés de CdI-2.
% Copyright (C) 2008, 2009
%   Laurent Claessens
% See the file fdl-1.3.txt for copying conditions.


\begin{corrige}{_II-2-02}

La matrice du système homogène est $\begin{pmatrix}
	0	&	a	\\ 
	-a	&	0	
\end{pmatrix}$, et les valeurs propres sont $\lambda=\pm ia$. Les vecteurs propres correspondants sont 
\begin{equation}
	\begin{aligned}[]
		v_1&=\begin{pmatrix}
	1	\\ 
	i	
\end{pmatrix},&v_2&=\begin{pmatrix}
	1	\\ 
	-i	
\end{pmatrix}.
	\end{aligned}
\end{equation}
Le système homogène a donc comme solutions
\begin{equation}
	x_H=C_1\begin{pmatrix}
	1	\\ 
	i	
\end{pmatrix} e^{iat}+C_2\begin{pmatrix}
	1	\\ 
	-i	
\end{pmatrix} e^{-iat}.
\end{equation}
Les solutions réelles sont 
\begin{equation}
	x_H=\begin{pmatrix}
	A	\\ 
	B	
\end{pmatrix}\cos(at)+\begin{pmatrix}
	B	\\ 
	-A	
\end{pmatrix}\sin(at).
\end{equation}
Le principe de la variation des constantes consiste à poser $A=A(t)$ et $B=B(t)$. La dérivation de $x$ ne pose pas trop de problèmes :
\begin{equation}
	x'=\begin{pmatrix}
	A'+aB	\\ 
	B'-aA	
\end{pmatrix}\cos(at)+\begin{pmatrix}
	B'-aA	\\ 
	-A'-aB	
\end{pmatrix}\sin(at),
\end{equation}
que l'on remet dans le système de départ. Ce que nous obtenons est
\begin{subequations}
\begin{numcases}{}
A'\cos(at)+B'\sin(at)=-2a\\
B'\cos(at)-A'\sin(at)=a.
\end{numcases}
\end{subequations}
Afin de trouver les fonctions inconnues $A$ et $B$, nous commençons par résoudre ce système algébriquement par rapport à $A'$ et $B'$. Nous obtenons
\begin{subequations}
\begin{numcases}{}
	A'(t)=-a\sin(at)-2a\cos(at)\\
	B'(t)=-2a\sin(at)+a\cos(at).
\end{numcases}
\end{subequations}
Ces deux équations sont maintenant solubles par simple quadrature.
\begin{subequations}
\begin{numcases}{}
A(t)=\cos(at)-2\sin(at)+C_1\\
B(t)=2\cos(at)+\sin(at)+C_2.
\end{numcases}
\end{subequations}
La solution générale du système est donc
\begin{equation}
	\begin{aligned}[]
	x&=\begin{pmatrix}
	\cos(at)-2\sin(at)+C_1	\\ 
	2\cos(at)+\sin(at)+C_2	
\end{pmatrix}\cos(at)+
\begin{pmatrix}
	2\cos(at)+\sin(at)+C_2	\\ 
	-\cos(at)+2\sin(at)-C_1\\	
\end{pmatrix}\\&=
\begin{pmatrix}
	1+C_2\sin(at)+C_1\cos(at)	\\ 
	2+C_2\cos'(at)-C_1\sin(at)
\end{pmatrix}.
	\end{aligned}
\end{equation}

Ceci est pour la variation des constantes. Il y a cependant une façon plus facile de résoudre le système. 

D'abord, le système se récrit
\begin{subequations}
\begin{numcases}{}
x'=a(y-2)\\
y'=-a(x-1),
\end{numcases}
\end{subequations}
ce qui nous incite à faire le changement de variable $u=x-1$ et $v=y-2$. Le système devient
\begin{subequations}
\begin{numcases}{}
u'=av\\
v'=-au.
\end{numcases}
\end{subequations}
En dérivant la première équation, $u''=-a^2u$, ce qui donne tout ce suite
\begin{equation}
	u=C_1\cos(at)+C_2\sin(at).
\end{equation}
Ensuite $v$ se trouve en faisant $v=u'/a$.

\end{corrige}
