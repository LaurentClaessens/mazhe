% This is part of the Exercices et corrigés de mathématique générale.
% Copyright (C) 2010
%   Laurent Claessens
% See the file fdl-1.3.txt for copying conditions.

\begin{corrige}{DerrivePartielle-0000}

	\begin{enumerate}

		\item
			Les dérivées partielles sont faciles:
			\begin{equation}
				\begin{aligned}[]
					\frac{ \partial f }{ \partial x }&=2\\
					\frac{ \partial f }{ \partial y }&=1.
				\end{aligned}
			\end{equation}
			Pour dériver dans une direction, il faut considérer le vecteur de norme $1$ dans la direction donnée. Ici $\| v \|=\sqrt{4+9}=\sqrt{13}$, et la dérivée demandée vaut
			\begin{equation}
				\frac{ \partial f }{ \partial v }=\frac{ 2 }{ \sqrt{13} }\frac{ \partial f }{ \partial x }+\frac{ -3 }{ \sqrt{13} }\frac{ \partial f }{ \partial y }=\frac{1}{ \sqrt{13} }.
			\end{equation}
			
		\item
		\item
		\item
			Les dérivées partielles sont
			\begin{equation}
				\begin{aligned}[]
					\frac{ \partial f }{ \partial x }&=\ln(y)\\
					\frac{ \partial f }{ \partial y }&=\frac{ x }{ y }.
				\end{aligned}
			\end{equation}
			En y substituant le point $x=2$, $y=2$, nous avons $\partial_xf(P)=\ln(2)$ et $\partial_yf(P)=1$.

			Le vecteur normalisé dans la direction $v$ est $(1/\sqrt{2},-1/\sqrt{2})$. Donc nous avons
			\begin{equation}
				\frac{ \partial f }{ \partial v }(P)=\frac{1}{ \sqrt{2} }\ln(2)-\frac{1}{ \sqrt{2} }.
			\end{equation}
		\item
			Les dérivées partielles sont
			\begin{equation}
				\begin{aligned}[]
					\frac{ \partial f }{ \partial x }&=yz\cos(xyz)&\frac{ \partial f }{ \partial x }(P)=-1\\
					\frac{ \partial f }{ \partial y }&=xz\cos(xyz)&\frac{ \partial f }{ \partial y }(P)=-\pi\\
					\frac{ \partial f }{ \partial z }&=xy\cos(xyz)&\frac{ \partial f }{ \partial z }(P)=-\pi
				\end{aligned}
			\end{equation}
			Le vecteur normé dans la direction donnée est $v=(2/3,1/3,2/3)$, et la dérivée directionnelle est
			\begin{equation}
				\frac{ \partial f }{ \partial v }(P)=-\frac{ 2 }{ 3 }-\pi.
			\end{equation}
			

	\end{enumerate}
	

\end{corrige}
