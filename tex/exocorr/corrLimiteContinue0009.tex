\begin{corrige}{LimiteContinue0009}

	Pour rappel, une fonction est continue en $(a,b)$ si $\lim_{(x,y)\to(a,b)}f(x,y)=f(a,b)$. Donc si, comme c'est souvent le cas dans cet exercice,  $f(0,0)=0$, nous devons vérifier si la limite de $f(x,y)$ lorsque $(x,y)$ tend vers $(0,0)$ est égale à zéro ou non. Si oui, la fonction est continue, si non, la fonction n'est pas continue.

  \begin{enumerate}
  \item Cette fonction est continue en l'origine. Pour le voir nous passons aux coordonnées polaires
    \begin{equation}
      \begin{aligned}
        \lim_{(x,y)\to(0,0)} \frac{x^4y}{x^4+y^6}=&\lim_{r\to 0}\frac{r^5 \cos^4\theta\sin\theta}{r^4(\cos^4\theta+r^2\sin^6\theta}=\\
        &=\lim_{r\to 0}\frac{r^5 \cos^4\theta\sin\theta}{r^4(\cos^4\theta+r^2\sin^6\theta}=\\
        &=\lim_{r\to 0}r \underbrace{\frac{\sin\theta}{1 +r^2\tan^4\theta\sin^2\theta}}_{A}.
      \end{aligned}
     \end{equation}
      Pour faire le dernier passage nous avons supposé $\cos\theta\neq 0$, cela ne pose pas de problèmes parce que nous pouvons considérer l'axe vertical séparément et il est facile de voir que $\lim_{y\to 0}f(0,y)=0$.  La fonction $A$ est borné dans l'intervalle $[-1,1]$, donc la limite de $f$ existe et vaut zéro. 
      \item 

	      Cette fonction n'est pas continue en l'origine. Il suffit de considérer la limite le long de la courbe $(t^{3/2},t)$
        \begin{equation}
          \lim_{t\to 0} \frac{t^{4+3/2}}{2t^6}
        \end{equation}
        n'est clairement pas zéro.
        \item
		
	      Cette fonction doit être testée en tous les points de la forme $(a,b)=(a,0)$ parce que c'est en ces points que l'on a un «problème», à savoir un zéro dans un dénominateur. 
		
		La fonction $f$ est le produit de $y^2$ et d'une fonction bornée. On a alors 
          \begin{equation}
            \begin{aligned}
              0\leq |y^2|\left|\sin\frac{x}{y}\right|\leq |y|^2.
            \end{aligned}  
          \end{equation}
          Quand $y$ tend vers zéro, $|f|$ est coincée entre deux fonctions qui tendent vers zéro, elle; ne peut pas s'échapper et donc doit tendre vers zéro. Ce théorème est dit <<des deux gendarmes>> (<<dei due carabinieri>> en italien) ou <<de l'étau>>.
          \item Passons aux coordonnées polaires : $f(r\cos\theta, r\sin\theta )= r\cos\theta e^\theta$. La fonction $f$ est le produit de $x$ et d'une fonction qui prends ses valeurs entre $-e^{2\pi}$ et $e^{2\pi}$ et est donc continue partout. 
          
  \end{enumerate}

\end{corrige}
