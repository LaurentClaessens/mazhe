% This is part of Exercices de mathématique pour SVT
% Copyright (c) 2011
%   Laurent Claessens et Carlotta Donadello
% See the file fdl-1.3.txt for copying conditions.

\begin{exercice}\label{exoExamenDecembre2010-0005}

On considère l'équation différentielle linéaire 
\begin{equation}\label{nonhomog}
  y'=-xy+x.
\end{equation}
\begin{enumerate}
\item On commence par considérer l'équation homogène associée
  \begin{equation}\label{homog}
    y'=-xy.
  \end{equation}
  \begin{enumerate}
  \item Dire si l'équation \eqref{homog} a une solution constante. Si oui, la trouver. 
  \item Trouver la solution générale de l'équation \eqref{homog}.
  \end{enumerate}
\item  Trouver  une solution particulière $P(x)$ de \eqref{nonhomog} de la forme $P(x)=ax+b$, avec $a$ et $b$ dans $\eR$. 
\item Vérifiez que la fonction $1-3e^{-x^2/2}$ est une solution du problème suivant
  \begin{equation}
    \left\{\begin{array}{l}
      y'=-xy+x,\\
      y\big(\sqrt{\ln(3)}\big)= 1-\sqrt{3}.
    \end{array}\right.
  \end{equation}
%  \item[\textbf{Bonus :}] Trouver la solution générale de \eqref{nonhomog}.
\end{enumerate}


\corrref{ExamenDecembre2010-0005}
\end{exercice}
