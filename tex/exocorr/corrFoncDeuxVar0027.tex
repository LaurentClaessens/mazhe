% This is part of the Exercices et corrigés de mathématique générale.
% Copyright (C) 2010
%   Laurent Claessens
% See the file fdl-1.3.txt for copying conditions.

\begin{corrige}{FoncDeuxVar0027}

	\begin{enumerate}

		\item
			Calculer les dérivées partielles au point $(x,y)=(-2,-1)$ :
			\begin{equation}
				\begin{aligned}[]
					\frac{ \partial f }{ \partial x }&=8x=-16\\
					\frac{ \partial f }{ \partial y }&=18y=-18.
				\end{aligned}
			\end{equation}
			Le plan tangent est donné par
			\begin{equation}
				\begin{aligned}[]
					T_{(-2,-1)}(x,y)&=25+-16(x+2)-18(y+1)\\
						&=-16x-18y-25.
				\end{aligned}
			\end{equation}
			En guise de petite vérification, noter que $T_{(-2,-1)}(-2,-1)=25=f(-2,-1)$.

			Des vecteurs directeurs du plan peuvent être trouvés en prenant deux vecteurs dans le plan translaté à l'origine :
			\begin{equation}
				U(x,y)=-16x-18y.
			\end{equation}
			Deux points dans le plan $U$ sont par exemple
			\begin{equation}
				\begin{aligned}[]
					P_1&=(1,0,-16)\\
					P_2&=(0,1,-18).
				\end{aligned}
			\end{equation}
			Pour trouver un vecteur perpendiculaire au plan $U$ nous prenons le produit vectoriel de $P_1$ et $P_2$ :
			\begin{equation}
				v=P_1\times P_2=\begin{vmatrix}
					e_x	&	e_y	&	e_z	\\
					1	&	0	&	-16	\\
					0	&	1	&	-18
				\end{vmatrix}=\begin{pmatrix}
					16	\\ 
					18	\\ 
					1	
				\end{pmatrix}.
			\end{equation}
			Toute droite normale au plan est parallèle à ce vecteur.

			Afin de trouver la droite normale au plan $T$ passant par le point $(-2,-1,25)$ nous faisons une translation :
			\begin{equation}
				r(\lambda)=\begin{pmatrix}
					-2	\\ 
					-1	\\ 
					25	
				\end{pmatrix}+\lambda\begin{pmatrix}
					16	\\ 
					18	\\ 
					1	
				\end{pmatrix}=
				\begin{pmatrix}
					16\lambda-2	\\ 
					18\lambda-1	\\ 
					\lambda+25	
				\end{pmatrix}.
			\end{equation}
			
		\item
		\item
			Nous commençons par calculer les dérivées partielles qui serviront de «coefficients angulaires» pour le plan tangent. En l'occurrence les dérivées partielles sont
			\begin{equation}
				\begin{aligned}[]
					\frac{ \partial f }{ \partial x }&=\frac{1}{ x+y }\\
					\frac{ \partial f }{ \partial y }&=\frac{ -x }{ xy+y^2 }.
				\end{aligned}
			\end{equation}
			Nous évaluons ces dérivées partielles en $(x,y)=(0,2)$ :
			\begin{equation}
				\begin{aligned}[]
					(\partial_xf)(0,2)&=\frac{ 1 }{2}\\
					(\partial_yf)(0,1)&=0.
				\end{aligned}
			\end{equation}
			L'équation du plan tangent est alors
			\begin{equation}
				T_P(x,y)=f(P)+x\cdot\frac{ \partial f }{ \partial x }(P)+(y-2)\frac{ \partial f }{ \partial y }(P).
			\end{equation}
			Le second terme est nul et il reste 
			\begin{equation}
				T_{(0,2)}(x,y)=\ln\left( \frac{ 1 }{2} \right)+\frac{ x }{ 2 }.
			\end{equation}
			
			Pour trouver la normale, nous allons trouver deux vecteurs dans le plan et en prendre le produit vectoriel. Attention cependant : il faut réellement prendre des vecteurs dans le plan, ce qui est délicat pour ce plan parce qu'il ne passe pas par l'origine. Pour éviter tout problèmes, nous commençons par prendre le plan
			\begin{equation}
				U(x,y)=\frac{ x }{ 2 }.
			\end{equation}
			Deux vecteurs de ce plan sont par exemple
			\begin{equation}
				\begin{aligned}[]
					v&=(1,0,\frac{ 1 }{2})\\
					w&=(0,1,0).
				\end{aligned}
			\end{equation}
			Le vecteur perpendiculaire est alors donné par le produit vectoriel
			\begin{equation}
				u\times w=
				\begin{vmatrix}
					e_x	&	e_y	&	e_z	\\
					1	&	0	&	1/2	\\
					0	&	1	&	0
				\end{vmatrix}=
				\begin{pmatrix}
					-1/2	\\ 
					0	\\ 
					1	
				\end{pmatrix}.
			\end{equation}
			La droite que nous recherchons est celle qui est parallèle à ce vecteur et qui passe par le point $P=(0,2,1)$. L'équation paramétrique est donnée par
			\begin{equation}
				r(\lambda)=
				\begin{pmatrix}
					0	\\ 
					2	\\ 
					1	
				\end{pmatrix}+\lambda
				\begin{pmatrix}
					-1/2	\\ 
					0	\\ 
					1	
				\end{pmatrix}=
				\begin{pmatrix}
					-\lambda/2	\\ 
					2	\\ 
					1+\lambda	
				\end{pmatrix}.
			\end{equation}

	\end{enumerate}

\end{corrige}
