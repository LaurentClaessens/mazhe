% This is part of the Exercices et corrigés de CdI-2.
% Copyright (C) 2008, 2009
%   Laurent Claessens
% See the file fdl-1.3.txt for copying conditions.


\begin{corrige}{_I-1-16}

Cet exercice est énoncé (de façon un peu différente) et corrigé sur le site \href{http://www.bibmath.net}{bibmath.net}\footnote{L'énoncé : \href{http://www.bibmath.net/exercices/bde/analyse/distrib/testeno.pdf}{http://www.bibmath.net/exercices/bde/analyse/distrib/testeno.pdf},\\ et la correction : \href{http://www.bibmath.net/exercices/bde/analyse/distrib/testcor.pdf}{http://www.bibmath.net/exercices/bde/analyse/distrib/testcor.pdf}}. La présente correction s'en inspire largement.

Nous posons
\begin{equation}
	u_n(x)=\frac{ C_n }{ n! }x^nf(\frac{ x }{ t_n }).
\end{equation}
Prouvons que la série des $u_n$ converge normalement (ce qui est plus fort que uniformément). Pour cela, nous majorons $| u_n(x) |$ pour tout $x$. Si $| x/t_n |\geq 1$, il n'y a pas de problèmes : $u_n(x)=0$. Sinon, nous avons
\begin{equation}		\label{EqMajorationBorel16}
	| u_n(x) |\leq\frac{ | C_n | }{ n! }\| f \|_{\infty}| t_n |^n
\end{equation}
où la norme supremum $\| f \|_{\infty}$ existe parce que $f$ est à support compact. Nous choisissons les nombres $t_n$ de telle manière à avoir en même temps $t_n\leq 1$ et $| C_n |t_n\leq 1$. Pour cela, nous posons
\begin{equation}
	t_n=\begin{cases}
	1	&	\text{si }|C_n|\leq 1\\
	1/|C_n|	&	 \text{si }| C_n |\geq 1
\end{cases}
\end{equation}
Nous avons donc
\begin{equation}
	| u_n(x) |\leq \| f \|_{\infty}\frac{1}{ n! }
\end{equation}
qui est le terme général d'une série convergente. La série $u=\sum_n u_n$ étant normalement convergente, elle est uniformément convergente. Afin de prouver que nous pouvons la dériver terme à termes, nous prouvons que pour tout $m$, la série $\sum_n u_n^{(m)}$ converge normalement. D'abord, remarquons que $\| f^{(m)} \|_{\infty}$ existe parce que $f^{(m)}$ est à support compact.

En utilisant la règle de Leibnitz, nous trouvons la série des dérivées $m$ièmes :
\begin{equation}
	u_n^{(m)}(x)=\sum_{l=0}^m {l\choose m} \frac{ x^{n-l} }{ (n-l)! }\frac{ C_n }{ t_n^{m-l} }f^{(m-l)}(\frac{ x }{ t_n }).
\end{equation}
Afin de majorer cela de la même façon que nous avons trouvé la majoration \eqref{EqMajorationBorel16}, nous écrivons
\begin{equation}		\label{EqPreMajBorel}
	\frac{ C_nx^{n-l} }{ t_n^{m-l} }=\frac{ C_n }{ t_n }\frac{ x^{n-l} }{ t_n^{n-l} }t_n^{n-m-1}
\end{equation}
dont chacun des trois facteurs est plus petit ou égal à $1$ lorsque $n>m$. Comprenons bien que nous nous fixons un degré de dérivation $m$, et puis que nous voulons sommer sur $n$ la série $\sum_n u^{(k)}(x)$. Les $m$ premier termes valent quelque chose qui ne nous intéresse pas (dans les séries, nous pouvons toujours modifier les premiers termes sans changer la convergence), tandis que les termes suivants se majorent grâce au fait que \eqref{EqPreMajBorel} est plus petit ou égal à $1$ pour ces termes :
\begin{equation}
	| u_n^{(m)}(x) |\leq \sum_{l=0}^m{l\choose m} \| f^{(k-l)} \|_{\infty}\frac{1}{ (n-l)! }.
\end{equation}
Il suffit de poser $M_1=\max\{ \| f^{(m-l)} \|_{\infty} \}_{l=0,\ldots,m}$ et $M_2=\max\{ {l\choose m} \}_{l=0,\ldots,m}$ pour trouver la borne
\begin{equation}
	| u_n^{(m)}(x) |\leq kM_1M_2\frac{1}{ (n-m)! }, 
\end{equation}
dont la série converge. Ce que nous avons donc prouvé est que
\begin{equation}
	\| u_n^{(m)} \|_{\infty}\leq M_n^{(m)}
\end{equation}
où la somme $\sum_nM_n^{(m)}$ converge. La série $\sum_nu_n^{(m)}$ converge donc normalement et uniformément (voir remarque 1 de la page I.17).

Nous pouvons donc dériver terme à terme. Il reste à déterminer la valeur de la $m$ième dérivée en $x=0$. Fixons un ordre $m$ de dérivation et prouvons que pour tout $\epsilon>0$, nous avons $| u^{(m)}(0)-C_m |\leq \epsilon$. Soit donc $\epsilon>0$ et prenons un entier $N>m$ tel que pour tout $x$,
\begin{equation}
	\left| \sum_{n\geq N}u_n^{(m)}(x) \right| \leq \epsilon.
\end{equation}
Maintenant, nous considérons un voisinage de zéro dans lequel
\begin{equation}
	u_n^{(m)}(x)=C_n\frac{ x^{n-m} }{ (n-m)! }
\end{equation}
pour tout $n<N$. À ce moment, nous avons pour tout $x$ dans ce voisinage
\begin{equation}
	\left| u^{(m)}(x)-\sum_{n<N}\frac{ x^{n-m} }{ (n-m)! }C_n \right| =\left| \sum_{n\geq N}u_n^{(m)}(x) \right| \leq\epsilon.
\end{equation}
En particulier, en $x=0$ nous avons
\begin{equation}
	| u^{(m)}(0)-C_m |\leq\epsilon.
\end{equation}
Nous en déduisons que $u^{(m)}(0)=C_m$.


{\bf Note} Dans \href{http://www.bibmath.net/exercices/bde/analyse/distrib/testcor.pdf}{le corrigé} que j'ai suivit, il est dit que sur un voisinage de zéro, $u_n(x)=\frac{ C_n }{ n! }x^n$. Il me semble que cela n'est pas toujours vrai parce que si $C_n\to\infty$, alors $t_n\to 0$ et tout voisinage de zéro contiendra un $x$ tel que pour un certain $n$, nous ayons $|x/t_n|\in [\frac{ 1 }{2},1]$ sur lequel $f$ ne vaut pas $1$. C'est pour cela que je fais appel à un $N$ et un $\epsilon$ afin de prouver que la différence entre $u_n(x)=\frac{ C_n }{ n! }x^n$ et la réalité n'est pas si grande que ça.

\end{corrige}
