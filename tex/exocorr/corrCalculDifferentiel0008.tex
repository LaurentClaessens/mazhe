\begin{corrige}{CalculDifferentiel0008}

	Notez que dans tous les points de cet exercice, les dérivées données sont des fonctions $C^{\infty}$. Donc les fonctions qu'on cherche sont également $C^{\infty}$; elles vérifient en particulier le théorème \ref{Schwarz}. Une façon de s'assurer qu'un énoncé est impossible est de vérifier que $\partial^2_{xy}f\neq\partial^2_{yx}f$. Si les dérivées partielles croisées ne sont pas égales, alors il est impossible de trouver une fonction qui satisfait les contraintes.

	\begin{enumerate}
		\item
			Le fait de vérifier $\partial_xf(x,y)=-\sin(x)\sin(y)$ implique que
			\begin{equation}
				f(x,y)=\int -\sin(x)\sin(y)dx=\sin(y)\cos(x)+c_1(y).
			\end{equation}
			La subtilité à comprendre est que la constante d'intégration est une fonction qui ne dépend pas de $x$, mais qui peut dépendre de $y$ ! Ce qui nous avons donc obtenu est qu'il existe une fonction $c_1(y)$ telle que
			\begin{equation}	\label{CD81fc1}
				f(x,y)=\sin(y)\cos(x)+c_1(y).
			\end{equation}
			Pour voir ce que nous dit la seconde condition, nous pouvons calculer $\partial_yf$ en utilisant la forme \eqref{CD81fc1} :
			\begin{equation}
				\frac{ \partial f }{ \partial y }(x,y)=\cos(y)\cos(x)+c'_1(y).
			\end{equation}
			Si nous voulons que cela soit toujours égal à $\cos(y)\cos(x)$, il faut $c'_1(y)=0$ et donc que $c_1$ soit une constante. Toutes les fonctions répondant à la question sont donc de la forme
			\begin{equation}
				f(x,y)=\sin(y)\cos(x)+C
			\end{equation}
			pour une constante $C$.

		\item
			La condition sur $\partial_xf$ dit que
			\begin{equation}
				f(x,y)=x^2y+xy^3+c(y)
			\end{equation}
			pour une certaine fonction $c$. La condition sur $\partial_yf$ nous dit alors que $c'(y)=0$ et donc que $c$ est constante.

			En dérivant $2xy+y^3$ par rapport à $y$, nous trouvons
			\begin{equation}
				\frac{ \partial  }{ \partial y }\frac{ \partial f }{ \partial x }=2x+3y^2,
			\end{equation}
			tandis qu'en dérivant $x^2+3xy^2$ par rapport à $x$ nous trouvons
			\begin{equation}
				\frac{ \partial  }{ \partial x }\frac{ \partial f }{ \partial y }=2x+3y^2,
			\end{equation}
			qui est la même chose.

			Nous avons donc, dans ce cas, $\partial^2_{yx}f=\partial^2_{xy}f$ par le thé.

		\item
			La condition $\partial_xf=-y$ impose $f(x,y)=-yx+c_1(y)$, tandis que la condition $\partial_yf(x,y)=x$ impose $f(x,y)=yx+c_2(x)$. Les deux sont incompatibles. Il n'y a donc pas de solutions à cet exercice.

			À noter que $\partial^2_{yx}f(x,y)=-1$ tandis que $\partial^2_{xy}f(x,y)=1$. Les deux dérivées secondes sont dont différentes.
		\item
			Encore une fois les deux conditions sont incompatibles parce que nous trouvons
			\begin{equation}
				\begin{aligned}[]
					f(x,y)&=xy+\sin(x)+c_1(y)\\
					f(x,y)&=x^2y+\frac{ y^3 }{ 3 }+c_2(x).
				\end{aligned}
			\end{equation}
			Il est certes possibles d'arranger les fonctions $c_1(y)$ et $c_2(x)$ pour obtenir les termes $\sin(x)$ et $\frac{ y^3 }{ 3 }$, mais il n'est cependant pas possible de les arranger pour obtenir les termes $xy$ et $x^2y$.

			Ici encore, les dérivées secondes ne sont pas égales : $\partial^2_{yx}f=1$ tandis que $\partial_{xy}f=2x$.


	\end{enumerate}
	

\end{corrige}
