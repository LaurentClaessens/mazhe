% This is part of Exercices de mathématique pour SVT
% Copyright (c) 2010-2011
%   Laurent Claessens et Carlotta Donadello
% See the file fdl-1.3.txt for copying conditions.

\begin{corrige}{TD3-0012}
    Avant de nous lancer dans l'action, nous pouvons déterminer quelles sont les limites possibles. Le «candidats limites» sont les $u$ tels que
    \begin{equation}
        u=2u(1-bu).
    \end{equation}
    La première solution est $u=0$. Ensuite nous pouvons simplifier l'équation par $u$, ce qui donne l'équation $1=2-2bu$, et donc la solution $u=\frac{1}{ 2b }$. Nous retenons que les limites possibles de la suite $(u_n)$ sont zéro et $\frac{1}{ 2b }$.
    \begin{enumerate}
        \item
            Si $u_0=0$, alors la suite reste nulle tout le temps parce que $u_n$ est en facteur dans la définition de $u_{n+1}$. Si par contre $u_0=\frac{1}{ b }$, alors $1-bu_0=0$, et par conséquent $u_1=0$. Dès que $u_1=0$, tous les termes suivants de la suite seront nuls.
        \item
            Nous savons que les limites possibles (si la suite est convergente) sont $0$ et $\frac{1}{ 2b }$. Dans le cas où $u_0$ est strictement entre $0$ et $\frac{1}{ 2b }$, si nous voulons prouver que la limite existe et vaut $\frac{1}{ 2b }$.
            
            La stratégie sera de prouver que la suite est croissante et bornée. En effet, si elle est croissante et borné, elle converge. Mais si le premier terme est plus grand que $0$ et qu'elle est croissante, elle ne peut pas converger vers $0$. Elle devra donc converger vers $\frac{1}{ 2b }$.
        
            \begin{description}
                \item[Bornée] 
                    Montrons que si $u_k<\frac{1}{ 2b }$, alors $u_{k+1}<\frac{1}{ 2b }$. Demander
                    \begin{equation}
                        2u_k(1-bu_k)<\frac{1}{ 2b }
                    \end{equation}
                    revient à demander
                    \begin{equation}
                        -2bu_k^2+2u_k-\frac{1}{ 2b }<0.
                    \end{equation}
                    Étudions donc la fonction $f(x)=-2bx^2+2x-\frac{1}{2b }$. C'est une fonction qui s'annule pour les valeurs suivantes de $x$ :
                    \begin{equation}
                        \frac{ -2\pm\sqrt{4-4(-2b) \left( -\frac{1}{ 2b } \right)  } }{ -4b }=\frac{1}{ 2b }.
                    \end{equation}
                    C'est une parabole (fonction du second degré) qui ne s'annule qu'une seule fois, et qui est donc toujours positive ou bien toujours négative. Dans notre cas, elle est toujours négative parce que $f(0)=-\frac{1}{ 2b }<0$.
                    
                    Cela est une très bonne nouvelle parce que nous avons démontré que \emph{quelle que soit la valeur de $u_k$}, nous avons \emph{toujours} $u_{k+1}<\frac{1}{ 2b }$. Ce qui nous intéresse pour l'instant est que la suite soit bornée par $\frac{1}{ 2b }$.
        
                \item[Bornée, alternative]% Par Pauline Klein

                    Calculons la différence \( u_{n+1}-\frac{1}{ 2b }\) et montrons que cette différence est toujours négative. Nous avons
                    \begin{subequations}
                        \begin{align}
                            u_{n+1}-\frac{1}{ 2b }&=2u_n(1-bu_n)-\frac{1}{ 2b }\\
                            &=-2bu_n^2+2u_n-\frac{1}{ 2b }\\
                            &=-\frac{1}{ 2b }\big( 4b^2u_n^2-4bu_n+1 \big)\\
                            &=-\frac{1}{ 2b }(2bu_n-1)^2,
                        \end{align}
                    \end{subequations}
                    qui est effectivement toujours négatif. Par conséquent \( u_{n+1}-\frac{1}{ 2b }<0\) et \( u_{n+1}<\frac{1}{ 2b }\).

                \item[Croissante]
                    Calculons
                    \begin{equation}
                        \frac{ u_{k+1} }{ u_k }=2(1-bu_k)>2\left( 1-b\frac{1}{ 2b } \right)=1.
                    \end{equation}
                    Pour cette majoration, nous avons utilisé le fait que $1-bu_k>1-b\frac{1}{ 2b }$ parce que si $u_k<\frac{1}{ 2b }$, alors $1-u_k>1-\frac{1}{ 2b }$.
                \item[Conclusion]
                    Lorsque $u_n$ est entre $0$ et $\frac{1}{ 2b }$, nous avons montré qu'elle est croissante et bornée. Elle est donc convergente. Or les seules bornes possibles sont $0$ et $\frac{1}{ 2b }$. La suite ne peut donc converger que vers $\frac{1}{ 2b }$.
            \end{description}


        \item
            La bonne surprise est que si $u_k>\frac{1}{ 2b }$, alors $u_{k+1}<\frac{1}{ 2b }$, parce que nous avons déjà vu que de toutes façons la fonction $-2bu_n^2+2u_n$ ne retourne que des valeurs plus petites que $\frac{1}{ 2b }$. Donc si $u_0>\frac{1}{ 2b }$, il n'en reste pas moins que $u_1<\frac{1}{ 2b }$, et que par conséquent la suite converge vers $\frac{1}{ 2b }$ par le point précédent.
    \end{enumerate}

\end{corrige}
