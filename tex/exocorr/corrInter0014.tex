% This is part of the Exercices et corrigés de mathématique générale.
% Copyright (C) 2009
%   Laurent Claessens
% See the file fdl-1.3.txt for copying conditions.
\begin{corrige}{Inter0014}

\begin{enumerate}

\item


La courbe $4y=x^3$ s'écrit tout aussi bien $y=\frac{ x^3 }{ 3 }$, et la surface de révolution correspondante est
\begin{equation}
	S=2\pi\int_0^1\sqrt{1+\left( \frac{ 3x^2 }{ 4 } \right)^2}\frac{ x^3 }{ 4 }dx
\end{equation}
Le changement de variable miracle qu'il faut voir est $u=1+\frac{ 9x^4 }{ 16 }$, parce que $dx=\frac{ 16 }{ 27 }\frac{1}{ x^3 }du$, ce qui fait que les $x^3$ se simplifient et l'intégrale à calculer devient
\begin{equation}
	2\pi\int\sqrt{u}du=\frac{ 8\pi }{ 27 }\frac{ u^{3/2} }{ 3/2 }=\frac{\pi}{ 3\cdot 27 }\left( 1+\frac{ 9x^4 }{ 16 } \right)^{3/2}.
\end{equation}
Nous avons donc
\begin{equation}
	S=\frac{ 4\pi }{ 27 }\left[ \left( 1+\frac{ 9x^{4} }{ 16 } \right) \right]_0^1=\frac{ 61\pi }{ 432 }.
\end{equation}

\item
La courbe donnée est
\begin{equation}
	x=\frac{ 2\ln(y)-y^2 }{ 4 }.
\end{equation}
Il est important d'écrire $x=x(y)$, et non $y=y(x)$ comme d'habitude, parce que nous allons maintenant faire une rotation le long de l'axe $Oy$ au lieu de $Ox$. Cela fait qu'il faut changer $x\leftrightarrow y$ dans toutes les formules utilisées d'habitude.
\begin{equation}
	\begin{aligned}[]
		I(y)	&=\int\sqrt{1+\left( \frac{ \frac{ 2 }{ y }-2y }{ 4 } \right)^2}\frac{ 2\ln(y)-y^2 }{ 4 }du\\
			&=\int \frac{ y^2+1 }{ 2y }\left(  \frac{ 2\ln(y)-y^2 }{ 4 }\right)dy\\
			&=\int\frac{ y\ln(y) }{ 4 }dy+\int\frac{ \ln(y) }{ 4y }dy-\frac{ y^3 }{ 8 }-\frac{ y }{ 8 }.
	\end{aligned}
\end{equation}
La seconde intégrale est une intégrale remarquable : c'est $\int ff'$ avec $f=\ln$, donc l'intégrale vaut $\frac{ \ln(y)^2 }{2}$. La seule difficulté qui reste est de calculer
\begin{equation}
	J=\int y\ln(y)dy.
\end{equation}
Pour cela, nous faisons par partie :
\begin{equation}
	\begin{aligned}[]
		u&=\ln(y)		&dv&=y\\
		du&=\frac{1}{ y }dy	&v&=\frac{ y^2 }{ 2 }.
	\end{aligned}
\end{equation}
Alors, nous avons
\begin{equation}
	J=\frac{ y^2\ln(y) }{ 2 }-\frac{ 1 }{2}\int ydy=\frac{ y^2\ln(y) }{ 2 }-\frac{ y^2 }{ 4 }.
\end{equation}
En remettant les bouts ensemble,
\begin{equation}
	I(y)=\frac{ y^2\ln(y) }{ 8 }-\frac{ y^2 }{ 8 }+\frac{ \ln(y)^2 }{ 8 }-\frac{ y^4 }{ 32 },
\end{equation}
et la surface demandée vaut
\begin{equation}
	S=2\pi\big( I(3)-I(1) \big)=\frac{ \pi }{ 4 }\big( \ln(3)^2+9\ln(3)-28 \big).
\end{equation}
\end{enumerate}


\end{corrige}
