% This is part of Exercices et corrigés de CdI-1
% Copyright (c) 2011
%   Laurent Claessens
% See the file fdl-1.3.txt for copying conditions.

\begin{corrige}{OutilsMath-0003}

	Le segment $AC$ a une longueur $1$, et le segment $HC$ une longueur $\frac{ 1 }{2}$ parce que nous sommes dans un triangle équilatéral. Si nous nommons $x$ la longueur de la hauteur $AH$, le théorème de Pythagore dit que
	\begin{equation}
		\left( \frac{ 1 }{2} \right)^2+x^2=1^2,
	\end{equation}
	donc $x^2=1-\frac{1}{ 4 }=\frac{ 3 }{ 4 }$, d'où le fait que
	\begin{equation}
		\sin(\unit{60}{\degree})=\frac{ \sqrt{3} }{ 2 }.
	\end{equation}
	Afin de trouver le cosinus de \unit{30}{\degree} nous utilisons la relation
	\begin{equation}
		\sin^2(30)+\cos^2(30)=1,
	\end{equation}
	donc $\cos^2(\unit{30}{\degree})=\frac{1}{ 4 }$, et le résultat en découle.

\end{corrige}
