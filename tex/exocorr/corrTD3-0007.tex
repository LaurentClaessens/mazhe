% This is part of Exercices de mathématique pour SVT
% Copyright (C) 2010
%   Laurent Claessens et Carlotta Donadello
% See the file fdl-1.3.txt for copying conditions.

\begin{corrige}{TD3-0007}

	Ici nous ne pouvons pas utiliser la proposition \ref{Propufulimite} pour trouver des candidats limites parce que la fonction qui lie $u_{n+1}$ à $u_n$ n'est pas continue.
	\begin{enumerate}
		\item
			Nous avons $u_0=3$ et
			\begin{equation}
				f(y)=\begin{cases}
					\frac{ 1 }{2}	&	\text{si }y<2\\
					\frac{1}{ y }	&	 \text{si }y\geq 2.
				\end{cases}
			\end{equation}
			Calculons quelques termes.
			\begin{equation}
				\begin{aligned}[]
					u_0&=3\\
					u_1&=f(u_0)u_0=f(3)\cdot 3=\frac{1}{ 3 }3=1\\
					u_2&=f(u_1)u_1=f(1)=\frac{ 1 }{2}\\
					u_3&=f(u_2)u_2=f(\frac{ 1 }{2})\frac{ 1 }{2}=\frac{1}{ 4 }.
				\end{aligned}
			\end{equation}
			Ce que nous remarquons est que, à partir du moment où la suite passe en dessous de $1$, le nombre est divisé par deux à chaque pas. Cela est à cause du fait que dès que $u_n<1$, alors $u_{n+1}=f(u_n)u_n=\frac{ 1 }{2}u_n$.

			La suite tend donc vers zéro.
		\item
			Cette fois nous avons
			\begin{equation}
				f(y)=\begin{cases}
					2	&	\text{si }u<2\\
					\frac{ 4 }{ y }	&	 \text{si }y\geq 2
				\end{cases}
			\end{equation}
			Calculons quelques termes :
			\begin{equation}
				\begin{aligned}[]
					u_0&=1\\
					u_1&=2u_0=2\\
					u_2&=f(2)u_2=\frac{ 4 }{ 2 }2=4\\
					u_3&=f(4)\cdot 4=1\cdot 4=4.
				\end{aligned}
			\end{equation}
			La suite reste alors bloquée à $4$. En effet si $u_n=4$, alors
			\begin{equation}
				u_{n+1}=f(u_n)u_n=\frac{ 4 }{ u_n }\cdot 4=\frac{ 4 }{ 4 }\cdot 4=4.
			\end{equation}
			La limite est donc $4$.
	\end{enumerate}
	
	Notez que dans cet exercice nous avons eu de la chance : rien qu'en calculant les premiers termes, nous avons pu comprendre comment se comporte la suite et déterminer la limite \ldots\ ce ne sera pas toujours le cas.

\end{corrige}
