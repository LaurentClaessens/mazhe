% This is part of Exercices et corrigés de CdI-1
% Copyright (c) 2010-2011, 2024
%   Laurent Claessens
% See the file fdl-1.3.txt for copying conditions.


\begin{corrige}{reserve0001}

	Le seul problème se trouve éventuellement en $x=0$. En utilisant la règle de L'Hôpital, nous voyons très vite que $f$ y est continue. En ce qui concerne $f'$, nous avons
	\begin{equation}
		f'(x)=\frac{ x\cosh(x)-\sinh(x) }{ x^2 }.
	\end{equation}
	Encore une fois, la règle de L'Hôpital nous dit que cette fonction est continue en $x=0$.

	Pour passer aux dérivées d'ordre supérieur, nous remarquons qu'elles peuvent toujours s'écrire sous la forme
	\begin{equation}	\label{Eq1907fnchshQ}
		f^{(n)}(x)=\frac{ P(x)\sinh(x)+Q(x)\cosh(x) }{ x^{n+1} }.
	\end{equation}
	Cela se voit par récurrence en utilisant la règle de Leibniz pour la dérivation de produits. La limite de \eqref{Eq1907fnchshQ} lorsque $x\to 0$ se calcule en faisant $n+1$ fois la règle de L'Hôpital. À ce moment, le dénominateur est devenu $1$ et le numérateur est toujours une combinaison de polynômes, de sinus et cosinus hyperboliques.

\end{corrige}
