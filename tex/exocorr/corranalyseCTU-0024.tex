% This is part of Analyse Starter CTU
% Copyright (c) 2014
%   Laurent Claessens,Carlotta Donadello
% See the file fdl-1.3.txt for copying conditions.

\begin{corrige}{analyseCTU-0024}

\begin{enumerate}
\item Pour répondre à cette question on commence par calculer la fonction dérivée de $f$, qui est $f'(x) = 2x^2 -x -6$. L'étude du signe de $f'$ ne présente aucune difficulté : c'est une fonction négative entre $x=-3/2$ et $x=2$ et positive ailleurs. Du coup, la fonction $f$ est strictement monotone sur les trois intervalles $I_1= ]-\infty, -3/2[$ , $I_2 =]-3/2, 2[ $ et $I_3 = ]2, +\infty[$, croissante sur $I_1$ et $I_3$ et décroissante sur $I_2$. Il est facile de vérifier que $\lim_{x\to\pm\infty}f(x) = \pm\infty$, $f(-3/2) = 15,291\bar 6$ et $f(2) = 1$. Puisque $f$ est une fonction continue, le théorème des valeurs intermédiaires nous dit que la valeur $0$ sera atteinte une seule fois, pour un $x$ qui est dans l'intervalle $I_1$.
\item Il faut calculer $f(-3) $ et $f(-4)$ et remarquer que $f(-3)$ est positive et $f(-4)$ négative. La valeur de $x$ pour laquelle $f(x) = 0$ est donc comprise entre $-4$ et $-3$.
\end{enumerate}


\end{corrige}
