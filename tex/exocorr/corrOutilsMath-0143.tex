% This is part of Outils mathématiques
% Copyright (c) 2012
%   Laurent Claessens
% See the file fdl-1.3.txt for copying conditions.

\begin{corrige}{OutilsMath-0143}

    Nous avons d'abord
    \begin{equation}
        \sigma'(t)=\begin{pmatrix}
            1    \\ 
            2t    
        \end{pmatrix},
    \end{equation}
    et ensuite
    \begin{equation}
        F\big( \sigma(t) \big)=F(t,t^2)=\begin{pmatrix}
            2t^3+t^4    \\ 
            2t^3+t^2    
        \end{pmatrix}.
    \end{equation}
    Du coup l'intégrale demandée vaut
    \begin{equation}
        \int_0^1F\big( \sigma(t) \big)\cdot\sigma'(t)dt=\int_{0}^1\begin{pmatrix}
            2t^3+t^4    \\ 
            2t^3+t^2    
        \end{pmatrix}\cdot\begin{pmatrix}
            1    \\ 
            2t    
        \end{pmatrix}dt=\int_0^1(2t^3+t^4)+2t(2t^3+t^2)dt=2.
    \end{equation}
    En ce qui concerne le potentiel, nous devons trouver une fonction \( f(x,y)\) telle que
    \begin{subequations}
        \begin{numcases}{}
            \frac{ \partial f }{ \partial x }=2xy+y^2\\
            \frac{ \partial f }{ \partial y }=2xy+x^2.
        \end{numcases}
    \end{subequations}
    En intégrant la première il est vite vu que la fonction 
    \begin{equation}
        f(x,y)=x^2y+xy^2
    \end{equation}
    fait l'affaire. L'intégrale vaut alors
    \begin{equation}
        \int_{\sigma}f=f\big( \sigma(1) \big)-f\big( \sigma(0) \big)=f(1,1)-f(0,0)=2-0=2.
    \end{equation}
    Cela confirme le résultat obtenu plus haut.

\end{corrige}
