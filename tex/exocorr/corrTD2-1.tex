% This is part of Exercices de mathématique pour SVT
% Copyright (C) 2010, 2019
%   Laurent Claessens et Carlotta Donadello
% See the file fdl-1.3.txt for copying conditions.

\begin{corrige}{TD2-1}

	\begin{enumerate}
		\item
            En utilisant \href{http://www.sagemath.org}{sage},
            \begin{verbatim}
----------------------------------------------------------------------
| Sage Version 4.7.1, Release Date: 2011-08-11                       |
| Type notebook() for the GUI, and license() for information.        |
----------------------------------------------------------------------
sage: f(x)=(x**2+x-2)/(x**2+2*x-3)
sage: f.limit(x=1)
x |--> 3/4
            \end{verbatim}
            Si nous voulons la calculer à la main, nous commençons par remarquer qu'en remplaçant \( x\) par \( 1\) nous tombons sur l'indétermination \( 0/0\). L'idée est de factoriser le numérateur et le dénominateur. Pour cela nous cherchons les racines. Pour le numérateur, les racines sont données par
            \begin{equation}
                x_{\pm}=\frac{ -1\pm\sqrt{1-4\cdot(-2)} }{ 2 }=\frac{ -1\pm3 }{2},
            \end{equation}
            c'est-à-dire \( -2\) et \( 1\). Le numérateur se factorise donc en \( (x+2)(x-1)\). De la même façon le dénominateur se factorise en \( (x-1)(x+3)\). Nous avons donc
            \begin{equation}
                \frac{ x^2+x-2 }{ x^2+2x-3 }=\frac{ (x-1)(x+2) }{ (x-1)(x+3) }=\frac{ x+2 }{ x+3 }.
            \end{equation}
            Maintenant en remplaçant \( x\) par \( 1\) nous trouvons le résultat \( 3/4\).
		\item
            Nous utilisons la technique du binôme conjugué :
            \begin{equation}
                \frac{ \sqrt{x+1}-1 }{ x }=\frac{ \left( \sqrt{x+1}-1 \right)\left( \sqrt{x+1}+1 \right) }{ x\big( \sqrt{x+1}+1 \big) }=\frac{ 1 }{ \sqrt{x+1}+1 }.
            \end{equation}
            La limite lorsque \( x\to 0\) vaut alors \( 1/2\).
		\item
			Ceci est une limite de base qu'il faut connaitre : la limite est $1$.
		\item
			En remplaçant, nous obtenons $\frac{ -\infty }{  0}$, et donc la limite est $-\infty$ sans indétermination.
		\item
			Ici il n'y a pas d'indéterminations : en remplaçant on trouve $\frac{ 1 }{ 1+0 }=1$.
		\item
			Lorsqu'il y a un sinus tout seul, un truc courant est de diviser et multiplier par $x$. Ici nous avons $\sin^2(x)$, donc nous allons multiplier et diviser par $x^2$ :
			\begin{equation}
				\frac{ x^2( e^{x^2}-1) }{ x^2\sin^2(x) }.
			\end{equation}
			Lorsqu'on prend la limite pour $x\to 0$ nous savons que $\frac{ x^2 }{ \sin^2(x) }$ tend vers $1$. Nous nous retrouvons donc à devoir calculer
			\begin{equation}
				\lim_{x\to 0} \frac{  e^{x^2}-1 }{ x^2 }=\lim_{y\to 0}\frac{ e^y-1 }{ y }, 
			\end{equation}
			où nous avons posé $y=x^2$. La dernière limite est une des limites remarquables. Il est également possible de la faire en utilisant la règle de L'Hôpital :
            \begin{equation}
                \lim_{x\to 0} \frac{  e^{x^2}-1 }{ x^2 }=\lim_{x\to 0} \frac{ 2x e^{x^2} }{ 2x }=\lim_{x\to 0}  e^{x^2}=1.
            \end{equation}
		\item
			Ici il n'y a pas tellement d'indéterminations non plus : lorsque $x\to 0$, le logarithme tend vers $-\infty$, et le premier terme tende vers zéro. En même temps, le second terme tend vers $\frac{1}{ 0-1 }=-1$, donc la différence tend vers $1$.
		\item
			Nous multiplions et divisons par le binôme conjugué $1+\cos(x)$, et nous trouvons
			\begin{equation}
                \frac{ 1-\cos(x) }{ x^2 }=\frac{ \big( 1-\cos(x) \big)\big( 1+\cos(x) \big) }{ x^2\big( 1+\cos(x) \big) }=\frac{ 1-\cos^2(x) }{ x^2\big( 1+\cos(x) \big) }=\frac{ \sin^2(x) }{ x^2 }\frac{1}{ 1+\cos(x) }.
			\end{equation}
            La limite du premier facteur est une limite remarquable qui vaut \( 1\). Le second facteur tend vers \( \frac{ 1 }{2}\). La limite du tout vaut par conséquent \( \frac{ 1 }{2}\).
		\item
			Nous allons calculer le logarithme de la limite à calculer :
            \begin{equation}        \label{EqBtkUpe}
				\spadesuit=\ln\left[ \left( 1+\frac{1}{ x^2 } \right)^x \right]=x\ln\left( 1+\frac{1}{ x^2 } \right).
			\end{equation}
			Maintenant en posant $t=\frac{1}{ x^2 }$, nous avons
			\begin{equation}
				\spadesuit=\frac{1}{ \sqrt{t} }\ln(1+t),
			\end{equation}
            dont nous devons prendre la limite lorsque $t\to \infty$. Cela est une des limites remarquables, qui vaut zéro\footnote{Le logatithme tend plus vite vers \( \infty\) que les puissances.}.

			Donc la limite que nous cherchons est
			\begin{equation}
				\lim_{x\to 0} e^{\spadesuit}= e^{0}=1.
			\end{equation}

            La limite \eqref{EqBtkUpe} peut également être calculée en utilisant la règle de L'Hôpital :
            \begin{equation}
                x\ln\left( 1+\frac{1}{ x^2 } \right)=\frac{ \ln\left( 1+\frac{1}{ x^2 } \right) }{ 1/x },
            \end{equation}
            en prenant la dérivée du numérateur et du dénominateur nous devons calculer
            \begin{equation}
                \lim_{x\to 0} \frac{ \frac{ -2 }{ \left( 1+\frac{1}{ x^2 } \right)x^2 } }{ -1/x^2 }=\lim_{x\to 0} \frac{ 2 }{ 1+\frac{1}{ x^2 } }=0.
            \end{equation}

        \item
            Nous refaisons la même manipulation que pour le point \ref{ExobgEPck}. Maintenant nous devons calculer
            \begin{equation}
                \lim_{t\to 0} \frac{ \ln(1+t) }{ \sqrt{t} }.
            \end{equation}
            Cette limite peut être déterminée à l'aide de la \wikipedia{fr}{Règle_de_L'Hôpital}{règle de L'Hôpital}. Les dérivées du numérateur et du dénominateur sont respectivement \( \frac{1}{ 1+t }\) et \( \frac{1}{ 2\sqrt{t} }\), par conséquent
            \begin{equation}
                \lim_{t\to 0} \frac{ \ln(1+t) }{ \sqrt{t} }=\lim_{t\to 0} \frac{ 2\sqrt{t} }{ 1+t }=0.
            \end{equation}
            La limite demandée est donc \(  e^{0}=1\).
			
	\end{enumerate}

\end{corrige}
