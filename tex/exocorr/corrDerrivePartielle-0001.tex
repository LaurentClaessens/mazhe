% This is part of the Exercices et corrigés de mathématique générale.
% Copyright (C) 2010,2016
%   Laurent Claessens
% See the file fdl-1.3.txt for copying conditions.

\begin{corrige}{DerrivePartielle-0001}

	La fonction $f(x,y)=xy$ est un exemple qui fonctionne bien parce que
	\begin{equation}
		\frac{ \partial f }{ \partial x }(x,y)=y.
	\end{equation}
	Nous avons alors $\partial_xf(a,b)=b$ et $\partial_xf(b,a)=a$ qui ne sont pas égaux. Cela prouve que la relation \eqref{subEqDPZUa} n'est pas correcte.

	Prouvons maintenant que la relation \eqref{subEqDPZUb} est vraie. En écrivant la définition de la dérivée partielle, nous avons
	\begin{equation}
		\begin{aligned}[]
			\frac{ \partial f }{ \partial x }&=\lim_{h\to0}\frac{ f(a+h,b)-f(a,b) }{ h }\\
			&=\lim_{h\to 0}\frac{ f(b,a+h)-f(b,a) }{ h }\\
			&=\frac{ \partial f }{ \partial y }(b,a)
		\end{aligned}
	\end{equation}
	où nous avons utilisé le fait que $f(a+h,b)=f(b,a+h)$.
\end{corrige}
