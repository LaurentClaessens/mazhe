% This is part of Analyse Starter CTU
% Copyright (c) 2014
%   Laurent Claessens,Carlotta Donadello
% See the file fdl-1.3.txt for copying conditions.

\begin{corrige}{autoanalyseCTU-37devoir}
 \begin{enumerate}
   \item[(2)] \begin{enumerate}
  \item[(b)] On a établi plus t\^ot dans l'exercice que la solution générale de l'équation homogène est donc $\displaystyle \mathcal{Y}_h  = \left\{ e^{x}\left(C_1\cos(2x) +C_2\sin(2x)\right)  \,:\, C_1,\, C_2 \in \eR \right\}.$

On cherche une solution particulière $y_p$ de l'équation non homogène $y''-2y'+5y=x$. Comme le membre de droite est un polyn\^one de degré un, $y_p(x)$ sera  de la forme $Ax + B$. En injectant $y_p$ dans l'équation nous obtenons 
\begin{equation*}
  -2A + 5\left(Ax+B\right) = x,
\end{equation*}
qui correspond au système de deux équations pour les deux inconnues $A$ et $B$
\begin{equation*}
    5A =1 ,\text{ et, }-2A +5B = 0,
\end{equation*}
On a alors $A=1/5$ et $B=2/25$, et $y_p = \displaystyle \frac{1}{5}x +\frac{2}{25}$.

La solution générale de l'équation  $y''-2y'+5y=x$ est la somme de $\mathcal{Y}_h$ et $y_p$, c'est-à-dire 
\begin{equation*}
  \mathcal{Y} = \left\{ e^{x}\left(C_1\cos(2x) +C_2\sin(2x)\right)+ \frac{1}{5}x +\frac{2}{25} \ \,:\, C_1,\, C_2 \in \eR \right\}.
\end{equation*}

Les conditions initiales fixées nous donnent le système suivant pour trouver les valeurs de $C_1$ et $C_2$ correspondants à la solution particulière demandée :
\begin{equation*}
  \begin{cases}
    C_1+\frac{2}{25}= 0,  & \quad\text{ qui correspond à la condition }y(0)=0,\\
    C_1+2C_2+\frac{1}{5} = 0 & \quad\text{ qui correspond à la condition }y'(0)=0.
  \end{cases}
\end{equation*}
Donc $C_1= -2/25$ et $C_2 = -3/50$ et $y(x) = e^{x}\left(-\frac{2}{25}\cos(2x) -\frac{3}{50}\sin(2x)\right)+ \frac{1}{5}x+\frac{2}{25}$.
  \item[(c)] Pour obtenir une solution particulière de cette équation il suffit de sommer les solutions particulières trouvées pour les équations des points (Z.a) et (2.b) de cet exercice (on exploite ici le fait que l'équation soit linéaire). La solution générale de cette équation est donc 
 \begin{equation*}
  \mathcal{Y} = \left\{ e^{x}\left(C_1\cos(2x) +C_2\sin(2x)\right)+ \frac{1}{5} \cos(x) -\frac{1}{10}\sin(x)+\frac{1}{5}x +\frac{2}{25} \ \,:\, C_1,\, C_2 \in \eR \right\}.
\end{equation*}
Les conditions initiales fixées nous donnent le système suivant pour trouver les valeurs de $C_1$ et $C_2$ correspondants à la solution particulière demandée :
\begin{equation*}
  \begin{cases}
    C_1+\frac{7}{25}= 0,  & \quad\text{ qui correspond à la condition }y(0)=0,\\
    C_1+2C_2+\frac{1}{10} = 0 & \quad\text{ qui correspond à la condition }y'(0)=0.
  \end{cases}
\end{equation*}
Donc $C_1= -7/25$ et $C_2 = 9/100$ et 
\[y(x) = e^{x}\left(-\frac{7}{25}\cos(2x) +\frac{9}{100}\sin(2x)\right)+ \frac{1}{5} \cos(x) -\frac{1}{10}\sin(x)+ \frac{1}{5}x+\frac{2}{25}.
\]
  \end{enumerate}
  \end{enumerate}

\end{corrige}   
