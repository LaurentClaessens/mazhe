% This is part of the Exercices et corrigés de mathématique générale.
% Copyright (C) 2009,2015
%   Laurent Claessens
% See the file fdl-1.3.txt for copying conditions.
\begin{corrige}{Lineraire0037}

	Comme ce sont des matrices symétriques, il existe une base dans laquelle elles sont diagonales. Et de plus, cette base est elle-même orthonormale, donc le changement de base est donné par une matrice orthogonale.

	Voyons les vecteurs propres de la matrice
	\begin{equation}
		A=\begin{pmatrix}
			3	&	-1	&	1	\\
			-1	&	5	&	-1	\\
			1	&	-1	&	3
		\end{pmatrix},
	\end{equation}
	Nous avons $\det(1-\lambda\mtu)=-(\lambda-6)(\lambda-3)(\lambda-2)$. Nous devons résoudre successivement les systèmes $Av=6v$, $Av=3v$ et $Av=2v$. Les vecteurs trouvés sont respectivement
	\begin{equation}
		\begin{aligned}[]
			\begin{pmatrix}
				1	\\ 
				-2	\\ 
				1	
			\end{pmatrix},
			&&\begin{pmatrix}
				1	\\ 
				1	\\ 
				1	
			\end{pmatrix},&&\begin{pmatrix}
				1	\\ 
				0	\\ 
				-1	
			\end{pmatrix}.
		\end{aligned}
	\end{equation}
	En calculant les produits scalaires, il est vite vu que ces trois vecteurs sont deux à deux orthogonaux. La matrice qui fait passer de la base canonique à la base de ces trois vecteurs s'obtient en mettant simplement ces trois vecteurs en colonnes :
	\begin{equation}
		\begin{pmatrix}
			1	&	1	&	1	\\
			-2	&	1	&	0	\\
			1	&	1	&	-1
		\end{pmatrix}.
	\end{equation}
	Hélas, cette matrice n'est pas orthogonale parce que les trois vecteurs obtenus ne sont pas de norme $1$. Il faut donc un peu les modifier. La norme du premier est $\sqrt{1^2+4=1}=\sqrt{6}$, donc en divisant le vecteur par $\sqrt{6}$, nous avons un vecteur de norme $1$. La norme du second est $\sqrt{3}$ et celle du troisième est $\sqrt{2}$, donc la matrice orthogonale recherchée est
	\begin{equation}
		\begin{pmatrix}
			1/\sqrt{6}	&	1/\sqrt{3}	&	1\sqrt{2}	\\
			-2/\sqrt{6}	&	1/\sqrt{3}	&	0	\\
			1/\sqrt{6}	&	1/\sqrt{3}	&	-1/\sqrt{2}
		\end{pmatrix}.
	\end{equation}
\end{corrige}
