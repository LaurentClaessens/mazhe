\begin{corrige}{CalculDifferentiel0021}

	\begin{enumerate}

		\item
			Nous avons vu autour de l'équation \eqref{EqcddzuiiDifcy} que $\partial_xf(0,0)=1$ et $\partial_yf(0,0)=-1$. Par conséquent la différentielle de $f$ en $(0,0)$, si elle existe, vaut $T(x,y)=0$. Afin de voir si cela est bien la différentielle, nous calculons
			\begin{equation}
				\begin{aligned}[]
					\lim_{(x,y)\to(0,0)}\frac{ \frac{ x^3-y^3 }{ x^2-y^2 }-(x-y) }{ \sqrt{x^2+y^2} }&=\lim_{(x,y)\to(0,0)}\frac{ -xy^2+yx^2 }{ (x^2+y^2)^{3/2} }\\
					&=\lim_{r\to 0} r^3\frac{ \cos\theta\sin\theta(\cos\theta-\sin\theta) }{ r^3 }.
				\end{aligned}
			\end{equation}
			Cette dernière limite étant dépendante de $\theta$, nous en déduisons que la limite qui définit la différentielle n'existe pas. La fonction n'est donc pas différentiable en $(0,0)$.
	\end{enumerate}
	

\end{corrige}
