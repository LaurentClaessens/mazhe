% This is part of the Exercices et corrigés de CdI-2.
% Copyright (C) 2008, 2009
%   Laurent Claessens
% See the file fdl-1.3.txt for copying conditions.


\begin{exercice}\label{exo_I-2-4}

Lorsqu'un point matériel pesant est assujetti à décrire une courbe polie d'équations
\begin{equation}
	s\mapsto\big( x(s),y(s),z(s) \big)
\end{equation}
où $s$ est la longueur d'arc, le mouvement est décrit par (expression du temps de parcours) :
\begin{equation}
	t-t_0=\int_{s_0}^s\frac{ ds }{ \sqrt{ 2g(h-z(s)) } }
\end{equation}
où $g$ est l'accélération de la pesanteur et $h$ une constante. Étudier l'existence de l'intégrale 
\begin{equation}
	\int_{s_0}^{s_1}  \frac{ ds }{ \sqrt{ 2g(h-z(s)) } }
\end{equation}
lorsque $z(s_1)=h$, en supposant que $z(s)$ est de classe $C^2$.

\corrref{_I-2-4}
\end{exercice}
