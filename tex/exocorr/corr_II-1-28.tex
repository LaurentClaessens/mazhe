% This is part of the Exercices et corrigés de CdI-2.
% Copyright (C) 2008, 2009
%   Laurent Claessens
% See the file fdl-1.3.txt for copying conditions.


\begin{corrige}{_II-1-28}

La théorie se trouve à la page II.85 du cours. En suivant ces notations, nous avons ici $f(y')=1/y'^2$, dont l'ensemble de définition est $\eR_0$. Donc pour tout $c\in\eR_0$, la droite
\begin{equation}		\label{EqII128lignes}
	y_c(t)=ct+\frac{1}{ c^2 }
\end{equation}
est solution. Nous trouvons les constantes $c$ qui correspondent au problème de Cauchy $y(t_0)=y_0$ en résolvant l'équation
\begin{equation}		\label{EqTrouverFDroitesII128}
	c^2y_0-c^3t_0-1=0.
\end{equation}
En vertu de ce qui est écrit au bas de la page II.86, l'enveloppe est donnée paramétriquement par
\begin{subequations}
	\begin{numcases}{}
	t(\lambda)=-f'(\lambda)\\
	y(\lambda)=-f'(\lambda)\lambda+f(\lambda)
	\end{numcases}
\end{subequations}
où $f(\lambda)=1/\lambda^2$. En remplaçant,
\begin{subequations}
	\begin{numcases}{}
	t(\lambda)=\frac{ 2 }{ \lambda^3 }\\
	y(\lambda)=\frac{ 3 }{ \lambda^2 }.
	\end{numcases}
\end{subequations}
La relation $t(\lambda)$ peut être inversée : $\lambda=\sqrt[3]{2/t}$, et donc
\begin{equation}
	y(t)=3\left( \frac{ 2 }{ t } \right)^{-2/3}.
\end{equation}
Cette dernière équation est plus belle sous la forme
\begin{equation}
	y_E^3(t)=\frac{ 27 }{ 4 }t^2.
\end{equation}

Étudions maintenant les problèmes de Cauchy proposés.

\begin{enumerate}

\item
$y(1)=2$.
Étant donné que le point $(1,2)$ n'est pas sur $y_E$, regardons s'il ne serait pas sur une des droites, c'est-à-dire : résolvons $y_c(1)=2$ par rapport à $c$. Nous trouvons les solutions
\begin{equation}
	\begin{aligned}[]
		c_1&=1	&& c_2&=\frac{ -\sqrt{5}+1 }{ 2 }&c_3&=\frac{ \sqrt{5}+1 }{2}.
	\end{aligned}
\end{equation}
Les trois solutions correspondantes s'obtiennent en mettant $c_i$ dans l'équation des droites \eqref{EqII128lignes}.


\item
$y(2)=3$.
Le point $(2,3)$ est sur l'enveloppe, et l'équation \eqref{EqTrouverFDroitesII128} à résoudre pour trouver les droites est
\begin{equation}
	3c^2-2c^3-1=0,
\end{equation}
les solution sont alors $c=-\frac{1}{ 2 }$ et $c=1$. Les droites solutions sont donc
\begin{subequations}
	\begin{numcases}{}
	y_1(t)	=\frac{ 1 }{2}t+4\\
	y_2(t)=t+1.
	\end{numcases}
\end{subequations}

\item
$y(0)=1$. Le point $(0,1)$ n'est pas sur l'enveloppe, et les droites sont données par $c=\pm 1$.

\item
$y(0)=-1$. Ce n'est pas sur l'enveloppe, et il n'y a pas de solutions à l'équation \eqref{EqTrouverFDroitesII128}. Pas du tout de solutions dans ce cas-ci.

\item
$y(3)=-2$. Cela n'est pas sur l'enveloppe, et une seule droite $c=-1$, c'est-à-dire $y=-t+1$.

\end{enumerate}

\end{corrige}
