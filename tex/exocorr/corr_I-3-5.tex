% This is part of the Exercices et corrigés de CdI-2.
% Copyright (C) 2008, 2009
%   Laurent Claessens
% See the file fdl-1.3.txt for copying conditions.


\begin{corrige}{_I-3-5}

Nous considérons la fonction 
\begin{equation}
	f(x,t)=\frac{  e^{t} }{  \sin(xt)^{1/3} }.
\end{equation}

La stratégie que nous allons suivre est la suivante :
\begin{enumerate}
	\item montrer que $\int_{0}^1f(x,t)dt$ existe $\forall x\in]0,1[$,
	\item on montre que $F(x)$ est continue par le coup du compact pour $x\in]0,1[$,
	\item pour étudier la dérivabilité, on pose
	\begin{equation}
		G(x)=\int_0^1\frac{ \partial f }{ \partial x }(x,t)dt,
	\end{equation}
	dont on prouve d'abord l'existence et ensuite l'uniforme convergence.
\end{enumerate}

D'abord, montrons que pour tout $x\in ]0,1]$, l'intégrale de la fonction $f_x\colon t\mapsto f(x,t)$ par rapport à $t$ existe. Pour cela nous allons utiliser le corollaire \ref{CorAlphaLCasInteabf}. En nous souvenant que la page 191 du cours de première dit que $\lim_{x\to 0}\sin(x)/x=1$, nous avons
\begin{equation}
		\lim_{t\to0}t^{1/3}f(x,t)=\lim_{t\to 0}\left( \frac{ xt }{ \sin(xt) } \right)^{\frac{ 1 }{ 3 }}\frac{ e^t }{ x^{1/3} }=\frac{1}{ x^{1/3} }<\infty.
\end{equation}
Nous appliquons le corollaire \ref{CorAlphaLCasInteabf} chaque fonction $f_x(t)=f(x,t)$, dont nous concluons que pour tout $x\neq 0$,
\begin{equation}
	\int_0^1 f_x(t)dt
\end{equation}
existe. Ceci montre que la fonction $F(x)$ est bien définie. Nous étudions maintenant sa continuité. Nous faisons le coup du compact. Soit donc un compact de $]0,1]$ dont le minimum est $x_0>0$, nous avons la majoration
\begin{equation}
	f(x,t)\leq\frac{ e^t }{ \sin(x_0t)^{1/3} },
\end{equation}
dont l'intégrale existe : ce n'est autre que $F(x_0)$ dont nous venons de prouver l'existence. Par le critère de Weierstrass, l'intégrale $\int_0^1f(x,t)\,dt$ est uniformément convergente sur le compact considéré. Par le coup du compact et le théorème \ref{ThoInDerrtCvUnifFContinue}, la fonction $F$ est alors continue sur l'ensemble $]0,1]$.

Nous pouvons maintenant étudier la dérivabilité sous le signe intégral de $F$. Pour cela, considérons
\begin{equation}		\label{EqIntPartialAvUtCorr35}
	G(x)=\int_0^1\frac{ \partial f }{ \partial x }(x,t)dt=-\int_0^1\frac{ t e^{t} }{ 3 }\frac{ \cos(xt) }{ \sin(xt)^{4/3} }.
\end{equation}
Nous allons prouver que cette intégrale existe et qu'elle converge uniformément en $x$ sur tout compact de $]0,1$. Ainsi, elle sera la dérivée de $F$ et continue, ce qui prouve que $F$ est $C^1$.

Pour prouver l'existence, nous procédons comme pour $F$ :
\begin{equation}
	\lim_{t\to 0}t^{1/3}\left| \frac{ \partial f }{ \partial x } \right| =\frac{ e^t }{ 3 }\frac{ \cos(xt) }{ x^{4/3} }\left( \frac{ xt }{ \sin(xt) } \right)^{\frac{ 4 }{ 3 }}=\frac{ 1 }{ 3x^{4/3} }.
\end{equation}
Nous en déduisons que l'intégrale \eqref{EqIntPartialAvUtCorr35} existe pour tout $x\neq 0$.

Maintenant, nous voudrions bien borner $\frac{ cos(xt) }{ \sin(xt)^{4/3} }$ de façon indépendante de $x$ pour $x\in K\subset ]0,1[$. Nous serions donc content que cette fonction soit soit croissante soit décroissante sur $]0,1[$. En réalité, elle est décroissante parce que si nous posons
\begin{equation}
		l(y)=\frac{ \cos(y) }{ \sin(y)^{4/3} },
\end{equation}
alors 
\begin{equation}
	l'(y)=-\frac{ \sin(y)\cos(y)^{4/3}+\frac{ 4 }{ 3 }\sin(y)^{1/3}\cos(y) }{ \sin(y)^{8/3} },
\end{equation}
qui est négatif tant que $y\in[0,\frac{ \pi }{ 2 }]$. Nous pouvons donc majorer la dérivée de $f$ comme ceci :
\begin{equation}
	\left| \frac{ \partial f }{ \partial x } \right| (x,t)\leq\frac{ te^t }{ 3 }\frac{ \cos(x_0t) }{ \sin(x_0t)^{4/3} }.
\end{equation}
L'intégrale du membre de droite existe : c'est $G(x_0)$ dont nous venons de prouver l'existence. Donc $G(x)$ converge uniformément sur tout compact de $]0,1]$. Cela prouve deux choses. Premièrement, $G(x)$ est la dérivée de $F(x)$, et secondement $G$ est continue. Ces deux points sont exactement le fait que $F$ est $C^1$ sur $]0,1]$.

\end{corrige}
