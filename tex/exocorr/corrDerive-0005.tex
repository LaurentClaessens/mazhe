% This is part of Outils mathématiques
% Copyright (c) 2011, 2019
%   Laurent Claessens
% See the file fdl-1.3.txt for copying conditions.

\begin{corrige}{Derive-0005}

    L'angle proche de $46\degree$ dont nous connaissons le cosinus est $45\degree$. En passant aux radians,
    \begin{equation}
        \cos(46\degree)=\cos\big( \frac{ \pi }{ 4 }+\frac{ \pi }{ 180 } \big).
    \end{equation}
    Nous pouvons donc donner une approximation de $\cos(46\degree)$ en partant de la fonction cosinus en $\frac{ \pi }{ 4 }$ et de sa dérivée :
    \begin{equation}
        \begin{aligned}[]
            \cos\big( \frac{ \pi }{ 4 }+\frac{ \pi }{ 180 } \big)&\simeq\cos\big( \frac{ \pi }{ 4 } \big)-\frac{ \pi }{ 180 }\sin\big( \frac{ \pi }{ 4 } \big)\\
            &=\frac{ \sqrt{2} }{2}-\frac{ \pi }{ 180 }\frac{ \sqrt{2} }{2}\\
            &=\frac{1}{ \sqrt{2} }\left( 1-\frac{ \pi }{ 180 } \right).
        \end{aligned}
    \end{equation}
    Notez toutefois que la technique a ses limites. Nous avons certes réussi à exprimer le cosinus en faisant disparaitre les fonctions trigonométriques; mais il n'en reste pas mois que pour avoir une réelle approximation numérique, il faut encore être capable de calculer $\sqrt{2}$ et $\pi$.

    Pour vérification :
    \begin{verbatim}
----------------------------------------------------------------------
| Sage Version 4.6.1, Release Date: 2011-01-11                       |
| Type notebook() for the GUI, and license() for information.        |
----------------------------------------------------------------------
sage: x=23*pi/90 
sage: exacte=cos(x)
sage: approximation=(1/sqrt(2))*(1-pi/180)
sage: numerical_approx(exacte-approximation)
-0.000107069232665902
    \end{verbatim}
    La valeur exacte est donc un peu plus petite que la valeur calculée. Cela s'explique par le fait que, si on regarde la courbe de cosinus (figure \ref{LabelFigFnCosApprox}), on voit que la pente s'accélère\footnote{En effet, la dérivée de la dérivée est $-\cos(x)$ qui à cet endroit est négatif, donc la dérivée, c'est-à-dire la pente de la tangente, devient de plus en plus négative.}, ce qui a pour effet que la tangente a tendance à surévaluer la fonction.

    \newcommand{\CaptionFigFnCosApprox}{La fonction cosinus autour de $\frac{ \pi }{2}$.}
    \input{auto/pictures_tex/Fig_FnCosApprox.pstricks}

\end{corrige}
