% This is part of the Exercices et corrigés de mathématique générale.
% Copyright (C) 2010
%   Laurent Claessens
% See the file fdl-1.3.txt for copying conditions.

\begin{corrige}{FoncDeuxVar0025}

	Nous voulons voir ce que ça donne sur $f$  d'ajouter $1$ à $x$ pour la fonction
	\begin{equation}
		f(x,y)=60x^{1/2}y^{1/3}
	\end{equation}
	lorsque $x=900.000$ et $y=1000$. Nous utilisons la formule des accroissements finis avec les dérivées partielles :
	\begin{equation}		\label{EqDVVCqccf}
		f(900.000+1,1000)=(900.000,1000)+1\cdot\frac{ \partial f }{ \partial x }(900.000,1000).
	\end{equation}
	La dérivée partielle vaut
	\begin{equation}
		\frac{ \partial f }{ \partial x }=\frac{ 60 }{ 2\sqrt{x} }y^{1/3}.
	\end{equation}
	En remplaçant avec le nombres donnés, la formule \eqref{EqDVVCqccf} donne
	\begin{equation}
		\begin{aligned}[]
			f(x,y)&=60\cdot 300\sqrt{10}\cdot 10=18.000\sqrt{10}\\
			(\partial_xf)(900.000,1000)&=\frac{ 60\cdot 10 }{ 2\cdot 300\sqrt{10} }=\frac{1}{ \sqrt{10} }.
		\end{aligned}
	\end{equation}
	Donc en ajoutant $1000$ euros au capital, on n'augmente que de $1/\sqrt{10}$ la productivité.


\end{corrige}
