% This is part of Exercices et corrigés de CdI-1
% Copyright (c) 2011
%   Laurent Claessens
% See the file fdl-1.3.txt for copying conditions.

\begin{corrige}{OutilsMath-0030}

    Il s'agit d'appliquer la définition \ref{DefDiffrdrrOM}, et plus particulièrement la formule \eqref{EqDefDiffmapdfOM}.
    \begin{enumerate}
        \item
            D'abord les dérivées partielles :
            \begin{subequations}
                \begin{align}
                    \frac{ \partial f }{ \partial x }(x,y)&=2xy+2x^2y\\
                    \frac{ \partial f }{ \partial y }(x,y)&=x^2+x^3.
                \end{align}
            \end{subequations}
            Donc
            \begin{equation}
                \begin{aligned}[]
                    df_{(a,b)}\begin{pmatrix}
                        u_1    \\ 
                        u_2    
                    \end{pmatrix}&=\frac{ \partial f }{ \partial x }(a,b)u_1+\frac{ \partial f }{ \partial y }(a,b)u_2\\
                    &=(2ab+3a^2b)u_1+(a^2+a^3)u_2.
                \end{aligned}
            \end{equation}
        \item
            Les dérivées partielles :
            \begin{equation}
                \begin{aligned}[]
                    \frac{ \partial f }{ \partial x }(x,y)=-2x\sin(x^2+y^4)+y\cos(xy)\\
                    \frac{ \partial f }{ \partial y }(x,y)=-4y^3\sin(x^2+y^4)+x\cos(xy)
                \end{aligned}
            \end{equation}
            Nous avons de toutes façons que les deux sont nulles lorsque $x=y=0$, dont la différentielle est nulle :
            \begin{equation}
                df_{(0,0)}\begin{pmatrix}
                    u_1    \\ 
                    u_2    
                \end{pmatrix}=0.
            \end{equation}
        \item
            Les dérivées partielles sont
            \begin{equation}
                \begin{aligned}[]
                    \frac{ \partial f }{ \partial x }&=2xyz+yz e^{xyz}\\
                    \frac{ \partial f }{ \partial y }&=x^2z+2yz^2+xz e^{xyz}\\
                    \frac{ \partial f }{ \partial z }&=x^2y+2y^2z+xy e^{xyz}.
                \end{aligned}
            \end{equation}
            Si on pose $x=1$, $y=0$ et $z=1$, nous trouvons que seule $\partial_yf$ est non nulle et vaut deux. Donc
            \begin{equation}
                df_{(1,0,1)}\begin{pmatrix}
                    u_1    \\ 
                    u_2    \\ 
                    u_3    
                \end{pmatrix}=2u_2.
            \end{equation}
    \end{enumerate}

\end{corrige}
