% This is part of Exercices et corrigés de CdI-1
% Copyright (c) 2011
%   Laurent Claessens
% See the file fdl-1.3.txt for copying conditions.

\begin{corrige}{IntMult0004}

	$\rho^2=a^2\cos(2\theta)$. La surface délimitée par cette courbe contient deux parties : une à gauche de l'axe vertical et une à droite. Nous allons calculer l'aire de la partie à droite. Afin d'obtenir l'aire totale, il faudra multiplier par deux.
	
	Pour chaque $\theta$, le rayon du domaine va de $0$ à $a\sqrt{\cos(2\theta)}$. Étant donné que $r$ est toujours positif, les angles ne vont que de $-\pi/4$ à $\pi/4$, donc
\begin{equation}
	I=\int_{-\pi/4}^{\pi/4}d\theta\int_0^{a\sqrt{\cos(2\theta)}}r\,dr=\frac{ a^2 }{ 2 }.
\end{equation}
	L'aire totale vaut donc $a^2$.

\end{corrige}
