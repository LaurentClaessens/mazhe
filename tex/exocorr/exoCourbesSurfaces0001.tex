\begin{exercice}\label{exoCourbesSurfaces0001}

	Les courbes qui ont les mêmes graphes ne sont pas toujours équivalentes. Quelque exemples.
	\begin{enumerate}
		\item
 On considère les courbes planes $([0,1], f)$ et  $([-1,1], g)$ définies par
 \begin{equation}
	 \begin{aligned}[]
		 f(t) = (t, t) &&\text{et}&& g(t)= (t^2, t^2).
	 \end{aligned}
 \end{equation}
Montrer que les deux courbes ont le même graphe (image), mais ne sont pas équivalentes.

\item
Même question avec les courbes $([0, 2 \pi], f_1)$ et  $([0, 6\pi], g_1)$ définies par $ f_1(t) = (\cos t, \sin t)$   et $ \vec{g_1}(t)= (\cos t, \sin t).$

\item
Montrer que les courbes $(\mathopen] -\pi , \pi \mathclose[, f)$ et $(\eR, g)$ définies par 
\begin{equation}
	\begin{aligned}[]
		f (t) = ( \cos t, \sin t) &&\text{et}&&  g(t) = \left( \frac{1-t^2}{1 + t^2}, \frac{ 2t}{ 1 + t^2} \right)
	\end{aligned}
\end{equation}
sont équivalentes.
			
	\end{enumerate}

    Indice : parmi les formules de trigonométries, nous rappelons 
    \begin{subequations}
        \begin{align}
            \frac{ 1-\tan^2\left( \frac{ x }{ 2 } \right) }{ 1+\tan^2\left( \frac{ x }{2} \right) }&=\cos(x)\\
            \sin(2x)&=2\sin(x)\cos(x).
        \end{align}
    \end{subequations}

\corrref{CourbesSurfaces0001}
\end{exercice}
