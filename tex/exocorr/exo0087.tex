% This is part of Exercices et corrigés de CdI-1
% Copyright (c) 2011,2013, 2025
%   Laurent Claessens
% See the file fdl-1.3.txt for copying conditions.

\begin{exercice}\label{exo0087}

	À propos de fonctions lipschitziennes.
	\begin{enumerate}
		\item
		      Montrer qu'une application lipschitzienne est continue.

		\item
		      Montrer qu'une application $f \colon \eR \to \eR$, $x \mapsto ax+b$ est lipschitzienne.  Quelle est la plus petite constante $L$ qui convienne?

		\item
		      Montrer que les fonctions $z \mapsto |z|$, $z \mapsto \overline z$, $z \mapsto {\rm Re\,} z$ et $z \mapsto {\rm Im\,} z$ sont lipschitziennes.  Quelle sont les plus petites constantes $L$ qui conviennent?

		\item
		      Montrer que la fonction $d(A,\cdot)\colon X \to \eR$ de l'exercice \ref{exo0086} est lipschitzienne.

	\end{enumerate}

	\corrref{0087}
\end{exercice}
