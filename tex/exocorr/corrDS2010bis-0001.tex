% This is part of Exercices de mathématique pour SVT
% Copyright (C) 2010
%   Laurent Claessens et Carlotta Donadello
% See the file fdl-1.3.txt for copying conditions.

\begin{corrige}{DS2010bis-0001}

	\begin{enumerate}
		\item
			Pour la fonction  $f$, nous avons un dénominateur qui ne peut pas s'annuler : $1-x^2\neq 0$, c'est à dire $x^2\neq 1$. Le domaine est donc 
			\begin{equation}
				\Dom(f)=\eR\setminus\{ -1,1 \}.
			\end{equation}
			Pour rappel, $x^2=1$ lorsque $x=1$ ou $x=-1$.

			En ce qui concerne la fonction $g$, la seule condition est que $x>0$ à cause du logarithme.
		\item
			\begin{enumerate}
				\item
					Pour $f\circ g$, nous avons
					\begin{equation}
						(f\circ g)(x)=f\big( g(x) \big)=\frac{1}{ 1-g(x)^2 }=\frac{1}{ 1-\ln(x)^2 }.
					\end{equation}
					Dans cette expression nous voyons une fraction et un logarithme. Il y aura donc deux conditions.
					\begin{enumerate}
						\item
							Pour le dénominateur, $1-\ln(x)^2\neq 0$. Cela demande $\ln(x)^2\neq 1$, et donc les deux conditions
							\begin{equation}
								\begin{aligned}[]
									\ln(x)&\neq 1\\
									\ln(x)&\neq -1.
								\end{aligned}
							\end{equation}
							La première demande $x\neq e$, tandis que la seconde demande $x\neq -e$.
						\item
							Pour le logarithme, la condition est $x>0$.
					\end{enumerate}
					En résumé, nous devons avoir $x>0$ en même temps que $x\neq e$ et $x\neq -e$. Le domaine de $f\circ g$ est donc
					\begin{equation}
						\Dom\big( f\circ g \big)=\mathopen] 0 , \infty \mathclose[\setminus\{ e \}.
					\end{equation}
					Notez que la condition $x\neq -e$ est redondante par rapport à la condition $x>0$.

				\item
					Pour $g\circ f$, nous avons
					\begin{equation}
						(g\circ f)(x)=g\big( f(x) \big)=\ln\big( f(x) \big)=\ln\left( \frac{1}{ 1-x^2 } \right).
					\end{equation}
					Ici nous voyons une fraction et un logarithme. Il y aura donc deux conditions.
					\begin{enumerate}
						\item
							Pour la fraction, le dénominateur ne peut pas s'annuler : $1-x^2\neq 0$. Cela demande $x\neq 1$ et $x\neq -1$.
						\item
							Pour le logarithme, nous devons avoir $\frac{1}{ 1-x^2 }>0$, c'est à dire $1-x^2>0$. 
					\end{enumerate}
					Le domaine est donc
					\begin{equation}
						\Dom(g\circ f)=\mathopen] -1 , 1 \mathclose[.
					\end{equation}
			\end{enumerate}
			
	\end{enumerate}

\end{corrige}
