% This is part of the Exercices et corrigés de mathématique générale.
% Copyright (C) 2009
%   Laurent Claessens
% See the file fdl-1.3.txt for copying conditions.
\begin{corrige}{Lineraire0020}

	\begin{enumerate}

		\item
			C'est la base canonique, donc le vecteur a pour composantes $(1,2)$.

		\item
			Il faut résoudre
			\begin{equation}
				\lambda_1\begin{pmatrix}
					0	\\ 
					1	
				\end{pmatrix}+\lambda_2
				\begin{pmatrix}
					1	\\ 
					0	
				\end{pmatrix}=\begin{pmatrix}
					1	\\ 
					2	
				\end{pmatrix}.
			\end{equation}
			Facile, la réponse est que les coordonnées $(\lambda_1,\lambda_2)=(2,1)$.
		\item
			À résoudre, le système
			\begin{equation}
				\left\{
				\begin{array}{ll}
					a+b=1\\
					2a+b=2
				\end{array}
				\right..
			\end{equation}
			La solution est $a=1$ et $b=0$, donc les coordonnées sont $(1,0)$.

		\item
			À résoudre :
			\begin{equation}
				\lambda_1\begin{pmatrix}
					1	\\ 
					1	
				\end{pmatrix}+\lambda_2\begin{pmatrix}
					2	\\ 
					1	
				\end{pmatrix}=\begin{pmatrix}
					1	\\ 
					2	
				\end{pmatrix},
			\end{equation}
			la solution est $(\lambda_1,\lambda_2)=(1,)1$.

		\item
			Le système est
			\begin{equation}
				\left\{
				\begin{array}{ll}
					-a+4b=1\\
					3a=2
				\end{array}
				\right.,
			\end{equation}
			donc $(2/3,-5/12)$.

	\end{enumerate}
	

\end{corrige}
