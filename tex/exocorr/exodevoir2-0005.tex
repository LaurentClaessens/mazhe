\begin{exercice}\label{exodevoir2-0005}
 
Soit $f:\mathbb{R}^2\to\mathbb{R}$ la fonction définie par 
\[
f(x,y)=\left\{
\begin{array}{ll}
  \frac{xy}{|x|+y^2},\qquad & (x,y)\neq (0,0),\\
  0, &(x,y)=(0,0). 
\end{array}\right.
\]
\begin{enumerate}
%\item Démontrer que $f$ est continue en $(0,0)$.
\item Calculer les dérivées partielles de $f$ en $(0,0)$. 
\item Montrer que $f$ n'est pas différentiable. 
%\item Soit $\vect{v}_\theta=(\cos\theta,\sin\theta)$. Calculer $\frac{\partial f}{\vect{v}_\theta}(0,0)$ pour tout $\theta$ dans l'intervalle $[0,2\pi]$. Attention, puisque $f$ n'est pas différentiable on ne peut pas utiliser la formule $\nabla f\cdot \vect{v}_\theta$, il faut utiliser définition.
\end{enumerate}

\corrref{devoir2-0005}
\end{exercice}
