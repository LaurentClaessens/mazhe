% This is part of Analyse Starter CTU
% Copyright (c) 2014
%   Laurent Claessens,Carlotta Donadello
% See the file fdl-1.3.txt for copying conditions.

\begin{exercice}\label{exoanalyseCTU-0001}

Quelques questions à propos d'intervalles et de croissance.

\begin{enumerate}
    \item
        Donner un intervalle ouvert contenant le nombre \( -4\).
    \item
        Donner un exemple de fonction décroissante mais non strictement décroissante.
    \item
        Supposons que la fonction \( f\) dont le graphe est donné ci-dessous soit dérivable. Dresser le tableau de signe de la dérivée de \( f\).

        \begin{center}
            \input{auto/pictures_tex/Fig_TVXooWoKkqV.pstricks}
        \end{center}
\end{enumerate}


\corrref{analyseCTU-0001}
\end{exercice}
