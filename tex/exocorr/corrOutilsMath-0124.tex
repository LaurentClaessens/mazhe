% This is part of Outils mathématiques
% Copyright (c) 2011, 2024
%   Laurent Claessens
% See the file fdl-1.3.txt for copying conditions.

\begin{corrige}{OutilsMath-0124}

	\begin{enumerate}
		\item
		      En oubliant pas la règle de Leibniz, les dérivées partielles sont données par
		      \begin{subequations}        \label{Subeqsderpartzutz}
			      \begin{align}
				      \frac{ \partial f }{ \partial x } & =2y^3z\cos(xyz)+2xyz              \\
				      \frac{ \partial f }{ \partial y } & =2xy^2z\cos(xyz)+x^2z+4y\sin(xyz) \\
				      \frac{ \partial f }{ \partial z } & =2xy^3\cos(xyz)+x^2y.
			      \end{align}
		      \end{subequations}
		\item
		      Étant donné que ces dérivées partielles sont continues, la différentielle de \( f\) au point \( (1,1,\pi)\) est l'application
		      \begin{equation}
			      df_{(1,1,\pi)}(u_1,u_2,u_4)=\frac{ \partial f }{ \partial x }(1,1,\pi)u_1+\frac{ \partial f }{ \partial y }(1,1,\pi)u_2+\frac{ \partial f }{ \partial z }(1,1,\pi)u_3.
		      \end{equation}
		      Ici en remplaçant \( x=1\), \( y=1\) et $z=\pi$ dans les équations \eqref{Subeqsderpartzutz} nous trouvons
		      \begin{subequations}
			      \begin{align}
				      \frac{ \partial f }{ \partial x }(1,1,\pi) & =0    \\
				      \frac{ \partial f }{ \partial y }(1,1,\pi) & =-\pi \\
				      \frac{ \partial f }{ \partial z }(1,1,\pi) & =-1   \\
			      \end{align}
		      \end{subequations}
		\item
		      En ce qui concerne l'approximation nous avons
		      \begin{equation}
			      f(1+10^{-2},1-10^{-3},\pi+10^{-4}\simeq f(1,1,\pi)+10^{-3}\pi-10^{-4}
		      \end{equation}
		      avec \( f(1,1,\pi)=\pi\).
	\end{enumerate}

\end{corrige}
