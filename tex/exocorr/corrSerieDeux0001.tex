% This is part of Exercices et corrections de MAT1151
% Copyright (C) 2010,2014
%   Laurent Claessens
% See the file LICENCE.txt for copying conditions.

\begin{corrige}{SerieDeux0001}

	Nous devons vérifier les trois conditions de la définition \pageref{DefOYPooZIoWnI}.
	\begin{enumerate}

		\item
			Si $\sup_{| x |=1}\{ |\alpha(x)| \}=0$, c'est que $\alpha(x)=0$ pour tout $x$ de norme $1$, et nous en déduisons que $\alpha(x)=0$ pour tout $x$.
		\item
			Nous avons
			\begin{equation}
				\begin{aligned}[]
					\| \lambda\alpha \|&=\sup\{ | (\lambda\alpha)(x) | \}\\
					&=\sup\{ | \lambda |\cdot| \alpha(x) | \}\\
					&=| \lambda |\sup\{ | \alpha(x) | \}\\
					&=| \lambda |\cdot\| \alpha \|.
				\end{aligned}
			\end{equation}
		\item
			Nous avons
			\begin{equation}
				\begin{aligned}[]
					\| \alpha+\beta \|&=\sup_{| x |=1}\{ | \alpha(x)+\beta(x) | \}\\
					&\leq\sup \{ | \alpha(x) |+| \beta(x) | \}\\
					&\leq \sup\{ | \alpha(x) | \}+\sup\{ | \beta(x) | \}\\
					&=\| \alpha \|+\| \beta \|.
				\end{aligned}
			\end{equation}

	\end{enumerate}
	Dans ces calculs, notez qu'on a utilisé le fait que la valeur absolue est une norme sur $\eR$.

\end{corrige}
