% This is part of Exercices et corrigés de CdI-1
% Copyright (c) 2011
%   Laurent Claessens
% See the file fdl-1.3.txt for copying conditions.

\begin{corrige}{OptimSS0005}

La fonction à minimiser est $V(x,y,z)=xyz$, qui se réduit à un problème à deux variables en utilisant la contrainte,
\begin{equation}
	V(x,y)=4xy-2x^2y-\frac{ 4 }{ 3 }xy^2,
\end{equation}
qui est une fonction usuelle de $\eR^2$ vers $\eR$ à maximiser. Les équations pour les points critiques sont
\begin{equation}
	\begin{aligned}[]
		\partial_xV&=4y-4xy-\frac{ 4 }{ 3 }y^2=0\\
		\partial_yV&=4x-2x^2-\frac{ 8 }{ 3 }xy=0.
	\end{aligned}
\end{equation}
Évidement, nous refusons toute solution avec $x$ ou $y$ nuls, donc nous pouvons simplifier la première par $y$ et la seconde par $x$, ce qui donne le système
\begin{subequations}
\begin{numcases}{}
2-2x-\frac{ 2 }{ 3 }y=0\\
2-x-\frac{ 4 }{ 3 }y=0.
\end{numcases}
\end{subequations}
La solution est $(\frac{ 2 }{ 3 },1)$. Étant donné que nous regardons $V$ sur un compact, et que le maximum n'est certainement pas sur un des bords ($V$ s'y annule), le maximum global (qui existe par compacité) doit être à l'intérieur, et donc sur un maximum local. Il ne peut donc être que $(\frac{ 2 }{ 2 },1)$.

Nous pouvons le vérifier directement:
\begin{equation}
	d^2V(x,y)=\begin{pmatrix}
	-4y	&	4-4x-\frac{ 8 }{ 3 }y	\\ 
	4-4x-\frac{ 8 }{ 3 }y	&	-\frac{ 8 }{ 3 }x,	
\end{pmatrix}
\end{equation}
donc
\begin{equation}
	d^2V(\frac{ 3 }{ 3 },1)=4\begin{pmatrix}
	-1	&	-1/9	\\ 
	-1/9	&	-2/3	
\end{pmatrix},
\end{equation}
et ses valeurs propres sont
\begin{equation}
	\lambda_{\pm}=\frac{ -15\pm\sqrt{13} }{ 18 },
\end{equation}
qui sont, effectivement, toutes deux négatives, ce qui prouve un maximum global.


\end{corrige}
