% This is part of Exercices et corrigés de CdI-1
% Copyright (c) 2011
%   Laurent Claessens
% See the file fdl-1.3.txt for copying conditions.

\begin{corrige}{OutilsMath-0091}

    Ici, il était possible de remarquer que le champ de vecteur proposé dérivait du potentiel $V(x,y,z)=x\sin(yz)$. En remarquant cela, la réponse à la première question est $V(4,1,0)-V(0,0,0)=0-0=0$. Le rotationnel est également automatiquement nul.

    Si on ne remarque pas que le champ dérive d'un potentiel, il faut un peu plus travailler.

    \begin{enumerate}
        \item
            Le chemin est $\sigma(t)=(4t,t,0)$, et la dérivée vaut $\sigma'(t)=(4,1,0)$. Nous avons aussi
            \begin{equation}
                F\big( \sigma(t) \big)=\begin{pmatrix}
                    0    \\ 
                    0    \\ 
                    4t^2    
                \end{pmatrix}.
            \end{equation}
            Nous avons donc 
            \begin{equation}
                F\big( \sigma(t) \big)\cdot \sigma'(t)=0
            \end{equation}
            et l'intégrale est donc nulle.

        \item
            Le rotationnel est donné par le déterminant
            \begin{equation}
                \nabla\times F=\begin{vmatrix}
                    e_x    &   e_y    &   e_z    \\
                    \partial_x    &   \partial_y    &   \partial_z    \\
                    \sin(yz)    &   xz\cos(yz)    &   xy\cos(yz)
                \end{vmatrix}.
            \end{equation}
            En calculant les dérivées, on remarque que tout s'annule.

        \item
            En ce qui concerne la divergence,
            \begin{equation}
                \nabla\cdot F=-xz^2\sin(yz)-xy^2\sin(yz),
            \end{equation}
            et par conséquent
            \begin{equation}
                (\nabla\cdot F)\big( \frac{1}{ \sqrt{3} },\pi,\frac{1}{ 4 } \big)=-\frac{ 1 }{2}\sqrt{\frac{ 2 }{ 3 }}\left( \frac{1}{ 16 }+\pi^2 \right).
            \end{equation}
            
    \end{enumerate}
    
    Note. Pour calculer la divergence à l'aide de Sage, on peut faire comme ceci :
    \begin{verbatim}
sage: var('x,y,z')
(x, y, z)
sage: fx=sin(y*z)
sage: fy=x*z*cos(y*z)
sage: fz=x*y*cos(y*z)
sage: divergence=symbolic_expression(fx.diff(x)+fy.diff(y)+fz.diff(z)).function(x,y,z)
sage: divergence(1/sqrt(3),pi,1/4)
-1/6*pi^2*sqrt(2)*sqrt(3) - 1/96*sqrt(2)*sqrt(3)
    \end{verbatim}

\end{corrige}
