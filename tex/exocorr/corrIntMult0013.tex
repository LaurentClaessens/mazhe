% This is part of Exercices et corrigés de CdI-1
% Copyright (c) 2011
%   Laurent Claessens
% See the file fdl-1.3.txt for copying conditions.

\begin{corrige}{IntMult0013}

	Nous passons à des coordonnées presque cylindriques :
	\begin{equation}
		\left\{
		\begin{array}{ll}
			x=r\cos(\theta)\\
			y=\sqrt{2}u\\
			z=r\sin(\theta)
		\end{array}
		\right.,
	\end{equation}
	dont le jacobien est $\sqrt{2}r$. Les bornes du domaine deviennent $0<u<\sqrt{2}$ et $r^2-u^4=1$, donc les intégrales à calculer sont
	\begin{equation}
		\int_0^{\sqrt{2}}du\int_0^{2\pi}d\theta\int_0^{\sqrt{1+u^4}}\sqrt{2}rf(r,\theta,u)dr.
	\end{equation}
	\begin{enumerate}

		\item
			L'intégrale de la fonction $1$ est très simple et donne $\frac{ 18\pi }{ 5 }$.
		\item
			La fonction $y^2=2u^2$ demande le calcul suivant :
			\begin{equation}
				\int_0^{\sqrt{2}}du\int_0^{2\pi}d\theta\int_0^{\sqrt{1+u^4}}2u^2\sqrt{2}rdr=\frac{ 152\pi }{ 21 }.
			\end{equation}

	\end{enumerate}

\end{corrige}
