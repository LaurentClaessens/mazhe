% This is part of Exercices et corrigés de CdI-1
% Copyright (c) 2011-2012
%   Laurent Claessens
% See the file fdl-1.3.txt for copying conditions.

\begin{corrige}{0032}

\begin{enumerate}
\item 
La fonction $\eR \to \eR : x \mapsto x$ est partout dérivable et partout continue. Cela a été démontré à titre d'exemple dans la partie rappels.

\item
La fonction $x \mapsto \abs x$ est continue sur $\eR$ et dérivable sur $\eR\setminus\{ 0 \}$. Commençons par montrer la non-dérivabilité en $0$.

Analysons la limite suivante
\begin{equation}
  \limite x 0 \frac{\abs x - \abs 0}{x-0} =
  \begin{arrowcases}
    \limite[x > 0] x 0 \frac{\abs x}{x} = \limite[x > 0] x 0
    \frac{x}{x}
    = \limite[x > 0] x 0 1 = 1\\
    \lim_{_{\substack{x\to 0\\x<0}}} \frac{\abs x}{x} = \limite[x < 0] x 0
    \frac{-x}{x} = \limite[x < 0] x 0 -1 = -1
  \end{arrowcases}
\end{equation}
On observe que deux restrictions de cette limite fournissent deux
valeurs différentes, ce qui montre que la limite n'existe pas ; la
fonction n'est donc pas dérivable en $0$.

Montrons la continuité en $0$, c'est-à-dire
\begin{equation*}
  \limite x 0 \abs x = \abs 0 = 0
\end{equation*}
ce qui, par définition, équivaut à
\begin{equation*}
  \forall \epsilon > 0, \exists \delta > 0 : \forall x \in \eR
  \abs{x-0}< \delta \Rightarrow \abs{\abs x - \abs 0} = \abs x < \epsilon
\end{equation*}
qui est vérifiée pour $\delta \pardef \epsilon$.

Reste à voir la dérivabilité en $x = a \neq 0$. Pour fixer les idées,
prenons $a < 0$ et choisissons une boule $B$ autour de $a$,
complètement contenue dans $\eR^-$ (c'est possible car $a < 0$). La
notion de limite étant locale, nous avons :
\begin{equation*}
  \limite x a \frac{\abs x - \abs a}{x-a} = \limite[x \in B] x a \frac{-x -
    (-a)}{x-a} = -1
\end{equation*}
où on a utilisé le fait que $a, x \in \eR^-$ pour avoir $\abs x = -x$
et $\abs a = -a$. Dès lors la dérivée en $a < 0$ existe et vaut $-1$ ;
un argument similaire montre que la dérivée en $a > 0$ existe et vaut
$1$.

\item
Voyons que la fonction, notons-la $f$, n'est pas continue (donc pas dérivable) en $0$. En effet, la limite
\begin{equation*}
  \limite[x > 0] x 0 f(x) = \limite[x > 0] x 0 \frac 1 x = +\infty
\end{equation*}
n'est pas égale à $f(0) = 0$. 

Il n'est pas nécessaire de tester la dérivabilité en zéro : si elle n'est pas continue, elle n'est pas dérivable. Pour les curieux, calcul suivant prouve néanmoins directement que la fonction n'est pas dérivable en zéro:
\begin{equation}
		\lim_{\epsilon\to 0}\frac{ f(\epsilon)-f(0) }{ \epsilon }	=\lim_{\epsilon\to0}\frac{ \epsilon\sin(\frac{1}{ \epsilon })-0 }{ \epsilon }
			=	\lim_{\epsilon\to 0}\sin(\frac{1}{ \epsilon }),
\end{equation}
qui n'est pas définie.

Par contre, $f$ est dérivable en tout point $a \in \eR\setminus\{ 0 \}$, car dans
une boule assez petite autour de $a$, $f(x) = \frac 1 x$ est
dérivable. Or la dérivabilité est une notion locale, donc $f$ est
dérivable en $a \neq 0$, et donc également continue.

\item
Montrons que $x \mapsto x^2$ est dérivable
sur $\eR^2$. En effet si $a \in \eR$, alors
\begin{equation*}
  \limite[x\neq a] x a \frac{x^2 - a^2}{x-a} = \limite[x \neq a] x a (x
  + a) = 2 a
\end{equation*}
donc la dérivée existe en tout point $a \in \eR$ et vaut $2a$.

Remarquons que cela colle avec la formule usuelle~:
\begin{equation*}
  {(x^2)}^\prime = 2 x
\end{equation*}
\end{enumerate}


\end{corrige}
