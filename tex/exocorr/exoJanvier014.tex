% This is part of the Exercices et corrigés de mathématique générale.
% Copyright (C) 2009
%   Laurent Claessens
% See the file fdl-1.3.txt for copying conditions.
\begin{exercice}\label{exoJanvier014}

On considère un fil (de fer, par exemple) de longueur \unit{10}{\centi\meter} que l'on coupe en morceaux. On utilise le premier morceau pour former un carré. Avec le second, on construit un cercle. Où faut il couper le fil pour que la somme des aires de ces deux figures géométriques soit minimales ?

\corrref{Janvier014}
\end{exercice}
