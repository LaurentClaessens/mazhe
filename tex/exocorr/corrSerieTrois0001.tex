% This is part of Exercices et corrections de MAT1151
% Copyright (C) 2010
%   Laurent Claessens
% See the file LICENCE.txt for copying conditions.

\begin{corrige}{SerieTrois0001}

	L'étude de la convergence passe par l'étude de la quantité
	\begin{equation}
		| x(d)-x_n(d_n) |
	\end{equation}
	qui devrait être petite quand $d_n$ est proche de $d$. En ajoutant et retranchant $x_n(d)$
	\begin{equation}		\label{Eqtroiszerounbaspt}
		\begin{aligned}[]
			| x(d)-x_n(d_n) |&=\big| x(d)+x_n(d)-x_n(d)-x_n(d_n) \big|\\
				&\leq \big| x(d)-x_n(d) \big|+\big| x_n(d)-x_n(d_n) \big|
		\end{aligned}
	\end{equation}
	Nous allons étudier les deux termes séparément.

	D'abord, $F_n$ est un problème stable. Cela implique que $F_n(x_n,d_n)=0$ a une unique solution $x(d)$. Or nous avons que
	\begin{equation}
		\begin{aligned}[]
			F_n(x,d_n)&=0\\
			F_n(x_n,d_n)&=0.
		\end{aligned}
	\end{equation}
	La première ligne est la forte consistance et la seconde est la définition de $x_n$ comme solution de $F_n(x_n,d_n)$. L'unicité de la solution implique donc que $x=x_n$ en tant que fonctions. Le premier terme de la somme \eqref{Eqtroiszerounbaspt} est donc nul.

	Pour le second terme, nous savons que la fonction $x_n$ est continue au point $d_n$ parce que $x_n$ est solution du problème stable $F_n$. Donc il existe un $\delta$ qui fait que $\big| x_n(d)-x_n(d_n) \big|\leq \epsilon$ lorsque $| d-d_n |\leq \delta$.
	
\end{corrige}
