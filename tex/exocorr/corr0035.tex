% This is part of Exercices et corrigés de CdI-1
% Copyright (c) 2011,2014
%   Laurent Claessens
% See the file fdl-1.3.txt for copying conditions.

\begin{corrige}{0035}


Dans ces exercices, les fonctions données sont dérivables et à dérivée
continue sur $\eR_0$ car pour $a \in \eR_0$, il existe toujours une
boule autour de $a$ dans laquelle la fonction est composée de
fonctions dérivables ($\sin$, $\frac 1x$, $\ldots$). L'intérêt de
l'exercice est donc d'établir (ou réfuter) la continuité et la
dérivabilité en $0$.

\begin{enumerate}

\item 
 Notons $f$ cette fonction. $f$ n'est pas continue en $0$ car
\begin{equation*}
  \limite[x \neq 0] x 0 f(x) = \limite[x \neq 0] x 0 0 = 0 \neq f(0)
\end{equation*}
En particulier $f$ n'est pas dérivable en $0$ (et donc la continuité
de sa dérivée n'a pas de sens en $0$).

\item
Dans ce cas-ci, la limite \og restreinte\fg{}
\begin{equation*}
  \limite[x \neq 0] x 0 f(x) = \limite[x \neq 0] x 0 \sin\left(\frac 1 x\right)
\end{equation*}
n'existe pas puisque, par exemple, pour la suite de terme général
\begin{equation*}
  x_k = \frac 1 {\frac \pi 2 + 2k \pi}
\end{equation*}
on a bien $\limite k \infty x_k = 0$ mais
\begin{equation*}
  \limite k \infty f(x_k) = 1 \neq f(0)
\end{equation*}
puisque pour tout $k \in \eN$, $f(x_k) = 1$. Donc la fonction n'est pas continue.

\item
Montrons que la fonction, notée $f$, est continue en $0$. Pour tout $x$ réel, nous avons
\begin{equation*}
  0 \leq \abs{f(x)} = \abs x \abs{\sin \left(\frac 1 x\right)} \leq
  \abs x
\end{equation*}
ce qui montre que $\limite x 0 f(x) = 0$ par la règle de l'étau.

Par ailleurs, $f$ n'est pas dérivable en $0$ car la limite
\begin{equation*}
  \limite[x \neq 0] x 0 \frac{f(x) - f(0)}{x-0} = \limite[x\neq 0] x 0 \sin\left(\frac{1}{x}\right)
\end{equation*}
n'existe pas, comme on l'a vu précédemment.

\item
Montrons que cette fonction, notée $f$, est
dérivable en $0$ (ce qui prouvera qu'elle y est continue). Calculons
\begin{equation*}
  \limite[x \neq 0] x 0 \frac{f(x)}{x} = \limite[x\neq 0] x 0 x \sin
  \left(
    \frac 1 x
  \right) = 0
\end{equation*}
où la dernière égalité a été montrée à l'exercice précédent. Nous
avons donc $f^\prime(0) = 0$.

Par ailleurs, en utilisant les règles de calcul usuelles sur les
dérivées, nous obtenons pour $x \neq 0$
\begin{equation*}
  f^\prime(x) = \sin
  \left(
    \frac 1 x
  \right) - \frac 1 x \cos
  \left(
    \frac 1 x
  \right)
\end{equation*}
qui est une fonction ne possédant pas de limite en $0$ puisque, par exemple,
si $x_k$ est tel que
\begin{equation*}
  \frac 1{x_k} = \frac\pi4 + 2k\pi
\end{equation*}
alors la suite $(x_k)_{k\in\eN}$ tend vers $0$, mais $f^\prime(x_k)$ tend vers $+\infty$ lorsque $k \rightarrow +\infty$. La dérivée n'est donc pas continue en zéro.

\end{enumerate}

\end{corrige}
