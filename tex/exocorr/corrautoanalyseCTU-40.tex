% This is part of Analyse Starter CTU
% Copyright (c) 2014
%   Laurent Claessens,Carlotta Donadello
% See the file fdl-1.3.txt for copying conditions.

\begin{corrige}{autoanalyseCTU-40}

\begin{enumerate}
\item \begin{enumerate}
\item Nous allons utiliser la formule de Taylor-Young \eqref{EqTJRooUbsyzJ}. Soit $f(x) = e^{3x}$, alors le développement cherché est 
\[
f(x) = f(0) + f'(0)x + \frac{f''(0)}{2}x^2 + x^2\alpha(x) = 1 + 3x +\frac{9}{2}x^2 +  x^2\alpha(x).
\]
\begin{remark}
  Un développement similaire a été fait dans le cours dans l'exemple \ref{compose1} en utilisant la règle de développement d'une fonction composée. 
\end{remark}
\item \[\lim_{x\to 0}\dfrac{e^{3x}-1}{x} = \lim_{x\to 0}\frac{3x +\frac{9}{2}x^2 +  x^2\alpha(x)}{x} =3.\]
\end{enumerate}
\item \begin{enumerate}
\item Ce développement limité est dans le tableau des développements limités à conna\^{i}tre. Pour le calculer on peut utiliser encore une fois la formule de Taylor-Young \eqref{EqTJRooUbsyzJ}. Soit $f(x) = \ln(1+x)$, on a
  \begin{equation*}
    \begin{aligned}
      f(x) =& f(0) + f'(0)x + \frac{f''(0)}{2}x^2 + \frac{f'''(0)}{3!} x^3 + \frac{f^{(4)}(0)}{4!} x^4 + x^4 \alpha(x)=\\
      & = 0 + x -\frac{x^2}{2} + \frac{x^3}{3} -\frac{x^4}{4} + x^4 \alpha(x).
    \end{aligned}
  \end{equation*}
\item  \[\lim_{x\to0}\frac{\ln (1+x)}{x} = \lim_{x\to0}\frac{x -\frac{x^2}{2} + \frac{x^3}{3} -\frac{x^4}{4} + x^4 \alpha(x)}{x} = 1.\]

\end{enumerate}
\end{enumerate}

\end{corrige}   
