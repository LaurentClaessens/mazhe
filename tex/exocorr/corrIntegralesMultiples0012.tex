\begin{corrige}{IntegralesMultiples0012}

 La première inégalité est vérifiée parce que la fonction $f(\xi)=e^\xi$ est positive pour tout $\xi $ dans $\eR$ et $D_R$ a une aire inférieure à l'aire de $K_R$. Les deux intégrales peuvent être considérées comme  le volumes des régions solides entre le plan $x$-$y$ et le graphe  de $e^{-x^2+y^2}$ au dessus de $D_R$ et de $K_R$ respectivement 
\begin{equation}
\iint_{D_R} \int_0^{e^{-(x^2+y^2)}} 1\, dz \, dx\, dy \leq \iint_{K_R} \int_0^{e^{-(x^2+y^2)}}1\,dz\, dx\, dy. 
\end{equation}
Les deux solides sont égaux au dessus de $D_R$ mais l'un de deux continue à l'extérieur de $D_R$ : son volume sera alors plus importante. 

La deuxième inégalité se démontre de la même façon. Il faut juste remarquer que l'exponentielle est une fonction positive et que l'aire de $K_R$ est inférieure à l'aire de $D_{2R}$.

Nous voulons maintenant résoudre la deuxième partie de l'exercice. Il faut d'abord comprendre le lien entre le trois intégrales dans l'inégalité et l'intégrale 
\[
\int_0^R e^{-t^2}\,dt.
\] 

En fait, ce dernier est la racine carrée de  l'intégrale au milieu 
\begin{equation}
 \iint_{K_R} e^{-(x^2+y^2)}\, dx\, dy =\int_0^Re^{-x^2}\,dx\,\int_0^Re^{-y^2}\,dy= \left(\int_0^Re^{-t^2}\,dt\right)^2.
\end{equation}
Les intégrales sur $D_R $ et $D_{2R}$ sont simples à calculer en coordonnées polaires 
\begin{equation}
  \iint_{D_R} e^{-(x^2+y^2)}\, dx\, dy =\int_{0}^{2\pi}\int_0^R e^{-r^2} r\, dr \, d\theta =\pi [e^{-r^2} ]_0^R.
\end{equation}

Comme les inégalités \eqref{gaussian} sont valides pour tout $R$ la limite qu'on veut calculer est coincée entre  deux limites identiques
\begin{equation}
  \lim_{R\to +\infty}\left(\pi(e^{-R^2}-1)\right)^{1/2}\leq \lim_{R\to +\infty}\int_0^Re^{-t^2}\,dt\leq \lim_{R\to +\infty}\left(\pi(e^{-4R^2}-1)\right)^{1/2}.
\end{equation}
Nous obtenons alors 
\begin{equation}
  \lim_{R\to +\infty}\int_0^Re^{-t^2}\,dt=\sqrt{\pi}.
\end{equation}
 \end{corrige}
