% This is part of the Exercices et corrigés de mathématique générale.
% Copyright (C) 2009
%   Laurent Claessens
% See the file fdl-1.3.txt for copying conditions.
\begin{corrige}{Lineraire0035}

	En posant $a=1$, le déterminant de la matrice est
	\begin{equation}
		\begin{vmatrix}
			1	&	0	&	c	\\
			2	&	1	&	3(b-1)	\\
			-3b	&	c	&	1
		\end{vmatrix}=2c^2+3c+1,
	\end{equation}
	qui s'annule lorsque $c=-1/2$ ou bien lorsque $c=-1$.

	Lorsque $c=-1$, les valeurs propres sont données par l'équation
	\begin{equation}
		\begin{vmatrix}
			1-\lambda	&	0	&	-1	\\
			2	&	1-\lambda	&	3(b-1)	\\
			-3b	&	-1	&	1-\lambda
		\end{vmatrix}=-(\lambda-3)\lambda^2=0.
	\end{equation}
	Les valeurs propres sont $\lambda_1=0$ et $\lambda-2=3$. Noter que valeur $\lambda_1$ est une racine double. Afin de trouver les vecteurs propres pour la valeur $\lambda_1=$, nous résolvons maintenant l'équation $Mv=0$ :
	\begin{equation}
		\left(\begin{array}{ccc|c}
			 1	&	0	&	-1	&	0	\\
			  2	&	1	&	3(b-1)	&	\\
			   -3b	&	-1	&	1	&		 
		   \end{array}\right)\sim
		   \left(\begin{array}{ccc|c}
			    1	&	0	&	-1	&	0	\\
			     2-3b	&	0	&	3(b-1)+1	&	\\
			      1-3b	&	-1	&	0	&	0	 
		      \end{array}\right).
	\end{equation}
	Nous voyons que la deuxième ligne est $(2-3b)$ fois la première, donc elle saute. Il reste $z=x$ et $y=(1-3b)x$, et donc l'espace engendré par le vecteur
	\begin{equation}
		\begin{pmatrix}
			1	\\ 
			1	\\ 
			1-3b	
		\end{pmatrix}.
	\end{equation}
	Nous passons maintenant à $\lambda_2=3$, c'est-à-dire à résoudre l'équation $Mv=3v$. Sous forme matricielle nous avons
	\begin{equation}
		\left(\begin{array}{ccc|c}
			 -2	&	0	&	-1	&	0	\\
			  2	&	-2	&	3(b-1)	&	0\\
			   -3b	&	-1	&	-2	&	0	 
		   \end{array}\right),
	\end{equation}
	dont la solution fournit le vecteur
	\begin{equation}
		\begin{pmatrix}
			-1/2	\\ 
			(3b-4)/2	\\ 
			1	
		\end{pmatrix}.
	\end{equation}
	Le fait qu'il n'y ait que deux vecteurs propres linéairement indépendants fait qu'il n'y a pas de bases de vecteurs propres, et donc la matrice n'est pas diagonalisable.

\end{corrige}
