% This is part of Analyse Starter CTU
% Copyright (c) 2014
%   Laurent Claessens,Carlotta Donadello
% See the file fdl-1.3.txt for copying conditions.

\begin{corrige}{starterST-0006}

\begin{enumerate}
\item On a $\Dom_f = \{x\in\eR \text{ tels que } x-1\geq 0\} = [1, +\infty[$ et  $\Dom_g = \{x\in\eR \text{ tels que } x> 0, \text{ et } \ln(x)\neq 0\} = ]0,1[\cup]1, +\infty[$.
\item Les expressions explicites de $h_1$ et $h_2$ sont 
\[
h_1 (x) = f\circ g (x) = \sqrt{\frac{1}{\ln(x)} -1},
\]
et 
\[
h_2(x)=g\circ f (x)=\frac{1}{\ln\left(\sqrt{x-1}\right)}.
\]
Leurs ensembles de définition sont 
\begin{equation*}  
  \begin{aligned}
    \Dom_{h_1} =& \left\{x\in\eR \text{ tels que }  \ln(x)\neq 0  \text{ et }\frac{1}{\ln(x)}-1\geq 0\right\}\\
    & = \left\{x\in\eR \text{ tels que } 0 <\ln(x)\leq 1\right\}  = ]1, e]. 
  \end{aligned}
\end{equation*}
et 
\begin{equation*}
  \begin{aligned}
    \Dom_{h_2} =& \left\{x\in\eR \text{ tels que }  \sqrt{x-1}> 0, \text{ et } \ln(\sqrt{x-1})\neq 0 \right\}\\
    & = \left\{x\in\eR \text{ tels que } x>1 \text{ et } x-1\neq 1\right\}  = ]1,2[\cup]2,+\infty[. 
  \end{aligned}
\end{equation*}
\item Pour trouver l'expression des fonctions dérivées de $h_1$ et $h_2$, il faut utiliser les formules dans l'encadré \ref{formulesderivation}. 
  \begin{equation*}
    h'_1(x) = -\frac{1}{2x\ln^2(x)\sqrt{\frac{1}{\ln(x)} -1}},
  \end{equation*}
  \begin{equation*}
   h'_2(x) = -  \frac{1}{2(x-1)\ln^2(\sqrt{x-1})}.
  \end{equation*}
\end{enumerate}
   


\end{corrige}


