% This is part of Exercices de mathématique pour SVT
% Copyright (c) 2010-2011
%   Laurent Claessens et Carlotta Donadello
% See the file fdl-1.3.txt for copying conditions.

\begin{corrige}{TD4-0003}

	En ce qui concerne les domaines, pour la première nous devons avoir $x\in\mathopen] -1 , 1 \mathclose[$ à cause de la racine et du dénominateur. Étant donné que la fonction tangente peut prendre n'importe quelle valeur, il n'y a pas de conditions sur la fonction $\arctan$.

	Pour le domaine de $f_2$, c'est $x\in\mathopen[ -1 , 1 \mathclose]$ parce que ce sont les seules valeurs possibles pour la fonction sinus.

	En utilisant les règles de dérivation composées et les résultats de l'exercice \ref{exoTD4-0002}, nous voyons que $f_1'(x)=f_2'(x)=\frac{1}{ \sqrt{1-x^2} }$.

	Les deux fonctions ont la même dérivée, de plus, elles ont la même valeur en $0$ parce que
	\begin{equation}
		f_1(0)=\arctan(0)=0
	\end{equation}
	tandis que
	\begin{equation}
		\arcsin(0)=0
	\end{equation}
	parce que c'est en zéro que le sinus vaut zéro.	Ce sont deux fonctions qui ont le même domaine (à part les points extrêmes) et qui sont égales en un point. Elles sont donc égales sur leur domaine commun, c'est à dire sur $\mathopen] -1 , 1 \mathclose[$.

\end{corrige}
