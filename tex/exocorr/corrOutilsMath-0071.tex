% This is part of Exercices et corrigés de CdI-1
% Copyright (c) 2011
%   Laurent Claessens
% See the file fdl-1.3.txt for copying conditions.

\begin{corrige}{OutilsMath-0071}

    Trouvons d'abord le plan passant par $(0,0,0)$ qui est perpendiculaire au vecteur donné. Ce la est $z=ax+by$. Nous cherchons donc $a$ et $b$ tels que
    \begin{equation}
        \begin{pmatrix}
            x    \\ 
            y    \\ 
            ax+by    
        \end{pmatrix}\cdot\begin{pmatrix}
            2    \\ 
            1    \\ 
            5    
        \end{pmatrix}=0
    \end{equation}
    pour tout $x$ et $y$. L'équation est 
    \begin{equation}
        (2+5a)x+(1+5b)y=0.
    \end{equation}
    Nous devons donc avoir $a=-2/5$ et $b=-1/5$.

    Maintenant nous cherchons le plan de la forme
    \begin{equation}
        z=-\frac{ 2 }{ 5 }x-\frac{1}{ 5 }y+c
    \end{equation}
    qui passe par le point $(1,4,7)$. Il faut donc
    \begin{equation}
        7=-\frac{ 2 }{ 5 }1-\frac{1}{ 5 }4+c,
    \end{equation}
    ce qui donne $c=\frac{ 41 }{ 5 }$.

\end{corrige}
