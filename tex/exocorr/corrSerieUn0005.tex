% This is part of Exercices et corrections de MAT1151
% Copyright (C) 2010, 2019
%   Laurent Claessens
% See the file LICENCE.txt for copying conditions.

\begin{corrige}{SerieUn0005}

	Le conditionnement absolu est donné par la norme de la dérivée. En ce qui concerne le \wikipedia{fr}{Conditionnement_(psychologie)}{conditionnement} relatif, nous avons 
	\begin{equation}		\label{EqSUZCKfg}
		\begin{aligned}[]
			K_{f_1\circ f_2}(d)&=| (f_1\circ f_2)'(d) |\frac{ | d | }{ | (f_1\circ f_2)(d) | }\\
			&=f_1'\big( f_2(d) \big)g'(d)\frac{ | d | }{ | (f_1\circ f_2)(d) | }
		\end{aligned}
	\end{equation}
	Mais le conditionnement de $f_1$ au point $f_2(d)$ est donné par
	\begin{equation}
		K_{f_1}\big( f_2(d) \big)=f_1'\big( f_2(d) \big)\frac{ | f_2(d) | }{ | f_1\big( f_2(d) \big) | },
	\end{equation}
	nous pouvons faire apparaitre cette expression dans \eqref{EqSUZCKfg} en multipliant et divisant par $f_2(d)$. Ainsi nous avons
	\begin{equation}
		K_{f_1\circ f_2}(d)=K_{f_1}\big( f_2(d) \big)f_2'(d)\frac{ | d | }{ f_2(d) }=K_{f_1}\big( f_2(d) \big)K_{f_2}(d).
	\end{equation}
	
	Cependant, pour cette formule nous avons besoin que les fonctions $f_1$ et $f_2$ soient telles que l'approximation du conditionnement par la dérivée fonctionne.

\end{corrige}
