% This is part of the Exercices et corrigés de mathématique générale.
% Copyright (C) 2010
%   Laurent Claessens
% See the file fdl-1.3.txt for copying conditions.

\begin{corrige}{exoMatrices-0001}

	D'abord le fait que le rang soit plus petit que l'ordre implique que le déterminant est nul, et que nous avons une valeur propre nulle. La multiplicité de cette valeur propre ne peut être que $1$. Si elle était de multiplicité deux, le rang n'aurait été que de un.

	En ce qui concerne les autres valeurs propres, on ne peut rien dire. En effet n'importe quelle matrice de la forme
	\begin{equation}
		\begin{pmatrix}
			0	&	0	&	0	\\
			0	&	a	&	0	\\
			0	&	0	&	b
		\end{pmatrix}
	\end{equation}
	avec $a$ et $b$ non nuls est dans les hypothèses de l'exercice.

\end{corrige}
