% This is part of Exercices et corrigés de CdI-1
% Copyright (c) 2011
%   Laurent Claessens
% See the file fdl-1.3.txt for copying conditions.

\begin{corrige}{OutilsMath-0035}

    Étant donné que $f$ est une fonction, $\nabla\cdot f$ et $\nabla\times f$ n'ont pas de sens. Donc seul $\nabla f$ a un sens et est un champ de vecteurs. Par conséquent, $\nabla\times(\nabla f)$ et $\nabla\cdot(\nabla f)$ ont un sens tandis que $\nabla(\nabla f)$ n'a pas de sens.

    Étant donné que $F$ est un champ de vecteurs, $\nabla F$ n'a pas de sens tandis que $\nabla\times F$ et $\nabla\cdot F$ ont un sens. Le premier est un nouveau champ de vecteurs tandis que le second est une fonction. En ce qui concerne $\nabla\times F$,
    \begin{enumerate}
        \item
            $\nabla(\nabla\times F)$, non;
        \item
            $\nabla\cdot(\nabla\times F)$, oui;
        \item
            $\nabla\times(\nabla\times F)$, oui.
    \end{enumerate}
    En ce qui concerne $\nabla\times F$ qui est une fonction,
    \begin{enumerate}
        \item
            $\nabla(\nabla\cdot F)$, oui;
        \item
            $\nabla\cdot(\nabla\cdot F)$, non;
        \item
            $\nabla\times(\nabla\cdot F)$, non.
    \end{enumerate}



\end{corrige}
