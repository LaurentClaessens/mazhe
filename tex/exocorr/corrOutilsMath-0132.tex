% This is part of Outils mathématiques
% Copyright (c) 2012
%   Laurent Claessens
% See the file fdl-1.3.txt for copying conditions.

\begin{corrige}{OutilsMath-0132}

    Ce champ de vecteurs est donné en coordonnées polaires. Pour l'intégrale, si \( \gamma\) est le chemin qui parcours le cercle,
    \begin{equation}
        \int_{\gamma}F=\int_{0J^{2\pi}}F\big( \gamma(t) \big)\cdot \gamma'(t)dt=0
    \end{equation}
    parce que le champ de vecteurs est toujours dans la direction \( e_r\) (c'est-à-dire radial) alors que \( \gamma'\) est toujours dans la direction \( e_{\theta}\) qui lui est perpendiculaire.

    En ce qui concerne le rotationnel et la divergence, il faut utiliser les formules \eqref{EqtBnoCwOM} et \eqref{EqgRxJKdOM} avec ici \( F_{\theta}=0\) et \( F_r=\frac{1}{ r^2 }\). Les résultats sont
    \begin{subequations}
        \begin{align}
            \nabla\times F&=0\\
            \nabla\cdot F&=\frac{1}{ r }\left( \frac{ \partial  }{ \partial r }\frac{1}{ r } \right)=-\frac{1}{ r^3 }.
        \end{align}
    \end{subequations}

\end{corrige}
