% This is part of Exercices et corrigés de CdI-1
% Copyright (c) 2011,2013, 2025
%   Laurent Claessens
% See the file fdl-1.3.txt for copying conditions.

\begin{corrige}{0073}

	Prenons un recouvrement par des ouverts. Parmi tous les ouverts, il y en a au moins un qui contient $x$, disons $A_1$. Cet ouvert contient une boule de rayon $r$. Prenons maintenant $K\in\eN$ tel que $n>K$ implique $| x_n-x |\leq r$. Tous les éléments de la suite à partir de $K$ (y compris la limite) sont donc contenus dans $A_1$.

	Il ne reste plus que $K$ éléments à mettre dans un nombre finis d'ouverts. Ça c'est facile : on prend un ouvert par point.

	Si cet exercice t'a plu, alors l'exercice \ref{exo0088} va surement te donner tout plein de plaisir.

\end{corrige}
