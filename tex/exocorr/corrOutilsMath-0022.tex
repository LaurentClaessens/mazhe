% This is part of Exercices et corrigés de CdI-1
% Copyright (c) 2011
%   Laurent Claessens
% See the file fdl-1.3.txt for copying conditions.

\begin{corrige}{OutilsMath-0022}

    \begin{enumerate}
        \item
            $\frac{ \partial f }{ \partial x }=-y\sin(xy)$, $\frac{ \partial f }{ \partial y }=-x\sin(y)$.
        \item
            \begin{equation}
                \begin{aligned}[]
                    \frac{ \partial \sqrt{R^2-x^2-y^2} }{ \partial x }&=\frac{ -x }{ \sqrt{R^2-x^2-y^2} }\\
                    \frac{ \partial \sqrt{R^2-x^2-y^2} }{ \partial y }&=\frac{ -y }{ \sqrt{R^2-x^2-y^2} }
                \end{aligned}
            \end{equation}
    \end{enumerate}

    \begin{verbatim}
sage: f(x,y)=cos(x*y)                                                                                                                                                                
sage: f.diff(x)
(x, y) |--> -y*sin(x*y)
sage: f.diff(y)
(x, y) |--> -x*sin(x*y)
sage: var('R')
R
sage: f(x,y)=sqrt(R^2-x^2-y^2)
sage: f.diff(x)
(x, y) |--> -x/sqrt(R^2 - x^2 - y^2)
sage: f.diff(y)
(x, y) |--> -y/sqrt(R^2 - x^2 - y^2)
    \end{verbatim}
    Notez l'utilisation de \info{f.diff(x)} pour la dérivée partielle dans la direction $x$ et \info{f.diff(y)} pour celle dans la direction $y$. 
    
    Notez aussi qu'il a fallu utiliser \info{var('R')} pour déclarer que \info{R} sera une variable.
\end{corrige}
