% This is part of (almost) Everything I know in mathematics
% Copyright (c) 2010-2017, 2019, 2021-2023
%   Laurent Claessens
% See the file fdl-1.3.txt for copying conditions.

\section{Differentiable manifolds}
%+++++++++++++++++++++++++++++++++

Most of the results about differential geometry come from \cite{kobayashi, madore, Helgason, ms_book, dgbook}.

\subsection{Definition, charts}
%------------------------------

\begin{definition}		\label{DEFooVMWRooGQYJwl}
	Let \( \mA\) be a class of functions as \(  C^{\infty}\), \( C^k\), or analytic. A $n$-dimensional \( \mA\)-\defe{manifold}{differentiable!manifold}\index{manifold} is a set $M$ and a set of maps $\{(\mU_{\alpha},\varphi_{\alpha})\}_{\alpha\in I}$ where each set $\mU_{\alpha}$ is open in $\eR^n$ and the maps $\dpt{\varphi_{\alpha}}{\mU_{\alpha}}{M}$ are injective and satisfy the three following conditions:

	\begin{enumerate}
		\item \label{ITEMooUOXXooAzRrAk}
		      every $x\in M$ is contained in at least one set $\varphi_{\alpha}(\mU_{\alpha})$,
		\item 	\label{ITEMooQHIYooBISPjL}
		      for any two charts $\dpt{\varphi_{\alpha}}{\mU_{\alpha}}{M}$ and $\dpt{\varphi_{\beta}}{\mU_{\beta}}{M}$, the set
		      \begin{equation}
			      \varphi_{\alpha}^{-1}( \varphi_{\alpha}(\mU_{\alpha})\cap\varphi_{\beta}(\mU_{\beta}) )
		      \end{equation}
		      is an open subset of $\mU_{\alpha}$,
		\item \label{ITEMooHICSooDrPwuV}
		      the map
		      \begin{equation}
			      \dpt{  (\varphi_{\beta}^{-1}\circ\varphi_{\alpha})  }{   \varphi_{\alpha}^{-1}( \varphi_{\alpha}(\mU_{\alpha})\cap\varphi_{\beta}(\mU_{\beta})  )   }{\mU_{\beta}}
		      \end{equation}
		      is in the class \( \mA\) as map from $\eR^n$ to $\eR^n$.
	\end{enumerate}
	In most of the cases, we will use \(  C^{\infty}\) or analytic manifolds. The expression ``smooth manifold'' means \(  C^{\infty}\) manifold.

	The maps \( (U_{\alpha}, \varphi_{\alpha})\) are said ``definition charts'', but the definition \ref{DEFooQLPIooPGagtz} and the proposition \ref{PROPooUDVFooEJeluM} will show that they are not really special.
\end{definition}

\begin{example}
	Any open set of $\eR^n$ is a smooth manifold if we choose the identity map as maps.
\end{example}

Most of surfaces $z=f(x,y)$ in $\eR^3$ are manifolds, depending on certain regularity conditions on~$f$.

\begin{definition}[\cite{MonCerveau}]       \label{DEFooQLPIooPGagtz}
	Let \( M\) be a manifold with its maps \(  \{ (U_{\alpha}, \varphi_{\alpha}) \}_{\alpha\in I}   \). A \( C^k\)-\defe{chart}{chart} of \( M\) is a couple \( (V,\psi)\) where \( V\) is open in \( \eR^n\) and \( \psi\colon V\to M\) is such that for every \( \alpha\in I\), the maps
	\begin{equation}
		\psi^{-1}\circ \varphi_{\alpha}\colon \varphi_{\alpha}^{-1}\big( \psi(V) \big)\to V
	\end{equation}
	and
	\begin{equation}
		\varphi_{\alpha}^{-1}\circ \psi\colon \psi^{-1}\big( \varphi_{\alpha}(U_{\alpha}) \big)\to U_{\alpha}
	\end{equation}
	are of class \( C^k\).
\end{definition}

\begin{definition}[\cite{MonCerveau}]       \label{DEFooUFHTooTXUVpN}
	Let \( \mA\) be a class of functions (\( C^k\), smooth, analytic). Let \( M,N\) be two manifolds. We consider the charts \( (U,\varphi)\) and \( (V,\phi)\) of \( M\) and \( N\). We say that a map \(f\colon M\to N \) is in the class \( \mA\) with respect to these charts if the map
	\begin{equation}
		\phi^{-1}\circ f\circ\varphi\colon \varphi^{-1}\big( \phi(V) \big)\to V
	\end{equation}
	is in the class \( \mA\).

	If \( A\) is a set of charts of \( M\) and \( B\) is a set of charts of \( N\), we say that \( f\) is in the class \( \mA\) if it is in the class \( \mA\) for every choice of charts in \( A\) and \( B\).
\end{definition}

\begin{proposition}		\label{PROPooCWPAooKDnwHR}
	If \( f\) and \( g\) are \( C^k\) functions on \( M\), then the functions \( f+g\) and \( fg\) are \( C^k\).
\end{proposition}
\noproof


\begin{definition}[\cite{MonCerveau}]       \label{DEFooMLNQooEgEfdq}
	Let \( M\) be a manifold. An \defe{atlas}{atlas} for \( M\) is a set of charts\footnote{Definition \ref{DEFooQLPIooPGagtz}.} \( \{  (U_j,\phi_j)  \}_{j\in J}\) such that \( \bigcup_{j\in J}\phi_j(U_j)=M\).
\end{definition}

\begin{normaltext}
	An atlas is a set of charts which is sufficient to reach each point of \( M\). It is however not required that every open sets in \( M\) is the image of a chart in the atlas.
\end{normaltext}

\begin{propositionDef}          \label{PROPooUFGQooACIjVL}
	Let \( A\) be an atlas\footnote{Definition \ref{DEFooMLNQooEgEfdq}.} of \( M\) and \( B\) an atlas of \( N\). If a map \( f\colon M\to N\) is in the class\footnote{Being in a class, definition \ref{DEFooUFHTooTXUVpN}.} \( \mA\) for these atlas, it is in the same class \( \mA\) for every charts of \( M\) and \( N\).

	In this case, one say that the map \( f\) is in the class \( \mA\).
\end{propositionDef}

\begin{lemma}       \label{LEMooGAMVooIWUzmy}
	Let \( (M,\{ (U_{\alpha}, \varphi_{\alpha}) \}_{\alpha\in I}) \) be a manifold. If \( (V,\psi)\) is a chart, the set
	\begin{equation}
		\psi^{-1}\big(\varphi_{\alpha}(U_{\alpha})\big)
	\end{equation}
	is open in \(\eR^n\).
\end{lemma}

\begin{proof}
	By definition \ref{DEFooQLPIooPGagtz}, the map \( \varphi_{\alpha}\circ \psi \colon V\to U_{\alpha}\) is \( C^k\) and, in particular, continuous\footnote{Definition \ref{DefOLNtrxB}\ref{ITEMooEHGWooDdITRV}.}. Thus if \( \mO\) is open in \( U_{\alpha}\), then \( (\varphi_{\alpha}^{-1}\circ\psi)^{-1}(\mO)\) is open in \( V\). Since \( V\) is open in \( \eR^n\), an open set in \( V\) is open in \( \eR^n\). The set \( U_{\alpha}\) is in particular open in \( U_{\alpha}\), thus the part
	\begin{equation}
		(\varphi_{\alpha}^{-1}\circ \psi)^{-1}(U_{\alpha})=(\psi^{-1}\circ\varphi_{\alpha})(U_{\alpha})
	\end{equation}
	is open in \( \eR^n\).
\end{proof}

\begin{proposition}[\cite{MonCerveau}]      \label{PROPooUDVFooEJeluM}
	If \( (V_1,\psi_1)\) and \( (V_2,\psi_2)\) are \( C^k\)-charts, then the map
	\begin{equation}
		\psi_2^{-1}\circ \psi_1\colon \psi_1^{-1}\big( \psi_2(V_2) \big)\to V_2
	\end{equation}
	is of class \( C^k\).
\end{proposition}

\begin{proof}
	Let \( q\in \psi_1^{-1}\big( \psi_2(V_2) \big)\). We will prove that \( \psi_2^{-1}\circ\psi_1\) is \( C^k\) on a neighbourhood of \( q\).  Let \( p=\psi_1(q)\) and \( (U_{\alpha},\varphi_{\alpha})\) be a definition chart around \( p\). We consider
	\begin{equation}
		A=\varphi_{\alpha}(U_{\alpha})\cap\psi_1(V_1)\cap\psi_2(V_2).
	\end{equation}
	The point \( p\) belongs to \( A\). We set \( U'=\varphi_{\alpha}^{-1}(A)\), \( V_1'=\psi_1^{-1}(A)\) and \( V_2'=\psi_2^{-1}(A)\). We show that
	\begin{equation}
		\psi_2^{-1}\circ\psi_1\colon V_1'\to V_2'
	\end{equation}
	is \( C^k\).

	We have
	\begin{equation}
		\psi_2^{-1}\circ\psi_1=\psi_2^{-1}\circ\varphi\circ\varphi^{-1}\circ\psi_1.
	\end{equation}
	Since \( \psi_1\) and \( \psi_2\) are charts, the maps \( \psi_2^{-1}\circ \varphi\) and \( \varphi^{-1}\circ\psi_1\) are \( C^k\), so that the compound function is \( C^k\) by theorem \ref{ThoAGXGuEt}.
\end{proof}

\begin{lemma}       \label{LEMooOPPJooXezOHS}
	Let \( M\) be a \( C^k\) manifold and \( p\in M\). There exists a chart \( (V,\psi)\) of \( M\) around \( p\) such that \( 0\in V\) and \( p=\psi(0)\).
\end{lemma}

\begin{proof}
	By definition there exists a chart \( (U_{\alpha},\varphi_{\alpha})\) around \( p\) and \( u\in U_{\alpha}\) such that \( p=\varphi_{\alpha}(u)\). Let \( V=U-u\) and \( \psi(x)=\varphi_{\alpha}(x+u)\).

	Then \( V\) is an open set and \( \psi(0)=\varphi_{\alpha}(u)=p\).
\end{proof}

\subsection{Topology}
%--------------------

\begin{propositionDef}      \label{DEFooHGNOooNqGmxE}
	Let \( M\) be a manifold. A subset $V\subset M$ is \defe{open}{topology!on manifold} if for every chart $\dpt{\varphi}{\mU}{M}$, the set $\varphi^{-1}(V\cap\varphi(\mU))$ is open in $\mU$.

	The set of open sets in \( M\) is a topology.
\end{propositionDef}

\begin{proof}
	First we prove that the open system defines a topology. For this, remark that $\varphi_{\alpha}^{-1}$ is injective (if not, there should be some multivalued points). Then $\varphi^{-1}(A\cap B)=\varphi^{-1}(A)\cap\varphi^{-1}(B)$. If $V_1$ and $V_2$ are open in $M$, then
	\begin{equation}
		\varphi^{-1}(V_1\cap V_2\cap\varphi(\mU))=\varphi^{-1}(V_1\cap\varphi(\mU))\cap\varphi^{-1}(V_2\cap\varphi(\mU))
	\end{equation}
	which is open in $\eR^n$. The same property works for the unions.
\end{proof}

\begin{theorem}     \label{THOooIAXUooDqMrav}
	Let \( M\) be a manifold. Its topology has the following properties.
	\begin{enumerate}
		\item the charts maps are continuous,
		\item the sets $\varphi_{\alpha}(\mU_{\alpha})$ are open.
	\end{enumerate}
\end{theorem}

\begin{proof}
	We proof the continuity of $\dpt{\varphi}{\mU}{M}$; for an open set $V$ in $M$, we have to show that $\varphi^{-1}(V)$ is open in $\mU\subset\eR^n$. But the definition of the topology on $M$, is precisely the fact that $\varphi^{-1}(V\cap\varphi(\mU))$ is open.
\end{proof}

\begin{lemma}       \label{LEMooGDMZooLCtnuA}
	The topology of \( \eR^n\) as manifold is the same as the usual one.
\end{lemma}

%--------------------------------------------------------------------------------------------------------------------------- 
\subsection{Regularity}
%---------------------------------------------------------------------------------------------------------------------------

\begin{definition}      \label{DEFooMELXooEkEnwz}
	If $M$ and $M$ are two analytic manifolds, a map $\dpt{\phi}{M}{N}$ is \defe{regular}{regular}\label{PgDefRegular} at $p\in M$ if it is analytic at $p$ and $\dpt{d\phi_p}{T_pM}{T_{\phi(p)}N}$ is injective.
\end{definition}

\begin{definition}[\cite{BIBooXDPUooVeTGwz}]        \label{DEFooFNTHooEwsqXB}
	Let \( M\) and \( N\) be \( C^k\)-manifolds. We say that a map \( h\colon M\to N\) is \( C^k\) if the two following conditions hold:
	\begin{enumerate}
		\item
		      The map \( h\) is continuous\footnote{With respect to the topology of the definition \ref{DEFooHGNOooNqGmxE}.}.
		\item
		      for every \( p\in M\), there exists charts \( (U,\varphi)\) around \( p\) and \( (V,\psi)\) around \( h(p)\) such that
		      \begin{enumerate}
			      \item
			            \( h\big( \varphi(U) \big)\subset\psi(V)\)
			      \item       \label{SUBITEMooXQFUooRxMVnw}
			            the map \( \psi^{-1}\circ h\circ\varphi\colon U \to V \) is \( C^k\) (definition \ref{DefPNjMGqy}).
		      \end{enumerate}
	\end{enumerate}
\end{definition}

For the sake of the following lemma, we say «\( mC^k\)» for «manifold»-\( C^k\) (definition \ref{DEFooFNTHooEwsqXB}) and «\( uC^k\)» for «usual» \( C^k\) (definition \ref{DefPNjMGqy} for normed vector spaces). The lemma will say that the two notions are the same, so that we can only say «\( C^k\)».
\begin{lemma}[\cite{MonCerveau}]
	We consider the manifolds \( M=\eR^m\) and \( N=\eR^n\) with the identify as charts. A map \( f\colon M\to N\) is \( mC^k\) if and only if it is \( uC^k\).
\end{lemma}

\begin{proof}
	For the sake of notations, we set the charts \( U=\eR^m\), \( V=\eR^n\) and \( \psi_m\colon U\to M\), \( \psi_n\colon V\to N\). The maps \( \psi_n\) and \( \psi_m\) are the identity.
	\begin{subproof}
		\spitem[\( \Rightarrow\)]
		The hypothesis that \( f\colon M\to N\) is \( mC^k\) says that the map \( \psi_n^{-1}\circ f\circ\psi_m\) is \( uC^k\). Thus the map
		\begin{equation}
			f=\psi_n\circ\psi_n^{-1}\circ f\circ\psi_m\circ\psi_m^{-1}
		\end{equation}
		is \( uC^k\) too.
		\spitem[\( \Leftarrow\)]
		Since \( f\) is \( uC^k\) and since \( \psi_n\) and \( \psi_m\) are the identity, the map \( \psi_n^{-1}\circ f\circ\psi_m\) is \( uC^k\), which means that \( f\) is \( mC^k\).
	\end{subproof}
\end{proof}

%--------------------------------------------------------------------------------------------------------------------------- 
\subsection{Product of manifolds}
%---------------------------------------------------------------------------------------------------------------------------

\begin{definition}      \label{DEFooYOLXooDPrnHa}
	TODO: Definition of \( M\times N\) if \( M\) and \( N\) are manifolds.
	%TODOooQYWUooJWOFaM
\end{definition}

%TODOooUHUNooFRaiIx Il faut prouver que la définition de produit de variété et de produit de groupe donne bien une structure de groupe de Lie au produit.

\begin{proposition}[\cite{MonCerveau}]      \label{PROPooCHVLooVFScOl}
	Let \( \mA\) be a class of functions: \( C^k\), smooth or analytic.  Let \( M\) and \( N\) be \( \mA\)-manifolds.
	\begin{enumerate}
		\item
		      The permutation
		      \begin{equation}
			      \begin{aligned}
				      \sigma\colon M\times N & \to N\times M \\
				      (p,q)                  & \mapsto (q,p)
			      \end{aligned}
		      \end{equation}
		      is in the class \( \mA\).
		\item       \label{ITEMooRFFAooRSeBPl}
		      The projection
		      \begin{equation}
			      \begin{aligned}
				      \pi_1\colon M\times N & \to M     \\
				      (p,q)                 & \mapsto p
			      \end{aligned}
		      \end{equation}
		      is in the class \( \mA\).
		\item
		      Let \( p\in M\). The inclusion map
		      \begin{equation}
			      \begin{aligned}
				      \iota\colon N & \to M\times N \\
				      q             & \mapsto (p,q)
			      \end{aligned}
		      \end{equation}
		      is in the class \( \mA\).
	\end{enumerate}
\end{proposition}

%+++++++++++++++++++++++++++++++++++++++++++++++++++++++++++++++++++++++++++++++++++++++++++++++++++++++++++++++++++++++++++ 
\section{Tangent vector}
%+++++++++++++++++++++++++++++++++++++++++++++++++++++++++++++++++++++++++++++++++++++++++++++++++++++++++++++++++++++++++++


\begin{definition}
	An open set \( \Omega\subset \eR^n\) is a manifold with the identify chart \( (\Omega, \varphi)\) where \( \phi\colon \Omega\to \Omega\) is the identity.
\end{definition}

\begin{lemma}
	Let \( \gamma\colon \eR\to M\) be a \( C^1\) path. Let \( f\colon M\to \eR\) be a \( C^1\)-function. Then the map \( f\circ \gamma\colon \eR\to \eR\) is \( C^1\).
\end{lemma}

\begin{proof}
	Let \( a\in \eR\) and \( p=\gamma(a)\). We consider a chart \( (U,\varphi)\) of \( M\) around \( p\). We decompose
	\begin{equation}
		f\circ \gamma= f\circ \varphi\circ \varphi^{-1}\circ\gamma.
	\end{equation}
	Since \( \varphi\) is a chart, the maps \( f\circ\varphi\) and \( \varphi^{-1}\circ\gamma\) are \( C^1\).
\end{proof}


\subsection{Tangent vector}
%--------------------------

\begin{definition}      \label{DEFooJJVIooDUBwDJ}
	Let \( \gamma\colon \eR\to M\) be a \( C^1\) path\footnote{The concept of map of class \( C^k\) between two manifolds is the definition \ref{DEFooFNTHooEwsqXB}.}. We consider the operator
	\begin{equation}
		\begin{aligned}
			\nabla_{\gamma}\colon C^k(M,\eR) & \to \eR                                   \\
			f                                & \mapsto \Dsdd{ (f\circ\gamma)(t) }{t}{0}.
		\end{aligned}
	\end{equation}
	We define
	\begin{equation}
		T_aM=\{ \nabla_{\gamma}\tq \gamma\in C^k(\eR,M),\gamma(0)=a \}.
	\end{equation}
	This is the \defe{tangent space}{tangent space} at \( a\).
\end{definition}

If \( \gamma\colon \eR\to M\) is a path, we also use the notation
\begin{equation}        \label{EQooJQVRooLziKoH}
	\nabla_{\gamma}=\Dsdd{ \gamma(t) }{t}{0}.
\end{equation}
If one sees \( \eR\) as a manifold, then the expression \( \Dsdd{ 2t+1 }{t}{0}\) can stand for the number \( 1\) (usual derivative of \( t\mapsto 2t+1\) at \( t=0\)) or for the operator
\begin{equation}
	\Dsdd{ 2t+1 }{t}{0}\phi=2\phi'(1).
\end{equation}

\begin{remark}      \label{REMooJQFHooQuoZxt}
	The notation \( \gamma'(0)\) for the tangent vector to the curve \( \gamma\) has to be taken with caution. In particular, \( \gamma'(0)\) is not defined by the limit
	\begin{equation}        \label{EQooVMGFooFUCNEY}
		\lim_{\epsilon\to 0} \frac{ \gamma(\epsilon)-\gamma(0) }{ \epsilon }
	\end{equation}
	because when \( M\) is a manifold, there is in general no notion of difference between the points of \( M\), so that the difference \( \gamma(\epsilon)-\gamma(0)\) has no meaning.

	The only definition of \( \gamma'(0)\) is as differential operator.
\end{remark}

\begin{lemma}[\cite{MonCerveau}]        \label{LEMooMHSQooQyTZCg}
	Let \( \gamma\colon I\to M\) be a \( C^1\) path. Let \( u\in \eR\). We consider the path
	\begin{equation}
		\begin{aligned}
			\sigma\colon I & \to M               \\
			t              & \mapsto \gamma(ut).
		\end{aligned}
	\end{equation}
	Then we have \( \sigma'(t_0)=u\gamma'(t_0)\).
\end{lemma}


%---------------------------------------------------------------------------------------------------------------------------
\subsection{Vector space structure on the tangent space}
%---------------------------------------------------------------------------------------------------------------------------

If \( X\) and \( Y \) are elements of \( T_pM\), and if \( \lambda\in \eR\) the definitions of \( \lambda X\) and \( X+Y\) are just the usual definitions:
\begin{equation}
	\begin{aligned}
		\lambda X\colon  C^{\infty}(M) & \to \eR              \\
		f                              & \mapsto \lambda X(f)
	\end{aligned}
\end{equation}
and
\begin{equation}
	\begin{aligned}
		X+Y\colon  C^{\infty}(M) & \to \eR            \\
		f                        & \mapsto X(f)+Y(f).
	\end{aligned}
\end{equation}
The real questions is: are \( \lambda X\) and \( X+Y\) elements of \( T_pM\) ?

\begin{proposition}[\( T_pM\) is  a vector space\cite{MonCerveau}]  \label{PROPooEJBWooSbvypo}
	Let \( M\) be a \( C^k\) manifold.
	\begin{enumerate}
		\item
		      The set \( T_pM\) is  a vector space
		\item
		      The dimension of \( T_pM\) is the same as the dimension of \( M\).
	\end{enumerate}
\end{proposition}

\begin{proof}
	Let \( p\in M\) and a chart \( (U,\varphi)\) of \( M\) around \( p\) such that \( p=\varphi(0)\) (lemma \ref{LEMooOPPJooXezOHS}). We consider \( X,Y\in T_pM\) and paths \( \gamma\colon \eR\to M\), \( \sigma\colon \eR\to M\) such that \( X=\nabla_{\gamma}\) and \( Y=\nabla_{\sigma}\).

	\begin{subproof}
		\spitem[Sum]


		We aim to find a path \( s\colon \eR\to M\) such that \( \nabla_s=X+Y\). We set
		\begin{equation}        \label{EQooYNAOooKkKmxo}
			\begin{aligned}
				s\colon \eR & \to M                                                                                           \\
				t           & \mapsto \varphi\Big( \varphi^{-1}\big( \gamma(t) \big)+\varphi^{-1}\big( \sigma(t) \big) \Big).
			\end{aligned}
		\end{equation}
		This path satisfy \( s(0)=\varphi\Big( \varphi^{-1}(p)+\varphi^{-1}(p) \Big)=\varphi(0)=p\). In order to prove that \( \nabla_s=X+Y\) we consider \( f\in C^k(M,\eR)\) and we compute \( \nabla_s(f)\) :
		\begin{subequations}        \label{SUBEQSooKOGNooGISCax}
			\begin{align}
				\Dsdd{ (f\circ s)(t) }{t}{0} & =\Dsdd{ (f\circ\varphi)\Big( \varphi^{-1}\big( \gamma(t) \big)+\varphi^{-1}\big( \sigma(t) \big) \Big) }{t}{0}              \\
				                             & =\sum_k\partial_k(f\circ \varphi)(0)\Dsdd{ \varphi^{-1}\big( \gamma(t) \big)_k+\varphi^{-1}\big( \sigma(t) \big)_k }{t}{0}.
			\end{align}
		\end{subequations}
		We used the theorem \ref{THOooKBTYooWFtoSF} with the maps \( f\circ \varphi\colon \eR^n\to \eR\) and
		\begin{equation}
			\begin{aligned}
				g\colon \eR & \to \eR^n                                                                    \\
				t           & \mapsto \varphi^{-1}\big( \gamma(t) \big)+\varphi^{-1}\big( \sigma(t) \big).
			\end{aligned}
		\end{equation}
		We focus on one term:
		\begin{subequations}
			\begin{align}
				\sum_k\partial_k(f\circ \varphi)(0)\Dsdd{ \varphi^{-1}\big( \gamma(t) \big)_k }{t}{0} & =\Dsdd{ (f\circ\varphi)\big( \varphi^{-1}(\gamma(t)) \big) }{t}{0} \\
				                                                                                      & =\Dsdd{ f\big( \gamma(t) \big) }{t}{0}                             \\
				                                                                                      & =\nabla_{\gamma}(f).
			\end{align}
		\end{subequations}
		The two terms of \eqref{SUBEQSooKOGNooGISCax} sum to \( \nabla_{\gamma}(f)+\nabla_{\sigma}(f)\), so that \( \nabla_s=\nabla_{\gamma}+\nabla_{\sigma}=X+Y\).

		\spitem[Product]

		We ail to find a path \( s\colon \eR\to M\) such that \( \nabla_s=\lambda X\). The answer is easy: \( s(t)=\gamma(\lambda t)\). Indeed:
		\begin{equation}
			\Dsdd{ (f\circ s)(t) }{t}{0}=\Dsdd{ (f\circ \gamma)(\lambda t) }{t}{0}=\lambda \Dsdd{ (f\circ \gamma)(t) }{t}{0}=\lambda X(f).
		\end{equation}
		We used the lemma \ref{LEMooXHVBooHYjXdq}.
	\end{subproof}
\end{proof}

\begin{proposition}[\cite{MonCerveau}]     \label{PROPooJVSQooGvNqIx}
	Let \( V\) be a \( n\)-dimensional vector space.
	\begin{enumerate}
		\item
		      \( V\) has a structure of \( n\)-dimensional manifold.
		\item
		      For every \( a\in V\), the map
		      \begin{equation}
			      \begin{aligned}
				      i\colon V & \to T_aV                    \\
				      v         & \mapsto \sum_iv_i\partial_i
			      \end{aligned}
		      \end{equation}
		      is a vector space isomorphism.
	\end{enumerate}
\end{proposition}

%--------------------------------------------------------------------------------------------------------------------------- 
\subsection{Tangent vector on a product manifold}
%---------------------------------------------------------------------------------------------------------------------------

\begin{lemma}       \label{LEMooTONEooFiysTA}
	Let \( M\) and \( N\) be \( C^k\) manifolds\footnote{The product of manifolds is defined in \ref{DEFooYOLXooDPrnHa}.}. Let \( \gamma_M\colon I\to M\) and \( \gamma_N\colon I\to N\) be \( C^k\) paths. We define
	\begin{equation}
		\begin{aligned}
			\gamma\colon I & \to M\times N                                \\
			t              & \mapsto \big( \gamma_M(t),\gamma_N(t) \big).
		\end{aligned}
	\end{equation}
	We have:
	\begin{enumerate}
		\item
		      A vector space isomorphism \( T_{(a,b)(M\times N)}\simeq T_aM\times T_bN\).
		\item
		      Under that isomorphism,
		      \begin{equation}
			      \Dsdd{ \gamma(t) }{t}{0}=\big( \gamma_M'(0),\gamma_N'(0) \big).
		      \end{equation}
	\end{enumerate}
\end{lemma}

%+++++++++++++++++++++++++++++++++++++++++++++++++++++++++++++++++++++++++++++++++++++++++++++++++++++++++++++++++++++++++++ 
\section{Differential of a map}
%+++++++++++++++++++++++++++++++++++++++++++++++++++++++++++++++++++++++++++++++++++++++++++++++++++++++++++++++++++++++++++

\begin{propositionDef}[Differential of a map\cite{MonCerveau}]      \label{DEFooDRGUooDPFIJa}
	Let \( M\), \( N\) be \( C^1\)-manifolds. Let \( a\in M\). Let \( \phi\in C^1(M,N)\). We consider two paths \( \gamma\) and \( \sigma\) such that \( \gamma(0)=\sigma(0)=a\) and
	\begin{equation}
		\nabla_{\gamma}=\nabla_{\sigma}.
	\end{equation}
	Then
	\begin{equation}
		\nabla_{\phi\circ \gamma}=\nabla_{\phi\circ \sigma}.
	\end{equation}
	We define
	\begin{equation}        \label{EQooQNZPooMVaSQC}
		\begin{aligned}
			d\phi_a\colon T_aM & \to T_{\phi(a)}N                   \\
			\nabla_{\gamma}    & \mapsto \nabla_{\phi\circ \gamma}.
		\end{aligned}
	\end{equation}
	We have the formula
	\begin{equation}        \label{EQooEWMRooFsSVpb}
		d\phi_a(\nabla_{\gamma})f=\nabla_{\phi\circ \gamma}(f)=\Dsdd{ (f\circ\phi\circ\gamma)(t) }{t}{0}
	\end{equation}
	where \( a=\gamma(0)\).
\end{propositionDef}

\begin{proof}
	Let \( f\in C^1(N,\eR)\). The map \( f\circ\phi\colon M\to \eR\) belongs to \( C^1(M,\eR)\) so that we can apply \( \nabla_{\gamma}\) and \( \nabla_{\sigma}\) on it. By hypothesis,
	\begin{equation}
		\nabla_{\gamma}(f\circ \phi)=\nabla_{\sigma}(f\circ \phi).
	\end{equation}
	Using the definition of \( \nabla\),
	\begin{equation}
		\nabla_{\gamma}(f\circ\phi)=\Dsdd{ (f\circ \phi\circ \gamma)(t) }{t}{0}=\nabla_{\phi\circ\gamma}(f).
	\end{equation}
	and
	\begin{equation}
		\nabla_{\sigma}(f\circ\phi)=\Dsdd{ (f\circ \phi\circ \sigma)(t) }{t}{0}=\nabla_{\phi\circ\sigma}(f).
	\end{equation}
\end{proof}

\begin{proposition}[\cite{MonCerveau}]      \label{PROPooALATooGgcVQV}
	Let \( M\), \( N\) be \( C^1\)-manifolds. Let \(I \) be an interval around \( 0\) in \( \eR\) and a \( C^1\) path \( \gamma\colon I\to M\). We consider a \( C^1\) map \( \phi\colon M\to N\) and a function \( f\colon N\to \eR\).

	We have the formula
	\begin{equation}        \label{EQooYHBZooUqInIC}
		\Dsdd{ (f\circ\phi)\big( \gamma(t) \big) }{t}{t_0}=d\phi_{\gamma(t_0)}\big( \gamma'(t_0) \big)f,
	\end{equation}
	or
	\begin{equation}        \label{EQooVLUIooAbGZEi}
		d\phi_{\gamma(t_0)}\big( \gamma'(t_0) \big)=\Dsdd{ (\phi\circ \gamma)(t) }{t}{t_0}.
	\end{equation}
\end{proposition}

\begin{proof}
	It's a computation. We write \( s(t)=\gamma(t_0+t)\) and
	\begin{subequations}
		\begin{align}
			\Dsdd{ (f\circ\phi)\big( \gamma(t) \big) }{t}{t_0} & =\Dsdd{ (f\circ\phi)\big( s(t) \big) }{t}{0}        \label{SUBEQooMYOQooTBvFTn} \\
			                                                   & =d\phi_{s(0)}\big( s'(0) \big)f        \label{SUBEQooNINCooIwqrdP}              \\
			                                                   & =d\phi_{\gamma(t_0)}\big( \gamma'(t_0) \big)f.
		\end{align}
	\end{subequations}
	Justifications.
	\begin{itemize}
		\item For \eqref{SUBEQooMYOQooTBvFTn}.  The left hand side of is a classical derivative of the map \( f\circ\phi\circ \gamma\colon I\to \eR\).
		\item For \eqref{SUBEQooNINCooIwqrdP}.  This is formula \eqref{EQooEWMRooFsSVpb}.
	\end{itemize}
\end{proof}

\begin{lemma}[\cite{MonCerveau}]        \label{LEMooBOZBooNJMccB}
	Let \( M,N\) be \( C^1\)-manifolds. We consider \( C^1\)-maps \( f\colon M\to N\) and \( \phi\colon N\to \eR\). Let \( \gamma\colon \eR\to M\) be a \( C^1\)-path. We have
	\begin{equation}
		\nabla_{\gamma}(\phi\circ f)=\nabla_{f\circ \gamma}(\phi).
	\end{equation}
\end{lemma}

\begin{proof}
	By definition,
	\begin{equation}
		\nabla_{\gamma}(\phi\circ f)=\Dsdd{ (\phi\circ f\circ \gamma)(t) }{t}{0}=\Dsdd{ \phi\big( (f\circ \gamma)(t) \big) }{t}{0}=\nabla_{f\circ\gamma}(\phi).
	\end{equation}
\end{proof}

Here we prove that \( df_a\) is linear. The proposition \ref{PROPooPEMLooPQcywG} will provide the reciprocal map : \( (df_a)^{-1}=(df^{-1})_{f(a)}\).
\begin{proposition}     \label{PROPooNRLVooChhiIS}
	Let \( M\) and \( N\) be \( C^1\)-manifolds. Let \( f\in C^1(M,N)\) and \( a\in M\). The map \( df_a\colon T_aM\to T_{f(a)}N\) is linear.
\end{proposition}

\begin{proof}
	We consider two tangent vectors \( \nabla_{\gamma}\) and \( \nabla_{\sigma}\) to \( M\) at \( a\). We know that there exists a path \( s\colon \eR\to M\) such that \( \nabla_{\gamma}+\nabla_{\sigma}=\nabla_s\). This is proposition \ref{PROPooEJBWooSbvypo}, see equation \eqref{EQooYNAOooKkKmxo}.

	Let \( \phi\in C^1(N,\eR)\) be a test function; we have
	\begin{subequations}
		\begin{align}
			df_a(\nabla_{\gamma}+\nabla_{\sigma})\phi & =df_a(\nabla_s)\phi  \label{EQooIMZIooQlZODR}                                           \\
			                                          & =\nabla_s(\phi\circ f)     \label{EQooXMHHooGpbAge}                                     \\
			                                          & =\nabla_{\gamma}(\phi\circ f)+\nabla_{\sigma}(\phi\circ f)                              \\
			                                          & =\nabla_{f\circ \gamma}(\phi)+\nabla_{f\circ \sigma}(\phi)     \label{EQooHMZAooArgfTN} \\
			                                          & =df_a(\nabla_{\gamma})\phi+df_a(\nabla_{\sigma})\phi.
		\end{align}
	\end{subequations}
	Justifications:
	\begin{itemize}
		\item For \eqref{EQooIMZIooQlZODR}: definition od the path \( s\).
		\item For \eqref{EQooXMHHooGpbAge}: definition \eqref{EQooQNZPooMVaSQC} of \( df_a\).
		\item For \eqref{EQooHMZAooArgfTN}: lemma \ref{LEMooBOZBooNJMccB}.
	\end{itemize}
\end{proof}


The following lemma will be generalized to vector fields in \ref{LEMooZWFAooDlYaJm}.
\begin{lemma}       \label{LEMooSCVHooYPiGse}
	Let \( \varphi\colon U\to M\) be a chart around \( a\in M\) with \( \varphi(s)=a\) (\( s\in U\)). We have \( v\in T_aM\) if and only if there exists reals numbers \( \{ v_k \}_{k=1,\ldots, n}\) such that
	\begin{equation}        \label{EQooNEDSooOhyrCZ}
		v(f)=\sum_{k=1}^nv_k\partial_k(f\circ \varphi)(s).
	\end{equation}
\end{lemma}

\begin{proof}
	Two parts.
	\begin{subproof}
		\spitem[\( \Rightarrow\)]
		Let \( \gamma\colon \eR\to M\) be a path for the vector \( v\): \( v=\nabla_{\gamma}\). By the regularity hypothesis\footnote{The manifold \( M\) is \( C^k\), the charts maps are \( C^k\) and all that.}, the maps \( f\circ \varphi\colon U\to \eR\) and \( \varphi^{-1}\circ\gamma\colon \eR \to U \) are differentiable usual maps.

		Thus, using the equalities of lemma \ref{LemdfaSurLesPartielles} and theorem \ref{THOooIHPIooIUyPaf}, we can write
		\begin{subequations}
			\begin{align}
				\nabla_{\gamma}f & =\Dsdd{ (f\circ\gamma)(t) }{t}{0}                                                          \\
				                 & =\Dsdd{ (f\circ\varphi)\circ(\varphi^{-1}\circ\gamma)(t) }{t}{0}                           \\
				                 & =d(f\circ \varphi)_{(\varphi^{-1}\circ\gamma)(0)}\big( (\varphi^{-1}\circ\gamma)'(0) \big) \\
				                 & =d(f\circ \varphi)_{s}\big( (\varphi^{-1}\circ\gamma)'(0) \big)                            \\
				                 & =\sum_k\partial_k(f\circ \varphi)(s)(\varphi^{-1}\circ\gamma)'(0)_k
			\end{align}
		\end{subequations}
		Let \( v_k=(\varphi^{-1}\circ\gamma)'(0)_k\) and we have the result.
		\spitem[$ \Leftarrow$ ]
	\end{subproof}
\end{proof}


\begin{proposition}		\label{PROPooLHMSooMMXrSS}
	Let \( M\) be a \( n\)-dimensional manifold. Let \( X\in T_pM\) be a tangent vector. There exists a unique \( v\in \eR^n\) such that
	\begin{equation}
		X=\frac{d}{dt} \left[ \varphi_{\alpha}(s_0+tv)  \right]_{t=0}
	\end{equation}
	where \( s_0=\varphi_{\alpha}^{-1}(p)\).
\end{proposition}

\noproof

\begin{lemma}       \label{LEMooZXEFooZgXbNP}
	Let \( \varphi\colon U\to M\) be a chart around \( a\in M\) with \( \varphi(s)=a\) (\( s\in U\)). Let \( v\in T_aM\). We have
	\begin{equation}
		v(f)=\sum_{k=1}^nv_k\partial_k(f\circ \varphi)(s)
	\end{equation}
	with \( v= (\varphi^{-1}\circ \gamma)'(0)\).
\end{lemma}
% This lemma is a duplicate of PROPooLHMSooMMXrSS.

\noproof

\begin{normaltext}      \label{NORMooXAJGooDNyxjv}
	If \( v\in \eR^n\) and if \( a\in \eR^n\), we can speak of \( v\in T_a\eR^n\) with the abuse of notation \( v=\nabla_{\gamma}\) where
	\begin{equation}
		\begin{aligned}
			\gamma\colon \eR & \to \eR^n     \\
			t                & \mapsto a+tv.
		\end{aligned}
	\end{equation}
	You have to keep in mind that \( v\) is an element of \( \eR^n\) (a list of numbers) while \( \nabla_{\gamma}\) is an element of \( T_a\eR^n\) (a differential operator). Writing «\( v=\nabla_{\gamma}\)» is an abuse of notation.

	The object \( \nabla_{\gamma}\) is what one could name «the vector \( v\) tied to the point \( a\)».
\end{normaltext}

This remark is formalised by the following proposition which provides a canonical isomorphism between \( \eR^n\) and \( T_a\eR^n\).

\begin{proposition}     \label{PROPooRXIIooFmhqJd}
	Let \( M\) be a \( C^k\) manifold and \( a\in M\). We consider a chart \( \varphi\colon U\to M\) around \( a\). For \( v\in \eR^n\) we define
	\begin{equation}
		\gamma_{a,v}(t)=\varphi\big( \varphi^{-1}(a)+tv \big).
	\end{equation}
	\begin{enumerate}
		\item
		      The map \( \gamma_{a,v}\colon \eR\to M\) is \( C^k\).
		\item
		      The map
		      \begin{equation}
			      \begin{aligned}
				      \psi\colon \eR^n & \to T_aM                      \\
				      v                & \mapsto \nabla_{\gamma_{a,v}}
			      \end{aligned}
		      \end{equation}
		      is a bijection.
		\item
		      We have the handy equalities
		      \begin{equation}
			      \psi(v)f=d\varphi_s(v)f=\nabla_{s,v}(f)=\Dsdd{ (f\circ\varphi)\big( \varphi^{-1}(a)+tv \big) }{t}{0}=\sum_kv_k\partial_k(f\circ\varphi)\big( \varphi^{-1}(x) \big)
		      \end{equation}
		      where \( \nabla_{s,v}\) is the operator defined in proposition \ref{PROPooRXIIooFmhqJd}.
	\end{enumerate}
\end{proposition}

\begin{proof}
	The map \( \gamma_{a,v}\) is continuous as composed of continuous maps. The set \( \eR\) is a manifold with the identity as charts. Thus the condition \ref{DEFooFNTHooEwsqXB}\ref{SUBITEMooXQFUooRxMVnw} to be checked reduces to
	\begin{equation}
		\tilde \gamma_{a,v}=\varphi^{-1}\circ\gamma_{a,v}.
	\end{equation}
	We consider the \(  C^{\infty}\) map
	\begin{equation}
		\begin{aligned}
			l\colon \eR & \to \eR^n                   \\
			t           & \mapsto \varphi^{-1}(a)+tv.
		\end{aligned}
	\end{equation}
	We have \( \tilde \gamma_{a,v}=\varphi^{-1}\circ\varphi \circ l\). Thus \( \tilde \gamma\) is \(  C^{\infty}\).

	Remain to prove that \( \psi\) is a bijection.
	\begin{subproof}
		\spitem[Injective]
		Suppose that \( \psi(v)=\psi(w)\). For every \( f\in C^k(M)\) we have \( \nabla_{a,v}(f)=\nabla_{a,w}(f)\), which means
		\begin{equation}        \label{EQooMXJWooQKpzCG}
			\Dsdd{ f\Big( \varphi\big( \varphi^{-1}(a)+tv \big) \Big) }{t}{0}=\Dsdd{ f\Big( \varphi\big( \varphi^{-1}(a)+tw \big) \Big) }{t}{0}.
		\end{equation}
		We apply this to the function \( f=\pr_k\circ\varphi^{-1}\) :
		\begin{equation}
			f\Big( \varphi\big( \varphi^{-1}(a)+tv \big) \Big)=\varphi^{-1}(a)_k+tv_k,
		\end{equation}
		so that
		\begin{equation}
			\nabla_{a,v}(f)=\Dsdd{ \varphi^{-1}(a)_k+tv_k }{t}{0}=v_k.
		\end{equation}
		The equation \eqref{EQooMXJWooQKpzCG} implies \( v_k=w_k\) for every \( k\).
		\spitem[Surjective]
		The map \( f\circ \varphi\colon U \to \eR \) is a usual \( C^k\) function. The formulas of the lemma \ref{LemdfaSurLesPartielles} are valid; in particular the ones concerning the directional derivative. We have
		\begin{subequations}
			\begin{align}
				\nabla_{a,v}(f) & =\Dsdd{ (f\circ \varphi)\big( \varphi^{-1}(a)+tv \big) }{t}{0}   \\
				                & =\partial_v(f\circ \varphi)\big( \varphi^{-1}(a) \big)           \\
				                & =\sum_k v_k\partial_k(f\circ\varphi)\big( \varphi^{-1}(a) \big).
			\end{align}
		\end{subequations}
		This is the general form \eqref{EQooNEDSooOhyrCZ} for the action of a tangent vector on \( f\).
	\end{subproof}
\end{proof}

\begin{proposition}     \label{PROPooKMCGooDEuaWz}
	Let \( \varphi\colon U\to M\) be a chart satisfying \( \varphi(s)=a\).
	\begin{enumerate}
		\item       \label{ITEMooSFUBooNXgGuu}
		      The set\footnote{We use the abuse of notation of \ref{NORMooXAJGooDNyxjv}.} \( \{ d\varphi_s(e_i) \}_{i=1,\ldots, n}\) is a basis of \( T_aM\).
		\item       \label{ITEMooPYPVooKkHrkQ}
		      For every \( X\in T_aM\), there exists an unique \( v\in \eR^n\) such that
		      \begin{equation}
			      X=d\varphi_{s}(v).
		      \end{equation}
	\end{enumerate}
\end{proposition}

\begin{proof}
	For part \ref{ITEMooSFUBooNXgGuu}, we have to prove that the vectors of the form \( d\varphi_s(e_i)\) are linearly independent and spanning \( T_aM\).
	\begin{subproof}
		\spitem[Spanning]
		Lemma \ref{LEMooSCVHooYPiGse} says that, if \( v\in T_aM\), there exist numbers \( v_k\) such that
		\begin{equation}
			v(f)=\sum_{k=1}^nv_k\partial_k(f\circ\varphi)(s)
		\end{equation}
		We consider the path
		\begin{equation}
			\begin{aligned}
				\gamma_k\colon \eR & \to U           \\
				t                  & \mapsto s+te_k.
			\end{aligned}
		\end{equation}
		Now we have
		\begin{subequations}
			\begin{align}
				v(f) & =\sum_{k=1}^nv_k\partial_k(f\circ\varphi)(s)             \\
				     & =\sum_kv_k\Dsdd{ (f\circ\varphi)(s+te_k) }{t}{0}         \\
				     & =\sum_kv_k\Dsdd{ (f\circ\varphi\circ\gamma_k)(t) }{t}{0} \\
				     & =\sum_kv_kd\varphi_{\gamma_k(0)}(\nabla_{\gamma_k})f     \\
				     & =\sum_kv_kd\varphi_s(\nabla_{\gamma_k})f                 \\
				     & =\sum_kv_kd\varphi_s(e_k).
			\end{align}
		\end{subequations}
		The last equality is the abuse of notation explained in \ref{NORMooXAJGooDNyxjv}.
		\spitem[Independent]
		We suppose that
		\begin{equation}
			\sum_{k=1}^nv_kd\varphi_s(e_k)=0
		\end{equation}
		for some numbers \( v_k\). It means that for every \( C^k\) functions \( f\colon M\to \eR\) we have
		\begin{equation}
			\sum_{k}v_k\partial_k(f\circ \varphi)(s)=0.
		\end{equation}
		Since the function
		\begin{equation}
			\begin{aligned}
				\pr_i\colon \eR^n & \to \eR     \\
				x                 & \mapsto x_i
			\end{aligned}
		\end{equation}
		is \(  C^{\infty}\), we can choose \( f=\pr_k\circ\varphi^{-1}\). Then we have
		\begin{subequations}
			\begin{align}
				0 & =\sum_kv_k\partial_k(\pr_i\circ\varphi^{-1}\circ\varphi)(s) \\
				  & =\sum_kv_k\partial(s\mapsto s_i)(s)                         \\
				  & =\sum_kv_k\delta_{ki}                                       \\
				  & =v_i.
			\end{align}
		\end{subequations}
		We conclude that \( v_i=0\) for every \( i\) and we are done.
	\end{subproof}
	Now we prove part \ref{ITEMooPYPVooKkHrkQ}. From part \ref{ITEMooSFUBooNXgGuu} there is a set of numbers \( v_i\) such that
	\begin{equation}
		X=\sum_iv_i
	\end{equation}
	We pose \( v=\sum_iv_ie_i\) and we use the linearity of \( d\varphi_s\) :
	\begin{equation}
		X=\sum_iv_id\varphi_s(e_i)=\sum_id\varphi_s(v_ie_i)=d\varphi_s\big( \sum_iv_ie_i \big)=d\varphi_s(v).
	\end{equation}
	For the unicity, let \( v,w\in \eR^n\) such that \( d\varphi_s(v)=d\varphi_s(w)\). Thus we have \( d\varphi_s(v-w)=0\) which implies \( v=w\).
\end{proof}

\begin{lemma}[\cite{MonCerveau}]
	Let \( a\) and \( b\) be different points in the manifold \( M\). Then
	\begin{equation}
		T_aM\cap T_bM=\{ 0 \}.
	\end{equation}
\end{lemma}

\begin{proof}
	The operator which maps every function to zero belongs to \( T_aM\) and \( T_bM\). This is easy. The tricky part is the contrary. Let \( v\in T_aM\) and \( w\in T_bM\) both non zero. We consider charts \( \varphi\colon U\to M\) and \( \psi\colon V\to M\) such that
	\begin{itemize}
		\item \( U\cap V=\emptyset\)
		\item \( \varphi(U)\cap \psi(V)=\emptyset\)
		\item \( a=\varphi(s)\) and \( b=\psi(t)\) with \( s\in U\) and \( t\in V\).
	\end{itemize}
	Let \( \phi\in C^k(M,\eR)\). By lemma \ref{LEMooSCVHooYPiGse} we have
	\begin{equation}        \label{EQooTQKZooIeQNaU}
		v(\phi)=\sum_kv_k\partial_k(\phi\circ \varphi)(s)
	\end{equation}
	and
	\begin{equation}
		w(\phi)=\sum_kw_k\partial_k(\phi\circ \varphi)(t).
	\end{equation}
	The trick now is to build a function \( \phi\) for which \( v(\phi)\neq w(\phi)\).

	Let \( r>0\) such that \( \overline{ B(s,r) }\subset U\). We use the Urysohn lemma \ref{PROPooBOZIooAhKbPs} to create a function \( p\colon U\to \eR\) such that
	\begin{itemize}
		\item \( p=1\) on a neighbourhood of \( s\),
		\item \( p=0\) outside \( \overline{ B(s,r) }\),
		\item \( p\in  C^{\infty}(U)\).
	\end{itemize}
	We also consider a \(  C^{\infty}\) function \( q\colon U\to \eR\) such that \( (\partial_kq)(s)=\alpha_k\) for some numbers \( \alpha_k\) to be fixed later.

	Finally we build
	\begin{equation}
		\begin{aligned}
			\phi\colon M & \to \eR                                                                       \\
			x            & \mapsto \begin{cases}
				                       (pq)\big( \varphi^{-1}(x) \big) & \text{if }  x\in \varphi(U) \\
				                       0                               & \text{otherwise. }
			                       \end{cases}
		\end{aligned}
	\end{equation}
	This function is \( C^k\). Indeed there are two possibilities : \( x\in \varphi(U)\) or \( x\notin\varphi(U)\). In the first case \( \phi\) is the composition of \( pq\) (which is \(  C^{\infty}\)) with \( \varphi^{-1}\) which is \( C^k\). If \( x\notin\varphi(U)\), then it is in particular outside \( \varphi\big( \overline{ B(s,r) } \big)\) which is closed.

	The set of points outside of \( \varphi\big( \overline{ B(s,r) } \big)\) is open. Thus there is a neighbourhood of \( x\) which does not intersect \( \varphi\big( \overline{ B(r,s) } \big)\). The function \( \phi\) is zero in this neighbourhood, so that \( \phi\) is \(  C^k\).

	We can compute the values of \( v(\phi)\) and \( w(\phi)\). The easiest if \( w(\phi)=0\) because \( \phi=0\) on a neighbourhood of \( b\). For \( v(\phi)\) we use the formula \ref{EQooTQKZooIeQNaU}. We have
	\begin{equation}
		\partial_k(\phi\circ\varphi)(s)=\partial_k(q)(s)=\alpha_k
	\end{equation}
	because \( p=1\) on a neighbourhood of \( \varphi^{-1}(a)\). Thus we have
	\begin{equation}
		v(\phi)=\sum_kv_k\alpha_k.
	\end{equation}
	We can choose the \( \alpha_k\) in such a way that \( v(\phi)\neq 0\) because \( v\neq 0\).
\end{proof}

\begin{proposition}       \label{PROPooLJYEooMjevio}
	Let $M$ be a submanifold of the manifold $N$. If $p\in M$, then there  exists a coordinate system $\{x_1,\ldots,x_n\}$ on a neighbourhood of $p$ in $N$ such that $x_1(p)=\ldots=x_n(p)=0$ and such that the set
	\[
		U=\{q\in V\tq x_j(q)=0\,\forall\, m+1\leq j\leq n\}
	\]
	gives a local chart of $M$ containing $p$.
\end{proposition}

The sense of this proposition is that one can put $p$ at the center of a coordinate system on $N$ such that $M$ is just a submanifold of $N$ parametrised by the fact that its last $m-n$ components are zero.

\begin{lemma}\label{lem:var_cont_diff}
	Let $V,M$ be two manifolds and $\varphi\colon V\to M$, a differentiable map. We suppose that $\varphi(V)$ is contained in a submanifold $S$ of $M$. If $\dpt{\varphi}{V}{S}$ is continuous\footnote{This hypothesis states the continuity for the topology of \( S\), which is different from the continuity with respect to the topology of \( M\).}, then it is differentiable.
\end{lemma}

\begin{proof}
	Let $p\in V$. By proposition~\ref{PROPooLJYEooMjevio}, we have  a coordinate system $\{x_1,\ldots,x_m\}$ valid on a neighbourhood $N$ of $\varphi(p)$ in $M$ such that the set
	\[
		\{r\in N\tq x_j(r)=0\, \forall s<j\leq m  \}
	\]
	with the restriction of $(x_1,\ldots x_s)\in N_S$ form a local chart which contains $\varphi(p)$. From the continuity of $\varphi$, there exists a chart $(W,\psi)$ around $p$ such that $\varphi(W)\subset N_S$. The coordinates $x_j(\varphi(q))$ are differentiable functions of  the coordinates of $q$ in $W$. In particular, the coordinates $x_j(\varphi(q))$ for $1\leq j\leq s$ are differentiable and $\dpt{\varphi}{V}{S}$ is differentiable because its expression in a chart is differentiable.
\end{proof}

A consequence of this lemma: if $V$ and $S$ are submanifolds of $M$ with $V\subset S$, and if $S$ has the induced topology from $M$, then $V$ is a submanifold of $S$. Indeed, we can consider the inclusion $\dpt{\iota}{V}{S}$: it is differentiable from $V$ to $M$ and continuous from $V$ to $S$ then it is differentiable from $V$ to $S$ by the lemma. Thus $V=\iota^{-1}(S)$ is a submanifold of $S$ (this is a classical result of differential geometry).

%--------------------------------------------------------------------------------------------------------------------------- 
\subsection{Chain rule and inverse}
%---------------------------------------------------------------------------------------------------------------------------

\begin{lemma}       \label{LEMooEGITooXbAPDe}
	If \( M\) is a \( C^k\) manifold, and if \( \id\colon M\to M\) is the identity map, we have
	\begin{equation}
		d\id_a=\id_{T_aM}.
	\end{equation}
	In other words, the differential of the identity map is the identity.
\end{lemma}

\begin{lemma}[Chain rule\cite{BIBooJMRFooTAhhcg}]       \label{LEMooGRRAooXxDMuw}
	Let \( M_i\) be \( C^k\) manifolds. If the maps \( g\colon M_1\to M_2\) and \( f\colon M_2\to M_3 \) are \( C^k\), thus the composition \( f\circ g\) is \( C^k\) and for every \( a\in M_1\) we have
	\begin{equation}
		d(f\circ g)_a=df_{g(a)}\circ dg_a.
	\end{equation}
\end{lemma}


\begin{proposition}[\cite{BIBooJMRFooTAhhcg}]       \label{PROPooPEMLooPQcywG}
	Let \( f\colon M\to N\) be a diffeomorphism between the \( C^k\) manifolds \( M\) and \( N\). For every \( a\in M\), the map \( df_a\colon T_aM\to T_{f(a)}N\) is a vector space isomorphism and the inverse is given by
	\begin{equation}
		(df_a)^{-1}=(df^{-1})_{f(a)}.
	\end{equation}
\end{proposition}

\begin{proof}
	The linearity of \( df_a\) is the proposition \ref{PROPooNRLVooChhiIS}. Since \( f\) is a diffeomorphism we have the equality \( f^{-1}\circ f=\id_M\). Using the chain rule of lemma \ref{LEMooGRRAooXxDMuw} and the differential of the identity of lemma \ref{LEMooEGITooXbAPDe}, we get
	\begin{equation}
		(df^{-1})_{f(a)}\circ df_a=\id.
	\end{equation}
	The same with \( f\circ f^{-1}=\id\) provides
	\begin{equation}
		df_a\circ(df^{-1})_{f(a)}=\id.
	\end{equation}
	This proves that \( df_a\) is invertible and that its inverse is \( (df^{-1})_{f(a)}\).
\end{proof}

%--------------------------------------------------------------------------------------------------------------------------- 
\subsection{Topology on a tangent space}
%---------------------------------------------------------------------------------------------------------------------------

\begin{propositionDef}[\cite{BIBooDLJSooYWJiIT}]        \label{PROPooHJOXooMGANfd}
	Let \( M\) be a \( C^k\) manifold. Let \( a\in M\) and \( \varphi\colon U\to M\) be a chart around \( a\). We define \( s=\varphi^{-1}(a)\). For \( v\in T_aM\) we define
	\begin{equation}
		\| v \|_{T_aM}=\| (d\varphi_s)^{-1}(v) \|_{\eR^n}.
	\end{equation}
	This is a norm on the vector space \( T_aM\).

	The topology on \( T_aM\) is the one induced by this norm\footnote{Keep in mind that, since \( T_aM\) is finite dimensional, all the norm are equivalent (theorem \ref{ThoNormesEquiv}), so that this norm is not special.}.
\end{propositionDef}

\begin{proof}
	Several points.
	\begin{subproof}
		\spitem[\( \| v \|\geq 0\)]
		From the very definition, yes.
		\spitem[\( \| v \|=0\) si et seulement si \( v=0\)]
		If \( \| v \|=0\), then \( \| (d\varphi_s)^{-1}(v) \|_{\eR^n}=0\). Since the norm on \( \eR^n\) is a norm, this implies \( (d\varphi_s)^{-1}(v)=0\). And since \( d\varphi_s\) is a linear bijection, we conclude \( v=0\).
		\spitem[\( \| \lambda v \|=| \lambda |\| v \|\)]
		Because \( d\varphi_s^{-1}\) is linear.
		\spitem[\( \| v+w \|\leq \| v \|+\| v \|\)]
		Because \( d\varphi_s^{-1}\) is linear and the corresponding property on \( \eR^n\).
	\end{subproof}
\end{proof}

Now we are allowed to write \( \| v \|\) when \( v\in T_aM\). But we have to keep in mind that it depends on the choice of a local chart.

%+++++++++++++++++++++++++++++++++++++++++++++++++++++++++++++++++++++++++++++++++++++++++++++++++++++++++++++++++++++++++++ 
\section{Vector field}
%+++++++++++++++++++++++++++++++++++++++++++++++++++++++++++++++++++++++++++++++++++++++++++++++++++++++++++++++++++++++++++

\begin{lemma}       \label{LEMooXFNQooXwCMNB}
	Let \( M\) be a manifold and \( (U,\varphi)\) be a chart and \( \gamma\) be a path. We set \( a=\varphi^{-1}\big( \gamma(0) \big)\), \(v= (\varphi^{-1}\circ\gamma)'(0)\) and
	\begin{equation}
		\begin{aligned}
			\sigma\colon \eR & \to M                  \\
			t                & \mapsto \varphi(a+tv).
		\end{aligned}
	\end{equation}
	Then we have \( \nabla_{\gamma}=\nabla_{\sigma}\).
\end{lemma}

\begin{proof}
	Let \( \phi\in C^k(M, \eR)\).
	\begin{subequations}
		\begin{align}
			\nabla_{\sigma}(\phi) & =\Dsdd{ \phi\big( \sigma(t) \big) }{t}{0}                                                          \\
			                      & =\Dsdd{ (\phi\circ \varphi)\big( a+t(\varphi^{-1}\circ\gamma)'(0) \big) }{t}{0}                    \\
			                      & =\sum_k\partial_k(\phi\circ\varphi)(a)(\varphi^{-1}\circ\gamma)'(0)_k  \label{SUBEQooFDVFooFzACbX} \\
			                      & =\Dsdd{ (\phi\circ\varphi)\big( (\varphi^{-1}\circ\gamma)(t) \big) }{t}{0}                         \\
			                      & =\Dsdd{ (\phi\circ\varphi\circ\varphi^{-1}\circ\gamma)(t) }{t}{0}                                  \\
			                      & =\Dsdd{ (\phi\circ\gamma)(t) }{t}{0}                                                               \\
			                      & =\nabla_{\gamma}(\phi).
		\end{align}
	\end{subequations}
	For \eqref{SUBEQooFDVFooFzACbX}: the maps \( \phi\circ\varphi\colon U\to \eR\) and \( \varphi^{-1}\circ\gamma\colon \eR\to U\) are \( C^k\) maps, so that we can apply the formula \eqref{EQooZMAUooIusxgD}:
\end{proof}

\begin{lemma}       \label{LEMooGPCBooXMTddG}
	Let \( M\) be a manifold, \( (U,\varphi)\) be a chart, \( a\in U\) and \( v,w\in \eR^n\). We define \( \gamma_v(t)=\varphi(a+tv)\) and \( \gamma_w(t)=\varphi(a+tw)\). If \( v\neq w\) then
	\begin{equation}
		\nabla_{\gamma_v}\neq \nabla_{\gamma_w}.
	\end{equation}
\end{lemma}

\begin{proof}
	We suppose \( v\neq w\). If \( w\) is a multiple of \( v\), an adaptation of the «product» part of proposition \ref{PROPooEJBWooSbvypo} shows that \( \nabla_{\gamma_v}\neq \nabla_{\gamma_w}\).

	Of \( v\) and \( w\) are nor aligned, we consider a basis \( \{ e_1,\ldots, e_n \}\) of \( \eR^n\) such that \( e_1=v\) and \( e_2=w\) (theorem \ref{THOooOQLQooHqEeDK}). Now we consider the function
	\begin{equation}
		\begin{aligned}
			f\colon \eR^n & \to \eR          \\
			x             & \mapsto (x-a)_1.
		\end{aligned}
	\end{equation}
	We will show that the result of \( \nabla_{\gamma_v}\) and \( \nabla_{\gamma_w}\) on the function \( f\circ\varphi^{-1}\) are not equal. First we have
	\begin{subequations}
		\begin{align}
			\nabla_{\gamma_v}(f\circ\varphi^{-1}) & =\Dsdd{ (f\circ\varphi^{-1})\big( \gamma_v(t) \big) }{t}{0} \\
			                                      & =\Dsdd{ (f\circ\varphi^{-1}\circ\varphi)(a+tv) }{t}{0}      \\
			                                      & =\Dsdd{ f(a+tv) }{t}{0}                                     \\
			                                      & =\Dsdd{ (tv)_1 }{t}{0}                                      \\
			                                      & =\Dsdd{ t }{t}{0}                                           \\
			                                      & =1.
		\end{align}
	\end{subequations}
	In the same way we get
	\begin{equation}
		\nabla_{\gamma_w}(f\circ\varphi^{-1})=\Dsdd{ f(a+tw) }{t}{0}=\Dsdd{ (tw)_1 }{t}{0}=\Dsdd{ 0 }{t}{0}=0.
	\end{equation}
\end{proof}


\begin{proposition}     \label{PROPooMEPPooRonxuh}
	Let \( M\) be a manifold, \( (U, \varphi)\) a chart. If \( s\in U\) the map\footnote{We use the notation \eqref{EQooJQVRooLziKoH}.}
	\begin{equation}        \label{EQooJMTJooZNzREy}
		\begin{aligned}
			\psi\colon \eR^n & \to T_{\varphi(s)}M                  \\
			v                & \mapsto \Dsdd{ \varphi(s+tv) }{t}{0}
		\end{aligned}
	\end{equation}
	is a vector space isomorphism.
\end{proposition}

\begin{proof}
	We need to prove that \( \psi\) is surjective, injective and linear.
	\begin{subproof}
		\spitem[Surjective] Lemma \ref{LEMooXFNQooXwCMNB}.
		\spitem[Injective] Lemme \ref{LEMooGPCBooXMTddG}.
		\spitem[Linear]
		Let \( v,w\in \eR^n\). If set \( \sigma(t)=s+t(v+w)\), we have
		\begin{subequations}
			\begin{align}
				\psi(v+w)\phi & =\Dsdd{ (\phi\circ \varphi)\big( s+t(v+w) \big) }{t}{0}                                                                \\
				              & =\Dsdd{ (\phi\circ \varphi)\circ \sigma(t) }{t}{0}                                                                     \\
				              & =\sum_k\partial_k(\phi\circ\varphi)\big( \sigma(0) \big)\sigma'(0)_k                                                   \\
				              & =\sum_k\partial_k(\phi\circ\varphi)\big( \sigma(0) \big)(v_k+w_k)                                                      \\
				              & =\sum_k\partial_k(\phi\circ\varphi)\big( \sigma(0) \big)v_k+\sum_k\partial_k(\phi\circ\varphi)\big( \sigma(0) \big)w_k \\
				              & =\Dsdd{ (\phi\circ\varphi)\big( \sigma(0)+tv \big) }{t}{0}+\Dsdd{ (\phi\circ\varphi)\big( \sigma(0)+tw \big) }{t}{0}   \\
				              & =\psi(v)\phi+\psi(w)\phi.
			\end{align}
		\end{subequations}
		In the same way, if \( \lambda\in \eR\) we set \( \sigma(t)=s+t\lambda v\) and we have
		\begin{subequations}
			\begin{align}
				\psi(\lambda v)\phi & =\Dsdd{ (\phi\circ\varphi)(s+t\lambda v) }{t}{0}        \\
				                    & =\Dsdd{ (\phi\circ\varphi)\big( \sigma(t) \big) }{t}{0} \\
				                    & =d(\phi\circ\varphi)_{\sigma(0)}\sigma'(0)              \\
				                    & =\lambda d(\phi\circ\varphi)_{\sigma(0)}(v)             \\
				                    & =\lambda\psi(v)\phi.
			\end{align}
		\end{subequations}
	\end{subproof}
\end{proof}

\begin{proposition}		\label{PROPooXURIooYPytwa}
	Let \( M\) be a \( C^k\) manifold, and \( \varphi\colon U\to M\) be a local chart. A map \( X\colon M\to TM\) is a \( C^k\) vector field on \( \varphi(U)\) if and only it can be written under the form
	\begin{equation}
		X_x=\sum_{i=1}^nX_i(x)\partial_i
	\end{equation}
	for some \( C^k\) maps \( X_i\colon M\to \eR\).
\end{proposition}

\begin{lemma}[\cite{MonCerveau}]        \label{LEMooZWFAooDlYaJm}
	Let \( M\) be a \( C^k\) manifold, and \( \varphi\colon U\to M\) be a local chart. A map \( X\colon M\to TM\) is a \( C^k\) vector field on \( \varphi(U)\) if and only it can be written under the form
	\begin{equation}
		X_x(f)=\sum_{k=1}^nv_k(x)\partial_k(f\circ\varphi)\big( \varphi^{-1}(x) \big)
	\end{equation}
	for some \( C^k\) maps \( v_k\colon M\to \eR\).
\end{lemma}

\begin{lemma}       \label{LEMooIQZWooOSLNXB}
	If \( x\in M\) and \( v\in \eR^n\). The isomorphism \( \psi\) of proposition \ref{PROPooMEPPooRonxuh} satisfies
	\begin{equation}        \label{EQooBVOBooBTfYWC}
		\psi(v)f=\sum_kv_k\partial_k(f\circ\varphi)(s)
	\end{equation}
	where \( s=\varphi^{-1}(x)\).
\end{lemma}

\begin{proof}
	We use the linearity of proposition \ref{PROPooMEPPooRonxuh}:
	\begin{subequations}
		\begin{align}
			\psi(v)f & =\sum_kv_k\psi(e_k)f                                  \\
			         & =\sum_kv_k\Dsdd{ f\big( \varphi(s+te_k) \big) }{t}{0} \\
			         & =\sum_kv_k\Dsdd{ (f\circ\varphi)(s+te_k) }{t}{0}      \\
			         & =\sum_kv_k\partial_k(f\circ\varphi)(s).
		\end{align}
	\end{subequations}
\end{proof}

\begin{theorem}     \label{THOooTSQXooLvJMQb}
	Let \( M\) be a manifold. We consider the definition maps \( \{ U_{\alpha}, \varphi_{\alpha} \}\) of \( M\). Then the set \( TM\) becomes a manifold with the maps \( V_{\alpha}=U_{\alpha}\times \eR^n\) and
	\begin{equation}
		\begin{aligned}
			\psi_{\alpha}\colon U_{\alpha}\times \eR^n & \to TM                                         \\
			(x,v)                                      & \mapsto \Dsdd{ \varphi_{\alpha}(x+tv) }{t}{0}.
		\end{aligned}
	\end{equation}
\end{theorem}


%--------------------------------------------------------------------------------------------------------------------------- 
\subsection{Vector field}
%---------------------------------------------------------------------------------------------------------------------------

\begin{definition}[Vector field]        \label{DEFooAATTooLhNqDb}
	Let \( M\) be a \(  C^{\infty}\) manifold. A \defe{vector field}{vector field} is a map \( X\colon M\to TM\) such that \( X(p)\in T_pM\) for every \( p\in M\).

	We will often write \( X_p\) instead of \( X(p)\).
\end{definition}

\begin{proposition}         \label{PROPooGYWRooPIyocN}
	Let \( M\) be a \( C^{k+1}\) manifold, \( X\) be a \( C^k\) vector field and \( f\colon M \to \eR\) be a \( C^k\) function. Then the map
	\begin{equation}
		\begin{aligned}
			X(f)\colon M & \to \eR        \\
			p            & \mapsto X_p(f)
		\end{aligned}
	\end{equation}
	is \( C^{k-1}\).
\end{proposition}

\begin{lemma}       \label{LEMooLNIAooCmbLQp}
	Let \( X\) be a smooth vector field on \( M\). Let \( p\in M\) such that \( X_p\neq 0\). There exists a local chart \( \varphi\colon U\to M\) around \( p\) such that \( X=\partial_1\) in that chart. More precisely, for every \( q\in \varphi(U)\) and every smooth function \( f\) on \( M\) we have
	\begin{equation}
		X_q(f)=\partial_1(f\circ \varphi)\big( \varphi^{-1}(q) \big).
	\end{equation}
\end{lemma}


%+++++++++++++++++++++++++++++++++++++++++++++++++++++++++++++++++++++++++++++++++++++++++++++++++++++++++++++++++++++++++++ 
\section{Tangent and cotangent bundle}
%+++++++++++++++++++++++++++++++++++++++++++++++++++++++++++++++++++++++++++++++++++++++++++++++++++++++++++++++++++++++++++

If $M$ is a $n$ dimensional manifold, as set the tangent bundle\index{tangent!space} is the \emph{disjoint} union of tangent spaces
\begin{equation}
	TM=\bigcup_{x\in M}T_xM.
\end{equation}

\begin{theorem}
	The tangent bundle admits a $2n$ dimensional manifold structure for which the projection
	\begin{equation}
		\begin{aligned}
			\pi \colon TM & \to M     \\
			T_pM          & \mapsto p
		\end{aligned}
	\end{equation}
	is a submersion.
\end{theorem}

The structure is easy to guess. If $\dpt{\varphi_{\alpha}}{\mU_{\alpha}}{M}$ is a coordinate system on $M$ (with $\mU_{\alpha}\subset\eR^n$), we define $\dpt{\psi_{\alpha}}{\mU_{\alpha}\times \eR^n}{TM}$ by
\[
	\psi( \underbrace{x_1,\ldots x_n}_{\in\mU_{\alpha}},\underbrace{a_1,\ldots a_n}_{\in\eR^n}  )
	=\sum_i a_i\left.\dsd{}{x_i}\right|_{\varphi(x_1,\ldots,x_n)}.
\]
The map $\psi_{\beta}^{-1}\circ\psi_{\beta}$ is differentiable because
\[
	(\psi_{\beta}^{-1}\circ\psi_{\beta})(x,a)=( y(x),\sum_i a_i\left.\dsd{y_j}{x_i}\right|_{y(x)}  )
\]
which is a composition of differentiable maps. The set $TM$ endowed with this structure is called the \defe{tangent bundle}{tangent!bundle}.


%--------------------------------------------------------------------------------------------------------------------------- 
\subsection{Decomposition of vectors}
%---------------------------------------------------------------------------------------------------------------------------

If \( X,Y\in T_pM\) are tangent vectors, one can define \( X+Y\) and \( \lambda X\) for every \( \lambda\in\eR\). The second one is easy:
\begin{equation}
	\lambda X=\Dsdd{ X(\lambda t) }{t}{0}.
\end{equation}
In order to define the sum of two vectors one has to consider a neighbourhood \( \mU\) of \( p\) in \( M\) and a chart \( \varphi\colon \mU\to \mO\) where \( \mO\) is an open set in \( \eR^n\). Then one consider a basis \( \{ e_i \}_{1\leq i\leq n}\) of \( \eR^n\) at the point \( \varphi(p)\). With these choices we define the ``basis'' path
\begin{equation}
	\gamma_i(t)=\varphi^{-1}(te_i)
\end{equation}
and we write
\begin{equation}
	\partial_i=\frac{ \partial  }{ \partial x_i }=\Dsdd{ \varphi^{-1}(te_i) }{t}{0}.
\end{equation}
The vectors \( \partial_i\) form a basis of \( T_pM\) in the sense of the following lemma.

\begin{lemma}       \label{LEMooXDESooHXzIJU}
	The action of a vector \( X\in T_pM\) on a function \( f\colon M\to \eR\) can be decomposed into
	\begin{equation}
		Xf=\sum_{i=1}^n X_i(\partial_if)
	\end{equation}
	with \( X_i\in\eR\)
\end{lemma}

\begin{proof}
	Let \( \varphi\colon M\to \eR^n\) be a chart of a neighbourhood of \( p\) with \( \varphi(p)=0\). We determine the value of \( X_i\) using the function
	\begin{equation}
		f_i(x)=\varphi(x)_i,
	\end{equation}
	that is the \( i\)th component of the point \( \varphi(x)\in\eR^n\). Then if we write \( \varphi\big( X(t) \big)=\sum_j a_j(t)e_j\) we have
	\begin{subequations}
		\begin{align}
			X(f_i)=\Dsdd{ f_i\big( X(t) \big) }{t}{0}=\Dsdd{ \big[ \sum_ja_j(t)e_j \big]_i }{t}{0}=\Dsdd{ ai_(t) }{t}{0}=a_i'(0).
		\end{align}
	\end{subequations}
	Notice that \( a_i(0)=0\) since \( X(0)=p\) and \( \varphi(p)=0\). The combination \( f\circ\varphi^{-1}\) is an usual function from \( \eR^n\) to \( \eR\), so that we can use the chain rule on it. The following computation thus make sense:
	\begin{subequations}
		\begin{align}
			Xf & =\Dsdd{ f\big( X(t) \big) }{t}{0}                                                                                                                                                            \\
			   & =\Dsdd{ f\Big( \varphi^{-1}\varphi\big( X(t) \big) \Big) }{t}{0}                                                                                                                             \\
			   & =\Dsdd{ (f\circ\varphi^{-1})\big( \sum_ja_j(t)e_j \big) }{t}{0}                                                                                                                              \\
			   & =\sum_k \frac{ \partial (f\circ\varphi^{-1}) }{ \partial x_k }\big( \underbrace{\sum_ja_j(0)e_j}_{=\varphi(p)=0} \big)\underbrace{\frac{ d\big[ \sum_ja_j(t)e_j \big]_k  }{ dt }}_{=a'_k(0)} \\
			   & =\sum_k a'_k(0)\frac{ \partial (f\circ\varphi^{-1}) }{ \partial x_k }(0).
		\end{align}
	\end{subequations}
	Now using the definition of a derivative of a function \( \eR^n\to \eR\) and of the ``basis'' tangent vector \( \partial_k\),
	\begin{subequations}
		\begin{align}
			\frac{ \partial (f\circ\varphi^{-1}) }{ \partial x_k }(0) & =\Dsdd{ (f\circ\varphi^{-1})(te_k) }{t}{0} \\
			                                                          & =\partial_k f
		\end{align}
	\end{subequations}
	At the end of the day we have
	\begin{equation}
		Xf=\sum_k a'_k(0)\partial_kf.
	\end{equation}
\end{proof}


\begin{proposition}		\label{PROPooCGKRooLjlULU}
	This lemma allows us to define the sum in \( T_pM\) as\quext{This is not really true because we still have to prove that for every choice of \( X_i\), there exists a path \( \alpha\) such that \( \alpha'(0)=\sum_iX_i\partial_i\).}      %TODOooNHVGooLYbUkg
	\begin{equation}
		\left( \sum_kX_k\partial_k \right)+\left( \sum_kY_k\partial_k \right)=\sum_k (X_k+Y_k)\partial_k
	\end{equation}
	when \( X_k\) and \( Y_k\) are reals.

	The tangent space \( T_pM\) is thus a vector space.
\end{proposition}

\begin{proposition}		\label{PROPooAAAXooKAMsfK}
	Let \( M\) be a \( n\)-dimensional manifold and \( p=\varphi_{\alpha}(s_0)\in M\). The vectors
	\begin{equation}
		\partial_i=\frac{d}{dt} \left[ \varphi_{\alpha}(s_0+te_i)  \right]_{t=0}
	\end{equation}
	form a basis of the vector space\footnote{Proposition \ref{PROPooCGKRooLjlULU} for the vector space structure on \( T_pM\).} \( T_pM\).
\end{proposition}


\begin{lemma}       \label{LEMooVCSJooEuDZFz}
	Let \( M\) and \( N\) be smooth manifolds of dimension \( m\) and \( n\) with charts \( \varphi\colon U\to M\) and \( \psi\colon V\to N\) around \( p\in M\) and \( f(p)\in N\). We consider basis \( \{ e_i \}_{i=1,\ldots, m}\) of \( \eR^m\) and \( \{ e'_{\alpha} \}_{\alpha=1,\ldots, n}\) of \( \eR^n\).

	The matrix of \( df_p\colon T_pM\to T_{f(p)}N\) in the basis \( \{ d\varphi_{\varphi^{-1}(p)}(e_i) \}\) and \( \{ d\psi_{\psi^{-1}(f(p))}(e'_{\alpha}) \}\) is the same as the matrix of \( d(\psi^{-1}\circ f\circ\varphi)_{\varphi^{-1}(p)}\) as map from \( \eR^m\) to \( \eR^n\).
\end{lemma}

\begin{proof}
	Let subdivise.
	\begin{subproof}
		\spitem[Notations]
		As a preliminary remark, the fact that the proposed sets are basis is the proposition \ref{PROPooKMCGooDEuaWz}\ref{ITEMooSFUBooNXgGuu}. For the notations, we write
		\begin{subequations}
			\begin{align}
				\frac{ \partial  }{ \partial x_i }        & =d\varphi_{\varphi^{-1}(p)}(e_i),                \\
				\frac{ \partial  }{ \partial y_{\alpha} } & =d\psi_{\psi^{-1}\big( f(p) \big)}(e'_{\alpha}).
			\end{align}
		\end{subequations}
		\spitem[Component]
		Let \( v\in T_{f(p)}N\). We prove that
		\begin{equation}        \label{EQooISXNooJOzUmS}
			v_{\alpha}=\Big( (d\psi^{-1})_{f(p)}v \Big)_{\alpha}
		\end{equation}
		where in the left-hand side we are speaking of component with respect to the basis \( \{ \partial_{y_{\alpha}} \}\) while in the right-hand side, the ones with respect to the basis \( \{ e'_{\alpha} \}\).

		First we decompose \( v\):
		\begin{equation}
			v=\sum_{\alpha}v_{\alpha}\frac{ \partial  }{ \partial y_{\alpha} }=\sum_{\alpha}v_{\alpha}d\psi_{\psi^{-1}\big( f(p) \big)}e'_{\alpha},
		\end{equation}
		then we apply \( d\psi^{-1}_{f(p)}\) to that equation:
		\begin{equation}
			d\psi^{-1}_{f(p)}v=\sum_{\alpha}v_{\alpha}e'_{\alpha}.
		\end{equation}
		Taking the \( \alpha\)\th\ component on both side we have our result \eqref{EQooISXNooJOzUmS}.
		\spitem[Matrix]
		The matrix of a linear map is defined by the proposition \ref{PROPooGXDBooHfKRrv}. In our case,
		\begin{equation}
			(df_p)_{\alpha i}=\left( df_p\big( \frac{ \partial  }{ \partial x_i } \big) \right)_{\alpha} =\Big( df_p\circ d\varphi_{\varphi^{-1}(p)}e_i \Big)_{\alpha}.
		\end{equation}
		Using the formula \eqref{EQooISXNooJOzUmS},
		\begin{subequations}
			\begin{align}
				(df_p)_{\alpha i} & =\Big( df_p\circ d\varphi_{\varphi^{-1}(p)}e_i \Big)_{\alpha}                          \\
				                  & =\big( (d\psi^{-1})_{f(p)}\circ df_p\circ d\varphi_{\varphi^{-1}(p)}e_i \big)_{\alpha} \\
				                  & =\big( (d\psi^{-1})_{f(p)}\circ df_p\circ d\varphi_{\varphi^{-1}(p)} \big)_{\alpha i}  \\
			\end{align}
		\end{subequations}
	\end{subproof}
\end{proof}

\subsection{Commutator of vector fields}

\begin{lemma}       \label{LEMooPSWEooVKLWMQ}
	If \( X\) is a \( c^k\) vector field on the \( C^k\) manifold \( M\) and if \( f\) is a \( c^k\) function, then the formula
	\begin{equation}
		(Xf)(x)=X_x(f)
	\end{equation}
	defines a \( C^{k-1}\) function \( Xf\) on \( M\).
\end{lemma}

\begin{propositionDef}      \label{DEFooHOTOooRaPwyo}
	Let \( M\) be a \( C^k\) manifold with \( k\geq 2\). Let $X$, $Y\in\cvec(M)$.
	\begin{enumerate}
		\item       \label{ITEMooZKKUooQjYftU}
		      For every \( x\in M\), the operator \( [X,Y]_x\) defined by
		      \begin{equation}        \label{EQooDSKWooXdjPPP}
			      [X,Y]_xf=X_x(Yf)-Y_x(Xf)
		      \end{equation}
		      is an element of \( T_xM\).

		      Here, \( Yf\) and \( Xf\) are defined by virtue of lemma \ref{LEMooPSWEooVKLWMQ}.
		\item       \label{ITEMooPGPLooQrKxWY}
		      The map \( x\mapsto [X,Y]_x\) is a vector field of class \( C^{k-1}\).
	\end{enumerate}

	The so-defined vector field \( [X,Y]\) is the \defe{commutator}{commutator of vector fields} of \( X\) and \( Y\).
\end{propositionDef}

\begin{proof}
	Point by point.
	\begin{subproof}
		\spitem[\ref{ITEMooZKKUooQjYftU}]
		From lemma \ref{LEMooZWFAooDlYaJm}, we have \( C^k\) functions \( v_k\colon M\to \eR\) such that
		\begin{equation}
			X_x(f)=\sum_kv_k(x)\partial_k(f\circ \varphi)\big( \varphi^{-1}(x) \big),
		\end{equation}
		and the same for \( Y\) :
		\begin{equation}
			Y_y(f)=\sum_lw_l(y)\partial_l(f\circ\varphi)\big( \varphi^{-1}(y) \big).
		\end{equation}
		We need to compute \( X_x(Yf)\), that is
		\begin{equation}        \label{EQooAHMYooWRttQr}
			X_x(Yf)=\sum_kv_k(x)\partial_k(Yf\circ\varphi)\big( \varphi^{-1}(x) \big).
		\end{equation}
		The easy part of that is
		\begin{equation}
			(Yf\circ\varphi)(s)=(Yf)\big( \varphi(s) \big)=\sum_l(w_l\circ\varphi)(s)\partial_l(f\circ \varphi)(s).
		\end{equation}
		This is a product of two functions \( w_l\circ\varphi\colon U\to \eR\) and \( f\circ\varphi\colon U\to \eR\). For computing the partial derivative, we use the usual Leibnitz rule :
		\begin{equation}
			\partial_k(Yf\circ\varphi)(s)=\sum_{l}\partial_k(w_l\circ\varphi)(s)\partial_l(f\circ\varphi)(s)+\sum_l(w_l\circ\varphi)(s)\partial_k\partial_l(f\circ\varphi)(s).
		\end{equation}
		We can put that in \eqref{EQooAHMYooWRttQr}, with the definition \( s=\varphi^{-1}(x)\in U\subset \eR^n\) :
		\begin{subequations}
			\begin{align}
				X_x(Yf) & =\sum_kv_k(x)\sum_l\partial_k(w_l\circ\varphi)(s)\partial_l(f\circ\varphi)(s)    \\
				        & \quad +\sum_{kl}v_k(x)(w_l\circ\varphi)(s)\partial_k\partial_l(f\circ\varphi)(s) \\
				        & =\sum_{kl}v_k(x)\partial_k(w_l\circ\varphi)(s)\partial_l(f\circ\varphi)(s)       \\
				        & \quad+\sum_{kl}v_k(x)w_l(x)\partial_k\partial_l(f\circ\varphi)(s).
			\end{align}
		\end{subequations}
		The expression of \( Y_x(Xf)\) is the same, permuting \( v\) and \( w\). The commutator has \( 4\) terms :
		\begin{equation}
			\begin{aligned}[]
				[X,Y]_xf & =\sum_{kl}v_k(x)\partial_k(w_l\circ\varphi)(s)\partial_l(f\circ\varphi)(s)       \\
				         & \quad+\sum_{kl}v_k(x)w_l(x)\partial_k\partial_l(f\circ\varphi)(s)                \\
				         & \quad -\sum_{kl}w_k(x)\partial_k(v_l\circ\varphi)(s)\partial_l(f\circ\varphi)(s) \\
				         & \quad -\sum_{kl}w_k(x)v_l(x)\partial_k\partial_l(f\circ\varphi)(s).
			\end{aligned}
		\end{equation}
		By virtue of theorem \ref{Schwarz}, the two terms with second derivatives cancel out because the maps are of class \( C^k\) with \( k\geq 2\). Only two terms remain :
		\begin{equation}
			[X,Y]_xf=\sum_l\big[ \sum_kv_k(x)\partial_k(w_l\circ\varphi)(s)-w_k(x)\partial_k(v_l\circ\varphi)(s) \big].
		\end{equation}
		We pose
		\begin{equation}
			\begin{aligned}
				u_l\colon M & \to \eR                                                                                                                                               \\
				x           & \mapsto  \sum_k\big[ v_k(x)\partial_k(w_l\circ\varphi)\big( \varphi^{-1}(x) \big)-w_k(x)\partial_k(v_l\circ\varphi)\big( \varphi^{-1}(x) \big) \big].
			\end{aligned}
		\end{equation}
		For each \( x\) we have
		\begin{equation}        \label{EQooVUOKooAyGoae}
			[X,Y]_x(f)=\sum_lu_l(x)\partial_l(f\circ\varphi)(s).
		\end{equation}
		By lemma \ref{LEMooSCVHooYPiGse}, this means that \( [X,Y]_x\in T_xM\).

		\spitem[\ref{ITEMooPGPLooQrKxWY}]

		Now the functions \( u_l\) are of of class \( C^{k-1}\) (theorem \ref{THOooPZTAooTASBhZ}) which satisfies, so that the lemma \ref{LEMooPSWEooVKLWMQ} makes \( [X,Y] \) a vector field of class \( C^{k-1}\).
	\end{subproof}
\end{proof}

\begin{lemma}       \label{LEMooPWMUooRalWxC}
	Let \( M\) be a \( C^k\) manifold. Let \( \varphi\colon U\to M\) be a chart around \( a\in M\). Let \( \psi_a\colon \eR^n\to T_aM \) be the map defined in proposition \ref{PROPooMEPPooRonxuh}.

	A map \( X\colon M\to TM\) is a \( C^k\) vector field on \( \varphi(U)\) if and only if there exist a \( C^k\) function \( v\colon M\to \eR^n\) such that
	\begin{equation}
		X=\psi\circ v
	\end{equation}
	where it is understood that \( (\psi\circ v)(x)=\psi_x\big( v(x) \big)\).
\end{lemma}


\begin{lemma}
	Let \( M\) be a smooth manifold\footnote{When one deal with commutators, the natural setting is (at least) \(  C^{\infty}\) manifolds since the bracket diminishes by \( 1\) the regularity of the vector fields.}. Let \( X,Y\) be smooth vector fields given by \( X=\psi\circ v\) and \( Y=\psi\circ w\). Then we have
	\begin{equation}
		[\psi\circ v,\psi\circ w]=\psi\circ u
	\end{equation}
	with
	\begin{equation}
		\begin{aligned}
			u_l\colon \varphi(U) & \to \eR                    \\
			x                    & \mapsto X_x(w_l)-Y_x(v_l).
		\end{aligned}
	\end{equation}
	This equation can be shorthanded into
	\begin{equation}
		u=X(w)-Y(v).
	\end{equation}
\end{lemma}

\begin{proof}
	We start from the expression \ref{EQooVUOKooAyGoae} in which we substitute the values of \eqref{LEMooZWFAooDlYaJm} :
	\begin{equation}
		u_l(x)=\sum_k\underbrace{v_k(x)\partial_k(w_l\circ\varphi)(s)}_{=X_x(w_l)}-w_k(x)\partial_k(v_l\circ\varphi)(s)=X_x(w_l)-Y_x(v_l).
	\end{equation}
\end{proof}

%+++++++++++++++++++++++++++++++++++++++++++++++++++++++++++++++++++++++++++++++++++++++++++++++++++++++++++++++++++++++++++ 
\section{Submanifold}
%+++++++++++++++++++++++++++++++++++++++++++++++++++++++++++++++++++++++++++++++++++++++++++++++++++++++++++++++++++++++++++

\begin{definition}      \label{DEFooLQHWooMOTgzq}
	If $M$ is a differentiable manifold and $S$, a subset of $M$, we say that $S$ is a \defe{submanifold}{submanifold} of dimension $k$ if $\forall\,p\in S$, there exists a chart $\dpt{\varphi}{\mU}{M}$ around $p$ such that
	\begin{equation}        \label{EQooLWQRooQJCQbA}
		\varphi^{-1}(\varphi(\mU)\cap S)=\eR^k\cap\mU:=\{(x_1,\ldots,x_k,0\ldots,0)\in\mU\}.
	\end{equation}
\end{definition}

Let \( k\leq n\) in \( \eN\). We consider the maps
\begin{equation}
	\begin{aligned}
		\proj_k\colon \eR^n & \to \eR^k                  \\
		x                   & \mapsto (x_1,\ldots, x_k).
	\end{aligned}
\end{equation}
and
\begin{equation}
	\begin{aligned}
		j\colon \eR^k & \to \eR^n                             \\
		x             & \mapsto (x_1,\ldots, x_k,0,\ldots 0).
	\end{aligned}
\end{equation}

\begin{proposition}[\cite{BIBooDUPSooZjcTHL}]
	Let \( S\) be a smooth submanifold of the smooth manifold \( M\). Let \( p\in S\). We consider a map \( \varphi\colon U\to M\) around \( p\) such that\footnote{we write the condition \ref{EQooLWQRooQJCQbA} in a more condensed way.}
	\begin{equation}
		\varphi^{-1}\big( \varphi(U)\cap S \big)=U\cap j(\eR^k).
	\end{equation}
	Let \( X=\varphi(U)\cap S\) and \( Y=(\proj_k\circ\varphi^{-1})(X)\). Then we define
	\begin{equation}
		\begin{aligned}
			\psi\colon Y & \to X                        \\
			x            & \mapsto (\varphi\circ j)(x).
		\end{aligned}
	\end{equation}
	We have:
	\begin{enumerate}
		\item
		      \( \psi\) is bijective,
		\item
		      for every chart \( \varphi_{\alpha}\colon U_{\alpha}\to M\) of \( M\), the maps
		      \begin{equation}        \label{EQooBAGFooDnpctJ}
			      \psi^{-1}\circ\varphi_{\alpha}\colon \varphi_{\alpha}^{-1}\big( \psi(Y) \big)\to Y
		      \end{equation}
		      and
		      \begin{equation}        \label{EQooKQIUooDCCczD}
			      \varphi_{\alpha}^{-1}\circ\psi\colon \psi^{-1}\big( \varphi_{\alpha}(U_{\alpha}) \big)\to U_{\alpha}
		      \end{equation}
		      are smooth.
	\end{enumerate}
\end{proposition}

\begin{proof}
	In several parts.
	\begin{subproof}
		\spitem[\( \psi\) is injective]
		Let \( a,b\in Y\subset \eR^k\) such that \( \psi(a)=\psi(b)\), that is \( (\varphi\circ j)(a)=(\varphi\circ j)(b)\). Since \( \varphi\) and \( j\) are injective, we have \( a=b\).
		\spitem[\( j\circ\proj_k=\id|_{j(\eR^k)}\)]
		If \( a\in j(\eR^k)\), we have \( a=(a_1,\ldots, a_k,0,\ldots, 0)\), so that
		\begin{equation}
			j\circ\proj_k(a)=j(a_1,\ldots, a_k)=(a_1,\ldots, a_k,0,\ldots, 0)=a.
		\end{equation}
		\spitem[\( \psi\) is surjective]
		Let \( x\in X\). By definition of \( Y\), we have \( y=\proj_k\big( \varphi^{-1}(x) \big)\in Y\). We prove that \( \psi(y)=x\). We have
		\begin{equation}        \label{EQooFFSTooCddIyX}
			\psi(y)=(\psi\circ\proj_k\circ\varphi^{-1})(x)=(\varphi\circ j\circ\proj_k\circ\varphi^{-1})(x).
		\end{equation}
		Since \( x\in X=\varphi(U)\cap S\) we have \( \varphi^{-1}(x)\subset U\cap j(\eR^k)\). We know that \( j\circ\proj_k=\id|_{j(\eR^k)}\), so that
		\begin{equation}
			(j\circ\proj_k)\big( \varphi^{-1}(x) \big)=\varphi^{-1}(x).
		\end{equation}
		We continue \eqref{EQooFFSTooCddIyX}:
		\begin{equation}
			\psi(y)=(\varphi\circ j\circ\proj_k\circ\varphi^{-1})(x)=\varphi\big( \varphi^{-1}(x) \big)=x.
		\end{equation}
		So \( \psi\) is surjective. And we proved that \( \psi\) is a bijection.

		By the way, we have a formula for the inverse of \( \psi\) :
		\begin{equation}        \label{EQooQSZUooWsAqCz}
			\psi^{-1}=\proj_k\circ\varphi^{-1}\colon X\to Y.
		\end{equation}
		\spitem[First smoothness]
		We prove that \eqref{EQooBAGFooDnpctJ} is smooth. First notice that
		\begin{equation}
			\varphi_{\alpha}\Big( \varphi_{\alpha}^{-1}\big( \psi(Y) \big) \Big)\subset\psi(Y)=X=\varphi(U)\cap S\subset \varphi(U).
		\end{equation}
		Thus it makes sense to write
		\begin{equation}        \label{EQooOTSUooXHGAZz}
			\psi^{-1}\circ\varphi_{\alpha}=\psi^{-1}\circ\varphi\circ\varphi^{-1}\circ\varphi_{\alpha}
		\end{equation}
		as maps defined on \( \varphi_{\alpha}^{-1}\big( \psi(Y) \big)\). Since \( \varphi\) and \( \varphi_{\alpha}\) are charts, the map \( \varphi^{-1}\circ\varphi_{\alpha}\) is smooth.

		We know the inverse of \( \psi\) from equation \eqref{EQooQSZUooWsAqCz}. We have
		\begin{equation}
			\psi^{-1}\circ\varphi=\proj_k\circ\varphi^{-1}\circ\varphi=\proj_k,
		\end{equation}
		which is smooth.

		Equation \eqref{EQooOTSUooXHGAZz} is now the composition of two smooth functions.
		\spitem[Second smoothness]
		We prove that \eqref{EQooKQIUooDCCczD} is smooth. We have
		\begin{equation}
			\varphi_{\alpha}^{-1}\circ\psi=\varphi_{\alpha}^{-1}\circ\varphi\circ j
		\end{equation}
		while \( \varphi_{\alpha}^{-1}\), \( \varphi\) and \( j\) are smooth.
	\end{subproof}
\end{proof}

The following proposition shows that a submanifold is a manifold for itself.
\begin{proposition}[\cite{MonCerveau}]      \label{PROPooRZIHooXIhnpq}
	Let \( S\) be a smooth submanifold of the smooth manifold \( M\). For each \( p\in S\), we consider the set \( \{ \varphi_{p,i} \}_{i\in I}\) of the charts \( \varphi_{p,i}\colon U_{p,i}\to M\) around \( p\) such that
	\begin{equation}
		\varphi_{p,i}^{-1}\big( \varphi(U_{p,i})\cap S \big)=U_{p,i}\cap j(\eR^k).
	\end{equation}
	Let \( X_{p,i}=\varphi_{p,i}(U_{p,i})\cap S\) and \( Y_{p,i}=(\proj_k\circ\varphi_{p,i}^{-1})(X_{p,i})\). Then we define
	\begin{equation}
		\begin{aligned}
			\psi_{p,i}\colon Y_{p,i} & \to X_{p,i}                        \\
			x                        & \mapsto (\varphi_{p,i}\circ j)(x).
		\end{aligned}
	\end{equation}
	The couple \( \big( S,\{ (U_{p,i},\varphi_{p,i}) \}_{i\in I} \big)\) is a manifold\footnote{Well. The index set \( I\) may depend on \( p\), but the notations are already complicated enough.}.

	When \( S\) is a submanifold, we will always consider this manifold structure on \( S\).
\end{proposition}

\begin{definition}      \label{DEFooZKUIooXWVGvh}
	Let a map $\dpt{f}{M_1}{M_2}$.
	\begin{enumerate}
		\item
		      It is an \defe{immersion}{immersion} at $p\in M_1$ if $\dpt{df_p}{T_pM_1}{T_{f(p)}M_2}$ is injective\footnote{Differential of map, definition \ref{DEFooDRGUooDPFIJa}.}.
		\item
		      It is a \defe{submersion}{submersion} if $df_p$ is surjective.
	\end{enumerate}
\end{definition}



\begin{proposition}[\cite{BIBooDUPSooZjcTHL}]       \label{PROPooEWUCooTStAvb}
	If \( S\) is a submanifold of \( M\), the inclusion map \( \iota\colon S \to M\) is an immersion\footnote{Definition \ref{DEFooZKUIooXWVGvh}}.
\end{proposition}

\begin{proposition}     \label{PROPooZACHooCNgLSl}
	A manifold \( S\) is a submanifold of \( M\) if \( S\subset M\) (as sets) and the identity \( \iota\colon S\to M\) is regular\footnote{Definition \ref{DEFooMELXooEkEnwz}.}.
\end{proposition}

\begin{proposition}\label{prop:topo_sub_manif}
	The own topology of a submanifold is finer than the induced one from the manifold.
	\index{topology!on submanifold}
\end{proposition}

\begin{proof}
	Let $M$ be a manifold of dimension $n$ and $N$ a submanifold\footnote{In the whole proof, we should say ``there exists a sub-neighbourhood such that\ldots``} of dimension $k<n$. We consider $V$, an open subset of $N$ for the induced topology, so $V=N\cap\mO$ for a certain open subset $\mO$ of $M$. The aim is to show that $V$ is an open subset in the topology of $N$.

	Let us define $\mP=\varphi^{-1}(\mO)$.  The charts of $N$ are the projection to $\eR^k$ of the ones of $M$. We have to consider $W=\varphi^{-1}(V)$, since $N$ is a submanifold, $\varphi^{-1}(\mO\cap N)=\eR^k\cap\mP$. It is clear that $W=\eR^k\cap\mP$ is an open subset of $\eR^k$ because it is the projection on the $k$ first coordinates of an open subset of $\eR^n$.

	The subset $V$ of $N$ will be open in the sense of the own topology of $N$ if $\varphi'{}^{-1}(V\cap\varphi'(\mU'))$ is open in $\eR^k$ where $\varphi'$ is the restriction of $\varphi$ to his $k$ first coordinates: $\varphi'(a)=\varphi(a,0)$ and $\mU'$ is the projection of $\mU$.
\end{proof}


\begin{proposition}\label{prop:subvar_ouvert}
	A submanifold is open if and only if it has the same dimension as the main manifold.
\end{proposition}

\begin{proof}
	\subdem{Necessary condition}
	We consider some charts $\dpt{\varphi_i}{U_i}{M}$ on some open subsets $U_i$ of $\eR^n$. If $N$ is open in $M$, then this can be written as
	\[
		N=\bigcup_iU_i.
	\]
	If we choose the charts on $M$ in such a manner that $\dpt{\varphi_i}{U_i\cap \eR^k}{N}$ are charts of $N$, we must have $\varphi_i(U_i\cap\eR^k)=\varphi_i(U_i)$. Then it is clear that $k=n$ is necessary.
	\subdem{Sufficient condition}
	If $N$ has same dimension as $M$, the charts $\dpt{\varphi_i}{U_i}{M}$ are trivially restricted to $N$.
\end{proof}

The following result allow to extend a smooth function defined on a submanifold to an open set of the «larger» manifold.
\begin{proposition}     \label{PROPooOTZQooIfboXV}
	Let \( N\) be a submanifold of \( M\) and \( f\in  C^{\infty}(N)\). Let \( p\in N\). There exists a neighbourhood \( W\) of \( p\) in \( M\) and a function \( \tilde f\in  C^{\infty}(W)\) such that
	\begin{equation}
		\tilde f(n)=f(n)
	\end{equation}
	for every \( n\in N\).
\end{proposition}

\begin{proof}
	Since \( N\) is a submanifold of \( M\), the definition \ref{DEFooLQHWooMOTgzq} provides a chart \( \varphi\colon U\to M\) around \( p\) such that
	\begin{equation}
		\varphi^{-1}\big( \varphi(U)\cap N \big)=\{ (x_1,\ldots, x_n,0,\ldots, 0) \}.
	\end{equation}
	From the function \( f\colon N\to \eR\) we consider
	\begin{equation}
		\begin{aligned}
			f_1\colon \varphi^{-1}\big( \varphi(U)\cap N \big) & \to \eR        \\
			f_1                                                & =f\circ\varphi
		\end{aligned}
	\end{equation}
	This is the function \( f\) seen trough the chart. The function \( f_1\) is only defined on the ``\( N\)'' part of the chart, but can be extended as
	\begin{equation}
		\begin{aligned}
			\tilde f_1\colon U & \to \eR                                   \\
			(x_1,\ldots, x_m)  & \mapsto f_1(x_1,\ldots, x_n,0,\ldots, 0),
		\end{aligned}
	\end{equation}
	which is a good definition since \( (x_1,\ldots, x_n,0,\ldots, 0)\) is in \( \varphi^{-1}\big( \varphi(U)\cap N \big)\).

	Finally we write
	\begin{equation}
		\begin{aligned}
			\tilde f\colon \varphi(U) & \to \eR                       \\
			\tilde f                  & =\tilde f_1\circ\varphi^{-1}.
		\end{aligned}
	\end{equation}
	This is the extension we are searching for. Indeed it is defined on \( \varphi(U)\) which is an open set in \( M\) which contains \( p\) and if \( q\in N\cap\varphi(U)\) we have \( q=\varphi(x_1,\ldots, x_n,0,\ldots, 0)\) and then
	\begin{subequations}
		\begin{align}
			\tilde f(q) & =(\tilde f_\circ\varphi^{-1})\varphi(x_1,\ldots, x_n,0,\ldots, 0) \\
			            & =\tilde f_1(x_1,\ldots, x_n,0,\ldots, 0)                          \\
			            & =f_1(x_1,\ldots, x_n,0,\ldots, 0)                                 \\
			            & =(f\circ\varphi)(x_1,\ldots, x_n,0,\ldots, 0)                     \\
			            & =f(q).
		\end{align}
	\end{subequations}
	Thus \( \tilde f=f\) on \( \varphi(U)\cap N\).
\end{proof}
