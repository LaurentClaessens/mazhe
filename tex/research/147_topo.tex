% This is part of (almost) Everything I know in mathematics
% Copyright (c) 2013,2016, 2019, 2021-2022, 2025
%   Laurent Claessens
% See the file fdl-1.3.txt for copying conditions.

\section{Main definitions}
%+++++++++++++++++++++++++

As for notations, when we have a topological space $X$ and a point $x\in X$, we denote by $\mV(x)$ the set of neighbourhoods of $x$. If $F$ is a subset of $X$, we denote by $\overline{F}$ the closure of $F$ and by $\Int{F}$, its interior.

\begin{definition}
	Let $\mO$ be a topology on a set $X$. A subset $\mB\subset\mO$ (the elements of $\mB$ are subsets of $X$) is a \defe{basis of topology}{basis!of topology} $\mO$ of $X$ when
	any element of $\mO$ is an union of elements of $\mB$.
\end{definition}

\begin{definition}
	If $X$ is a topological space, a subset $A\subset X$ is a \defe{topological subspace}{topological!subspace} when we consider on $A$ the \defe{induced topology}{topology!induced}\index{induced!topology} from $X$. This is the topology on $A$ where then open sets are the subsets of $A$ which can be written as $A\cap\mO$ for an open set $\mO\subset X$ in the sense of the topology on $X$.
\end{definition}

\begin{remark}
	In some literature, a topological subspace of $X$ is just a subset $A$ endowed with a topology for which the inclusion $\dpt{\iota}{A}{X}$ is continuous. This is a less restrictive condition.
\end{remark}

From this definition, we see that the topology on a topological subspace is at least the induced one. Indeed if $\mO$ is an open subset of $X$, the set $i^{-1}(\mO)$ must be open in $A$, but it is clear that $i^{-1}(\mO)=A\cap \mO$.

%---------------------------------------------------------------------------------------------------------------------------
\subsection{Separability axioms}	\index{axiom!separability}
%---------------------------------------------------------------------------------------------------------------------------

The following definitions are the \defe{separability axioms}{separability!axiom}. A topological space $X$ is said to be $T_0$\index{$T_0$ topological space} if for every pair of points in $X$, at least one of them has an open neighbourhood which does not contain the other.

The topological space $X$ is $T_1$\index{$T_1$ topological space} if every point is closed or, equivalently, when for every two points, each has a neighbourhood which does not contain the other.

The topological space $X$ is $T_2$\index{$T_2$ topological space} if every two points belong to disjoint neighbourhoods.


%---------------------------------------------------------------------------------------------------------------------------
\subsection{Axioms of countability}	\index{countability axioms}\index{axiom!countability}
%---------------------------------------------------------------------------------------------------------------------------

A topological space $X$ is
\begin{enumerate}
	\item \defe{first countable}{first countable} if every point has a countable neighbourhood basis,
	\item \defe{second countable}{second countable} if the topology has a countable basis,
	\item \defe{separable}{separable} if there exists a countable dense subspace.
\end{enumerate}


\begin{lemma}
	If $\dpt{\varphi}{X}{Y}$ is a linear map between two normed vector spaces such that
	\[
		\| \varphi(A) \|\leq \| A \|,
	\]
	then $\varphi$ is continuous.
	\label{lem:lin_vec_cont}
\end{lemma}

\begin{proof}
	Let $\mU$ be open in $Y$, namely, consider that it is a ball centered at $y\in Y$ with ray $\delta$; if $x\in\varphi^{-1}(\mU)$, we have to prove that $A$ posses a neighbourhood contained in $\varphi^{-1}(\mU)$. Let's see the condition on $a\in X$ for $x+a$ to belongs to $\varphi^{-1}(\mU)$. We have
	\[
		\| \varphi(x+a)-y \|\leq\| \varphi(x)-y \|+\| a \|,
	\]
	but $\| \varphi(x)-y \|\leq \delta$. Taking $\| a \|$ suitably small, it is always possible to $x+a$ to belong to $\varphi^{-1}(\mU)$.
\end{proof}

\begin{definition}

	If $V$ and $W$ are topological vector spaces, the map $\dpt{f}{V}{W}$ is \defe{uniformly continuous}{uniformly!continuous}\index{continuous!uniformly} when for every neighbourhood $B$ of zero in $W$, there exists a neighbourhood $A$ of zero in $V$ such that
	\[
		v_1-v_2\in A\Rightarrow f(v_1)-f(v_2)\in B.
	\]
	\label{def:unif_cont}
\end{definition}


\begin{definition}
	A topological space is \defe{separable}{separable!topological space} when it admits a countable basis.
\end{definition}

\begin{lemma}
	Let $\Gamma$ be a family of subsets of a set $X$ such that $X=\bigcup_{A\in\Gamma}A$.

	There exists a topology on $X$ of basis $\Gamma$ if and only if the intersection of any \emph{finite} subfamily of $\Gamma$ can be written as the union of elements of $\Gamma$.
	\label{lem:topo_base}
\end{lemma}

\begin{lemma}
	A metric space $(E,d)$ is separable if and only if it contains a dense sequence.
	\label{lem:sep_metric}
\end{lemma}

\begin{proof}
	\subdem{Necessary condition} If the countable basis is $(B_i)$, we take a $x_i$ in each $B_i$. For any $x\in E$ and open set $A$ containing $x$, there is a $B_i\subset A$ and then a $x_i\in A$. Thus the sequence $(x_i)$ is dense in $E$.
	\subdem{Sufficient condition}
	Let $(x_i)$ be the dense sequence. We consider $B_{i,k}$, the open ball centred in $x_i$ of radius $1/k$. This is a countable system of open set in $E$. Let $A$ be an open set in $E$ and $S$, the union of all the $B_{i,k}$ which are contained in $A$. It is clear that $S\subset A$. Let $x\in A$ and $D$, an open ball of radius $\delta$ chosen in order to have $D\subset A$. We choose $k\in\eN$ such that $1/k<\delta$ and $x_i$ such that $d(x_i,x)<1/k$. Then $x\in S$; namely $x$ belongs to a ball of center $x_i$ and of radius $1/k$.
\end{proof}

\begin{definition}
	A subset of a topological space is \defe{connected}{connected part in topological space} if it cannot be written as an union of disjoint open sets.
\end{definition}

If $A$ is a part of the topological space $X$, the \defe{boundary}{boundary} is the subset of $X$ defined by
\[
	\Fr(A)=\overline{A}\cap\overline{\complement A}
\]

\begin{lemma}
	Every path connected part of a topological space is connected.
\end{lemma}

\begin{theorem}
	Let $E$ be a topological space and $A,B$ two connected parts of $E$. If $A$ intersects $B$ and $\complement B$, then $A$ intersects $\Fr(B)$.\label{tho:doine}
\end{theorem}


\begin{proposition}
	Let $X$ be a topological space and $A$, $B$ two subsets of $X$ such that $A\cup B$ is closed and there exist $A'$, $B'$, two disjoint sets containing respectively the closure of $A$ and $B$. Then $A$ and $B$ are separately closed.\label{prop:sep_ferme}
\end{proposition}

\begin{proof}
	We consider a converging sequence in $A$. If $x$ is the limit, $x\in A\cup B$ and $x\in\overline{A}\subset A'$. But it is clear from the assumptions that $A'\cap(A\cup B)=A$. Then $x\in A$, which proves that $A$ is closed.
\end{proof}

\begin{proposition}
	Let $X$ be a topological space, $K\subset X$ a compact and $\dpt{\phi}{K}{X}$ a continuous map. Then $\phi(K)$ is compact in $X$.
\end{proposition}

\begin{definition}		\label{DEFooERRQooNOpCXD}
	A family $\{ X_i \}_{i\in I}$ of subsets of a topological space $E$ is said to be \defe{locally finite}{locally finite} if each $x\in E$ has a neighbourhood $V$ such that $V\cap X_i \neq \emptyset$ for only finitely many values of $i$.
\end{definition}

\begin{lemma}[\cite{MonCerveau}]		\label{LEMooGRPUooVhmqDH}
	Let \( \{ X_i \}_{i\in I}\) be an open locally finite\footnote{Definition \ref{DEFooERRQooNOpCXD}.} covering of a topological space \( X\). Then \( \{ \bar X_i \}_{i\in I}\) is a locally finite covering.
\end{lemma}

\begin{proof}
	Let \( x\in X\). We have an open neighbourhood \( V\) such that \( I'=\{ i\in I\tq V\cap X_i\neq\emptyset \}\) is finite. We show that \( \{ i\in I\tq V\cap \bar X_i\neq \emptyset \}\) is \( I'\).

	Let \( i\in I\) and \(a\in V\cap \bar X_i\). Since \( a\in V\) and since  \( V\) is open, \( a\) has an open neighbourhood \( A\subset V\). Since \( a\in\bar X_i\), we have \( A\cap X_i\neq \emptyset\). Thus \( V\cap X_i\neq \emptyset\) and \(  i\in I'\).
\end{proof}

\begin{definition}	\label{DefParacompact}
	A topological space $E$ is \defe{paracompact}{paracompact} if every open cover admits a locally finite subcovering.
\end{definition}


\begin{definition}
	A topological space $E$ is \defe{locally compact}{compact!locally} if $\forall x\in E$, there exists a compact neighbourhood of $x$ in $E$.
\end{definition}

For example, a (non-empty) open subset of $\eR^n$ is \emph{never} compact\footnote{The compacts subsets of $\eR^n$ are the closed and bounded subsets.} but \emph{always} locally compact.

\begin{definition}
	A subset $A$ of a metric space is \defe{relatively compact}{compact!relatively} when $\overline{A}$ is compact.
\end{definition}

\begin{definition}
	An \defe{Hausdorff space}{Hausdorff space} is a topological space in which any two distinct elements have disjoint neighbourhoods.
\end{definition}

\begin{lemma}
	If a set is Hausdorff for a topology $\tau_1$ and compact in a weaker topology $\tau_2$, then $\tau_1=\tau_2$.\label{lem:Hausweak}
\end{lemma}

\begin{proposition}
	If $K$ is a compact topological space and $E$ a separated\quext{Comment on dit un espace s\'epar\'e?} one and if $\dpt{f}{K}{E}$ is a continuous bijection, then $f$ is an homeomorphism. \label{lem:wiki}
\end{proposition}

\begin{definition}
	A space is \defe{normal}{normal!space} when for all disjoint closed subset $E$ and $F$ of the topological space $X$, there exist neighbourhoods $\mU$ and $\mV$ of $E$ and $F$ which are disjoint too. A topological result says that a compact Hausdorff space is normal.
\end{definition}

\begin{lemma}\label{lem:Urysohn}
	If $X$ is a normal space, $F$ a closed subset of $X$ and $\mU$ an open set containing $F$, then there exists a continuous function $\dpt{f}{X}{[0,1]}$ such that $f(F)=0$ and $f(X\setminus\mU)=1$.
\end{lemma}

\begin{definition}
	A metric space is \defe{complete}{complete!metric space} when any Cauchy sequence is converging in the space.
\end{definition}

Pay attention not to be confused with a characterization of a \emph{closed} subset: the limit of any converging sequence lies in the space.

\begin{definition}\label{def:separe}
	Let $E$ and $F$ be two spaces in duality\quext{I'm not sure what it means.}; if $A\subset E$ and $B\subset F$, we say that $A$ \defe{separates}{separate} $B$ if $\forall x\neq y\in B$, $\exists a\in A$ such that $\hat{a}(x)\neq\hat{a}(y)$ where the hat denotes the duality.
\end{definition}

For an Hausdorff $X$ space, we denote by $C(X)$\nomenclature{$C(X)$}{Space of continuous functions}\label{pg_def_Cz } the space of all continuous functions on $X$. For a locally compact space $X$, the set $C_0(X)$\nomenclature{$C_0(X)$}{Space of continuous functions which decrease to zero at infinity in a certain sense} contains continuous functions which decrease to zero at infinity in the sense that for any $\varepsilon>0$, there exists a compact set $K$ such that $|f(X)|<\varepsilon$ for all $x$ outside $K$.

\begin{definition}
	If $X$ is an Hausdorff compact space, one can define the \defe{supremum norm}{supremum!norm} on $C(X)$ by
	\begin{equation}
		\|f\|_{\infty}:=\sup_{x\in X}|f(x)|.
	\end{equation}
	\label{def:sup_norm}
\end{definition}

\begin{definition}
	Let $E$ be a vector space on $\eR$ or $\eC$. A \defe{seminorm}{seminorm} on $E$ is a map $\dpt{p}{E}{\eR}$ such that $\forall x,y\in E$, $\forall\lambda\in\eC$,
	\begin{enumerate}
		\item $p(x)\geq 0$
		\item $p(\lambda x)=|\lambda|p(x)$
		\item $p(x+y)\leq p(x)+p(y)$.
	\end{enumerate}
\end{definition}
A seminorm is a norm without the condition  $p(x)=0$ if and only if $x=0$.

\begin{definition}
	An \defe{\ecart}{ecart}\quext{C'est pas joli la traduction du mot ``\'Ecart``.} on a set $E$ is a map $\dpt{d}{E\times E}{\eR}$ such that $\forall x,y,z\in E$,
	\begin{enumerate}
		\item $d(x,x)=0$
		\item $d(x,y)\geq 0$
		\item $d(x,y)=d(y,x)$
		\item $d(x,z)\leq d(x,y)+d(y,z)$.
	\end{enumerate}
\end{definition}
It is a distance minus the condition  $d(x,y)=0$ if and only if $x=y$.
It is clear that $d(x,y)=p(x-y)$ is an \ecart\ on $E$ when $p$ is a seminorm. Moreover, this $d$ is invariant under the translations and fulfils $d(\lambda x,\lambda y)=|\lambda|d(x,y)$.

\begin{proposition}
	Let $p$ be a seminorm on a topological vector space. Then $p$ is continuous if and only if there exists a neighbourhood of $o$ in which $p$ is bounded.
	\label{prop:semi_norm_cont}
\end{proposition}


From a family of \ecarts, we can define a topology. Let $(d_{\alpha})_{\alpha\in I}$ be a family of \ecarts\ on the set $E$. For an element $a\in E$, a \emph{finite} subset $(\alpha_j)_{j=1\ldots m}$ of $I$ and some numbers $(r_j)_{j=1\ldots m}>0$, we write
\[
	B(a; (\alpha_j),(r_j))=\{x\in E\ast d_{\alpha_j}(a,x)<r_j,\forall j=1,\ldots,m\}.
\]
\label{topo_semi_norm} A subset $U\subset E$ is said \defe{open}{topology!with seminorms} by respect of the family $(\alpha_i)_{i\in I}$ when for any $x\in U$, there exists a sub-family $(\alpha_j)$ and some numbers $(r_j)>0$ such that $B(x;(\alpha_j),(r_j))\subset U$.

\begin{definition}
	When a topology can be defined from a family of \ecarts, it is said an \defe{uniformisable}{uniformisable topology} topology.
\end{definition}

If we have a family $(p_{\alpha})_{\alpha\in I}$ of seminorm, we can build the \ecarts\ $d_{\alpha}(x,y)=p_{\alpha}(x-y)$. One can show that this topological structure is compatible with the vector space structure.


\begin{definition}
	A metric space is \defe{separable}{separable} if there exists a subset at most countable which is everywhere dense.
\end{definition}

\begin{proposition}
	If $E$ is a metric locally compact space, then the following are equivalent:
	\begin{enumerate}
		\item There exists an increasing sequence $(A_n)$ of open relatively compacts subset of $E$ such that $\overline{A_n}\subset A_{n+1}$ and $E=\bigcup_n A_n$,

		\item $E$ is a countable union of compact subsets,

		\item $E$ is separable.
	\end{enumerate}
\end{proposition}

If $A$ is a topological space with an equivalence relation $\sim$ and the canonical projection $\dpt{\varphi}{A}{A/\sim}$, then $V\subset A/\sim$ is open in then \defe{quotient topology}{quotient topology}\index{topology!quotient} if and only if $\varphi^{-1}(V)\subset A$ is open.

\section{Topology and convergence}
%+++++++++++++++++++++++++++++++++

When we have a topological space $X$, we can define a \defe{convergence notion}{convergence!in topological spaces} for the sequences.

\begin{definition}
	We say that the sequence $(x_n)$ in $X$ converges to $x\in X$ if
	$\forall V\in\mV(x)$, $\exists n_0$ such that $n\geq n_0$ implies $x_n\in V$. \label{def:convergence}
\end{definition}

\begin{definition}
	If $X$ and $Y$ are two topological spaces, a map $\dpt{f}{X}{Y}$ is said \defe{continuous}{continuous} if $f^{-1}(\mU)$ is open in $Y$ for any open set $\mU$ in $X$.
\end{definition}

If $f$ is continuous and $\mU$ open in $X$, it is not true that $f(\mU)$ is open in $Y$. Simple counter-example are given by the constants functions from $\eR$ to $\eR$. So it is false that continuity preserves the openness. But it preserves the convergence.

\begin{proposition}
	Let $a_k$ be a converging ($a_k\to a$) sequence in $X$ and $\dpt{g}{X}{Y}$, a continuous map. Then $g(a)=\lim_{k\to\infty}g(a_k)$.
	\label{prop:continu_cv}
\end{proposition}
\begin{proof}
	If $\mU$ is an open in $Y$ which contains $g(a)$, $g^{-1}(\mU)$ is an open subset of $X$ which contains $a$. Then from a certain $N$, $a_k\in g^{-1}(\mU)$, so that $g(a_k)\in\mU$; it proves that the sequence $g(a_k)$ in $Y$ converges to $g(a)$ in the sense of the topology of $Y$.
\end{proof}

Let us define a notion which depend only on the convergence notion.

\begin{definition}
	An \defe{usinage}{usinage} around $x\in X$ is a subset $\mU\subset X$ which contains $x\in\mU$ and such that for any convergent sequence $(a_n)\to x$, there exists $n_0$ such that $n>n_0$ implies $a_n\in\mU$.
\end{definition}

\begin{lemma}
	Let $X$ be a metric topological space. The topology is completely determined by his convergence notion.

	An other way to state this proposition: there is only one metric topology which gives rise to a given convergence notion. \label{prop:usinage}
\end{lemma}
\begin{proof}
	We will show that, provided that the topology comes from a metric, the notion of neighbourhood is the same as the notion of usinage.

	On the one hand, from topological definition~\ref{def:convergence} of convergence, it is clear that any neighbourhood is an usinage.

	On the other hand, any usinage must contains a metric ball centred at $x$ and then is a neighbourhood. Indeed, if $\mU$ doesn't contains any ball, then
	\[
		\forall\epsilon>0,\exists a(\epsilon)\text{ such that } d(a(\epsilon),x)<\epsilon \text{ and } a(\epsilon)\notin\mU.
	\]
	%
	In such a case, any sequence $( a(\epsilon_1),\ldots )$ with $\epsilon_n\to 0$ contradict the fact that $\mU$ is an usinage.
\end{proof}

\begin{proposition}
	For a given notion of usinage (i.e. for a given notion of convergence) on any set, there exists a topology (maybe no metric!) for which the usinage form a basis.
\end{proposition}

\begin{proof}
	This is a direct application of lemma~\ref{lem:topo_base} and the fact that the intersection of two usinages is an usinage.
\end{proof}

The fact that a convergence notion gives rise to a topology is summarized by the following scheme:
\[
	\xymatrix   { &&\text{Basis of topology (always)} \\
		\text{Convergence}\ar[r]& \text{Usinage}\ar[ru]\ar[rd]\\
		&& \text{Unique metric topology (when it exists)} }
\]


\begin{definition}
	Let $\mU$ be an open subset of $\eR^n$. A increasing sequence $(K_m)$ of compacts subset of $\mU$ is \defe{fundamental}{fundamental!sequence of subset} if $K_m\subset\Int{K}_{m+1}$ for all $m$ and
	\[
		\bigcup_m K_m=\mU
	\]
\end{definition}

%+++++++++++++++++++++++++++++++++++++++++++++++++++++++++++++++++++++++++++++++++++++++++++++++++++++++++++++++++++++++++++ 
\section{Locally convex}
%+++++++++++++++++++++++++++++++++++++++++++++++++++++++++++++++++++++++++++++++++++++++++++++++++++++++++++++++++++++++++++

\begin{definition}[\cite{BIBooJRWYooHjYhfQ}]        \label{DEFooCGJBooSvDpyC}
	A topological vector space is called \defe{locally convex}{locally convex} if the origin has a neighborhood basis consisting of convex sets.
\end{definition}

\begin{proposition}
	A topological vector space is locally convex if and only if its topology can be defined from a family of seminorms.
\end{proposition}

\begin{lemma}[\cite{MonCerveau}]        \label{LEMooDVZWooWKRQWC}
	Let \( X\) be a locally convex vector space. Let \( K\) be compact in \( X\) and let \( U\) be an open set containing \( 0\). There exists \( \lambda\in \eR\) such that \( K\subset \lambda U\).
\end{lemma}

\begin{lemma}   \label{LEMooAHUMooBwxzPj}
	Let \( U\) be convex, \( u_1, u_2\in U\) and \( a,b\in \eR^+\). Then there exists \( u\in U\) such that
	\begin{equation}
		au_1+bu_2=(a+b)u.
	\end{equation}
\end{lemma}

\begin{proof}
	We have
	\begin{equation}
		au_1+bu_2=(a+b)\left( \frac{ a }{ a+b }u_1+\frac{ b }{ a+b }u_2 \right).
	\end{equation}
	Since \( a,b\in \eR^+\), the coefficients \( a/(a+b)\) and \( b/(a+b)\) belong to \( \mathopen[ 0 , 1 \mathclose]\) and they sum to \( 1\). From convexity of \( U\), the element \( \frac{ a }{ a+b }u_1+\frac{ b }{ a+b }u_2\) belongs to \( U\).
\end{proof}

\begin{proposition}    \label{prop:topo_E}
	On $\scrE(\mU)$, there exists one and only one structure of locally convex metrisable Hausdorff space whose convergence notion is:

	$(f_k)\to 0$ in $\scrE(\mU)$ when for any compact $K\subset\mU$ and any multi-indices $\nu$, the sequence of restriction $(D^{\nu}f_k|_K)$ converges uniformly to $0$.

	Here the $D^{\nu}$ denotes the multi-derivation with respect to the coordinates contained in the multi-index $\nu$.
\end{proposition}

\begin{proof}
	The unicity part is given by lemma~\ref{prop:usinage}. Now, we will build such a topology. First remark that we can find a fundamental sequence $K_m$ in $\mU$. Next, for any couple of integer $s\geq 0$, $m>0$ and for any function $f\in\scrE^{(r)}(\mU)$ with $r\geq s$, we define
	\[
		p_{s,m}(f)=\sup_{\substack{ x\in K_m\\|\nu|\leq s}}|(D^{\nu}f)(x)|.
	\]

	One can see that these $p_{s,m}$ are seminorms, moreover $p_{s,m}\leq p_{r,m}$. Remark that if $p_{0,m}(f)=0$ for any $m$, $f=0$ on any $K_m$, and then $f=0$ in $\mU$. Thus $p_{0,m}=0$ for all $m$ implies $p_{s,m}(f)=0$.

	Now, we consider only the $p_{s,m}$ with $0\leq s\leq m$ and the induced topology on $\scrE(\mU)$.

	First, we prove that this topology is Hausdorff. Let us consider $f$, $g\in\scrE(\mU)$, and suppose that any ball centred at $f$ contains $g$. Then we must obtain $f=g$. For any sequences $(s_i)$, $(r_i)$ with $0\leq s_i\leq r_i$ and for any sequence $(\epsilon_j)>0$,
	\[
		g\in B(f;(p_{s_i,r_i}),(\epsilon_j))=
		\{ x\in\scrE(\mU)|\, d_{p_{s_i,r_i}}(f,x)<\epsilon_j,\,\forall j \}.
	\]
	In particular, for a sequence $\epsilon_j\to 0$ and $s_i=0$, the condition becomes $p_{0,r_i}(f-g)<\epsilon_j$ for any $f$. Then $p_{0,r_i}(f-g)=0$ and $f=g$.

	By the way, remark that this topology is almost the ``supremum norm``\ topology, then one can guess that the convergence notion will be something as the uniform convergence. There are just some subtleties as ``for any compact``\ or ``for any derivatives``. But these are just what we need in the statement. Let us be more precise.

	Definition~\ref{ITEMooJDUUooVMvAOn} of a compact subset makes that any compact subset of $\mU$ is a subset of one of the $K_m$. This makes our topology independent of the choice of the fundamental sequence $K_m$.

	Let us consider $(f_k)$, a sequence in $\scrE(\mU)$. The condition to converge to $0$ in the sense of our new topology is that if $A$ is an open subset of $\scrE(\mU)$ which contains $0$, there exists a $k_0$ such that $k>k_0$ implies $f_k\in A$.

	Then for any real sequences $(\epsilon_j)>0$ and $p_{s_i,r_i}$ with $0\leq s_i\leq r_i$, $f_k\in B( 0;(p_{s_i,r_i}),(\epsilon_j) )$. Thus,

	$\exists k_0$ such that $k>k_0$ implies that $\forall s_j,r_j$,
	\[
		\sup_{ \substack{ x\in K_{r_j}\\|\nu|\leq s_j } }|D^{\nu}f_k(x)|<\epsilon_j.
	\]
	Taking a sequence $\epsilon_j\to 0$, we get the thesis.
\end{proof}


\begin{definition}
	A subset $H\subset\scrE(\mU)$ is \defe{bounded}{bounded!subset of $\scrE(\mU)$} is each of the seminorm $p_{s,m}$ is bounded in $H$.
	\label{def:bounded}
\end{definition}
One can see that this notion only depend on the topology of $\scrE(\mU)$. Let us once again give some properties without proof.

\begin{proposition}
	There exists a sequence of functions in $\scrE(\mU)$ with compact support which is dense in each $\scrE^{(r)}(\mU)$ and in $\scrE(\mU)$. With others words, the spaces $\scrE^{(r)}(\mU)$ and $\scrE(\mU)$ are Fréchet separable spaces.
	\label{prop:E_Frechet}
\end{proposition}

\begin{proposition}
	Any bounded subset of $\scrE(\mU)$ is relatively compact in $\scrE(\mU)$.
\end{proposition}

\begin{proposition}
	For any multi-index $\nu$, the linear map $\dpt{D^{\nu}}{\scrE(\mU)}{\scrE(\mU)}$, $f\to D^{\nu}f$ is continuous.
\end{proposition}

%--------------------------------------------------------------------------------------------------------------------------- 
\subsection{Topological group}
%---------------------------------------------------------------------------------------------------------------------------



\begin{remark}\label{rem:ouvert}
	From the existence of an unique inverse for any element of $G$, the multiplication and the inversion are also open maps.
\end{remark}

\begin{definition}[Uniform continuity]      \label{DEFooUEBTooDqipcL}
	Let \( G\) be a topological group\footnote{Definition \ref{DEFooCHZVooHnvSgW}.} and \( (E,d)\) a metric space. A map \( f\colon G\to E\) is \defe{uniformly continuous}{uniform continuity} if for every \( \epsilon>0\), there exists a neighbourhood \( V\) of \( e\) in \( E\) such that
	\begin{equation}
		y^{-1}x\in V\Rightarrow d\big( f(x),f(y) \big)<\epsilon.
	\end{equation}
\end{definition}

\begin{proposition} \label{PROPooSHBAooVRdAFM}
	If \( K\) is compact in the topological group \( G\), and if \( f\) is continuous on \( K\), then it is uniformly continuous\footnote{Definition \ref{DEFooUEBTooDqipcL}.} on \( K\).
\end{proposition}

\section{Metrisable groups}\label{sec:metrisable_groups}
%+++++++++++++++++++++++++++

Let $G$ be any group. The function $\dpt{f}{G\times G}{\eR}$ is said \defe{left invariant}{left invariant!function on a group} if for any $x$, $y$, $z\in G$, $f(xy,xz)=f(y,z)$. If $G$ is abelian, the notion of left invariant and of right invariant are the same; in this case, we say it to be \defe{translation invariant}{translation!invariant function on a group}.

For example, when a distance is left invariant, the left translations are isometries. For example, in a normed space vector, the distance $d(x,y)=\|x-y\|$ is always translation invariant: $\| (x+a)-(y+a) \|=\|x-y\|$.



\begin{proposition}
	Let $G$ be a topological group\footnote{Definition \ref{DEFooCHZVooHnvSgW}.}.
	\begin{itemize}
		\item $G$ is metrisable if and only if there exists a countable fundamental system of neighbourhoods of $e$ (the neutral) whose intersection is only $e$.
		\item In this case, the topology of $G$ ca be defined by a left invariant distance or a right invariant distance.
	\end{itemize}
\end{proposition}




On a metrisable topological group, one has thus two types of distance. But in general, one can not find a distance which defines the topology being left and right invariant.

\begin{proposition}
	Let $G$ be a metrisable group. A sequence $(a_n)$ in $G$ is a Cauchy sequence for a left invariant distance if and only if

	$\forall\, V\in\mV(e)$, $\exists n_0>0$ such that $x_n^{-1} x_m\in V$ $\forall n,m\geq n_0$.
	\label{prop:Cauchy_metrisable}
\end{proposition}

The point is that if a sequence is a Cauchy sequence for a left invariant distance, then it is a Cauchy sequence for \emph{any} left invariant distance. This yields the important conclusion: in metrisable groups, the notion of Cauchy sequence is a topological notion which only depends on the neighbourhood notion.

A right invariant Cauchy sequence is the same but with $x_nx_m^{-1}$ instead of $x_n^{-1} x_m$. By definition of a topological group, the map $x\to x^{-1}$ is continuous. Thus, if all the left Cauchy sequences are convergent, then all the right Cauchy sequences are convergent. In this case, the group $G$ is said \defe{complete}{complete!topological group}.

If the group is abelian, we have only one notion of Cauchy sequence, and it is independent of the metric.

\section{One point compactification}  \label{sec:compactific}
%---------------------------

\begin{definition}  \label{DEFooAKWJooWKcYav}
	Let $X$ be a non compact topological space. We consider the ``extended'' space $\tilde X:=X\cup\{ \infty\}$ on which we put the following topology. Open sets of $X$ are open in $\tilde X$ and when $\mO\subset\tilde X$ contains $\infty$, it is open if and only if the complementary in is compact in $X$. It is easy to see that $\tilde X$ is compact. It is the \defe{one point compactification}{one point compactification} of $X$.
\end{definition}

Let us now consider $\tilde X$, any compact space and $X=\tilde X\setminus\{\infty\}$ where $\infty$ is a point in $\tilde X$. If we put on $X$ the induced topology from $\tilde X$, we get a locally compact space whose compactification is $\tilde X$.

\begin{example}
	When \( X=\eR\) we add \emph{one} infinity to get the compactification \( \tilde \eR=\eR\cup\{ \infty \}\) which is isomorphic to the circle \( S^1\).

	This is not the same space as \( \bar \eR=\eR\cup\{ +\infty,-\infty \}\) which is not compact.
\end{example}

In this context we make the difference between \( \infty\) and \( +\infty\). When we write \( +\infty\) we are speaking of the infinity in \( \bar \eR\) while writing \( \infty\) with no sign we are speaking about the infinity in the one point compactification. The difference is important since for the one point compactification of \( \eR\) we have the limit
\begin{equation}
	\lim_{x\to 0} \frac{1}{ x }=\infty
\end{equation}
while for the topology on \( \bar \eR\) the limite does not exist.

%---------------------------------------------------------------------------------------------------------------------------
\subsection{Example: the Riemann sphere}
%---------------------------------------------------------------------------------------------------------------------------
\label{SEBSECooLJSEooNlyFYv}

We already defined the Riemann sphere as a projective space in the definition~\ref{DEFooSZGNooTzFYbh}.

\begin{definition}      \label{DEFooMAHAooGUYyqU}
	The \defe{Riemann sphere}{Riemann!sphere} is the one point compactification of \( \eC\). We denote it by \( \hat \eC\).
\end{definition}
See the definition~\ref{DEFooAKWJooWKcYav} for the topology.

\begin{proposition}     \label{PROPooDTPKooYcSzYq}
	Let \( \gamma\colon \eR\to \hat \eC\) be a path in \( \hat \eC\). We have \( \lim_{t\to \infty} \gamma(t)=\infty\) if and only if \( \| \gamma(t) \|\to +\infty\).
\end{proposition}

\begin{proof}
	To be clear: the first limit is a limit in \( \hat \eC\) for its one point compactification topology while the second limit is an usual limit in \( \eR\).
	\begin{subproof}
		\spitem[\( \Rightarrow\)]
		Let \( M>0\). Since \( \gamma(t)\to \infty\), there exists a \( T\) such that \( t>T\) implies
		\begin{equation}
			\gamma(t)\in\overline{ B(O,M) }^c
		\end{equation}
		because this is an open set around \( \infty\). For that \( t>T\) we thus have \( \| \gamma(t) \|>M\). This proves that \( \| \gamma(t) \|\to +\infty\).
		\spitem[\( \Leftarrow\)]
		Let \( \mO\) be an open neighbourhood of \( \infty\) in \( \hat \eC\). This \( \mO^c\) is compact in \( \eC\); \( \mO^c\) is then bounded\footnote{The theorem~\ref{ThoXTEooxFmdI} applies because \( \eC\) has the topology of \( \eR^2\).} and there exists \( M\) such that \( \mO^c\subset B(0,M)\).

		There also exists \( T\) such that \( t>T\) implies \( \| \gamma(t) \|>M\). Thus for \( t>T\) we also have \( \gamma(t)\in B(0,M)^c\subset \mO\).
	\end{subproof}
\end{proof}

As far as arithmetic is concerned, the proposition~\ref{PROPooDTPKooYcSzYq} makes us  define \( \frac{1}{ 0 }=\infty\) and \( \frac{ 1 }{ \infty }=0\).

\begin{remark}
	\lstinputlisting{tex/sage/sageSnip002.sage}
	Sage makes the difference between \( +\infty\), \( -\infty\) and \( \infty\).

	In the first case, \( 1/x\) is seen in \( \hat \eC\) and the limit is \( \infty\), while in the second case, the limit is understood in \( \eR\) and sage provides \( +\infty\).
\end{remark}

\begin{definition}[Pole of a complex function \cite{ooXQVIooTVOqFo}]       \label{DEFooPXYYooOMZYOT}
	Suppose \( U\) is an open subset of the complex plane \( \eC\), $p$ is an element of $U$ and \( f\colon U\setminus\{ p \}\to \eC\) is an holomorphic function  over its domain. If there exists a holomorphic function \( g\colon U\to \eC\), such that $g(p)$ is nonzero, and a positive integer $n$, such that for all \( z\in U\setminus\{ p \}\),
	\begin{equation}
		f(z)=\frac{ g(z) }{ (z-p)^n }
	\end{equation}
	holds, then $p$ is called a \defe{pole}{pole} of \( f\). The smallest such $n$ is called the \defe{order}{order!of a pole}. A pole of order 1 is called a \defe{simple pole}{pole!simple}.

	A function \( f\colon \hat \eC\to \eC\) which is holomorphic on a neighbourhood of \( \infty\) has a \defe{pole}{pole!at infinity} at \( z=\infty\) if the function \( z\mapsto f(1/z)\) has a pole at \( z=0\).
\end{definition}

%+++++++++++++++++++++++++++++++++++++++++++++++++++++++++++++++++++++++++++++++++++++++++++++++++++++++++++++++++++++++++++
\section{Topological approximation}
%----------------------------------------------------------------------------------------------------------------------------
\subsection{Introduction}

In this section, we follow \cite{Landi}. Let consider a particle moving on a circle $S^1$, and suppose that we have three detectors, each of them cover a surface, let
\begin{align*}
	\mU_1 & =]-\frac{ \pi }{ 3 },\frac{ 2\pi }{ 3 }[ & \mU_2 & =]\frac{ \pi }{ 3 },\frac{ 4\pi }{ 3 }[, & \mU_3 & =]\pi,2\pi[.
\end{align*}
If the detectors $\mU_1$ and $\mU_2$ are react, we know that the particle is in $\mU_1\cap\mU_2$ (which is an open set), but no more; while if the detector $\mU_1$ reacts alone, we know that the particle \emph{is not} in $\mU_2$ or $\mU_3$, so that the particle belongs to
\[
	\mU_1\setminus\big( \mU_2\cap\mU_3 \big),
\]
which is closed. Our experimental setting allows us to distinguish six parts of $S^1$ that we name in the following way
\begin{align*}
	\mU_1\cap\mU_3 & \mapsto \alpha & \mU_1\setminus (\mU_2\cup\mU_3) & \mapsto a  \\
	\mU_1\cap\mU_2 & \mapsto \beta  & \mU_2\setminus (\mU_1\cup\mU_3) & \mapsto b  \\
	\mU_2\cap\mU_3 & \mapsto \gamma & \mU_3\setminus (\mU_1\cup\mU_2) & \mapsto c.
\end{align*}
Notice that each point in $S^1$ belongs to one and only one of these sets. One says that a point of $S^1$ is open when it belongs to $\alpha$, $\beta$ or $\gamma$ and that it is closed when it belongs to $a$, $b$, $c$. Now let us consider the six point space
\[
	P=\{ \alpha,\beta,\gamma,a,b,c \},
\]
on which we put the topology induced from $S^1$. A basis of that topology is given by
\[
	\{ \alpha \},\{ \beta \},\{ \gamma \},\{ \alpha,a,\beta \},\{ \alpha,c,\gamma \},\{ \beta,b,\gamma \}.
\]

%----------------------------------------------------------------------------------------------------------------------------
\subsection{Generalization}

Let $(X,\mU_{\lambda})$, a topological space. We define the equivalence relation $x\sim y$ if and only if $x\in\mU_{\lambda}\Leftrightarrow y\in\mU_{\lambda}$ for every $\lambda$. In other words, $x$ is equivalent to $y$ when they cannot be distinguished by the topology. Now we consider the quotient space
\[
	P_{\mU}(X)=X/\sim
\]
with its quotient topology: $A\subset P_{\mU}(X)$ is open if and only if $\pi^{-1}(A)$ is open in $X$ where $\pi\colon X\to P_{\mU}(X)$ is the canonical projection. It is the finest topology in which $\pi$ is continuous.

When $X$ is compact, the covering $\mU_{\lambda}$ can be finite, so that $P_{\mU}(X)$ is finite. If $X$ is only locally finite, the space $P_{\mU}(X)$ will be countable and it is said to be \defe{unitary}{unitary!topological space}. Notice that $P_{\mU}(X)$ is not Hausdorff in general, for example in $P_6(S^1)$, the point $a$ cannot be separated from $\alpha$. Neither it is $T_1$ because points $\alpha$, $\beta$ and $\gamma$ are open (in genera, it is even possible to get points which are open neither closed). The space $P_{\mU}(X)$ is however always $T_0$.

%----------------------------------------------------------------------------------------------------------------------------
\subsection{Order and topology}

When $P$ is a finite topological space, the collections $\tau$ of open sets in $P$ is closed under arbitrary unions and intersections. For each $x\in P$ we can consider the set
\begin{equation}
	\Lambda(x)=\bigcap\{ \mO\in\tau\tq x\in\mO \}
\end{equation}
which is the smallest neighbourhood of $x$. We define the relation $x\preceq y$ if and only if $\Lambda(x)\subseteq\Lambda(y)$, or equivalently if and only if every open neighbourhood of $y$ contains $x$. Yet another way to express it is $x\preceq y$ if and only if $y\in\overline{\{ x \}}$ where the bar denotes the closure. The relation $\preceq$ is reflexive and transitive.

The space $P$ is $T_0$ by assumptions, so that for every couple of points $x,y\in P$, there always is a neighbourhood of $x$ which does not contain $y$ (or \emph{vice versa}). That makes the relation antisymmetric:
\[
	x\preceq y\text{ and }y\preceq\,\Rightarrow x=y.
\]
So the relation $\preceq$ is reflexive, antisymmetric and transitive. Such a relation is a \defe{partial order}{partial!order}; a partially ordered set is often called \defe{poset}{poset} for short.

The construction can be carried in the inverse sense: if $(P,\preceq)$ is a poset, we define $\Lambda(x)=\{ y\in P\tq y\preceq x \}$ and we define an open set in $P$ as an union of such subsets. The resulting topological space is $T_0$.

A map $f\colon P\to Q$ between two posets is continuous if and only if it preserves order:
\begin{equation}
	x\preceq_Py\Rightarrow f(x)\preceq_Q f(y).
\end{equation}
When $x\preceq y$ and $x\neq y$, we write $x\prec y$.

With every poset, there is a \defe{Hasse diagram}{hasse diagram} associated. It is obtained by drawing one point for each element of the poset on different levels and lines between them respecting the following two rules:
\begin{enumerate}
	\item if $x\prec y$, then the point associated with $x$ has to be one level below the one of $y$,
	\item if $x\prec y$ and if there does not exist $z$ such that $x\prec z\prec y$ then the point of $x$ in one level bollow the one of $y$ and the two points are connected by a line.
\end{enumerate}
Let us draw the Hasse diagram for $P_6(S^1)=\{ a,b,c,\alpha,\beta,\gamma \}$. We have
\begin{align*}
	\Lambda(\alpha) & =\{ \alpha \}         & \Lambda(\beta) & =\{ \beta \}          & \Lambda(\gamma) & =\{ \gamma \}          \\
	\Lambda(a)      & =\{ \alpha,a,\beta \} & \Lambda(b)     & =\{ \beta,b,\gamma \} & \Lambda(c)      & =\{ \alpha,c,\gamma \}
\end{align*}
and
\begin{align*}
	\alpha & \prec a, & \alpha & \prec c & \beta & \prec a & \beta & \prec b & \gamma & \prec b & \gamma & \prec c,
\end{align*}
so that the Hasse diagram of $P_6(S^1)$ is
\begin{center}
	%The result is on figure~\ref{LabelFigHasseAGdfdy}. % From file HasseAGdfdy
	%\newcommand{\CaptionFigHasseAGdfdy}{<+Type your caption here+>}
	\input{auto/pictures_tex/Fig_HasseAGdfdy.pstricks}
\end{center}
The smallest open set containing a point $x$ is the set of points below $x$ which are connected to $x$ by a sequence of lines.


%+++++++++++++++++++++++++++++++++++++++++++++++++++++++++++++++++++++++++++++++++++++++++++++++++++++++++++++++++++++++++++
\section{Partition of unity}
%+++++++++++++++++++++++++++++++++++++++++++++++++++++++++++++++++++++++++++++++++++++++++++++++++++++++++++++++++++++++++++

\begin{definition}[\cite{ooQCDSooCpqDvB}]       \label{DEFooKFXLooFRLaBG}
	Let \( X\) be a topological space. A \defe{partition of unity}{partition!of unity} of \( X\) is a family of continuous functions \( \{ \psi_j \}_{j\in J}\) such that
	\begin{enumerate}
		\item
		      \( \psi_j\colon X\to \mathopen[ 0 , 1 \mathclose]\);
		\item
		      for every \( x\in X\) there exists a neighbourhood of \( x\) in \( X\) in which only a finite number of the \( \psi_j\)'s is non zero;
		\item
		      for every \( x\in X\) we have
		      \begin{equation}
			      \sum_{j\in J}\psi_j(x)=1.
		      \end{equation}
	\end{enumerate}
\end{definition}

\begin{definition}[\cite{ooQCDSooCpqDvB}]
	Let \( X\) be a topological space and \( \{ U_i \}_{i\in I}\) be a locally finite cover of \( X\). A partition of unity is \defe{subordinate}{partition!of unity!subordinate} to that cover if it is indexed by \( I\) (\( J=I\) in the definition~\ref{DEFooKFXLooFRLaBG}) and such that \( \supp(\psi_i)\subset U_i\) for every \( i\in I\).
\end{definition}

\begin{theorem}[\cite{ooQCDSooCpqDvB}]      \label{THOooPCHDooITWKpC}
	Let \( \Omega\) be an open set in \( \eR^d\) and \( \{ U_i \}_{i\in I}\) be an open cover of \( \Omega\). There exists
	\begin{enumerate}
		\item
		      a \(  C^{\infty}\) partition of unity \( \{ \psi_j \}_{j\in J}\) such that \( \supp(\psi_j)\) is compact in one of the \( U_i\);
		\item       \label{ITEMooFGMJooQPLqGY}
		      a \(  C^{\infty}\) partition of unity \( \{ \alpha_i \}_{i\in I}\) subordinated to the cover, such that for every compact \( K\) only a finite number of these \( \psi_i\)'s is non zero.
	\end{enumerate}
\end{theorem}
% C'est probablement mieux de démontrer THOooUGQCooFVySMP à la place de celui-ci.

\begin{remark}
	This theorem does not furnish a smooth compactly supported partition of unity subordinated to the given cover. Either you choose the partition to be compactly supported, either you choose them subordinated to the cover.
\end{remark}

\begin{definition}[\cite{ooQCDSooCpqDvB}]
	Let \( X\) be a topological space and \( \{ U_i \}_{i\in I}\) be a locally finite cover of \( X\). A partition of unity is \defe{subordinate}{partition!of unity!subordinate} to that cover if it is indexed by \( I\) (\( J=I\) in the definition~\ref{DEFooKFXLooFRLaBG}) and such that \( \supp(\psi_i)\subset U_i\) for every \( i\in I\).
\end{definition}

\begin{corollary}  \label{CORooMSWPooCxvuhm}
	If \( \Omega\) is bounded and \( \{U_i \}_{i\in I}\) is an open cover of \( \Omega\), there exists a partition of unity subordinated to \( \{ U_i \}_{i\in I}\) such that each \( \psi_i\) belongs to \( \swD(U_i)\).
\end{corollary}

\begin{proof}
	If \( \Omega\) is bounded in \( \eR^d\) we can consider \( U'_i=U_i\cap \Omega\) and use the point~\ref{ITEMooFGMJooQPLqGY} of theorem~\ref{THOooPCHDooITWKpC} for the cover \( \{ U'_i \}_{i\in I}\). So we have a partition of unity subordinated to that cover with supports in the \( U'_i\)'s. Since the support is closed and the \( U_i\)'s are bounded, the supports are compact. The functions of this partition of unity are also subordinated to the original \( U_i\)'s.
\end{proof}

\begin{definition}[\cite{BIBooRNREooMaKTGj}]        \label{DEFooTUYIooCRkgDm}
	Let \( X\) be a topological space. A part \( \mA\subset C^0(X)\) is \defe{normal}{normal!family of functions} is for every disjoint parts \( A,B\subset X\), there exists \( f\colon X\to \mathopen[ 0 , 1 \mathclose]\) such that
	\begin{enumerate}
		\item
		      \( f(X)\in\mathopen[ 0 , 1 \mathclose]\),
		\item
		      \( f(A)=\{ 0 \}\)
		\item
		      \( f(B)=\{ 1 \}\).
	\end{enumerate}
\end{definition}

\begin{lemma}       \label{LEMooDZGCooIGFXnA}
	If \( X\) is a locally compact topological space, the part \( C^0_c(X)\) is normal\footnote{Definition \ref{DEFooTUYIooCRkgDm}.}
\end{lemma}

\begin{definition}      \label{DEFooEDFIooHKnGdE}
	Let \( X\) be a topological space and \( U=\{ U_1,\ldots, U_n \}\) be an open covering of \( X\). A \defe{partition of unity subordinated}{partition of unity} to \( U\) is a set of functions \( \{ \phi_i \}_{i=1,\ldots, n}\) such that
	\begin{enumerate}
		\item
		      \( \phi_i(X)\subset \mathopen[ 0 , 1 \mathclose]\)
		\item
		      \( \supp(\phi_i)\subset U_i\)
		\item
		      \( \sum_{i=1}^n\phi(x)=1\) for every \( x\in X\).
	\end{enumerate}
	If \( \mA\subset C^0(X)\) we say that \( \{ \phi_i \}\) is a \( \mA\)-partition of unity of \( \phi_i\in \mA\) for every \( i\).
\end{definition}

\begin{theorem}[\cite{BIBooRNREooMaKTGj}]       \label{THOooUGQCooFVySMP}
	Let \( X\) be a topological space. Let \( \mA\subset C^0(X)\) be
	\begin{enumerate}
		\item
		      normal
		\item
		      closed under finite sums
		\item
		      if \( f,g\in\mA\) and \( g(x)\neq 0\) for every \( x\in X\), then \( f/g\in \mA\).
	\end{enumerate}
	For every finite open covering \( U_1,\ldots, U_n\) of \( X\), there exists a \( \mA\)-partition of unity\footnote{Definition \ref{DEFooEDFIooHKnGdE}.} subordinated to \( \{ U_i \}\).
\end{theorem}
