% This is part of Giulietta
% Copyright (c) 2013-2015, 2019-2022
%   Laurent Claessens
% See the file fdl-1.3.txt for copying conditions.

Here are the results which relate Lie groups and Lie algebras.

%+++++++++++++++++++++++++++++++++++++++++++++++++++++++++++++++++++++++++++++++++++++++++++++++++++++++++++++++++++++++++++ 
\section{Lie algebra of a Lie group}
%+++++++++++++++++++++++++++++++++++++++++++++++++++++++++++++++++++++++++++++++++++++++++++++++++++++++++++++++++++++++++++

\begin{lemmaDef}        \label{DEFooSSDYooOwjHso}
    Let \( G\) be a smooth Lie group and \( X\in T_eG\). The vector field \( X^L\) defined by
    \begin{equation}
        X^L_g=dL_Xv
    \end{equation}
    is smooth. This is the \defe{left invariant}{left invariant vector field} vector field associated with the vector \( X\in T_eG\).
\end{lemmaDef}

\begin{propositionDef}      \label{DEFooKDCPooZOJsMD}
    Let \( G\) be a smooth Lie group. If \( X,Y\in T_eG\) we define the bracket\footnote{On the right, this is the bracket of vector fields defined in \ref{DEFooHOTOooRaPwyo}.}
    \begin{equation}
        [X,Y] = [X^L,Y^L]_e.
    \end{equation}
    The set \( T_eG\) with this bracket is a Lie algebra. This is the \defe{Lie algebra of the Lie group}{Lie algebra of a Lie group} \( G\). It will usually be denoted by \( \lG\).

    The topology on \( \lG=T_eG\) is the usual one of the tangent spaces, definition \ref{PROPooHJOXooMGANfd}.
\end{propositionDef}

\begin{proof}
    We know from proposition \ref{PROPooEJBWooSbvypo} that \( T_eG\) is a vector space. We have to define a Lie bracket on it. For that we use the left-invariant vector field. Let \( X\in T_eG\) and \( g\in M\) we define
    \begin{equation}
        X^L_g=dL_gX
    \end{equation}
    where \( L_g\colon G\to G\) is the left translation: \( L_g(h)=gh\). If \( X,Y\in T_eG\) we define
    \begin{equation}
        [X,Y]=[X^L,Y^L]_e
    \end{equation}
    where the bracket on the right hand side is the commutator of vector field defined in \ref{DEFooHOTOooRaPwyo}. It defines a Lie algebra structure by the proposition \ref{PROPooSWQSooSEfTuX}.
\end{proof}

In order to make the notations clear, let us write the formula explicitly. If \( X,Y\in T_eG\) are given by \( X=\alpha'(0)\) and \( Y=\beta'(0)\) we have
\begin{subequations}        \label{SUBEQSooHKWMooQbeStl}
    \begin{align}
        (XY)f&=X(Y(f))\\
        &=\Dsdd{ (Yf)\big( \alpha(t) \big) }{t}{0}\\
        &=\Dsdd{ Y_{\alpha(t)}(f) }{t}{0}\\
        &=\Dsdd{ Y^L_{\alpha(t)}(f) }{t}{0}\\
        &=\DDsdd{ f\big( \alpha(t)\beta(u) \big) }{t}{0}{s}{0}.
    \end{align}
\end{subequations}

Now a great theorem without proof:
\begin{theorem} \label{tho:loc_isom}
Two Lie groups are locally isomorphic if and only if their Lie algebras are isomorphic.
\end{theorem}

\begin{theorem}		\label{ThoSubGpSubAlg}		\label{tho:gp_alg}
If $G$ is a Lie group, then
\begin{enumerate}
\item\label{ThoSubGpSubAlgi} if $\lH$ is the Lie algebra of a Lie subgroup $H$ of $G$, then it is a subalgebra of $\lG$,
\item Any subalgebra of $\lG$ is the Lie algebra of one and only one connected Lie subgroup of $G$.
\end{enumerate}

\begin{probleme}
À mon avis, il faut dire ``connexe et simplement connexe'', et non juste ``connexe''.
\end{probleme}

\end{theorem}
\begin{proof}

\subdem{First item}
Let $\dpt{i}{H}{G}$ be the identity map; it is a homomorphism from $H$ to $G$, thus $di_e$ is a homomorphism from $\lH$ to $\lG$. Conclusion: $\lH$ is a subalgebra of $\lG$.

\subdem{Characterization for $\lH$}
Before to go on with the second point, we derive an important characterization of $\lH$:
\begin{equation}\label{eq:path_alg}
\lH=\{X\in\lG:\text{the map } t\to\exp tX\text{ is a path in $H$}\}.
\end{equation}
For that, consider $\dpt{\exp_H}{\lH}{H}$ and $\dpt{\exp_G}{\lG}{G}$; from unicity of the exponential, for any $X\in\lH$, $\exp_HX=\exp_GX$, so that one can simply write ``$\exp$''\ instead of ``$\exp_h$''\ or ``$\exp_G$''.

Now, if $X\in\lH$, the map $t\to\exp tX$ is a curve in $H$. But it is not immediately clear that such a curve in $H$ is automatically build from a vector in $\lH$ rather than in $\lG$.  More precisely, consider a $X\in\lG$ such that $t\to\exp tX$ is a path (continuous curve) in H. By lemma~\ref{lem:var_cont_diff}, the map $t\to\exp tX$ is differentiable and thus by derivation, $X\in\lH$.
The characterisation \eqref{eq:path_alg} is proved.

Thus $\lH$ is a Lie subalgebra of $\lG$.

\subdem{Second item}
For the second part, we consider $\lH$ any subalgebra of $\lG$ and $H$, the smallest subgroup of $G$ which contains $\exp\lH$. We also consider a basis $\{X_1,\ldots,X_n\}$ of $\lG$ such that $\{X_{r+1},\ldots,X_n\}$ is a basis of $\lH$.

By corollary~\ref{cor:/24}, the set of linear combinations of elements of the form $X(M)$ with $M=(0,\ldots,0,m_{r+1},\ldots,m_r)$ form a subalgebra of $U(\lG)$. If $X=x_1X_1+\cdots+x_nX_n$, we define $|X|=(x_1^2+\cdots+x_n^2)^{1/2}$ ($x_i\in\eR$).

Let us consider a $\delta>0$ such that $\exp$ is a diffeomorphism (normal neighbourhood) from $B_{\delta}=\{X\in\lG:|X|<\delta\}$ to a neighbourhood $N_e$ of $e\in G$ and such that $\forall x,y,xy\in N_e$,
\begin{equation}\label{eq:coord_xy}
   (xy)_k=\sum_{M,N}C^{[k]}_{MN}x^My^N
\end{equation}
holds\footnote{The validity of this second condition is assured during the proof of theorem~\ref{tho:loc_isom} which is not given here.}. We note $V=\exp(\lH\cap B_{\delta})\subset N_e$. The map
\[
   \exp(x_{r+1}X_{r+1}+\cdots+x_nX_n)\to(x_{r+1},\ldots,x_n)
\]
is a coordinate system on $V$ for which $V$ is a connected manifold. But $\lH\cap B_{\delta}$ is a submanifold of $B_{\delta}$, then $V$ is a submanifold of $N_e$ and consequently of~$G$.

Let $x$, $y\in V$ such that $xy\in N_e$ (this exist: $x=y=e$); the canonical coordinates of $xy$ are given by \eqref{eq:coord_xy}. Since $x_k=y_k=0$ for $1\leq k\leq r$, $(xy)_k=0$ for the same $k$ because for $(xy)_k$ to be non zero, one need $m_1=\ldots=m_r=n_1=\ldots=n_r=0$ -- otherwise, $x^M$ or $y^N$ is zero. Now we looks at $C^{[k]}_{MN}$ for such a $k$ (say $k=1$ to fix ideas): $[k]=(\delta_{11},\ldots,\delta_{1k})=(1,0,\ldots,0)$ and by definition of the $C$'s,
\[
   X(M)X(N)=\sum_PC_{MN}^PX(P).
\]
But we had seen that the set of the $X(A)$ with $A=(0,\ldots,0,a_{r+1},\ldots,a_n)$ form a subalgebra of $U(\lG)$. Then, only terms with $P=(0,\ldots,0,p_{r+1},\ldots,p_n)$ are present in the sum; in particular, $C_{MN}^{[k]}=0$ for $k=1,\ldots,r$. Thus $VV\cap N_e\subset V$.

The next step is to consider $\mV$, the set of all the subset of $H$ whose contains a neighbourhood of $e$ in $V$. We can check that this fulfils the six axioms of a topological group\index{topological!group}:

\begin{enumerate}
\item The intersection of two elements of $\mV$ is in $\mV$;
\item the intersection of all the elements of $\mV$ is $\{e\}$;
\item any subset of $H$ which contains a set of $\mV$ is in $\mV$;
\item If $\mU\in\mV$, there exists a $\mU_1\in\mV$ such that $\mU_1\mU_1\subset\mU$ because $VV\cap N_e\subset V$;
\item if $\mU\in\mV$, then $\mU^{-1}\in\mV$ because the inverse map is differentiable and transforms a neighbourhood of $e$ into a neighbourhood of $e$;
\item if $\mU\in\mV$ and $h\in H$, then $h\mU h^{-1}\in\mV$.
\end{enumerate}

To see this last item, we denote by $\log$ the inverse map of $\dpt{\exp}{B_{\delta}}{N_e}$. By definition of $V$, it sends $V$ on $\lH\cap B_{\delta}$. If $X\in\lG$, there exists one and only one $X'\in\lG$ such that $he^{tX}h^{-1}=e^{tX'}$ for any $t\in\eR$. Indeed we know that $he^{X}h^{-1}=e^{\Ad_hX}$, then $X'$ must satisfy $e^{tX'}=e^{\Ad_htX}$. If it is true for any $t$, then, by derivation, $X'=\Ad_hX$.

The map $X\to X'$ is an automorphism of $\lG$ which sent $\lH$ on itself. So one can find a $\delta_1$ with $0<\delta_1<\delta$ such that
\[
   h\exp({B_{\delta_1}\cap\lH})h^{-1}\subset V.
\]
Indeed, $he^{\lH} h^{-1}\subset\lH$, so that taking $\delta_1<\delta$, we get the strict inclusion. We can choose $\delta_1$ even smaller to satisfy $he^{B_{\delta_1}}h^{-1}\subset N_e$. Since the map $X\to\log(he^{X}h^{-1})$ from $B_{\delta_1\cap\lH}$ to $B_{\delta}\cap\lH$ is regular, the image of $B_{\delta_1}\cap\lH$ is a neighbourhood of $0$ in $\lH$. Thus $he^{B_{\delta_1}\cap\lH}h^{-1}$ is a neighbourhood of $e$ in $V$. Finally, $h\mU h^{-1}\in\mV$ and the last axiom of a topological group is checked.

This is important because there exists a topology on $H$ such that $H$ becomes a topological group and $\mV$ is a family of neighbourhood of $e$ in $H$. In particular, $V$ is a neighbourhood of $e$ in $H$.

For any $z\in G$, we define the map $\dpt{\phi_z}{zN_e}{B_{\delta}}$ by
\begin{equation}
  \phi_z(ze^{x_1X_1+\cdots+x_nX_n})=(x_1,\ldots,x_n),
\end{equation}
and we denote by $\varphi_z$ the restriction of $\phi_z$ to $zV$. If $z\in H$, then $\varphi_z$ sends the neighbourhood $zV$ of $z$ in $H$ to the open set $B_{\delta}\cap\lH$ in $\eR^{n-r}$. Indeed, an element of $zV$ is a $ze^Z$ with $Z\in\lH\cap B_{\delta}$ which is sent by $\varphi_z$ to an element of $\lH\cap B_{\delta}$. (we just have to identify $x_1X_1+\cdots+x_nX_n$ with $(x_1,\ldots,x_n)$).

Moreover, if $z_1,z_2\in H$, the map $\varphi_{z_1}\circ\varphi_{z_2}^{-1}$ is the restriction to an open subset of $\lH$ of $\phi_{z_1}\circ\phi_{z_2}$. Then $\varphi_{z_1}\circ\varphi_{z_2}^{-1}$ is differentiable. Conclusion: $(H,\varphi_z: z\in H)$ is a differentiable manifold.

Recall that the definition of $\lH$ was to be a subalgebra of $\lG$; therefore $V=e^{\lH\cap B_{\delta}}$ is a submanifold of $G$. But the left translations are diffeomorphism of $H$ and $H$ is the smallest subgroup of $G$ containing $e^{\lH}$. Thus $H$ is a manifold on which the multiplication is diffeomorphic and consequently, $H$ is a Lie subgroup of $G$.

Rest to prove that the Lie algebra of $H$ is $\lH$ and the unicity part of the theorem.

We know that $\dim H=\dim\lH$ and moreover for $i>r$, the map $t\to\exp tX_i$ is a curve in $H$. Now, the fact that $\lH$ is the set of $X\in\lG$ such that $t\to\exp tX$ is a path in $H$ show that $X_i\in\lH$. Then the Lie algebra of $H$ is $\lH$ and $H$ is a connected group because it is generated by $\exp\lH$ which is a connected neighbourhood of $e$ in $H$.

We turn our attention to the unicity part. Let $H_1$ be a connected Lie subgroup of $G$ such that $T_eH_1=\lH$. Since $\exp_{\lH}X=\exp_{\lH_1}X$, $H=H_1$ as set. But $\exp$ is a differentiable diffeomorphism from a neighbourhood of $0$ in $\lH$ to a neighbourhood of $e$ in $H$ and $H_1$, so as Lie groups, $H$ and $H_1$ are the same.

Let us consider an element $X\in\lG$ such that $\exp tX\in H$ for every $t\in\eR$, and the map $\dpt{\varphi}{\eR}{G}$, $\varphi(t)=\exp tX$. This is continuous, then there exists a connected neighbourhood $\mU$ of $0$ in $\eR$ such that $\varphi(\mU)\subset V$. Then $\varphi(\mU)\subset H\cap V$ and the connectedness of $\varphi(\mU)$ makes $\varphi(\mU)\subset\exp\mU_h$. But $\exp\mU_h$ is an arbitrary small neighbourhood of $e$ in $H$; the conclusion is that $\varphi$ is a continuous map from $\eR$ into $H$. Indeed, we had chosen $X$ such that $\exp tX\in H$.

Moreover, we know that
\[
  e^{(t_0+\epsilon)X}=e^{t_0X}e^{\epsilon X},
\]
but $\exp \epsilon X$ can be as close to $e$ as we want (this proves the continuity at $t_0$). Then $\varphi$ is a path in $H$.

In definitive, we had shown that $\exp tX\in H$ implies that $t\to\exp tX$ is a path. Now equation \eqref{eq:path_alg} gives the result.
\end{proof}

\begin{corollary}
Let $G$ be a Lie group and $H_1$, $H_2$, two subgroups both having a finite number of connected components (each for his own topology). If $H_1=H_2$ as sets, then $H_1=H_2$ as Lie groups.
\end{corollary}

\begin{proof}
The proposition shows that $H_1$ and $H_2$ have same Lie algebra. But any Lie subalgebra of $\lG$ is the Lie algebra of exactly one connected subgroup of $G$ (theorem~\ref{tho:gp_alg}). Then as Lie groups, ${H_1}_0={H_2}_0$. Since $H_1$ and $H_2$ are topological groups, the equality of they topology on one connected component gives the equality everywhere (because translations are differentiable).
\end{proof}

\label{pg:ex_topo_Lie}
Consider the group $T=S^1\times S^1$ and the continuous map $\dpt{\gamma}{\eR}{T}$ given by
\[
  \gamma(t)=(e^{it},e^{i\alpha t})
\]
with a certain irrational $\alpha$ in such a manner that $\gamma$ is injective and $\Gamma=\gamma(\eR)$ is dense in $T$.

The subset $\Gamma$ is not closed because his complementary in $T$ is not open: any neighbourhood of element $p\in T$ which don't lie in $\Gamma$ contains some elements of $\Gamma$. We will show that the inclusion map $\dpt{\iota}{\Gamma}{T}$ is continuous. An open subset of $T$ is somethings like
\[
  \mO=(e^{iU},e^{iV})
\]
where $U,V$ are open subsets of $\eR$. It is clear that
\[
   \iota^{-1}(\mO)=\{ \gamma(t)\tq t\in U+2k\pi,\alpha t\in V+2m\pi \},
\]
but the set of elements $t$ of $\eR$ which satisfies it is clearly open. Then $\Gamma$ has at least the induced topology from $T$ (as shown in proposition~\ref{prop:topo_sub_manif}). In fact, the own topology of $\Gamma$ is \emph{more} than the induced: the open subsets of $\Gamma$ whose are just some small segments clearly doesn't appear in the induced topology. Thus the present case is an example (and not a counter-example) of theorem~\ref{tho:H_ferme}.

This example show the importance of the condition for a topological subspace to have \emph{exactly} the induced topology. If not, any Lie subgroup were a topological Lie subgroup because a submanifold has at least the induced topology. We will go further with this example after the proof.

%+++++++++++++++++++++++++++++++++++++++++++++++++++++++++++++++++++++++++++++++++++++++++++++++++++++++++++++++++++++++++++ 
\section{Matrix Lie group and its algebra}
%+++++++++++++++++++++++++++++++++++++++++++++++++++++++++++++++++++++++++++++++++++++++++++++++++++++++++++++++++++++++++++
\label{SECooTSAJooNtjgMD}

In this section we deal with Lie groups made from matrices, that is subgroups of \( \GL(n, \eC)\) (typically \( \SO(n)\) or \( \SU(n)\)) and their Lie algebra. We will denote the identity either by \( e\) or by \( \mtu\).

\begin{normaltext}      \label{NORMooHZGKooJEiamo}
    It is time to reread the remark \ref{REMooJQFHooQuoZxt}. In this section, when \( \gamma\) is a path in the matrix group \( G\), we denote by \( \gamma'(0)\) the ``usual'' derivative of \( \gamma\): that is the component-wise derivative; not the differential operator.

    We denote by \( D_{\gamma}\) the differential operator
    \begin{equation}
        \begin{aligned}
            D_{\gamma}\colon  C^{\infty}(G)&\to \eR \\
            f&\mapsto \Dsdd{ f\big( \gamma(t) \big) }{t}{0}. 
        \end{aligned}
    \end{equation}

    We aim to study the link between \( D_{\gamma}\) and \( \gamma'(0)\).

    From the Lie group of matrix \( G\) we can build (at least) two Lie algebras\footnote{Definition \ref{DEFooVBPKooGxlDBn}.}:
    \begin{itemize}
        \item The usual Lie algebra of the group: \( T_eG\) with the definition \ref{DEFooKDCPooZOJsMD}. As set, this is
            \begin{equation}
                T_eG=\{ D_{\gamma}\st \gamma(0)=e \}
            \end{equation}
            with the implicit that \( \gamma\) is a smooth path in \( G\).
        \item 
            The set of ``usual'' derivatives of the paths in \( G\):
            \begin{equation}
                G'=\{ \gamma'(0)\tq \gamma(0)=e \}.
            \end{equation}
            This is a set of matrices on which we can use the bracket \( [X,Y]=XY-YX\) (matrix product). We will see the following facts.
            \begin{itemize}
                \item 
                    The set \( G'\) is a Lie algebra in proposition \ref{PROPooUKITooLnEKZW},
                \item
                    The Lie algebras \( G'\) and \( T_eG\) are isomorphic as Lie algebras in theorem \ref{THOooWQGMooHyjRtx} for the case \( G=\GL(n,\eC)\)
                \item
                    When \( H\) is a Lie subgroup of \( \GL(n,\eC)\), the Lie algebras \( H'\) and \( T_eH\) are isomorphic as Lie algebras in proposition \ref{PROPooSQHLooGQAykc} for the Lie subgroups of \( \GL(n,\eC)\).
            \end{itemize}
    \end{itemize}
\end{normaltext}

\begin{lemma}[\cite{MonCerveau}]
    Let \( G\) be a matrix Lie group, et \( g\in G\) and \( X\in G'\). Then \( gXg^{-1}\in G'\).
\end{lemma}

\begin{proof}
    Let \( x\colon \eR\to G\) be a smooth path such that \( X=x'(0)\). Then we the derivative of the path given by the matrix product
    \begin{equation}
        t\mapsto gx(t)g^{-1}
    \end{equation}
    is \( gXg^{-1}\).
\end{proof}

\begin{lemma}[\cite{MonCerveau}]        \label{LEMooHQUYooSoiKbI}
    Let \( G\) be a matrix Lie group. Then \( G'\) is a vector space on \( \eR\).
\end{lemma}

\begin{proof}
    Let \( X,Y\in G'\) be the derivatives of the paths \( x\) and \( y\). If we set \( \varphi_1(t)=x(t)y(t)\) we have
    \begin{equation}
        \varphi_1'(0)=x'(0)y(0)+x(0)y'(0).
    \end{equation}
    Since \( x(0)=y(0)=e\) we have \( \varphi'(0)=X+Y\), so that \( X+Y\in G'\).

    For the product by a scalar, let the path \( \varphi_2(t)=x(\lambda t)\). The component-wise derivative
    \begin{equation}
        \varphi_2'(0)=\lambda x'(0)=\lambda X,
    \end{equation}
    so that \( \lambda X\in G'\).
\end{proof}

\begin{proposition}     \label{PROPooUKITooLnEKZW}
    Let \( G\) be a matrix Lie group. The vector space \( G'\) is a Lie algebra for the matrix commutator.
\end{proposition}

\begin{proof}
    We already know that \( G'\) is a real vector space by lemma \ref{LEMooHQUYooSoiKbI}. The fact that \( (X,Y)\mapsto XY-YX\) satisfies the axioms of a Lie algebra is easy to check. The only point is to show that if \( X,Y\in G'\), then \( [X,Y]=XY-YX\in G'\).

    Let
    \begin{equation}        \label{EQooJDTLooGWsDiq}
        \varphi(t)=x(t)Yx(-t).
    \end{equation}
    This is for sure a path in the full matrix vector space, and this is derivable because \( x\) is derivable while the matrix product is linear. So the derivative \( \varphi'(0)\) is still a matrix. The question is: why \( \varphi'(0)\in G'\) ?

    By lemma \ref{LEMooHQUYooSoiKbI}, for each \( t\) we have
    \begin{equation}
        \frac{ \varphi(t)-\varphi(0) }{ t }\in G'.
    \end{equation}
    Now, \( G'\) is a vector subspace of \( \eM(n,\eC)\) which is finite dimensional; is is thus closed and the limit belongs to \( G'\).

    Is is now a simple computation to show that \( \varphi'(0)=[X,Y]\).
\end{proof}

\begin{normaltext}
The following theorem is a Giulietta's masterpiece in the following sense:
\begin{itemize}
    \item It is fundamental because the Lie algebra isomorphism between \( T_eGL(n,\eR)\) and the matrices is used everywhere one says «The Lie algebra of $\SO(3)$ is the set of skew-symmetric traceless matrices».
    \item
        Either I'm idiot, either I never seen that theorem even stated (let alone being proved)\footnote{There is in fact a third possibility:  this theorem is a classic one but cannot be found \emph{on internet}.}.
    \item
        I think that the fundamental misunderstanding\footnote{Once again, either I'm idiot either everybody is wrong but me\ldots well \ldots} is that in the context of Lie groups, people \emph{define} \( [X,Y]\) as being \( \ad(X)Y\) while \( \ad\) is defined as the ``second differential'' of \( \AD(g)h=ghg^{-1}\). In that case, obviously we get \( [X,Y]=XY-YX\) with the matrix product. This way fails to make the link with the commutator of vector fields as defined by \ref{DEFooHOTOooRaPwyo}.
    \item
        So you must read this proof with much care and write me if you see any mistake or unclear point.
\end{itemize}
\end{normaltext}
So here it is with the notations explained in \ref{NORMooHZGKooJEiamo}.
    

\begin{theorem}     \label{THOooWQGMooHyjRtx}
    Let \( G=\GL(n,\eC)\) be the group of invertible matrices. The map
    \begin{equation}
        \begin{aligned}
            \phi\colon G'&\to T_eG \\
            \gamma'(0)&\mapsto D_{\gamma} 
        \end{aligned}
    \end{equation}
    is 
    \begin{enumerate}
        \item
            well defined,
        \item
            bijective,
        \item
            linear,
        \item
            a Lie algebra isomorphism.
    \end{enumerate}
\end{theorem}

\begin{proof}
    Several points to be proved.
    \begin{subproof}
        \spitem[\( \phi\) is well defined]
            Let \( \alpha\) and \( \beta\) be paths in \( G\) such that \( \alpha'(0)=\beta'(0)\) and let \( f\colon G\to \eR\) be a smooth function. We have to prove that \( D_{\alpha}(f)=D_{\beta}(f)\).

            We consider a chart \( \varphi\colon \mU\to \mO\) where \( \mU\) is a neighbourhood of \( 0\) in \( \eR^m\) and \( \mO\) is a neighbourhood of \( e\) in \( \GL(n,\eC)\). We suppose that \( \varphi(0)=e\). We set \( \tilde f=f\circ \varphi\), \( \tilde \alpha=\varphi^{-1}\circ \alpha\) and \( \tilde \beta=\varphi^{-1}\circ\beta\). We have
            \begin{subequations}
                \begin{align}
                    D_{\alpha}(f)&=\Dsdd{ f\big( \alpha(t) \big) }{t}{0}\\
                    &=\Dsdd{ \tilde f\big( \tilde \alpha(t) \big) }{t}{0}\\
                    &=\sum_{i=1}^m\frac{ \partial \tilde f }{ \partial x_i }\big( \tilde \alpha(0) \big)\tilde \alpha_i(0).
                \end{align}
            \end{subequations}
            Since \( \tilde \alpha(0)=\tilde \beta(0)\) we still have to prove that \( \tilde \alpha_i'(0)=\tilde \beta_i'(0)\). As you remember, \( \tilde \alpha\) is a map from \( \eR\) to \( \eR^m\), so that the following derivative is quite usual:
            \begin{subequations}
                \begin{align}
                    \tilde \alpha'(0)&=\Dsdd{ (\varphi^{-1}\circ \alpha)(t) }{t}{0}\\
                    &=d\varphi^{-1}_{\alpha(0)}\big( \alpha'(0) \big)\\
                    &=d\varphi^{-1}_{\beta(0)}\big( \beta'(0) \big).
                \end{align}
            \end{subequations}
            Thus the map \( \phi\) is well defined.
        \spitem[\( \phi\) is linear]
            This is from the linearity of the derivation.
        \spitem[\( \phi\) is injective]
            If \( \phi(\alpha')=\phi(\beta')\), then \( D_{\alpha}(f)=D_{\beta}(f)\) for every function \( f\). In that case,
            \begin{equation}
                \sum_{i=1}^m\frac{ \partial \tilde f }{ \partial x_i }(e)\tilde \alpha_i'(0)=\sum_{i=1}^m\frac{ \partial \tilde f }{ \partial x_i }(e)\tilde \beta_i'(0).
            \end{equation}
            That equation must be satisfied for every function. Taking the projection on the components, we get \( \tilde \alpha_i'(0)=\tilde b_i'(0)\), which means \( \alpha'(0)=\beta'(0)\) because \( \varphi^{-1}\) is bijective.
        \spitem[\( \phi\) is surjective]
            Every element of \( T_eG\) is of the form \( D_{\alpha}\) for some path \( \alpha\), so \( \phi\) is surjective.
        \spitem[\( \phi\) is a Lie algebra isomorphism]
            Let \( X,Y\in G'\) being the derivative of the paths \( \alpha\) and \( \beta\). We have to prove that
            \begin{equation}
                [\phi(X),\phi(Y)]=\phi[X,Y].
            \end{equation}
            If \( t\) is small enough, the paths
            \begin{subequations}
                \begin{align}
                    \alpha(t)=\mtu+tX\\
                    \beta(t)=\mtu+tY\\
                \end{align}
            \end{subequations}
            are good ones because \( \det(\mtu)\neq 0\), so that the determinant of \( \mtu+tX\) remains different from zero when \( t\) is small, whatever \( X\) is. So \( \alpha\) and \( \beta\) are paths in \( \GL(n,\eC)\). Using the general definition in differential geometry,
            \begin{subequations}        \label{SUBEQSooCYRDooFOdLrn}
                \begin{align}
                    [\phi(X),\phi(Y)]f&=[\phi(X)^L,\phi(Y)^L]_ef\\
                    &=\phi(X)^L_e\big( \phi(Y)^L(f) \big)-\phi(Y)^L_e\big( \phi(X)^L(f) \big) \label{SUBEQooOPUAooZYsZlX}.
                \end{align}
            \end{subequations}
            We focus on the first term:
            \begin{subequations}        \label{SUBEQooTUNFooFkDmuP}
                \begin{align}
                    \phi(X)^L\big( \phi(Y)^L(f) \big)&=\Dsdd{ \phi(Y)^L_{\phi(X)^L_e(t)}(f) }{t}{0}\\
                    &=\DDsdd{ f\big( (\mtu+tX)(\mtu+sY) \big) }{t}{0}{s}{0}\\
                    &=\DDsdd{ f(\mtu+tX+sY+tsXY) }{t}{0}{s}{0}\\
                    &=\Dsdd{ df_{\mtu+tX}\big( (\mtu+tX)Y \big) }{t}{0} \label{SUBEQooLHPBooTnXiZd}\\
                    &=\Dsdd{ df_{\mtu+tX}(Y) }{t}{0}+\Dsdd{ df_{\mtu+tX}(tXY) }{t}{0}   \label{SUBEQooMXJJooBFTLsM}
                \end{align}
            \end{subequations}
            where we have used the linearity of \( df_{\mtu+tX}\) and where \( XY\) stands for the matrix product. In the expression \eqref{SUBEQooLHPBooTnXiZd}, the symbol \( df\) stands for the differential of \( f\) as function from \( \eM(n,\eC)\) (as vector space), not for the differential of \( f\) on \( G\) as manifold. This is why we are allowed to put an expression as the matrix \( Y\) as argument of \( df_{\mtu+tX}\) while \( Y\) is not an element of \( T_{\mtu+tX}G\).

            The expression \eqref{SUBEQooMXJJooBFTLsM} is still made of two terms. The second one is
            \begin{equation}
                \Dsdd{ df_{\mtu+tX}(tXY) }{t}{0}=\Dsdd{ tdf_{\mtu+tX}(XY) }{t}{0}=df_{\mtu}(XY)
            \end{equation}
            where we used the Leibnitz rule\footnote{In general, notice that \( \Dsdd{ tf(t) }{t}{0}=f(0)\)}.

            The first term in \eqref{SUBEQooMXJJooBFTLsM} is computed as
            \begin{equation}
                    \Dsdd{ df_{\mtu+tX}(Y) }{t}{0}=\DDsdd{ f(\mtu+tX+sY) }{t}{0}{s}{0}.
            \end{equation}
            We set 
            \begin{equation}
                \begin{aligned}
                    \gamma\colon \eR^2&\to G \\
                    (t,s)&\mapsto \mtu+tX+sY, 
                \end{aligned}
            \end{equation}
            so that
            \begin{subequations}
                \begin{align}
                    \Dsdd{ df_{\mtu+tX}(Y) }{t}{0}&=\DDsdd{ f(\mtu+tX+sY) }{t}{0}{s}{0}\\
                    &=\DDsdd{ (\tilde f\circ\varphi^{-1}\circ\gamma)(t,s) }{t}{0}{s}{0}\\
                    &=\DDsdd{ g(t,s) }{t}{0}{s}{0}
                \end{align}
            \end{subequations}
            where the function \( g=\tilde f\circ\varphi^{-1}\circ \gamma\) is a smooth function from \( \eR^2\) to \( \eR\).        

            The expression \eqref{SUBEQooTUNFooFkDmuP} is now
            \begin{equation}
                \phi(X)^L\big( \phi(Y)^L(f) \big)=\DDsdd{ g(t,s) }{t}{0}{s}{0}+df_{\mtu}(XY).
            \end{equation}
            The commutator we have to compute, with the same computations is
            \begin{equation}
                [\phi(X),\phi(Y)]f=\DDsdd{ g(t,s) }{t}{0}{s}{0}+df_{\mtu}(XY)-\DDsdd{ g(s,t) }{t}{0}{s}{0}-df_{\mtu}(YX).
            \end{equation}
            The function \( g\) being \(  C^{\infty}\), the derivative commute and the corresponding termes annihilate each other and we are left with
            \begin{equation}
                [\phi(X),\phi(Y)]f=df_{\mtu}(XY)-df_{\mtu}(YX)=df_{\mtu}(XY-YX)
            \end{equation}
            where we used the linearity of the differential.

            In the other sense,
            \begin{equation}
                \phi[X,Y]f=\Dsdd{ f(\mtu+tXY-tYX) }{t}{0}=df_{\mtu}\big( [X,Y] \big)
            \end{equation}
            where, once again, \( df\) stands for the ``usual'' differential.
    \end{subproof}
\end{proof}

Ok. This is proved for \( G=\GL(n,\eC)\), the full matrix group. What about subgroups ? Here is the result.

\begin{proposition}[\cite{MonCerveau}]      \label{PROPooSQHLooGQAykc}
    Let \( H\) be a closed Lie subgroup of \( \GL(n,\eC)\). With the same notations as above, the map
    \begin{equation}
        \begin{aligned}
            \phi\colon H'&\to T_eH \\
            \gamma'(0)&\mapsto D_{\gamma} 
        \end{aligned}
    \end{equation}
    is a Lie algebra isomorphism.
\end{proposition}

\begin{proof}
    We have to prove that
    \begin{equation}        \label{EQooRLBBooYgHhtH}
        \phi[X,Y]f=[\phi(X),\phi(Y)]f
    \end{equation}
    for every \( X,Y\in H'\) and \( f\in  C^{\infty}(H)\). For that, we will see the left and right hand sides of \eqref{EQooRLBBooYgHhtH} in \( G=\GL(n,\eC)\), and use the already proved result, theorem \ref{THOooWQGMooHyjRtx}.

    If \( X,Y\in H'\) we know from proposition \ref{PROPooUKITooLnEKZW} that \( [X,Y]\in H'\). Thus there exists a path \( \gamma\colon \eR\to H\) such that \( [X,Y]=\gamma'(0)\). We consider the extension\footnote{The proposition \ref{PROPooOTZQooIfboXV} can be used since \( H\) is a submanifold of \( G\) by \ref{PROPooFXZJooCOFXZX}.} \( \tilde f\colon W\to \eR\) of \( f\) such that \( \tilde f=f\) on \( H\) and \( W\) is an open set around \( e\) in \( \GL(n,\eC)\). For the sake of making things complicated we also define \( \tilde \gamma=\iota\circ \gamma\) where \( \iota\colon H\to \GL(n,\eC)\) is the inclusion. With all that we have
    \begin{equation}
        \phi[X,Y]f=\Dsdd{ f\big( \gamma(t) \big) }{t}{0}=\Dsdd{ \tilde f\big( \tilde \gamma(t) \big) }{t}{0}=\clubsuit.
    \end{equation}
    At this point, notice that \( [X,Y]\in \GL(n,\eC)'\) and \( [X,Y]=\tilde \gamma'(0)\), so that if we consider the map \( \tilde \phi\colon \GL(n,\eC)\to T_e\GL(n,\eC)\) we also have
    \begin{equation}
        \clubsuit=\Dsdd{ \tilde f\big( \tilde \gamma(t) \big) }{t}{0}=\tilde \phi[X,Y]\tilde f=\big[ \tilde \phi(X),\tilde \phi(Y) \big]\tilde f
    \end{equation}
    where we used the result \ref{THOooWQGMooHyjRtx} on \( \GL(n,\eC)\).

    We still have to prove that \( \tilde \phi(X)\tilde \phi(Y)\tilde f=\phi(X)\phi(Y)f\). Using, among others the formula \ref{SUBEQSooHKWMooQbeStl} adapted to \( \tilde \phi(X)\) instead of \( X\):
    \begin{subequations}
        \begin{align}
            \tilde \phi(X)\tilde \phi(Y)\tilde f&=\Dsdd{ \big( \tilde \phi(Y)^L\tilde f \big)\big( \alpha(t) \big) }{t}{0}\\
            &=\Dsdd{ \tilde \phi(Y)^L_{\alpha(t)}\tilde f }{t}{0}\\
            &=\DDsdd{ \tilde f\big( \alpha(t)\beta(u) \big) }{t}{0}{s}{0}.
        \end{align}
    \end{subequations}
    At this point, notice that \( \alpha(t)\) and \( \beta(u)\) are elements in \( H\) which is a group, so \( \tilde f\big( \alpha(t)\beta(u) \big)=f\big( \alpha(t)\beta(u) \big)\). Thus
    \begin{subequations}
        \begin{align}
            \tilde \phi(X)\tilde \phi(Y)\tilde f&=\DDsdd{ \tilde f\big( \alpha(t)\beta(u) \big) }{t}{0}{s}{0}\\
            &=\DDsdd{ f\big( \alpha(t)\beta(u) \big) }{t}{0}{s}{0}\\
            &=\phi(X)\phi(y)f.
        \end{align}
    \end{subequations}
\end{proof}

\begin{lemma}[\cite{MonCerveau}]
    Let \( G\) be a Lie group of matrices and \( X\in T_eG\) such that 
    \begin{equation}
        df_e(X)=0
    \end{equation}
    for every smooth function \( f\colon G\to \eR\). Then \( X=0\).
\end{lemma}

\begin{proof}
    We consider the functions \( \pr_{ij}\colon G\to \eR\) defined by \( \pr_{ij}(A)=A_{ij}\). If \( g\colon \eR\to G\) is a path, for every \( t\) we have \( \pr_{ij}g(t)=g(t)_{ij}\) and then
    \begin{equation}
        \Dsdd{ \pr_{ij}g(t) }{t}{0}=g'(0)_{ij}.
    \end{equation}
    Then we build
    \begin{equation}
        \begin{aligned}
            f\colon G&\to \eR \\
            A&\mapsto \pr_{11}(A)\pr_{ij}(A). 
        \end{aligned}
    \end{equation}
    If \( g\colon \eR\to G\) is a path such that \( g(0)=e\) and \( g'(0)=X\), then we have
    \begin{subequations}
        \begin{align}
            \Dsdd{ f\big( g(t) \big) }{t}{0}&=\Dsdd{ \pr_{11}\big( g(t) \big)\pr_{ij}\big( g(t) \big) }{t}{0}\\
            &=\pr_{11}g(0)\Dsdd{ \pr_{ij}g(t) }{t}{0}+\Dsdd{ \pr_{11}g(t) }{t}{0}\pr_{ij}g(0)\\
            &=X_{ij}+\delta_{ij}X_{11}\\
            &=X_{ij}+\delta_{ij}X_{11}.
        \end{align}
    \end{subequations}
    We know that this is zero for every choice of \( ij\):
    \begin{equation}
        X_{ij}+\delta_{ij}X_{11}=0
    \end{equation}
    In particular with \( i=j=1\) we have \( 2X_{11}=0\), so that \( X_{11}=0\). Then we are left with \( X_{ij}=0\) for every \( ij\).
\end{proof}

\section{Fundamental vector field}\label{sec:fond_vec}
%++++++++++++++++++++++++++++++++++++

\begin{definition}
    If $\yG$ is the Lie algebra\footnote{Lie algebra of a Lie group, definition \ref{DEFooKDCPooZOJsMD}.} of a Lie group $G$ acting on a manifold $M$ (the action of $g$ on $x$ being denoted by $x\cdot g$), the \defe{fundamental vector field}{fundamental!vector field} associated with $A\in\yG$ is given by
    \begin{equation}			\label{EqDefChmpFond}
       A^*_x=\Dsdd{ x\cdot e^{-tA} }{t}{0}.
    \end{equation}
\end{definition}

If the action of $G$ is transitive, the fundamental vectors at point $x\in M$ form a basis of $T_xM$. More precisely, we have the

\begin{lemma}
For any $v\in T_xM$, there exists a $A\in\yG$ such that $v=A^*_x$, in other terms
\[
  \Span\{ A^*_{x}\tq A\in\yG \}=T_{x}M.
\]
\label{LemFundSpansTan}
\end{lemma}

\begin{proof}
The vector $v$ is given by a path $v(t)$ in $M$. Since the action is transitive, one can write $v(t)=x\cdot c(t)$ for a certain path $c$ in $G$ which fulfills $c(0)=e$. We have to show that $v$ depends only on $c'(0)\in\yG$. We consider
\begin{equation}  \label{eq_def_RGM}
\begin{aligned}
 R\colon G\times M&\to M \\
R(g,x)&= x\cdot g,
\end{aligned}
\end{equation}
so
\begin{equation}\label{eq:v_Rc}
   v=\Dsdd{ R(c(t),x) }{t}{0}=dR_{(e,x)}\big[  (d_tc(t),x)+(c(0),x)   \big].
\end{equation}

\end{proof}

\begin{lemma}\label{lem:As_Bs_A_B}
If $A$, $B\in\yG$ are such that $A^*=B^*$, and if the action is effective, then $A=B$.
\end{lemma}

\begin{proof}
 We consider once again the map \eqref{eq_def_RGM} and we look at
\[
  v=\Dsdd{ R(c(t),x) }{t}{0}
   =(dR)_{(e,x)}\Dsdd{ (c(t),x) }{t}{0},
\]
keeping in mind that $c(t)=e^{-tA}$. In order to treat this expression, we define
\begin{subequations}
\begin{align}
  \dpt{R_1}{G}{M},\quad  R_1(h)&=R(h,x),\\
  \dpt{R_2}{M}{M},\quad  R_2(y)&=R(g,y).
\end{align}
\end{subequations}
So
\[
  v=dR_1(X)+dR_2(0)=dR_1c'(0)
\]
and the assumption $A^*_x=B^*_x$ becomes $dR_1 A=dR_1 B$. This makes, for small enough $t$, 
\begin{equation}
    R_1(e^{tA}e^{-tB})=x\cdot e^{tA}e^{-tB}=x; 
\end{equation}
if the action is effective, it imposes $A=B$.
\end{proof}

\begin{lemma}
If we consider the action of a matrix group, $R_g$ acts on the fundamental field by
\[
  dR_g(A^*_{\xi})=\big( \Ad(g^{-1})A \big)^*_{\xi\cdot g}.
\]
\label{lem:dRgAstar}
\end{lemma}

\begin{proof}
Just notice that $e^{-t\Ad(g^{-1})A}=\AD_{g^{-1}}(e^{-tA})=g^{-1} e^{-tA}g$, thus
\begin{equation}
  \big( \Ad(g^{-1})A \big)^*_{\xi\cdot g}=\Dsdd{ \xi\cdot ge^{-t\Ad(g^{-1})A} }{t}{0}=dR_g(A^*_{\xi}).
\end{equation}
\end{proof}

%+++++++++++++++++++++++++++++++++++++++++++++++++++++++++++++++++++++++++++++++++++++++++++++++++++++++++++++++++++++++++++ 
\section{Invariant vector fields}
%+++++++++++++++++++++++++++++++++++++++++++++++++++++++++++++++++++++++++++++++++++++++++++++++++++++++++++++++++++++++++++

\begin{definition}
    Let \( G\) be a group. The \defe{left translation}{left translation} by \( g\) on \( G\) is the map
    \begin{equation}
        \begin{aligned}
            L_g\colon G&\to G \\
            h&\mapsto gh. 
        \end{aligned}
    \end{equation}
    The \defe{right translation}{right translation} by \( g\) on \( G\) is the map
    \begin{equation}
        \begin{aligned}
            R_g\colon G&\to G \\
            h&\mapsto hg. 
        \end{aligned}
    \end{equation}
\end{definition}

\begin{definition}[\cite{BIBooUGWHooPbodCu}]        \label{DEFooYHKXooVoJalX}
    If $G$ is a Lie group, a vector field $X\in\Gamma^{\infty}(TG)$ is \defe{left invariant}{left invariant!vector field} if
    \begin{equation}
        (dL_g) X= X,
    \end{equation}
    which means that for every \( g,h\in G\),
    \begin{equation}
        (dL_h)_gX_g=X_{hg}.
    \end{equation}
    In the same way, the vector field \( Y\) is \defe{right invariant}{right!invariant!vector field} if
    \begin{equation}
        (dR_g)Y=Y.
    \end{equation}
    \index{invariant vector field}
\end{definition}

When \( X\in T_eG\), we define the associated left-invariant vector field \( X^L\) by
\begin{equation}        \label{DEFooYPUIooAzcdjP}
    X^L_g=(dL_g)_eX.
\end{equation}

\begin{lemma}[\cite{BIBooFLEXooPgvAlz}]       \label{LEMooWTNRooCjlYMJ}
    Let \( G\) be a Lie group and \( X_1,\ldots, X_k\) be linearly independent vectors in its Lie algebra \( \lG\). For every \( g\in G\), the vectors \( X_i^L(g)\) are linearly independent in \( T_gG\).
\end{lemma}

\begin{proof}
    Let \( \lambda_i\) be reals such that \( \sum_{i=1}^k\lambda_iX_i^L(g)=0\); in other terms,
    \begin{equation}
        0=(dL_g)_e\big( \sum_i\lambda_iX_i \big).
    \end{equation}
    We apply \( (dL_{g^{-1}})_e\) on both side: \( 0=\sum_i\lambda_iX_i\). By hypothesis, \( \lambda_i=0\) for every \( i\). This proves that the vectors \( X^L_i(g)\) are linearly independent.
\end{proof}

\begin{theorem}[\cite{BIBooUGWHooPbodCu}]
	The map \( \varphi\colon X\mapsto X^L\) where \( X^L_g=(dL_g)_eX\) is a bijection from \( T_eG\) to the set of left-invariant vector fields.
\end{theorem}

\begin{proof}
    Two parts.
    \begin{subproof}
        \spitem[Surjective]
            Let \( X\) be a left-invariant vector field. We have \( X=(X_e)^L\) because
            \begin{equation}
                (X_e)^L_g=(dL_g)X_e=X_g.
            \end{equation}
            The first equality is the definition of the left-invariant associated vector field (equation \eqref{DEFooYPUIooAzcdjP} applied to \( X_e\)) and the second equality is the fact that \( X\) is left-invariant. Thus \( X\) is the left-invariant vector field associated with \( X_e\).
        \spitem[Injective]
            Let \( X,Y\in T_eG\) be such that \( X^L=Y^L\). In particular \( X^L_e=Y^L_e\), which means \( X=Y\).
    \end{subproof}
\end{proof}

\begin{proposition}[\cite{BIBooUGWHooPbodCu, MonCerveau}]
    Let \( G\) be a Lie group. The map
    \begin{equation}
        \begin{aligned}
            \varphi\colon G\times \lG&\to TG \\
            (g,X)&\mapsto X^L_g 
        \end{aligned}
    \end{equation}
    is a bijection.

    Moreover for each \( g\in G\), the map
    \begin{equation}
        \begin{aligned}
            \varphi_g\colon \lG&\to T_gG \\
           X&\mapsto X^L_g 
        \end{aligned}
    \end{equation}
    is a vector space isomorphism.
\end{proposition}

\begin{proof}
    Several points.
    \begin{subproof}
        \spitem[\( \varphi\) is surjective]
            Let \( X\in TG\); there is some \( g\in G\) such that \( X\in T_gG\). Since \( X=(dL_g)_e(dL_{g^{-1}})_gX\) we have
            \begin{equation}
                X=(dL_{g^{-1}}X)^L_g=\varphi(g,dL_{g^{-1}}X).
            \end{equation}
        \spitem[\( \varphi\) is injective]
            If \( \varphi(g,X)=\varphi(h,Y)\), we have \( X_g^L=Y^L_h\), so that \( g=h\). The equality  \( X_g^L=Y_g^L\) means \( (dL_g)_eX=(dL_g)_eY\). Applying \( (dL_{g^{-1}})_g\) on both sides we get \( X=Y\).
        \spitem[\( \varphi_g\) is bijective]
            These are the same verifications.
        \spitem[\( \varphi_g\) is linear]
            The map \( \varphi_g\) is nothing else than \( (dL_g)_e\), so it is linear.
    \end{subproof}
\end{proof}

%+++++++++++++++++++++++++++++++++++++++++++++++++++++++++++++++++++++++++++++++++++++++++++++++++++++++++++++++++++++++++++
\section{Exponential map}
%+++++++++++++++++++++++++++++++++++++++++++++++++++++++++++++++++++++++++++++++++++++++++++++++++++++++++++++++++++++++++++

%--------------------------------------------------------------------------------------------------------------------------- 
\subsection{Integral curve}
%---------------------------------------------------------------------------------------------------------------------------

The integral curve of a vector field on a manifold is given by definition \ref{PROPooJACTooXBSxfE}. Here we are dealing with special manifolds: Lie groupe. From definition \ref{DEFooYHKXooVoJalX} we know that one can create vector fields (invariant) on \( G\) from an element of \( T_eG\).

We want to prove that the vector space of left invariant vector fields is isomorphic to the tangent vector space \( T_eG\) to \( G\) at identity. If \( X\in T_eG\), we introduce the left invariant vector field \( X^L=dLX\), more explicitly:
\begin{equation}
    X^L_g=\Dsdd{ gX(t) }{t}{0}.
\end{equation}
Then we consider \( \alpha_X\colon I\to G\) the integral curve of maximal length to \( X^L\) trough \( X_e\). Here, \( I\) is the interval on which \( \alpha_X\) is defined. This is the solution of
\begin{subequations}        \label{SUBEQSooBJUVooEeOVVA}
    \begin{numcases}{}
        \Dsdd{ \alpha_X(t_0+t) }{t}{0}=X_{\alpha_X(t_0)}\\
        \alpha_X(0)=e.
    \end{numcases}
\end{subequations}

\begin{proposition}     \label{PROPooWEYCooCvyHNr}
    Let \( X\in T_eG\). The integral curve of \( X^L\) is defined on \( \eR\) and for every \( s,t\in\eR\),
    \begin{equation}
        \alpha_X(s+t)=\alpha_X(s)\alpha_X(t).
    \end{equation}
\end{proposition}

\begin{proof}
    Let \( \alpha\) be any integral curve for \( X^L\) and \( y\in G\). If we put \( \alpha_1(t)=y\alpha(t)\), we have
    \begin{equation}
        \Dsdd{ \alpha_1(t) }{t}{0}=X^L_y,
    \end{equation}
    so that \( \alpha_1\) is an integral curve for \( X^L\) trough the point \( y\).

    Let now \( I\) be the maximal domain of \( \alpha_X\), and \( t_1\in I\). If we set \( x_1=\alpha_X(t_1)\), the path
    \begin{equation}
         \alpha_1(t)=x_1\alpha_X(t)
    \end{equation}
    is an integral curve of \( X^L\) trough \( x_1\) and has the same maximal definition domain \( I\). On the other hand, the maximal integral curve starting at \( e\) being \( \alpha_X\), the maximal integral curve starting at \( \alpha_X(t_1)\) is
    \begin{equation}
        \alpha_2\colon t\mapsto \alpha_X(t+t_1).
    \end{equation}
    Its domain is \( I-t_1\), but since it starts at \( x_1\), it has to be the same as \( \alpha_1\), then \( I\subset I-t_1\) which proves that \( I=\eR\).

    For each \( s\) and \( t\) in \( \eR\), the maximal integral curve starting at \( \alpha_X(s)\) can be written as
    \begin{equation}
        c(t)=\alpha_X(s)\alpha_X(t)
    \end{equation}
    as well as
    \begin{equation}
        d(t)=\alpha_X(s+t),
    \end{equation}
    so again by unicity, \( \alpha_X(s+t)=\alpha_X(s)\alpha_X(t)\).
\end{proof}

\begin{proposition} \label{PROPooUXFQooIwimav}
    The flow\footnote{Definition \ref{PROPooJACTooXBSxfE}.} of a smooth left-invariant vector field \( X\) is given by
    \begin{equation}
        \Phi(t,g)=g\Phi(t,e).
    \end{equation}
\end{proposition}

\begin{proof}
    In few steps.
    \begin{subproof}
        \spitem[Left invariance]
        % -------------------------------------------------------------------------------------------- 
         
        The left invariance means that $(dL_g)_hX_h=X_{gh}$ for every \( h\in G\). We write that condition with \( h=\Phi_e(t_0)\):
        \begin{equation}        \label{EQooPGWKooWNKslJ}
            (dL_g)_{\Phi_e(t_0)}X_{\Phi_e(t_0)}(f)=X_{g\Phi_e(t_0)}(f).
        \end{equation}
        for every smooth function \( f\colon M\to \eR\).
        \spitem[The path]
        % -------------------------------------------------------------------------------------------- 
        In order to prove that \( \Phi_g(t)=\Phi_e(t)\) we consider the path
        \begin{equation}
            \begin{aligned}
                \gamma\colon I&\to M \\
                t&\mapsto g\Phi_e(t) 
            \end{aligned}
        \end{equation}
        and we prove that it satisfy the properties to be the integral curve of \( X\) at \( g\), that is the two conditions of definition \ref{PROPooJACTooXBSxfE}.
        \spitem[First condition]
        % -------------------------------------------------------------------------------------------- 
        This is the easy one: \( \gamma(0)=g\Phi_e(0)=ge=g\).
        \spitem[Second condition]
        % -------------------------------------------------------------------------------------------- 
        We apply \( \gamma'(0)\) to a function \( f\):
        \begin{subequations}
            \begin{align}
                \gamma'(0)f&=\Dsdd{ (f\circ \gamma)(t) }{t}{t_0}\\
                &=\Dsdd{ f\big( g\Phi_e(t) \big) }{t}{t_0}\\
                &=\Dsdd{ (f\circ L_g)\big( \Phi_e(t) \big) }{t}{t_0}\\
                &=(dL_g)_{\Phi_e(t_0)}\big( \Phi_e'(t_0) \big)f     &\text{proposition \ref{PROPooALATooGgcVQV}}\\
                &=(dL_g)_{\Phi_e(t_0)}X_{\Phi_e(t_0)}(f)        \label{SUBEQooOMDOooFZaCLp}\\
                &=X_{g\Phi_e(t_0)}(f)       &\text{by \eqref{EQooPGWKooWNKslJ}}\\
                &=X_{\gamma(t_0)}(f).
            \end{align}
        \end{subequations}
        Justifications.
        \begin{itemize}
            \item For \eqref{SUBEQooOMDOooFZaCLp}, by definition of an integral curve, \( \Phi'_e(t_0)=X_{\Phi_e(t_0)}\)
        \end{itemize}
    \end{subproof}
\end{proof}

\begin{proposition}
    The flows of \( X^L\) and \( X^R\) are defined on \( \eR\).
\end{proposition}

%---------------------------------------------------------------------------------------------------------------------------
\subsection{Integral curve and exponential}
%---------------------------------------------------------------------------------------------------------------------------

\begin{definition}[Exponential map]     \label{DEFooOLLZooMHRgsz}
    If \( G\) is a Lie group with algebra \( \lG\), we define the \defe{exponential}{exponential from a Lie algebra} is the map
    \begin{equation}\label{EqdefExpoLieTgFGp}
        \begin{aligned}
            \exp\colon \lG&\to G \\
            X&\mapsto \Phi^{X^L}_e(1)
        \end{aligned}
    \end{equation}
    where \( \Phi\) is the flow defined in \ref{PROPooJACTooXBSxfE}.
\end{definition}

\begin{normaltext}
    This definition works on Lie groups thanks to the group structure that allows to build a natural vector field \( X^L\) from the data of a single vector \( X\). On general manifolds, one has not a notion of exponential. However, if one has a Riemannian manifold, one consider the geodesic.

    In the case of groups for which the Killing form defines a scalar product, the notion of exponential associated with the Riemannian structure propagated from the Killing form coincides with the definition \eqref{EqdefExpoLieTgFGp}.
\end{normaltext}


\begin{proposition}     \label{PROPooMAGKooInwNom}
    The map \( \exp\colon \lG\to G\) is continuous.
\end{proposition}

%--------------------------------------------------------------------------------------------------------------------------- 
\subsection{Flow and exponential}
%---------------------------------------------------------------------------------------------------------------------------

\begin{lemma}       \label{LEMooEQFHooRjUAin}
    We have
    \begin{equation}
        \exp(0)=e.
    \end{equation}
\end{lemma}

The following proposition is a generalization of \ref{PROPooKDKDooCUpGzE}.
\begin{proposition}     \label{PROPooNRVJooEDCpOI}
    If \( X\in \lG\) and \( s,t\in \eR\) we have
    \begin{equation}
        e^{sX} e^{tX}= e^{(s+t)X}.
    \end{equation}
\end{proposition}

\begin{lemma}       \label{LEMooRPHVooAtZJnz}
    Let \( G\) be a Lie group with algebra \( \lG\). If \( n\in \eN\) we have
    \begin{equation}
        \exp(X)^n=\exp(nX).
    \end{equation}
\end{lemma}

\begin{proof}
    Apply \( n\) times the proposition \ref{PROPooNRVJooEDCpOI}.
\end{proof}

\begin{lemma}       \label{LEMooLMTZooCvunSl}
    Let \( G\) be a Lie group and \( X\in G\). We have
    \begin{equation}        \label{EQooNBENooPXLENs}
        X^R_g=\Dsdd{  e^{tX}g }{t}{0}
    \end{equation}
    and
    \begin{equation}
        X^L_g=\Dsdd{  ge^{tX} }{t}{0}
    \end{equation}
\end{lemma}

\begin{normaltext}      \label{NORMooSATDooIhwXXr}
    We will often write the relation \eqref{EQooNBENooPXLENs} under the form
    \begin{equation}
        X^R_g(t)= e^{tX}g.
    \end{equation}
    This is a way to implies that \( t\mapsto  e^{tX}g\) is a path for the vector \( X^R_g\). It is a common abuse of notation to write the vector and a path representing the vector with the same symbol.
\end{normaltext}

%--------------------------------------------------------------------------------------------------------------------------- 
\subsection{Invariant vector and derivation}
%---------------------------------------------------------------------------------------------------------------------------

You may want to know how the exponential can be used to write some formulas linking left-invariant vector field and derivation of functions. Here you are.

\begin{normaltext}
    Let \( X\in \lG\), \( g\in G\) and \( u\in \eR\). Let \( f\colon G\to \eR\) be a smooth function. Using the abuse of notation described in \ref{NORMooSATDooIhwXXr} and the proposition \ref{PROPooNRVJooEDCpOI},
    \begin{subequations}
        \begin{align}
            (X^Lf)(g e^{uX})&=\Dsdd{ f\big( X^L_{g e^{uX}}(t) \big) }{t}{0}\\
            &=\Dsdd{ f\big( g e^{uX} e^{tX} \big) }{t}{0}\\
            &=\Dsdd{ f\big( g e^{(t+u)X)} \big)}{t}{0}\\
            &=\Dsdd{ f(g e^{tX}) }{t}{u}.
        \end{align}
    \end{subequations}
    The formula
    \begin{equation}
        (X^Lf)(g e^{uX})=\Dsdd{ f(g e^{tX}) }{t}{u}
    \end{equation}
    means that \( X^L\) derives \( f\) in the direction of the path \(  e^{tX}\) at right.
\end{normaltext}

\begin{normaltext}
    By the way, we recall that, if \( f\) is a function and \( X\) a vector field, \( (Xf)\) is a new function, given by
    \begin{equation}
        (Xf)(a)=X_a(f).
    \end{equation}
    In that sense we can write combinations like \( XYf\) or \( (X^2+X)f\) where \( X\) and \( Y\) are vector fields.
\end{normaltext}

\begin{proposition}[\cite{BIBooPBAMooNcYhCM}]       \label{PROPooKSIDooVIFkiM}
    Let \( G\) be a Lie group with Lie algebra \( \lG\). We consider \( X,Y\in \lG\) and a smooth function \( f\colon G\to \eR\). We have\quext{My source \cite{BIBooPBAMooNcYhCM} seems to write \( (X^R)^n(Y^R)^m\) instead of \( (X^R)^n(Y^L)^m\). Let me know where I'm wrong.}
    \begin{equation}
        \big( (X^R)^n(Y^L)^mf \big)( e^{sX} e^{tY})=\frac{ d^n }{ du^n }\frac{ d^m }{ dv^m }\Big( f( e^{uX} e^{vY}) \Big)_{\substack{u=s\\v=t}}.
    \end{equation}
\end{proposition}

\begin{proof}
    We have to do a proof by induction on \( (n,m)\). We start with \( (n,m)=(0,0)\) and we prove the steps \( (n,m)\to (n+1,m)\) and \( (n,m)\to (n,m+1)\).

    \begin{subproof}
        \spitem[\( (0,0)\)]
            With \( (n,m)=(0,0)\) we are okay.
        \spitem[\( (n+1,m)\)]
            We have
            \begin{equation}
                \Big( (X^R)^{n+1}(Y^L)^mf \Big)( e^{sX} e^{tY})=\big( (X^R)(X^R)^n(Y^L)^mf \big)( e^{sX} e^{tY}).
            \end{equation}
            We will apply the induction hypothesis on the function \( (X^R)^n(Y^L)^mf\), but in a first time we just apply the vector field \( X^R\) to the function \( (X^R)^n(Y^L)^m\) and we evaluate at \(  e^{sX} e^{tY}\). Here is a couple of computations:
            \begin{subequations}
                \begin{align}
                    \Big( (X^R)(X^R)^n(Y^L)^mf \Big)( e^{sX} e^{tY})&=\Dsdd{  \Big( (X^R)^n(Y^L)^mf \Big)\big( X^R_{ e^{sX} e^{tY}}(u) \big)  }{u}{0}\\
                    &=\Dsdd{  \Big( (X^R)^n(Y^L)^mf \Big)(  e^{uX} e^{sX} e^{tY} )  }{u}{0}\\
                    &=\Dsdd{  \Big( (X^R)^n(Y^L)^mf \Big)( e^{uX} e^{tY})  }{u}{s}.
                \end{align}
            \end{subequations}
            At this point we use the induction hypothesis:
            \begin{subequations}
                \begin{align}
                    \Dsdd{  \Big( (X^R)^n(Y^L)^mf \Big)( e^{uX} e^{tY})  }{u}{s}&=\frac{ d }{ du }\left( \frac{ d^n }{ dw^n }\frac{ d^m }{ dv^m }\big( f( e^{wX} e^{vY}) \big)_{\substack{w=u\\v=t}}  \right)_{u=s}\\
                    &=\frac{ d^{n+1} }{ dw^{n+1} }\frac{ d^m }{ dv^m }\left( f( e^{wX} e^{vY}) \right)_{\substack{w=s\\v=t}}.
                \end{align}
            \end{subequations}
        \spitem[\( (n,m+1)\)]
            Same kind of computations.
    \end{subproof}
\end{proof}



\begin{lemma}       \label{LEMooWKFIooRHsrFX}
    Let \( G\) be an analytic Lie group with algebra \( \lG\). We consider a basis \( \{ e_i \}_{i=1,\ldots, n}\) of \( \lG\) and the functions
    \begin{equation}
        \begin{aligned}
            f_i\colon U&\to \eR \\
            \exp(x_1e_1+\ldots+x_ne_n)&\mapsto x_i 
        \end{aligned}
    \end{equation}
    defined on a normal neighbourhood\footnote{Definition \ref{}.} \( U\) of \( e\).
    
    If \( X,Y\in \lG\) satisfy
    \begin{equation}
        Xf_i=Yf_i
    \end{equation}
    for every \( i\), then \( X=Y\).
\end{lemma}

\begin{proof}
    If \( X=\sum_kX_ke_k\) we have
    \begin{equation}
        X(f_i)=\Dsdd{ f_i( e^{tX}) }{t}{0}=\Dsdd{ f_i\big(  e^{t\sum_kX_ke_k} \big) }{t}{0}=\Dsdd{ tX_i }{t}{0}=X_i.
    \end{equation}
\end{proof}



%+++++++++++++++++++++++++++++++++++++++++++++++++++++++++++++++++++++++++++++++++++++++++++++++++++++++++++++++++++++++++++ 
\section{Exponential as bijection on a neighborhood}
%+++++++++++++++++++++++++++++++++++++++++++++++++++++++++++++++++++++++++++++++++++++++++++++++++++++++++++++++++++++++++++

\begin{proposition}[\cite{BIBooJOSEooGOmqoQ}]     \label{PROPooYFZZooLUOuOj}
    Let \( G\) be a smooth Lie group. There exists a neighbourhood \( U\) of \( 0\) in \( \lG\) and a neighbourhood \( V\) of \( e\) in \( G\) such that
    \begin{equation}
        \exp\colon U\to V
    \end{equation}
    is a \(  C^{\infty}\)-diffeomorphism\footnote{\( \exp\) is \(  C^{\infty}\), invertible and he inverse is \(  C^{\infty}\) as well.}.
\end{proposition}


\begin{proposition}     \label{PROPooAICDooQcmPZB}
    Let \( G\) be an analytic Lie group. There exists a neighbourhood \( U\) of \( 0\) in \( \lG\) and a neighbourhood \( V\) of \( e\) in \( G\) such that
    \begin{equation}
        \exp\colon U\to V
    \end{equation}
    is an analytic diffeomorphism\footnote{\( \exp\) is analytic, invertible and its inverse is analytic too.}.
\end{proposition}
% TODOooQVZJooQeMwAM Prouver PROPooAICDooQcmPZB en sa basant sur THOooFMFLooCnLJPr, et quand c'est fait, supprimer ce dernier.

\begin{theorem}     \label{THOooFMFLooCnLJPr}
    Let \( G\) be an analytic Lie group. There exist a neighbourhood \( U\) of \( 0\) in \( \lG\) and \( V\) of \( e\) in \( G\) such that
    \begin{equation}
        \exp\colon U\to V
    \end{equation}
    is an analytic diffeomorphism.
\end{theorem}

\begin{proof}
    Let $\{X_1,\ldots,X_n\}$ be a basis of $\lG$ such that $X_i\in\lM$ for $1\leq i\leq r$ and $X_j\in\lN$ for $r<j\leq n$. We consider $\{t_1,\ldots,t_n\}$, the canonical coordinates of $\exp(x_1X_1+\cdots+x_rX_r)\exp(x_{r+1}X_{r+1}+\cdots+x_nX_n)$ in this coordinate system. By properties of the exponential, the function $\varphi_j$ defined by $t_j=\varphi_j(x_1,\ldots,x_n)$ is differentiable at $(0,\ldots,0)$. If $x_i=\delta_{ij}s$, then $t_i=\delta_{ij}s$ and the Jacobian of
    \[
    \dsd{(\varphi_1,\ldots,\varphi_n)}{(x_1,\ldots,x_n)}
    \]
    is $1$ for $x_1=\ldots=x_n=0$. Thus $d\varphi_e$ is a diffeomorphism and so $\varphi$ is a locally diffeomorphic.
\end{proof}


%--------------------------------------------------------------------------------------------------------------------------- 
\subsection{Analytic Lie group, Taylor formula}
%---------------------------------------------------------------------------------------------------------------------------

In this subsection we study the analytic functions over an analytic Lie group.

\begin{lemma}[\cite{BIBooPBAMooNcYhCM}]     \label{LEMooPILVooHQbtAH}
    Let \( G\) be an analytic Lie group. We consider an analytic function \( f\colon G\to \eR\), an element \( X\in \lG\), a basis \( \{ X_i \}\) of \( \lG\) and \( g\in G\). There exists an absolutely converging power series \( P\) such that
    \begin{equation}
        f(g e^{x_1X_1+\ldots +x_nX_n})=P(x_1,\ldots, x_n).
    \end{equation}
\end{lemma}

\begin{proof}
    First we make the proof for \( g=e\).

    We consider a basis \( \{ e_i \}\) of \( \lG\). Let \( U\) be a neighbourhood of \( 0\) in \( \lG\) and \( V\) a neighbourhood of \( e\) in \( G\) such that \( \exp\colon U\to V\) is an analytic diffeomorphism\footnote{By proposition \ref{PROPooAICDooQcmPZB}.}.

    We consider \( U'\), the open set in \( \eR^n\) which correspond to \( U\) via the basis \( \{ e_i \}\). The map
    \begin{equation}
        \begin{aligned}
            \varphi\colon U'&\to V \\
            (x_1,\ldots, x_n)&\mapsto \exp(x_1e_1+\ldots+x_ne_n)
        \end{aligned}
    \end{equation}
    is analytic chart of \( V\).

    The fact that \( f\) is analytic means that the composition of \( f\) with the charts are analytic. In our case, the map \( \tilde f =f\circ\varphi\) is analytic from \( U'\subset \eR^n\) to \( \eR\). Thus there exists an absolutely converging power series \( P\) such that
    \begin{equation}
        \tilde f(x_1,\ldots, x_n)=P(x_1,\ldots, x_n).
    \end{equation}
    We conclude:
    \begin{equation}
        f\big( \exp(x_1e_1+\ldots +x_ne_n) \big)=f\big( \varphi(x_1,\ldots, x_n) \big)=P(x_1,\ldots, x_n).
    \end{equation}
    
    If \( g\) is not \( e\), we consider the neighbourhood \( gV\) and the map
    \begin{equation}
        \begin{aligned}
            \varphi\colon U&\to gV \\
            (x_1,\ldots, x_n)&\mapsto g\exp(x_1e_1+\ldots +x_ne_n)
        \end{aligned}
    \end{equation}
    is a chart, so that
    \begin{equation}
        f(g e^{x_1e_1+\ldots +x_ne_n})=\tilde f(x_1,\ldots, x_n)
    \end{equation}
    which is a power series.
\end{proof}

\begin{proposition}[Taylor formula\cite{BIBooPBAMooNcYhCM}]     \label{PROPooIYWQooZJtKiu}
    Let \( G\) be an analytic Lie group. We suppose that \( f\colon G\to \eR\) is an analytic functions. For \( g\in G\) and \( X\in \lG\) we have
    \begin{equation}
        f(g e^{X})=\sum_{k=0}^{\infty}\frac{1}{ n! }\big( (X^R)^nf \big)(g).
    \end{equation}
\end{proposition}

\begin{proof}
    We know from proposition \ref{LEMooPILVooHQbtAH} that \( f(g e^{X})=P(x_1,\ldots, x_n)\) for some power series \( P\). We consider a neighbourhood \( U\) of \( 0\) in \( \lG\) and \( V\) of \( g\) in \( G\) such that
    \begin{equation}
        \begin{aligned}
            \varphi\colon U&\to V \\
            X&\mapsto  ge^{X} 
        \end{aligned}
    \end{equation}
is an analytic diffeomorphism (i.e. an analytic chart for \( G\) around \( g\)). Let \( X\in U\) and \( \delta\) such that \( tX\in U\) for all \( t\in \mathopen] -\delta , \delta \mathclose[\). Notice that \( \delta>1\). Now, \( X\) being fixed, the value of \( P(tx_1,\ldots, tx_n)\) is an absolutely convergent power series of \( t\). We have
    \begin{equation}
        f(g e^{tX})=P(tx_1,\ldots, tx_n)=\sum_{k=0}^{\infty}\frac{ a_m }{ m! }t^m
    \end{equation}
    for some constants \( a_m\in \eR\).

    But considering the function
    \begin{equation}
        \begin{aligned}
            r\colon \eR&\to \eR \\
            t&\mapsto f(g e^{tX}), 
        \end{aligned}
    \end{equation}
    there is an unicity of its power series expansion; thus \( a_m\) is the \( m\)-th derivative of \( r\) at \( t=0\).

    But we also know from proposition \ref{PROPooKSIDooVIFkiM} that
    \begin{equation}
        \big( (X^L)^mf \big)(g e^{tX})=\frac{ d^m }{ du^m }\big( f(g e^{uX}) \big)_{u=t};
    \end{equation}
    taking that at \( t=0\) we have
    \begin{equation}
        a_m=\big( (X^L)^mf \big)(g)
    \end{equation}
    and the Taylor formula
    \begin{equation}
        f(g e^{tX})=\sum_{k=0}^{\infty}\frac{1}{ k! }\frac{ d^k }{ du^k }\big( f(g e^{uX}) \big)_{u=0}t^m.
    \end{equation}
    Finally taking \( t=1\) (recall that \( \delta>1\), so it is valid):
    \begin{equation}
        f(g e^{X})=\sum_{k=0}^{\infty}\frac{1}{ k! }\frac{1}{ k! }\big( (X^L)^kf \big)(g).
    \end{equation}
\end{proof}

\begin{lemma}[\cite{BIBooPBAMooNcYhCM}]     \label{LEMooMJBRooMOuJpa}
    Let \( G\) be an analytic Lie group with Lie algebra \( \lG\). For \( X,Y\in \lG\) we have:
    \begin{enumerate}
        \item       \label{ITEMooHVOIooKDrUSw}
            \( \exp(tX)\exp(tY)=\exp\big( t(X+Y)+\frac{ t^2 }{2}[X,Y]+t^2\alpha(t) \big)\),
        \item       \label{ITEMooWIQIooHphJcP}
            \( \exp\big( t(X+Y) \big)=\exp(tX)\exp(tY)\exp(t\alpha(t))\)
        \item       \label{ITEMooVMDCooExpIrp}
            \( \exp(-tX)\exp(-tY)\exp(tX)\exp(tY)=\exp\big( t^2[X,Y]+t^3\alpha(t) \big)\).
    \end{enumerate}
    In both formulas, \( \alpha\) is a function \( \alpha\colon \eR\to \lG\) satisfying \( \lim_{t\to 0} \alpha(t)=0\).
\end{lemma}

\begin{proof}
    Several steps.
    \begin{subproof}
        \spitem[A good function]
        
            Let \( \{ e_i \}_{i=1,\ldots, n}\) be a basis of \( \lG\). We consider a neighbourhood \( U\) of \( 0\) in \( \lG\) and \( V\) of \( e\) in \( G\) such that \( \exp\colon U\to V\) is an analytic diffeomorphism. On that \( U\) we consider the function
            \begin{equation}
                \begin{aligned}
                    f\colon U&\to \eR \\
                    \exp(x_1e_1+\ldots +x_ne_n)&\mapsto x_i 
                \end{aligned}
            \end{equation}
            for some fixed \( i\). This function is analytic and satisfies \( f(e)=0\). 
        \spitem[Some Taylor expansions] 
            Using proposition \ref{PROPooKSIDooVIFkiM} we have
            \begin{equation}
                \big( (X^R)^n(X^L)^mf \big)( e^{sX} e^{tY})=\frac{ d^n }{ du^n }\frac{ d^m }{ dv^m }\big( f( e^{uX} e^{vY}) \big)_{\substack{u=s\\v=t}}.
            \end{equation}
            Considering the function \( q(s,t)=f( e^{sX} e^{tY})\), we have the Taylor expansion
            \begin{equation}        \label{EQooNBOIooRxlZmP}
                f( e^{sX} e^{tY})=q(s,t)=\sum_{m,n\geq 0}\frac{ s^n }{ n! }\frac{ t^m }{ m! }\big( (X^R)^n(Y^L)^mf \big)(e)=\sum_{m,n\geq 0}\frac{ s^n }{ n! }\frac{ t^m }{ m! }\big( X^nY^mf \big)(e).
            \end{equation}
            Here the second equality is due to the fact that \( (X^Lf)(e)=(X^Rf)(e)=X(f)\).

        \spitem[The function \( Z\)]

            On the other hand, when \( t\) is small enough, the element \(  e^{tX} e^{tY}\) belongs to a normal neighbourhood of \( e\), so that there exists an element \( Z(t)\in \lG\) satisfying
            \begin{equation}
                e^{tX} e^{tY}= e^{Z(t)}.
            \end{equation}
            The element \( Z(t)\) is given by
            \begin{equation}
                Z(t)=\exp^{-1}\big(  e^{tX} e^{tY} \big).
            \end{equation}
            Since the exponential is an analytic diffeomorphism\footnote{Proposition \ref{PROPooAICDooQcmPZB}.} (the inverse is analytic), \( Z\) is an analytic function around \( t=0\). Thus there exists a function \( \alpha\colon \eR\to \lG\) such that
            \begin{equation}        \label{EQooRPGGooXtZzFy}
                Z(t)=tZ_1+t^2Z_2+t^2\alpha(t)
            \end{equation}
            and \( \lim_{t\to 0} \alpha(t)=0\). Notice that \( Z(0)=0\), which explain the absence of constant term in \eqref{EQooRPGGooXtZzFy}.

        \spitem[A formula for \( f\big(  e^{Z(t)} \big)\)]

            We pose \( Z_1=\sum_ka_{1k}e_k\), \( Z_2=\sum_ka_{2k}e_k\) and \( \alpha(t)=\sum_k\sigma_k(t)e_k\), so that
            \begin{equation}
                Z(t)=\sum_k\big( ta_{1k}+t^2a_{2k}+t^2\alpha_k(t) \big)e_k.
            \end{equation}
            Applying \( f\) we have
            \begin{equation}
                f\big(  e^{Z(t)} \big)=ta_{1i}+t^2a_{2i}+t^2\alpha_i(t)=f\big(  e^{tZ_1+t^2Z_2} \big)+t^2\alpha_i(t).
            \end{equation}
            
        \spitem[Some more Taylor expansions]

            We use the Taylor expansion of proposition \ref{PROPooIYWQooZJtKiu} with \( g=e\) and \( X=Z(t)\):
            \begin{equation}        \label{EQooSFKOooDAavVy}
                f( e^{Z(t)})=\sum_k\frac{1}{ k! }\big( [tZ^L_1+t^2Z_2^L]^kf \big)(e)+t^2\alpha_i(t).
            \end{equation}
            Once again we can drop the \( L\) exponent since \( (X^Lf)(e)=X(f)\). We collect out of \eqref{EQooSFKOooDAavVy} the terms with \( t\) and \( t^2\):
            \begin{equation}        \label{EQooEYUSooTDntym}
                f( e^{tX} e^{tY})=f( e^{Z(t)})=tZ_1(f)+t^2 Z_2 +\frac{ t^2 }{2}Z_1^2 +t^2\beta(t)
            \end{equation}
            with \( \lim_{t\to 0} \beta(t)=0\).

        \spitem[Comparison]

            The formulas \eqref{EQooNBOIooRxlZmP} with \( s=t\) and \eqref{EQooEYUSooTDntym} are Taylor expansions of the same quantity. They are equal; we copy them here:
            \begin{equation}
                \sum_{m,n}\frac{ t^{m+n} }{ m!n! }\big( (X^R)^n(Y^L)^mf \big)(e)=tZ_1(f)+t^2 Z_2 +\frac{ t^2 }{2}Z_1^2 +t^2\beta(t)
            \end{equation}
            On the left hand side, the terms with \( t\) and \( t^2\) are obtained when \( (n,m)\) is among the possibilities $(0,1)$, $(1,0)$, $(2,0)$, $(0,2)$, and $(1,1)$. Collecting we have on the left
            \begin{equation}
                (X+Y)f+XYf+\frac{ 1 }{2}X^2f+\frac{ 1 }{2}Y^2f
            \end{equation}
            where we used the fact that \( \big( (X^R)^2f \big)(e)=X(Xf)=X^2f\).

            Using lemma \ref{LEMooWKFIooRHsrFX} we have \( Z_1f=(X+Y)f\), so that \( Z_1=X+Y\) and then
            \begin{equation}
                \frac{ 1 }{2}[X,Y]=Z_2.
            \end{equation}
    \end{subproof}
    At this point we proved that
    \begin{equation}
        e^{tX} e^{tY}= e^{t(X+Y)+\frac{ t^2 }{2}[X,Y]+t^2\alpha(t)}.
    \end{equation}
    This is \ref{ITEMooHVOIooKDrUSw}.

    For point \ref{ITEMooWIQIooHphJcP}, we are searching for a function \( \beta\) such that 
    \begin{equation}
        e^{tX} e^{tY} e^{t\beta(t)}= e^{t(X+Y)}.
    \end{equation}
    We replace in the left-hand side the value of \(  e^{tX} e^{tY}\) given by the point \ref{ITEMooHVOIooKDrUSw} (this is the reason why we write \( \beta\) instead of \( \alpha\)) and we isolate \(  e^{t\beta(t)}\):
    \begin{equation}        \label{EQooLTMBooVIChyC}
        e^{t\beta(t)}= e^{t(X+Y)} e^{-t(X+Y)-t^2[X,Y]/2-t^2\alpha(t)}.
    \end{equation}
    So now our aim is to show that the right-hand side of \eqref{EQooLTMBooVIChyC} can be written as only one exponential with an argument of the form \( t\beta(t)\) satisfying \( \beta(t)\to 0\). For that, we use \ref{ITEMooHVOIooKDrUSw} once again with \( X+Y\) instead of \( X\) and \( -(X+Y)-t[X,Y]/2-t\alpha(t)\) instead of \( Y\). What we get is
    \begin{subequations}
        \begin{align}
            e^{t\beta(t)}&=\exp\big( t(-t[X,Y]/3-t\alpha(t))+\frac{ t^2 }{2}\big[ X+Y,-(X+Y)-t[X,Y]/2-t\alpha(t) \big] \big)\\
            &=\exp\big( -\frac{ t^2 }{2}[X,Y]  -t^2\alpha(t)-\frac{ t^3 }{ 4 }\big[ X+Y,[X,Y] \big]-\frac{ t^3 }{ 2 }\alpha(t)  \big).
        \end{align}
    \end{subequations}
    We are done with \ref{ITEMooWIQIooHphJcP}.

et

    \ref{ITEMooVMDCooExpIrp}

\end{proof}


%--------------------------------------------------------------------------------------------------------------------------- 
\subsection{Exponential and subspace}
%---------------------------------------------------------------------------------------------------------------------------

\begin{lemma}[\cite{BIBooTKQTooGjFxwB}]     \label{LEMooFXKBooRnzZKQ}
    Let \( G\) be an analytic Lie group and \( H\) be a closed subgroup of \( G\). The set
    \begin{equation}
        \lH=\{ X\in \lG\tq  e^{tX}\in H\forall t\in \eR \}
    \end{equation}
    is a vector subspace of \( \lG\).
\end{lemma}

\begin{proof}
    Let \( X,Y\in \lH\), \( \lambda\in \eR\) and \( t\in \eR\). The element \( \lambda X\) belongs to \( \lH\). We have to prove that \( X+Y\in \lH\).

    For every \( n\in \eN\), we have \(  e^{tX/n} e^{tY/n}\in H\) because \( H\) is a subgroup. For every \( n\) we also have
    \begin{equation}
        \big(  e^{tX/n} e^{tY/n} \big)^n\in H.
    \end{equation}
    we have to following computation :
    \begin{subequations}        \label{SUBEQSooMDRVooQXBwiS}
        \begin{align}
            \big(  e^{tX/n} e^{tY/n} \big)^n&=\left[ \exp\left( \frac{ t(X+Y) }{ n }+\frac{ t^2 }{ n^2 }\alpha(t/n) \right)  \right]^n \label{SUBEQooUYNCooJVIWMi}\\
            &=\exp\left( t(X+Y)+\frac{ t^2 }{n  }\alpha(t/n) \right)    \label{SUBEQooUYKKooXtGaxL}
        \end{align}
    \end{subequations}
    Justifications.
    \begin{itemize}
        \item For \eqref{SUBEQooUYNCooJVIWMi}. This is lemma \ref{LEMooMJBRooMOuJpa}\ref{ITEMooWIQIooHphJcP}.
        \item For \eqref{SUBEQooUYKKooXtGaxL}. The \( n\) enters the exponential from lemma \ref{LEMooRPHVooAtZJnz}.
    \end{itemize}
    We have the limit
    \begin{equation}        \label{EQooTJWDooVJsDJt}
        \lim_{n\to \infty} \exp\left( t(X+Y)+\frac{ t^2 }{n  }\alpha(t/n) \right) =  e^{t(X+Y)}
    \end{equation}
    because the exponential is continuous\footnote{See \ref{PROPooYFZZooLUOuOj}.} and the properties of \( \alpha\). Since the limit exists on the right hand side of \eqref{SUBEQSooMDRVooQXBwiS}, the limit exists on the left hand side too.

    The limit
    \begin{equation}
        \lim_{n\to \infty} \big(  e^{tX/n} e^{tY/n} \big)^n
    \end{equation}
    is a limit of elements in \( H\). Since \( H\) is closed, this is an element of \( H\). We deduce that \eqref{EQooTJWDooVJsDJt} is an element of \( H\).
\end{proof}

\begin{lemma}[\cite{BIBooTKQTooGjFxwB}]     \label{LEMooFDIIooCkSJpY}
    Let \( G\) be an analytic Lie group and \( H\) be a closed subgroup of \( G\). We consider the set
    \begin{equation}
        \lH=\{ X\in \lG\tq  e^{tX}\in H\forall t\in \eR \}.
    \end{equation}

    Let \( (X_i)\) be a sequence in \( \lG\) such that\footnote{The topology and the norm on \( \lH\) are given in definition \ref{PROPooHJOXooMGANfd}.}
    \begin{enumerate}
        \item       \label{ITEMooWQZAooTTnenM}
            \( X_i\to 0\)
        \item
            \(  e^{X_i}\in H\) for every \( i\).
        \item       \label{ITEMooFLLRooCUXnHb}
            The limit \( \lim_{i\to \infty} \frac{ X_i }{ \| X_i \| }\) exists. We name it \( X\).
    \end{enumerate}
    Then \( X\in \lH\).
\end{lemma}

\begin{proof}
    We consider \( n_i=\integer(t/\| X_i \|)\) where \( \integer\) is the «integer part» defined in \ref{LEMooLEXTooGAQxGB}, and we write
    \begin{equation}
        n_iX_i=\integer\big( \frac{ t }{ \| X_i \| } \big)X_i=\| X_i \|\integer\big( \frac{ t }{ \| X_i \| } \big)\frac{ X_i }{ \| X_i \| }.
    \end{equation}
    Using the hypothesis \ref{ITEMooWQZAooTTnenM} and \ref{ITEMooFLLRooCUXnHb} and the limit of lemme \ref{LEMooLSJZooDauTkc} we have
    \begin{equation}
        \lim_{i\to \infty} n_iX_i=\lim_{i\to \infty} \| X_i \|\integer\big( \frac{ t }{ \| X_i \| } \big)\frac{ X_i }{ \| X_i \| }=tX.
    \end{equation}
    
    Using the continuity of the exponential (it commutes with the limit), we have
    \begin{subequations}
        \begin{align}
            e^{tX}&=\exp\big( \lim_{i\to \infty} n_iX_i \big)\\
            &=\lim_{i\to \infty}  e^{n_iX_i}\\
            &=\lim_{i\to \infty} \exp(X_i)^{n_i}.
        \end{align}
    \end{subequations}
    Since \( X_i\in \lH\), we have \( \exp(X_i)\in H\), and \( \exp(X_i)^{n_i}\in H\) because \( H\) is a group.

    Thus the sequence \( \exp(X_i)^{n_i}\) is a sequence contained in the closed space \( H\) and which has a limit (\( \exp(tX)\)). Thus the limit is contained in \( H\) by proposition \ref{PROPooBBNSooCjrtRb}. We conclude that \(  e^{tX}\in H\).
\end{proof}

\begin{lemma}[\cite{BIBooJOSEooGOmqoQ}]     \label{LEMooNPLBooBrwNWY}
    Let \( G\) be a Lie group, \( \lG\) its Lie algebra and \( H\), a closed subgroup of \( G\). We consider
    \begin{equation}
        \lH=\{ X\in \lG\tq \exp(tX)\in H\forall t\in \eR \}.
    \end{equation}
    There exists a neighbourhood \( \mO\) of \( 0\) in \( \lH\) and \( V\) of \( e\) in \( H\) such that 
    \begin{equation}
        \exp\colon U\to V
    \end{equation}
    is  a bijection.
\end{lemma}

\begin{proof}
    Several steps.
    \begin{subproof}
        \spitem[The map \( \phi\)]
    Let \( \lH'\) be a subspace of \( \lG\) such that \( \lG=\lH\oplus\lH'\) and consider the map
    \begin{equation}
        \begin{aligned}
            \phi\colon \lH\oplus\lH'&\to G \\
            X+Y&\mapsto \exp(X)\exp(Y). 
        \end{aligned}
    \end{equation}
\spitem[\( \phi\colon \lG\to G \) is a local bijection]
    % -------------------------------------------------------------------------------------------- 
    Let \( X\in \lH\) and \( Y\in \lH'\). Using the Leibnitz rule \ref{PROPooAXYRooWVhXRa} we have
    \begin{equation}
        d\phi_0(X+Y)=\Dsdd{ \phi(tX,tY) }{t}{0}=\Dsdd{ \exp(tX)\exp(tY) }{t}{0}=X+Y.
    \end{equation}
    In other words, \( d\phi_0=\id\). Since the exponential is continuous\footnote{Proposition \ref{}.}, the local inversion theorem \ref{THOooDWEXooMClWVi} says that \( \phi\) is a local bijection.

    Let \( \mO\) be a neighbourhood of \( 0\) in \( \mG\) and \( v\) a neighbourhood of \( e\) in \( G\) such that \( \phi\colon \mO\to V\) is a bijection.

    \spitem[\( \phi\) is a local bijection between \( \lH\) and \( H\)]
    % -------------------------------------------------------------------------------------------- 

    We recall that the definition \ref{PROPooMAGKooInwNom} says that the topology on \( \lG\) is any norm topology. Suppose that \( \phi\) is not a local bijection between \( \lH\) and \( H\). That means that if \( U\) is a neighbourhood of \( 0\) in \( \lH\) and \( V\) is a neighbourhood of \( e\) in \( H\), then \( \phi\) is not a bijection between \( U\) and \( V\).

    Let \( \mO_i=B(0,r_i)\) in \( \lG\) with \( r_i\to 0\) be as small as \( \mO_i\subset \mO\) for every \( i\). Let \( V_i=\phi(\mO_i)\). The part \( B(0,r_i)\cap \lH\) is a neighbourhood of \( 0\) in \( \lH\) and \( V_i\cap H\) is a neighbourhood of \( e\) in \( H\). This \( \phi\) is not a bijection between \( B(0,r_i)\cap\lH\) and \( V_i\cap H\).

    Since \( \phi(\lH)\subset H\), we have \( \phi\big( B(0,r_i)\cap \lH \big)\subset V_i\cap \lH\). Since the map \( \phi\) is injective and not bijective, it has to be not surjective. Thus there exists \( h_i\in (V_i\cap H)\setminus\phi\big( B(0,r_i)\cap \lH \big)\). Now every element of \( V_i\) is the image of an element of \( B(0,r_i)\), thus there exits \( Z_i\in B(0,r_i)\) such that \( \phi(Z_i)=h_i\).

    The element \( Z_i\) can be decomposed into \( Z_i=X_i+Y_i\) with \( X_i\in \lH\) and \( Y_i\in \lH'\). Since \( h_i\) is not in the image of \( B(0,r_i)\cap\lH\) we deduce that \( Y_i\neq 0\). Let us summarize:
    \begin{itemize}
        \item \( X_i+Y_i\stackrel{\lG}{\longrightarrow}0\)
        \item
            \( \phi(X_i+Y_i)\in H\)
        \item
            \( Y_i\neq 0\)
        \item
            \( \exp(X_i)\in H\).
    \end{itemize}
    We have
    \begin{equation}
        \underbrace{\phi(X_i+Y_i)}_{\in \lH}=\underbrace{\exp(X_i)}_{\in \lH}\exp(Y_i),
    \end{equation}
    and then \( \exp(Y_i)\in\lH\).

    Now we consider the limit
    Lemme \ref{LEMooFDIIooCkSJpY} says that the limit \( Y=\lim_{i\to \infty} Y_i/\| Y_i \|\) exists and belongs to \( \lH\). But \( \lH'\) is closed (because finite dimensional vector subspace), so that a limit in \( \lH'\) remains in \( \lH'\). We deduce that
    \begin{equation}
        Y\in\lH\cap\lH'=\{ 0 \}.
    \end{equation}
    This is impossible because \( \| Y \|=1\).

    Then \( \phi\) is a local bijection between \( \lH\) and \( H\): there exists a neighbourhood \( \mO\) of \( 0\) in \( \lH\) and a neighbourhood \( V\) of \( e\) in \( H\) such that \( \phi\colon \mO\to V\) is a bijection.

    \spitem[Conclusion]
    % -------------------------------------------------------------------------------------------- 
      
    The map \( \exp\colon \mO\to V\) is bijective because, on \( \lH\) we have \( \exp=\phi\).
    
    \end{subproof}
\end{proof}

\begin{lemma}[\cite{BIBooJOSEooGOmqoQ}]       \label{LEMooOBIMooVvIDnb} 
 Let $G$ be a Lie group and $H$, a Lie subgroup of $G$ ($\lG$ and $\lH$ are the corresponding Lie algebras). If $H$ is a topological subspace of $G$, then there exists an open neighbourhood $U$ of $0$ in $\mG$ such that
 \begin{enumerate}
 \item $\exp$ is a diffeomorphism between $U$ and an open neighbourhood of $e$ in~$G$,
 \item
     \( \exp\colon U\cap\lH\to \exp(U)\cap H\) is a smooth diffeomorphism.
 \end{enumerate}
\end{lemma}
 

\begin{lemma}  \label{LEMooEBQUooKXkCda} 
Let $\lG$ admit a direct sum decomposition (as vector space) $\lG=\lM\oplus\lN$. Then there exists open and bounded neighbourhoods $\mU_m$ and $\mU_n$ of $0$ in $\lM$ and $\lN$ such that the map
		\begin{equation}
		\begin{aligned}
			\phi \colon \mU_m\times\mU_n &\to G\\
			(A,B)&\mapsto e^Ae^B
		\end{aligned}
	\end{equation}
is a diffeomorphism between $\mU_m\times\mU_n$ and an open neighbourhood of $e$ in $G$.
\end{lemma}


\begin{corollary}
Let $G$ be a Lie group, and $K$, $H$ two differentiable subgroups of $G$. We suppose $K\subset H$. Then $K$ is a differentiable subgroup of the Lie group $H$.
\end{corollary}

\begin{proof}
The Lie algebras of $K$ and $H$ are respectively denoted by $\lK$ and $\lH$. We denote by $K^*$ the differentiable subgroup of $H$ which has $\lK$ as Lie algebra. The differentiable subgroups $K$ and $K^*$ have same Lie algebra, and then coincide as Lie groups.
\end{proof}

\begin{lemma}[\cite{Lie}]		\label{lemsur5d}
    Let $G$, $H$ be two Lie groups with algebras\footnote{Lie algebra of a Lie group, definition \ref{DEFooKDCPooZOJsMD}.} $\mG$ and $\mH$. Let $\dpt{\phi}{G}{H}$ be a group morphism which is differentiable at $e\in G$. Then for every $X\in\lG$, the following formula holds:
    \begin{equation}
		\phi(\exp X)=\exp(d\phi_eX).
    \end{equation}
\end{lemma}

\begin{theorem}[Cartan\cite{BIBooGZHEooBPsXQy,BIBooFLEXooPgvAlz}]     \label{THOooDEJHooVKJYBL}
    Let \( G\) be a smooth Lie group. If \( H\) is a closed subgroup of \( G\).

    \begin{enumerate}
        \item
        There exists a unique structure of smooth manifold on \( H\) such that the inclusion map \( \iota\colon H\to G\) is a smooth embedding\footnote{Definition \ref{DEFooQLGLooNyXaOV}.}.
\item
    With that structure, \( H\) is a smooth Lie subgroup of \( G\).
    \end{enumerate}
\end{theorem}

\begin{proof}
    By proposition \ref{PROPooYFZZooLUOuOj}  \ldots
    %TODOooYVFYooQwfUiy: finir cette preuve
\end{proof}

%--------------------------------------------------------------------------------------------------------------------------- 
\subsection{Campbell-Baker-Hausdorff formula}
%---------------------------------------------------------------------------------------------------------------------------

\begin{theorem}     \label{THOooYJPEooSpKHnC}
    Let \( G\) be a Lie group with Lie algebra \( \lG\). For every \( X,Y\in \lG\), there exists \( Z\in \lG\) such that
    \begin{equation}
        e^{X} e^{Y}= e^{Z}.
    \end{equation}
\end{theorem}
%TODOooVIFLooKbKMRv: we need an explicit formula, or at least prove analyticity, because it is used at ooPFNFooFvYtHr

%--------------------------------------------------------------------------------------------------------------------------- 
\subsection{Smooth and analytic Lie group}
%---------------------------------------------------------------------------------------------------------------------------

We recall that a manifold is a set with its charts. Usually we write ``Let \( M\) be a manifold'' where we should write ``Let \( (M,\{ \varphi_{\alpha} \}_{\alpha\in I})\) be a manifold''. Changing the set of charts can completely change the properties of a manifold. That subtlety is crucial in the following theorem which basically says ``every smooth Lie group is analytic''.
\begin{theorem}[\cite{BIBooSYWXooYSgEgW}]       \label{THOooSQVCooCyEPOS}
    Let \( (G,\{ \varphi_{\alpha} \}_{\alpha\in A})\) be a smooth manifold. We suppose that \( G\) is a group for which the multiplication and the inverse are smooth functions.

    \begin{enumerate}
        \item
            There exists an atlas \( \{ \phi_i \}_{i\in I}\) of \( G\) for which the transition functions are analytic.
        \item
            The multiplication and the inverse of \( G\) are analytic on the manifold \( \big( G,\{ \phi_i \}_{i\in I} \big)\).
    \end{enumerate}
\end{theorem}

\begin{probleme}
    We have to complete the proof; in particular we have to state and prove all the quoted results.
\end{probleme}


\begin{proof}
    We initiate by considerate a vector space isomorphism \( \sigma\colon \eR^n\to \lG\) where \( \lG\) is the Lie algebra of \( G\). Let \( U'\) be a neighbourhood of \( 0\) in \( \lG\) from which the exponential is a smooth diffeomorphism. We write \( U=\sigma^{-1}(U')\). For \( g\in G\) we consider the chart
    \begin{equation}
        \begin{aligned}
            \phi_g\colon U&\to G \\
            v&\mapsto g\exp\big( \sigma(v) \big). 
        \end{aligned}
    \end{equation}
    Here are some facts\quext{that need some justifications.}.
%TODOooBZLSooJeHmXD: provide some justifications to these facts.
    \begin{enumerate}
        \item
            These maps \( \phi_g\) are smooth and then are charts of \( (G,\{ \varphi_{\alpha} \})\).
        \item
            The set \( \{ \phi_g \}_{g\in G}\) is an atlas of \( (G,\{ \varphi_{\alpha} \})\).
        \item
            The transitions functions \( \phi_h^{-1}\circ\phi_g\) are analytic\footnote{As functions from an open set of \( \eR^n\) to an open set of \( \eR^n\).} where they are defined.
        \item
            The multiplication and the inverse are analytic for the manifold \( (G,\{ \phi_g \}_{g\in G})\).
    \end{enumerate}
    Here is an hint for the last point. Let \( g,h\) be in a neighbourhood of \( e\) such that \( g,h\in \phi_e(U)\). Let \( g=\exp(X_0)\) and \( h=\exp(X)\). We look at the map
    \begin{equation}
        \begin{aligned}
            \mu\colon \lG&\to G \\
            X&\mapsto g\exp(X). 
        \end{aligned}
    \end{equation}
    The element \( \phi_e\big( \mu(X) \big)\) satisfy
    \begin{equation}
        \exp(X_0)\exp(X)=\exp\Big( \phi_e^{-1}\big( \mu(X) \big) \Big).
    \end{equation}
    In other terms, \( \phi_e^{-1}(X)\) is given by the Campbell-Baker-Hausdorff formula, which is analytic\quext{This affirmation should require some more work.}, theorem \ref{THOooYJPEooSpKHnC}.
    % Analyticity of CBH used here ooPFNFooFvYtHr
\end{proof}

\begin{normaltext}      \label{NORMooKCBMooGWQZJY}
    From now on, we will always consider analytic Lie groups. The interesting point is that, if we show that some charts are smooth, theorem \ref{THOooSQVCooCyEPOS} allows us to say that the group is analytic by choosing the correct atlas.
\end{normaltext}


%--------------------------------------------------------------------------------------------------------------------------- 
\subsection{Topological Lie subgroup}
%---------------------------------------------------------------------------------------------------------------------------

\begin{remark}
A \textit{topological} Lie subgroup\index{topological!Lie subgroup} is stronger that a common Lie subgroup because it needs to be a topological subgroup: it must carry \emph{exactly} the induced topology. In our definition of a Lie group, this feature doesn't appears.
\end{remark}

\begin{theorem} \label{THOooXVXBooZDJzQo}
    Let $G$ be a Lie group whose Lie algebra is $\lG$ and $H$, a closed subgroup (not specially a \emph{Lie} subgroup) of $G$. Then there exists one and only one analytic structure on $H$ for which $H$ is a topological Lie subgroup of $G$.
\end{theorem}


\begin{proof}
    The unicity part comes from the corollary~\ref{CORooMCWWooXkpkNO}.

    We will work with
    \begin{equation}\label{eq:lH_de_G}
  \lH=\{X\in\lG\tq \forall t\in\eR,\, e^{tX}\in H\}.
  %TODOooVKBWooYUnOkA:  finir la preuve.
\end{equation}
\end{proof}

%--------------------------------------------------------------------------------------------------------------------------- 
\subsection{Action from Lie algebra to Lie group}
%---------------------------------------------------------------------------------------------------------------------------

A very important point\cite{ooOLNIooDLmxkR} is that when \( G\) is acting on $M$, one can reconstruct the action of \( G\) only knowing the action of \( \lG\). Let \( X\in \lG\) and \( x\in M\). We consider the path
\begin{equation}
    \begin{aligned}
        \gamma\colon \eR&\to M \\
        t&\mapsto \exp(tX)(x).
    \end{aligned}
\end{equation}
This map satisfies \( \gamma(0)=x\). We also have, using proposition~\ref{PROPooWEYCooCvyHNr},
\begin{equation}
    \gamma(t_0+u)=\big( \exp(uX)\circ\exp(t_0X)\big)(x)=\exp(uX)\gamma(t_0).
\end{equation}
In that, we used the fact that \( G\) acts on \( M\), so that we have transformed the product inside the group \( \exp\big( (u+t_0)X \big)= \exp(uX)\exp(t_0x) \) into a composition of map.  Then
\begin{equation}
    \Dsdd{ \gamma(t_0+u) }{u}{0}=X\big( \gamma(t_0) \big)
\end{equation}
We conclude that \( \gamma\) satisfies the differential equation
\begin{equation}        \label{EQooFGSIooUplbmN}
    \gamma'(t)=-X\big( \gamma(t) \big).
\end{equation}
When \( M=\eR^n\), the Cauchy-Lipschitz theorem~\ref{ThokUUlgU} provides unicity of the solution on a maximal domain providing the map \( X\colon \eR^n\to \eR^n\) has nice properties.

\begin{normaltext}      \label{NORMooMGAUooIoLtjW}
    If we have to determine the transformations of \( \eR^n\) that satisfies some properties, the strategy is then the following:
    \begin{itemize}
        \item Suppose the searched group to be a connected Lie group.
        \item Write the condition with the groupe element \( \exp(tX)\) and differentiate with respect to \( t\). This point is what physicist call ``consider an infinitesimal transformation and neglect the higher order terms''.
        \item This provides an equation for \( X\). Typically a differential equation for the map \( X\colon \eR^n\to \eR^n\). Solve it.
        \item The group action is then retrieved solving the differential equation \eqref{EQooFGSIooUplbmN}.
    \end{itemize}
    Using that technique we will determine the isometries of \( \eR^n\) in proposition~\ref{PROPooDVIWooAFDNPy} and determine the conformal group around definition~\ref{DEFooVKNBooFBWQQM}.  % position 10906-29466: provides a more precise reference to the result instead of the definition.
\end{normaltext}

\begin{remark}
    When the Lie algebra is made of linear transformations, the last differential equation to solve is actually exponentiating a matrix.
\end{remark}

%---------------------------------------------------------------------------------------------------------------------------
\subsection{Example: determining the smooth isometries of the flat vector space}
%---------------------------------------------------------------------------------------------------------------------------

We know from theorem~\ref{ThoDsFErq} that the isometries of \( \eR^n\) are affine functions. We give now an alternative proof of that result.

\begin{proposition}     \label{PROPooDVIWooAFDNPy}
    The smooth\footnote{In fact we only need \( C^2\).} isometries of \( (\eR^n,\| . \|)  \)  are affine maps.
\end{proposition}

\begin{proof}
    The condition for a diffeomorphism \( \phi\colon \eR^n\to \eR^n\) to be an isometry is
    \begin{equation}        \label{EQooRKYWooFIKfYZ}
        \| \phi(x)- \phi(y) \|^2=\| x-y \|^2.
    \end{equation}
    We write \( \phi_t(x)= e^{-tX}x\) and take the derivative of \eqref{EQooRKYWooFIKfYZ} with respect to \( t\) at \( t=0\) taking into account that \( \phi_0(x)=x\):
    \begin{equation}        \label{EQooXEKMooGOktOj}
        \big( X(y)-X(x) \big)\cdot (x-y)=0.
    \end{equation}
    We used the fact that \( \Dsdd{ \phi_t(x) }{t}{0}=-X(x)\).

    We write the condition \eqref{EQooXEKMooGOktOj} with \( tx\) and take the derivative with respect to \( t\): \( dX_0(x)\cdot y+X(y)\cdot x=0\). The same with \( y\) gives
    \begin{equation}
        dX_0(x)\cdot y+dX_0(y)\cdot x=0.
    \end{equation}
    Taking \( x=e_i\) and \( y=e_j\) this equation reads
    \begin{equation}
        \frac{ \partial X_j }{ \partial x_i }+\frac{ \partial X_i }{ \partial x_j }=0.
    \end{equation}
    With \( i=j\) we get \( \frac{ \partial X_i }{ \partial x_i }=0\). The we compute
    \begin{equation}
        \frac{ \partial  }{ \partial x_j }\frac{ \partial X_i }{ \partial x_j }=-\frac{ \partial  }{ \partial x_j }\left( \frac{ \partial X_j }{ \partial x_i } \right)=-\frac{ \partial  }{ \partial x_i }\frac{ \partial X_j }{ \partial x_j }=0.
    \end{equation}
    We used the fact that \( X_j\) is of class \( C^2\) in order to permute the derivatives (lemma~\ref{LemPermDerrxyz}). We proved that
    \begin{equation}
        \frac{ \partial^2 X_i  }{ \partial x_j }=0
    \end{equation}
    for all \( i,j\). Thus \( X\) is linear.
\end{proof}

%--------------------------------------------------------------------------------------------------------------------------- 
\subsection{Other stuff}
%---------------------------------------------------------------------------------------------------------------------------

The concept of normal neighbourhood will be widely used for the study of the relations between a Lie group and its algebra. Let $M$ be a differentiable manifold. If $V$ is a neighbourhood of zero in $T_pM$ on which the exponential $\dpt{\exp_p}{T_pM}{M}$ is a diffeomorphism, then $\exp_pV$ is  \defe{normal neighbourhood}{normal!neighbourhood} of $p$.

\begin{lemma}
Let $\lG$ be a Lie algebra and $A$, a linear operator on $\lG$ (see as a common vector space) such that $\forall t\in\eR$, the map $e^{tA}$ is an automorphism of $\lG$. Then $A$ is a derivation of $\lG$.
\label{lem:autom_derr}
\end{lemma}

\begin{proof}
Let us consider $X$, $Y\in\lG$;  the assumption is
\[
  e^{tA}[X,Y]=[e^{tA}X,e^{tA}Y].
\]
Since $e^{tA}$ is a linear map, it has a ``good behavior''\ with the derivations:
\[
\Dsddc{e^{tA}[X,Y]}{t}{0}=\Dsddc{e^{tA}}{t}{0}[X,Y]=A[X,Y].
\]
Using on the other hand the linearity of $\ad$, we can see
\[
  (\ad(e^{tA}X))(e^{tA}Y)
\]
as a product ``matrix times vector''. Then
\begin{equation}
\begin{split}
  \Dsddc{[e^{tA}X,e^{tA}Y]}{t}{0}&=\Dsddc{(\ad e^{tA}X)Y}{t}{0}+\Dsddc{(\ad X)(e^{tA}Y)}{t}{0}\\
                                 &=(\ad AX)Y+(\ad X)(AY).
\end{split}
\end{equation}
Finally, $A[X,Y]=[AX,Y]+[X,AY]$.

\end{proof}

As notational convention, if $G$ and $H$ are Lie groups, their Lie algebra are denoted by $\lG$ and $\lH$.

\begin{lemma}		\label{LemAlgEtGroupesGenere}
	Let $\lG$ be a Lie algebra ans $\lS$ be a subset of $\lG$. The algebra of the group generated by $ e^{\lS}$ is the algebra generated by $\lS$.
\end{lemma}

Invariant vector fields are also often used in order to transport a structure from the identity of a Lie group to the whole group by $A_g(X_g)=A_e(dL_{g^{-1}}X_g)$ where $A_e$ is some structure and $X_g$, a vector at $g$.


\begin{corollary}\label{Ad_e}
An useful formula:
\[
   \Ad(e^X)=e^{\ad X}.
\]
\end{corollary}

\begin{corollary}
Another useful corollary of lemma~\ref{lemsur5d} is the particular case $\phi=\AD(e^X)$:
\[
   e^Xe^Ye^{-X}=e^{Ad(e^Y)X}.
\]
\label{cor:eXeYe-X}
\end{corollary}

\begin{proposition}
	Let $G$ be a connected Lie group.
	\begin{enumerate}

		\item
			All the left invariant vector fields are complete. That means that the map $X\mapsto  e^{X}$ is defined for every $X\in \mG$.
		\item
			The map $\exp\colon \mG\to G$ is a local diffeomorphism in a neighbourhood of $0$ in $\mG$.
	\end{enumerate}
\end{proposition}

\begin{proof}
	\begin{enumerate}

		\item
			The flow is a one parameter subgroup. Thus if $ e^{tX}$ is defined for $t\in[0,a]$, by composition, $ e^{2a}$ is defined. So $ e^{tX}$ is defined for every value of $t$ in $\eR$.
		\item
			Let us consider the manifold $G\times \mG$ and the vector field $\Xi$ defined by
			\begin{equation}
				\Xi_{(g,X)}=\tilde X_g\oplus 0\in T_g(G)\oplus T_X\mG\simeq T_{(g,X)}(G\times \mG).
			\end{equation}
			The flow of that vector field is given by
			\begin{equation}
				\Phi_t(g,X)=\big( g\exp(tX),X \big).
			\end{equation}
			In particular, $\Xi$ is a complete vector field, and we consider the global diffeomorphism
			\begin{equation}
				\begin{aligned}
					\Phi_1\colon G\times \mG&\to G\times \mG \\
					(g,X)&\mapsto \big( g\exp(X),X \big).
				\end{aligned}
			\end{equation}
			On the point $(e,X)$ we have $\Phi_1(e,X)=(\exp(X),X)$. Thus the exponential is the projection on the first component of $\Phi_1(e,X)$ and we can write
			\begin{equation}
				\exp(X)=\pr_1\circ\Phi_1(e,X).
			\end{equation}
			It is a smooth function since both the projection and $\Phi_1$ are smooth.

            Now, the differential $(d\exp)_0$ is the identity on $\mG$, so that the theorem of inverse function\footnote{Theorem \ref{ThoXWpzqCn}.} makes $\exp$ a local diffeomorphism.
	\end{enumerate}
\end{proof}


\begin{theorem}
For any $p\in M$, there exist a $\delta>0$ and a neighbourhood $W$ of $p$ in $M$ such that for every $q\in W$, we have

\begin{itemize}
\item $\exp_q$ is a diffeomorphism on $B\bdelta(0)\subset T_qM$,
\item $\exp_q B\bdelta(0)$ contains $W$
\end{itemize}
\end{theorem}
This theorem says that everywhere on a differentiable manifold, one can find a neighbourhood which is a normal neighbourhood of each of its points. Such a neighbourhood is said a \emph{totally} normal neighbourhood.

\begin{lemma}
In a Lie group, $e$ is an isolated fixed point for the inversion.
\end{lemma}

\begin{proof}
One can use an exponential map in a neighbourhood of $e$. In this neighbourhood, an element $g$ can be written as $g=e^X$ for a certain $X\in\lG$. The equality $g=g^{-1}$ gives (because the exponential is a diffeomorphism) $X=-X$, so that $X=0$ and $g=e$.
\end{proof}

%--------------------------------------------------------------------------------------------------------------------------- 
\subsection{Lie algebra of a Lie subgroup}
%---------------------------------------------------------------------------------------------------------------------------

\begin{probleme}
    The proposition \ref{PROPooCRKMooIIbKUM} is the same as \ref{PROPooLUEJooAAyHVh}, but with more assumptions. Why ?
\end{probleme}

% note pour moi-même: j'ai déjà téléchargé BIBooDUPSooZjcTHL et tout mis en un seul fichier 18F.pdf
% théorème 1.3 page 90 dans lecture 18.
\begin{proposition}[\cite{BIBooDUPSooZjcTHL}]       \label{PROPooLUEJooAAyHVh}
    Let \( G\) be a Lie group and \( H\) be a Lie subgroup of \( G\). If \( \lG\) is the Lie algebra of \( G\) and \( \lH\) the one of \( H\), then we have
    \begin{equation}
        \lH=\{ X\in \lG\tq \exp(tX)\in H\,\forall t\in \eR \}.
    \end{equation}
\end{proposition}

%+++++++++++++++++++++++++++++++++++++++++++++++++++++++++++++++++++++++++++++++++++++++++++++++++++++++++++++++++++++++++++ 
\section{Lifting a Lie subalgebra}
%+++++++++++++++++++++++++++++++++++++++++++++++++++++++++++++++++++++++++++++++++++++++++++++++++++++++++++++++++++++++++++

\begin{theorem}[Chevalley\cite{BIBooFLEXooPgvAlz}]       \label{THOooXALIooGiPVdD}
    We consider a class \( \mA\) (smooth, analytic) of functions. Let \( G\) be a \( \mA\)-Lie group and \( \lH\) be a Lie subalgebra of \( \lG\). There exists an unique connected \( \mA\)-Lie subgroup \( H\) whose Lie algebra is \( \lH\).
\end{theorem}

\begin{proof}
    Let \( X_1,\ldots, X_k\) be a basis of \( \lH\) in \( \lG\). We consider the corresponding left invariant vector fields \( X_i^L(g)=(dL_g)_e(X_i)\). 
    \begin{subproof}
    \spitem[A distribution]
    By lemma \ref{LEMooWTNRooCjlYMJ}, the vectors \( X_i^L(g)\) are linearly independents in \( T_gG\). Thus by setting
    \begin{equation}
        \mD_g=\Span\{ X_i^L(g) \}
    \end{equation}
    we define a \( k\) dimensional distribution\footnote{Definition \ref{DEFooYOMHooZJvsSt}.} on \( G\).
\spitem[Involutive]
    Since \( (dL_g)_e\) is linear we have
    \begin{equation}
        [X_i^L,X_j^L]=[X_i,X_j]^L,
    \end{equation}
    and since \( \lH\) is a Lie algebra, we have \( [X_i,X_j]\in \lH\), so that \( [X_i,X_j]^L\in\mD\).
\spitem[Frobenius]
    The Frobenius theorem \ref{THOooVRDYooIusxwW} says that \( \mD\) is integrable : there exists an unique maximal integral connected manifold \( H\) trough \( e\). There exists in particular an immersion \( i\colon H\to G\).
\spitem[\( H\) is a group]
        Note that, if \( g\in G\), then \( L_g(H)\) is still an integral manifold of \( \mD\). Let \( h_1,h_2\in H\). We have
        \begin{equation}
            h_1=L_{h_1}(e)\in H\cap L_{h_1}(H)
        \end{equation}
        because \( h_1\in H\) and \( L_{h_1}(e)\in L_{h_1}(H)\). We have:
        \begin{itemize}
            \item 
                The integral manifold \( H\) is maximal,
            \item
                 \( L_{h_1}(\mD)\) is still an integral manifold
             \item
                 The intersection \( L_{h_1}(\mD)\cap H\) is not empty.
        \end{itemize}
        From these properties we deduce \( L_{h_1}(H)\subset H\). Since \( h_2\in H\), we have in particular \( L_{h_1}(h_2)\in H\), so that \( h_1h_2\in H\).

        For the inverse, let \( h\in H\). We have \( L_{h^{-1}}(h)=e\in H\). So \( L_{h^{-1}}(H)\) is an integral manifold with an intersection with \( H\). Thus \( L_{h^{-1}}(H)\subset H\). In particular \( h^{-1}=L_{h^{-1}}(e)\in H\).

        We conclude that \( H\) is a subgroup of \( G\).
    \spitem[\( G\) is a Lie group]
        The manifold structure on \( H\) is the one of submanifold of \( G\). The fact that the multiplication and the inverse are in the class \( \mA\) as maps on the manifold \( G\) implies that they are in the same class \( \mA\) for the manifold \( H\).

        Thus \( H\) is a \( \mA\)-Lie group.
    \spitem[The inclusion is injective]
        This is always true for an inclusion map.
    \spitem[\( H\) is a Lie group]
        The manifold structure on \( H\) is given by the charts \( \psi_i\) of proposition \ref{PROPooRZIHooXIhnpq} which are composition of charts of \( G\) with the inclusion. The product and the inverse of \( H\) are the restrictions of the product and the inverse on \( G\); just check that their composition with the charts \( \psi_i\) are smooth.
    \spitem[The inclusion is an immersion]
        The inclusion of a submanifold is an immersion from proposition \ref{PROPooEWUCooTStAvb}.
    \spitem[\( H\) is a Lie subgroup of \( G\)]
        The conditions of the definition \ref{DEFooGCHDooHUMSju} are satisfied.
    \end{subproof}
    The existence part is proven. Now we prove the unicity.

    Let \( K\) be a Lie subgroup of \( G\) whose algebra is \( \lH\).
    \begin{subproof}
    \spitem[\( T_kK=\mD_k\)]
        The map \( (dL_g)_e\colon T_eG\to T_gG\) is an isomorphism of vector spaces. Thus we have
        \begin{equation}
            T_kK=(dL_k)_e(\lH)=\Span\{ (dL_k)_e(X_i) \}=\Span\{ X_i^L(k) \}=\mD_k.
        \end{equation}
    \spitem[Maximality]
        Thus \( K\) is an integral manifold of \( \mD\) trough \( e\). Thanks to the maximality of \( H\), we have \( K\subset H\). Since \( T_eK=T_eH\), the inclusion \( \iota\colon K\to H\) is a local isomorphism. There exists a neighbourhood \( V\) of \( e\) in \( G\) such that \( K\cap V=H\cap V\).
         
        Since a connected Lie group is generated by any neighbourhood of \( e\), we have \( K=H\).
    \end{subproof}
\end{proof}


%+++++++++++++++++++++++++++++++++++++++++++++++++++++++++++++++++++++++++++++++++++++++++++++++++++++++++++++++++++++++++++ 
\section{Covering}
%+++++++++++++++++++++++++++++++++++++++++++++++++++++++++++++++++++++++++++++++++++++++++++++++++++++++++++++++++++++++++++

\begin{lemma}[\cite{BIBooFLEXooPgvAlz}]     \label{LEMooSYVQooTjkgBL}
    Let \( G\) and \( H\) be connected Lie groups. We consider map \( \Phi\colon G\to H\) such that
    \begin{enumerate}
        \item
            \( \Phi\) is a smooth Lie group diffeomorphism\footnote{See definition \ref{DefAQIQooYqZdya}. Everywhere ``smooth'' means \(  C^{\infty}\).}.
             \item
                 The map \( d\Phi_e\colon \lG\to \lH\) is bijective.
    \end{enumerate}
    Then \( \Phi\) is a covering\footnote{Definition \ref{DEFooQBDWooVVrkkh}.}.
\end{lemma}

\begin{proof}
    Several points.
    \begin{subproof}
    \spitem[\( \Phi\) is surjective]
        Let \( h\in H\). From proposition \ref{PROPooYFZZooLUOuOj}, there exists a neighbourhood \( V\) of \( e\) in \( H\) on which the exponential is surjective. From proposition \ref{PropUssGpGenere}, there exist \( h_1,\ldots, h_n\) in \( V\) such that \( h=\prod_{i=1}^nh_i\).

        Since \( \exp\colon \lH\to V\) is surjective, there exist \( Y_i\in \lH\) such that \( h_i=\exp(Y_i)\). We know that \( d\Phi\colon \lG\to \lH\) is surjective, so there exist \( X_i\in\lG\) such that \( Y_i=d\Phi_e(X_i)\). Thus we have 
        \begin{equation}
            h_i=\exp\big( d\Phi(X_i) \big)=\Phi\big( \exp(X_i) \big)
        \end{equation}
        from lemma \ref{lemsur5d}.

        As far as a product is concerned,
        \begin{subequations}
            \begin{align}
                h_ih_j&=\Phi\big( \exp(X_i) \big)\Phi\big( \exp(X_j) \big)\\
                &=\Phi\big( \exp(X_i)\exp(X_j) \big)    \label{SUBEQooRTXGooZNrvUY}\\
                &=\Phi\big( \exp(Z) \big).      \label{SUBEQooFHKGooGHUITs}
            \end{align}
        \end{subequations}
        Justifications.
        \begin{itemize}
            \item For \eqref{SUBEQooRTXGooZNrvUY}. The map \( \Phi\) is a group morphism.
            \item For \eqref{SUBEQooFHKGooGHUITs}. The element \( Z\) is given by the Campbell-Baker-Hausdorff formula, theorem \ref{THOooYJPEooSpKHnC}.
        \end{itemize}
    \spitem[Continuous]
        The map \( \Phi\) is continuous as part of the definition of smooth diffeomorphism.
    \spitem[The third condition]
        Since the differential \( d\Phi_e\colon \lG\to \lH\) is bijective, the inversion theorem \ref{THOooDWEXooMClWVi} says that there exist open neighbourhood \( U_0\) of \( e\) and \( V_0\) of \( \Phi(e)\) such that  the restriction \( \Phi\colon U_0\to V_0\) is bijective.

        Since \( U_0\) and \( U_0^{-1}\) are open sets, the intersection is still open and still contain \( e\). We consider \( U=U_0\cap U_0^{-1}\), and \( V=\Phi(U)\). Now the restriction \( \Phi\colon U\to V\) is still a bijection.

        Let \( \Gamma=\Phi^{-1}(e)\). Since \( \Phi\) is not injective, this \( \Gamma\) can be a set.
        \begin{subproof}
        \spitem[\( \Gamma\) is a subgroup]
            Let \( g_1,g_2\in \Gamma\). Since \( \Phi\) is a group morphism, we have \( \Phi(g_1g_2)=\Phi(g_1)\Phi(g_2)=ee=e\), so that \( g_1g_2\in \Gamma\).
        \spitem[As an union]
            Let \( g\in\Phi^{-1}(V)\). There exists a \( g_0\in U\) such that \( \Phi(g_0)=\Phi(g)\). Recall that \( \Phi\) is not injective; \( g_0\) and \( g\) are element of \( G\) that are mapped on the same point in \( V\). Since \( \Phi\) is a morphism,
            \begin{equation}
                \Phi(gg_0^{-1})=\Phi(g)\Phi(g_0^{-1})=\Phi(g)\Phi(g_0)^{-1}=e.
            \end{equation}
            Thus \( gg_0^{-1}\in \Gamma\). Since \( U^{-1}=U\) we have proved that if \( g\in \Phi^{-1}(V)\), there exists \( a\in \Gamma\) and \( s\in U\) such that \( as=g\). In other words,
            \begin{equation}        \label{EQooEZBDooBIfRJx}
                \Phi^{-1}(V)=\bigcup_{a\in \Gamma}L_a(U).
            \end{equation}
        \spitem[Disjoint]
            It remain to be proven that the union \eqref{EQooEZBDooBIfRJx} is disjoint. Let \( a,b\in \Gamma\) such that \( L_a(U)\cap L_b(U)\neq \emptyset\). We'll prove the \( a=b\). Let \( x\in L_a(U)\cap L\b(U)\); we have
            \begin{equation}
                L_{b^{-1}}(x)\in L_{b^{-1} a}(U)\cap U.
            \end{equation}
            So, with \( c=b^{-1} a\), we have \( c\in\Gamma\) and
            \begin{equation}        \label{EQooVNIFooBtjxRi}
                L_c(U)\cap U\neq\emptyset.
            \end{equation}
            Thus there exist \( p_1,p_2\in U\) such that \( cp_1=p_2\) (these are two ways to write an element of \eqref{EQooVNIFooBtjxRi}). In particular we have \( \Phi(cp_1)=\Phi(p_2)\). Since \( c\in \Gamma\) we also have
            \begin{equation}
                \Phi(p_2)= \Phi(cp_1)=\Phi(c)\Phi(p_1)=\Phi(p_1).
            \end{equation}
            But \( \Phi\colon U\to V\) is injective. We deduce that \( p_1=p_2\), so that \( c=e\) and \( b^{-1}a=e\). Thus \( a=b\).
        \end{subproof}
    \end{subproof}
\end{proof}

\begin{proposition}[\cite{BIBooFSPWooQoosA}]        \label{PROPooIORNooLeuXPW}
    Let \( G,H\) be smooth Lie groups. If \( f\colon G\to H\) is a smooth morphism\footnote{The map is smooth with respect to the manifold structure while being a morphism of groups.}, then it has constant rank.
\end{proposition}

\begin{proof}
    Let \( G\) and \( H\) be Lie groups, let \( f\colon G\to H\) be a smooth morphism and \( g_0\in G\). We will prove that \( df_{g_0}\) has the same rank as \( df_e\).

    Since \( f\) is a morphism we have
    \begin{equation}
        (f\circ L_{g_0})(g)=f(g_0g)=f(g_0)f(g)=L_{f(g_0)}\big( f(g) \big)=\big( L_{f(g_0)}\circ f \big)(g).
    \end{equation}
    Thus we have \( f\circ L_{g_0}=L_{f(g_0)}\circ f\). We take the differential on both sides. On the left hand side:
    \begin{equation}
        d(f\circ L_{g_0})_e=df_{L_{g_0}(e)}\circ(dL_{g_0})_e,
    \end{equation}
    and on the right hand side:
    \begin{equation}
        d\big( L_{f(g_0)}\circ f \big)=(dL_{f(g_0)})_{f(g_0)}\circ df_e
    \end{equation}
    We know from lemma \ref{LEMooPIUFooHjyXln} that for every \( g,h\in G\), the differential \( (dL_h)_g\) is a vector space isomorphism. The composition of a linear map with a vector space isomorphism does not change the rank. Thus
    \begin{equation}
        \rank\Big[ (dL_{f(g_0)})_{f(g_0)}\circ df_e \Big]=\rank(df_e),
    \end{equation}
    and 
    \begin{equation}
        \rank\big[ df_g\circ(dL_{g_0})_e \big]=\rank(df_g),
    \end{equation}
    so that \( \rank(df_g)=\rank(df_e)\).
\end{proof}

\begin{proposition}[\cite{BIBooFLEXooPgvAlz}]    
    Let \( f\colon G\to H\) be a smooth morphism of Lie groups\footnote{This is a morphism of groups which is smooth.}. Then
    \begin{enumerate}
        \item
            The set \( \ker(f)\) is a closed subgroup of \( G\).
        \item
            The group \( \ker(f)\) is a Lie group.
        \item
            The Lie algebra of \( \ker(f)\) is \( \ker(df)\).
    \end{enumerate}
\end{proposition}

\begin{proof}
    In several parts.
    \begin{subproof}
        \spitem[\( \ker(f)\) is a subgroup]
            If \( g_1,g_2\in \ker(f)\), then
            \begin{equation}
                f(g_1g_2)=f(g_1)f(g_2)=ee=e,
            \end{equation}
            so that \( g_1g_2\in\ker(f)\). Moreover \( f(e)=e\), so that \( e\in \ker(f)\). This proves that \( \ker(f)\) is a subgroup of \( G\).
        \spitem[\( \ker(f)\) is closed]
            We are going to prove that \( G\setminus\ker(f)\) is open. For that we consider \( g\in G\setminus\ker(f)\) and we prove that there exists a neighbourhood of \( g\) contained in \( G\setminus\ker(f)\).

            Let \( g\in G\setminus\ker(f)\). We know that \( f(g)\neq e\) in \( H\), so that we can consider an open neighbourhood \( V\) of \( f(g)\) in \( H\) such that \( e\notin V\).

            Since \( f\) is continuous, \( f^{-1}(V)\) is an open set which contains \( g\). We have \( f^{-1}(V)\cap \ker(f)=\emptyset\) because if \( x\in f^{-1}(V)\cap\ker(f)\), we would have \( f(x)\in V\) and \( f(x)=e\) which is impossible. We deduce that \( f^{-1}(V)\cap\ker(f)=\emptyset\). Thus \( G\setminus\ker(f)\) contains \( f^{-1}(V)\) which is a neighbourhood of \( g\). We conclude that \( G\setminus \ker(f)\) is open, so that \( \ker(f)\) is closed.
        \spitem[\( \ker(f)\) is a Lie group]
            From the Cartan theorem \ref{THOooDEJHooVKJYBL}, we know that \( \ker(f)\) is a Lie subgroup of \( G\).
        \spitem[Lie algebra]
            Proposition \ref{PROPooLUEJooAAyHVh} says that the Lie algebra of \( \ker(f)\) is
            \begin{equation}
                \lK=\{ X\in \lG\tq \exp(tX)\in \ker(f)\,\forall t\in \eR \}.
            \end{equation}
            We have to prove that \( \lK=\ker(df)\). Let \( X\in \lG\). We have:
            \begin{subequations}
                \begin{align}
                    X\in\ker(df)&\Leftrightarrow df(X)=0\\
                    &\Leftrightarrow df(tX)=0\,\forall t\in \eR\\
                    &\Leftrightarrow \exp\big( df(tX) \big)=e\,\forall t\in \eR       \label{SUBEQooVVVKooVHdrfC}     \\
                    &\Leftrightarrow f\big( \exp(tX) \big)=e\,\forall t\in \eR  \label{SUBEQooCUSGooKTlkJn}     \\
                    &\Leftrightarrow tX\in\ker(f)\,\forall t\in \eR     \\
                    &\Leftrightarrow X\in\lK.
                \end{align}
            \end{subequations}
            Justifications.
            \begin{itemize}
                \item For \eqref{SUBEQooVVVKooVHdrfC}. The exponential is a bijection between a neighbourhood of \( 0\) in \( \lG\) and a neighbourhood of \( e\) in \( G\) (proposition \ref{PROPooYFZZooLUOuOj}). Thus if \( X\neq 0\) we can choose \( t_1,t_2\in \eR\) such that \( \exp(t_1X)\neq \exp(t_2X)\).
                \item For \eqref{SUBEQooCUSGooKTlkJn}. Lemma \ref{lemsur5d}.
            \end{itemize}
    \end{subproof}
\end{proof}

\begin{proposition}[\cite{BIBooFLEXooPgvAlz}]
    A continuous morphism between Lie groups is smooth.
\end{proposition}

\begin{proof}
    Let \( \phi\colon G\to H\) be a continuous morphism. We consider
    \begin{equation}
        \Gamma=\{ (g,h)\in G\times H\tq \phi(g)h^{-1}=e \}.
    \end{equation}
    \begin{subproof}
    \spitem[\( \Gamma\) is a subgroup of \( G\times H\)]
        Let \( (g_1,h_1)\) and \( (g_2,h_2)\) be in \( \Gamma\). We have \( (g_1,h_1)(g_2,h_2)=(g_1g_2,h_1h_2)\) while
        \begin{equation}
            \phi(g_1g_2)(h_1h_2)^{-1}=\phi(g_1)\underbrace{\phi(g_2)h_2^{-1}}_{=e}h_1^{-1}=e
        \end{equation}
    \spitem[\( \Gamma\) is closed]
        We consider
        \begin{equation}
            \begin{aligned}
                f\colon G\times H&\to H \\
                (g,h)&\mapsto \phi(g)h^{-1}. 
            \end{aligned}
        \end{equation}
        This is a continuous map because \( \phi\) is continuous (as well as the product and the inverse in \( H\)) and \( \Gamma=\ker(f)\). Thus \( \Gamma\) is closed.
    \spitem[\( \Gamma\) is a Lie subgroup]
        The Cartan theorem \ref{THOooDEJHooVKJYBL} says that \( \Gamma\) being a closed subgroup, it is a smooth Lie subgroup.
    \spitem[The projection]
        We introduce the projection
        \begin{equation}
            \begin{aligned}
                p\colon \Gamma&\to G \\
                (g,h)&\mapsto g
            \end{aligned}
        \end{equation}
        and we given some properties.
        \begin{subproof}
        \spitem[Smooth]
            By proposition \ref{PROPooCHVLooVFScOl}, the projection is smooth.
        \spitem[Bijective]
            Let \( (g_1,h_1),(g_2,h_2)\in \Gamma\) such that \( p(g_1,h_1)=p(g_2,h_2)\). Then \( g_1=g_2\). But for each \( i\) we have \( \phi(g_i)=h_i\), so \( h_1=\phi(g_1)=\phi(g_2)=h_2\).
        \spitem[Morphism]
            Because
            \begin{equation}
                p\big( (g_1,h_1)(g_2,h_2) \big)=p(g_1g_2,h_1h_2)=g_1g_2=p(g_1,h_1)p(g_2,h_2).
            \end{equation}
            For the inverse, \( (g,h)^{-1}=(g^{-1},h^{-1})\), so that 
            \begin{equation}
                p\big( (g,h)^{-1}\big)=p(g^{-1},h^{-1})=g^{-1}=p(g,h)^{-1}.
            \end{equation}
            So \( p\) is a group morphism.
        \end{subproof}
    \spitem[Smooth diffeomorphism]
        The differential of \( p\) at \( (e,e)\) is the map\footnote{Proposition \ref{PROPooKITOooTcsIiu}.} \( dp_{(e,e)}\colon \lG\times \lH\to \lG\) given by
        \begin{equation}
            dp_{(e,e)}(X,Y)=\Dsdd{ p( e^{tX},  e^{tX}) }{t}{0}=\Dsdd{  e^{tX} }{t}{0}=X,
        \end{equation}
        so that \( dp_{(e,e)}\) is surjective on \( \lG\). Its rank is the dimension of \( \lG\).

        Proposition \ref{PROPooIORNooLeuXPW} shows that \( p\) has constant rank. We know that this rank is maximal, so that the theorem \ref{THOooDWEXooMClWVi} thus says that \( p\) is a local smooth diffeomorphism everywhere.

        Since \( p\) is invertible, it is a global smooth diffeomorphism.
    \spitem[Conclusion]
        We have
        \begin{equation}
            \phi=\pr_2\circ p^{-1}.
        \end{equation}
        Since \( p^{-1}\) and \( \pr_2\) are smooth, the map \( \phi\) is smooth.
    \end{subproof}
\end{proof}

\begin{theorem}[\cite{BIBooFLEXooPgvAlz}]       \label{THOooZAEYooXCdxKI}
    Let \( G\) and \( H\) be Lie groups. We suppose that \( G\) is connected and simply connected. Let \( \rho\colon \lG\to \lH\) be a morphism of Lie algebra.

    There exists an unique smooth morphism \( f\colon G\to H\) such that \( df_e=\rho\).
\end{theorem}

\begin{proof}
    Let \( \lS\) be the graph of \( \rho\):
    \begin{equation}
        \lS=\{ (X,Y)\in \lG\oplus\lH\tq Y=\rho(X) \}.
    \end{equation}
    Since \( \rho\) is linear, the set \( \lS\) is a vector subspace of \( \lG \oplus\lH\). Moreover \( \lS\) is a Lie subalgebra. Indeed, if \( Y_i=\rho(X_i)\) we have
    \begin{equation}
        [Y_1,Y_2]=\big[ \rho(X_1),\rho(X_2) \big]=\rho\big( [X_1,X_2] \big)
    \end{equation}
    and then
    \begin{equation}
        \big[ (X_1,Y_1),(X_2,Y_2) \big]=\big( [X_1,X_2],[Y_1,Y_2] \big)=\Big( [X_1,X_2],\rho\big( [X_1,X_2] \big) \Big)\in\lS.
    \end{equation}
    The Chevalley theorem \ref{THOooXALIooGiPVdD} says that there exists a unique connected Lie subgroup \( S\) of \( G\times H\) whose Lie algebra is \( \lS\).

    We consider the inclusion \( \iota\colon S\to G\times H\) and the map
    \begin{equation}
        \begin{aligned}
            \varphi=\pr_1\circ\iota\colon S &\to G \\
            (g,h)&\mapsto g.
        \end{aligned}
    \end{equation}
    This is a smooth morphism, so that its differential is a morphism of Lie algebra.
    \begin{subproof}
    \spitem[\( d\varphi_{(e,e)}\) is bijective]
        We study the map
        \begin{equation}
            d\varphi_{(e,e)}=(d\pr_1)_{(e,e)}\circ d\iota_{(e,e)}\colon \lS\to \lG.
        \end{equation}
        \begin{subproof}
        \spitem[Injection]
            Let \( (X_1,Y_1)\) and \( (X_2,Y_2)\) be elements of \( \lS\) such that \( d\varphi_{(e,e)}(X_1,Y_1)=d\varphi_{(e,e)}(X_2,Y_2)\). Since \( d\iota_{(e,e)}\) is the identity, we have
            \begin{equation}
                (d\pr_1)_{(e,e)}(X_1,Y_2)=(d\pr_1)_{(e,e)}(X_2,Y_2).
            \end{equation}
            We deduce \( X_1=X_2\). By definition of \( \lS\) we also have \( Y_i=\rho(X_i)\) and then
            \begin{equation}
                Y_1=\rho(X_1)=\rho(X_2)=Y_2.
            \end{equation}
        \spitem[Surjection]
            Let \( X\in\lG\). We consider \( Y=\rho(X)\), so that \( (X,Y)\in \lS\) and
            \begin{equation}
                d\varphi_{(e,e)}(X,Y)=X.
            \end{equation}
        \end{subproof}
    \spitem[Diffeomorphism]
        We know that \( \varphi\colon S\to G\) is a smooth morphism and that \( d\varphi_{(e,e)}\) is bijective. Theorem \ref{THOooDWEXooMClWVi} concludes that \( \varphi\) is a local smooth diffeomorphism. So there exists a neighbourhood \( V_1\) of \( (e,e)\) in \( S\) and a neighbourhood \( V_2\) of \( e\) in \( G\) such that \( V_2=\varphi(V_1)\). By proposition \ref{PropUssGpGenere}, the map \( \varphi\) is surjective on the connected component of \( e\) in \( G\). Since \( G\) is connected, \( \varphi\) is a global smooth diffeomorphism.
    \spitem[The map \( \Phi\)]
        We define
        \begin{equation}
            \begin{aligned}
                \Phi\colon G&\to H \\
                \Phi&=\pr_2\circ \varphi^{-1}.
            \end{aligned}
        \end{equation}
        As composition of smooth morphisms, the map \( \Phi\) is a smooth morphism. We have to prove that \( d\Phi_e=\rho\).
    \spitem[The map \( d\varphi^{-1}_e\)]
        Let \( X\in\lG\). We have
        \begin{subequations}
            \begin{align}
                d\varphi_{(e,e)}\big( X,\rho(X) \big)&=(d\pr_1)_e\circ d\iota_{(e,e)}\big( X,\rho(X) \big)\\
                &=(d\pr_1)_e\big( X,\rho(X) \big)\\
                &=X.
            \end{align}
        \end{subequations}
        The proposition \ref{PROPooPEMLooPQcywG} concludes that 
        \begin{equation}
            (d\varphi^{-1})_e(X)=\big( X,\rho(X) \big).
        \end{equation}
    \spitem[Conclusion]
        Now we conclude with the differential of \( \Phi\); if \( X\in\lG\) we have
        \begin{subequations}
            \begin{align}
                d\Phi_e(X)&=(d\pr_2)_{\varphi^{-1}(e)}\circ (d\varphi^{-1})_e(X)\\
                &=(d\pr_2)_{\varphi^{-1}(e)}\big( X,\rho(X) \big)\\
                &=\rho(X)
            \end{align}
        \end{subequations}
        and \( \Phi\) is the map we were searching for.
    \end{subproof}
\end{proof}


\begin{proposition}     \label{PROPooCRKMooIIbKUM}
Let $G$ be a Lie group and $H$, a Lie subgroup of $G$ ($\lG$ and $\lH$ are the corresponding Lie algebras). We suppose that $H$ has at most a countable number of connected components. Then
\begin{equation}
  \lH=\{ X\in\lG:\forall t\in\eR,e^{tX}\in H \}
\end{equation}
\end{proposition}

\begin{proof}
We will once again use the lemma ~\ref{LEMooEBQUooKXkCda} with $\lN=\lH$ and $\lM$, a complementary vector space of $\lH$ in $\lG$. We define
\[
   V=\exp\mU_m\exp\mU_h
\]
where $\mU_m$ and $\mU_h$ are the sets given by the lemma. We consider on $V$ the induced topology from $G$. If we define
\[
   \mA=\{A\in\mU_m:e^{A}\in H\},
\]
we have
\begin{equation}\label{eq:union_A}
   H\cap V=\bigcup_{A\in\mA}e^{A}e^{\mU_h}.
\end{equation}
First, the definition of $V$ makes clear that the elements of the form $\exp A\exp\mU_h$ are in $V$. They are also in $H$ because $\exp A\in H$ (definition of $\mA$) and $\exp\mU_h$ still by definition. In order to see the inverse inclusion, let us consider a $h\in H\cap V$. We know that
\begin{equation}\label{eq:AB_to_exp}
(A,B)\to\exp A\exp B
\end{equation}
is a diffeomorphism between $\mU_m\times\mU_h$ and a neighbourhood of $e$ in $G$ which we called $V$. Thus any element of $V$ (\emph{a fortiori} in $V\cap H$) can be written as $\exp A\exp B$ with $A\in\mU_m$ and $B\in\mU_h$. Then $h=e^Ae^B$ for some $A\in\mU_m$, $B\in\mU_h$. Since $H$ is a group and $e^B\in H$, in order the product to belongs to $H$, $e^A$ must lies in $H$: $A\in\mA$.

\begin{remark}\label{rem:union_disj}
Note that since \eqref{eq:AB_to_exp} is diffeomorphic, the union in right hand side of \eqref{eq:union_A} is disjoint. Each member of this union is a neighbourhood in $H$ because it is a set $h\exp\mU_h$ where $\exp\mU_h$ is a neighbourhood of $e$ in $H$.
\end{remark}

Now we consider the map $\dpt{\pi}{V}{\mU_m}$,
\[
  \pi(e^{X}e^Y)=X
\]
if $X\in\mU_m$ and $Y\in\mU_h$. This is a continuous map which sends $H\cap V$ into $\mA$. The identity component of $H\cap V$ (in the sense of topology of $V$) is sent to a countable subset of $\mU_m$. Indeed by remark~\ref{rem:union_disj}, identity component of $H\cap V$ is only one of the terms in the union \eqref{eq:union_A}, namely $A=0$. But we know that $\pi^{-1}(o)=\exp\mU_h$, thus $\exp\mU_h$ is the identity component of $H\cap V$ for the topology of $V$.
\end{proof}



Now we take back our example with $G=S^1\times S^1$, $H=\gamma(\eR)$. In this case, the theorem doesn't works. Let us see why as deep as possible. We have $\lG=\eR\oplus\eR=\eR^2$ and $\lH=\eR$, a one-dimensional vector subspace of $\lG$. ($\lH$ is a ``direction ''\ in $\lG$) First, we build the neighbourhood $V$ of $0$ in $\lG$. It is standard to require that $\exp$ is diffeomorphic between $V$ and an open around $(1,1)\in S^1\times S^1$. It also must satisfy $e^{V\cap\lH}=e^V\cap H$. This second requirement is impossible.

Intuitively. We can see $V\subset\lG$ as a little disk tangent to  the torus. The exponential map deposits it on the torus, as well that $e^V$ covers a little area on $G$. Then $e^V\cap H$ is one of these amazing open subset of $\Gamma$ which are dense in a certain domain of $G$.

On the other hand, $V\cap\lH$ is just a little vector in $\lH$; the exponential deposits it on a small line in $G$. This is not the same at all. Then lemma~\ref{LEMooOBIMooVvIDnb} fails in our case. Let us review the proof of this lemma until we find a problem.

Let $W_0\subset\lG$  be a neighbourhood of $0$ which is in bijection with an open around $e$ in $G$. We consider $N_0$, an open subset of $H$ such that $N_0\subset W_0$ and $N_0$ is in bijection with $N_e$, a neighbourhood of $e$ in $G$. Until here, no problems. But now the proof says that there exists an open $U_e$ in $G$ such that $N_e=U_e\cap H$. This is false in our case. Indeed, $N_e=e^{N_0}$ is just a segment in $G$ while any subset of $G$ of the form $U_e\cap H$ is an ``amazing''\ open.

So we see that deeply, the obstruction for a Lie subgroup to be a topological Lie subgroup resides in the fact that the topology of a submanifold is \emph{more} than the induced topology, so that we can't automatically find the open $U_e$ in $G$.


Note that two groups which have the same Lie algebra are not necessarily isomorphic. For example the sphere $S^2$ and $\eR^2$ both have $\eR^2$ as Lie algebra. But two groups with same Lie algebra are locally the same. More precisely, we have the following lemma.

\begin{lemma}
If $G$ is a Lie group and $H$, a topological subgroup of $G$ with the same Lie algebra ($\lH=\lG$), then there exists a common neighbourhood $A$ of $e$ of $G$ and $G$ on which the products in $G$ and $H$ are the same.
\end{lemma}

\begin{proof}
The exponential is a diffeomorphism between $U\subset\lG$ and $V\subset G$ and between $U'\subset\lH$ and $W\subset H$ (obvious notations). We consider an open $\mO\subset\lH$ such that $\mO\subset U\subset U'$. The exponential is diffeomorphic from $\mO$ to a certain open $A$ in $G$ and $H$. Since $H$ is a subgroup of $G$, the product $e^Xe^Y$ of elements in $A$ is the same for $H$ and $G$. (cf error~\ref{err:gp_meme_alg})
\end{proof}

Under the same assumptions, we can say that $H$ contains at least the whole $G_0$ because it is generated by any neighbourhood of the identity. Since $H$ is a subgroup, the products keep in $H$.

For a semisimple Lie group, the Lie algebras $\partial(\lG)$ and $\ad(\lG)$ are the same. Then $\Int(\lG)$ contains at least the identity component of $\Aut(\lG)$. Since $\Int(\lG)$ is connected, for a semisimple group, it is the identity component of $\Aut(\lG)$.

\begin{proposition}     \label{ProplGcompactKillNeg}\label{prop:compact_Killing}
    Let $\lG$ be a real Lie algebra.
    \begin{enumerate}
    \item If $\lG$ is semisimple, then $\lG$ is compact if and only if  the Killing form is strictly negative definite.
    \item If it is compact then it is a direct sum
    \begin{equation}
    \lG=\mZ\oplus [\lG,\lG]
    \end{equation}
    where $\mZ$ is the center of $\lG$ and the ideal $[\lG,\lG]$ is compact and semisimple.
    \end{enumerate}
\end{proposition}

\begin{proof}
\subdem{If the Killing form is nondegenerate}
We consider $\lG$, a Lie algebra whose Killing form is strictly negative definite. Up to some dilatations (and a sign), this is the euclidian metric. Then $O(B)$, the group of linear transformations which leave $B$ unchanged is compact in the topology of $\GL(\lG)$: this is almost the rotations. From equation \eqref{eq:Aut_Iso}, $\Aut(\lG)\subset O(B)$. With this, $\Aut(\lG)$ is closed in a compact, then it is compact. Then $\Int(\lG)$ is closed in $\Aut(\lG)$ --here is the assumption of semi-simplicity-- and $\Int(\lG)$ is compact.
\subdem{If $\lG$ is compact}
Since $\lG$ is compact, $\Int(\lG)$ is compact in the topology of $\Aut(\lG)$; then there exists an $\Int(\lG)$-invariant quadratic form $Q$. In a suitable basis $\{X_1,\ldots,X_n\}$ of $\lG$, we can write this form as
\[
   Q(X)=\sum x_i^2
\]
for $X=\sum x_iX_i$. In this basis the elements of $\Int(\lG)$ are orthogonal matrices and the matrices of $\ad(\lG)$ are skew-symmetric matrices (the Lie algebra of orthogonal matrices). Let us consider a $X\in\lG$ and denote by $a_{ij}(X)$ the matrix of $\ad(X)$. We have
\begin{equation}
\begin{split}
  B(X,X)=\tr(\ad X\circ\ad X)
        =\sum_i\sum_ja_{ij}(X)a_{ji}(X)
    =-\sum_{ij}a_{ij}(X)^2\leq 0.
\end{split}
\end{equation}
Then the Killing form is negative definite\footnote{Here we use ``negative definite''\ and ``\emph{strictly} negative definite''; in some literature, the terminology is slightly different and one says ``\emph{semi} negative definite''\ and ``negative definite''.}. On the other hand, $B(X,X)=0$ implies $\ad(X)=0$ and $X\in\mZ(\lG)$. Thus $\lG^{\perp}\subset\mZ$. If $\lG$ is semisimple, this center is zero; this conclude the first item of the proposition.

Now $\mZ$ is an ideal and corollary~\ref{cor:decomp_ideal} decomposes $\lG$ as
\begin{equation}
  \lG=\mZ\oplus\lG'.
\end{equation}
Let us suppose that the restriction of $B$ to $\lG'\times\lG'$ is actually the Killing form on $\lG'$ (we will prove it below). Then the Killing form on $\lG'$ is strictly negative definite; then $\lG'$ is compact.

Now we prove that the Killing form on $\lG$ descent to the Killing form on~$\lG'$. Remark that $\mZ$ is invariant under all the automorphism. Indeed consider $Z\in\mZ$, i.e.  $[X,Z]=0$. If $\sigma$ is an automorphism,
\[
   [X,\sigma Z]=\sigma[\sigma^{-1} X,Z]=0.
\]
Here the difference between $\Int(\lG)$ and $\Aut(\lG)$ is the fact that $\Int(\lG)$ is compact; then we can construct a $\Int(\lG)$-invariant quadratic form $Q$, but not a $\Aut(\lG)$-invariant one. We consider an orthogonal complement (with respect to $Q$) $\lG'$ of $\mZ$:
\begin{equation}
   \lG=\lG'\oplus_{\perp}\mZ.
\end{equation}
The algebra $\lG'$ is also invariant because for any $Z\in\mZ$,
\[
Q(Z,\sigma X)=Q(\sigma^{-1}(Z),X)=0.
\]
It is also clear that $\mZ$ is invariant under $\ad\lG$ because $(\ad X)Z=0$. Finally $\lG'$ is invariant as well under $\ad(\lG)$. Indeed $a\in\ad(\lG)$ can be written as $a=a'(0)$ for a path $a(t)\in\Int(\lG)$. We identify $\lG$ and his tangent space (as vector spaces),
\[
  aX=\Dsdd{ a(t)X }{t}{0}.
\]
If $X\in\lG'$, $a(t)X\in\lG'$ for any $t$ because $\lG'$ is invariant under $\Int(\lG)$\footnote{As physical interpretation, if something is invariant under a group of transformations, it is invariant under the infinitesimal transformations as well.}. Thus $a(t)X$ is a path in $\lG'$ and his derivative is a vector in $\lG'$.

All this make $\lG'$ an ideal in $\lG$; then the Killing form descent by lemma~\ref{lem:Killing_descent_ideal}. Now if $X\in\lG$, we have
\begin{equation}
  B(X,X)=\tr(\ad X\circ\ad X)
        =\sum_{ij}a_{ij}(X)a_{ji}(X)
    =-\sum_{ij}a_{ij}(X)^2;
\end{equation}
then $B(X,X)\leq 0$ and the equality holds if and only if $\ad X=0$ i.e. if and only if $X\in\mZ$. Thus $B$ is strictly negative definite on $\lG'$.

Up to now we have proved that $\lG'$ is semisimple (because $B$ is nondegenerate) and compact (because $B$ is strictly negative definite).

It remains to be proved that $\lG'=[\lG,\lG]=\dD(\lG)$. From corollary~\ref{cor:decomp_ideal}, $\dD\lG$ has a complementary $\lA$ which is also an ideal: $\lG=\dD\lG+\lA$. Then $[\lG,\lA]\subset\dD\lG$ and $[\lG,\lA]\subset\lA\cap\dD\lG:\{0\}$. Then $\lA\subset\mZ$, so that
\begin{align}\label{eq:G_Z_B}
   \lG=\mZ+\dD\lG&&\text{(non direct sum)}.
\end{align}
Now we have to prove that the sum is actually direct. The ideal $\mZ$ has a complementary ideal $\lB$: $\lG=\mZ\oplus\lB$ and
\[
   \dD\lG=[\lG,\lG]\subset\underbrace{[\lG,\mZ]}_{=0}+[\lG,\lB]\subset\lB.
\]
Then $\dD\lG\subset\lB$ which implies that $\dD\lG\cap\mZ=\{0\}$ because the sum $\lG=\mZ\oplus\lB$ is direct. Then the sum \eqref{eq:G_Z_B} is direct.

\end{proof}

\begin{proposition}
A real Lie algebra $\lG$ is compact if and only if one can find a compact Lie group $G$ which Lie algebra is isomorphic to $\lG$.
\label{prop:alg_grp_compact}
\end{proposition}

\begin{proof}
\subdem{Direct sense} Since $\lG$ is compact, $\lG=\mZ\oplus\dD\lG$ with $\dD\lG=\lG'$ compact and semisimple; in particular, the center of $\lG'$ is $\{0\}$. Since $\mZ$ is compact and abelian, it is isomorphic to the torus $S^1\times\ldots\times S^1$. Since $\lG'$ is compact, $\Int(\lG')$ is compact, but the Lie algebra if $\Int(\lG')$ is --by definition--  $\ad(\lG')$. The center of a semisimple Lie algebra is zero; then $\ad X'=0$ implies $X=0$ (for $X\in\lG'$). Then $\ad$ is an isomorphism between $\lG'$ and $\ad\lG'$.

All this shows that --up to isomorphism-- $\mZ$ and $[\lG,\lG]$ are Lie algebras of compact groups. We know from lemma~\ref{lemLeibnitz} that the Lie algebra of $G\times H$ is $\lG\oplus\lH$. Thus, here, $\lG$ is the Lie algebra of the compact group $S^1\times\ldots\times S^1\times\Int(\lG)$.
\subdem{Reverse sense}
We consider a compact group $G$ and we have to see the its Lie algebra $\lG$ is compact. If $G$ is connected, $\Ad_G$ is an analytic homomorphism from $G$ to $\Int(\lG)$. If $G$ is not connected, the Lie algebra of $G$ is $T_eG_0$ ($G_0$ is the identity component of $G$) where $G_0$ is connected and compact because closed in a compact.
\end{proof}

\begin{proposition}
Let $\lG$ be a real Lie algebra and $\mZ$, the center of $\lG$. We consider $\lK$, a compactly embedded in $\lG$. If $\lK\cap\mZ=\{0\}$ then the Killing form of $\lG$ is strictly negative definite on $\lK$.
\label{prop:K_Z_Killing}
\end{proposition}

\begin{proof}
Let $B$ be the Killing form on $\lG$ and $K$ the analytic subgroup of $\Int(\lG)$ whose Lie algebra is $\ad_{\lG}(\lK)$. By assumption, $K$ is a compact Lie subgroup of $\GL(\lG)$. Then there exists a quadratic form on $\lG$ invariant under $K$, and a basis in which the endomorphisms $\ad_{\lG}(T)$ for $T\in\lK$ are skew-symmetric because the matrices of $K$ are orthogonal. If the matrix of $\ad T$ is $(a_{ij})$, then
\begin{equation}
   B(T,T)=\sum_{ij}a_{ij}(T)a_{ji}(T)
         =-\sum_{ij}a_{ij}^2(T)\leq 0,
\end{equation}
and the equality hold only if $\ad T=0$ i.e. if $T\in\mZ$. From the assumptions, $\lK\cap\mZ=\{0\}$; then $B(T,T)=0$ if and only if $T=0$.
\end{proof}
