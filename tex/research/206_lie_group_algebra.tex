% This is part of Giulietta
% Copyright (c) 2013-2015, 2019
%   Laurent Claessens
% See the file fdl-1.3.txt for copying conditions.


Here are the results which relate Lie groups and Lie algebras.

%+++++++++++++++++++++++++++++++++++++++++++++++++++++++++++++++++++++++++++++++++++++++++++++++++++++++++++++++++++++++++++ 
\section{Lie algebra of a Lie group}
%+++++++++++++++++++++++++++++++++++++++++++++++++++++++++++++++++++++++++++++++++++++++++++++++++++++++++++++++++++++++++++

\begin{propositionDef}      \label{DEFooKDCPooZOJsMD}
    If \( G\) is a Lie group, its tangent space on at the identity is a Lie algebra. 
    
    This is the \defe{Lie algebra}{Lie algebra of a Lie group} is is tangent space at identity. From a notational point of view, this is written
    \begin{equation}
        \lG=T_eG.
    \end{equation}
\end{propositionDef}

\begin{proof}
    We know from proposition \ref{PROPooOHLQooCNetuD} that \( T_eG\) is a vector space. We have to define a Lie bracket on it. For that we use the left-invariant vector field. Let \( X\in T_eG\) and \( g\in M\) we define
    \begin{equation}
        X^L_g=dL_gX
    \end{equation}
    where \( L_g\colon G\to G\) is the left translation: \( L_g(h)=gh\). If \( X,Y\in T_eG\) we define
    \begin{equation}
        [X,Y]=[X^L,Y^L]_e
    \end{equation}
    where the bracket on the right hand side is the commutator of vector field defined in \ref{DEFooHOTOooRaPwyo}. It defines a Lie algebra structure by the proposition \ref{PROPooSWQSooSEfTuX}.
\end{proof}

Now a great theorem without proof:
\begin{theorem} \label{tho:loc_isom}
Two Lie groups are locally isomorphic if and only if their Lie algebras are isomorphic.
\end{theorem}

\begin{theorem}		\label{ThoSubGpSubAlg}		\label{tho:gp_alg}
If $G$ is a Lie group, then
\begin{enumerate}
\item\label{ThoSubGpSubAlgi} if $\lH$ is the Lie algebra of a Lie subgroup $H$ of $G$, then it is a subalgebra of $\lG$,
\item Any subalgebra of $\lG$ is the Lie algebra of one and only one connected Lie subgroup of $G$.
\end{enumerate}

\begin{probleme}
À mon avis, il faut dire ``connexe et simplement connexe'', et non juste ``connexe''.
\end{probleme}

\end{theorem}
\begin{proof}

\subdem{First item}
Let $\dpt{i}{H}{G}$ be the identity map; it is a homomorphism from $H$ to $G$, thus $di_e$ is a homomorphism from $\lH$ to $\lG$. Conclusion: $\lH$ is a subalgebra of $\lG$.

\subdem{Characterization for $\lH$}
Before to go on with the second point, we derive an important characterization of $\lH$:
\begin{equation}\label{eq:path_alg}
\lH=\{X\in\lG:\text{the map } t\to\exp tX\text{ is a path in $H$}\}.
\end{equation}
For that, consider $\dpt{\exp_H}{\lH}{H}$ and $\dpt{\exp_G}{\lG}{G}$; from unicity of the exponential, for any $X\in\lH$, $\exp_HX=\exp_GX$, so that one can simply write ``$\exp$''\ instead of ``$\exp_h$''\ or ``$\exp_G$''.

Now, if $X\in\lH$, the map $t\to\exp tX$ is a curve in $H$. But it is not immediately clear that such a curve in $H$ is automatically build from a vector in $\lH$ rather than in $\lG$.  More precisely, consider a $X\in\lG$ such that $t\to\exp tX$ is a path (continuous curve) in H. By lemma~\ref{lem:var_cont_diff}, the map $t\to\exp tX$ is differentiable and thus by derivation, $X\in\lH$.
The characterisation \eqref{eq:path_alg} is proved.

Thus $\lH$ is a Lie subalgebra of $\lG$.

\subdem{Second item}
For the second part, we consider $\lH$ any subalgebra of $\lG$ and $H$, the smallest subgroup of $G$ which contains $\exp\lH$. We also consider a basis $\{X_1,\ldots,X_n\}$ of $\lG$ such that $\{X_{r+1},\ldots,X_n\}$ is a basis of $\lH$.

By corollary~\ref{cor:/24}, the set of linear combinations of elements of the form $X(M)$ with $M=(0,\ldots,0,m_{r+1},\ldots,m_r)$ form a subalgebra of $U(\lG)$. If $X=x_1X_1+\cdots+x_nX_n$, we define $|X|=(x_1^2+\cdots+x_n^2)^{1/2}$ ($x_i\in\eR$).

Let us consider a $\delta>0$ such that $\exp$ is a diffeomorphism (normal neighbourhood) from $B_{\delta}=\{X\in\lG:|X|<\delta\}$ to a neighbourhood $N_e$ of $e\in G$ and such that $\forall x,y,xy\in N_e$,
\begin{equation}\label{eq:coord_xy}
   (xy)_k=\sum_{M,N}C^{[k]}_{MN}x^My^N
\end{equation}
holds\footnote{The validity of this second condition is assured during the proof of theorem~\ref{tho:loc_isom} which is not given here.}. We note $V=\exp(\lH\cap B_{\delta})\subset N_e$. The map
\[
   \exp(x_{r+1}X_{r+1}+\cdots+x_nX_n)\to(x_{r+1},\ldots,x_n)
\]
is a coordinate system on $V$ for which $V$ is a connected manifold. But $\lH\cap B_{\delta}$ is a submanifold of $B_{\delta}$, then $V$ is a submanifold of $N_e$ and consequently of~$G$.

Let $x$, $y\in V$ such that $xy\in N_e$ (this exist: $x=y=e$); the canonical coordinates of $xy$ are given by \eqref{eq:coord_xy}. Since $x_k=y_k=0$ for $1\leq k\leq r$, $(xy)_k=0$ for the same $k$ because for $(xy)_k$ to be non zero, one need $m_1=\ldots=m_r=n_1=\ldots=n_r=0$ -- otherwise, $x^M$ or $y^N$ is zero. Now we looks at $C^{[k]}_{MN}$ for such a $k$ (say $k=1$ to fix ideas): $[k]=(\delta_{11},\ldots,\delta_{1k})=(1,0,\ldots,0)$ and by definition of the $C$'s,
\[
   X(M)X(N)=\sum_PC_{MN}^PX(P).
\]
But we had seen that the set of the $X(A)$ with $A=(0,\ldots,0,a_{r+1},\ldots,a_n)$ form a subalgebra of $U(\lG)$. Then, only terms with $P=(0,\ldots,0,p_{r+1},\ldots,p_n)$ are present in the sum; in particular, $C_{MN}^{[k]}=0$ for $k=1,\ldots,r$. Thus $VV\cap N_e\subset V$.

The next step is to consider $\mV$, the set of all the subset of $H$ whose contains a neighbourhood of $e$ in $V$. We can check that this fulfils the six axioms of a topological group\index{topological!group}:

\begin{enumerate}
\item The intersection of two elements of $\mV$ is in $\mV$;
\item the intersection of all the elements of $\mV$ is $\{e\}$;
\item any subset of $H$ which contains a set of $\mV$ is in $\mV$;
\item If $\mU\in\mV$, there exists a $\mU_1\in\mV$ such that $\mU_1\mU_1\subset\mU$ because $VV\cap N_e\subset V$;
\item if $\mU\in\mV$, then $\mU^{-1}\in\mV$ because the inverse map is differentiable and transforms a neighbourhood of $e$ into a neighbourhood of $e$;
\item if $\mU\in\mV$ and $h\in H$, then $h\mU h^{-1}\in\mV$.
\end{enumerate}

To see this last item, we denote by $\log$ the inverse map of $\dpt{\exp}{B_{\delta}}{N_e}$. By definition of $V$, it sends $V$ on $\lH\cap B_{\delta}$. If $X\in\lG$, there exists one and only one $X'\in\lG$ such that $he^{tX}h^{-1}=e^{tX'}$ for any $t\in\eR$. Indeed we know that $he^{X}h^{-1}=e^{\Ad_hX}$, then $X'$ must satisfy $e^{tX'}=e^{\Ad_htX}$. If it is true for any $t$, then, by derivation, $X'=\Ad_hX$.

The map $X\to X'$ is an automorphism of $\lG$ which sent $\lH$ on itself. So one can find a $\delta_1$ with $0<\delta_1<\delta$ such that
\[
   h\exp({B_{\delta_1}\cap\lH})h^{-1}\subset V.
\]
Indeed, $he^{\lH} h^{-1}\subset\lH$, so that taking $\delta_1<\delta$, we get the strict inclusion. We can choose $\delta_1$ even smaller to satisfy $he^{B_{\delta_1}}h^{-1}\subset N_e$. Since the map $X\to\log(he^{X}h^{-1})$ from $B_{\delta_1\cap\lH}$ to $B_{\delta}\cap\lH$ is regular, the image of $B_{\delta_1}\cap\lH$ is a neighbourhood of $0$ in $\lH$. Thus $he^{B_{\delta_1}\cap\lH}h^{-1}$ is a neighbourhood of $e$ in $V$. Finally, $h\mU h^{-1}\in\mV$ and the last axiom of a topological group is checked.

This is important because there exists a topology on $H$ such that $H$ becomes a topological group and $\mV$ is a family of neighbourhood of $e$ in $H$. In particular, $V$ is a neighbourhood of $e$ in $H$.

For any $z\in G$, we define the map $\dpt{\phi_z}{zN_e}{B_{\delta}}$ by
\begin{equation}
  \phi_z(ze^{x_1X_1+\cdots+x_nX_n})=(x_1,\ldots,x_n),
\end{equation}
and we denote by $\varphi_z$ the restriction of $\phi_z$ to $zV$. If $z\in H$, then $\varphi_z$ sends the neighbourhood $zV$ of $z$ in $H$ to the open set $B_{\delta}\cap\lH$ in $\eR^{n-r}$. Indeed, an element of $zV$ is a $ze^Z$ with $Z\in\lH\cap B_{\delta}$ which is sent by $\varphi_z$ to an element of $\lH\cap B_{\delta}$. (we just have to identify $x_1X_1+\cdots+x_nX_n$ with $(x_1,\ldots,x_n)$).

Moreover, if $z_1,z_2\in H$, the map $\varphi_{z_1}\circ\varphi_{z_2}^{-1}$ is the restriction to an open subset of $\lH$ of $\phi_{z_1}\circ\phi_{z_2}$. Then $\varphi_{z_1}\circ\varphi_{z_2}^{-1}$ is differentiable. Conclusion: $(H,\varphi_z: z\in H)$ is a differentiable manifold.

Recall that the definition of $\lH$ was to be a subalgebra of $\lG$; therefore $V=e^{\lH\cap B_{\delta}}$ is a submanifold of $G$. But the left translations are diffeomorphism of $H$ and $H$ is the smallest subgroup of $G$ containing $e^{\lH}$. Thus $H$ is a manifold on which the multiplication is diffeomorphic and consequently, $H$ is a Lie subgroup of $G$.

Rest to prove that the Lie algebra of $H$ is $\lH$ and the unicity part of the theorem.

We know that $\dim H=\dim\lH$ and moreover for $i>r$, the map $t\to\exp tX_i$ is a curve in $H$. Now, the fact that $\lH$ is the set of $X\in\lG$ such that $t\to\exp tX$ is a path in $H$ show that $X_i\in\lH$. Then the Lie algebra of $H$ is $\lH$ and $H$ is a connected group because it is generated by $\exp\lH$ which is a connected neighbourhood of $e$ in $H$.

We turn our attention to the unicity part. Let $H_1$ be a connected Lie subgroup of $G$ such that $T_eH_1=\lH$. Since $\exp_{\lH}X=\exp_{\lH_1}X$, $H=H_1$ as set. But $\exp$ is a differentiable diffeomorphism from a neighbourhood of $0$ in $\lH$ to a neighbourhood of $e$ in $H$ and $H_1$, so as Lie groups, $H$ and $H_1$ are the same.
\end{proof}

