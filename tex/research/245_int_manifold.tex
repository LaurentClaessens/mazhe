% This is part of Giulietta
% Copyright (c) 2010-2017, 2019, 2021-2022, 2024
%   Laurent Claessens
% See the file fdl-1.3.txt for copying conditions.


%+++++++++++++++++++++++++++++++++++++++++++++++++++++++
\section{Orientation and volume forme}
%+++++++++++++++++++++++++++++++++++++++++++++++++++++++

%-------------------------------------------------------
\subsection{Orientation}
%----------------------------------------------------


\begin{propositionDef}[\cite{BIBooTGSMooYDnTxb}]		\label{PROPooIDYRooSXTBtw}		\label{DEFooGMOQooVzqFxy}
	Let \( M\) be  a smooth manifold. Let two charts \(\varphi_{\alpha} \colon  U_{\alpha}\to M  \) and \(\varphi_{\beta} \colon U_{\beta}\to M  \). We say that \( (U_{\alpha}, \varphi_{\alpha})\sim (U_{\beta}, \varphi_{\beta})\) if
	\begin{equation}
		\det\big( d(\varphi_{\alpha}^{-1}\circ \varphi_{\beta})_x \big)>0
	\end{equation}
	for every \( x\in U_{\alpha}\cap U_{\beta}\).

	This is an equivalence relation\footnote{Definition \ref{DefHoJzMp}}.
\end{propositionDef}

\begin{definition}[\cite{BIBooTGSMooYDnTxb}]		\label{DEFooAQPOooJeSRAt}	\label{DEFooQZJPooOwMbim}
	An \defe{orientation}{orientation on manifold} on a smooth manifold is an atlas in which all the charts are equivalent\footnote{Equivalence in the sense of definition \ref{PROPooIDYRooSXTBtw}.}.

	If such an atlas exists, we say that the manifold \( M\) is \defe{orientable}{orientable}.
\end{definition}


\begin{propositionDef}[\cite{BIBooTGSMooYDnTxb}]		\label{PROPooMKHKooUTiUjF}
	We say that two orientations \( \mA\) and \( \mB\) are \defe{equivalent}{equivalence of orientation} if \( \mA\cup\mB\) is an orientation.

	This is an equivalence relation.
\end{propositionDef}


\begin{lemma}[\cite{MonCerveau}]		\label{LEMooYMIYooUDOhyR}
	Let \( \{ (U_{\alpha},\varphi_{\alpha}) \}_{\alpha\in \Lambda}\) be charts of the smooth manifold \( M\). For \( \alpha,\beta\in\Lambda\), we write \( M_{\alpha\beta}=\varphi_{\alpha}(U_{\alpha})\cap \varphi_{\beta}(U_{\beta})\), and we define the map
	\begin{equation}
		\begin{aligned}
			f_{\alpha\beta}\colon M_{\alpha\beta} & \to \eR                                                                                          \\
			x                                     & \mapsto \det\Big( d(\varphi_{\alpha}^{-1}\circ \varphi_{\beta})_{\varphi_{\beta}^{-1}(x)} \Big).
		\end{aligned}
	\end{equation}

	\begin{enumerate}
		\item		\label{ITEMooLHBBooFxUTHJ}
		      If \( \alpha,\beta,\sigma\in \Lambda\) and \( x\in M_{\alpha\beta}\cap M_{\alpha\sigma}\cap M_{\sigma\beta}\), we have
		      \begin{equation}
			      f_{\alpha\beta}(x)=f_{\alpha\sigma}(x)f_{\sigma\beta}(x)
		      \end{equation}
		\item		\label{ITEMooURXTooRWnYsV}
		      For every \( x\in M_{\alpha\beta}\), we have
		      \begin{equation}
			      f_{\alpha\beta}(x)\neq 0.
		      \end{equation}
	\end{enumerate}
\end{lemma}

\begin{proof}
	Several points.
	\begin{subproof}
		\spitem[For \ref{ITEMooLHBBooFxUTHJ}]
		%-----------------------------------------------------------

		Using the fact that the differential and the determinant are multiplicative (lemma \ref{LEMooGRRAooXxDMuw} and proposition \ref{PropYQNMooZjlYlA}\ref{ItemUPLNooYZMRJy}), we have
		\begin{subequations}
			\begin{align}
				f_{\alpha\beta}(x) & =\det\Big( d(\varphi_{\alpha}^{-1}\circ \varphi_{\beta})_{\varphi_{\beta}^{-1}(x)} \Big)                                                                                                                                           \\
				                   & =\det\Big(  d(\varphi_{\alpha}^{-1}\circ  \varphi_{\sigma}\circ\varphi_{\sigma}^{-1}\circ  \varphi_{\beta})_{\varphi_{\beta}^{-1}(x)}  \Big)                                                                                       \\
				                   & =\det\Big(  d(\varphi_{\alpha}^{-1}\circ  \varphi_{\sigma})_{      \varphi_{\sigma}^{-1}\circ\varphi_{\beta}\circ\varphi_{\beta}^{-1}(x)    }\circ d(\varphi_{\sigma}^{-1}\circ  \varphi_{\beta})_{\varphi_{\beta}^{-1}(x)}  \Big) \\
				                   & =\det\Big(  d(\varphi_{\alpha}^{-1}\circ  \varphi_{\sigma})_{      \varphi_{\sigma}^{-1}   }\circ d(\varphi_{\sigma}^{-1}\circ  \varphi_{\beta})_{\varphi_{\beta}^{-1}(x)}  \Big)                                                  \\
				                   & = f_{\alpha\sigma}(x)f_{\sigma\beta}(x).
			\end{align}
		\end{subequations}
		\spitem[For \ref{ITEMooURXTooRWnYsV}]
		%-----------------------------------------------------------
		The trick is to compute \( f_{\alpha\alpha}\) using the point \ref{ITEMooLHBBooFxUTHJ}. We have
		\begin{equation}
			f_{\alpha\alpha}(x)=f_{\alpha s}(x)f_{s\alpha}(x).
		\end{equation}
		On the other hand, \( f_{\alpha\alpha}(x)=\det(\id)\neq 0\). A product is non vanishing only when the two factors are non vanishing. Thus \( f_{\alpha s}(x)\neq 0\).
	\end{subproof}
\end{proof}

\begin{lemma}[\cite{MonCerveau}]		\label{LEMooXMIJooCRspfz}
	Let \( M\) be an orientable smooth manifold. Let \( \{ (U_{\alpha}, \varphi_{\alpha}) \}_{\alpha\in \Lambda}\) and \( \{ (U_s,\phi_s) \}_{s\in S}\) be orientations on \( M\). For every \( \alpha,\beta\in \Lambda\) and \( s,t\in S\) we have
	\begin{equation}
		f_{\beta t}(x)f_{\alpha s}(x)>0.
	\end{equation}
\end{lemma}

\begin{proof}
	Using both points of lemma \ref{LEMooYMIYooUDOhyR}, we have
	\begin{equation}
		f_{\beta t}(x)=f_{\beta \alpha}(x)f_{\alpha s}(x).
	\end{equation}
	Since \( \Lambda\) is an orientation, \( f_{\beta\alpha}(x)>0\) and by lemma \ref{LEMooYMIYooUDOhyR}\ref{ITEMooURXTooRWnYsV}, we deduce hat \( f_{\beta t}(x)\) and \( f_{\alpha s}(x)\) have the same sign.
\end{proof}


\begin{normaltext}		\label{NORMooPXHIooHTFJoa}
	When we have two charts \( (U_{\alpha},\varphi_{\alpha})\) and \( (U_{\beta},\varphi_{\beta})\), we can write the corresponding «basis» vector fields
	\begin{equation}
		(\partial_{\alpha, i})(x)=\frac{d}{dt} \left[ \varphi_{\alpha}\big( \varphi_{\alpha}^{-1}(x)+te_i \big)  \right]_{t=0},
	\end{equation}
	and
	\begin{equation}
		(\partial_{\beta, i})(x)=\frac{d}{dt} \left[ \varphi_{\beta}\big( \varphi_{\beta}^{-1}(x)+te_i \big)  \right]_{t=0}.
	\end{equation}
\end{normaltext}

%-------------------------------------------------------
\subsection{Connected atlas}
%----------------------------------------------------

\begin{definition}[\cite{MonCerveau}]		\label{DEFooIQUXooOdVpLS}
	Let \( M\) be a smooth manifold with an atlas \( \{ (U_{\alpha}, \varphi_{\alpha}) \}_{\alpha\in \Lambda}\). For \( \alpha\in \Lambda\) we write \( M_{\alpha}=\varphi_{\alpha}(U_{\alpha})\). Let \( x\in M\). We write
	\begin{equation}
		S_x=\{ y\in M \tq \exists \alpha_1,\ldots,\alpha_n\in\Lambda\tq
		\begin{cases}
			x\in M_{\alpha_1} \\
			y\in M_{\alpha_n} \\
			M_{\alpha_i}\cap M_{\alpha_{i+1}}\neq \emptyset.
		\end{cases}
		\}
	\end{equation}
\end{definition}

\begin{lemma}[\cite{MonCerveau}]		\label{LEMooIFABooVGFYfI}
	Let \( M\) be a smooth manifold and \( x,y\in M\).
	\begin{enumerate}
		\item	\label{ITEMooCMDVooBobapJ}
		      If \( y\in S_x\), then \( x\in S_y\).
		\item		\label{ITEMooIIBSooNNXIFj}
		      If \( y\in S_x\), then \( S_y=S_x\).
		\item		\label{ITEMooNGSEooIZuiMU}
		      If \( S_x\cap S_y\neq\emptyset\) then \( S_x=S_y\).
	\end{enumerate}
	The relation \( x\sim y\) if \( x\in S_y\) is an equivalence relation.
\end{lemma}

\begin{proof}
	Two parts.
	\begin{subproof}
		\spitem[For \ref{ITEMooCMDVooBobapJ}]
		%-----------------------------------------------------------
		Since \( y\in S_x\), we have a chain \( (\alpha_1,\ldots,\alpha_n)\) such that \( x\in M_{\alpha_1}\) and \( y\in S_{\alpha_n}\) with \( M_{\alpha_i}\cap S_{\alpha_{i+1}}\neq \emptyset\). The chain \( (\alpha_n,\ldots,\alpha_1)\) shows that \( x\in S_y\).
		\spitem[For \ref{ITEMooIIBSooNNXIFj}]
		%-----------------------------------------------------------
		Let \( y\in S_x\).
		\begin{subproof}
			\spitem[\( S_y\subset S_x\)]
			%-----------------------------------------------------------
			Let \( a\in S_y\). We have a chain \( \alpha_1,\ldots,\alpha_n\) with \( y\in M_{\alpha_1}\) and \( a\in M_{\alpha_n}\). Since \( y\in S_x\) we also have a chain \( \beta_1,\ldots,\beta_m\) such that \( x\in M_{\beta_1}\) and \( y\in M_{\beta_m}\). Since \( M_{\beta_m}\cap M_{\alpha_1}\neq \emptyset\) we have the chain \( (\beta_1,\ldots,\beta_m,\alpha_1,\ldots,\alpha_n)\) such that \( x\in M_{\beta_1}\) and \( a\in M_{\alpha_n}\). Thus \( a\in S_x\).
			\spitem[\( S_x\subset S_y\)]
			%-----------------------------------------------------------
			We know that \( y\in S_x\). By \ref{ITEMooCMDVooBobapJ} we also have \( x\in S_y\). The inclusion we just proved reads now \( S_x\subset S_y\).
		\end{subproof}
		\spitem[For \ref{ITEMooNGSEooIZuiMU}]
		%-----------------------------------------------------------
		Let \( a\in S_x\cap S_y\). Since \( a\in S_x\) we have \( S_a=S_y\), and from \( a\in S_y\) we deduce \( S_a=S_y\).
	\end{subproof}

	The fact that \( x\sim y\) is an equivalence relation is now an immediate check.
\end{proof}

\begin{proposition}[\cite{MonCerveau}]		\label{PROPooFTFCooYTKaPF}
	Let \( M\) be a connected smooth manifold. For every \( a\in M\) we have \( M=S_a\).
\end{proposition}

\begin{proof}
	The set \( \{ S_x \}_{x\in M}\) is a covering of \( M\) by open parts. Let \( A=S_a\) and \( B=\bigcup_{\substack{ x\in M \\ a\not\in S_x }  }S_x \).
	\begin{subproof}
		\spitem[\( M=A\cup B\)]
		%-----------------------------------------------------------
		We prove that \( M=A\cup B\). Let \( x\in M\). There are two possibilities: \( x\in S_a\) or not. If \( x\in S_a\), then \( x\in A\). Suppose that \( x\not\in S_a\). Then \( a\not\in S_x\) (lemma \ref{LEMooIFABooVGFYfI}\ref{ITEMooCMDVooBobapJ}) and then \( x\in B\).
		\spitem[\( B=\emptyset\)]
		%-----------------------------------------------------------
		Suppose that \( A\) and \( B\) are both non empty. From connectedness of \( M\), we have \( A\cap B\neq\emptyset\). Let \( y\in A\cap B\). We have \( y\in S_a\) and there exists \( x\in M\) such that \( a\neq S_x\) and \( y\in S_x\). We have
		\begin{subequations}
			\begin{align}
				y\in S_a & \Rightarrow a\in S_y  \\
				y\in S_x & \Rightarrow x\in S_y.
			\end{align}
		\end{subequations}
		Using lemma \ref{LEMooIFABooVGFYfI}\ref{ITEMooIIBSooNNXIFj} we deduce \( S_a=S_x=S_y\). In particular \( a\in S_x\) which is a contradiction.

		Thus \( A\) and \( B\) are not both non empty. The part \( A\) contains \( a\) and is then non empty. We conclude that \( B\) is empty.

		\spitem[Conclusion]
		%-----------------------------------------------------------
		We have \( M=A\cup B=A=S_a\).
	\end{subproof}
\end{proof}

\begin{lemma}[\cite{MonCerveau}]		\label{LEMooSDQUooWEagbY}
	Let \( \mA= \{ (U_{\alpha},\varphi_{\alpha}) \}_{\alpha\in \Lambda}\) be an atlas. For each \( \alpha\in \Lambda\) we consider the set \( \{ U_{\alpha,i} \}_{i\in I_{\alpha}}\) of its connected components.

	Then the set
	\begin{equation}
		\mA_c=\{ (U_{\alpha,i},\varphi_{\alpha,i}) \}_{\alpha\in\Lambda, i\in I_{\alpha}}
	\end{equation}
	where \(\varphi_{\alpha,i} \colon U_{\alpha,i}\to M  \) is the restriction of \( \varphi_{\alpha}\) is an atlas.
\end{lemma}

\begin{proposition}[\cite{MonCerveau}]		\label{PROPooHQSSooDCjlIX}
	Let \( \mA\) be an orientation. Then the atlas \( \mA_c\) is an orientation and is the same orientation as \( \mA\).
\end{proposition}

\begin{proof}
	We consider two charts \( \varphi_{\alpha,i}\) and \( \varphi_{\beta,j}\) of \( \mA_c\) (\( i\in I_{\alpha}\) and \( j\in I_{\beta}\)). Let \( x\in M_{\alpha,i}\cap M_{\beta,j}\). Since the connected components are open, \( x\) has a neighbourhood in \( M_{\alpha,i}\cap M_{\beta,j}\) and we have
	\begin{equation}
		d(\varphi_{\alpha,i}^{-1}\circ\varphi_{\beta,j})_x=d(\varphi_{\alpha}^{-1}\circ \varphi_{\beta}).
	\end{equation}
\end{proof}

%-------------------------------------------------------
\subsection{Volume form}
%----------------------------------------------------

\begin{proposition}[\cite{MonCerveau}]		\label{PROPooBIVHooXOycnS}
	Let \( M\) be a smooth manifold. There exists an orientation \( \{ (U_{\alpha},\varphi_{\alpha})_{\alpha\in \Lambda} \}\) and a partition of unity \( \{ \phi_i \}_{i\in I}\) such that
	\begin{enumerate}
		\item
		      The orientation \( \{ (U_{\alpha},\varphi_{\alpha}) \}_{\alpha\in \Lambda}\) is countable and locally finite.
		\item
		      The partition of unity \( \{ \phi \}_{i\in I}\)  is countable and subordinate to \( \mW=\{ \varphi_{\alpha}(U_{\alpha}) \}_{\alpha\in \Lambda}\).
		\item
		      The supports \( \supp(\phi_i)\) are compacts.
	\end{enumerate}
\end{proposition}

\begin{definition}[\cite{BIBooTGSMooYDnTxb}]			\label{DEFooOBZEooMZauZF}
	A \defe{volume form}{volume form} on a smooth \(n \)-dimensional manifold is a smooth nowhere vanishing \( n\)-differential form.
\end{definition}

\begin{lemma}[\cite{MonCerveau}]		\label{LEMooQGVMooSHXUmD}
	Let \( \omega\) be a \( n\)-differential form on \( M\). Let \(\varphi_{\alpha} \colon U_{\alpha} \to M  \) be a chart. There exists a smooth function \(f \colon M_{\alpha}\to \eR   \) such that
	\begin{equation}
		\omega_x=f(x)\partial^*_1\wedge\ldots\wedge\partial^*_n
	\end{equation}
	for every \( x\in M_{\alpha}\).
\end{lemma}

This lemma is a combination of proposition \ref{PROPooUGLOooTULnDK} for the existence of \( f\) and definition \ref{DEFooZELVooFfosEn} for the smoothness.


\begin{lemma}[\cite{MonCerveau}]	\label{LEMooKJRCooCklXie}
	If \( \omega\) is a volume form, then
	\begin{equation}
		\omega_x(\partial_{\alpha, 1},\ldots,\partial_{\alpha,n})\neq 0
	\end{equation}
	for every \( \alpha\in \Lambda\) and \( x\in M_{\alpha}\).
\end{lemma}

\begin{proof}
	From lemma \ref{LEMooQGVMooSHXUmD}, we have \( a\in \eR\) such that \( \omega_x=a\partial^*_{\alpha,1}\wedge\ldots\wedge\partial^*_{\alpha,n}\). Notice that \( a\neq 0\) because \( \omega_x\neq 0\). Proposition \ref{PROPooRRSZooJXOApq} shows that
	\begin{equation}		\label{EQooRRHNooBVgAnF}
		(\partial^*_{\alpha,1}\wedge\ldots\wedge\partial^*_{\alpha,n})(\partial_{\alpha,1},\ldots,\partial_{\alpha,n})=\det\big( \partial_{\alpha,i}^*(\partial_{\alpha,j}) \big)=\det(\delta_{ij})=1.
	\end{equation}
	Thus \( \omega_x(\partial_{\alpha,1},\ldots,\partial_{\alpha,n})=a\neq 0\).
\end{proof}

\begin{lemmaDef}		\label{LEMooELOHooYtXSEH}
	Let \( \omega\) be a volume form on the connected manifold \( M\). Let \( \mA=\{ (U_{\alpha},\varphi_{\alpha})\}_{\alpha\in\Lambda}\) be an orientation on \( M\). If there exists \( x\in M\) and \( \alpha\in \Lambda\) such that
	\begin{equation}
		\omega_x(\partial_{\alpha,1},\ldots,\partial_{\alpha,n})>0,
	\end{equation}
	then
	\begin{equation}
		\omega_y(\partial_{\beta,1},\ldots,\partial_{\beta,n})>0
	\end{equation}
	for every \( \beta\in\Lambda \) and \( y\in M_{\beta}=\varphi_{\beta}(U_{\beta})\).

	In that case we say that \( \omega\) is \defe{oriented}{orientation of a volume form} \( \mA\)-positively.
\end{lemmaDef}

\begin{proof}

	Two parts. In the first part we show that, keeping the same point, we can change the chart, and in the second part we will show how we can change the point.

	\begin{subproof}
		\spitem[Changing chart]
		%-----------------------------------------------------------


		Let \( x\in M_{\alpha}\cap M_{\beta}\). From lemma \ref{LEMooQGVMooSHXUmD}, there exists \(a \eR  \) such that
		\begin{equation}
			\omega_x=a\partial^*_{\beta,1}\wedge\ldots\wedge\partial^*_{\beta,n}.
		\end{equation}
		Using lemma \ref{LEMooSZTOooBIzMCc},
		\begin{equation}
			\omega_x(\partial_{\alpha,1},\ldots,\partial_{\alpha,n})=a(\partial^*_{\beta,1}\wedge\ldots\wedge\partial^*_{\beta,n})(\partial_{\alpha,1},\ldots,\partial_{\alpha,n})=a\det\big( d(\varphi_{\beta}^{-1}\circ\varphi_{\alpha})_x \big).
		\end{equation}
		Since \( \mA\) is an orientation, the determinant is strictly positive. By hypothesis, the left hand side is strictly positive. We deduce that \( a>0\). Using the same computation as in \eqref{EQooRRHNooBVgAnF}, we find
		\begin{equation}
			\omega_x(\partial_{\beta,1},\ldots,\partial_{\beta,n})=a\det(\id)=a>0.
		\end{equation}

		\spitem[Changing the point]
		%-----------------------------------------------------------
		Let \( x,y\in M\) We consider the connected orientation \( \mA_c=\{ (U_i,\varphi_i) \}_{i\in I}\) associated with \( \mA\) (proposition \ref{PROPooHQSSooDCjlIX}). Since \( M\) is connected, \( S_x=M\) (proposition \ref{PROPooFTFCooYTKaPF}) and there exists a chain \( i_0,\ldots,i_n\in I\)  and points \( a_k\in \varphi_{i_k}(U_k)\cap\varphi_{i_{k+1}}(U_{k+1})\) such that \( x\in M_{i_0}\) and \( y\in M_{i_n}\).


		The map
		\begin{equation}
			\begin{aligned}
				f\colon \varphi_{i_0}(U_{i_0}) & \to \eR                                                    \\
				s                              & \mapsto \omega_s(\partial_{i_0,1},\ldots,\partial_{i_0,n})
			\end{aligned}
		\end{equation}
		satisfy:
		\begin{enumerate}
			\item
			      It is continuous
			\item
			      It's domain is connected,
			\item
			      It does not vanish (lemma \ref{LEMooKJRCooCklXie}).
			\item
			      Let \( \beta\in\Lambda\) such that \( U_{i_0}\) is one of the connected components of \( U_{\beta}\). We have
			      \begin{equation}
				      f(x)=\omega_x(\partial_{i_0,1},\ldots,\partial_{i_0,n})=\omega_x(\partial_{\beta,1},\ldots,\partial_{\beta,n})>0.
			      \end{equation}
			      The last inequality is the first part of this lemma "Changing charts".
		\end{enumerate}
		Thus \( f(s)>0\) for every \( s\in \varphi_{i_0}(U_{i_0})\) (intermediate value theorem \ref{PROPooGURQooAwKNUJ}). In particular \( f(a_0)>0\). Now we change of chat at \( a_0\) to use the chart \( (U_{i_1},\varphi_{i_1})\) and we continue\footnote{The reader is assumed to be able to write a correct recursion.}.
	\end{subproof}
\end{proof}


\begin{theorem}[\cite{BIBooJMRFooTAhhcg}]	\label{THOooQEFUooQTtPDD}
	A smooth manifold is orientable if and only if it accepts a volume form\footnote{Volume form, definition \ref{DEFooOBZEooMZauZF}.}.
\end{theorem}

\begin{proof}
	Two parts.
	\begin{subproof}
		\spitem[\( \Leftarrow\)]
		%-----------------------------------------------------------
		Let \( \omega\in\Omega^n(M)\) be such that \( \omega_x\neq 0\) for every \( x\in M\). Let \( \{ (U_{\alpha},\varphi_{\alpha}) \}_{\alpha\in \Lambda}\) be an atlas of \( M\). We will build an atlas \( \{ (V_x,\phi_x) \}_{x\in M}\) of \( M\) indexed by the elements of \( M\).

		Let \( x\in M\). There exists \( \alpha\in \Lambda\) such that \( x\in \varphi_{\alpha}(U_{\alpha})\). With that chart we have (lemma \ref{LEMooQGVMooSHXUmD})
		\begin{equation}
			\omega_y=f(y)\partial^*_{\alpha, 1}\wedge\ldots \wedge\partial^*_{\alpha,n}
		\end{equation}
		for every \( y\in M_{\alpha}\). Since \( \omega\) is non vanishing, \( f(x)\neq 0\). If \( f(x)>0\), we consider \( (V_x,\phi_x)=(U_{\alpha}, \varphi_{\alpha})\). If \( f(x)<0\) we have to twist a little bit. We consider the linear map \(\tau \colon \eR^n\to \eR^n  \) which permutes \( e_1\) and \( e_2\), that is:
		\begin{equation}
			\begin{aligned}
				\tau\colon \eR^n & \to \eR^n                        \\
				e_i              & \mapsto \begin{cases}
					                           e_2 & \text{if } i=1     \\
					                           e_1 & \text{if }i=2      \\
					                           e_i & \text{otherwise. }
				                           \end{cases}
			\end{aligned}
		\end{equation}
		Notice that \( \tau^{-1}=\tau\). We set \( V_x=\tau(U_{\alpha})\) and \( \phi_x=\varphi_{\alpha}\circ\tau\). Following the notations of \ref{NORMooPXHIooHTFJoa}, we have
		\begin{subequations}
			\begin{align}
				(\partial_{x,1})(y) & =\frac{d}{dt} \left[ \phi_x\big( \phi_x^{-1}(y)+te_1 \big)  \right]_{t=0}                                            \\
				                    & =\frac{d}{dt} \left[ (\varphi_{\alpha}\circ \tau)\big( (\tau\circ\varphi_{\alpha}^{-1})(y)+te_1 \big)  \right]_{t=0} \\
				                    & =\frac{d}{dt} \left[ \varphi_{\alpha}\big( \varphi_{\alpha}^{-1}(y)+te_2 \big)  \right]_{t=0}                        \\
				                    & =(\partial_{\alpha}, 2)(y).
			\end{align}
		\end{subequations}
		In the same way we have \( \partial_{x,1}=\partial_{\alpha,1}\) and \( \partial_{x,i}=\partial_{\alpha,i}\) for \( i>2\). Thus we have
		\begin{equation}
			\omega_x=f(x)\partial^*_{\alpha,1}\wedge\partial^*_{\alpha,2}\wedge\ldots \wedge\partial^*_{\alpha,n}=f(x)\partial^*_{x,2}\wedge\partial^*_{x,1}\wedge\ldots\wedge\partial^*_{x,n}=-f(x)\partial^*_{x,1}\wedge\ldots\wedge\partial^*_{x,n}.
		\end{equation}
		Thus using the atlas \( \{ (V_x,\phi_x) \}_{x\in M}\), for each \( y\in M\), we have a chart \( (V_y,\phi_y)\) and a smooth map \(g_y \colon M_y\to \eR  \) such that
		\begin{equation}
			\omega_x=g_y(x)\partial^*_{x,1}\wedge\ldots\wedge\partial^*_{x,n}
		\end{equation}
		with \( g_y(x)>0\) for every \( x\in M_y\).

		Now we prove that \( \{ (V_x,\phi_x) \}\) is an orientation of \( M\). Let \( z\in M_{xy}=\phi_(V_x)\cap \phi_y(V_y)\). We have
		\begin{equation}
			\omega_z=g_x(z)\partial_{x,1}^*\wedge\ldots\partial^*_{x,n}=g_y(z)\partial_{y,1}^*\wedge\ldots\wedge\partial_{y,n}^*.
		\end{equation}
		Applying to \( (\partial_{x,1},\ldots,\partial_{x,n})\) we have
		\begin{equation}
			\omega_z(\partial_{x,1},\ldots,\partial_{x,n})=g_x(z)=g_y(z)(\partial_{y,1}^*\wedge\ldots\wedge\partial_{y,n}^*)(\partial_{x,1},\ldots,\partial_{x,n}).
		\end{equation}
		Using proposition \ref{PROPooRRSZooJXOApq} and lemma \ref{LEMooSZTOooBIzMCc} on the right hand side, we get
		\begin{equation}
			g_x(z)=g_y(z)\det\Big( \partial_{y,i}^*(\partial_{x,j}) \Big)=g_y(z)\det\big( d(\varphi_y^{-1}\circ\varphi_x)_z \big).
		\end{equation}
		The charts \( \varphi_x\) and \( \varphi_y\) were chosen in such a way that \( g_x(z)>0\) and \( g_y(z)>0\). Thus
		\begin{equation}
			\det\big( d(\varphi_y^{-1}\circ\varphi_x)_z \big)>0.
		\end{equation}
		Thus the maps \( \{ \varphi_x \}_{x\in M}\) are an atlas and an orientation.
		\spitem[\( \Rightarrow\)]
		%-----------------------------------------------------------
		Let \( \mA=\{ (U_{\alpha}, \varphi_{\alpha}) \}_{\alpha\in \Lambda}\) be an orientation on \( M\). For each \( \alpha\in \Lambda\) we consider the differential form
		\begin{equation}
			\omega_{\alpha}=\partial_{\alpha,1}^*\wedge\ldots \wedge\partial_{\alpha,n}^*.
		\end{equation}
		We consider a partition of unity as in proposition \ref{PROPooOVQHooMNRQDH}, and we set \( \omega=\sum_{\alpha\in \Lambda}\rho_{\alpha}\omega_{\alpha}\). We show that this is nowhere vanishing. Let \( m\in M\). Since \( \{ \supp(\rho_{\alpha}) \}_{\alpha\in \Lambda}\) is locally finite, the set \( \Lambda'=\{ \alpha\in \Lambda\tq \rho_{\alpha}(m)\neq 0 \}\) is finite. We have
		\begin{equation}
			\omega(m)=\sum_{\alpha\in\Lambda'}\rho_{\alpha}(m)\omega_{\alpha}(m).
		\end{equation}
		Let \( \beta\in \Lambda'\). We have
		\begin{subequations}
			\begin{align}
				\omega_m(\partial_{\beta,1},\ldots,\partial_{\beta,n}) & = \sum_{\alpha\in\Lambda'}\rho_{\alpha}(m) \omega_{\alpha}(\partial_{\beta,1},\ldots,\partial_{\beta,n}) \\
				                                                       & =\sum_{\alpha\in\Lambda'}\rho{\alpha}(m)\det\big( d(\varphi_{\alpha}^{-1}\circ\varphi_{\beta})_m \big).
			\end{align}
		\end{subequations}
		Since \( \{ \varphi_{\alpha} \}_{\alpha\in\Lambda}\) is an orientation, we have \( \det\big( d(\varphi_{\alpha}^{-1}\circ\varphi_{\beta})_m \big)>0\) for every \( \alpha\in \Lambda'\), and since we have only a finite number of them, we consider the minimum:
		\begin{equation}
			s=\min_{\alpha\in \Lambda'}\det\big( d(\varphi^{-1}_{\alpha}\circ\varphi_{\beta})_m \big).
		\end{equation}
		Now we have
		\begin{subequations}
			\begin{align}
				\sum_{\alpha\in\Lambda}\rho_{\alpha}(m)\omega_{\alpha}(m)(\partial_{\beta,1},\ldots,\partial_{\beta,n}) & = \sum_{\alpha\in\Lambda'}\rho_{\alpha}(m)\det\big( d(\varphi_{\alpha}^{-1}\circ\varphi_{\beta})_m \big) \\
				                                                                                                        & \geq s\sum_{\alpha\in \Lambda'}\rho_{\alpha}(m)                                                          \\
				                                                                                                        & =s\sum_{\alpha\in\Lambda}\rho_{\alpha}(m)                                                                \\
				                                                                                                        & =s >0.
			\end{align}
		\end{subequations}
	\end{subproof}
\end{proof}

\begin{lemma}[\cite{MonCerveau}]	\label{LEMooHLYAooAeEnZb}
	Let \( \mA\) and \( \mA'\) be orientations on a connected manifold. Let \( \omega\) be a \( \mA\)-oriented volume form. The volume form \( \omega\) is \( \mA'\)-oriented if and only if \( \mA\) and \( \mA'\)  are equivalent orientations.
\end{lemma}

\begin{proof}
	Let \( (\varphi_{\alpha}, U_{\alpha})\in\mA\) and \( (U_{\beta},\varphi_{\beta})\in\mA'\). We can write \( \omega_x=f(x)\partial^*_{\alpha,1}\wedge\ldots\partial_{\alpha,n}^*\) with \( f(x)>0\), and
	\begin{subequations}
		\begin{align}
			\omega_x(\partial_{\beta,1},\ldots,\partial_{\beta,n}) & =f(x)(\partial^*_{\alpha,1}\wedge\ldots\partial_{\alpha,n})(\partial_{\beta,1},\ldots,\partial_{\beta,n})                                       \\
			                                                       & =f(x)\det\big( d(\varphi_{\alpha}^{-1}\circ \varphi_{\beta})_x \big).                                     & \text{lem. \ref{LEMooSZTOooBIzMCc}}
		\end{align}
	\end{subequations}
	Thus we have equivalence between :
	\begin{enumerate}
		\item
		      \( \omega\) is \( \mA'\)-oriented,
		\item
		      \( \omega_x(\partial_{\beta,1},\ldots,\partial_{\beta,n})>0\)
		\item
		      $\det\big( d(\varphi_{\alpha}^{-1}\circ\varphi_{\beta}) \big)>0$,
		\item
		      the orientations \( \mA\) and \( \mA'\) are equivalent.
	\end{enumerate}
\end{proof}

\begin{proposition}[\cite{BIBooTGSMooYDnTxb,BIBooXNMMooNOLQEL,MonCerveau}]		\label{PROPooNCNJooHFngBW}
	If a connected manifold accepts one orientation, then it accepts exactly two classes of orientation.
\end{proposition}

\begin{proof}
	Let \( \mA\) and \( \mA'\) be two different orientations. Let \( \omega\) be a \( \mA\)-oriented volume form, and \( \mB\) a third orientation. Since \( \omega\) is \( \mA\)-oriented and since \( \mA'\) is not equivalent to \( \mA\), the volume form \( \omega\) is \( \mA'\)-negative.


	If \( \omega\) is \( \mB\)-oriented, then \( \mB\sim\mA\) (lemma \ref{LEMooHLYAooAeEnZb}). If \( \omega\) is \( \mB\)-negative, then \( \mB\) is equivalent to \( \mA'\).

	We have shown that the orientation \( \mB\) is equivalent to \( \mA\) or \( \mA'\).
\end{proof}

%+++++++++++++++++++++++++++++++++++++++++++++++++++++++
\section{Manifold with boundary}
%+++++++++++++++++++++++++++++++++++++++++++++++++++++++


%-------------------------------------------------------
\subsection{Some facts about \( \eR^n_+\)}
%----------------------------------------------------

We already proved that \( \eR^n\) is \( C^{\infty}\)-diffeomorphic to \( B(0,1)\).

We write
\begin{equation}
	\eR_+^n=\{ (x_1,\ldots,x_n)\in \eR^n\tq x_n\geq 0 \}.
\end{equation}
We consider the topology induced from \( \eR^n\), and the differential structure of definition \ref{DEFooLXVNooIpoIBp}.

\begin{lemma}[\cite{MonCerveau}]	\label{LEMooBICBooFUQyYA}
	The boundary\footnote{Definition \ref{DEFooACVLooRwehTl}.} of \( \eR^n_+\) is
	\begin{equation}
		\partial \eR^n_+=\{ x\in \eR^n_+\tq x_n=0 \}.
	\end{equation}
\end{lemma}


\begin{proposition}[\cite{BIBooGOXFooBAddeQ,BIBooUCJHooPpuntW, MonCerveau}]	\label{PROPooPUREooSkeyxs}
	There are no diffeomorphism\footnote{Definition \ref{DefAQIQooYqZdya}. Bijection, differentiable, invertible with differentiable inverse in the sens of definition \ref{DEFooLXVNooIpoIBp}.} between \( \eR^n_+\) and \( \eR^n\).
\end{proposition}

\begin{proof}
	We suppose that such a diffeomorphism exists. Let \(f \colon \eR^n_+\to \eR^n  \) be a diffeomorphism. Let \( a\in \eR^n_+\) be such that \( a_n=0\). There exists open parts \( U\) of \( \eR^n_+\) and \( V\) of \( \eR^n\) and a diffeomorphism \(\hat f \colon U\to V  \) such that \( a\in U\). We use the mean value theorem \ref{LEMooYQZZooVybqjK} on \( \hat f\) :
	\begin{equation}	\label{EQooTKGQooQmlOhv}
		\hat f(a+h)=\hat f(a)+d\hat f_a(h)+\alpha(h).
	\end{equation}

	Since \( U\cap \eR^n_+\) is open in \( \eR^n_++\) and since the reciprocal of \( f\) is continuous, the part \( f(U\cap \eR^n_+)\) is open in \( \eR^n\) (and contains \( f(a)\)). Let \( r>0\) be such that
	\begin{equation}
		B\big( \hat f(a), r \big)=   B\big( f(a),r \big)\subset f\big( U\cap \eR^n_+ \big).
	\end{equation}
	From equation \eqref{EQooTKGQooQmlOhv} we get
	\begin{equation}
		\| \hat f(a+\epsilon e_n)-\hat f(a) \|\leq \epsilon\| df_a \|+\| \alpha(\epsilon e_n) \|.
	\end{equation}
	If \( \epsilon\) is small enough we have
	\begin{equation}
		\hat f(a+\epsilon e_n)\in B\big( \hat f(a),r \big)\subset f\big( U\cap \eR^n_+ \big).
	\end{equation}
	Let \( \epsilon<0\) small enough. There exists \( u\in U\cap \eR^n_+\) such that \( f(u)=\hat f(a+\epsilon e_n)\). But on \( U\cap \eR^n_+\) we have \( f=\hat f\), so that
	\begin{equation}		\label{EQooSZLJooBjICmu}
		\hat f(u)=\hat f(a+\epsilon e_n)
	\end{equation}
	Notice that \( u\neq a+\epsilon e_n\) because \( u_n\geq 0\) while \( (a+\epsilon e_n)_n=\epsilon<0\). Equation \eqref{EQooSZLJooBjICmu} contradicts that \( \hat f\) is bijective. Contradiction. Finished.
\end{proof}

\begin{lemma}[\cite{MonCerveau}]	\label{LEMooEVSVooKJrgKz}
	If \( A\) is open in \( \eR^n_+\) contains an element of \( \partial\eR^n_+\), this is not an open part of \( \eR^n\).
\end{lemma}

\begin{proof}
	Let \( a\in A\) be such that \( a_n=0\). We suppose that \( A\) is open in \( \eR^n\). Then there exists \( r>0\) such that \( B(a,r)\subset A\). The element \( a-re_n\) belongs to \( B(a,r)\) but \( (a-re_n)_n=a_n-r<0\). Thus \( B(a,r)\) contains an element outside \( \eR^n_+\). Contradiction.
\end{proof}

\begin{lemma}[\cite{MonCerveau}]	\label{LEMooWZOBooGrvNGQ}
	Let \( X\) be a topological space, \( A\) be open in \( \eR^n_+\), and \( a\in A\) satisfying \( a_n>0\). We consider an homeomorphism \(f \colon A\to X  \). Then there exists a neighbourhood \( U\) of \( a\) in \( \eR^n\) such that \( U\subset A\) and \(f \colon U\to X  \) is homeomorphic.
\end{lemma}

\begin{proof}
	Several subsets.
	\begin{enumerate}
		\item
		      Since \( A\) is open in \( \eR^n_+\), there exists an open \( U_1\) of \( \eR^n\) such that \( A=U\cap \eR^n_+\). There exists \( r_1>0\) such that \( B(a,r_1)\subset U_1\).
		\item

		      Since the projection \( x\mapsto x_n\) is continuous on \( \eR^n\), there exists an open neighbourhood \( U_2\) of \( a\) in \( \eR^n\) such that \( y_n>0\) for every \( y\in U_2\). Let \( r_2>0\) be such that \( B(a,r_2)\subset U_2\).
		\item
		      The definition of continuity on a closed part (definition \ref{DEFooLXVNooIpoIBp}) says that there exists an open \( U_3 \) of \( \eR^n\) containing \( A\) and an homeomorphic extension \(\tilde f \colon U_3\to X  \).
	\end{enumerate}
	Taking \( r=\min\{r_1,r_2\}\) we have
	\begin{equation}
		B(a,r)\subset U_1\cap\eR^n_+=A\subset U_3.
	\end{equation}
	The extension \( \tilde f\) is thus defined on \( B(a,r)\). And, since \( B(a,r)\subset U_2\), we have \( \tilde f(x)=f(x)\) for every \( x\in B(a,r)\). Thus the map \( f \colon B(a,r)\to X  \) is an homeomorphism.
\end{proof}

\begin{lemma}[\cite{BIBooSMBNooTmuXpu}]	\label{LEMooLPXFooUBLCqA}
	Let \( U\) and \( V\) be open parts of \( \eR^n_+\). Let \(f \colon U\to V  \) be a \( C^1\)-diffeomorphism\footnote{Definition \ref{DEFooLXVNooIpoIBp}.}. Then \( f(x)_n=0\) if and only if \( x_n=0\).
\end{lemma}

\begin{proof}
	Since \( U\) and \( V\) are open in \( \eR^n_+\), there are open parts \( A\) and \( B\) of \( \eR^n\) such that \( U=A\cap \eR^n_+\) and \( V=B\cap \eR^n_+\). We also consider the smooth map
	\begin{equation}
		\begin{aligned}
			\sigma\colon \eR^n & \to \eR      \\
			x                  & \mapsto x_n.
		\end{aligned}
	\end{equation}
	Now we prove the two  parts.
	\begin{subproof}
		\spitem[Suppose \( a_n>0\)]
		%-----------------------------------------------------------
		Since \( \sigma(a)>0\), there exists an open \( B'\) of \( \eR^n\) such that \( \sigma>0\) on \( B\). Taking \( B=B'\cap A\), we have
		\begin{equation}
			a\in B\subset A
		\end{equation}
		where \( B\) is open of \( \eR^n\) and \( \sigma|_B>0\). What about \( f(B)\) ? We use the smooth invariance of the domain theorem \ref{THOooGUELooSBQhRu} with the manifolds \( M=R^n\) and \( N=\eR^n\). Since \( B\) is open and \(f \colon B\to f(B)  \) is a \( C^1\)-diffeomorphism, the part \( f(B)\) is open in \( \eR^n\).

		We show that \( f(y)_n>0\) for every \( y\in f(B)\). Suppose that \( y\in f(B)\) satisfy \( y_n=0\). Since \( f(B)\) is an open of \( \eR^n\), we have an open \( S\) of \( \eR^n\) satisfying \( y\in S\subset f(B)\). Since \( y_n=0\), there are points \( z\) of \( S\) such that \( z_n<0\). This is a contradiction because \( z\in S\subset f(B)\subset V\subset \eR^n_+\).

		\spitem[Suppose \( f(a)_n>0\)]
		%-----------------------------------------------------------
		Since \( f^{-1}\) is a \( C^1\)-diffeomorphism as well as \( f\) itself, we use the previous point to say that \( f^{-1}\big( f(a) \big)_n>0\). Thus we have \( a_n>0\).
	\end{subproof}
\end{proof}

%-------------------------------------------------------
\subsection{Manifold with boundary}
%----------------------------------------------------

\begin{definition}[Topological manifold with boundary\cite{BIBooZBZTooYmHemH}]	\label{DEFooNETXooZaZMcm}
	A topological space \( M\) is a \defe{topological manifold with boundary}{topological manifold!with boundary} if it is Hausdorff and every element of \( M\) has a neighbourhood homeomorphic either to \( \eR^n\) or \( \eR^{n}_+\).

	Let \( M\) be a manifold with boundary. The \defe{boundary}{boundary} of \( M\) is set of points of \( M\) which have no neighbourhood homeomorphic to \( \eR^n\). And we denote by \( \partial M\) the boundary of \( M\).
\end{definition}

\begin{normaltext}
	The definition of \( \partial M\) is not a particular case of the general definition \ref{DEFooACVLooRwehTl} of a boundary because every point of a topological space is in the interior of the space itself.

	The reason of the definition is that the points of \( \partial M\) are the ones who are images of points in \( \eR^n_+\) that are on \( \partial \eR^n_+\) as we will see in proposition \ref{LEMooLPXFooUBLCqA}.
\end{normaltext}


\begin{theorem}[Smooth manifold with boundary\cite{BIBooZBZTooYmHemH}]	\label{THOooKKBLooIrITBe}
	Let \( M\) be a topological manifold with boundary. A \defe{chart}{chart with boundary} is a map \(\varphi \colon U\to M  \) where \( U\) is either homeomorphic to \( \eR^n\) or \( \eR^n_+\)\footnote{In fact \( U\) could be only homeomorphic to \( \eR^n_+\), but it would complicate everything. In particular the characterisation of \( \partial M\) as the points with coordinates \( x_n=0\).}

	A set \( \mA=\{ (U_{\alpha},\varphi_{\alpha}) \}_{\alpha\in \Lambda}\) is an atlas if
	\begin{enumerate}
		\item
		      For every \( m\in M\), there exists \( \alpha\in \Lambda\) such that \( m\in \varphi_{\alpha}(U_{\alpha})\),
		\item
		      The maps
		      \begin{equation}
			      \varphi_{\alpha}^{-1}\circ\varphi_{\beta} \colon \varphi_{\beta}^{-1}(M_{\alpha}\cap M_{\beta})\to U_{\alpha}
		      \end{equation}
	\end{enumerate}
	are homeomorphic in the sense of definition \ref{DEFooLXVNooIpoIBp}.

	If these maps are smooth, the manifold \( M\) is a \defe{smooth manifold with boundary}{smooth manifold with boundary}.
\end{theorem}


\begin{proposition}[\cite{MonCerveau}]	\label{PROPooWQHTooXhkWdZ}
	Let \( M\) be a smooth manifold with boundary and let \( (U_{\alpha}, \varphi_{\alpha})\) be a chart with \( U_{\alpha}=\eR^n_+\). Then \( m\in \partial M\cap \varphi_{\alpha}(U_{\alpha})\) if and only if \( \varphi_{\alpha}^{-1}(m)_n=0\).
\end{proposition}

\begin{proof}
	Two parts.
	\begin{subproof}
		\spitem[\( \Rightarrow\)]
		%-----------------------------------------------------------
		We suppose that \( m\in \partial M\cap \varphi_{\alpha}(U_{\alpha})\). Since \( \varphi_{\alpha}^{-1}(m)\in \eR^n_+\), we have \( \varphi_{\alpha}^{-1}(m)_n\geq 0\). Suppose that \( \varphi_{\alpha}^{-1}(m)>0\). Let \( a=\varphi_{\alpha}^{-1}(m)\). Since \( \varphi_{\alpha}^{-1}\) is continuous, there exists a neighbourhood \( U\) of \(a\in \eR^n \) such that \( y_n>0\) for every \( y\in U\). The restriction \(\varphi_{\alpha} \colon U\to M  \) is then a chart around \( m\) defined on a open part of \( \eR^n\), which is not possible because \( m\in\partial M\).

		\spitem[\( \Leftarrow\)]
		%-----------------------------------------------------------
		Suppose that \( \varphi_{\alpha}^{-1}(m)_n=0\). We have to prove that \( m\in\partial M\), in other words, every chart around \( m\) has to be from \( \eR^n_+\); there exists no charts from an open set of \( \eR^n\) to a neighbourhood of \( m\) in \( M\).

		Let \(\varphi_{\beta} \colon U_{\beta}\to M  \) be an other chart around \( m\). We write \( M_{\alpha\beta}=\varphi_{\alpha}(U_{\alpha})\cap \varphi_{\beta}(U_{\beta})\). By definition of a chart, the map
		\begin{equation}
			\varphi_{\alpha}^{-1}\circ \varphi_{\beta} \colon \varphi_{\beta}^{-1}(M_{\alpha\beta})\to \varphi_{\alpha}^{-1}(M_{\alpha\beta})
		\end{equation}
		is a \( C^{\infty}\)-diffeomorphism.

		Let \( q=\varphi_{\beta}^{-1}(m)\). We have
		\begin{equation}
			(\varphi_{\alpha}^{-1}\circ\varphi_{\beta})(q)_n=\varphi_{\alpha}^{-1}(m)_n=0,
		\end{equation}
		so that \( q_n=0\) by lemma \ref{LEMooLPXFooUBLCqA}.

		There are two possibilities : \( U_{\beta}\) is an open of \( \eR^n_+\) or \( U_{\beta}\) is an open of \( \eR^n\). These possibilities are not exclusive : \( B(10,1)\) is open in \( \eR^n\) and in \( \eR^n_+\).

		\begin{subproof}
			\spitem[If \( U_{\beta}\) is open in \( \eR^n_+\)]
			%-----------------------------------------------------------
			The part \( U_{\beta}\) is an open of \( \eR^n_+\) containing an element of \( \partial \eR^n_+\) (the point \( q\)). By lemma \ref{LEMooEVSVooKJrgKz}, the part \( U_{\beta}\) cannot be an open of \( \eR^n\). Thus \( U_{\beta}\) is \( \eR^n_+\).

			\spitem[If \( U_{\beta}\) is open in \( \eR^n\)]
			%-----------------------------------------------------------
			We set again \( q=\varphi_{\beta}^{-1}(m)\). Using the proposition  \ref{PROPooFEJXooMjsvFo}, we consider a \( C^{\infty}\)-diffeomorphism \(\psi \colon \eR^n\to B(s,r)  \) with \( s_n>0\) and \( r<\| s \|\) (so \( B(s,r)\) keeps in the interior of \( \eR^n_+\)).

			Since \( U_{\beta}\) is open in \( \eR^n\), there exists \( \delta>0\) such that \( B(q,\delta)\subset U_{\beta}\). We choose \( \delta\) small enough to have \( \varphi_{\beta}(\big( B(q,\delta) \big)\subset \varphi_{\alpha}(U_{\alpha})\). We write
			\begin{enumerate}
				\item
				      \( A=\psi\big( B(q,\delta) \big)\),
				\item
				      \( U_{\alpha}'=(\varphi_{\alpha}^{-1}\circ\varphi_{\beta})\circ\psi^{-1}(A)\).
			\end{enumerate}
			The map
			\begin{equation}
				\sigma=\varphi_{\alpha}^{-1}\circ\varphi_{\beta}\circ\psi^{-1} \colon A\to U_{\alpha}'
			\end{equation}
			is a \( C^{\infty}\)-diffeomorphism between \( A\) which is open in \( \eR^n_+\) and \( U_{\alpha}'\) which is open in \( \eR^n\). The point \( a=\psi(q)\) belongs to \( A\) and satisfy
			\begin{equation}
				\sigma(a)=\sigma\big( \psi(q) \big)=\varphi_{\alpha}^{-1}(m).
			\end{equation}
			Thus \( \sigma(a)_n=\varphi_{\alpha}^{-1}(m)_n=0 \). This is a contradiction because the map \( \psi\) takes its values in \( B(s,r)\) which does not intersect \( \partial \eR^n_+\).
		\end{subproof}
	\end{subproof}
\end{proof}

\begin{proposition}[\cite{BIBooZBZTooYmHemH,MonCerveau}]	\label{PROPooTENAooDxIAbf}
	If \( M\) is a smooth manifold with boundary, then the boundary \( \partial M\) has a structure of smooth manifold.
\end{proposition}

\begin{proof}
	Let \( \{ (U_{\alpha}, \varphi_{\alpha}) \}_{\alpha\in \Lambda}\) be an atlas for \( M\). We consider the "boundary" charts:
	\begin{equation}
		\Lambda'=\{ \alpha\in\Lambda\tq U_{\alpha}=\eR^n_+ \},
	\end{equation}
	and, for each \( \alpha\in \Lambda'\), the maps
	\begin{equation}
		\begin{aligned}
			\phi_{\alpha}\colon r(U_{\alpha}) & \to \partial M                                 \\
			(x_1,\ldots,x_{n-1})              & \mapsto \varphi_{\alpha}(x_1,\ldots,x_{n-1},0)
		\end{aligned}
	\end{equation}
	where \( r\) is the map
	\begin{equation}
		\begin{aligned}
			r\colon \eR^n    & \to \eR^{n-1}                 \\
			(x_1,\ldots,x_n) & \mapsto (x_1,\ldots,x_{n-1}).
		\end{aligned}
	\end{equation}
	Defining
	\begin{equation}
		\begin{aligned}
			s\colon \eR^{n-1} & \to \eR^n      \\
			x'                & \mapsto (x',0)
		\end{aligned}
	\end{equation}
	we have
	\begin{subequations}		\label{SUBEQSooMASYooFfkvRT}
		\begin{align}
			\phi_{\alpha}=\varphi_{\alpha}\circ s \\
			\phi_{\alpha}^{-1}=r\circ\varphi_{\alpha}^{-1}.
		\end{align}
	\end{subequations}


	We are going to use the theorem \ref{THOooFIHIooLiSUxH} to show that the maps \( \{ \big( r(U_{\alpha}), \phi_{\alpha} \big) \}_{\alpha\in \Lambda'}\) turn \( \partial M\) into a topological manifold.
	\begin{subproof}
		\spitem[\( r(U_{\alpha})\) is open in \(\eR^{n-1}\)]
		%-----------------------------------------------------------
		Since \( U_{\alpha}=\eR^{n}_+\), we have \( r(U_{\alpha})=\eR^{n-1}\) which is open in \( \eR^{n-1}\).

		\spitem[Union is \( \partial M\)]
		%-----------------------------------------------------------
		Let \( m\in \partial M\). There is a \( \alpha\in \Lambda'\) such that \( m=\varphi_{\alpha}(q)\) with \( q\in U_{\alpha}\). From proposition \ref{PROPooWQHTooXhkWdZ}, we have \( q_n=0\) and then \( m=\phi_{\alpha}\big( r(q) \big)\).

		\spitem[\( \phi\) is injective]
		%-----------------------------------------------------------
		Suppose \( \phi_{\alpha}(x)=\phi_{\alpha}(y)\). This means
		\begin{equation}
			\varphi_{\alpha}(x_1,\ldots,x_{n-1},0)=\varphi_{\alpha}(y_1,\ldots,y_{n-1},0)
		\end{equation}
		and then \( (x_1,\ldots,x_{n-1},0)=(y_1,\ldots,y_{n-1},0)\) because \( \varphi_{\alpha}\) is injective.

		\spitem[\( \phi_{\alpha}^{-1}(\ldots)\) is open in \( U_{\alpha}\)]
		%-----------------------------------------------------------
		We have to prove that \( \phi_{\alpha}^{-1}\Big( \phi_{\alpha}\big( r(U_{\alpha}) \big)\cap \phi_{\beta}\big( r(U_{\beta}) \big)  \Big)\) is open in \( r(U_{\alpha})=\eR^{n-1}\).

		By hypothesis, the part
		\begin{equation}
			\varphi_{\alpha}^{-1}\big( \varphi_{\alpha}(U_{\alpha})\cap \varphi_{\beta}(U_{\beta}) \big)
		\end{equation}
		is open in \( U_{\alpha}\). Thus the part
		\begin{equation}
			A=(r\circ \varphi_{\alpha}^{-1})  \big( \varphi_{\alpha}(U_{\alpha})\cap \varphi_{\beta}(U_{\beta}) \big)
		\end{equation}
		is open in \( \eR^{n-1}\). Notice that
		\begin{equation}
			A=\phi_{\alpha}^{-1}\Big( \varphi_{\alpha}(U_{\alpha})\cap \varphi_{\beta}(U_{\beta}) \Big)=\phi_{\alpha}^{-1} \Big( \varphi_{\alpha}(U_{\alpha})\cap \varphi_{\beta}(U_{\beta})\cap\partial M \Big)
		\end{equation}
		because \( \phi_{\alpha}\) takes its values in \( \partial M\). For \( x\in U_{\alpha}\), there are two possibilities: either \( x_n=0\) or not. In the first case, \( \varphi_{\alpha}(x)\in \partial M\), in the second case, \( \varphi_{\alpha}(x)\not\in \partial M\) (proposition \ref{PROPooWQHTooXhkWdZ}). Defining \( U'_{\alpha}=\{ x\in U_{\alpha}\ tq x_n=0 \}\) and \( U_{\beta}'=\{ x\in U_{\beta}\tq x_n=0 \}\) we have
		\begin{equation}
			A=\phi_{\alpha}^{-1} \Big( \varphi_{\alpha}(U'_{\alpha})\cap \varphi_{\beta}(U'_{\beta}) \Big)
		\end{equation}
		Let \( (x',0)\in U'_{\alpha}\). We have
		\begin{equation}
			\varphi_{\alpha}(x',0)=\phi_{\alpha}(x')=\phi_{\alpha}\big( r(x',0) \big),
		\end{equation}
		and the same with \( \beta\) instead of \( \alpha\). Thus
		\begin{equation}		\label{EQooWVGBooMzuLYh}
			A=\phi_{\alpha}^{-1} \Big( \varphi_{\alpha}(U'_{\alpha})\cap \varphi_{\beta}(U'_{\beta}) \Big)=\phi_{\alpha}^{-1}\Big( \phi_{\alpha}\big( r(U_{\alpha}) \big)\cap\phi_{\beta}\big( r(U_{\beta}) \big) \Big).
		\end{equation}
		Since \( A\) is open, the right hand side of \eqref{EQooWVGBooMzuLYh} is open too.

		\spitem[Continuity and smoothness]
		%-----------------------------------------------------------
		Using the relations \eqref{SUBEQSooMASYooFfkvRT} we have
		\begin{equation}
			\phi_{\alpha}^{-1}\circ\phi_{\beta}=r\circ \varphi_{\alpha}^{-1}\circ\varphi_{\beta}\circ s.
		\end{equation}
		Since \( \varphi_{\alpha}^{-1}\circ\varphi_{\beta}\) is smooth, the whole is smooth.
	\end{subproof}
	Now the theorem \ref{THOooFIHIooLiSUxH} produces a structure of \( n-1\)-dimensional topological manifold for \( \partial M\). The fact that the transition maps \( \phi_{\alpha}^{-1}\circ\phi_{\beta}\) is smooth makes \( \partial M\) a smooth manifold.
\end{proof}

\begin{theorem}[Induced orientation\cite{BIBooZBZTooYmHemH}]	\label{THOooENBHooOfBTlA}
	Si \( \{ (U_{\alpha}, \varphi_{\alpha}) \}_{\alpha\in \Lambda}\) is an orientation on the \( n\)-dimensional smooth manifold \( M\), then the maps \( \{ r(U_{\alpha}), \phi_{\alpha} \}_{\alpha\in \Lambda'}\) is an orientation on \( \partial M\).

	If \( n\) is even, then this is the \defe{induced orientation}{induced orientation} is that one. If \( n\) is odd, the induced orientation is the other one.
\end{theorem}

\begin{proof}
	We have to check that
	\begin{equation}
		\det\big( d(\phi_{\alpha}^{-1}\circ\phi_{\beta})_{\phi^{-1}_{\beta}(m)} \big)>0
	\end{equation}
	for every \( \mu\in\partial M\) for which it makes sense.
\end{proof}


%+++++++++++++++++++++++++++++++++++++++++++++++++++++++
\section{Integral of differential forms}
%+++++++++++++++++++++++++++++++++++++++++++++++++++++++


\begin{proposition}[\cite{BIBooTGSMooYDnTxb}]	\label{PROPooGVQDooNwnGXs}
	Let \( M\) be a smooth orientable manifold. Let \( \mA= \{ (U_{\alpha},\varphi_{\alpha}) \}_{\alpha\in \Lambda}\) be an orientation on \( M\). We consider a differential form \( \omega\) supported on \( S\subset M\). Let \( \alpha,\beta\in\Lambda\) such that \( S\subset\varphi_{\alpha}(U_{\alpha})\cap\varphi_{\beta}(U_{\beta})\). We define the maps \(f \colon M_{\alpha}\to \eR  \) and \(g \colon M_{\beta}\to \eR  \) such that
	\begin{subequations}
		\begin{align}
			\omega_x=f(x)\partial_{\alpha,1}^*\wedge\ldots\wedge\partial_{\alpha,n}^* \\
			\omega_x=g(x)\partial_{\beta,1}^*\wedge\ldots\wedge\partial_{\beta,n}^*.
		\end{align}
	\end{subequations}
	Then we have
	\begin{equation}
		\int_{U_{\alpha}}(f\circ \varphi_{\alpha})=\int_{U_{\beta}}g\circ\varphi_{\beta}.
	\end{equation}
\end{proposition}

\begin{proof}
	This is a change of variable in the integral: theorem \ref{THOooUMIWooZUtUSg}\ref{ITEMooAJGDooGHKnvj}. First we have
	\begin{equation}
		\int_{U_{\alpha}}f\circ \varphi_{\alpha}=\int_{\varphi_{\alpha}^{-1}(S)}f\circ \varphi_{\alpha},
	\end{equation}
	and since
	\begin{equation}
		\varphi_{\alpha}^{-1}(S)=(\varphi_{\alpha}^{-1}\circ\varphi_{\beta})\big( \varphi_{\beta}^{-1}(S) \big),
	\end{equation}
	we try the change of variable with
	\begin{equation}
		\begin{aligned}
			\phi\colon \varphi_{\beta}^{-1}(S) & \to \varphi_{\alpha}(S)                      \\
			\phi                               & =\varphi_{\alpha}^{-1}\circ \varphi_{\beta}.
		\end{aligned}
	\end{equation}
	As far as the domain are concerned,
	\begin{equation}
		\phi^{-1}\big( \varphi_{\alpha}^{-1}(S) \big)=\varphi_{\beta}^{-1}\big( (\varphi_{\alpha}\circ\varphi_{\alpha}^{-1})(S) \big)=\varphi_{\beta}^{-1}(S).
	\end{equation}
	For the integral we have
	\begin{equation}
		\int_{\varphi_{\alpha}^{-1}(S)}s\circ\varphi_{\alpha}=\int_{\varphi_{\beta}^{-1}(S)}(f\circ\varphi_{\beta})J_{\phi}
	\end{equation}
	where \( J_{\phi}\) is the function
	\begin{equation}
		J_{\phi}(u)=\det\big( d\phi_u \big).
	\end{equation}
	Now we work out the link between \( f\) and \( g\) with lemma \ref{LEMooSZTOooBIzMCc}:
	\begin{equation}
		g(x)=\omega_x(\partial_{\beta,1},\ldots,\partial_{\beta,n})=f(x)\det\big( d\phi_{\varphi_{\beta}^{-1}(x)} \big).
	\end{equation}
	Thus we have
	\begin{equation}
		(g\circ\varphi_{\beta})(u)=(f\circ\varphi_{\beta})(u)\det\big( d\phi_u \big)=(f\circ\varphi_{\beta})(u)J_{\phi}(u),
	\end{equation}
	and we have the result.
\end{proof}


\begin{propositionDef}[Integral of a differential form on a manifold]       \label{DEFooOMQLooGiJWZS}
	Let \( M\) be a manifold. Let \(  \{ (\varphi_i,U_i) \}_{i\in I}\) and \( \{ (\psi_j,V_j) \}_{j\in J}\) be two atlas of \( M\). We consider \( \{ f_i \}_{i\in I}\) and \( \{ g_j \}_{j\in J}\) be partition of the unity subordinated to the respective two atlases.

	Then we have
	\begin{equation}
		\sum_{i\in I}\int_{\varphi_i(U_i)}f_i\omega=\sum_{j\in J}\int_{\psi_j(V_j)}g_j\omega.
	\end{equation}
	This number is the \defe{integral of \( \omega\) on \( M\)}{integral of a differential form} and is denoted by
	\begin{equation}
		\int_M\omega
	\end{equation}
\end{propositionDef}


%+++++++++++++++++++++++++++++++++++++++++++++++++++++++++++++++++++++++++++++++++++++++++++++++++++++++++++++++++++++++++++
\section{Integration of a differential form}
%+++++++++++++++++++++++++++++++++++++++++++++++++++++++++++++++++++++++++++++++++++++++++++++++++++++++++++++++++++++++++++

%---------------------------------------------------------------------------------------------------------------------------
\subsection{Open set in \( \eR^n\)}
%---------------------------------------------------------------------------------------------------------------------------

Let \( U\) be an open set of \( \eR^n\). A differential form of degree \( n\) over \( U\) can always be written under the form
\begin{equation}
	\omega_x=f(x)dx_1\wedge\ldots\wedge dx_n;
\end{equation}
this is proposition~\ref{ProprbjihK}.

\begin{definition}      \label{DEFooEYRFooRQTmRF}
	The integral of \( \omega\) on \( U\) is
	\begin{equation}
		\int_{U}f\,dx_1\wedge\ldots\wedge dx_n=\int_Uf
	\end{equation}
	The second integral is the integral of a function on \( \eR^n\), that is definition~\ref{DefTVOooleEst} where the measure is the Lebesgue measure on \( \eR^n\).
\end{definition}

\begin{lemma}[Change of variable]       \label{LEMooNCYSooXtnCKq}
	Let \( f\colon V\to U\) be a diffeomorphism of open sets in \( \eR^n\) and \( \omega\) be a \( n\)-form on \( U\). Then we have
	\begin{equation}
		\int_U\omega=\int_{f^{-1}(U)}f^*\omega
	\end{equation}
	if \( \det(f)>0\). A sign change if \( \det(df)<0\).
\end{lemma}

\begin{proof}
	Let, for \( y\in U\), write the form \( \omega\) as \( \omega_y=h(y)dy_1\wedge\ldots\wedge dy_n\). Taking \( v_i\in \Gamma(TV)\) we have
	\begin{subequations}
		\begin{align}
			(f^*\omega)_x(v_1,\ldots, v_n) & =\omega_{f(x)}\big( df_xv_1,\ldots, df_xv_n \big)                       \\
			                               & =h\big( f(x) \big)\det\begin{pmatrix}
				                                                       df_xv_1 \\
				                                                       \vdots  \\
				                                                       df_xv_n
			                                                       \end{pmatrix}                                    \\
			                               & =(h\circ f)(x)\det(df_x)\det\begin{pmatrix}
				                                                             v_1    \\
				                                                             \vdots \\
				                                                             v_n
			                                                             \end{pmatrix}                              \\
			                               & =(h\circ f)(x)\det(df_x)(dx_1\wedge\ldots\wedge dx_n)(v_1,\ldots, v_n).
		\end{align}
	\end{subequations}
	Thus
	\begin{equation}
		f^*\omega= (h\circ f)\det(df)dx_1\wedge\ldots\wedge dx_n
	\end{equation}
	Using the usual change of variable theorem~\ref{THOooUMIWooZUtUSg}\ref{ITEMooAJGDooGHKnvj} (and taking a sign if \( \det(df)<0\) because there is an absolute value in around the jacobian in \eqref{EQooLYAWooTArAZR}) :
	\begin{equation}
		\int_{f^{-1}(U)}f^*\omega=\int_V(h\circ f)\det(df)=\int_{f(V)}h=\int_Uh=\int_U\omega.
	\end{equation}
\end{proof}

That is for integrating a differential form on an open set of \( \eR^n\). In order to integrate on a manifold we ``simply'' use a pull-back with a chart system. There will be three complications
\begin{itemize}
	\item If an atlas is made from more than one chart, what about the intersections ?
	\item Independence with respect to the choice of the chart.
	\item Integrating a vector field (that is not obviously a \( n\)-form).
\end{itemize}

%---------------------------------------------------------------------------------------------------------------------------
\subsection{One chart on a manifold}
%---------------------------------------------------------------------------------------------------------------------------

We suppose \( (M,g)\) to be a \( n\)-dimensional Riemannian manifold and \( S\) to be a \( (n-1)\)-dimensional submanifold. We suppose that both are inside only one chart
\begin{equation}
	\phi\colon U\subset \eR^n\to M
\end{equation}
and
\begin{equation}
	\varphi\colon A\subset \eR^{n-1}\to S.
\end{equation}
We also consider a differential form \( \omega\in \Wedge^n(T^*M)\) and \( \sigma\in\Wedge^{n-1}(T^*M)\). These are respectively \( n\) and \( n-1\) differential forms on \( M\).  We also consider \( v\), a vector field on \( M\) and \( \tau\), a \( 1\)-form on \(M\).

Let us see what is possible to integrate.

\begin{definition}[\cite{ooMLEZooCKxedX}]       \label{DEFooPDRCooPiBklC}
	Let \( \omega\) be a \( n\)-form defined on \( \phi(U)\) (vanishing everywhere else). Its integral is :
	\begin{equation}
		\int_{\phi(U)}\omega=\int_U\phi^*\omega.
	\end{equation}
	The last integral is an integral of type \( \int_{U}F(x_1,\ldots, x_n)dx_1\wedge\ldots \wedge dx_n\) on an open set in \( \eR^n\). That is definition~\ref{DEFooEYRFooRQTmRF}.
\end{definition}

This definition is nothing if it depend on the parametrisation. The following proposition show slightly more than the independence.
\begin{proposition}[\cite{MonCerveau,ooBTXRooUEBLMV}]       \label{PROPooNJCLooMqeeeX}
	Let be the charts \( \phi\colon U\to M\) and \( \psi\colon V\to N\) and a map \( f\colon M\to N\). The whole is supposed to be minimal :
	\begin{equation}
		f\big( \phi(U) \big)=\psi(V).
	\end{equation}
	Then we have the ``change of variable'' formula :
	\begin{equation}
		\int_{\phi(U)}\omega=\int_{\psi(V)}(f^{-1})^*\omega.
	\end{equation}
\end{proposition}

\begin{proof}
	By definition \( \int_{\phi(U)}\omega=\int_U\phi^*\omega\) and we have the diffeomorphism
	\begin{equation}
		\phi^{-1}\circ f^{-1}\circ \psi\colon V\to U,
	\end{equation}
	so that we can use the result of lemma~\ref{LEMooNCYSooXtnCKq} :
	\begin{equation}
		\int_U\phi^*\omega=\int_{(\phi^{-1}\circ f^{-1}\circ \psi)^{-1}(U)}  (\phi^{-1}\circ f^{-1}\circ \psi)^*\phi^*\omega=\int_{(\psi^{-1}\circ f\circ \phi )U}\psi^*(f^{-1})^*\omega=\int_{\psi^{-1}(N)}\psi^*(f^{-1})^*\omega.
	\end{equation}
	The last integral is the definition of an integral on \( N\) :
	\begin{equation}
		\int_{\psi^{-1}(N)}\psi^*(f^{-1})^*\omega=\int_N(f^{-1})^*\omega.
	\end{equation}
\end{proof}

Here is the lemma that shows the independence of definition~\ref{DEFooPDRCooPiBklC} with respect to the change of chart system.
\begin{lemma}
	Let \( \varphi\colon V\to M\) be a chart such that \( \varphi(V)\cap \varphi(U)=N\) is not empty. We define \( U'=\phi^{-1}(N)\) and \( V'=\varphi^{-1}(N)\). Then
	\begin{equation}        \label{EQooLSZMooPcyMWN}
		\int_{\phi(U')}\omega=\int_{\varphi(V')}\omega.
	\end{equation}
\end{lemma}
This lemma allows us to write \( \int_N\omega\) the common value of both sides of \eqref{EQooLSZMooPcyMWN}.

\begin{proof}
	Taking \( f=\id\) and two charts for the same open set in \( M\) in proposition~\ref{PROPooNJCLooMqeeeX} shows the result.
\end{proof}

%---------------------------------------------------------------------------------------------------------------------------
\subsection{On manifold that require a finite atlas}
%---------------------------------------------------------------------------------------------------------------------------

We restrict ourself to manifolds that accept a finite atlas.

\begin{definition}[\cite{ooBTXRooUEBLMV}]      \label{DEFooITDTooWwrPPr}
	If \( \omega\) is a \( n\)-form on \( M\) and if \( \{ f_{\alpha} \} \) is a partition of unity\footnote{See theorem~\ref{THOooPCHDooITWKpC}.} subordinate to the finite atlas \( \{ U_{\alpha} \}\) then
	\begin{equation}
		\int_M\omega=\sum_{\alpha}\int_{\phi_{\alpha}(U_{\alpha})}f_{\alpha}\omega.
	\end{equation}
\end{definition}

We show that this definition does not depend on the choice of the partition of unity.
\begin{lemma}[\cite{MonCerveau,ooBTXRooUEBLMV}] \label{LEMooCMIZooHhHaHV}
	The definition~\ref{DEFooITDTooWwrPPr} is independent of the choice of atlas and partition of unity.
\end{lemma}

\begin{proof}
	Let \(  \{ U_{\alpha},\phi_{\alpha},f_{\alpha} \}_{\alpha\in A}  \) and \( \{ V_i,\varphi_i,g_i \}_{i\in I}\) be two choices of atlas, charts and subordinate partition of unity. We have to show that
	\begin{equation}        \label{EQooPVQZooHvbioJ}
		\sum_{\alpha\in A}\int_{\phi_{\alpha}(U_{\alpha})}f_{\alpha}\omega=\sum_{i\in I}\int_{\varphi_i(V_i)}g_i\omega.
	\end{equation}
	Since \( \{ g_i \}\) is a partition of unity,
	\begin{equation}
		\spadesuit=\sum_{\alpha}\int_{\phi_{\alpha}(U_{\alpha})}f_{\alpha}\omega=\sum_{\alpha}\int_{\phi_{\alpha}(U_{\alpha})}\sum_ig_if_{\alpha}\omega.
	\end{equation}
	Since the atlas are finite, the sums are finite and can be permuted with the integral. Moreover the function \( g_if_{\alpha}\) is nonzero only on \( \phi_{\alpha}(U_{\alpha})\cap\varphi_i(V_i)\) so that the integral can be taken on \( \phi_{\alpha}(U_{\alpha})\), \( \phi_{\alpha}(U_{\alpha})\cap\varphi_i(V_i)\) or \( \varphi_i(V_i)\). We have
	\begin{subequations}
		\begin{align}
			\spadesuit=\sum_i\sum_{\alpha}\int_{\phi_{\alpha(U_{\alpha})}}g_if_{\alpha}\omega & =  \sum_i\sum_{\alpha}\int_{\phi_{\alpha(U_{\alpha})}\cap \varphi_i(V_i)}g_if_{\alpha}\omega \\
			                                                                                  & =  \sum_i\sum_{\alpha}\int_{\varphi_i(V_i)}g_if_{\alpha}\omega                               \\
			                                                                                  & = \sum_i\int_{\varphi_i(V_i)}g_i\sum_{\alpha}f_{\alpha}\omega                                \\
			                                                                                  & =\sum_i\int_{\varphi_i(V_i)}g_i\omega.
		\end{align}
	\end{subequations}
\end{proof}
The common values of both sides of \eqref{EQooPVQZooHvbioJ} is denoted by \( \int_M\omega\).

The following is not really a definition, but a particular case of~\ref{DEFooPDRCooPiBklC}. The integral of a \( n-1\)-form on a \( (n-1)\)-submanifold is
\begin{equation}        \label{EQooYPOGooRYOXQe}
	\int_S\sigma=\int_A\varphi^*\sigma.
\end{equation}
Once again the last integral is an integral of a \( n-1\)-form on an open set in \( \eR^{n-1}\).

\begin{definition}[\cite{ooMLEZooCKxedX}]       \label{DEFooAXFXooWiMLKP}
	The integral of a \( 1\)-form on a \( n-1\) dimensional submanifold is :
	\begin{equation}
		\int_S\tau=\int_S\hodge\tau
	\end{equation}
	where \( \hodge\) is the Hodge dual defined by~\ref{DEFooUOJQooSzKjNR}.
\end{definition}
The last integral is the integral of a \( (n-1)\)-form on a \( (n-1)\)-submanifold, given by \eqref{EQooYPOGooRYOXQe}.

\begin{definition}      \label{DEFooAXZGooJairMQ}
	The integral of a vector field on a \( (n-1)\)-submanifold is :
	\begin{equation}
		\int_Sv=\int_Sv^{\flat}
	\end{equation}
	where \( v^{\flat}\) is the \( 1\)-form defined by the musical isomorphism \eqref{EQooBTWXooTqoNxa}.
\end{definition}

The following proposition provides a much more explicit formula for the integral of a vector field.

\begin{proposition}     \label{PROPooETLZooAVsrwy}
	Let \( \varphi\colon A\subset \eR^{n-1}\to \eR^n\) be an hypersurface and \( X\) be a vector field in \( \eR^n\). Then
	\begin{subequations}
		\begin{align}
			\int_SX & =\int_A\det\big( X,\frac{ \partial \varphi }{ \partial y_1 },\ldots, \frac{ \partial \varphi }{ \partial y_{n-1} } \big)  \label{SUBEQooWJSPooImJjQN} \\
			        & =\int_A X\cdot\det\begin{pmatrix}
				                            e_1 & \ldots                    & e_n \\
				                                & \partial_{y_1}\varphi     &     \\
				                                & \vdots                    &     \\
				                                & \partial_{y_{n-1}}\varphi &
			                            \end{pmatrix}                                                                        \\
			        & =\int_A X\cdot n
		\end{align}
	\end{subequations}
	where \( \{ y_1,\ldots, y_{n-1} \}\) are the coordinates on \( A\) and \( n\) is the normal vector to the parametrization.
\end{proposition}
Note : thanks to lemma~\ref{LEMooCMIZooHhHaHV}, the value of \( n\) can depend on the choice of coordinates, but the integral will not depend.

\begin{proof}
	If \( X=\sum_{i=1}^nX_i\partial_i\), then \( X^{\flat}=\sum_{i}X_idx_i\) and its Hodge dual is
	\begin{equation}
		\sum_{i}(-1)^i dx_1\wedge\ldots\wedge\widehat{dx_i}\wedge\ldots\wedge dx_n
	\end{equation}
	where the hat denotes a factor that is not present. Using the definitions~\ref{DEFooAXZGooJairMQ},~\ref{DEFooAXFXooWiMLKP} and~\ref{DEFooPDRCooPiBklC} it remains to integrate
	\begin{equation}
		\int_A\sum_i(-1)^iX_i\varphi^*\big( dx_1\wedge\ldots\wedge\widehat{dx_i}\wedge\ldots\wedge dx_n \big).
	\end{equation}
	If \( u_1,\ldots, u_{n-1}\) are vectors on \( A\) (that is on \( T_xA\) where \( x\) is the integration variable) we have
	\begin{subequations}
		\begin{align}
			\varphi^*(dx_1\wedge\ldots\wedge \widehat{dx_i}\wedge\ldots\wedge dx_n)(u_1,\ldots, u_{n-1}) & = (dx_1\wedge\ldots\wedge \widehat{dx_i}\wedge\ldots\wedge dx_n)(d\varphi u_1,\ldots, d\varphi u_{n-1}) \\
			                                                                                             & =\det\big( \tau_id\varphi u_1,\ldots, \tau_id\varphi u_{n-1} \big)
		\end{align}
	\end{subequations}
	where we used the lemma~\ref{LEMooICRXooFKPCRd}.

	What lies in the integral is the \( (n-1)\) differential form
	\begin{subequations}        \label{EQooEVAPooSbRfaj}
		\begin{align}
			(u_1,\ldots, u_{n-1})\mapsto \sum_{i}(-1)^iX_i & \det\big(    \tau_id\varphi u_1,\ldots, \tau_id\varphi u_{n-1}  \big) \\
			                                               & =\det\big( X,d\varphi u_1,\ldots, d\varphi u_{n-1} \big).
		\end{align}
	\end{subequations}
	Since this is a \( (n-1)\) differential form over \( \eR^{n-1}\), this has to be proportional to \( dy_1\wedge\ldots dy_{n-1}\). The proportionality factor is found by applying \eqref{EQooEVAPooSbRfaj} to the basis \( \{ e_1,\ldots, e_n \}\). Since \( d\varphi(e_i)=\frac{ \partial \varphi }{ \partial y_i }\) we have the proportionality factor
	\begin{equation}
		\det\left( X,\frac{ \partial \varphi }{ \partial y_1 },\ldots, \frac{ \partial \varphi }{ \partial y_n } \right)
	\end{equation}
	and the integral to be computed is
	\begin{equation}
		\int_A\det\left( X,\frac{ \partial \varphi }{ \partial y_1 },\ldots, \frac{ \partial \varphi }{ \partial y_n } \right)dy_1\wedge\ldots\wedge dy_{n-1}=\int_A\det\left( X,\frac{ \partial \varphi }{ \partial y_1 },\ldots, \frac{ \partial \varphi }{ \partial y_n } \right).
	\end{equation}
	The formula \eqref{SUBEQooWJSPooImJjQN} is proven. The two others are application of lemma~\ref{LEMooFRWKooVloCSM}.
\end{proof}

\begin{example}
	Let us make the example with \( n=3\). We have
	\begin{equation}
		\varphi^*(dx\wedge dy)(v_1,v_2)=(dx\wedge dy)(d\varphi v_1 , d\varphi v_2)=\det
		\begin{pmatrix}
			d\varphi(v_1)_x & d\varphi(v_2)_x \\
			d\varphi(v_1)_y & d\varphi(v_2)_y
		\end{pmatrix},
	\end{equation}
	and then
	\begin{equation}
		\sum_i(-1)^iX_i \varphi^*(   \Wedge_{k\neq i}dx_k    )(v_1,v_2)=\sum_i(-1)^iX_i
		\begin{pmatrix}
			d\varphi(v_1)_x & d\varphi(v_2)_x \\
			d\varphi(v_1)_y & d\varphi(v_2)_y
		\end{pmatrix}=
		\det\begin{pmatrix}
			X_1 & d\varphi (v_1)_x & d\varphi(v_2)_x \\
			X_2 & d\varphi (v_1)_y & d\varphi(v_2)_y \\
			X_3 & d\varphi (v_1)_z & d\varphi(v_2)_z
		\end{pmatrix}
	\end{equation}
\end{example}

%---------------------------------------------------------------------------------------------------------------------------
\subsection{Integrating by part}
%---------------------------------------------------------------------------------------------------------------------------

\begin{proposition}[\cite{MonCerveau}]
	Let \( \Omega\) be an open set in an manifold \( M\) of dimension \( n\) and \( \varphi\colon A\to \eR^n\) be a parametrisation of the boundary \( \partial\Omega\) with tangent vector field \( n\) (defined on \( \partial\Omega\)). Let \( u,v\in  C^{\infty}(M)\). Then we have
	\begin{equation}        \label{EQooQSMNooKHwbqp}
		\int_{\partial \Omega}uv\,n_j=\int_{\Omega}\frac{ \partial u }{ \partial x_j }v+\int_{\Omega}u\frac{ \partial v }{ \partial x_j }.
	\end{equation}
\end{proposition}

\begin{proof}
	We use the Stokes formula (theorem~\ref{ThoATsPuzF}) on the \( (n-1)\)-form
	\begin{equation}
		\omega=uv\,dx_1\wedge\ldots\wedge\widehat{dx_j}\wedge\ldots\wedge dx_n,
	\end{equation}
	and we know from example~\ref{EXooCIYIooFPMLMU} that \( \omega=(-1)^{j+1}\hodge dx_j\). On the other hand,
	\begin{equation}
		d\omega=\sum_k\frac{ \partial (uv) }{ \partial x_k }dx_j\wedge dx_1\wedge\ldots\wedge\widehat{dx_j}\wedge\ldots\wedge dx_n=\frac{ \partial (uv) }{ \partial x_j }(-1)^{j+1}dx_1\wedge\ldots\wedge dx_n.
	\end{equation}
	We can use the Stokes formula :
	\begin{equation}
		\int_{\partial \Omega} uv dx_1\wedge\ldots\wedge\widehat{dx_j}\wedge\ldots\wedge dx_n=  (-1)^{j+1} \int_{\Omega}\frac{ \partial (uv) }{ \partial x_j }.
	\end{equation}
	The left-hand side can be transformed as
	\begin{equation}
		\int_{\partial\Omega}\hodge(dx_j)=\int_{\partial\Omega}uv\partial_j=\int_{\partial\Omega}uv\,n_j
	\end{equation}
	where we used the definition~\ref{DEFooAXFXooWiMLKP} and the proposition~\ref{PROPooETLZooAVsrwy}.

	The coefficients \( (-1)^{j+1}\) simplify and the derivation of product produce the result.
\end{proof}

\begin{example}     \label{EXooWLUVooNamnKG}
	If we integrate by part the function \( u\frac{ \partial^2 v }{ \partial x_j^2 }\) we have
	\begin{equation}
		\int_{\omega}u\frac{ \partial^2 }{ \partial x_j^2 }=-\int_{\Omega}\frac{ \partial u }{ \partial x_j }\frac{ \partial v }{ \partial x_j }+\int_{\partial \Omega}u\frac{ \partial v }{ \partial x_j }n_j.
	\end{equation}
	Summing over \( j\) we have the interesting formula
	\begin{equation}        \label{EQooJLDTooIMtxEX}
		\int_{\Omega}u\Delta v=-\int_{\Omega}\nabla u\cdot\nabla v+\int_{\partial \Omega}u\frac{ \partial v }{ \partial n }
	\end{equation}
	where \( \Delta v=\sum_j\frac{ \partial^2v }{ \partial x_j^2 }\) and \( \frac{ \partial v }{ \partial n }\) is a notation for \( \nabla v\cdot n\).
\end{example}
