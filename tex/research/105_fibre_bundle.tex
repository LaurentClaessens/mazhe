% This is part of (almost) Everything I know in mathematics
% Copyright (c) 2013-2014, 2020, 2023-2024
%   Laurent Claessens
% See the file fdl-1.3.txt for copying conditions.

\section{Vector bundle}
%++++++++++++++++++++++

\begin{definition}[\cite{MonCerveau,BIBooZBHSooLbtepQ}]		\label{DEFooFRPLooTxMUzg}
	Let \( M\) be a smooth manifold with charts \( \{ \varphi_{\alpha} \colon I_{\alpha}\to M   \}_{\alpha\in \tau}\). A \( k\)-dimensional \defe{vector bundle}{vector bundle} on \( M\) is the given of
	\begin{enumerate}
		\item		\label{ITEMooIVKKooNEfxVi}
		      A vector space \( V_x\) for each \( x\in M\).

		      Formally, we suppose that the vector space \( V_x\) has the form \( V_x=\{ (x,v)\tq v\in W_x \}\) for some vector space \( W_x\). The reason is that we want the union \( E=\bigcup_{x\in M}V_x\) to be disjoint.
		\item
		      For each chart \(\varphi_{\alpha} \colon U_{\alpha}\to M  \), a map
		      \begin{equation}
			      \sigma_{\alpha} \colon U_{\alpha}\times \eR^k\to \bigcup_{x\in \varphi_{\alpha}(U_{\alpha})}V_x.
		      \end{equation}
	\end{enumerate}

	With the following conditions.
	\begin{enumerate}[label=(\alph*)]
		\item	\label{ITEMooVIYBooMXFaKb}
		      The maps \( \sigma_{\alpha}\) give a manifold structure on \( E\).
		\item	\label{ITEMooXHOBooERubro}
		      The projection
		      \begin{equation}
			      \begin{aligned}
				      \pi\colon E & \to M     \\
				      (x,v)       & \mapsto x
			      \end{aligned}
		      \end{equation}
		      is smooth for the differential structure of \( E\) given by the atlas \( \{ \sigma_{\alpha} \}_{\alpha\in\tau}\).
		\item		\label{ITEMooXJIVooVJTorK}
		      For every \( s\in U_{\alpha}\) and \( v\in \eR^k\) we have \( \sigma_{\alpha}(s,v)\in V_{\varphi_{\alpha}(s)}\).
		\item \label{ITEMooOBCMooOCqoQo}
		      For each \( s\in U_{\alpha}\), the map
		      \begin{equation}
			      \begin{aligned}
				      \tau_{\alpha,s}\colon \eR^k & \to V_{\varphi_{\alpha}(s)}  \\
				      v                           & \mapsto \sigma_{\alpha}(s,v)
			      \end{aligned}
		      \end{equation}
		      is a vector space isomorphism (linear bijection).
		\item		\label{ITEMooPNLXooLQWGqc}
		      The map \(g_{\alpha\beta} \colon U_{\alpha}\cap U_{\beta}\to \Fun(\eR^k,\eR^k)  \) defined by
		      \begin{equation}
			      (\sigma_{\alpha}^{-1}\circ \sigma_{\beta})(s,v)=\big( s,g_{\alpha\beta}(s)v \big)
		      \end{equation}
		      belongs to \( C^{\infty}\big( U_{\alpha}\cap U_{\beta},\GL(\eR,k) \big)\).
	\end{enumerate}
	Most of the time, we will denote the whole vector bundle only by the couple \( (E,\pi)\).

	The maps \( g_{\alpha\beta}\) are the \defe{transition functions}{transition function}.
\end{definition}

\begin{definition}		\label{DEFooDSKKooEFgmPU}
	Let \( E_1\), \( E_2\) be two vector bundles over the same manifold \( M\). A map \(f \colon E_1\to E_2  \) is a \defe{vector bundle isomorphism}{vector bundle isomorphism} if
	\begin{enumerate}
		\item
		      \( f\) is a manifold smooth diffeomorphism.
		\item
		      \( \pi_1\circ f=\pi_2\)
		\item
		      \( f\big( \pi_1^{-1}(x) \big)=\pi_2^{-1}(x)\) for every \( x\in M\).
		\item
		      The restriction \(f \colon \pi_1^{-1}(x)\to \pi_2^{-1}(x)   \) is a vector space isomorphism.
	\end{enumerate}
\end{definition}


\begin{proposition}		\label{PROPooJZLEooYFgjnY}
	Let \( V\) be a \( k\)-dimensional vector space. Let \(f \colon \eR^k\to V  \) be an isomorphism and \( M\), a manifold. We define
	\begin{equation}
		V_x=\{ (x,v)\tq v\in V \}
	\end{equation}
	and
	\begin{equation}
		\begin{aligned}
			\sigma_{\alpha}\colon U_{\alpha}\times \eR^k & \to \bigcup_{x\in \varphi_{\alpha}(U_{\alpha})} \\
			(s,v)                                        & \mapsto \big( \varphi_{\alpha}(s),f(v) \big).
		\end{aligned}
	\end{equation}
	We have:
	\begin{enumerate}
		\item		\label{ITEMooYHUVooXUSryK}
		      This is a vector bundle structure.
		\item		\label{ITEMooCZLRooIJjQUn}
		      If \( f\) and \( g\) are two isomorphisms between \( \eR^k\) and \( V\), the corresponding vector bundle are isomorphic\footnote{Definition \ref{DEFooDSKKooEFgmPU}}.
	\end{enumerate}
\end{proposition}

\begin{proof}
	For \ref{ITEMooYHUVooXUSryK}, this is a long verification\quext{Make it, write it and send me your proof, so that I can add it here.}.
	%TODOooNKRPooHQgEtj

	We prove \ref{ITEMooCZLRooIJjQUn}. We name \( E_1\) the vector bundle created by \( f\) and \( E_2\) the one created by \( g\). We define
	\begin{equation}
		\begin{aligned}
			h\colon E_1 & \to E_2                                   \\
			(x,v)       & \mapsto \big( x,(f\circ g^{-1})(v) \big).
		\end{aligned}
	\end{equation}
	It satisfies \( \pi_1\circ h=\pi_2\) because \( \pi_i(x,v)=x\) for every \( v\). The map \( h\) is a bijection because \( f\) and \( g\) are bijections. Since \( f\) and \( g\) are linear, the restriction map \(h \colon V_x\to V_x  \) is a vector space isomorphism.

	We prove that \( h\) is a manifold isomorphism: \( h\) and \( h^{-1}\) have to be smooth.
	\begin{subproof}
		\spitem[\( h\) is smooth]
		%-----------------------------------------------------------
		In order to say that \(h \colon E_1\to E_2  \) is smooth, we have to check that
		\begin{equation}
			\sigma_{f,\alpha}^{-1}\circ h\circ \sigma_{g,\beta} \colon  U_{\alpha}\cap U_{\beta}\times \eR^k  \to U_{\alpha}\cap U_{\beta}\times \eR^k
		\end{equation}
		is smooth. We have
		\begin{subequations}
			\begin{align}
				\sigma_{f,\alpha}^{-1}\circ h\circ\sigma_{g,\beta}(s,v) & = (\sigma_{f,\alpha}\circ h)\big( \varphi_{\beta}(s),g(v) \big)                       \\
				                                                        & =\sigma_{f,\alpha}^{-1}\Big( \varphi_{\beta}(s),(f\circ g^{-1})\big( g(v) \big) \Big) \\
				                                                        & =\sigma_{f,\alpha}^{-1}\big( \varphi_{\beta}(s),f(v) \big)                            \\
				                                                        & = \big( (\varphi_{\alpha}^{-1}\circ \varphi_{\beta})(s),v \big).
			\end{align}
		\end{subequations}
		The last expression is smooth with respect to \( (s,v)\).
		\spitem[\( h^{-1}\) is smooth]
		%-----------------------------------------------------------
		The verification is the same for \( \sigma_{g,\alpha}^{-1}\circ h^{-1}\circ\sigma_{g,\beta}\).
	\end{subproof}
\end{proof}

\subsection{Transition functions}
%--------------------------------

\begin{proposition}
	Let \( (E,\pi)\) be a vector bundle. The transition functions satisfy
	\begin{equation}\label{eq:g_compat}
		g_{\alpha\beta}\circ g_{\alpha\gamma}\circ g_{\gamma\alpha}=\mtu.
	\end{equation}
\end{proposition}

\subsection{Inverse construction}\label{subsec:inv_g}
%-------------------------------------------

Let us consider a manifold $M$, an open covering $\{\mU_{\alpha}:\alpha\in I\}$ and some functions $\dpt{g\bab}{\mU\bab}{GL(r,\eK)}$ which fulfill relations \eqref{eq:g_compat}. We will build a vector bundle $E\stackrel{p}{\longrightarrow}M$ whose transition functions are the $g_{\alpha\beta}$'s. Let $\tilde{E}$ be the disjoint union
\[
	\tilde{E}=\bigsqcup_{\alpha\in I}\mU_{\alpha}\times\eK^r,
\]
i.e. triples of the form $(x,v,\alpha)\in M\times\eK^r\times I$ with the condition that $x\in\mU_{\alpha}$. We define an equivalence relation on $\tilde{E}$ by $(x,v,\alpha)\sim(y,w,\beta)$ if and only if $x=y$ and $w=g\bab(x)v$. Next, we define $E=\tilde{E}/\sim$ and $\dpt{\omega}{\tilde{E}}{E}$, the canonical projection. The projection $\dpt{p}{E}{M}$ is naturally defined by $p([x,v,\alpha])=x$. The chart diffeomorphism is $\dpt{\varphi_{\alpha}}{\mU_{\alpha}\times\eK^r}{p^{-1}(\mU_{\alpha})}$,
\[
	\varphi_{\alpha}(x,v)=\omega(x,v,\alpha).
\]
Now we have to prove that $E$ endowed with the $\varphi_{\alpha}$'s is a vector bundle.

First we prove that $\varphi_{\alpha}$ is surjective. For this we remark that a general element in $p^{-1}(\mU_{\alpha})$ can be written under the form $\omega(x,v,\alpha)$ with $x\in\mU\bab$. But
\begin{equation}
	\begin{split}
		\varphi_{\alpha}(x,g\bab(x)w)&=\omega(x,g\bab(x)w,\alpha)\\
		&=\omega(x,g_{\alpha\beta}(x)g\bab(x)w,\beta)\\
		&=\omega(x,w\beta),
	\end{split}
\end{equation}
then $\varphi_{\alpha}$ is surjective. Now we suppose $\varphi_{\alpha}(x,v)=\varphi_{\alpha}(y,w)$. Then $\omega(x,v,\alpha)=\omega(y,w,\alpha)$ and $x=y$, $w=g_{\alpha\alpha}v$ which immediately gives $v=w$. Then $\varphi_{\alpha}$ is injective.

Finally, we have
\begin{equation}
	(\varphi\alpha\circ\varphi_{\beta}^{-1})(\omega(x,v,\alpha))=\varphi_{\alpha}(x,g_{\alpha\beta}(x)v)
	=\omega(x,g_{\alpha\beta}(x)v,\alpha),
\end{equation}
which proves that the maps $g$ are the transition functions of the vector bundle $E$.

\subsection{Equivalence of vector bundle}
%----------------------------------------

Let $E\stackrel{p}{\longrightarrow}M$ and $F\stackrel{p'}{\longrightarrow}M$ be two vector bundles on $M$. They are \defe{equivalent}{equivalence!of vector bundle} if there exists a smooth diffeomorphism $\dpt{f}{E}{F}$ such that

\begin{itemize}
	\item $p'\circ f=p$,
	\item $\dpt{f|_{E_x}}{E_x}{F_x}$ is a vector space isomorphism.
\end{itemize}

Let $E$ and $F$ be two equivalent vector bundles, $\{\mU_{\alpha}\tq \alpha\in I\}$, an open covering which trivialize $E$ and $F$ in the same time and $\phi^E_{\alpha}$, $\phi^F_{\alpha}$ the corresponding trivializations. A map $\dpt{f}{E}{F}$ reads ``in the trivialization''\ as $\dpt{\phi^F_{\alpha}\circ f|_{p^{-1}(\mu_{\alpha})}\circ\phi^E{}^{-1}_{\alpha}}{\mU_{\alpha}\times\eK^r}{\mU_{\alpha}\times\eK^r}$ and defines a map $\dpt{\lambda_{\alpha}}{\mU_{\alpha}}{GL(r,\eK)}$ by
\begin{equation}
	(\phi^F_{\alpha}\circ f|_{p^{-1}(\mu_{\alpha})}\circ\phi^E{}^{-1}_{\alpha})(x,v)=(x,\lambda_{\alpha}(x)v).
\end{equation}
If we denote by $g^E$ the transition functions for $E$ (and $g^F$ for $F$),
\[
	\phi^F_{\alpha}\circ\phi^F_{\beta}{}^{-1}= (\phi^F_{\alpha}\circ f\circ\phi_{\alpha}^E{}^{-1})\circ
	(\phi^E_{\alpha}\circ\phi^E_{\beta}{}^{-1})\circ
	(\phi^E_{\beta}\circ f^{-1}\circ\phi_{\beta}^E{}^{-1}),
\]
so that
\begin{equation}\label{eq:g_l_g_l}
	g\bab^F(x)=\lambda_{\alpha}(x) g^E\bab(x)\lambda(x)^{-1}.
\end{equation}

Once again we have an inverse construction. We consider a vector bundle $E$ on $M$ with transition functions $g^E$ and some maps $\dpt{\lambda_{\alpha}}{\mU_{\alpha}}{GL(r,\eK)}$; then we define $g^F\bab(x)$ by equation \eqref{eq:g_l_g_l}.

From subsection~\ref{subsec:inv_g}, one can construct a vector bundle $F$ on $M$ whose transition functions are these $g^F$. With the trivializations $\phi^F$ of $F$, one can define $\dpt{f}{E}{F}$ by
\[
	(\phi^F_{\alpha}\circ f\circ\phi^E_{\alpha}{}^{-1})(x,v)=(x,\lambda_{\alpha}(x)v).
\]

When a basis space $B$ is given, we denote by $\Vect(B)$ the set of isomorphism classes of vector bundles over $B$. In the complex case, we denote it by $\Vect_{\eC}(B)$.

\begin{proposition}
	Any vector bundle over $\eR^n$ is trivial.
\end{proposition}

\begin{proof}
	Let $\dpt{p}{F}{M}$ be a vector bundle on $M=\eR^n$ and $\{\mU_{\alpha}\}$ be covering of $\eR^n$ by local trivializations. Now consider a partition of unity\index{partition of unity} related to the covering $\mU_{\alpha}$: a set of functions $\dpt{f_{\alpha}}{M}{\eR}$ such that
	\begin{itemize}
		\item $f_{\alpha}>0$,
		\item $\forall x\in M$, one can find a neighbourhood of $x$ in which only a \emph{finite} number of $f_{\alpha}$ is non zero,
		\item $\forall x\in M$, $\sum_{\alpha} f_{\alpha}(x)=1$.
		\item $f_{\alpha}=0$ outside of $\mU_{\alpha}$.
	\end{itemize}
	Using that partition of unity, we build the trivialization function $\dpt{f}{F}{\eR^n\times V}$ by $f(l)=(x,\sum_{\alpha} f_{\alpha}(x)\phi_{\alpha x}(l))$.
\end{proof}

The following two propositions have some importance in K-theory.
\begin{proposition}		\label{PropEoplusEprimetriv}
	Let $\pi\colon E\to B$ be a complex vector bundle over a basis compact, Hausdorff, connected basis $B$. Then there exists a vector bundle $E'$ such that $E\oplus E'$ is trivial.
\end{proposition}

\begin{proposition}		\label{PropmapfEEsun}
	Let $f\colon A\to B$ be a map between the topological spaces $A$ and $B$, and consider a vector bundle $\pi\colon E\to B$. Then there exists one and only one vector bundle $\pi'\colon E'\to A$ and a map $f'\colon E'\to E$ such that $f'|_{E'_x}\colon E'_x\to E_{f(x)}$ is an isomorphism. The vector bundle $E'$ is unique up to isomorphism.
\end{proposition}
Proofs can be found in \cite{VB_and_K}. Let us denote by $f^*(E)$ the function given by proposition~\ref{PropmapfEEsun}. It satisfies the following properties
\begin{equation}		\label{EqPropfstarEVect}
	\begin{split}
		(fg)^*(E)		&=g^*\big( f^*(E) \big)\\
		\id^*(E)		&=E\\
		f^*(E_1\oplus E_2)	&=f^*(E_1)\oplus f^*(E_2)\\
		f^*(E_1\otimes E_2)	&=f^*(E_1)\otimes f^*(E_2).
	\end{split}
\end{equation}



\subsection{Sections of vector bundle}
%-------------------------------------

\begin{definition}		\label{DEFooULPGooZeVheH}
	A \defe{section}{section!of vector bundle} of the vector bundle $p\colon E\to M$ is a smooth map $\dpt{s}{M}{E}$ such that $p\circ s=\id|_M$. The set of all the sections is denoted by $\Gamma^{\infty}(M)$ or simply $\Gamma(E)$.\nomenclature{$\Gamma(E)$}{Space of sections of the vector bundle $E$}
\end{definition}

\begin{proposition}		\label{PROPooMCJWooNAqmIQ}
	Let \( E= \big(  \{ V_x \}_{x\in M},  \{ \sigma_{\alpha} \}  \big)\) be a vector bundle. We consider a set of sections \(e_i \colon M\to E  \) such that for each \( x\in M\), the set \( \{ e_i(x) \}\) is a basis of \( V_x\).

	For every \( v\in \Gamma^{\infty}(E)\), there exist smooth functions \( v_i\in C^{\infty}(M)\) such that
	\begin{equation}
		v(x)=\sum_iv_i(x)e_i(x)
	\end{equation}
	for every \( x\in M\).
\end{proposition}

If $(\mU_{\alpha},\phi_{\alpha})$ is a local trivialization, one can describe the section $s$ by a function $\dpt{s_{\alpha}}{\mU_{\alpha}}{V}$ defined by $\phi_{\alpha}(s(x))=(x,s_{\alpha}(x))$, or equivalently by
\[
	s(x)=\phi_{\alpha}^{-1}(x,s_{\alpha}(x)).
\]
As usual when we define such a local quantity, we have to ask ourself how are related $s_{\alpha}$ and $s_{\beta}$ on $\mU_{\alpha}\cap\mU_{\beta}$. The best is $s_{\alpha}=s_{\beta}$, but most of the time it is not. Here, we compute
\[
	\phi_{\beta}\circ\phi_{\alpha}^{-1}\circ\phi_{\alpha}(s(s))=(x,g_{\alpha\beta}(x)s_{\alpha}(x)),
\]
which is obviously also equal to $(x,s_{\beta}(x))$. Then
\begin{equation}\label{eq:tr_sec}
	s_{\beta}(x)=g_{\alpha\beta}(x)s_{\alpha}(x)
\end{equation}
without summation.

%-------------------------------------------------------
\subsection{Tensor product}
%----------------------------------------------------

\begin{propositionDef}[\cite{MonCerveau,BIBooPQXJooItBJGZ}]		\label{DEFooCSDZooJuzGuE}
	Let \( M \) be a smooth manifold. We consider two vector bundles\footnote{Definition \ref{DEFooFRPLooTxMUzg}.} \( V_x^{(i)}\), \( \sigma_{\alpha}^{(i)}  \colon U_{\alpha}\times \eR^{k_i}\to \bigcup_{x\in M}V_x^{(i)} \) with transition functions \(g_{\alpha\beta}^{(i)} \colon U_{\alpha}\cap U_{\beta}\to \GL(\eR,k)  \).

	Let \(r \colon \eR^{k_1}\otimes \eR^{k_2}\to \eR^{k_1k_2}  \) be a vector space isomorphism\footnote{For the dimension, see the proposition \ref{PROPooTHDPooWgjUwk}.}.

	We consider the following :
	\begin{enumerate}
		\item
		      \( V_x=V_x^{(1)}\otimes V_x^{(2)}\),
		\item
		      \( T=\bigcup_{x\in M}V_x\)
		\item For each \( \alpha\) we define
		      \begin{equation}		\label{EQooXHURooBpOMlU}
			      \begin{aligned}
				      \sigma_{\alpha}\colon U_{\alpha}\times \eR^{k_1k_2} & \to T                                                                                   \\
				      \big( s, r(\sum_{ij}a_{ij} v_i\otimes w_j) \big)    & \mapsto \sum_{ij}a_{ij}\sigma_{\alpha}^{(1)}(s,v_i)\otimes \sigma_{\alpha}^{(2)}(s,w_j)
			      \end{aligned}
		      \end{equation}
		      for every \( s\in U_{\alpha}\), \( v_i\in \eR^{k_1}\) and \( w_j\in \eR^{k_2}\).
	\end{enumerate}

	We have:
	\begin{enumerate}
		\item
		      The set	\( T\) with the charts \( \sigma_{\alpha} \) is a smooth vector bundle of dimension \( k_1k_2\) over \( M\).
		\item
		      The vector bundle does not depend on the choice of the isomorphism \( r\) in the sense that two vector bundles build from two different isomorphisms \( r_1\) and \( r_2\) are equivalent.
		      %TODOooTXYEooCxyiwi Je dois définir la notion d'équivalence, et le prouver.
		      % C'est la définition DEFooDSKKooEFgmPU
	\end{enumerate}
\end{propositionDef}

\begin{proof}
	Since the spaces \( V^{(i)}_x\) are distinct, the spaces \( V_x=V_x^{(1)}\otimes V_x^{(2)}\) are distinct.
	We have to check the conditions of definition \ref{DEFooFRPLooTxMUzg}.
	\begin{subproof}
		\spitem[Condition \ref{ITEMooVIYBooMXFaKb}]
		%-----------------------------------------------------------
		We have to check the conditions of definition \ref{DEFooVMWRooGQYJwl}.
		\begin{subproof}
			\spitem[Condition \ref{ITEMooUOXXooAzRrAk}]
			%-----------------------------------------------------------
			A generic element of \( T\) has the form
			\begin{equation}
				\xi=\sum_{ij}a_{ij} v_i\otimes w_j=\sum_{ij} (a_{ij}v_i)\otimes w_j
			\end{equation}
			with \( v_i\in V_x^{(1)}\) and \( w_j\in V_x^{(2)}\) for some \( x\in M\). There exists a \( \alpha\) such that \( x\in \varphi_{\alpha}(U_{\alpha})\). Let \( s\in U_{\alpha}\) such that \( x=\varphi_{\alpha}(s)\). There exist vectors \( v'_{ij}\in \eR^{k_1}\) such that
			\begin{equation}
				a_{ij}v_i=\sigma_{\alpha}^{1}(s,v'_{ij}),
			\end{equation}
			and \( w'_j\in \eR^{k_2}\) such that \( w_j=\sigma_{\alpha}^{2}(s,w'_j)\). Now we have
			\begin{subequations}
				\begin{align}
					\sigma_{\alpha}\big( s,r(\sum_{ij}v'_{ij}\otimes w'_j) \big) & =\sum_{ij}\sigma_{\alpha}^{(1)}(s,v'_{ij})\otimes \sigma_{\alpha}^{(2)}(s,w'_j) \\
					                                                             & = \sum_{ij}a_{ij}v_i\otimes w_j                                                 \\
					                                                             & =\xi.
				\end{align}
			\end{subequations}
			This proves that \( \xi\) belongs to \( \sigma_{\alpha}(U_{\alpha}\times \eR^{k_1k_2})\).
			\spitem[Condition \ref{ITEMooQHIYooBISPjL}]
			%-----------------------------------------------------------
			We have to prove that
			\begin{equation}
				\sigma_{\alpha}^{-1}\big( \sigma_{\alpha}(U_{\alpha}\times \eR^{k_1k_2})\cap  \sigma_{\beta}(U_{\beta}\times \eR^{k_1k_2}) \big)
			\end{equation}
			is open in \( U_{\alpha}\times \eR^{k_1k_2}\).

			First we prove that
			\begin{equation}		\label{EQooNWFCooWcxQxa}
				\sigma_{\alpha}(U_{\alpha}\times \eR^{k_1k_2})\cap \sigma_{\beta}(U_{\beta}\times \eR^{k_1k_2})=\bigcup_{x\in \varphi_{\alpha}(U_{\alpha})\cap \varphi_{\beta}(U_{\beta})}V_x^{(1)}\otimes V_x^{(2)}.
			\end{equation}
			\begin{subproof}
				\spitem[First inclusion]
				%-----------------------------------------------------------
				Let \( \xi\in \sigma_{\alpha}(U_{\alpha}\times \eR^{k_1k_2})\cap \sigma_{\beta}(U_{\beta}\times \eR^{k_1k_2})\). There exists \( s\in U_{\alpha}\) and \( v\in \eR^{k_1k_2}\) such that, using the condition \ref{DEFooFRPLooTxMUzg}\ref{ITEMooXJIVooVJTorK},
				\begin{equation}
					\xi=\sigma_{\alpha}(s,v)\in V_{\varphi_{\alpha}(s)}=V_{\varphi_{\alpha}(s)}^{(1)}\otimes V_{\varphi_{\alpha}(s)}^{(2)}.
				\end{equation}
				In the same way, there exists \( t\in U_{\beta}\) and \( w\in \eR^{k_1k_2}\) such that
				\begin{equation}
					\xi=\sigma_{\beta}(t,w)\in V_{\varphi_{\beta}(t)}=V_{\varphi_{\beta}(t)}^{(1)}\otimes V_{\varphi_{\beta}(t)}^{(2)}.
				\end{equation}
				We deduce that \( \varphi_{\alpha}(s)=\varphi_{\beta}(t)\), which is, by the way, \( \pi(\xi)\). Thus, setting \( x=\varphi_{\alpha}(s)\) we have
				\begin{equation}
					\xi\in V_x^{(1)}\otimes V_{x}^{(2)}.
				\end{equation}
				\spitem[Second inclusion]
				%-----------------------------------------------------------
				Let \( \xi\in V_x^{(1)}\otimes V_x^{(2)}\) with \( x\in \varphi_{\alpha}(U_{\alpha})\cap \varphi_{\beta}(U_{\beta})\). We can write \( \xi=\sum_{ij}a_{ij}v_i\otimes w_j\) with \( v_i\in V_x^{(1)}\) and \( w_j\in V_x^{(2)}\). Since the map \(\sigma_{\alpha}^{(1)} \colon U_{\alpha}\times \eR^{k_1}\to \bigcup_{x\in M}V_x^{(1)}  \) is a chart, there exist \( v'_i\in \eR^{k_1}\) such that
				\begin{equation}
					v_i=\sigma_{\alpha}^{(1)}\big( \varphi_{\alpha}^{-1}(x),v'_i \big).
				\end{equation}
				In the same way, there exists \( w'_{ij}\in \eR^{k_2}\) such that
				\begin{equation}
					a_{ij}w_j=\sigma_{\alpha}^{(2)}\big( \varphi_{\alpha}^{-1}(x),w'_{ij} \big).
				\end{equation}
				Few computations:
				\begin{subequations}
					\begin{align}
						\xi & =\sum_{ij}a_{ij}v_i\otimes w_j                                                                                                                     \\
						    & =\sum_{ij}v_i\otimes (a_{ij}w_j)                                                                                                                   \\
						    & =\sum_{ij}\sigma_{\alpha}^{(1)}\big( \varphi_{\alpha}^{-1}(x),v'_i \big)\otimes \sigma_{\alpha}^{(2)}\big( \varphi_{\alpha}^{-1}(x), w'_{ij} \big) \\
						    & =\sigma_{\alpha}\Big( \varphi_{\alpha}^{-1}(x),r\big( \sum_{ij}v'_i\otimes w'_{ij} \big) \Big)                                                     \\
						    & \in \sigma_{\alpha}(U_{\alpha}\times \eR^{k_1k_2}).
					\end{align}
				\end{subequations}
				In the same way we see that \( \xi\in \sigma_{\beta}(U_{\beta}\times \eR^{k_1k_2})\). Equation \eqref{EQooNWFCooWcxQxa} is proven.
			\end{subproof}
			Now let \( A\) be a part of \( U_{\alpha}\). We prove that
			\begin{equation}		\label{EQooJCUAooYxMMvC}
				\sigma_{\alpha}^{-1}\big( \bigcup_{x\in \varphi_{\alpha}(A)}V_x^{(1)}\otimes V_x^{(2)} \big)=A\times \eR^{k_1k_2}.
			\end{equation}
			\begin{subproof}
				\spitem[First inclusion]
				%-----------------------------------------------------------
				Let \( \xi\in V_x^{(1)}\otimes V_x^{(2)}\) and let \( s\in U_{\alpha}\) be such that \( x=\varphi_{\alpha}(s)\). We have \( \xi=\sum_{ij}a_{ij} v_i\otimes w_j\) with \( v_i= \sigma_{\alpha}^{(1)}(s,v'_i)\) and \( a_{ij}w_j=\sigma_{\alpha}^{(2)}(s,w'_{ij})\). Now we have
				\begin{subequations}
					\begin{align}
						\xi & =\sum_{ij}\sigma_{\alpha}^{(1)}(s,v'_i)\otimes \sigma_{\alpha}^{(2)}(s,w_{ij}') \\
						    & =\sigma_{\alpha}\big( s,r(\sum_{ij}v'_i\otimes w'_{ij}) \big)                   \\
						    & \in \sigma_{\alpha}\big( \varphi_{\alpha}^{-1}(x),\eR^{k_1k_2} \big).
					\end{align}
				\end{subequations}
				\spitem[Second inclusion]
				%-----------------------------------------------------------
				Let \( (s,\omega)\in A\times \eR^{k_1k_2}\). We have to prove that \( \sigma_{\alpha}(s,\omega)\in \bigcup_{x\in \varphi_{\alpha}(A)}V_x^{(1)}\otimes V_x^{(2)}\). For that, notice that \( \omega\in \eR^{k_1k_2}\) can be written as \( \omega=r\big( \sum_{ij}a_{ij}v_i\otimes w_j \big)\) for some \( v_i\in \eR^{k_1}\) and \( w_j\in \eR^{k_2}\). Now compute
				\begin{equation}
					\sigma_{\alpha}(s,\omega)=\sum_{ij}a_{ij}\sigma_{\alpha}^{(1)}(s,v_i)\otimes \sigma_{\alpha}^{(2))}(s,w_j).
				\end{equation}
				Since \( \sigma_{\alpha}^{(1)}(s,v_i)\in V_{\varphi_{\alpha}(s)}\) and \( \sigma_{\alpha}^{(2)}(s,w_j)\in V_{\varphi_{\alpha}(s)}^{(2)}\) we found that
				\begin{equation}
					\sigma_{\alpha}(s,\omega)\in V_{\varphi_{\alpha}(s)}^{(1)}\otimes V_{\varphi_{\alpha}(s)}^{(2)}
				\end{equation}
				The equality \eqref{EQooJCUAooYxMMvC} is proved.
			\end{subproof}

			Recall what we have in equations \eqref{EQooNWFCooWcxQxa} and \eqref{EQooJCUAooYxMMvC} :
			\begin{subequations}
				\begin{align}
					\sigma_{\alpha}(U_{\alpha}\times \eR^{k_1k_2}) & \cap \sigma_{\beta}(U_{\beta}\times \eR^{k_1k_2})                                                                                                                                                  \\
					                                               & =\bigcup_{x\in \varphi_{\alpha}(U_{\alpha})\cap \varphi_{\beta}(U_{\beta})}V_x^{(1)}\otimes V_x^{(2)}                                                                                              \\
					                                               & = \Big( \bigcup_{x\in \varphi_{\alpha}(U_{\alpha})}V_x^{(1)}\otimes V_x^{(2)} \Big)\cap\Big( \bigcup_{x\in \varphi_{\beta}(U_{\beta})}V_x^{(1)}\otimes V_x^{(2)}  \Big)		\label{SUBEQooPFDVooPuzjYl}
				\end{align}
			\end{subequations}
			and
			\begin{equation}		\label{EQooSYTEooEfyxPi}
				\sigma_{\alpha}^{-1}\big( \bigcup_{x\in \varphi_{\alpha}(A)}V_x^{(1)}\otimes V_x^{(2)} \big)    =A\times \eR^{k_1k_2}.
			\end{equation}
			The map \( \sigma_{\alpha}^{-1}\) is injective, so that we can use proposition \ref{PROPooBXVSooZXmwKC}. Now we make the final computation:
			\begin{subequations}
				\begin{align}
					 & \sigma_{\alpha}^{-1}\big( \sigma_{\alpha}(U_{\alpha}\times \eR^{k_1k_2})  \big)                                                                                                                                                                                    \\
					 & = \sigma_{\alpha}^{-1}\Big(  \bigcup_{x\in \varphi_{\alpha}(U_{\alpha})}V_x^{(1)}\otimes V_x^{(2)}  \Big) \cap   \sigma_{\alpha}^{-1}\Big(  \bigcup_{x\in \varphi_{\beta}(U_{\beta})}V_x^{(1)}\otimes V_x^{(2)}  \Big) & \text{by eq. \eqref{SUBEQooPFDVooPuzjYl}} \\
					 & = \big( U_{\alpha}\times \eR^{k_1k_2} \big) \cap \big( U_{\beta}\times \eR^{k_1k_2} \big)                                                                                                                              & \text{by eq. \eqref{EQooSYTEooEfyxPi}}    \\
					 & =\big( U_{\alpha}\cap U_{\beta}\big)\times \eR^{k_1k_2}.
				\end{align}
			\end{subequations}
			We know that \( \big( U_{\alpha}\cap U_{\beta}\big)\times \eR^{k_1k_2}\) is open since \( \varphi_{\alpha}\) and \( \varphi_{\beta}\) are charts of \( M\).
			\spitem[Condition \ref{ITEMooHICSooDrPwuV}]
			%-----------------------------------------------------------
			We have to prove that
			\begin{equation}
				\sigma_{\beta}^{-1}\circ\sigma_{\alpha}  \colon  \sigma_{\alpha}^{-1}\Big( \sigma_{\alpha}(U_{\alpha}\times \eR^{k_1k_2})\cap \sigma_{\beta}(U_{\beta}\times \eR^{k_1k_2}) \Big)      \to  U_{\beta}\times \eR^{k_1k_2}
			\end{equation}
			is smooth.

			\begin{subproof}
				\spitem[First step]
				%-----------------------------------------------------------

				We have :
				\begin{equation}		\label{EQooNROCooARFfPa}
					(\sigma_{\beta}^{-1}\circ \sigma_{\alpha})\Big( s,r\big( \sum_{ij}a_{ij}v_i\otimes w_j \big)\Big)
					=\sigma_{\beta}^{-1}\Big( \sum_{ij}a_{ij}\sigma_{\alpha}^{(1)}(s,v_i)\otimes \sigma_{\alpha}^{(2)}(s,w_j) \Big).
				\end{equation}

				\spitem[Notation \( A_1\) and \( A_2\)]
				%-----------------------------------------------------------
				Small point about the notations. When we have a map \(A \colon U_{\alpha}\times \eR^{k_1}\to U_{\beta}\times \eR^{k_2}  \), we write \( A_1(s,v)\) the \( U_{\beta}\) component of \( A(s,v)\) and \( A_2(s,v)\) the component in \( \eR^k_2\).

				With that notation we have
				\begin{equation}		\label{EQooSBLDooRxPKQR}
					(\sigma_{\beta}^{(1)-1}\circ\sigma_{\alpha}^{(1)})_1(s,v)=(\varphi_{\beta}^{-1}\circ \varphi_{\alpha})(s).
				\end{equation}
				Indeed, by definition \ref{DEFooFRPLooTxMUzg}\ref{ITEMooOBCMooOCqoQo}, we have \( \sigma_{\alpha}^{(1)}(s,v)\in V_{\varphi_{\alpha}(s)}^{(1)}\), and by the same point,
				\begin{equation}
					\sigma_{\beta}^{(1)-1}\big( V_{\varphi_{\alpha}(s)} \big)=\big( (\varphi_{\beta}^{-1}\circ\varphi_{\alpha})(s), \eR^{k_1} \big).
				\end{equation}
				The first component is \( (\varphi_{\beta}^{-1}\circ \varphi_{\alpha})(s)\) as expected.

				\spitem[Next step]
				%-----------------------------------------------------------
				We continue \eqref{EQooNROCooARFfPa} by proving
				\begin{equation}		\label{EQooGVGEooRbGZUd}
					\begin{aligned}[]
						 & \sigma_{\beta}^{-1}\Big( \sum_{ij}a_{ij}\sigma_{\alpha}^{(1)}(s,v_i)\otimes \sigma_{\alpha}^{(2)}(s,w_j) \Big)                                  \\
						 & \quad = \Big( (\varphi_{\beta}^{-1}\circ\varphi_{\alpha})(s),r\big( \sum_{ij}a_{ij}(\sigma_{\beta}^{(1)-1}\circ \sigma_{\alpha}^{(1)})_2(s,v_i)
						\otimes
						(\sigma_{\beta}^{(2)-1}\circ\sigma_{\alpha}^{(2)})_2(s,w_j)
						\big) \Big).
					\end{aligned}
				\end{equation}

				Notice that, setting \( x=(\varphi_{\beta}^{-1}\circ\varphi_{\alpha})(s)\), the equation \eqref{EQooSBLDooRxPKQR} says that \(   (\sigma_{\beta}^{(1)-1}\circ\sigma_{\alpha}^{(1)})_1(s,v_i)  =x   \), so that
				\begin{equation}		\label{EQooZEBYooXtCaUP}
					\big( x,  (\sigma_{\beta}^{(1)-1}\circ\sigma_{\alpha}^{(1)})_2(s,v_i) \big) = (\sigma_{\beta}^{(1)-1}\circ\sigma_{\alpha}^{(1)})(s,v_i).
				\end{equation}

				Now we prove \eqref{EQooGVGEooRbGZUd} by applying \( \sigma_{\beta}\) to the right hand side :
				\begin{subequations}
					\begin{align}
						                                                                                            & \sigma_{\beta}  \Big( x,r\big( \sum_{ij}a_{ij}(\sigma_{\beta}^{(1)-1}\circ \sigma_{\alpha}^{(1)})_2(s,v_i)
						\otimes
						(\sigma_{\beta}^{(2)-1}\circ\sigma_{\alpha}^{(2)})_2(s,w_j)
						\big) \Big)                                                                                                                                                                                              \\
						                                                                                            & \quad  = \sum_{ij}a_{ij}
						\sigma_{\beta}^{(1)}\big( x,  (\sigma_{\beta}^{(1)-1}\circ\sigma_{\alpha}^{(1)})_2(s,v_i) \big)
						\otimes
						\sigma_{\beta}^{(2)}\big( x,  (\sigma_{\beta}^{(2)-1}\circ\sigma_{\alpha}^{(2)})_2(s,w_i) \big)                                                                                                          \\
						                                                                                            & \quad  = \sum_{ij}a_{ij}
						\sigma_{\beta}^{(1)}\big(   (\sigma_{\beta}^{(1)-1}\circ\sigma_{\alpha}^{(1)})(s,v_i) \big)
						\otimes
						\sigma_{\beta}^{(2)}\big(   (\sigma_{\beta}^{(2)-1}\circ\sigma_{\alpha}^{(2)})(s,w_i) \big) & \text{eq. \eqref{EQooZEBYooXtCaUP}}                                                                        \\
						                                                                                            & \quad = \sum_{ij}a_{ij}\sigma_{\alpha}^{(1)}(s,v_i)\otimes \sigma_{\alpha}^{(2)}(s,w_j).
					\end{align}
				\end{subequations}
				\spitem[Conclusion]
				%-----------------------------------------------------------
				We have
				\begin{subequations}		\label{SUBEQSooRYKSooBkBmaF}
					\begin{align}
						 & (\sigma_{\beta}^{-1}\circ\sigma_{\alpha})\Big( s,r\big( \sum_{ij}a_{ij}v_i\otimes w_j \big) \Big)                                                                                                                               \\
						 & \quad =\sigma_{\beta}^{-1}\Big( \sum_{ij}a_{ij}   \sigma_{\alpha}^{(1)}(s,v_i)\otimes \sigma_{\alpha}^{(2)}(s,w_j)  \Big)                                                                                                       \\
						 & \quad = \Big( (\varphi_{\beta}^{-1}\circ\varphi_{\alpha})(s),r\big[  \sum_{ij}a_{ij}(\sigma_{\beta}^{(2)-1}\circ\sigma_{\alpha}^{(1)})_2(s,v_i)\otimes (\sigma_{\beta}^{(1)-1}\circ\sigma_{\alpha}^{(2)})_2(s,w_j) \big] \Big).
					\end{align}
				\end{subequations}
				Since the maps \( \varphi_{\alpha}\), \( \varphi_{\beta}\) are charts of \( M\), the composition \( \varphi_{\beta}^{-1}\circ\varphi_{\alpha}\) is smooth. The same way, the maps \( \sigma_{\alpha}^{(i)}\) and \( \sigma_{\beta}^{(i)} \) are charts of \( \bigcup_{x\in M}V_x^{(i)}\), the composition \( \sigma_{\beta}^{(i)-1}\circ \sigma_{\alpha}^(i)\) is smooth.
			\end{subproof}
		\end{subproof}
		\spitem[Condition \ref{ITEMooXHOBooERubro}]
		%-----------------------------------------------------------
		We have to prove that\footnote{Here we use the fact that the spaces \( V_x^{(i)}\) have the form \( \{ (x,v)\tq v\in W_x^{(i)} \}\), so that when we write \( v_i\otimes w_j\), we could have written \( (x,v_i)\otimes (x,w_j)\). Just keep in mind that the base point \( x\) is contained in the vector \( v_i\) in the following formula.}
		\begin{equation}
			\begin{aligned}
				\pi\colon \bigcup_{x\in M}V_x^{(1)}\otimes V_x^{(2)} & \to M     \\
				\sum_{ij}a_{ij}v_i\otimes w_j                        & \mapsto x
			\end{aligned}
		\end{equation}
		is smooth. In other words, for each \( \alpha\) and \( \beta\), the map
		\begin{equation}
			\varphi_{\beta}^{-1}\circ \pi\circ \sigma_{\alpha} \colon U_{\alpha}\to U_{\beta}
		\end{equation}
		must be smooth. We have
		\begin{subequations}
			\begin{align}
				(\varphi_{\beta}^{-1}\circ\pi\circ\sigma_{\alpha}) & \big( s,r(\sum_{ij}a_{ij}v_i\otimes w_j) \big)                                                                                                                        \\
				                                                   & =(\varphi_{\beta}^{-1}\circ\pi)\Big( \underbrace{\sum_{ij}a_{ij}\sigma_{\alpha}^{(1)}(s,v_i)\otimes \sigma_{\alpha}^{(2)}(s,w_j)}_{\in V_{\varphi_{\alpha}(s)}} \Big) \\
				                                                   & =(\varphi_{\beta}^{-1}\circ\varphi_{\alpha})(s).
			\end{align}
		\end{subequations}
		So we have
		\begin{equation}
			(\varphi_{\beta}^{-1}\circ\pi\circ\sigma_{\alpha})(s,\xi)=(\varphi_{\beta}^{-1}\circ\varphi_{\alpha})(s),
		\end{equation}
		which leads us to \( \varphi_{\beta}^{-1}\circ\pi\circ\sigma_{\alpha}=\varphi_{\beta}^{-1}\circ\varphi_{\alpha}\circ\pr\) where \(\pr \colon \eR^n\times \eR^{k_1k_2}\to \eR^n  \) is the projection. Since \( \pr\) is smooth and \( \varphi_{\beta}^{-1}\circ\varphi_{\alpha}\) is smooth, we conclude that \( \varphi_{\beta}^{-1}\circ\pi\circ\sigma_{\alpha}\) is smooth.
		\spitem[Condition \ref{ITEMooXJIVooVJTorK}]
		%-----------------------------------------------------------
		Since \( \sigma_{\alpha}^{(i)}\) is a chart for the vector bundle \( \bigcup_{x\in M}V_x^{(i)}\) we have \( \sigma_{\alpha}^{i}(s,v)\in V_{\varphi_{\alpha}(s)}\). Thus for right hand side of \eqref{EQooXHURooBpOMlU} is an element of \( V_{\varphi_{\alpha}(s)}^{(1)}\otimes V_{\varphi_{\alpha}(s)}^{(2)}=V_{\varphi_{\alpha}(s)}\).
		\spitem[Condition \ref{ITEMooOBCMooOCqoQo}]
		%-----------------------------------------------------------
		We fix \( s\in U_{\alpha}\). We have to prove that the map
		\begin{equation}
			\begin{aligned}
				\tau_{\alpha,s}\colon \eR^{k_1k_2} & \to V_{\varphi_{\alpha}(s)}    \\
				\xi                                & \mapsto \sigma_{\alpha}(s,\xi)
			\end{aligned}
		\end{equation}
		is a linear bijection. If \( \xi=r\big( \sum_{ij}a_{ij}v_i\otimes w_j \big)\),
		\begin{equation}
			\tau_{\alpha,s}(\xi)=\sum_{ij}a_{ij}\sigma_{\alpha}^{(1)}(s,v_i)\otimes \sigma_{\alpha}^{(2)}(s,w_j).
		\end{equation}
		\begin{subproof}
			\spitem[Linear]
			%-----------------------------------------------------------
			For the linearity, for \( \lambda\in \eR\) we have
			\begin{subequations}
				\begin{align}
					\tau_{\alpha,s}(\lambda\xi) & =\tau_{\alpha,s}\Big( r\big( \sum_{ij}a_{ij}(\lambda v_i)\otimes w_j \big) \Big)         \\
					                            & =\sum_{ij}a_{ij}\sigma_{\alpha}^{(1)}(s,\lambda v_i)\otimes \sigma_{\alpha}^{(2)}(s,w_j) \\
					                            & =\lambda\sum_{ij}\sigma_{\alpha}^{(1)}(s,v_i)\otimes \sigma_{\alpha}^{(2)}(s,w_j)        \\
					                            & =\lambda\tau_{\alpha,s}(\xi)
				\end{align}
			\end{subequations}
			because \( \sigma_{\alpha}^{(1)}(s,\lambda v_i)=\lambda\sigma_{\alpha}^{(1)}(s,v_i)\). The same kind of computations hold to prove that \( \tau_{\alpha,s}(\xi_1+\xi_2)=\tau_{\sigma,s}(\xi_1)+\tau_{\alpha,s}(\xi_2)\).
			\spitem[Injective]
			%-----------------------------------------------------------
			Suppose \( \tau_{\alpha,s}(\xi)=0\). Let \( \{e_i\}\) be a basis of \( V\) and \( \{e'_j\}\) be a basis of \( W\). By proposition \ref{PROPooTHDPooWgjUwk}\ref{ITEMooQCILooUncdGl}, the set \( \{e_i\otimes e'_j\}\) is a basis of \( V\otimes W\). We write \( \xi=r\big( \sum_{ij}a_{ij}e_i\otimes e'_j \big)\), so that
			\begin{equation}		\label{EQooVGWPooTRDuJW}
				\sum_{ij}a_{ij}\sigma_{\alpha}^{(1)}(s,e_i)\otimes \sigma_{\alpha}^{(2)}(s,e'_j)=0.
			\end{equation}
			Since \( v\mapsto \sigma_{\alpha}^{(i)}(s,v)\) is a linear bijection, the set \( \{ \sigma_{\alpha}^{(i)}(s,v) \}\) is a basis of \( V_{\varphi_{\alpha}(s)}^{(i)}\). We conclude that the set
			\begin{equation}
				\{ \sigma_{\alpha}^{(1)}(s,e_i)\otimes \sigma_{\alpha}^{(2)}(s,e'_j) \}
			\end{equation}
			is a basis of \( V_{\varphi_{\alpha}(s)}=V_{\varphi_{\alpha}(s)}^{(1)}\otimes V_{\varphi_{\alpha}(s)}^{(2)}\). Thus equation \eqref{EQooVGWPooTRDuJW} implies that \( a_{ij}=0\) for every \( i,j\). We conclude that \( \xi=0\) and that \( \tau_{\alpha,s}\) is injective.

			\spitem[Surjective]
			%-----------------------------------------------------------
			The set \(			\{ \sigma_{\alpha}^{(1)}(s,e_i)\otimes \sigma_{\alpha}^{(2)}(s,e'_j) \}  \) is a basis and is in the image of \( \tau_{\alpha,s}\) which is linear. Thus \( \tau_{\alpha,s}\) is surjective.
		\end{subproof}
		\spitem[Condition \ref{ITEMooPNLXooLQWGqc}]
		%-----------------------------------------------------------
		Two points to be proven.
		\begin{subproof}
			\spitem[\( g_{\alpha\beta}(s)\in \GL(\eR,k_1k_2)\)]
			%-----------------------------------------------------------
			The condition
			\begin{equation}
				(\sigma_{\alpha}^{-1}\circ\sigma_{\beta})(s,v)=\big( s,g_{\alpha\beta}(s)v \big)
			\end{equation}
			for every \( v\in \eR^{k_1k_2}\) says
			\begin{equation}
				g_{\alpha\beta}(s)v=(\tau_{\alpha,s}^{-1}\circ\tau_{\alpha,s})(v).
			\end{equation}
			Thus \( g_{\alpha\beta}(s)\) is linear and invertible. This means that for every \( s\), the map \( g_{\alpha\beta}(s)\) belongs to \(\GL(\eR, k_1k_2)\).
			\spitem[\( g_{\alpha\beta}\) is smooth]
			%-----------------------------------------------------------
			We are going to prove that
			\begin{equation}\label{EQooAUMXooCoqPTO}
				g_{\alpha\beta}(s)=r\circ\big( g_{\alpha\beta}^{(1)}(s)\otimes g_{\alpha\beta}^{(2)}(s) \big)\circ r^{-1}.
			\end{equation}
			Let \( \{ e_i \}_{i\in I}\) be a basis of \( \eR^k_1\) and \( \{ e'_j \}_{j\in J}\) be a basis of \( \eR^{k_2}\). We consider \( v=r(e_i\otimes e'_j)\) and we compute
			\begin{subequations}		\label{EQSooUIMFooMBbBvp}
				\begin{align}
					\sigma_{\alpha}\Big( s,  r\circ\big( g_{\alpha\beta}^{(1)}(s)\otimes g_{\alpha\beta}^{(2)}(s) \big)\circ r^{-1}v \Big)
					 & =\sigma_{\alpha}\Big(  s, r\circ\big( g_{\alpha\beta}^{(1)}(s)\otimes g_{\alpha\beta}^{(2)}(s) \big)(e_i\otimes e'_j)   \Big)                                                          \\
					 & =\sigma_{\alpha}\Big( s,r\big( g_{\alpha\beta}^{(1)}(s)e_i\otimes g_{\alpha\beta}^{(2)}(s)e'_j \big) \Big)                                                                             \\
					 & = \sigma_{\alpha}^{(1)}\big( s,g_{\alpha\beta}^{(1)}(s)e_i \big)\otimes \sigma_{\alpha}^{(2)}\big( s,g_{\alpha\beta}^{(2)}e'_j \big)                                                   \\
					 & = \sigma_{\beta}^{(1)}(s,e_i)\otimes \sigma_{\beta}^{(2)}(s,e'_j)                                                                    & \text{cf. justif.}  \label{SUBEQooSZDEooLQBGoQ} \\
					 & =\sigma_{\beta}\big( s,r(e_i\otimes e'_j) \big)                                                                                                                                        \\
					 & =\sigma_{\beta}\big( s,r(v) \big).
				\end{align}
			\end{subequations}
			Justifications.
			\begin{itemize}
				\item
				      For \eqref{SUBEQooSZDEooLQBGoQ}. We have \( \big( s,g_{\alpha\beta}^{(1)}(s)e_i \big)=\big( \sigma_{\alpha}^{(1)-1}\circ \sigma_{\beta} \big)(s,e_i)\), and similar for \( e'_j\).
			\end{itemize}
			Putting \( \sigma_{\alpha}\) on the right hand side the equalities \eqref{EQSooUIMFooMBbBvp} show that
			\begin{equation}
				(\sigma_{\alpha}^{-1}\circ \sigma_{\beta})\big( s,r(v) \big)=\Big( s,r\circ\big( g_{\alpha\beta}^{(1)}(s)\otimes g_{\alpha\beta}^{(2)}(s) \big)r^{-1}v \Big).
			\end{equation}
			The left hand side can be expressed in terms og \( g_{\alpha\beta}(s)\). We have
			\begin{equation}
				\big( g_{\alpha\beta}(s)\circ r \big)(e_i\otimes e'_j)=r\circ \big( g_{\alpha\beta}^{(1)}(s)\otimes g_{\alpha\beta}^{(2)}(s) \big)(e_i\otimes e'_j).
			\end{equation}
			Since the operators \( g_{\alpha\beta}(s)\circ r\) and \( r\circ\big(   g_{\alpha\beta}^{(1)}(s)\otimes g_{\alpha\beta}^{(2)}(s) \big)\) are linear and since the elements \( e_i\otimes e'_j\) form a basis we have the operator equality
			\begin{equation}
				g_{\alpha\beta}(s)\circ r=r\circ \big( g_{\alpha\beta}^{(1)}(s)\otimes g_{\alpha\beta}^{(2)}(s) \big),
			\end{equation}
			which is the announced equality \eqref{EQooAUMXooCoqPTO}.
			\spitem[Conclusion]
			%-----------------------------------------------------------
			The maps \( g_{\alpha\beta}^(1)\) and \( g_{\alpha\beta}^{(2)}\) are smooths by hypothesis. By proposition \ref{PROPooLANVooKPiLuu}, the tensor product is smooth. The maps \( r\) and \( r^{-1}\) are linear on finite dimensional vector spaces; they are smooth too.
		\end{subproof}
	\end{subproof}
\end{proof}

\begin{proposition}[\cite{MonCerveau}]		\label{PROPooEYSWooOeQNyX}
	The tensor product\footnote{Definition \ref{DEFooCSDZooJuzGuE}.} of vector bundles is associative : if \( E_1\), \( E_2\) and \( E_3\) are vector bundles, then
	\begin{equation}
		(E_1\otimes E_2)\otimes E_3=E_1\otimes (E_2\otimes E_3).
	\end{equation}
	%TODOooTXYEooCxyiwi Je dois définir la notion d'équivalence, et le prouver.
\end{proposition}


%+++++++++++++++++++++++++++++++++++++++++++++++++++++++
\section{Tensor bundle}
%+++++++++++++++++++++++++++++++++++++++++++++++++++++++

\begin{definition}
	Let \( (k,l)\) be integers. The \defe{\( k,l\)-tensor bundle}{tensor bundle} is
	\begin{equation}
		\bigotimes^{k,l}(TM)=(TM)^{\otimes k}\otimes (T^*M)^{\otimes l}.
	\end{equation}
	A \defe{\( (k,l)\)-tensor field}{tensor field} is a smooth section of \( \bigotimes^{k,l}(TM)\).
\end{definition}

\begin{proposition}		\label{PROPooIMOJooKzEDXA}
	The set \( TM\) becomes a vector bundle with the charts
	\begin{equation}
		\begin{aligned}
			\sigma_{\alpha}\colon U_{\alpha}\times \eR^n & \to TM                                                              \\
			(s,v)                                        & \mapsto  \frac{d}{dt} \left[ \varphi_{\alpha}(s+tv)  \right]_{t=0}.
		\end{aligned}
	\end{equation}
\end{proposition}

\noproof


%-------------------------------------------------------
\subsection{Dual basis}
%----------------------------------------------------


The proposition \ref{PROPooAAAXooKAMsfK} says that the vectors
\begin{equation}
	\partial_i=\frac{d}{dt} \left[ \varphi_{\alpha}(s_0+te_i)  \right]_{t=0}
\end{equation}
form a basis of \( T_pM\) where \( p=\varphi_{\alpha}(s_0)\). We denote by \( \{ \partial_{i}^* \}\) the dual basis.

\begin{normaltext}
	When \( I=(i_1,\ldots,i_k)\) is a multiindex, we write
	\begin{equation}
		\partial^*_{\wedge I}=\partial_{i_1}\wedge\ldots\wedge\partial_{i_k}.
	\end{equation}
\end{normaltext}

\begin{normaltext}
	The object \( \partial^*_{\wedge I}\) is build from the elements of the basis \( \partial_i^*\) which is the dual of the basis
	\begin{equation}
		\partial_i=\frac{d}{dt} \left[ \varphi_{\alpha}(s+te_i)  \right]_{t=0}.
	\end{equation}
	Thus \( \partial^*_{\wedge I}\) is a function of \( s\).
\end{normaltext}

\begin{lemma}[\cite{MonCerveau}]		\label{LEMooKCBSooPjEwEl}
	Let \( (U,\varphi)\) be a chart of a smooth manifold. We have the local expression
	\begin{equation}
		\partial^*_i=df_i
	\end{equation}
	where \( f_i\) is the function
	\begin{equation}
		\begin{aligned}
			f_i\colon M & \to \eR                    \\
			q           & \mapsto \varphi^{-1}(q)_i.
		\end{aligned}
	\end{equation}
\end{lemma}

\begin{proof}
	Let \( q=\varphi(x)\in M\). At \( q\) we have \( \partial_j=\frac{d}{dt} \left[ \varphi(x+te_j)  \right]_{t=0}\) and, using the definition \( f_i=\proj_i\circ\varphi^{-1}\), we have
	\begin{subequations}
		\begin{align}
			df_i(\partial_j) & =\frac{d}{dt} \left[ f_i\big( \varphi(x+te_j) \big)  \right]_{t=0}                 \\
			                 & =\frac{d}{dt} \left[ (\proj_i\circ\varphi^{-1}\circ\varphi)(x+te_j)  \right]_{t=0} \\
			                 & =\frac{d}{dt} \left[ \proj_i(x+te_j)  \right]_{t=0}                                \\
			                 & =\frac{d}{dt} \left[ x_i+t\delta_{ij}  \right]_{t=0}                               \\
			                 & =\delta_{ij},
		\end{align}
	\end{subequations}
	which proves that \( df_i=\partial^*_i\).
\end{proof}

\begin{lemma}[\cite{MonCerveau}]  \label{LEMooGEFSooVlPLOs}
	Let \( f\) be a smooth function over the manifold \( M\). We consider a chart \( (U,\varphi)\). The differential of \( f\) is
	\begin{equation}
		df=\sum_i(\partial_if)\partial^*_i.
	\end{equation}
\end{lemma}

\begin{proof}
	Let \( X\) be the vector \( X=\sum_i\partial_i\). Since \( \{ \partial^*_i \}\) is the dual basis, we express the components of \( X\) under the form \( X_i=\partial_i^*(X)\). Now, by definition,
	\begin{equation}
		df(X)=X(f)=\sum_iX_i\partial_if=\sum_i(\partial_if)\partial_i^*(X).
	\end{equation}
	Since it is valid for every vector \( X\), we have the equality.
\end{proof}

\begin{lemma}		\label{LEMooIAASooDnnhRA}
	The map
	\begin{equation}
		\partial^*_{\wedge I}\colon U_{\alpha} \to \Wedge^k(T^*M)
	\end{equation}
	is smooth and satisfy
	\begin{equation}
		\partial_{\wedge I}^*(s)\in\Wedge^k\big( T^*_{\varphi_{\alpha}(s)}M \big).
	\end{equation}
\end{lemma}

\begin{normaltext}		\label{NORMooDQEZooHaJPoa}
	Let \(\varphi_{\alpha} \colon U_{\alpha}\to M  \) be a chart. We introduce two maps to make the link between the real world and the manifold world. One for the forms :
	\begin{equation}
		\begin{aligned}
			\tau_{\alpha,s}\colon \Wedge^k(\eR^n)^* & \to \Wedge^kT^*_{\varphi_{\alpha}(s)}M                \\
			\sum_{I\in C_k}\omega_Ie^*_{\wedge I}   & \mapsto \sum_{I\in C_k}\omega_I\partial^*_{\wedge I}.
		\end{aligned}
	\end{equation}
	An one for the vectors:
	\begin{equation}
		\begin{aligned}
			\tau_{\alpha,s}\colon \eR^n & \to T_{\varphi_{\alpha}(s)}M \\
			v                           & \mapsto \sum_iv_i\partial_i.
		\end{aligned}
	\end{equation}
	These two maps have the same name «\( \tau_{\alpha,s}\)», but their arguments are different enough not to cause confusion.
\end{normaltext}


\begin{propositionDef}[Exterior bundle]		\label{DEFooZELVooFfosEn}
	Let \( M\) be a \( n\)-dimensional manifold. We write\footnote{This provides a basis of \( \Wedge^kT^*m\), see proposition \ref{PROPooUGLOooTULnDK}.}
	\begin{equation}
		C_k=\{ (i_1,\ldots,i_k)\tq 1\leq i_1<\ldots <i_k\leq n \}.
	\end{equation}
	The set \( \Wedge^kT^*M\) is a vector bundle with the charts
	\begin{equation}		\label{EQooOMWMooKKWhqf}
		\begin{aligned}
			\sigma_{\alpha}\colon U_{\alpha}\times \Wedge^k(\eR^n)^* & \to \Wedge^kT^*M                                                                       \\
			(s,\sum_{I\in C_k}\omega_Ie^*_{\wedge I})                & \mapsto \Big( \varphi_{\alpha}(s),  \sum_{I\in C_k}\omega_I\partial^*_{\wedge I}\Big).
		\end{aligned}
	\end{equation}
	In other words,
	\begin{equation}
		\sigma_{\alpha}(s,\omega')=\big( \varphi_{\alpha}(s),\tau_{\alpha,s}(\omega') \big).
	\end{equation}
	Two notes.
	\begin{enumerate}
		\item
		      There is an hidden choice of isomorphism between \( \eR^m\) and \( \Wedge^k\eR^n\) with \( m=\binom{ n }{ k }\).
		\item
		      The vector space \( V_{\varphi_{\alpha}(s)}\) is the set of elements of the form \( \big( \varphi_{\alpha}(s),\omega \big)\) with \( \omega\in \Wedge^kT^*_{\varphi_{\alpha}(s)}M\). The purpose of that is to make explicit the fact that all the vector spaces \( V_x\) are distinct. See the point \ref{ITEMooIVKKooNEfxVi} of the definition \ref{DEFooFRPLooTxMUzg} of a vector bundle.
	\end{enumerate}
\end{propositionDef}

\noproof


%+++++++++++++++++++++++++++++++++++++++++++++++++++++++
\section{Differential form}
%+++++++++++++++++++++++++++++++++++++++++++++++++++++++

\begin{definition}		\label{DEFooBRSSooWOgoov}
	A \defe{\( k\)-differential-form}{differential form} is a section of \( \Wedge^kT^*M\). We denote by \( \Omega^k(M)\) the set of \( k\)-differential forms.

	When the context is clear, we could simply say "\( k\)-form" instead of \( k\)-differential form.
\end{definition}

\begin{normaltext}
	Formally, a \( k\)-differential form \( \omega\) is a smooth map
	\begin{equation}
		\omega \colon M\to M\times \Wedge^kT^*M
	\end{equation}
	where the \( M\)-component of \( \omega\) is the identity.

	We will, however, often make the abuse of notation and drop that \( M\)-component.
\end{normaltext}

\begin{propositionDef}		\label{PROPooQOUXooZfFHNb}
	If \( \omega\in \Omega^k(M)\) and \( \eta\in \Omega^l(M)\) we define \( \omega\wedge \eta\) pointwise by
	\begin{equation}
		(\omega\wedge \eta)_p=\omega_p\wedge \eta_p
	\end{equation}
	where the exterior product is defined by \ref{DEFooCTSPooZRIufr}.

	We have \( \omega\wedge\eta\in\Omega^{k+l}(M)\).
\end{propositionDef}

\begin{lemma}[\cite{MonCerveau}]		\label{LEMooUVRQooVZDKir}
	If \( f\) is a function and \( \omega\) is a \( k\)-differential form we have
	\begin{equation}		\label{EQooYHCKooNLgJWG}
		f\wedge \omega=f\omega.
	\end{equation}
	where, on the left hand side, \( f\) is seen as a \( 0\)-differential form while, on the right hand side this is the pointwise product.
\end{lemma}

\begin{proof}
	We compute the left-hand side of \eqref{EQooYHCKooNLgJWG} at \( x\in M\). From definition \ref{PROPooQOUXooZfFHNb},
	\begin{equation}
		(f\wedge \omega)_x=f(x)\wedge \omega_x.
	\end{equation}
	Now we are reduced to check \( (\lambda\otimes \omega)=\lambda\omega\) for \( \lambda\in \eR\) and \( \omega\in \Wedge^k(V^*)\) for some vector space \( V\). Using the definitions \ref{DEFooCTSPooZRIufr} and \ref{DEFooHAKAooIfbsEy},
	\begin{subequations}
		\begin{align}
			\lambda\wedge \omega & =\frac{ (0+k)! }{ 0!k! }\Alt(\lambda\otimes \omega)                    \\
			                     & =\Alt(\lambda\otimes \omega)                                           \\
			                     & =\frac{1}{ k!}\sum_{\pi\in S_k}(-1)^{\pi}(\lambda\otimes\omega)^{\pi}.
		\end{align}
	\end{subequations}
	Each term in the sum is subject to the relation \( (\lambda\otimes \omega)(v_1,\ldots,v_k)=\lambda\omega(v_1,\ldots,v_k)\).
\end{proof}

\begin{proposition}		\label{PROPooNOGKooQedJba}
	Let \( \omega\in \Omega^k(M)\). There exist functions \( \omega_I\in C^{\infty}(M)\) for each
	\begin{equation}
		I\in C_k=\{ (i_1,\ldots,i_k)\tq 1\leq i_1<\ldots <i_k\leq n \}
	\end{equation}
	such that
	\begin{equation}
		\omega_x=\sum_{I\in C_k}\omega_I(x)\partial^*_{\wedge I}.
	\end{equation}
	Notes:
	\begin{itemize}
		\item
		      When \( I=(i_1,\ldots,i_k)\), the notation \( a_{\wedge I}\) means \( a_{i_1}\wedge \ldots\wedge a_{i_k}\).
		\item
		      Even if we don't write it explicitly, \( \partial^*_i\) is a vector field, and then a function of \( x\).
	\end{itemize}
\end{proposition}

\noproof


\begin{proposition}			\label{PROPooHTKMooVzYQdW}
	We consider a \( k\)-form \( \omega\in \Omega^k(M)\) and smooth vector fields \( X_1,\ldots,X_k\). The map
	\begin{equation}
		\begin{aligned}
			\omega(X_1,\ldots,X_k)\colon M & \to \eR                                            \\
			x                              & \mapsto \omega_x\big( (X_1)_x,\ldots,(X_k)_x \big)
		\end{aligned}
	\end{equation}
	is smooth.
\end{proposition}

\begin{proof}

	We have to prove that \( \omega(X_1,\ldots,X_k)\circ \varphi_{\alpha}\) is smooth as function from \( U_{\alpha}\) to \( \eR\). Using proposition \ref{PROPooNOGKooQedJba}, we have smooth functions \( \omega_I\in C^{\infty(M,\eR)}\) such that
	\begin{equation}
		\omega_x=\sum_{i\in C_k}\omega_I(x)\partial^*_{\wedge I}.
	\end{equation}

	From proposition \ref{PROPooXURIooYPytwa}, there exist smooth functions \( X_{ij}\in C^{\infty(M)}\) such that
	\begin{equation}
		X_i(x)=\sum_jX_{ij}(x)\partial_j.
	\end{equation}


	We have
	\begin{subequations}
		\begin{align}
			\omega(X_1,\ldots,X_k)\varphi_{\alpha}(s) & =\sum_{I\in C_k}(\omega_I\circ\varphi_{\alpha})(s)\partial_{\wedge I}^*\Big( (X_1\circ\varphi_{\alpha})(s),\ldots,(X_k\circ \varphi_{\alpha})(s) \Big)                                               \\
			                                          & =\sum_I(\omega_I\circ\varphi_{\alpha})(s)(\partial^*_{i_1}\wedge\ldots\partial_{i_k}^*)  \Big( (X_1\circ\varphi_{\alpha})(s),\ldots,(X_k\circ \varphi_{\alpha})(s) \Big).		\label{SUBEQooKPRLooFzreMJ}
		\end{align}
	\end{subequations}
	The function \( \omega_I\circ\varphi_{\alpha}\) is smooth. The rest of \eqref{SUBEQooKPRLooFzreMJ} are sum of products of elements of the form\footnote{We write the dependence of \( \partial_i^*\) in \( x=\varphi_{\alpha}(s)\).}
	\begin{subequations}
		\begin{align}
			\partial_{i_m}^*(x)\Big( (X_l\circ\varphi_{\alpha})(s) \Big) & = \partial^*_{i_m}(x)\sum_{j=1}^n(X_{lj\circ\varphi_{\alpha}})(s)\partial_j(x) \\
			                                                             & = \sum_j (X_{lj}\circ \varphi_{\alpha})(s)\delta_{j,i_m}                       \\
			                                                             & = (X_{l,i_m}\circ\varphi_{\alpha})(s).
		\end{align}
	\end{subequations}
	The last line is a smooth function of \( s\) since \( X_{ij}\) are smooth. The proposition \ref{PROPooCWPAooKDnwHR} about sum and product of smooth functions makes \( \omega(X_1,\ldots,X_k)\) smooth.
\end{proof}


\begin{lemma}		\label{LEMooMEZCooGXpOOa}
	Let \( v\in \eR^n\). The maps of \ref{NORMooDQEZooHaJPoa} satisfy
	\begin{enumerate}
		\item		\label{ITEMooZSFDooAuUnPd}
		      \begin{equation}
			      \tau_s(e^*_i)\big( \tau_s(v) \big)=e_i^*(v)=v_i.
		      \end{equation}
		\item		\label{ITEMooSVEFooBKhHnc}
		      For every miltiindex \( I=(i_1,\ldots,i_k)\) and \( X_j\in T_{\varphi_{\alpha}(s)}M\), we have
		      \begin{equation}
			      \tau_{\alpha,s}(e^*_{\wedge I})(X_1,\ldots,X_k)=e^*_{\wedge I}\big( \tau_{\alpha,s}^{-1}(X_1),\ldots,\tau_{\alpha,s}^{-1}(X_k) \big).
		      \end{equation}
		      \item\label{ITEMooQUZCooGcJrLe}
		      For every \( \omega'\in \Wedge^k(\eR^n)^*\) and \( X_i\in T_{\varphi_{\alpha}(s)M}\) we have
		      \begin{equation}
			      \tau_{\alpha,s}(\omega')(X_1,\ldots,X_k)=\omega'\big( \tau_{\alpha,s}^{-1}(X_1),\ldots,\tau_{\alpha,s}^{-1}(X_k) \big).
		      \end{equation}
	\end{enumerate}
\end{lemma}

\begin{proof}
	Each point is a generalisation of the previous one.
	\begin{subproof}
		\spitem[For \ref{ITEMooZSFDooAuUnPd}]
		%-----------------------------------------------------------
		We have \( e^*_i\in \Wedge^k\eR^n\) with \( k=1\); we can apply \( \tau_s\) to it. We have
		\begin{equation}
			\tau_s(e^*_i)\big( \tau_s(v) \big)=\partial_i^*\big( \sum_jv_j\partial_j \big)=\sum_jv_j\delta_{ij}=v_i=e^*_i(v).
		\end{equation}
		\spitem[For \ref{ITEMooSVEFooBKhHnc}]
		%-----------------------------------------------------------
		Let \( X_j\in T_{\varphi_{\alpha}(s)}M\). Thanks to proposition \ref{PROPooAAAXooKAMsfK}, we write \( X_j=\sum_lX_{jl}\partial_l\). Applying \( \partial^*_m\) we have
		\begin{equation}		\label{EQooBGWWooXQqAdc}
			\partial_m^*(X_j)=\sum_lX_{jl}\delta_{ml}=X_{jm}.
		\end{equation}
		Now we can compute. On the one hand,
		\begin{subequations}
			\begin{align}
				\tau_{\alpha,s}(e^*_{\wedge I})(X_1,\ldots,X_k) & = \partial^*_{\wedge I}(X_1,\ldots,X_k)                                                                                                              \\
				                                                & =(\partial_{i_1}^*\wedge \ldots \wedge \partial_{i_k}^*)(X_1,\ldots,X_k)                                                                             \\
				                                                & =k!\Alt(\partial^*_{i_1}\otimes \ldots\otimes \partial_{i_k}^*)(X_1,\ldots,X_k)                               & \text{lem. \ref{LEMooZJJLooFGuguy}.} \\
				                                                & = \sum_{\pi\in S_k}(-1)^{\pi}(\partial_{i_1}^*\otimes \ldots \partial_{i_k}^*)(X_{\pi(1)},\ldots, X_{\pi(k)}) & \text{def. \ref{DEFooHAKAooIfbsEy}}  \\
				                                                & = \sum_{\pi\in S_k}(-1)^{\pi} \partial_{i_1}^*(X_{\pi(1)})\ldots \partial_{i_k}(X_{\pi(k)})                                                          \\
				                                                & = \sum_{\pi\in S_k}(-1)^{\pi}X_{i_1\pi(1)}\ldots X_{i_k\pi(k)}                                                & \text{by \eqref{EQooBGWWooXQqAdc}.}
			\end{align}
		\end{subequations}
		On the other hand we can make the same calculation\quext{I did not. Check and send me a message saying if it works or not.} for \( e^*_{\wedge I}\big( \tau^{-1}_{\alpha,s}(X_1),\ldots,\tau_{\alpha,s}^{-1}(X_k) \big)\) using the fact that
		\begin{equation}
			e^*_i\big( \tau_{\alpha,s}(X_j) \big)=e^*_i\big( \sum_lX_{jl}e_l \big)=X_{ji}.
		\end{equation}
		\spitem[For \ref{ITEMooQUZCooGcJrLe}]
		%-----------------------------------------------------------
		Using the linearity of \( \tau_{\alpha,s}\) and point \ref{ITEMooSVEFooBKhHnc}.
	\end{subproof}
\end{proof}

\begin{lemma}		\label{LEMooOLSHooZNRoZs}
	Let \( \omega\in \Omega^k(M)\). There exists a smooth \(\omega' \colon U_{\alpha}\to \Wedge^k(\eR^n)^*  \) such that
	\begin{equation}
		\omega_{\varphi_{\alpha}(s)}=\Big( \varphi_{\alpha}(s),\tau_{\alpha,s}\big( \omega'(s) \big) \Big).
	\end{equation}
\end{lemma}

\begin{proof}
	Saying that the map \(\omega \colon M\to M\times \Wedge^kT^*M  \) is smooth means that the map \( f=\sigma_{\beta}^{-1}\circ \omega\circ\varphi_{\alpha}\) is smooth as map from \( U_{\alpha}\) to \( U_{\beta}\). In particular with \( \beta=\alpha\) we are looking at \( f(s)=(\sigma_{\alpha}^{-1}\circ \omega\circ \varphi_{\alpha})(s)\):
	\begin{subequations}
		\begin{align}
			f(s) & = \sigma_{\alpha}^{-1}\big( (\omega\circ \varphi_{\alpha})(s) \big) \\
			     & = \sigma_{\alpha}^{-1}\big( \varphi_{\alpha}(s),\xi(s) \big)        \\
			     & = \big( s,\omega'(s) \big)
		\end{align}
	\end{subequations}
	for some \( \xi(s)\in \Wedge^k(T^*_{\varphi_{\alpha}(s)})\) and \( \omega'(s)\in \Wedge^k(\eR^n)^*\). Since \( f\) is smooth, the map
	\begin{equation}
		\omega' \colon U_{\alpha}\to \Wedge^k(\eR^n)^*
	\end{equation}
	is smooth. From the definition of \( f\),
	\begin{equation}
		(\omega\circ \varphi_{\alpha})(s)=(\sigma_{\alpha}\circ f)(s)=\sigma_{\alpha}\big( s,\omega'(s) \big)=\Big( \varphi_{\alpha}(s),\tau_{\alpha,s}\big( \omega'(s) \big) \Big).
	\end{equation}
	We proved that
	\begin{equation}		\label{EQooMAINooBKsaex}
		(\omega\circ \varphi_{\alpha})(s)=\Big( \varphi_{\alpha}(s),\tau_{\alpha,s}\big( \omega'(s) \big) \Big).
	\end{equation}
\end{proof}

\begin{proposition}		\label{PROPooFWEQooIqTFSa}
	If \(\omega \colon U_{\alpha}\to \Wedge^k(\eR^n)^*  \) is smooth, then the map
	\begin{equation}
		\begin{aligned}
			f\colon U_{\alpha} & \to \Wedge^k\big( T^*_{\varphi_{\alpha}(s)}M \big) \\
			s                  & \mapsto \tau_{\alpha,s}\big( \omega(s) \big)
		\end{aligned}
	\end{equation}
	is smooth.
\end{proposition}

\begin{proof}
	Proposition \ref{PROPooDXTOooKYDOiI} says that there exist smooth functions \(\omega_I \colon U_{\alpha}\to \eR  \) such that
	\begin{equation}
		\omega(s)=\sum_{I\in C_k}\omega_I(s)e^*_{\lambda I}.
	\end{equation}
	Thus
	\begin{equation}
		\tau_{\alpha,s}\big( \omega(s) \big)=\sum_{I\in C_k}\omega_I(s)\partial^*_{\wedge I}.
	\end{equation}
	The functions \( \omega_I\) are smooth as well as \( \partial_{\wedge I}^*\) (lemma \ref{LEMooIAASooDnnhRA}).
\end{proof}

The following definition is in the same spirit of definition \ref{DEFooFWIHooJiTnwM}.
\begin{definition}		\label{DEFooZMQNooRNhWXk}
	Let \( \omega'\in \Wedge^k(\eR^n)^*\), \( p=\varphi_{\alpha}(s)\in M\) and \( X\in T_pM\). We define \( i_X(\omega)\) by
	\begin{equation}
		i_X(\omega)(X_1,\ldots,X_{k-1})=\omega(X,X_1,\ldots,X_{k-1}).
	\end{equation}

	Let \( X\) be a smooth vector field and \( \omega\) a smooth \( k\)-differential form. We defin \( i_X(\omega)\) for each \( p\in M\) by
	\begin{equation}
		i_X(\omega)_p=i_{X_p}(\omega_p).
	\end{equation}
\end{definition}

\begin{proposition}		\label{PROPooLCFAooLrbqeh}
	Let \( \omega'\in \Wedge^k(\eR^n)^*\), \( v\in \eR^n\) and \( p=\varphi_{\alpha}(s)\in M\). We have
	\begin{equation}
		\tau_{\alpha,s}\big( i_{v}(\omega') \big)=i_{\tau_{\alpha,s}(v)}\big( \tau_{\alpha,s}(\omega') \big).
	\end{equation}
\end{proposition}

\begin{proof}
	Using lemma \ref{LEMooMEZCooGXpOOa}\ref{ITEMooQUZCooGcJrLe} twice (once in each direction) we have
	\begin{subequations}
		\begin{align}
			i_v(\omega')(X_1,\ldots,X_{k-1}) & =i_v(\omega')\big( \tau_{\alpha,s}^{-1}(X_1),\ldots,\tau_{\alpha,s}^{-1}(X_{k_1}) \big)                                   \\
			                                 & =\omega'\big( v,\tau_{\alpha,s}^{-1}(X_1),\ldots,\tau_{\alpha,s}^{-1}(X_{k-1}) \big)                                      \\
			                                 & =\omega'\big( \tau_{\alpha,s}^{-1}\tau_{\alpha,s}(v),\tau_{\alpha,s}^{-1}(X_1),\ldots,\tau_{\alpha,s}^{-1}(X_{k-1}) \big) \\
			                                 & =(\tau_{\alpha,s}\omega')\big( \tau_{\alpha,s}(v),X_1,\ldots,X_{k-1} \big)                                                \\
			                                 & =i_{\tau_{\alpha,s}(v)}\big( \tau_{\alpha,s}(\omega') \big)(X_1,\ldots,X_{k-1}).
		\end{align}
	\end{subequations}
\end{proof}

\begin{lemma}[\cite{MonCerveau}]		\label{LEMooEKYDooKkgZJB}
	Let \(v \colon U_{\alpha}\to \eR^n  \) and \(\omega \colon U_{\alpha}\to \Wedge^k(\eR^n)^*  \) be smooth. Then the map
	\begin{equation}
		\begin{aligned}
			r\colon U_{\alpha} & \to \Wedge^{k-1}(\eR^n)^* \\
			s                  & \mapsto i_{v(s)}\omega(s)
		\end{aligned}
	\end{equation}
	is smooth.
\end{lemma}

\begin{proof}
	Notice that \( \Wedge^{k-1}(\eR^n)^*\) is a part of \( \End\Big( (\eR^n)^{k-1},\eR \Big)\), thus we can consider
	\begin{equation}
		r \colon U_{\alpha}\to \End\big( (\eR^n)^{k-1},\eR \big).
	\end{equation}
	If \( \omega(s)=\sum_{I=(i_1,\ldots,i_k)}(s)e^*_{\otimes I}\), proposition \ref{LEMooOBUVooBHpDZX} gives
	\begin{equation}
		r(s)=\sum_{J=(j_1,\ldots,j_{k-1})}\omega'_J(s)e^*_{\otimes J}
	\end{equation}
	with
	\begin{equation}
		\omega'_J=\sum_i\omega_{iJ}(s)v(s)_i.
	\end{equation}
	Thus we write
	\begin{equation}		\label{EQooQKQHooGXUzHY}
		r(s)=\sum_J\sum_i\omega_{iJ}(s)v(s)_ie^*_{\otimes J}.
	\end{equation}
	The map \( s\mapsto \omega_{iJ(s)}\) is smooth as well as \( s\mapsto v(s)_i\). The whole expression \eqref{EQooQKQHooGXUzHY} is smooth.
\end{proof}

\begin{propositionDef}[\cite{MonCerveau}]		\label{PROPooTBFLooVCqycE}
	Let \( X\) be a smooth vector field and \( \omega\) a smooth \( k\)-differential form. Then the map
	\begin{equation}
		i_X(\omega) \colon M\to \Wedge^k(T^*M)
	\end{equation}
	is smooth and is then a \( k\)-differential form.
\end{propositionDef}

\begin{proof}
	Let \( p=\varphi_{\alpha}(s)\in M\). We have, for some \( \omega'(s)\in \Wedge^k(\eR^n)^*\),
	\begin{subequations}
		\begin{align}
			\big( i_X(\omega) \big)_p & = i_{X_p}(\omega_p)                                                                                       \\
			                          & =i_{X_p} \tau_{\alpha,s}\big( \omega'(s) \big)                                                            \\
			                          & =\tau_{\alpha,s}\Big( i_{\tau_{\alpha,s}^{-1}}(X_p)\omega' \Big) & \text{prop. \ref{PROPooLCFAooLrbqeh}.}
		\end{align}
	\end{subequations}
	Since \( \tau_{\alpha,s}\) is smooth by proposition \ref{PROPooFWEQooIqTFSa}, it remains to prove that the map
	\begin{equation}
		\begin{aligned}
			r\colon U_{\alpha} & \to \Wedge^{k-1}(\eR^n)^*                                           \\
			s                  & \mapsto i_{\tau_{\alpha,s}^{-1}(X_{\varphi_{\alpha}(s)})}\omega'(s)
		\end{aligned}
	\end{equation}
	is smooth. Since \( X\) is a smooth vector field, proposition \ref{PROPooXURIooYPytwa} says that we can write
	\begin{equation}
		X_{\varphi_{\alpha}(s)}=\sum_{i=1}^nX_i\big( \varphi_{\alpha}(s) \big)\partial_i
	\end{equation}
	for some smooth functions \(X_i \colon M\to \eR  \). With these notations we have
	\begin{equation}
		\tau_{\alpha,s}^{-1}(X_{\varphi_{\alpha}(s)})=\sum_i(X_i\circ \varphi_{\alpha,s})(s)e_i.
	\end{equation}
	The latter is smooth with respect to \( s\).
\end{proof}


%-------------------------------------------------------
\subsection{Interior product}
%----------------------------------------------------

In the algebraic setting (without regularity and manifolds), the interior product we introduced in definition \ref{DEFooFWIHooJiTnwM}, and we already had the Leibnitz rule in proposition \ref{PROPooYCRXooSOsqCb}. Recall: the we denote by \( \Omega^k(M)\) the space of \( k\)-differential forms on the smooth manifold \( M\). These are the sections of \( \Wedge^k(T^*M)\), see definition \ref{DEFooBRSSooWOgoov}.

\begin{proposition}[\cite{BIBooJMRFooTAhhcg}]		\label{PROPooIQIUooTDNJdB}
	Let \( X\) be a smooth vector field. Let \( \omega\in \Omega^k(M)\) and \( \eta\in \Omega^l(M)\) be \( k\) and \( l\) differential forms.

	Then we have
	\begin{equation}
		i_X(\omega\wedge \eta)=i_X(\omega)\wedge \eta+(-1)^k\omega\wedge i_X(\eta).
	\end{equation}
\end{proposition}

\begin{proof}
	For each \( p\in M\) we have
	\begin{subequations}
		\begin{align}
			\Big( i_X(\omega\wedge \eta) \Big)_p & = i_{X_p}(\omega_p\wedge \eta_p)                                                                                         \\
			                                     & = i_{X_p}(\omega_p)\wedge\eta_p+(-1)^k\omega_p\wedge(i_{X_p}\eta)                & \text{prop. \ref{PROPooYCRXooSOsqCb}} \\
			                                     & = \Big( i_X(\omega)\wedge \eta \Big)_p+(-1)^k\Big( \omega\wedge i_X\eta \Big)_p.
		\end{align}
	\end{subequations}
\end{proof}


%-------------------------------------------------------
\subsection{Exterior derivative}
%----------------------------------------------------

\begin{proposition}[exterior derivative\cite{MonCerveau,BIBooANSPooJQsBvY}]		\label{PROPooTXFRooMtaVrU}
	There exists a unique set of maps \(d \colon \Omega^k(M)\to \Omega^{k+1}(M)  \) (\( k\in \eN\)) such that
	\begin{enumerate}
		\item		\label{ITEMooPQYCooHMjgIR}
		      \( d\) is \( \eR\)-linear :  \( d(\lambda \omega)=\lambda d\omega\) and \( d(\omega+\eta)=d\omega+d\eta\) for every \( \lambda\in \eR\) and \( \omega,\eta\in \Omega^k(M)\).
		\item		\label{ITEMooRKJRooWwHUiS}
		      If \( f\in\Omega^0(M)=C^{\infty}(M)\), then \( df\) is the usual differential of \( f\).
		\item	\label{ITEMooBBFLooWpZJuT}
		      Leibniz rule : if \( \omega\in \Omega^k(M)\), we have \( d(\omega\wedge \eta)=(d\omega)\wedge\eta=(-1)^k\omega\wedge d\eta\)
		\item		\label{ITEMooJXMFooHFTFYi}
		      \( d(d\omega)=0\).
	\end{enumerate}
\end{proposition}

\begin{proof}
	First unicity, then existence.
	\begin{proofpart}
		Unicity
	\end{proofpart}
	We initiate by proving the unicity by showing that \( d\) has a fixed form in a chart. Let \( (U,\varphi)\) be a chart. From \ref{PROPooNOGKooQedJba} a generic differential form reads \( \omega_x=\sum_I\omega_I(x)\partial^*_{\wedge I}\).

	\begin{subproof}
		\spitem[\( d(\partial^*_{\wedge I})=0\)]
		%-----------------------------------------------------------
		Let \( I=(i_1,\ldots,i_k)\) and compute \( d(\partial^*_{\wedge I})\). We denote \( \omega=\partial^*_{i_1}\wedge\ldots\wedge\partial^*_{i_{k-1}}\) :
		\begin{subequations}
			\begin{align}
				d(\partial^*_{\wedge I}) & = d\big( \omega\wedge \partial^*_{i_{k}} \big)                                                                                    \\
				                         & = (d\omega)\wedge\partial^*_{i_{k}}+(-1)^{k-1}\omega\wedge d(\partial^*_{i_{k}}) & \text{by item \ref{ITEMooBBFLooWpZJuT}.}       \\
				                         & = d(\partial^*_{i_1}\wedge \ldots\partial^*_{i_{k-1}})\wedge\partial^*_{i_k}     & \text{see bellow}. \label{SUBEQooOWTHooIZcNWw}
			\end{align}
		\end{subequations}
		Justification for \eqref{SUBEQooOWTHooIZcNWw}. From lemma \ref{LEMooKCBSooPjEwEl}, \( \partial^*_{i_{k}}=df_{i_k}\), so that item \ref{ITEMooJXMFooHFTFYi} gives \( d(\partial^*_{i_{k+1}})=d(df_i)=0\).

		By recursion we have \( d(\partial^*_{\wedge I})=0\).
		\spitem[Computation of \( d(f\partial^*_{\wedge I})\)]
		%-----------------------------------------------------------
		From lemma \ref{LEMooUVRQooVZDKir} we use the wedge product \(  f\partial^*_{\wedge I}=f\wedge \partial^*_{\wedge I} \) and we can use the Leibniz formula of item \ref{ITEMooBBFLooWpZJuT} :
		\begin{equation}
			d(f\partial^*_{\wedge I})=d(f\wedge\partial^*_{\wedge I})=df\wedge \partial^*_{\wedge I}+f\wedge \underbrace{d(\partial^*_{\wedge I})}_{=0}=df\wedge\partial^*_{\wedge I}.
		\end{equation}
		\spitem[The general case]
		%-----------------------------------------------------------
		A general differential \( k\)-form \( \omega\) is given by proposition \ref{PROPooNOGKooQedJba} which is a real linear combination of elements of the form \( f\partial^*_{\wedge I}\).
		\begin{equation}		\label{EQooMGAVooCndFmv}
			d\omega=d\big( \sum_Ia_I\partial^*_{\wedge I} \big)=\sum_Ida_I\wedge \partial^*_{\wedge I}.
		\end{equation}
		\spitem[Unicity]
		%-----------------------------------------------------------
		If a map \( d\) satisfying all the conditions exists, it has to be of the form
		\begin{equation}
			d\big( \sum_Ia_I\partial^*_{\wedge I} \big)=\sum_Ida_I\wedge \partial^*_{\wedge I}
		\end{equation}
		on each coordinate patch. This proves the unicity.
	\end{subproof}
	\begin{proofpart}
		Local existence
	\end{proofpart}
	For each chart \( (U_{\alpha}, \varphi_{\alpha})\) we define the operators
	\begin{equation}
		\begin{aligned}
			d_{\alpha} \colon \Omega^k\big( \varphi_{\alpha}(U_{\alpha}) \big) & \to \Omega^{k+1}\big( \varphi_{\alpha}(U_{\alpha}) \big) \\
			\sum_Ia_I\partial^*_{\wedge I}                                     & \mapsto \sum_Ida_I\wedge\partial^*_{\wedge I},
		\end{aligned}
	\end{equation}
	and we prove that they satisfy the conditions.
	\begin{subproof}
		\spitem[Property \ref{ITEMooPQYCooHMjgIR}]
		%-----------------------------------------------------------
		We have
		\begin{equation}
			d(\lambda \omega)  =\big( \sum_I(\lambda a_I)\partial^*_{\wedge I} \big)
			=\sum_Id(\lambda a_I)\wedge\partial^*_{\wedge I}
			=\lambda\sum_Ida_I\wedge \partial^*_{\wedge I}.
		\end{equation}
		Same kind of computations for \( d(\omega+\eta)=d\omega+d\eta\).
		\spitem[Property \ref{ITEMooRKJRooWwHUiS}]
		%-----------------------------------------------------------
		If \( \omega\) is a \( 0\)-form, the expression \eqref{EQooMGAVooCndFmv} has no sum over \( I\) and reduces to \( \omega=f\).
		\spitem[Property \ref{ITEMooBBFLooWpZJuT}]
		%-----------------------------------------------------------
		Let \( \omega=\sum_{I\in C_k}a_I\partial^*_{\wedge I}\) and \( \eta=\sum_{J\in C_l}b_J\partial^*_{\wedge J}\). Since the wedge product is linear we can group the sums:
		\begin{equation}
			\omega\wedge \eta=\sum_I\sum_Ja_Ib_J\partial^*_{\wedge I}\wedge \partial^*_{\wedge J}.
		\end{equation}
		Using the product rule for usual differential, we have
		\begin{subequations}
			\begin{align}
				d_{\alpha}(\omega\wedge\eta) & =\sum_{I,J}d(a_Ib_J)\wedge(\partial^*_{\wedge I}\wedge \partial^*_{\wedge J})                                                                                                                               \\
				                             & =\sum_{I,J}b_J(da_I)\wedge(\partial^*_{\wedge I}\wedge \partial^*_{\wedge J})+\sum_{I,J}(a_Idb_J)\wedge(\partial^*_{\wedge I}\wedge \partial^*_{\wedge J})                                                  \\
				                             & =\sum_{I,J}(da_I\wedge \partial^*_{\wedge I})\wedge(b_J\partial^*_{\wedge J})+\sum_{I,J}a_I(-1)^k\partial^*_{\wedge I}\wedge db_J\wedge\partial^*_{\wedge J} & \text{cf. bellow}		\label{SUBEQooFXOAooCziZnU} \\
				                             & = d\omega\wedge\eta+(-1)^k\omega\wedge d\eta.
			\end{align}
		\end{subequations}
		For \eqref{SUBEQooFXOAooCziZnU}. In the first sum, the function \( b_J\) commutes with the operation \( \wedge\). In the second sum, the differential form \( db_J\) commutes with the wedge products giving a minus sign. This explains the \( (-1)^k\).
		\spitem[\( d(df)=0\)]
		%-----------------------------------------------------------
		Let \(f \colon M\to \eR  \) be a smooth function. Using lemma \ref{LEMooGEFSooVlPLOs}, we compute \( d_{\alpha}(df)\) :
		\begin{equation}
			d_{\alpha}(df)=\sum_i\sum_j(\partial_j\partial_if)\partial^*_j\wedge\partial^*_i.
		\end{equation}
		Since \( \partial_i\partial_if\) is symmetric with respect to \( i,j\) and \( \partial^*_j\wedge\partial^*_i\) is anti-symmetric with respect to \( i,j\), the sum is zero.
		\spitem[Property \ref{ITEMooJXMFooHFTFYi}]
		%-----------------------------------------------------------
		Let \( \omega=\sum_Ia_I\partial^*_{\wedge I}\). First we have
		\begin{equation}
			d_{\alpha}\omega=\sum_Ida_I\wedge\partial^*_{\wedge I}.
		\end{equation}
		Then
		\begin{equation}
			d_{\alpha}(d_{\alpha\omega})=\sum_I\underbrace{d(da_I)}_{=0}\wedge\partial^*_{\wedge I}-\sum_Ida_I\wedge \underbrace{d(\partial^*_{\wedge I})}_{=0}.
		\end{equation}
	\end{subproof}
	\begin{proofpart}
		Global existence
	\end{proofpart}
	By the unicity part, we have \( d_{\alpha}=d_{\beta}\) on the part \( \varphi_{\alpha}(U_{\alpha})\cap\varphi_{\beta}(U_{\beta})\). Thus the local formula
	\begin{equation}
		d\big( \sum_Ia_I\partial^*_{\wedge I} \big)  =\sum_Ida_I\wedge\partial^*_{\wedge I}
	\end{equation}
	globally defines \( d\).
\end{proof}

\begin{proposition}[\cite{MonCerveau,BIBooEJXFooVcandW,BIBooFZPAooVLUjdi,BIBooTJJBooWeUZEO}]		\label{PROPooUOQRooAdvYqx}
	Let \( \omega\) be a \( k\)-differential form and \( X_1,\ldots,X_{k+1}\) be vector fields. We have
	\begin{equation}
		\begin{aligned}[]
			(d\omega)(X_1,\ldots,X_{k+1}) & =\sum_i(-1)^{i+1}X_i\big( \omega(X_1,\ldots,\hat X_i,\ldots X_{k+1}) \big)                              \\
			                              & \quad+\sum_{i<j}(-1)^{i+j}\omega\big( [X_i,X_j],X_1,\ldots\hat X_i,\ldots,\hat X_j,\ldots,X_{k+1} \big)
		\end{aligned}
	\end{equation}
	where \( d\omega\) is the exterior derivative defined by proposition \ref{PROPooTXFRooMtaVrU}.
\end{proposition}

\begin{proof}
	We defined \(s \colon \Omega^k(M)\to \Omega^{k+1}(M)  \) by
	\begin{equation}		\label{EQooZPNOooAXitUC}
		\begin{aligned}[]
			(s\omega)(X_1,\ldots,X_{k+1}) & =\sum_i(-1)^{i+1}X_i\big( \omega(X_1,\ldots,\hat X_i,\ldots X_{k+1}) \big)                              \\
			                              & \quad+\sum_{i<j}(-1)^{i+j}\omega\big( [X_i,X_j],X_1,\ldots\hat X_i,\ldots,\hat X_j,\ldots,X_{k+1} \big)
		\end{aligned}
	\end{equation}
	and we will show that \( s\) has the same local form as \( d\):
	\begin{equation}
		s(f\partial^*_{\wedge I})=df\wedge\partial^*_{\wedge I}.
	\end{equation}
	Then, as we seen in proposition \ref{PROPooTXFRooMtaVrU}, it makes \( s\) satisfy the defining properties of the exterior derivative and the unicity part will conclude that \( s=d\).


	\begin{proofpart}
		\( s(\omega)\) is \( C^{\infty}(M)\)-linear
	\end{proofpart}
	In this part, we prove that \( s(\omega)(fX_1,X_2,\ldots,X_{k+1})=fs(\omega)(X_1,\ldots,X_{k+1})\) for every smooth function \(f \colon M\to \eR  \).

	\begin{subproof}
		\spitem[Splitting the sums]
		%-----------------------------------------------------------

		We split the sums in \eqref{EQooZPNOooAXitUC}. We separate the terms \( i=1\) in the sums in equation \eqref{EQooZPNOooAXitUC}:
		\begin{equation}
			\begin{aligned}[]
				s(\omega)(fX_1,X_2,\ldots,X_{k+1}) & =fX_1\big( \omega(X_2,\ldots,X_{k+1}) \big)                                                   \\
				                                   & \quad +\sum_{i>1}(-1)^{i+1}X_i\big( \omega(fX_1,X_2,\ldots,\hat X_i,\ldots,x_{k+1})\big)      \\
				                                   & \quad + \sum_{j>1}(-1)^{1+j}\omega\big( [fX_1,X_j],X_2,\ldots,\hat X_j,\ldots,X_{k+1} \big)   \\
				                                   & \quad +\sum_{1<i<j}(-1)^{i+j}\omega\big( [X_i,X_j],fX_1,\ldots,\hat X_j,\ldots,X_{k+1} \big).
			\end{aligned}
		\end{equation}
		Since \( \omega\) is \( C^{\infty}(M)\)-linear, we can move \( f\) outside \( \omega\) when it is a simple coefficient:
		\begin{equation}
			\begin{aligned}[]
				s(\omega)(fX_1,X_2,\ldots,X_{k+1}) & =fX_1\big( \omega(X_2,\ldots,X_{k+1}) \big)                                                   \\
				                                   & \quad +\sum_{i>1}(-1)^{i+1}X_i\big(f \omega(X_1,X_2,\ldots,\hat X_i,\ldots,x_{k+1})\big)      \\
				                                   & \quad + \sum_{j>1}(-1)^{1+j}\omega\big( [fX_1,X_j],X_2,\ldots,\hat X_j,\ldots,X_{k+1} \big)   \\
				                                   & \quad +\sum_{1<i<j}(-1)^{i+j}f\omega\big( [X_i,X_j],X_1,\ldots,\hat X_j,\ldots,X_{k+1} \big).
			\end{aligned}
		\end{equation}
		\spitem[Apply \( X_i\)]
		%-----------------------------------------------------------
		Using the product rule,
		\begin{equation}
			\begin{aligned}[]
				X_i\big( f\omega(X_1,\ldots,\hat X_i,\ldots , X_{k+1}) \big) & =(X_if)\omega(X_1,\ldots,\hat X_i,\ldots,X_{k+1})                \\
				                                                             & \quad+fX_i\big( \omega(X_1,\ldots,\hat X_i,\ldots,X_{k+1}) \big)
			\end{aligned}
		\end{equation}
		\spitem[The commutator]
		%-----------------------------------------------------------
		If \( v\) is a vector we have
		\begin{subequations}
			\begin{align}
				[fX_1,X_j](v) & =(fX_1)(X_jv)-X_j\big( X_j(X_1v) \big) \\
				              & =(fX_1X_j)(v)-X_j(f)X_1(v)-fX_jX_1v    \\
				              & =f[X_1,X_j](v)-X_j(f)X_1(v).
			\end{align}
		\end{subequations}
		So we have \( [fX_1,X_k]=f[X_1,X_j]-X_j(f)X_1\). Beware: the expression \( f(X_1v)\) means the function \( x\mapsto f(x)(X_1)_xv\), not the function \( f\) applied to \( X_1v\) (which makes no sense).

		Notice that \(  X_j(f)\) is a function and can get out of \( \omega\).
		\spitem[Last computation]
		%-----------------------------------------------------------
		Using the small computations we've just made,
		\begin{subequations}
			\begin{align}
				s(\omega)(fX_1,\ldots,X_{k+1}) & =fX_1\big( \omega(X_2,\ldots,X_{k+1}) \big)                                                                                                                             \\
				                               & \quad + \sum_{i>1}(-1)^{i+1}(X_if)\omega(X_1,\ldots,\hat X_i,\ldots,X_{k+1})                                                                                            \\
				                               & \quad + \sum_{i>1}(-1)^{i+1}fX_1\big( \omega(X_1,\ldots,\hat X_i,\ldots,X_{k+1}) \big)                           \label{SUBEQooXSVIooWhHJNA}                            \\
				                               & \quad +\sum_{j>1}(-1)^{1+j}f\omega\big( [X_1,X_j], X_2,\ldots, \hat X_j,\ldots,X_{k+1} \big)                                                                            \\
				                               & \quad - \sum_{j>1}(-1)^{1+j}X_j(f)\omega(X_1,\ldots,\hat X_j,\ldots,X_{k+1})                                                               \label{SUBEQooIBWYOooXfVAWt} \\
				                               & \quad + \sum_{1<i<j}(-1)^{i+j}f\omega\big( [X_i,X_j],X_1,\ldots,\hat X_i,\ldots,\hat X_j,\ldots,X_{k+1} \big).
			\end{align}
		\end{subequations}
		The terms \eqref{SUBEQooXSVIooWhHJNA} and \eqref{SUBEQooIBWYOooXfVAWt} simplify. The other can be grouped to get
		\begin{subequations}
			\begin{align}
				s(\omega)(fX_1,\ldots,X_{k+1}) & = \sum_i(-1)^{i+1}fX_i\big( \omega(X_1,\ldots,\hat X_i,\ldots,X_{k+1}) \big)                                       \\
				                               & \quad\nonumber+\sum_{i<j}(-1)^{i+j}f\omega\big( [X_i,X_j],X_1,\ldots,\hat X_j,\ldots,\hat X_j,\ldots,X_{k+1} \big) \\
				                               & =fs(\omega)(X_1,\ldots,X_{k+1}).
			\end{align}
		\end{subequations}
		Do you believe that one can make the same work with \( X_l\) instead of \( X_1\) ? Okay we are done for the first part: \( s(\omega)\) is \( C^{\infty}(M)\)-linear.
	\end{subproof}

	\begin{proofpart}
		Local expression
	\end{proofpart}

	Several steps.

	\begin{subproof}
		\spitem[Setting]
		%-----------------------------------------------------------


		Since \( s(\omega)\) is \( C^{\infty}(M)\)-linear, it is sufficient to compute its local expression when acting on the coordinate vectors \(  (\partial_{i_1},\ldots,\partial_{i_{k+1}})   \). So consider a multiindex \( J=(j_1,\ldots,j_{k+1})\), let\footnote{We use compact notations like \( \partial_I=(\partial_{i_1},\ldots,\partial_{i_k})\) and \( \partial_{\wedge I}=\partial_{i_1}\wedge \ldots\wedge \partial_{i_k}\).}
		\begin{equation}
			\omega=\sum_{I\in C_k}a_I(\partial_{\wedge I}^*)
		\end{equation}
		and let's compute \( s(\omega)(\partial_J)\).

		\spitem[Commutators]
		%-----------------------------------------------------------


		Since \( [\partial_i,\partial_i]=0\) for every \( i\) and \( j\), the term containing the commutator in \( s(\omega)\) vanishes when applied to \( \partial_J\). We are left with
		\begin{equation}
			s(\omega)(\partial_J)=\sum_l(-1)^{l+1}\sum_I\partial_{j_l}\big( a_I\partial^*_{\wedge I}(\partial_{j_1},\ldots,\hat\partial_{j_l},\ldots,\partial_{j_{k+1}}) \big).
		\end{equation}
		Notice that the numbers \( i_1<i_{2}<\ldots<i_k\) because the sum is over \( I\in C_k\), see proposition \ref{PROPooNOGKooQedJba}.
		\spitem[Intermediate result]
		%-----------------------------------------------------------
		If \( I\) is a multiindex, if \( \eta\) is a \( 1\)-form and if \( k\) is not part of \( I\), let us prove that
		\begin{equation}	\label{EQooLMTCooYAEiOy}
			(\eta\otimes \partial^*_{\wedge I})(\partial_k,\partial_I)=(\eta\wedge\partial^*_{\wedge I})(\partial_k,\partial_I).
		\end{equation}
		The wedge product \( \eta\wedge\partial^*_{\wedge I}\) contains terms with \( \eta\otimes\wedge^*_{\wedge I}\)  and with \( \partial^*_{\wedge I}\wedge \eta\). The second term is zero because \( \partial_k\) is part of the arguments of \( \partial^*_{\wedge I}\) while \( k\) is not part of \( I\).


		\spitem[If \( J\) is sorted]
		%-----------------------------------------------------------
		Let us suppose that \( j_1<\ldots <j_{k+1}\). We have:
		\begin{equation}
			s(\omega)(\partial_J)=\sum_l(-1)^{l+1}\sum_I\partial_{j_l}\big( a_I\partial^*_{\wedge I}(\partial_{J_{\hat l}}) \big)
		\end{equation}
		where \( J_{\hat l}=(j_1,\ldots,\hat j_l,\ldots,j_{k+1})\). The only terms in the sum over \( I\) which do not vanish are the ones with
		\begin{equation}		\label{EQooLOAXooIJERyY}
			\{ i_1,\ldots,i_k \}=\{ j_1,\ldots,j_{k+1} \}\setminus\{ j_l \}.
		\end{equation}
		Since the multiindices \( I\) and \( J\) are supposed to be sorted, the set equality \eqref{EQooLOAXooIJERyY} says that only one term is non zero: the term \( (i_1,\ldots,i_k)=(j_1,\ldots,\hat j_l,\ldots,j_{k+1})=J_{\hat l}\). We have:
		\begin{equation}
			s(\omega)(\partial_J)=\sum_l(-1)^{l+1}\partial_{j_l}\big( a_{J_{\hat l}}\partial^*_{\wedge J_{\hat l}}(\partial_{J_{\hat l}}) \big).
		\end{equation}
		In this expression, the factor \( \partial^*_{\wedge J_{\hat l}}(\partial_{J_{\hat l}})\) is a constant. The derivative \( \partial_{i_l}\) does not act on it. We continue the computation:
		\begin{subequations}
			\begin{align}
				s(\omega)(\partial_J) & =\sum_l(-1)^{l+1}\partial_{j_l}\big( a_{J_{\hat l}}\partial^*_{\wedge J_{\hat l}}(\partial_{J_{\hat l}}) \big)                                                                                             \\
				                      & =\sum_l(-1)^{l+1}\underbrace{\partial_{j_l}(a_{J_{\hat l}})}_{=(da_{J_{\hat l}})(\partial_{j_l})}\partial^*_{\wedge J_{\hat l}}(\partial_{J_{\hat l}})                                                     \\
				                      & = \sum_l(-1)^{l+1}(da_{J_{\hat l}})(\partial_{j_l})(\partial^*_{j_1}\wedge\ldots\hat\partial^*_{j_l}\ldots\wedge \partial^*_{j_{k+1}})(\partial_{j_1},\ldots,\hat\partial_{j_l},\ldots,\partial_{j_{k+1}}) \\
				                      & =\sum_l(-1)^{l+1}\big[   (da_{I_{\hat l}}\otimes \partial_{\wedge J_{\hat l}})   \big](\partial_{j_l}, \partial_{J_{\hat l}})
			\end{align}
		\end{subequations}
		We use the intermediate result \eqref{EQooLMTCooYAEiOy} :
		\begin{subequations}
			\begin{align}
				s(\omega)(\partial_J) & =\sum_{l}(-1)^{l+1}(da_{J_{\hat l}}\wedge \partial^*_{\wedge J_{\hat l}})(\partial_{j_l},\partial_{J_{\hat l}})                      \\
				                      & =\sum_l(da_{J_{\hat l}}\wedge \partial^*_{\wedge J_{\hat l}})(\partial_{j_1},\ldots, \partial_{j_{k+1}}).		\label{SUBEQooBADJooNmVtew}
			\end{align}
		\end{subequations}
		The second equality correspond to move the argument \( \partial_{j_l} \) to the \( l\)\ieme\ position: it requires \( l-1\) transpositions, each of them adding an extra minus sign.

		\spitem[Conclusion (\( J\) sorted)]
		%-----------------------------------------------------------
		We are still with \( J\) sorted. We have to compare \eqref{SUBEQooBADJooNmVtew} with the known expression of \( d\omega\). We have
		\begin{subequations}
			\begin{align}
				(d\omega)(\partial_J) & =\sum_I(da_I\wedge\partial^*_{\wedge I})(\partial_J)                                                                                                                \\
				                      & =\sum_I\sum_l(\partial_la_I)(\partial^*_l\wedge\partial^*_{\wedge I})(\partial_J)                   & \text{lem. \ref{LEMooGEFSooVlPLOs}}                           \\
				                      & =\sum_l\sum_I(\partial_la_I)(\partial^*_l\wedge\partial^*_{\wedge I})(\partial_J)                                                                                   \\
				                      & = \sum_l(\partial_la_{J_{\hat l}})\partial_l\wedge\partial^*_{\wedge J_{\hat l}}(\partial_J)        & \text{see bellow}                 \label{SUBEQooIKQUooVSVbCm} \\
				                      & = \sum_lda_{J_{\hat l}}\wedge\partial^*_{\wedge J_{\hat l}}(\partial_J).		\label{SUBEQooTVDQooSAfZYQ}
			\end{align}
		\end{subequations}
		Explanation for \eqref{SUBEQooIKQUooVSVbCm}. The only non vanishing term in the sum over \( I\) is the one with \( \{ l,i_1,\ldots,i_k \}=\{ j_1,\ldots,j_{k+1} \}\), which is \( I=J_{\hat l}\).

		Now we see that the expression \eqref{SUBEQooBADJooNmVtew} and \eqref{SUBEQooTVDQooSAfZYQ} are the same, so that
		\begin{equation}
			(d\omega)(\partial_J)=(s\omega)(\partial_J).
		\end{equation}
		\spitem[\( J\) is not sorted]
		%-----------------------------------------------------------
		We suppose now that \( J\) is not sorted. Let \( \pi\) be a permutation such that \( \pi(J)\) is sorted. Using the fact that \( s(\omega) \) and \( d\omega\) are anti-symmetric, we have:
		\begin{subequations}
			\begin{align}
				s(\omega)(\partial_J) & =(-1)^{\pi}s(\omega)(\partial_{\pi(J)}) \\
				                      & =(-1)^{\pi}(d\omega)(\partial_{\pi(J)}) \\
				                      & =(d\omega)(\partial_J).
			\end{align}
		\end{subequations}
	\end{subproof}
\end{proof}
