%+++++++++++++++++++++++++++++++++++++++++++++++++++++++++++++++++++++++++++++++++++++++++++++++++++++++++++++++++++++++++++
\section{Polar cone}
%+++++++++++++++++++++++++++++++++++++++++++++++++++++++++++++++++++++++++++++++++++++++++++++++++++++++++++++++++++++++++++

Let $\hH$ be an Hilbert space and $A$ a part of $\hH$. A part $\hH^+$ is \defe{self polar}{self polar cone} if
\begin{equation}
	\hH^+=\{ \xi\in\hH\tq\langle \xi, \eta\rangle \geq 0\forall \eta\in \hH^+ \}.
\end{equation}
The notation is motivated by the fact mentioned on page \ref{PgConeAndPositive} that a cone implies a notion of positivity.

\begin{proposition}
	Let $\hH^+$ be a self polar cone in the Hilbert space $\hH$. Define 
	\begin{equation}
		\hH^J=\{ \xi\in\hH\tq\langle \xi, \eta\rangle \in\eR\forall\eta\in\hH^+ \}.
	\end{equation}
	Then
	\begin{enumerate}
		\item
			We have $\hH=\hH^J\oplus i\hH^J$.
		\item
			If we define $J\colon \hH\to \hH$, $J(\xi+i\eta)=\xi-i\eta$ for $\xi,\eta\in\hH^J$, then $J$ is an anti-unitary involution and $\hH^J$ is an ordered vector space.
		\item\label{ItemPropDecompJordanH}
			An $J$-real element $\xi\in\hH^J$ can be written under the form $\xi=\xi_+-\xi_-$ where $\xi_+$ and $\xi_-$ are positive orthogonal elements of $\hH^+$.
	\end{enumerate}
\end{proposition}
The point \ref{ItemPropDecompJordanH} is called \defe{Jordan decomposition}{Jordan decomposition!in Hilbert space}.


%+++++++++++++++++++++++++++++++++++++++++++++++++++++++++++++++++++++++++++++++++++++++++++++++++++++++++++++++++++++++++++
\section{Representations of Lie groups on Hilbert spaces}
%+++++++++++++++++++++++++++++++++++++++++++++++++++++++++++++++++++++++++++++++++++++++++++++++++++++++++++++++++++++++++++

Let $G$ be a Lie group. A \defe{linear representation}{representation!on Hilbert space} of $G$ is a pair $(\hH,U)$ where $\hH$ is an Hilbert space and $U$ is a map $U\colon G\to \GL(\hH)$ such that
\begin{equation}
	\begin{aligned}[]
		U(gg')&=U(g)\circ U(g')\\
		U(e)&=\id
	\end{aligned}
\end{equation}

\begin{definition}
	If $\hH$ is an Hilbert space, the representation is \defe{unitary}{unitary!representation} if
	\begin{enumerate}

		\item
			for every element $g\in G$ the operator $U(g)$ is unitary:
			\begin{equation}
				\langle U(g)\phi, U(g)\psi\rangle =\langle \phi, \psi\rangle 
			\end{equation}
			for every $\phi,\psi\in\hH$.

		\item
			for every $\psi\in\hH$, the map
			\begin{equation}
				\begin{aligned}
					G&\to \hH \\
					g&\mapsto U(g)\psi 
				\end{aligned}
			\end{equation}
			is continuous.
	\end{enumerate}
\end{definition}

A famous example is the \defe{regular left representation}{representation!regular left}. The vector space is $ C^{\infty}(G,\eC)$ and the representation is given by
\begin{equation}
	\begin{aligned}
		L\colon G&\to \End_{\eC}\big(  C^{\infty}(G,\eC) \big) \\
		L_g(u)&=L_{g^{-1}}^*u
	\end{aligned}
\end{equation}
where $L_{g^{-1}}^*u$ is the function defined by
\begin{equation}
	\big(  L^*_{g^{-1}}u  \big)(x)=u(g^{-1}x)
\end{equation}
for $x\in G$. 

That is an infinite dimensional representation.

%---------------------------------------------------------------------------------------------------------------------------
\subsection{Induced representation}
%---------------------------------------------------------------------------------------------------------------------------
\label{SubSecInducrepresBBGC}

Let $B$ be a subgroup of $G$ and let $(V,\chi)$ be a representation of $B$. We are going to build a representation of $G$ from the data of $B$ and $\chi$. First we consider the space of $G$-\defe{equivariant}{equivariant} functions in $G$:
\begin{equation}
	C^{\infty}(G,\eC)^B=\{ u\in C^{\infty}(G,\eC)\tq u(gb)=\chi(b^{-1})u(g) \}.
\end{equation}
This space is invariant under the regular left representations of $G$ because if $u\in C^{\infty}(G,\eC)^B$ and $g\in G$,
\begin{equation}
	\begin{aligned}[]
		(L_gu)(xb)&=u(g^{-1}xb)\\
		&=\chi(b^{-1})u(g^{-1}x)\\
		&=\chi(b^{-1})(L_gu)(x).
	\end{aligned}
\end{equation}
The restriction of the representation $L$ to $C^{\infty}(G,\eC)^B$ is the \defe{induced representation}{induced!representation}.

Let us now suppose that
\begin{enumerate}

	\item
		the subgroup $B$ is closed
	\item
		the representation $\chi$ is one dimensional.
\end{enumerate}

Let now consider a Lie group $G$ and a closed subgroup $B$ which admits an unitary representation 
\begin{equation}
	\begin{aligned}
		\chi\colon B&\to \gU(1) \\
		b&\mapsto \chi(b).
	\end{aligned}
\end{equation}
We look at the quotient
\begin{equation}
	Q=G/B
\end{equation}
with the natural action
\begin{equation}
	\begin{aligned}
		\tau\colon G\times Q&\to Q \\
		(g,[x])&\mapsto [gx]=\tau_g[x].
	\end{aligned}
\end{equation}
In general, $\varphi(\tau_gq)$ is not equal to $g\varphi(q)$, we denote by $\beta$ the difference:
\begin{equation}			\label{EqDefBetavptaunotEqual}
	\varphi(\tau_g q)=g\varphi(q)\beta(q,g).
\end{equation}
\begin{lemma}
	The function $\beta$ does not depend on $q$: $\beta(g,q)=\beta(g,q')$.
\end{lemma}
\begin{proof}
	No proof.
\end{proof}

We suppose that $Q$ has a volume form $dq\in\Omega^{\dim(Q)}(Q)$ which is $G$-invariant, i.e.
\begin{equation}
	\tau^*_g(dq)=dq
\end{equation}
for every $g\in G$.

We also suppose that there exists a global section $\varphi\colon Q\to G$ which provides the global diffeomorphism
\begin{equation}
	\begin{aligned}
		Q\times B&\to G \\
		(q,b)&\mapsto \varphi(q)b. 
	\end{aligned}
\end{equation}
In that case we have a linear isomorphism
\begin{equation}		\label{EqIsomTildeHat}
	\begin{aligned}
		C^{\infty}(G,\eC)^B&\simeq  C^{\infty}(Q,\eC) \\
		\hat f&\mapsto \tilde f 
	\end{aligned}
\end{equation}
where $\hat f$ is defined from $\tilde f$ by the formula
\begin{equation}		\label{EqIsomtildehat}
	\hat f\big( \varphi(q)b \big)=\chi(b^{-1})\tilde f(q).
\end{equation}
This definition makes sense because $\varphi(q)b$ is a general element of $G$. We have to prove that the function $\hat f$ is equivariant. For that consider $x=\varphi(g)b\in G$ and $b_0\in B$ and compute
\begin{equation}
	\begin{aligned}[]
		\hat f\big( \varphi(q)bb_0 \big)&=\chi(bb_0)^{-1}\tilde f(q)\\
		&=\chi(b_0^{-1})\chi(b^{-1})\tilde f(q)\\
		&=\chi(b_0^{-1})\tilde f(x).
	\end{aligned}
\end{equation}
The map $\tilde f\mapsto\hat f$ given by \eqref{EqIsomtildehat} is moreover surjective. Indeed, if $\hat f$ is any element of $ C^{\infty}(G,\eC)^B$, it satisfy
\begin{equation}
	\hat f\big( \varphi(q)b \big)=\chi(b^{-1})\tilde f(q),
\end{equation}
So that $\hat f$ is the image of a function $\tilde f$.

We can look at the compact supported functions $\cdD(Q)= C^{\infty}_c(Q,\eC)$ under the isomorphism \eqref{EqIsomTildeHat}.  First, we want to know how does the regular representation goes from $ C^{\infty}(G,\eC)^B$ to $ C^{\infty}(Q,\eC)$. Let $\tilde u\in\cdD(Q)$ and look at the equivariant function $\hat u$ (formula \eqref{EqIsomtildehat}). We act with $g\in G$, and we use the function $\beta$ (equation \eqref{EqDefBetavptaunotEqual}):
\begin{equation}
	\begin{aligned}[]
		\big( U(g)\hat u \big)\big( \varphi(q) \big)&=\hat u\big( g^{-1}\varphi(q) \big)\\
		&=\hat u\big( \varphi(g^{-1}\cdot q)\beta(q,g^{-1})^{-1} \big)\\
		&=(\chi\circ\beta)(q,g^{-1})\tilde u(g^{-1}\cdot q)\\
		&=\underbrace{(\chi\circ\beta)(q,g^{-1})}_{\in \gU(1)}\big( \tau_{g^{-1}}^*\tilde u \big)(q).
	\end{aligned}
\end{equation}
Now, since $\tau_{g^{-1}}$ is a diffeomorphism, the function $\tau_{g^{-1}}^*\tilde u$ belongs to $\cdD(Q)$. The fact to multiply by $\chi\big( \beta(q,g^{-1}) \big)$ does not changes the compactness of the support.

We define the representation $\tilde U$ as
\begin{equation}
	\tilde U(g)\tilde u=\chi\big( \beta(g^{-1}) \big)\tau_{g^{-1}}^*\tilde u.
\end{equation}

%---------------------------------------------------------------------------------------------------------------------------
\subsection{Unitary induced representation}
%---------------------------------------------------------------------------------------------------------------------------
\label{SubSecUnitInducedPrep}

Let $\tilde u,\tilde v\in\cdD(Q)$, we consider the product in $L^2(Q,dq)$:
\begin{equation}
	\langle \tilde u, \tilde v\rangle =\int_Q\overline{\tilde u(q)}\tilde v(q)dq.
\end{equation}
If we look at the action of $\tilde U(g)$, we have
\begin{equation}
	\begin{aligned}[]
		\langle \tilde U(g)\tilde u, \tilde U(g)\tilde v\rangle &=\int_Q\overline{ \chi\big( \beta(q,g^{-1}) \big)\tilde u(\tau_{g^{-1}}q) }\chi\big( \beta(q,g^{-1}) \big)\tilde v(\tau_{g^{-1}}q)dq\\
		&=\int_Q(\overline{ \chi }\chi)(\overline{ \tilde u }\cdot \tilde v)(\tau_{g^{-1}}q)dq.
	\end{aligned}
\end{equation}
The factor $\overline{ \chi }\chi$ is equal to one. If we perform the change of variable $q'=\tau_{g^{-1}}q$, we have $\tau_{g^{-1}}^*dq'=dq'$ because we suppose that the measure is invariant. At the end of the day we have
\begin{equation}
	\langle \tilde U(g)\tilde u, \tilde U(g)\tilde v\rangle =\int_Q(\overline{ \tilde u }\tilde v)(q')dq'=\langle \tilde u, \tilde v\rangle 
\end{equation}
Since every unitary operator is continuous and since $\cdD(Q)$ is dense in $L^2(Q,dq)$, we can extend $\tilde U(g)$ into a continuous operator on $L^2(Q,dq)$, so that we get a representation
\begin{equation}
	\Big( \hH,\tilde U \Big).
\end{equation}
where $\hH$ stands for $L^2(Q,dq)$. Let $\tilde u\in\hH$ and consider the map
\begin{equation}
	\begin{aligned}
		\alpha(\tilde u)\colon G&\to \hH \\
		g&\mapsto \tilde U(g)\tilde u.
	\end{aligned}
\end{equation}
This is the \defe{coaction}{coaction}. As composition of continuous maps, the map $g\mapsto \tilde U(g)\tilde u$ is continuous, so that $\tilde U$ is an unitary representation.

In a compact notation, we write
\begin{equation}
	\tilde U=m_{\beta}\circ\tau^*
\end{equation}
where $m_{\beta}$ is the multiplication by $\beta$. That representation is the \defe{unitary induced}{unitary!induced representation} defined from $B$ and $\chi$.

