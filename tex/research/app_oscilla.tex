% This is part of (almost) Everything I know in mathematics
% Copyright (c) 2013-2014,2018
%   Laurent Claessens
% See the file fdl-1.3.txt for copying conditions.

%\label{app_oscilla}

\section{Operator symbol}
%+++++++++++++++++++++++++

Operator symbol are natural framework to build oscillatory integral. Most of this theory comes from the book \cite{Dieu7} of Dieudonné.

\subsection{A case without problem}
%----------------------------------
Let $X$ be an open subset of $\eR^{N'}$ and $\dpt{A_{\nu}}{X}{M_{N''\times N'}}$ with $|\nu|\leq m$, some $\Cinf$ maps. A differential operator is a map $\dpt{P}{\cdE_{\eR}(X)^{N'}}{\cdE_{\eR}(X)^{N''}}$ of the following form:
\[
	P(u)=\sum_{|\nu|\leq m}A_{\nu}\cdot D^{_{\nu}}u,
\]
where the dot denotes a product matrix times vector. We suppose that the support of $u$ is
compact in $X$, so that it can be extended by $0$ in $\eR^n\setminus X$ in order to get a function in $\scrD_{\eR}(X)^{N'}$ that we will also denote by $u$. One can compute the Fourier transform of $u$: $\mF u_k\in\scrC(\eR^n)$, and $\mF u\in\scrC(\mR^n)^{N'}$. A main property of Fourier transform is that
\[
	D^{\nu}u=\int_{\eR^n} e^{2\pi ix\cdot\xi}(2\pi u\xi)^{\nu}(\mF u)(\xi)d\xi.
\]
If we pose
\[
	A(x,\xi)=\sum_{|\nu|\leq m}(2\pi i\xi)^{\nu}A_{\nu}(x),
\]
$P(u)$ can be written as
\[
	P(u)=\int_{\eR^n} e^{2\pi ix\cdot\xi}A(x,\xi)\cdot(\mF u)(\xi)d\xi.
\]
This integral makes only sense because we had carefully chosen allows the regularity conditions: as far as integral over $\xi$ is concerned, the functions $A(x,\xi)$ are polynomial with coefficients in $\cdE_{\eR}(X)^{N'N''}$ while the exponential contains a scalar product. The purpose of oscillatory integral is to generalise these two circumstances.

\subsection{A problem}
%---------------------

Let us consider the integral $\int_1^{\infty}\frac{1}{x}e^{ix}dx$. How to give a sense to that? Since $e^{ix}=-i\partial_xe^{ix}$, an integration by parts gives
\begin{equation}
	\int_1^{\infty}\frac{1}{x}e^{ix}=-i\int_1^{\infty}\frac{1}{x}\partial_x(e^{ix})
	=ie^{-i}-\int_1^{\infty}\frac{1}{x^2}e^{ix}
\end{equation}
where the last integral exists in the usual sense.

\subsection{Basic definitions}
%-----------------------------

\begin{definition}
	Let $X$ be an open subset of $\eR^n$, $\dpt{a}{X\times\eR^N}{\eR}$ a $\Cinf$ function and $m$ be any real. We say that $a$ is an \defe{operator symbol}{operator!symbol} of order $m$ in $X\times\eR^N$ when

	$\forall L\subset X$ compact, $\forall$ multi-indices $\alpha=(\alpha_1,\ldots,\alpha_n)$, $\beta=(\beta_1,\ldots,\beta_n)$,

	$\exists$ constant $c_{\alpha,\beta,L}>0$ such that $\forall (x,\xi)\in L\times\eR^N$,
	\begin{equation}\label{eq:cond_symbol}
		|D_x^{\alpha} D_{\xi}\hbeta a(x,\xi)|\leq c_{\alpha,\beta,L}(1+|\xi|)^{m-|\beta|},
	\end{equation}
	where $|\xi|^2:=\sum_{k=1}^{N}|\xi_k|^2$.
\end{definition}

Note that in $X\times \eR^N$, $X\subset\eR^n$ with $n$ and $N$ not necessarily equals. For short, we will often say ``symbol''\ instead of ``operator symbol''; the set of symbols of order $m$ in $X\times\eR^N$  is a vector space denoted by $\mS^m(X\times\eR^N)$\label{pg:defmS}.

For terminology issues, we say that a property of the point $(x,\xi)\in X\times\eR^N$ is \emph{true when $|\xi|$ is large} if $\forall$ compact $L\subset X$, there exists $r_L>0$ such that the property is true $\forall(x,\xi)$ with $x\in L$ and $|\xi|\geq r_L$.

\noindent This allows us to re-express the definition of a symbol. We say that $a\in\Cinf(X\times\eR^N)$ is a symbol when

$\forall$ multi-indices $\alpha,\beta$,

$(1+|\xi|)^{-m+|\beta|}D_x^{\alpha}D_{\xi}^{\beta}a(x,\xi)$ is bounded when $|\xi|$ is large.

\noindent Indeed, the statement that $f(x,\xi)$ is such that $(1+|\xi|)^{-q}f(x,\xi)$ is bounded when $|\xi|$ is large is the existence of a constant $c$ (which depend on $L\subset X$) such that $f(x,\xi)\leq c(1+|\xi|)^q$.

\begin{proposition}
	A function in $\cdE(X\times\eR^N)$ which is zero for large $| \xi |$ is a symbol for all order.
\end{proposition}
\begin{proof}
	The assumption is: for all compact subset $L\subset X$, the restriction of $a$ to $L\times\eR^N$ have a compact support. Let us fix a compact $L$ and multi-indices $\alpha,\beta$; then on $L\times\eR^N$, $a$ and $D_x^{\alpha} D_{\xi}\hbeta a$ have a compact support. Then it is bounded by continuity. The same makes $|D_x^{\alpha} D_{\xi}\hbeta a(x,\xi)|(1+|\xi|)^{-m+|\beta|}$ bounded and then it can be majored by a constant $c_{\alpha\beta L}$.
\end{proof}

More generally, for same reason, if $a\in\mS^m(X\times\eR^N)$ and $b=a$ when $|\xi|$ is large, then $b\in\mS^m(X\times\eR^N)$.

The function
\[
	\sigma(\xi)=\us{1+\xi^2}
\]
is a symbol of order $2$. Here, there are no $x$ part and $\xi\in\eR$. The problem is to find a $c_{\beta}$ for any $\beta$. For $\beta=0$, the condition \eqref{eq:cond_symbol} becomes $\sigma(\xi)\leq c(1+\xi)^2$, which is true. For $|\beta|=1$, $c=2$ works because
\[
	\frac{2\xi}{(1+\xi^2)^2}\leq 2(1+\xi).
\]
It is clear that it will always woks because the degree of the denominator becomes bigger a bigger as $|\beta|$ grows.

This is a special case of a more general situation.

\begin{proposition}
	A complex function of $\cdE(X\times\eR^N)$ which is positively homogeneous of
	degree $m$ when $\abxi$ is large is a symbol of order $m$.
\end{proposition}\label{prop:23.16.4}
\begin{proof}
	The assumption is that $\forall$ compact $L\subset X$, $\exists r_L>0$ such that $\forall x\in L$, $\abxi\geq r_L$, and for all $\lambda\geq 1$,
	\[
		a(x,\lambda\xi)=\lambda^m a(x,\xi).
	\]
	Let us define $b(x,\xi)=a(x,\lambda\xi)$. We have $(D_{\xi} b)(x,\xi)=\lambda(D_{\xi} a)(x,\xi)$, then
	\[
		(D_{\xi}\hbeta b)(x,\xi)=\lambda^{|\beta|}(D\hbeta_{\xi} a)(x,\lambda\xi).
	\]

	Since $b(x,\xi)=a(x,\lambda\xi)=\lambda^ma(x,\xi)$, we have $(D^{\beta}_{\xi}b)(x,\xi)=\lambda^m(D_{\xi}^{\beta}a)(x,\xi)$. By equalizing both expression of $(D_{\xi}\hbeta b)(x,\xi)$, we find
	\[
		(D_x^{\alpha} D_{\xi}\hbeta a)(x,\lambda\xi)=\lambda^{m-|\beta|}(D_x^{\alpha} D_{\xi}\hbeta a)(x,\xi),
	\]
	when $x\in L$, $\abxi\geq r_L$ and $\lambda\geq 1$.

	On the (compact) sphere $\abxi=r_L$, $|(D_x^{\alpha} D_{\xi}\hbeta a)(x,\xi)|$ is bounded. Let $c$ be a majoration, depending only on $\alpha,\beta$ and $L$. Any $\eta\in\eR^N$ with $|\eta|\geq r_L$ can be written as $\eta=\lambda\xi$ with $\lambda\geq 1$ and $\abxi=r_L$. Then
	\begin{equation}\label{eq:majo}
		|(D_x^{\alpha} D_{\xi}\hbeta a)(x,\eta)|\leq c\lambda^{m-|\beta|}.
	\end{equation}

	It is cleat that $1+|\eta|\geq|\eta|\geq|\eta|/\abxi=\lambda$. Then, for $m-|\beta|>0$, the majoration \eqref{eq:majo} keep if we replace $\lambda^{m-|\beta|}$ by  $(1+|\eta|)^{m-|\beta|}$.

	If $m-|\beta|<0$, the replacement of $\lambda^{m-|\beta|}$ by  $(1+|\eta|)^{m-|\beta|}$ need to change the constant $c$. More precisely, it raises the question to find a constant $k$ such that $|\eta|^{-r}\leq k(1+|\eta|)^{-r}$. It is easy to see that any
	\[
		k>\left(\frac{r_L}{1+r_L}\right)^{-r}
	\]
	works. Finally,
	\begin{equation}
		|(D_x^{\alpha} D_{\xi}\hbeta a)(x,\eta)|\leq c\lambda^{m-|\beta|}
		\leq ck(1+|\eta|)^{m-|\beta|}.
	\end{equation}
\end{proof}

This proposition speaks about a function which is homogeneous when $|\xi|$ is large. There exist functions which are homogeneous although not symbol because of problems of continuity at $0$. For example, $a(x,\xi)=|\xi|$ when $N=1$. It is not $\Cinf$ at zero.
\begin{remark}
	A function in $\cdE(X)$ is a $\Cinf$ function on $X\times\eR^N$ which is positively homogeneous of degree zero. Then
	\[
		\cdE(X)\subset\mS^0(X\times\eR^N).
	\]
\end{remark}

We now give an useful result without proof.
\begin{proposition}
	Let $a$ be a $\Cinf$ function which is \emph{only} defined for large $\abxi$ and such that for all $\alpha,\beta$,
	\[
		(1+\abxi)^{-m+|\beta|}D_x^{\alpha} D_{\xi}\hbeta a(x,\xi)
	\]
	is bounded for large $\abxi$.

	Then there exists a symbol $b$ of order $m$ on $X\times \eR^N$ such that $a=b$ when $\abxi$ is large.
\end{proposition}

For notational convenience, we define
\begin{equation} \label{eq:def_ablambda}
	a_{\lambda}(x,\xi)=a(x,\lambda\xi).
\end{equation}

Let us clearly compute a derivative of $a_{\lambda}$. The notation $(D_{\xi}a_{\lambda})(x_0,\xi_0)$ has to be read as ``The derivative of $a_{\lambda}$ with respect to his second argument at point $(x_0,\xi_0)$''.  If $M_{\lambda}(x,\xi)=(x,\lambda\xi)$,
\begin{equation}
	\begin{split}
		(D_{\xi}a_{\lambda})(x_0,\xi_0)&=D_{\xi}(a\circ M_{\lambda})(x_0,\xi_0)\\
		&=(D_{\xi}a)(M_{\lambda}(x_0,\xi_0))\cdot \frac{dM_{\lambda}}{d\xi}(x_0,\xi_0)\\
		&=\lambda(D_{\xi}a)(x_0,\lambda\xi_0).
	\end{split}
\end{equation}

\begin{theorem}\label{tho:dieu23.16.6}
	A function $a\in\cdE(X\times\eR^N)$ is a symbol of order $m$ if and only if the set of the restrictions of the functions $\lambda^{-m}a_{\lambda}$ (with $\lambda\geq 1$) to $X\times(\eR^N\setminus\{o\})$ is bounded in the sense of definition~\ref{def:bounded}   in the Fréchet space $\cdE(X\times(\eR^N\setminus\{o\}))$.
\end{theorem}

\begin{proof}
	Let $a$ be a symbol of order $m$. By definition of $a_{\lambda}$,
	\[
		D_x^{\alpha} D_{\xi}\hbeta a_{\lambda}(x,\xi)=\lambda^{|\beta|}(D_x^{\alpha} D_{\xi}\hbeta a)(x,\lambda\xi),
	\]
	but any compact in $\Xrnz$ is contained in a compact which can be written as $L\times G$ with $L$ compact in $X$ and $G=\{\xi\in\eR^N\,|\, r\leq\abxi\leq R\}$ with $0<r<R$.

	Since $a$ is a symbol, we have
	\[
		|\DxaDxb a(x,\lambda\xi)|\leq  c_{\alpha,\beta,L}(1+\lambda\abxi)^{m-|\beta|};
	\]
	if we multiply it by $(1/\lambda)^{m-|\beta|}$, we find
	\begin{equation}
		\begin{split}
			\lambda^{|\beta|-m}|\DxaDxb a(x,\lambda\xi)|
			&\leq c_{\alpha,\beta,L}\left(\us{\lambda}+\abxi\right)^{m-|\beta|}\\
			&\leq \begin{cases}
				c_{\alpha,\beta,L}(1+R)^{m-|\beta|} & \text{if }|\beta|\leq m \\
				c_{\alpha,\beta,L}r^{m-|\beta|}     & \text{if }|\beta|> m.
			\end{cases}
		\end{split}
	\end{equation}

	Now, we want to know if $\{ \lambda^{-m}a_{\lambda}|\lambda\geq 1 \}$ is bounded in $\cdE(\Xrnz)$. The computation is
	\begin{equation}
		\begin{split}
			p_{sj}(\lambda^{-m}a_{\lambda})&=\sup_{ \substack{(x,\xi)\in K_j\\ \alpha,\beta\text{ st } |\alpha|+|\beta|\leq s }}|\DxaDxb(\lambda^{-m}a_{\lambda})(x,\xi)|\\
			&=\lambda^{-m}\sup_{\ldots}\lambda^{|\beta|}(\DxaDxb a)(x,\lambda\xi)|\\
			&=\lambda^{-m+|\beta|}\sup_{\ldots}|(\DxaDxb a)(x,\lambda\xi)|.
		\end{split}
	\end{equation}
	Then
	\begin{equation}
		p_{s,m}(\lambda^{-m} a_{\lambda})\leq \begin{cases}
			c_{\alpha,\beta,L}(1+R)^{m-|\beta|} & \text{if }|\beta|\leq m \\
			c_{\alpha,\beta,L}r^{m-|\beta|}     & \text{if }|\beta|> m.
		\end{cases}
	\end{equation}

	Now, we prove the reverse sense. Let us suppose that $p_{sj}$ is bounded, \emph{i.e.}
	\[
		| \lambda^{-m+|\beta|}(\DxaDxb a)(x,\lambda\xi)|\leq A
	\]
	with $x\in L$, $r\leq\abxi\leq R$ and $\lambda\geq 1$. Then, for $x\in L$ and $\abxi\geq r$,
	\begin{equation}\label{eq:2409r1}
		|(\DxaDxb a)(x,\xi)|\leq A\left(\frac{\abxi}{r}\right)^{m-|\beta|}.
	\end{equation}

	Since for $\abxi\geq r$,
	\[
		\frac{r}{1+r}\leq\frac{\abxi}{1+\abxi}\leq 1,
	\]
	one can ``forget''\ the $(1/r)^{m-|\beta|}$ in \eqref{eq:2409r1}: it can be absorbed in a constant which depend on $\beta$. By the same as in the proof of proposition~\ref{prop:23.16.4}, we can also replace $\abxi^{m-|\beta|}$ by $(1+\abxi)^{m-|\beta|}$; this proves that $a$ is a symbol of order $m$.

\end{proof}

\subsection{Topology on \texorpdfstring{$\mS^m(X\times\eR^N)$}{SmXRN}}
%---------------------------------------------

We will endow the vector space $\mS^m(X\times\eR^N)$ with a locally convex topological structure.

Let $(p_k)$ be the family of seminorms defining the topology of $\cdE(X\times\eR^N)$; we consider their restriction to $\mS^m(X\times\eR^N)$. We also consider $(p'_k)$, the one which defines the topology of $\cdE(\Xrnz)$. Now we pose for $a\in\mS^m(\Xrnz)$,
\[
	q_k(a)=\sup_{\lambda\geq 1}p'_k(\lambda^{-m}a_{\lambda}).
\]
One can see that these $q_k$ are seminorms. Finally, we put on $\mS^m(X\times\eR^N)$ the topology defined by the $p_k$ \emph{and} the $q_k$.

First remark: this topology on $\mS^m$ is finer\angl than the one which is induced from $\cdE(X\times\eR^N)$. In particular, $\mS(X\times\eR^N)$ is metrisabe\angl.

\begin{lemma}
	Each seminorm defining the topology of a Fréchet space is continuous.
	\label{lem:q_cont}
\end{lemma}
\begin{proof}
	Let $q$ be one of them. By proposition~\ref{prop:semi_norm_cont}, it is sufficient to find a neighbourhood of $0$ on which $q$ is bounded. Let us show that
	\[
		B(0;q,r)=\{x\in\mS^m|q(x)<r\}
	\]
	works. Indeed, it is an open set by definition of the topology, we just have to prove that this contains $0$, \emph{i.e.} we have to find a $r$ such that $q(0)<r$.
	\[
		q(0)=sup_{\lambda\geq 1}p'(0)=p'(0)
	\]
	where $p$ defines the topology of $\cdE(\Xrnz)$. In proposition~\ref{prop:topo_E}, we see that $p'(f)=sup_{\ldots}|(D^{\ldots}f)(x)|$. It is clear that this is zero when $f\equiv 0$.


\end{proof}

\begin{proposition}
	For this topology, $\mS^m(X\times\eR^N)$ is a Fréchet space.
\end{proposition}
\begin{proof}
	We already know that $\mS^m(X\times\eR^N)$ is locally convex space because its topology is defined by seminorms. It is also Hausdorff because its topology is finer than the one of $\cdE$ which is itself separable. Now, we consider $\cdE(X\times \eR^N)$ as an additive group, so that we can use the theory of metrisable groups developed in point~\ref{sec:metrisable_groups}. Thus we can speak of \defe{Cauchy sequences}{Cauchy sequences}: a sequence $(f_n)$
	in $\cdE(X\times \eR^N)$ is Cauchy when

	$\forall\,V\in\mV(0)$, $\exists\,n_0$ such that $\forall\,m,n\geq n_0$, $f_n-f_m\in V$,

	\noindent where $\dpt{0}{X\times \eR^N}{\eR}$ is the null function.

	From proposition~\ref{prop:E_Frechet}, we know that $\cdE(\mU)$ is a Fréchet space; in particular, it is complete for the distances which defines the Cauchy sequences (the ones whose are invariant under translations).

	Let us consider a Cauchy sequence $(a^n)$ in $\mS^m(X\times\eR^N)$. It converges (in the sense of $\cdE(X\times \eR^N)$) to $b\in\cdE(X\times \eR^N)$. A Cauchy sequence is always contained in a compact set, and by lemma~\ref{lem:q_cont}, $q_k$ is continuous for any $k$. Thus for each $k$ the sequence (in $\eR$) given by $(q_k(a^n))$ is bounded (continuous function on a compact).

	So, for each $j$, the set $\{p'_k(\lambda^{-m}a^n_{\lambda})\}$ is bounded when $\lambda,n\geq 1$. If we let $n$ goes to infinity, we find that $\{p'_k(\lambda^{-m}b_{\lambda}):\lambda>1\}$ is bounded, and then $b\in\mS^m(X\times\eR^N)$.

	We had just proved that $b$ where the limit of $(a^n)$ in the sense of $\cdE(X\times\eR^n)$; we yet have to see that $b$ is also the limit in the sense of $\mS^m(X\times\eR^n)$. As $(a^n)$ is a Cauchy sequence, proposition~\ref{prop:Cauchy_metrisable} assures us that

	$\forall\,V\in\mV(0)$, $\exists\,n_0$ such that $\forall\,m,n\geq 0$, $f_n-f_m\in V$.

	\noindent In particular, a neighbourhood $V$ of $0$ is for example $\{x\in\mS^m|q_k(x)<r\}$. Then $\forall\epsilon>0$, $\exists\,n_0$ such that $i$, $j>0$ implies $q_k(a^i-a^j)<\epsilon$. So $b$ is the limit of $(a^n)$ when $n\to\infty$ in the sense of $\mS^m(X\times\eR^N)$.

\end{proof}

\begin{lemma}   \label{LemHIUsKABh}
	Let $B$ be a bounded set in $\cdE(\mU)$ and $A>0$ such that $f(z)>A$ for all $f\in B$ and for all $z\in\mU$. Then the set $\{f^s\}_{f\in B}$ is bounded in $\cdE(\mU)$.
\end{lemma}

\begin{proof}
	We have to show that $(D^{\nu}(f^s))(z)$ is bounded when $f$ runs over $B$ and $z$ keeps in a compact subset of $\mU$. Let's use an induction on $|\nu|$ in order to prove that
	\[
		D^{\nu}(f^s)=f^{s-|\nu|}P_{\nu}\big(  (d^{\rho}f_{\rho\leq\nu})   \big)
	\]
	where $P_{\nu}$ is a polynomial with coefficients independent of $f$. Indeed, let $\nu=\{\nu_0,\sigma\}$ where $\nu_0$ is a multi-index and $\sigma$ a single index.
	\begin{equation}
		\begin{split}
			D_{\nu}(f^s)=D_{\sigma}D_{\nu_0}(f^s)&=D^{\sigma}\left( f^{s-|\nu_0|}P_{\nu_0}\big( (D^{\rho}f)_{\rho\leq\nu_0} \big) \right)\\
			&=(s-|\nu_0|)f^{s-|\nu_0|-1}(D^{\sigma}f)P_{\nu_0}\left( (D^{\rho}f)_{\rho\leq\nu_0} \right)\\
			&\quad    +f^{s-|\nu_0|}D^{\sigma}\left( P_{\nu_0}(D^{\rho}f)_{\rho\leq\nu_0} \right).
		\end{split}
	\end{equation}
	The quantity $P_{\nu}\left( (D^{\rho}f)_{\rho\leq\nu} \right)$ remains bounded because $|f(z)|^{-1}\leq A^{-1}$ and $z$ belongs to a compact set. The function $f$ take bounded values (continuous on a compact set) and while $s-|\nu|>0$, the function $f^{s-|\nu|}$ is bounded too. If $s-|\nu|<0$, there can be a problem, but th condition $f(z)\geq A$ avoid this case.
\end{proof}

\begin{theorem} \label{tho:lenumf}
	Multiplication and derivations on symbols behave rather well:
	\begin{enumerate}
		\item \label{enufi}  For all multi-index $\gamma$, $\delta$, the map $a\to D^{\gamma}_xD^{\delta}_{\xi}a$ is a continuous linear map from $\mS^m$ to $\mS^{m-|\delta|}$.
		\item \label{enufii} The map  $(a,b)\to ab$ is continuous and bilinear from $\mS^{m}\times\mS^{m'}$ into $\mS^{m+m'}$.
		\item \label{enufiii} Take $a\in\mS^m$. The function $a^{-1}$ is defined and is a symbol of order $-m$ if and only if for all compact $H\subset X$ there exists a constant $c_H>0$ such that, in $H\times\eR^n$, the condition
		      \[
			      |a(x,\xi)|\geq c_H(1+|\xi|)^m
		      \]
		      holds. In this case, for all real number $s$, we have $|a|^s\in\mS^{sm}$.
	\end{enumerate}
\end{theorem}

We only prove the point~\ref{enufiii}.

\begin{proof}
	\subdem{Necessary condition}
	Let $a\in\mS^m$ such that $a^{-1}$ is well defined as symbol of order $-m$. By the definition of a symbol,
	\begin{equation}
		\left| \frac{1}{a(x,\xi)}  \right|\leq c_H(1+|\xi|)^{-m}
	\end{equation}
	for a certain $c_H$ determined by the compact $H$ in which $x$ moves. Then $|a(x,\xi)|\geq c_H^{-1}(1+|\xi|)^m$ and the claim follows

	\subdem{Sufficient condition} Let $a\in\mS^m$ and for each compact $H\subset X$, there exists a $c_H>0$ such that $\forall(x,\xi)\in H\times \eR^N$,
	\[
		|a(x,\xi)|\geq c_H(1+|\xi|)^m.
	\]
	By point~\ref{enufii}, $|a|^2\in\mS^{2m}$ and $a^{-1}=\bar a (|a|^3)^{-1}$. Let us suppose that the theorem is proved when $a(x,\xi)>0$ on $X\times\eR^N$. So suppose that $a<0$ somewhere. In this case, $a^{-1}=\bar a(|a|^2)^{-1}$ where $|a|^2$ is positive in such a manner that our assumption makes it a symbol of order $-2m$; now by~\ref{enufii}, $a^{-1}$ is a symbol of order $-m$.

	So we can restrict ourself to the case where $a(x,\xi)>0$ in $X\times\eR^N$. It is clear that $\lambda^{-ms}(a_{\lambda}^s)=(\lambda^{-m}a_{\lambda})^s$. From theorem~\ref{tho:dieu23.16.6} we just have to prove that
	$\lambda^{-m}a_{\lambda})^s$ is a bounded set when $s=-1$. Lemma~\ref{LemHIUsKABh} shows it for all $s$.

\end{proof}

\begin{proposition}
	Let $Y$ be an open subset of $\eR^p$,
	\[
		\dpt{\psi}{Y\times\eR^P}{X}
	\]
	and
	\[
		\dpt{\theta}{Y\times\eR^P}{\eR^N},
	\]
	two $C^{\infty}$ functions such that for large $|\xi|$, the function $\psi$ is positively homogeneous of degree zero while $\theta$ is of degree $1$ with respect to $\xi$.

	Then for all symbol $a\in\mS^m(X\times\eR^N)$, the function
	\begin{equation}
		\begin{aligned}
			b \colon Y\times\eR^P & \to \eR\
			(x',\xi')             & \mapsto a(\psi(x',\xi'),\theta(x',\xi'))
		\end{aligned}
	\end{equation}
	is a symbol of $\mS^m(Y\times\eR^P)$ and the linear map $\mS^m(X\times\eR^N)\to\mS^m(Y\times\eR^P)$ given by $a\to a\circ(\psi,\theta)$ is continuous.


\end{proposition}

\begin{proof}
	We pose $F=(\psi,\theta)$, so that $b=a\circ F$. Homogeneity assumptions make that when $\lambda\geq 1$ and when $|\xi|$ is large,
	\[
		\lambda^{-m}b_{\lambda}=(\lambda^{-m}a_{\lambda})\circ F.
	\]
	Then the problem reduces to prove that if $B$ is bounded in $\cdE(X\times\eR^N_0)$, then the set $f\circ F$ is bounded when $f$ runs over $B$.
\end{proof}


\begin{lemma}
	Let $m<m'\in\eR$. Then the canonical injection
	\[
		\dpt{\id}{\mS^m}{\mS^{m'}}
	\]
	is continuous.

\end{lemma}


\begin{proof}
	Let $\mO$ be an open set in $\mS^{m'}$; we have to show that $\id^{-1}(\mO)=\mO\cap\mS^{m}$ is open in $\mS^m$. Let $p_k$ be the family of seminorm defining the topology on $\cdE(X\times\eR^N)$ and $p'_k$ the one of $\cdE(X\times\eR^N_0)$ and then
	\[
		q_k(a)=\sup_{\lambda\geq 1}p'_k(\lambda^{-m}a_{\lambda}).
	\]
	The topology of $\mS$ is defined from $p_k$ and $q_k$. We define the ball
	\[
		B(a,d,r)=\{x\in E\tq d(a,x)<r\}.
	\]
	The property for $\mO\subset\mS^{m'}$ to be open is that one of these ball is included in $\mO$. If a ball build with one of the $p_k$'s is included in $\mO$, there are no problems because the seminorms $p_k$ are also in the definition of the topology on $\mS^m$. Let
	\[
		B(a,d_k^{m'},r)=\{x\in\mS^{m'}\tq q_k^{m'}(a-x)<r\}.
	\]
	The candidate ball of $\mS^m$ to be included in $\mO$ is
	\[
		\mO'=\{x\in\mS^m\tq q_k^m(a-x)<\overline{r}\}
	\]
	for a certain $\overline{r}<r$. Let us prove that $q_k^m(y)\leq q_k^{m'}(y)$. Here, $k=(n,s)$.
	\begin{equation}
		\begin{split}
			q_k^m(y)&=\sup_{\lambda\geq 1}\sup_{\substack{x\in K_n\\|\nu|<s\|}}| D^{\nu}(\lambda^{-m}y_{\lambda})(x) |\\
			&=\sup_{\lambda\geq 1}\lambda^{m'-m}\sup_{\substack{x\in K_n\\|\nu|<s\|}}| D^{\nu}(\lambda^{-m'}y_{\lambda})(x) |\\
			&\leq\sup_{\lambda\geq 1}p'_k(\lambda^{-m'}y_{\lambda})\\
			&=q_k^{m'}(y).
		\end{split}
	\end{equation}

\end{proof}

\begin{lemma}
	The adherence of $\cdD(X\times\eR^N)$ in $\mS^m(X\times \eR^N)$ contains $\mS^{m'}(X\times\eR^N)$ for all $m'<m$.
	\label{lem:DadhS}
\end{lemma}

One can characterize the topology by another choice of seminorms as the following proposition shows.

\begin{proposition}
	The topology of Fréchet space on $\mS^(X\times\eR^N)$ is defined by the seminorms
	\[
		p_{K,\alpha,\beta}(a)=\sup_{x\in K,\xi\in\eR^N}(1+|\xi|)^{-m+|\beta|}| D_x^{\alpha}D_{\xi}^{\alpha}a(x,\xi) |.
	\]
	\label{prop:topo_alter}
\end{proposition}

\subsection{Asymptotic expansions}
%--------------------------------------

Note that $\cdD(X\times \eR^N)\subset\mS^{-\infty}(X\times\eR^N)$ because a compact supported function is always a symbol of all order. Let us give without proof the two following results \cite{Dieu7}.


\begin{lemma}
	Let $a\in\mS^{m'}(X\times\eR^N)$ a symbol whose support is contained in $H\times\eR^N$ where $H$ is a compact subset of $X$. Let $h\in\cdD(X\times\eR^N)$, a function equals to $1$ for $x\in H$ and $|\xi|\leq A$ for a certain constant $A>0$. Then for all $m>m'$,
	\[
		\lim_{q\to\infty}(h_{1/q}a)=a
	\]
	in the sense of the convergence in $\mS^m$.
	\label{lem:limha}
\end{lemma}

\begin{proposition}
	The adherence of $\cdD(X\times\eR^N)$ in $\mS^m(X\times\eR^N)$ contains $S^{m'}(X\times\eR^N)$ for all $m'<m$.
\end{proposition}

The main theorem is

\begin{theorem}
	Consider a strictly increasing sequence in $\eR$ $(m_j)$ with $\lim_{j\to\infty}m_j=-\infty$. Let, for each $j$, $a_j\in\mS^{m_j}(X\times\eR^N)$. Then there exists $a\in \mS^{m_0}$ such that for all $k$,
	\begin{equation}  \label{eq:dev_ass}
		a-\sum_{j<k}a_j\in\mS^{m_k}(X\times\eR^N).
	\end{equation}
	Furthermore if $a'$ shares this property with $a$, then $a-a'\in\mS^{-\infty}(X\times\eR^N)$.
	\label{tho:dev_ass}
\end{theorem}


\begin{proof}
	The last assertion is easy. Let us subtract equations \eqref{eq:dev_ass} for two such symbols: for all $k$,
	\[
		a-a'\in\mS^{m_k}(X\times\eR^N).
	\]
	Then $a-a'\in\mS^{X\times\eR^N}$.

	Now consider a locally finite countable covering $(\mU_{\alpha})$ of open relatively compact sets and $(f_{\alpha})$, a partition of unity for this covering. Let us suppose that, for each $\alpha$, we know a symbol $b_{\alpha}$ of order $m_0$ which is zero outside $\mU_{\alpha}\times\eR^N$ and such that for all $k$,
	\[
		b_{\alpha}=\sum_{i<k}f_{\alpha}a_j\in\mS^{m_k}(X\times\eR^N).
	\]
	In other words, let us suppose that we have the answer for the symbols $f_{\alpha}a_j$ instead of the symbols $a_j$. Then the symbol $a=\sum_{\alpha}b_{\alpha}$ answer the question for the symbols $a_j$. Indeed
	\begin{equation}
		\begin{split}
			\sum_{\alpha}b_{\alpha}-\sum_ja_j=\sum_{\alpha}(b_{\alpha}-\sum_jf_{\alpha}a_j)
		\end{split}
	\end{equation}
	where the sums are pointwise \emph{finite} sums because the covering $(\mU_{\alpha})$ is locally finite.

	So we are left to consider $a_j$ vanishing outside a set $H_j\times\eR^N$ where $H_j$ is a compact subset of $X$. For $j\geq 1$, let us define
	\begin{equation}
		a_{j,q}=(1-h_{j,1/q})a_j\in\mS^{m_j}
	\end{equation}
	where $h_j\in\cdD(X\times\eR^N)$ fulfils $h_j(x,\xi)=1$ when $x\in H_j$ and $0\leq|\xi|\leq 1$. By $h_{i,1/q}$, we mean $(h_j)_{1/q}$ in the sense of equation \eqref{eq:def_ablambda}. Since it has a compact support, it is a symbol for all orders. From lemma~\ref{lem:limha},
	\[
		\lim_{q\to\infty}(a_{j,q})=\lim_{q\to\infty}a_j-\lim_{q\to\infty}h_{j,1/q}a_j=0.
	\]
	Proposition~\ref{prop_suiteFk} makes that for all $r\geq 0$, the sum $\sum_{j\geq r}a_{j,q_j}$ converges in $\mS^{m_r}$. We are now going to prove that the symbol $a=\sum_{j\geq 0}a_{j,q_j}$ is the one needed by the theorem. First remark that $a_{j,q}$ has support contained in $H_j\times\eR^N$. For all $q>0$,
	\[
		a_j-a_{j,q}\in\cdD(X\times\eR^N)\subset\mS^{-\infty}(X\times\eR^N).
	\]
	Now,
	\[
		\Big( a-\sum_{j<k}a_j \Big)-\Big( a-\sum_{j<k}a_{j,q_j} \Big)=\sum_{j<k}\Big( a_{j,q_j}-a_j \Big)\in\mS^{-\infty}(X\times\eR^N),
	\]
	then
	\[
		a-\sum_{j<j}a_{j,q_j}=\sum_{l\geq k}a_{l,q_l}\in\mS^{m_k}
	\]
	from lemma~\ref{prop_suiteFk}.

\end{proof}

When equation \eqref{eq:dev_ass} holds , we say that the sum $\sum_ja_j$ is an \defe{asymptotic expansion}{asymptotic expansion} of the symbol $a$ and we write $a\sim\sum_ja_j$. In this case, for all $k$ we have
\begin{equation} \label{eq:assa}
	a-\sum_{j<k}a_j\in\mS^{m_k},
\end{equation}
thus for any multi-index $\nu$, the following holds
\[
	D^{\nu}a\sim\sum_jD^{\nu}a_j
\]
as can be seen by derivation of equation \eqref{eq:assa} and using point~\ref{enufi} of theorem~\ref{tho:lenumf}.

\subsection{Tempered functions}
%------------------------------

A function $f\in\cdE(X\times\eR^N)$ is \defe{tempered with respect to $|\xi|$}{tempered!function} if for all compact $L\subset X$, and for all pair of multi-index $(\alpha,\beta)$, there exists constants $c(\alpha,\beta,L)$ and $p(\alpha,\beta,L)$ such that
\[
	|D_x^{\alpha}D_{\xi}f(x,\xi)|\leq c(\alpha,\beta,L)(1+|\xi|)^{p(\alpha,\beta,L)}
\]
for all $(x,\xi)\in L\times\eR^N$.


\begin{lemma}
	Let $c\in\cdE(X\times\eR^N)$ be a tempered function with respect to $\xi$ such that for all compact $L\subset X$ and $q>0\in\eN$, there exists a constant $C_{qL}$ such that
	\[
		|c(x,\xi)|\leq C_{qL}(1+|\xi|)^{-q}
	\]
	for all $(x,\xi)\in L\times\eR^N$. Then $c\in\mS^{-\infty}(X\times\eR^N)$.
	\label{lem:csymbol}
\end{lemma}

The following proposition gives a link between tempered functions and symbols.

\begin{proposition}
	Let $X$ be an open set in $\eR^N$ and $(m_j)\in\eR$ a strictly decreasing sequence such that $\lim_{j\to\infty}m_j=-\infty$. For each $m_j$, we consider a symbol $a_j\in\mS^{m_j}$. Let $a\in\cdE(X\times\eR^N)$, a tempered function with respect to $\xi$ and suppose that for all compact $L\subset X$, there exists a decreasing sequence $(q_k)\in\eR$ with $\lim_{k\to\infty}q_k=-\infty$ and for each $k$, a constant $C_{kL}$ such that
	\begin{equation}
		|a(x,\xi)-\sum_{j<k}a_j(x,\xi)|\leq C_{kL}(1+|\xi|)^{q_k}
	\end{equation}
	for all $(x,\xi)\in L\times\eR^N$.

	Then $a$ is a symbol of order $m_0$ and admits the asymptotic expansion
	\[
		a\sim\sum_ja_j.
	\]

\end{proposition}

\begin{proof}
	Theorem~\ref{tho:dev_ass} holds for symbols $a_j$, then there exists a symbol $b\in\mS^{m_0}$ such that $b\sim\sum_{j}a_j$. So we have to prove that $a-b\in\mS^{\infty}$ because, in this case,
	\[
		a-\sum_{j<k}a_j=(a-b)-(\sum_{j<k}a_j-b)\in\mS^{m_k}.
	\]
	On the one hand, the function $a-b$ is tempered (because a symbol is always tempered). On the other hand, for all choice of $k$, we can write
	\begin{equation}
		\begin{aligned}
			|(a-b)(x,\xi)| & =|a-\sum_{j<k}a_j-b+\sum_{j<k}a_j|            \\
			               & \leq|a-\sum_{j<k}a_j|+|b-\sum_{j<k}a_j|       \\
			               & \leq C_{kL}(1+|\xi|)^{q_k}+D_L(1+|\xi|)^{m_k} \\
			               & \leq C'_{kL}(1+|\xi|)^{q'_k}
		\end{aligned}
	\end{equation}
	where $C'_{kL}=\max(C_{kL},D_L)$ and $q'_k=\max(q_k,m_k)$. Let us now fix $q>0$; since $q_k,m_k\to-\infty$, there exists $k$ such that $q'_k<-q$. For this one, we have
	\[
		|(a-b)(x,\xi)|\leq C''_{qL}(1+|\xi|)^{-q}.
	\]
	Lemma~\ref{lem:csymbol} concludes the proof.


\end{proof}


\section{First construction of oscillatory integrals}
%++++++++++++++++++++++++++++++++++++++++++++++++++++

\subsection{Construction}
%------------------------

Let us consider $X$ be an open set in $\eR^N$, a continuous map $\dpt{\varphi}{X\times\eR^N}{\eR}$ and a symbol $a\in\mS^{m}(X\times\eR^N)$. For any compact $L\subset X$, the integral
\[
	\int_{L\times\eR^N}e^{i\varphi(x,\xi)}a(x,\xi)dxd\xi
\]
converges when $m<-N$. Indeed
\begin{equation}
	\int |e^{i\varphi}a|\leq\int|e^{i\varphi}||a|
	\leq\int C(1+|\xi|)^m.
\end{equation}
The integral over $L$ is a constant on a compact while the one on $\eR^N$ is performed with spherical coordinates, sop we are left with $\int_{\eR^+}(1+r)^mr^{N-1}$. Under these assumptions, the form
\begin{equation}  \label{eq:formaprol}
	a\to\int_{L\times\eR^N}e^{i\varphi}a
\end{equation}
is linear. In order to see that is is continuous too, we will prove that there exists a neighbourhood of zero in $S^m(X\times\eR^N)$ in which any $a$ fulfils the majoration
\[
	|a(x,\xi)|\leq C(1+|\xi)^m
\]
for all $x(x,\xi)\in L\times\eR^N$ and a certain constant $C$. Let us use the seminorms given in proposition~\ref{prop:topo_alter}  and consider the neighbourhood
\[
	\mO=\{ a\in\mS^m\tq\sup_{x\in K,\xi\in\eR^N}(1+| \xi |)^{-m}| a(x,\xi) |<\varepsilon \}
\]
for a certain compact $K\subset X$ and $\varepsilon>0$. A symbol $a$ in $\mO$ fulfils in particular
$| a(x,\xi) |<\varepsilon(1+| \xi |)^m$. Then the linear form \eqref{eq:formaprol} is bounded and thus continuous.

\begin{definition} \label{def:phase}

	A function $\dpt{\varphi}{X\times\eR^N_0}{\eR}$ is a \defe{phase function}{phase!function} if

	\begin{enumerate}
		\item $\varphi\in C^{\infty}(X\times\eR^N_0)$,
		\item the $n+N$ first derivatives $D'_j\varphi$ and $D''_k\varphi$ doesn't vanish at same point of $X\times\eR^N_0$, in other words, $\varphi$ is not singular
		\item for all $(x,\xi)\in X\times\eR^N_0$ and all $\lambda>0$, the formula $\varphi(x,\lambda\xi)=\lambda\varphi(x,\xi)$ holds.
	\end{enumerate}

\end{definition}
Here, $D'_j$ denotes the derivative with respect to $x_j$ and $D''_k$ the one with respect to $\xi_k$. The third point implies that the prolongation $\varphi(X\times\{  0\})=0$ in continuous in $X\times\eR^N$ but not $C^{\infty}$ as the example $\varphi(x,\xi)=| \xi |$ shows with $N=1$.

An important example of phase function is given when $N=n$ by $\varphi (x,\xi)=-2\pi \scal{x}{\xi}$; it gives rise to the Fourier transforms.

An important result will help us to define oscillatory integrals

\begin{lemma}  \label{lem:defL}
	Let $\varphi$ be a phase. There exists symbols $a_k\in \mS^0$, $b_j\in\mS^{-1}$ and $c\in\mS^{-\infty}$ ($1\leq j\leq N$ and $1\leq j\leq n$) such that the differential operator
	\[
		L=\sum_{k=1}^Na_kD''_k+\sum_{j=1}^nb_jD'_j+c\mtu
	\]
	on $X\times\eR^N$ fulfil
	\[
		Le^{i\varphi}=e^{i\varphi}
	\]
	in $X\times\eR^N_0$. The transposed $L^t$ of $L$ is given by
	\[
		L^t=-\sum_{a_k}D''_k-\sum_{j=1}^nb_jD'_j+c'\mtu
	\]
	where $c'=c-\sum D''_ka_k-\sum D'_jb_j\in\mS^{-1}$.

	For all $u\in\cdE(X\times\eR^N)$ and $r\in\eN^+_0$,
	\[
		\big( (L^t)^ru \big)(x,\xi)=\sum_{| \alpha |+| \beta |\leq r}g_{\alpha\beta}(x,\xi)D_x^{\alpha}D_{\xi}^{\beta} u (x,\xi)
	\]
	where each $g_{\alpha\beta}$ is a symbol of order $-r+| \beta |$ on $X\times\eR^N$, independent of $u$. In other words,
	$(L^t)^r= \sum_{| \alpha |+| \beta |\leq r}g_{\alpha\beta}D_x^{\alpha}D_{\xi}^{\beta}$.

	In particular $a\to (L^t)ra$ is a continuous linear map from $\mS^m$ to $\mS^{m-r}$.

\end{lemma}

We will not prove all assertions.

\begin{proof}
	We consider the function
	\[
		g(x,\xi)=| \xi |^2\sum_{k=1}^N\big( D''_k\varphi(x,\xi) \big)^2+\sum_{j=1}^n\big( D'_j\varphi(x,\xi)^2.
	\]
	Since $(D''_k\varphi)(x,\lambda\xi)=(D''_k\varphi)(x,\xi)$, the function $g$ is positively homogeneous of degree $2$ with respect to $\xi$ and from the fact the $\varphi$ has no critical points, there is always at least one no zero term the two sums, then $g(x,\xi)\neq 0$ on $X\times\eR^N_0$. Now we consider a $C^{\infty}$ function $\dpt{h}{\eR}{\eR}$ with compact support (hence a symbol of all order) and equals to $1$ in a neighbourhood of zero.

	We are going to study the function $\frac{1-h(| \xi |)}{g(x,\xi)}$. In a neighbourhood of zero, the numerator is zero; then we don't mind with the values of $1/g(x,\xi)$ when $| \xi |\to 0$. Proposition~\ref{prop:23.16.4} shows that $g\mS^2(X\times\eR^N)$. On the other hand the inequality
	\[
		| g(x,\xi)\geq C_H(1+| \xi |) |
	\]
	holds for a certain constant $C_H$ because both side are of degree two. Then point~\ref{enufi} of theorem~\ref{tho:lenumf} assures that $1/g$ is a symbol of $ \mS^{-2}$. So $\frac{1-h(| \xi |)}{g(x,\xi)}\in\mS^{-2}$. Now we claim that $Le^{i\varphi}=e^{i\varphi}$ when
	\begin{equation}
		\begin{aligned}
			a_k(x,\xi) & =| \xi |^2\frac{1-h(| \xi |)}{ig(x,\xi)}D''_k\varphi(x,\xi) \\
			b_j(x,\xi) & =\frac{1-h(| \xi |)}{ig(x,\xi)}D'_j\varphi(x,\xi)           \\
			c(x,\xi)   & =h(| \xi |).
		\end{aligned}
	\end{equation}
	A simple counting shows that $a_k\in\mS^0$ and $b_j\in\mS^{-2}$. The computation is easy:
	\begin{equation}
		\begin{split}
			(Le^{i\varphi})(x,\xi)&=\sum_ka_k(x,\xi)(D''_ke^{i\varphi})(x,\xi)\\
			&\quad+\sum_j(x,\xi)(D'_je^{i\varphi})(x,\xi)\\
			&\quad+c(x,\xi)e^{i\varphi(x,\xi)}\\
			&=e^{i\varphi}\frac{1-h(| \xi |)}{ig}\Big( i\sum_k| \xi |^2 (D''_k\varphi)^2+\sum_ji(D'_j\varphi)^2 \Big)+h(| \xi |)e^{i\varphi}\\
			&=e^{i\varphi}.
		\end{split}
	\end{equation}
	We don't prove the other assertions.

\end{proof}

The theorem which defines the oscillatory integrals is the following

\begin{theorem}
	Let $\varphi$ be a phase on $X\times\eR^N$. For all $m\in\eR$ and for all compact $H\subset X$, the $\eC$-linear form $\cdD(X\times\eR^N)\to\eC$
	\begin{equation}
		a\to \int_{H\times\eR^N}e^{i\varphi(x,\xi)}a(x,\xi)dxd\xi
	\end{equation}
	admits an unique extension into a continuous map $S^m(X)\times\eR^N\to\eC$.

\end{theorem}

\begin{proof}\label{eskdfml}
	As far as integration is concerned, the set $X\times\{ 0 \}$ is negligible, then one can apply the formula $Le^{i\varphi}=e^{i\varphi}$ although id is only true on $X\times\eR^N_0$. From construction of $L$ and the definition of the transpose,
	\[
		\scald{e^{i\varphi}}{w}=\scald{Le^{i\varphi}}{w}=\scald{e^{i\varphi}}{L^tw},
	\]
	then in $\cdD(X\times\eR^N)$, the studied linear form reads
	\begin{equation} \label{eq:intaprol}
		a\to\int_{H\times\eR^N}e^{i\varphi}(L^t)^ra
	\end{equation}
	for any integer $r>0$. If one choses $r>m+N$, then $(L^t)^ra\in\mS^{m-r}$ if $a\in\mS^m$ and the integral exists and is a continuous function of $a$ from discussion below equation \eqref{eq:formaprol}. Then the integral in equation \eqref{eq:intaprol} defines a prolongation of the map $u\to\int e^{i\varphi}u$ into the whole $\mS^m$ when $r>m+N$.

	Lemma~\ref{lem:DadhS} makes that any element in $\mS^mX\times\eR^N$ is adherent to $\cdD(X\times\eR^N)$, then a continuous prolongation is unique.

\end{proof}

The (prolonged) linear form
\[
	a\to\int_{H\times\eR^N}e^{i\varphi}a
\]
is continuous at all $a\in\mS^m$ for all $m\in\eR$. If $a\in\mS^{\infty}(X\times\eR^N)$, we write it by
\begin{equation}
	a\to \osiint_{H\times\eR^N}e^{i\varphi(x,\xi)}a(x,\xi)dxd\xi,
\end{equation}
and it is called an \defe{oscillatory integral}{oscillatory integral}. It's value is given by formula
\[
	\osiint_{H\times\eR^N}e^{i\varphi}a=\int_{H\times\eR^N}e^{i\varphi}(L^t)^ra
\]
with $r>m+N$. Let $h\in\cdD(X\times\eR^N)$ equals to $1$ at $(x,\xi)$ if $x\in H$ and $| \xi |<R$; then lemma~\ref{lem:limha} implies $\lim_{q\to\infty}h_{1/q}a=a$ in the sense of $\mS^m$ when $a\in\mS^{m'}$ and $m>m'$. Since the oscillatory integral is continuous for each $\mS^m$, one can commute the limit and the integral:
\[
	\osiint_{H\times\\eR^N}e^{i\varphi}a=\lim_{q\to\infty}e^{i\varphi}h_{1/q}a.
\]

\subsection{Parametric oscillatory integral}
%-------------------------------------------

Let us introduce a parameter in the integral. We suppose the parameter belongs to an open set $\mU\subset\eR^N$ and we consider a symbol $a\in\mS^m(X\times\mU\times\eR^N)$. For each $z\in\mU$, the function $(x,\xi)\to a(x,z,\xi)$ is a symbol of order $m$ in $X\times\eR^N$. For all compact sets $H\subset X$ and $K\subset\mU$, there exists a constant $C_{HK}$ (independent of $z\in\mU$) such that
\begin{equation}
	| D^{\alpha}_xD^{\beta}_{\xi}a(x,z,\xi) |\leq C_{HK}(1+| \xi |)^{m-| \beta |}
\end{equation}
for all $(x,z,\xi)\in H\times K\times\eR^N$. This comes from the fact that all compact in $X\times\mU$ is the product of a compact in $X$ by a compact in $\mU$.


\begin{theorem}
	Let $X$ be an open set in $\eR^N$, $\mU$ an open in $\eR^q$ and $\dpt{\varphi}{X\times\mU\times\eR^N}{\eR}$ be a continuous function, $C^{\infty}$ in $X\times\mU\times\eR^N_0$ such that for all $z\in\mU$, the function $(x,\xi)\to(x,z,\xi)$ is a phase on $X\times\eR^N$. Then

	\begin{enumerate}
		\item for all $a\in\mS^m(X\times\mU\times\eR^N)$ and all compact $H\subset X$, the map
		      \[
			      z\to F_a(z)=\osiint_{H\times\eR^N}e^{i\varphi(x,z,\xi)}a(x,z,\xi)dxd\xi
		      \]
		      is $C^{\infty}$ in $\mU$ and $a\to F_a$ is continuous from $\mS^m(X\times\mU\times\eR^N)$ to $\cdE(\mU)$,

		\item the function $\varphi$ is a phase on $X\times\mU\times\eR^N$ and for all compact $K\subset\mU$,
		      \begin{equation} \label{eq:intKFa}
			      \int_K F_a(z)dz=\osiint_{H\times K\times\eR^N}e^{i\varphi}a.
		      \end{equation}

	\end{enumerate}

\end{theorem}

\begin{proof}
	Let us begin by proving that $\varphi$ is a phase on $X\times\mU\times\eR^N$. From assumptions it is $C^{\infty}$ on $X\times\mU\times\eR^N_0$, the positive homogeneity comes from the fact that for each $z\in\mU$, the function $(x,\xi)\to(x,z,\xi)$ is positively homogeneous and the partial derivatives cannot vanish at same point because the one with respect to $x$ and $\xi$ yet doesn't.

	Now we define $\varphi_z=\varphi(x,z,\xi)$ and the operator $L_z$ given by lemma~\ref{lem:defL}. Then for all $u\in\cdE(X\times\mU\times\eR^N)$ and for all $r>0$,
	\[
		\big((L_z^t)^ru\big)(x,z,\xi)=\sum_{| \alpha |+| \beta |}g_{\alpha\beta}(x,z,\xi)D_x^{\alpha}D_{\xi}^{\beta}u(x,z,\xi)
	\]
	where $g_{\alpha\beta}$ is a symbol of order $-r+| \beta |$ in $X\times\mU\times\eR^N$, see proof of lemma~\ref{lem:defL}. Since $L_ze^{i\varphi_z}=e^{i\varphi_z}$, the equation
	\[
		F_a(z)=\int_{H\times\eR^N}e^{i\varphi}(L_z^t)^ra
	\]
	holds for all $a\in\cdD(X\times\mU\times\eR^N)$. If $a$ is just a symbol, the problem is no more complicated because the definition is
	\[
		\osiint_{H\times\eR^N}e^{i\varphi}a=\int_{H\times\eR^N}e^{i\varphi}(L_z^t)^a,
	\]
	with $r>m+N$. So we have to prove the continuity of
	\[
		z\to\int_{H\times\eR^N}e^{i\varphi}(L_z^t)^ra.
	\]
	We do it by using proposition~\ref{prop:fdefint}. First, for all $z\in\mU$, the map $(x,\xi)\to e^{i\varphi(x,z,\xi)(L_z^t)^r}a(x,z,\xi)$ is integrable from our choice of $r$. Second, the map $z\to e^{i\varphi(x,z,\xi)}(L_z^t)^ra(x,z,\xi)$ is continuous on the whole $\mU$. The third assumption of proposition~\ref{prop:fdefint} is the true point.
	\begin{equation}
		\begin{split}
			| e^{i\varphi}(L^t)^ra |	&\leq | L_z^ta |\\
			&=\Big| \sum_{| \alpha |+| \beta |\leq r}g_{\alpha\beta}(x,z,\xi)D_x^{\alpha}D_{\xi}^{\beta}a(x,z,\xi)    \Big|\\
			&\leq \sum | g_{\alpha\beta}(x,z,\xi) | |D_x^{\alpha}D_{\xi}^{\beta}a(x,z,\xi) |\\
			&\leq C_{\alpha\beta}C'_{\alpha\beta}(1+| \xi |)^{m-r}
		\end{split}
	\end{equation}
	which integrable over $H\times\eR^N$ because $r>m+N$.

	Now we prove equation \eqref{eq:intKFa}. For this, we write the definition of the oscillatory integral of the right hand side and we compute:
	\begin{equation}
		\begin{split}
			\osiiint_{H\times K\times\eR^N}e^{i\varphi}a	&=\iiint_{H\times K\times\eR^N}e^{i\varphi}(L^t)^ra \\
			&=\iiint_{H\times K\times\eR^N}L^re^{i\varphi_z}a\\
			&=\iiint_{H\times K\times\eR^N}L_z^re^{i\varphi_z}\\
			&=\iiint_{H\times K\times\eR^N}e^{i\varphi_z(x,\xi)}(L_z^t)^ra(x,z,\xi)dxdzd\xi\\
			&=\int_K\osiint_{H\times\eR^N}e^{i\varphi_z}a\,dz\\
			&=\int_KF_a(z)dz.
		\end{split}
	\end{equation}
	We don't give the rest of the proof.

\end{proof}

We give without proof the formula
\begin{equation}
	D_z^{\alpha}F_a(z)=\osiint_{H\times\eR^N}D_z^{\alpha}(e^{i\varphi}a).
\end{equation}

Now let us consider the situation in which $z\in\mU$ is the only variable in front of $\xi$. More precisely, consider a function $\dpt{\varphi}{\mU\times\eR^N}{\eR}$ which is $C^{\infty}$ on $\mU\times\eR^N_0$ and continuous on $\mU\times\eR^N$ such that for all $z\in\mU$, the function $\xi\to\varphi(z,\xi)$ is positively homogeneous of degree $1$, without critical points, i.e. $\varphi_z$ is a phase in the sense of definition~\ref{def:phase} with $X=\emptyset$. Then for all symbol $a\in\mS^m(\mU\times\eR^N)$, the map
\[
	z\to F_a(z)=\osint_{\eR^N}e^{i\varphi}a
\]
is $C^{\infty}$ on $\mU$ and  $a\to F_a$ is continuous from $S^m(\mU\times\eR^N)$ into $\cdE(\mU)$. One can moreover commutes the integral and the derivative
\[
	D_z^{\alpha}F_a(z)=\osint_{\eR^N}D^{\alpha}_z\big( e^{i\varphi(z,\xi)}a(z,\xi) \big)d\xi
\]
The function $\varphi$ is a phase on $\mU\times\eR^N$ and for all compact part $K$ of $\mU$, one has
\[
	\int_KF_a=\osiint_{K\times\eR^N}e^{i\varphi}a.
\]

\section{Other constructions of oscillatory integral}
%+++++++++++++++++++++++++++++++++++++++++++++++++++++

\subsection{Derivation definition}
%---------------------------------

This construction is very the same as the first one, and comes from \cite{Rieffel}.

We consider a strongly continuous action $\tau$ of $V$ on $ C_u(V,A)$. Let $\svec^A(V)$ be the set of smooth vectors for this action; if $W=V\times V$, the set $\svec^A(W)$ is well defined and we choose on $V$ a measure such that the unit cube has measure $1$. Let $\dpt{\varphi}{V\times V}{\eR}$, $\varphi(u,v)=2\pi u\cdot v$; we want to give a sense to the integral
\[
	\iint_{V\times V}F(u,v)e^{i\varphi(u,v)}dudv
\]
when $F\in\svec^A(W)$. As usual, we first consider the case where $F$ is compactly supported; in this case the integral makes sense with usual definitions. An integration by part gives
\[
	\iint_{V\times V}(\partial_{k,u}F)e^{i\varphi}=-\iint_{V\times V}2\pi i v_k Fe^{i\varphi}.
\]
Taking a second derivative and making the same with the variable $v$,
\[
	\iint_{V\times V}(\partial_{sv}\partial_{rv}+\partial_{ku}\partial_{lu})Fe^{i\varphi}=-4\pi^2\iint_{V\times V}(u_su_r+v_kv_l)Fe^{i\varphi};
\]
taking $s=r$ and $k=l$, and making sum, we find
\begin{equation} \label{eq:int_Delta}
	\iint_{V\times V}(\Delta F)e^{i\varphi}=-4\pi^2\iint_{V\times V} (u\cdot u+v\cdot v)Fe^{i\varphi}.
\end{equation}
Now we pose $w=(u,v)\in W$ and $Qw=u\cdot v$;  equation \eqref{eq:int_Delta} reads
\begin{equation} \label{eq:int_unmDelta}
	\int_W[(1-\Delta/4\pi^2)F](w)e^{iQw}=\int_WF(w)(1+w\cdot w)e^{iQw}.
\end{equation}
This equation only holds for compact supported functions $F$. In particular, it is also true for $F'=F/(1+w^2)$ where $w^2$ stands for $w\cdot w$. For notational convenience, we write $K(w)=(1+w^2)^{-1}$ and $M_K$ the operator of pointwise multiplication by $K(w)$. With $F'$ instead of $F$, equation \eqref{eq:int_unmDelta} gives $\int_W(1- \Delta/4\pi^2)(M_KF)e^{iQ}=\int_W Fe^{iQ}$. Making $k$ times all the work,
\begin{equation}
	\int_W [(1-\Delta/4\pi^2)M_K]^kFe^{iQ}=\int_WFe^{iQ}.
\end{equation}

Now let us see how to write $[(1-\Delta/4\pi^2)K]^kF$. Direct computation shows that
\[
	\Delta(KF)=K(\Delta F-2KF+8K^3w^2F)=K\sum_{| \nu |\leq 2}B_{\nu}\partial^{\nu}F
\]
where $B_{\nu}$ are bounded functions. By induction, we see that
\[
	[(1-\Delta )K]^kF=K^k\sum_{| \nu |\leq 2k}B_{\nu}\partial^{\nu}F.
\]
So one can write
\begin{equation} \label{eq:defintosc}
	\int_WFe^{iQ}=\int_WK^k\big( \sum_{| \nu |\leq 2k}B_{\nu}\partial^{\nu}F \big)e^{iQ}
\end{equation}
for all $k$ and all function $F$ with compact support. But we know that $K^k$ is integrable when $k>d$, and if $F$ is  a smooth vector for the action \eqref{eq:def_act_tau}, it is bounded on any bounded set (and all the derivatives of $F$ too). So the function $K^k\sum_{\nu}B_{\nu}\partial^{\nu}Fe^{iQ}$ is the product of an integrable function by a bounded function; it is then integrable. The conclusion is that the right hand side of equation \eqref{eq:defintosc} makes sense for all $F\in\svec^A(W)$. We take it as \emph{definition} of the left hand side.

\subsubsection{Value estimations for the integral}
%////////////////////////////////////////////////////

Let us suppose that $F$ has a compact support $S(F)$ and consider the real numbers $c_k=\max_{| \nu |\leq 2k}\nu!\| B_{\nu} \|_{\infty}$ where $\| . \|_{\infty}$ denotes the usual supremum norm.

\begin{lemma} \label{lem:born_osci_un}
	For all $k>0$, there exists a constant $c_k$ such that for all $F\in\svec^A(W)$ with compact support $S(F)$ and all $j$,
	\[
		\| \int_WFe^{iQ} \|_j\leq c_K\| F \|_{j,2k}\int_{S(F)}K^k
	\]
\end{lemma}

\begin{proof}

	From definition \eqref{eq:defintosc},
	\begin{equation}
		\| \int_W(w)Fe^{iQw} \|_j \leq\int_{S(F)}\| K(w)^k \|_j\sum_{| \nu |\leq 2k} \|B_{\nu}(w)\partial^{\nu}F(w) \|,
	\end{equation}
	We can majore $\|B_{\nu}(w)\|_j$ by $\| B_{\nu} \|_{\infty}=\sup_{w\in W}\| B_{\nu}(w) \|_j$\quext{Faut voir si ce $\sup$ est prit aussi sur les $j$, et je crois bien que oui; en tout cas, je l'utilise} which can at his turn be majored by $c_k$; the rest of the sum  is precisely $\| F \|_{j,2k}$. Since $K(w)>0$, we can forget the norm around $K^k$.

\end{proof}

Note that when $S(F)$ goes to infinity, the integral $\int_WK^k$ goes to zero and so does the integral $\int_WFe^{iQ}$. Intuitively, it comes from the fact that when $w\to\infty$, the function $F$ doesn't makes anything\footnote{Because it is a function in $A^{\infty}$}  while function $e^{iQ}$ oscillate rapidly and provokes cancellations.

Let us suppose that $S(F)$ is contained in a ball $B(w_0,r)$ of radius $r$ around the point $w_0$ and suppose that $| w_0 |<r$. In this case, $K$ takes its bigger value when $w^2=(| w_0 |-r)^2$. But there exists a constant $c'$ (depending on $r$) such that $1+(| w_0 |-r)^2>c'(1+| w_0 |^2)$. Taking volume of $B(w_0,r)$ into account ,we can compute the following majoration
\begin{equation}
	\begin{split}
		\int_{B(w_0,r)}K^k(w)&\leq vol\big(B(w_0,r)\big)\left( \frac{1}{1+(| w_0 |-r)^2} \right)^k\\
		&\leq \left( \frac{vol\big( B(w_0,r) \big)}{c'}^k \right)\left( \frac{1}{1+| w_0 |^2} \right)^k\\
		&=cK(w_0)^k
	\end{split}
\end{equation}
where $c$ is a new constant which take into account all other constants. If $| w_0 |>r$, one can redefine the constant in such a way that this majoration still holds. Using inequality of lemma~\ref{lem:born_osci_un} and absorbing once again all constant in $c$, we find the

\begin{proposition}
	For all $F\in\svec^A(W)$ with support in $B(w_0,r)$ and for all $j$, there exists a constant $c$ depending only on $k$ and $r$ such that
	\begin{equation}
		\| \int_WFe^{iQ} \|_j\leq c\| F \|_{j,2k}K(w_0)^k.
	\end{equation}
	In particular, $c$ doesn't depends on $j$.
\end{proposition}


\subsection{Lattice way to define oscillatory integrals}
%//////////////////////////////////////////////////////////

The latter proposition holds for a compact supported function; in order to see what happens when the support becomes non compact, choose a basis of $W$ and let $L$ be the lattice of points with integer components in this basis. Consider a function $\varphi_0\in C^{\infty}_c(W)$, a smooth compact supported function on $W$ and $\Phi(w)=\sum_{p\in L}\varphi_0(w+p)$. We choose $\varphi_0$ in such a way that $\Phi$ vanishes nowhere. Let $\varphi=\varphi_0/\Phi$ and $\varphi_p(w)=\varphi(w+p)$ for $p\in L$. By construction, $\sum_{p\in L}\varphi_p=1$ and each compact in $W$ intersect only finitely many support of $\varphi_p$.

Let $F\in\svec^A(W)$; if we suppose $S(\varphi)\subset B(0,r)$, then $S(F,\varphi_p)\subset B(p,r)$ and the proposition can be used with $F\varphi_p$. By virtue of the Leibnitz rule on $F\varphi_p$, we see that $\| \partial^{\nu}(F\varphi_p) \|_j$ is domined by a linear combination of terms of the form $\| \partial^{\nu}F \|_j\| \partial^{\rho}\varphi_p \|_{\infty}$ with $| \mu+\rho |\leq | \nu |$. But for all $p\in L$ and all multi-index $\rho$, $\| \partial^{\rho}\varphi \|_{\infty}=\| \partial^{\rho}\varphi \|_{\infty}$. Then if we fix $r$ and $\varphi$, the value of $\| F\varphi_p \|_{j,2k}$ is bounded by a sum from which  $\| \partial^{\rho}\varphi \|_{\infty}$ get out while the remaining sum is $\| F \|_{j,2k}$. By redefining the constant $c_k$, we get
\begin{equation} \label{eq:pr_abs_conv}
	\| \int_WF\varphi_pe^{iQ} \|\leq c_kK(p)^k\| F \|_{j,2k}.
\end{equation}
We can find a $k$ large enough that $\sum_{p\in L}K(p)^k$. For such a $k$, formula \eqref{eq:pr_abs_conv} proves the absolute convergence of the sum $\sum_{p\in L}\int_W(F\varphi_p)e^{iQ}$.

\begin{definition} \label{def:intFeiQ}
	When $F\in\svec^A(W)$ has no compact support, we define
	\begin{equation}
		\int_WFe^{iQ}=\sum_{p\in L}\int_W(F\varphi_p)e^{iQ}.
	\end{equation}

\end{definition}
When $F$ has a compact support, only finitely many terms are non zero and we can permute the sum and the integral. Since for all $x$, $\sum_{p\in L}(F\varphi_p)(x)=F(X)$, we find back the usual integral. For such a $k$, we can sum equation \eqref{eq:pr_abs_conv} over $p\in L$ and redefine $c_k$ and find
\[
	\| \int_W Fe^{iQ} \|\leq c_k\| F \|_{j,2k}.
\]

\begin{proposition}   \label{prop:leqd_m}
	For all $k>d$ and all positive sequence $(r_i)\to\infty$, there exists a real sequence $(d_i)\to 0$ such that for all $F\in\svec^A(W)$ vanishing on $B(0,r_m)$, the inequality
	\[
		\| \int_WFe^{iQ} \|_j\leq d_m\| F \|_{j,2k}.
	\]
	holds.
\end{proposition}

Stated in a more natural way, the statement is that if $F=0$ in $B(0,r)$, then there exists a constant $d$ such that $\| \int_WFe^{iQ} \|_j\leq d\| F \|_{j,2k}$. The dependence of $r$ with respect to $r$ is such that $lim_{r\to\infty}d(r)=0$. This corresponds to the intuition that, since $e^{iQ}$, oscillates more and more rapidly with distance to origin, the contribution to the integral is mainly contained in the values of $F$ near zero.

\begin{proof}
	Let $E_m=\{ p\in L\tq S(\varphi_p)\nsubseteq B(0,r_m) \}$. If $p\in E_m$, then $\varphi_p$ takes non zero values outside $r_m$. In the expression $\int_WFe^{iQ}=\sum_{p\in L}\int-WF\varphi_pe^{iQ}$, we can reduce the sum because $F$ is zero on $B(o,r_m)$: the terms with $p\notin E_m$ are zero. Then, using \eqref{eq:pr_abs_conv},
	\[
		\| \int_WFe^{iQ} \|_j\leq\sum_{p\in E_m}\| \int_W\varphi_pFe^{iQ} \|_j\leq\sum_{p\in E_m}c_kK(p)^k\| F \|_{j,2k}.
	\]
	For suitably large $k$ (i.e. $k>d$ as in the assumptions), convergence of the sequence
	\[
		d_m=c_k\sum_{p\in E_m}K(p)^k
	\]
	can be studied by the integral $\int_{\complement B(0,r_m)}K(p)^kdp$ which converges to zero when $m\to\infty$.
\end{proof}

\subsubsection{Independence with respect to choices}
%///////////////////////////////////////////////////

We have to show that definition~\ref{def:intFeiQ} don't depend on the choice of $L$ nor $\varphi$. For this, we will find a sequence of numbers independent of $L$ and $\varphi$ which converges to $\int_WFe^{iQ}$. Let $(\psi_i)\in C^{\infty}_c$, a sequence of functions with $\psi_i=1$ on $B(0,r_i)$ for a certain sequence $(r_m\in\eR^+)\to\infty$. For each $k$, we requires that the functions $\psi_i$ are equibounded for the norm $2k$ in the sense that there exists a real $b_k$ such that $\| \psi_i \|_{2k}\leq b_k$ for all $i$. Such a sequence can be build by choosing $\psi_1$ and defining $\psi_i(w)=\psi_1(w/r_i)$.

Remark that for each $i$, the sum $F\psi_i=\sim_{p\in L}F\psi_i\varphi_p$ is pointwise a finite because $\psi_i$ has compact support; then we can permute sum and integral and write
\[
	\int_W \sum_pF\psi_i\varphi_pe^{iQ}=\sum_p\int_WF\psi_i\varphi_ie^{iQ}.
\]
This allows us to write down the following majoration
\begin{equation}
	\begin{aligned}
		\| \int_WFe^{iQ}-\int_WFe^{iQ}\psi_i \|_j & =\| \sum_{p\in L}\int_WF\varphi_p(1-\psi_i)e^{iQ} \|_j &  & \textrm{Usual integral}       \\
		                                          & =\| \int_WF(1-\psi_i)e^{iQ} \|_j                       &  & \textrm{Oscillatory integral} \\
		                                          & \leq d_i\| F(1-\psi_i) \|_{j,2k}                                                          \\
		                                          & =d_i\| F \|_{j,2k}(1+b_k).
	\end{aligned}
\end{equation}
The latter converges to zero when $i\to\infty$. Then
\begin{equation}
	\lim_{m\to\infty}\int_WF\psi_me^{iQ}=\int_WFe^{iQ}
\end{equation}
where the left hand side don't depend on $L$ and $\varphi$.

One can take this equation as \emph{definition} of the right hand side.

\begin{probleme}
	Let \( f\in L^2(\eR^n)\). Does the integral
	\begin{equation}
		\int  e^{i\xi x}f(x)dx
	\end{equation}
	always exist in the sense of the oscillatory integral?

	If yes, does the value of that integral equal the value of the Fourier transform of \( f\) as defined by extension from \( L^1\cap L^2\) by~\ref{THOooJLCDooAjTvJf}?
\end{probleme}

\section{Decomposition into direct sum}
%++++++++++++++++++++++++++++++++++++++

Let $W=W_0\oplus W_1$ be a direct sum decomposition; we denote by $w=w_0+w_1$ the corresponding decomposition for the vectors $w\in W$. Let $B_i$ be a basis of $W_i$ and $L_i$ be the corresponding lattice. Since $B_0\cup B_1$ is a basis of $W$, we can consider the lattice $L_0\times L_1$ of $W$. Let $\varphi\in C^{\infty}_x$ be a positive function on $W_1$ with $\sum_{p\in L_0}\varphi_p=0$ and the same for $\varphi'$ for $W_1$. The function $\lambda(w)=\varphi(w_0)\varphi'(w_1)$ belongs to $C^{\infty}_c(W)$ and since the sums are pointwise finite,
\[
	\sum_{p\in L}\lambda_p(w)=\sum_{p\in L}\varphi(w_0+p_0)\varphi'(w_1+p_1)=1.
\]
Therefore the function $\lambda$ has the required properties to define the oscillatory integral
\begin{equation}  \label{eq:osc_int_phip}
	\begin{split}
		\int_WFe^{iQ}&=\sum_{p\in L}\int_W\lambda_pe^{iQ}\\
		&=\sum_{\substack{p_0\in L_0\\p_1\in L_1}}F(w)\varphi_{p_0}(w_0)\varphi'_{p_1}(w_1)e^{iQw}dw
	\end{split}
\end{equation}
with an absolutely convergent sum, there are no ordering problem in the summation.

\begin{proposition}
	Let $k>d$ and an increasing sequence $(r_i\in\eR^+)\to\infty$. Then there exists a sequence $(d_i\in\eR)\to 0$ such that, if $F\in\svec^A(V\times V)$ and $F(u,v)=0$ when $u\in B(0,r_m)$,  then
	\[
		\| \iint_{V\times V}e^{i\phi(u,v)} \|_j\leq d_m\| F \|_{j,2k}
	\]
	Here, $\phi(u,v)=2\pi u\cdot v$ is the former $\varphi$.
\end{proposition}

\begin{proof}
	Let $E_m=\{ p\tq S(\varphi_p)\nsubseteq B(o,r_m) \}$. The computation is performed in rather the same way as in proposition~\ref{prop:leqd_m}:
	\begin{equation}  \label{eq:re_fppe}
		\| \iint_{V\times V}Fe^{iQ} \|_j=\| \sum_{p\in E_m}\sum_q F(u,v)\lambda_{p+q}(u+v)e^{i\phi}  \|
	\end{equation}
	where $\lambda_{p+q}(u+v)=\varphi_p(u)\varphi'_q(v)$ because $(u,v)$ and $(p,q)$ are seen as vectors of $W$. If $\varphi$ has support contained in $B(r,0)$, the function $F\varphi_p\varphi'_q$ has support contained in the ball centered at $p+q$; the proposition
	\ref{lem:born_osci_un} gives
	\[
		\| \iint_{V\times V}F\varphi_p\varphi_q \|\leq c_k\| F\varphi_p\varphi'_q \|K(p+q)^k.
	\]
	If we suppose that the sum $W=W_0\oplus W_1$ is orthogonal, $K(p+q)=(1+p\cdot p+q\cdot q)$, a simple redefinition of $c_k$, leads to $c_k\| F\varphi_p\varphi'_q \|\leq c_k\| F \|$. So we can majore equation \eqref{eq:re_fppe} by
	\[
		\sum_{p\in E_m}\sum_qc_k\| F \|_{j,2k}(1+p^2+q^2)^{-k}.
	\]
	Now the definition $d_m=\sum_{p\in E_m}\sum_qc_k(1+p^2+q^2)^{-k}$ gives the thesis.
\end{proof}

\begin{proposition}\
	Let us consider a sequence $(\psi_i)\in C^{\infty}(W_0)$ with $\psi_i=1$ on $B(0,r_i)$ and $\| \psi_i \|_{2k}\leq b_k$ for all $i$. Then for all $F\in\svec^A(W)$, we have
	\begin{equation}
		\int Fe^{iQ}=\lim_{m\to\infty}\int F(w)\psi_m(w_0)e^{iQw}.
	\end{equation}

\end{proposition}


\begin{proof}
	We consider $\varphi$ and $\varphi'$ as in equation \eqref{eq:osc_int_phip} and we write $\varphi_p$ instead of $\varphi_p\circ\pr_0$ where $\pr_0$ is the projection into $W_0$. Then
	\begin{equation}
		\begin{split}
			\int_WFe^{iQ}-\int_W(\psi_m\circ\pr_0)e^{iQ}&=\sum_{p,q}\int_WF\varphi_p\varphi'_qe^{iQ}
			-\sum_{p,q}F(\psi_m\circ\pr_0)\varphi_p\varphi'_qe^{iQ}\\
			&=\sum_{p;q}\int_WF(1-\psi_m\pr_0)\varphi_p\varphi'_qe^{iQ}
		\end{split}
	\end{equation}
	which converges absolutely. By setting $E_m=\{ p\in L_0\tq S(\varphi_p)\nsubseteq B(0,r_m) \}$, we can reduce the sum over $p$ to $E_m$ and majore the norm of the latter expression by $\sum_{p\in E_m}\sum_qK(p+q)^k$. It converges to $0$ when $m\to \infty$.

\end{proof}

Let us now suppose that we have functions $(\psi'_i)$ on $W_1$ with the usual properties. Writing the latter proposition with $F\psi'_n$ instead of $F$, we find that, for all $b$,
\[
	\lim_{m\to\infty}\int_WF(w)\psi_m(w_0)\psi'_n(w_1)e^{iQw}=\int_WF(w)\psi'_n(w_1)e^{iQw}
\]
and inverting $W_1$ with $W_2$ in the preceding proposition, we see that the right hand side converges to $0$ when $n\to\infty$. So we find
\begin{equation} \label{eq:lim_mn_int}
	\lim_{m,n\to\infty}\int_WF(\psi_m\circ\pr_0)(\psi'_n\circ\pr_1)e^{iQ}=\int_WFe^{iQ},
\end{equation}
and we can take the limit of $n$ before $m$ if we want.


\subsection{Integral over \texorpdfstring{$V\times V$}{VV}}
%////////////////////////////////////////

Let an orthogonal direct sum decomposition $V=V_0\oplus V_1$ The decomposition $W=V_0\times V_1\times V_0\times V_1$ is naturally orthogonal. Consider an usual function $\dpt{F}{V\times V}{A}$ such that $F\big( (v_0+v_1),(v'_0,v'_1) \big)$ don't depend on $v_0$.

Let $(\psi_i)$ and $(\psi_j)$ be sequences of functions such that equation \eqref{eq:lim_mn_int} gives
\begin{equation} \label{eq:run}
	\iint F(u,v)e^{iQ(u,v)}=\lim_{mnjk}\iint F(u_1,v)\psi_m(u_0)\psi'_n(u_1)\psi_j(v_0)\psi'_k(v_1)e^{u_0\cdot v_0+u_1\cdot v_1}.
\end{equation}
We define the \defe{Fourier transform}{fourier transform} of $\psi_m$ by
\[
	\hat\psi_m(\xi)=\int_{V_0}\psi_m(u_0)e^{2\pi i u_0\cdot\xi}du_0.
\]
The main property is the inverse transform theorem:
\[
	\psi_m(v)=\int_{V_0}\hat\psi_m(\xi)e^{-2\pi i\xi\cdot v}d\xi=\psi_m(v).
\]
%
Equation \eqref{eq:run} can be rewritten as
\newcommand{\pheqrdeuxun}{\lim_{mnjk}\iint_{V\times V}F(u_1,v)\int_V}
\newcommand{\pheqrdeuxdeux}{\lim_{mnjk}\iiint_{V\times V\times V}F(u_1,v_1+v_0)}
\begin{equation} \label{eq:rdeux}
	\begin{split}
		&\pheqrdeuxun\hat\psi_m(\xi)e^{-2\pi i\xi\cdot u_0}\psi'_n(u_1)\psi_j(v_0)\psi'_k(v_1)\\
		&\phantom{\pheqrdeuxun}e^{2\pi i(u_0\cdot v_0+u_1\cdot v_1)}dudud\xi\\
		&=\pheqrdeuxdeux\hat\psi_m(\xi)\psi'_n(u_1)\psi_j(v_0)\psi'_k(v_1)\\
		&\phantom{pheqrdeuxdeux}e^{2\pi i[u_0\cdot(v_1-\xi)+u_1\cdot v_1]}dvdud\xi.\\
		\intertext{The integral over $u_0$ gives a Dirac delta at $\xi-v_0$}
		&=\lim_{mnjk}\iint F(u_1,v_1+v_1)\hat\psi_m(v_0)\psi'_n(u_1)\psi_j(v_0)\psi'_k(v_1)e^{2\pi i u_1\cdot v_1}dudv.
	\end{split}
\end{equation}

We can suppose that the functions $\psi_m$ are chosen in such a way that $(\hat\psi_m)$ is an approximation of the delta distribution. Indeed, if we choose $\psi_1$ equals to $1$ in $B(0,r_1)$, the sequence $\psi_m\to 1$ whose Fourier transform is the Dirac delta. In this case, the limit $m\to\infty$ gives $v_0=0$ and the limit $j\to \infty$ becomes trivial. Equation \eqref{eq:rdeux} becomes
\begin{equation}
	\begin{split}
		\quad&\lim_{nk}\iint F(u_1,v_1)\hat\psi(v_0)\psi'_n(u_1)\psi'_k(v_1)e^{2\pi i u_1\cdot v_1}\\
		&=\iint_{V_1\times V_1}F(u_1,v_1)e^{iQ(u_1,v_1)}.
	\end{split}
\end{equation}
In definitive, we had shown

\begin{proposition} \label{prop:F_pas_uz}
	Let $F\in\svec^A(V\times V)$ and an orthogonal decomposition $V=V_0\oplus V_1$ such that $F(u_0+u_1,v_0+v_1)$ don't depend on $u_0$. Then
	\[
		\iint_{V\times V}Fe^{iQ}=\iint_{V_1\times V_1}F(u_1,v-1)e^{iQ(u_1,v_1)}.
	\]

\end{proposition}


There exists a somewhat degenerate interesting case: $V_1=\{ 0 \}$ with a measure $1$. In this case the assumption is that $F(u,v)$ don't depend on $u$ at all. In this case,
\[
	\iint_{V_1\times V_1}F(u_1,v_1)e^{2\pi i u_1\cdot v_1}=\int_{V_1}F(v_1)dv_1=F(0).
\]
The result is that if $F\in\svec^A(V\times V)$ don't depend on his first variable (i.e. $F\in\svec^A(V)$), then
\[
	\int_{V\times V}Fe^{iQ}=F(0).
\]


\subsection{Normal operator}
%--------------------------

Let $T$ be a linear operator on $V$ and a decomposition $V=\ker T\oplus V_1$. In particular, $F(Tu,v)$  doesn't depends on $u_O$ (if we denote $\ker T$ by $V_0$) and we can apply proposition~\ref{prop:F_pas_uz}:
\[
	\int F(T_u,v)e^{iQ}=\iint_{V_1\times V_1}F(Tu_1,v_1)e^{iQ(u_1,v_1)}.
\]
Note that in general, $Tu_1$ don't belong to $V_1$. We now suppose that $T$ is \defe{normal}{normal!operator} in the sense that $[T,T^t]=0$. It is invertible on $V_1$. Then when we have a normal operator in an integral, one can always suppose that it is invertible because the kernel part of the space disappears.

Let $T$ be invertible on $V$, $F\in\svec^A(V\times V)$ and an usual sequence $(\psi_m)$ of functions. We have
\begin{equation}
	\begin{split}
		\iint F(Tu,v)r^{iQ(u,v)}&=\lim_{m,n\to\infty}\iint F(Tu,v)\psi_m(u)\psi_n(v)e^{iQ}\\
		&=\lim_{m,n\to\infty}(\det T)^{-1}\iint F(u,v)\psi_m(T^{-1}u)\psi_n(v)e^{2\pi i T^{-1} u\cdot v}\\
		&=\lim_{m,n\to\infty}\iint F(u,T^tv)\psi_m(T^{-1}u)\psi_n(T^tv)e^{iQ}\\
		\intertext{but the sequences $(\psi_m\circ T^{-1})$ and $(\psi_n\circ T^t)$ have same fundamental properties as $(\psi_m)$, then taking the limit,}
		&=\iint F(u,T^tv)e^{iQ}.
	\end{split}
\end{equation}
If $T$ is normal but not invertible, the first integral restricts to $V_1$ and for all normal operator, the equality\quext{Je ne comprends pas pourquoi.}
\[
	\iint F(Tu,v)e^{iQ}=\iint F(u,T^tv)e^{iQ}.
\]
hold. From polar decomposition of any operator into two normal operators, we conclude that this equation holds for all linear operator on $V$.

\begin{proposition}
	If $\dpt{S}{A}{C}$ is a continuous linear map from $A$ to a Fréchet space $C$, then $S\circ F\in\svec^C(W)$ and
	\begin{equation}
		S( \int_WFe^{iQ} )=\int_W(S\circ F)e^{iQ}.
	\end{equation}

\end{proposition}

\begin{proof}
	The absolute convergence of the defining sum for the left hand side integral allows us to permute linear operator and sum. Since for each term the integral is an usual integral over a compact set, we can permute the continuous operator $S$ and the integral:
	\begin{equation}
		S\sum_{p\in L}\int F\varphi_pe^{iQ}=\sum S\int F\varphi_pe^{iQ}=\sum\int (S\circ F)\varphi_pe^{iQ}.
	\end{equation}


\end{proof}




%%%%%%%%%%%%%%%%%%%%%%%%%%%%
%
\section{Pseudo-differentiable operators}
%
%%%%%%%%%%%%%%%%%%%%%%%%


\subsection{Differential operator}
%---------------------------------

Let $\dpt{ \pi }{ E }{ M }$ be a vector bundle of rank $k$ on a compact manifold of dimension $n$. We denote by $\Gamma(E)$ the $ C^{\infty}(M)$-module of $ C^{\infty}$ sections of $E$. The module structure is given by $(f\cdot\psi)(x)=f(x)\psi(x)$ where the product in the right hand side is the product $\eC\times E_x\to E_x$ which defines the vector space structure on $E_x=\pi^{-1}(x)$.

A \defe{differential operator}{differential!operator}\index{operator!differential} of rank $m$  is a linear operator $\dpt{ P }{ \Gamma(E) }{ \Gamma(E) }$ for which there exists local coordinates $(x_1,\cdots,x_n)$ on $M$ in which $P$ is written as
\begin{equation}
	P=\sum_{| \alpha |\leq m}A_{\alpha}(x)(-i)^{| \alpha |}\frac{ \partial^{| \alpha |} }{ \partial x^{\alpha} }
\end{equation}
where $\alpha=(\alpha_1,\cdots,\alpha_n)$ is a multi-index with $0\leq\alpha_j\leq m$ and $| \alpha |=\sum_{j=1}^n\alpha_j$. Each $A_{\alpha}$ is a $k\times k$ matrix of $ C^{\infty}$ functions on $M$ and $A_{\alpha}\neq 0$ for at least one $\alpha$ with $| \alpha |=m$.

Let $\xi\in T^*_xM$ written under the form $\xi=\sum_j\xi_j\,dx_j$ in the previously given local coordinates. The \defe{complete symbol}{symbol!complete}\index{complete!symbol} of $P$ applied to $\xi$ is the following polynomial combination of coordinates of $\xi$:
\begin{equation}
	p^P(x,\xi)=\sum_{j=0}^m p^P_{m-j}(x,\xi)
\end{equation}
where
\[
	p^P_{m-j}=\sum_{| \alpha |\leq(m-j)}A_{\alpha}(x)\xi^{\alpha}
\]
where $\xi^{\alpha}=\xi_1^{\alpha_1}\cdots\xi_n^{\alpha_n}$ and $\partial^{| \alpha |}_{\alpha}=\partial_{x_1}^{\alpha_1}\circ\cdots\circ\partial^{\alpha_n}_{x_n}$.  The \defe{principal symbol}{principal!symbol}\index{symbol!principal} of $P$ is the leading term, namely
\begin{equation}
	\sigma^P(x,\xi)=p^P_m(x,\xi)=\sum_{| \alpha |=m}A_{\alpha}(x)\xi^{\alpha}.
\end{equation}
For each $\xi\in T^*_xM$, the principal symbol gives rise to a map $\dpt{ \sigma_x^P }{ E_x }{ E_x }$ because $A_{\alpha}(x)$ is a $k\times k$ matrix:
\[
	\sigma_x^P(\xi)v=\sum_{| \alpha |=m}\xi^{\alpha}A_{\alpha}(x)v.
\]

\begin{definition}  \label{DefGLpDEHy}
	The operator $P$ is \defe{elliptic}{elliptic operator}\index{operator!elliptic} if for each non zero $\xi\in T^*M$, the map $\sigma_x^P(\xi_x)$ are all invertible.
\end{definition}

If $(M,g)$ is a Riemannian manifold, the fact for $P$ to be elliptic is equivalent to be invertible on the \defe{cosphere}{cosphere}
\[
	S^*M=\{ (x,\xi)\in T^*M\tq g_x(\xi,\xi)=1 \}\subset T^*M.
\]

%---------------------------------------------------------------------------------------------------------------------------
\subsection{Laplace operator}
%---------------------------------------------------------------------------------------------------------------------------

As first example, we see the Laplace operator. Let $(M,g)$ be a Riemannian manifold and its Laplace operator $\dpt{ \Delta }{  C^{\infty}(L) }{  C^{\infty}(M) }$ where $ C^{\infty}(M)$ is seen as $\Gamma(\eC\times M)$ is
\[
	\Delta f=-g^{\mu\nu}\partial^2_{\mu\nu}f+\text{lower order terms}
\]
Its principal symbol is $\sigma^{\Delta}(x,\xi)=g^{\mu\nu}\xi_{\mu}\xi_{\nu}=\| \xi \|^2$, which is invertible of all $\xi\neq 0$. Thus Laplace operator is elliptic of order $2$.

An other example of pseudo-differential operator is the Dirac operator. We will show that it is an elliptic operator in subsection~\ref{subSecREctBOh}.

%+++++++++++++++++++++++++++++++++++++++++++++++++++++++++++++++++++++++++++++++++++++++++++++++++++++++++++++++++++++++++++
\section{Invariant differential operators on Lie groups}
%+++++++++++++++++++++++++++++++++++++++++++++++++++++++++++++++++++++++++++++++++++++++++++++++++++++++++++++++++++++++++++

This section is intended to understand the paper \cite{QuantifKhalerian}. Let $G$ be a connected Lie group and $\lG$ be its Lie algebra.

An endomorphism $P\colon C^{\infty}(G,A)\to  C^{\infty}(G,A)$ of the space of functions over $G$ with values in a vector space $A$ is said to be \defe{\hypertarget{HyperDefLeftInvar}{left invariant}}{left invariant!operator}\index{differential!operator!left invariant} if
\begin{equation}  \label{EqDefLxinvarop}
	L_{x}(P\psi)=P(L_{x}\psi)
\end{equation}
for all $x\in G$ and $\psi\in C^{\infty}(G,A)$. Here, $L$ is the regular left representation of $G$ on $ C^{\infty}(G,A)$.

\begin{lemma}
	If $X\in\lG$ and if $\tilde X$ is the associated left invariant vector field, the operator $P$ defined by
	\begin{equation}
		(P\psi)(x)=\tilde X_{x}\psi
	\end{equation}
	is left invariant.
\end{lemma}

\begin{proof}
	On the one hand
	\[
		\big( L_{x}(P\psi) \big)(y)=(P\psi)(xy)=\tilde X_{xy}\psi,
	\]
	on the other hand,
	\begin{equation}
		\begin{aligned}[]
			P(L_{x}\psi)(y)=\tilde X_{y}(L_{x}\psi)
			=\dsdd{ (L_{x}\psi)(y e^{tX}) }{t}{0}
			=\dsdd{ \psi(xy e^{tX}) }{t}{0}
			=\tilde X_{xy}\psi.
		\end{aligned}
	\end{equation}
\end{proof}

In fact, if $\Diff^G(G)$\nomenclature{$\Diff^{G}(G)$}{space of $G$-left invariant differential operators on $G$} denotes the space of $G$-left invariant differential operators on $G$, we have a morphism
\begin{equation}		\label{EqMorphismUgDiff}
	\begin{aligned}
		\mU(\lG) & \to \Diff^{G}(G)  \\
		X        & \mapsto \tilde X.
	\end{aligned}
\end{equation}
The operator $P$ is \defe{left invariant}{left invariant!differential operator} when it satisfies
\begin{equation}
	(Pf)(x)=P\big( L_x^*f \big)(e).
\end{equation}
As far as notations are concerned, we denote by $(L^*_xP)$\nomenclature[F]{$L^*_xP$}{Left translated differential operator} the operator defined by
\begin{equation}
	(L^*_xP)(f)=P(L^*_xf).
\end{equation}
Every differential operator $P\colon  C^{\infty}(G)\to  C^{\infty}(G)$ can be written as a sum of products $\sum_ip_i\tilde X_i$ of functions $p_i\in C^{\infty}(G)$ by operators $\tilde X_i\in\mU(\lG)$. The operator $P$ is left invariant if and only if the functions $p_i$ are constant.

As a consequence we have the
\begin{corollary}	\label{CorUisomDiff}
	The map $\mU(\lG)\to\Diff^G(G)$ is an isomorphism.
\end{corollary}

We denote by $\biDiff^G(G)$\nomenclature{$\biDiff(G)$}{Bidifferential operators on $G$} the space of $G$-left invariant \defe{bidifferential operators}{bidifferential operator} on $G$. These are differential operators
\begin{equation}
	\begin{aligned}
		A\colon  C^{\infty}(G)\times C^{\infty}(G) & \to  C^{\infty}(G)     \\
		u\otimes v                                 & \mapsto A(u\otimes v).
	\end{aligned}
\end{equation}
Since $A$ is a differential operator, the function $A(u,v)$ only depends on the class $u\otimes v$. The left invariance means that
\begin{equation}		\label{EqDefLeftInvarbiDiff}
	L_x\big( A(u\otimes v) \big)=A\big( L_x(u\otimes v) \big)
\end{equation}
for every $x\in G$.

\begin{proposition}		\label{PropbidiffUU}
	The space of left invariant bidifferential operators $\biDiff^G(G)$ is canonically isomorphic to the tensor product $\mU(\lG)\otimes\mU(\lG)$.
\end{proposition}

\begin{proof}
	Let $A$ be a bidifferential operator and $\{ X_i \}$ be a basis of $\lG$. For every multi-index $a=(a_1,\ldots,a_n)$ we write $X^a=X^{a_1}\cdots X^{a_n}$ the corresponding element in $\mU(\lG)$. Thus there exist functions $A_{ab}\in C^{\infty}(G)$ such that
	\begin{equation}	\label{EqdefAabdiffopgp}
		A(u\otimes v)=\sum_{ab}A_{ab}(\tilde X^au)(\tilde X^bv)
	\end{equation}
	for every $u,v\in C^{\infty}(G)$.

	Now we are supposing that $A$ is left invariant and we will prove that the functions $A_{ab}(x)$ are constant. We evaluate the definition \eqref{EqDefLeftInvarbiDiff} at $g\in G$:
	\begin{equation}
		\begin{aligned}[]
			A(u\otimes v)(xg) & =A\big( L_x(u\otimes v) \big)(g)                    \\
			                  & =A\big( (L^*_xu)\otimes (L^*_xv) \big)(g)           \\
			                  & =A_{ab}(g)\tilde X^a_g(L^*_xu)\tilde X^b_g(L^*_xv).
		\end{aligned}
	\end{equation}
	If we consider that equation at $g=e$, we have
	\begin{equation}
		A(u\otimes v)(x)=A_{ab}(e)\tilde X^a_e(L^*_xu)\tilde X^b_e(L^*_xv)=A_{ab}(e)\tilde X^a_xu\tilde X^b_xv
	\end{equation}
	where we used the invariance property \eqref{EqInvarUgField} on each of the operators $X^a\in\mU(\lG)$. Now, the expression \eqref{EqdefAabdiffopgp} evaluated at $x$ gives
	\begin{equation}
		A(u\otimes v)(x)=A_{ab}(x)(\tilde X_x^au)(\tilde X_x^bv),
	\end{equation}
	so that $A_{ab}(x)=A_{ab}(e)$ when $A$ is left invariant.
\end{proof}

\begin{proof}[Alternative proof]
	Using the same notations, we have
	\begin{equation}		\label{EaAuvDeuxDun}
		A(u\otimes v)(x)=A_{ab}(x)\big( \tilde X^a_x\otimes\tilde X^b_x \big)(u\otimes v)=A_{ab}(x)\big( \tilde X^a_xu \big)\big( \tilde X^b_xv \big).
	\end{equation}
	Using the left invariance of $A$, on the other hand, we have
	\begin{equation}		\label{EqAuvUnDun}
		A(u\otimes v)(x)=A\big( L^*_xu\otimes L^*_xv \big)(e)=A_{ab}(e)\big( \tilde X^a_eL^*_xu \big)\otimes\big( \tilde X^b_eL^*_xv \big),
	\end{equation}
	but
	\begin{equation}
		\tilde X^a_eL^*_xu=\Dsdd{ (L^*_xu)( e^{t\tilde X^a}) }{t}{0}=\Dsdd{ u\big( x e^{t\tilde X^a} \big) }{t}{0}=\tilde X^a_xu,
	\end{equation}
	thus the expression \eqref{EqAuvUnDun} becomes
	\begin{equation}
		A_{ab}(e)\big( \tilde X^a_xu \big)\big( \tilde X^a_xv \big),
	\end{equation}
	and the comparison with \eqref{EaAuvDeuxDun} shows that $A_{ab}(x)=A_{ab}(e)$, so that the coefficients $A_{ab}$ are constant.
\end{proof}

\begin{remark}  \label{REMooGIFYooTphiex}
	If $\tilde X$ is the left invariant differential operator associated with $X\in\lG$, the Leibnitz rules reads
	\begin{equation}		\label{EqXfgDeltaUnif}
		\tilde X(fg)=\widetilde{\Delta(X)}(f\otimes g)
	\end{equation}
	where \( \Delta\) is the coproduct defined in~\ref{SUBSECooTKZAooWVXXug}.
\end{remark}

Let us now consider the space $\Diff(G\times G)$ of differential operators on $G\times G$. These operators an operator in $\Diff(G\times G)$ acts on the space of functions $ C^{\infty}(G\times G)$. Such an operator reads $X\cdot Y$ with $X,Y\in\mU(\lG)$. If $f\in C^{\infty}(G\times G)$,
\begin{equation}
	(Pf)(x,y)=(X\cdot Y f)(x,y)
\end{equation}
where $X$ acts on the first variable and $Y$ acts on the second variable. The space $\Diff(G\times G)$ is then naturally isomorphic to $\mU(\lG)\otimes \mU(\lG)$ as an operator $P\in\Diff(G\times G)$ can be written as $P=X\otimes Y$. This acts on tensor product of functions by
\begin{equation}
	P(u\otimes v)=Xu\otimes Yv.
\end{equation}
Since they are both isomorphic to $\mU(\lG)\otimes\mU(\lG)$, the spaces of left invariant operators $\Diff^{G\times G}(G\times G)$ and $\biDiff^G(G)$ are isomorphic.

Let us make that isomorphism more explicit. First, an element $P\in\Diff(G\times G)$ provides the element $X\otimes Y\in\mU(\lG)\otimes\mU(\lG)$. The latter produces the bidifferential operator $A\in\biDiff(G)$ defined by $A(u\otimes v)(x)=X_xuX_xv$. Thus we define
\begin{equation}		\label{EqDefAlphaDiffbiDiff}
	\begin{aligned}
		\alpha\colon \Diff(G\times G) & \to \biDiff(G) \\
		\alpha(P)(u\otimes v)(x)      & =X_xuY_xv.
	\end{aligned}
\end{equation}

\begin{lemma}
	The map \eqref{EqDefAlphaDiffbiDiff} is surjective.
\end{lemma}

\begin{proof}
	If $A\in\biDiff(G)$ is given by $A(u\otimes v)(x)=X_xuY_xv$, we have $\alpha(P_A)=A$ with $P_A\in\Diff(G\times G)$ given on $f\in C^{\infty}(G\times G)$ by
	\begin{equation}
		(P_Af)(x,y)=(XYf)(x,y)
	\end{equation}
	where $X$ acts on the first variable and $Y$ on the second.
\end{proof}

\begin{proposition}
	If we restrict to the left invariant operators, the map $\alpha\colon \Diff^{G\times G}(G\times G)\to \biDiff^G(G)$ is an isomorphism.
\end{proposition}

\begin{proof}
	First, remark that if $A\in\biDiff(G)$ is invariant, then an operator $P_A$ such that $\alpha(P_A)=A$ is also invariant because of proposition~\ref{PropbidiffUU} which states that
	\begin{equation}
		A(u\otimes v)(x)=A_{ab}(x)(\tilde X^a_x\otimes \tilde X^b_x)(u\otimes v)
	\end{equation}
	is invariant only when $A_{ab}$ are constant functions. Thus the map $\alpha$ is surjective from $\Diff^{G\times G}(G\times G)$ to $\biDiff^G(G)$.

	Now we have to prove that the map $\alpha$ is injective from the subspace of invariant operators. Let $P\in\Diff(G\times G)$ be an operator in the kernel of $\alpha$, so if $P$ reads
	\begin{equation}
		(Pf)(x,y)=c(x,y)(\tilde X\tilde Yf)(x,y),
	\end{equation}
	we have
	\begin{equation}
		(\alpha P)(u\otimes v)(x)=c(x,x)\tilde X_xu\tilde Y_xv=0
	\end{equation}
	for every $x\in G$ and every $u,v\in C^{\infty}(B)$. In that case, $c(x,x)$ has to vanish.

	What we proved is that the kernel of $\alpha$ is the set of operators with $c(x,x)=0$. In other words, the kernel is the set of operators $P$ such that
	\begin{equation}	\label{EqPfxxzero}
		(Pf)(x,x)=0
	\end{equation}
	for every function $f\in C^{\infty}(G\times G)$.

	Let us now suppose that $P$ is left invariant and compute $(Pg)(x,y)$ using the left invariance. We have
	\begin{equation}
		(Pg)(x,y)=dL_{x\times y}(Pg)(e,e)=P(L^*_{x\times y}g)(e,e)=0
	\end{equation}
	because of \eqref{EqPfxxzero} applied to the function $f=L^*_{x\times y}g$.
\end{proof}


%+++++++++++++++++++++++++++++++++++++++++++++++++++++++++++++++++++++++++++++++++++++++++++++++++++++++++++++++++++++++++++
\section{Pseudo differential operators}
%+++++++++++++++++++++++++++++++++++++++++++++++++++++++++++++++++++++++++++++++++++++++++++++++++++++++++++++++++++++++++++

\subsection{Composition}
%-----------------------

If $A$ and $B$ are pseudo-differential operators with symbols $a$ and $b$, the symbol of the composition is given by
\begin{equation}		\label{EqCompPSDSymb}
	c(x,\xi)\sim \sum_{\alpha\in\eN^n} \frac{ (-i)^{| \alpha |} }{ \alpha! }\partial_{\xi}^{\alpha}a\partial_x^{\alpha}b.
\end{equation}


\subsection{Fourier transform and pseudo-differential operator}
%--------------------------------------------------------------

Let $\dpt{ \psi }{ M }{ E }$ be a local section and $P$ a differential operator. One can prove that
\begin{subequations} \label{eq:PspiFour}
	\begin{align}
		(P\psi)(x)    & =\frac{1}{ (2\pi)^{n/2} } \int e^{i\xi\cdot x}p(x,\xi)\hat\psi(\xi)\,d\xi \\
		\hat\psi(\xi) & = \frac{1}{ (2\pi)^{n/2} }\int e^{-i\xi\cdot y}\psi(y)\,dy
	\end{align}
\end{subequations}
where $\xi\cdot x=\sum_j\xi_jx_j$ is the usual scalar product. So a pseudo-differential operator has to be written under the form
\begin{equation}
	(Au)(x)=(2\pi)^n\iint e^{i(x-y)\cdot \xi}a(x,\xi)u(y)\,dy\,d\xi
\end{equation}
and $a$ is the \defe{symbol}{symbol!of a pseudo-differential operator} of $A$.


Let us see a simple example: $P\psi=\partial_1\psi$. In this case, $p(x,\xi)=\xi_1$ and equations \eqref{eq:PspiFour} give
\[
	(P\psi)(x)=(2\pi)^{-n}\iint e^{i\xi\cdot(x-y)}\xi^1\psi(y)\,d\xi\,dy,
\]
an integration by part gives
\[
	(2\pi)^{-n}\iint e^{i\xi\cdot(x-y)}(-\frac{ i }{ \xi_1 })\xi_1(\partial_{y_1}\psi)(y)\,d\xi\,dy.
\]
The integration over $\xi$ produce de Dirac distribution centred at $x-y$, i.e. a factor $(2\pi)^{-n}\delta(x-y)$ and the integral over $y$ leads to $(\partial_1)\psi(x)$.

To define pseudo-differential operator, we begin by only consider the trivial vector bundle over $\eR^n$ and thus functions $\dpt{ u }{ \eR^n }{ \eC^k }$.

\begin{definition}
	A \defe{pseudo-differential}{pseudo-differential operator}\index{operator!pseudo-differential} operator of order $m$ is an operator $P$ which can be written under the form
	\begin{equation}
		(Pu)(x)=(2\pi)^{-n/2}\int e^{i\xi\cdot x}p(x,\xi)\hat u(\xi)\,d\xi
	\end{equation}
	where
	\begin{equation}
		\hat u(\xi)=(2\pi)^{-n/2}\int e^{-i\xi\cdot y}u(y)\,dy
	\end{equation}
	and $p\in\mS^m$, the space defined at page \pageref{pg:defmS}.
\end{definition}

The set of all pseudo-differential operators is denoted by $\mP^m$\nomenclature{$\mP$}{Set of pseudo-differential operator}. If $p\in\mS^m$ is the symbol associated with $P\in\mP^m$, the \defe{principal symbol}{principal!symbol} of $P$ is the class $\sigma^P=[p]\in\mS^m/\mS^{m-1}$. Operators with associated symbol in $\mS^{\-\infty}$ are call \defe{smoothing operator}{smoothing operator} and we denote their space by $\mP^{-\infty}$.

A pseudo-differential operator $Q$ is a \defe{\wikipedia{en}{Parametrix}{\hypertarget{DefParametrix}{parametrix}}}{parametrix} for the pseudo-differential operator $P$ is the operators $PQ-\mtu$ and $QP-\mtu$ are compact operators.


\subsection{Example} \label{pg_exem_psdo}
%-------------------

The operator $[1-(2\pi)^{-2}\Delta]^{1/2}$ is the pseudo-differential operator on $\eR^N$ with symbol $p(\xi)=(1+\xi^2)^{1/2}$ where $\xi^2$ stands for $| \xi |^2$. Indeed if $P$ is the pseudo-differential operator with this symbol
\[
	\begin{aligned}
		(Pu)(x) & =\int e^{2i\pi\xi\cdot x} (1+\xi^2)^{1/2}\hat u(\xi)\,d\xi                                                                      \\
		        & =\int e^{2i\pi\xi\cdot x}\int e^{-2i\pi\xi\cdot y}[1-(2\pi)^{-2}\Delta]^{1/2}u(y)\,dy\,d\xi &  & \text{by \eqref{eq_umdpi_spi}} \\
		        & =\int e^{2i\pi\xi\cdot(x-y)}                                                                                                    \\
		        & =\int\delta(x-y)[1-(2\pi)^{-2}\Delta]^{1/2}u(y)\,dy                                                                             \\
		        & =[1-(2\pi)^{-2}\Delta]^{1/2}u(x).
	\end{aligned}
\]

\subsection{Trace operators}  \label{subsec_traceop}
%---------------------------

A pseudo-differential operator $P$ is expressed by its symbol $p$ and the formula
\[
	(Pu)(x)=\int  e^{2i\pi \xi\cdot x}p(x,\xi)\hat y(\xi)\,d\xi.
\]
Assume that $P$ can also be written under the form
\[
	(Pu)(x)=\int K(x,y)u(y)\,dy
\]
and let us equalize these two expressions for all $u$ (for example in $H^{1/2}(\eR^n)$. When it makes sense,
\begin{equation}
	K(x,y)=\int e^{2i\pi \xi\cdot (x-y)}p(x,\xi)\,d\xi.
\end{equation}
The \defe{trace}{trace!of an operator}\nomenclature[F]{$\tr P$}{Trace of the operator $P$} of $P$ is defined by
\begin{equation}
	\tr P=\int K(x,y)\,dx
	=\iint p(x,\xi)\,dx\,d\xi.
\end{equation}
When the latter converges, one says that $P$ is a \emph{trace operator}.

\subsection{Asymptotic expansions}
%------------------------------------

We say that the pseudo-differential operator $A$ is \defe{classical}{classical!pseudo-differential operator} and we write $A\in\Psi^p(M)$\nomenclature[F]{$\Psi^n(M)$}{Space of classical pseudo-differential operators on $M$} if the symbols accepts the asymptotic expansion
\[
	a(x,\xi)\sim\sum_{j=0}^{\infty}a_{p-j}(x,\xi)
\]
where each of the $a_r$ is $r$-homogeneous with respect to $\xi$: $a_r(x,t\xi)=t^ra_r(x,\xi)$. Although the whole definitions are made in local coordinates, one can show that the \defe{principal symbol}{principal!symbol} is a globally defined function over the cotangent bundle $T^*M$. The operator is elliptic if $a_n(x,\xi)$ is invertible for all $\xi\neq 0$.

Notice that an element of $\Psi^p$ has an asymptotic expansion which begins with order $p$, so that the spaces $\Psi^d$ fulfil $\Psi^{p-1}\subset\Psi^p$. Now the \defe{algebra of classical pseudo-differential operators}{algebra!of classical $\Psi$DO} is the quotient
\begin{equation}		\label{EqDefmPalgOpeClass}
	\mP=\Psi^{-\infty}/\Psi^{\infty}.
\end{equation}

%+++++++++++++++++++++++++++++++++++++++++++++++++++++++++++++++++++++++++++++++++++++++++++++++++++++++++++++++++++++++++++
\section{Dirac and Laplace type operators}

Let $(M,\partial M)$ be a manifold with boundary and $V$, a vector bundle over $M$. A second order differential operator $\Delta$ on $V$ is of \defe{Laplace type}{laplace type operator} if its principal symbol is the metric tensor, i.e. if it has a local expression of the form
\begin{equation}
	\Delta = -(g^{ij}\partial^2_{ij}+A^k\partial_k+B)
\end{equation}
where $a\in \Gamma(TM\otimes\End(V))$ and $B\in\Gamma(\End(V))$. A first order operator $D$ on $\Gamma(V)$ is of \defe{Dirac type}{dirac type operator} if its principal symbol defines a Clifford module over $V$. Such an operator has a local expression
\begin{equation}			\label{EqFormGeneDirac}
	D=\gamma^i\partial_i-r
\end{equation}
with $\gamma\in\Gamma(TM\otimes\End(V))$ and $r\in\Gamma(\End(V))$. The condition is that
\begin{equation}
	\gamma^i\gamma^j+\gamma^j\gamma^i=2g^{ij}\id|_V.
\end{equation}

\begin{proposition}
	The operator $D$ is of Dirac type if and only if $D^2$ is of Laplace type.
\end{proposition}

\begin{proof}
	If $D$ is of Dirac type, up to lower order terms we have
	\[
		D^2=\gamma^i\gamma^j\partial^2_{ij}+\ldots=\frac{ 1 }{2}(\gamma^i\gamma^j+\gamma^j\gamma^i)\partial^2_{ij}+\ldots=-g^{ij}\partial^2_{ij}+\ldots
	\]
\end{proof}
It is however not true that every Laplace type operator is the square of a Dirac operator.

\section{Wodzicki residue}
%+++++++++++++++++++++++++

Let $M$ be a $n$-dimensional manifold. In that case the term of order $(-n)$ in the asymptotic expansion
\[
	a(x,\xi)\sim\sum_{j=0}^{\infty}a_{n-j}(x,\xi)
\]
of the principal symbol of the pseudo-differential operator $A$ is specially important. Let $\mU\subset M$ be a coordinate open set on which the cotangent bundle is trivial. One can see $a_{-n}$ as a smooth function on $T^*\mU\invtible$ (the set of sections from which we remove the zero section).

We define
\[
	\alpha(x,\xi)=a_{-n}(x,\xi)d\xi_1\wedge\cdots\wedge d\xi_n\wedge dx^1\wedge\cdots\wedge dx^n.
\]
That form is invariant under dilatations $\xi\to t\xi$ because $a_{-n}(x,t\xi)=t^{-n}a_{-n}(x,\xi)$. Let us consider $R=\sum_j \xi_j\partial_{\xi_j}$, the dilatation generator.

From formula $\mL_R=\iota_Rd+d\iota_R$, we have $d\iota_R\alpha=\mL_R\alpha=0$ where $\mL$ denotes the Lie derivative \eqref{liesurforme}. If we write $dx=d^nx=dx^1\wedge\cdots\wedge dx^n$, we have
\[
	(\iota_R\alpha)(x,\xi)=a_{-n}(x,\xi)\sigma_n\wedge dx
\]
where $\sigma_{\xi}=\sum (-1)^{j-1}\xi_j\,d\xi_1\wedge\cdots\widehat{d\xi_j}\wedge\cdots\wedge d\xi_n$. For each particular $x\in M$, $\sigma_{\xi}$ is a volume form on the unit sphere $| \xi |=1$ on $T^*_xM$. We can integrate $\iota_R\alpha$ on such a sphere:
\[
	\int_{| \xi |=1}a_{-n}(x,\xi)\sigma_{\xi}.
\]
One can prove that under the coordinate change $w\to y=\phi(x)$, $\xi\to\eta=\phi'(x)^t\xi$, we have $a_{-n}(x,\xi)\to \tilde a_{-n}(y,\eta)$ and
\[
	\int_{| \eta |=1}\tilde a_{-n}(y,\eta)\sigma_{\eta}=| \det\phi'(x) |\int_{| \xi |=1}a_{-n}(x,\xi)\sigma_{\xi}.
\]
That shows that the quantity
\begin{equation}
	\left( \int_{| \xi |=1}a_{-n}(x,\xi)\sigma_{\xi} \right)dx
\end{equation}
is a $1$-density over $M$ that we call $\ResW(x)A$. The \defe{Wodzicki residue}{wodzicki residue} is the integral of that over $M$:
\begin{equation}
	\ResW A=\int_M\Res_W(x)A=\int_{S^*M}\iota_R\alpha=\int_{S^*M}a_{-n}(x,\xi)\sigma_{\xi}dx
\end{equation}
where $S^*M=\{ (x,\xi)\in T^*M\tq | \xi |=1 \}$. Notice that the latter integral can diverge.

\begin{proposition}
	The operation $\ResW$ is a trace on the algebra of classical pseudo-differential operators. That means that $\Res_W[A,B]=0$ whenever $A$, $B$ are classical pseudo-differential operators.
\end{proposition}

\begin{proof}
	If $A\in\Psi^d(M)$ and $B\in\Psi^r(M)$, then the product $AB$ belongs to the space $\Psi^{d+r}(M)$ and the commutator $P=[A,B]$,  seen as in $\Psi^{-\infty}$, has symbol (see equation \eqref{EqCompPSDSymb})
	\[
		p(x,y)\sim\sum_{| \alpha |>0}\frac{ (-i)^{| \alpha |} }{ \alpha! }\big( \partial^{\alpha}_{\xi}a\partial_x^{\alpha}b-\partial^{\alpha}_{\xi}b\partial_x^{\alpha}a \big).
	\]
	We can suppose that $A$ and $B$ have compact support because the aim is to integrate them with a partition of unity.
\end{proof}

\begin{probleme}
	Unfinished proof
\end{probleme}

\begin{proposition}
	The operation $\ResW$ is the unique trace on the algebra $\mP$ defined in equation~\ref{EqDefmPalgOpeClass}.
\end{proposition}
\begin{proof}
	No proof.
\end{proof}
