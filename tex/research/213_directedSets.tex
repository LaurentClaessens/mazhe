% This is part of Giulietta
% Copyright (c) 2016, 2018, 2021-2022
%   Laurent Claessens
% See the file fdl-1.3.txt for copying conditions.

%++++++++++++++++++++++++++++++++++++++++++++++++++++++++++++++++++++++++++++++++++++++++++++++++++++++++++++++++++++++++++++
\section{Directed sets and net}

\begin{definition}
    A \defe{directed set}{directed set} is a pre-ordered set (i.e. a set with a reflexive and transitive binary relation) such that every pair of elements has an upper bound. 
\end{definition}
As a consequence, when \( a_1,\ldots a_n\) are elements of the directed set \( A\), then there exists a \( a\in A\) such that \( a\geq a_i\).

\begin{definition}
    A \defe{net}{net} is a map \( A\to X\) from a directed set to a topological space. We denote by \( x_{\alpha}\) the element of \( X\) which corresponds to \( \alpha\in A\).
\end{definition}

As example, if \( S\) is any set, the set \( A\) of finite subsets of \( S\) with the inclusion is an example of net.

There is a notion of \defe{convergence}{convergence!of a net} of net. We say that the net \( \alpha\mapsto x_\alpha\) converges to \( x\) and we write \( x_{\alpha}\to x\) if and only if for every open set \( \mU\subseteq X\) containing \( x\), there exists a \( \alpha\in A\) such that \( \alpha'\geq \alpha\) implies \( x_{\alpha'}\in\mU\).

So a topology implies a convergence notion for nets, as well as for sequences. However, there exists different topologies which have the same notion of convergence of sequences, but two topologies having the same notion of convergence of nets are the same.


%+++++++++++++++++++++++++++++++++++++++++++++++++++++++++++++++++++++++++++++++++++++++++++++++++++++++++++++++++++++++++++
\section{Homotopy group}

Let \( X\) be a topological space with a base point \( b\), and \( S^n\) be the \( n\)-sphere. The \( n\)th \defe{group of homotopy}{group!homotopy}\index{homotopy group} on the point \( b\) of \( X\) is\nomenclature{\( \pi_n(X,b)\)}{Homotopy group of \( X\) on the base point \( b\)}
\begin{equation}
    \pi_n(X,b)=\{ \text{homotopy classes of maps }f\colon S^n\to X \text{ such that } f(a)=b \}.
\end{equation}
The classes are taken up to homotopy, i.e. continuous deformations. In an equivalent way, \( \pi_n(X,b)\) can be seen as the set of classes of maps \( p\colon [0,1]^n\to X\) which sent the whole border of the cube to \( b\).

%+++++++++++++++++++++++++++++++++++++++++++++++++++++++++++++++++++++++++++++++++++++++++++++++++++++++++++++++++++++++++++
\section{Covering spaces}
%+++++++++++++++++++++++++++++++++++++++++++++++++++++++++++++++++++++++++++++++++++++++++++++++++++++++++++++++++++++++++++

\begin{definition}      \label{DEFooQBDWooVVrkkh}
    Let \( M\) be an Hausdorff connected topological space. Let \( X\) be a topological space. A map \( \pi\colon X\to M\) is a \defe{covering}{covering map} if
    \begin{enumerate}
        \item
            it is continuous
        \item
            it is surjective
        \item
            for every \( p\in M\), there exists a neighbourhood \( V\) of \( p\) such that \( \pi^{-1}(V)\) is a union of disjoint open sets \( \{ A_i \}\) in \( X\) such that the restriction \( \pi\colon A_i\to V\) is homeomorphic\footnote{Definition \ref{DEFooYPGQooMAObTO}.}.
    \end{enumerate}
\end{definition}

\begin{proposition}[lifting property]\index{lifting property!covering space}
Let \( \rho\colon C\to X\) be a covering and \( \gamma\colon [0,1]\to X\), a continuous map. Let \( c\in \rho^{-1}\big( \gamma(0) \big)\). Then there exists one unique path \( \sigma\) in \( C\) such that \( \sigma\circ\rho=\gamma\) and \( \sigma(0)=0\).
\end{proposition}
\begin{proof}
No proof.
\end{proof}
If \( x\) and \( y\) in \( X\) are connected by a path, the lifted path provides a bijection between the fibres \( \rho^{-1}(x)\) and \( \rho^{-1}(y)\).

\begin{proposition}     \label{PROPooPWIFooFAZhVe}
    If \( M\) is a simply connected manifold, then every covering is an homeomorphism.
\end{proposition}

%---------------------------------------------------------------------------------------------------------------------------
\subsection{Universal covering}
%---------------------------------------------------------------------------------------------------------------------------

\begin{definition}
    One says that a covering \( q\colon D\to X\) is \defe{universal}{universal!covering}\index{covering!universal} if \( D\) is simply connected. 
\end{definition}

The following proposition states that an universal covering is a covering that covers all other coverings.

\begin{proposition}
Let \( q\colon D\to X\) be an universal covering, and \( \rho\colon C\to X\) be a covering of \( X\) with \( C\) being connected. Then there exists a covering map \( f\colon D\to C\) such that \( \rho\circ f=q\).
\end{proposition}

The following proposition states that the universal covering is essentially unique.

\begin{proposition}
Let \( q_i\colon D_i\to X\) (with \( i=1,2\)) be two universal coverings of the topological space \( X\). Then there exists an homeomorphism \( f\colon D_1\to D_2\) such that \( q_2\circ f=q_1\).
\end{proposition}

%---------------------------------------------------------------------------------------------------------------------------
					\subsection{Monodromy action}
%---------------------------------------------------------------------------------------------------------------------------
\label{sssMonodromyact}

Let \( \rho\colon C\to X\) be a covering with \( C\) being connected and locally arc connected. First, that shows that \( X\) has these two properties too. Now, let \( x\in X\) and \( c\in\rho^{-1}(x)\) and a path \( \gamma\colon [0,1]\to X\) with \( \gamma(0)=\gamma(1)=x\). By the lifting property, that path lifts to an unique path in \( C\) starting at \( c\), while it is not guarantee that the lifted path will \emph{end} at \( x\). One only knows that the lifted path will end in \( \rho^{-1}(x)\).

It turns out that the end point of the lifted path only depends on the class of \( \gamma\) in \( \pi(X,x)\). Thus we define an action if \( \pi(X,x)\) on the fibre over \( x\). This is the \defe{monodromy action}{monodromy action}. Notice that by taking that action pointwise on \( X\), the group \( \pi(X)\) acts on \( C\).

%+++++++++++++++++++++++++++++++++++++++++++++++++++++++++++++++++++++++++++++++++++++++++++++++++++++++++++++++++++++++++++ 
\section{Haar measure}
%+++++++++++++++++++++++++++++++++++++++++++++++++++++++++++++++++++++++++++++++++++++++++++++++++++++++++++++++++++++++++++

\begin{theorem}[Kakutani fixed point\cite{BIBooNYIOooTIiwbz,BIBooAWWDooUzosWY}]     \label{THOooWXQFooQrWcLY}
    Let \( X\) be an Hausdorff locally convex\footnote{Locally convex, definition \ref{DEFooCGJBooSvDpyC}.} topological vector space. Let \( K\) be compact in \( X\) and \( \mS\) be a group of linear endomorphisms of \( X\) such that
    \begin{enumerate}
        \item
            \( \mS\) is equicontinuous\footnote{Definition \ref{DEFooDHQDooFfIvsX}.} on \( K\),
        \item
            \( \mS(K)\subset K\).
    \end{enumerate}
    There exists a point \( p\in K\) such that \( T(p)=p\) for every \( T\in \mS\).
\end{theorem}

\begin{proof}
    We'll use the Zorn lemma. Let 
    \begin{equation}
        \mA=\big\{  L\subset K\tq L \text{ is non empty, compact, convex and} \mS(L)\subset L   \big\}.
    \end{equation}
    The set \( \mA\) is ordered with \( L_1\leq L_2\) if and only if \( L_2\subset L_1\) (this is the opposite of what we usually do).

    \begin{subproof}
    \spitem[\( (\mA, \leq)\) is inductive]
        Let \( \mF\subset \mA\) be a totally ordered subset. We prove that 
        \begin{equation}
            M=\bigcap_{A\in \mF}A
        \end{equation}
        is a upper bound of \( \mF\).
        \begin{subproof}
        \spitem[Finite intersection property]
            We prove that the family \( \mF\) has the finite intersection property\footnote{Definition \ref{DEFooCESGooZkACqs}.}. Let \( \mI\) be finite in \( \mF\). The set \( (\mI,\leq)\) is totally ordered and the lemma \ref{LEMooPCRFooXRGrUr} says that \( \mI\) has a maximum. Here the order is the inverse inclusion, so that the maximum is in fact the smaller set, which is contained in all the others. In other words, the maximum, which is an element of \( \mI\) is the intersection \( \bigcap_{A\in\mI}A\).
        \spitem[\( M\) is non empty]
            The family \( \mF\) is a family of closed\footnote{Compacts are closed, see \ref{LemnAeACf}.} parts of \( X\) which has the finite intersection property. Thus the intersection of \( \mF\) is non empty by theorem \ref{THOooCQSQooDuasqo}.
            \spitem[\( M\) is compact] 
            Since \( X\) is Hausdorff, every intersection of compacts parts is compact by the proposition \ref{PROPooQWHSooXeJOkT}.
        \spitem[\( M\) is convex]
            Proposition \ref{PROPooJOCEooUKhkqQ} says that every intersection of convex is convex.
        \spitem[\( \mS(M)\subset\mS(M)\)]
            Let \( x\in M\). For every \( A\in \mF\), we have \( x\in A\). Since \( \mS(A)\subset A\) we also have \( \mS(x)\in A\) and \( \mS(x)\in \bigcap_{A\in\mF}A\).
        \end{subproof}
    \spitem[Zorn, definition of \( K_1\)] 
        Since \( \mA\) is inductive we use the Zorn lemma \ref{LemUEGjJBc} and we let \( K_1\) be a maximal element of \( \mA\).
    \spitem[If \( K_1\) has only one point]
        Let us suppose that \( K_1\) contains only one element \( p\). We prove that \( T(p)=p\) for every \( T\in\mS\). Since \( K_1\in \mA\) we have \( T(K_1)\subset K_1\) and since \( K_1=\{ p \}\) we have \( T(p)\in \{ p \}\) which means \( T(p)=p\).
    \end{subproof}
    We suppose that \( K_1\) has at least \( 2\) points and we will reach a contradiction. As far as the notations are concerned, we write \( K_1-K_1=\{ k_1-k_2\tq k_1,k_2\in K_1 \}\).
    \begin{subproof}
    \spitem[\( K_1-K_1\) has a non zero element]
        The part \( K_1\) contains at least two elements. Let \( k_1\) and \( k_2\) be two distinct elements in \( K_1\). Then \( k_1-k_2\neq 0\) in \( K_1-K_1\).
    \spitem[Definition of \( V\)]
        From lemma \ref{LEMooMDTNooThlHJl}, we consider in \( X\) a neighbourhood \( V\) of \( 0\) such that \( \bar V\) is not contained in \( K_1-K_1\).
    \spitem[Definition of \( V_1\)]
        Since \( X\) is locally convex, \( 0\) has a basis of convex neighbourhoods. Let \( V_1\subset V\) be a convex neighbourhood of \( 0\).
    \spitem[Definition of \( U_1\)]
        The set \( \mS\) is equicontinuous on \( K\) and then equicontinuous on \( K_1\). There exists a neighbourhood \( U_1\) of \( 0\) in \( X\) such that for every \( k_1,k_2\in K_1\) such that \( k_2-k_1\in U_1\), we have \( \mS(k_1-k_2)\subset  V_1\). In other words,
        \begin{equation}        \label{EQooPELQooJgcrzr}
            \mS\big( U_1\cap(K_1-K_1) \big)\subset V_1.
        \end{equation}
    \spitem[Open map]
        Since \( \mS\) is a group, each of the linear maps \( T\in\mS\) is invertible and its inverse is part of \( \mS\). In other words, the elements of \( \mS\) are continuous linear maps with continuous inverse.

        In particular, if \( \mO\) is open in \( X\), then \( T(\mO)\) is open in \( X\).
    \spitem[Définition of \( U_2\)]
        We define
        \begin{equation}       
            U_2=\Conv\Big( \mS\big(  U_1\cap(K_1-K_1)  \big) \Big).
        \end{equation}
        Since the elements \( T\in \mS\) are linear and inversible, we have \( T(A\cap B)=T(A)\cap T(B)\) and \( T\big( \Conv(A) \big)=\Conv\big( T(A) \big)\) for every parts \( A,B\subset X\). Thus we have
        \begin{equation}
            \mS(U_2)=U_2.
        \end{equation}
    \spitem[\( K_1-K_1\subsetneq U_2\)]
        The set \( U_2\) is the convex hull of \( \mS\big( U_1\cap (K_1-K_1) \big)\) which is a part of the convex set \( V_1\) by \eqref{EQooPELQooJgcrzr}. Thus \( U_2\subset V_1\). We have the inclusions chain
        \begin{equation}
            U_2\subset V_1\subset V.
        \end{equation}
        Since \( K_1-K_2\) is not included in \( V\), it is not included in \( U_2\).
    \spitem[Definition of \( U\)]
        Since the operators \( T\) are continuous and since \( T(U_2)\subset U_2\), we also have \( \mS(\bar U_2)\subset \bar U_2\). The part \( U_2 \) is an open convex containing \( 0\) and \( K_1-K_2\) is compact. Thus lemma \ref{LEMooDVZWooWKRQWC} implies the existence of \( \lambda>0\) such that \( K_1-K_2 \subset \lambda U_2\). Let
        \begin{equation}
            \delta=\inf\{ \lambda>0\tq K_1-K_1\subset \lambda U_2 \}.
        \end{equation}
        Notice that \( \delta<\infty\) because of lemma \ref{LEMooDVZWooWKRQWC} and \( \delta >1\) because \( K_1-K_1\nsubset U_2\) . We define
        \begin{equation}
            U=\delta U_2.
        \end{equation}
    \spitem[Some sums]
        Let \(   0<\epsilon<1   \). For every \( \lambda>\delta\) we have \( K_1-K_1\subset \lambda U_2\). In particular with \( \lambda= (1+\epsilon)\delta\). Thus
        \begin{equation}
            K_1-K_1\subset (1+\epsilon)\delta U_2=(1+\epsilon)U.
        \end{equation}
        In the same way, for every \( \lambda<\delta\) we have \( K_1-K_1\nsubset \lambda U_2\). Thus
        \begin{equation}        \label{EQooNYNTooQScVJL}
            K_1-K_2\nsubset (1-\epsilon)\bar U.
        \end{equation}
        
        For \( k\in K_1\) we consider \( \mO_k=\frac{ 1 }{2}U+k\). Since \( 0\in U\) the set \( \{ \mO_k \}_{k\in K_1}\) is an open covering of \( K_1\). The set \( K_1\) being compact we can extract a finite subcovering: there exists \( k_1,\ldots, k_n\in K_1\) such that
        \begin{equation}
            K_1\subset \bigcup_{i=1}^n(\frac{ 1 }{2}U+k_i).
        \end{equation}
        Let
        \begin{equation}
            p=\frac{ k_1+\ldots +k_n }{ n }.
        \end{equation}
        Let \( k\in K_1\). Two statements:
        \begin{enumerate}
            \item
                For each \( l\) we have \( k_l-k\in (1+\epsilon)U\).
            \item
                There exists \( i\) such that \( k_i-k\in \frac{ 1 }{2}U\).
        \end{enumerate}
        Thus there exist \( u_1,\ldots, u_n\in U\) such that
        \begin{subequations}        \label{EQSooZFRQooJoFhXF}
            \begin{numcases}{}
                k_i-k=\frac{ 1 }{2}u_i\\
                k_l-k=(1+\epsilon)u_l.
            \end{numcases}
        \end{subequations}
        Inserting the values of \( k_i\) from \eqref{EQSooZFRQooJoFhXF} in the definition of \( p\),
        \begin{subequations}
            \begin{align}
                p&=\frac{1}{ n }\sum_{j=1}^nk_j\\
                &=\frac{1}{ n }\Big( \frac{ 1 }{2}u_i+k+\sum_{j\neq i}\big[ (1+\epsilon)u_l+k \big] \Big)\\
                &=\frac{1}{ n }\Big( \frac{ 1 }{2}u_i+\sum_{j\neq i}(1+\epsilon)u_l \Big)
            \end{align}
        \end{subequations}
        The set \( U\) is convexe because \( U=\delta U_2\) where \( U_2\) is convex. Thus if \( u_1, u_2\in U\) we have \( \frac{ u_1+u_2 }{2}\in U\) and \( u_1+u_2\in 2U\). In the same way, there exists \( u'\in U\) such that
        \begin{equation}
            \sum_{j\neq i}u_l=(n-1)u'.
        \end{equation}
        So we write
        \begin{equation}
            p=\frac{1}{ n }\big( \frac{ 1 }{2}u_i +(n-1)(1+\epsilon)u' \big)+k.
        \end{equation}
        Using lemma \ref{LEMooAHUMooBwxzPj}, there exists \( u\in U\) such that
        \begin{equation}
            p=\frac{1}{ n }\big( \frac{ 1 }{2}+(n-1)(1+\epsilon) \big)u+k.
        \end{equation}
        We write that with \( \epsilon=\frac{1}{ 4(n-1) }\). The coefficient becomes
        \begin{equation}
            \frac{ 1 }{2}+(n-1)(1+\epsilon)=\frac{ 4n-1 }{ 4 }
        \end{equation}
        and we have proven that, for each \( k\in K_1\), there exists \( u\in U\) such that
        \begin{equation}
            p=\frac{ 4n-1 }{ 4n }u+k.
        \end{equation}
    \spitem[Definition of \( K_2\)]
        Let, for each \( k\in K_1\),
        \begin{equation}
            A_k=\frac{ 4n-1 }{ 4n }\bar U+k
        \end{equation}
        and
        \begin{equation}
            K_2=K_1\cap\bigcap_{k\in K_1}A_k.
        \end{equation}
        We give some properties of \( K_2\).
        \begin{subproof}
    \spitem[\( K_2\) is non empty]
        We know that \( p\in K_2\).
    \spitem[\( K_2\subsetneq K_1\)]
        From what we said around equation \eqref{EQooNYNTooQScVJL}, there exists \( k_1,k_2\in K_1\) such that
        \begin{equation}
            k_1-k_2\notin\left( 1-\frac{ 4n-1 }{ 4n } \right)\bar U,
        \end{equation}
        which means that \( k_1\notin A_{k_2}\). In particular \( k_1\notin K_2\). Thus the inclusion \( K_2\subset K_1\) is strict.
    \spitem[\( K_2\) is compact and convex]
        The sets \( K_1\) and \( A_k\) are closed. The set \( K_2\) is closed as intersection of closed sets. Since \( K_2\) is closed in the compact \( K_1\), it is compact (lemma \ref{LemnAeACf}\ref{ITEMooNKAKooQoNddr}).

        The set \( K_2\) is convex as intersection of convexes.
    \spitem[\( \mS(K_2)\subset K_2\)]
        Let \( T\in \mS\) and \( u\in\bar U\). We have \( u=\delta u_2\) with \( u_2\in\bar U_2\). Then
        \begin{equation}
            T(u)=\delta T(u_2)\in \delta T(\bar U_2)\subset \delta\bar U_2=\bar U
        \end{equation}
        because \( T(U_2)\subset U_2\) and \( T\) is continuous. We proved that \( T(\bar U)\subset \bar U\).

        Let \( a\in K_2\). We have \( a\in K_1\) and, since \( a\in A_k\) for each \( k\in K_1\), we have a map \( u\colon K_1\to \bar U\) such that
        \begin{equation}        \label{EQooFDOZooNTThIj}
            \lambda u(k)+k=a
        \end{equation}
        where \( \lambda=\frac{ 4n-1 }{ 4n }\). 
        Consider \( l\in K_1\) and let us show that \( T(a)\in A_{l}\). Since \( T(K_1)=K_1\), there exists \( l'\in K_1\) such that \( T(l')=l\). Writing \eqref{EQooFDOZooNTThIj} with \( l'\) we have
        \begin{equation}
            T(a)=\lambda T\big( u(l') \big)+T(l')=\lambda T\big( u(l') \big)+l.
        \end{equation}
        Since \( u(l')\in \bar U\) and \( T(\bar U)\subset \bar U\), we have
        \begin{equation}
            T(a)\in \lambda\bar U+l=A_l.
        \end{equation}
        We have proved that \( T(a)\in A_l\) for each \( l\in K_1\).
        \end{subproof}
    \spitem[Conclusion]
        The part \( K_2\) belong to \( \mA\) and satisfies the strict inclusion \( K_2\subsetneq K_1\). This means that \( K_2>K_1\), which contradicts the maximality of \( K_1\).

        We conclude that \( K_1\) has only one point, so that \( \mS\) has a unique fixed point as shown before.
    \end{subproof}
\end{proof}

We recall the regular left action. If \( G\) is a group, \( u\) is a map \( u\colon G\to E\) and \( s\in G\), we write
\begin{equation}
    (L_su)(x)=u(s^{-1}x).
\end{equation}

\begin{definition}[Haar measure]        \label{DEFooSBRZooUbzMnN}
    Let \( G\) be a compact topological group\footnote{Definition \ref{DEFooCHZVooHnvSgW}.}. A linear form \( m\colon C^0(G,\eC)\to \eC\) is a \defe{Haar measure}{Haar measure} if it satisfies
    \begin{enumerate}
        \item
            \( m(f)\geq 0\) for every \( f\geq 0\).
        \item
            \( m(\mtu)=1\) where \( \mtu\) is the constant function \( \mtu(x)=1\).
        \item
            \( m(L_sf)=m(f)\) for every \( f\in C^0(G,\eC)\) and \( s\in G\).
    \end{enumerate}
\end{definition}

\begin{lemma}[\cite{BIBooYIEQooXEgwMv}]     \label{LEMooRWOFooDOSUYo}
    A Haar measure on a compact topological group \( G\) is continuous on \( \big( C^0(G,\eC), \| . \|_{\infty} \big)\).
\end{lemma}

\begin{proof}
    Let \( f\in C^0(G,\eC)\), and let \( m\) be a Haar measure on \( G\). 

    Support for the moment that \( f\) is real valued. For every \( x\in G\) we have
    \begin{equation}
        -\| f \|\leq f(x)\leq \| f \|.
    \end{equation}
    If we consider \( \| f \|\) as the constant function on \( G\) we can write the inequalities \( -\| f \|\leq f\leq \| f \|\) in \( C^0(G,\eC)\).  Since \( m\) is positive, we can apply \( m\) to these inequalities :
    \begin{equation}
        m(-\| f \|)\leq m(f)\leq m(\| f \|).
    \end{equation}
    By linearity of \( m\) we have \( -\| f \|m(\mtu)\leq m(f)\leq \| f \|m(\mtu)\), and then
    \begin{equation}
        -\| f \|\leq m(f)\leq \| f \|.
    \end{equation}
    
    Now we suppose that \( f\) is complex-valued and that \( f(x)=u(x)+iv(x)\) where \( u\) and \( v\) are real-valued. We have:
    \begin{subequations}        \label{SUBEQooKMQEooOoydWP}
        \begin{align}
            | m(f) |&=| m(u)+im(v) |\\
            &\leq | m(u) |+| m(v) |\\
            &\leq \| u \|+\| v \|\\
            &\leq 2\| f \|.
        \end{align}
    \end{subequations}
    We used the result for real valued functions and lemma \ref{LEMooLPRZooUPsWTR}. The inequality \eqref{SUBEQooKMQEooOoydWP} shows that \( m\) is continuous at \( 0\). Since \( m\) is linear, it is continuous.
\end{proof}

\begin{theorem}[Haar measure\cite{BIBooNYIOooTIiwbz,BIBooYIEQooXEgwMv}]       \label{THOooHZNRooLWmJMB}
    Let \( G\) be a compact topological group. 
    \begin{enumerate}
        \item
            There exists a unique Haar measure\footnote{Definition \ref{DEFooSBRZooUbzMnN}.} on \( G\).
        \item
            A Haar measure satisfies \( m(R_sf)=m(f)\).
    \end{enumerate}
\end{theorem}

\begin{proof}
    Here is a summary of the proof.
    \begin{enumerate}
        \item
            We prove the existence by building a Haar measure \( m\).
        \item
            We show that \( m\) satisfies \( m(R_sf)=m(f)\).
        \item
            We show the unicity by showing that every Haar measure is equal to \( m\).
    \end{enumerate}

    For \( f\in C^0(G,\eC)\) we consider \( C_f\), the set of convex combinations of the functions of the form \( L_s(f)\). In other words, \( g\in C_f\) is there exists \( \{ (a_i,s_i)\in \eR^+\times G \}_{i=1,\ldots, N}\) such that \( a_i>0\), \( \sum_ia_i=1\) and \( g(x)=\sum_ia_if(s_ix)\).

    Since \( f\) is continuous in the compact space \( G\), we define
    \begin{equation}
        \| f \|=\max_{x\in G}| f(x) |.
    \end{equation}
    If \( g\in C_f\) we have \( \| g \|\leq \| f \|\) because
    \begin{subequations}
        \begin{align}
            \| g \|&=\max_{x\in G}| \sum_ia_if(s_ix) |\\
            &\leq\sum_i\max_{x\in G}| a_if(s_ix) |\\
            &\leq \sum_ia_i\max_{x\in G}| f(s_ix) |\\
            &=\sum_ia_i\| f \|\\
            &=\| f \|.
        \end{align}
    \end{subequations}
    For \( x\in G\) we also define
    \begin{equation}
        C_f(x)=\{ g(x)\tq g\in C_f \}.
    \end{equation}
    This is a bounded set in \( \eC\) because each element is bounded by \( \| f \|\). The closure \( \overline{ C_f(x) }\) is then compact in \( \eC\). The function \( f\) being continuous, it is uniformly continuous on \( \overline{ C_f(x) }\) (proposition \ref{PROPooSHBAooVRdAFM}.). Let \( \epsilon>0\) and an open neighbourhood \( V\) of \( e\) as in the definition of uniform continuity. If \( x,y\in V\) we have
    \begin{equation}
        | (L_sf)(x)-(L_sf)(y) |=| f(s^{-1}x)-f(s^{-1} y) |<\epsilon
    \end{equation}
    because \( (s^{-1}y)^{-1}(s^{-1}x)=y^{-1}x\).

    This shows that the set \( \{ L_s(f) \}_{s\in G}\) is equicontinuous\footnote{Definition \ref{DEFooDHQDooFfIvsX}.}. Now we prove that \( C_f\) is equicontinuous. Let \( g\in C_f\) and \( x,y\in V\). We have
    \begin{subequations}
        \begin{align}
            | g(x)-g(y) |&=\Big|   \sum_ia_i\big[ f(s_ix)-f(x_iy) \big]   \Big|\\
            &\leq \sum_ia_i| s_{s_i^{-1}}(x)-L_{s_i^{-1}}(y) |\\
            &\leq \sum_ia_i\epsilon\\
            &=\epsilon,
        \end{align}
    \end{subequations}
    so \( C_f\) is equicontinuous. The Ascoli theorem \ref{ThoKRbtpah} implies that \( C_f\) is relatively compact in \( C^0(G,\eC)\), which means that \( K_f=\overline{ C_f }\) is compact.

    The action
    \begin{equation}
        \begin{aligned}
            L\colon G\times C^0(G,\eC)&\to C^0(G,\eC) \\
            (g,u)&\mapsto L_g(u) 
        \end{aligned}
    \end{equation}
    is isometric: \( \| L_s(u) \|=\| u \|\) and leaves \( C_f\) invariant.
    \begin{subproof}
    \spitem[\( L_s(C_f)\subset C_f\)]
        We write
        \begin{equation}
            \begin{aligned}
                \psi\colon \eR^N\times G^N&\to C_f \\
                (a_i), (s_i)&\mapsto \sum_ia_iL_{s_i}(f). 
            \end{aligned}
        \end{equation}
        If \( g\in C_f\) we have
        \begin{equation}
                g(x)=\psi\big( (a_i), (s_i) \big)(x)=\sum_ia_if(s_ix),
        \end{equation}
        so that
        \begin{subequations}
            \begin{align}
                (L_sg)(x) &=\sum_ia_if(s_is^{-1}x)\\
                &=\psi\big( (a_i),(s_is^{-1}) \big)(x).
            \end{align}
        \end{subequations}
        This shows that \( L_s(g)\in C_f\).
    \spitem[\(C_f\subset L_s( C_f)\)]
        The same kind of computation shows that
        \begin{equation}
            \psi\big( (a_i),(s_i) \big)=L_s\Big( \psi\big( (a_i), (s_is) \big) \Big).
        \end{equation}
    \spitem[Kakutani]
        Let us summarise some facts.
        \begin{itemize}
            \item
                \( L_s(K_f)=K_f\) by lemma \ref{LEMooSCIIooRyRrHA}. 
            \item 
                \( K_j\) is compact and convex in \( C^0(G,\eC)\).
            \item
                \( G\) is a group of linear transformations acting on \( K_f\) : \( G(K_f)\subset K_f\).
            \item
                \( G\) is isomectric on \( K_f\), and then equicontinuous on \( K_f\) by lemma \ref{LEMooMIHJooUhvPgM}.
        \end{itemize}
        The Kakutani theorem \ref{THOooWXQFooQrWcLY} is then valid and there exists a fixed point \( h\in K_f\) for the action of \( G\). In other words, \( L_s(h)=h\) for every \( s\in G\). The function \( h\) is constant because for every \( s\in G\) we have
        \begin{equation}
            h(s)=(L_{s^{-1}}h)(e)=h(e).
        \end{equation}
        Thus \( K_f\) contains one constant function. We prove now that it contains only one constant.
    \spitem[\( K_f\) contains only one constant]
        We have just seen that \( K_f\) contains a constant function. We are going to show that \( K_f\) contains only one constant. Let \( c\in \eC \) be such a constant, and let \( \epsilon>0\). Since \( c\in K_f=\bar C_j\), there exists an element \( g\in C_f\) such that \( \| c-g \|<\epsilon\). More precisely, there exists \( a_i>0\) and \( \{ s_1,\ldots, s_N \}\in G\) such that
        \begin{subequations}        \label{SUBEQSooDGOPooQLxVjw}
            \begin{numcases}{}
                g(x)=\sum_{i=1}^Na_jf(s_ix)\\
                \sum_{i=1}^Na_i=1\\
                | c- g(x) |<\epsilon.
            \end{numcases}
        \end{subequations}
        The third condition holds for every \( x\in G\).

        We can make the whole work with the right action instead of the left action of \( G\) on \( C^0(G,\eC)\). That leads us to build \( C'_f\), \( K'_f\) and a constant function \( c'\in K'_f\), elements \( b_j\geq 0\) and \( t_j\in G\) such that
        \begin{subequations}        \label{SUBEQSooBQIZooApBecK}
            \begin{numcases}{}
                | c'-\sum_{j=1}^Nb_jf(xt_j) |<\epsilon\\
                \sum_{j=1}^Nb_j=1.
            \end{numcases}
        \end{subequations}
        You may wonder why the conditions \eqref{SUBEQSooBQIZooApBecK} hold for the same \( N\) as the conditions \eqref{SUBEQSooDGOPooQLxVjw}. In fact, we took \( N\) as the maximum of the two and we set \( a_i=0\) or \( b_i=0\) for the indices \( i\) between the «correct» \( N\)  and \( N\).

        Let us fix \( i\), multiply the first condition \eqref{SUBEQSooBQIZooApBecK} by \( a_i\) and set \( x=s_i\) :
        \begin{equation}
            | c'a_i-\sum_{j=1}^Na_ib_jf(s_it_j) |<\epsilon a_i.
        \end{equation}
        Making the sum over \( i\) we get\footnote{We can enter the sum in the module because if \( | a_i |<b_i\), then \( | \sum_ia_i |\leq \sum\i| a_i |<\sum_i b_i\).}
        \begin{equation}
            | \sum_ic'a_i-\sum_{ij}a_ib_jf(s_it_j) |<\sum_i\epsilon a_i,
        \end{equation}
        and then
        \begin{equation}
            | c'-\sum_{ij}a_ib_jf(s_it_j) |<\epsilon.
        \end{equation}
        We make the same from \eqref{SUBEQSooDGOPooQLxVjw} : multiply by \( b_j\), \( x=t_j\) and sum over \( j\) :
        \begin{equation}
            | c-\sum_{ij}a_ib_jf(s_it_j) |<\epsilon.
        \end{equation}
        And we deduce that \( c=c'\) because
        \begin{equation}
            | c-c' |\leq | c-\sum_{ij}a_ib_jf(s_it_j) |+| \sum_{ij}a_ib_jf(s_it_j)-c' |<2\epsilon.
        \end{equation}
        
        We have shown that if \( c\) is a constant in \( K_f\) and if \( c'\) is a constant in \( K'_f\), then \( c=c'\). We deduce that the set of constants in \( K_f\) and \( K'_f\) are the same and contain only one element.

    \spitem[Definition of \( m\)]
        We denote by \( m(f)\) the unique \( c\in K_f\) such that \( \| c-C_f \|=0\). This defines a map
        \begin{equation}
            m\colon C^0(G,\eC)\to \eC.
        \end{equation}
        We prove now a lot of properties.
    \spitem[\( m(1)=1\)]
        We consider the constant function \( 1\). A function in \( C_1\) has the form
        \begin{equation}
            g(x)=\sum_ia_i1(x_ix)=\sum_ia_i=1.
        \end{equation}
        Thus \( C_1=K_1=\{ 1 \}\), so that \( m(1)=1\).
    \spitem[\( m(f)\geq 0\) if \( f\geq 0\)]
            An element of \( C_f\) has the form \( g(x)=\sum_ia_if(s_ix)\) where \( a_i\geq 0\). Thus all functions of \( C_f\) are positive. A function in \( K_f\) cannot take a strictly negative value. In particular, the constant one is positive.
        \spitem[\( m(\lambda f)=\lambda m(f)\)]
            An element of \( C_{\lambda f}\) has the form \( g(x)=\sum_ia_i(\lambda f)(s_ix)=\lambda\sum_ia_if(s_ix)\), so that \( C_{\lambda f}=\lambda C_f\) and \( K_{\lambda f}=\lambda K_f\). If \( c\) is the constant in \( K_f\), then \( \lambda c\) is the constant in \( K_{\lambda f}\).
        \spitem[\( m(L_sf)=m(f)\)]
            An element in \( C_{L_sf}\) has the form
            \begin{equation}
                g(x)=\sum_ia_i(L_sf)(s_ix)=\sum_if(s^{-1}s_ix).
            \end{equation}
            This shows that \( C_{L_sf}\subset C_f\). The same shows that \( C_f\subset C_{L_sf}\).
        \spitem[\( m(R_sf)=m(f)\)]
            Same as \( m(L_sf)=m(f)\).
        \spitem[\( m\) is linear]
            We have already shown that \( m(\lambda f)=\lambda m(f)\). So let \( f,g\in C^0(G,\eC)\) and let us prove that \( m(f+g)=m(f)+m(g)\). Let \( \epsilon>0\); we consider some reals \( a_i\) and some elements \( s_i\in G\) such that
            \begin{equation}        \label{EQooDDZXooWIPAhX}
                | m(f)-\sum_ia_if(s_ix) |<\epsilon,
            \end{equation}
            and we define
            \begin{equation}
                h(x)=\sum_ia_ig(s_ix).
            \end{equation}
            We have \( h\in C_g\), so that \( C_h\subset C_g\) and \( K_h\subset K_g\). Since there is only one constant in \( K_g\) and only one constant in \( K_h\), we deduce that \( m(h)=m(g)\). There exist a function \( l\in C_h\) such that \( \| h-l \|_{\infty}<\epsilon\); in other words we can choose \( b_j\in \eR^+\) and \( t_j\in G\) such that
            \begin{equation}
                | m(h)-\sum_{j}b_jh(t_jx) |<\epsilon
            \end{equation}
            for every \( x\in G\). We inject therein the definition of \( h\) and the fact that \( m(h)=m(g)\) :
            \begin{equation}        \label{EQooZZOSooQwJtjp}
                | m(g)-\sum_jb_j\sum_ia_ig(s_it_jx) |<\epsilon.
            \end{equation}
            We multiply \eqref{EQooDDZXooWIPAhX} by \( b_j\), we evaluate it at \(t_jx\) and we make the sum over \( j\) :
            \begin{equation}
                | \underbrace{\sum_jb_jm(f)}_{=m(f)}-\sum_{ij}a_ib_jf(s_it_jx) |<\epsilon.
            \end{equation}
            Combining with \eqref{EQooZZOSooQwJtjp} we have the majorations :
            \begin{subequations}
                \begin{align}
                    | m(f)+m(g)-\sum_{ij}a_ib_j(f+g)(s_it_jx) |&\leq | m(f)-\sum_{ij}a_ib_jf(s_it_jx) |\\
                                                                &\quad+| m(g)-\sum_{ij}a_ib_jg(s_it_jx) |   \nonumber\\
                    &<2\epsilon.
                \end{align}
            \end{subequations}
            This proves that the constant \( m(f)+m(g)\) belong to \( K_{f+g}\). Since \( K_{f+g}\) contains only one constant, which is \( m(f+g)\), we deduce that \( m(f)+m(g)=m(f+g)\).
    \end{subproof}
    Up to now we build a Haar measure \( m\colon C^0(G,\eC)\to \eC\) which satisfies \( m(R_sf)=m(f)\). We prove the unicity. 

    Let \( m_2\colon C^0(G,\eC)\to \eC\) be an other Haar measure on \( G\). We will prove that \( m_2(f)=m(f)\) for every \( f\). 
    \begin{subproof}
    \spitem[\( m_2\) is constant on \( C_f\)]
        Let \( g\in C_f\). We have \( g(x)=\sum_ia_iL_{s_i(f)}(x)\), so that
        \begin{equation}
            m_2(g)=\sum_ia_im_2\big(L_{s_i}(f)\big)=\sum_ia_im_2(f)=m_2(f).
        \end{equation}
        We used the left invariance of \( m_2\) and the fact that \( \sum_ia_i=1\).
    \spitem[\( m_2\) is constant on \( K_f\)]
        We know from lemma \ref{LEMooRWOFooDOSUYo} that \( m_2\) is continuous on \( C^0(G,\eC)\). Let \( g,h\in K_f\). There exists sequences \( (g_i)\) ang \( (h_i)\) in \( C_f\) such that \( g_i\stackrel{C^0(G,\eC)}{\longrightarrow}g\) and \( h_i\stackrel{C^0(G,\eC)}{\longrightarrow} h\). For every \( i\) we have \( m_2(g_i)=m_2(h_i)\) (because \( m_2\) is constant on \( C_f\)). By continuity of \( m_2\) we also have \( m_2(g)=m_2(h)\).
    \spitem[\( m_2=m\)]
        Let \( f\in C^0(G,\eC)\). We consider the constant function 
        \begin{equation}
            \begin{aligned}
                g\colon G&\to \eC \\
                x&\mapsto m(f). 
            \end{aligned}
        \end{equation}
        This function belongs to \( K_f\); this is even, by definition, the only constant function in \( K_f\). Thus \( m_2(f)=m_2(g)\) because \( m_2\) is constant on \( K_f\). Now we have
        \begin{equation}
            m_2(f)=m_2(g)=m_2\big( m(f)\mtu \big)=m(f)m_2(\mtu)=m(f).
        \end{equation}
    \end{subproof}
    Thus every Haar measure on \( G\) is equal to \( m\) and we deduce that there exists only one Haar measure on \( G\).
\end{proof}

\begin{definition}      \label{DEFooXIKEooWOxHlr}
    Let \( G\) be a topological group. A \defe{Haar measure}{Haar measure} on \( G\) is a positive measure\footnote{Definition \ref{DefBTsgznn}.} \( \mu\) such that
	\begin{enumerate}
		\item
		      \( \mu(gA)=\mu(A)\) for every \( g\in G\) and measurable part \( A\subset G\),
		\item
		      \( \mu(K)<\infty\) for every compact part \( K\subset G\).
	\end{enumerate}
    If the group \( G\) is itself compact, we requite the measure to be normalized : \( \mu(G)=1\).
\end{definition}


%TODOooBQYGooFNefUp Revoir la partie unicité. À mon avis c'est possible que la truibu soit également unique.
\begin{theorem}[Riesz-Markov representation theorem\cite{BIBooMGTQooQQVmPG,BIBooWEPLooDsLEvu, MonCerveau}]      \label{THOooTWZWooHqGDAx}
    Let \( X\) be a Hausdorff locally compact space. Let \( m\) be a positive linear form ob \( C^0_c(X, \eR)\). There exists a unique measured\quext{I'm not completely sure about the unicity of the \( \sigma\)-algebra. At least if \( (\tribA_1,\mu_1)\) and \( (\tribA_2,\mu_2)\) satisfy the requirements, we have \( \mu_1=\mu_2\) on \( \tribA_1\cap\tribA_2\). Please, contact me if you have an opinion.} space \( (X,\tribA, \mu)\)  containing the Borel sets such that
    \begin{enumerate}
        \item       \label{ITEMooAKLQooWLetDk}
            The integral build on \( \mu\) satisfies
            \begin{equation}
                m(f)=\int_X fd\mu
            \end{equation}
            for every \( f\in C^0_c(X,\eR)\).
        \item       \label{ITEMooFLYHooTNUENu}
            \( \mu(K)<\infty\) for every compact set \( K\subset X\).
        \item       \label{ITEMooKDTLooJuUTaW}
            If \( E\in\tribA\) and \( \mu(E)<\infty\), then
            \begin{equation}
                \mu(E)=\inf\{ \mu(V)\tq E\subset V \text{ and \( V\) is open in \( X\) }\},
            \end{equation}
        \item
            If \( E\in\tribA\) and \( \mu(E)<\infty\), then
            \begin{equation}
                \mu(E)=\sup\{ \mu(K)\tq K\subset E \text{ and \( K\) is compact in \( X\) }\},
            \end{equation}
        \item
            The measured space \( (X,\tribA,\mu)\) is complete\footnote{Definition \ref{DefBWAoomQZcI}.}
    \end{enumerate}
\end{theorem}

\begin{proof}
    We initiate with the unicity.
    \begin{center}
        Unicity
    \end{center}
    Let \( \tribA\) be a \( \sigma\)-algebra containing the Borel set and let \( \mu_1\), \( \mu_2\) be measures on \( (X,\tribA)\) satisfying the conditions.
    \begin{subproof}
    \spitem[Unicity on compacts parts]

    Let \( K\) be compact in \( X\). Since it is closed, it is a Borel set and \( K\in \tribA\). Let \( \epsilon>0\). By \ref{ITEMooFLYHooTNUENu} and \ref{ITEMooKDTLooJuUTaW} there exists an open set \( V\) such that \( \mu_1(V)<\mu_1(K)+\epsilon\). From Urysohn's lemma \ref{LEMooECTNooKagaRU}, there exists a map \( f\in C^0_c(X,\eR)\) such that \( f(K)=\{ 1 \}\) and \( f(x)=0\) when \( x\notin V\). We have
    \begin{subequations}
        \begin{align}
            \mu_2(K)&=\int_X\mtu_Kd\mu_2        \label{SUBEQooWVQVooLbgJDZ}\\
            &\leq \int_Xfd\mu_2\\
            &=m(f)                  \label{SUBEQooXKADooTpXNlS}\\
            &= \int_Xfd\mu_1\\
            &\leq \int_X\mtu_{V}d\mu_1\\
            &=\mu_1(V)\\
            &<\mu_1(K)+\epsilon.
        \end{align}
    \end{subequations}
    Justifications.
    \begin{itemize}
        \item For \eqref{SUBEQooWVQVooLbgJDZ}. Link between integral and measure, lemma \ref{LemooPJLNooVKrBhN}.
        \item For \eqref{SUBEQooXKADooTpXNlS}. Property \ref{ITEMooAKLQooWLetDk}.
    \end{itemize}
    Thus for every \( \epsilon>0\) we have \( \mu_2(K)\leq \mu_1(K)+\epsilon\). Permuting the roles of \( \mu_1\) and \( \mu_2\) we also get \( \mu_1(K)\leq \mu_2(K)+\epsilon\).

    We conclude that \( \mu_1(K)=\mu_2(K)\).
\spitem[Unicity]
    Let \( E\in\tribA\). We have
    \begin{equation}
        \{ \mu_1(K)\tq \text{ \( K\) is compact in \( V\)} \} = \{ \mu_2(K)\tq \text{ \( K\) is compact in \( V\)} \},
    \end{equation}
    thus \( \mu_1(V)=\mu_2(V)\).
    \end{subproof}
    \begin{center}
        Existence
    \end{center}
    Existence is the big part. Let us introduce some notations. If \( f\in C^0_c(X,\eR)\) and \( E\subset X\) we write
    \begin{enumerate}
        \item \( f\prec E\) if \( \supp(f)\subset E\) and \( f(X)\subset \mathopen[ 0 , 1 \mathclose]\).
        \item \( E\prec f\) if \( f(E)=\{ 1 \}\) and \( f(X)\subset \mathopen[ 0 , 1 \mathclose]\).
    \end{enumerate}
    For open sets \( V\subset X\) we define
    \begin{equation}
        \mu^o(V)=\sup\{ m(f)\tq f\prec V \}.
    \end{equation}
    Then, for every part \( E\) we define
    \begin{equation}
        \mu^+(E)=\inf\{ \mu^o(V)\tq \text{ \( E\subset V\) and \( V\) open} \},
    \end{equation}
    and
    \begin{equation}        \label{EQooSIKUooJeeIqI}
        \mu^-(E)=\sup\{ \mu^+(K)\tq \text{ \( K\subset E\) and \( K\) compact} \},
    \end{equation}
    and
    \begin{equation}
        \tribA_F=\{ E\subset X\tq \mu^-(E)=\mu^+(E)<\infty \},
    \end{equation}
    and
    \begin{equation}
        \tribA=\{ E\subset X\tq E\cap K\in \tribA_F\,\text{\( \forall\) compact \( K\) in \( X\)} \},
    \end{equation}
    and finally, for \( E\in \tribA\) we define
    \begin{equation}
        \mu(E)=\mu^+(E).
    \end{equation}
    We have to prove that \( (X,\tribA,\mu)\) is a complete measured space which satisfies all the conditions.
    \begin{subproof}
    \spitem[\( \mu^o\) is increasing]     \label{ITEMooOMDSooQiYmaP}
        Let \( V_1\subset V_2\) be open sets. If \( f\prec V_2\), then \( f\prec V_1\), so that
        \begin{equation}
            \{ m(f)\tq f\prec V_1 \}\subset \{ m(f)\tq f\prec V_2 \}
        \end{equation}
        and we deduce that \( \mu^o(V_1)\leq \mu^o(V_2)\).
    \spitem[\( \mu^+\) is increasing]
        Let \( E\subset F\). Then \( \{ V\tq F\subset V \}\subset \{ V\tq E\subset V \}\). The infimum of the largest set is smaller, then \( \mu^+(E)\leq\mu^+(F)\).
    \spitem[\( \mu^-\) is increasing]
        Same argument as for \( \mu^+\), but with the inclusions inverted.
    \spitem[If \( V\) is open, \( \mu^o(V)=\mu^+(V)\)]        \label{ITEMooBRBTooAJBxGO}
        Let \( V\) be open and let us show \( \mu^o(V)=\mu^+(V)\). Per definition
        \begin{equation}
            \mu^+(V)=\inf\{ \mu^o(W)\tq \text{ \( U\subset W\) and \( W\) is open} \}.
        \end{equation}
        If \( W\) is open with \( V\subset W \) we have \( \mu^o(V)\leq \mu^o(W)\). Thus \( \mu^o(V)\) is a lower bound of the set whose infimum is \( \mu^+(V)\). Thus \( \mu^o(V)\leq \mu^+(V)\).

        Since \( V\subset V\), the number \( \mu^o(V)\) is part of the set \( \{ \mu^o(W)\tq \text{ \( V\subset W\) and \( W\) is open} \}\). Thus \( \mu^+(V)\leq \mu^o(V)\).
    \spitem[If \( K\) is compact, \( \mu^+(K)<\infty\)]       \label{ITEMooQVOGooQvTTzw}
        From lemma \ref{LEMooAXESooYvyesg}, we consider an open set \( V\) such that \( K\subset V\) and \( \bar V\) is compact. Urysohn's lemma provides a function \( g\) such that \( \bar V\prec g\). If \( f\in C^0_c(X,\eR)\) satisfies \( f\prec V\), we have \( f\leq g\). Indeed if \( x\in\bar V\), \( g(x)=1\geq f(x)\) and if \( x\notin \bar V\), then \( f(x)=0\leq g(x)\).

        We have
        \begin{equation}
            \mu^+(V)=\mu^o(V)=\sup\{ m(f)\tq f\prec V \}.
        \end{equation}
        And since \( K\subset V\) we have
        \begin{equation}
            \mu^+(K)\leq \mu^+(V)=\sup\{ m(f)\tq f\prec V \}\leq m(g)<\infty.
        \end{equation}
    \spitem[\( \mu^+(K)=\inf\{ m(f):\, K\prec f \}\)]
        Let \( \epsilon>0\). The definition is \( \mu^+(K)=\inf\{ \mu^o(V)\tq K\subset V \}\). There exists an open set \( V\) such that \( \mu^+(K)>\mu^o(V)-\epsilon\) and \( K\subset V\). From Urysohn's lemma we consider a function \( f\in C^0_c(X)\) such that \( K\prec f\prec V\). In particular \( \mu^o(V)\geq m(f)\). Putting the two inequalities together,
        \begin{equation}
            m(f)\leq \mu^o(V)<\mu^+(K)+\epsilon.
        \end{equation}
        We proved that for every \( \epsilon>0\), there exists \( f\) such that \( K\prec f\) and \( m(f)<\mu^+(K)+\epsilon\). This means that
        \begin{equation}
            \mu^+(K)\geq\inf\{ m(f)\tq K\prec f \}.
        \end{equation}
        Now we prove the reverse inequality.

        We want to prove that
        \begin{equation}
            \inf\{ \mu^o(V): K\subset V \}\leq \inf\{ m(f):K\prec f \}.
        \end{equation}
        Let \( \epsilon>0\) and \( f\in C^0_c(X)\) such that \( K\prec f\). We will build an open set \( V\) such that
        \begin{subequations}
            \begin{numcases}{}
                \mu^o(V)<m(f)+\epsilon\\
                K\subset V.
            \end{numcases}
        \end{subequations}
        Let \( 0<\alpha<1\) and \( V_{\alpha}=\{ x\tq f(x)>\alpha \}\). This is an open set satisfying \( K\subset V_{\alpha}\). We have
        \begin{equation}
            \mu^+(V_{\alpha})=\mu^o(V_{\alpha})=\sup\{ m(g): g\prec V_{\alpha} \}.
        \end{equation}
        Let \( g\prec V_{\alpha}\). For every \( x\in V_{\alpha}\) we have
        \begin{equation}
            \alpha g(x)\leq \alpha <f(x),
        \end{equation}
        and, if \( x\notin V_{\alpha}\) we have \( g(x)=0\). Thus \( \alpha g\leq f\) and
        \begin{equation}
            m(\alpha g)\leq m(f).
        \end{equation}
        That inequality is valid for every \( g\prec V_{\alpha}\), so that
        \begin{equation}
            \alpha \sup\{ m(g): g\prec V_{\alpha} \}\leq m(f).
        \end{equation}
        In other words,
        \begin{equation}
            \alpha \mu^o(V_{\alpha})\leq m(f)
        \end{equation}
        for every \( 0<\alpha<1\). For \( \alpha=1-\epsilon\) we have
        \begin{equation}
            \mu^o(V_{\alpha})\leq m(f)+\epsilon \mu^o(V_{\alpha}).
        \end{equation}
        Each \( V_{\alpha}\) is contained in the compact \( \supp(f)\), if \( s=\mu^+\big( \supp(f) \big)\), we have \( s<\infty\) and \( \mu^o(V_{\alpha})=\mu^+(V_{\alpha})<s\) for every \( \alpha\). Writing \( \epsilon'\) instead of \( \epsilon\), we have proved that for every \( \epsilon'\), we have
        \begin{equation}
            \mu^o(V_{1-\epsilon'})\leq m(f)+s\epsilon'.
        \end{equation}
        Now let \( \epsilon>0\). Taking \( \epsilon'\) as small as \( \epsilon' s<\epsilon\) we have
        \begin{equation}
            \mu^o(V_{1-\epsilon'})\leq m(f)+\epsilon's< m(f)+\epsilon.
        \end{equation}
        
        So for every \( \epsilon>0\), there exists an open \( V_{\alpha}\) such that
        \begin{subequations}
            \begin{numcases}{}
                \mu^o(V_{\alpha})< m(f)+\epsilon\\
                K\subset V_{\alpha}.
            \end{numcases}
        \end{subequations}
        This shows that
        \begin{equation}
            \inf\{ \mu^o(V):\text{ \( V\) is open and \( K\subset V\)} \}< m(f)+\epsilon,
        \end{equation}
        and than that
        \begin{equation}
            \mu^+(K)<m(f)+\epsilon.
        \end{equation}
        Since this is true for every \( \epsilon\), we deduce
        \begin{equation}
            \mu^+(K)\leq m(f),
        \end{equation}
        and then
        \begin{equation}
            \mu^+(K)\leq \inf\{ m(f): K\prec f \}.
        \end{equation}
    \spitem[\( K\in \tribA_F\)]       \label{ITEMooSMHRooIxgdeO}
        We already have \( \mu^+(K)<\infty\) (point \ref{ITEMooQVOGooQvTTzw}). The equality \( \mu^-(K)=\mu^+(K)\) is from the definition \eqref{EQooSIKUooJeeIqI} itself: the set \( K\) realises the supremum.
    \spitem[\( K\in \tribA\)]
        If \( E\) is compact, the set \( E\cap K\) is compact (proposition \ref{LEMooFJZDooSxYWVW}) and then belongs to \( \tribA_F\). Thus if \( E\) is compact, \( E\in \tribA\).
    \spitem[ \(\mu^+(V)= \mu^o(V)=\mu^-(V)\)]     \label{ITEMooEDOSooPwvyAO}
        Let \( V\) be open in \( X\). The equality \( \mu^+(V)=\mu^o(V)\) is the point \ref{ITEMooBRBTooAJBxGO}.

        Let \( 0<\alpha<\mu^o(V)\). From the definition of \( \mu^o(V)\), there exists \( f\prec V\) such that \( m(f)>\alpha\). We write \( K=\supp(f)\); since \( f\prec V\) we have \( K\subset V\).

        Let \( W\) be an open set with \( K\subset W\). We have \( \supp(f)=K\subset W\) and then \( f\prec W\), so that $m(f)\leq \mu^o(W)$. Here is a summary of the inequalities so far:
        \begin{equation}
            \alpha\leq m(f)\leq \mu^o(W).
        \end{equation}
        These inequalities hold for every open \( W\) such that \( K\subset W\). Thus we have
        \begin{equation}
                \alpha\leq \inf\{ \mu^o(W):\text{ \( W\) is open and \( K\subset W\)} \} =\mu^+(K) \leq \mu^+(V)= \mu^o(V).
        \end{equation}
        These inequalities hold for every \( \alpha<\mu^o(V)\). So for every \( \alpha<\mu^o(V)\), there exists a compact \( K\subset V\) such that \( \alpha\leq \mu^+(K)\leq \mu^o(V)\). We deduce that
        \begin{equation}
            \sup\{ \mu^+(K):\text{ \( K\) compact in \( V\)} \}\geq \mu^o(V).
        \end{equation}
        
        For the reverse inequality, if \( K\subset V\) we have \( \mu^+(K)\leq \mu^+(V)=\mu^o(V)\).
    \spitem[If \( \mu^o(X)<\infty\), then \( V\in \tribA_F\)]
        Let \( V\) be open in \( V\). From points \ref{ITEMooBRBTooAJBxGO} and \ref{ITEMooEDOSooPwvyAO} we have \( \mu^o(V)=\mu^+(V)=\mu^-(V)\). Since \( \mu^o\) is increasing (point \ref{ITEMooOMDSooQiYmaP}), we have
        \begin{equation}
            \mu^-(V)=\mu^+(V)=\mu^o(V)<\mu^o(X)<\infty,
        \end{equation}
        so \( V\in\mA_F\).
    \spitem[\( \mu^+(V_1\cup V_2)\leq \mu^+(V_1)+\mu^+(V_2)\)]        \label{ITEMooLPYWooONCYTi}
        Let \( V_1\) and \( V_2\) be open sets in \( X\). Since \( V_1\cup V_2\) is open, using point \ref{ITEMooBRBTooAJBxGO} we have
        \begin{equation}
            \mu^+(V_1\cup V_2)=\mu^o(V_1\cup V_2)=\sup\{ m(f):f\prec (V_1\cup V_2) \}. 
        \end{equation}
        Let \( \epsilon>0\). There exists \( g\in C^0_C(X)\) such that \( g\prec V_1\cup V_2\) and 
        \begin{equation}        \label{LEMooFYAAooDaOYUy}
            \mu^o(V_1\cup V_2)-\epsilon\leq m(g).
        \end{equation}
        Let \( K=\supp(g)\); we have \( K\subset V_1\cup V_2\).

        We consider \( V_1\cup V_2\) as topological vector space with the induced topology. Since \( V_1\cup V_2\) is open in \( X\), the induced topology makes no difficulties: the open parts of \( V_1\cup V_2\) are the same as the open parts of \( X\). The sets \( V_1\) and \( V_2\) themselve form an open covering of \( V_1\cup V_2\). We use a \( C_c^0(V_1\cup V_2)\)-partition of unity\footnote{Theorem \ref{THOooUGQCooFVySMP} and lemma \ref{LEMooDZGCooIGFXnA}.}, i.e. functions \( \phi_i\colon V_1\cup V_2\to \mathopen[ 0 , 1 \mathclose]\) such that \( \phi_i\in C^0_c(V_1\cup V_2)\) and \( \supp(\phi_i)\subset V_i\). 

        We consider the extensions
        \begin{equation}
            \begin{aligned}
                h_i\colon X&\to \mathopen[ 0 , 1 \mathclose] \\
                x&\mapsto \begin{cases}
                    \phi_i(x)    &   \text{is } x\in V_i\\
                    0    &    \text{otherwise. }
                \end{cases}
            \end{aligned}
        \end{equation}
        The functions \( h_i\) belong to \( C^0_c(X)\) (lemma \ref{LEMooTUQIooEyTLBa}). Here are their properties:
        \begin{enumerate}
            \item
                \( h_i\in C^0_c(X)\)
            \item
                \( h_i\prec V_i\)
            \item
                \( h_1+h_2=1\) on \( V_1\cup V_2\) and, in particular, on \( K\).
            \item
                \( h_ig\prec V_i\) because \( \supp(h_i)\subset V_i\)
            \item
                \( h_1g+h_2g=g\) on \( K \).
        \end{enumerate}
        We have
        \begin{equation}
            m(h_ig)\leq \sup\{ m(f)\tq f\prec V_i \}=\mu^o(V_i).
        \end{equation}
        Using the linearity of \( m\),
        \begin{equation}
            m(g)=m(h_1g)+m(h_2g)\leq \mu^o(V_1)+\mu^o(V_2)
        \end{equation}
        Combining with \eqref{LEMooFYAAooDaOYUy} we find
        \begin{equation}
            \mu^o(V_1\cup V_2)-\epsilon\leq m(g)\leq \mu^o(V_1)+\mu^o(V_2).
        \end{equation}
        Since it is true for every \( \epsilon>0\) we deduce
        \begin{equation}
            \mu^o(V_1\cup V_2)\leq \mu^o(V_1)+\mu^o(V_2).
        \end{equation}
        
    \spitem[\( \mu^+(\bigcup_i E_i)\leq \sum_i\mu^+(E_i)\)]       \label{ITEMooQZTEooJhWFna}
        Let \( E_i\) be parts of \( X\). Let \( \epsilon>0\). There exists an open set \( V_i\) such that \( E_i\subset V_i\) and
        \begin{equation}
            \mu^o(V_i)<\mu^+(E_i)+\frac{ \epsilon }{ 2^i }.
        \end{equation}
        Let \( E=\bigcup_iE_i\) and \( V=\bigcup_iV_i\). We have \( E\subset V\) while \( V\) is open as union of open parts. Thus \( \mu^+(E)\leq \mu^+(V)\). Let \( f\prec V\). We have \( \mu^+(V)=\mu^o(V)\leq m(f)\). Using point \ref{ITEMooLPYWooONCYTi}, and the geometric series\footnote{Proposition \ref{PROPooWOWQooWbzukS}\ref{ITEMooBJHBooBMEmiG}.}
        \begin{equation}
            \mu^o(V)\leq \sum_i\mu^o(V_i)\leq \sum_i\big( \mu^+(E_i)+\frac{ \epsilon }{ 2^i } \big)=\sum_i\mu^+(E_i)+\epsilon.
        \end{equation}
        Since this is valid for every \( \epsilon\), we deduce
        \begin{equation}
            \mu^+(E)\leq \mu^+(V)\leq \sum_i\mu^+(E_i).
        \end{equation}
    \spitem[\( \mu^+(A)+\mu^+(B)=\mu^+(A\cup B)\) when \( A,B\) are disjoint compacts] \label{ITEMooCVWSooCmPoPL}

        Let \( A,B\) be disjoint compacts. We know from point \ref{ITEMooSMHRooIxgdeO} that \( A,B\in \tribA\), so that \( \mu(A)=\mu^+(A)=\mu^-(A)\), and the same for \( B\). Point \ref{ITEMooQZTEooJhWFna} says that \( \mu(A\cup B)\leq \mu(A)+\mu(B)\). We have to prove the reverse inequality.

        Let \( \epsilon>0\). We consider an open set \( W\) such the \( K\subset W\) and 
        \begin{equation}
            \mu^+(W)=\mu(K)+\epsilon.
        \end{equation}
        The set \( A\) is closed\footnote{Lemma \ref{LemnAeACf}.} and \( W\) is open. Thus \( V=W\setminus A\) is open. Since \( A\) and \( B\) are disjoint, we also have \( B\setminus V\).

        The set \( V\) is open and contains the compact \( B\). By lemma \ref{LEMooKYMKooPxZjWN}, there exists an open set \( V_2\) such that
        \begin{equation}
            B\subset V_2\subset \bar V_2\subset V
        \end{equation}
        with \( \bar V_2\) being compact.

        Let \( V_1=W\setminus \bar V_2\). Since \(  \bar V_2\) is closed and \( W\) is open, \( V_1\) is open. Moreover we have \( V_1\cap V_2=\emptyset\). Since \( A\subset W\) and since \( \bar V_2\cap A=\emptyset\), \( A\subset V_1\). Let us summarise the important properties of \( V_1\) and \( V_2\) :
        \begin{itemize}
            \item \( A\subset V_1\)
            \item \( B\subset V_2\)
            \item \( V_1\cap V_2=\emptyset\).
        \end{itemize}
        Let finally
        \begin{subequations}
            \begin{align}
                V'_1=V_1\cap\big( W\setminus B \big)\\
                V'_2=V_2\cap\big( W\setminus A \big).
            \end{align}
        \end{subequations}
         These open sets have the same properties as \( V_1\) and \( V_2\)

         Let \( f_i\prec V'_i\) such that \( \mu^+(V'_i)<m(f_i)+\epsilon\). Since \( V'_1\cap V'_2=\emptyset\) we also have 
         \begin{equation}
            f_1+f_2\prec V_1\cup V_2\subset A\cup B\subset W.
         \end{equation}
         Now we can make the small computation
         \begin{subequations}
             \begin{align}
                 \mu^+(A)+\mu^+(B)&\leq \mu^+(V'_1)+\mu^+(V'_2)\\
                 &\leq m(f_1)+m(f_2)+2\epsilon\\
                 &=m(f_1+f_2)+2\epsilon\\
                 &\leq \mu^+(W)+2\epsilon\\
                 &\leq \mu^+(K)+3\epsilon.
             \end{align}
         \end{subequations}
         These inequalities hold for every \( \epsilon>0\), then \( \mu^+(A)+\mu^+(B)\leq \mu^+(K)\).
     \spitem[\( \mu^+(V\cup K)=\mu^+(V)+\mu^+(K)\) if \( V\cap K=\emptyset\)]         \label{ITEMooZSUIooPrPhtD}

         Let \( V\) be open and \( K\) be compact such that \( V\cap K=\emptyset\). If \( L\) is compact in \( V\), we have \( L\cap K=\emptyset \) and then \( \mu^+(K\cup L)=\mu^+(K+\mu^+(L)\). Using point \ref{ITEMooCVWSooCmPoPL}, we have:
         \begin{subequations}
             \begin{align}
                 \mu^+(V\cup K)&=\sup\{ \mu^+(M)\tq \text{  \( M\) is compact and \( M\subset V\cup K\) } \}\\
                 &\geq \sup\{ \mu^+(L\cup K)\tq \text{ \( L\) is compact and \( L\subset V\) } \}\\
                 &=\sup\{ \mu^+(K)+\mu^+(L)  \tq \text{ \( L\) is compact and \( L\subset V\) } \}\\
                 &= \mu^+(K)+ \sup\{\mu^+(L)  \tq \text{ \( L\) is compact and \( L\subset V\) } \}\\
                 &=\mu^+(K)+\mu^+(V).
             \end{align}
         \end{subequations}
         We proved that \( \mu^+(V\cup K)\geq \mu^+(K)+\mu^+(V)\). The reverse inequality is from point \ref{ITEMooQZTEooJhWFna}.
     \spitem[\( \mu^+(A\cap V)+\mu^+(A\setminus V)\leq \mu^+(A)\), \( A\) open]
        Let \( A,V\) be open. We consider \( K\subset A\cap V\) and we define \( W=A\setminus K\). The set \( W\) is open and we have the inclusions
        \begin{equation}
            A\setminus V\subset A\setminus K=W.
        \end{equation}
        Thus we have the following computation:
        \begin{subequations}        \label{SUBEQSooOSJAooBgpKec}
            \begin{align}
                \mu^+(K)+\mu^+(A\setminus V)&\leq \mu^+(K)+\mu^+(A\setminus K)\\
                &=\mu^+\big( K\cup (A\setminus K) \big)     \label{SUBEQooUEFQooMbLiTC}\\
                &=\mu^+(A).
            \end{align}
        \end{subequations}
        Justifications:
        \begin{itemize}
            \item For \eqref{SUBEQooUEFQooMbLiTC}: point \ref{ITEMooZSUIooPrPhtD}.
        \end{itemize}
        Taking the supremum over compacts \( K \) in \( A\cap V\) we have
        \begin{subequations}
            \begin{align}
                \mu^o(A\cap V)+\mu^+(A\setminus V)&=\mu^-(A\cap V)+\mu^+(A\setminus V)      \label{SUBEQooHHXTooAcuBNl}\\
                &=\sup\{ \mu^+(K)\tq \text{\( K\) is compact and \( K\subset A\cap V\)} \}\\
                &\leq \mu^+(A).     \label{SUBEQooRLOHooVCPVeG}
            \end{align}
        \end{subequations}
        Justifications:
        \begin{itemize}
            \item For \eqref{SUBEQooHHXTooAcuBNl}: point \ref{ITEMooEDOSooPwvyAO}
            \item For \eqref{SUBEQooRLOHooVCPVeG}: inequality \eqref{SUBEQSooOSJAooBgpKec}
        \end{itemize}
        <++>
        
     \spitem[\( \mu^+(E\cap V)+\mu^+(E\setminus V)\leq \mu^+(E)\), \( E\) arbitrary]
    \end{subproof}
\end{proof}



\begin{theorem} \label{ThoBZBooOTxqcI}
    Let \( G\) be a compact topological group accepting a countable basis of topology.
    \begin{enumerate}
        \item
            \( G\) accepts a unique normalized Haar measure.
        \item
            The Haar measure is unimodular:
	\begin{equation}
		\mu(Ag)=\mu(gA)=\mu(A)
	\end{equation}
    for every measurable subset \( A\subset G\) and every \( g\in G\).
    \end{enumerate}
\end{theorem}
\index{Haar measure}

%TODOooUCMXooQQNzcu
% Faire une réponse complète à
% https://math.stackexchange.com/questions/3318978/hermitian-representations-of-su2

%TODOooVHRGooIpYykY
% et aussi à
% https://math.stackexchange.com/questions/4419760/measure-from-a-linear-functional-riesz
