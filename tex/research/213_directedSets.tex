% This is part of Giulietta
% Copyright (c) 2016, 2018, 2021-2022
%   Laurent Claessens
% See the file fdl-1.3.txt for copying conditions.

%++++++++++++++++++++++++++++++++++++++++++++++++++++++++++++++++++++++++++++++++++++++++++++++++++++++++++++++++++++++++++++
\section{Directed sets and net}

\begin{definition}
    A \defe{directed set}{directed set} is a pre-ordered set (i.e. a set with a reflexive and transitive binary relation) such that every pair of elements has an upper bound. 
\end{definition}
As a consequence, when $a_1,\ldots a_n$ are elements of the directed set $A$, then there exists a $a\in A$ such that $a\geq a_i$.

\begin{definition}
    A \defe{net}{net} is a map $A\to X$ from a directed set to a topological space. We denote by $x_{\alpha}$ the element of $X$ which corresponds to $\alpha\in A$.
\end{definition}

As example, if $S$ is any set, the set $A$ of finite subsets of $S$ with the inclusion is an example of net.

There is a notion of \defe{convergence}{convergence!of a net} of net. We say that the net $\alpha\mapsto x_\alpha$ converges to $x$ and we write $x_{\alpha}\to x$ if and only if for every open set $\mU\subseteq X$ containing $x$, there exists a $\alpha\in A$ such that $\alpha'\geq \alpha$ implies $x_{\alpha'}\in\mU$.

So a topology implies a convergence notion for nets, as well as for sequences. However, there exists different topologies which have the same notion of convergence of sequences, but two topologies having the same notion of convergence of nets are the same.


%+++++++++++++++++++++++++++++++++++++++++++++++++++++++++++++++++++++++++++++++++++++++++++++++++++++++++++++++++++++++++++
\section{Homotopy group}

Let $X$ be a topological space with a base point $b$, and $S^n$ be the $n$-sphere. The $n$th \defe{group of homotopy}{group!homotopy}\index{homotopy group} on the point $b$ of $X$ is\nomenclature{$\pi_n(X,b)$}{Homotopy group of $X$ on the base point $b$}
\begin{equation}
    \pi_n(X,b)=\{ \text{homotopy classes of maps }f\colon S^n\to X \text{ such that } f(a)=b \}.
\end{equation}
The classes are taken up to homotopy, i.e. continuous deformations. In an equivalent way, $\pi_n(X,b)$ can be seen as the set of classes of maps $p\colon [0,1]^n\to X$ which sent the whole border of the cube to $b$.

%+++++++++++++++++++++++++++++++++++++++++++++++++++++++++++++++++++++++++++++++++++++++++++++++++++++++++++++++++++++++++++
\section{Covering spaces}
%+++++++++++++++++++++++++++++++++++++++++++++++++++++++++++++++++++++++++++++++++++++++++++++++++++++++++++++++++++++++++++

\begin{definition}      \label{DEFooQBDWooVVrkkh}
    Let \( M\) be an Hausdorff connected topological space. Let \( X\) be a topological space. A map \( \pi\colon X\to M\) is a \defe{covering}{covering map} if
    \begin{enumerate}
        \item
            it is continuous
        \item
            it is surjective
        \item
            for every \( p\in M\), there exists a neighbourhood \( V\) of \( p\) such that \( \pi^{-1}(V)\) is a union of disjoint open sets \( \{ A_i \}\) in \( X\) such that the restriction \( \pi\colon A_i\to V\) is homeomorphic\footnote{Definition \ref{DEFooYPGQooMAObTO}.}.
    \end{enumerate}
\end{definition}

\begin{proposition}[lifting property]\index{lifting property!covering space}
Let $\rho\colon C\to X$ be a covering and $\gamma\colon [0,1]\to X$, a continuous map. Let $c\in \rho^{-1}\big( \gamma(0) \big)$. Then there exists one unique path $\sigma$ in $C$ such that $\sigma\circ\rho=\gamma$ and $\sigma(0)=0$.
\end{proposition}
\begin{proof}
No proof.
\end{proof}
If $x$ and $y$ in $X$ are connected by a path, the lifted path provides a bijection between the fibres $\rho^{-1}(x)$ and $\rho^{-1}(y)$.

\begin{proposition}     \label{PROPooPWIFooFAZhVe}
    If \( M\) is a simply connected manifold, then every covering is an homeomorphism.
\end{proposition}

%---------------------------------------------------------------------------------------------------------------------------
\subsection{Universal covering}
%---------------------------------------------------------------------------------------------------------------------------

\begin{definition}
    One says that a covering $q\colon D\to X$ is \defe{universal}{universal!covering}\index{covering!universal} if $D$ is simply connected. 
\end{definition}

The following proposition states that an universal covering is a covering that covers all other coverings.

\begin{proposition}
Let $q\colon D\to X$ be an universal covering, and $\rho\colon C\to X$ be a covering of $X$ with $C$ being connected. Then there exists a covering map $f\colon D\to C$ such that $\rho\circ f=q$.
\end{proposition}

The following proposition states that the universal covering is essentially unique.

\begin{proposition}
Let $q_i\colon D_i\to X$ (with $i=1,2$) be two universal coverings of the topological space $X$. Then there exists an homeomorphism $f\colon D_1\to D_2$ such that $q_2\circ f=q_1$.
\end{proposition}

%---------------------------------------------------------------------------------------------------------------------------
					\subsection{Monodromy action}
%---------------------------------------------------------------------------------------------------------------------------
\label{sssMonodromyact}

Let $\rho\colon C\to X$ be a covering with $C$ being connected and locally arc connected. First, that shows that $X$ has these two properties too. Now, let $x\in X$ and $c\in\rho^{-1}(x)$ and a path $\gamma\colon [0,1]\to X$ with $\gamma(0)=\gamma(1)=x$. By the lifting property, that path lifts to an unique path in $C$ starting at $c$, while it is not guarantee that the lifted path will \emph{end} at $x$. One only knows that the lifted path will end in $\rho^{-1}(x)$.

It turns out that the end point of the lifted path only depends on the class of $\gamma$ in $\pi(X,x)$. Thus we define an action if $\pi(X,x)$ on the fibre over $x$. This is the \defe{monodromy action}{monodromy action}. Notice that by taking that action pointwise on $X$, the group $\pi(X)$ acts on $C$.

%+++++++++++++++++++++++++++++++++++++++++++++++++++++++++++++++++++++++++++++++++++++++++++++++++++++++++++++++++++++++++++ 
\section{Haar measure}
%+++++++++++++++++++++++++++++++++++++++++++++++++++++++++++++++++++++++++++++++++++++++++++++++++++++++++++++++++++++++++++

\begin{theorem}[Kakutani fixed point\cite{BIBooNYIOooTIiwbz,BIBooAWWDooUzosWY}]     \label{THOooWXQFooQrWcLY}
    Let \( X\) be an Hausdorff locally convex\footnote{Locally convex, definition \ref{DEFooCGJBooSvDpyC}.} topological vector space. Let \( K\) be compact in \( X\) and \( \mS\) be a group of linear endomorphisms of \( X\) such that
    \begin{enumerate}
        \item
            \( \mS\) is equicontinuous\footnote{Definition \ref{DEFooDHQDooFfIvsX}.} on \( K\),
        \item
            \( \mS(K)\subset K\).
    \end{enumerate}
    There exists a point \( p\in K\) such that \( T(p)=p\) for every \( T\in \mS\).
\end{theorem}

\begin{proof}
    We'll use the Zorn lemma. Let 
    \begin{equation}
        \mA=\big\{  L\subset K\tq L \text{ is non empty, compact, convex and} \mS(L)\subset L   \big\}.
    \end{equation}
    The set \( \mA\) is ordered with \( L_1\leq L_2\) if and only if \( L_2\subset L_1\) (this is the opposite of what we usually do).

    \begin{subproof}
    \item[\( (\mA, \leq)\) is inductive]
        Let \( \mF\subset \mA\) be a totally ordered subset. We prove that 
        \begin{equation}
            M=\bigcap_{A\in \mF}A
        \end{equation}
        is a upper bound of \( \mF\).
        \begin{subproof}
        \item[Finite intersection property]
            We prove that the family \( \mF\) has the finite intersection property\footnote{Definition \ref{DEFooCESGooZkACqs}.}. Let \( \mI\) be finite in \( \mF\). The set \( (\mI,\leq)\) is totally ordered and the lemma \ref{LEMooPCRFooXRGrUr} says that \( \mI\) has a maximum. Here the order is the inverse inclusion, so that the maximum is in fact the smaller set, which is contained in all the others. In other words, the maximum, which is an element of \( \mI\) is the intersection \( \bigcap_{A\in\mI}A\).
        \item[\( M\) is non empty]
            The family \( \mF\) is a family of closed\footnote{Compacts are closed, see \ref{LemnAeACf}.} parts of \( X\) which has the finite intersection property. Thus the intersection of \( \mF\) is non empty by theorem \ref{THOooCQSQooDuasqo}.
            \item[\( M\) is compact] 
            Since \( X\) is Hausdorff, every intersection of compacts parts is compact by the proposition \ref{PROPooQWHSooXeJOkT}.
        \item[\( M\) is convex]
            Proposition \ref{PROPooJOCEooUKhkqQ} says that every intersection of convex is convex.
        \item[\( \mS(M)\subset\mS(M)\)]
            Let \( x\in M\). For every \( A\in \mF\), we have \( x\in A\). Since \( \mS(A)\subset A\) we also have \( \mS(x)\in A\) and \( \mS(x)\in \bigcap_{A\in\mF}A\).
        \end{subproof}
    \item[Zorn, definition of \( K_1\)] 
        Since \( \mA\) is inductive we use the Zorn lemma \ref{LemUEGjJBc} and we let \( K_1\) be a maximal element of \( \mA\).
    \item[If \( K_1\) has only one point]
        Let us suppose that \( K_1\) contains only one element \( p\). We prove that \( T(p)=p\) for every \( T\in\mS\). Since \( K_1\in \mA\) we have \( T(K_1)\subset K_1\) and since \( K_1=\{ p \}\) we have \( T(p)\in \{ p \}\) which means \( T(p)=p\).
    \end{subproof}
    We suppose that \( K_1\) has at least \( 2\) points and we will reach a contradiction. As far as the notations are concerned, we write \( K_1-K_1=\{ k_1-k_2\tq k_1,k_2\in K_1 \}\).
    \begin{subproof}
    \item[\( K_1-K_1\) has a non zero element]
        The part \( K_1\) contains at least two elements. Let \( k_1\) and \( k_2\) be two distinct elements in \( K_1\). Then \( k_1-k_2\neq 0\) in \( K_1-K_1\).
    \item[Definition of \( V\)]
        From lemma \ref{LEMooMDTNooThlHJl}, we consider in \( X\) a neighbourhood \( V\) of \( 0\) such that \( \bar V\) is not contained in \( K_1-K_1\).
    \item[Definition of \( V_1\)]
        Since \( X\) is locally convex, \( 0\) has a basis of convex neighbourhoods. Let \( V_1\subset V\) be a convex neighbourhood of \( 0\).
    \item[Definition of \( U_1\)]
        The set \( \mS\) is equicontinuous on \( K\) and then equicontinuous on \( K_1\). There exists a neighbourhood \( U_1\) of \( 0\) in \( X\) such that for every \( k_1,k_2\in K_1\) such that \( k_2-k_1\in U_1\), we have \( \mS(k_1-k_2)\subset  V_1\). In other words,
        \begin{equation}        \label{EQooPELQooJgcrzr}
            \mS\big( U_1\cap(K_1-K_1) \big)\subset V_1.
        \end{equation}
    \item[Open map]
        Since \( \mS\) is a group, each of the linear maps \( T\in\mS\) is invertible and its inverse is part of \( \mS\). In other words, the elements of \( \mS\) are continuous linear maps with continuous inverse.

        In particular, if \( \mO\) is open in \( X\), then \( T(\mO)\) is open in \( X\).
    \item[Définition of \( U_2\)]
        We define
        \begin{equation}       
            U_2=\Conv\Big( \mS\big(  U_1\cap(K_1-K_1)  \big) \Big).
        \end{equation}
        Since the elements \( T\in \mS\) are linear and inversible, we have \( T(A\cap B)=T(A)\cap T(B)\) and \( T\big( \Conv(A) \big)=\Conv\big( T(A) \big)\) for every parts \( A,B\subset X\). Thus we have
        \begin{equation}
            \mS(U_2)=U_2.
        \end{equation}
    \item[\( K_1-K_1\subsetneq U_2\)]
        The set \( U_2\) is the convex hull of \( \mS\big( U_1\cap (K_1-K_1) \big)\) which is a part of the convex set \( V_1\) by \eqref{EQooPELQooJgcrzr}. Thus \( U_2\subset V_1\). We have the inclusions chain
        \begin{equation}
            U_2\subset V_1\subset V.
        \end{equation}
        Since \( K_1-K_2\) is not included in \( V\), it is not included in \( U_2\).
    \item[Definition of \( U\)]
        Since the operators \( T\) are continuous and since \( T(U_2)\subset U_2\), we also have \( \mS(\bar U_2)\subset \bar U_2\). The part \( U_2 \) is an open convex containing \( 0\) and \( K_1-K_2\) is compact. Thus lemma \ref{LEMooDVZWooWKRQWC} implies the existence of \( \lambda>0\) such that \( K_1-K_2 \subset \lambda U_2\). Let
        \begin{equation}
            \delta=\inf\{ \lambda>0\tq K_1-K_1\subset \lambda U_2 \}.
        \end{equation}
        Notice that \( \delta<\infty\) because of lemma \ref{LEMooDVZWooWKRQWC} and \( \delta >1\) because \( K_1-K_1\nsubset U_2\) . We define
        \begin{equation}
            U=\delta U_2.
        \end{equation}
    \item[Some sums]
        Let \(   0<\epsilon<1   \). For every \( \lambda>\delta\) we have \( K_1-K_1\subset \lambda U_2\). In particular with \( \lambda= (1+\epsilon)\delta\). Thus
        \begin{equation}
            K_1-K_1\subset (1+\epsilon)\delta U_2=(1+\epsilon)U.
        \end{equation}
        In the same way, for every \( \lambda<\delta\) we have \( K_1-K_1\nsubset \lambda U_2\). Thus
        \begin{equation}        \label{EQooNYNTooQScVJL}
            K_1-K_2\nsubset (1-\epsilon)\bar U.
        \end{equation}
        
        For \( k\in K_1\) we consider \( \mO_k=\frac{ 1 }{2}U+k\). Since \( 0\in U\) the set \( \{ \mO_k \}_{k\in K_1}\) is an open covering of \( K_1\). The set \( K_1\) being compact we can extract a finite subcovering: there exists \( k_1,\ldots, k_n\in K_1\) such that
        \begin{equation}
            K_1\subset \bigcup_{i=1}^n(\frac{ 1 }{2}U+k_i).
        \end{equation}
        Let
        \begin{equation}
            p=\frac{ k_1+\ldots +k_n }{ n }.
        \end{equation}
        Let \( k\in K_1\). Two statements:
        \begin{enumerate}
            \item
                For each \( l\) we have \( k_l-k\in (1+\epsilon)U\).
            \item
                There exists \( i\) such that \( k_i-k\in \frac{ 1 }{2}U\).
        \end{enumerate}
        Thus there exist \( u_1,\ldots, u_n\in U\) such that
        \begin{subequations}        \label{EQSooZFRQooJoFhXF}
            \begin{numcases}{}
                k_i-k=\frac{ 1 }{2}u_i\\
                k_l-k=(1+\epsilon)u_l.
            \end{numcases}
        \end{subequations}
        Inserting the values of \( k_i\) from \eqref{EQSooZFRQooJoFhXF} in the definition of \( p\),
        \begin{subequations}
            \begin{align}
                p&=\frac{1}{ n }\sum_{j=1}^nk_j\\
                &=\frac{1}{ n }\Big( \frac{ 1 }{2}u_i+k+\sum_{j\neq i}\big[ (1+\epsilon)u_l+k \big] \Big)\\
                &=\frac{1}{ n }\Big( \frac{ 1 }{2}u_i+\sum_{j\neq i}(1+\epsilon)u_l \Big)
            \end{align}
        \end{subequations}
        The set \( U\) is convexe because \( U=\delta U_2\) where \( U_2\) is convex. Thus if \( u_1, u_2\in U\) we have \( \frac{ u_1+u_2 }{2}\in U\) and \( u_1+u_2\in 2U\). In the same way, there exists \( u'\in U\) such that
        \begin{equation}
            \sum_{j\neq i}u_l=(n-1)u'.
        \end{equation}
        So we write
        \begin{equation}
            p=\frac{1}{ n }\big( \frac{ 1 }{2}u_i +(n-1)(1+\epsilon)u' \big)+k.
        \end{equation}
        Using lemma \ref{LEMooAHUMooBwxzPj}, there exists \( u\in U\) such that
        \begin{equation}
            p=\frac{1}{ n }\big( \frac{ 1 }{2}+(n-1)(1+\epsilon) \big)u+k.
        \end{equation}
        We write that with \( \epsilon=\frac{1}{ 4(n-1) }\). The coefficient becomes
        \begin{equation}
            \frac{ 1 }{2}+(n-1)(1+\epsilon)=\frac{ 4n-1 }{ 4 }
        \end{equation}
        and we have proven that, for each \( k\in K_1\), there exists \( u\in U\) such that
        \begin{equation}
            p=\frac{ 4n-1 }{ 4n }u+k.
        \end{equation}
    \item[Definition of \( K_2\)]
        Let, for each \( k\in K_1\),
        \begin{equation}
            A_k=\frac{ 4n-1 }{ 4n }\bar U+k
        \end{equation}
        and
        \begin{equation}
            K_2=K_1\cap\bigcap_{k\in K_1}A_k.
        \end{equation}
        We give some properties of \( K_2\).
        \begin{subproof}
    \item[\( K_2\) is non empty]
        We know that \( p\in K_2\).
    \item[\( K_2\subsetneq K_1\)]
        From what we said around equation \eqref{EQooNYNTooQScVJL}, there exists \( k_1,k_2\in K_1\) such that
        \begin{equation}
            k_1-k_2\notin\left( 1-\frac{ 4n-1 }{ 4n } \right)\bar U,
        \end{equation}
        which means that \( k_1\notin A_{k_2}\). In particular \( k_1\notin K_2\). Thus the inclusion \( K_2\subset K_1\) is strict.
    \item[\( K_2\) is compact and convex]
        The sets \( K_1\) and \( A_k\) are closed. The set \( K_2\) is closed as intersection of closed sets. Since \( K_2\) is closed in the compact \( K_1\), it is compact (lemma \ref{LemnAeACf}\ref{ITEMooNKAKooQoNddr}).

        The set \( K_2\) is convex as intersection of convexes.
    \item[\( \mS(K_2)\subset K_2\)]
        Let \( T\in \mS\) and \( u\in\bar U\). We have \( u=\delta u_2\) with \( u_2\in\bar U_2\). Then
        \begin{equation}
            T(u)=\delta T(u_2)\in \delta T(\bar U_2)\subset \delta\bar U_2=\bar U
        \end{equation}
        because \( T(U_2)\subset U_2\) and \( T\) is continuous. We proved that \( T(\bar U)\subset \bar U\).

        Let \( a\in K_2\). We have \( a\in K_1\) and, since \( a\in A_k\) for each \( k\in K_1\), we have a map \( u\colon K_1\to \bar U\) such that
        \begin{equation}        \label{EQooFDOZooNTThIj}
            \lambda u(k)+k=a
        \end{equation}
        where \( \lambda=\frac{ 4n-1 }{ 4n }\). 
        Consider \( l\in K_1\) and let us show that \( T(a)\in A_{l}\). Since \( T(K_1)=K_1\), there exists \( l'\in K_1\) such that \( T(l')=l\). Writing \eqref{EQooFDOZooNTThIj} with \( l'\) we have
        \begin{equation}
            T(a)=\lambda T\big( u(l') \big)+T(l')=\lambda T\big( u(l') \big)+l.
        \end{equation}
        Since \( u(l')\in \bar U\) and \( T(\bar U)\subset \bar U\), we have
        \begin{equation}
            T(a)\in \lambda\bar U+l=A_l.
        \end{equation}
        We have proved that \( T(a)\in A_l\) for each \( l\in K_1\).
        \end{subproof}
    \item[Conclusion]
        The part \( K_2\) belong to \( \mA\) and satisfies the strict inclusion \( K_2\subsetneq K_1\). This means that \( K_2>K_1\), which contradicts the maximality of \( K_1\).

        We conclude that \( K_1\) has only one point, so that \( \mS\) has a unique fixed point as shown before.
    \end{subproof}
\end{proof}

We recall the regular left action. If \( G\) is a group, \( u\) is a map \( u\colon G\to E\) and \( s\in G\), we write
\begin{equation}
    (L_su)(x)=u(s^{-1}x).
\end{equation}

\begin{theorem}[Haar measure\cite{BIBooNYIOooTIiwbz}]
    Let \( G\) be a compact topological group\footnote{Definition \ref{DEFooCHZVooHnvSgW}.}. There exists a unique linear form \( m\colon C^0(G,\eC)\to \eC\) such that
    \begin{enumerate}
        \item
            \( m(f)\geq 0\) for every \( f\geq 0\).
        \item
            \( m(1)=1\),
        \item
            \( m(L_sf)=m(f)\) for every \( f\in C^0(G,\eC)\) and \( s\in G\).
        \item
            \( m(R_sf)=f(f)\).
    \end{enumerate}
\end{theorem}

\begin{proof}
    For \( f\in C^0(G,\eC)\) we consider \( C_f\), the set of convex combinations of the functions of the form \( L_s(f)\). In other words, \( g\in C_f\) is there exists \( \{ (a_i,s_i)\in \eR^+\times G \}_{i=1,\ldots, N}\) such that \( a_i>0\), \( \sum_ia_i=1\) and \( g(x)=\sum_ia_if(s_ix)\).

    Since \( f\) is continuous in the compact space \( G\), we define
    \begin{equation}
        \| f \|=\max_{x\in G}| f(x) |.
    \end{equation}
    If \( g\in C_f\) we have \( \| g \|\leq \| f \|\) because
    \begin{subequations}
        \begin{align}
            \| g \|&=\max_{x\in G}| \sum_ia_if(s_ix) |\\
            &\leq\sum_i\max_{x\in G}| a_if(s_ix) |\\
            &\leq \sum_ia_i\max_{x\in G}| f(s_ix) |\\
            &=\sum_ia_i\| f \|\\
            &=\| f \|.
        \end{align}
    \end{subequations}
    For \( x\in G\) we also define
    \begin{equation}
        C_f(x)=\{ g(x)\tq g\in C_f \}.
    \end{equation}
    This is a bounded set in \( \eC\) because each element is bounded by \( \| f \|\). The closure \( \overline{ C_f(x) }\) is then compact in \( \eC\). The function \( f\) being continuous, it is uniformly continuous on \( \overline{ C_f(x) }\) (proposition \ref{PROPooSHBAooVRdAFM}.). Let \( \epsilon>0\) and an open neighbourhood \( V\) of \( e\) as in the definition of uniform continuity. If \( x,y\in V\) we have
    \begin{equation}
        | (L_sf)(x)-(L_sf)(y) |=| f(s^{-1}x)-f(s^{-1} y) |<\epsilon
    \end{equation}
    because \( (s^{-1}y)^{-1}(s^{-1}x)=y^{-1}x\).

    This shows that the set \( \{ L_s(f) \}_{s\in G}\) is equicontinuous\footnote{Definition \ref{DEFooDHQDooFfIvsX}.}. Now we prove that \( C_f\) is equicontinuous. Let \( g\in C_f\) and \( x,y\in V\). We have
    \begin{subequations}
        \begin{align}
            | g(x)-g(y) |&=\Big|   \sum_ia_i\big[ f(s_ix)-f(x_iy) \big]   \Big|\\
            &\leq \sum_ia_i| s_{s_i^{-1}}(x)-L_{s_i^{-1}}(y) |\\
            &\leq \sum_ia_i\epsilon\\
            &=\epsilon,
        \end{align}
    \end{subequations}
    so \( C_f\) is equicontinuous. The Ascoli theorem \ref{ThoKRbtpah} implies that \( C_f\) is relatively compact in \( C^0(G,\eC)\), which means that \( K_f=\overline{ C_f }\) is compact.

    The action
    \begin{equation}
        \begin{aligned}
            L\colon G\times C^0(G,\eC)&\to C^0(G,\eC) \\
            (g,u)&\mapsto L_g(u) 
        \end{aligned}
    \end{equation}
    is isometric: \( \| L_s(u) \|=\| u \|\) and leaves \( C_f\) invariant.
    \begin{subproof}
    \item[\( L_s(C_f)\subset C_f\)]
        We write
        \begin{equation}
            \begin{aligned}
                \psi\colon \eR^N\times G^N&\to C_f \\
                (a_i), (s_i)&\mapsto \sum_ia_iL_{s_i}(f). 
            \end{aligned}
        \end{equation}
        If \( g\in C_f\) we have
        \begin{equation}
                g(x)=\psi\big( (a_i), (s_i) \big)(x)=\sum_ia_if(s_ix),
        \end{equation}
        so that
        \begin{subequations}
            \begin{align}
                (L_sg)(x) &=\sum_ia_if(s_is^{-1}x)\\
                &=\psi\big( (a_i),(s_is^{-1}) \big)(x).
            \end{align}
        \end{subequations}
        This shows that \( L_s(g)\in C_f\).
    \item[\(C_f\subset L_s( C_f)\)]
        The same kind of computation shows that
        \begin{equation}
            \psi\big( (a_i),(s_i) \big)=L_s\Big( \psi\big( (a_i), (s_is) \big) \Big).
        \end{equation}
    \end{subproof}
    
\end{proof}
<++>

\begin{definition}      \label{DEFooXIKEooWOxHlr}
    Let \( G\) be a topological group. A \defe{Haar measure}{Haar measure} on \( G\) is a positive measure\footnote{Definition \ref{DefBTsgznn}.} \( \mu\) such that
	\begin{enumerate}
		\item
		      \( \mu(gA)=\mu(A)\) for every \( g\in G\) and measurable part \( A\subset G\),
		\item
		      \( \mu(K)<\infty\) for every compact part \( K\subset G\).
	\end{enumerate}
    If the group \( G\) is itself compact, we requite the measure to be normalized : \( \mu(G)=1\).
\end{definition}

\begin{theorem} \label{ThoBZBooOTxqcI}
    Let \( G\) be a compact topological group accepting a countable basis of topology.
    \begin{enumerate}
        \item
            \( G\) accepts a unique normalized Haar measure.
        \item
            The Haar measure is unimodular:
	\begin{equation}
		\mu(Ag)=\mu(gA)=\mu(A)
	\end{equation}
    for every measurable subset \( A\subset G\) and every \( g\in G\).
    \end{enumerate}
\end{theorem}
\index{Haar measure}
