% This is part of Giulietta
% Copyright (c) 2020
%   Laurent Claessens
% See the file fdl-1.3.txt for copying conditions.

\section{Cyclic cohomology}
%+++++++++++++++++++++++++
\label{SecCyclicHomol}

Let $\cA$ be an algebra over $\eC$. We consider the complex $(C_{\lambda}^n,b)$ defined by
\begin{itemize}
	\item $C_{\lambda}^n$ is the set of $(n+1)$-linear functionals $\varphi$ on $\cA$ such that
	      \begin{equation}
		      \varphi(a^1,\cdots,a^{n},a^0)=(-1)^n\varphi(a^0,\cdots,a^{n}),
	      \end{equation}
	\item the \defe{Hochschild coboundary}{Hochschild!coboundary} $b$ is defined by
	      \begin{equation}
		      \begin{split}
			      (b\varphi)(a^0,\cdots,a^{n+1})=&\sum_{j=0}^{n}(-1)^j\varphi(a^0,\cdots,a^{j}a^{j+1},\cdots,a^{n+1})\\
			      &+(-1)^{n+1}\varphi(a^{n+1}a^0,\cdots,a^{n}).
		      \end{split}
	      \end{equation}
	      We denote by $C^0_{\lambda}$ the set of linear functionals.

\end{itemize}
We denote by $HC^*(\cA)$ the cohomology of this complex.

\begin{definition}		\label{DefCycleCoh}
	A $n$-dimensional \defe{cycle}{cycle} is a triple $(\Omega,d,\int)$ with
	\begin{itemize}
		\item $\Omega=\sum_{j=0}^{n}\Omega^j$ is a graded algebra over $\eC$,
		\item $d$ is  graduate derivative of degree $1$,
		\item $\int\colon \Omega^n\to \eC$ is a \defe{closed trace}{closed!trace}, that is $\int d\omega=0$.
	\end{itemize}
	When $\cA$ is an algebra over $\eC$, we say that a cycle \emph{over $\cA$} is a cycle endowed with a homomorphism $\rho\colon \cA\to \Omega^0$.
\end{definition}

An important subset of $C^n_{\lambda}$ is the set
\begin{equation}
	Z^n_{\lambda}(\cA)=\{ \varphi\tq b\varphi=0 \}.
\end{equation}
When $n=0$, the condition $b\varphi=0$ becomes $(b\varphi)(a^0,a^1)=\varphi(a^0a^1)-\varphi(a^1a^0)=0$, in such a way that the elements of $Z^0_{\lambda}$ are exactly the traces on $\cA$.

If $(\Omega,d,\int)$ is a cycle and $\rho\colon \cA\to \Omega^0$ a homomorphism, then the \defe{character}{character!of a homomorphism} of $\rho$ is the functional $\tau$ defined by
\begin{equation}
	\tau(a^0,\cdots,a^{n})=\int \rho(a^0)d\big( \rho(a^1) \big)\cdots d\big( \rho(a^{n}) \big).
\end{equation}


\subsection{Example: de Rham homology}
%--------------------------------------

Let $M$ be a differentiable manifold and $C$, a closed de Rham $q$-current on $M$ with $q\leq\dim M$. If $\omega$ is a form on $M$, we denote by $C(\omega)$ the evaluation of $C$ on $\omega$. We consider $\Omega^{i}= C^{\infty}\big( M,\Wedge^i T^*M \big)$, the space of differential forms of degree $i$. It provides a graded differential algebra $(\Omega,d)$ on which we can put the trace $\int\colon \Omega^q\to \eC$,
\[
	\int \omega=C(\omega).
\]

\subsection{Hochschild cohomology}
%---------------------------------

Let $\cA$ be an algebra over $\eC$ and $\cA_{\cun}=\cA\oplus\eC\cun$ be the algebra obtained by adduction of an unity to $\cA$. We define
\[
	\Omega^1(\cA)=\cA_{\cun}\times_{\eC}\cA
\]
and an $\cA$-bimodule structure by
\[
	x\Big( (a+\lambda\cun)\times b \Big)y:=(xa+\lambda x)\times by-(xa+\lambda xb)\times y
\]
with $a$, $b$, $x$, $y\in\cA$ and $\lambda\in\eC$. Then we consider
\begin{equation}
	\begin{aligned}
		d\colon \cA & \to \Omega^1(\cA) \\
		da          & = 1\otimes a
	\end{aligned}
\end{equation}
which can be checked to be a derivation.

\begin{proposition}
	Let $\modE$ be a $\cA$-bimodule and $\delta\colon \cA\to \modE$, a derivation. Then there exists a bimodule morphism $\rho\colon \Omega^1(\cA)\to \modE$ such that $\delta=\rho\circ d$.
\end{proposition}

\begin{proof}
	No proof.
\end{proof}

This proposition says that $\big( \Omega^1(\cA),d \big)$ is an universal derivation in a $\cA$-bimodule. From $\Omega^1(\cA)$, we define $\Omega^0(\cA)=\cA$.
\[
	\Omega^n(\cA)=\Omega^1(\cA)\otimes_{\cA}\cdots\otimes_{\cA}\Omega^1(\cA),
\]
and the differential extends to an unique graded derivation of $\Omega^*(\cA)$ such that $d^2=0$. Note the isomorphism
\begin{equation}
	\begin{aligned}
		j\colon \cA_{\cun}\otimes\cA^{\otimes n}               & \to\Omega^n(\cA)                                       \\
		(a^0+\lambda\cun)\otimes a^1\otimes\cdots\otimes a^{n} & \mapsto a^0da^1\cdots da^{n}+\lambda da^1\cdots da^{n}
	\end{aligned}
\end{equation}
for each $a^{j}\in\cA$ and $\lambda\in\eC$.

\begin{lemma}
	The cohomology of the complex $\big( \Omega^*(\cA),d \big)$ is zero in any dimension, including zero.
\end{lemma}
\begin{proof}
	No proof.
\end{proof}

A product in $\Omega^*(\cA)$ is defied as usual by juxtaposition and rearrangement using the fact that $d$ is a derivation, see for example the proof of proposition~\ref{prop_modMununique}:
\[
	\begin{split}
		(a^0da^1\cdots da^{n})(a^{n+1}da^{n+2}\cdots da^{n})&=\sum_{j=1}^{n}(-1)^{n-j}a^0da^1\cdots d(a^{j}a^{j+1})\cdots da^{n}da^{n+1}\cdots da^m\\
		&\quad+(-1)^na^0a^1 da^2\cdots da^{m}.
	\end{split}
\]
It is nothing else than define the product in such a way that $\Omega^*(\cA)$ is a right $\cA$-module and $d$ a derivation.


\begin{proposition}
	Let $\tau$ be a $(n+1)$-linear functional on $\cA$. The three following properties, where $a^0,\cdots a^n$ are some elements of $\cA$, are equivalent
	\begin{enumerate}
		\item There exists a cycle $(\Omega,d,\int)$ of dimension $n$ and a homomorphism $\rho\colon \cA\to \Omega^0$ such that
		      \[
			      \tau(a^0,\cdots,a^{n})=\int \rho(a^0)d\big( \rho(a^1) \big)d\big( \rho(a^{n}) \big)
		      \]
		      for all $a^0,\cdots,a^{n}\in\cA$. In other words, $\varphi$ is the character of a cycle.
		\item There exists a closed graded trace $\hat{\tau}$ of dimension $n$ on $\Omega^*(\cA)$ such that
		      \[
			      \tau(a^0,\cdots,a^{n})=\hat{\tau}(a^0da^1,\cdots,da^{n})
		      \]
		\item The functional $\tau$ satisfies $b\tau=0$ and $\tau\circ\gamma=\epsilon(\gamma)\tau$ where $\gamma$ is any cyclic permutation of $\{ 0,1,\cdots,n \}$ and $\epsilon(\gamma)$ is its parity, or more explicitly,
		      \[
			      \tau(a^1,\cdots,a^{n},a^0)=(-1)^n\tau(a^0,\cdots,a^{n})
		      \]
		      and
		      \[
			      \sum_{i=0}^{n}(-1)^n\tau(a^0,\cdots a^ia^{i+1},\cdots,a^{n+1})+(-1)^{n+1}\tau(a^{n+1}a^0,\cdots,a^{n})=0.
		      \]
		      The most compact way to express this condition is just $\tau\in Z^n_{\lambda}$.
	\end{enumerate}
\end{proposition}

\subsection{Hochschild groups of cohomology}
%--------------------------------------------

\begin{definition}
	Let $\cA^e=\cA\otimes\cA^0$, the tensor product of $\cA$ with its opposite algebra. Let $\modM$ be a module over $\cA$. The \defe{Hochschild cohomology}{Hochschild!cohomology} is defined as follows.

	\begin{itemize}
		\item
		      Let $C^n(\cA,\modM)$ is the set of $n$-linear maps from $\cA$ to $\modM$.
		\item A cochain is an element of $C^*(\cA,\modM)$ and the differential is given by
		      \begin{equation}
			      \begin{split}
				      (bT)(a^1,\cdots,a^{n+1})&=a^1T(a^2,\cdots,a^{n+1})+\sum_{i=1}^{n}(-1)^iT(a^1,\cdots,a^ia^{i+1},\cdots,a^{n+1})\\
				      &\quad +(-1)^{n+1}T(a^1,\cdots,a^{n})a^{n+1}.
			      \end{split}
		      \end{equation}
		\item
		      The Hochschild cohomology of $\cA$ with coefficients in $\modM$ is finally defined as the cohomology of the complex $\big( C^*(\cA,\modM),b \big)$
	\end{itemize}
\end{definition}

Let us study a particular case. The space $\cA^*$ of functionals on $\cA$ becomes a $\cA$-bimodule when one defines $(a\varphi b)(c)=\varphi(abc)$ for each $a$, $b$, $c\in\cA$.

Let $T\in C^n(\cA,\cA^*)$; this is a $n$-linear functional from $\cA$ to $\cA^*$ which can be seen as a $(n+1)$-linear function $\tau_T$ on $\cA$ by
\[
	\tau_T(a^0,a^1,\cdots,a^{n})=T(a^1,\cdots,a^{n})(a^0).
\]
If one defines $b$ acting on $\tau$ by
\begin{align}
	b(\tau)(a^0,\cdots,a^{n+1}) & =\sum_{i=0}^{n}(-1)^i\tau(a^0,\cdots,a^ia^{i+1},\cdots,a^{n/1}) \\
	                            & \quad+(-1)^{n+1}\tau(a^{n+1}a^0,\cdots,a^{n}),
\end{align}
we have $\tau_{bT}=b\tau_T$.

Let $A\colon C^n(\cA,\cA^*)\to C^n(\cA,\cA^*)$, the linear map defined by
\[
	(A\varphi)=\sum_{\gamma\in\Gamma}\epsilon(\gamma)(\varphi\circ\gamma)
\]
where $\Gamma$ is the group of cyclic permutations of $\{ 0,1,\cdots,n \}$, and $\epsilon(\gamma)$, the parity of $\gamma$. The image of $A$ is $C_{\lambda}^n(\cA)$.

\begin{lemma}
	If one defines $b'\colon C^n(\cA,\cA^*)\to C^{n+1}(\cA,\cA^*)$ by
	\begin{equation}
		(b'\varphi)(a^0,\cdots,a^{n+1})=\sum_{j=0}^{n}(-1)^j\varphi(a^0,\cdots,a^ja^{j+1},\cdots,a^{n}),
	\end{equation}
	we have $b\circ A=A\circ b'$.
\end{lemma}
\begin{proof}
	No proof.
\end{proof}

As corollary,
\begin{corollary}
	The complex $\big( C_{\lambda}^n(\cA),b \big)$ is a subcomplex of the Hochschild complex.
\end{corollary}
\begin{proof}
	No proof.
\end{proof}

\subsection{Homomorphisms}
%--------------------------

Let $\cA$ and $\cB$ be two algebras. From a homomorphism $\rho\colon \cA\to \cB$, one induces a homomorphism of complex $\rho^*\colon C_{\lambda}^n(\cB)\to C_{\lambda}^n(\cA)$ with the definition
\[
	(\rho^*\varphi)(a^0,\cdots,a^{n})=\varphi\big( \rho(a^0),\cdots,\rho(a^{n}) \big).
\]
In order for $\rho^*$ to pass to quotient and define a homomorphism $\rho^*\colon HC^n(\cB)\to HV^n(\cA)$, we need that, for each $\varphi$, a certain $\eta$ fulfills $\rho^*b\varphi=b'\eta$.

\begin{proposition}
	Let $u\in\cA\invtible$, and $\theta\colon \cA\to \cA$ defined by $\theta(x)=uxu^{-1}$. Then the induced map $\theta^*\colon HC^*(\cA)\to HC^*(\cA)$ is the identity.
\end{proposition}
\begin{proof}
	No proof.
\end{proof}
The so defined map $\theta$ is called the \defe{inner automorphism}{inner!automorphism} associated with $u$.

\begin{lemma}
	Let $\cA$ be an unital algebra for which there exists a homomorphism $\rho$ and an invertible element $X$ in $M_2(\cA)$ such that
	\[
		X\begin{pmatrix}
			a & 0       \\
			0 & \rho(a)
		\end{pmatrix}
		X^{-1}=
		\begin{pmatrix}
			0 & 0       \\
			0 & \rho(a)
		\end{pmatrix}
	\]
	for all $a\in\cA$. Then $HC^n(\cA)=0$ for all $n$.
\end{lemma}
\begin{proof}
	No proof.
\end{proof}

We say that the cycle $(\Omega,d,\int)$ \defe{vanishes}{vanishing cycle} if the algebra $\Omega^0$ fulfills the hypothesis of the latter lemma.


\subsection{The cup product}
%---------------------------


In general, the space $\Omega^*(\cA\otimes\cB)$ is different to the space $\Omega^*(\cA)\otimes\Omega^*(\cB)$, but the first if a left $\cA$-module and a right $\cB$-module and $\Omega^*(\cA\otimes\cB)$ is the universal algebra with this property. So we have a homomorphism $\pi\colon \Omega^*(\cA\otimes\cB)\to \Omega^*(\cA)\otimes\Omega^*(\cB)$ such that $\pi\circ d_{\cA\otimes\cB}=(d_{\cA}\otimes d_{\cB})\circ\pi$.

Let now $\varphi$ be a $(n+1)$-linear functional on $\cA$. One define the linear functional $\hat{\varphi}$ on $\Omega^n(\cA)$ by the formula
\begin{equation}
	\hat{\varphi}\circ j\big( (a^0+\lambda\cun)\otimes a^1\otimes\cdots\otimes a^{n} \big)=\varphi(a^0,a^1,\cdots,a^{n}).
\end{equation}
This definition can be of course be adapted to $\cB$. Now we consider $\varphi\in C^n(\cA,\cA^*)$ and $\psi\in C^m(\cB,\cB^*)$ and the corresponding hat functions $\hat{\varphi}\colon \Omega^n(\cA)\to \cA^*$ and $\hat{\psi}\colon \Omega^m(\cB)\to \cB^*$. We can also consider the tensor product $\hat{\varphi}\otimes\hat{\psi}$. The homomorphism $\pi\colon \Omega^*(\cA\otimes\cB)\to \Omega^*(\cA)\otimes\Omega^*(\cB)$ allows us to consider the composition map $(\hat{\varphi}\otimes \hat{\psi})\circ \pi$ which has to be the hat function of a certain multilinear functional on $\cA\otimes\cB$. The latter is denoted by $\varphi  \cuppr \psi$:
\begin{equation}
	\widehat{\varphi\cuppr\psi}=(\hat{\varphi}\otimes\hat{\psi})\circ\pi.
\end{equation}
The map $\varphi\cuppr\psi$ and is called the \defe{cup product}{cup product} of $\varphi$ and $\psi$.

An element of $\Omega^l(\cA\otimes\cB)$ is of the form $\omega_1\otimes\omega_2\otimes\cdots\otimes\omega_l$ with $\omega_i\in\Omega^1(\cA\otimes\cB)$. But an element of the latter is of the form
\[
	\big( (a\otimes b)+\lambda\cun \big) \otimes(a'\otimes b')
\]
where $a$, $a'\in\cA$ and $b$, $b'\in\cB$. Notice that taking $b=b'=1$, one can identify the result to an element of $\Omega^1(\cA)$. Hence an element of $\Omega^n(\cA)\otimes\Omega^m(\cB)$ can be seen as a very special element of $\Omega^{n+m}(\cA\otimes\cB)$. From this point of view, we can see $\pi$ as a map
\[
	\pi\colon \Omega^{n+m}(\cA\otimes\cB)\to \Omega^n(\cA)\otimes\Omega^m(\cB).
\]

\begin{proposition}
	The cup product enjoys the following main properties.
	\begin{enumerate}
		\item The map $\varphi\otimes\psi\mapsto \varphi\cuppr\psi$ is a homomorphism
		      \[
			      HC^n(\cA)\otimes HC^m(\cB)\to HC^{n+m}(\cA\otimes\cB).
		      \]
		\item The character of the tensor product of two cycles is the cup product of their character.
	\end{enumerate}

\end{proposition}
