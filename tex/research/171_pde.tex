% This is part of (almost) Everything I know in mathematics
% Copyright (c) 2016
%   Laurent Claessens
% See the file fdl-1.3.txt for copying conditions.

%+++++++++++++++++++++++++++++++++++++++++++++++++++++++++++++++++++++++++++++++++++++++++++++++++++++++++++++++++++++++++++ 
\section{Basic definitions}
%+++++++++++++++++++++++++++++++++++++++++++++++++++++++++++++++++++++++++++++++++++++++++++++++++++++++++++++++++++++++++++

We consider PDE's for functions \( u\colon \eR^d\to \eR\). We denote by \( D^lu\) the set of partial derivatives of order \( l\) :
\begin{equation}
    (D^lu)(x)=\{ (D^{\alpha}u)(x) \}_{| \alpha |=l}.
\end{equation}

Some definitions.
\begin{definition}
    A PDE of order \( k\) is \defe{linear}{linear!PDE} if it is under the form
    \begin{equation}
        \sum_{| \alpha |\leq k}a_{\alpha}(x)(\partial^{\alpha}u)(x)+f(x)=0.
    \end{equation}
    That equation is \defe{homogeneous}{homogeneous!linear PDE} if \( f=0\).
\end{definition}

\begin{definition}
    A PDE is \defe{semi-linear}{semi-linear!PDE} is it is linear only with respect to the highest derivatives :
    \begin{equation}
        \sum_{| \alpha |=k}a_{\alpha}(\partial^{\alpha}u)(x)+F\big( x, (Du)(x),\ldots, D^{k-1}u(x) \big)=0
    \end{equation}
    where \( F\colon \Omega\times \eR^{\ldots}\to \eR\) is some function. The dots here represent the number of different partial derivative of order \( 0\) to \( k-1\) one has to consider.
\end{definition}

\begin{definition}
    A PDE of order \( k\) is \defe{quasi-linear}{quasi!linear PDE} if it is linear with respect to the higher order derivatives, with a coefficient that can depend to the lower order derivatives of the unknown :
    \begin{equation}
        \sum_{| \alpha |=k}a_{\alpha}\big( x,u(x),(Du)(x),\ldots, (D^{k-1})u(x) \big)(\partial^{\alpha}u)(x)+F\big( x,u(x),\ldots, (D^{k-1}u)(x) \big)=0.
    \end{equation}
\end{definition}

\begin{example}
    The equation
    \begin{equation}
        x(\partial_xu)(x,y)+y(partial_yu)(x,y)+u\sin(xy)=1
    \end{equation}
    is linear of order \( 1\). Indeed the coefficient of the partial derivatives of all orders  (\( 0\) and \( 1\)) depend on \( x\) and \( y\), but not of \( u\) or the derivatives of \( u\). 
\end{example}

\begin{example}
    The equation
    \begin{equation}
        x(\partial_xu)(x,y)+y(\partial_yu)(x,y)+u^2\sin(xy)=1
    \end{equation}
    is semi-linear of order \( 1\). Indeed the coefficient of the partial derivatives of order \( 1\) (the higher order) depend on \( x\) and \( y\) but not of \( u\) or the derivatives of \( u\). However, the derivatives of lower order (here : \( 0\)) are not linear. 
\end{example}

\begin{example}
    The heat equation
    \begin{equation}
        \frac{ \partial u }{ \partial t }-\frac{ \partial^2u }{ \partial x^2 }=1
    \end{equation}
    is linear ans non homogeneous.
\end{example}

\begin{example}
    The equation
    \begin{equation}
        \frac{ \partial u }{ \partial t }-\frac{ \partial  }{ \partial x }\left( c(u)\frac{ \partial u }{ \partial x } \right)=0
    \end{equation}
    is an abuse of notations for asking 
    \begin{equation}
        \frac{ \partial u }{ \partial t }(x,t)-\frac{ \partial x }{ \partial  }\left( (c\circ u)\times \frac{ \partial u }{ \partial x } \right)(x,t).
    \end{equation}
    for every \( t\) and \( x\). This is a second order equation whose highest order term is
    \begin{equation}
        (c\circ u)\frac{ \partial^2u }{ \partial x^2 }
    \end{equation}
    The coefficient of \( \partial^2_xu\) depend on \( u\), but not on the second order derivatives. So this is a quasi-linear equation.
\end{example}

%+++++++++++++++++++++++++++++++++++++++++++++++++++++++++++++++++++++++++++++++++++++++++++++++++++++++++++++++++++++++++++ 
\section{Principal symbol}
%+++++++++++++++++++++++++++++++++++++++++++++++++++++++++++++++++++++++++++++++++++++++++++++++++++++++++++++++++++++++++++

Let the semi-linear equation
\begin{equation}
    \sum_{| \alpha |=k}a_{\alpha}(x)(\partial^{\alpha}u)(x)+F\big( x,u(x),(Du)(x),\ldots,(D^{k-1}u)(x) \big).
\end{equation}
The differential operator associated with that equation is
\begin{equation}       \label{EQooNSOMooCfAdvt}
    \sum_{| \alpha |=k}a_{\alpha}\partial^k +F
\end{equation}
where we admit quite a notational shortcut to say that \( F\) applies on \( u\) by
\begin{equation}
    (Fu)(x)=F  \big( x,u(x),(Du)(x),\ldots,(D^{k-1}u)(x) \big).
\end{equation}

\begin{definition}
    The \defe{principal symbol}{symbol!principal} of the operator \eqref{EQooNSOMooCfAdvt} is the function
    \begin{equation}
        \begin{aligned}
            \sigma\colon \eR^d\times \eR^d&\to \eR \\
            (x,\xi)&\mapsto \sum_{| \alpha |=k}a_{\alpha}(x)\xi^{\alpha} 
        \end{aligned}
    \end{equation}
    where
    \begin{equation}
        \xi^{\alpha}=\xi_1^{\alpha_1}\ldots \xi_d^{\alpha_d}.
    \end{equation}
\end{definition}

\begin{definition}
    The characteristic associated with the symbol \( \sigma\) is the set of surfaces \( S\subset \eR^d\) of the form
    \begin{equation}
        S=\{ x\in \eR^d\st \phi(x)=0 \}
    \end{equation}
    where \( \phi\colon \eR^d\to \eR\) satisfies
    \begin{enumerate}
        \item
            \( \sigma\big( x,(\nabla \phi)(x) \big)=0\) for every \( x\in\eR^d\),
        \item
            \( (\nabla\phi)(x)\neq 0\) for every \( x\in S\).
    \end{enumerate}
\end{definition}
