\section{BF theory and relativity}
%++++++++++++++++++++++++++++++++++

References for BF theory, Palatini formalism and spin foam models are \cite{bkindep,degesols,itospinfoam,BenyAshHamil}. Let $M$ be a smooth $n$-dimensional manifold and consider the principal bundle
\begin{equation}
\xymatrix{%
 G   \ar@{~>}[r]		&	P\ar[d]\\
   				&	M
 }
\end{equation}
with $G$ being a connected semisimple Lie group with Lie algebra $\mG$. We consider the following as the basics fields of the theory:
\begin{itemize}
\item a connection $\omega$ on $P$,
\item a $\Ad(P)$-valued $(n-2)$-form $E$ on $M$.
\end{itemize}
In order to form the wedge product $E\wedge F$, we express both fields as $\mG$-valued forms on $M$ using a local coordinate chart $\sigma_{\alpha}\colon \mU_{\alpha}\to P$ (see pages \pageref{PgLocSecCurv} and \pageref{PgLocSecConn} ). We pose $A_{\alpha}=\sigma_{\alpha}^*\omega$ and $F_{\alpha}=\sigma_{\alpha}^*\Omega$ where $\Omega$ is the curvature of $\omega$.

There exists a $\eta\in\Wedge^{n-2}T_x^*M$ such that $E(x)=[\xi,X]\otimes\eta$; using the same section $\sigma_{\alpha}$, we find an element $X_{\alpha}\in\mG$ by the condition
\[
  E(x)=[\sigma_{\alpha}(x),X_{\alpha}]\otimes\eta.
\]
Hence we see $E$ as a $(n-2)$ form with values in $\mG$ as
\begin{equation}
E_{\alpha}(X_1,\cdots,X_{n-2})=\eta(X_1,\cdots,X_{n-2})X_{\alpha}(x).
\end{equation}
We are now able, using the wedge product of subsection~\ref{SecVectValFiffFor}, to see $E_{\alpha}\wedge F_{\alpha}$ as a $n$-form with values in $\mG\otimes\mG$. The use of the Killing form $\mG\otimes\mG\to\eR$ is what we write $\tr$. From now we drop all indices of local coordinates.

When we write $E\wedge F$, we mean the $4$-form defined by
\begin{align*}
(E\wedge F)(v_1,v_2,v_3,v_4)&= E(v_1,v_2)\otimes F(v_3,v_4) - E(v_1,v_3)\otimes F(v_2,v_4)  \\
	&\quad+E(v_1,v_4)\otimes F(v_2,v_3) + E(v_2,v_3)\otimes F(v_1,v_4)  \\
	&\quad-E(v_2,v_4)\otimes F(v_1,v_2) + E(v_3,v_4)\otimes F(v_1,v_2).
\end{align*}
Obviously $E\wedge F\neq F\wedge E$, but taking the trace (i.e changing the tensor product by the Killing form),
\begin{equation}
\tr(E\wedge F+F\wedge E)=2\tr(E\wedge F).
\end{equation}
The action that we want to study first is
\begin{equation}
S(E,\omega)=\int_M\tr(E\wedge F)
\end{equation}
where $F$ is the curvature associated with the connection $\omega$. When we vary that action with respect to $E$, i.e. when we impose
\[
  \Dsdd{ S(E+tE',\omega) }{t}{0}=0
\]
for every $(n-2)$ form $E'$, we find $F=0$ as equation of motion. A small computation shows that
\[
  \Dsdd{ F(\omega+t\omega') }{t}{0}=dA'+A\wedge A'=d_AA',
\]
where $d_A$ denotes the exterior covariant derivative associated with the connection $\omega$. Now the action principle imposes that for every connection $\omega'$,
\[
0=\Dsdd{ S(E,\omega+t\omega') }{t}{0}
		=\int_M\tr(E\wedge d_AA')
		=\int_M\tr(d_AE\wedge A')
\]
where we used an integral by part. So the equation of motion are
\begin{align}
F&=0&\text{and}&&d_AE&=0.
\end{align}

The inclusion of a cosmological constant in the game is done by modifying the action as
\begin{equation}
S_{BF}(A,E)=\int_M\tr\big( E\wedge F+\frac{ \Lambda }{ 12 }E\wedge E \big).
\end{equation}
Using the remark above, the equations of motion for the BF action are
\begin{align}
d_AE&=0&\text{and}&&F+\frac{ \Lambda }{ 6 }E=0.
\end{align}

The BF action with vanishing cosmological constant, $S_{BF}(\omega,E)=\int_M\tr(E\wedge F)$ has an interesting symmetry. Let $\eta$ be a $(n-3)$-$\Ad(P)$-valued form, and consider the transformation
\begin{align}		\label{EqSymPalaAEdAeta}
A&\mapsto A&\text{and}&&E&\mapsto E+d_A\eta.
\end{align}
We have
\[
\int_M\tr\big( E\wedge F+d_A\eta\wedge F \big)=\int_M\tr\big( E\wedge F+(-1)^{n-3+1}\eta\wedge d_AF \big)
\]
where we used an integral by part. The Bianchi identity $d_AF=0$ makes the last term vanishes, so that transformation \eqref{EqSymPalaAEdAeta} actually is a symmetry of the Palatini action without cosmological constant.


\section{Palatini formalism}
%+++++++++++++++++++++++++++

Let $M$ be a smooth four dimensional manifold on which we consider an oriented vector bundle $\mT$ which is isomorphic to $TM$ (without being \emph{canonically} isomorphic) endowed with a nondegenerate metric $\eta$ with fixed signature. We say that $\mT$ is the \defe{internal space}{internal!space} and that $\eta$ is the \defe{internal metric}{internal!metric}.

The basic fields of the theory are
\begin{itemize}
\item a $\mT$-valued $1$-form $e$ on $M$,
\item and a metric-preserving connection $\nabla$ on $\mT$.
\end{itemize}
One can define a metric on $M$ from the internal metric by the formula
\begin{equation}
  g(X,Y)=\eta\big( e(X),e(Y) \big).
\end{equation}
In order to do that, we have to see $e$ as a map $e\colon TM\to \mT$. We know that $e\in\Gamma(\mT\otimes T^*M)$, so that for each $x\in M$, we can write $e(x)=\sum_iv_i\otimes\omega_i$, so that we define
\[
  e(x)X=\sum_i\omega_i(X)v_i
\]
for all $X\in T_xM$. Notice that $g$ is nondegenerate if and only if $e$ is an isomorphism.

We are now going to express the basic fields as forms with values in the exterior algebra bundle $\Wedge\mT$. For $e$, it is easy: if $e(x)=v_i\otimes \omega_i$, we consider $e(x)=v_i^*\otimes\omega_i$.

We know from propositions~\ref{PropFormnabXthe} and~\ref{Propformconnve} that, with a choice of a local trivialization, the connection can be expressed in terms of a $\gl(V)$-valued $1$-form $A_{\alpha}$ on $\mU_{\alpha}\subset M$ where $V$ is the vector space on which $\mT$ is modeled. Here, the connection form takes its values in $\so(\eta)$ because of the assumption of metric preserving.

Passing to the curvature, we have $F\in\Gamma(\so(\eta)\otimes\Wedge^2T^*M)$. Now an element of $\mG=\so(\eta)$ can be seen as an element of $\Wedge^2\mT$ by the following formula:
\begin{equation}
S(v,w)=\eta(v,Sw)-\eta(Sv,w)
\end{equation}
where $S\in\so(\eta)$ and $v$, $w\in \mT_x$. In fact, the knowledge of that $2$-form allows to rebuild the element $S$ because when $v$ and $w$ run over a basis $\{ e_i \}$ of $V$, we find the numbers
\[
  (Se_j)_{i}-(Se_i)_i=S_{ij}-S_{ji}=2S_{ij}
\]
because $S$ has to be skew symmetric.

The orientation of $M$ provides an everywhere non vanishing volume form, this is an element of $\Wedge^4\mT$. Now if $\beta$ is another element of $\Wedge^4\mT$, it has to be a multiple of $\mu$. When $H$ is a $\Wedge^4\mT$-valued $n$-form on $M$, we define $\tr(H)$ as the usual $n$-form on $M$ defined by
\[
  H(X_1,\cdots,X_n)=\tr(H)(X_1,\cdots,X_n)\mu.
\]
Now the following action, which is the \defe{Palatini action}{Palatini action}, makes sense as integral of a $n$-form on a $n$-dimensional manifold:
\begin{equation}
S_{Pal}(\nabla,e)=\int_M\tr\left( e\wedge e\wedge F+\frac{ \Lambda }{ 12 }e\wedge e\wedge e\wedge e \right).
\end{equation}
Here the wedge product is a composition of wedge product as forms \emph{with values in} $\Wedge\mT$ and the wedge product as \emph{elements of} $\Wedge\mT$, so that
\[
  (e\wedge f)(X,Y)=e(X)\wedge f(y)-e(Y)\wedge f(X)
\]
where the wedge of the right hand side is taken in the sense of $\Wedge \mT$, so that we actually have $e\wedge f=f\wedge e$. The variation with respect to $\nabla$ provides the equation of motion
\begin{equation}
  d_A(e\wedge e)=0,
\end{equation}
 while the variation with respect to $e$ provides
\[
  0=\Dsdd{ S_{Pal}(\nabla,e+te') }{t}{0}=\int_M\tr\big( 2e'\wedge e\wedge F+\frac{ \Lambda }{ 12 }4e'\wedge e\wedge e\wedge e \big),
\]
or
\begin{equation}
e\wedge\big( F+\frac{ \Lambda }{ 3 }e\wedge e \big)=0.
\end{equation}

%----------------------------------------------------------------------------------------------------------------------------
\subsection{Other point of view}

Let $M$ be the four dimensional physical space and $E$, a four dimensional vector bundle. We suppose to know a local trivialization $E=M\times\eR^4$ and we consider the Lorentzian metric $\eta$ on each of the $\eR^4$. Finally we consider a soldering form $e$: for each point $x$, the map $e_x\colon T_xM\to E_x$ is an isomorphism. If $\partial_i$ is the local basis of $TM$, we use the local basis $\xi_i=e(\partial_i)$ as local coordinates on $E$.

These data allows us to define a metric on $M$ by
\begin{equation}
  g(x,Y)=\eta(eX,eY).
\end{equation}
The second dynamical variable will be an exterior derivative $D$ on $E$. Any section $s\in \Gamma(E)$ reads $s(x)=s^i\xi_i$ and the action of the exterior derivative reads
\begin{equation}
  D_Xs=\big( X(s^i)+s^j\omega_j^i(X) \big)\xi_i.
\end{equation}
At this point, we cannot define a torsion for $D$ because we do not have a canonical soldering form on $E$. However we pose the hypothesis that $D$ preserves the metric in the sense that
\begin{equation}		\label{EqDefDXpreserveMetr}
d_X\big( \eta(s,t) \big)=\eta(D_Xs,t)+\eta(s,D_Xt)
\end{equation}
for all $s$, $t\in\Gamma(E)$. Since $\eta(s,t)$ is a function on $M$, we have $D_X\big( \eta(s,t) \big)=X\big( \eta(s,t) \big)$. So we have on the one hand $\eta(D_Xs,t)=\eta_{ij}\Big( X(s^i)+s^k\omega^i_k \Big)t^j$ and on the other hand,
\[
  D_X\big( \eta(s,t) \big)=X^i\eta_{kl}\big( \partial_i(s^k)t^l+s^k\partial_it^l \big).
\]
Putting these two expressions into \eqref{EqDefDXpreserveMetr}, we find
\begin{equation}
\eta_{ij}\omega_k^i=-\eta_{ki}\omega^i_j,
\end{equation}
so that $\omega\in\Omega^1(M,\so(1,3))$. Now we use the isomorphism $e$ to build a connection on $TM$:
\begin{equation}
\nabla_XY=e^{-1}\big( D_X(eY) \big)
\end{equation}
where we see $eY$ as a section of $E$ by $(eY)(x)=e(Y_x)$. The Christoffel symbols of $\nabla$ and $D$ are defined as usual by $\nabla_{\partial_i}(\partial_j)=\tilde\Gamma_{ij}^k\partial_k$, and $\omega_j^k(\partial_i)=\Gamma_{ij}^k$. It is simple to see that numerically, $\Gamma_{ij}^k=\tilde\Gamma_{ij}^k$. Indeed
\[
  D_{\partial_i}(\xi_j)	=\xi_k\otimes\big( \partial_i(\delta_j^k)+\delta-j^l\omega_l^k(\partial_i) \big)=\xi_k\otimes\omega+j^k(\partial_i)=\xi_k\otimes\Gamma_{ij}^k,
\]
so that $\nabla_{\partial_j}(\partial_j)=e^{-1}\big( \xi_k\otimes\Gamma_{ij}^k\big)=\Gamma_{ij}^k\partial_k$.

The connection $\nabla$ is compatible with the metric because
\[
\begin{split}
\nabla_Z\big( g(X,Y) \big)	&=Z\big( \eta(eX,eY) \big)\\
				&=\eta\big( \underbrace{D_Z(eX)}_{=e(\nabla-ZX)},eY \big)+\eta\big( eX,D_Z(eY) \big)\\
				&=g(\nabla_ZX,Y)+g(X,\nabla_ZY).
\end{split}
\]
