The concept of representation of a group is already defined in \ref{DEFooXVMSooXDIfZV}.

%+++++++++++++++++++++++++++++++++++++++++++++++++++++++++++++++++++++++++++++++++++++++++++++++++++++++++++++++++++++++++++ 
\section{Definitions}
%+++++++++++++++++++++++++++++++++++++++++++++++++++++++++++++++++++++++++++++++++++++++++++++++++++++++++++++++++++++++++++

\begin{definition}      \label{DEFooAFSAooGDSDBb}
    The representation \( (V,T)\) is \defe{faithful}{faithful representation} if the map \( T\colon G\to \GL(V)\) is injective. It means that \( T(g)=\id\) only when \( g\) is the neutral in \( G\).
\end{definition}

When \( G\) is a subgroup of \( \GL(V)\) we say that
\begin{equation}
    \begin{aligned}
        \rho\colon G&\to \GL(V) \\
        g&\mapsto g 
    \end{aligned}
\end{equation}
is the \defe{definition representation}{definition representation}. This is in particular the case when \( G\) is a group of matrix; the definition representation of \( G\) is then the action of these matrices on \( \eR^n\) or \( \eC^n\). This is a representation by proposition \ref{PROPooHNQOooSzeEFG}.

\begin{definition}
    Let \( G\) be a group. Two representations \( (T_1,V_1)\), \( (T_2,V_2)\) of \( G\) are \defe{equivalent}{equivalent representations} is there exists a vector space isomorphism\footnote{Linear bijection} \( m\colon V_1\to V_2\) such that
    \begin{equation}
        T_2(g)\circ m=m\circ T_1(g)
    \end{equation}
    for every \( g\in G\).
\end{definition}

\begin{definition}
    If \( T\colon G\to \GL(V)\) is a representation, a subspace \( W\subset V\) is \defe{invariant}{invariant subspace} under \( T\) if
    \begin{equation}
        T(g)W\subset W
    \end{equation}
    for every \( g\in G\).

    A representation is \defe{irreducible}{irreducible representation} if \( V\) and \( \{ 0 \}\) are the only invariant subspaces of \( V\).
\end{definition}

\begin{lemma}[Schur\cite{BIBooYTTJooYpPYLT}]
    Let \( (T,V)\) and \( (S,W)\) be two irreducible representations of \( G\). Let \( \alpha\colon V\to W\) be a linear map such that
    \begin{equation}
        \alpha\circ T(g)=S(g)\circ \alpha
    \end{equation}
    for every \( g\in G\). If \( \alpha\neq 0\), then \( \alpha\) is a bijection.
\end{lemma}

\begin{proof}
    Let \( L\) be the image of \( \alpha\): \( L=\alpha(V)\).
    \begin{subproof}
        \item[\( L\) is invariant under \( S\)]
            Let \( y\in \alpha(V)\): there exist \( x\in V\) such that \( y=\alpha(x)\). Then we have
            \begin{equation}
                S(g)\big( \alpha(x) \big)=\big( \alpha\circ T(g) \big)x=\alpha\big( T(g)x \big)\in \alpha(V).
            \end{equation}
        \item[Two possibilities]
            Since \( S\) is irreducible, there are two possibilities: \( \alpha(V)=\{ 0 \}\) and \( \alpha(V)=W\).
        \item[First: \( \alpha(V)=\{ 0 \}\)]
            In this case, \( \alpha=0\).
        \item[Second: \( \alpha(V)=W\)]
            This means that \( \alpha\) is surjective. We show that \( \alpha\) is injective too. The space \( \ker(\alpha)\) is invariant under \( T\) because, if \( z\in \ker(\alpha)\), then
            \begin{equation}
                \alpha\big( T(g)z \big)=S(g)\big( \alpha(z) \big)=S(g)(0)=0.
            \end{equation}
            There are two possibilities: \( \ker(\alpha)=\{ 0 \}\) and \( \ker(\alpha)=V\). The possibility \( \ker(\alpha)=V\) says \( \alpha=0\) while we are in the case \( \alpha(V)=W\). So we deduce \( \ker(\alpha)=\{ 0 \}\) which means that \( \alpha\) is injective.
    \end{subproof}
\end{proof}

\begin{theorem}[\cite{BIBooYTTJooYpPYLT}]   \label{THOooXHVHooDQdgDI}
    Let \( (T,V)\) be a finite dimensional representation of \( G\). Let \( \alpha\in \End(V)\) be such that
    \begin{equation}
        \alpha\circ T(g)=T(g)\circ \alpha.
    \end{equation}
    There exists \( \lambda\in \eC\) such that \( \alpha=\lambda\id\).
\end{theorem}

\begin{proof}
    The endomorphism \( \alpha\) has at least one eigenvalue \( \lambda\in \eC\) (proposition \ref{PROPooLXGSooXmVcVG}). Let \( V_{\lambda}=\{ x\in V\tq \alpha(x)=\lambda x \}\) be the associated eigenspace.

    We prove that \( V_{\lambda}\) is an invariant subspace: if \( x\in V_{\lambda}\), then \( T(g)x\) has eigenvalue \( \lambda\) for \( \alpha\). Let's do that:
    \begin{equation}
        \alpha\big( T(g)x \big)=T(g)\big( \alpha(x) \big)=T(g)(\lambda x)=\lambda T(g)x.
    \end{equation}
    
    Since \( T\) is irreducible, the invariant subspace \( V_{\lambda}\) is \( \{ 0 \}\) or \( V\). The space \( V_{\lambda}\) has at least dimension one because \( \lambda\) is an eigenvalue. Thus \( V_{\lambda}=V\) and \( \alpha(x)=\lambda x\).
\end{proof}

%--------------------------------------------------------------------------------------------------------------------------- 
\subsection{Representations of \( \gU(1)\)}
%---------------------------------------------------------------------------------------------------------------------------

\begin{theorem}[\cite{BIBooYTTJooYpPYLT}]       \label{THOooFFJGooCekFQc}
    Every irreducible representation of an abelian group has dimension \( 1\).
\end{theorem}

\begin{proof}
    Let \( G\) be an abelian group and \( (T,V)\) an irreducible representation of \( G\). If \( g_1,g_2\in G\) we have \( g_1g_2=g_2g_1\) and thus
    \begin{equation}
        T(g_1)T(g_2)=T(g_2)T(g_1).
    \end{equation}
    The endomorphism \( \alpha=T(g_1)\) commutes with every \( T(g)\) and is then a multiple of \( \id\) by the theorem \ref{THOooXHVHooDQdgDI}. So there exists a map \( \lambda\colon G\to \eC\) such that \( T(g)=\lambda(g)\id\).

    Every subspace of \( V\) is invariant. In particular a subspace of dimension \( 1\) is the whole space \( V\).
\end{proof}

\begin{proposition}
    Let \( V\) be a complex vector space of dimension \( 1\). Let \( \xi\neq 0\in V\). If \( \langle ., .\rangle \) is an hermitian product on \( V\), there exists \( m\in \eR\) such that
    \begin{equation}
        \langle z_1\xi, z_2\xi\rangle =mz_1\bar z_2.
    \end{equation}
\end{proposition}

\begin{proof}
    Every element of \( V\) can be written under the form \( z\xi\) for some \( z\in \eC\). The properties of an hermitian product say
    \begin{equation}
        \langle z_1\xi, z_2\xi\rangle =z_1\bar z_2\langle \xi, \xi\rangle 
    \end{equation}
    and \( \langle \xi, \xi\rangle \in \eR\).
\end{proof}

\begin{proposition}[Irreducible representations of \( U(1)\)]       \label{PROPooLWWEooUmqbRA}
    Let \( k\in \eZ\). We consider
    \begin{equation}        \label{EQooXPXKooJasMyY}
        \begin{aligned}
            T_k\colon U(1)&\to \GL(\eC) \\
            T_k(g)z&=g^kz.
        \end{aligned}
    \end{equation}
    \begin{enumerate}
        \item
            The formula \eqref{EQooXPXKooJasMyY} defines a representation of \( U(1)\).
        \item
            The representation \( T_k\) is irreducible.
        \item
            The representation \( T_k\) is continuous.
        \item
            If \( k\neq l\), then the representations \( T_k\) and \( T_l\) are not equivalent.
        \item
            Every continuous irreducible representation of \( U(1)\) is equivalent to one of them.
    \end{enumerate}
\end{proposition}

\begin{proof}
    Several points.
    \begin{subproof}
        \item[It is a representation]
            Since \( U(1)\) is abelian, \( (g_1g_2)^k=g_1^kg_2^k\).
        \item[Irreducible]
            The representation \( T_k\) is irreducible because the vector space is \( \eC\) which has dimension \( 1\).
        \item[Continuous]
        \item[We have all of them]
            Let \( (\rho, V)\) be an irreducible representation of \( U(1)\). Since \( U(1)\) is abelian, \( \dim(V)=1\) (theorem \ref{THOooFFJGooCekFQc}). So there exist a function \( \lambda\colon U(1)\to \eC\) such that 
            \begin{equation}
                \rho(g)=\lambda(g)\id.
            \end{equation}
            For \( x,y\in \eR\) we have \( \rho\big(  e^{ix} e^{iy} \big)=\rho( e^{ix})\rho( e^{iy})\). We introduce the notation
            \begin{equation}
                \alpha=\lambda\circ\varphi
            \end{equation}
            where \( \varphi\colon \eR\to U(1)\) is \( \varphi(x)= e^{ix}\).
    \end{subproof}
\end{proof}
<++>
