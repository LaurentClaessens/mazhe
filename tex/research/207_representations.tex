The concept of representation of a group is already defined in \ref{DEFooXVMSooXDIfZV}.

%+++++++++++++++++++++++++++++++++++++++++++++++++++++++++++++++++++++++++++++++++++++++++++++++++++++++++++++++++++++++++++ 
\section{Some results}
%+++++++++++++++++++++++++++++++++++++++++++++++++++++++++++++++++++++++++++++++++++++++++++++++++++++++++++++++++++++++++++

\begin{definition}      \label{DEFooAFSAooGDSDBb}
    The representation \( (V,T)\) is \defe{faithful}{faithful representation} if the map \( T\colon G\to \GL(V)\) is injective. It means that \( T(g)=\id\) only when \( g\) is the neutral in \( G\).
\end{definition}

When \( G\) is a subgroup of \( \GL(V)\) we say that
\begin{equation}
    \begin{aligned}
        \rho\colon G&\to \GL(V) \\
        g&\mapsto g 
    \end{aligned}
\end{equation}
is the \defe{definition representation}{definition representation}. This is in particular the case when \( G\) is a group of matrix; the definition representation of \( G\) is then the action of these matrices on \( \eR^n\) or \( \eC^n\). This is a representation by proposition \ref{PROPooHNQOooSzeEFG}.
