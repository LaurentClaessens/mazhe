% This is part of (almost) Everything I know in mathematics and physics
% Copyright (c) 2013-2014, 2019
%   Laurent Claessens
% See the file fdl-1.3.txt for copying conditions.

\section{Iwasawa decomposition of Lie groups}%\label{sec:Iwasawa}
%+++++++++++++++++++++++++++++++++++++++++++++++

In this section, we show the main steps of the Iwasawa decomposition for a semisimple Lie group. For proofs, the reader will see \cite{Knapp} VI.4 and \cite{Helgason} III,\S\ 3,4 and VI,\S\ 3. In the whole section, $G$ denotes a semisimple group, and $\lG$ its real (finite dimensional) Lie algebra. The two main examples that are widely used are $\SL(2,\eR)$ and $\SO(2,n)$.

\subsection{Cartan decomposition}
%-------------------------------

If $\lG$ is a finite dimensional Lie algebra and $X$, $Y\in\lG$, the composition of the adjoint $\dpt{\ad X\circ \ad Y}{\lG}{\lG}$ makes sense.
\begin{definition}
An involutive automorphism $\theta$ on a \emph{real} semi simple Lie algebra $\lG $ for which the form $B_{\theta}$,
\begin{equation}
          B_{\theta}(X,Y):=-B(X,\theta Y)
\end{equation}
($B$ is the Killing form on $\lG$) is positive definite is a \defe{Cartan involution}{Cartan!involution}.
\index{involution!Cartan}
\end{definition}

\begin{proposition}
There exists a Cartan involution for every real semisimple Lie algebra.
\end{proposition}

\begin{probleme}
The theorem 4.1 in \cite{Helgason} is maybe a proof of this proposition.
\end{probleme}

See \cite{Helgason}, theorem 4.1.  Since $\theta^2=id$, the eigenvalues of a Cartan involution are $\pm 1$, and we can define the \defe{Cartan decomposition}{Cartan!decomposition}\index{decomposition!Cartan} $\lG$
\begin{equation}
                    \lG=\lK\oplus\lP
\end{equation}
into $\pm1$-eigenspaces of $\theta$ in such a way that $\theta=(-\id)|_{\lP}\oplus \id|_{\lK}$. These eigenspaces are subject to the following commutation relations:
\begin{equation} \label{Ieq:comm_KP}
[\lK,\lK]\subseteq\lK,\quad[\lK,\lP]\subseteq\lP,\quad [\lP,\lP]\subseteq\lK.
\end{equation}
%
The dimension of a maximal abelian subalgebra of $\lP$ is the \defe{rank}{rank of a Lie algebra} of $\lG$. One can prove that it does not depend on the choices (Cartan involution and maximal abelian subalgebra). We denote by $\lA$ such a maximal abelian subalgebras of $\lP$.


\begin{lemma}
If $\lG_0$ is a real semisimple Lie algebra and $\theta$ a Cartan involution, then for all $X\in\lG_0$,
\begin{equation}
                (\ad X)^*=-\ad(\theta X),
\end{equation}
where the star on an operator on $\lG$ is defined by
\begin{equation}
                  B_{\theta}(X,AY)=B_{\theta}(A^*X,Y).
\end{equation}
\end{lemma}

\begin{lemma}
The set of operators $\ad(\lA)$ is an abelian algebra whose elements are self-adjoint.
\end{lemma}

\begin{proof}
We have to prove that $(\ad H)^*=(\ad H)$ and $[\ad H,\ad I]=0$ for every $H$, $I\in \lA$.  First, note that $H\in\lA\subset\lP$, thus $\theta H=-H$, and $(\ad H)^*=-\ad(\theta H)=\ad H$.

For the second, $\ad H\circ \ad I=\ad(H\circ I)$ so that $[\ad H,\ad I]=\ad[H,I]=0$ because $\lA$ is abelian.
\end{proof}

\subsection{Root space decomposition}
%--------------------------------

From the lemma, the operators $\ad(H)$ with $H\in\lA$ are simultaneously diagonalisable. That means that there exists a basis $\{ X_i \}$ of $\lG$ and linear maps $\lambda_i\colon \lA\to \eR$ such that
\[
    \ad(H)X_i=\lambda_i(H)X_i.
\]
 For each $\lambda\in\lA^*$, we define
\begin{equation}
        \lG_{\lambda}=\{X\in\lG|(\ad H)X=\lambda(H)X,\forall H\in\lA\}.
\end{equation}
Elements $0\neq\lambda\in\lA^*$ such that $\lG_{\lambda}\neq 0$ are called \defe{restricted roots}{root!restricted} of $\lG$. The set of restricted roots is denoted by $\Sigma$.

\begin{proposition}     \label{prop:somme_de_G}
 The restricted root together with $\lA$ itself span the whole space:
\begin{equation}    \label{eq:somme_de_G}
             \lG=\lG_0 \oplus_{\lambda\in\Sigma}\lG_{\lambda},
\end{equation}
This decomposition is called the \defe{restricted root space decomposition}{root!space!decomposition}\index{decomposition!root space}.
\end{proposition}

\begin{proof}
We first prove that the sum is direct. If the sum is not so, we can find a $H^*\in\lG_0$ and $X_i\in\lG_{\lambda_i}$ ($\lambda_i\in\Sigma$) such that
\begin{equation}\label{eq:2906r2}
              H^*+\sum_iX_i=0
\end{equation}
Let us consider
\[
      N=\{H\in\lG_0|\textrm{ the $\lambda_i(H)$ are all differents and not zero}\}
\]
A $H$ which is not in $N$ fulfils some relations as $\lambda_i(H)=\lambda_j(H)$ which are linear equations, so the complement of $N$ is an union of hyperplanes and thus $N$ is not empty. This allows us to consider a $H\in N$.

We have choice the $X_i$ in $\lG_{\lambda_i}$, \emph{i.e.}
\begin{equation} \label{eq:2906r1}
(\ad A)X_i=\lambda_i(A)X_i
\end{equation}
 for all $A\in\lA$. In other words, $X_i$ diagonalise $\ad A$ with eigenvalues $\lambda_i(A)$. Now, let us consider $\ad H$ for a $H\in N$. Since all the $\lambda_i(H)$ are different and not zero, the equation \eqref{eq:2906r1} implies that all the $X_i$ (and $H^*$) are in separate eigenspaces of $\ad H$. Thus they are linearly independent, hence the equation \eqref{eq:2906r2} is not possible. The sum \eqref{eq:somme_de_G} is thus a direct sum. For the rest of
  the proof, see~\cite{Helgason} theorem 4.2.
\end{proof}

Other properties of the root spaces are listed in the following proposition.
\begin{proposition}
The spaces $\lG_{\lambda_i}$ satisfy also:
\begin{enumerate}
\item $[\lG_{\lambda},\lG_{\mu}]\subseteq\lG_{\lambda+\mu}$,
\item $\theta\lG_{\lambda}=\lG_{-\lambda}$; in particular, if $\lambda$ belongs to $\Sigma$, then $-\lambda$ belongs to $\Sigma$ too,
\item $\lG_0=\lA\oplus\mZ_{\lK}(\lA)$ orthogonally.
\end{enumerate}
\end{proposition}

\begin{probleme}
Il faut définir quelque part ce qu'est cet espace $\mZ_{\lK}(\lA)$
\end{probleme}

%---------------------------------------------------------------------------------------------------------------------------
\subsection{Positivity, convex cone and partial ordering}
%---------------------------------------------------------------------------------------------------------------------------
\label{SubsecPosiCconePartOrder}

\begin{definition}[\cite{Knapp}]
    Let $V$ be a vector space. A \defe{positivity notion}{positive!element!in a vector space} is the data of a subset $V^+$ of $V$ such that
    \begin{enumerate}
        \item
            for every nonzero $v\in V$, we have $v\in V^+$ \emph{xor} $-v\in V^+$,
        \item
            for every $v$, $w\in V^+$ and every $\lambda\in\eR^+$, the elements $v+w$ and $ \lambda v$ are positive.
    \end{enumerate}
\end{definition}

If $v\in V^+$, we say that $v$ is \defe{positive}{positive!vector} and we note $v>0$.

\begin{definition}      \label{DefConvexCone}
    A \defe{convex cone}{convex!cone} in a vector space \(A\) is a subset $C$ such that
    \begin{enumerate}
        \item
            $x\in C$ and $t\in\eR^+$ imply $tx\in C$, \label{enuli}
        \item
            $x,y\in C$ implies $x+y\in C$,\label{enulii}
        \item
            $C\cap(-C)=\{ 0 \}$.\label{enuliii}
    \end{enumerate}
\end{definition}

To a convex cone $C$ is attached a notion of positivity by defining $x\geq 0$ if and only if $x\in C$. The converse is also true: if we have a notion of positivity on $V$, we define the corresponding convex cone by\label{PgConeAndPositive}
\begin{equation}
    V^+=\{ x\in V\tq x\geq 0 \}.
\end{equation}

A \defe{linear partial ordering relation}{ordering!linear partial relation} is a partial ordering $\leq$ such that
\begin{itemize}
    \item $A\leq B$ implies $A+C\leq B+C$ for all $C$,
    \item $\lambda A\leq\lambda B$ for all $\lambda\in\eR^+$.
\end{itemize}
From a positivity notion gives rise to a linear partial ordering on \(V\) by defining \(x\geq y\) if and only if \(y-x\geq 0\).

\subsection{Iwasawa decomposition}
%--------------------------------

Let us consider a notion of positivity on $\lA^*$ and denote by $\Sigma^+$ the set of positive roots. We define
\begin{equation}
      \lN:=\oplus_{\lambda\in\Sigma^+}\lG_{\lambda}.
\end{equation}
The \defe{Iwasawa decomposition}{Iwasawa!decomposition}\index{decomposition!Iwasawa} is given by the following theorem (\cite{Knapp}, theorem 5.12):

\begin{theorem}
Let $G$ be a linear connected semisimple group and $A=\exp\lA$, $N=\exp\lN$ where $\lA$ and $\lN$ are the previously defined algebras. Then $A$, $N$ and $AN$ are simply connected subgroups of $G$ and the multiplication map
\begin{equation}
\begin{aligned}
  \phi\colon A\times N\times K&\to G \\
 (a,n,k)&\mapsto ank
\end{aligned}
\end{equation}
is a global analytic diffeomorphism. In particular, the Lie algebra $\lG$ decomposes as vector space direct sum
\begin{equation}
            \lG=\lA\oplus\lN\oplus\lK.
\end{equation}
 The group $AN$ is a solvable subgroup of $G$ which is called the \defe{Iwasawa group}{Iwasawa!group}, or Iwasawa component of $G$.
\label{ThoIwasawaVrai}
\end{theorem}

\begin{remark}
It can be proved that this theorem is independent of the choices: the Cartan involution, the maximal abelian subalgebra $\lA$ and the notion of positivity.
\end{remark}
Notice that $A$, $N$ and $K$ are unique up to isomorphism. Their matricial representation of course depend on choices.

This theorem from \cite{Helgason}, chapter VI, Theorem 3.4. will be useful.

\begin{theorem}
The Lie algebra $\lA\oplus\lK$ is solvable.
\end{theorem}
This theorem implies that the group $AN$ is solvable.\quext{J'esère que ce que je raconte ici n'est pas trop débile pcq j'ai pas été fouiller à fond.} Before to go into concrete situations, let us prove an useful property of the $\lK$ part of $\lG$:

\begin{theorem}
\[
      \Stab(\lK)=K
\]
 for the adjoint action of $G$ on $\lK$.
\label{tho:Stab_K}
\end{theorem}
The proof of it is given by two lemmas. \cite{Humphreys}

\begin{lemma}
For any $k\in K$,
\[
   \Ad(k)\lK=\lK,
\]
\label{lem:stab_1}
\end{lemma}
and
\begin{lemma}
 If for any $L\in\lK$, $\Ad(x)L$ belongs to $\lK$, then $x\in K$.
\label{lem:stab_2}
\end{lemma}

\begin{proof}[Proof of lemma~\ref{lem:stab_1}]
   Let us take a $L\in\lK$ and define $M\in K$ $k=e^M$. We have $\Ad(k)L=e^{\ad M}L$. But in general, we have the relations \eqref{Ieq:comm_KP} which give $e^{\ad M}L\in\lK$. Then $\Ad(k)\lK\subset\lK$.

   In order to show that $\lK\subset\Ad(k)\lK$, let us consider a $L\in\lK$. We have to find a $N\in\lK$ such that $\Ad(k)N=L$. It is clear that $N=\Ad(k^{-1})L$ fulfils the conditions.
\end{proof}

\begin{proof}[Proof of lemma~\ref{lem:stab_2}]
Let us consider $X\in\lG$ such that $x=e^X$. We have $e^{\ad X}L\in\lK$ for all $L\in\lK$. This implies that all the terms of the expansion of $e^{\ad X}L$ are in $\lK$. In particular, $[X,L]\in\lK$ for all $L\in\lK$. Let us consider the Cartan decomposition of $X$: $X=X_k+X_p$. We need $X$ such that
\[
   [X_k,L]+[X_p,L]\in\lK
\]
for any $L\in\lK$. But inclusions \eqref{Ieq:comm_KP} make $[X_p,L]\in\lP$. Then the $X_p$ part of $X$ must vanish (because $\lG=\lK\oplus\lP$ is a direct sum).
\end{proof}

\section{Representations}
%++++++++++++++++++++++++++++++
Source: \cite{Lie_groups}

We are interested in the adjoint representation on a common vector space; we will not discuss the importance of some more complicated features as the ``locally convex''\ condition. We only mention it.

\begin{definition}
If $V$ is a locally convex space, a \defe{continuous representation}{representation!of Lie group} of a Lie group $G$ on $V$ is a left invariant action $\dpt{\pi}{G\times V}{V}$ such that for any $x\in G$, the map $\dpt{\pi(x)}{V}{V}$ is a linear endomorphism of $V$.
\end{definition}

If $\lG$ is a Lie algebra, a \defe{representation}{representation!of Lie algebra}\index{Lie!algebra!representation} of $\lG$ in $V$ is a bilinear map $\dpt{\sigma}{\lG}{\End(V)}$ such that
\begin{equation}
    \sigma([X,Y])v=[\sigma(X),\sigma(Y)]v=\sigma(X)\sigma(Y)v-\sigma(Y)\sigma(X)v.
\end{equation}
In other words, $\dpt{\sigma}{\lG}{\End{V}}$ is an algebra homomorphism.

A vector space equipped with a representation of a Lie algebra $\lG$ is a \defe{$\lG$-module}{$g$-module@$\protect\lG$-module}. A \emph{complete} locally convex space equipped with a representation of a Lie group is a \defe{$G$-module}{$G$-module}.

If $\pi$ is a representation of $G$ in a (eventually complex) vector space $V$, an \defe{invariant subspace}{invariant!vector subspace} is a vector subspace $W\subset V$ such that $\pi(x)W\subset W$ for any $x\in G$. A continuous representation in a complete locally compact vector space $V$ is \defe{irreducible}{irreducible!representation} if $\{0\}$ and $V$ are the only two invariant closed subspaces of $V$.

In the case of finite dimensional vector space, any subspace is closed; in this class, we find back the usual notion of irreducibility.

An \defe{unitary}{unitary!representation}\index{representation!unitary} representation of $G$ is a continuous representation $\pi$ of $G$ in a complex (or real) Hilbert space $H$ such that $\pi(x)$ is unitary for any $x\in G$. This is: $\pi$ is unitary if and only if $\forall x\in G$, $v$, $w\in H$,
\begin{equation}
\scal{\pi(x)v}{w}=\scal{v}{\pi(x)^{-1} w}.
\end{equation}
A continuous and finite dimensional representation is \defe{unitarisable}{representation!unitarisable} if there exists an hermitian product for which the representation is unitary.

Now a great proposition without proof:

\begin{proposition}
Let $G$ be a compact Lie group\quext{Verifie s'il faut que ce soit de Lie}. Then every representation on a finite dimensional vector space is unitarisable.
\end{proposition}

%+++++++++++++++++++++++++++++++++++++++++++++++++++++++++++++++++++++++++++++++++++++++++++++++++++++++++++++++++++++++++++
\section{Other results about Cartan algebras}
%+++++++++++++++++++++++++++++++++++++++++++++++++++++++++++++++++++++++++++++++++++++++++++++++++++++++++++++++++++++++++++



\begin{lemma}
A Cartan subalgebra of a semisimple complex Lie algebra is maximally abelian.
\end{lemma}

\begin{proof}
    If \( \lH\) is a Cartan subalgebra of \( \lG\), proposition~\ref{PropCartanLzXtjs} provides \( H_0\in\lG\) such that \( \lH=\lG_0(H_0)\); in particular \( H_0\in\lH\). We are going to prove that if \( H_1,H_2\in\lH\), then for every \( Y\in\lG\) we have \( B\big( [H_1,H_2],Y \big)=0\), so that the non degeneracy of the Killing form will conclude that \( [H_1,H_2]=0\).


    Let $X\in\lG(H_0,\lambda)$, $H\in\lH$. The map $\ad X\circ\ad H$ sends $\lG(H_0,\mu)$ to $\lG(H_0,\lambda+\mu)$. If we choose a basis of $\lG$ made up with basis of the spaces $\lG(H_0,\lambda_i)$ (by the primary decomposition theorem) it is clear that $B(H,X)=\tr(\ad H\circ\ad X)=0$. In particular with \( H=[H_1,H_2]\) we get \( B\big( [H_1,H_2],X \big)=0\).

    On the other hand, $\lH$ is solvable because it is nilpotent. Since the adjoint action provides a representation of \( \lH\) on \( \lH\), corollary~\ref{cor:de_Lie_Vu} says that we have  basis of $\lH$ in which all the matrices of are upper triangular. Now if $A$, $B$ and $C$ are upper triangular matrices, $ABC$ and $BAC$ have same elements on the diagonal;in particular they traces are the equal: $\tr(ABC)=\tr(BAC)$. Let us consider $H_1$, $H_2$, $H\in\lH$ By Jacobi, $\ad[H_1,H_2]=[\ad H_1,\ad H_2]$, then
    \begin{equation}
    \begin{split}
    \tr(\ad[H_1,H_2]\ad H)&=\tr(\ad H_1\ad H_2\ad H)-\tr(\ad H_2\ad H_1\ad H)\\
    &=\tr(\ad H_2\ad H_1\ad H)-\tr(\ad H_1\ad H_2\ad H)\\
    &=0.
    \end{split}
    \end{equation}
    Up to now we had seen that $B([H_1,H_2],H)=0$ and $B(H,X)=0$ if $X\in\oplus_{\lambda\neq 0}\lG(H_0,\lambda)$. In the latter, we can consider $[H_1,H_2]$ as $H$. Then
    \[
    B([H_1,H_2],Y)=0
    \]
    for all $Y\in\lG$. Then $[H_1,H_2]=0$ because the Killing form is nondegenerate ($\lG$ is semisimple). This proves that $\lH$ is abelian.

    Now it remains to see that $\lH$ is contained in no larger abelian subalgebra of $\lG$. For this, we naturally consider a larger abelian subalgebra $\lH'$ of $\lG$. For any $H'\in\lH'$ and $H\in\lH$, we have $[H,H']=0$. In particular $[H',H_0]=0$; the property
    \[
    \lH=\lG(H_0,0)=\{X\in\lG\tq (\ad H_0)^kX=0\text{ for a cerain }k\in\eN\}.
    \]
    makes $H'\in\lH$.
\end{proof}


\begin{proposition}\label{prop:G_x_central}
    Let $\lG$ be a Lie algebra, $x\in\lG$ and\nomenclature{$\lG^x$}{An algebra derived from $\lG$}
        \begin{equation}
            \lG^x=\{y\in\lG\tq\exists n\in\eN:(\ad x)^ny=0\}.
        \end{equation}
    Then $\lG^x$ is a subalgebra of $\lG$ which is its own centralizer in $\lG$.
\end{proposition}

\begin{proof}
    Since $\ad(x)$ is a derivation\footnote{Definition \ref{DEFooDUEUooZLhKdv}.} of $\lG$
\[
(\ad x)^n([u,v])=\sum_{k=0}^n \binom{n}{k} [ (\ad x)^ku,(\ad x)^{n-k}v ];
\]
then $[\lG^x,\lG^x]\subset\lG^x$. This proves that $\lG^x$ is a subalgebra of $\lG$. Let $y\in\lG$ be such that $[y,\lG^x]\subset\lG^x$. Clearly $[x,y]\in\lG^x$ (because $x\in\lG^x$) then $(\ad x)^ny=(\ad x)^{n-1}[x,y]$, so that $y\in\lG^x$.
\end{proof}

\begin{proposition}
Let $\lG$ be a Lie algebra and $x\in\lG$. Then there exists a subspace $\lG_x$ of $\lG$ such that $\lG=\lG_x\oplus\lG^x$ and $[\lG^x,\lG_x]\subset\lG_x$.
\label{prop:G_x_G_x}
\end{proposition}

\begin{proof}
We claim that the space is given by
\begin{equation}
\lG_x=(\ad x)^p\lG
\end{equation}
where $p$ is taken large enough to have $(\ad x)^p\lG=(\ad x)^{p+1}\lG$. The lemma and the discussion below show the correctness of the definition of $\lG_x$ and that $\lG=\lG_x\oplus\lG^x$. It remains to be proved that $[\lG^x,\lG_x]\subset\lG_x$. For we will prove (by induction with respect to $m$) for any $m$ that $(\ad x)^my=0$ implies $(\ad y)\lG_x\subset\lG_x$.

For $m=1$, the induction assumption becomes $[x,y]=0$ and Jacobi gives $\ad x\circ\ad y=\ad y\circ\ad x$, then $(\ad y)\lG_x=(\ad x)^p(\ad y)\lG\subset\lG_x$. Now we suppose that $(\ad x)^{m-1}z=0$ implies $(\ad z)\lG_x\subset\lG_x$ and we consider $y\in\lG$ such that $(\ad x)^my=0$ and $u\in\lG_x$. We are going to show that $(\ad y)u\in\lG_x$. Let $f$ be the characteristic polynomial of $\ad x$:
\[
f(t)=\det\big( \ad x-t\mtu \big)
\]
where $\ad x$ and $\mtu$ are taken on $\lG_x$. Since $(\ad x)u=0$, $f(0)\neq 0$ and by the Cayley-Hamilton theorem, $f(\ad x)u=0$. Then
\begin{equation}
( f(\ad x)\ad y )u=(  f(\ad x)\ad y-(\ad y)f(\ad x)   )u,
\end{equation}
and, on the other hand, $\forall q\in\eN$,
\[
(\ad x)^q\ad y-\ad y(\ad x)^q=\sum_{r=0}^{q-1}(\ad x)^r(\ad[x,y])(\ad x)^{q-r-1}.
\]
It follows that $f(\ad x)\ad y-(\ad y)f(\ad x)$ is a linear combination of terms of the form
\[
(\ad x)^a(\ad[x,y])(\ad x)^b
\]
and the induction hypothesis shows that $f(\ad x)(\ad y)u\in\lG_x$.

Now we consider a $n$ such that $(\ad x)^n\lG^x=0$; the fact that $f(0)\neq 0$ implies the existence of polynomials $g(t)$ and $h(t)$ such that $g(t)t^n+h(t)f(t)=1$. If we decompose $(\ad y)u=v+w$ with respect to $\lG=\lG_x\oplus\lG^x$ we find
\begin{equation}
\begin{split}
(\ad y)u&=[ g(\ad x)(\ad x)^n+h(\ad x)f(\ad x) ](\ad y)u\\
&=f(\ad x)(\ad x)^nv+h(\ad x)f(\ad x)(\ad y)u\in\lG_x.
\end{split}
\end{equation}

\end{proof}

\begin{proposition}
Let $\lG$ be a Lie algebra and $x\in\lG$ such that $\lG^x$ is as small as possible. Then $\lG^x$ is a Cartan subalgebra.
\end{proposition}

\begin{proof}
From proposition~\ref{prop:G_x_central}, it is sufficient to prove that $\lG^x$ is nilpotent. Let $y\in\lG^x$ and $f_y(t)$ be the characteristic polynomial of $\ad y$. Since it is a subalgebra, $\lG^x$ is stable under $\ad y$ and proposition~\ref{prop:G_x_G_x} makes $\lG_x$ also stable under $\ad y$. Then $\ad y$ can be written under a bloc-diagonal form with respect to the decomposition $\lG=\lG_x\oplus\lG^x$, so that the characteristic polynomial can be factorised as
\begin{equation}
f_y(t)=g_y(t)h_y(t)
\end{equation}
where $g_y$ and $h_y$ are the characteristic polynomials of the restrictions of $\ad y$ to $\lG^x$ and $\lG_x$. Let $(y_1,\ldots,y_m)$ be a basis of $\lG^x$ and $t^n$, the greatest power of $t$ which divide all the $g_y(t)$ with $y\in\lG^x$. The coefficient of $t^n$ in $g_{c^iy_i}(t)$ is a polynomial with respect to the $c^i$ because of the expression
\[
g_{c^iy_i}(t)=\det\Big( \ad(c^iy_i)-t\mtu \Big).
\]
Let $u$ be this polynomial: $g_{c^iy_i}(t)=\ldots+u(c^1,\ldots,c^m)t^n$. By definition of $n$, this is not an identically zero polynomial and there are no terms with $t^{n-1}$. For the same reasons, we have a polynomial $v$ such that
\begin{equation}
h_{c^iy_i}(0)=v(c^1,\ldots,c^m).
\end{equation}
We know that none of the non-zero elements in $\lG_x$ are annihilated by $\ad x$ (because of the definition of $\lG^x$). Then $h_x(0)\neq 0$ and $v$ is not identically zero. With all this we can find some $c^i\in\eC$ such that $u(c^1,\ldots,c^m)v(c^1,\ldots,c^m)\neq 0$. If $y=c^iy_i$, the coefficient of $t^n$ in $f_y(t)$ is $u(c)v(c)\neq 0$, so that $f_y(t)$ is not divisible by $t^{n+1}$.

But in the other hand $\lG^x$ has minimal dimension, then $\dim\lG^y\geq m=\dim\lG^x$. Moreover $t^{\dim\lG^y}$ divide $f_y(t)$ because there is a certain power of $\ad y$ which has zero as eigenvalue with multiplicity $\dim\lG^y$\quext{This is not a good reason.}. Since $f_y(t)$ can not be divided by $t^{n+1}$ this shows that $n+1>\dim\lG^y$ and $n\geq\dim\lG^y\geq m$.

Now we consider $y$, any element of $\lG^x$. From the fact that $t^n$ divide all the $g_y(t)$ and that $n\geq m$, we see that $t^m$ divide $g_y(t)$. But the degree of $g_y(t)$ is $\dim\lG^x=m$. Finally, $g_y(t)=m$ and $\ad y$ is nilpotent on $\lG^x$ for any $y\in\lG^x$.

The Engel's theorem ~\ref{tho:Engel} makes $\lG^x$ nilpotent.
\end{proof}


The following holds for complex or real Lie algebras and comes from \cite{Sagle} see also \cite{SamelsonNotesLieAlg}. We denote by $\eK$ the base field of $\lG$, i.e. $\eR$ or $\eC$. For $X\in\lG$ and $\lambda\in\eK$ we define
\begin{equation}
\lG(X,\lambda)=\{Y\in\lG\tq (\ad X-\lambda\mtu)^nY=0\textrm{ for a certain $n\in\eN$}\}.
\end{equation}
A first useful result is given in
\begin{lemma}
If $Z\in\lG$, then
\[
[\lG(Z,\lambda),\lG(Z,\mu)]\subset\lG(Z,\lambda+\mu),
\]
in particular $\lH$ is a subalgebra of $\lG$.
\label{lem:lambda_mu_plus}
\end{lemma}

\begin{proof}
We consider $X_{\lambda}\in\lG(Z,\lambda)$ and $X_{\mu}\in\lG(Z,\mu)$. We have
\begin{equation}
\begin{split}
\big( \ad Z-(\lambda+\mu)I\big)[X_{\lambda},X_{\mu}]&=[ (\ad Z-\lambda I)X_{\lambda},X_{\mu} ]\\
            &\quad+[X_{\lambda}, (\ad Z-\mu I)X_{\mu} ].
\end{split}
\end{equation}
By induction,
\begin{equation}
\big( \ad Z-(\lambda+\mu)\mtu\big)^n[X_{\lambda},X_{\mu}]=\sum_{i=0}^{\infty}\binom{n}{i}
[ (\ad Z-\lambda I)^iX_{\lambda} ,(\ad Z-\mu I)^{n-i}X\mu].
\end{equation}
It will become zero for large enough $n$.
\end{proof}


An element $X\in\lG$ is \defe{regular}{regular} if $\dim\lG(X,0)$ is minimum\angl. This minimum is the \defe{rank}{rank of a Lie algebra} of $\lG$.

\begin{proposition}
    If $X\in\lG$ is a regular element then the algebra
    \begin{equation}
        \lH=\lG(X,0)=\{Y\in\lG\tq(\ad X)^nY=0\text{ for some }n\in\eN\}
    \end{equation}
    is nilpotent.
\end{proposition}

\begin{proof}
We have to show that for any $H\in\lH$, the endomorphism $\ad H$ of $\lH$ is nilpotent. Consider the characteristic polynomial of $\ad X$
\[
p(t)=\det(\ad X-t\mtu)=t^rq(t)
\]
where $t^r$ is the maximal factorization of $t$; in other words, $q(t)$ is not divisible by $t$ and $r=\dim\lH$. In particular
\begin{equation}
\lH=\{Y\in\lG\tq (\ad X)^rY=0\}.
\end{equation}
Let
\begin{equation}
\lK=\{Y\in\lG\tq q(\ad X)Y=0\}
\end{equation}
From the Cayley-Hamilton theorem (\ref{ThoCayleyHamilton}), $p(\ad X)=0$, then $(\ad X)^rq(\ad X)=0$ and $\lG=\lH\oplus\lK$. Moreover $\lH$ and $\lK$ are $\ad X$-invariants: $(\ad X)\lH\subseteq\lH$ and $(\ad X)\lK\subseteq\lK$.

Every weight of $\ad X$ are in $\eC$. As we know that $\lH$ is Cartan in $\lG$ if and only if $\lHeC$ is Cartan in $\lGeC$, we can suppose that $\lG$ is a complex algebra by considering $\lGeC$ if $\lG$ is real. So all the weight are in the base field and we can define
\[
\lK=\sum_{\lambda\in\Delta}\lG(X,\lambda).
\]
where $\Delta$ is the set of all the non zero weight of $\ad X$. A property\quext{Que je dois encore faire, cf Sagle} of the weight space is that
\[
\lG=\lG(X,\lambda_1)\oplus\ldots\oplus\lG(X,\lambda_m)
\]
if the $\lambda_i$'s are the weight of $\ad X$. Now we prove that $\sum_{\lambda\neq 0}\lG(X,\lambda)=\lK$. First consider a $Y\in\lG(X,\lambda)$ which can be decomposed as $Y=H+K$ with $H\in\lH$ and $K\in\lK$. Then $(\ad X-\lambda\mtu)^n(H+K)=(\ad X-\lambda\mtu)^nH+(\ad X-\lambda\mtu)^nK$ where the first term is not zero (because $H\in\lH$) and lies in $\lH$ while the second term lies in $\lK$. Then the sum can be zero only if $H=0$.

% Il faut continuer la preuve à partir du bas de /100: la démonstration de l'inclusion inverse.

\end{proof}

Let $\lG$ be a complex semisimple Lie algebra, $H\in\lG$ and $0=\lambda_0,\lambda_1,\ldots,\lambda_r$, the eigenvalues of $\ad H$. For any $\lambda\in\eC$, one can consider
\begin{equation}
\lG(H,\lambda)=\{X\in\lG\tq (\ad H-\lambda I)^kX=0\}.
\end{equation}
From the Jordan decomposition, $g(H,\lambda)=0$ except if $\lambda$ is one of the $\lambda_i$, and
\begin{equation}\label{eq:g_sum_g_H}
\lG=\bigoplus_{i=0}^{r}g(H,\lambda_i).
\end{equation}

An element $H\in\lG$ is \defe{regular}{regular} if
\[
\dim g(H,0)=\min_{X\in\lG}\dim g(X,0).
\]
Let $H_0$ be a regular element and $\lH=\lG(H_0,0)$.

\begin{lemma}
    The algebra $\lH=\lG(H_0,0)$ is nilpotent
\end{lemma}

\begin{proof}
Let $0=\lambda_0,\lambda_1,\ldots,\lambda_r$ be the eigenvalues of $\ad H_0$ and
\[
\lG'=\sum_{i=1}^{r} \lG(H_0,\lambda_i)
\]
which is a subspace of $\lG$. From the lemma,
\[
[\lG(H_0,0),\lG(H_0,\lambda_i)]\subset\lG(H_0,\lambda_i)\subset\lG'.
\]
For each $H\in\lH$, we denote $H'$, the restriction of $\ad H$ to $\lG'$ and $d(H)=\det H'$. The function $H\to d(H)$ is a polynomial on $\lH$ in the sense of the coordinates on $\lH$ as vector space. If $H'_0$ has a zero eigenvalue we would have $\ad(H_0)X=0$ for some $X\in\lG'$. In this case $[H_0,X]=0$, but $X\in\lG(H_0,\lambda_i)$, then for a certain $k$, $(\ad H_0-\lambda_i)^kX=0$, so that $\lambda_iX=0$. Since $\lG$ is defined by excluding $\lambda_0$, $X=0$. Thus $H_0'$ has only non zero eigenvalues and $d(H_0)\neq 0$.

We know that a polynomial which is zero on an open set is identically zero; then on any open set of $\lH$, $d$ has a non zero value somewhere. In particular,
\[
S=\{H\in\lH\tq d(H)\neq 0\}
\]
is dense in $\lH$. We consider a $H\in S$. The endomorphism $H'$ has only non zero eigenvalues, so that $\lG(H,0)\subset\lH$ from lemma~\ref{lem:lambda_mu_plus}; but $H_0$ is regular, then $\lG(H,0)\subset\lH$. Thus the restriction of $\ad H$ to $\lH$ is nilpotent because it is nilpotent on $\lG(H,0)$\quext{Ce paragraphe n'est pas vraiment clair\ldots}.

If $l=\dim\lH$, then $(\ad_{\lH}H)^l=0$ because $\ad_{\lH}H$ is nilpotent. By continuity, this equation is true for any $H\in\lH$ from the density of $S$ in $\lH$. Then $\lH$ is nilpotent.

\end{proof}


Here is an alternative proof (that I do not really understand) for theorem~\ref{TholGCartalphaplusbeta}.
\begin{theorem}
Let $\lG$ be a complex Lie algebra with Cartan subalgebra $\lH$. Then $\lG_0=\lH$.
\end{theorem}

\begin{proof}
Since $\lH$ is Cartan, it is nilpotent. So $\lH\subset\lG_0$. If $v\in\lG_0$, there exists a $n$ such that for any $z\in\lH$, $(\ad z)^nv=0$. The fact that $\lH$ is nilpotent makes $(\ad z_n)\circ\ldots\circ(\ad z_1)v=0$ for any $z\in\lG_0$ and for all $z_1,\ldots,z_n\in\lH$. If we write $(\ad z_1)v$ with $v\in\lG_0\setminus\lH$, we can always choose $z_1$ in order the result to \emph{not} be $\lH$. Next we can choose $z_2\in\lH$ such that $(\ad z_2)\circ(\ad z_1)v$ is also not in $\lH$ and so on\ldots Since $\lG_0$ is nilpotent, we always finish on zero. If $n$ is the maximum of adjoint that we can take before to fall into zero; we have
\[
[h, (\ad z_{n-1})\circ(\ad z_1)v ]=0
\]
for all $h\in\lH$ and with a good choice of $z_i$, it contradicts the fact that $\lH$ is Cartan.
\end{proof}
