% This is part of (almost) Everything I know in mathematics
% Copyright (c) 2013-2015, 2019
%   Laurent Claessens
% See the file fdl-1.3.txt for copying conditions.

Most of this chapter come (often very directly) from \cite{Helgason}. Other sources are \cite{Knapp_reprez,DirkEnvFiniteDimNilLieAlg,SamelsonNotesLieAlg,SternLieAlgebra}.

Do you know what is violet and commutative? Answer in the footnote\footnote{An abelian grape!}.

\section{Lie groups}
%+++++++++++++++++++++++++++++

\begin{definition}      \label{DEFooGDWTooTvINuw}
    A \defe{Lie group}{Lie!group}\index{group!Lie} is a manifold $G$ endowed with a group structure such that the inversion map $\dpt{i}{G}{G}$, $i(x)=x^{-1}$ and the multiplication $\dpt{m}{G\times G}{G}$, $m(x,y)=xy$ are differentiable. 
\end{definition}

 It is immediate to see that $g\mapsto g^{-1}$ is a smooth homeomorphism and that, for any fixed $g_0, g_1$, the maps
\[
\begin{split}
   g&\mapsto g_0g,\\
g&\mapsto gg_0,\\
g&\mapsto g_0gg_1
\end{split}
\]
are smooth homeomorphisms. When $A\subset G$, we define $A^{-1}=\{ g^{-1}\tq g\in G \}$.

\subsection{Connected component of Lie groups}
%---------------------------------------------

\begin{proposition}		\label{PropUssGpGenere} 
If $G$ is a connected Lie group and $\mU$, a neighbourhood of the identity $e$, then $G$ is generated by $\mU$ in the sense that $\forall g\in G$, there exists a \emph{finite} number of $g_{i}\in \mU$ such that
\[
  g=g_1\ldots g_n.
\]
Notice that the number $n$ is function of $g$ in general.
\end{proposition}

\begin{proof}
Eventually passing to a subset, we can suppose that $\mU$ is open. In this case, $\mU^{-1}$ is open because it is the image of $\mU$ under the homeomorphism $g\mapsto g^{-1}$. Now we consider $V=\mU\cap\mU^{-1}$. The main property of this set is that $V=V^{-1}$. Let
\[
  [V]=\{ g_1\ldots g_n\tq g_{i}\in V \};
\]
we will prove that $[V]=G$ by proving that it is closed and open in $G$ (the fact that $G$ is connected then concludes).

We begin by openness of $[V]$. Let $g_0=g_1\cdots g_n\in[V]$. We know that $g_0V$ is open because the multiplication by $g_0$ is an homeomorphism. It is clear that $g_0V\subset [V]$ and that $g_0=g_0e\in g_0V$. Hence $g_0\in g_0V\subseteq[V]$. It proves that $[V]$ is open because $g_0V$ is a neighbourhood of $g_0$ in $[V]$.

We now turn our attention to the closeness of $[V]$. Let $h\in\overline{ [V] }$. The set $hV$ is an open set which contains $h$ and $hV\cap [V]\neq \emptyset$ because an open which contains an element of the closure of a set intersects the set (it is almost the definition of the closure). Let $g_0\in hV\cap[V]$. There exists a $h_{1}\in V$ such that $g_0=hh_1$. For this $h_1$, we have $hh_1=g_0=g_1\cdots g_n$, and therefore
\[
  h=g_1\cdots g_n h_1^{-1}\in[V].
\]
This proves that $h\in[V]$ because $h_1^{-1}\in V$ from the fact that $V=V^{-1}$.
\end{proof}
 Remark that this proof emphasises the topological aspect of a Lie group: the differential structure was only used to prove thinks like that $A^{-1}$ is open when $A$ is open.

\begin{proposition}
Let $G$ be a Lie group and $G_0$, the identity component of $G$. We have the following:
\begin{enumerate}
\item $G_0$ is an open invariant subgroup of $G$,
\item $G_0$ is a Lie group,
\item the connected components of $G$ are lateral classes of $G_0$. More specifically, if $x$ belongs to the connected component $G_1$, then $G_1=xG_0=G_0x$.
\end{enumerate}

\end{proposition}

\begin{proof}
We know that when $M_{1}$ is open in the manifold $M$, one can put on $M_{1}$ a differential structure of manifold of same dimension as $M$ with the induced topology. Since $G_0$ is open, it is a smooth manifold. In order for $G_0$ to be a Lie group, we have to prove that it is stable under the inversion and that $gh\in G_0$ whenever $g$, $h\in G_0$.

First, $G_0^{-1}$ is connected because  it is homeomorphic to $G_0$ in $G$. The element $e$ belongs to the intersection of $G_0$ and $G_0^{-1}$, so $G_0\cup G_0^{-1}$ is connected as non-disjoint union of connected sets. Hence $G_0\cup G_0^{-1}=G_0$ and we conclude that $G_0^{-1}\subseteq G_0$. The set $G_0G_0$ is connected because it is the image of $G_0\times G_0$ under the multiplication map, but $e\in G_0G_0$, so $G_0G_0\subseteq G_0$ and  $G_0$ is thus closed for the multiplication. Hence $G_0$ is a Lie group.

For all $x\in G$, we have $e=xex^{-1}\in xG_0x^{-1}$, but $xG_0x^{-1}$ is connected. Hence $xG_0x^{-1}\subseteq G_0$, which proves that $G_0$ is an invariant subset of $G$.

Lateral classes $xG_0$ are connected because the left multiplication is an homeomorphism. They are moreover \emph{maximal} connected subsets because, if $xG_0\subset H$ (proper inclusion) with a connected $H$, then $G_0\subset x^{-1}H$ (still proper inclusion). But the definition of $G_0$ is that this proper inclusion is impossible. Therefore, the sets of the form $xG_0$ are maximally connected sets. It is clear that $\bigcup_{g\in G}gG_0=G$.

Notice that the last point works with $G_0x$ too.
\end{proof}

%+++++++++++++++++++++++++++++++++++++++++++++++++++++++++++++++++++++++++++++++++++++++++++++++++++++++++++++++++++++++++++
\section{Two words about Lie algebra}
%+++++++++++++++++++++++++++++++++++++++++++++++++++++++++++++++++++++++++++++++++++++++++++++++++++++++++++++++++++++++++++

\subsection{What is  \texorpdfstring{$g^{-1} dg$}{g-1dg}?}\label{SubSecgmudg}
%--------------------------------

The expression $g^{-1} dg$ is often written in the physical literature. In our framework, the way to gives a sense to this expression is to consider it pointwise acting on a tangent vector. More precisely, the framework is the data of a manifold $M$, a Lie group $G$ and a map $\dpt{g}{M}{G}$. Pointwise, we have to apply $g(x)^{-1} dg_x$ to a tangent vector $v\in T_xM$.

Note that $\dpt{dg_x}{T_xM}{T_{g(0)}G\neq T_eG}$, so $dg_x\notin \yG$. But the product $g(x)^{-1} dg_x v$ is defined by
\begin{equation}
	g(x)^{-1} dg_x v=\Dsdd{ g(x)^{-1} g(v(t)) }{t}{0}\in\yG.
\end{equation}


%---------------------------------------------------------------------------------------------------------------------------
\subsection{Adjoint map}
%---------------------------------------------------------------------------------------------------------------------------

The ideas of this short note comes from \cite{Lie}. A more traumatic definition of the adjoint group can be found in \cite{Helgason}, chapter II, \S 5. 

\begin{definition}
    Let $G$ be a Lie group, and $\mG$, its Lie algebra. We define the \defe{adjoint map}{adjoint!map} at the point $x\in G$ by
    \begin{equation}
        \begin{aligned}
            \AD_x\colon G&\to G \\
            \AD_xy&=xyx^{-1}
        \end{aligned}
    \end{equation}
\end{definition}

Then we define
\[
\dpt{Ad_x:=(d\AD_x)_e}{\mG}{\mG};
\]
the chain rule applied on $\AD_{xy}=\AD_x\circ\AD_y$ leads to $Ad_{xy}=Ad_x\circ Ad_y$, and thus we can see $Ad$ as a group homomorphism $\dpt{Ad}{G}{GL(\mG)}$, $Ad(x)=Ad_x$.

\begin{definition}
This homomorphism is the \defe{adjoint representation}{adjoint!representation!Lie group on its Lie algebra}\index{representation!adjoint} of the group $G$ in the vector space $\mG$.
\end{definition}


Finally, we define
\[
 \dpt{ad:=d(Ad)_1}{\mG}{L(\mG,\mG)}
\]
 where we identify $T_1GL(\mG)$ with $L(\mG,\mG)$.

\begin{lemma}\label{LEMooEALFooJOeOgk}
    If $\dpt{f}{G}{G}$ is an automorphism of $G$ (i.e.: $f(xy)=f(x)f(y)$), then $df_e$ is an automorphism of $\mG$: $df[X,Y]=[df X,dfY]$
\end{lemma}

\begin{proof}
First, remark that $f(\AD_xy)=\AD_{f(x)}f(y)$. Now, $\Ad_x X=(d\AD_x)_eX$, so that one can compute:
\begin{equation}
\begin{split}
   df(\Ad_xX)&=\Dsdd{f(\AD_xX(t))}{t}{0}\\
             &=\Dsdd{   \AD_{f(x)}f(X(t))  }{t}{0}\\
	     &=(d\AD_{f(x)})_{f(e)}df X\\
	     &=\Ad_{f(x)}df X.
\end{split}
\end{equation}
On the other hand, we need to understand how does the $\ad$ work.
\[
  \ad XY=\Dsdd{\Ad_{X(t)}}{t}{0}Y=\Dsdd{\Ad_{X(t)}Y}{t}{0}
\]
because $\dpt{\Ad_{X(t)}}{\mG}{\mG}$ is linear, so that $Y$ can enter the derivation (for this, we identify $\mG$ and $T_X\mG$). Since $\Ad_{X(t)}Y$ is a path in $\mG$ the \emph{true space} is
\[
(\ad X)Y=\Dsdd{ \Ad_{X(t)}Y }{t}{0}\in T_{[X,Y]}\mG\simeq\mG.
\]
For the same reason of linearity, $df$ can get in the derivative in the expression $df\Dsdd{  \Ad_{X(t)}Y  }{t}{0}$. Thus
\begin{equation}
\begin{split}
(\ad X)Y&=\Dsdd{  df\big(\Ad_{X(t)}Y\big)  }{t}{0}\\
        &=\Dsdd{  \Ad_{ f(X(t)) }df Y  }{t}{0}\\
	&=\Dsdd{ \Ad_{f(X(t))} }{t}{0}df Y\\
	&=\ad(dfX)df Y\\
	&=[dfX,df Y]
\end{split}
\end{equation}
because $f(X(t))$ is a path which gives $df X$.
\end{proof}

\begin{corollary}
An automorphism of a semisimple Lie group is an isometry for the Killing metric. Stated in other words,
    \begin{equation}\label{eq:Aut_Iso}
        \Aut(G)\subset\Iso G.
    \end{equation}
\end{corollary}

\begin{proof}
    By lemma~\ref{LEMooEALFooJOeOgk}, if $f$ is an automorphism of $G$, $df$ is an automorphism of $\mG$. Now, by proposition~\ref{prop:auto_2}, $f$ is an isometry of $G$.
\end{proof}

\begin{theorem}
The Killing form is bi-invariant\index{Killing!form!bi-invariance} on $G$.
\label{tho:bi_invariance}
\end{theorem}

\begin{proof}
Because of the left invariance,
\[
  B(dR_gX,dR_gY)=B(dL_{g^{-1}}dR_gX,dL_{g^{-1}}dR_gY)=B(\Ad_{g^{-1}}X,\Ad_{g^{-1}}Y).
\]
But $\Ad_{g^{-1}}=d(\AD_{g^{-1}})$ and $\AD_{g^{-1}}$ is an automorphism of $G$. Thus by lemma~\ref{LEMooEALFooJOeOgk} and proposition~\ref{prop:auto_2},
\begin{equation}                    \label{eq_KillAdinvariant}
B\big(\Ad(g^{-1})X,\Ad(g^{-1})Y\big)=B(X,Y).
\end{equation}

\end{proof}

One can show that $[X,Y]$ is tangent to the curve
\begin{equation}
  c(t)=e^{-\sqrt{s}X}e^{-\sqrt{s}Y}e^{\sqrt{s}X}e^{\sqrt{s}Y}.
\end{equation}

\begin{lemma}
	In the case of Lie algebra, the bracket is given by the derivative of the adjoint action:
	\begin{equation}
		\Dsdd{ \Ad( e^{tX})Y }{t}{0}=[X,Y]
	\end{equation}
\end{lemma}

\begin{proof}
	Let us make $[\tilde X,\tilde Y]_e$ act on a function $f$. Using the definition \eqref{EqDefLieDerivativeVect} and the property of theorem~\ref{ThoLieDerrComm}, we have
	\begin{equation}
		\begin{aligned}[]
			[\tilde X,\tilde Y]_ef&=\Dsdd{ (d\varphi_{-t}^X)\tilde Y }{t}{0}f\\
			&=\Dsdd{ (d\varphi_{-t}^X)_{\varphi_t^X(e)}\big( \tilde Y_{\varphi_t^X(e)} \big) }{t}{0}f\\
			&=\Dsdd{ \tilde Y_{ e^{tX}}\cdot(f\circ\varphi_{-t}^X) }{t}{0}
		\end{aligned}
	\end{equation}
	Now, we use the fact that, by definition, $\varphi_t^X(x)=x e^{tX}$, so that $\varphi_s^Y( e^{tX})= e^{tX} e^{sY}$ and we get
	\begin{equation}
		\begin{aligned}[]
			[\tilde X,\tilde Y]_ef&=\Dsdd{ \Dsdd{ f\big( \varphi_{-t}^X( e^{tX} e^{sY}) \big) }{s}{0} }{t}{0}\\
			&=\Dsdd { \Dsdd{ f( e^{tX} e^{sY} e^{-tX}) }{s}{0} } {t}{0}\\
			&=\Dsdd{ \Dsdd{ f\big(  e^{s\Ad( e^{tX})Y} \big) }{s}{0} }{t}{0}\\
			&=\Dsdd{ \big( \Ad( e^{tX})Y \big)_e\cdot f }{t}{0}
		\end{aligned}
	\end{equation}

\end{proof}


\section{Fundamental vector field}\label{sec:fond_vec}
%++++++++++++++++++++++++++++++++++++

If $\yG$ is the Lie algebra of a Lie group $G$ acting on a manifold $M$ (the action of $g$ on $x$ being denoted by $x\cdot g$), the \defe{fundamental vector field}{fundamental!vector field} associated with $A\in\yG$ is given by
\begin{equation}			\label{EqDefChmpFond}
   A^*_x=\Dsdd{ x\cdot e^{-tA} }{t}{0}.
\end{equation}
We always suppose that the action is effective. If the action of $G$ is transitive, the fundamental vectors at point $x\in M$ form a basis of $T_xM$. More precisely, we have the

\begin{lemma}
For any $v\in T_xM$, there exists a $A\in\yG$ such that $v=A^*_x$, in other terms
\[
  \Span\{ A^*_{x}\tq A\in\yG \}=T_{x}M.
\]
\label{LemFundSpansTan}
\end{lemma}

\begin{proof}
The vector $v$ is given by a path $v(t)$ in $M$. Since the action is transitive, one can write $v(t)=x\cdot c(t)$ for a certain path $c$ in $G$ which fulfills $c(0)=e$. We have to show that $v$ depends only on $c'(0)\in\yG$. We consider
\begin{equation}  \label{eq_def_RGM}
\begin{aligned}
 R\colon G\times M&\to M \\
R(g,x)&= x\cdot g,
\end{aligned}
\end{equation}
so
\begin{equation}\label{eq:v_Rc}
   v=\Dsdd{ R(c(t),x) }{t}{0}=dR_{(e,x)}\big[  (d_tc(t),x)+(c(0),x)   \big].
\end{equation}

\end{proof}



\begin{lemma}
If $A$, $B\in\yG$ are such that $A^*=B^*$, and if the action is effective, then $A=B$.
\label{lem:As_Bs_A_B}
\end{lemma}

\begin{proof}
 We consider once again the map \eqref{eq_def_RGM} and we look at
\[
  v=\Dsdd{ R(c(t),x) }{t}{0}
   =(dR)_{(e,x)}\Dsdd{ (c(t),x) }{t}{0},
\]
keeping in mind that $c(t)=e^{-tA}$. In order to treat this expression, we define
\begin{subequations}
\begin{align}
  \dpt{R_1}{G}{M},\quad  R_1(h)&=R(h,x),\\
  \dpt{R_2}{M}{M},\quad  R_2(y)&=R(g,y).
\end{align}
\end{subequations}
So
\[
  v=dR_1(X)+dR_2(0)=dR_1c'(0)
\]
and the assumption $A^*_x=B^*_x$ becomes $dR_1 A=dR_1 B$. This makes, for small enough $t$, $R_1(e^{tA}e^{-tB})=x\cdot e^{tA}e^{-tB}=x$; if the action is effective, it imposes $A=B$.
\end{proof}

\begin{lemma}
If we consider the action of a matrix group, $R_g$ acts on the fundamental field by
\[
  dR_g(A^*_{\xi})=\big( \Ad(g^{-1})A \big)^*_{\xi\cdot g}.
\]
\label{lem:dRgAstar}
\end{lemma}

\begin{proof}
Just notice that $e^{-t\Ad(g^{-1})A}=\AD_{g^{-1}}(e^{-tA})=g^{-1} e^{-tA}g$, thus
\begin{equation}
  \big( \Ad(g^{-1})A \big)^*_{\xi\cdot g}=\Dsdd{ \xi\cdot ge^{-t\Ad(g^{-1})A} }{t}{0}=dR_g(A^*_{\xi}).
\end{equation}
\end{proof}

%+++++++++++++++++++++++++++++++++++++++++++++++++++++++++++++++++++++++++++++++++++++++++++++++++++++++++++++++++++++++++++
\section{Exponential map}
%+++++++++++++++++++++++++++++++++++++++++++++++++++++++++++++++++++++++++++++++++++++++++++++++++++++++++++++++++++++++++++

\begin{definition}
    A \defe{topological group}{topological!group} is a group $G$ equipped with a topological structure such that the maps $(x,y)\in G^2\to xy\in G$ and $x\in G\to x^{-1}\in G$ are continuous.
\end{definition}

\begin{remark}\label{rem:ouvert}
From the existence of an unique inverse for any element of $G$, the multiplication and the inversion are also open maps.
\end{remark}

\begin{definition}
    A \defe{Lie group}{Lie!group} is a group $G$ which is in the same times an manifold such that the group operations (multiplication and inverse) are smooth.

    A Lie group is \defe{analytic}{analytic Lie group} if the manifold is analytic and the group operations are analytic.
\end{definition}

\subsection{Invariant vector fields}\index{invariant!vector field}
%-----------------------------------

\begin{definition}[\cite{BIBooUGWHooPbodCu}]
    If $G$ is a Lie group, a vector field $X\in\Gamma^{\infty}(TG)$ is \defe{left invariant}{left invariant!vector field} if
    \begin{equation}
        (dL_g) X= X,
    \end{equation}
    which means that for every \( g,h\in G\),
    \begin{equation}
        (dL_h)_gX_g=X_{hg}.
    \end{equation}
    In the same way, the vector field \( Y\) is \defe{right invariant}{right!invariant!vector field} if
    \begin{equation}
        (dR_g)Y=Y.
    \end{equation}
\end{definition}

When \( X\in T_eG\), we define the associated left-invariant vector field \( X^L\) by
\begin{equation}        \label{DEFooYPUIooAzcdjP}
    X^L_g=(dL_g)_eX.
\end{equation}

\begin{theorem}[\cite{BIBooUGWHooPbodCu}]
	The map \( \varphi\colon X\mapsto X^L\) where \( X^L_g=(dL_g)_eX\) is a bijection from \( T_eG\) to the set of left-invariant vector fields.
\end{theorem}

\begin{proof}
    Two parts.
    \begin{subproof}
        \item[Surjective]
            Let \( X\) be a left-invariant vector field. We have \( X=(X_e)^L\) because
            \begin{equation}
                (X_e)^L_g=(dL_g)X_e=X_g.
            \end{equation}
            The first equality is the definition of the left-invariant associated vector field (equation \eqref{DEFooYPUIooAzcdjP} applied to \( X_e\)) and the second equality is the fact that \( X\) is left-invariant. Thus \( X\) is the left-invariant vector field associated with \( X_e\).
        \item[Injective]
            Let \( X,Y\in T_eG\) be such that \( X^L=Y^L\). In particular \( X^L_e=Y^L_e\), which means \( X=Y\).
    \end{subproof}
\end{proof}

\begin{proposition}[\cite{BIBooUGWHooPbodCu, MonCerveau}]
    Let \( G\) be a Lie group. The map
    \begin{equation}
        \begin{aligned}
            \varphi\colon G\times \lG&\to TG \\
            (g,X)&\mapsto X^L_g 
        \end{aligned}
    \end{equation}
    is a bijection.

    Moreover for each \( g\in G\), the map
    \begin{equation}
        \begin{aligned}
            \varphi_g\colon \lG&\to T_gG \\
           X&\mapsto X^L_g 
        \end{aligned}
    \end{equation}
    is a vector space isomorphism.
\end{proposition}

\begin{proof}
    Several points.
    \begin{subproof}
        \item[\( \varphi\) is surjective]
            Let \( X\in TG\); there is some \( g\in G\) such that \( X\in T_gG\). Since \( X=(dL_g)_e(dL_{g^{-1}})_gX\) we have
            \begin{equation}
                X=(dL_{g^{-1}}X)^L_g=\varphi(g,dL_{g^{-1}}X).
            \end{equation}
        \item[\( \varphi\) is injective]
            If \( \varphi(g,X)=\varphi(h,Y)\), we have \( X_g^L=Y^L_h\), so that \( g=h\). The equality  \( X_g^L=Y_g^L\) means \( (dL_g)_eX=(dL_g)_eY\). Applying \( (dL_{g^{-1}})_g\) on both sides we get \( X=Y\).
        \item[\( \varphi_g\) is bijective]
            These are the same verifications.
        \item[\( \varphi_g\) is linear]
            The map \( \varphi_g\) is nothing else than \( (dL_g)_e\), so it is linear.
    \end{subproof}
\end{proof}

%--------------------------------------------------------------------------------------------------------------------------- 
\subsection{Flow and exponential}
%---------------------------------------------------------------------------------------------------------------------------

\begin{proposition} \label{PROPooUXFQooIwimav}
    Let \( \Phi\) be the flow of the left-invariant vector field \( X\). We have
    \begin{equation}
        \Phi(t,g)=g\Phi(t,e).
    \end{equation}
\end{proposition}

\begin{proposition}     \label{PROPooZHBOooGTLXsi}
    Let \( G\) be a Lie group and \( X\in\lG\) (the Lie algebra\footnote{Definition \ref{}.}).
    \begin{enumerate}
        \item
            There exists a unique \(  C^{\infty}\) group homomorphism \( h_X\colon (\eR,+)\to G\) such that \( \Dsdd{ h_x(t) }{t}{0}=X\).
        \item
            The path \( h_X\) is the maximal integral curve of \( X^L\) and \( X^R\) for the initial condition \( h_X(0)=e\).
        \item
            The flows of \( X^L\) and \( X^R\) are defined on \( \eR\).
    \end{enumerate}
\end{proposition}

\begin{definition}
    If \( G\) is a Lie group with algebra \( \lG\), we define the \defe{exponential}{exponential from a Lie algebra} is the map
    \begin{equation}
        \begin{aligned}
            \exp\colon \lG&\to G \\
            X&\mapsto h_X(1) 
        \end{aligned}
    \end{equation}
    where \( h_X\) is the homomorphism defined by the proposition \ref{PROPooZHBOooGTLXsi}. We often write \(  e^{X} \) for \( \exp(X)\).
\end{definition}


The following proposition is a generalization of \ref{PROPooKDKDooCUpGzE}.
\begin{proposition}     \label{PROPooNRVJooEDCpOI}
    If \( X\in \lG\) and \( s,t\in \eR\) we have
    \begin{equation}
        e^{sX} e^{tX}= e^{(s+t)X}.
    \end{equation}
\end{proposition}

\begin{lemma}       \label{LEMooLMTZooCvunSl}
    Let \( G\) be a Lie group and \( X\in G\). We have
    \begin{equation}        \label{EQooNBENooPXLENs}
        X^R_g=\Dsdd{  e^{tX}g }{t}{0}
    \end{equation}
    and
    \begin{equation}
        X^L_g=\Dsdd{  ge^{tX} }{t}{0}
    \end{equation}
\end{lemma}

\begin{normaltext}      \label{NORMooSATDooIhwXXr}
    We will often write the relation \eqref{EQooNBENooPXLENs} under the form
    \begin{equation}
        X^R_g(t)= e^{tX}g.
    \end{equation}
    This is a way to implies that \( t\mapsto  e^{tX}g\) is a path for the vector \( X^R_g\). It is a common abuse of notation to write the vector and a path representing the vector with the same symbol.
\end{normaltext}

\begin{proposition}     \label{PROPooYFZZooLUOuOj}
    Let \( G\) be a Lie group. There exists a neighbourhood \( U\) of \( 0\) in \( \lG\) and a neighbourhood \( V\) of \( e\) in \( G\) such that
    \begin{equation}
        \exp\colon U\to V
    \end{equation}
    is a \(  C^{\infty}\) diffeomorphism\footnote{\( \exp\) is \(  C^{\infty}\), invertible and he inverse is \(  C^{\infty}\) as well.}.
\end{proposition}

\begin{proposition}     \label{PROPooAICDooQcmPZB}
    Let \( G\) be an analytic Lie group. There exists a neighbourhood \( U\) of \( 0\) in \( \lG\) and a neighbourhood \( V\) of \( e\) in \( G\) such that
    \begin{equation}
        \exp\colon U\to V
    \end{equation}
    is an analytic diffeomorphism\footnote{\( \exp\) is analytic, invertible and he inverse is analytic too.}.
\end{proposition}


%--------------------------------------------------------------------------------------------------------------------------- 
\subsection{Invariant vector and derivation}
%---------------------------------------------------------------------------------------------------------------------------

You may want to know how the exponential can be used to write some formulas linking left-invariant vector field and derivation of functions. Here you are.

\begin{normaltext}
    Let \( X\in \lG\), \( g\in G\) and \( u\in \eR\). Let \( f\colon G\to \eR\) be a smooth function. Using the abuse of notation described in \ref{NORMooSATDooIhwXXr} and the proposition \ref{PROPooNRVJooEDCpOI},
    \begin{subequations}
        \begin{align}
            (X^Lf)(g e^{uX})&=\Dsdd{ f\big( X^L_{g e^{uX}}(t) \big) }{t}{0}\\
            &=\Dsdd{ f\big( g e^{uX} e^{tX} \big) }{t}{0}\\
            &=\Dsdd{ f\big( g e^{(t+u)X)} \big)}{t}{0}\\
            &=\Dsdd{ f(g e^{tX}) }{t}{u}.
        \end{align}
    \end{subequations}
    The formula
    \begin{equation}
        (X^Lf)(g e^{uX})=\Dsdd{ f(g e^{tX}) }{t}{u}
    \end{equation}
    means that \( X^L\) derives \( f\) in the direction of the path \(  e^{tX}\) at right.
\end{normaltext}

\begin{normaltext}
    By the way, we recall that, if \( f\) is a function and \( X\) a vector field, \( (Xf)\) is a new function, given by
    \begin{equation}
        (Xf)(a)=X_a(f).
    \end{equation}
    In that sense we can write combinations like \( XYf\) or \( (X^2+X)f\) where \( X\) and \( Y\) are vector fields.
\end{normaltext}

\begin{proposition}[\cite{BIBooPBAMooNcYhCM}]       \label{PROPooKSIDooVIFkiM}
    Let \( G\) be a Lie group with Lie algebra \( \lG\). We consider \( X,Y\in \lG\) and a smooth function \( f\colon G\to \eR\). We have\quext{My source \cite{BIBooPBAMooNcYhCM} seems to write \( (X^R)^n(Y^R)^m\) instead of \( (X^R)^n(Y^L)^m\). Let me know where I'm wrong.}
    \begin{equation}
        \big( (X^R)^n(Y^L)^mf \big)( e^{sX} e^{tY})=\frac{ d^n }{ du^n }\frac{ d^m }{ dv^m }\Big( f( e^{uX} e^{vY}) \Big)_{\substack{u=s\\v=t}}.
    \end{equation}
\end{proposition}

\begin{proof}
    We have to do a proof by induction on \( (n,m)\). We start with \( (n,m)=(0,0)\) and we prove the steps \( (n,m)\to (n+1,m)\) and \( (n,m)\to (n,m+1)\).

    \begin{subproof}
        \item[\( (0,0)\)]
            With \( (n,m)=(0,0)\) we are okay.
        \item[\( (n+1,m)\)]
            We have
            \begin{equation}
                \Big( (X^R)^{n+1}(Y^L)^mf \Big)( e^{sX} e^{tY})=\big( (X^R)(X^R)^n(Y^L)^mf \big)( e^{sX} e^{tY}).
            \end{equation}
            We will apply the induction hypothesis on the function \( (X^R)^n(Y^L)^mf\), but in a first time we just apply the vector field \( X^R\) to the function \( (X^R)^n(Y^L)^m\) and we evaluate at \(  e^{sX} e^{tY}\). Here is a couple of computations:
            \begin{subequations}
                \begin{align}
                    \Big( (X^R)(X^R)^n(Y^L)^mf \Big)( e^{sX} e^{tY})&=\Dsdd{  \Big( (X^R)^n(Y^L)^mf \Big)\big( X^R_{ e^{sX} e^{tY}}(u) \big)  }{u}{0}\\
                    &=\Dsdd{  \Big( (X^R)^n(Y^L)^mf \Big)(  e^{uX} e^{sX} e^{tY} )  }{u}{0}\\
                    &=\Dsdd{  \Big( (X^R)^n(Y^L)^mf \Big)( e^{uX} e^{tY})  }{u}{s}.
                \end{align}
            \end{subequations}
            At this point we use the induction hypothesis:
            \begin{subequations}
                \begin{align}
                    \Dsdd{  \Big( (X^R)^n(Y^L)^mf \Big)( e^{uX} e^{tY})  }{u}{s}&=\frac{ d }{ du }\left( \frac{ d^n }{ dw^n }\frac{ d^m }{ dv^m }\big( f( e^{wX} e^{vY}) \big)_{\substack{w=u\\v=t}}  \right)_{u=s}\\
                    &=\frac{ d^{n+1} }{ dw^{n+1} }\frac{ d^m }{ dv^m }\left( f( e^{wX} e^{vY}) \right)_{\substack{w=s\\v=t}}.
                \end{align}
            \end{subequations}
        \item[\( (n,m+1)\)]
            Same kind of computations.
    \end{subproof}
\end{proof}

%--------------------------------------------------------------------------------------------------------------------------- 
\subsection{Analytic Lie group, Taylor formula}
%---------------------------------------------------------------------------------------------------------------------------

In this subsection we study the analytic functions over an analytic Lie group.

\begin{lemma}[\cite{BIBooPBAMooNcYhCM}]     \label{LEMooPILVooHQbtAH}
    Let \( G\) be an analytic Lie group. We consider an analytic function \( f\colon G\to \eR\), an element \( X\in \lG\), a basis \( \{ X_i \}\) of \( \lG\) and \( g\in G\). There exists an absolutely converging power series \( P\) such that
    \begin{equation}
        f(g e^{x_1X_1+\ldots +x_nX_n})=P(x_1,\ldots, x_n).
    \end{equation}
\end{lemma}

\begin{proof}
    First we make the proof for \( g=e\).

    We consider a basis \( \{ e_i \}\) of \( \lG\). Let \( U\) be a neighbourhood of \( 0\) in \( \lG\) and \( V\) a neighbourhood of \( e\) in \( G\) such that \( \exp\colon U\to V\) is an analytic diffeomorphism\footnote{By proposition \ref{PROPooAICDooQcmPZB}.}.

    We consider \( U'\), the open set in \( \eR^n\) which correspond to \( U\) via the basis \( \{ e_i \}\). The map
    \begin{equation}
        \begin{aligned}
            \varphi\colon U'&\to V \\
            (x_1,\ldots, x_n)&\mapsto \exp(x_1e_1+\ldots+x_ne_n)
        \end{aligned}
    \end{equation}
    is analytic chart of \( V\).

    The fact that \( f\) is analytic means that the composition of \( f\) with the charts are analytic. In our case, the map \( \tilde f =f\circ\varphi\) is analytic from \( U'\subset \eR^n\) to \( \eR\). Thus there exists an absolutely converging power series \( P\) such that
    \begin{equation}
        \tilde f(x_1,\ldots, x_n)=P(x_1,\ldots, x_n).
    \end{equation}
    We conclude:
    \begin{equation}
        f\big( \exp(x_1e_1+\ldots +x_ne_n) \big)=f\big( \varphi(x_1,\ldots, x_n) \big)=P(x_1,\ldots, x_n).
    \end{equation}
    
    If \( g\) is not \( e\), we consider the neighbourhood \( gV\) and the map
    \begin{equation}
        \begin{aligned}
            \varphi\colon U&\to gV \\
            (x_1,\ldots, x_n)&\mapsto g\exp(x_1e_1+\ldots +x_ne_n)
        \end{aligned}
    \end{equation}
    is a chart, so that
    \begin{equation}
        f(g e^{x_1e_1+\ldots +x_ne_n})=\tilde f(x_1,\ldots, x_n)
    \end{equation}
    which is a power series.
\end{proof}

\begin{proposition}[Taylor formula\cite{BIBooPBAMooNcYhCM}]     \label{PROPooIYWQooZJtKiu}
    Let \( G\) be an analytic Lie group. We suppose that \( f\colon G\to \eR\) is an analytic functions. For \( g\in G\) and \( X\in \lG\) we have
    \begin{equation}
        f(g e^{X})=\sum_{k=0}^{\infty}\frac{1}{ n! }\big( (X^R)^nf \big)(g).
    \end{equation}
\end{proposition}

\begin{proof}
    We know from proposition \ref{LEMooPILVooHQbtAH} that \( f(g e^{X})=P(x_1,\ldots, x_n)\) for some power series \( P\). We consider a neighbourhood \( U\) of \( 0\) in \( \lG\) and \( V\) of \( g\) in \( G\) such that
    \begin{equation}
        \begin{aligned}
            \varphi\colon U&\to V \\
            X&\mapsto  ge^{X} 
        \end{aligned}
    \end{equation}
is an analytic diffeomorphism (i.e. an analytic chart for \( G\) around \( g\)). Let \( X\in U\) and \( \delta\) such that \( tX\in U\) for all \( t\in \mathopen] -\delta , \delta \mathclose[\). Notice that \( \delta>1\). Now, \( X\) being fixed, the value of \( P(tx_1,\ldots, tx_n)\) is an absolutely convergent power series of \( t\). We have
    \begin{equation}
        f(g e^{tX})=P(tx_1,\ldots, tx_n)=\sum_{k=0}^{\infty}\frac{ a_m }{ m! }t^m
    \end{equation}
    for some constants \( a_m\in \eR\).

    But considering the function
    \begin{equation}
        \begin{aligned}
            r\colon \eR&\to \eR \\
            t&\mapsto f(g e^{tX}), 
        \end{aligned}
    \end{equation}
    there is an unicity of its power series expansion; thus \( a_m\) is the \( m\)-th derivative of \( r\) at \( t=0\).

    But we also know from proposition \ref{PROPooKSIDooVIFkiM} that
    \begin{equation}
        \big( (X^L)^mf \big)(g e^{tX})=\frac{ d^m }{ du^m }\big( f(g e^{uX}) \big)_{u=t};
    \end{equation}
    taking that at \( t=0\) we have
    \begin{equation}
        a_m=\big( (X^L)^mf \big)(g)
    \end{equation}
    and the Taylor formula
    \begin{equation}
        f(g e^{tX})=\sum_{k=0}^{\infty}\frac{1}{ k! }\frac{ d^k }{ du^k }\big( f(g e^{uX}) \big)_{u=0}t^m.
    \end{equation}
    Finally taking \( t=1\) (recall that \( \delta>1\), so it is valid):
    \begin{equation}
        f(g e^{X})=\sum_{k=0}^{\infty}\frac{1}{ k! }\frac{1}{ k! }\big( (X^L)^kf \big)(g).
    \end{equation}
\end{proof}

\begin{lemma}       \label{LEMooWKFIooRHsrFX}
    Let \( G\) be an analytic Lie group with algebra \( \lG\). We consider a basis \( \{ e_i \}_{i=1,\ldots, n}\) of \( \lG\) and the functions
    \begin{equation}
        \begin{aligned}
            f_i\colon U&\to \eR \\
            \exp(x_1e_1+\ldots+x_ne_n)&\mapsto x_i 
        \end{aligned}
    \end{equation}
    defined on a normal neighbourhood \( U\) of \( e\).
    
    If \( X,Y\in \lG\) satisfy
    \begin{equation}
        Xf_i=Yf_i
    \end{equation}
    for every \( i\), then \( X=Y\).
\end{lemma}

\begin{proof}
    If \( X=\sum_kX_ke_k\) we have
    \begin{equation}
        X(f_i)=\Dsdd{ f_i( e^{tX}) }{t}{0}=\Dsdd{ f_i\big(  e^{t\sum_kX_ke_k} \big) }{t}{0}=\Dsdd{ tX_i }{t}{0}=X_i.
    \end{equation}
\end{proof}

\begin{lemma}[\cite{BIBooPBAMooNcYhCM}]
    Let \( G\) be an analytic Lie group with Lie algebra \( \lG\). For \( X,Y\in \lG\) we have:
    \begin{enumerate}
        \item       \label{ITEMooHVOIooKDrUSw}
            \( \exp(tX)\exp(tY)=\exp\big( t(x+Y)+\frac{ t^2 }{2}[X,Y]+t^2\alpha(t) \big)\),
        \item       \label{ITEMooWIQIooHphJcP}
            \( \exp\big( t(X+Y) \big)=\exp(tX)\exp(tY)\exp(t\alpha(t))\)
        \item       \label{ITEMooVMDCooExpIrp}
            \( \exp(-tX)\exp(-tY)\exp(tX)\exp(tY)=\exp\big( t^2[X,Y]+t^3\alpha(t) \big)\).
    \end{enumerate}
    In both formulas, \( \alpha\) is a function \( \alpha\colon \eR\to \lG\) satisfying \( \lim_{t\to 0} \alpha(t)=0\).
\end{lemma}

\begin{proof}
    Several steps.
    \begin{subproof}
        \item[A good function]
        
            Let \( \{ e_i \}_{i=1,\ldots, n}\) be a basis of \( \lG\). We consider a neighbourhood \( U\) of \( 0\) in \( \lG\) and \( V\) of \( e\) in \( G\) such that \( \exp\colon U\to V\) is an analytic diffeomorphism. On that \( U\) we consider the function
            \begin{equation}
                \begin{aligned}
                    f\colon U&\to \eR \\
                    \exp(x_1e_1+\ldots +x_ne_n)&\mapsto x_i 
                \end{aligned}
            \end{equation}
            for some fixed \( i\). This function is analytic and satisfies \( f(e)=0\). 
        \item[Some Taylor expansions] 
            Using proposition \ref{PROPooKSIDooVIFkiM} we have
            \begin{equation}
                \big( (X^R)^n(X^L)^mf \big)( e^{sX} e^{tY})=\frac{ d^n }{ du^n }\frac{ d^m }{ dv^m }\big( f( e^{uX} e^{vY}) \big)_{\substack{u=s\\v=t}}.
            \end{equation}
            Considering the function \( q(s,t)=f( e^{sX} e^{tY})\), we have the Taylor expansion
            \begin{equation}        \label{EQooNBOIooRxlZmP}
                f( e^{sX} e^{tY})=q(s,t)=\sum_{m,n\geq 0}\frac{ s^n }{ n! }\frac{ t^m }{ m! }\big( (X^R)^n(Y^L)^mf \big)(e)=\sum_{m,n\geq 0}\frac{ s^n }{ n! }\frac{ t^m }{ m! }\big( X^nY^mf \big)(e).
            \end{equation}
            Here the second equality is due to the fact that \( (X^Lf)(e)=(X^Rf)(e)=X(f)\).

        \item[The function \( Z\)]

            On the other hand, when \( t\) is small enough, the element \(  e^{tX} e^{tY}\) belongs to a normal neighbourhood of \( e\), so that there exists an element \( Z(t)\in \lG\) satisfying
            \begin{equation}
                e^{tX} e^{tY}= e^{Z(t)}.
            \end{equation}
            The element \( Z(t)\) is given by
            \begin{equation}
                Z(t)=\exp^{-1}\big(  e^{tX} e^{tY} \big).
            \end{equation}
            Since the exponential is an analytic diffeomorphism\footnote{Proposition \ref{PROPooAICDooQcmPZB}.} (the inverse is analytic), \( Z\) is an analytic function around \( t=0\). Thus there exists a function \( \alpha\colon \eR\to \lG\) such that
            \begin{equation}        \label{EQooRPGGooXtZzFy}
                Z(t)=tZ_1+t^2Z_2+t^2\alpha(t)
            \end{equation}
            and \( \lim_{t\to 0} \alpha(t)=0\). Notice that \( Z(0)=0\), which explain the absence of constant term in \eqref{EQooRPGGooXtZzFy}.

        \item[A formula for \( f\big(  e^{Z(t)} \big)\)]

            We pose \( Z_1=\sum_ka_{1k}e_k\), \( Z_2=\sum_ka_{2k}e_k\) and \( \alpha(t)=\sum_k\sigma_k(t)e_k\), so that
            \begin{equation}
                Z(t)=\sum_k\big( ta_{1k}+t^2a_{2k}+t^2\alpha_k(t) \big)e_k.
            \end{equation}
            Applying \( f\) we have
            \begin{equation}
                f\big(  e^{Z(t)} \big)=ta_{1i}+t^2a_{2i}+t^2\alpha_i(t)=f\big(  e^{tZ_1+t^2Z_2} \big)+t^2\alpha_i(t).
            \end{equation}
            
        \item[Some more Taylor expansions]

            We use the Taylor expansion of proposition \ref{PROPooIYWQooZJtKiu} with \( g=e\) and \( X=Z(t)\):
            \begin{equation}        \label{EQooSFKOooDAavVy}
                f( e^{Z(t)})=\sum_k\frac{1}{ k! }\big( [tZ^L_1+t^2Z_2^L]^kf \big)(e)+t^2\alpha_i(t).
            \end{equation}
            Once again we can drop the \( L\) exponent since \( (X^Lf)(e)=X(f)\). We collect out of \eqref{EQooSFKOooDAavVy} the terms with \( t\) and \( t^2\):
            \begin{equation}        \label{EQooEYUSooTDntym}
                f( e^{tX} e^{tY})=f( e^{Z(t)})=tZ_1(f)+t^2 Z_2 +\frac{ t^2 }{2}Z_1^2 +t^2\beta(t)
            \end{equation}
            with \( \lim_{t\to 0} \beta(t)=0\).

        \item[Comparison]

            The formulas \eqref{EQooNBOIooRxlZmP} with \( s=t\) and \eqref{EQooEYUSooTDntym} are Taylor expansions of the same quantity. They are equal; we copy them here:
            \begin{equation}
                \sum_{m,n}\frac{ t^{m+n} }{ m!n! }\big( (X^R)^n(Y^L)^mf \big)(e)=tZ_1(f)+t^2 Z_2 +\frac{ t^2 }{2}Z_1^2 +t^2\beta(t)
            \end{equation}
            On the left hand side, the terms with \( t\) and \( t^2\) are obtained with \( (n,m)=(0,1),(1,0), (2,0), (0,2), (1,1)\). Collecting we have on the left
            \begin{equation}
                (X+Y)f+XYf+\frac{ 1 }{2}X^2f+\frac{ 1 }{2}Y^2f
            \end{equation}
            where we used the fact that \( \big( (X^R)^2f \big)(e)=X(Xf)=X^2f\).

            Using lemma \ref{LEMooWKFIooRHsrFX} we have \( Z_1f=(X+Y)f\), so that \( Z_1=X+Y\) and then
            \begin{equation}
                \frac{ 1 }{2}[X,Y]=Z_2.
            \end{equation}
    \end{subproof}
    At this point we proved that
    \begin{equation}
        e^{tX} e^{tY}= e^{t(X+Y)+\frac{ t^2 }{2}[X,Y]+t^2\alpha(t)}.
    \end{equation}
    This is \ref{ITEMooHVOIooKDrUSw}.

    For point \ref{ITEMooWIQIooHphJcP}, we are searching for a function \( \beta\) such that 
    \begin{equation}
        e^{tX} e^{tY} e^{t\beta(t)}= e^{t(X+Y)}.
    \end{equation}
    We replace in the left-hand side the value of \(  e^{tX} e^{tY}\) given by the point \ref{ITEMooHVOIooKDrUSw} (this is the reason why we write \( \beta\) instead of \( \alpha\)) and we isolate \(  e^{t\beta(t)}\):
    \begin{equation}        \label{EQooLTMBooVIChyC}
        e^{t\beta(t)}= e^{t(X+Y)} e^{-t(X+Y)-t^2[X,Y]/2-t^2\alpha(t)}.
    \end{equation}
    So now our aim is to show that the right-hand side of \eqref{EQooLTMBooVIChyC} can be written as only one exponential with an argument of the form \( t\beta(t)\) satisfying \( \beta(t)\to 0\). For that, we use \ref{ITEMooHVOIooKDrUSw} once again with \( X+Y\) instead of \( X\) and \( -(X+Y)-t[X,Y]/2-t\alpha(t)\) instead of \( Y\). What we get is
    \begin{subequations}
        \begin{align}
            e^{t\beta(t)}&=\exp\big( t(-t[X,Y]/3-t\alpha(t))+\frac{ t^2 }{2}\big[ X+Y,-(X+Y)-t[X,Y]/2-t\alpha(t) \big] \big)\\
            &=\exp\big( -\frac{ t^2 }{2}[X,Y]  -t^2\alpha(t)-\frac{ t^3 }{ 4 }\big[ X+Y,[X,Y] \big]-\frac{ t^3 }{ 2 }\alpha(t)  \big).
        \end{align}
    \end{subequations}
    We are done with \ref{ITEMooWIQIooHphJcP}.
    

et

    \ref{ITEMooVMDCooExpIrp}

\end{proof}


%--------------------------------------------------------------------------------------------------------------------------- 
\subsection{Other stuff}
%---------------------------------------------------------------------------------------------------------------------------

The concept of normal neighbourhood will be widely used for the study of the relations between a Lie group and its algebra. Let $M$ be a differentiable manifold. If $V$ is a neighbourhood of zero in $T_pM$ on which the exponential $\dpt{\exp_p}{T_pM}{M}$ is a diffeomorphism, then $\exp_pV$ is  \defe{normal neighbourhood}{normal!neighbourhood} of $p$.

\begin{lemma}
Let $\lG$ be a Lie algebra and $A$, a linear operator on $\lG$ (see as a common vector space) such that $\forall t\in\eR$, the map $e^{tA}$ is an automorphism of $\lG$. Then $A$ is a derivation of $\lG$.
\label{lem:autom_derr}
\end{lemma}

\begin{proof}
Let us consider $X$, $Y\in\lG$;  the assumption is
\[
  e^{tA}[X,Y]=[e^{tA}X,e^{tA}Y].
\]
Since $e^{tA}$ is a linear map, it has a ``good behavior''\ with the derivations:
\[
\Dsddc{e^{tA}[X,Y]}{t}{0}=\Dsddc{e^{tA}}{t}{0}[X,Y]=A[X,Y].
\]
Using on the other hand the linearity of $\ad$, we can see
\[
  (\ad(e^{tA}X))(e^{tA}Y)
\]
as a product ``matrix times vector''. Then
\begin{equation}
\begin{split}
  \Dsddc{[e^{tA}X,e^{tA}Y]}{t}{0}&=\Dsddc{(\ad e^{tA}X)Y}{t}{0}+\Dsddc{(\ad X)(e^{tA}Y)}{t}{0}\\
                                 &=(\ad AX)Y+(\ad X)(AY).
\end{split}
\end{equation}
Finally, $A[X,Y]=[AX,Y]+[X,AY]$.

\end{proof}

As notational convention, if $G$ and $H$ are Lie groups, their Lie algebra are denoted by $\lG$ and $\lH$.

\begin{lemma}		\label{LemAlgEtGroupesGenere}
	Let $\lG$ be a Lie algebra ans $\lS$ be a subset of $\lG$. The algebra of the group generated by $ e^{\lS}$ is the algebra generated by $\lS$.
\end{lemma}

Invariant vector fields are also often used in order to transport a structure from the identity of a Lie group to the whole group by $A_g(X_g)=A_e(dL_{g^{-1}}X_g)$ where $A_e$ is some structure and $X_g$, a vector at $g$.


\begin{proposition}
	Let $G$ be a Lie group and $\mG$ the vector space of its left invariant vector fields.
	\begin{enumerate}

		\item
			The map
			\begin{equation}
				\begin{aligned}
					\mG\colon &\to T_eG \\
					\tilde X&\mapsto \tilde X_e
				\end{aligned}
			\end{equation}
			is a vector space isomorphism.
		\item
			We have $[\mG,\mG]\subset \mG$ and $\mG$ is a Lie algebra. Here, the commutator is the bracket of vector fields.

	\end{enumerate}
\end{proposition}
\begin{proof}
	No proof.
\end{proof}

A vector field \( X\) on a Lie group \( G\) is \defe{left invariant}{left invariant!vector field} if \( dL_g(X)=X\) for every \( g\in G\). Here \( L_g\colon G\to G\) is the left translation defined by \( L_g(h)=gh\). More explicitly, the left invariance is expressed by
\begin{equation}
    \Dsdd{ gX_h(t) }{t}{0}=X_{gh}
\end{equation}
where \( X_h(t)\) is the path defining the tangent vector \( X_h\in T_h G\).

We want to prove that the vector space of left invariant vector fields is isomorphic to the tangent vector space \( T_eG\) to \( G\) at identity. If \( X\in T_eG\), we introduce the left invariant vector field \( X^L=dLX\), more explicitly:
\begin{equation}
    X^L_g=\Dsdd{ gX(t) }{t}{0}.
\end{equation}
Then we consider \( \alpha_X\colon I\to G\) the integral curve of maximal length to \( X^L\) trough \( X_e\). Here, \( I\) is the interval on which \( \alpha_X\) is defined. This is the solution of
\begin{subequations}
    \begin{numcases}{}
        \Dsdd{ \alpha_X(t_0+t) }{t}{0}=X_{\alpha_X(t_0)}\\
        \alpha_X(0)=e.
    \end{numcases}
\end{subequations}

\begin{proposition}     \label{PROPooWEYCooCvyHNr}
    Let \( X\in T_eG\). The integral curve has \( \eR\) as domain and for every \( s,t\in\eR\),
    \begin{equation}
        \alpha_X(s+t)=\alpha_X(s)\alpha_X(t).
    \end{equation}
\end{proposition}

\begin{proof}
    Let \( \alpha\) be any integral curve for \( X^L\) and \( y\in G\). If we put \( \alpha_1(t)=y\alpha(t)\), we have
    \begin{equation}
        \Dsdd{ \alpha_1(t) }{t}{0}=X^L_y,
    \end{equation}
    so that \( \alpha_1\) is an integral curve for \( X^L\) trough the point \( y\).

    Let now \( I\) be the maximal domain of \( \alpha_X\), and \( t_1\in I\). If we set \( x_1=\alpha_X(t_1)\), the path
    \begin{equation}
         \alpha_1(t)=x_1\alpha_X(t)
    \end{equation}
    is an integral curve of \( X^L\) trough \( x_1\) and has the same maximal definition domain \( I\). On the other hand, the maximal integral curve starting at \( e\) being \( \alpha_X\), the maximal integral curve starting at \( \alpha_X(t_1)\) is
    \begin{equation}
        \alpha_2\colon t\mapsto \alpha_X(t+t_1).
    \end{equation}
    Its domain is \( I-t_1\), but since it starts at \( x_1\), it has to be the same as \( \alpha_1\), then \( I\subset I-t_1\) which proves that \( I=\eR\).

    For each \( s\) and \( t\) in \( \eR\), the maximal integral curve starting at \( \alpha_X(s)\) can be written as
    \begin{equation}
        c(t)=\alpha_X(s)\alpha_X(t)
    \end{equation}
    as well as
    \begin{equation}
        d(t)=\alpha_X(s+t),
    \end{equation}
    so again by unicity, \( \alpha_X(s+t)=\alpha_X(s)\alpha_X(t)\).
\end{proof}


%---------------------------------------------------------------------------------------------------------------------------
\subsection{Integral curve and exponential}
%---------------------------------------------------------------------------------------------------------------------------

\begin{definition}
    The \defe{exponential}{exponential!on Lie group} is the map
    \begin{equation}        \label{EqdefExpoLieTgFGp}
        \begin{aligned}
            \exp\colon T_eG&\to G \\
            X&\mapsto \alpha_X(1).
        \end{aligned}
    \end{equation}
    where \( \alpha_X\) is the integral curve to \( X^L\) such that \( \alpha_X(0)=e\).
\end{definition}

This definition works on Lie groups thanks to the group structure that allows to build a natural vector field \( X^L\) from the data of a single vector \( X\). On general manifolds, one has not a notion of exponential. However, if one has a Riemannian manifold, one consider the geodesic.

In the case of groups for which the Killing form defines a scalar product, the notion of exponential associated with the Riemannian structure propagated from the Killing form coincides with the definition \eqref{EqdefExpoLieTgFGp}.

A very important point\cite{ooOLNIooDLmxkR} is that when \( G\) is acting on $M$, one can reconstruct the action of \( G\) only knowing the action of \( \lG\). Let \( X\in \lG\) and \( x\in M\). We consider the path
\begin{equation}
    \begin{aligned}
        \gamma\colon \eR&\to M \\
        t&\mapsto \exp(tX)(x).
    \end{aligned}
\end{equation}
This map satisfies \( \gamma(0)=x\). We also have, using proposition~\ref{PROPooWEYCooCvyHNr},
\begin{equation}
    \gamma(t_0+u)=\big( \exp(uX)\circ\exp(t_0X)\big)(x)=\exp(uX)\gamma(t_0).
\end{equation}
In that, we used the fact that \( G\) acts on \( M\), so that we have transformed the product inside the group \( \exp\big( (u+t_0)X \big)= \exp(uX)\exp(t_0x) \) into a composition of map.  Then
\begin{equation}
    \Dsdd{ \gamma(t_0+u) }{u}{0}=X\big( \gamma(t_0) \big)
\end{equation}
We conclude that \( \gamma\) satisfies the differential equation
\begin{equation}        \label{EQooFGSIooUplbmN}
    \gamma'(t)=-X\big( \gamma(t) \big).
\end{equation}
When \( M=\eR^n\), the Cauchy-Lipschitz theorem~\ref{ThokUUlgU} provides unicity of the solution on a maximal domain providing the map \( X\colon \eR^n\to \eR^n\) has nice properties.

\begin{normaltext}      \label{NORMooMGAUooIoLtjW}
    If we have to determine the transformations of \( \eR^n\) that satisfies some properties, the strategy is then the following:
    \begin{itemize}
        \item Suppose the searched group to be a connected Lie group.
        \item Write the condition with the groupe element \( \exp(tX)\) and differentiate with respect to \( t\). This point is what physicist call ``consider an infinitesimal transformation and neglect the higher order terms''.
        \item This provides an equation for \( X\). Typically a differential equation for the map \( X\colon \eR^n\to \eR^n\). Solve it.
        \item The group action is then retrieved solving the differential equation \eqref{EQooFGSIooUplbmN}.
    \end{itemize}
    Using that technique we will determine the isometries of \( \eR^n\) in proposition~\ref{PROPooDVIWooAFDNPy} and determine the conformal group around definition~\ref{DEFooVKNBooFBWQQM}.  % position 10906-29466: provides a more precise reference to the result instead of the definition.
\end{normaltext}

\begin{remark}
    When the Lie algebra is made of linear transformations, the last differential equation to solve is actually exponentiating a matrix.
\end{remark}

%---------------------------------------------------------------------------------------------------------------------------
\subsection{Example: determining the smooth isometries of the flat vector space}
%---------------------------------------------------------------------------------------------------------------------------

We know from theorem~\ref{ThoDsFErq} that the isometries of \( \eR^n\) are affine functions. We give now an alternative proof of that result.

\begin{proposition}     \label{PROPooDVIWooAFDNPy}
    The smooth\footnote{In fact we only need \( C^2\).} isometries of \( (\eR^n,\| . \|)  \)  are affine maps.
\end{proposition}

\begin{proof}
    The condition for a diffeomorphism \( \phi\colon \eR^n\to \eR^n\) to be an isometry is
    \begin{equation}        \label{EQooRKYWooFIKfYZ}
        \| \phi(x)- \phi(y) \|^2=\| x-y \|^2.
    \end{equation}
    We write \( \phi_t(x)= e^{-tX}x\) and take the derivative of \eqref{EQooRKYWooFIKfYZ} with respect to \( t\) at \( t=0\) taking into account that \( \phi_0(x)=x\):
    \begin{equation}        \label{EQooXEKMooGOktOj}
        \big( X(y)-X(x) \big)\cdot (x-y)=0.
    \end{equation}
    We used the fact that \( \Dsdd{ \phi_t(x) }{t}{0}=-X(x)\).

    We write the condition \eqref{EQooXEKMooGOktOj} with \( tx\) and take the derivative with respect to \( t\): \( dX_0(x)\cdot y+X(y)\cdot x=0\). The same with \( y\) gives
    \begin{equation}
        dX_0(x)\cdot y+dX_0(y)\cdot x=0.
    \end{equation}
    Taking \( x=e_i\) and \( y=e_j\) this equation reads
    \begin{equation}
        \frac{ \partial X_j }{ \partial x_i }+\frac{ \partial X_i }{ \partial x_j }=0.
    \end{equation}
    With \( i=j\) we get \( \frac{ \partial X_i }{ \partial x_i }=0\). The we compute
    \begin{equation}
        \frac{ \partial  }{ \partial x_j }\frac{ \partial X_i }{ \partial x_j }=-\frac{ \partial  }{ \partial x_j }\left( \frac{ \partial X_j }{ \partial x_i } \right)=-\frac{ \partial  }{ \partial x_i }\frac{ \partial X_j }{ \partial x_j }=0.
    \end{equation}
    We used the fact that \( X_j\) is of class \( C^2\) in order to permute the derivatives (lemma~\ref{LemPermDerrxyz}). We proved that
    \begin{equation}
        \frac{ \partial^2 X_i  }{ \partial x_j }=0
    \end{equation}
    for all \( i,j\). Thus \( X\) is linear.
\end{proof}

\section{Universal enveloping algebra}  \label{subsec:env_alg}
%----------------------------------------

Let $\mA$ be a Lie algebra. One knows that the composition law $(X,Y)\to[X,Y]$ is often non associative. In order to build an associative Lie algebra which ``looks like''\ $\mA$, one considers $T(\mA)$, the tensor algebra of $\mA$ (as vector space) and $\mJ$ the two-sided ideal in $T(\mA)$ generated by elements of the form
\[
   X\otimes Y-Y\otimes X -[X,Y]
\]
for $X$, $Y\in\mA$. The \defe{universal enveloping algebra}{universal!enveloping algebra} of $\mA$ is the quotient \nomenclature{$U(\mA)$}{Universal enveloping algebra}
\begin{equation}
     U(\mA)=T(\mA)/\mJ.
\end{equation}
For $X\in\mA$, we denote by $X^*$\nomenclature{$X^*$}{Image of a tensor in the universal enveloping algebra} the image of $X$ by canonical projection $\dpt{\pi}{T(\mA)}{U(\mA)}$ and by $1$ the unit in $U(\mA)$. One has $1\neq 0$ if and only if $\mA\neq\{0\}$.

\begin{proposition}[\cite{Helgason}]
Le $V$ be a vector space on $K$. Then there is a natural bijection between the representations of $\mA$\index{representation! of $U(\mA)$} on $V$ and the ones of $U(\mA)$ on $V$. If $\rho$ is a representation of $\mA$ on $V$, the corresponding $\rho^*$ of $U(\mA)$ is given by
\[
   \rho(X)=\rho^*(X^*)
\]
($X\in\mA$).
\end{proposition}

Let $\{X_1,\ldots,X_n\}$ be a basis of $\mA$. For a $n$-uple of complex numbers $(t_i)$, one defines
\begin{equation}
X^*(t)=\sum_{i=1}^nt_iX^*_i.
\end{equation}
On the other hand, we consider a $n$-uple of positive integers $M=(m_1+\ldots m_n)$, and the notation
\begin{equation}
\begin{split}
   |M|&=m_1+\cdots+m_n\\
   t^M&=t_1^{m_1}\cdots t_n^{m_n}.
\end{split}
\end{equation}

When $|M|>0$, we denote by $X^*(M)\in U(\mA)$ the coefficient of $t^M$ in the expansion of $(|M|!)^{-1} (X^*(t))^{|M|}$. If $|M|=0$, the definition is $X^*(0)=1$. Once again a proposition without proof.

\begin{proposition}
The smallest vector subspace of $U(\mA)$ which contains all the elements of the form $X^*(M)$ is $U(\mA)$ itself:
\[
   U(\mA)=\Span\{ X^*(M):M\in \eN^n \}.
\]
\end{proposition}

\begin{corollary} \label{cor:/24}
    Let $\mA$ be a Banach algebra of dimension $n$, $\mB$ a Banach subalgebra of dimension $n-r$ and a basis $\{X_1,\ldots,X_n\}$ of $\mA$ such that the $n-r$ last basis vectors are in $\mB$. We denotes by $B$ the vector subspace of $U(\mA)$ spanned by the elements of the form $X^*(M)$ with $m=(0,\ldots,0,m_{r+1},\ldots,m_n)$. Then $B$ is a subalgebra of $U(\mA)$.
\end{corollary}

\begin{definition}
Two Lie groups $G$ and $G'$ are \defe{isomorphic}{isomorphism!of Lie groups} when there exists a differentiable group isomorphism between $G$ and $G'$.

They are \defe{locally isomorphic}{locally!isomorphic!Lie groups} when there exists neighbourhoods $\mU$ and $\mU'$of $e$ and $e'$ and a differentiable diffeomorphism $\dpt{f}{\mU}{\mU'}$ such that

$\forall x,y,xy\in\mU$, $f(xy)=f(x)f(y)$, \\and

$\forall x',y',x'y'\in\mU'$, $f^{-1}(x'y')=f^{-1}(x')f^{-1}(y')$.
\end{definition}

The following universal property of the \emph{universal} enveloping algebra explains the denomination:
\begin{proposition}
Let $\dpt{\sigma}{\mG}{\mU(\mG)}$ the canonical inclusion and $A$ an unital complex associative algebra. A linear map $\dpt{\varphi}{\mG}{A}$ such that
\begin{equation}
\varphi[X,Y]=\varphi(X)\varphi(Y)-\varphi(Y)\varphi(X)
\end{equation}
can be extended in only one way to an algebra homomorphism $\dpt{\varphi_0}{\mU(\mG)}{A}$ such that $\varphi_0\circ\sigma=\varphi$ and $\varphi(1)=1$
\label{prop:extunifmap}
\end{proposition}
For a proof, see \cite{Knapp_reprez}.

\subsection{Adjoint map in \texorpdfstring{$\mU(\mG)$}{U(G)}}   \label{ssadjunif}
%/////////////////////////////////////////////

We know that $\dpt{ \Ad(g) }{ \mG }{ \mG }$ fulfils
\[
  \Ad(g)[X,y]=[  \Ad(g)X,\Ad(g)Y  ],
\]
and we can define $\dpt{ \Ad(g) }{ \mG }{ \mU(\mG) }$ by $\Ad(g)X=X$ where in the right hand side, $X$ denotes the class of $X$ for the quotients of the tensor algebra which defines the universal enveloping algebra.

When $[A,B]$ is seen in $\mU(\mG)$, we have $[A,B]=A\otimes B-B\otimes A$. Then $\dpt{ \Ad(g) }{ \mG }{ \mU(\mG) }$ fulfils proposition~\ref{prop:extunifmap} and is extended in an unique way to $\dpt{ \Ad(g) }{ \mU(\mG) }{ \mU(\mG) }$ with $\Ad(g)1=1$.

\begin{lemma}
    If $D\in\mU(\mG)$, the following properties are equivalent:
    \begin{itemize}
        \item $D\in\mZ(\mG)$
        \item $D\otimes X=X\otimes D$ for all $X\in \mG$
        \item $e^{\ad X}D=D$ for all $X\in\mG$
        \item $\Ad(g)D=D$ for all $g\in G$.
    \end{itemize}
     \label{lem:equivDAd}
\end{lemma}

%---------------------------------------------------------------------------------------------------------------------------
\subsection{Invariant fields}
%---------------------------------------------------------------------------------------------------------------------------

If $X\in\lG$, we have the associated left invariant vector field on $G$ given by $\tilde X_x=dL_xX$. That field is left invariant as operator on the functions because
\begin{equation}
    \tilde X_x(u)=\tilde X_e(L^*_xu)
\end{equation}
as the following computation shows
\begin{equation}
        \tilde X_e(L^*u)=\Dsdd{ (L_x^*u)\big(  e^{tX} \big) }{t}{0}
        =\Dsdd{ u\big( x e^{tX} \big) }{t}{0}
        =\Dsdd{ u\big( \tilde X_x(t) \big) }{t}{0}
        =\tilde X_x(u)
\end{equation}
because the path defining $\tilde X_x$ is $x e^{tX}$.

We can perform the same construction in order to build left invariant fields based on $\mU(\lG)$. If $X$ and $Y$ are elements of $\lG$, the  differential operator on $ C^{\infty}(G)$ associated to $XY\in\mU(\lG)$ is given by
\begin{equation}
    (XY)(f)=\DDsdd{ f\big( X(s)Y(t) \big) }{t}{0}{s}{0}
\end{equation}
The path defining the field $\widetilde{XY}$ is
\begin{equation}
    \widetilde{XY}_x=xX(s)Y(t).
\end{equation}
Thus we have
\begin{equation}        \label{EqInvarUgField}
    \widetilde{(XY)}_e(L^*u)=\widetilde{(XY)}_xu
\end{equation}

\begin{lemma}       \label{LemAdesthioo}
    If \( X,Y\in\lG\) we have
    \begin{equation}
        [\ad(X),\ad(Y)]=\ad([X,Y]).
    \end{equation}
\end{lemma}

\begin{proof}
    Let \( f\in\lG\) and compute the action of \( [\ad(X),\ad(Y)]\):
    \begin{subequations}
        \begin{align}
            [\ad(X),\ad(Y)]f&=\ad(X)[Yf,fY]-\ad(Y)(Xf-fX)\\
            &=(XY-YX)f+f(YX-XY)\\
            &=\ad([X,Y])f.
        \end{align}
    \end{subequations}
\end{proof}

%---------------------------------------------------------------------------------------------------------------------------
                    \subsection{Representation of Lie groups}
%---------------------------------------------------------------------------------------------------------------------------

\begin{proposition}
    Let $G$ be a Lie group and $\mG$ its Lie algebra. A representation $\varphi\colon G\to \End(V)$ of the group induces a representation $\phi\colon \mU(\mG)\to \End(V)$ of the universal enveloping algebra with the definitions
    \begin{subequations}
        \begin{align}
            \phi(X)     &=d\varphi_e(X),\\
            \phi(XY)    &=\phi(X)\circ\phi(Y)
        \end{align}
    \end{subequations}
    where $e$ is the unit in $G$ and $X$, $Y$ are any elements of $\mG$.
\end{proposition}

\begin{proof}
    We have
    \begin{equation}
        \phi(X)=\Dsdd{ \varphi( e^{tX})v }{t}{0}=d\varphi_e(X)v.
    \end{equation}
    Notice that, by linearity of the action of $\varphi( e^{tX})$ on $v$, one can leave $v$ outside the derivation. Now, neglecting the second order terms in $t$ in the derivative, and using the Leibnitz formula, we have
    \begin{equation}
        \begin{aligned}[]
            \phi([X,Y])v    &=  \Dsdd{ \varphi( e^{tXY} e^{-tXY}) }{t}{0}v\\
                    &=  \Dsdd{ \varphi( e^{tXY})\varphi(\mtu) }{t}{0}v+\Dsdd{ \varphi(\mtu)\varphi( e^{-tXY}) }{t}{0}v\\
                    &=  \phi(XY)v-\phi(YX)v\\
                    &=  \big( \phi(X)\phi(Y)-\phi(Y)\phi(X) \big)v\\
                    &=  [\phi(X),\phi(Y)]v,
        \end{aligned}
    \end{equation}
    which is the claim.
\end{proof}

\section{Lie subgroup}
%------------------------

\begin{definition}[\cite{BIBooXYAIooSLNqBk}]        \label{DEFooGCHDooHUMSju}
    Let \( G\) be a Lie group. A \defe{Lie subgroup}{Lie subgroup} of \( G\) is a part \( H\) of \( G\) such that
    \begin{enumerate}
        \item
            The part \( H\) is a Lie group for its own (in particular has a manifold structure).
        \item
            The inclusion map \( \iota\colon H\to G\) is an injective immersion\footnote{Definition \ref{DEFooZEWNooMVOzWI}.}.
        \item
            The inclusion map \( \iota\colon H\to G\) is a group homomorphism.
    \end{enumerate}
\end{definition}

\begin{proposition}     \label{PROPooFXZJooCOFXZX}
    A Lie subgroup is a submanifold\footnote{Definition \ref{DEFooLQHWooMOTgzq}.}. 
\end{proposition}

\begin{proof}
    Let \( H\) be a Lie subgroup of the Lie group \( G\); we have to prove that \( H\) is a submanifold of \( G\). As far as the notation are concerned we write \( \dim(H)=m\) and \( \dim(G)=n\). Let \( h\in H\). We know from the definition \ref{DEFooGCHDooHUMSju} that the inclusion \( \iota\colon H\to G\) is an injective immersion, and in particular the map
    \begin{equation}
        d\iota_h\colon T_hH\to T_{\iota(h)}G
    \end{equation}
    is injective. In particular the rank of \( d\iota_h\) is equal to \( \dim(H)\) and the rank theorem \ref{THOooSWKVooTJQsXc} applies. There exists charts \( \varphi\colon U\to H\) and \( \psi\colon V\to G\) such that
    \begin{equation}
        \begin{aligned}
            \psi^{-1}\circ\iota\circ\varphi\colon U&\to V \\
            (x_1,\ldots, x_m)&\mapsto (x_1,\ldots, x_m,0,\ldots, 0). 
        \end{aligned}
    \end{equation}
    
    What we need is a map \( \alpha\colon W\to G\) such that
    \begin{equation}
        \alpha^{-1}\big( \alpha(W)\cap H \big)=\{ (x_1,\ldots, x_m,0,\ldots, 0) \}\cap W.
    \end{equation}
    
    We prove that \( \alpha=\psi\) with \( W=V\) works. Indeed suppose that \( x\in V\) satisfies \( \psi(x)\in H\). In that case we have \( \psi(x)=\varphi(x')\) for some \( x'\in U\) and
    \begin{equation}
            \psi^{-1}(\psi(x))=\psi^{-1}\big( \varphi(x') \big)=(\psi^{-1}\circ\iota\circ\varphi)(x')=(x'_1,\ldots, x'_m,0,\ldots, 0).
    \end{equation}
\end{proof}

\begin{theorem}[Cartan\cite{BIBooGZHEooBPsXQy}]     \label{THOooDEJHooVKJYBL}
    Let \( G\) be a Lie group. If \( H\) is a closed subgroup of \( G\), then it has an unique structure of differential manifold such that the inclusion map \( \iota\colon H\to G\) is a smooth embedding.
\end{theorem}

\begin{corollary}
If $H_1$ and $H_2$ are two Lie subgroups of the Lie group $G$  such that $H_1=H_2$ as topological groups, then $H_1=H_2$ as Lie groups.
 \label{cor:top_subgroup}
\end{corollary}

\begin{proposition}
Let $G_1$ and $G_2$ be two Lie groups with same Lie algebra such that $\pi_0(G_1)=\pi_0(G_2)$ and $\pi_1(G_1)=\pi_1(G_2)$, then $G_1$ and $G_2$ are isomorphic.
\end{proposition}

\begin{proof}
The assumptions of equality of Lie algebras and of the $\pi_0$ make that the universal covering $\tilde G_1$ and $\tilde G_2$ of $G_1$ and $G_2$ are the same. But we know that $G_i=\tilde G_i/\pi_1(G_i)$. Now equality $\pi_1(G_1)=\pi_1(G_2)$ concludes that $G_1=G_2$.
\end{proof}

\begin{proposition}     \label{PROPooWMKGooKftzGv}
    There exists a differentiable structure on \( \SU(n)\) such that \( \SU(n)\) is a Lie subgroup of \( \GL(n,\eC)\).
\end{proposition}

%+++++++++++++++++++++++++++++++++++++++++++++++++++++++++++++++++++++++++++++++++++++++++++++++++++++++++++++++++++++++++++ 
\section{Semi-direct product of Lie groups}
%+++++++++++++++++++++++++++++++++++++++++++++++++++++++++++++++++++++++++++++++++++++++++++++++++++++++++++++++++++++++++++

\begin{definition}
A subgroup $H$ is \defe{normal}{normal!subgroup} in the group $G$ if for any $g\in G$ and $a\in H$, $gag^{-1}\in H$.
\end{definition}

If $G$ is a group, $N$ a normal subgroup and $L$ a subgroup, we have $LN=NL$ where, by notation, if $A$ and $B$ are subsets of $G$, $AB=\{xy|x\in A,y\in B\}$.

If $N$ and $L$ are groups, an \defe{extension}{extension!of group} of $N$ by $G$ is a short exact sequence
\begin{equation}
\xymatrix{ e \ar[r]& N \ar[r]^{\displaystyle i} & G \ar[r]^{\displaystyle \pi} & L \ar[r]^L  & e }
\end{equation}
which means that

\begin{enumerate}
\item $i$ is injective because only $e_N$ is sent to $e_G$,
\item $\pi$ is surjective because the whole $L$ is sent to $e$.
\end{enumerate}
One often say that $G$ is an extension of $N$ by $L$. In the most common case, $i$ is the inclusion, $L=G/N$ and $\pi$ is the natural projection.

We say that the extension is \defe{split}{split!extension} when there exists a \emph{split homomorphism} $\dpt{\rho}{L}{G}$ such that $\rho\circ\pi=\id_G$.

\begin{definition}
We say that $G$ is the \defe{semidirect product}{semi-direct product!of Lie groups} of $N$ and $L$ when any $g\in G$ can be written in one and only one way as $g=nl$ with $n\in N$ and $l\in L$.
\end{definition}


\begin{definition}
A \defe{Lie group homomorphism}{homomorphism!of Lie group} between $G$ and $G'$ is a map $\dpt{u}{G}{G'}$ which is a group homomorphism and a morphism between $G$ and $G'$ as differentiable manifolds.
\end{definition}

\begin{lemma}
Any continuous (group) homomorphism between two Lie groups is a \emph{Lie} group homomorphism.
\end{lemma}

We consider $G$, a connected Lie group; $N$, a closed normal subgroup; and $L$, a connected immersed Lie group. Moreover, we suppose that $G$ is semidirect product of $N$ and $L$.

\begin{proposition}
The restriction to $L$ of the canonical projection $\dpt{\pi}{G}{G/N}$ is continuous for the induced topology from $G$ to $L$.
\end{proposition}
\begin{proof}
      The definition of an open set $\mU$ in $G/N$ is that $\pi^{-1}(\mU)$ is open in $G$. Then it is clear that $\pi$ is continuous. The matter is to check it for $\pi|_L$. Let $\mU$ be a subset of $\pi(L)$. It is unclear that $\pi^{-1}(\mU)\subset L$, but it is true that $\pi|_L^{-1}(\mU)\subset L$.

      As far as the induced topology on $L$ is concerned, $A\subset L$ is open when $A=\mO\cap L$ for a certain open set $\mO$ in $G$.

      Let $\mU$ be an open subset of $\pi|_L(L)$; this is $\pi^{-1}(\mU)$ is open in $G$. We have to compare $\pi^{-1}(\mU)$ and $\pi|_L^{-1}(\mU)$. Since

    \[
        \pi|_L^{-1}(\mU)=\{x\in L|\pi(x)\in\mU\},
    \]
    we have $\pi|_L^{-1}(\mU)=\pi^{-1}(\mU)\cap L$. But $\pi^{-1}(\mU)$ is open in $G$, then $\pi^{-1}(\mU)\cap L$ is open in $L$.
\end{proof}

\begin{proposition}
The group $G$ is the semidirect product of $N$ and $L$ if and only if $G=NL$ and $N\cap L=\{e\}$.
\end{proposition}

\begin{proof}

If $G$ is semidirect product of $N$ and $L$, $G=NL$ is clear. In this case, if $e\neq z\in N\cap L$, $z=ez=ze$, thus $z\in G$ can be written in two ways as $xy$ with $x\in N$ and $y\in L$.

For the converse, let us consider $n'l'=nl$. Then $x^{-1} x'=yy'{}^{-1}\in N\cap L=\{e\}$. Thus $x'=x$ and $y'=y$.
\end{proof}

Now, we consider $N$, a normal subgroup of $G$. If $\dpt{\pi}{G}{G/N}$ is the canonical homomorphism, the restriction $\dpt{\pi|_L}{L}{G/N}$ is an isomorphism. Indeed, on the one hand, this is surjective because $G=NL$ yields
$[g]=[nl]=[l]=\pi|_L(l)$. On the other hand, $\pi|_L(l)=\pi|_L(l')$ implies that $l=nl'$ for a certain $n\in N$. Then $ll'{}^{-1}=n\in N\cap L=\{e\}$. So $n=e$ and $l=l'$.

\begin{remark}
If $N$ is any normal subgroup of $G$, there doesn't exist in general any subgroup $L$ of $G$ such that $G$ should be the semidirect product of $N$ and $L$.
\end{remark}

If $G$ is the semidirect product of $N$ and $L$, for any $y\in L$, $\dpt{\sigma_y}{x}{yxy^{-1}}$ is an automorphism of $N$. The point is that $\sigma_y(a)\in N$ for all $a\in N$ because $N$ is a normal subgroup.

It is also clear that $\forall\,u,v\in L$, $\sigma_{uv}=\sigma_u\circ\sigma_v$. Then $\dpt{\sigma}{L}{\Aut N}$\footnote{$\Aut N$ is the set of all the automorphism of $N$.} is a homomorphism. Moreover, the data of $\sigma$, $N$ and $L$ determines the law in $G$ (provided the fact that the product $NL$ is seen as formal) because any element of $G$ can be written as $nl$; thus a product $GG$ is $(nl)(n'l')=(n\sigma_y(n'))(ll')$

\begin{proposition}
Let $N$ and $L$ be two Lie groups and $\dpt{\sigma}{L}{\Aut N}$ a homomorphism. With the law
\[
   (x,y)(x',y')=(x\sigma_y(x'),yy'),
\]
the set $S=N\times L$ is a group.

\end{proposition}
\begin{proof}
\end{proof}

The set $N\times L$ endowed with this inner product is denoted
\[
   N\times_{\sigma}L.
\]

\begin{proposition}
If $G$ is the semidirect product of $N$ and $L$, then $G$ is isomorphic to $N\times_{\sigma} L$.
\end{proposition}

\begin{proof}
    The isomorphism is $\dpt{T}{N\times_{\sigma}L}{G}$, $T(x,y)=xy$. On the one hand, it is bijective because an element of $G$ can be written as $nl$ with $n\in N$ and $l\in L$ in only one way. On the other hand, it is easy to check that $T( (x,y)(x',u') )=T(x,y)T(x',y')$.
\end{proof}

One can now give the final definition. Let us consider two connected Lie groups $N$, $L$ and a Lie group  homomorphism $\dpt{\sigma}{L}{\Aut N}$. The map $N\times L\to N$, $(x,y)\to\sigma_y(x)$ is $\Cinf$. So, the group structure on $N\times L$ given by
\begin{equation}\label{eq:prod_semi_direct}
   (x,y)(x',y')=(x\sigma_y(x'),yy')
\end{equation}
is compatible with the $\Cinf$ structure of $N\times L$ (seen as a Lie group). The manifold $N\times L$ endowed with the group structure \eqref{eq:prod_semi_direct} is the \defe{semidirect product}{semi-direct product!of Lie groups} on $N$ and $L$; this is denoted by
\[
   N\times_{\sigma}L.
\]

%---------------------------------------------------------------------------------------------------------------------------
\subsection{Introduction by exact short sequence}
%---------------------------------------------------------------------------------------------------------------------------

%///////////////////////////////////////////////////////////////////////////////////////////////////////////////////////////
\subsubsection{General setting}
%///////////////////////////////////////////////////////////////////////////////////////////////////////////////////////////

Let $G_0$, $G_1$ and $G_2$ be tree connected Lie groups. A \defe{short exact sequence}{exact sequence!short} between them is two group homomorphisms
\begin{equation}
    \begin{aligned}[]
        \iota&\colon G_0\to G_1\\
        \pi&\colon G_1\to G_2
    \end{aligned}
\end{equation}
such that $\Image(\iota)=\Kernel(\pi)$. In that case, one says that $G_1$ is an \defe{extension}{extension of Lie groups} of $G_2$ by $G_0$.

Since the group $\iota(G_0)$ is the kernel of an homomorphism, it is normal and we write $\iota(G_0)\lhd G_1$\nomenclature[G]{$A\lhd B$}{$A$ is a normal subgroup of $B$}. Moreover, $\iota(G_0)=\pi^{-1}(e_2)$ and is then closed in $G_1$. As group, we have
\begin{equation}
    G_2=G_1/\iota(G_0).
\end{equation}

The extension is \defe{split}{extension of Lie groups!split} if there exists a Lie group homomorphism  $j\colon G_2\to G_1$ such that
\begin{equation}
    \pi\circ j=\id|_{G_2}.
\end{equation}
This condition imposes $j$ to be injective. In that case we have an action of $G_2$ on $G_0$ defined by
\begin{equation}
    \begin{aligned}
        R\colon G_2&\to \Aut(G_0) \\
        R_{g_2}(g_0)&=\iota^{-1}\Big( \AD\big( j(g_2) \big)\iota(g_0) \Big).
    \end{aligned}
\end{equation}
Notice that $\AD\big(j(g_2)\big)\iota(g_0)$ belongs to $\iota(G_0)$ because the latter is normal.

As manifold we consider
\begin{equation}
    G=G_0\times G_2
\end{equation}
and we define the multiplication law
\begin{equation}
    \begin{aligned}
        \cdot\colon G\times G&\to G \\
        (g_0,g_2)\cdot(g'_0,g'_2)&=\big( g_0 R_{g_2}(g'_0),g_2g'_2 \big).
    \end{aligned}
\end{equation}
For associativity we have
\begin{equation}
    (g_0,g_2)\cdot\big( (g'_0,g'_2)\cdot (g''_0,g''_2)) \big)=\Big(  g_0R_{g_2}\big( g'_0R_{g'_2}(g''_0) \big),g_2g'_2g''_2  \Big)
\end{equation}
while
\begin{equation}
    \begin{aligned}[]
        \big( (g_0,g_2)\cdot(g'_0,g'_2) \big)\cdot(g''_0,g''_2)=\big( g_0R_{g_2}(g'_0)R_{g_2g'_2}(g''_0),(g'_2g''_2) \big).
    \end{aligned}
\end{equation}
Thus the product is associative if and only if
\begin{equation}
    g_0R_{g_2}\big( g'_0R_{g'_2}(g''_0) \big)=\big( g_0R_{g_2}(g'_0) \big)R_{g_2g'_2}(g''_0).
\end{equation}
That equality is in fact true because $R$ is a morphism from $G_2$ to $\Aut(G_0)$, so that $R_{g_2}R_{g'_2}=R_{g_2g'_2}$.

The neutral in $G$ is $(e_0,e_2)$.

Since $R_{g_2}(g_0)$ is smooth with respect to both variables, the product is smooth. In that way, $G$ becomes a Lie group named the \defe{semi direct product}{semi direct product} of $G_2$ by $G_0$ and is denoted by
\begin{equation}
    G_0 \rtimes_RG_2.
\end{equation}
All the construction is still valid when $R$ is an homomorphism which does not comes from a split extension.

We define the product $G_0\times G_2\to G$ by
\begin{equation}
    g_0\cdot g_2=(g_0,e_2)\cdot(e_0,g_2)
\end{equation}

The diagram
\begin{equation}
    \xymatrix{%
            &       G_1 \ar[dr]^{\pi}           &\\
            G_0 \ar[ur]^{\iota}\ar[rd]_{\id\times\{ e \}}   &       & G_2\\
            &       G \ar@{.>}[uu]^{\varphi}\ar[ru]_{\pr_2}     &\\
       }
\end{equation}
suggests us to define the map
\begin{equation}
    \begin{aligned}
        \varphi\colon G_0\times G_2&\to G_1 \\
        (g_0,g_2)&\mapsto \iota(g_0)j(g_2)
    \end{aligned}
\end{equation}
This is a Lie group homomorphism because on the one hand
\begin{equation}
    \varphi(g_0,g_2)\cdot \varphi(g'_0,g'_2)=\iota(g_0)j(g_2)\cdot \iota(g'_0)j(g'_2),
\end{equation}
while on the other hand
\begin{equation}
    \begin{aligned}[]
        \varphi\big( (g_0,g_2)\cdot (g'_0,g'_2) \big)&=\varphi\big( g_0R_{g_2}(g'_0),g_2g'_2 \big)\\
        &=\varphi\Big( g_0\iota^{-1}\big( \AD(j(g_2))\iota(g'_0) \big),g_2g'_2 \Big)\\
        &=\iota\Big(g_0\iota^{-1}\Big( \AD(j(g_2))\iota(g'_0) \Big))j(g_2g'_2)\\
        &=\iota(g_0)j(g_2)\iota(g'_0)j(g'_2)
    \end{aligned}
\end{equation}
because $\iota$ and $j$ are homomorphisms.

The Leibnitz rule on $\iota(g_0)j(g_2)$ provides the differential
\begin{equation}
    (d\varphi)_e=(d\iota)_{e_0}\oplus(dj)_{e_2}.
\end{equation}
This is injective because $j$ is injective. The kernel of $\varphi$ is the set
\begin{equation}
    \Kernel(\varphi)=\{ (g_0,g_2)\tq \iota(g_0)=j(g_2)^{-1} \}.
\end{equation}
Since $\iota(G_0)$ and $j(G_2)$ have no intersections\footnote{They are transverse because $j\circ \pi=\id|_{G_2}$.} (a part the identity), we have that the kernel reduces to the identity:
\begin{equation}
    \Kernel(\varphi)=\{ e \}.
\end{equation}
The differentials provide the diagram
\begin{equation}
    \xymatrix{
    \mG_0\ar[r]^{(d\iota)_{e_0}}    &   \mG_1\ar[r]^{(d\pi)_{e_1}}  &   \mG_2\ar@<2pt>[l]^{(dj)_{e_2}}.
    }
\end{equation}
We have $(d\pi)_{e_1}\circ (dj)_{e_2}=\id|_{\mG_2}$ and the map
\begin{equation}
    (d\varphi)_e\colon \mG_1\to \mG_0\oplus\mG_2
\end{equation}
is an algebra homomorphism (as differential of group homomorphism). It is also an isomorphism by dimension counting. The inverse theorem then shows that $\varphi$ is a local diffeomorphism: $\varphi(G)$ contains a neighborhood of the identity and then is surjective by proposition~\ref{PropUssGpGenere}.

We conclude that $\varphi$ is a Lie group isomorphism.

