% This is part of (almost) Everything I know in mathematics and physics
% Copyright (c) 2013-2018, 2022
%   Laurent Claessens
% See the file fdl-1.3.txt for copying conditions.

\section{Groups of transformations}
%-------------------------------------

\begin{definition}
Let $M$ be an Hausdorff space and $G$, a topological group. It is a \defe{topological group of transformations}{topological!group of transformations}\index{transformation!group (topological)} of $M$ if to any element $g\in G$, one associates an homeomorphism $\dpt{g}{M}{M}$, denoted by dot ($p\to g\cdot p$) such that
$\forall p\in M$; $\forall g,h\in G$,
\begin{enumerate}
\item $(gh)\cdot p=g\cdot(h\cdot p)$,
\item the map $G\times M\to M$; $(g,p)\to g\cdot p$ is continuous.
\end{enumerate}
\label{DefTopoGpTransfo}
\end{definition}
If $g\cdot=id$ only for $g=e$, then the action is \defe{effective}{effective!action}.\index{action!of a group on a manifold}


The \defe{isotropy group}{isotropy!group} of $M$ (with respect to the action of $G$) is naturally defined by
\begin{equation}
   H=\{g\in G\tq g\cdot p=p\}.
\end{equation}

The unit sphere is an example. Let's consider the vector $\overrightarrow{1}_z$. It is clear that the action of $\SO(3)$ on this vector covers the whole sphere. But there is a $\SO(2)$ subgroup of rotations which leaves $\overrightarrow{1}_z$ unchanged. So the sphere is given by the quotient 
\begin{equation}
  S^2=\frac{\SO(3)}{\SO(2)}.
\end{equation}

\begin{corollary}
Let $G$ and $X$ be two locally compact groups. We suppose $G$ to have countable basis. Then any homomorphism $\dpt{\psi}{G}{X}$ is open.
\end{corollary}

\begin{proof}
By terminology, when we say that a group has some topological property (as the local compactness here), we suppose that the group is a topological group.

For any $g\in G$, we can build an homeomorphism (see remark~\ref{rem:ouvert}) $\dpt{\varphi_g}{X}{X}$, $x\to \psi(g)x$, so that $G$ is  a transitive topological group of transformation on $X$. Let us denote by $f$ the identity of $X$. We know that $\varphi_g$ is continuous, open and satisfies
\[
   \varphi_g(f)=\psi(g)f=\psi(g).
\]

If we define $H=\{h\in G\tq \psi(h)=f\}$, $\psi(gh)=\psi(g)$ for any $h\in H$, so that $\psi$ descent to a map $\dpt{\psi}{X/H}{X}$. This is precisely the map which theorem~\ref{tho:homeo_action} assure us to be an homeomorphism.

\end{proof}

\section{Lie groups of transformations}
%----------------------------------------

\begin{definition}
Let $G$ be a Lie group and $M$, an analytic manifold. We say that $G$ is a \defe{Lie group of transformation}{Lie!group!of transformation}\index{transformation!Lie group} of $M$ when

\begin{enumerate}
\item $G$ is a topological group of transformation of $M$,
\item the map $G\times M\to M$, $(g,p)\to g\cdot p$ is analytic.
\end{enumerate}
\label{DefLieGpTransfo}
\end{definition}

In particular, for any $g\in G$, $p\to g\cdot p$ is a diffeomorphism $M\to M$. When $G$ is a Lie transformation group on $M$, for any $X\in\lG$, we define a \defe{fundamental vector field}{fundamental!vector field} $X^{\dag}\in\cvec(M)$ by\footnote{Remark, as usual, that some literature (in particular in \cite{Helgason}) gives it without the minus sign.}
\begin{equation}
  X^{\dag}_p=\Dsdd{e^{-tX}\cdot p}{t}{0}
\end{equation}
The existence comes from the fact that $(g,p)\to g\cdot p$ and the exponential are analytic and on a function $f\in\Cinf(M)$, the vector field acts as
\[
   (X^{\dag} f)(p)=\lim_{t\to\infty}\frac{ f(e^{-tX}\cdot p)-f(p) }{t}.
\]

\begin{theorem}
Consider $G$, a Lie transformation group on $M$ and $X$, $Y\in\lG$. Then
\begin{equation}
  [X^{\dag},Y^{\dag}]=\lim_{t\to 0}\us{t}(Y^{\dag}-dg_t Y^{\dag})
\end{equation}
where $g(t)=g_t=e^{tX}$.
\end{theorem}

\begin{proof}
We consider $f\in\Cinf(M)$ and $q\in M$. The function $F$ defined by
\begin{equation}
  F(f,q)=f(e^{-tX}\cdot q)
\end{equation}
is analytic with respect to $t$, so that
\begin{equation}\label{eq:def_F_h}
  F(t,q)-F(0,q)=t\int^1_0\left(\dsd{F}{t}\right)(st,q)ds=h(t,q)t
\end{equation}
for a certain function $h\in\Cinf(\eR\times M)$ which satisfies $h(0,q)=(X^{\dag} f)(q)$.
Naturally, $g_t$ can be seen as a map $\dpt{g_t}{M}{M}$ by the action. Then $dg_t$ is a linear map $\dpt{dg_t}{T_qM}{T_{g_t\cdot q}M}$  (we voluntary omit the index $q$ which was fixed; formally, we speak about $(dg_t)_q$)
\[
   dg_tv=\Dsdd{g_t\cdot v(u)}{u}{0}.
\]
Thus in order to compute $(dg_t Y^{\dag})_p$, we have to consider $Y^{\dag}$ at $r=g_t^{-1}\cdot p$. Consider a path $\dpt{v_r}{[0,1]}{M}$ such that $v'_r(0)=Y^{\dag}_r$ and $v_r(0)=r$. So,
\begin{equation}
\begin{split}
(dg_t\cdot Y^{\dag})_pf&=\Dsdd{ f(g_t\cdot v_r(u)) }{u}{0}\\
                     &=\Dsdd{ (f\circ g_t\circ v_r)(u) }{u}{0}\\
		     &=\Dsdd{ F(t,v_r(u)) }{u}{0}\\
		     &=\Dsdd{ F(0,v_r(u))+h(t,v_r(u))t }{u}{0}
\end{split}
\end{equation}
The two terms are computed separately:
\[
   \Dsdd{ F(0,v_r(u)) }{u}{0}=\Dsdd{ f(v_r(u)) }{u}{0}=Y^{\dag}_r(f),
\]
and
\[
    \Dsdd{ th(t,v_r(u)) }{u}{0}=t(Y^{\dag} h)_{(t,v_r(0))}.
\]
Finally,
\begin{equation}\label{eq:Y_h}
  (dg_t\cdot Y^{\dag})_pf=Y^{\dag}_{g_t^{-1}\cdot p}(p)+t(Y^{\dag} h)_{(t,g_t^{-1}\cdot p)}.
\end{equation}
Now we can compute:
\begin{equation}
\begin{aligned}
  \lim_{t\to 0}\us{t}\left( (Y^{\dag}-dg_tY^{\dag})_pf \right)
                     &=\lim_{t\to 0}\us{t}\left\{ (Y^{\dag} f)p-(Y^{\dag} f)(g_t^{-1}\cdot p) \right\}\\
		     &\quad+\lim_{t\to 0}\us{t}\left\{ (Y^{\dag} f)(g_t^{-1}\cdot p)
		                                      -(Y^{\dag}(f\circ g_t) )(g_t^{-1}\cdot p)
					       \right\}\\
   &=\Dsdd{ (Y^{\dag} f)(g_t^{-1}\cdot p) }{t}{0}-\lim_{t\to 0}
                \left(    (Y^{\dag} h)(t,g_t^{-1}\cdot p)       \right).
\end{aligned}
\end{equation}
The latter equality comes from  \eqref{eq:Y_h}. The first term is computed as following ($Y^{\dag}(f)$ is a function):
\begin{equation}
\Dsdd{ Y^{\dag}(f)(g_t^{-1}\cdot p) }{t}{0}=\Dsdd{ (Y^{\dag}(f)\circ g_t^{-1})(p) }{t}{0}
                                        =\left( Y^{\dag}(f) \right)_p(g_t^{-1})'(0)
					=\left( X^{\dag} Y^{\dag}(f)\right)_p
\end{equation}
In the expression $(Y^{\dag} h)(t,g_t^{-1}\cdot p)$, we have to consider the dependence on $t$ as a parameter: the vector $Y^{\dag}$ only acts on the ``second slot''\ of $h$. From definition \eqref{eq:def_F_h} of $h$,
\[
   h(t,g_t^{-1}\cdot p)=\us{t}\left(  F(f,g_t^{-1}\cdot p)-F(0,g_t^{-1}\cdot p)
                             \right),
\]
but $F(0,q)=f(q)$ and $F(t,g_t^{-1}\cdot p)=f(p)$, then
\[
  f(t,g_t^{-1}\cdot p)=\us{t}\left( f(p)-f(e^{-tX}\cdot p) \right)
\]
Taking the limit for small $t$, it becomes
\[
  \Dsdd{f(e^{-tX}\cdot p)}{t}{0}=(X^{\dag} f)_p
\]

\end{proof}

%---------------------------------------------------------------------------------------------------------------------------
\subsection{Action of the Lie algebra}
%---------------------------------------------------------------------------------------------------------------------------

\begin{normaltext}
    We will study the infinitesimal action of the two dimensional conformal group on $\eC$ in~\ref{NORMooHDLPooQBfEif} and what follows.
\end{normaltext}

\begin{definition}[\cite{ooUUIGooOsgprP}]       \label{DEFooUYOZooWdcClz}
    If the Lie group \( G\) acts on the manifold \( M\), then its Lie algebra \( \lG\) has an action
    \begin{equation}
        \begin{aligned}
            X\colon M&\to TM \\
            x&\mapsto \Dsdd{  \exp^{-tX}\cdot x }{t}{0}\in T_xM.
        \end{aligned}
    \end{equation}
\end{definition}
If you want to know how to integrate this to an action of the group, see the differential equation \eqref{EQooFGSIooUplbmN}.

Thus an element of the Lie algebra \( \lG\) acts on a function \( f\in C^{\infty}(M)\) as \( X(f)\in C^{\infty}(M)\) by
\begin{equation}        \label{EQooQDLAooZXnWta}
    X_x(f)=\Dsdd{ f\big(  e^{-tX}x \big) }{t}{0}=df_x\big( -X(x) \big)=-df_x\big( X(x) \big)
\end{equation}
where \( X_x(f)\) is a notation shortcut for \( X(f)(x)\). Here \( X(x)\) stands for the map \( X\colon M\to M\) defined by \( X(x)= e^{tX}x\).

\section{Cosets and homogeneous spaces}
%-----------------------------------------

Proposition~\ref{propHelgason4.3} takes its interest in the setting of homogeneous space. An \defe{homogeneous spaces}{homogeneous!space} is a smooth manifolds $M$ which admits a Lie group of transformations. If $p_{0}\in M$ and $H$ is the stabilizer of $p_{0}$, proposition~\ref{propHelgason4.3} says that $H$ is closed and therefore theorem~\ref{tho:struc_anal} makes $G$ a Lie group of transformations of $G/H$. Hence the latter becomes a homogeneous space. The map $\alpha$ of proposition~\ref{propHelgason4.3} gives an isomorphism of homogeneous spaces.

This allow us to see a homogeneous space as the quotient of a group by a closed subgroup.

\section{Isotropy group}
%--------------------------

If one has a Lie group $G$ and a closed subgroup $H$, we know from theorem~\ref{THOooXVXBooZDJzQo} that $H$ is a topological Lie subgroup of $G$. We naturally consider this structure and the analytic structure on $G/H$ given by~\ref{tho:struc_anal}. For this structure of the coset space $G/H$, the group $H$ is the \defe{isotropy group}{isotropy!group}\index{group!isotropy}. We denote by $\dpt{\tau(x)}{G/H}{G/H}$ ($x\in G$) the diffeomorphism $\tau(x)[y]=[xy]$. The group $H^*$ of the linear transformations $(d\tau(h))_{\pi(e)}$ ($h\in H$) is the \defe{linear isotropy group}{isotropy!Linear group}.


Let $N$ be a Lie subgroup of $G$. Then the subset $N\cap H$ is closed
\footnote{If $H'$ denotes the complementary of $H$ in $G$ (which is open in $G$), the complementary $N\cap H'$ of $N\cap H$ is open in $N$ for the induced topology of $N$ from $G$.}. Then $N\cap H$ is closed in $N$ and we look at $N/(N\cap H)$. We can exhibit a bijection between this and the orbit of $\pi(e)$ in $G/H$ with respect to the action of $N$:
\begin{equation}
  \{ n\pi(e)\tq n\in N \}\simeq N/(N\cap H)
\end{equation}
by the map $\psi$ given by $\psi(n\pi(e))=\overline{n}$. Here the $\overline{ x }$ denotes the class of $x$ with respect to $N\cap H$. Note that for $n\in N$, $\overline{n}\neq\overline{e}$ because there are \emph{a priori} elements in $N\setminus H$. The map $\psi$ is well defined because $m[e]=[m]=[n]$ if $m=nh$ for a certain $h\in H$. Then $\psi([nh])=\overline{nh}$. But in order for $nh$ to belongs to $N$, one needs $h\in N\cap H$; then $\overline{nh}=\overline{n}$. For the same reason, $\psi$ is injective. The surjectivity is clear.

\begin{proposition}			\label{prop:orbit_N_ss_var}
	In this context,
	\begin{enumerate}
		\item The orbit of $e$ by $N$ in $G/H$ is $N/(N\cap H)$. It is submanifold of $G/H$.
		\item If $N$ is a topological subgroup of $G$ and if $H$ is compact, then the submanifold $N/(N\cap H)$ is a closed topological submanifold of $G/H$.
	\end{enumerate}
\end{proposition}

\begin{proof}
\subdem{First item}
We denote by $\overline{n}$, the class of $n$ with respect to $N\cap H$ and by $[g]$, the class of $g$ with respect to $H$.
The following diagram is commutative:
\begin{equation}\label{eq:dig_4.4}
 \xymatrix{
    N  \ar[d]_{\displaystyle\pi_1}\ar[r]^{\displaystyle i} &  G \ar[d]^{\displaystyle\pi}\\
    N/(N\cap H)\ar[r]^-{\displaystyle I} &       G/H
  }
\end{equation}
where $\dpt{\pi_1}{N}{N/(N\cap H)}$ and $\dpt{\pi}{G}{G/H}$ are canonical projections; $\dpt{i}{N}{G}$ is the inclusion; and $\dpt{I}{N/(N\cap H)}{G/H}$ is defined by $\overline{n}\to [i(n)]$. Indeed, $\pi(i(n))=[i(n)]=I( \overline{n} )=I(\pi_1(n))$ for any $n\in N$.

If $\lN$ is the Lie algebra of $N$ and $\lH$ the one of $H$, $\lH_1=\lH\cap\lN$ is the Lie algebra of $N\cap H$. We consider $\lN_1$ and  $\lG_1$ such that $\lN_1\oplus\lH_1=\lN$ and $\lG_1\oplus(\lH\oplus\lN_1)=\lG$. Let us show why is the sum $\lH\oplus\lN_1$ direct. First remark that $\lH\cap\lN_1$ because $\lH\cap\lN_1\subset\lH\cap\lN=\lH_1$, but $\lH_1\cap\lN_1=\{0\}$. Immediately, the sum $\lG=\lG_1\oplus(\lH\oplus\lN_1)$ is direct.

Now we apply lemma~\ref{lem:vois_U} to the decomposition $\lN=\lH_1\oplus\lN_1$; this give us a submanifold $B_N\subset N$ which contains $e$ and on which $\pi_1$ is diffeomorphic to an open neighbourhood of $\pi_1(e)$ in $N/(N\cap H)$. The same with $\lG=\lH\oplus(\lN_1+\lG_1)$ gives $B_G\subset G$, a submanifold around $e$ on which $\pi$ is diffeomorphic to a neighbourhood of $\pi(e)$ in $G/H$. We can see $B_N$ as a submanifold of $B_G$.

We denotes $V_1=\pi_1(B_N)$, $V=\pi(B_G)$ and $I_{V_1}$, the restriction of $I$ to $V_1$. The Jacobian of $I_{V_1}$ at $\pi(e)$ has a rank equal to $\dim\big( N/(N\cap H) \big)$. Indeed we can write $I_{V1}$ as $I_{V_1}=\pi\circ i\circ\pi^{-1}$ and $\pi_1$ is a diffeomorphism, so that $\pi^{-1}$ don't change the dimension. The fact that the Jacobian at identity is non zero on a neighbourhood makes it regular on this neighbourhood and the analyticity make it regular everywhere. The characterization for a submanifold given by proposition \ref{PROPooZACHooCNgLSl} gives the first item.

\subdem{Second item}
We know that $N$ is a submanifold of $G$; the commutative diagram \eqref{eq:dig_4.4} shows that $I$ is an homeomorphism because the topologies on $G/H$ and $N/(N\cap H)$ are made in order to $\pi$ and $\pi_1$ are continuous and open. If $N$ is a topological subgroup of $G$, an open subset of $N$ is written under the form $N\cap\mO$ where $\mO$ is open in $G$. If $\overline{n}\in N/(N\cap H)$, $I(\overline{n})=[i(n)]$.

Now we show that $N/(N\cap H)$ is closed. Consider a sequence $(p_k)$ in $N/(N\cap H)$ which converges to $q\in G/H$. The aim is to show that $q\in N/(N\cap H)$. We take a $g\in G$ such that $\pi(g)=q$; we can suppose that the whole sequence $(p_k)$ is in the neighbourhood $g\cdot V$ of $q$. In order to see that it is a neighbourhood, recall that $\pi$ is a diffeomorphism from $V$ to an open neighbourhood of $e$ in $G/H$, thus $V$ is an open neighbourhood of $\pi(e)$ in $G/H$.

Since $\pi$ is diffeomorphic, there exists a sequence $g_k\in gB_G$ such that $\pi(g_k)=p_k$. It satisfies $\lim g_k=g$. On the other hand, for each $k$, $\exists n_k\in N$ such that $\pi_1(n_k)=p_k$; then $\pi(g_k)=p_k=\pi_1(n_k)$, then there exists $h_k\in H$ such that $g_k=n_kh_k$. But $H$ is compact, then $h_k$ is a converging sequence (by eventually passing to a subsequence). Since $g_k$ and $h_k$ converge, $n_k$ also converges. But $N$ is closed in $G$, then $n^*=\lim n_k\in N$. Finally $\pi_1(n^*)=q$, so that $N/(N\cap H)$ is closed.
\end{proof}
