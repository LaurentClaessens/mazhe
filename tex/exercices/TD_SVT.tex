% This is part of Mes notes de mathématique
% Copyright (c) 2015-2016
%   Laurent Claessens
% See the file fdl-1.3.txt for copying conditions.

\chapter{Mathématique générale pour le starter SVT (Besançon)}

Contributeurs :
\begin{description}
    \item[L'équipe SVT] Rédaction de la plupart des exercices.
    \item[Laurent Claessens] \LaTeX, corrections et publication
    \item[Carlotta Donadello] \LaTeX, corrections
    \item[Pauline Klein] détectrice de coquilles.
\end{description}

Nous avons effectué une certaine «classification» des exercices en y ajoutant des petits symboles.
\begin{enumerate}
	\item Le symbole \minsyndical\ marque les exercices à faire à tout prix. Il faut les faire tous.
	\item Le symbole \boringexo\ signifie que l'exercice ne va en principe pas apporter de nouvelles techniques. Ces exercices sont à faire après avoir fait les exercices de type \minsyndical, si on a encore des doutes.
	\item Le symbole \coolexo\ indique que l'exercice sera plus difficile et qu'il vaut mieux l'éviter avant d'avoir bien compris les exercices de type \minsyndical.
	\item Le symbole \mortelexo\ est appliqué aux exercices qui sont des compléments de la matière, mais qui ne sont pas à faire de façon obligatoire.
\end{enumerate}

%+++++++++++++++++++++++++++++++++++++++++++++++++++++++++++++++++++++++++++++++++++++++++++++++++++++++++++++++++++++++++++
\section{Théorie}
%+++++++++++++++++++++++++++++++++++++++++++++++++++++++++++++++++++++++++++++++++++++++++++++++++++++++++++++++++++++++++++
% This is part of Exercices de mathématique pour SVT
% Copyright (c) 2010,2014
%   Laurent Claessens et Carlotta Donadello
% See the file fdl-1.3.txt for copying conditions.

%--------------------------------------------------------------------------------------------------------------------------- 
\subsection{Suites}
%---------------------------------------------------------------------------------------------------------------------------

\begin{proposition}		\label{Propufulimite}
	Soit $(u_n)_{n\in\eN}$, une suite définie par récurrence par 
	\begin{equation}
		u_{n+1}=f(u_n)
	\end{equation}
	où $f$ est une fonction suffisamment gentille\footnote{Nous ne rentrons pas dans les détails des hypothèses exactes. Sachez qu'il faut au moins que la fonction soit continue et bien définie.}. Si la suite $(u_n)_{n\in\eN}$ converge vers un nombre réel $u$, alors $u$ est une solution de l'équation  $u=f(u)$. 
\end{proposition}
Lorsque $u$ est une solution de $u=f(u)$ on dit que $u$ est un point fixe de $f$. Attention : la proposition (\ref{Propufulimite}) ne garantit pas l'existence de la limite, ni que toute solution de $u=f(u)$ soit une limite. D'ailleurs l'équation $u=f(u)$ peut avoir plusieurs solutions sans que la suite n'ai de limites finis. Cela est le cas lorsque la suite diverge vers $\pm\infty$. 

 \begin{example}
	 Soit la suite définie par
	 \begin{equation}
		 \begin{cases}
		 u_{n+1}=u_n^2\\
		u_0=2.
		 \end{cases}
	 \end{equation}
	 Les premiers termes sont $u_0=2$, $u_1=2^2=4$, $u_3=4^2=16$, etc. Cette suite diverge. Pourtant l'équation $u=u^2$ a des solutions : $u=0$ et $u=1$.

	 Si au lieu d'avoir $u_0=2$, on avait eu $u_0=\frac{ 1 }{2}$, alors nous aurions $u_2=\frac{ 1 }{ 4 }$, $u_3=\frac{1}{ 16 }$, etc. Cette suite converge vers $0$, qui est bien solution de $u=f(u)$.
\end{example}

\begin{proposition}		\label{Propsuiteborncv}
Toute suite monotone et bornée converge. En particulier, si une suite est décroissante et bornée vers le bas, alors elle converge. De même, si une suite est  croissante et bornée vers le haut, alors elle converge. 
\end{proposition}

%--------------------------------------------------------------------------------------------------------------------------- 
\subsection{Techniques pour majorer et minorer}
%---------------------------------------------------------------------------------------------------------------------------

Dans de nombreux exercices sur les suites, une difficulté est de majorer ou minorer des expression contenant $u_n$ sachant que $u_n$ est dans un certain intervalle. La technique la plus puissante pour ce faire demande une utilisation intensive des dérivées; nous n'allons pas parler de cela ici, mais sachez que ça existe.

\begin{example}
	Trouver	des bornes pour la quantité
	\begin{equation}		\label{EqExpuumulnuB}
		a_n=u_n\big( 1-\ln(u_n) \big)
	\end{equation}
	sachant que $1<u_n<e$.

	Trouvons une borne supérieure pour l'expression \eqref{EqExpuumulnuB}, c'est à dire, trouvons $M$ tel que nous soyons certain d'avoir $a_n<M$. Pour ce faire, nous remplaçons tous les $u_n$ par la valeur qui rend l'expression la plus grande possible. Le premier $u_n$ doit être remplacé par $e$. Le $u_n$ qui se trouve dans le logarithme doit par contre être remplacé par $1$ parce qu'il arrive dans terme qui se soustrait; pour minorer, il faut soustraire la quantité la plus petite possible. Nous pouvons donc dire que 
	\begin{equation}
		a_n<e(1-\ln(1))=e.
	\end{equation}
	
	De la même façon, si nous voulons minorer $a_n$, c'est à dire trouver un $m$ tel que $m<a_n$, nous devons remplacer les $u_n$ par les valeurs qui rendent $a_n$ le plus petit possible. Le premier $u_n$ doit être remplacé par $1$, tandis que le second doit être remplacé par $e$ (pour soustraire le plus possible). Nous trouvons
	\begin{equation}
		a_n>1(1-\ln(e))=0.
	\end{equation}
	Nous avons donc
	\begin{equation}
		0<u_n\big( 1-\ln(u_n) \big)<e
	\end{equation}
	dès que $1<u_n<e$.
	
	Notez que cela ne sont pas les bornes optimales. Il est possible (en travaillant plus) de prouver que, sous les mêmes hypothèses, $u_n\leq 1$.
\end{example}



%%%%%%%%% Fonctions %%%%%%%%%%%%%%%%%%%%%%
\section{Rappels : exponentielles et logarithmes}
% This is part of Exercices de mathématique pour SVT
% Copyright (c) 2011
%   Laurent Claessens et Carlotta Donadello
% See the file fdl-1.3.txt for copying conditions.

\begin{definition}
  Soit $a>0$ un nombre réel. La fonction exponentielle en base $a$ est la fonction définie par $x\mapsto a^x$. 
\end{definition}

Le domaine de $x\mapsto a^x$ est $\eR$. La fonction exponentielle est croissante si $a>1$, décroissante si $0<a<1$, constante si $a=1$. Son image est 
\begin{itemize}
\item $]0,+\infty[$ si $a\neq 1$,
    \item $\{1\}$ si $a=1$.
\end{itemize}

La fonction exponentielle satisfait les propriétés suivantes pour tous $x$ et $y$ dans $\eR$ :
\begin{itemize}
\item $a^0=1$ ;
  \item $a^{x+y}=a^xa^y$ ;
    \item $\displaystyle a^{x-y}=\frac{a^x}{a^y}$ ;
      \item $\displaystyle a^{xy}= (a^x)^y$. 
\end{itemize}

Si $a\neq 0$ la fonction exponentielle est strictement monotone sur $\eR$ et par conséquence elle admet une fonction réciproque. D' où la définition suivante

\begin{definition}
  Soit $a>0$, $a\neq 1$. La fonction logarithme de base $a$, $\log_{a}$, est définie par la relation $x= a^{\log_{a}x}$. 
\end{definition}

La fonction logarithme satisfait les propriétés suivantes pour tous $x$ et $y$ dans $\eR$ :
\begin{itemize}
\item $\log_a 1=0$ ;
  \item $\log_a x+\log_a y=\log_a (xy)$ ;
    \item $\displaystyle \log_a x-\log_a y=\log_a\left(\frac{x}{y}\right)$ ;
      \item $\displaystyle \log_a (x^y)= y\log_a x$. 
\end{itemize}

En outre, la formule suivante permet de <<changer de base>> :
\begin{equation}
  \log_a(x)=\frac{\log_b x }{\log_b a}.
\end{equation}
Cela est particulièrement important parce que nous permet d'établir la relation entre le logarithme en base $a$ et le logarithme népérien (de base $e$). 



\section{Exponentielles et logarithmes}
\Exo{logarithme-0003}
\Exo{logarithme-0001}
\Exo{logarithme-0002}


\section{Fonctions et graphes}
\Exo{TD1_1}
\Exo{TD1_2}
\Exo{TD1_3}
\Exo{TD1_4}

\section{Limites du côté de l'infini}
\Exo{SVT-0001}
\Exo{TD3-0003}

\section{Limite de suites}
\Exo{TD3-0001}
\Exo{TD3-0002}
\Exo{TD3-0004}
\Exo{TD3-0005}
\Exo{TD3-0006}
\Exo{TD3-0007}
\Exo{TD3-0008}
\Exo{TD3-0009}
\Exo{TD3-0010}
\Exo{TD3-0011}
\Exo{TD3-0012}
\Exo{TD3-0013}
\Exo{TD3-0014}

\section{Étude de fonctions, première partie}
\Exo{TD2-1}
\Exo{TD2A-2}
\Exo{TD2B_1}
\Exo{TD2-2}

\section{Étude de fonctions, suite}
\Exo{TD4-0001}
\Exo{TD4-0002}
\Exo{TD4-0003}
\Exo{TD4-0004}
\Exo{TD4-0005}

\section{Intégration}
\Exo{TD5-00001}
\Exo{TD5-00002}
\Exo{TD5-00003}

\Exo{TD5-a-0001}
\Exo{TD5-a-0002}
\Exo{TD5-a-0003}
\Exo{TD5-0001}
\Exo{TD5-0002}
\Exo{TD5-0003}
\Exo{TD5-0004}
\Exo{TD5-0005}

\section{Équations différentielles}
\Exo{SVT-0004}
\Exo{SVT-0003}
\Exo{TD6A-0002}
\Exo{TD6A-0003}

\Exo{TD6b-0001}
\Exo{TD6b-0002}
\Exo{TD6b-0003}
\Exo{TD6b-0004}

\Exo{SVT-0005}
\Exo{TD6-0001}
\Exo{TD6-0002}
\Exo{TD6-0003}
\Exo{TD6-0004}

\Exo{TD6A-0001}

\section{Révisions}
\Exo{revisions-0001}
\Exo{revisions-0002}
\Exo{revisions-0003}
\Exo{revisions-0004}

\section{Anciennes interrogations}
% This is part of Exercices de mathématique pour SVT
% Copyright (c) 2010-2011,2016
%   Laurent Claessens et Carlotta Donadello
% See the file fdl-1.3.txt for copying conditions.

Pour rappel, toutes les interrogations, devoirs surveillés et examens se font avec uniquement du papier et de quoi écrire (pas de notes, pas de calculatrices ou autre équipements). Les réponses doivent être justifiées un minimum.

%---------------------------------------------------------------------------------------------------------------------------
\subsection{Septembre 2010}
%---------------------------------------------------------------------------------------------------------------------------

\Exo{interro-0002}
\Exo{interro-0003}
\Exo{interro-0004}
\Exo{interro-0005}
\Exo{interro-0007}
\Exo{interro-0008}

%---------------------------------------------------------------------------------------------------------------------------
\subsection{DS octobre 2010, un}
%---------------------------------------------------------------------------------------------------------------------------

\Exo{DS2010-1-0001}
\Exo{DS2010-1-0002}
\Exo{DS2010-1-0003}
\Exo{DS2010-1-0004}
\Exo{DS2010-1-0005}

%---------------------------------------------------------------------------------------------------------------------------
\subsection{DS octobre 2010, deux}
%---------------------------------------------------------------------------------------------------------------------------

\Exo{DS2010bis-0001}
\Exo{DS2010bis-0002}
\Exo{DS2010bis-0003}
\Exo{DS2010bis-0004}
\Exo{DS2010bis-0005}

\clearpage

%---------------------------------------------------------------------------------------------------------------------------
\subsection{Épreuve complémentaire décembre 2010}
%---------------------------------------------------------------------------------------------------------------------------

\subsubsection{A traiter}

\Exo{ExamenDecembre2010-0001}
\Exo{ECdecembre2010-0001}
\Exo{ECdecembre2010-0002}
\Exo{ECdecembre2010-0003}
\Exo{ECdecembre2010-0004}

%TODO : est-ce qu'il y avait 5 exercices ? En tout cas ici cet exercice est vide \ldots
%---------------------------------------------------------------------------------------------------------------------------
\subsection{Examen décembre 2010}
%---------------------------------------------------------------------------------------------------------------------------

\Exo{Examen-0001}
\Exo{ExamenDecembre2010-0002}
\Exo{ExamenDecembre2010-0003}
\Exo{ExamenDecembre2010-0004}
\Exo{ExamenDecembre2010-0005}

\clearpage

%---------------------------------------------------------------------------------------------------------------------------
\subsection{DS décembre 2011}
%---------------------------------------------------------------------------------------------------------------------------

\Exo{Exosenvrac-0001} 
\Exo{Exosenvrac-0015}
\Exo{Exosenvrac-0015A}
\Exo{Exosenvrac-0006}
\Exo{Exosenvrac-0009}

%--------------------------------------------------------------------------------------------------------------------------- 
\subsection{Épreuve complémentaire, 15 décembre 2010}
%---------------------------------------------------------------------------------------------------------------------------

\Exo{EC-0003}
\Exo{EC-0004}


