
\InternalLinks{limite et continuité}    \label{THEMEooGVCCooHBrNNd}
\begin{enumerate}
	\item
	      Limite d'une fonction en un point : définition \ref{DefYNVoWBx}. Il n'y a pas unicité en général comme le montre l'exemple \ref{EXooSHKAooZQEVLB} dans un espace non séparé.
	\item
	      Caractérisation de la limite dans \( \eR\), proposition \ref{PropAJQQooQQClfp}.
	\item
	      Unicité de la limite d'une suite dans un espace séparé : proposition \ref{PropUniciteLimitePourSuites}. Unicité de la limite d'une fonction, toujours dans le cas d'un espace séparé : proposition \ref{PropFObayrf}.
	\item
	      La proposition~\ref{PropRBCiHbz} donne l'unicité de la limite dans le cas des espaces duaux pour la topologie \( *\)-faible. La proposition~\ref{PropFObayrf} nous dira qu'il y a unicité dès que l'espace d'arrivée est séparé.
	\item
	      Définition de la continuité d'une fonction en un point et sur une partie de l'espace de départ : définition~\ref{DefOLNtrxB}.
	\item
	      Continuité sur une partie si et seulement si continue en chaque point, c'est le théorème~\ref{ThoESCaraB}.
	\item
	      Voir l'exemple~\ref{EXooKREUooLeuIlv} traité en détail pour la (non) continuité d'une fonction qui fait un saut en un point.
	\item
	      La fonction \( f(x,y)=x+y\) est continue, lemme \ref{LEMooGKIPooWgpFTB}.
\end{enumerate}
