\InternalLinks{normes}      \label{THEMEooUJVXooZdlmHj}

\begin{description}
    \item[Définition] Espace vectotiel normé : définition~\ref{DefNorme}.
    \item[Équivalence de norme]

        \begin{enumerate}
        \item
            Définition de l'équivalence de norme~\ref{DefEquivNorm}.
\item
    La proposition~\ref{PropLJEJooMOWPNi} sur l'équivalence des normes \( \| . \|_2\), \( \| . \|_1\) et \( \| . \|_{\infty}\)  dans \( \eR^n\).
\item
     En général pour les normes \( \| . \|_p\), il y a des inégalités dans \ref{THOooPPDPooJxTYIy} et \ref{CORooMBQMooWBAIIH}; voir aussi le thème \ref{THEMEooUJVXooZdlmHj}.
 \item
     La proposition \ref{PROPooQZTNooGACMlQ} donne l'inégalité \( \| x \|_q\leq n^{\frac{1}{ q }-\frac{1}{ p }}\| x \|_p\) dès que \( 0<q<p\).
\item
    Toutes les normes sur un espace vectoriel de dimension finie sont équivalentes, théorème~\ref{ThoNormesEquiv}.
\item
    Montrer que le problème \( a-b\) est stable dans l'exemple~\ref{ExooXJONooTAYZVc}.
\item
    La proposition~\ref{PROPooWZJBooTPLSZp} donnant \( \rho(A)\leq \| A \|\) utilise l'équivalence de toutes les normes sur un espace vectoriel de dimension finie.

        \end{enumerate}

    \item[Norme opérateur et d'algèbre] voir le thème~\ref{THEMEooOJJFooWMSAtL}.

\end{description}
