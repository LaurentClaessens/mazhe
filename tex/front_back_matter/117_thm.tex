\InternalLinks{différentielle}
\begin{description}
	\item[Généralités]
		\begin{enumerate}
			\item
			      La différentielle est définie en général pour des espaces vectoriels normés par la proposition \ref{DefDifferentiellePta}
			\item
			      Différentielle d'une application linéaire, lemme \ref{LEMooZSNMooCfjzOB}.
			\item
			      Nous parlons de différentielle en dimension finie et donnons une interprétation géométrique en~\ref{SEBSECooLPRQooJRQCFL}.
			\item
			      La recherche d'extrémums d'une fonction sur \( \eR^n\) passe par la seconde différentielle, proposition~\ref{PropoExtreRn}.
			\item
			      Lien entre différentielle seconde (hessienne) et convexité en la proposition~\ref{PROPooBMIRooFkQSAb} et le corolaire \ref{CORooMBQMooWBAIIH}.
			\item
			      La différentielle est liée aux dérivées partielles par les formules données au lemme~\ref{LemdfaSurLesPartielles}
			      \begin{equation}
				      df_a(u)=\frac{ \partial f }{ \partial u }(a)=\Dsdd{ f(a+tu) }{t}{0}=\sum_{i=1}^mu_i\frac{ \partial f }{ \partial x_i }(a)=\nabla f(a)\cdot u.
			      \end{equation}
			      Je ne vous cache pas que cette suite d'égalités est une de mes préférées.
		\end{enumerate}
	\item[Différentielle et dérivées partielles]
		À propos de fonctions de classe \( C^k\), définition \ref{DefPNjMGqy}.
		\begin{enumerate}
			\item
			      Une fonction est de classe \( C^1\) si et seulement si ses dérivées partielles sont continues, théorème \ref{THOooBEAOooBdvOdr}.
			\item
			      Une fonction est \( C^n\) si et seulement si ses dérivées partielles sont \( C^{n-1}\), c'est le théorème \ref{THOooPZTAooTASBhZ}.
			\item
			      Différentiabilité en un seul point si les dérivées partielles sont continues en ce point : proposition \ref{PROPooUDJLooHwzjQF}.
		\end{enumerate}
	\item[Fonctions composées]
		À propos de la formule \( d(f\circ g)_a=dg_{f(a)}\circ df_a\), il y a deux théorèmes très semblables.
		\begin{enumerate}
			\item
			      Le théorème \ref{THOooIHPIooIUyPaf} insiste sur des hypothèses locales.
			\item
			      Le théorème \ref{ThoAGXGuEt} fait des hypothèses plus globales pour s'alléger l'esprit, mais fait une récurrence pour dire que \( f\circ g\) est de classe \( C^r\) si \( f\) et \( g\) le sont.
		\end{enumerate}
\end{description}
