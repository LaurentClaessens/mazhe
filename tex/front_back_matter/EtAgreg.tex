% This is part of Mes notes de mathématique
% Copyright (c) 2015-2017
%   Laurent Claessens
% See the file fdl-1.3.txt for copying conditions.

%+++++++++++++++++++++++++++++++++++++++++++++++++++++++++++++++++++++++++++++++++++++++++++++++++++++++++++++++++++++++++++
\section*{Ce cours à l'agrégation ?}
%+++++++++++++++++++++++++++++++++++++++++++++++++++++++++++++++++++++++++++++++++++++++++++++++++++++++++++++++++++++++++++

Peut-on utiliser ce cours pour \textbf{les oraux d'\href{http://agreg.org/}{agrégation}} (de mathématiques) ?

La réponse :
\begin{enumerate}
    \item
        Oui de façon certaine si vous achetez les livres en ligne. Pour savoir comment, allez voir sur la page dédiée de mon site :
        \begin{center}
            \url{http://laurent.claessens-donadello.eu/frido.html}
        \end{center}
    \item
        Non de façon certaine pour télécharger la dernières version du Frido et imprimer vous-même une version modifiée non distribuée.
\end{enumerate}
Entre ces deux extremes, la licence FDL (qui autorise la recommercialisation) vous offre de nombreuses possibilités et variations sur le thème de «j'apporte des modifications au Frido et je le vends sur un site d'impression à la demande». Si ce n'est la lettre, en tout cas l'esprit du règlement de l'agrégation est assez restrictif de ce point de vue.
