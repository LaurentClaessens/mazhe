\InternalLinks{formes bilinéaires et quadratiques}      \label{THEMEooOAJKooEvcCVn}
\begin{enumerate}
	\item
	      Les formes bilinéaires sont définies en~\ref{DEFooEEQGooNiPjHz}.
	\item
	      Forme quadratique, définition \ref{DefBSIoouvuKR}.
	\item
	      Équivalence de forme quadratiques, définition \ref{DEFooOLWYooMwhMJp}. Deux formes quadratiques sont équivalentes si et seulement si elles ont même signature, proposition \ref{PROPooBWXMooLsgyKm}.
	\item
	      Une isométrie d'une forme bilinéaire est affine ou linéaire, théorème \ref{ThoDsFErq}.
	\item
	      Forme bilinéaire dégénérée, définition \ref{DEFooNUBFooLfCqaK}.
	\item
	      Une forme bilinéaire est non-dégénérée si et seulement si sa matrice associée est inversible, c'est la proposition \ref{PROPooQHHPooSqpgcb}.
	\item
	      Une isométrie d'une forme bilinéaire est linéaire ou affine par le théorème \ref{ThoDsFErq}.
	\item
	      Toute forme quadratique admet des bases orthogonales, théorème \ref{THOooIDMPooIMwkqB} pour le cas général; proposition \ref{PROPooUKRUooGRIDHt} pour le cas de \( \eR^n\), en se basant sur le théorème spectral.
	\item
	      Base \( q\)-orthogonale pour une forme quadratique, théorème \ref{THOooIDMPooIMwkqB}.
	\item
	      Le concept de projection orthogonale est la définition \ref{ThoWKwosrH} en dimension finie et la définition \ref{ThoProjOrthuzcYkz} dans le cas des espaces de Hilbert.
	\item
	      Produit hermitien, définition \ref{DefMZQxmQ}. Opérateur hermitien (\( A^{\dag}=A\)), opérateur unitaire (\( A^{\dag}A=1\)), définition \ref{DEFooKEBHooWwCKRK}
\end{enumerate}

\begin{description}
	\item[matrice]\hspace{1cm}
	\begin{enumerate}
		\item
		      Matrice d'une forme quadratique, définition \ref{DEFooAOGPooXWXUcN}.
		\item
		      Théorème de Sylvester à propos de signature (définition \ref{DEFooWDCLooDkRYLK}) de forme quadratique réelle : \ref{ThoQFVsBCk}.
	\end{enumerate}
	\item[orthogonalité]\hspace{1cm}
	\begin{enumerate}
		\item
		      Il existe une base \( q\)-orhogonale, proposition \ref{PROPooRERSooFHwWtB} et théorème \ref{THOooIDMPooIMwkqB}.
		\item
		      Définition de l'orthogonal \( A^{\perp}\) et du noyau \( \ker(b)\), définitions \ref{DEFooENZGooXMWfUy}, \ref{DEFooQQBQooKJdwxO}.
		\item
		      Diverses propriétés comme \( A^{\perp}=\Span(A)^{\perp}\) et \( V\cap V^{\perp}=\ker(b_V)\), propositions \ref{PROPooWPKRooUAnVzd} et \ref{PROPooSACFooQTsiJL}.
		\item
		      En dimension finie, \( (V^{\perp})^{\perp}=V\), lemme \ref{LEMooYYGLooYIDmoa}.
		\item
		      Nous avons \( \dim(V)+\dim(V^{\perp})=\dim(E)\), lemme \ref{LEMooRXMMooAvvOjF}.
		\item
		      Pour une forme quadratique strictement définie positive ou négative, \( E=F\oplus E^{\perp}\), lemme \ref{LEMooUOZOooYvEcji}.
	\end{enumerate}
\end{description}
