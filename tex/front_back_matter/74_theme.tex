\InternalLinks{fonction puissance}      \label{THEMEooBSBLooWcaQnR}

En ce qui concerne la dérivation :
\begin{enumerate}
    \item
        La formule de dérivation pour \( x\mapsto x^q\) avec \( q\in \eQ\) est la proposition \ref{PROPooSGLGooIgzque}. 
    \item
        La dérivation de \( x\mapsto x^{\alpha}\) avec \( \alpha\in \eR\) est la proposition \ref{PROPooKIASooGngEDh}. Si elle est tellement loin, c'est parce qu'elle nécessite de permuter une limite de fonctions avec une dérivée.
    \item
        Pour la formule générale de dérivation de \( x\mapsto a^x\) demande de savoir les logarithmes (proposition \ref{PROPooKUULooKSEULJ}).
\end{enumerate}

Une définition alternative de la fonction puissance serait de poser directement
\begin{equation*}
    a^x=e^{x\ln(a)}.
\end{equation*}
De là les propriétés se déduisent facilement. Dans cette approche, les choses se mettent dans l'ordre suivant :
\begin{itemize}
    \item Définir \( \exp(x)\) par sa série pour tout \( x\).
    \item Démontrer que \( \exp(q)=\exp(1)^q\) pour tout rationnel \( q\) (première partie de la proposition \ref{PropCELWooLBSYmS}).
    \item Définir \( e=\exp(1)\).
    \item Définir, pour \( x\) irrationnel, \( a^x=\exp(x\ln(a))\).
    \item Prouver que \( e^x=\exp(x)\) pour tout \( x\).
\end{itemize}


