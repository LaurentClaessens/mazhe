
\InternalLinks{logarithme}	\label{THEMEooLogarithme}
\begin{enumerate}
	\item
	      Le logarithme pour les réels strictement positifs \( \ln\colon \mathopen] 0 , \infty \mathclose[\to \eR\) est donné en la définition~\ref{DEFooELGOooGiZQjt}; c'est l'application réciproque de \( \exp\).
	\item
	      Les principales propriétés sont dans la proposition \ref{PROPooLAOWooEYvXmI} : \( \ln(xy)=\ln(x)+\ln(y)\) etc.
	\item
	      Dérivée : \( \ln'(x)=\frac{1}{ x }\), proposition \ref{PROPooPDJLooXphpEM}.
	\item
	      La proposition \ref{PROPooKPBIooJdNsqX} donne la série
	      \begin{equation}
		      \ln(1+x)=\sum_{k=1}^{\infty}\frac{ (-1)^{k+1} }{ k }x^k.
	      \end{equation}
	\item
	      L'exemple \ref{EXooYMEEooMGpUNM} donne l'encadrement \( 0.644\leq \ln(2)\leq 0.846\).
	\item
	      La proposition~\ref{PropKKdmnkD} dit que toute matrice complexe admet un logarithme. En particulier une série explicite est donnée pour le logarithme d'un bloc de Jordan\footnote{Jordan, théorème \ref{ThoGGMYooPzMVpe}.}.
	\item
	      Sur les complexes, le logarithme \( \ln \colon \eC^*\to \eC\) est la définition~\ref{DEFooWDYNooYIXVMC}. Attention : ce n'est pas la seule définition possible.
	\item
	      La série harmonique diverge à vitesse logarithmique, et la série des inverses des nombres premiers, c'est encore plus lent : théorème~\ref{ThonfVruT}.
      \item
          Détermination du logarithme le long d'un chemin dans \( \eC\), définition \ref{DEFooOCDGooGyvvWi}.
      \item
          Un logarithme continu d'une fonction, définition \ref{DEFooBBGFooCEdsFR}.
      \item
          Théorème de Borsuk \ref{THOooTCUMooEByCKg} : il y a toujours un logarithme continu le long d'un chemin homotope à une constante (et il y a même équivalence avec admettre une extension sur \( \eC^*\)).
      \item
          Si \( z\in \eC\), nous avons $\int\frac{1}{ x+z }dx=\ln(x+z)$, proposition \ref{PROPooNIJVooKueuYJ}.
      \item
          La proposition \ref{PROPooCFMFooXjlhfV} dit qu'une application \( f\) a un logarithme continu si et seulement si \( \Ind(f\circ\gamma,0)=0\).
\end{enumerate}

