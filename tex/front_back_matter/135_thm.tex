
\InternalLinks{dualité}     \label{THEMEooULGFooPscFJC}

Ne pas confondre dual algébrique et dual topologique d'un espace vectoriel.

\begin{enumerate}
    \item
        Définition du dual, \ref{DefJPGSHpn}.
    \item
        Le dual d'un espace de Banach est de Banach, proposition \ref{PROPooOVGGooNffWJW}.
	\item
	      Définition de la base duale \ref{DEFooTMSEooZFtsqa}.
	\item
	      Base préduale (existence, unicité) : proposition \ref{PROPooDBPGooPagbEB}.
	\item
	      Théorème de représentation de Riez \ref{ThoLPQPooPWBXuv} : \( L^p=(L^q)'\), et en particulier \( L^{\infty}=(L^1)'\).
	\item
	      Il n'est pas vrai que \( (L^{\infty})'=L^1\), voir la proposition \ref{PROPooXXRQooNSBZOi}.
      \item
          Dans un espace de Banach\footnote{Espace de Banach, définition \ref{DefVKuyYpQ}.}, \( \| x \|=\max_{\substack{\varphi\in E'\\\| \varphi \|=1}}| \varphi(x) |\), proposition \ref{PROPooFJPXooWrjbuH}.
\end{enumerate}

\begin{description}
	\item[Dual topologique et algébrique]
	      Ils sont définis par~\ref{DefJPGSHpn}. Le dual algébrique est l'ensemble des formes linéaires, et le dual topologique ne considère que les formes linéaires continues (en dimension infinie, les applications linéaires ne sont pas toutes continues).
	\item[Topologie]
	      Une topologie possible sur le dual d'un espace vectoriel topologique est celle \( *\)-faible de la définition~\ref{DefHUelCDD}.

	      Nous comparons les topologies faibles et de la norme en la section~\ref{SECooKOJNooQVawFY}.
	\item[Théorèmes de dualité]
	      Quelques théorèmes établissent des dualités entre des espaces courants.
	      \begin{enumerate}
		      \item
		            Le théorème de représentation de Riesz~\ref{ThoQgTovL} pour les espaces de Hilbert.
		      \item
		            La proposition~\ref{PropOAVooYZSodR} pour les espaces \( L^p\big( \mathopen[ 0 , 1 \mathclose] \big)\) avec \( 1<p<2\).
		      \item
		            Le théorème de représentation de Riesz~\ref{ThoLPQPooPWBXuv} pour les espaces \( L^p\) en général.
	      \end{enumerate}
	      Tous ces théorèmes donnent la dualité par l'application \( \Phi_x=\langle x, .\rangle \).

\end{description}
