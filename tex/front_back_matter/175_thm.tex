\InternalLinks{changement de variables}
Il n'existe rien en mathémaitque qui s'appelle «changement de variables». Il n'existe que des compositions de fonctions. Ce snobisme terminologique étant, voici un certain nombre de résultats de changement de variables.
\begin{enumerate}
	\item
	      Dans des intégrales, théorème \ref{THOooUMIWooZUtUSg}.
	\item
	      Dans des limites, le lemme \ref{LEMooAHIGooJhpPvo} donne \( \lim_{x\to a} f(x)=\lim_{x\to b}f(x+a-b)\) si la limite existe.
	\item
	      Dans les limites, la proposition \ref{PROPooXWNDooGHDXTa} dit que \( \lim_{x\to b}f(x+a)=\lim_{x\to a+b}f(x)\).
	\item
	      Dans une équation aux dérivées partielles, exemple \ref{EQooPGDPooTjiVhB}.
	\item
	      Limite de fonction composée \( \lim_{x\to a}(f\circ g) \), propositions \ref{PROPooFGWXooFjvTYj} et \ref{PROPooKNVHooXlQyVA}.
\end{enumerate}
