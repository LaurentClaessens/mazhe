
\InternalLinks{extension de corps et polynômes} \label{THEMEooZYKFooQXhiPD}
    \begin{enumerate}
        \item
            Définition d'une extension de corps~\ref{DEFooFLJJooGJYDOe}.
        \item
            Pour l'extension du corps de base d'un espace vectoriel et les propriétés d'extension des applications linéaires, voir la section~\ref{SECooAUOWooNdYTZf}.
        \item
            Extension de corps de base et similitude d'application linéaire (ou de matrices, c'est la même chose), théorème~\ref{THOooHUFBooReKZWG}.
        \item
            Extension de corps de base et cyclicité des applications linéaires, corollaire~\ref{CORooAKQEooSliXPp}.
        \item
            À propos d'extensions de \( \eQ\), le lemme~\ref{LemSoXCQH}.
        \item
            Corps de rupture : définition~\ref{DEFooVALTooDJJmJv} existence par la proposition~\ref{PROPooUBIIooGZQyeE}. Il n'y a pas unicité.
        \item
            Corps de décomposition : définition~\ref{DEFooEKGZooSkvbum}. Attention : le plus souvent nous parlons de corps de décomposition d'un seul polynôme. Cette définition est un peu surfaite. Existence par la proposition~\ref{PROPooDPOYooFHcqkU} qui le donne même comme extension par toutes les racines, et unicité à isomorphisme près par le théorème~\ref{THOooQVKWooZAAYxK}, énoncé de façon plus simple (mais pas indépendante !) en la proposition~\ref{PropTMkfyM}.
    \end{enumerate}

Un trio de résutats d'enfer est :
\begin{enumerate}
    \item
        Dans un anneau principal qui n'est pas un corps, un idéal est maximal si et seulement si il est engendré par un irréductible (proposition~\ref{PropomqcGe}).
    \item
        Dans un anneau, un idéal \( I\) est maximam si et seulement si \( A/I\) est un corps (proposition~\ref{PROPooSHHWooCyZPPw})
    \item
        Si \( \eK\) est un corps, \( \eK[X]\) est principal (lemme~\ref{LEMooIDSKooQfkeKp}).
\end{enumerate}
