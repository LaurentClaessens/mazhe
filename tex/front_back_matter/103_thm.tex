\InternalLinks{normes}      \label{THEMEooUJVXooZdlmHj}

\begin{description}
	\item[Définitions]
	      \begin{enumerate}
		      \item
		            Espace vectoriel normé : définition~\ref{DefNorme}.
		      \item
		            Produit scalaire, définition \ref{DefVJIeTFj}.
		      \item
		            Norme associée à un produit scalaire (cas réel), théorème \ref{ThoAYfEHG}.
		      \item
		            Norme associée à une forme sesquilinéaire\footnote{Définition \ref{DefMZQxmQ}.} (cas complexe), proposition \ref{PROPooSSYJooHAXAnC}.
		      \item
		            Identité du parallélogramme \( \| x+y \|^2+\| x-y \|^2=2\| x \|^2+2\| y \|^2\), proposition \ref{PROPooSSYJooHAXAnC}.
		      \item
		            Produit scalaire sur \( \eR^n\), définition \ref{PROPooSKVRooDGVCYj}. Norme sur \( \eR^n\), définition \ref{DEFooJAGXooMgaUsR}.
		      \item
		            Forme hermitienne sur \( \eC^n\), définition \ref{PROPooMWUCooMbJuaJ}. Norme sur \( \eC^n\), définition \ref{DEFooGUXNooXwCsrq}.
		      \item
		            La proposition \ref{PROPooCLZRooIRxCnZ} donne les normes \( \| x \|_1\), \( \| x \|_2\) et \( \| x \|_{\infty}\) sur \( \eR^n\).
		      \item
		            Sur \( \eR^n\), la proposition \ref{PROPooUDFTooQyhAtq} dit que \( \| x \|_p\) est une norme.
		      \item
		            Norme d'une application \( n\)-multilinéaire, définition \ref{DEFooTAUWooNDJJEO}.
	      \end{enumerate}
	\item[Topologie]
	      \begin{enumerate}
		      \item
		            Métrique associée à une norme \( d(x,y)=\| x-y \|\), définition \ref{LEMooWGBJooYTDYIK}.
		      \item
		            La topologie d'un espace vectoriel normé est la topologie métrique du théorème \ref{ThoORdLYUu}.
		      \item
		            Les boules sont les boules métrique définies en \eqref{EQooYCWSooIhibvd}.
	      \end{enumerate}
	\item[Inégalités]
	      \begin{enumerate}
		      \item
		            En général pour les normes \( \| . \|_p\), il y a des inégalités dans \ref{THOooPPDPooJxTYIy} et \ref{CORooMBQMooWBAIIH}.
		      \item
		            La proposition \ref{PROPooQZTNooGACMlQ} donne l'inégalité \( \| x \|_q\leq n^{\frac{1}{ q }-\frac{1}{ p }}\| x \|_p\) dès que \( 0<q<p\).
	      \end{enumerate}
	\item[Équivalence de norme]

	      \begin{enumerate}
		      \item
		            Définition de l'équivalence de norme~\ref{DefEquivNorm}.
		      \item
		            La proposition~\ref{PropLJEJooMOWPNi} sur l'équivalence des normes \( \| . \|_2\), \( \| . \|_1\) et \( \| . \|_{\infty}\)  dans \( \eR^n\).
		      \item
		            Toutes les normes sur un espace vectoriel de dimension finie sont équivalentes par le théorème \ref{ThoNormesEquiv}.
	      \end{enumerate}
	\item[Autres]
	      \begin{enumerate}
		      \item
		            Montrer que le problème \( a-b\) est stable dans l'exemple~\ref{ExooXJONooTAYZVc}.
		      \item
		            La proposition~\ref{PROPooWZJBooTPLSZp} donnant \( \rho(A)\leq \| A \|\) utilise l'équivalence de toutes les normes sur un espace vectoriel de dimension finie (théorème \ref{ThoNormesEquiv}.).
		      \item
		            La norme \( x\mapsto \| x \|\) est une application continue, proposition \ref{PROPooYMCUooERvDpk}.
	      \end{enumerate}
	\item[Norme opérateur et d'algèbre] voir le thème~\ref{THEMEooOJJFooWMSAtL}.

\end{description}
