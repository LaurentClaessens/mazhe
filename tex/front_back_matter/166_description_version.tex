% Voir aussi le fichier 205_version_description.tex pour la version
% en anglais.

\thispagestyle{empty}

Plusieurs versions et extensions de ce document.
\begin{description}

    \item[La version courante]

        Vous trouverez une version dédiée à l'agrégation régulièrement mise à jour à l'adresse suivante :
        \begin{center}
            \url{http://laurent.claessens-donadello.eu/pdf/lefrido.pdf}
        \end{center}

    \item[Aux oraux d'agrégation]

        Une version est en vente en 4 volumes, voir la page dédiée
        \begin{center}
            \url{http://laurent.claessens-donadello.eu/frido.html}
        \end{center}
        ainsi que l'erratum :
        \begin{center}
            \url{https://tuleap.net/plugins/git/lefrido/lefrido?a=blob&f=erratum.md}\\
            \url{https://github.com/LaurentClaessens/mazhe/blob/master/erratum.md}
        \end{center}

    \item[La version la plus complète]

        Une version plus complète, comprenant le Frido, des exercices ainsi que de la mathématique de niveau recherche :
        \begin{center}
            \url{http://laurent.claessens-donadello.eu/pdf/giulietta.pdf}
        \end{center}

    \item[Tout ce qu'il faut savoir pour recompiler soi-même]
        Pour savoir comment recompiler ce document à l'identique, il faut lire
        \begin{center}
            \url{https://tuleap.net/plugins/git/lefrido/lefrido?a=blob&f=COMPILATION.md}\\
            \url{https://github.com/LaurentClaessens/mazhe/blob/master/COMPILATION_frido.md}
            \url{https://github.com/LaurentClaessens/mazhe/blob/master/COMPILATION_giulietta.md}
        \end{center}

\end{description}
