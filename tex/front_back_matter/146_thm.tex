
\InternalLinks{rang}	\label{THEMEooDimensionRang}
\begin{enumerate}
	\item Définition pour une application linéaire : \ref{DefALUAooSPcmyK}, pour une matrice : \ref{DEFooCSGXooFRzLRj}. L'équivalence est la proposition \ref{PROPooEGNBooIffJXc}.
	\item Le théorème du rang, théorème~\ref{ThoGkkffA}
	\item Pour une applicaton linéaire entre deux espaces vectoriels de même dimension finie, il est équivalent d'être injectif, surjectif ou bijectif, c'est le corolaire \ref{CORooCCXHooALmxKk}.
	\item Pour prouver que des matrices sont équivalentes et pour les mettre sous des formes canoniques, nous avons le lemme \ref{LemZMxxnfM} et son corolaire \ref{CorGOUYooErfOIe}.
	\item Tout hyperplan de \( \eM(n,\eK)\) coupe \( \GL(n,\eK)\), corolaire~\ref{CorGOUYooErfOIe}. Cela utilise la forme canonique sus-mentionnée.
	\item Le lien entre application duale et orthogonal de la proposition~\ref{PropWOPIooBHFDdP} utilise la notion de rang.
	\item Le lemme \ref{LEMooDFFDooJTQkRu} parle de commutant et utilise la notion de rang. Ce lemme sert à prouver diverses conditions équivalentes à être un endomorphisme cyclique dans le théorème \ref{THOooGLMSooYewNxW}.
\end{enumerate}
