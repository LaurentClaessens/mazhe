\InternalLinks{suites et séries}

\begin{description}
	\item[Opérations sur les suites]
		\begin{enumerate}
			\item
			      Les limites \( x_k+y_k\to x+y\) et \( \lambda x_k\to \lambda x\) sont la proposition \ref{PROPooZRCBooKiJhDg}.
			\item
			      La limite \( x_ky_k\to xy\) est la proposition \ref{PROPooIQOAooJPMoDD}.
		\end{enumerate}
	\item[Suites]
		La proposition \ref{PROPooOSXCooJWXkWH} donne une caractérisation de limite de suite dans un espace vectoriel normé.
		\begin{enumerate}
			\item
			      Les suites adjacentes, c'est la définition \ref{DEFooDMZLooDtNPmu}.
			\item
			      Les séries alternées, théorème \ref{THOooOHANooHYfkII}. Il s'agit de dire que \( \sum_{k=0}^{\infty}(-1)^ka_k\) converge quand \( a_k\) est décroissante et tend vers zéro.
			\item
			      Le concept de suite adjacente sert à étudier la série de Taylor de \( \ln(x+1)\), voir le lemme \ref{LEMooWMGGooRpAxBa} et ce qui l'entoure.
			\item
			      La définition de la convergence absolue est la définition~\ref{DefVFUIXwU}.
			\item
			      Une suite réelle croissante et majorée converge, proposition \ref{LemSuiteCrBorncv}.
			\item
			      Toute suite dans un compact admet une sous-suite convergente, théorème \ref{THOooRDYOooJHLfGq}.
			\item
			      Pour tout réel, il existe une suite croissante de rationnels qui y converge, proposition \ref{PropSLCUooUFgiSR}.
		\end{enumerate}
	\item[Produit de Cauchy]
		\begin{enumerate}
			\item
			      Dans une algèbre normée, proposition \ref{PROPooFMEXooCNjdhS},
			\item
			      Dans \( \eC\), théorème \ref{ThokPTXYC}.
		\end{enumerate}
	\item[Calcul de suites]
		\begin{enumerate}
			\item
			      Somme : \( x_n+y_n\to x+y\) est la proposition \ref{PROPooICZMooGfLdPc}.
		\end{enumerate}
	\item[Série]
		Les séries sont en général dans la section \ref{SECooYCQBooSZNXhd}.
		\begin{enumerate}
			\item
			      Quelques séries usuelles en \ref{SUBSECooDTYHooZjXXJf} : série harmonique, géométrique, de Riemann, et la mythique arithmético-géométrique.
			      \begin{enumerate}
				      \item
				            La série est associative : \( \sum_k(a_k+b_k)=\sum_ka_k+\sum_kb_k\). C'est la proposition \ref{PROPooUEBWooUQBQvP}.
				      \item
				            La série harmonique diverge : \( \sum_k\frac{1}{ k }=\infty\), exemple \ref{PROPooBAIWooKxMLvh}.
				      \item
				            La série géométrique : \( \sum_{k=0}^Nq^k=\frac{ 1-q^{N+1} }{ 1-q }\), proposition \ref{PROPooWOWQooWbzukS}.
				      \item
				            Une autre cool série : \( \sum_{k=1}^N\frac{ 1 }{ k(k+1) }=\frac{ N }{ N+1 }\), lemme \ref{LEMooKDHPooPlFTIT}.
			      \end{enumerate}
			\item
			      Critère des séries alternées, théorème \ref{THOooOHANooHYfkII}.
			\item
			      Convergence d'une série implique convergence vers zéro du terme général, proposition~\ref{PROPooYDFUooTGnYQg}.
			\item
			      Dans une algèbre normée : \( (\sum_{k=0}^{\infty}a_k)b=\sum_{k=0}^{\infty}(a_kb)\), proposition \ref{PROPooMZZQooEhQsgQ}.
			\item
			      Produit de Cauchy : théorème \ref{ThokPTXYC} et proposition \ref{PROPooFMEXooCNjdhS}.
		\end{enumerate}

	\item[Sommes infinies]
		En ce qui concerne les sommes finies, la notation \( \sum_{i=1}^N\) est définie en \ref{DEFooNEVNooJlmJOC}. Pour permuter les termes d'une somme avec un élément du groupe symétrique, nous avons la proposition \ref{PROPooJBQVooNqWErk}.

		Voici quelques résultats à propos de sommes infinies.% cette phrase est là pour le mot-clef ``somme infinie''.
		\begin{enumerate}
			\item
			      Une somme indexée par un ensemble quelconque est la définition \ref{DefIkoheE}.
			\item
			      La proposition \ref{PROPooOYNRooQFpBly} donne une caractérisation pour le sommes de réels positifs.
			\item
			      La définition de la somme d'une infinité de termes est donnée par la définition~\ref{DefGFHAaOL}.
			\item
			      Une somme de termes positifs indexée par un ensemble indénombrable est toujours infinie par le lemme \ref{LEMooQIMGooOUpZjk}.
			\item
			      si la série converge, on peut regrouper ses termes sans modifier la convergence ni la somme (associativité);
			      pour les sommes infinies l'associativité et la commutativité dans une série sont perdues. Néanmoins, il subsiste que
			      \begin{enumerate}
				      \item
				            si la série converge absolument, on peut modifier l'ordre des termes sans modifier la convergence ni la somme (commutativité, proposition~\ref{PopriXWvIY}).
			      \end{enumerate}
			\item Permuter une somme infinie avec une application linéaire : \( f(\sum_{i\in I}v_i)=\sum_{i\in I}f(v_i)\), c'est la proposition \ref{PROPooWLEDooJogXpQ}.
		\end{enumerate}
	\item[Série entières]
		\begin{enumerate}
			\item
			      Rayon de convergence, définition \ref{DefZWKOZOl}.
			\item
			      Convervence absolue à l'intérieur du rayon de convergence, lemme d'Abel \ref{LemmbWnFI}.
			\item
			      La fonction définie par la série entière  \(z\mapsto \sum_{k=0}^{\infty}a_nz^n\) est holomorphe dans son disque de convergence par la proposition \ref{PropSNMEooVgNqBP}.
			\item
			      La série entière pour \( \frac{1}{ 1-z^k }\), pour \( \frac{1}{ \omega-z }\) et pour \( \frac{1}{ (\omega-z)^k }\) sont dans le lemme \ref{LemPQFDooGUPBvF}.
		\end{enumerate}
\end{description}
