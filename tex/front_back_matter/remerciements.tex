%+++++++++++++++++++++++++++++++++++++++++++++++++++++++++++++++++++++++++++++++++++++++++++++++++++++++++++++++++++++++++++
\section{Auteurs, contributeurs, sources et remerciements}
\label{SECooINTROremerciements}
%+++++++++++++++++++++++++++++++++++++++++++++++++++++++++++++++++++++++++++++++++++++++++++++++++++++++++++++++++++++++++++

Les remerciements, dans chaque catégorie, sont mis dans l'ordre chronologique approximatif.

%---------------------------------------------------------------------------------------------------------------------------
\subsection{Ceux qui ont travaillé sur le Frido}
\label{SUBooINTROremerciementsFrido}
%---------------------------------------------------------------------------------------------------------------------------

\begin{enumerate}
	\item
	      Carlotta Donadello pour l'ensemble du cours de CTU de géométrie analytique 2010-2011. Une grosse partie de «analyse réelle» vient de là.
	\item
	      Les étudiants de géométrie analytique en CTU 2010-2011 ont détecté d'innombrables coquilles. Les étudiants du cours présentiel de géométrie analytique 2011-2012 ont signalé un certain nombre d'incorrections dans les exercices et les corrigés. Les agrégatifs de Besançon 2011-2012 pour leurs plans et leurs développements.
	\item
	      Lilian Besson pour m'avoir signalé un paquet de fautes, et quelques points pas clairs en statistiques.
	\item
	      Plouf qui m'a signalé une coquille dans le fil \href{http://passeurdesciences.blog.lemonde.fr/2014/01/24/la-selection-scientifique-de-la-semaine-numero-106}{la-selection-scientifique-de-la-semaine-numero-106}.
	\item
	      Benjamin de Block pour des coquilles et une mise au point sur les conventions à propos de \( \eR^+\) et \( (\eR^+)^*\).
	\item
	      Olivier Garet pour avoir répondu à plein de questions de probabilités.
	\item
	      François Gannaz pour de la relecture et une version plus claire de la preuve (et de l'énoncé) de la proposition~\ref{PROPooIOQEooGMcCJm}.
	\item
	      Danarmk pour des réponses à des questions dans les commentaires (allongement pour éviter un Overfull hbox) \url{http://linuxfr.org/nodes/110155/comments/1675589}. Et aussi pour \href{ https://github.com/LaurentClaessens/mazhe/issues/87 }{ une discussion } à propos de la topologie sur \( \swD(\Omega)\).
	\item
	      Cédric Boutilier pour des réponses à des questions de probabilité statistique. \url{https://github.com/LaurentClaessens/mazhe/issues/16}
	\item
	      Remsirems pour des réponses à des questions d'analyse\footnote{\url{http://linuxfr.org/nodes/110155/comments/1675813}}
	\item
	      Bertrand Desmons pour plusieurs patchs rendant plus clairs de nombreux passages sur les suites de Cauchy dans \( \eQ\).
	\item
	      Anthony Ollivier pour m'avoir fait remarquer qu'il n'est pas vrai que \( A[X]\) est euclidien lorsque \( A\) est intègre (contre-exemple : \( A=\eZ\)). Ça fait une faute de moins dans le Frido.
	\item
	      ybailly pour avoir détecté un bon nombre de coquilles dans la partie sur les ensembles de nombres.
	\item
	      Éric Guirbal pour le remplacement de \info{frenchb} par \info{french}.
	\item
	      cdrcprds pour une réponse à une question d'algèbre, démonstration à l'appui à propos de \href{https://github.com/LaurentClaessens/mazhe/issues/52#issuecomment-333251728}{pgcd}. Également pour sa relecture sans pitié de la partie sur la cardinalité (en particulier \( A\approx A\setminus B\)) et pour avoir pointé l'utilité du théorème de comparabilité cardinale.
	\item
	      Antoine Bensalah pour avoir répondu à une question sur Lax-Milgram tout en même temps que pointé une erreur dans la démonstration et fourni l'exemple~\ref{EXooTTBDooUNhBOc} sur l'optimalité de l'inégalité.
	\item
	      Guillaume Deschamps pour ses remarques à propos du fait que le chapitre «constructions des ensembles» est très ardu.
	\item
	      Guillaume Barriere pour sa relecture attentive jusqu'aux corps.
	\item
	      Samy Clementz pour avoir découvert une faute dans la définition de mesure positive sur un espace mesurable.
	\item
	      Sylvain Rousseau pour avoir clarifié une construction dans le théorème de Bower version \(  C^{\infty}\).
	\item
	      Maxmax pour des typos dans l'index thématique.
	\item
	      Laurent Choulette pour une typo dans les propriétés du neutre d'un groupe.
	\item
	      Pierre Lairez pour la démonstration du théorème d'inversion de limite et de dérivée \ref{THOooXZQCooSRteSr} sans passer par les intégrales (et les lemmes correspondants à propos du module de continuité).
	\item
	      Gregory Berhuy pour des réponses d'algèbres dans les catégories facile, moyen et difficile.
	\item
	      Benoît Tran pour avoir signalé un paquet de typos dans la démonstration de l'ellipsoïde de John-Loewner et ses dépendances.
	\item
	      Provatiscus pour avoir signalé un paquet de choses pas claires, et surtout pour avoir trouvé une faute dans la démonstration du fait qu'une fonction continue sur \( \eQ\) se prolonge en une fonction continue sur \( \eR\). Et pour cause : cet énoncé est faux. \url{https://github.com/LaurentClaessens/mazhe/issues/124}
	\item
	      William pour l'environnement \info{example} qui gère correctement le triangle.
	\item
	      Colin Pitrat pour de nombreuses remarques, typos et relecture de théorèmes.
	\item
	      Bruno Turgeon pour une très belle moisson de fautes de frappe (euphémisme pour dire «mon ignorance crasse de l'orthographe»).
	\item
	      Sacha Dhénin pour une belle quantité de fautes de frappes et pour avoir soulevé quelques points pas clairs (par exemple la définition du sous-groupe engendré dans le cas de la partie vide).
	\item
	      Patrice Goyer pour m'avoir signalé quelques fautes et pas mal de points pas clairs dans les polynômes et dans la théorie généraliste des ensembles. Et pour m'avoir fait remarquer (deux fois) que mon script de déploiement ne marchait pas.
	\item
	      Alain Vigne pour une quantité (presque) indénombrable de fautes d'orthographe et de mauvaises tournures de phrase dans les espaces vectoriels, la construction de nombres, la théorie des groupes et les anneaux. Il m'a également pointé quelques fautes et points vraiment pas clairs dans des démonstrations dont la correction a permis de bien améliorer la qualité du texte.
	\item
	      Quentin Guyot pour une quantité de typos proche du nombre de Graham dont beaucoup me semblent impossible à détecter sans une lecture attentive; (au moins) huit entrées dans l'erratum lui sont dues (au sens où c'est lui qui les a trouvées, pas qu'il les a causées).
	\item
	      Jean Abou Samra pour la démonstration de la connextité par arcs \( C^1\).
	\item
	      jperon pour le classement de l'index en comptant les é comme des é. Voir \cite{BIBooHVJLooVHqAss}.
\end{enumerate}

%---------------------------------------------------------------------------------------------------------------------------
\subsection{Aide directe, mais pas volontairement sur le Frido}
\label{SUBooINTROremerciementsDirecte}
%---------------------------------------------------------------------------------------------------------------------------

\begin{enumerate}
	\item
	      Plein de monde pour diverses contributions à des énoncés d'exercices. Pierre Bieliavsky pour les énoncés d'analyse numérique (MAT1151 à Louvain la Neuve 2009-2010). Jonathan Di Cosmo pour certaines corrections de MAT1151. François Lemeux, exercices sur les normes de matrices et correction de coquilles.  Martin Meyer et Mustapha Mokhtar-Kharroubi pour certains exercices du cours \emph{Outils mathématiques} (surtout ceux des DS et examens).
	\item
	      Nicolas Richard et Ivik Swan pour les parties des exercices et rappels de calcul différentiel et intégral (Université libre de Bruxelles, 2003-2004) qui leurs reviennent.
	\item
	      Carlotta Donadello pour la partie géométrie analytique : topologie dans \( \eR^n\), courbes, intégrales, limites. (Université de Franche-Comté 2010-2012)
	\item
	      Le forum usenet de math, en particulier pour la construction des corps fini dans le fil « Vérifier qu'un polynôme est primitif » initié le 20 décembre 2011.
	\item
	      Mihai Bostan nous a donné ses notes manuscrites de son cours présentiel de géométrie analytique 2009-2010. (Presque) Toute la structure du cours de géométrie analytique lui est due (qui est maintenant fondue un peu partout dans les chapitres d'analyse).
\end{enumerate}

%---------------------------------------------------------------------------------------------------------------------------
\subsection{Des gens qui ont fait un travail qui m'a bien servi}
\label{SUBooINTROremerciementsAutres}
%---------------------------------------------------------------------------------------------------------------------------

\begin{enumerate}
	\item
	      Arnaud Girand pour avoir mis ses développements bien faits en ligne. Une bonne vingtaine de résultats un peu partout dans ces notes viennent de lui.
	\item
	      Le site \url{http://www.les-mathematiques.net} m'a donné les preuves de nombreux résultats.
	\item
	      Pierre Monmarché pour son document en ligne tout plein de développements, et des réponses à des questions.
	\item
	      Tous les contributeurs du Wikipédia francophone (et aussi un peu l'anglophone) doivent être remerciés. J'en ai pompé des quantités astronomiques; des articles utilisés sont cités à divers endroits du texte, mais ce n'est absolument pas exhaustif.
	\item
	      Les intervenants du fil «\href{http://www.les-mathematiques.net/phorum/read.php?2,302266}{Antisymétrisation, alterné, déterminant et caractéristique}» sur \texttt{les-mathematiques.net} m'ont bien aidé pour la section sur les déterminants~\ref{SecGYzHWs} (bien qu'ils ne le savent pas).
	\item
	      Xavier Mauquoy pour l'énoncé et la preuve du théorème~\ref{THOooYXJIooWcbnbm}.
	\item
	      David Revoy pour les dessins de Pepper\&Carrot \href{https://www.peppercarrot.com/fr/article285/episode-8-pepper-s-birthday-party}{de la couverture}.
\end{enumerate}

J'ai souvent donné entre parenthèses à côté des mots « théorème », « lemme » ou « proposition » une ou plusieurs références vers les sources de la preuve que je donne. Ce sont parfois des liens vers la bibliographie; parfois aussi des liens hypertextes vers des sites, des blogs, etc. Tous ces gens ont fait du bon boulot. Sans toute cette « communauté », l'internet serait mort\footnote{Cette dernière phrase doit être comprise comme un appel à ne pas utiliser Moodle et autres iCampus pour diffuser vos cours de math, ou en tout cas pas comme moyen exclusif.}.

%+++++++++++++++++++++++++++++++++++++++++++++++++++++++++++++++++++++++++++++++++++++++++++++++++++++++++++++++++++++++++++
\section{Originalité}
\label{SECooINTROoriginalite}
%+++++++++++++++++++++++++++++++++++++++++++++++++++++++++++++++++++++++++++++++++++++++++++++++++++++++++++++++++++++++++++

Ces notes ne sont pas originales par leur contenu : ce sont toutes des choses qu'on trouve facilement sur internet; je crois que la bibliographie est éloquente à ce sujet. Ce cours se distingue des autres sur les points suivants.
\begin{description}
	\item[La longueur] J'ai décidé de ne pas me soucier de la taille du fichier. Il fera cinq mille pages s'il le faut, mais il restera en un bloc. Étant donné qu'il n'existe qu'une seule mathématique, il ne m'a pas semblé intéressant de produire une division artificielle entre l'analyse, la géométrie ou l'algèbre. Tous les résultats d'une branche peuvent (et sont) être utilisés dans toutes les autres branches.

	      Dans cette optique, je me suis évertué à ne créer que des références «vers le haut». À moins d'oubli de ma part\footnote{Par exemple pour les théorèmes pour lesquels je n'ai pas lu ni a fortiori écrit de preuves.}, il n'y a aucun endroit du texte qui dépend d'un lemme démontré plus bas. Le fait qu'un théorème \( B\) soit plus bas qu'un théorème \( A\) signifie qu'on peut démontrer \( A\) sans savoir \( B\).

	\item[La licence] Ce document est publié sous une licence libre. Elle vous donne explicitement le droit de copier, modifier et redistribuer.

	\item[Les mises à jour] Ce document est régulièrement mis à jour. Des fautes d'orthographe sont corrigées (presque) chaque jour. Si vous me signalez une faute de mathématique, elle sera corrigée.
	\item[Transparence] Je ne fais pas semblant que ces notes soient parfaites. Les points sur lesquels je ne suis pas sûr, les preuves que j'ai inventées moi-même sont clairement indiqués pour inciter le lecteur à redoubler de prudence. Une liste de questions à résoudre est incluse en la section~\ref{SecooCKWWooBFgnea}. Voir \ref{SECooWVHBooCaYoXP} pour plus de détails.
\end{description}

%+++++++++++++++++++++++++++++++++++++++++++++++++++++++++++++++++++++++++++++++++++++++++++++++++++++++++++++++++++++++++++
\section{Les choses qui doivent vous faire tiquer}
\label{SECooWVHBooCaYoXP}
%+++++++++++++++++++++++++++++++++++++++++++++++++++++++++++++++++++++++++++++++++++++++++++++++++++++++++++++++++++++++++++

Un cours de math doit toujours être lu attentivement, surtout si vous avez l'intention de resservir à un jury le fruit de vos lectures. Dans ce livre, trois éléments doivent vous faire redoubler de prudence.

\begin{description}
	\item[La référence \cite{MonCerveau}]
	      D'abord les références à \cite{MonCerveau} indiquent qu'une bonne partie de ce qui suit est de l'invention personnelle de l'auteur. Cela ne veut évidemment pas dire que c'est moi qui ai découvert le résultat. Ça veut dire que je n'ai pas trouvé le résultat ou certaines parties de la preuve.
	\item[Les notes en bas de page]
	      Certaines notes en bas de page sont écrites dans une fonte spéciale\quext{Les notes comme celle-ci signifient qu'il y a certaines choses dont je ne suis pas sûr.}. Elles indiquent des points sur lesquels je doute ou des étapes de calculs que je ne parviens pas à reproduire en suivant mes sources. Lorsque vous voyez une telle note, redoublez de prudence, allez voir la source, et écrivez-moi si vous pouvez résoudre le problème.
	\item[Les environnements dédiés]
	      Et enfin certains problèmes sont indiqués plus longuement dans un environnement dédié en petits caractères comme ceci :

	      \begin{probleme}
		      Les choses écrites comme ceci sont des questions ou des éléments sur lesquels j'ai un doute. Lisez-les attentivement. Ces notes mentionnent des points que personnellement je n'oserais pas affirmer plein d'aplomb à un jury d'agrégation.
	      \end{probleme}
\end{description}

%+++++++++++++++++++++++++++++++++++++++++++++++++++++++++++++++++++++++++++++++++++++++++++++++++++++++++++++++++++++++++++
\section{Quelques choix qui peuvent provoquer des quiproquos}
\label{SECooINTROquiproquos}
%+++++++++++++++++++++++++++++++++++++++++++++++++++++++++++++++++++++++++++++++++++++++++++++++++++++++++++++++++++++++++++

Comme tout cours de mathématique, ce cours fait des choix qui sont parfois discutables. Voici quelques points sur lesquels les choix faits ici ne sont peut-être pas ceux fait par tout le monde. Ce sont donc des points sur lesquels vous devez faire attention pour éviter les quiproquos lors par exemple d'un oral dans un concours.

\begin{enumerate}
	\item
	      Nous utilisons la définition «épointée» de limite d'une fonction en un point. Elle diffère de celle donnée par le ministère de l'enseignement en France. Si votre but est de passer un concours d'enseignement en France, vous devriez lire~\ref{SECooNJSGooGSAtdV}; dans tous les autres cas, la définition prise ici est celle qu'il vous faut.
	\item
	      Un compact est une partie d'un espace topologique pour lequel tout recouvrement par des ouverts admet un sous-recouvrement fini. Le fait d'être séparable n'est pas inclus dans la définition de compact. De nombreux textes français incluent la séparabilité dans la compacité.
	\item
	      Le logarithme sur \( \eC\) est une application \( \ln\colon \eC^*\to \eC\) définie partout sauf en zéro. Elle n'est donc pas continue sur la fameuse demi-droite. À ne pas confondre avec une \emph{détermination} du logarithme qui est par définition continue et donc non définie sur la demi-droite.

	      Cela est un choix très discutable. La raison de donner à la notation «\( \ln\)» cette signification est simplement de suivre l'usage de Sage. Pour Sage, \( \ln(-1)\) existe et vaut \( i\pi\).

	      Voir les remarques~\ref{REMooFBLLooDnkmjR}.
	\item
	      Le mot «corps» n'implique pas la commutativité bien que tous les corps du Frido soient commutatifs, et nous n'utilisons pas la terminologie «anneau à division». Voir la section \ref{SECooPBZVooCVInFT} et la discussion~\ref{NORMooGPWRooIKJqqw}.
\end{enumerate}

%+++++++++++++++++++++++++++++++++++++++++++++++++++++++++++++++++++++++++++++++++++++++++++++++++++++++++++++++++++++++++++ 
\section{Autres choix pas spécialement standards}
\label{SECooINTROchoixPasStandards}
%+++++++++++++++++++++++++++++++++++++++++++++++++++++++++++++++++++++++++++++++++++++++++++++++++++++++++++++++++++++++++++

Nous listons ici quelques choix qui n'induisent pas de différences ou d'incompatibilité avec les autres cours, mais qui doivent être compris et justifiés.

\begin{enumerate}
	\item
	      Nous n'utilisons pas les notations \( o(x)\) ou autres \( O(N^2)\). D'abord parce que je n'ai jamais très bien compris comment elles fonctionnent, et ensuite (surtout) parce que ces notations induisent en erreur. Ce sont des notations qui cachent, sous des notations a peu près intuitive, l'utilisation de théorèmes pas simples.

	      Écrire
	      \begin{equation}
		      f(x)=P(x)+o(x^2),
	      \end{equation}
	      c'est un peu comme quand on écrit (horreur !)
	      \begin{equation}
		      F(x)=\int f(x)dx+C.
	      \end{equation}
	      Où est le \( x\) à droite ? Quel est le statut de \( C \) ?

	      Même chose pour la notation \( f(x)=P(x)+o(x^2)\). Le \( x\) de \( o(x^2)\) est-il le \( x\) qu'on a à gauche ? Si \( g(x)=Q(x)+o(x^2)\), est-ce le même \( o\) que celui de \( f\) ?
	\item
	      Nous allons être plus calme avec la notation \( A[X]\) pour les polynômes sur l'anneau \( A\), et encore moins \( A[X_1,\ldots, X_n]\) pour les polynômes de \( n\) variables. Au lieu de cela nous utilisons \( \Poly(A)\) et \( \Poly_n(A)\).

	      Est-ce que vous diriez que \( A[X]=A[Y]\) ? Quelle est exactement la nature de \( X\) dans \( P=X^2+1\) ou dans \( P(X)=X^2+1\) ? Si \( P\in A[X]\) vaut \( P(X)=X^2\)  et si \( Q\in A[Y]\) vaut \( Q(Y)=Y^2\), est-ce que vous oseriez écrire \( P=Q\) ?
\end{enumerate}

%--------------------------------------------------------------------------------------------------------------------------- 
\subsection{Mathématique intéressante}
\label{SUBooINTROmathInteressantes}
%---------------------------------------------------------------------------------------------------------------------------

\begin{definition}      \label{DEFooDABVooKdDyBw}
	Une notion mathématique est \defe{intéressante}{intéressante} si elle permet de répondre à une question que l'on peut se poser sans connaitre la notion.
\end{definition}

\begin{example}	\label{EXEooINTROmathInteressantes1}
	Étant donné un segment dans le plan, quels sont les triangles rectangles dont ce segment est l'hypoténuse ?

	Nous n'avons pas besoin de cercles pour poser cette question. Mais nous avons besoin de connaissances sur les cercles pour y répondre. Donc l'étude des cercles est intéressante.
\end{example}

\begin{example}	\label{EXEooINTROmathInteressantes2}
	Comment fonctionne la gravitation ?

	Cette question peut être posée sans connaitre de calcul tensoriel, d'équations différentielles ou d'intégrales. Et pourtant, tous ces concepts sont utiles pour y répondre.
\end{example}

%+++++++++++++++++++++++++++++++++++++++++++++++++++++++++++++++++++++++++++++++++++++++++++++++++++++++++++++++++++++++++++
\section{Sage est là pour vous aider}
\label{SECooINTROsage}
%+++++++++++++++++++++++++++++++++++++++++++++++++++++++++++++++++++++++++++++++++++++++++++++++++++++++++++++++++++++++++++

Il existe de nombreux logiciels de mathématique. Notre préféré est \href{http://www.sagemath.org}{Sage} pour une raison très précise : Sage est (en simplifiant) un module pour Python. Donc quand on travaille en Sage, on dispose de tout Python. La syntaxe et la structure de Sage ne sont pas \emph{ad hoc} pour faire des maths, et ce qu'on apprend en Sage peut être recyclé pour faire n'importe quoi : navigateur web, script de manipulation de texte, traitement d'image, réseau neuronaux, \ldots

%Par ailleurs, le vingt et unième siècle est déjà largement entamé; si vous vous lancez dans une carrière scientifique, il vous faudra maitriser l'informatique un peu plus solidement qu'être virtuose es trouver le trajet le plus court en bus sur votre téléphone.

Sage est un logiciel disponible pour l'épreuve de modélisation de l'agrégation de mathématique; il y a donc de bonnes chances que vous en ayez l'usage.

%---------------------------------------------------------------------------------------------------------------------------
\subsection{Lancez-vous dans Sage}
\label{SUBooINTROsageLancezVous}
%---------------------------------------------------------------------------------------------------------------------------


\begin{enumerate}
	\item
	      Aller sur \url{http://www.sagemath.org},
	\item
	      créer un compte,
	\item
	      créer des feuilles de calcul et s'amuser !!
\end{enumerate}

Il y a beaucoup de \href{http://lmgtfy.com/?q=sage+documentation}{documentation} sur le \href{http://www.sagemath.org}{site officiel}\footnote{\href{http://www.sagemath.org}{http://www.sagemath.org}}, et nous vous conseillons particulièrement le livre \cite{ooBLMMooWTPsQy}.

Si vous comptez utiliser régulièrement ce logiciel, je vous recommande \emph{chaudement} de \href{http://mirror.switch.ch/mirror/sagemath/index.html}{l'installer} sur votre ordinateur.

%---------------------------------------------------------------------------------------------------------------------------
\subsection{Exemples de ce que Sage peut faire pour vous}
\label{SUBooINTROsageExemples}
%---------------------------------------------------------------------------------------------------------------------------

Ce livre est émaillé de petits bouts de code en Sage montrant ses différentes fonctionnalités là où nous en avons besoin\footnote{Soit un vrai besoin comme tracer un graphique en 3D, soit de la paresse comme calculer une grosse dérivée.}. Voici une liste (non exhaustive) de ce que Sage peut faire pour vous.

\begin{enumerate}

	\item
	      Calculer des limites de fonctions, exemples~\ref{ExBCRXooDVUdcf} et~\ref{ExCWDRooKxnjGL}.
	\item
	      Tracer des graphes de fonctions, exemple~\ref{ExCWDRooKxnjGL}.
	\item
	      Tracer des courbes en trois dimensions, voir exemple~\ref{ExempleTroisDxxyy}. Notez que pour cela vous devez installer aussi le logiciel Jmol. Pour Ubuntu, c'est dans le paquet \info{icedtea6-plugin}.
	\item
	      Calculer des dérivées partielles de fonctions à plusieurs variables, voir exemple~\ref{exJMGTooZcZYNy}.
	\item
	      Résoudre des systèmes d'équations linéaires. Voir les exemples~\ref{exKGDIooVefujD} et~\ref{ExBGCEooPIQgGW}. Lire aussi \href{http://www.sagemath.org/doc/constructions/linear_algebra.html#solving-systems-of-linear-equations}{la documentation}.
	\item
	      Tout savoir d'une forme quadratique, voir exemple~\ref{exBNGVooIvKfTT}.
	\item
	      Calculer la matrice hessienne de fonctions de deux variables, déterminer les points critiques, déterminer le genre de la matrice hessienne aux points critiques et écrire les extrémums de la fonctions (sous réserve d'être capable de résoudre certaines équations), voir les exemples~\ref{exZHGRooTQpVpq} et~\ref{exHWIHooOAvaDQ}.
	\item
	      Indiquer une infinité de solutions à une équation en utilisant des paramètres. L'exemple \ref{exEEHPooKDxLTJ} montre ça avec une équation algébrique. Un exemple concernant des fonctions trigonométriques :
	      \begin{verbatim}
sage: solve(sin(x)/cos(x)==1,x,to_poly_solve=True)
[x == 1/4*pi + pi*z1]
sage: solve(sin(x)**2==cos(x)**2,x,to_poly_solve=True)
[sin(x) == cos(x), x == -1/4*pi + 2*pi*z86, x == 3/4*pi + 2*pi*z84]
        \end{verbatim}

	      Notez l'option \info{to\_poly\_solve=True} dans \info{solve}.

	\item
	      Calculer des dérivées symboliquement, voir exemple~\ref{exRNZKooUIOfPU}.
	\item
	      Calculer des approximations numériques comme dans l'exemple~\ref{exLFYFooNCXCJz}.
	\item
	      Calculer dans un corps de polynômes modulo comme \( \eF_p[X]/P\) où \( P\) est un polynôme à coefficients dans \( \eF_p\). Voir l'exemple~\ref{ExemWUdrcs}.
\end{enumerate}

Sage peut en général faire tout ce que vous êtes capable de faire à l'entrée en master et probablement bien plus, à la notable exception des limites à deux variables.

\begin{remark}	\label{REMooINTROsageAttention}
	Sage peut toutefois vous induire en erreur si vous n'y prenez pas garde parce qu'il sait des choses en mathématique que vous ne savez pas. Par conséquent il peut parfois vous donner des réponses (mathématiquement exactes) auxquelles vous ne vous attendez pas. Voir par exemple~\ref{ooOPWYooDDSZWx} pour le logarithme de nombres négatifs. Et aussi ceci :

	\lstinputlisting{tex/sage/sageSnip017.sage}

	Sage fait une différence entre \info{Infinity} et \info{+Infinity} et donne
	\begin{equation}
		\lim_{x\to 0} \frac{1}{ x }=\infty
	\end{equation}
	ainsi que
	\begin{equation}
		\lim_{x\to 0} \frac{1}{ x^2 }=+\infty.
	\end{equation}
\end{remark}

Voir aussi la compactification en un point d'Alexandroff \ref{PROPooHNOZooPSzKIN}.

%+++++++++++++++++++++++++++++++++++++++++++++++++++++++++++++++++++++++++++++++++++++++++++++++++++++++++++++++++++++++++++
\section{Comment contribuer et aider ?}
\label{SecooCKWWooBFgnea}
%+++++++++++++++++++++++++++++++++++++++++++++++++++++++++++++++++++++++++++++++++++++++++++++++++++++++++++++++++++++++++++

%-------------------------------------------------------
\subsection{Orthographe}
%----------------------------------------------------

Quelques règles.
\begin{enumerate}
	\item
	      Le Frido utilise l'orthographe rectifiée. Le plus visible est «corolaire» au lieu de «corollaire» et quelques accents circonflexes qui sautent. Par exemple : «convergence presque sure» au lieu de «presque sûre».
	\item
	      Le Frido écrit «à priori» et non «a priori» parce que si au bout de seulement quelques décennies on a le droit d'écrire «des pizzas» ou comme si ce n'était pas un mot italien, au bout de 1000 ans il faut se calmer en prétendant que «à priori» est du latin.
	\item
	      En terme de féminisation, on n'écrit pas «la.e lectrice.teur» mais on utilise
	      \begin{center}
		      \info{\textbackslash randomGender\{le lecteur\}\{la lectrice\}}
	      \end{center}
	      et une des deux formes sera choisie au hasard durant la compilation. Comme ça le Frido est neutre en moyenne. La macro est bien faite : elle choisira la même forme si on l'appelle deux fois dans le même paragraphe.
\end{enumerate}


%--------------------------------------------------------------------------------------------------------------------------- 
\subsection{Des preuves qui manquent}
\label{SUBooINTROhowtoContribProofs}
%---------------------------------------------------------------------------------------------------------------------------

Vous trouverez un peu partout des énoncés sans preuves. Certaines sont surement très faciles, et d'autres probablement assez compliquées. N'hésitez pas à rédiger une preuve et me l'envoyer.

Vous pouvez m'envoyer vos preuves sous forme de «c'est bien fait dans tel cours», avec une URL.

Ne me dites juste pas «c'est bien fait dans tel \emph{livre}». Je ne travaille pas à l'université, et je n'ai pas accès à une bibliothèque universitaire; je n'ai donc pas réellement accès à ces fameux «livres» dont tout le monde parle.

%--------------------------------------------------------------------------------------------------------------------------- 
\subsection{Du texte qui manque}
\label{SUBooINTROhowtoContribText}
%---------------------------------------------------------------------------------------------------------------------------

Vous remarquerez que de nombreuses pages du Frido sont des enchainements de théorèmes et démonstrations sans articulations. Autrement dit, il manque ce qu'à l'agrégation on dirait à l'oral quand on présente le plan. Si vous avez des idées de choses à ajouter ici ou là, faites-le moi savoir.

%--------------------------------------------------------------------------------------------------------------------------- 
\subsection{Des exemples qui manquent}
\label{SUBooINTROhowtoContribExamples}
%---------------------------------------------------------------------------------------------------------------------------

Si vous connaissez de bons exemples, faites-le moi savoir.


%-------------------------------------------------------
\subsection{Des choses qui n'aident pas}
%----------------------------------------------------

Il y a toutefois un type de contribution qui n'est pas très utile : les modifications de style ou de notations. Ne me faites pas un commit changeant 50 fichiers pour modifier «strictement définie positive» en «définie positive» ou pour améliorer le \( d\) dans \( \int f(x)dx\).

%--------------------------------------------------------------------------------------------------------------------------- 
\subsection{Trucs de programmation et de \LaTeX}
\label{SUBooINTROhowtoContribProg}
%---------------------------------------------------------------------------------------------------------------------------

\begin{enumerate}
	\item
	      Comment faire en sorte que les mots commençant par «é» soient avec les «e» dans l'index, et non avant les «a» ? Il me faudrait un mécanisme plus automatique que faire \info{machin@truc}.

	      Une fonction en Python qui prend en entrée le fichier bbl sous forme de string et qui ressort sous forme de string le fichier bbl modifié me convient.
	      %	\item
	      %        Créer un epub du Frido.  Attention : le Frido étant un truc assez compliqué, avant de répondre la première chose qui vous passe par la tête, assurez-vous que votre solution fait avancer les choses sur le Frido et non sur un petit document de test. Pour vous assurer que votre solution aide, vous pouvez la tester sur le fichier tex donné ici : \url{https://laurent.claessens-donadello.eu/pdf/défi_epub/}.

	      %          Pour les figures tikz, le document de test contient les sources tikz. Si ça vous aide, les fichiers \info{pdf} correspondant sont dans le répertoire \info{auto/pictures\_tikz}.

	      %          Vu que je n'ai pas de liseuse et que le sujet ne m'intéresse pas tellement, je suis preneur d'une solution clef en main; pas de conseils sur des pistes à explorer.
	\item
	      Écrire un script (en Python ou autre) qui prend en argument deux numéros ou noms de chapitres et qui retourne l'ensemble des lignes du premier qui contient des \info{ref} ou \info{eqref} dont le label correspondant est dans le second.

	      Attention : il faut tenir compte de \info{input} de façon récursive.

	      Bonus : calculer le hash sha1 de chaque ligne du résultat et ne pas l'afficher si il se trouve dans la liste du fichier \info{commons.py}.
\end{enumerate}

%+++++++++++++++++++++++++++++++++++++++++++++++++++++++++++++++++++++++++++++++++++++++++++++++++++++++++++++++++++++++++++
\section{Les politiques éditoriales}
\label{SECooINTROpolitics}
%+++++++++++++++++++++++++++++++++++++++++++++++++++++++++++++++++++++++++++++++++++++++++++++++++++++++++++++++++++++++++++

Certaines parties de ce texte ne respectent pas les politiques éditoriales. Ce sont des erreurs de jeunesse, et j'en suis le premier triste.

%---------------------------------------------------------------------------------------------------------------------------
\subsection{Licence libre}
\label{SUBooINTROpoliticsFreelic}
%---------------------------------------------------------------------------------------------------------------------------

Je crois que c'est clair.

%---------------------------------------------------------------------------------------------------------------------------
\subsection{pdflatex}
\label{SUBooINTROpoliticsPdflatex}
%---------------------------------------------------------------------------------------------------------------------------

Tout est compilable avec pdf\LaTeX. Pas de pstricks, de psfrag ou de ps<quoiquecesoit>.

%---------------------------------------------------------------------------------------------------------------------------
\subsection{utf8}
\label{SUBooINTROpoliticsUTF8}
%---------------------------------------------------------------------------------------------------------------------------

Je crois que c'est clair.

%---------------------------------------------------------------------------------------------------------------------------
\subsection{Notations}
\label{SUBooINTROpoliticsNotations}
%---------------------------------------------------------------------------------------------------------------------------

On essaie d'être cohérent dans les notations et les conventions. Pour la transformée de Fourier par exemple, je crois que la définition du produit scalaire dans \( L^2\), des coefficients de Fourier, de la transformation et de la transformation inverse sont cohérents. Cela demande, lorsqu'on suit un livre qui ne suit pas les conventions utilisées ici, de convertir parfois massivement.

%---------------------------------------------------------------------------------------------------------------------------
\subsection{De la bibliographie}
\label{SUBooINTROpoliticsBiblio}
%---------------------------------------------------------------------------------------------------------------------------

On évite d'écrire en haut de chapitre «les références pour ce chapitre sont \ldots». Il est mieux d'écrire au niveau des théorèmes, entre parenthèses, les références.

Lorsqu'on écrit l'énoncé d'un théorème sans retranscrire la démonstration, il faut mettre une référence vers un document \emph{en ligne} qui en contient la preuve. Il est vraiment fastidieux de chercher une preuve sur internet et de tomber sur des dizaines de documents qui donnent l'énoncé mais pas la preuve.

%---------------------------------------------------------------------------------------------------------------------------
\subsection{Faire des références à tout}
\label{SUBooINTROpoliticsInternalRefs}
%---------------------------------------------------------------------------------------------------------------------------

Lorsqu'un utilise le théorème des accroissements finis, il ne faut pas écrire «d'après le théorème des accroissements finis, blablabla». Il faut écrire un \verb+\ref+ explicite vers le résultat. Cela alourdit un peu le texte, mais lorsqu'on joue avec un texte de plus de 2000 pages, il est parfois laborieux de trouver le résultat qu'on cherche (surtout si il existe plusieurs versions d'un résultat et que l'on veut faire référence à une version particulière).

%---------------------------------------------------------------------------------------------------------------------------
\subsection{Des listes de liens internes}
\label{SUBooINTROpoliticsInternalLinks}
%---------------------------------------------------------------------------------------------------------------------------

Le début du Frido contient une espèce d'index thématique. Il serait bon de l'étoffer.

%---------------------------------------------------------------------------------------------------------------------------
\subsection{Pas de références vers le futur}
\label{SUBooINTROpoliticsNoFutureRef}
%---------------------------------------------------------------------------------------------------------------------------

Dans le Frido, \emph{aucune} preuve ne peut faire une référence vers un résultat prouvé plus bas. On n'utilise pas le théorème 10 dans la démonstration du théorème 7. Cela est une contrainte forte sur le découpage en chapitres et sur l'ordre de présentation des matières.

Il est bien entendu accepté et même encouragé de mettre des notes du type «Nous verrons plus loin un théorème qui \ldots». Tant que ce théorème n'est pas \emph{utilisé}, ça va.



% pour référence éventuelle plus tard:
% 6f32b464dea03707fd9fc0b7424ce0a1bedb1cb68eb603beed9ea6202a3c1ab0095d44798d3f0c5dfbeb60fcd402192e
