\InternalLinks{morphismes et compagnie}	\label{THEMEooMorphismes}
Un morphisme est un concept algébrique. Il s'agit d'une application (pas spécialement inversible) qui préserve la structure. Quand on parle de morphisme, il faut donc préciser la structure. On dit «morphisme de groupe», «morphisme d'espace vectoriel», «morphisme d'anneaux», etc.
\begin{enumerate}
	\item
	      Un morphisme de module est la définition \ref{DEFooHXITooBFvzrR}.
	\item
	      Un morphisme de groupes est la définition \ref{DEFooBEHTooMeCOTX}.
	\item
	      Un morphisme d'anneaux est la définition \ref{DEFooSPHPooCwjzuz}.
	\item
	      Un morphisme de corps est aussi la définition \ref{DEFooSPHPooCwjzuz}. Un corps n'a pas plus de structure qu'un anneau.
	\item
	      Un isomorphisme d'espaces topologiques est une application continue, inversible, dont l'inverse est continue, \ref{DEFooYPGQooMAObTO}. On dit aussi un homéomorphisme.
	\item
	      Un difféomorphisme est différentiable d'inverse différentiable, définition \ref{DefAQIQooYqZdya}.
	\item
	      Un \( C^k\)-difféomorphisme est une application \( C^k\) d'inverse \( C^k\). Définition \ref{DefAQIQooYqZdya}.
	      \ifbool{isGiulietta}{
	\item
	      Un morphisme de groupes de Lie est un morphisme de groupe \(  C^{\infty}\). Nous ne demandons pas que l'inverse ait une régularité particulière.
	      }{}
\end{enumerate}

\begin{normaltext}		\label{NORMooTXFWooApjnOY}
	Le mot «homomorphisme» signifie exactement «morphisme», et, sauf incohérence de ma part, il n'est pas utilisé dans le Frido. Si vous tombez sur ce mot quelque part, soyez \randomGender{prudent}{prudente}, vérifiez dans le contexte si le mot attendu est bien «morphisme», et écrivez-moi.
	%TODOooVXIAooHrgOZR. Faire cette vérification.
	% Voir https://github.com/LaurentClaessens/mazhe/issues/227
\end{normaltext}
