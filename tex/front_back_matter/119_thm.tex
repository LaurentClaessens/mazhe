
\InternalLinks{morphismes et compagnie}
\begin{enumerate}
	\item
	      Un morphisme est un concept algébrique. Il s'agit d'une application (pas spécialement inversible) qui préserve la structure. Quand on parle de morphisme, il faut donc préciser la structure. On dit «morphisme de groupe», «morphisme d'espace vectoriel», «morphisme d'anneaux», etc.
	\item
	      Morphisme de module, définition \ref{DEFooHXITooBFvzrR}.
      \item
          Morphisme de groupes, définition \ref{DEFooBEHTooMeCOTX}.
	\item
        Un isomorphisme d'espaces topologiques est une application continue, inversible, dont l'inverse est continue, \ref{DEFooYPGQooMAObTO}. On dit aussi un homéomorphisme.
	\item
	      Un difféomorphisme est différentiable d'inverse différentiable, définition \ref{DefAQIQooYqZdya}.
	\item
	      Un \( C^k\)-difféomorphisme est une application \( C^k\) d'inverse \( C^k\). Définition \ref{DefAQIQooYqZdya}.
	      \ifbool{isGiulietta}{
	\item
	      Un morphisme de groupes de Lie est un morphisme de groupe \(  C^{\infty}\). Nous ne demandons pas que l'inverse ait une régularité particulière.
	      }{}
\end{enumerate}
Le mot «homomorphisme» signifie exactement «morphisme», et, sauf incohérence de ma part, il n'est pas utilisé dans le Frido.
