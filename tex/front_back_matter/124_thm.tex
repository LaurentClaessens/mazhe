\InternalLinks{inégalités}
Dans \( \eC\) nous avons \( | a+ b|\leq | a |+| b |\) par la proposition \ref{PROPooUMVGooIrhZZg}\ref{ITEMooDVMDooFDmOur}.
\begin{description}
	\item[Inégalité de Clarkson]
		Pour \( f,g\in L^p\) nous avons l'inégalité de Clarkson \ref{PROPooJDOQooWsGlkr} :
		\begin{equation}
			\| \frac{ f+g }{2} \|_p^p+\| \frac{ f-g }{2} \|_p^p\leq \frac{ 1 }{2}\Big( \| f \|_p^p+\| g \|_p^p \Big).
		\end{equation}
	\item[Inégalité de Young]
		Nous avons
		\begin{equation}
			ab\leq \frac{ a^p }{ p }+\frac{ b^q }{ q }
		\end{equation}
		par la proposition \ref{PROPooCQUBooCvtMSi}.
	\item[Inégalité de Jensen]
		\begin{enumerate}
			\item
			      Une version discrète pour \( f\big( \sum_i\lambda_ix_i \big)\), la proposition~\ref{PropXIBooLxTkhU}.
			\item
			      Une version intégrale pour \( f\big( \int \alpha d\mu \big)\), la proposition~\ref{PropXISooBxdaLk}.
			\item
			      Une version pour l'espérance conditionnelle, la proposition~\ref{PropABtKbBo}.
		\end{enumerate}
	\item[Inégalité de Hölder]
		Il en existe de nombreuses versions et variations.
		\begin{enumerate}
			\item
			      Hölder pour \( L^p\): \( \| fg \|_1\leq \| f \|_p\| g \|_q\), proposition \ref{ProptYqspT}.
			\item
			      Hölder pour \( \ell^p\): \( \| x \|_q\leq n^{\frac{1}{ q }-\frac{1}{ p }}\| x \|_p\), proposition \ref{PROPooQZTNooGACMlQ}.
			\item
			      $\| x \|_{\infty}\leq \| x \|_p\leq n^{1/p}\| x \|_{\infty}$, théorème \ref{THOooPPDPooJxTYIy}
			\item
			      $\| x \|_p\leq n^{1/p} \| x \|_q$, corolaire \ref{CORooEZGHooACHOiB}.
		\end{enumerate}
	\item[Inégalité de Minkowsky]
		\begin{enumerate}
			\item
			      Pour une forme quadratique\footnote{Définition \ref{DefBSIoouvuKR}.} \( q\) sur \( \eR^n\) nous avons \( \sqrt{q(x+y)}\leq\sqrt{q(x)}+\sqrt{q(y)}\). Proposition~\ref{PropACHooLtsMUL}.
			\item
			      Si \( 1\leq p<\infty\) et si \( f,g\in L^p(\Omega,\tribA,\mu)\) alors \(  \| f+g \|_p\leq \| f \|_p+\| g \|_p\). Proposition~\ref{PropInegMinkKUpRHg}.
			\item
			      L'inégalité de Minkowsky sous forme intégrale s'écrit sous forme déballée
			      \begin{equation*}
				      \left[ \int_X\Big( \int_Y| f(x,y) |d\nu(y) \Big)^pd\mu(x) \right]^{1/p}\leq \int_Y\Big( \int_X| f(x,y) |^pd\mu(x) \Big)^{1/p}d\nu(y).
			      \end{equation*}
			      ou sous forme compacte
			      \begin{equation*}
				      \left\|   x\mapsto\int_Y f(x,y)d\nu(y)   \right\|_p\leq \int_Y  \| f_y \|_pd\nu(y)
			      \end{equation*}
			      C'est la proposition \ref{PROPooGZJZooXfZdqn}.
		\end{enumerate}
	\item[Transformée de Fourier]
		Pour tout \( f\in L^1(\eR^n)\) nous avons \( \| \hat f \|_{\infty}\leq \| f \|_1\), lemme~\ref{LEMooCBPTooYlcbrR}.
	\item[Inégalité des normes]
		Inégalité de normes : si \( f\in L^p\) et \( g\in L^1\), alors \( \| f*g \|_p\leq \| f \|_p\| g \|_1\), proposition~\ref{PROPooDMMCooPTuQuS}.
	\item[\( c_r\) inégalité] La proposition \ref{PROPooTIJYooJGxphQ} donne \( \big|  f(\omega)+g(\omega)  \big|^r\leq  c_r| f(\omega) |^r+c_r| g(\omega) |^r \).
\end{description}
