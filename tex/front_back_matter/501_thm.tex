\InternalLinks{manifolds}	\label{THEMEooManifolds}
\begin{enumerate}
	\item
	      Inverse function theorem: if the differential is bijective, the map is a local diffeomorphism, proposition \ref{THOooDWEXooMClWVi}.
	\item
	      There exist a unique \( v\) such that \( X=\frac{d}{dt} \left[ \varphi_{\alpha}(s_0+tv)  \right]_{t=0}\), proposition \ref{PROPooLHMSooMMXrSS} and lemma \ref{LEMooZXEFooZgXbNP}.
	\item
	      By lemma \ref{LEMooSCVHooYPiGse}, a vector \( v\in T_aM\) can be written as
	      \begin{equation}
		      v(f)=\sum_{k=1}^nv_k\partial_k(f\circ \varphi)(s).
	      \end{equation}
	\item
	      We can also write
	      \begin{equation}
		      X_x=\sum_{i=1}^nX_i(x)\partial_i
	      \end{equation}
	      for \( X_i\in C^k(M, \eR)\) by proposition \ref{PROPooXURIooYPytwa}.
	\item
	      The tangent space \( T_pM\) is a vector space of the same dimension as \( M\), proposition \ref{PROPooEJBWooSbvypo}.
	\item
	      The vectors \( \partial_i\) are a basis of \( T_pM\), proposition \ref{PROPooAAAXooKAMsfK}.
	\item
	      If \( X\) is a vector field, it can be written as
	      \begin{equation}
		      X_x(f)=\sum_{k=1}^nv_k(x)\partial_k(f\circ\varphi)\big( \varphi^{-1}(x) \big),
	      \end{equation}
	      with \( v_k\in C^k(M, \eR)\), lemma \ref{LEMooZWFAooDlYaJm}.
\end{enumerate}
