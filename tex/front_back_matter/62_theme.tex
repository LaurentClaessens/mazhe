\InternalLinks{exponentielle de matrice}
\begin{enumerate}
    \item
            Le lemme à propos d'exponentielle de matrice \ref{LemQEARooLRXEef} donne :
            \begin{equation}
                \|  e^{tA} \|\leq P\big( | t | \big)\sum_{i=1}^r e^{t\real(\lambda_i)}.
            \end{equation}
        \item
    La proposition \label{PropCOMNooIErskN} : si \( A\in \eM(n,\eR)\) a un polynôme caractéristique scindé, alors \( A\) est diagonalisable si et seulement si \( e^A\) est diagonalisable.
\item
    La section \ref{subsecXNcaQfZ} parle des fonctions exponentielle et logarithme pour les matrices. Entre autres la dérivation et les séries.
\item
    Pour résoudre des équations différentielles linéaires : sous-section \ref{SUBSECooMDKIooKaaKlZ}.
\item
    La proposition \ref{PropKKdmnkD} dit que l'exponentielle est surjective sur \( \GL(n,\eC)\).
\item

La proposition \label{PropFMqsIE} : si \( u\) est un endomorphisme, alors \( \exp(u)\) est un polynôme en \( u\).
\item
    Calcul effectif : sous-section \ref{SUBSECooGAHVooBRUFub}.
\item Proposition \ref{PROPooZUHOooQBwfZq} : si \( A\in\eM(n,\eC)\) alors $ e^{\tr(A)}=\det( e^{A}).$
\end{proposition}


\end{enumerate}
