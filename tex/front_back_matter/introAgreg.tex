% This is part of Mes notes de mathématique
% Copyright (c) 2011-2013,2016-2017
%   Laurent Claessens
% See the file fdl-1.3.txt for copying conditions.

%+++++++++++++++++++++++++++++++++++++++++++++++++++++++++++++++++++++++++++++++++++++++++++++++++++++++++++++++++++++++++++
\section{Originalité}
%+++++++++++++++++++++++++++++++++++++++++++++++++++++++++++++++++++++++++++++++++++++++++++++++++++++++++++++++++++++++++++

Ces notes ne sont pas originales par leur contenu : ce sont toutes des choses qu'on trouve facilement sur internet; je crois que la bibliographie est éloquente à ce sujet. Ce cours se distingue des autres sur les points suivants.
\begin{description}
    \item[La longueur] J'ai décidé de ne pas me soucier de la taille du fichier. Il fera cinq mille pages s'il le faut, mais il restera en un bloc. Étant donné qu'il n'existe qu'une seule mathématique, il ne m'a pas semblé intéressant de produire une division artificielle entre l'analyse, la géométrie ou l'algèbre. Tous le résultats d'une branche peuvent (et sont) être utilisés dans toutes les autres branches.

        Dans cette optique, je me suis évertué à ne créer que des références «vers le haut». À moins d'oubli de ma part\footnote{Par exemple pour les théorèmes pour lesquels je n'ai pas lu ni a fortiori écrit de preuves.}, il n'y a aucun endroit du texte qui dépend d'un lemme démontré plus bas. Le fait qu'un théorème \( B\) soit plus bas qu'un théorème \( A\) signifie qu'on peut démontrer \( A\) sans savoir \( B\).

    \item[La licence] Ce document est publié sous une licence libre. Elle vous donne explicitement le droit de copier, modifier et redistribuer. 

    \item[Les mises à jour] Ce document est régulièrement mis à jour. Des fautes d'orthographe sont corrigées (presque) chaque jour. Si vous me signalez une faute de mathématique, elle sera corrigée.
    \item[Transparence] Je ne fais pas semblant que ces notes soient parfaites. Les points sur lesquels je ne suis pas sûr, les preuves que j'ai inventées moi-même sont clairement indiqués pour inciter le lecteur à redoubler de prudence. Une liste de questions à résoudre est inclue en la section \ref{SecooCKWWooBFgnea}. De plus de nombreuses notes en bas de page en fonte \info{texttt}\quext{Comme celle-ci} indiquent des points sur lesquels je doute ou des étapes intermédiaires de calculs que je ne parviens pas à reproduire en suivant mes sources. Lorsque vous voyez une telle note, redoublez de prudence, et allez voir la source.
        
\end{description}

%+++++++++++++++++++++++++++++++++++++++++++++++++++++++++++++++++++++++++++++++++++++++++++++++++++++++++++++++++++++++++++ 
\section{Quelque choix discutables}
%+++++++++++++++++++++++++++++++++++++++++++++++++++++++++++++++++++++++++++++++++++++++++++++++++++++++++++++++++++++++++++

Comme tout cours de mathématique, ce cours fait des choix qui sont parfois discutables. Voici quelque points sur lesquels les choix faits ici ne sont peut-être pas ceux fait par tout le monde. Ce sont donc des points sur lesquels vous devez faire attention pour éviter les quiproquos lors par exemple d'un oral dans un concours.

\begin{enumerate}
    \item
        Les limites sont épointées. La notion de limite pointée n'est pas abordée, c'est volontaire.
    \item
        Un compact est un partie dont tout recouvrement par des ouverts admet un sous-recouvrement fini. Le fait d'être séparable n'est pas inclus à la définition de compact. De nombreux textes français incluent la séparabilité dans la compacité. 
    \item
        Le logarithme sur \( \eC\) est une application \( \ln\colon \eC^*\to \eC\) définie partout sauf en zéro. Elle n'est donc pas continue sur la fameuse demi-droite. À ne pas confondre avec une \emph{détermination} du logarithme qui est par définition continue et donc non définie sur la demi-droite.

        Cela est un choix très discutable. La raison de donner à la notation «\( \ln\)» cette signification est simplement de suivre l'usage de Sage. Pour Sage, \( \ln(-1)\) existe et vaut \( i\pi\).

        Voir les remarques \ref{REMooFBLLooDnkmjR}.
    \item
        Le mot «corps» n'implique pas la commutativité, et nous n'utilisons pas la terminologie «anneau à division». Voir la remarque \ref{REMooYRNUooYgBBKF} et la discussion \ref{NORMooGPWRooIKJqqw}.
\end{enumerate}

%+++++++++++++++++++++++++++++++++++++++++++++++++++++++++++++++++++++++++++++++++++++++++++++++++++++++++++++++++++++++++++
\section{Sage est là pour vous aider}
%+++++++++++++++++++++++++++++++++++++++++++++++++++++++++++++++++++++++++++++++++++++++++++++++++++++++++++++++++++++++++++

Il existe de nombreux logiciels de mathématique. Notre préféré est \href{http://www.sagemath.org}{Sage} pour une raison très précise : en tant que langage de programmation, Sage est python qui est un langage généraliste. La syntaxe et la structure de Sage ne sont pas \emph{ad hoc} pour faire de math, et ce qu'on apprend en Sage peut être recyclé pour faire n'importe quoi : navigateur web, script de manipulation de texte, traitement d'image, réseau neuronaux, \ldots

%Par ailleurs, le vingt et unième siècle est déjà largement entamé; si vous vous lancez dans une carrière scientifique, il vous faudra maitriser l'informatique un peu plus solidement qu'être virtuose es trouver le trajet le plus court en bus sur votre téléphone.

Sage est un logiciel disponible pour l'épreuve de modélisation de l'agrégation de mathématique; il y a donc de bonnes chances que vous en ayez l'usage.

%---------------------------------------------------------------------------------------------------------------------------
\subsection{Lancez-vous dans Sage}
%---------------------------------------------------------------------------------------------------------------------------


\begin{enumerate}
	\item
        Aller sur \url{http://www.sagenmath.org},
	\item
		créer un compte,
	\item
		créer des feuilles de calcul et s'amuser !!
\end{enumerate}

Il y a beaucoup de \href{http://lmgtfy.com/?q=sage+documentation}{documentation} sur le \href{http://www.sagemath.org}{site officiel}\footnote{\href{http://www.sagemath.org}{http://www.sagemath.org}}, et nous vous conseillons particulièrement le livre \cite{ooBLMMooWTPsQy}.

Si vous comptez utiliser régulièrement ce logiciel, je vous recommande \emph{chaudement} de \href{http://mirror.switch.ch/mirror/sagemath/index.html}{l'installer} sur votre ordinateur.

%---------------------------------------------------------------------------------------------------------------------------
\subsection{Exemples de ce que Sage peut faire pour vous}
%---------------------------------------------------------------------------------------------------------------------------

Ce livre est émaillé de petits bouts de code en Sage montrant ses différentes fonctionnalités là où nous en avons besoin\footnote{Soit un vrai besoin comme tracer un graphique en 3D, soit de la paresse comme calculer une grosse dérivée.}. Voici une liste (non exhaustive) de ce que Sage peut faire pour vous.
\begin{enumerate}

	\item
        Calculer des limites de fonctions, exemples \ref{ExBCRXooDVUdcf} et \ref{ExCWDRooKxnjGL}.
	\item
        Tracer des graphes de fonctions, exemple \ref{ExCWDRooKxnjGL}.
	\item
        Tracer des courbes en trois dimensions, voir exemple \ref{ExempleTroisDxxyy}. 
	\item
		Calculer des dérivées partielles de fonctions à plusieurs variables, voir exemple \ref{exJMGTooZcZYNy}.
	\item
        Résoudre des systèmes d'équations linéaires. Voir les exemples \ref{exKGDIooVefujD} et \ref{ExBGCEooPIQgGW}. Lire aussi \href{http://www.sagemath.org/doc/constructions/linear_algebra.html#solving-systems-of-linear-equations}{la documentation}.
	\item
        Tout savoir d'une forme quadratique, voir exemple \ref{exBNGVooIvKfTT}.
	\item
        Calculer la matrice Hessienne de fonctions de deux variables, déterminer les points critiques, déterminer le genre de la matrice Hessienne aux points critiques et écrire extrema de la fonctions (sous réserve d'être capable de résoudre certaines équations), voir les exemples \ref{exZHGRooTQpVpq} et \ref{exHWIHooOAvaDQ}.
	\item
        Indiquer une infinité de solutions à une équation en utilisant des paramètres, voir l'exemple \ref{exEEHPooKDxLTJ}. Pour les fonctions trigonométriques, 
        \begin{verbatim}
sage: solve(sin(x)/cos(x)==1,x,to_poly_solve=True)                                                         
[x == 1/4*pi + pi*z1]
sage: solve(sin(x)**2==cos(x)**2,x,to_poly_solve=True)
[sin(x) == cos(x), x == -1/4*pi + 2*pi*z86, x == 3/4*pi + 2*pi*z84]
        \end{verbatim}

        Notez l'option \info{to\_poly\_solve=true} dans \info{solve}.

	\item
        Calculer des dérivées symboliquement, voir exemple \ref{exRNZKooUIOfPU}.
	\item
        Calculer des approximations numériques comme dans l'exemple \ref{exLFYFooNCXCJz}.
    \item
        Calculer dans un corps de polynômes modulo comme \( \eF_p[X]/P\) où \( P\) est un polynôme à coefficients dans \( \eF_p\). Voir l'exemple \ref{ExemWUdrcs}.
\end{enumerate}

Sage peut en général faire tout ce que vous êtes capable de faire à l'entrée en master et probablement bien plus, à la notable exception des limites à deux variables.

\begin{remark}
    Sage peut toutefois vous induire en erreur si vous n'y prenez pas garde parce qu'il sait des choses en mathématique que vous ne savez pas. Par conséquent il peut parfois vous donner des réponses (mathématiquement exactes) auxquelles vous ne vous attendez pas. Voir par exemple \ref{ooOPWYooDDSZWx} pour le logarithme de nombres négatifs. Et aussi ceci :
    
\lstinputlisting{tex/sage/sageSnip017.sage}

Sage fait une différence entre \info{Infinity} et \info{+Infinity} et donne
\begin{equation}
    \lim_{x\to 0} \frac{1}{ x }=\infty
\end{equation}
ainsi que
\begin{equation}
    \lim_{x\to 0} \frac{1}{ x^2 }=+\infty.
\end{equation}
    .
\end{remark}
