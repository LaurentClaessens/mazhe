
\InternalLinks{gaussienne}
\begin{enumerate}
    \item
        Le calcul de l'intégrale
        \begin{equation*}
            \int_{\eR} e^{-x^2}dx=\sqrt{\pi }
        \end{equation*}
        est fait de deux façons dans l'exemple~\ref{EXooLUFAooGcxFUW}. Dans les deux cas, le théorème de Fubini~\ref{ThoFubinioYLtPI} est utilisé.
    \item
        Le lemme~\ref{LEMooPAAJooCsoyAJ} calcule la transformée de Fourier de $ g_{\epsilon}(x)=  e^{-\epsilon\| x \|^2}$ qui donne $\hat g_{\epsilon}(\xi)=\left( \frac{ \pi }{ \epsilon } \right)^{d/2} e^{-\| \xi \|^2/4\epsilon}$.
    \item
        Le lemme~\ref{LEMooTDWSooSBJXdv} donne une suite régularisante à base de gaussienne.
    \item
        Elle est utilisée pour régulariser une intégrale dans la preuve de la formule d'inversion de Fourier~\ref{PROPooLWTJooReGlaN}
\end{enumerate}


