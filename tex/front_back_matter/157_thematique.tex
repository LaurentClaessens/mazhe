% This is part of le Frido
% Copyright (c) 2016-2021
%   Laurent Claessens
% See the file fdl-1.3.txt for copying conditions.

%+++++++++++++++++++++++++++++++++++++++++++++++++++++++++++++++++++++++++++++++++++++++++++++++++++++++++++++++++++++++++++
\section*{Thèmes}
%+++++++++++++++++++++++++++++++++++++++++++++++++++++++++++++++++++++++++++++++++++++++++++++++++++++++++++++++++++++++++++

Ceci est une sorte d'index thématique.

% The macro '\InternalLink' writes in 'themestoc.tex' the lines like
% ~\ref {THTOC4} : méthode de Newton\\

% We open the file after the input (if before, we erase it) and
% we close at the end of this file.

\begin{multicols}{2}
\noindent
\ref {THTOC1} : Cardinalité\\
\ref {THTOC2} : tribu, algèbre de parties, \( \lambda \)-systèmes et co.\\
\ref {THTOC3} : théorie de la mesure\\
\ref {THTOC4} : intégration\\
\ref {THTOC5} : suites et séries\\
\ref {THTOC6} : polynôme de Taylor\\
\ref {THTOC7} : normes\\
\ref {THTOC8} : caractérisations séquentielles\\
\ref {THTOC9} : topologie produit\\
\ref {THTOC10} : espaces métriques, normés\\
\ref {THTOC11} : gaussienne\\
\ref {THTOC12} : compacts\\
\ref {THTOC13} : densité\\
\ref {THTOC14} : espaces de fonctions\\
\ref {THTOC15} : fonctions Lipschitz\\
\ref {THTOC16} : formule des accroissements finis\\
\ref {THTOC17} : limite et continuité\\
\ref {THTOC18} : Suite de fonctions\\
\ref {THTOC19} : Dérivation\\
\ref {THTOC20} : différentiabilité\\
\ref {THTOC21} : points fixes\\
\ref {THTOC22} : Intégration sur des variétés\\
\ref {THTOC23} : permuter des limites\\
\ref {THTOC24} : applications continues et bornées\\
\ref {THTOC25} : inégalités\\
\ref {THTOC26} : connexité\\
\ref {THTOC27} : suite de Cauchy, espace complet\\
\ref {THTOC28} : application réciproque\\
\ref {THTOC29} : déduire la nullité d'une fonction depuis son intégrale\\
\ref {THTOC30} : équations différentielles\\
\ref {THTOC31} : injections\\
\ref {THTOC32} : logarithme\\
\ref {THTOC33} : inversion locale, fonction implicite\\
\ref {THTOC34} : convexité\\
\ref {THTOC35} : fonction puissance\\
\ref {THTOC36} : dualité\\
\ref {THTOC37} : opérations sur les distributions\\
\ref {THTOC38} : convolution\\
\ref {THTOC39} : séries de Fourier\\
\ref {THTOC40} : transformée de Fourier\\
\ref {THTOC41} : méthode de Newton\\
\ref {THTOC42} : méthodes de calcul\\
\ref {THTOC43} : espaces vectoriels\\
\ref {THTOC44} : valeurs propres, définie positive\\
\ref {THTOC45} : norme matricielle, norme opérateur et rayon spectral\\
\ref {THTOC46} : série de matrices\\
\ref {THTOC47} : rang\\
\ref {THTOC48} : extension de corps et polynômes\\
\ref {THTOC49} : décomposition de matrices\\
\ref {THTOC50} : systèmes d'équations linéaires\\
\ref {THTOC51} : formes bilinéaires et quadratiques\\
\ref {THTOC52} : arithmétique modulo, théorème de Bézout\\
\ref {THTOC53} : polynômes\\
\ref {THTOC54} : zoologie de l'algèbre\\
\ref {THTOC55} : invariants de similitude\\
\ref {THTOC56} : réduction, diagonalisation\\
\ref {THTOC57} : endomorphismes cycliques\\
\ref {THTOC58} : déterminant\\
\ref {THTOC59} : polynôme d'endomorphismes\\
\ref {THTOC60} : exponentielle\\
\ref {THTOC61} : types d'anneaux\\
\ref {THTOC62} : sous-groupes\\
\ref {THTOC63} : groupe symétrique\\
\ref {THTOC64} : action de groupe\\
\ref {THTOC65} : classification de groupes\\
\ref {THTOC66} : produit semi-direct de groupes\\
\ref {THTOC67} : théorie des représentations\\
\ref {THTOC68} : isométries\\
\ref {THTOC69} : caractérisation de distributions en probabilités\\
\ref {THTOC70} : théorème central limite\\
\ref {THTOC71} : lemme de transfert\\
\ref {THTOC72} : probabilités et espérances conditionnelles\\
\ref {THTOC73} : dénombrements\\
\ref {THTOC74} : enveloppes\\
\ref {THTOC75} : équations diophantiennes\\
\ref {THTOC76} : techniques de calcul\\

\end{multicols}

\newwrite\themetoc
\immediate\openout\themetoc=themestoc.tex


% ATTENTION : il est très important que le titre «Thèmes» soit au haut d'une page et que le premier thème commence sur cette même page. Donc pas de texte trop long ici.
% La raison :
% Pour la division en volume, je prend le bloc 'thèmes' comme commençant à la page du premier. Cela est dû au fait que le titre n'est pas dans la TOC.

% Convention : les titres ne commencent pas par une majuscule
%  La raison : cela facilite les recherches dans le pdf (oui, je sais : on peut faire des recherches sans tenir compte des majuscules)

\InternalLinks{Cardinalité}
Le Frido\footnote{Ici je mets la référence \cite{MonCerveau}; pas parce qu'elle est utile ici, mais parce que je veux être sûr qu'elle soit numéro 1 de façon à être facilement reconnaissable.} ne définit pas la notion de nombre cardinal; ça nous mènerait trop loin. Au lieu de cela, nous allons nous contenter des notions d'équipotence, surpotence et subpotence, et démontrer un certain nombre de résultats en utilisant sans retenue le lemme de Zorn \ref{LemUEGjJBc}.
\begin{enumerate}
    \item
        Définition d'équipotence, surpotence et subpotence, notations \( A\succ B\) et \( A\approx B\), définition \ref{DEFooXGXZooIgcBCg}.
    \item
    Toute partie d'un ensemble fini est finie, lemme \ref{LEMooTUIRooEXjfDY}.
\item
Si \( A\) est un ensemble fini ou dénombrable, alors il existe une surjection \( \eN\to A\), lemme \ref{LEMooSRZWooASgEfy}.
\item
    Si \( A\) est un ensemble infini et si \( f\colon A\to B\) est une application injective, alors \( f(A)\) est infini, proposition \ref{PROPooWKSIooHcfYPN}.
\item
    Toute partie infinie de \( \eN\) est dénombrable, proposition \ref{PROPooOBKMooWEGCvM}
\item
    Une bijection \( \eN\to \eN\times \eN\), proposition \ref{PROPooLPKUooAlsYJg}.
\item
    Une décomposition de \( \eN\) en une infinité de parties équipotentes à \( \eN\), corolaire \ref{CORooNRPIooZPSmqa}.
\item
    Si il existe une surjection \( \eN\to A\), alors \( A\) est fini ou dénombrable, lemme \ref{LEMooDLWFooNAJbbq}.
\item
    Une union dénombrable d'ensembles finis ou dénombrables est finie ou dénombrable, proposition \ref{PROPooENTPooSPpmhY}
\item
    Tout ensemble infini contient une partie en bijection avec \( \eN\), proposition \ref{PROPooUIPAooCUEFme}.
\item
    Toute partie d'un ensemble fini est finie, et toute partie d'un ensemble dénombrable est finie ou dénombrable, proposition \ref{PropQEPoozLqOQ}.
\item
    Si \( A\succeq B\) et \( B\succeq A\), alors \( A\approx B\), théorème de Cantor-Schröder-Bernstein \ref{THOooRYZJooQcjlcl}
\item
    Le théorème de Cantor \ref{THOooJPNFooWSxUhd} dit qu'il n'existe pas de surjection d'un ensemble vers son ensemble des parties. On en déduit qu'il n'existe pas d'ensemble contenant tous les ensembles (corolaire \ref{CORooZMAOooPfJosM}).
\item 
    Si \( A\) est infini et si \( A\succeq B\), alors \( A\approx A\cup B\) par le lemme \ref{LEMooXMVDooIWLWis}.
\item
    Si \( S\) est un ensemble infini alors il existe une bijection \( \varphi\colon \{ 0,1 \}\times S\to S\), proposition \ref{PropVCSooMzmIX}.
\item
    Si \( A\) est infini, alors \( A\times \eN\approx A\), proposition \ref{PROPooFKBEooKXqujV}.
\item 
    Si \( A\) est infini et si \( B\prec A\), alors \( A\setminus B\approx A\), lemme \ref{LEMooIVCBooHWQiZB}.
\item
    Si \( A\) est infini, alors \( A\approx A\times A\), théorème \ref{THOooDGOVooRdURVi}.
\end{enumerate}

Il y a aussi des résultats de cardinalité autour des extensions de corps.
\begin{enumerate}
    \item
        Si \( \eK\) est un corps infini, alors \( \eK[X]\approx \eK\).
    \item
        Le théorème de Steinitz \ref{THOooEDQKooLEGlDv} affirme que tout corps admet une unique clôture algébrique. La preuve utilise pas mal de cardinalité ainsi que le lemme de Zorn \ref{LemUEGjJBc}.
\end{enumerate}

\InternalLinks{tribu, algèbre de parties, \( \lambda\)-systèmes et co.}  \label{INTooVDSCooHXLLKp}
    Il existe des centaines de notions de mesures et de classes de parties.
\begin{enumerate}
        \item
            Le plus souvent lorsque nous parlons de mesure est que nous parlons de mesure positive, définition~\ref{DefBTsgznn} sur un espace mesuré avec une tribu, définition~\ref{DefjRsGSy}.
        \item
            Une mesure extérieure est la définition~\ref{DefUMWoolmMaf}
        \item
            Une algèbre de partie : définition~\ref{DefTCUoogGDud}. Une mesure sur une algèbre de parties : définition~\ref{DefWUPHooEklLmR}. L'intérêt est que si on connait une mesure sur une algèbre de parties, elle se prolonge en une mesure sur la tribu engendrée par le théorème de prolongement de Hahn~\ref{ThoLCQoojiFfZ}.
        \item
            Un \( \lambda\)-système : définition~\ref{DefRECXooWwYgej}.
        \item
            Une mesure complexe : définition~\ref{DefGKHLooYjocEt}.
\end{enumerate}

En théorie de l'intégration, si \( X\) est une partie de \( \eR^n\), la convention est de considérer des fonctions
\begin{equation*}
    f\colon \big( X,\Lebesgue(X) \big)\to \big( \eR,\Borelien(\eR) \big).
\end{equation*}
Voir les points \ref{NORMooNFOMooYnaflN} et \ref{NORMooFZEDooIxSgLe} pour les conventions à ce propos.

À propos d'applications mesurables :
\begin{enumerate}
    \item
        Définition d'une application mesurable, définition \ref{DefQKjDSeC}.
    \item
        Une fonction continue est borélienne, théorème \ref{ThoJDOKooKaaiJh}.
\end{enumerate}


À propos de tribu induite:
\begin{enumerate}
    \item
        Définition \ref{DefDHTTooWNoKDP}.
    \item
        Les boréliens induits sont bien les boréliens de la topologie induite : \( \Borelien(Y)=\Borelien(X)_Y\), théorème \ref{ThoSVTHooChgvYa}.
\end{enumerate}

\InternalLinks{théorie de la mesure}       \label{THEMEooKLVRooEqecQk}
\begin{description}
    \item[Mesure] 
    À propos de mesure.
\begin{enumerate}
    \item
        Mesure positive, mesure finie et \( \sigma\)-finie, c'est la définition \ref{DefBTsgznn}.

    \item Le produit de tribus est donné par la définition~\ref{DefTribProfGfYTuR},     % Cette référence doit être vers le haut.
    \item
        Produit d'une mesure par une fonction, définition \ref{PropooVXPMooGSkyBo}.
    \item le produit d'espaces mesurés est donné par la définition~\ref{DefUMlBCAO}.     % Cette référence doit être vers le haut.
        \item
            Mesure de Lebesgue sur \( \eR\), définition~\ref{DefooYZSQooSOcyYN}.
        \item
            Une partie de \( \eR\) non mesurable au sens de Lebesgue : l'exemple \ref{EXooCZCFooRPgKjj}.
        \item
            Mesure de Lebesgue sur \( \eR^N\), définition~\ref{DEFooSWJNooCSFeTF}.
        \item
            Mesure à densité, définition~\ref{PropooVXPMooGSkyBo}.
\end{enumerate}
\item[Théorèmes d'approximation]
    Il est important de pouvoir approcher des fonctions continues ou \( L^p\) par des fonctions étagées, sinon on ne parvient pas à faire tourner la machine de l'intégration de Lebesgue.
    \begin{enumerate}
        \item
            Si \( (S,\tribA, \mu)\) est un espace mesuré et si \( f\colon S\to \mathopen[ 0 , +\infty \mathclose]\) est une fonction mesurable, le théorème fondamental d'approximation \ref{THOooXHIVooKUddLi} dit qu'il existe une suite croissante de fonctions étagées qui converge vers \( f\).
        \item
            Les fonctions simples sont denses dans \( L^p\), proposition \ref{PROPooUQUBooAWgNhm}.
        \item
            Encadrement d'un borélien \( A\) par un fermé \( F\) et un ouvert \( V\) par le lemme \ref{LEMooCGKXooYWjRwk} : \( F\subset A\subset V\) avec \( \mu(V\setminus F)<\epsilon\).
        \item
            Approximation \( L^p\) de la fonction caractéristique d'un borélien par une fonction continue par le théorème \ref{ThoAFXXcVa}.
    \end{enumerate}
\end{description}


\InternalLinks{intégration}     \label{THEMEooHINHooJaSYQW}

À propos d'intégration.
\begin{description}

\item[L'ordre dans lequel les choses sont faites]
\begin{itemize}
    \item 
        Nous commençons par considérer des fonctions \( f\colon \Omega\to \mathopen[ 0 , +\infty \mathclose]\) dans la définition \ref{DefTVOooleEst}.
    \item
        Nous donnerons ensuite quelques propriétés restreintes aux fonctions à valeurs positives, par exemple
        \begin{enumerate}
            \item
                La convergence monotone \ref{ThoRRDooFUvEAN},
            \item
                Lemme de Fatou \ref{LemFatouUOQqyk}.
            \item 
                (presque) linéarité pour les fonctions positives, théorème \ref{ThoooCZCXooVvNcFD}.
        \end{enumerate}
        \item
            La définition pour les fonctions à valeurs dans \( \eR\) puis \( \eC\) est \ref{DefTCXooAstMYl}.
        \item
            Pour les fonctions à valeurs dans un espace vectoriel, c'est la définition \ref{PROPooOFSMooLhqOsc}.
\end{itemize}
    \item[Quelque résultats] 
\begin{enumerate}
    \item
        Intégrale associée à une mesure, définition~\ref{DefTVOooleEst}
\item
    L'existence d'une primitive pour toute fonction continue est le théorème~\ref{ThoEXXyooCLwgQg}.
\item
    La définition d'une primitive est la définition~\ref{DefXVMVooWhsfuI}.
\item
    Primitive et intégrale, proposition~\ref{PropEZFRsMj}.
\item
    Intégrale impropre, définition~\ref{DEFooINPOooWWObEz}.
\end{enumerate}
\item[Intégrale et mesure]
    \begin{enumerate}
        \item
            L'intégrale de la fonction \( 1\) donne la mesure : \( \int_B1d\mu=\mu(B)\), c'est le lemme \ref{LemooPJLNooVKrBhN}.
        \item
            Le théorème de Radon-Nikodym \ref{THOooEFVUooGKApaV} donne une densité pour certaines mesures.
        \item
            Le produit d'une mesure par une fonction donné par la définition \ref{PropooVXPMooGSkyBo} introduit aussi une densité : \( (w\cdot \mu)(A)=\int_Awd\mu\).
    \end{enumerate}

\item[Autre résultats]
\begin{enumerate}
    \item
        Si \( A,B\subset \Omega\) sont des parties disjointes, alors $\int_{A\cup B}f=\int_Af+\int_Bf$, proposition \ref{PropOPSCooVpzaBt}.
    \item
        La \( \sigma\)-additivité dénombrable, $\int_{\bigcup_iA_i}fd\mu=\sum_{i=0}^{\infty}\int_{A_i}fd\mu$ est dans les propositions \ref{PROPooTFOAooJBwmCV} et \ref{PROPooDWYNooWKJmEV}.
\end{enumerate}
\end{description}


\InternalLinks{suites et séries}

\begin{description}
    \item[Suites] 
        Les suites réelles sont en général dans la proposition \ref{PropLimiteSuiteNum} et ce qui s'ensuit. Cette proposition est souvent prise comme définition lorsque seules les suites réelles sont considérées.
        \begin{enumerate}
    \item
        Les suites adjacentes, c'est la définition \ref{DEFooDMZLooDtNPmu}. 
    \item
        Les séries alternées, théorème \ref{THOooOHANooHYfkII}. Il s'agit de dire que \( \sum_{k=0}^{\infty}(-1)^ka_k\) converge quand \( a_k\) est décroissante et tend vers zéro.
    \item
        Le concept de suite adjacente sert à étudier la série de Taylor de \( \ln(x+1)\), voir le lemme \ref{LEMooWMGGooRpAxBa} et ce qui l'entoure.
    \item
        La définition de la convergence absolue est la définition~\ref{DefVFUIXwU}.
            \item
                Une suite réelle croissante et majorée converge, proposition \ref{LemSuiteCrBorncv}.
            \item
                Toute suite dans un compact admet une sous-suite convergente, théorème \ref{THOooRDYOooJHLfGq}.
            \item
                Pour tout réel, il existe une suite croissante de rationnels qui y converge, proposition \ref{PropSLCUooUFgiSR}.
        \end{enumerate}
    \item[Produit de Cauchy]
        \begin{enumerate}
            \item
                Dans une algèbre normée, proposition \ref{PROPooFMEXooCNjdhS},
            \item
                Dans \( \eC\), théorème \ref{ThokPTXYC}.
        \end{enumerate}
    \item[Calcul de suites]
        \begin{enumerate}
            \item
                Somme : \( x_n+y_n\to x+y\) est la proposition \ref{PROPooICZMooGfLdPc}.
        \end{enumerate}
    \item[Série] 
        Les séries sont en général dans la section \ref{SECooYCQBooSZNXhd}.
        \begin{enumerate}
    \item
        Quelques séries usuelles en \ref{SUBSECooDTYHooZjXXJf} : série harmonique, géométrique, de Riemann, et la mythique arithmético-géométrique.
        \begin{enumerate}
            \item
                La série est associative : \( \sum_k(a_k+b_k)=\sum_ka_k+\sum_kb_k\). C'est la proposition \ref{PROPooUEBWooUQBQvP}.
            \item
                La série harmonique diverge : \( \sum_k\frac{1}{ k }=\infty\), exemple \ref{EXooDVQZooEZGoiG}.
            \item
                La série géométrique : \( \sum_{k=0}^Nq^k=\frac{ 1-q^{N+1} }{ 1-q }\), proposition \ref{PROPooWOWQooWbzukS}.
            \item
                Une autre cool série : \( \sum_{k=1}^N\frac{ 1 }{ k(k+1) }=\frac{ N }{ N+1 }\), lemme \ref{LEMooKDHPooPlFTIT}.
        \end{enumerate}
    \item
        Critère des séries alternées, théorème \ref{THOooOHANooHYfkII}.
    \item
        Convergence d'une série implique convergence vers zéro du terme général, proposition~\ref{PROPooYDFUooTGnYQg}.
    \item
        Dans une algèbre normée : \( (\sum_{k=0}^{\infty}a_k)b=\sum_{k=0}^{\infty}(a_kb)\), proposition \ref{PROPooMZZQooEhQsgQ}.
    \item
        Produit de Cauchy : théorème \ref{ThokPTXYC} et proposition \ref{PROPooFMEXooCNjdhS}.
        \end{enumerate}

    \item[Sommes infinies]
        En ce qui concerne les sommes finies, la notation \( \sum_{i=1}^N\) est définie en \ref{DEFooNEVNooJlmJOC}. Pour permuter les termes d'une somme avec un élément du groupe symétrique, nous avons la proposition \ref{PROPooJBQVooNqWErk}.

        Voici quelques résultats à propos de sommes infinies.% cette phrase est là pour le mot-clef ``somme infinie''.
        \begin{enumerate}
            \item
Une somme indexée par un ensemble quelconque est la définition~\ref{DefHYgkkA}.
    \item
        La définition de la somme d'une infinité de termes est donnée par la définition~\ref{DefGFHAaOL}.
    \item
        Une somme de termes positifs indexée par un ensemble indénombrable est toujours infinie par le lemme \ref{LEMooQIMGooOUpZjk}.
  \item
      si la série converge, on peut regrouper ses termes sans modifier la convergence ni la somme (associativité);
      pour les sommes infinies l'associativité et la commutativité dans une série sont perdues. Néanmoins, il subsiste que
  \begin{enumerate}
  \item
      si la série converge absolument, on peut modifier l'ordre des termes sans modifier la convergence ni la somme (commutativité, proposition~\ref{PopriXWvIY}).
  \end{enumerate}
  \item Permuter une somme infinie avec une application linéaire : \( f(\sum_{i\in I}v_i)=\sum_{i\in I}f(v_i)\), c'est la proposition \ref{PROPooWLEDooJogXpQ}.
        \end{enumerate}
    \item[Série entières]
        \begin{enumerate}
            \item
                Rayon de convergence, définition \ref{DefZWKOZOl}.
            \item
                Convervence absolue à l'intérieur du rayon de convergence, lemme d'Abel \ref{LemmbWnFI}.
            \item
            La fonction définie par la série entière  \(z\mapsto \sum_{k=0}^{\infty}a_nz^n\) est holomorphe dans son disque de convergence par la proposition \ref{PropSNMEooVgNqBP}.
        \item
            La série entière pour \( \frac{1}{ 1-z^k }\), pour \( \frac{1}{ \omega-z }\) et pour \( \frac{1}{ (\omega-z)^k }\) sont dans le lemme \ref{LemPQFDooGUPBvF}.
        \end{enumerate}
\end{description}


\InternalLinks{polynôme de Taylor}

\begin{description}
    \item[Énoncés] 

        Il existe de nombreux énoncés du théorème de Taylor, et en particulier beaucoup de formules pour le reste.

    \begin{enumerate}
    \item
        Énoncé : théorème~\ref{ThoTaylor}.
    \item
        Une majoration du reste est dans le théorème \ref{THOooEUVEooXZJTRL}
    \item
        De classe \( C^2\) sur \( \eR^n\), proposition~\ref{PROPooTOXIooMMlghF}.
    \item
    Avec un reste donné par un point dans \( \mathopen] x , a \mathclose[\), proposition~\ref{PropResteTaylorc}.
        \item
            Avec reste intégral, proposition~\ref{PropAXaSClx} et théorème \ref{THOooDGCJooXKmFTT} pour le cas simple \( \eR\to \eR\).
        \item
            Le polynôme de Taylor généralise à l'utilisation de toutes les dérivées disponibles le résultat de développement limité donné par la proposition~\ref{PropUTenzfQ}.
        \item
            Pour les fonctions holomorphes, il y a le théorème~\ref{THOooSULFooHTLRPE} qui donne une série de Taylor sur un disque de convergence.
        \end{enumerate}

    \item[Utilisation]

        Des polynômes de Taylor sont utilisés pour démontrer des théorèmes par-ci par-là.

\begin{enumerate}
        \item
            Il est utilisé pour justifier la méthode de Newton autour de l'équation \eqref{EQooOPUBooYaznay}.
    \item
        On utilise pas mal de Taylor dans les résultats liant extrémum et différentielle/hessienne. Par exemple la proposition~\ref{PropoExtreRn}.
\end{enumerate}

\item[Quelques développements]

Voici quelques développements limités à savoir. Ils sont calculables en utilisant la formule de Taylor-Young (proposition~\ref{PropVDGooCexFwy}).
\begin{subequations}
    \begin{align*}
        e^x&=\sum_{k=0}^n\frac{ x^k }{ k! }+x^n\alpha(x)&\text{ordre } n, \text{proposition \ref{PROPooQBRGooAhGrvP}}\\
        \cos(x)&=\sum_{k=0}^p\frac{ (-1)^kx^{2k} }{ (2k)! }+x^{2p+1}\alpha(x)&\text{ordre } 2p+1,\text{proposition \ref{PROPooNPYXooTuwAHP}}\\
        \sin(x)&=\sum_{k=0}^p\frac{ (-1)^kx^{2k+1} }{ (2k+1)! }+x^{2p+2}\alpha(x)&\text{ordre } 2p+1,\text{proposition \ref{PROPooNPYXooTuwAHP}}\\
        \ln(1+x)&=\sum_{k=1}^n\frac{ (-1)^{k+1} }{ k }x^k+\alpha(x)x^n&\text{ordre }n,\text{proposition \ref{PROPooWCUEooJudkCV}}\\
        \ln(1+x)&=\sum_{k=1}^{\infty}\frac{ (-1)^{k+1} }{ k }x^k&\text{exact }\text{proposition \ref{PROPooKPBIooJdNsqX}}\\
        \ln(2)&=\sum_{k=1}^{\infty}\frac{ (-1)^{k+1} }{ k }&\text{exact }\text{proposition \ref{PROPooKPBIooJdNsqX}}\\
      (1+x)^l&=\sum_{k=0}^l\binom{ l }{ k }x^k&\text{exact si } l\text{ est entier.}\\
      (1+x)^l&=1+\sum_{k=1}^n\frac{l(l-1)\ldots(l-k+1) }{ k! }x^k+x^n\alpha(x)&\text{ordre } n.
    \end{align*}
\end{subequations}
  Dans toutes ces formules, la fonction \( \alpha\colon \eR\to \eR\) vérifie \( \lim_{t\to 0} \alpha(t)=0\).

Le développement limité en $0$ d'une fonction paire ne contient que les puissances de $x$ d'exposant pair. Voir comme exemple le développement de la fonction cosinus.

\end{description}

\InternalLinks{normes}      \label{THEMEooUJVXooZdlmHj}

\begin{description}
    \item[Définition] Espace vectoriel normé : définition~\ref{DefNorme}.
    \item[Équivalence de norme]

        \begin{enumerate}
        \item
            Définition de l'équivalence de norme~\ref{DefEquivNorm}.
\item
    La proposition~\ref{PropLJEJooMOWPNi} sur l'équivalence des normes \( \| . \|_2\), \( \| . \|_1\) et \( \| . \|_{\infty}\)  dans \( \eR^n\).
\item
     En général pour les normes \( \| . \|_p\), il y a des inégalités dans \ref{THOooPPDPooJxTYIy} et \ref{CORooMBQMooWBAIIH}; voir aussi le thème \ref{THEMEooUJVXooZdlmHj}.
 \item
     La proposition \ref{PROPooQZTNooGACMlQ} donne l'inégalité \( \| x \|_q\leq n^{\frac{1}{ q }-\frac{1}{ p }}\| x \|_p\) dès que \( 0<q<p\).
\item
    Toutes les normes sur un espace vectoriel de dimension finie sont équivalentes par le théorème \ref{ThoNormesEquiv}.
\item
    Montrer que le problème \( a-b\) est stable dans l'exemple~\ref{ExooXJONooTAYZVc}.
\item
    La proposition~\ref{PROPooWZJBooTPLSZp} donnant \( \rho(A)\leq \| A \|\) utilise l'équivalence de toutes les normes sur un espace vectoriel de dimension finie (théorème \ref{ThoNormesEquiv}.).

        \end{enumerate}

    \item[Norme opérateur et d'algèbre] voir le thème~\ref{THEMEooOJJFooWMSAtL}.

\end{description}

\InternalLinks{caractérisations séquentielles}  
    \begin{enumerate}
        \item
            Fonction séquentiellement continue, définition \ref{DefENioICV}.
        \item
            La continuité implique la continuité séquentielle, proposition \ref{fContEstSeqCont} et corolaire \ref{PropFnContParSuite}.
        \item
            Pour des espaces métrisables, la continuité séquentielle d'une fonction est équivalente à la continuité, proposition \ref{PropXIAQSXr}. 
        \item
            Une version spéciale pour \( \eR^m\) est donnée par le théorème \ref{ThoLimSuite}.
    \end{enumerate}


\InternalLinks{topologie produit}       \label{THEMEooYRIWooDXZnhX}
    \begin{enumerate}
        \item
            La définition de la topologie produit est~\ref{DefIINHooAAjTdY}.
        \item
            Pour les espaces vectoriels normés, le produit est donné par la définition~\ref{DefFAJgTCE}.
        \item
            L'équivalence entre la topologie de la norme produit et la topologie produit est le lemme~\ref{DefFAJgTCE}.
        \item
            Quand \( V\) et \( W\) sont des espaces métriques, la topologie considérée sur \( V\times W\) est celle de la définition \ref{DefFAJgTCE}. C'est à la fois la topologie de la norme produit et la topologie produit.
        \item
            La convergence dans un espace vectoriel est si et seulement si il y a convergence composante par composante, proposition \ref{PROPooNRRIooCPesgO}.
        \end{enumerate}

\InternalLinks{espaces métriques, normés}
\begin{enumerate}
    \item
        Un espace métrique est un ensemble muni d'une distance, définition \ref{DefMVNVFsX}.
    \item
        La distance entre un point et un ensemble est la définition \ref{DEFooGNNUooFUZINs}.
    \item
        Le théorème-définition~\ref{ThoORdLYUu} donne la topologie sur un espace métrique en disant que les boules ouvertes sont une base de la topologie (définition \ref{DEFooLEHPooIlNmpi}).
    \item
        La définition de la convergence d'une suite est la définition~\ref{DefXSnbhZX}.
    \item
        Dans un espace vectoriel normé, une application est continue si et seulement si elle est bornée, proposition~\ref{PROPooQZYVooYJVlBd}.
    \item
        Un espace vectoriel topologique\footnote{Définition \ref{DefEVTopologique}.} qui possède en tout point une base dénombrable de topologie accepte une distance, théorème \ref{THOooAGBXooZnvQLK}.
\end{enumerate}


\InternalLinks{gaussienne}
\begin{enumerate}
    \item
        Le calcul de l'intégrale
        \begin{equation*}
            \int_{\eR} e^{-x^2}dx=\sqrt{\pi }
        \end{equation*}
        est fait de deux façons dans l'exemple~\ref{EXooLUFAooGcxFUW}. Dans les deux cas, le théorème de Fubini~\ref{ThoFubinioYLtPI} est utilisé.
    \item
        Le lemme~\ref{LEMooPAAJooCsoyAJ} calcule la transformée de Fourier de $ g_{\epsilon}(x)=  e^{-\epsilon\| x \|^2}$ qui donne $\hat g_{\epsilon}(\xi)=\left( \frac{ \pi }{ \epsilon } \right)^{d/2} e^{-\| \xi \|^2/4\epsilon}$.
    \item
        Le lemme~\ref{LEMooTDWSooSBJXdv} donne une suite régularisante à base de gaussienne.
    \item
        Elle est utilisée pour régulariser une intégrale dans la preuve de la formule d'inversion de Fourier~\ref{PROPooLWTJooReGlaN}
\end{enumerate}


\InternalLinks{compacts}        \label{THEMEooQQBHooLcqoKB}
    \begin{description}

        \item[Propriétés générales]

            Quelques propriétés de compacts.

                \begin{enumerate}
    \item
        La définition d'un ensemble compact est la définition~\ref{DefJJVsEqs}.
    \item
        Ne pas confondre le compactifié d'Alexandrov \ref{PROPooHNOZooPSzKIN} avec la droite réelle achevée \ref{DEFooRUyiBSUooALDDOa}.
        %TODOooWMZLooIqtVdb
    \item
        Si \( M\) est un compact de \( A\times B\), alors \( M\subset K\times L\) où \( K\) est compact de \( A\) et \( L\) de \( B\), proposition \ref{PropGBZUooRKaOxy}.
    \item 
        Un fermé dans un compact est compact, lemme \ref{LemnAeACf}
    \item
        Tout compact d'un espace topologique séparé est fermé, lemme \ref{LemnAeACf}\ref{ITEMooAZWVooLyPDeY}.
    \item
        Dans un espace vectoriel réel de dimension finie, les compacts sont les fermés bornés par le théorème~\ref{ThoXTEooxFmdI}.
    \item
        Le théorème de Borel-Lebesgue \ref{ThoBOrelLebesgue} dit qu'un intervalle\footnote{Définition \ref{DefEYAooMYYTz}.} de \( \eR\) est compact si et seulement si il est de la forme \( \mathopen[ a , b \mathclose]\).
    \item
        Théorème des fermés emboîtés dans le cas compact, corolaire \ref{CORooQABLooMPSUBf}. À ne pas confondre avec celui dans le cas des espaces métrique, théorème \ref{ThoCQAcZxX}.
    \item
        L'image d'un compact par une fonction continue est un compact, théorème~\ref{ThoImCompCotComp}.
    \item
        Suites dans un compact
        \begin{enumerate}
            \item
                Toute suite dans un compact admet une sous-suite convergente, théorème \ref{THOooRDYOooJHLfGq}.
            \item
                Dans \( \eR^n\), toute suite dans un compact admet une sous-suite convergente, théorème \ref{ThoBolzanoWeierstrassRn}. La démonstration de ce théorème est non seulement plus compliquée que le cas général, mais utilise en plus le cas dans \( \eR\); lequel cas n'est pas démontré de façon directe dans le Frido.
            \item
                Un espace métrique est compact si et seulement si toute suite contient une sous-suite convergente. C'est le théorème de Bolzano-Weierstrass~\ref{ThoBWFTXAZNH}. La démonstration de ce théorème est indépendante.
        \end{enumerate}
    \item
        Une fonction continue sur un compact est bornée et atteint ses bornes, théorème~\ref{ThoWeirstrassRn}.
    \item
        Une fonction continue sur un compact y est uniformément continue, théorème de Heine \ref{PROPooBWUFooYhMlDp}.
                \end{enumerate}

        \item[Produits de compacts]
            À propos de produits de compacts. C'est un compact dans tous les cas métriques\quext{Si vous connaissez des exemples non métriques de produits de compacts qui ne sont pas compacts, écrivez-moi.}.
    \begin{enumerate}
    \item
        Les produits d'espaces métriques compacts sont compacts. Il s'agit du théorème de Tykhonov que nous verrons ce résultat dans les cas suivants.
        \begin{itemize}
    \item
         \( \eR\), lemme~\ref{LemCKBooXkwkte}.
    \item
        Produit fini d'espaces métriques compacts, théorème~\ref{THOIYmxXuu}.
    \item
        Produit dénombrable d'espaces métriques compacts, théorème~\ref{ThoKKBooNaZgoO}.
        \end{itemize}
    \end{enumerate}
    \end{description}

\InternalLinks{densité}         \label{THEooPUIIooLDPUuq}
\begin{enumerate}
    \item
        Densité de \( \eQ\) dans \( \eR\), proposition \ref{PropooUHNZooOUYIkn}.
    \item
        Densité des polynômes dans \( \Big( C^0\big( \mathopen[ 0 , 1 \mathclose] \big),\| . \|_{\infty} \Big)\), théorème de Bernstein~\ref{ThoDJIvrty}.
    \item
        Densité des polynômes dans \( \big( C^0(I),\| . \|_{\infty} \big)\) lorsque \( I=\mathopen[ a , b \mathclose]\), corolaire \ref{CORooCWLMooWwCOAP}.
    \item
        Densité de \( \swD(\eR^d)\) dans \( L^p(\eR^d)\) pour \( 1\leq p<\infty\), théorème~\ref{ThoILGYXhX}.
    \item
        Densité de \( \swS(\eR^d)\) dans l'espace de Sobolev \( H^s(\eR^d)\), proposition~\ref{PROPooMKAFooKDNTbO}.

    \item
        Densité de \( \swD(\eR^d)\) dans l'espace de Sobolev \( H^s(\eR^d)\), proposition~\ref{PROPooLIQJooKpWtnV}.

        Cela est utilisé pour le théorème de trace~\ref{THOooXEJZooBKtXBW}.
    \item
        Les applications étagées dans les applications mesurables (qui plus est avec limite croissante), théorème fondamental d'approximation~\ref{LempTBaUw}.
    \item
        Les fonctions continues à support compact dans \( L^2(I)\), théorème~\ref{ThoJsBKir}.
    \item
        Les polynômes trigonométriques sont denses dans \( L^p(S^1)\) pour \( 1\leq p<\infty\). Deux démonstrations indépendantes par le théorème~\ref{ThoDPTwimI} et le théorème~\ref{ThoQGPSSJq}.
\end{enumerate}
Les densités sont bien entendu utilisées pour prouver des formules sur un espace en sachant qu'elles sont vraies sur une partie dense. Mais également pour étendre une application définie seulement sur une partie dense. C'est par exemple ce qui est fait pour définir la trace \( \gamma_0\) sur les espaces de Sobolev \( H^s(\eR^d)\) en utilisant le théorème d'extension~\ref{PropTTiRgAq}.

Comme presque tous les théorèmes importants, le théorème de Stone-Weierstrass possède de nombreuses formulations à divers degrés de généralité.
\begin{itemize}
    \item Le lemme~\ref{LemYdYLXb} le donne pour la racine carré.
    \item Le théorème~\ref{ThoGddfas} donne la densité des polynômes dans les fonctions continues sur un compact.
    \item Le théorème~\ref{THOooMDILooGPXbTW} est une généralisation qui donne la densité uniforme d'une sous-algèbre de \( C(X,\eR)\) dès que \( X\) sépare les points.
    \item Le théorème \ref{ThoWmAzSMF} donne le même résultat pour la densité dans \( C(X,\eC)\).
    \item Le lemme~\ref{LemXGYaRlC} est une version pour les polynômes trigonométriques.
    \item
        Le lemme~\ref{LemYdYLXb} est un cas particulier du
        théorème~\ref{ThoGddfas}, mais nous en donnons une démonstration indépendante afin d'isoler la preuve
de la généralisation~\ref{ThoWmAzSMF}.
Une version pour les polynômes trigonométriques sera donnée dans le lemme~\ref{LemXGYaRlC}.
\end{itemize}
Le théorème de Stone-Weierstrass est utilisé, entre autres nombreuses choses, pour prouver la densité des polynômes trigonométriques dans les fonctions continues sur \( S^1\), voir la proposition \ref{PROPooTGBHooXGhdPR}.


\InternalLinks{espaces de fonctions}                \label{THEMooNMYKooVVeGTU}

En ce qui concerne les densités, voir le thème~\ref{THEooPUIIooLDPUuq}.
\begin{enumerate}
    \item
        \(  C^{\infty}(A)\) est l'ensemble des fonctions de classe \(  C^{\infty}\) sur \( A\). Les éléments de \(  C^{\infty}(A)\) peuvent être à valeurs dans \( \eR\) ou dans \( \eC\) selon le contexte.
    \item
        \( \swD(A)\) est l'ensemble des fonctions de classe \(  C^{\infty}\) dont le support est un compact dans \( A\).
\end{enumerate}
Pour les ensembles \( \swS(A)\), \( \swD(A)\) et \( L^p(A)\), les fonctions sont à valeurs dans \( \eC\). La raison est que, de toutes façons, le passage à la transformée de Fourier produit en général des fonctions à valeurs dans \( \eC\) même si les fonctions de départ sont à valeurs dans \( \eR\).

\begin{description}
    \item[Topologie]

        Les espaces de fonctions sont souvent munis de topologies définies par des seminormes.

        \begin{enumerate}
            \item
                La topologie des seminormes est la définition~\ref{DefPNXlwmi}.
            \item
                La définition~\ref{DefFGGCooTYgmYf} donne les topologies sur \(  C^{\infty}(\Omega)\), \( \swD(K)\) et \( \swD(\Omega)\).
            \item
                La topologie \( *\)-faible sur \( \swD'(\Omega)\) est donnée par la définition~\ref{DefASmjVaT}.
        \end{enumerate}

    \item[L'espace \( { L^2\big( \mathopen[ 0 , 2\pi \mathclose] \big) } \)]

        C'est un espace très important, entre autres parce qu'il est de Hilbert et est bien adapté à la transformée de Fourier.

        \begin{enumerate}
        \item
            Un rappel de la construction en \ref{NORMooUEIEooYtlFse}.
            \item
                Le produit scalaire \( \langle f, g\rangle \) est donné en \eqref{EQooBFKDooMkCZOt} et la base trigonométrique est \eqref{EQooKMYOooLZCNap}.
            \item
                La densité des polynômes trigonométriques dans \( L^p(S^2)\) est le théorème~\ref{ThoQGPSSJq} ou le théorème~\ref{ThoDPTwimI}, au choix.
            \item
                Une conséquence de cette densité est que le système trigonométrique est une base hilbertienne\footnote{Définition \ref{DEFooADQXooFoIhTG}.} de \( L^2\) par le lemme~\ref{LEMooBJDQooLVPczR}.
        \end{enumerate}

            L'espace \( L^2\) est discuté en analyse fonctionnelle, dans la section \ref{SECooEVZSooLtLhUm} et les suivantes parce que l'étude de \( L^2\) utilise entre autres l'inégalité de Hölder~\ref{ProptYqspT}.

        Le fait que \( L^2\) soit un espace de Hilbert est utilisé dans la preuve du théorème de représentation de Riesz~\ref{PropOAVooYZSodR}.
\end{description}
Si \( (\Omega,\tribA,\mu)\) est un espace mesuré, alors \( L^p(\Omega,\tribA,\mu)\) est un espace de Banach; c'est le théorème de Riesz-Fischer \ref{ThoGVmqOro}.


\InternalLinks{fonctions Lipschitz}
    \begin{enumerate}
    \item
        Définition :~\ref{DEFooQHVEooDbYKmz}.
    \item
        La notion de Lipschitz est utilisée pour définir la stabilité d'un problème, définition~\ref{DEFooYIFAooSJbMkC}.
    \end{enumerate}


    \InternalLinks{formule des accroissements finis}        \label{INTERNooXFNTooNNaOzP}
    Il en existe plusieurs formes :
    \begin{enumerate}
        \item
            Une version adaptée aux espaces normés de dimension finie, avec hypothèse de différentiabilité, est le théorème~\ref{val_medio_2}. La formule $\|f(b)-f(a)\|_n\leq \sup_{x\in[a,b]}\|df_x\|_{\aL(\eR^m,\eR^n)}\|b-a\|_m$.
        \item
            Une version pour les dérivées partielles est dans le lemme \ref{LEMooNMTAooLgMkgH}. Pour rappel, la définition de la dérivation partielle est \ref{DEFooCATTooTPLtpR}.
        \item
            La formule \( f(a+\epsilon e_i)=f(a)+\epsilon(\partial_if)(a)+\epsilon\alpha(\epsilon)\), proposition \ref{PROPooYYSMooUDxtlB}.
        \item
        L'existence de \( c\in \mathopen] a , b \mathclose[\) tel que
            \begin{equation}
                f'(c)=\frac{ f(b)-f(a) }{ b-a }
            \end{equation}
            est le théorème des accroissements finis proprement dit. C'est le théorème \ref{ThoAccFinis}.
        \item
            Il existe un \( c\) entre \( a\) et \( b\) tel que
            \begin{equation}
                f(b)=f(a)+(b-a)(\partial_{\beta}f)(c)
            \end{equation}
            où \( \beta=b-a\) est la proposition \ref{PROPooCAWBooINcNxj}.
        \item
            La formule \( f(a+h)=f(a)+hf'(a)+\alpha(h)\) pour une fonction \( \eR\to \eR\) en le théorème \ref{PropUTenzfQ}.
        \item
            Une généralisation pour les intervalles non bornés : théorème~\ref{THOooRIIBooOjkzMa}.
        \item
            Espaces vectoriels normés, théorème~\ref{ThoNAKKght}
    \end{enumerate}

\InternalLinks{limite et continuité}    \label{THEMEooGVCCooHBrNNd}

\begin{enumerate}
    \item
        Limite d'une fonction en un point : définition \ref{DefYNVoWBx}. Il n'y a pas unicité en général comme le montre l'exemple \ref{EXooSHKAooZQEVLB} dans un espace non séparé.
    \item
        Caractérisation de la limite dans \( \eR\), proposition \ref{PropAJQQooQQClfp}.
    \item
        Unicité de la limite d'une suite dans un espace séparé : proposition \ref{PropUniciteLimitePourSuites}. Unicité de la limite d'une fonction, toujours dans le cas d'un espace séparé : proposition \ref{PropFObayrf}.
    \item
        La proposition~\ref{PropRBCiHbz} donne l'unicité de la limite dans le cas des espaces duaux pour la topologie \( *\)-faible. La proposition~\ref{PropFObayrf} nous dira qu'il y a unicité dès que l'espace d'arrivée est séparé.
    \item
        Définition de la continuité d'une fonction en un point et sur une partie de l'espace de départ : définition~\ref{DefOLNtrxB}.
    \item
        Continuité sur une partie si et seulement si continue en chaque point, c'est le théorème~\ref{ThoESCaraB}.
    \item
        Voir l'exemple~\ref{EXooKREUooLeuIlv} traité en détail pour la (non) continuité d'une fonction qui fait un saut en un point.
    \item
        La fonction \( f(x,y)=x+y\) est continue, lemme \ref{LEMooGKIPooWgpFTB}.
\end{enumerate}

\InternalLinks{suite de fonctions}
\begin{enumerate}
    \item
        Une limite uniforme de fonctions continues est continue, proposition \ref{PropCZslHBx}.
    \item
        Sous certaines hypothèses, si \( f_i\to f\), alors \( f'_i\to f'\), théorème \ref{THOooXZQCooSRteSr}.
\end{enumerate}

\InternalLinks{dérivation}
\begin{enumerate}
    \item
        Définition de la dérivée, définition \ref{DEFooOYFZooFWmcAB}.
    \item
        Dérivée de fonction composée, proposition \ref{PROPooDONLooWthqRR} dans le cas réel.
    \item
        Dérivée partielle de fonction composée, théorème \ref{THOooKBTYooWFtoSF}.
    \item
        \( f(\lambda x)'=\lambda f'(x)\), lemme \ref{LEMooXHVBooHYjXdq}.
\end{enumerate}

\InternalLinks{différentiabilité}
\begin{description}
    \item[Généralités] 
\begin{enumerate}
    \item
        La différentielle est définie en général pour des espaces vectoriels normés par la proposition \ref{DefDifferentiellePta}
    \item
        Différentielle de fonction composée, formule \( f(g\circ f)_a=dg_{f(a)}\circ df_a\), théorème \ref{THOooIHPIooIUyPaf}.
    \item
        Nous parlons de différentielle en dimension finie et donnons une interprétation géométrique en~\ref{SEBSECooLPRQooJRQCFL}.
    \item
        La recherche d'extrémums d'une fonction sur \( \eR^n\) passe par la seconde différentielle, proposition~\ref{PropoExtreRn}.
    \item
        Lien entre différentielle seconde (hessienne) et convexité en la proposition~\ref{PROPooBMIRooFkQSAb} et le corolaire \ref{CORooMBQMooWBAIIH}.
    \item
        La différentielle est liée aux dérivées partielles par les formules données au lemme~\ref{LemdfaSurLesPartielles}
	\begin{equation}
        df_a(u)=\frac{ \partial f }{ \partial u }(a)=\Dsdd{ f(a+tu) }{t}{0}=\sum_{i=1}^mu_i\frac{ \partial f }{ \partial x_i }(a)=\nabla f(a)\cdot u.
	\end{equation}
    Je ne vous cache pas que cette suite d'égalités est une de mes préférées.
\end{enumerate}
\item[Différentielle et dérivées partielles]
    À propos de fonctions de classe \( C^k\), définition \ref{DefPNjMGqy}.
    \begin{enumerate}
    \item
        Une fonction est de classe \( C^1\) si et seulement si ses dérivées partielles sont continues, théorème \ref{THOooBEAOooBdvOdr}.
    \item
        Une fonction est \( C^n\) si et seulement si ses dérivées partielles sont \( C^{n-1}\), c'est le théorème \ref{THOooPZTAooTASBhZ}.
    \item
        Différentiabilité en un seul point si les dérivées partielles sont continues en ce point : proposition \ref{PROPooUDJLooHwzjQF}.
    \end{enumerate}
\item[Fonctions composées]
    À propos de la formule \( d(f\circ g)_a=dg_{f(a)}\circ df_a\), il y a deux théorèmes très semblables.
    \begin{enumerate}
        \item
            Le théorème \ref{THOooIHPIooIUyPaf} insiste sur des hypothèses locales.
        \item
            Le théorème \ref{ThoAGXGuEt} fait des hypothèses plus globales pour s'alléger l'esprit, mais fait une récurrence pour dire que \( f\circ g\) est de classe \( C^r\) si \( f\) et \( g\) le sont.
    \end{enumerate}
\end{description}

\InternalLinks{Morphismes et compagnie}     
\begin{enumerate}
    \item
        Un morphisme est un concept algébrique. Il s'agit d'une application (pas spécialement inversible) qui préserve la structure. Quand on parle de morphisme, il faut donc préciser la structure. On dit «morphisme de groupe», «morphisme d'espace vectoriel», «morphisme d'anneaux», etc.
    \item
        Un homéomorphisme est continue d'inverse continue. C'est ce qu'on appelle un isomorphisme d'espaces topologiques. Définition \ref{DEFooYPGQooMAObTO}.
    \item
        Un difféomorphisme est différentiable d'inverse différentiable, définition \ref{DefAQIQooYqZdya}.
    \item
        Un \( C^k\)-difféomorphisme est une application \( C^k\) d'inverse \( C^k\). Définition \ref{DefAQIQooYqZdya}.
    \ifbool{isGiulietta}{
    \item
        Un homomomorphisme de groupes de Lie est un morphisme de groupe \(  C^{\infty}\). Nous ne demandons pas que l'inverse ait une régularité particulière.
    }{}
\end{enumerate}
Le mot «homomorphisme» signifie exactement «morphisme», et, sauf incohérence de ma part, il n'est pas utilisé dans le Frido.

\InternalLinks{points fixes}        \label{THEMEooWAYJooUSnmMh}
    \begin{enumerate}
\item
    Il y a plusieurs théorèmes de points fixes.
    \begin{description}
        \item[Théorème de Picard]~\ref{ThoEPVkCL} donne un point fixe comme limite d'itérés d'une fonction Lipschitz. Il aura pour conséquence le théorème de Cauchy-Lipschitz~\ref{ThokUUlgU}, l'équation de Fredholm, théorème~\ref{ThoagJPZJ} et le théorème d'inversion locale dans le cas des espaces de Banach~\ref{ThoXWpzqCn}.
    \item[Théorème de Brouwer] qui donne un point fixe pour une application d'une boule vers elle-même. Nous allons donner plusieurs versions et preuves.
            \begin{enumerate}
                \item
                    Dans \( \eR^n\) en version \( C^{\infty}\) via le théorème de Stokes, proposition~\ref{PropDRpYwv}.
                \item
                    Dans \( \eR^n\) en version continue, en s'appuyant sur le cas \( C^{\infty}\) et en faisant un passage à la limite, théorème~\ref{ThoRGjGdO}.
                \item
                    Dans \( \eR^2\) via l'homotopie, théorème~\ref{ThoLVViheK}. Oui, c'est très loin. Et c'est normal parce que ça va utiliser la formule de l'indice qui est de l'analyse complexe\footnote{On aime bien parce que ça ne demande pas Stokes, mais quand même hein, c'est pas gratos non plus.}.
            \end{enumerate}
        \item[Théorème de Markov-Kakutani]\ref{ThoeJCdMP} qui donne un point fixe à une application continue d'un convexe fermé borné dans lui-même. Ce théorème donnera la mesure de Haar~\ref{ThoBZBooOTxqcI} sur les groupes compacts.
        \item[Théorème de Schauder] C'est une version valable en dimension infinie du théorème de Brouwer. Théorème \ref{ThovHJXIU} 
    \end{description}

\item Pour les équations différentielles
    \begin{enumerate}
        \item
            Le théorème de Schauder a pour conséquence le théorème de Cauchy-Arzela~\ref{ThoHNBooUipgPX} pour les équations différentielles.
        \item
            Le théorème de Schauder~\ref{ThovHJXIU} permet de démontrer une version du théorème de Cauchy-Lipschitz (théorème~\ref{ThokUUlgU}) sans la condition Lipschitz, mais alors sans unicité de la solution. Notons que de ce point de vue nous sommes dans la même situation que la différence entre le théorème de Brouwer et celui de Picard : hors hypothèse de type «contraction», point d'unicité.
    \end{enumerate}
\item
    En calcul numérique
    \begin{itemize}
        \item
            La convergence d'une méthode de point fixe est donnée par la proposition~\ref{PROPooRPHKooLnPCVJ}.
        \item
            La convergence quadratique de la méthode de Newton est donnée par le théorème~\ref{THOooDOVSooWsAFkx}.
        \item
            En calcul numérique, section~\ref{SECooWUVTooMhmvaW}
        \item
            Méthode de Newton comme méthode de point fixe, sous-section~\ref{SUBSECooIBLNooTujslO}.
    \end{itemize}

\item
    D'autres utilisations de points fixes.
\begin{itemize}
    \item
        Processus de Galton-Watson, théorème~\ref{ThoJZnAOA}.
    \item
        Dans le théorème de Max-Milgram~\ref{THOooLLUXooHyqmVL}, le théorème de Picard est utilisé.
\end{itemize}
\end{enumerate}


\InternalLinks{intégration sur des variétés}
\begin{description}
    \item[orientation]
        La notion d'orientation commence avec l'orientation des bases d'un espace vectoriel et continue jusqu'à orienter des variétés à partir de ses cartes.
        \begin{enumerate}
            \item
                Classe d'orientation sur les bases d'un espace vectoriel, définition \ref{DEFooNVRHooEBHUSu}.
            \item
                Orientation sur une surface, définition \ref{DEFooFTQLooXXbtOQ}.
            \item
                Variété orientable, définition \ref{DEFooSWREooNdQpdA}.
        \end{enumerate}
    \item[théorème de Stokes, théorème de Green et compagnie] 
        Tous ces théorèmes sont des conséquences plus ou moins directes de celui de Stokes, et des généralisations du théorème fondamental de l'analyse.
    \begin{enumerate}
        \item
            Forme générale, théorème~\ref{ThoATsPuzF}.
        \item
            Rotationnel et circulation, théorème~\ref{THOooIRYTooFEyxif}.
        \end{enumerate}
        Le théorème de Stokes peut être utilisé pour montrer le théorème de Brouwer, proposition~\ref{PropDRpYwv}.
\end{description}


\InternalLinks{permuter des limites}        \label{THEMEooJGEHooNzQkMT}
\begin{description}
    \item[Permuter des dérivées partielles]
        Si une fonction est de classe \( C^2\), le théorème de Schwarz \ref{Schwarz} dit que 
        \begin{equation}
            \partial_k\partial_lf=\partial_l\partial_kf.
        \end{equation}
    \item[Fonctions définies par une intégrale]
        Les théorèmes sur les fonctions définies par une intégrale, section~\ref{SecCHwnBDj}. Nous avons entre autres
        \begin{enumerate}
            \item
                \( \partial_i\int_Bf=\int_B\partial_if\), avec \( B\) compact, proposition~\ref{PropDerrSSIntegraleDSD}.
            \item
                Si \( f\) est majorée par une fonction ne dépendant pas de \( x\), nous avons le théorème~\ref{ThoKnuSNd} pour la continuité de \( x\mapsto \int_{\Omega}f(x,\omega)d\mu(\omega)\).
            \item
                Pour la fonction $F(x)=\int_{\Omega}f(x,\omega)d\mu(\omega)$, nous avons la dérivation sous l'intégrale par la formule de Leibnitz
                \begin{equation}
                    F'(a)=\int_{\Omega}\frac{ \partial f }{ \partial x }(a,\omega)d\mu(\omega)
                \end{equation}
                démontrée en le théorème \ref{ThoMWpRKYp}.

                Des variations avec des dérivées partielles et des différentielles sont dans \ref{PropDerrSSIntegraleDSD} et dans \ref{PropAOZkDsh}.
            \item
                Si \( f\colon \eC\times \Omega\to \eC\) est holomorphe (pour \( \eC\)), alors \( F\) est holomorphe et
    \begin{equation}
        F'(z)=\int_{\Omega}\frac{ \partial f }{ \partial z }(z,\omega)d\mu(\omega).
    \end{equation}
            \item 
                Pour des dérivées partielles multiples, nous avons la formule
                \begin{equation}
                    (\partial^{\alpha}F)(x)=\int_{\Omega}(\partial^{\alpha}f_{\omega})(a)d\mu(\omega)
                \end{equation}
                dans la proposition \ref{PROPooJKXJooLxgEGd}.
            \item
                Si l'intégrale est uniformément convergente, nous avons le théorème~\ref{ThotexmgE} qui donne la continuité de $F(x)=\int_{\Omega}f(x,\omega)d\mu(\omega)$.
            \item
                Pour dériver \( \int_Bg(t,z)dt\) avec \( B\) compact dans \( \eR\) et \( g\colon \eR\times \eC\to \eC\), il faut aller voir la proposition~\ref{PROPooZCLYooUaSMWA}.
            \item
                En ce qui concerne le \( x\) dans la borne, le théorème \ref{PropEZFRsMj} lie primitive et intégrale en montrant que \( F(x)=\int_a^xf(t)dt\) est une primitive de \( f\) (sous certaines conditions). Le théorème fondamental de l'analyse \ref{ThoRWXooTqHGbC} en est une conséquence.
            \item Si \( T\) est une \textbf{distribution}, alors nous avons
                \begin{equation}
                    T\big( x\mapsto (\partial_y^{\alpha}\phi)(x,y_0) \big)=\partial_y^{\alpha}\Big( T\big( x\mapsto \phi(x,y) \big) \Big)_{y=y_0}.
                \end{equation}
                C'est la proposition \ref{PROPooCNYTooWCKHpV}.
        \end{enumerate}
    \item[Limite et intégrale]
        \begin{enumerate}
            \item
        Théorème de la convergence monotone, théorème~\ref{ThoRRDooFUvEAN}.
        \end{enumerate}
    \item[Fubini]
        Le théorème de Fubini permet non seulement de permuter des intégrales, mais également des sommes parce que ces dernières peuvent être vues comme des intégrales sur \( \eN\) muni de la tribu des parties et de la mesure de comptage\footnote{Mesure de comptage, définition \ref{DEFooILJRooByDzhs}.}. Nous utilisons cette technique pour permuter une somme et une intégrale dans l'équation \eqref{EQooWOLOooFHSrsx}.
    \item
        L'utilisation de Fubini pour permuter des intégrales (sur deux variables différentes) ou deux sommes est expliquée dans \ref{NORMooKIRJooPvyPWQ}. 

        C'est par exemple utilisé pour permuter deux sommes dans le cadre des chaines de Markov en \ref{LEMooZIEPooXHGnvy}.
\begin{itemize}
    \item
        le théorème de Fubini-Tonelli~\ref{ThoWTMSthY} demande que la fonction soit mesurable et positive;
    \item
        le théorème de Fubini~\ref{ThoFubinioYLtPI} demande que la fonction soit intégrable (mais pas spécialement positive);
    \item
        le corolaire~\ref{CorTKZKwP} demande l'intégrabilité de la valeur absolue des intégrales partielles pour déduire que la fonction elle-même est intégrable.
\end{itemize}

\item[Limite et dérivées, différentielle]
    \begin{enumerate}
        \item
            Permuter limite et dérivée dans le cas \( \eR\to \eR\), théorème \ref{THOooXZQCooSRteSr}.
        \item
 Permuter limite et dérivées partielles, théorème \ref{ThoSerUnifDerr}.
        \item
            Permuter limite et différentielle, théorème \ref{ThoLDpRmXQ}.
    \end{enumerate}
    Quelques remarques sur les techniques de démonstration.
    \begin{enumerate}
        \item
            Le résultat fondamental \ref{THOooXZQCooSRteSr} est démontré sans recourir à des intégrales. Une preuve alternative, plus courte, avec des intégrales est donnée en \ref{NORMALooGYUEooKrYjyz}.
        \item
            Permuter limite et dérivée partielle, théorème \ref{ThoSerUnifDerr}.
        \item
            Permuter série et différentielle, théorème \ref{ThoLDpRmXQ}.
    \end{enumerate}
\item[Somme et dérivée]
    Permuter somme et différentielle, théorème \ref{ThoLDpRmXQ}.
\item[Limite et mesure]
    Une mesure n'est pas toujours une limite, mais la définition d'une mesure positive sur un espace mesurable parle de permuter limite et mesure : définition \ref{DefBTsgznn}\ref{ItemQFjtOjXiii}.
\end{description}


\InternalLinks{applications continues et bornées}       \label{THEMEooYCBUooEnFdUg}
\begin{enumerate}
    \item
        Une application linéaire non continue : exemple~\ref{ExHKsIelG} de \( e_k\mapsto ke_k\). Les dérivées partielles sont calculées en \eqref{EQooWNLOooJNRUMQ}.
    \item
        La dérivation sur les polynômes (exemple \ref{EXooDMVJooAJywMU}) donne un autre exemple d'application linéaire non continue.
    \item
        Une application linéaire est bornée si et seulement si elle est continue, proposition~\ref{PROPooQZYVooYJVlBd}.
\end{enumerate}


\InternalLinks{inégalités}
    Dans \( \eC\) nous avons \( | a+ b|\leq | a |+| b |\) par la proposition \ref{PROPooUMVGooIrhZZg}\ref{ITEMooDVMDooFDmOur}.
\begin{description}
    \item[Inégalité de Young]
        Nous avons
        \begin{equation}
            ab\leq \frac{ a^p }{ p }+\frac{ b^q }{ q }
        \end{equation}
        par la proposition \ref{PROPooCQUBooCvtMSi}.
    \item[Inégalité de Jensen]
        \begin{enumerate}
            \item
                Une version discrète pour \( f\big( \sum_i\lambda_ix_i \big)\), la proposition~\ref{PropXIBooLxTkhU}.
            \item
                Une version intégrale pour \( f\big( \int \alpha d\mu \big)\), la proposition~\ref{PropXISooBxdaLk}.
            \item
                Une version pour l'espérance conditionnelle, la proposition~\ref{PropABtKbBo}.
        \end{enumerate}
    \item[Inégalité pour les normes $ \ell^p$]
        \begin{enumerate}
            \item
                Hölder pour \( L^p\): \( \| fg \|_1\leq \| f \|_p\| g \|_q\), proposition \ref{ProptYqspT}.
            \item
                Hölder pour \( \ell^p\): \( \| x \|_q\leq n^{\frac{1}{ q }-\frac{1}{ p }}\| x \|_p\), proposition \ref{PROPooQZTNooGACMlQ}.
        \end{enumerate}
    \item[Inégalité de Minkowsky]
        \begin{enumerate}
            \item
                Pour une forme quadratique\footnote{Définition \ref{DefBSIoouvuKR}.} \( q\) sur \( \eR^n\) nous avons $\sqrt{q(x+y)}\leq\sqrt{q(x)}+\sqrt{q(y)}$. Proposition~\ref{PropACHooLtsMUL}.
            \item
                Si \( 1\leq p<\infty\) et si \( f,g\in L^p(\Omega,\tribA,\mu)\) alors \(  \| f+g \|_p\leq \| f \|_p+\| g \|_p\). Proposition~\ref{PropInegMinkKUpRHg}.
            \item
                L'inégalité de Minkowsky sous forme intégrale s'écrit sous forme déballée
                \begin{equation*}
                    \left[ \int_X\Big( \int_Y| f(x,y) |d\nu(y) \Big)^pd\mu(x) \right]^{1/p}\leq \int_Y\Big( \int_X| f(x,y) |^pd\mu(x) \Big)^{1/p}d\nu(y).
                \end{equation*}
                ou sous forme compacte
                \begin{equation*}
                    \left\|   x\mapsto\int_Y f(x,y)d\nu(y)   \right\|_p\leq \int_Y  \| f_y \|_pd\nu(y)
                \end{equation*}
        \end{enumerate}
    \item[Transformée de Fourier]
                Pour tout \( f\in L^1(\eR^n)\) nous avons \( \| \hat f \|_{\infty}\leq \| f \|_1\), lemme~\ref{LEMooCBPTooYlcbrR}.
    \item[Inégalité des normes]
        Inégalité de normes : si \( f\in L^p\) et \( g\in L^1\), alors \( \| f*g \|_p\leq \| f \|_p\| g \|_1\), proposition~\ref{PROPooDMMCooPTuQuS}.

\end{description}


\InternalLinks{connexité}
    \begin{enumerate}
        \item
            Définition~\ref{DefIRKNooJJlmiD}
        \item
            L'image d'un connexe par une fonction continue est connexe, proposition \ref{PropGWMVzqb}.
        \item
            Connexité par arcs, définition \ref{DEFooOXVCooBizpgK}.
        \item
            Une partie de \( \eR^2\) qui est connexe, mais pas connexe par arcs, proposition \ref{PROPooVXDNooPZYKPr}.
        \item
            Une partie de \( \eR\) est connexe si et seulement si elle est un intervalle, proposition \ref{PropInterssiConn}.
        \item
            Le groupe \( \SL(n,\eK)\) est connexe par arcs, proposition~\ref{PROPooALQCooLZCKrH}.
        \item
            Le groupe \( \GL(n,\eC)\) est connexe par arcs, proposition~\ref{PROPooVJNIooMByUJQ}.
        \item
            Le groupe \( \GL(n,\eR)\) a exactement deux composantes connexes par arcs, proposition~\ref{PROPooBIYQooWLndSW}.
        \item
            Le groupe \( \gO(n,\eR)\) n'est pas connexe, lemme~\ref{LEMooIPOVooZJyNoH}.
        \item
            Les groupes \( \gU(n)\) et \( \SU(n)\) sont connexes par arcs, lemme~\ref{LEMooQMXHooZQozMK}.
        \item
            Pour tout \( n\geq 2\), le groupe \( \SO(n)\) est connexe, le groupe \( \gO(n)\) a deux composantes connexes, proposition \ref{THOooYQFNooPaYmaP}.
        \item
            Connexité des formes quadratiques de signature donnée, proposition~\ref{PropNPbnsMd}.
        \end{enumerate}

\InternalLinks{suite de Cauchy, espace complet}     \label{THMooOCXTooWenIJE}

Nous parlons d'espaces topologiques complets. À ne pas confondre avec un espace mesuré complet, définition~\ref{DefBWAoomQZcI}.

\begin{enumerate}
    \item
        Corps complet : définition~\ref{DefKCGBooLRNdJf}\ref{ITEMooKZZYooDaidGU}, espace métrique complet : définition~\ref{DEFooHBAVooKmqerL}.
    \item
        La définition~\ref{DEFooXOYSooSPTRTn} donne la notion de suite de Cauchy dans un espace métrique.
    \item
        La définition~\ref{DefZSnlbPc} donne la notion de suite de \( \tau\)-Cauchy dans un espace vectoriel topologique.
    \item
        Deux espaces métriques (avec une distance) peuvent être isomorphes en tant qu'espaces topologiques, mais ne pas avoir les mêmes suites de Cauchy, exemple~\ref{EXooNMNVooXyJSDm}.
    \item
        La proposition~\ref{PropooUEEOooLeIImr} donne l'équivalence entre les suites de Cauchy et les suites \( \tau\)-Cauchy dans le cas des espaces vectoriels topologiques \emph{normés}.
    \item
        L'exemple~\ref{EXooNMNVooXyJSDm} est un exemple pire que simplement une suite de Cauchy qui ne converge pas. Le problème de convergence de cette suite n'est pas simplement que la limite n'est pas dans l'espace; c'est que la suite de Cauchy donnée ne convergerait même pas dans \( \eR\).
    \item
        Le théorème~\ref{ThoKHTQJXZ} est un théorème de complétion d'un espace métrique.
    \item
        Dans \( \eR\), une suite est convergente si et seulement si elle est de Cauchy, théorème \ref{THOooNULFooYUqQYo}\ref{ITEMooUUFCooIVtGgz}.
    \item
        Toute suite convergente dans un espace métrique est de Cauchy, proposition \ref{PROPooZZNWooHghltd}.
\end{enumerate}

Quelques espaces qui sont complets sont listés ci-dessous. Attention : la complétude est bien une propriété de la métrique; le même ensemble peut être complet pour une distance et pas pour une autre. Souvent, cependant la distance à considérer est donnée par  le contexte.
\begin{multicols}{2}
    \begin{enumerate}
        \item
            Les réels \( \eR\), théorème~\ref{THOooNULFooYUqQYo}.
        \item
            Un espace vectoriel normé sur un corps complet est complet, proposition~\ref{PROPooGJDTooXOoYfw}.
        \item
            La proposition~\ref{PropSYMEZGU} donne quelques espaces complets. Soit \( X\) un espace topologique métrique, \( (Y,d)\) un espace métrique complet. Alors les espaces
    \begin{enumerate}
        \item
            \( \big( C^0_b(X,Y),\| . \|_{\infty} \big)\)
        \item
            \( \big( C^0_0(X,Y),\| . \|_{\infty} \big)\)
        \item
            \( \big( C^k_0(X,Y),\| . \|_{\infty} \big)\)
    \end{enumerate}
    sont complets.
\item
    Le lemme~\ref{LemdLKKnd} dit que \( \big( C^0(A,B),\| . \|_{\infty}\big)\) est complet dès que \( A\) est compact et \( B\) est complet.

\item
    L'espace \( \swD(K)\) est complet tant pour la topologie des seminormes que pour la topologie métrique (qui sont les mêmes). C'est la proposition~\ref{PropQAEVcTi}.
\item
    L'espace \( \swS(\Omega)\) est complet et métrisable par la proposition~\ref{PropIIAcyDq}.
\item
    L'espace \( L^p(\Omega,\tribA,\mu)\) par le théorème \ref{ThoUYBDWQX}.
    \end{enumerate}
\end{multicols}

    La limite uniforme d'une suite de fonctions dérivables n'est pas spécialement dérivable. Même si les fonctions sont de classe \(  C^{\infty}\), la limite n'est pas spécialement mieux que continue. En effet, le théorème de Stone-Weierstrass~\ref{ThoGddfas} nous dit que les polynômes (qui sont \(  C^{\infty}\)) sont denses dans les fonctions continues sur un compact pour la norme uniforme. Vous ne pouvez donc pas espérer que \( \big( C^p(X,Y),\| . \|_{\infty} \big)\) soit complet en général.



\InternalLinks{application réciproque}
\begin{enumerate}
    \item
        Définition~\ref{DEFooTRGYooRxORpY}.
    \item
        Dans le cas des réels, des exemples sont donnés en~\ref{EXooCWYHooLEciVj}.
    \item
        Continuité, proposition~\ref{PropIntContMOnIvCont}.
    \item
        Théorème de la bijection~\ref{ThoKBRooQKXThd} (qui contient aussi de la continuité).
    \item
        Dérivabilité, proposition~\ref{PropMRBooXnnDLq}.
\end{enumerate}


\InternalLinks{déduire la nullité d'une fonction depuis son intégrale}
Des résultats qui disent que si \( \int f=0\) c'est que \( f=0\) dans un sens ou dans un autre.
\begin{enumerate}
    \item
        Il y a le lemme~\ref{Lemfobnwt} qui dit ça.
    \item
        Un lemme du genre dans \( L^2\) existe aussi pour \( \int f\varphi=0\) pour tout \( \varphi\). C'est le lemme~\ref{LemDQEKNNf}.
    \item
        Et encore un pour \( L^p\) dans la proposition~\ref{PropUKLZZZh}.
    \item
        Si \( \int f\chi=0\) pour tout \( \chi\) à support compact alors \( f=0\) presque partout, proposition~\ref{PropAAjSURG}.
    \item
        En utilisant le théorème de représentation de Riesz, on peut prouver que \( \int_{\Omega}f\chi=0\) imploque \( f=0\) pour tout \( f\in L^p\), proposition \ref{PropUKLZZZh}.
    \item
        La proposition~\ref{PropRERZooYcEchc} donne \( f=0\) dans \( L^p\) lorsque \( \int fg=0\) pour tout \( g\in L^q\).
    \item
        Une fonction \( h\in C^{\infty}_c(I)\) admet une primitive dans \(  C^{\infty}_c(I)\) si et seulement si \( \int_Ih=0\). Théorème~\ref{PropHFWNpRb}.
\end{enumerate}


\InternalLinks{équations différentielles}
L'utilisation des théorèmes de point fixe pour l'existence de solutions à des équations différentielles est fait dans le chapitre sur les points fixes.
\begin{enumerate}
    \item
            Le théorème de Schauder a pour conséquence le théorème de Cauchy-Arzela~\ref{ThoHNBooUipgPX} pour les équations différentielles.
        \item
            Le théorème de Schauder~\ref{ThovHJXIU} permet de démontrer une version du théorème de Cauchy-Lipschitz (théorème~\ref{ThokUUlgU}) sans la condition Lipschitz
        \item
            Le théorème de Cauchy-Lipschitz~\ref{ThokUUlgU} est utilisé à plusieurs endroits :
            \begin{itemize}
                \item
                    Pour calculer la transformée de Fourier de \(  e^{-x^2/2}\) dans le lemme~\ref{LEMooPAAJooCsoyAJ}.
            \end{itemize}
    \item
        Théorème de stabilité de Lyapunov~\ref{ThoBSEJooIcdHYp}.
    \item
        Le système proie-prédateur de Lotka-Volterra~\ref{ThoJHCLooHjeCvT}
    \item
        Équation de Schrödinger, théorème~\ref{ThoLDmNnBR}.
    \item
        L'équation \( (x-x_0)^{\alpha}u=0\) pour \( u\in\swD'(\eR)\), théorème~\ref{ThoRDUXooQBlLNb}.
    \item
        La proposition~\ref{PropMYskGa} donne un résultat sur \( y''+qy=0\) à partir d'une hypothèse de croissance.
    \item
        Équation de Hill \( y''+qy=0\), proposition~\ref{PropGJCZcjR}.
\end{enumerate}


\InternalLinks{injections}
\begin{enumerate}
        \item
            L'espace de Sobolev \( H^1(I)\) s'injecte de façon compacte dans \( C^0(\bar I)\), proposition~\ref{ThoESIyxfU}.
        \item
            L'espace de Sobolev \( H^1(I)\) s'injecte de façon continue dans \( L^2(I)\), proposition~\ref{ThoESIyxfU}.
        \item
            L'espace \( L^2(\Omega)\) s'injecte continument dans \( \swD'(\Omega)\) (les distributions), proposition~\ref{PROPooYAJSooMSwVOm}.
\end{enumerate}

\InternalLinks{logarithme}
\begin{enumerate}
    \item
    Le logarithme pour les réels strictement positifs \( \ln\colon \mathopen] 0 , \infty \mathclose[\to \eR\) est donné en la définition~\ref{DEFooELGOooGiZQjt}; c'est l'application réciproque de \( \exp\).
    \item
        Les principales propriétés sont dans la proposition \ref{PROPooLAOWooEYvXmI} : \( \ln(xy)=\ln(x)+\ln(y)\) etc.
    \item
        Dérivée : \( \ln'(x)=\frac{1}{ x }\), proposition \ref{PROPooPDJLooXphpEM}.
    \item
        La proposition \ref{PROPooKPBIooJdNsqX} donne la série
        \begin{equation}
            \ln(1+x)=\sum_{k=1}^{\infty}\frac{ (-1)^{k+1} }{ k }x^k.
        \end{equation}
    \item
        L'exemple \ref{EXooYMEEooMGpUNM} donne l'encadrement \( 0.644\leq \ln(2)\leq 0.846\).
    \item
        La proposition~\ref{PropKKdmnkD} dit que toute matrice complexe admet un logarithme. En particulier une série explicite est donnée pour le logarithme d'un bloc de Jordan.
    \item
        Sur les complexes, le logarithme \( \ln \colon \eC^*\to \eC\) est la définition~\ref{DEFooWDYNooYIXVMC}. Attention : ce n'est pas la seule définition possible.
    \item
        La série harmonique diverge à vitesse logarithmique, et la série des inverses des nombres premiers, c'est encore plus lent : théorème~\ref{ThonfVruT}.
\end{enumerate}

\InternalLinks{inversion locale, fonction implicite}
\begin{enumerate}
    \item
        Théorème d'inversion locale dans un Banach : théorème \ref{ThoXWpzqCn}.
    \item
        Fonction implicite dans un Banach : théorème~\ref{ThoAcaWho}.
    \item
        Utilisé pour montrer que le flot d'une équation différentielle est un \( C^p\)-difféomorphisme local, voir~\ref{NORMooWEWVooXbGmfE}. % position 1051229132
    \item
        Pour le théorème de Von Neumann~\ref{ThoOBriEoe}.
\end{enumerate}

\InternalLinks{convexité}

L'essentiel des résultats sur les fonctions convexes sont dans la section~\ref{SECooVZWWooUjxXYi}. On a surtout :
\begin{enumerate}
    \item
        Définition des fonctions convexes :~\ref{DefVQXRJQz} et~\ref{DEFooKCFPooLwKAsS} en dimension supérieure.
    \item
        En termes de différentielles,~\ref{PROPooYNNHooSHLvHp} pour la différentielle première et~\ref{CORooMBQMooWBAIIH} pour la hessienne.
    \item
        Une courbe paramétrée convexe est la définition~\ref{DEFooVQODooJSNYLw}.
    \item
        L'enveloppe convexe d'une courbe fermée simple et convexe :~\ref{PROPooWZITooTFiWsi}.
    \item
        Courbure et convexité d'une courbe paramétrée : section~\ref{SUBSECooNJOLooYuGRjA}.
    \item
        Une courbe paramétrée convexe est localement le graphe d'une fonction convexe par le lemme~\ref{LEMooGEVEooHxPTMO}.
    \item
        La convexité est utilisée dans la méthode du gradient à pas optimal de la proposition~\ref{PropSOOooGoMOxG}.
\end{enumerate}

En termes de parties convexes, on a : 
\begin{enumerate}
    \item
        Définition \ref{DEFooQQEOooAFKbcQ} d'une partie convexe d'un espace vectoriel.
    \item
        Une boule est convexe, proposition \ref{PROPooUQLUooDQfYLT}.
\end{enumerate}


\InternalLinks{fonction puissance}      \label{THEMEooBSBLooWcaQnR}

Il y a beaucoup de choses à dire\ldots

\begin{description}
    \item[Définition] 
        Nous considérons, pour \( a>0\), la fonction \( g_a\colon \eR\to \eR\) donnée par \( g_a(x)=a^x\). La définition de cette fonction se fait en de nombreuses étapes.
\begin{enumerate}
    \item
        \( a^n\) pour \( n\in \eN\) en la définition \ref{DEFooGVSFooFVLtNo}.
    \item 
        \( a^n\) pour \( n\in \eZ\) en la définition \ref{DEFooKEBIooZtPkac}.
    \item
        \( a^{1/n}\) pour \( n\in \eZ\) en la définition \ref{DEFooJWQLooWkOBxQ}.
    \item
        \( a^q\) pour \( q\in \eQ\) en la définition \ref{DEFooJWQLooWkOBxQ}.
    \item
        \( \sqrt[n]{ x }\) en la définition \ref{DEFooPOELooPouwtD}.
    \item
        La fonction \( g_a\) est Cauchy-continue sur \( \eQ\), c'est la proposition \ref{PROPooQRFSooVzYdJM}.
    \item
        \( a^x\) pour \( a>0\) et \( x\in \eR\) en la définition \ref{DEFooOJMKooJgcCtq}.
    \item
        \( a^z\) pour \( a>0\) et \( z\in \eC\) en la définition \ref{DEFooRBTDooNLcWGj}.
\end{enumerate}

\item[Quelques propriétés]
\begin{enumerate}
    \item
        Pour tout \( q\in \eQ\), il y a un \( \sqrt{ q }\) dans \( \eR\), proposition \ref{PROPooUHKFooVKmpte}.
    \item
        Pour \( a>0\) et \( x,y\in \eR\) nous avons $a^xa^y=a^{x+y}$, proposition \ref{PROPooVADRooLCLOzP}.
    \item
        Si \( a>0\) et \( x,y\in \eR\) nous avons \( (a^x)^y=(a^y)^x=a^{xy}\) par la proposition \ref{PROPooDWZKooNwXsdV}.
    \item
        La fonction puissance est strictement croissante (en ses deux arguments), proposition \ref{PROPooRXLNooWkPGsO}.
    \item
        La formule \( a^{-x}=1/a^x\) est la proposition \ref{PROPooVADRooLCLOzP}\ref{ITEMooSCJBooNVJZah}.
    \item
        La fonction puissance \( g_a(x)=a^x\) est continue, proposition \ref{PROPooVADRooLCLOzP}.
\end{enumerate}
\item[Croissance]
\begin{enumerate}
    \item
        La fonction puissance \( f_{\alpha}(x)=x^{\alpha}\) est strictement croissante, proposition \ref{PROPooRXLNooWkPGsO}
    \item
        La fonction puissance \( g_a(x)=a^x\) est strictement croissante, proposition \ref{PROPooVADRooLCLOzP}.
    \item
        \( \lim_{x\to \infty} x^{\alpha}=\infty\), proposition \ref{PROPooJRWCooGiXAYt}.
\end{enumerate}

\item[Dérivation]
    Comme toutes les choses sur la fonction puissance, les preuves sont assez différentes selon que l'on parle de \( a^x\) ou de \( x^a\).
\begin{enumerate}
    \item
        La fonction \( a^x\) est dérivable et sa dérivée vérifie \( g_a'(x)=g_a(x)g_a'(0)\), proposition \ref{PROPooMXCDooBffXbl}.
    \item
        La formule de dérivation pour \( x\mapsto x^q\) avec \( q\in \eQ\) est la proposition \ref{PROPooSGLGooIgzque}. 
    \item
        La dérivation de \( x\mapsto x^{\alpha}\) avec \( \alpha\in \eR\) est la proposition \ref{PROPooKIASooGngEDh}. Si elle est tellement loin, c'est parce qu'elle nécessite de permuter une limite de fonctions avec une dérivée.
    \item
        Pour la formule générale de dérivation de \( x\mapsto a^x\) demande de savoir les logarithmes (proposition \ref{PROPooKUULooKSEULJ}).
\end{enumerate}

\item[L'équation fonctionnelle]
    L'exponentielle et plus généralement la fonction puissance \( g_a(x)=a^x\) peut être introduite au moyen d'une équation fonctionnelle au lieu de l'équation différentielle usuelle. Cette fameuse équation fonctionnelle est
    \begin{equation}
        f(x+y)=f(x)f(y)
    \end{equation}
    en la définition \ref{DEFooPJKMooOfZzgy}.
\begin{enumerate}
    \item
        L'équivalence entre l'équation fonctionnelle et l'équation différentielle est donnée par la proposition \ref{PROPooGBUPooWtWaFI}.
    \item
        La fonction \( g_a(x)=a^x\) vérifie l'équation fonctionnelle \( g_a(x+y)=g_a(x)g_a(y)\) et les conséquences. C'est la définition \ref{DEFooPJKMooOfZzgy} et les choses qui suivent.
    \item
        L'équation fonctionnelle pour une fonction continue \( f\colon \eR\to S^1\) est traitée dans la proposition \ref{PROPooVJLYooOzfWCd}.
\end{enumerate}
\end{description}

Une définition alternative de la fonction puissance serait de poser directement
\begin{equation*}
    a^x=e^{x\ln(a)}.
\end{equation*}
De là les propriétés se déduisent facilement. Dans cette approche, les choses se mettent dans l'ordre suivant :
\begin{itemize}
    \item Définir \( \exp(x)\) par sa série pour tout \( x\).
    \item Démontrer que \( \exp(q)=\exp(1)^q\) pour tout rationnel \( q\) (première partie de la proposition \ref{PropCELWooLBSYmS}).
    \item Définir \( e=\exp(1)\).
    \item Définir, pour \( x\) irrationnel, \( a^x=\exp(x\ln(a))\).
    \item Prouver que \( e^x=\exp(x)\) pour tout \( x\).
\end{itemize}


\InternalLinks{dualité}     \label{THEMEooULGFooPscFJC}

Ne pas confondre dual algébrique et dual topologique d'un espace vectoriel.

\begin{enumerate}
    \item
        Définition de la base duale \ref{DEFooTMSEooZFtsqa}.
    \item
        Base préduale (existence, unicité) : proposition \ref{PROPooDBPGooPagbEB}.
    \item
        Théorème de représentation de Riez \ref{ThoLPQPooPWBXuv} : \( L^p=(L^q)'\), et en particulier \( L^{\infty}=(L^1)'\).
    \item
        Il n'est pas vrai que \( (L^{\infty})'=L^1\), voir la proposition \ref{PROPooXXRQooNSBZOi}.
\end{enumerate}

\begin{description}
    \item[Dual topologique et algébrique]
        Ils sont définis par~\ref{DefJPGSHpn}. Le dual algébrique est l'ensemble des formes linéaires, et le dual topologique ne considère que les formes linéaires continues (en dimension infinie, les applications linéaires ne sont pas toutes continues).
    \item[Topologie]
        Une topologie possible sur le dual d'un espace vectoriel topologique est celle \( *\)-faible de la définition~\ref{DefHUelCDD}.

        Nous comparons les topologies faibles et de la norme en la section~\ref{SECooKOJNooQVawFY}.
    \item[Théorèmes de dualité]
        Quelques théorèmes établissent des dualités entre des espaces courants.
\begin{enumerate}
    \item
        Le théorème de représentation de Riesz~\ref{ThoQgTovL} pour les espaces de Hilbert.
    \item
        La proposition~\ref{PropOAVooYZSodR} pour les espaces \( L^p\big( \mathopen[ 0 , 1 \mathclose] \big)\) avec \( 1<p<2\).
    \item
        Le théorème de représentation de Riesz~\ref{ThoLPQPooPWBXuv} pour les espaces \( L^p\) en général.
\end{enumerate}
Tous ces théorèmes donnent la dualité par l'application \( \Phi_x=\langle x, .\rangle \).

\end{description}


\InternalLinks{opérations sur les distributions}
\begin{enumerate}
    \item
        Convolution d'une distribution par une fonction, définition par l'équation \eqref{EQooOUXKooGHDSzL}.
    \item
        Dérivation d'une distribution, proposition-définition~\ref{PropKJLrfSX}.
    \item
        Produit d'une distribution par une fonction, définition~\ref{DefZVRNooDXAoTU}.
\end{enumerate}


\InternalLinks{convolution}
\begin{enumerate}
    \item
        Définition \ref{DEFooHHCMooHzfStu}, et principales propriétés sur \( L^1(\eR)\) dans le théorème \ref{THOooMLNMooQfksn}.
    \item
        Inégalité de normes : si \( f\in L^p\) et \( g\in L^1\), alors \( \| f*g \|_p\leq \| f \|_p\| g \|_1\), proposition~\ref{PROPooDMMCooPTuQuS}.
    \item
        \( \varphi\in L^1(\eR)\) et \( \psi\in\swS(\eR)\), alors \( \varphi * \psi\in \swS(\eR)\), proposition~\ref{PROPooUNFYooYdbSbJ}.
    \item
        Les suites régularisantes : \( \lim_{n\to \infty} \rho_n*f=f\) dans la proposition~\ref{PROPooYUVUooMiOktf}.
    \item
        Convolution d'une distribution par une fonction, définition par l'équation \eqref{EQooOUXKooGHDSzL}.
\end{enumerate}

\InternalLinks{séries de Fourier}       \label{THMooHWEBooTMInve}
\begin{itemize}
    \item Formule sommatoire de Poisson, proposition~\ref{ProprPbkoQ}.
    \item Inégalité isopérimétrique, théorème~\ref{ThoIXyctPo}.
    \item Fonction continue et périodique dont la série de Fourier ne converge pas, proposition~\ref{PropREkHdol}.

    \item
Nous allons montrer la convergence de \( \sum_{k\in \eZ}c_k(f) e^{inx}\) vers \( f(x)\) dans divers cas :
\begin{enumerate}
    \item
        Si \( f\) est continue et périodique, convergence au sens de Cesàro, théorème de Fejèr~\ref{ThoJFqczow}.
    \item
        Convergence au sens \( L^2\Big( \mathopen[ 0 , 2\pi \mathclose] \Big)\) dans le théorème~\ref{ThoYDKZLyv}.
    \item
        Si \( f\) est continue, périodique et si sa série de Fourier converge uniformément, théorème \ref{PropmrLfGt}.
    \item
        Si \( f\) est périodique et la série des coefficients converge absolument pour tout \( x\), proposition~\ref{PropSgvPab}.
    \item
        Si \( f\) est périodique et de classe \( C^1\), théorème~\ref{ThozHXraQ}.
\end{enumerate}
Il est cependant faux de croire que la continuité et la périodicité suffisent à obtenir une convergence, comme le montre la proposition~\ref{PropREkHdol}.
\end{itemize}


\InternalLinks{transformée de Fourier}
\begin{enumerate}
    \item
        Définition sur \( L^1\), définition~\ref{DEFooRIXGooECoIbx}.
    \item
        La transformée de Fourier d'une fonction \( L^1(\eR^d)\) est continue, proposition~\ref{PropJvNfj}.
    \item
    L'espace de Schwartz est stable par transformée de Fourier. L'application $\TF\colon \swS(\eR^d)\to \swS(\eR^d)$ est continue. Proposition ~\ref{PropKPsjyzT}
\item
    L'application \( \TF\colon \swS(\eR^d)\to \swS(\eR^d)\) est une bijection. Formule d'inversion, proposition \ref{PROPooLWTJooReGlaN}.
\end{enumerate}


\InternalLinks{méthode de Newton}
    \begin{enumerate}
        \item
            Nous parlons un petit peu de méthode de Newton en dimension \( 1\) dans~\ref{SECooIKXNooACLljs}.
        \item
            La méthode de Newton fonctionne bien avec les fonctions convexes par la proposition~\ref{PROPooVTSAooAtSLeI}.
        \item
            La méthode de Newton en dimension $n$ est le théorème~\ref{ThoHGpGwXk}.
       \item
            Un intervalle de convergence autour de \( \alpha\) s'obtient par majoration de \( | g' |\), proposition~\ref{PROPooRPHKooLnPCVJ}.
       \item
           Un intervalle de convergence quadratique s'obtient par majoration de \( | g'' |\), théorème~\ref{THOooDOVSooWsAFkx}.
       \item
           En calcul numérique, section~\ref{SECooIKXNooACLljs}.
       \item
           Méthode de Newton pour calculer \( \sqrt{ A }\), exemple \ref{EXooDLSVooMHPpcl}.
       \end{enumerate}


\InternalLinks{méthodes de calcul}
\begin{enumerate}
    \item
        Théorème de Rothstein-Trager~\ref{ThoXJFatfu}.
    \item
        Algorithme des facteurs invariants~\ref{PropPDfCqee}.
    \item
        Méthode de Newton, théorème~\ref{ThoHGpGwXk}
    \item
        Calcul d'intégrale par suite équirépartie~\ref{PropDMvPDc}.
\end{enumerate}

\InternalLinks{espaces vectoriels} 

\begin{enumerate}
    \item
        Existence d'une base. Pour un espace vectoriel quelconque, proposition \ref{PROPooHDCEooMhDjPi}.
    \item
        Théorème de la base incomplète. Pour un espace vectoriel quelconque, théorème \ref{THOooOQLQooHqEeDK}.
\end{enumerate}

\InternalLinks{valeurs propres, définie positive}        \label{THEMEooYEVLooWotqMY}
\begin{description}
    \item[À propos de valeurs propres]
        \begin{enumerate}
            \item
                Définition des valeurs propres d'une forme quadratique : définition \ref{DEFooGVGGooWQEIET}.
            \item
                Définition de valeur propre et vecteur propre pour un endomorphisme \( f\colon V\to V\), définition \ref{DefooMMKZooVcskCc}.
        \end{enumerate}
    \item[À propos de choses définies positives] 
\begin{enumerate}
    \item
        Une application bilinéaire est définie positive lorsque \( g(u,u)\geq 0\) et \( g(u,u)=0\) si et seulement si \( u=0\) est la définition~\ref{DEFooJIAQooZkBtTy}.
    \item
        Un opérateur ou une matrice est défini positif si toutes ses valeurs propres sont positives, c'est la définition~\ref{DefAWAooCMPuVM}.
    \item
        Pour une matrice symétrique, définie positive si et seulement si \( \langle Ax, x\rangle >0\) pour tout \( x\). C'est le lemme~\ref{LemWZFSooYvksjw}.
    \item
        Une application linéaire est définie positive si et seulement si sa matrice associée l'est. C'est la proposition~\ref{PROPooUAAFooEGVDRC}.
\end{enumerate}
Remarque : nous ne définissons pas la notion de matrice définie positive dans le cas d'une matrice non symétrique.
\end{description}

\InternalLinks{norme matricielle, norme opérateur et rayon spectral}     \label{THEMEooOJJFooWMSAtL}

Quelques définitions
\begin{enumerate}
    \item
        Définition de la norme opérateur : définition \ref{DefNFYUooBZCPTr}.
    \item
        Définition du rayon spectral~\ref{DEFooEAUKooSsjqaL}.
\end{enumerate}

    La norme matricielle n'est rien d'autre que la norme opérateur de l'application linéaire donnée par la matrice.

    \begin{enumerate}
        \item
            Lien entre norme matricielle et rayon spectral, le théorème~\ref{THOooNDQSooOUWQrK} assure que $\|A\|_2=\sqrt{\rho(A{^t}A)}$.
        \item
            Lien entre valeurs propres et norme opérateur : le lemme~\ref{LEMooNESTooVvUEOv} pour les matrices symétriques strictement définies positives donne \( \| A \|_2=\lambda_{max}\).
        \item
            Pour toute norme algébrique nous avons \( \rho(A)\leq \| A \|\), proposition~\ref{PROPooWZJBooTPLSZp}.
        \item
            Dans le cadre du conditionnement de matrice. Voir en particulier la proposition~\ref{PROPooNUAUooIbVgcN} qui utilise le théorème~\ref{THOooNDQSooOUWQrK}.
        \item
            Rayon spectral et convergence de méthode itérative, proposition~\ref{PROPooAQSWooSTXDCO}.
    \end{enumerate}

    Pour la norme opérateur nous avons les résultats suivants.

    \begin{enumerate}
        \item
            La majoration \( \| Au \|\leq \| A \|\| u \|\) est le lemme \ref{LEMooIBLEooLJczmu}.
        \item
            Définition d'une algèbre :~\ref{DefAEbnJqI} et pour une norme d'algèbre :~\ref{DefJWRWQue}.
        \item
            La norme opérateur est une norme d'algèbre, lemme \ref{LEMooFITMooBBBWGI}.
        \item
            Pour des espaces vectoriels normés, être borné est équivalent à être continu : proposition~\ref{PROPooQZYVooYJVlBd}.
        \item
            Le lemme à propos d'exponentielle de matrice~\ref{LemQEARooLRXEef} donne :
            \begin{equation*}
                \|  e^{tA} \|\leq P\big( | t | \big)\sum_{i=1}^r e^{t\real(\lambda_i)}.
            \end{equation*}
    \end{enumerate}

    La norme opérateur est utilisée pour donner une norme sur les produits tensoriels, définition \ref{DEFooEXXNooMgIpSV}.

    Une norme matricielle donne une topologie. Il y a donc également des liens entre rayon spectral et convergence de série. Dans cette optique, pour les séries de matrices, voir le thème~\ref{THEMEooPQKDooTAVKFH}.


\InternalLinks{série de matrices}       \label{THEMEooPQKDooTAVKFH}

\begin{enumerate}
    \item
        Rayon spectral et norme opérateur : thème~\ref{THEMEooOJJFooWMSAtL}.
    \item
        Exponentielle de matrices : thème~\ref{THEMEooKXSGooCsQNoY}.
    \item
        Série entière de matrices : section~\ref{secEVnZXgf}.
    \item
        Pour la série \( \sum_kA^k=(1-A)^{-1}\).
        \begin{itemize}
            \item Pour un espace de Banach : proposition~\ref{PropQAjqUNp}.
            \item Pour les matrices nilpotentes : proposition~\ref{PROPooWTFWooXHlmhp}.
            \item En lien avec le rayon spectral (si et seulement si \( \rho(A)<1\)) dans la proposition~\ref{THOooMNLGooKETwhh}.
            \item Le lemme~\ref{LemPQFDooGUPBvF} parle de la série entière \( \sum_{n\in \eN}z^{nk}=(1-z^k)^{-1}\).
        \end{itemize}
        Cette série est utilisée entre autres dans la proposition~\ref{PROPooZDMQooIZAbKK} pour prouver qu'une M-matrice irréductible vérifie \( A^{-1}>0\).
\end{enumerate}


\InternalLinks{rang}
    \begin{enumerate}
        \item Définition pour une application linéaire : \ref{DefALUAooSPcmyK}, pour une matrice : \ref{DEFooCSGXooFRzLRj}. L'équivalence est la proposition \ref{PROPooEGNBooIffJXc}.
        \item Le théorème du rang, théorème~\ref{ThoGkkffA}
        \item Pour une applicaton linéaire entre deux espaces vectoriels de même dimension finie, il est équivalent d'être injectif, surjectif ou bijectif, c'est le corolaire \ref{CORooCCXHooALmxKk}.
        \item Pour prouver que des matrices sont équivalentes et pour les mettre sous des formes canoniques, nous avons le lemme \ref{LemZMxxnfM} et son corolaire \ref{CorGOUYooErfOIe}.
        \item Tout hyperplan de \( \eM(n,\eK)\) coupe \( \GL(n,\eK)\), corolaire~\ref{CorGOUYooErfOIe}. Cela utilise la forme canonique sus-mentionnée.
        \item Le lien entre application duale et orthogonal de la proposition~\ref{PropWOPIooBHFDdP} utilise la notion de rang.
        \item Le lemme \ref{LEMooDFFDooJTQkRu} parle de commutant et utilise la notion de rang. Ce lemme sert à prouver diverses conditions équivalentes à être un endomorphisme cyclique dans le théorème \ref{THOooGLMSooYewNxW}.
        \end{enumerate}


\InternalLinks{extension de corps et polynômes} \label{THEMEooZYKFooQXhiPD}
    \begin{enumerate}
        \item
            Définition d'une extension de corps~\ref{DEFooFLJJooGJYDOe}.
        \item
            Définition de polynôme minimal: \ref{DefCVMooFGSAgL}.
        \item
            Pour l'extension du corps de base d'un espace vectoriel et les propriétés d'extension des applications linéaires, voir la section~\ref{SECooAUOWooNdYTZf}.
        \item
            Extension de corps de base et similitude d'application linéaire (ou de matrices, c'est la même chose), théorème~\ref{THOooHUFBooReKZWG}.
        \item
            Extension de corps de base et cyclicité des applications linéaires, corolaire~\ref{CORooAKQEooSliXPp}.
        \item
            À propos d'extensions de \( \eQ\), le lemme~\ref{LemSoXCQH}.
        \item
            Corps de rupture : définition~\ref{DEFooVALTooDJJmJv} existence par la proposition~\ref{PROPooUBIIooGZQyeE}. Il n'y a pas unicité.
        \item
            Corps de décomposition : définition~\ref{DEFooEKGZooSkvbum}. Attention : le plus souvent nous parlons de corps de décomposition d'un seul polynôme. Cette définition est un peu surfaite. Existence par la proposition~\ref{PROPooDPOYooFHcqkU} qui le donne même comme extension par toutes les racines, et unicité à isomorphisme près par le théorème~\ref{THOooQVKWooZAAYxK}, énoncé de façon plus simple (mais pas indépendante !) en la proposition~\ref{PropTMkfyM}.
        \item
            Si \( \eK\) est algébrique clos et si \( \alpha\colon \eK\to \eL\) est une extension algébrique, alors \( \alpha(\eK)=\eL\) par le lemme \ref{LEMooYVHKooWhewKp}.
    \end{enumerate}

Un trio de résultats d'enfer est :
\begin{enumerate}
    \item
        Dans un anneau principal qui n'est pas un corps, un idéal est maximal si et seulement si il est engendré par un irréductible (proposition~\ref{PropomqcGe}).
    \item
        Dans un anneau, un idéal \( I\) est maximal si et seulement si \( A/I\) est un corps (proposition~\ref{PROPooSHHWooCyZPPw})
    \item
        Si \( \eK\) est un corps, \( \eK[X]\) est principal (lemme~\ref{LEMooIDSKooQfkeKp}).
\end{enumerate}


\InternalLinks{décomposition de matrices}   \label{DECooWTAIooNkZAFg}
\begin{enumerate}
    \item
        Décomposition de Bruhat, théorème~\ref{ThoizlYJO}.
    \item
        Décomposition de Dunford, théorème~\ref{ThoRURcpW}.
    \item
        Décomposition polaire~\ref{ThoLHebUAU} des matrices symétriques et la proposition~\ref{PropWCXAooDuFMjn} pour la régularité.
\end{enumerate}


\InternalLinks{systèmes d'équations linéaires}
\begin{itemize}
    \item Algorithme des facteurs invariants~\ref{PropPDfCqee}.
    \item La méthode du gradient à pas optimal permet de résoudre par itérations \( Ax=b\) lorsque \( A\) est symétrique strictement définie positive. Il s'agit de minimiser une fonction bien choisie. Propositions~\ref{PROPooYRLDooTwzfWU} pour l'existence et~\ref{PropSOOooGoMOxG} pour la méthode.
\end{itemize}


\InternalLinks{formes bilinéaires et quadratiques}      \label{THEMEooOAJKooEvcCVn}
    \begin{enumerate}
\item
    Les formes bilinéaires sont définies en~\ref{DEFooEEQGooNiPjHz}.
\item
    Forme quadratique, définition~\ref{DefBSIoouvuKR}. Sa matrice, définition \ref{DEFooAOGPooXWXUcN}.
\item
    Équivalence de forme quadratiques, définition \ref{DEFooOLWYooMwhMJp}. Deux formes quadratiques sont équivalentes si et seulement si elles ont même signature, proposition \ref{PROPooBWXMooLsgyKm}.
\item
    Une isométrie d'une forme bilinéaire est affine ou linéaire, théorème \ref{ThoDsFErq}.
\item
    Forme bilinéaire dégénérée, définition \ref{DEFooNUBFooLfCqaK}.
\item
    Une forme bilinéaire est non-dégénérée si et seulement si sa matrice associée est inversible, c'est la proposition \ref{PROPooQHHPooSqpgcb}.
\item
    Une isométrie d'une forme bilinéaire est linéaire ou affine par le théorème \ref{ThoDsFErq}.
\item
    Toute forme quadratique admet des bases orthogonales, théorème \ref{THOooIDMPooIMwkqB} pour le cas général; proposition \ref{PROPooUKRUooGRIDHt} pour le cas de \( \eR^n\), en se basant sur le théorème spectral.
\item
    Théorème de Sylvester à propos de signature (définition \ref{DEFooWDCLooDkRYLK}) de forme quadratique réelle : \ref{ThoQFVsBCk}.
\item
    Base \( q\)-orthogonale pour une forme quadratique, théorème \ref{THOooIDMPooIMwkqB}.
    \item
        Le concept de projection orthogonale est la définition \ref{ThoWKwosrH} en dimension finie et la définition \ref{ThoProjOrthuzcYkz} dans le cas des espaces de Hilbert.
\end{enumerate}

\InternalLinks{arithmétique modulo, théorème de Bézout} \label{THEMEooNRZHooYuuHyt}
    \begin{enumerate}
        \item
            Pour \( \eZ^*\) c'est le théorème~\ref{ThoBuNjam}.
        \item
            Théorème de Bézout dans un anneau principal : corolaire~\ref{CorimHyXy}.
        \item
            Théorème de Bézout dans un anneau de polynômes : théorème~\ref{ThoBezoutOuGmLB}.
        \item
            En parlant des racines de l'unité et des générateurs du groupe unitaire dans le lemme~\ref{LemcFTNMa}. Au passage nous y parlerons de solfège.
        \item
            La proposition~\ref{PropLAbRSE} qui donne tout entier assez grand comme combinaisons de \( a \) et \( b\) à coefficients positifs est utilisée en chaines de Markov, voir la définition~\ref{DefCxvOaT} et ce qui suit.
        \item
            PGCD et PPCM sont dans la définition \ref{DefrYwbct}.
        \item
            Calcul effectif du PGCD puis des coefficients de Bézout : sous-sections~\ref{SUBSECooAEBLooFGJRkg} et~\ref{SUBSECooRHSQooEuBWbd}.
        \end{enumerate}


\InternalLinks{polynômes}

\begin{description}
    \item[Définitions]
        Soient un anneau \( A\), un corps \( \eK\), une extension \( \eL\) de \( \eK\) et un élément \( \alpha\in \eL\).
        \begin{enumerate}
            \item
                La définition la plus formelle est en tant que module produit \( A^{(\eN)}\), définition \ref{DEFooFYZRooMikwEL}. Le produit et l'évaluation sont définis en \ref{DEFooNXKUooLrGeuh} et la formule \( (PQ)(x)=P(x)Q(x)\) dans \ref{PROPooGDQCooHziCPH}.
            \item
                Si \( \eA\) est commutatif, alors \( \eA[X]\) est également commutatif, lemme \ref{LEMooWVUXooQlaepO}.
            \item
                En ce qui concerne la notation \( A[X]\), elle ne devrait pas être utilisée, voir \ref{SUBSECooLEKVooFBPSJz}. L'ensemble des polynômes sera noté \( \polyP(A)\) et ceux de degré \( n\) (définition \ref{DefDegrePoly}), \( \polyP_n(A)\).
            \item
                \( A[X]\), définition~\ref{DEFooFYZRooMikwEL}; l'anneau \( \eK[X]\) a même définition parce que c'est un cas particulier. L'évaluation d'un polynôme en un élément de l'anneau, \( P(\alpha)\) est définie en~\ref{DEFooNXKUooLrGeuh}.
            \item
                Liens entre \( \eK[\alpha]\), \( \eK[X]\), \( \eK(\alpha)\) et \( \eK(X)\) lorsque \( \alpha\) est transcendant, proposition~\ref{PROPooSYQWooFbfQtm}.

                Et la proposition~\ref{PropURZooVtwNXE} pour le cas où \( \alpha\) est algébrique\footnote{Définition \ref{LEMooLVPLooEkWYDN}.}.
            \item
                Si \( A\) est un anneau et si \( \alpha\) est un élément d'une extension de \( A\) (comme anneau), nous écrivons \( A[\alpha]\) pour le plus petit sous-anneau de \( B\) contenant \( A\) et \( \alpha\). C'est la définition~\ref{DEFooRFBFooKCXQsv}.
            \item
                \( \eK(X)\), le corps des fractions de \( \eK[X]\), définition~\ref{DEFooQPZIooQYiNVh}. Si \( R=P/Q\) dans \( \eK(X)\), l'évaluation est \( R(\alpha)=P(\alpha)Q(\alpha)^{-1}\), définition~\ref{DEFooZHBZooKlNfGZ}.
            \item
                \( \eK(\alpha)_{\eL}\) est le plus petit corps de \( \eL\) contenant \( \eK\) et \( \alpha\), définition~\ref{DEFooVSKGooMyeGel}.
            \item
                À propos de polynômes à plusieurs variables.
                \begin{itemize}
                    \item Anneau de polynômes : \( A[X_1,\ldots, X_n]\) est la définition~\ref{DEFooZNHOooCruuwI}. 
                    \item Corps engendré : \( \eK(\alpha_1,\ldots, \alpha_n)\) est la définition~\ref{DEFooRHRKooPqLNOp}. 
                    \item Corps des fractions rationnelles : \( \eK(X_1,\ldots, X_n)\) est la définition~\ref{DEFooOCPHooXneutp}.
                \end{itemize}
        \end{enumerate}

    \item[Coefficients dans un anneau commutatif]

        \begin{enumerate}
            \item
Les polynômes à coefficients dans un anneau commutatif  sont à la section~\ref{SECooVMABooVdhbPo}.
        \end{enumerate}


    \item[Coefficients dans un corps]
Les polynômes à coefficients dans un corps sont à la section~\ref{SECooFYOGooQHitgE}.
        \begin{enumerate}

\item
Nous parlons de l'idéal des polynômes annulateurs dans le théorème~\ref{ThoCCHkoU}.
            \item
                Le théorème~\ref{ThoCCHkoU} dit que \( \eK[X]\) est un anneau principal et que tous ses idéaux sont engendrés par un unique polynôme unitaire.
            \item
                Le polynôme minimal est irréductible, proposition~\ref{PropRARooKavaIT}.
            \item
                Quelques formules sur le \( \pgcd\), lemme~\ref{LemUELTuwK}.
        \end{enumerate}
    \item[Polynôme primitif]

        \begin{enumerate}
            \item
                Un polynôme est irréductible sur \( A\) si et seulement si il est irréductible et primitif sur le corps des fractions, corolaire~\ref{CORooZCSOooHQVAOV}.
        \end{enumerate}

    \item[Polynôme d'endomorphisme]
        C'est la section~\ref{SECooUEQVooLBrRiE}.

    \item[Racines et factorisation]

    \begin{enumerate}
        \item
            Si \( \eA\) est un anneau, la proposition~\ref{PropHSQooASRbeA} factorise une racine.
        \item
            Si \( \eA\) est un anneau, la proposition~\ref{PropahQQpA} factorise une racine avec sa multiplicité.
        \item
            Si \( \eA\) est un anneau, le théorème~\ref{ThoSVZooMpNANi} factorise plusieurs racines avec leurs multiplicités.
        \item
            Le théorème~\ref{ThoSVZooMpNANi} nous indique que lorsqu'on a autant de racines (multiplicité comprise) que le degré, alors nous avons toutes les racines.
        \item
            Si \( \eK\) est un corps et \( \alpha\) une racine dans une extension, le polynôme minimal de \( \alpha\) divise tout polynôme annulateur par la proposition~\ref{PropXULooPCusvE}.
        \item
            Le théorème~\ref{ThoLXTooNaUAKR} annule un polynôme de degré \( n\) ayant \( n+1\) racines distinctes.
        \item
            La proposition~\ref{PropTETooGuBYQf} nous annule un polynôme à plusieurs variables lorsqu'il a trop de racines.
        \item
            En analyse complexe, le principe des zéros isolés~\ref{ThoukDPBX} annule en gros toute série entière possédant un zéro non isolé.
        \item
            Polynômes irréductibles sur \( \eF_q\).
        \end{enumerate}

\end{description}

\InternalLinks{zoologie de l'algèbre}      \label{THEMEooVIQIooOcFAQS}

Nous listons ici un peu tous les termes qui arrivent en algèbre.

\begin{enumerate}
    \item[Structures]
        Les structures.
        \begin{enumerate}
            \item
                Algèbre, définition \ref{DefAEbnJqI}.
        \end{enumerate}
    \item[Éléments]
        Pour les éléments, nous avons :
        \begin{enumerate}
            \item
                Élément irréductible en \ref{DeirredBDhQfA}.
            \item
                Élément premier en \ref{DEFooZCRQooWXRalw}.
        \end{enumerate}
    \item[Anneaux]
        Pour les anneaux, nous avons :
        \begin{enumerate}
            \item
                Anneau factoriel en \ref{DEFooVCATooPJGWKq}.
            \item
                Anneau principal en \ref{DEFooGWOZooXzUlhK}.
            \item
                Anneau intègre en \ref{DEFooTAOPooWDPYmd}.
            \item
                Anneau noetherien en \ref{DEFooPWMHooCnrQuJ}.
        \end{enumerate}
    \item[Idéaux]
        Pour les idéaux, nous avons :
        \begin{enumerate}
            \item
                Idéal principal en \ref{DEFooMZRKooBPLAWH}.
            \item
                Idéal premier en \ref{DEFooAQSZooVhvQWv}.
        \end{enumerate}
\end{enumerate}



\InternalLinks{invariants de similitude}
    \begin{enumerate}
        \item
            Théorème~\ref{THOooDOWUooOzxzxm}.
        \item
            Pour prouver que la similitude d'applications linéaires résiste à l'extension du corps de base, théorème~\ref{THOooHUFBooReKZWG}.
        \item
            Pour prouver que la dimension du commutant d'un endomorphisme de \( E\) est de dimension au moins \( \dim(E)\), lemme~\ref{LEMooDFFDooJTQkRu}.
        \item
            Nous verrons dans la remarque~\ref{REMooPVLEooYDRXQI} à propos des invariants de similitude que toute matrice est semblable à la matrice bloc-diagonale constituées des matrices compagnon (définition~\ref{DEFooOSVAooGevsda}) de la suite des polynômes minimaux.
        \end{enumerate}


\InternalLinks{réduction, diagonalisation}
    Des résultats qui parlent diagonalisation
    \begin{enumerate}
        \item
            Définition d'un endomorphisme diagonalisable :~\ref{DefCNJqsmo}.
        \item
            Conditions équivalentes au fait d'être diagonalisable en termes de polynôme minimal, y compris la décomposition en espaces propres : théorème~\ref{ThoDigLEQEXR}.
        \item
            Diagonalisation simultanée~\ref{PropGqhAMei}, pseudo-diagonalisation simultanée~\ref{CorNHKnLVA}.
        \item
            Diagonalisation d'exponentielle~\ref{PropCOMNooIErskN} utilisant la décomposition de Dunford.
        \item
            Décomposition polaire théorème~\ref{ThoLHebUAU}. \( M=SQ\), \( S\) est symétrique, réelle, définie positive, \( Q\) est orthogonale.
        \item
            Décomposition de Dunford~\ref{ThoRURcpW}. \( u=s+n\) où \( s\) est diagonalisable et \( n\) est nilpotent, \( [s,n]=0\).
        \item
            Réduction de Jordan (bloc-diagonale)~\ref{ThoGGMYooPzMVpe}.
        \item
            L'algorithme des facteurs invariants~\ref{PropPDfCqee} donne \( U=PDQ\) avec \( P\) et \( Q\) inversibles, \( D\) diagonale, sans hypothèse sur \( U\). De plus les éléments de \( D\) forment une chaine d'éléments qui se divisent l'un l'autre.
        \end{enumerate}
        Le théorème spectral et ses variantes :
        \begin{enumerate}
            \item
                Théorème spectral, matrice symétrique, théorème~\ref{ThoeTMXla}. Via le lemme de Schur complexe \ref{LemSchurComplHAftTq}.
            \item
                Théorème spectral autoadjoint (c'est le même, mais vu sans matrices), théorème~\ref{ThoRSBahHH}
            \item
                Théorème spectral hermitien, lemme~\ref{LEMooVCEOooIXnTpp}.
            \item
                Théorème spectral, matrice normales, théorème~\ref{ThogammwA}.
            \end{enumerate}
        Pour les résultats de décomposition dont une partie est diagonale, voir le thème~\ref{DECooWTAIooNkZAFg} sur les décompositions.
        Réduction de quadriques : 
        \begin{enumerate}
            \item
                Réduction de Gauss, théorème \ref{THOooOMMFooKxqICS}.
        \end{enumerate}


\InternalLinks{endomorphismes cycliques}
    \begin{enumerate}
        \item
            Définition~\ref{DEFooFEIFooNSGhQE}.
        \item
            Son lien avec le commutant donné dans la proposition~\ref{PropooQALUooTluDif} et le théorème~\ref{THOooGLMSooYewNxW}.
        \item
            Utilisation dans le théorème de Frobenius (invariants de similitude), théorème~\ref{THOooDOWUooOzxzxm}.
        \end{enumerate}


\InternalLinks{déterminant}     \label{THMooUXJMooOroxbI}
    \begin{enumerate}
        \item
            Déterminant d'une matrice : définition \ref{DEFooYCKRooTrajdP}.
    \item
        Déterminant d'un endomorphisme~\ref{DefCOZEooGhRfxA}. 
    \item
        Le lemme \ref{LEMooEZFIooXyYybe} donne la formule $\det(f)=\sum_{\sigma\in S_n}\epsilon(\sigma)\prod_{i=1}^n\langle e_{\sigma(i)}, f(e_i)\rangle$.
    \item
        Principales propriétés algébriques du déterminant : la proposition \ref{PropYQNMooZjlYlA}.
    \item
        La formule \( \det(AB)=\det(A)\det(B)\) est la proposition \ref{PROPooHQNPooIfPEDH} pour des matrices et la proposition \ref{PropYQNMooZjlYlA}\ref{ItemUPLNooYZMRJy} pour les applications linéaires.
        \item
            Déterminant et manipulations de lignes et colonnes, section \ref{SUBSECooKMSVooBBHwkH} et les propositions qui précèdent à partir du lemme \ref{LEMooCEQYooYAbctZ} qui dit que \( \det(A)=\det(A^t)\).
    \item
        Les \( n\)-formes alternées forment un espace de dimension \( 1\), proposition~\ref{ProprbjihK}.
    \item
        Déterminant d'une famille de vecteurs~\ref{DEFooODDFooSNahPb}.
    \item
        Calcul d'un déterminant de taille \( 2\times 2\) : équation \eqref{EQooQRGVooChwRMd}.
    \item
        Interprétations géométriques
            \begin{enumerate}
        \item
            À propos d'orthogonalité, le déterminant est très lié au produit vectoriel en dimension~\( 3\). Et il le généralise en dimension supérieure.
            \begin{enumerate}
                \item
            Liaison au produit vectoriel (orthogonalité) dans la proposition~\ref{PROPooTUVKooOQXKKl}.
        \item
            En particulier le lemme~\ref{LEMooFRWKooVloCSM} nous dit comment un déterminant donne un vecteur orthogonal à une famille donnée de vecteurs.
            \end{enumerate}
        \item
            Déterminant et aires, volumes
            \begin{enumerate}
                \item
            Déterminant et mesure de Lebesgue : théorème~\ref{ThoBVIJooMkifod}.
                \item
                    Aire du parallélogramme : il y a la formule avec le produit vectoriel dans la proposition~\ref{PropNormeProdVectoabsint}, mais l'aire proprement dite, avec une intégrale est dans la proposition \ref{PROPooAVVNooOOlSzr}.
        \item
            Volume du parallélépipède avec le produit mixte et le déterminant \( 3\times 3\),~\ref{NORMooWWOKooWzScnZ}.
            \end{enumerate}
        \end{enumerate}
        Tant que nous en sommes dans les interprétations géométriques, il faut lier déterminant, produit vectoriel, orthogonalité et mesure en notant que l'élément de volume lors de l'intégration en dimension \( 3\) est donné par \eqref{EQooNYWSooZuvcPe} : \( dS=\| T_u\times T_v \|\) qui est la norme du produit vectoriel des vecteurs tangents au paramétrage.

        Nous voyons dans l'équation \eqref{EQooARMAooQPhQAL} que l'élément de volume pour une partie de dimension \( n\) dans \( \eR^m\) (à l'occasion d'y intégrer une fonction) est donné par un déterminant mettant en jeu les vecteurs tangents du paramétrage.
        \item
            Le déterminant de Vandermonde est à la proposition~\ref{PropnuUvtj}. Il est utilisé à divers endroits :
\begin{enumerate}
    \item
        Pour prouver que \( u\) est nilpotente si et seulement si \( \tr(u^p)=0\) pour tout \( p\) (lemme \ref{LemzgNOjY})
    \item
        Pour prouver qu'un endomorphisme possédant \( \dim(E)\) valeurs propres distinctes est cyclique (proposition~\ref{PropooQALUooTluDif}).
\end{enumerate}

   \end{enumerate}


\InternalLinks{polynôme d'endomorphismes}
    \begin{enumerate}
    \item Endomorphismes cycliques et commutant dans le cas diagonalisable, proposition~\ref{PropooQALUooTluDif}.
    \item Racine carrée d'une matrice hermitienne positive, proposition~\ref{PropVZvCWn}.
    \item Théorème de Burnside sur les sous-groupes d'exposant fini de \( \GL(n,\eC)\), théorème~\ref{ThooJLTit}.
    \item Décomposition de Dunford, théorème~\ref{ThoRURcpW}.
    \item Algorithme des facteurs invariants~\ref{PropPDfCqee}.
    \end{enumerate}

\InternalLinks{exponentielle}        \label{THEMEooKXSGooCsQNoY}

Toutes les exponentielles sont définies par la série
\begin{equation*}
    \exp(x)=\sum_{k=0}^{\infty}\frac{ x^k }{ k! },
\end{equation*}
tant que la somme a un sens.

\begin{description}
    \item[Réels]

Voici le plan que nous suivons dans le Frido :
\begin{itemize}
    \item L'exponentielle est définie par sa série en~\ref{DEFooSFDUooMNsgZY}.
    \item Nous démontrons qu'elle vérifie l'équation différentielle \( y'=y\), \( y(0)=1\) (théorème \ref{ThoKRYAooAcnTut}).
    \item Nous démontrons l'unicité de la solution à cette équation différentielle.
    \item Nous démontrons qu'elle est égale à \( x\mapsto y(1)^x\). Cela donne la définition du nombre \( e\) comme valant \( y(1)\).
    \item Nous définissons le logarithme comme l'application réciproque de l'exponentielle (définition~\ref{DEFooELGOooGiZQjt}).
    \item Les fonctions trigonométriques (sinus et cosinus) sont définies par leurs séries. Il est alors montré que \( e^{ix}=\cos(x)+i\sin(x)\).
\end{itemize}

    \item[Propriétés]
        \begin{itemize}
            \item 
        La formule \( a^{-x}=1/a^x\) est la proposition \ref{PROPooVADRooLCLOzP}\ref{ITEMooSCJBooNVJZah}.
        \end{itemize}

    \item[Complexes]

        \begin{enumerate}
            \item
                La définition de \( \exp(a+ib)\) est la définition \ref{DEFooSFDUooMNsgZY}.
            \item
                Les principales propriétés, dont \(  e^{z+w}= e^{z} e^{w}\), sont dans la proposition \ref{PropdDjisy}.
            \item
                Nous avons \(  e^{ix}= e^{iy}\) si et seulement si il existe \( k\in\eZ\) tel que \( y=x+2k\pi\), corolaire \ref{CORooTFMAooHDRrqi}.
            \item
                Le fait que \(  e^{i\theta}\) donne tous les nombres complexes de norme \( 1\) est la proposition~\ref{PROPooZEFEooEKMOPT}.
            \item
                Le groupe des racines de l'unité est donné par l'équation \eqref{EqIEAXooIpvFPe}.
        \end{enumerate}

    \item[Algèbre normée commutative]

        Pour la définition c'est la proposition~\ref{DEFooSFDUooMNsgZY} et pour la régularité \(  C^{\infty}\) c'est la proposition~\ref{PROPooTBDAooQouzSk}.

    \item[Idem non commutatif]

        Il y a une tentative de théorème~\ref{THOooFGTQooZPiVLO}, mais c'est principalement pour les matrices qu'il y a des résultats.

    \item[Matrices]

        De nombreux résultats sont disponibles pour les exponentielles de matrices.

\begin{enumerate}
    \item
        \( e^{sA} e^{tA}= e^{(s+t)A}\), proposition \ref{PROPooKDKDooCUpGzE}.
    \item
        Si \( A\) est une matrice, alors \( (e^tA)'(u)=Ae^{uA}\), proposition \ref{PROPooSDNNooQtHkhA}.
    \item
        Les sections \ref{secAOnIwQM} et \ref{SECooBYQBooZifJsg} parlent d'exponentielle de matrices.
    \item
        L'exponentielle donne lieu à une fonction de classe \(  C^{\infty}\), proposition~\ref{PropXFfOiOb}.
    \item
            Le lemme à propos d'exponentielle de matrice~\ref{LemQEARooLRXEef} donne :
            \begin{equation*}
                \|  e^{tA} \|\leq P\big( | t | \big)\sum_{i=1}^r e^{t\real(\lambda_i)}.
            \end{equation*}
        \item
            La proposition~\ref{PropCOMNooIErskN} : si \( A\in \eM(n,\eR)\) a un polynôme caractéristique scindé, alors \( A\) est diagonalisable si et seulement si \( e^A\) est diagonalisable.
\item
    La section~\ref{subsecXNcaQfZ} parle des fonctions exponentielle et logarithme pour les matrices. Entre autres la dérivation et les séries.
\item
    Pour résoudre des équations différentielles linéaires : sous-section~\ref{SUBSECooMDKIooKaaKlZ}.
\item
    La proposition~\ref{PropKKdmnkD} dit que l'exponentielle est surjective sur \( \GL(n,\eC)\).
\item

La proposition~\ref{PropFMqsIE} : si \( u\) est un endomorphisme, alors \( \exp(u)\) est un polynôme en \( u\).
\item
    Calcul effectif : sous-section~\ref{SUBSECooGAHVooBRUFub}.
\item Proposition~\ref{PROPooZUHOooQBwfZq} : si \( A\in\eM(n,\eC)\) alors $ e^{\tr(A)}=\det( e^{A}).$
    \item
        Les séries entières de matrices sont traitées autour de la proposition~\ref{PropFIPooSSmJDQ}.
\end{enumerate}


\end{description}

\InternalLinks{types d'anneaux}

\begin{enumerate}
    \item
        Définition d'anneau : définition \ref{DefHXJUooKoovob}.
    \item
        La définition d'anneau principal est \ref{DEFooGWOZooXzUlhK}; pour un idéal principal, c'est \ref{DEFooMZRKooBPLAWH}.
    \item
        \( \eZ\) est intègre, exemple~\ref{EXooLDXRooSxUAXs}, principal et euclidien (proposition~\ref{PROPooPJGLooQSrJTU}).
    \item
        \( \eZ[X]\) n'est pas principal (voir~\ref{ITEMooNQQMooSnuKvW}).
    \item   \label{ITEMooNQQMooSnuKvW}
        Si \( A\) est un anneau intègre\footnote{Définition \ref{DEFooTAOPooWDPYmd}.} qui n'est pas un corps, alors \( A[X]\) n'est pas principal, lemme~\ref{LEMooDJSUooJWyxCL}.
    \item
        L'anneau des fonctions holomorphes sur un compact donné est principal, proposition~\ref{PROPooVWRPooGQMenV}.
    \item
        L'anneau \( \eZ[i\sqrt{ 3 }]\) n'est pas factoriel, exemple~\ref{EXooCWJUooCDJqkr}.
    \item
        L'anneau \( \eZ[i\sqrt{ 5 }]\) n'est ni factoriel ni principal, exemple~\ref{EXooYCTDooGXAjGC}.
    \item
        Tous les idéaux de \( \eZ/6\eZ\) sont principaux, mais \( \eZ/6\eZ\) n'est pas principal. Exemple~\ref{EXooCJRPooYkWdyr}.
\end{enumerate}


\InternalLinks{sous-groupes}
\begin{enumerate}
    \item
        Théorème de Burnside sur les sous-groupes d'exposant fini de \( \GL(n,\eC)\), théorème~\ref{ThooJLTit}.
    \item
        Sous-groupes compacts de \( \GL(n,\eR)\), lemme~\ref{LemOCtdiaE} ou proposition~\ref{PropQZkeHeG}.
\end{enumerate}

\InternalLinks{groupe symétrique}       \label{THEMEooQEEWooXDhvhv}

\begin{enumerate}
    \item
        Définition~\ref{DEFooJNPIooMuzIXd}.
    \item
        La signature \( \epsilon\colon S_n\to \{ -1,1 \}\) est l'unique homomorphisme surjectif de \( S_n\) sur \( \{ -1,1 \}\), proposition \ref{ProphIuJrC}\ref{ITEMooBQKUooFTkvSu}.
    \item
        La table des caractères du groupe symétrique \( S_4\) est donné dans la section~\ref{SecUMIgTmO}.
    \item
        Le groupe symétrique \( S_4\) est le groupe des symétries affines du tétraèdre régulier\footnote{Définition \ref{DEFooMUUMooFVxKyb}.}, proposition~\ref{PROPooVNLKooOjQzCj}.
    \item
        Le groupe alterné \( A_5\) est l'unique groupe simple d'ordre \( 60\), proposition~\ref{PROPooUBIWooTrfCat}.
    \item
        La proposition~\ref{PROPooCPXOooVxPAij} donne la position du groupe alterné dans le groupe symétrique : \( A_n\) est un sous-groupe caractéristique de \( S_n\) et l'unique sous-groupe d'indice \( 2\).
\end{enumerate}

\InternalLinks{action de groupe}    \label{THEMEooKZHBooRCULcr}
    \begin{enumerate}
        \item Définition d'une action de groupe sur un ensemble :~\ref{DefActionGroupe}.
    \item Action du groupe modulaire sur le demi-plan de Poincaré, théorème~\ref{ThoItqXCm}.
    \item
        La formule de Burnside (théorème~\ref{THOooEFDMooDfosOw}) parle du nombre d'orbites pour l'action d'un groupe fini sur un ensemble fini.
    \item Des applications de la formule de Burnside : le jeu de la roulette et l'affaire du collier,~\ref{pTqJLY} et~\ref{siOQlG}.
    \item
        
        Le groupe symétrique  \( S_n\) agit sur l'anneau  \( \eK[T_1,\ldots, T_n]\), lemme \ref{LEMooIRVQooHvoNBq}.

    \end{enumerate}


\InternalLinks{classification de groupes}
\begin{enumerate}
    \item Structure des groupes d'ordre \( pq\), théorème~\ref{ThoLnTMBy}.
    \item Le groupe alterné est simple, théorème~\ref{ThoURfSUXP}.
    \item Définition~\ref{DEFooPRCHooVZdwST} d'un \( p\)-groupe.
    \item Théorème de Sylow~\ref{ThoUkPDXf}.
    \item Théorème de Burnside sur les sous-groupes d'exposant fini de \( \GL(n,\eC)\), théorème~\ref{ThooJLTit}.
    \item \( (\eZ/p\eZ)^*\simeq \eZ/(p-1)\eZ\), corolaire~\ref{CorpRUndR}.
\end{enumerate}



\InternalLinks{produit semi-direct de groupes}
    \begin{enumerate}
        \item
            Définition~\ref{DEFooKWEHooISNQzi}.         % Ce commentaire sert à rendre cette ligne unique ooQLCCooQfibcp
        \item
            Le corolaire~\ref{CoroGohOZ} donne un critère pour prouver qu'un produit \( NH\) est un produit semi-direct.
        \item
            L'exemple~\ref{EXooHNYYooUDsKnm} donne le groupe des isométries du carré comme un produit semi-direct.
        \item
            Le théorème~\ref{ThoLnTMBy} classifie les groupes d'ordre \( pq\) (\( p\), \( q\) premiers distincts) à grands coups de produit semi-directs.
        \item
            Le théorème~\ref{THOooQJSRooMrqQct} donne les isométries de \( \eR^n\) par \( \Isom(\eR^n)=T(n)\times_{\rho} O(n)\) où \( T(n)\) est le groupe des translations.
        \item
            La proposition~\ref{PROPooDHYWooXxEXvl} donne une décomposition du groupe orthogonal \( \gO(n)=\SO(n)\times_{\rho} C_2\) où \( C_2=\{ \id,R \}\) où \( R\) est de déterminant \( -1\).
        \item
            La proposition~\ref{PROPooTPFZooKtFxhg} donne \( \Aff(\eR^n)=T(n)\times_{\rho}\GL(n,\eR)\) où \( \Aff(\eR^n)\) est le groupe des applications affines bijectives de \( \eR^n\).
        \end{enumerate}


\InternalLinks{théorie des représentations}
\begin{enumerate}
    \item Définition \ref{DEFooXVMSooXDIfZV}.
    \item Table des caractères du groupe diédral, section~\ref{SecWMzheKf}.
    \item Table des caractères du groupe symétrique \( S_4\), section~\ref{SecUMIgTmO}.
\end{enumerate}


\InternalLinks{isométries}      \label{THMooVUCLooCrdbxm}

Il y a \( (\eR^n,\| . \|)\) et \( \eR^n,d\).

Les isométries de \( \| . \|\) sont linéaires, tandis que les isométries de la distance contiennent aussi les translations et les rotations de centre différent de l'origine.

Ne pas confondre une isométrie d'un espace affine avec une isométrie d'un espace euclidien\footnote{Définition \ref{DefLZMcvfj}.}. Les isométries d'un espace euclidien préservent le produit scalaire et fixent donc l'origine (lemme~\ref{LEMooYXJZooWKRFRu}). Les isométries des espaces affines par contre conservent les distances (définition~\ref{DEFooZGKBooGgjkgs}) et peuvent donc déplacer l'origine de l'espace vectoriel sur lequel il est modelé; typiquement les translations sont des isométries de l'espace affine mais pas de l'espace euclidien.

Parfois, lorsqu'on coupe les cheveux en quatre, il faut faire attention en parlant de \( \eR^n\) : soit on en parle comme d'un espace métrique (muni de la distance), soit on en parle comme d'un espace normé (muni de la norme ou du produit scalaire).

\begin{description}
    \item[Général] 
        Quelques résultats généraux et en vrac à propos d'isométries.
\begin{enumerate}
    \item
        Définition d'une isométrie pour une forme bilinéaire,~\ref{DEFooIQURooMeQuqX}. Pour une forme quadratique : définition~\ref{DEFooECTUooRxBhHf}.
    \item
        Définition du groupe orthogonal~\ref{DEFooUHANooLVBVID}, et le spécial orthogonal \( \SO(n)\) en la définition~\ref{DEFooJLNQooBKTYUy}. Le groupe \( \SO(2)\) est le groupe des rotations, par corolaire~\ref{CORooVYUJooDbkIFY}.
    \item
        La rotation \( R_A(\theta)\) d'un angle \( \theta\) autour du point \( A\in \eR^2\) est donnée par la définition \ref{DEFooADTDooKIZbrw}.
    \item
        La proposition \ref{PROPooOTIVooZpvLnb} donne à toute rotation \( R_0(\theta)\) une matrice de la forme connue. C'est autour de cela que nous définissons les angles, définition \ref{DEFooUPUUooKAPFrh}.
    \item
        Le groupe orthogonal est le groupe des isométries de \( \eR^n\), proposition~\ref{PropKBCXooOuEZcS}.
    \item
        Les isométries de l'espace euclidien sont affines,~\ref{ThoDsFErq}.
    \item
        Les isométries de l'espace euclidien comme produit semi-direct : $\Isom(\eR^n)\simeq T(n)\times_{\rho}\gO(n)$, théorème~\ref{THOooQJSRooMrqQct}.
    \item
        Isométries du cube, section~\ref{SecPVCmkxM}.
    \item
        Nous parlons des isométries affines du tétraèdre régulier dans la proposition~\ref{PROPooVNLKooOjQzCj}.
\end{enumerate}

    \item[Groupe diédral]
        Le groupe diédral est un peu central dans la théorie des isométries de \( (\eR^2,d)\) parce que beaucoup de sous-groupes finis des isométries de \( (\eR^2,d)\) sont en fait isomorphes au groupe diédral.
        \begin{enumerate}
    \item
        Générateurs du groupe diédral, proposition~\ref{PropLDIPoZ}.
    \item
        Un sous-groupe fini des isométries de \( (\eR^2,d)\) contenant au moins une réflexion est isomorphe au groupe diédral par le théorème \ref{THOooKDMUooUxQqbB}.
    \item
        Le théorème \ref{THOooAYZVooPmCiWI} dit que le groupe des isométries propres d'une partie quelconque de \( (\eR^2,d)\) est soit cyclique soit isomorphe au groupe diédral.
        \end{enumerate}
        
    \item[Isométries et réflexions]
        Dans un espace euclidien, toute isométrie peut être décomposée en réflexions autour d'hyperplans. Voici quelques énoncés à ce propos.
\begin{enumerate}
    \item
        Définition d'une réflexion dans \( \eR^2\) \ref{LEMooIJELooLWqBfE}.
    \item
        La caractérisation en termes de projection orthogonale est le lemme \ref{LEMooZSDRooUkNYer}; en termes de médiatrice c'est le lemme \ref{LEMooTCIEooXdyuHu}.
    \item
        Définition d'un hyperplan \ref{DEFooEWDTooQbUQws}.
    \item
        En dimension \( 2\), une rotation est définie comme composée de deux réflexions en la définition \ref{DEFooFUBYooHGXphm}.
    \item
        En dimension \( 2\), les réflexions ont un déterminant \( -1\) par le lemme \ref{LEMooSYZYooWDFScw}.
    \item
        Les isométries du plan \( (\eR^2,d)\) sont données dans le théorème \ref{THOooVRNOooAgaVRN}, et sont au plus 3 réflexions par le théorème \ref{THOooRORQooTDWFdv}.
    \item
        Décomposition des isométries de \( \eR^n\) en réflexions par le lemme \ref{LEMooJCDRooGAmlwp}.
    \item
        En particulier, les éléments de \( \SO(3)\) sont des compositions de deux réflexions par le corolaire \ref{CORooJCURooSRzSFb}.
    \item
        Une isométrie de \( \eR^n\) préserve l'orientation si et seulement si elle est la composition d'un nombre pair de réflexions. C'est le théorème \ref{THOooWBIYooCtWoSq}.
\end{enumerate}
\item[Sous-groupe fini]
    \begin{enumerate}
        \item
            Les sous-groupes finis des isométries de \( (\eR^2,d)\) sont cycliques, théorème \ref{THOooKDMUooUxQqbB}.
        \item
            Les sous-groupes finis de \( \SO(3)\) sont listés dans \ref{PROPooBHPNooHPlgwH}.
        \item
            Les sous-groupes finis de \( \SO(2)\) sont cycliques, lemme \ref{LEMooUKEVooAEWvlM}.
    \end{enumerate}
\end{description}


\InternalLinks{caractérisation de distributions en probabilités}
\begin{enumerate}
    \item
        La probabilité conjointe est la définition~\ref{DefFonrepConj}.
    \item
        La fonction de répartition est la définition~\ref{DefooYAZVooNdxDCx}.
    \item
        La fonction caractéristique est la définition~\ref{DefooEIVXooNtHLQQ}.
\end{enumerate}


\InternalLinks{théorème central limite}
\begin{enumerate}
    \item
        Pour les processus de Poisson, théorème~\ref{ThoCSuLLo}.
\end{enumerate}

\InternalLinks{lemme de transfert}      \label{THEMEooJREIooKEdMOl}

Il y a deux résultats qui portent ce nom. Le premier est dans la théorie de Fourier, le résultat \( \hat{f'}=i\xi \hat{f}\).
\begin{enumerate}
    \item
        Lemme~\ref{LemQPVQjCx} sur \( \swS(\eR^d)\)
    \item
        Lemme~\ref{LEMooAGBZooWCbPDd} pour \( L^2\).
\end{enumerate}

L'autre lemme de transfert est en théorie des tribus, le résultat $\sigma\big( f^{-1}(\tribC) \big)=f^{-1}\big( \sigma(\tribC) \big)$ du lemme \ref{LemOQTBooWGYuDU}. Celui-ci est d'ailleurs plutôt nommé «lemme de transport».


\InternalLinks{probabilités et espérances conditionnelles}

    Les deux définitions de base, sur lesquelles se basent toutes les choses conditionnelles sont :
    \begin{itemize}
        \item L'espérance conditionnelle d'une variable aléatoire sachant une tribu : \( E(X|\tribF)\) de la définition~\ref{ThoMWfDPQ}.
    \end{itemize}

    Les autres sont listées ci-dessous.
\begin{description}

    \item[Probabilité conditionnelle]. 

        Plusieurs probabilités conditionnelles.
        \begin{itemize}
        \item D'un événement en sachant un autre : la définition \ref{DEFooGJVHooVbhVYv} donne
            \begin{equation*}
                P(A|B)=\frac{ P(A\cap B) }{ P(B) }
            \end{equation*}
            Cela est la définition de base. L'autre est une définition dérivée.
        \item D'un événement vis-à-vis d'une variable aléatoire discrète. C'est par la définition~\ref{DEFooFRLFooNvXuPK} qui définit la variable aléatoire
\begin{equation*}
    P(A|X)(\omega)=P(A|X=X(\omega)).
\end{equation*}
Dans le cas continu, c'est la définition~\ref{DEFooIUJMooBAVtMW} :
\begin{equation*}
    P(A|X)=P(A|\sigma(X))=E(\mtu_A|\sigma(X)).
\end{equation*}

    \item D'un événement par rapport à une tribu. C'est la variable aléatoire
\begin{equation*}
    P(A|\tribF)=E(\mtu_A|\tribF).
\end{equation*}

        \end{itemize}
    \item[Espérances conditionnelles] 
        Plusieurs espérances conditionnelles.
        \begin{itemize}
            \item
                D'une variable aléatoire par rapport à un événement, définition \ref{DEFooOMLCooJgrbpx} :
                \begin{equation}
                    E(X|A)=\frac{ E(X\mtu_A) }{ P(A) }.
                \end{equation}
            \item d'une variable aléatoire par rapport à une tribu. La variable aléatoire \( E(X|\tribF)\) est la variable aléatoire \( \tribF\)-mesurable telle que
\begin{equation*}
    \int_BE(X|\tribF)=\int_BX
\end{equation*}
pour tout \( X\in \tribF\). Si \( X\in L^2(\Omega,\tribA,P)\) alors \( E(X|\tribF)=\pr_K(X)\) où \( K\) est le sous-ensemble de \( L^2(\Omega,\tribA,P)\) des fonctions \( \tribF\)-mesurables (théorème~\ref{ThoMWfDPQ}). Cela au sens des projections orthogonales.


    \item d'une variable aléatoire par rapport à une autre. La définition~\ref{DefooKIHPooMhvirn} est une variation sur le même thème :
\begin{equation*}
    E(X|Y)=E(X|\sigma(Y)),
\end{equation*}

%TODO : mettre cette définition à côté de celle du conditionnement par rapport à la tribu.
        \end{itemize}

\end{description}

Notons que partout, si \( X\) est une variable aléatoire, la notation «sachant \( X\)» est un raccourci pour dire «sachant la tribu engendrée par \( X\)».

Quelque formules.
\begin{enumerate}
    \item
        Pour l'espérance conditionnelle d'une variable aléatoire prenant seulement une quantité dénombrable de valeurs :
        $E(X|A)=\sum_{k=0}^{\infty}y_kP(X=y_k|A)$ par le lemme \ref{LEMooRTVBooCEeIxL}.
    \item
        La probabilité conditionnelle se factorise par rapport à l'union disjointe par le lemme \ref{LEMooRDXRooQLMRGF} : $P\big( \bigcup_{i=0}^{\infty}A_i|B \big)=\sum_{i=0}^{\infty}P(A_i|B)$.
\end{enumerate}

\InternalLinks{dénombrements}
\begin{itemize}
    \item Coloriage de roulette (\ref{pTqJLY}) et composition de colliers (\ref{siOQlG}).
    \item Nombres de Bell, théorème~\ref{ThoYFAzwSg}.
    \item Le dénombrement des solutions de l'équation \( \alpha_1 n_1+\ldots \alpha_pn_p=n\) utilise des séries entières et des décompositions de fractions en éléments simples\footnote{Éléments simples, lemme \ref{LEMooABJMooJTUpgV}.}, théorème \ref{THOooQDYWooCOiUMb}.
\end{itemize}


\InternalLinks{enveloppes}
    \begin{enumerate}
        \item
            L'ellipse de John-Loewner donne un ellipsoïde de volume minimum autour d'un compact dans \( \eR^n\), théorème~\ref{PropJYVooRMaPok}.
        \item
            Le cercle circonscrit à une courbe donne un cercle de rayon minimal contenant une courbe fermée simple, proposition~\ref{PROPDEFooCWESooVbDven}.
    \item Enveloppe convexe du groupe orthogonal~\ref{ThoVBzqUpy}.
    \item Enveloppe convexe d'une courbe fermée plane comme intersection des demi-plans tangents, proposition~\ref{PROPooOORPooCXrIQi}.
        \end{enumerate}


\InternalLinks{équations diophantiennes}
    \begin{enumerate}
        \item
            Équation \( ax+by=c\) dans \( \eN\), équation \eqref{EqTOVSooJbxlIq}.
        \item Dans~\ref{subsecZVKNooXNjPSf}, nous résolvons \( ax+by=c\) en utilisant Bézout (théorème~\ref{ThoBuNjam}).
        \item L'exemple~\ref{ExZPVFooPpdKJc} donne une application de la pure notion de modulo pour \( x^2=3y^2+8\). Pas de solutions.
        \item L'exemple~\ref{ExmuQisZU} résout l'équation \( x^2+2=y^3\) en parlant de l'extension \( \eZ[i\sqrt{2}]\) et de stathme.
        \item Les propositions~\ref{PropXHMLooRnJKRi} et~\ref{propFKKKooFYQcxE} parlent de triplets pythagoriciens.
        \item Le dénombrement des solutions de l'équation \( \alpha_1 n_1+\ldots \alpha_pn_p=n\) utilise des séries entières et des décompositions de fractions en éléments simples, théorème \ref{THOooQDYWooCOiUMb}.
        \item La proposition~\ref{PROPooLPKUooAlsYJg} donne une bijection \( \eN\times \eN\to \eN\) en résolvant dans \( \eN\) (entre autres) l'équation \( k=y^2+x\) pour \( k\) fixé.
        \end{enumerate}


\InternalLinks{changement de variables}
    Il n'existe rien en mathémaitque qui s'appelle «changement de variables». Il n'existe que des compositions de fonctions. Ce snobisme terminologique étant, voici un certain nombre de résultats de changement de variables.
    \begin{enumerate}
        \item
            Dans des intégrales, théorème \ref{THOooUMIWooZUtUSg}.
        \item
            Dans des limites, le lemme \ref{LEMooAHIGooJhpPvo} donne $\lim_{x\to a} f(x)=\lim_{x\to b}f(x+a-b)$ si la limite existe.
        \item
            Dans une équation aux dérivées partielles, exemple \ref{EQooPGDPooTjiVhB}.
    \end{enumerate}

\InternalLinks{techniques de calcul}        \label{THEMEooLTCIooGDIPnF}

Il y en a pour tous les gouts.

\begin{description}
    \item[Primitives et intégrales]
        Toute la section~\ref{SECooKSOFooEVKDLh} donne des trucs et astuces pour trouver des primitives et des intégrales.
    \item[Limite à deux variables]

        Les exemples de limites à plusieurs variables font souvent intervenir des coordonnées polaires (du théorème \ref{THOooBETSooXSQhdX}) ou autres fonctions trigonométriques. Ils sont donc placés beaucoup plus bas que la théorie.
        \begin{itemize}
            \item Méthode du développement asymptotique, sous-section~\ref{SUBSECooRAKKooAnpvkE}.
            \item Méthode des coordonnées polaires, sous-section~\ref{SUBSECooWCGMooPrXSpt}.
            \item Utilisation du théorème de la fonction implicite, dans l'exemple~\ref{EXooSDHDooJzDioW}.
        \end{itemize}

\end{description}

\immediate\closeout\themetoc


%+++++++++++++++++++++++++++++++++++++++++++++++++++++++++++++++++++++++++++++++++++++++++++++++++++++++++++++++++++++++++++
\section{Conventions sur les matrices et changement de bases}
%+++++++++++++++++++++++++++++++++++++++++++++++++++++++++++++++++++++++++++++++++++++++++++++++++++++++++++++++++++++++++++
\label{SECooBTTTooZZABWA}

%--------------------------------------------------------------------------------------------------------------------------- 
\subsection{Matrices et applications linéaires}
%---------------------------------------------------------------------------------------------------------------------------
\label{SUBSECooAFPDooOzXdGz}

Le lien entre matrice et application linéaire est donné par la définition \ref{DEFooJVOAooUgGKme}. L'application d'une matrice à un vecteur est \eqref{EQooQFVTooMFfzol}. Le lien le plus simple entre l'application linéaire et les éléments de matrice est donné par la proposition \ref{PROPooGXDBooHfKRrv}. Voici les relations :
\begin{subequations}
    \begin{align}
    T_{\alpha i}&=T(e_i)_{\alpha}\\
    T(e_i)&=\sum_{\alpha}T_{\alpha i}f_{\alpha}\\
    T(x)&=\sum_{i\alpha}T_{\alpha i}x_if_{\alpha}\\
    T(x)_{\alpha}&=\sum_{i}T_{\alpha i}x_i.
    \end{align}
\end{subequations}

Cela définit une application $\psi\colon \eM(n\times m, \eK)\to \aL(E,F)$ qui a plein de propriétés.
\begin{enumerate}
    \item
        C'est une bijection, proposition \ref{PROPooGXDBooHfKRrv}\ref{ITEMooHSMLooRJZref}.
    \item
        C'est un isomorphisme d'algèbre, proposition \ref{PROPooFMBFooEVCLKA}.
    \item
        C'est un isomorphisme d'espaces vectoriels, proposition \ref{PROPooIYVQooOiuRhX}.
    \item
        Isomorphisme d'algèbres et d'anneaux, proposition \ref{PROPooFMBFooEVCLKA}.
    \item
        Isomorphisme d'espaces topologiques, proposition \ref{PROPooFMBFooEVCLKA}.
\end{enumerate}

Lorsque nous avons une base orthonormée nous avons aussi les propositions \ref{PROPooZKWXooWmEzoA} et \ref{PROPooZKWXooWmEzoA} qui donnent des formules avec produit scalaire :
\begin{enumerate}
    \item
    $T_{\alpha i}=e_{\alpha}\cdot T(e_i)$
    \item
    $x\cdot Ay=\sum_{kl}A_{kl}x_ky_l$.
\end{enumerate}
où le point est le produit scalaire usuel de \( \eR^n\).

%---------------------------------------------------------------------------------------------------------------------------
\subsection{Le changement de base}
%---------------------------------------------------------------------------------------------------------------------------

Soit un espace vectoriel \( V\) muni de deux bases \( (e_i)_{i=1,\ldots, n}\) et \( (f_{\alpha})_{\alpha=1,\ldots, n}\). Le lemme \ref{LEMooIHZGooOZoYZd} donne le lien entre les vecteurs de base :
\begin{enumerate}
    \item
        $f_{\alpha}=\sum_iQ_{i\alpha}e_i$
    \item
        $e_i=\sum_{\alpha}Q^{-1}_{\alpha i}f_{\alpha}$
\end{enumerate}
La proposition \ref{PROPooNYYOooHqHryX} donne un certain nombre de formules pour les coordonnées des vecteurs :
\begin{enumerate}
    \item   
        \( y_{\alpha}=\sum_iQ^{-1}_{\alpha i}x_i\)
    \item  
        $x_i=\sum_{\alpha}Q_{i\alpha}y_{\alpha}$.
    \item
    $x_i=(Qy)_i$ 
    \item
    $x=Qy$
\end{enumerate}

La transformation de la matrice d'une application linéaire lors d'un changement de base est la proposition \ref{PROPooNZBEooWyCXTw}. Soit une application linéaire \( T\colon V\to V\) de matrices \( A\) et \( B\) dans les bases \( \{ e_i \}\) et \( \{ f_{\alpha} \}\). Si les bases sont liées par $f_{\alpha}=\sum_iQ_{i\alpha}e_i$, alors les matrices \( A\) et \( B\) sont liées par
    \begin{equation}
        B=Q^{-1}AQ.
    \end{equation}

%---------------------------------------------------------------------------------------------------------------------------
\subsection{Changement de base : matrice d'une forme bilinéaire}
%---------------------------------------------------------------------------------------------------------------------------

La proposition \ref{PROPooLBIOooUpzxXA} fait le changement de matrice d'une forme bilinéaire lors d'un changement de base. Si la matrice de \( q\) dans la base \( \{ e_i \}\) est \( A\) et celle dans la base \( \{ f_{\alpha} \}\) est \( B\), alors
\begin{equation}
    B=Q^tAQ.
\end{equation}
Pour comparaison avec la loi de transformation des matrices des applications linéaires, voir la remarque \ref{REMooNEJLooSqgeih}.

%+++++++++++++++++++++++++++++++++++++++++++++++++++++++++++++++++++++++++++++++++++++++++++++++++++++++++++++++++++++++++++ 
\section{Multiindice et liste d'indices}
%+++++++++++++++++++++++++++++++++++++++++++++++++++++++++++++++++++++++++++++++++++++++++++++++++++++++++++++++++++++++++++

\begin{normaltext}      \label{NORMooRRZCooMOKAzY}
    Je crois qu'il y a quelques incohérences de notations/dénominations dans le texte. En principe quand on parle de \( \eR^n\), un \defe{multiindice}{multiindice} \cite{BIBooYDMJooGDtdbo} est une vecteur d'entiers positifs à \( n\) composantes. Si \( \alpha=(2,1)\) alors nous avons la notation
    \begin{equation}
        \partial^{\alpha}f=\frac{ \partial^2  }{ \partial x_1 }\frac{ \partial  }{ \partial x_2 }f
    \end{equation}
    Cette notation pose problème lorsque, par exemple, \( \partial_1^2\partial_2f\neq \partial_1\partial_2\partial_1f\).

    Elle pose également problème lorsque l'on veut faire une récurrence sur l'ordre de dérivation en ajoutant une seule dérivation à la fois.

    C'est pourquoi nous introduisons le concept de \defe{liste d'indices}{liste d'indices}. En parlant de \( \eR^n\), une liste d'indices est un vecteur arbitrairement long (mais fini) d'entiers dans \( \{ 1,\ldots, n \}\). Si, dans \( \eR^7\), \( \alpha=(1,3,1,5)\), alors
    \begin{equation}
        \partial^{\alpha}f=\partial_1\partial_3\partial_1\partial_5f.
    \end{equation}

    Si \( \alpha\) est une liste d'indices de longueur \( p\), une \defe{queue de}{queue de liste d'indices} \( \alpha\) est une liste d'indices de longueur \( 0 < k \leq p\) de la forme \( (\alpha_{p-k+1}, \alpha_{p-k+2},\ldots, \alpha_p)\).
\end{normaltext}
