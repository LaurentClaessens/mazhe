% This is part of le Frido
% Copyright (c) 2016-2017
%   Laurent Claessens
% See the file fdl-1.3.txt for copying conditions.

%+++++++++++++++++++++++++++++++++++++++++++++++++++++++++++++++++++++++++++++++++++++++++++++++++++++++++++++++++++++++++++ 
\section*{Thèmes}
%+++++++++++++++++++++++++++++++++++++++++++++++++++++++++++++++++++++++++++++++++++++++++++++++++++++++++++++++++++++++++++

Ceci est une sorte d'index thématique. Les thèmes ici ne sont pas des suggestions de plans de leçon d'agrégation. Le fait que certains titres sont équivalents à des plans de leçons n'est cependant pas une coïncidence : deux personnes qui se posent la même question «comment regrouper les résultats pour que ceux qui se parlent soient mis ensemble ?» arrivent souvent à la même réponse. Cependant l'esprit est différent. L'esprit des leçons d'agrégation est l'esprit du concours, tandis que l'esprit des thèmes ici est de simplifier la recherche dans le Frido parce que ce bouquin est très gros : c'est un esprit «table des matières».

% The macro '\InternalLink' writes in 'themestoc.tex' the lines like
%  \ref {THTOC4} : méthode de Newton\\

% We open the file after the input (if before, we erase it) and 
% we close at the end of this file.

\begin{multicols}{2}
\noindent
\ref {THTOC1} : fonctions Lipschitz\\
\ref {THTOC2} : formule des accroissements finis\\
\ref {THTOC3} : polynôme de Taylor\\
\ref {THTOC4} : points fixes\\
\ref {THTOC5} : méthode de Newton\\
\ref {THTOC6} : enveloppes\\
\ref {THTOC7} : produit semi-direct de groupes\\
\ref {THTOC8} : racines de polynôme et factorisation de polynômes\\
\ref {THTOC9} : théorème de Bézout\\
\ref {THTOC10} : équations diophantiennes\\
\ref {THTOC11} : application réciproque\\
\ref {THTOC12} : extension de corps et polynômes\\
\ref {THTOC13} : rang\\
\ref {THTOC14} : topologie produit\\
\ref {THTOC15} : produit de compact\\
\ref {THTOC16} : connexité\\
\ref {THTOC17} : norme matricielle et rayon spectral\\
\ref {THTOC18} : norme opérateur\\
\ref {THTOC19} : diagonalisation\\
\ref {THTOC20} : sous-groupes\\
\ref {THTOC21} : tribu, algèbre de parties, \( \lambda \)-systèmes et co.\\
\ref {THTOC22} : mesure et intégrale\\
\ref {THTOC23} : intégration\\
\ref {THTOC24} : équivalence de normes\\
\ref {THTOC25} : espace \( L^2\) (L2)\\
\ref {THTOC26} : espaces \( L^p\) (Lp)\\
\ref {THTOC27} : théorème de Stokes, Green et compagnie\\
\ref {THTOC28} : invariants de similitude\\
\ref {THTOC29} : endomorphismes cycliques\\
\ref {THTOC30} : formes bilinéaires et quadratiques\\
\ref {THTOC31} : isométries\\
\ref {THTOC32} : déterminant\\
\ref {THTOC33} : polynôme d'endomorphismes\\
\ref {THTOC34} : action de groupe\\
\ref {THTOC35} : systèmes d'équations linéaires\\
\ref {THTOC36} : classification de groupes\\
\ref {THTOC37} : théorie des représentations\\
\ref {THTOC38} : décomposition de matrices\\
\ref {THTOC39} : méthodes de calcul\\
\ref {THTOC40} : équations différentielles\\
\ref {THTOC41} : dénombrements\\
\ref {THTOC42} : densité\\
\ref {THTOC43} : injections\\
\ref {THTOC44} : dualité\\
\ref {THTOC45} : opérations sur les distributions\\
\ref {THTOC46} : permuter des limites\\
\ref {THTOC47} : convolution\\
\ref {THTOC48} : séries de Fourier\\
\ref {THTOC49} : transformée de Fourier\\
\ref {THTOC50} : applications continues et bornées\\
\ref {THTOC51} : définie positive\\
\ref {THTOC52} : gaussienne\\
\ref {THTOC53} : inégalités\\
\ref {THTOC54} : théorème central limite\\
\ref {THTOC55} : lemme de transfert\\
\ref {THTOC56} : déduire la nullité d'une fonction depuis son intégrale\\
\ref {THTOC57} : caractérisation de distributions en probabilités\\
\ref {THTOC58} : probabilités et espérances conditionnelles\\

\end{multicols}

\newwrite\themetoc      
\immediate\openout\themetoc=themestoc.tex



% ATTENTION : il est très important que le titre «Thèmes» soit au haut d'une page et que le premier thème commence sur cette même page. Donc pas de texte trop long ici.
% La raison :
% Pour la division en volume, je prend le bloc 'thèmes' comme commençant à la page du premier. Cela est dû au fait que le titre n'est pas dans la TOC.

% Convention : les titres ne commencent pas par une majuscule
%  La raison : cela facilite les recherches dans le pdf (oui, je sais : on peut faire des recherches sans tenir compte des majuscules)


% ANALYSE


\InternalLinks{mesure et intégration}
\begin{description}
    \item[Mesure] 
    À propos de mesure.
\begin{enumerate}
    \item
        Mesure positive, mesure finie et \( \sigma\)-finie, c'est la définition \ref{DefBTsgznn}.

    \item Le produit de tribus est donné par la définition~\ref{DefTribProfGfYTuR},     % Cette référence doit être vers le haut.
    \item
        Produit d'une mesure par une fonction, définition \ref{PropooVXPMooGSkyBo}.
    \item le produit d'espaces mesurés est donné par la définition~\ref{DefUMlBCAO}.     % Cette référence doit être vers le haut.
        \item
            Mesure de Lebesgue sur \( \eR\), définition~\ref{DefooYZSQooSOcyYN}.
        \item
            Mesure de Lebesgue sur \( \eR^N\), définition~\ref{DEFooSWJNooCSFeTF}.
        \item

            Si \( f\colon S\to \mathopen[ 0 , +\infty \mathclose]\) est une fonction mesurable, le théorème fondamental d'approximation~\ref{THOooXHIVooKUddLi} donne une suite croissante de fonctions étagées qui converge vers \( f\).
        \item
            Mesure à densité, définition~\ref{PropooVXPMooGSkyBo}.
\end{enumerate}
\item[Intégration]
    À propos d'intégration.
    \begin{enumerate}
        \item
            Intégrale associée à une mesure, définition~\ref{DefTVOooleEst}
    \item
        L'existence d'une primitive pour toute fonction continue est le théorème~\ref{ThoEXXyooCLwgQg}.
    \item
        La définition d'une primitive est la définition~\ref{DefXVMVooWhsfuI}.
    \item
        Primitive et intégrale, proposition~\ref{PropEZFRsMj}.
    \item
        Intégrale impropre, définition~\ref{DEFooINPOooWWObEz}.
\end{enumerate}
            
\end{description}

  % intégration
\InternalLinks{suites et séries}

\begin{description}
    \item[Suites] 
        Les suites réelles sont en général dans la proposition \ref{PropLimiteSuiteNum} et ce qui s'ensuit. Cette proposition est souvent prise comme définition lorsque seules les suites réelles ne sont considérées.
        \begin{enumerate}
    \item
        Les suites adjacentes, c'est la définition \ref{DEFooDMZLooDtNPmu}. 
    \item
        Les séries alternées, théorème \ref{THOooOHANooHYfkII}. Il s'agit de dire que \( \sum_{k=0}^{\infty}(-1)^ka_k\) converge quand \( a_k\) est décroissante et tend vers zéro.
    \item
        Le concept de suite adjacente sert à étudier la série de Taylor de \( \ln(x+1)\), voir le lemme \ref{LEMooWMGGooRpAxBa} et ce qui l'entoure.
    \item
        La définition de la convergence absolue est la définition~\ref{DefVFUIXwU}.
            \item
                Une suite réelle croissante et majorée converge, proposition \ref{LemSuiteCrBorncv}.
            \item
                Toute suite dans un compact admet une sous-suite convergente, théorème \ref{THOooRDYOooJHLfGq}.
            \item
                Pour tout réel, il existe une suite croissante de rationnels qui y converge, proposition \ref{PropSLCUooUFgiSR}.
        \end{enumerate}
    \item[Série] 
        Les séries sont en général dans la section \ref{SECooYCQBooSZNXhd}.
        \begin{enumerate}
    \item
        Quelques séries usuelles en \ref{SUBSECooDTYHooZjXXJf} : série harmonique, géométrique, de Riemann, et la mythique arithmético-géométrique.
        \begin{enumerate}
            \item
                La série harmonique diverge : \( \sum_k\frac{1}{ k }=\infty\), exemple \ref{EXooDVQZooEZGoiG}.
            \item
                La série géométrique : \( \sum_{k=0}^Nq^k=\frac{ 1-q^{N+1} }{ 1-q }\), exemple \ref{ExZMhWtJS}.
            \item
                Une autre cool série : \( \sum_{k=1}^N\frac{ 1 }{ k(k+1) }=\frac{ N }{ N+1 }\), lemme \ref{LEMooKDHPooPlFTIT}.
        \end{enumerate}
    \item
        Critère des séries alternées, théorème \ref{THOooOHANooHYfkII}.
    \item
        Convergence d'une série implique convergence vers zéro du terme général, proposition~\ref{PROPooYDFUooTGnYQg}.
        \end{enumerate}
    \item[Sommes infinies]
        Nous pouvons dire plusieurs choses à propos d'une somme infinie.% cette phrase est là pour le mot-clef ``somme infinie''.
        \begin{enumerate}
            \item
Une somme indexée par un ensemble quelconque est la définition~\ref{DefHYgkkA}.
    \item
        La définition de la somme d'une infinité de termes est donnée par la définition~\ref{DefGFHAaOL}.
  \item
      si la série converge, on peut regrouper ses termes sans modifier la convergence ni la somme (associativité);
    Pour les sommes infinies l'associativité et la commutativité dans une série sont perdues. Néanmoins, il subsiste que
  \begin{enumerate}
  \item
      si la série converge absolument, on peut modifier l'ordre des termes sans modifier la convergence ni la somme (commutativité, proposition~\ref{PopriXWvIY}).
  \end{enumerate}
  \item Permuter une somme infinie avec une application linéaire : \( f(\sum_{i\in I}v_i)=\sum_{i\in I}f(v_i)\), c'est la proposition \ref{PROPooWLEDooJogXpQ}.
        \end{enumerate}
\end{description}
  % suites et séries

\InternalLinks{polynôme de Taylor}
    \begin{enumerate}
    \item
        Énoncé : théorème \ref{ThoTaylor}.
        \item
            Le polynôme de Taylor généralise à l'utilisation de toutes les dérivées disponibles le résultat de développement limité donné par la proposition \ref{PropUTenzfQ}.
        \item
            Pour les fonctions holomorphes, il y a le théorème \ref{THOooSULFooHTLRPE} qui donne une série de Taylor sur un disque de convergence.
        \item
            Il est utilisé pour justifier la méthode de Newton autour de l'équation \eqref{EQooOPUBooYaznay}.
        \end{enumerate}

  % polynôme de Taylor
\InternalLinks{séries de Fourier}       \label{THMooHWEBooTMInve}
\begin{itemize}
    \item Formule sommatoire de Poisson, proposition \ref{ProprPbkoQ}.
    \item Inégalité isopérimétrique, théorème \ref{ThoIXyctPo}.
    \item Fonction continue et périodique dont la série de Fourier ne converge pas, proposition \ref{PropREkHdol}.

    \item
Nous allons montrer la convergence de \( \sum_{k\in \eZ}c_k(f) e^{inx}\) vers \( f(x)\) dans divers cas :
\begin{enumerate}
    \item
        Si \( f\) est continue et périodique, convergence au sens de Cesaro, théorème de Fejèr \ref{ThoJFqczow}.
    \item
        Convergence au sens \( L^2\Big( \mathopen[ 0 , 2\pi \mathclose] \Big)\) dans le théorème \ref{ThoYDKZLyv}.
    \item
        Si \( f\) est continue, périodique et sa série de Fourier converge uniformément, théorème \ref{PropmrLfGt}.
    \item
        Si \( f\) est périodique et la série des coefficients converge absolument pour tout \( x\), proposition \ref{PropSgvPab}.
    \item
        Si \( f\) est périodique et de classe \( C^1\), théorème \ref{ThozHXraQ}.
\end{enumerate}
Il est cependant faux de croire que la continuité et la périodicité suffisent à obtenir une convergence, comme le montre dans la proposition \ref{PropREkHdol}.
\end{itemize}
  % séries de Fourier

\InternalLinks{équivalence de normes}
    \begin{enumerate}
\item
    La proposition \ref{PropLJEJooMOWPNi} sur l'équivalence des normes dans \( \eR^n\).
\item
    Montrer que le problème \( a-b\) est stable dans l'exemple \ref{ExooXJONooTAYZVc}.
\end{enumerate}

  %  normes

\InternalLinks{topologie produit}
    \begin{enumerate}
        \item
            La définition de la topologie produit est \ref{DefIINHooAAjTdY}.
        \item
            Pour les espaces vectoriels normés, le produit est donné par la définition \ref{DefFAJgTCE}.
        \item
            L'équivalence entre la topologie de la norme produit et la topologie produit est le lemme \ref{LEMooWVVCooIGgAdJ}.
        \end{enumerate}

  % topologie produit
\InternalLinks{espaces métriques, normés}
\begin{enumerate}
    \item
        Le théorème-définition~\ref{ThoORdLYUu} donne la topologie sur un espace métrique en disant que les boules ouvertes sont une base de la topologie (définition \ref{DefQELfbBEyiB}).
    \item
        La définition de la convergence d'une suite est la définition~\ref{DefXSnbhZX}.
    \item
        Dans un espace vectoriel normé, une application est continue si et seulement si elle est bornée, proposition~\ref{PROPooQZYVooYJVlBd}.
\end{enumerate}
  % espaces métriques, normés

    \InternalLinks{norme opérateur}     \label{THEMEooHSLLooBQpFAr}
    Pour la norme matricielle et le rayon spectral, voir le thème \ref{THEMEooOJJFooWMSAtL}.
    \begin{enumerate}
        \item
            Définition \ref{DefNFYUooBZCPTr}.
        \item 
            Définition d'une norme d'algèbre \ref{DefJWRWQue}.
        \item
            Pour des espaces vectoriels normée, être borné est équivalent à être continu : proposition \ref{PropmEJjLE}.
    \end{enumerate}

  % norme opérateur
\input{52_theme}  % gaussienne

\InternalLinks{compacts}        \label{THEMEooQQBHooLcqoKB}
    \begin{description}

        \item[Propriétés générales]

            Quelques propriétés de compacts.

                \begin{enumerate}
    \item
        La définition d'un ensemble compact est la définition~\ref{DefJJVsEqs}.
    \item
        Les compacts sont les fermés bornés par le théorème~\ref{ThoXTEooxFmdI}.
    \item
        L'image d'un compact par une fonction continue est un compact, théorème~\ref{ThoImCompCotComp}.
    \item
        Suites dans un compact
        \begin{enumerate}
            \item
                Toute suite dans un compact admet une sous-suite convergente, théorème \ref{THOooRDYOooJHLfGq}.
            \item
                Dans \( \eR^n\), toute suite dans un compact admet une sous-suite convergente, théorème \ref{ThoBolzanoWeierstrassRn}. La démonstration de ce théorèma est non seulement plus compliquée que le cas général, mais utilise en plus le cas dans \( \eR\); lequel cas n'est pas démontré de façon directe dans le Frido.
            \item
                Un espace métrique est compact si et seulement si toute suite contient une sous-suite convergente. C'est le théorème de Bolzano-Weierstrass~\ref{ThoBWFTXAZNH}. La démonstration de ce théorème est indépendante.
        \end{enumerate}
    \item
        Une fonction continue sur un compact est bornée et attein ses bornes, théorème~\ref{ThoWeirstrassRn}.
                \end{enumerate}

        \item[Produits de compacts]
            À propos de produits de compacts. C'est un compact dans tous les cas métriques\quext{Si vous connaissez des exemples non métriques de produits de compacts qui ne sont pas compacts, écrivez moi.}.
    \begin{enumerate}
    \item
        Les produits d'espaces métriques compacts sont compacts; c'est le théorème de Tykhonov. Nous verrons ce résultat dans les cas suivants.
        \begin{itemize}
    \item
         \( \eR\), lemme~\ref{LemCKBooXkwkte}.
    \item
        Produit fini d'espaces métriques compacts, théorème~\ref{THOIYmxXuu}.
    \item
        Produit dénombrable d'espaces métriques compacts, théorème~\ref{ThoKKBooNaZgoO}.
        \end{itemize}
    \end{enumerate}
    \end{description}
  % produit de compact

\InternalLinks{densité}
\begin{enumerate}
    \item 
        Densité des polynômes dans \( C^0\big( \mathopen[ 0 , 1 \mathclose] \big)\), théorème de Bernstein \ref{ThoDJIvrty}.
    \item
        Densité de \( \swD(\eR^d)\) dans \( L^p(\eR^d)\) pour \( 1\leq p<\infty\), théorème \ref{ThoILGYXhX}.
    \item
        Densité de \( \swS(\eR^d)\) dans l'espace de Sobolev \( H^s(\eR^d)\), proposition \ref{PROPooMKAFooKDNTbO}. 

    \item
        Densité de \( \swD(\eR^d)\) dans l'espace de Sobolev \( H^s(\eR^d)\), proposition \ref{PROPooLIQJooKpWtnV}. 

        Cela est utilisé pour le théorème de trace \ref{THOooXEJZooBKtXBW}.
\end{enumerate}
Les densités sont bien entendu utilisées pour prouver des formules sur un espace en sachant qu'elles sont vraies sur une partie dense. Mais également pour étendre une application définie seulement sur une partie dense. C'est par exemple ce qui est fait pour définir la trace \( \gamma_0\) sur les espaces de Sobolev \( H^s(\eR^d)\) en utilisant le théorème d'extension \ref{PropTTiRgAq}.

  % densité
\InternalLinks{espaces de fonctions}                \label{THEMooNMYKooVVeGTU}

En ce qui concerne les densités, voir le thème~\ref{THEooPUIIooLDPUuq}.


\begin{description}
    \item[Topologie]

        Les espaces de fonctions sont souvent munis de topologies définies par des semi-normes.

        \begin{enumerate}
            \item
                La topologie des semi-normes est la définition~\ref{DefPNXlwmi}.
            \item
                La définition~\ref{DefFGGCooTYgmYf} donne les topologies sur \(  C^{\infty}(\Omega)\), \( \swD(K)\) et \( \swD(\Omega)\).
            \item
                La topologie \( *\)-faible sur \( \swD'(\Omega)\) est donnée par la définition~\ref{DefASmjVaT}.
        \end{enumerate}

    \item[L'espace \( { L^2\big( \mathopen[ 0 , 2\pi \mathclose] \big) } \)]

        C'est un espace très important, entre autres parce qu'il est de Hilbert et est bien adapté à la transformée de Fourier.

        \begin{enumerate}
        \item
            Un rappel de la construction en \ref{NORMooUEIEooYtlFse}.
            \item
                Le produit scalaire \( \langle f, g\rangle \) est donné en \eqref{EQooBFKDooMkCZOt} et la base trigonométrique est \eqref{EQooKMYOooLZCNap}.
            \item
                La densité des polynômes trigonométriques dans \( L^p(S^2)\) est le théorème~\ref{ThoQGPSSJq} ou le théorème~\ref{ThoDPTwimI}, au choix.
            \item
                Une conséquence de cette densité est que le système trigonométrique est une base hilbertienne de \( L^2\) par le lemme~\ref{LEMooBJDQooLVPczR}.
        \end{enumerate}

            L'espace \( L^2\) est discuté en analyse fonctionnelle, dans la section \ref{SECooEVZSooLtLhUm} et les suivantes parce que l'étude de \( L^2\) utilise entre autres l'inégalité de Hölder~\ref{ProptYqspT}.

        Le fait que \( L^2\) soit une espace de Hilbert est utilisé dans la preuve du théorème de représentation de Riesz~\ref{PropOAVooYZSodR}.

\end{description}
  % espaces de fonctions
\input{1_theme}  % fonctions Lipschitz

\InternalLinks{formule des accroissements finis}
    Il en existe plusieurs formes :
    \begin{enumerate}
        \item
            Une version adaptée aux espaces de dimension finie est le théorème~\ref{val_medio_2}.
        \item
            Au premier ordre, proposition \ref{PropUTenzfQ}.
        \item
            Pour les fonctions \( \eR\to \eR\) en le théorème~\ref{ThoAccFinis}.
        \item
            Une généralisation pour les intervalles non bornés : théorème~\ref{THOooRIIBooOjkzMa}.
        \item
            Espaces vectoriels normés, théorème~\ref{ThoNAKKght}
    \end{enumerate}
  % formule des accroissements finis
\InternalLinks{différentiabilité}
\begin{enumerate}
    \item
        La recherche d'extrema d'une fonction sur \( \eR^n\) passe par la seconde différentielle, proposition \ref{PropoExtreRn}.
    \item
        La différentielle est liée aux dérivées partielles par les formules données au lemme \ref{LemdfaSurLesPartielles}
	\begin{equation}
        df_a(u)=\frac{ \partial f }{ \partial u }(a)=\Dsdd{ f(a+tu) }{t}{0}=\sum_{i=1}^mu_i\frac{ \partial f }{ \partial x_i }(a)=\nabla f(a)\cdot u.
	\end{equation}
\end{enumerate}
    % différentielle
\input{4_theme}  % points fixes

\InternalLinks{théorème de Stokes, Green et compagnie}
    \begin{enumerate}
        \item
            Forme générale, théorème \ref{ThoATsPuzF}.
        \item
            Rotationnel et circulation, théorème \ref{THOooIRYTooFEyxif}.
        \end{enumerate}

  % théorème de Stokes, Green et compagnie

\InternalLinks{permuter des limites}
\begin{enumerate}
    \item 
        Les théorèmes sur les fonctions définies par des intégrales, section \ref{SecCHwnBDj}. Nous avons entre autres
        \begin{enumerate}
            \item
                \( \partial_i\int_Bf=\int_B\partial_if\), avec \( B\) compact, proposition \ref{PropDerrSSIntegraleDSD}.
            \item
                Si \( f\) est majorée par une fonction ne dépendant pas de \( x\), nous avons le théorème \ref{ThoKnuSNd}.
            \item
                Si l'intégrale est uniformément convergente, nous avons le théorème \ref{ThotexmgE}.
            \item
                Pour dériver \( \int_Bg(t,z)dt\) avec \( B\) compact dans \( \eR\) et \( g\colon \eR\times \eC\to \eC\), il faut aller voir la proposition \ref{PROPooZCLYooUaSMWA}.
        \end{enumerate}
    \item 
        Théorème de la convergence monotone, théorème \ref{ThoRRDooFUvEAN}.
    \item
        Le théorème de Fubini permet non seulement de permuter des intégrales, mais également des sommes parce que ces dernières peuvent être vues comme des intégrales sur \( \eN\) muni de la tribu des parties et de la mesure de comptage. Nous utilisons cette technique pour permute une somme et une intégrale dans l'équation \eqref{EQooWOLOooFHSrsx}.
\begin{itemize}
    \item
        le théorème de Fubini-Tonelli \ref{ThoWTMSthY} demande que la fonction soit mesurable et positive;
    \item
        le théorème de Fubini \ref{ThoFubinioYLtPI} demande que la fonction soit intégrable (mais pas spécialement positive);
    \item
        le corollaire \ref{CorTKZKwP} demande l'intégrabilité de la valeur absolue des intégrales partielles pour déduire que la fonction elle-même est intégrable.
\end{itemize}

%TODO : des démonstrations de ces trois théorèmes seraient les bienvenues.

\end{enumerate}

  % permuter des limites

\InternalLinks{applications continues et bornées}
\begin{enumerate}
    \item
        Une application linéaire non continue : exemple \ref{ExHKsIelG} de \( e_k\mapsto ke_k\). Les dérivées partielles sont calculées en \eqref{EQooWNLOooJNRUMQ}.
    \item
        Une application linéaire est bornée si et seulement si elle est continue, proposition \ref{PropmEJjLE}.
\end{enumerate}

  % applications continues et bornées
\input{53_theme}  % inégalités

\InternalLinks{connexité}
    \begin{enumerate}
        \item
            Définition~\ref{DefIRKNooJJlmiD}
        \item
            Le groupe \( \SL(n,\eK)\) est connexe par arcs : proposition~\ref{PROPooALQCooLZCKrH}.
        \item
            Le groupe \( \GL(n,\eC)\) est connexe par arcs : proposition~\ref{PROPooVJNIooMByUJQ}.
        \item
            Le groupe \( \GL(n,\eC)\) est connexe par arcs, proposition~\ref{PROPooVJNIooMByUJQ}.
        \item
            Le groupe \( \GL(n,\eR)\) a exactement deux composantes connexes par arcs, proposition~\ref{PROPooBIYQooWLndSW}.
        \item
            Le groupe \( \gO(n,\eR)\) n'est pas connexe, lemme~\ref{LEMooIPOVooZJyNoH}.
        \item
            Les groupe \( \gU(n)\) et \( \SU(n)\) sont connexes par arcs, lemme~\ref{LEMooQMXHooZQozMK}.
        \item
            Le groupe \( \SO(n)\) est connexe mais ce n'est pas encore démontré, proposition~\ref{PROPooYKMAooCuLtyh}.
        \item
            Connexité des formes quadratiques de signature donnée, proposition~\ref{PropNPbnsMd}.
        \end{enumerate}
  % connexité
\InternalLinks{suite de Cauchy, espace complet}     \label{THMooOCXTooWenIJE}
\begin{enumerate}
    \item
        La définition \ref{DEFooXOYSooSPTRTn} donne la notion de suite de Cauchy dans un espace métrique.
    \item
        La définition \ref{DefZSnlbPc} donne la notion de suite de Cauchy dans un espace vectoriel topologique.
    \item
        Deux espaces métriques (avec une distance) peuvent être isomorphes en tant qu'espaces topologiques, mais ne pas avoir les mêmes suites de Cauchy, exemple \ref{EXooNMNVooXyJSDm}.
    \item
        La proposition \ref{PropooUEEOooLeIImr} donne l'équivalence entre la définition «topologique» et la définition usuelle dans le cas des espaces vectoriels topologiques \emph{normés}.
    \item
        L'exemple \ref{EXooNMNVooXyJSDm} est un exemple pire que simplement une suite de Cauchy qui ne converge pas. Le problème de convergence de cette suite n'est pas simplement que la limite n'est pas dans l'espace; c'est que la suite de Cauchy donnée ne convergerait même pas dans \( \eR\).
\end{enumerate}
    % Suite de Cauchy, espace complet

\InternalLinks{application réciproque}
    \begin{enumerate}
        \item
            Définition \ref{DEFooTRGYooRxORpY}.
        \item
            Dans le cas des réels, des exemples sont donnés en \ref{EXooCWYHooLEciVj}.
        \end{enumerate}

  % application réciproque

\InternalLinks{tribu, algèbre de parties, \( \lambda\)-systèmes et co.}  \label{INTooVDSCooHXLLKp}
    Il existe des centaines de notions de mesures et de classes de parties.
    \begin{enumerate}
        \item
            Le plus souvent lorsque nous parlons de mesure est que nous parlons de mesure positive, définition \ref{DefBTsgznn} sur un espace mesuré avec une tribu, définition \ref{DefjRsGSy}.
        \item
            Une mesure extérieure est la définition \ref{DefUMWoolmMaf}
        \item 
            Une algèbre de partie : définition \ref{DefTCUoogGDud}. Une mesure sur une algèbre de parties : définition \ref{DefWUPHooEklLmR}. L'intérêt est que si on connait une mesure sur une algèbre de parties, elle se prolonge en une mesure sur la tribu engendrée par le théorème de prolongement de Hahn \ref{ThoLCQoojiFfZ}.
        \item
            Un \( \lambda\)-système : définition \ref{DefRECXooWwYgej}.
        \item
            Une mesure complexe : définition \ref{DefGKHLooYjocEt}.
    \end{enumerate}


  % tribu, algèbre de parties, \( \lambda\)-systèmes et co.
\input{56_theme}  % déduire la nullité d'une fonction depuis son intégrale

\InternalLinks{mesure et intégrale}
\begin{enumerate}
        \item
            Mesure de Lebesgue, définition \ref{DefooYZSQooSOcyYN}
        \item
            Intégrale associée à une mesure, définition \ref{DefTVOooleEst}
        \item
            Mesure à densité, définition \ref{PropooVXPMooGSkyBo}.
\end{enumerate}

  % mesure et intégrale
\input{40_theme}  % équations différentielles
\input{43_theme}  % injections
\InternalLinks{logarithme}
\begin{enumerate}
    \item
        La proposition \ref{PROPooKPBIooJdNsqX} donne une série pour \( \ln(1-x)\).
    \item
        La proposition~\ref{PropKKdmnkD} dit que toute matrice complexe admet un logarithme. En particulier une série explicite est donné pour le logarithme d'un bloc de Jordan.
    \item
        Le logarithme pour les réels strictement positifs est donné en la définition~\ref{DEFooELGOooGiZQjt}.
    \item
        Sur les complexes, le logarithme \( \ln \colon \eC^*\to \eC\) est la définition~\ref{DEFooWDYNooYIXVMC}. Attention : ce n'est pas la seule définition possible.
    \item
        La série harmonique diverge à vitesse logarithmique, et la série des inverses des nombres premiers, c'est encore plus lent : théorème~\ref{ThonfVruT}.
\end{enumerate}
  % logarithmes
\InternalLinks{inversion locale, fonction implicite}
\begin{enumerate}
    \item Inversion locale dans \( \eR^n\) : théorème \ref{THOooQGGWooPBRNEX}. Pour un Banach c'est le théorème \ref{ThoXWpzqCn}.
    \item
        Fonction implicite dans un Banach : théorème \ref{ThoAcaWho}.
    \item
        Utilisé pour montrer que le flot d'une équation différentielle est un \( C^p\)-difféomorphisme local, voir \ref{NORMooWEWVooXbGmfE}. % position 1051229132
\end{enumerate}
 % inversion locale, fonction implicite

% ANALYSES FONCTIONNELLE


\InternalLinks{dualité}     \label{THEMEooULGFooPscFJC}

Ne pas confondre dual algébrique et dual topologique d'un espace vectoriel.

\begin{description}
    \item[Dual topologique et algébrique]
        Ils sont définis par~\ref{DefJPGSHpn}. Le dual algébrique est l'ensemble des formes linéaires, et le dual topologique ne considère que les formes linéaires continues (en dimension infinie, les applications linéaires ne sont pas toutes continues).
    \item[Topologie]
        Une topologie possible sur le dual d'un espace vectoriel topologique est celle \( *\)-faible de la définition~\ref{DefHUelCDD}.

        Nous comparons les topologies faibles et de la norme en la section~\ref{SECooKOJNooQVawFY}.
    \item[Théorèmes de dualité]
        Quelques théorèmes établissent des dualités entre des espaces courants.
\begin{enumerate}
    \item
        Le théorème de représentation de Riesz~\ref{ThoQgTovL} pour les espaces de Hilbert.
    \item
        La proposition~\ref{PropOAVooYZSodR} pour les espaces \( L^p\big( \mathopen[ 0 , 1 \mathclose] \big)\) avec \( 1<p<2\).
    \item
        Le théorème de représentation de Riesz~\ref{ThoLPQPooPWBXuv} pour les espaces \( L^p\) en général.
\end{enumerate}
Tous ces théorèmes donnent la dualité par l'application \( \Phi_x=\langle x, .\rangle \).

\end{description}
  % dualité
\input{45_theme}  % opérations sur les distributions

\InternalLinks{transformée de Fourier}
\begin{enumerate}
    \item
        Définition sur \( L^1\), définition~\ref{DEFooRIXGooECoIbx}.
    \item
        La transformée de Fourier d'une fonction \( L^1(\eR^d)\) est continue, proposition~\ref{PropJvNfj}.
    \item
    L'espace de Schwartz est stable par transformée de Fourier. L'application $\TF\colon \swS(\eR^d)\to \swS(\eR^d)$ est une bijection linéaire et continue. Proposition ~\ref{PropKPsjyzT}
\end{enumerate}
  % transformée de Fourier
\input{47_theme}  % convolution

% NUMÉRIQUE
\input{5_theme}  % méthode de Newton
\input{39_theme}  % méthodes de calcul

% ALGÈBRE
\InternalLinks{définie positive}        \label{THEMEooYEVLooWotqMY}
\begin{enumerate}
    \item
        Une application bilinéaire est définie positive lorsque \( g(u,u)\geq 0\) et \( g(u,u)=0\) si et seulement si \( u=0\) est la définition~\ref{DEFooJIAQooZkBtTy}.
    \item
        Un opérateur ou une matrice est défini positif si toutes ses valeurs propres sont positives, c'est la définition~\ref{DefAWAooCMPuVM}.
    \item
        Pour une matrice symétrique, définie positive implique \( \langle Ax, x\rangle >0\) pour tout \( x\). C'est le lemme~\ref{LemWZFSooYvksjw}.
    \item
        Une application linéaire est définie positive si et seulement si sa matrice associée l'est. C'est la proposition~\ref{PROPooUAAFooEGVDRC}.
\end{enumerate}
Remarque : nous ne définissons pas la notion de matrice définie positive dans le cas d'une matrice non symétrique.
  % définie positive

        \InternalLinks{norme matricielle et rayon spectral}     \label{THEMEooOJJFooWMSAtL}
    \begin{enumerate}
        \item
            Définition du rayon spectral \ref{DEFooEAUKooSsjqaL}.
        \item
            Lien entre norme matricielle et rayon spectral, le théorème \ref{THOooNDQSooOUWQrK} assure que $\|A\|_2=\sqrt{\rho(A{^t}A)}$.
        \item
            Pour toute norme algébrique nous avons \( \rho(A)\leq \| A \|\), proposition \ref{PropEDvSQsA}\ref{ITEMooVQQFooWrHWeO}.
        \item
            Dans le cadre du conditionnement de matrice. Voir en particulier la proposition \ref{PROPooNUAUooIbVgcN} qui utilise le théorème \ref{THOooNDQSooOUWQrK}.
    \end{enumerate}

  % norme matricielle et rayon spectral

\InternalLinks{rang}
    \begin{enumerate}
        \item Définition~\ref{DefALUAooSPcmyK}.
        \item Le théorème du rang, théorème~\ref{ThoGkkffA}
        \item Prouver que des matrices sont équivalentes et les mettre sous des formes canoniques, lemme~\ref{LemZMxxnfM} et son corolaire~\ref{CorGOUYooErfOIe}.
        \item Tout hyperplan de \( \eM(n,\eK)\) coupe \( \GL(n,\eK)\), corolaire~\ref{CorGOUYooErfOIe}. Cela utilise la forme canonique sus-mentionnée.
        \item Le lien entre application duale et orthogonal de la proposition~\ref{PropWOPIooBHFDdP} utilise la notion de rang.
        \item Le lemme \ref{LEMooDFFDooJTQkRu} parle de commutant et utilise la notion de rang. Ce lemme sert à prouver diverses conditions équivalentes à être un endomorphisme cyclique dans le théorème \ref{THOooGLMSooYewNxW}.
        \end{enumerate}
  % rang

\InternalLinks{extension de corps et polynômes}
    \begin{enumerate}
        \item
            Définition d'une extension de corps \ref{DEFooFLJJooGJYDOe}.
        \item
            Pour l'extension du corps de base d'un espace vectoriel et les propriétés d'extension des applications linéaires, voir la section \ref{SECooAUOWooNdYTZf}.
        \item
            Extension de corps de base et similitude d'application linéaire (ou de matrices, c'est la même chose), théorème \ref{THOooHUFBooReKZWG}.
        \item
            Extension de corps de base et cyclicité des applications linéaires, corollaire \ref{CORooAKQEooSliXPp}.
        \item 
            À propos d'extensions de \( \eQ\), le lemme \ref{LemSoXCQH}.
    \end{enumerate}

  % extension de corps et polynômes

\InternalLinks{décomposition de matrices}   \label{DECooWTAIooNkZAFg}
\begin{enumerate}
    \item 
        Décomposition de Bruhat, théorème \ref{ThoizlYJO}.
    \item 
        Décomposition de Dunford, théorème \ref{ThoRURcpW}. 
    \item 
        Décomposition polaire \ref{ThoLHebUAU}.
\end{enumerate}

  % décomposition de matrices

\InternalLinks{systèmes d'équations linéaires}
\begin{itemize}
    \item Algorithme des facteurs invariants \ref{PropPDfCqee}.
    \item Méthode du gradient à pas optimal \ref{PropSOOooGoMOxG}.
\end{itemize}

  % systèmes d'équations linéaires


\InternalLinks{racines de polynôme et factorisation de polynômes}
    \begin{enumerate}
        \item
            Si \( \eA\) est une anneau, la proposition \ref{PropHSQooASRbeA} factorise une racine.
        \item
            Si \( \eA\) est un anneau, la proposition \ref{PropahQQpA} factorise une racine avec sa multiplicité.
        \item
            Si \( \eA\) est un anneau, le théorème \ref{ThoSVZooMpNANi} factorise plusieurs racines avec leurs multiplicités.
        \item
            Si \( \eK\) est un corps et \( \alpha\) une racine dans une extension, le polynôme minimal de \( \alpha\) divise tout polynôme annulateur par la proposition \ref{PropXULooPCusvE}.
        \item
            Le théorème \ref{ThoLXTooNaUAKR} annule un polynôme de degré \( n\) ayant \( n+1\) racines distinctes.
        \item
            La proposition \ref{PropTETooGuBYQf} nous annule un polynôme à plusieurs variables lorsqu'il a trop de racines.
        \item
            En analyse complexe, le principe des zéros isolés \ref{ThoukDPBX} annule en gros toute série entière possédant un zéro non isolé.
        \item 
            Polynômes irréductibles sur \( \eF_q\).
        \end{enumerate}
  % racines de polynôme et factorisation de polynômes

\InternalLinks{formes bilinéaires et quadratiques}      \label{THEMEooOAJKooEvcCVn}
    \begin{enumerate}
\item
    Les formes bilinéaires sont définies en~\ref{DEFooEEQGooNiPjHz}.
\item
    Forme quadratique, définition~\ref{DefBSIoouvuKR}
\item
    Une isométrie d'une forme bilinéaire est affine ou linéaire, théorème \ref{ThoDsFErq}.
\item
    Forme bilinéaire dégénérée, définition \ref{DEFooNUBFooLfCqaK}.
\item
    Une forme bilinéaire est non-dénénérée si et seulement si sa matrice associée est inversible, c'est la proposition \ref{PROPooQHHPooSqpgcb}.
\item
    Une isométrie d'une forme bilinéaire est linéaire ou affine par le théorème \ref{ThoDsFErq}.
\end{enumerate}
  % formes bilinéaires et quadratiques


\InternalLinks{théorème de Bézout}
    \begin{enumerate}
        \item
            Pour \( \eZ^*\) c'est le théorème \ref{ThoBuNjam}.
        \item
            Théorème de Bézout dans un anneau principal : corollaire \ref{CorimHyXy}.
        \item
            Théorème de Bézout dans un anneau de polynômes : théorème \ref{ThoBezoutOuGmLB}.
        \item
            En parlant des racines de l'unité et des générateurs du groupe unitaire dans le lemme \ref{LemcFTNMa}. Au passage nous y parlerons de solfège.
        \end{enumerate}

  % théorème de Bézout

% ENDOMORPHISMES
\input{28_theme}  % invariants de similitude
\input{19_theme}  % diagonalisation

\InternalLinks{endomorphismes cycliques}
    \begin{enumerate}
        \item
            Définition \ref{DEFooFEIFooNSGhQE}.
        \item
            Son lien avec le commutant donné dans la proposition \ref{PropooQALUooTluDif} et le théorème \ref{THOooGLMSooYewNxW}.
        \item
            Utilisation dans le théorème de Frobenius (invariants de similitude), théorème \ref{THOooDOWUooOzxzxm}.
        \end{enumerate}

  % endomorphismes cycliques

\InternalLinks{déterminant}
    \begin{enumerate}
    \item
        Les \( n\)-formes alternées forment un espace de dimension \( 1\), proposition \ref{ProprbjihK}.
    \item
        Déterminant d'une famille de vecteurs \ref{DEFooODDFooSNahPb}.
    \item
        Déterminant d'un endomorphisme \ref{DefCOZEooGhRfxA}.
        \item
            Des interprétations géométriques du déterminant sont dans la section \ref{SECooSQRDooGifgQi}.
        \item
            Le déterminant de Vandermonde est à la proposition \ref{PropnuUvtj}. Il est utilisé à divers endroits :
\begin{enumerate}
    \item
        Pour prouver que \( \tr(u^p)=0\) pour tout \( p\) si et seulement si \( u\) est nilpotente (lemme \ref{LemzgNOjY}).
    \item
        Pour prouver qu'un endomorphisme possédant \( \dim(E)\) valeurs propres distinctes est cyclique (proposition \ref{PropooQALUooTluDif}).
\end{enumerate}

   \end{enumerate}

  % déterminant
\input{33_theme}  % polynôme d'endomorphismes
\InternalLinks{exponentielle de matrice}
\begin{enumerate}
    \item
            Le lemme à propos d'exponentielle de matrice \ref{LemQEARooLRXEef} donne :
            \begin{equation}
                \|  e^{tA} \|\leq P\big( | t | \big)\sum_{i=1}^r e^{t\real(\lambda_i)}.
            \end{equation}
        \item
    La proposition \label{PropCOMNooIErskN} : si \( A\in \eM(n,\eR)\) a un polynôme caractéristique scindé, alors \( A\) est diagonalisable si et seulement si \( e^A\) est diagonalisable.
\item
    La section \ref{subsecXNcaQfZ} parle des fonctions exponentielle et logarithme pour les matrices. Entre autres la dérivation et les séries.
\item
    Pour résoudre des équations différentielles linéaires : sous-section \ref{SUBSECooMDKIooKaaKlZ}.
\item
    La proposition \ref{PropKKdmnkD} dit que l'exponentielle est surjective sur \( \GL(n,\eC)\).
\item

La proposition \label{PropFMqsIE} : si \( u\) est un endomorphisme, alors \( \exp(u)\) est un polynôme en \( u\).
\item
    Calcul effectif : sous-section \ref{SUBSECooGAHVooBRUFub}.
\item Proposition \ref{PROPooZUHOooQBwfZq} : si \( A\in\eM(n,\eC)\) alors $ e^{\tr(A)}=\det( e^{A}).$
\end{proposition}


\end{enumerate}
  % exponentielle de matrice

% GÉOMÉTRIE

% GROUPES
\input{20_theme}  % sous-groupes

\InternalLinks{action de groupe}
    \begin{enumerate}
    \item Action du groupe modulaire sur le demi-plan de Poincaré, théorème \ref{ThoItqXCm}. 
    \end{enumerate}

  % action de groupe

\InternalLinks{classification de groupes}
\begin{enumerate}
    \item Structure des groupes d'ordre \( pq\), théorème \ref{ThoLnTMBy}.
    \item Le groupe alterné est simple, théorème \ref{ThoURfSUXP}.
    \item Définition \ref{DEFooPRCHooVZdwST} d'un \( p\)-groupe.
    \item Théorème de Sylow \ref{ThoUkPDXf}. Tout le théorème, c'est un peu long. On peut se contenter de la partie qui dit que \( G\) contient un \( p\)-Sylow.
    \item Théorème de Burnside sur les sous groupes d'exposant fini de \( \GL(n,\eC)\), théorème \ref{ThooJLTit}.
    \item \( (\eZ/p\eZ)^*\simeq \eZ/(p-1)\eZ\), corollaire \ref{CorpRUndR}.
\end{enumerate}

  % classification de groupes
\input{7_theme}  % produit semi-direct de groupes

\InternalLinks{théorie des représentations}
\begin{enumerate}
    \item Table des caractères du groupe diédral, section~\ref{SecWMzheKf}.
    \item Table des caractères du groupe symétrique \( S_4\), section~\ref{SecUMIgTmO}.
\end{enumerate}
  % théorie des représentations

\InternalLinks{isométries}      \label{THMooVUCLooCrdbxm}

Ne pas confondre une isométrie d'un espace affine avec une isométrie d'un espace euclidien. Les isométrie d'un espace euclidien préservent le produit scalaire et fixent donc l'origine (lemme~\ref{LEMooYXJZooWKRFRu}). Les isométrie des espaces affines par contre conservent les distances (définition~\ref{DEFooZGKBooGgjkgs}) et peuvent donc déplacer l'origine de l'espace vectoriel sur lequel il est modelé; typiquement les translations sont des isométries de l'espace affine mais pas de l'espace euclidien.

Parfois, lorsqu'on coupe les cheveux en quatre, il faut faire attention en parlant de \( \eR^n\) : soit on en parle comme d'un espace métrique (muni de la distance), soit on en parle comme d'un espace normé (muni de la norme ou du produit scalaire).

\begin{enumerate}
    \item
        Définition d'une isométrie pour une forme bilinéaire,~\ref{DEFooGGTYooXsHIZj}. Pour une forme quadratique : définition~\ref{DEFooECTUooRxBhHf}.
    \item
        Définition du groupe orthogonal~\ref{DEFooUHANooLVBVID}, et le spécial orthogonal \( \SO(n)\) en la définition~\ref{DEFooJLNQooBKTYUy}. Le groupe \( \SO(2)\) est le groupe des rotations, par corollaire~\ref{CORooVYUJooDbkIFY}.
    \item
        Le lemme~\ref{LEMooHRESooQTrpMz} donne à toute rotation une matrice de la forme connue. C'est autour de cela que nous définissons les angles.
    \item
        Le groupe orthogonal est le groupe des isométries de \( \eR^n\), proposition~\ref{PropKBCXooOuEZcS}.
    \item
        Les isométries de l'espace euclidien sont affines,~\ref{ThoDsFErq}.
    \item
        Les isométries de l'espace euclidien comme produit semi-direct : $\Isom(\eR^n)\simeq T(n)\times_{\rho}\gO(n)$, théorème~\ref{THOooQJSRooMrqQct}.
    \item
        Isométries du cube, section~\ref{SecPVCmkxM}.
    \item
        Générateurs du groupe diédral, proposition~\ref{PropLDIPoZ}.
\end{enumerate}
  % isométries


% PROBA-STAT
\input{57_theme}  % caractérisation de distributions en probabilités
\input{54_theme}  % théorème central limite

\InternalLinks{lemme de transfert}      \label{THEMEooJREIooKEdMOl}
Il s'agit du résultat \( \hat{f'}=i\xi \hat{f}\).
\begin{enumerate}
    \item
        Lemme \ref{LemQPVQjCx} sur \( \swS(\eR^d)\)
    \item
        Lemme \ref{LEMooAGBZooWCbPDd} pour \( L^2\).
\end{enumerate}

  % lemme de transfert


\InternalLinks{probabilités et espérances conditionnelles}

    Les deux définitions de base, sur lesquelles se basent toutes les choses conditionnelles sont :
    \begin{itemize}
        \item La probabilité conditionnelle d'un événement en sachant un autre : \( P(A|B)\) de la définition~\ref{DEFooGJVHooVbhVYv}.
        \item L'espérance conditionnelle d'une variable aléatoire sachant une tribu : \( E(X|\tribF)\) de la définition~\ref{ThoMWfDPQ}.
    \end{itemize}

    Les autres sont listées ci-dessous.
\begin{description}

    \item[La probabilité conditionnelle d'un événement par rapport à un autre] donnée dans la proposition~\ref{DEFooGJVHooVbhVYv} est le nombre
\begin{equation}
    P(A|B)=\frac{ P(A\cap B) }{ P(B) }
\end{equation}

\item[La probabilité conditionnelle d'un événement vis-à-vis d'une variable aléatoire discrète] est par la définition~\ref{DEFooFRLFooNvXuPK} la variable aléatoire donnée par
\begin{equation}
    P(A|X)(\omega)=P(A|X=X(\omega)).
\end{equation}
Dans le cas continu, c'est la définition~\ref{DEFooIUJMooBAVtMW} :
\begin{equation}
    P(A|X)=P(A|\sigma(X))=E(\mtu_A|\sigma(X)).
\end{equation}

\item[L'espérance conditionnelle d'une variable aléatoire par rapport à une tribu] \( E(X|\tribF)\) est la variable aléatoire \( \tribF\)-mesurable telle que
\begin{equation}
    \int_BE(X|\tribF)=\int_BX
\end{equation}
pour tout \( X\in \tribF\). Si \( X\in L^2(\Omega,\tribA,P)\) alors \( E(X|\tribF)=\pr_K(X)\) où \( K\) est le sous-ensemble de \( L^2(\Omega,\tribA,P)\) des fonctions \( \tribF\)-mesurables (théorème~\ref{ThoMWfDPQ}). Cela au sens des projections orthogonales.

\item[La probabilité conditionnelle d'un événement par rapport à une tribu] est la variable aléatoire
\begin{equation}
    P(A|\tribF)=E(\mtu_A|\tribF).
\end{equation}

\item[L'espérance conditionnelle d'une variable aléatoire par rapport à une autre] de la définition~\ref{DefooKIHPooMhvirn} est une variation sur le thème :
\begin{equation}
    E(X|Y)=E(X|\sigma(Y)),
\end{equation}

%TODO : mettre cette définition à côté de celle du conditionnement par rapport à la tribu.

\end{description}

Notons que partout, si \( X\) est une variable aléatoire, la notation «sachant \( X\)» est un raccourcis pour dire «sachant la tribu engendrée par \( X\)».
  % probabilités et espérances conditionnelles



\input{41_theme}  % dénombrements
\input{6_theme}  % enveloppes
\input{10_theme}  % équations diophantiennes


\immediate\closeout\themetoc

