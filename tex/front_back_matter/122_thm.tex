\InternalLinks{permuter des limites}        \label{THEMEooJGEHooNzQkMT}
\begin{description}
	\item[Permuter des dérivées partielles]
		Pour permuter les dérivées partielles, la clef est d'être \( C^k\).
		\begin{enumerate}
			\item
			      Si une fonction est de classe \( C^2\), le théorème de Schwarz \ref{Schwarz} dit que
			      \begin{equation}
				      \partial_k\partial_lf=\partial_l\partial_kf.
			      \end{equation}
			\item
			      Pour une fonction de classe \( C^m\), on peut permuter \( m\) dérivées partielles avec la propostion \ref{PROPooYGJDooOqibbh}.
		\end{enumerate}
	\item[Fonctions définies par une intégrale]
		Les théorèmes sur les fonctions définies par une intégrale, section~\ref{SecCHwnBDj}. Nous avons entre autres
		\begin{enumerate}
			\item
			      \( \partial_i\int_Bf=\int_B\partial_if\), avec \( B\) compact, proposition~\ref{PropDerrSSIntegraleDSD}.
			\item
			      Si \( f\) est majorée par une fonction ne dépendant pas de \( x\), nous avons le théorème~\ref{ThoKnuSNd} pour la continuité de \( x\mapsto \int_{\Omega}f(x,\omega)d\mu(\omega)\).
			\item
			      Pour la fonction \( F(x)=\int_{\Omega}f(x,\omega)d\mu(\omega)\), nous avons la dérivation sous l'intégrale par la formule de Leibniz
			      \begin{equation}
				      F'(a)=\int_{\Omega}\frac{ \partial f }{ \partial x }(a,\omega)d\mu(\omega)
			      \end{equation}
			      démontrée en le théorème \ref{ThoMWpRKYp} ou \ref{ThoDerSousIntegrale}.

			      Des variations avec des dérivées partielles et des différentielles sont dans \ref{PropDerrSSIntegraleDSD} et dans \ref{PropAOZkDsh}.
			\item
			      Si \( f\colon \eC\times \Omega\to \eC\) est holomorphe (pour \( \eC\)), alors \( F\) est holomorphe et
			      \begin{equation}
				      F'(z)=\int_{\Omega}\frac{ \partial f }{ \partial z }(z,\omega)d\mu(\omega).
			      \end{equation}
			\item
			      Pour des dérivées partielles multiples, nous avons la formule
			      \begin{equation}
				      (\partial^{\alpha}F)(x)=\int_{\Omega}(\partial^{\alpha}f_{\omega})(a)d\mu(\omega)
			      \end{equation}
			      dans la proposition \ref{PROPooJKXJooLxgEGd}.
			\item
			      Si l'intégrale est uniformément convergente, nous avons le théorème~\ref{ThotexmgE} qui donne la continuité de \( F(x)=\int_{\Omega}f(x,\omega)d\mu(\omega)\).
			\item
			      Pour dériver \( \int_Bg(t,z)dt\) avec \( B\) compact dans \( \eR\) et \( g\colon \eR\times \eC\to \eC\), il faut aller voir la proposition~\ref{PROPooZCLYooUaSMWA}.
			\item
			      En ce qui concerne le \( x\) dans la borne, le théorème \ref{PropEZFRsMj} lie primitive et intégrale en montrant que \( F(x)=\int_a^xf(t)dt\) est une primitive de \( f\) (sous certaines conditions). Le théorème fondamental de l'analyse \ref{ThoRWXooTqHGbC} en est une conséquence.
			\item Si \( T\) est une \textbf{distribution}, alors nous avons
			      \begin{equation}
				      T\big( x\mapsto (\partial_y^{\alpha}\phi)(x,y_0) \big)=\partial_y^{\alpha}\Big( T\big( x\mapsto \phi(x,y) \big) \Big)_{y=y_0}.
			      \end{equation}
			      C'est la proposition \ref{PROPooCNYTooWCKHpV}.
		\end{enumerate}
	\item[Dérivée et intégrale]
		\begin{enumerate}
			\item
			      Dériver \( \int_{\Omega}f(x,\omega)d\omega\), théorème \ref{ThoDerSousIntegrale}.
		\end{enumerate}
	\item[Limite et intégrale]
		\begin{enumerate}
			\item
			      Théorème de la convergence monotone, théorème~\ref{ThoRRDooFUvEAN}.
			\item
			      \( \int \sum_n f_n=\sum_n\int f_n\) dans le corolaire \ref{CorNKXwhdz}.
		\end{enumerate}
	\item[Fubini]
		Le théorème de Fubini permet non seulement de permuter des intégrales, mais également des sommes parce que ces dernières peuvent être vues comme des intégrales sur \( \eN\) muni de la tribu des parties et de la mesure de comptage\footnote{Mesure de comptage, définition \ref{DEFooILJRooByDzhs}.}. Nous utilisons cette technique pour permuter une somme et une intégrale dans l'équation \eqref{EQooWOLOooFHSrsx}.
	\item
	      L'utilisation de Fubini pour permuter des intégrales (sur deux variables différentes) ou deux sommes est expliquée dans \ref{NORMooKIRJooPvyPWQ}.

	      C'est par exemple utilisé pour permuter deux sommes dans le cadre des chaines de Markov en \ref{LEMooZIEPooXHGnvy}.
	      \begin{itemize}
		      \item
		            le théorème de Fubini-Tonelli~\ref{ThoWTMSthY} demande que la fonction soit mesurable et positive;
		      \item
		            le théorème de Fubini~\ref{ThoFubinioYLtPI} demande que la fonction soit intégrable (mais pas spécialement positive);
		      \item
		            le corolaire~\ref{CorTKZKwP} demande l'intégrabilité de la valeur absolue des intégrales partielles pour déduire que la fonction elle-même est intégrable.
	      \end{itemize}

	\item[Limite et dérivées, différentielle]
		\begin{enumerate}
			\item
			      Permuter limite et dérivée dans le cas \( \eR\to \eR\), théorème \ref{THOooXZQCooSRteSr}.
			\item
			      Permuter limite et dérivées partielles, théorème \ref{ThoSerUnifDerr}.
			\item
			      Permuter limite et différentielle, théorème \ref{ThoLDpRmXQ}.
		\end{enumerate}
		Quelques remarques sur les techniques de démonstration.
		\begin{enumerate}
			\item
			      Le résultat fondamental \ref{THOooXZQCooSRteSr} est démontré sans recourir à des intégrales. Une preuve alternative, plus courte, avec des intégrales est donnée en \ref{NORMALooGYUEooKrYjyz}.
			\item
			      Permuter limite et dérivée partielle, théorème \ref{ThoSerUnifDerr}.
			\item
			      Permuter série et différentielle, théorème \ref{ThoLDpRmXQ}.
		\end{enumerate}
	\item[Somme et dérivée]
		Permuter somme et différentielle, théorème \ref{ThoLDpRmXQ}.
	\item[Limite et mesure]
		Une mesure n'est pas toujours une limite, mais la définition d'une mesure positive sur un espace mesurable parle de permuter limite et mesure : définition \ref{DefBTsgznn}\ref{ItemQFjtOjXiii}.
	\item[Norme et intégrale]
		L'inégalité de Minkowski \eqref{EQooRAIDooBhDMPe} permute la norme \( L^p\) et l'intégrale. Il n'y a cependant pas vraiment permutation, mais inégalité. C'est déjà mieux que rien.
	\item[Distribution et intégrale] C'est le théorème \ref{THOooKVEHooIBaZGR}.
\end{description}

Le corolaire \ref{CORooSGMXooCNfGNk} dit que \( \lim_{h\to 0}\int_a^{a+h}f=0\).
