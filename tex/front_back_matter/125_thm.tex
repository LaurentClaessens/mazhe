\InternalLinks{connexité}
\begin{enumerate}
	\item
	      Définition~\ref{DefIRKNooJJlmiD}
	\item
	      L'image d'un connexe par une fonction continue est connexe, lemme \ref{LemConncontconn}.
	\item
	      Connexité par arcs, définition \ref{DEFooOXVCooBizpgK}.
	\item
	      Si \( U\) est connexe et si \( U\subset S\subset \bar U\), alors \( S\) est connexe, proposition \ref{PROPooSCKNooRbewdv}.
	\item
	      Une partie de \( \eR^2\) qui est connexe, mais pas connexe par arcs, proposition \ref{PROPooVXDNooPZYKPr}.
	\item
	      Une partie de \( \eR\) est connexe si et seulement si elle est un intervalle, proposition \ref{PropInterssiConn}.
	\item
	      Le groupe \( \SL(n,\eK)\) est connexe par arcs, proposition~\ref{PROPooALQCooLZCKrH}.
	\item
	      Le groupe \( \GL(n,\eC)\) est connexe par arcs, proposition~\ref{PROPooVJNIooMByUJQ}.
	\item
	      Le groupe \( \GL(n,\eR)\) a exactement deux composantes connexes par arcs, proposition~\ref{PROPooBIYQooWLndSW}.
	\item
	      Le groupe \( \gO(n,\eR)\) n'est pas connexe, lemme~\ref{LEMooIPOVooZJyNoH}.
	\item
	      Les groupes \( \gU(n)\) et \( \SU(n)\) sont connexes par arcs, lemme~\ref{LEMooQMXHooZQozMK}.
	\item
	      Pour tout \( n\geq 2\), le groupe \( \SO(n)\) est connexe, le groupe \( \gO(n)\) a deux composantes connexes, proposition \ref{THOooYQFNooPaYmaP}.
	\item
	      Connexité des formes quadratiques de signature donnée, proposition~\ref{PropNPbnsMd}.
	\item
	      Dans un espace vectoriel normé, les connexes par arcs sont connexes par arcs \( C^1\), proposition \ref{PROPooGRXXooWcZyJG}.
\end{enumerate}
