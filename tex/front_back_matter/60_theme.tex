\InternalLinks{suites et séries}

\begin{description}
    \item[Suites] 
        Les suites réelles sont en général dans la proposition \ref{PropLimiteSuiteNum} et ce qui s'ensuit. Cette proposition est souvent prise comme définition lorsque seules les suites réelles ne sont considérées.
        \begin{enumerate}
    \item
        Les suites adjacentes, c'est la définition \ref{DEFooDMZLooDtNPmu}. 
    \item
        Les séries alternées, théorème \ref{THOooOHANooHYfkII}. Il s'agit de dire que \( \sum_{k=0}^{\infty}(-1)^ka_k\) converge quand \( a_k\) est décroissante et tend vers zéro.
    \item
        Le concept de suite adjacente sert à étudier la série de Taylor de \( \ln(x+1)\), voir le lemme \ref{LEMooWMGGooRpAxBa} et ce qui l'entoure.
    \item
        La définition de la convergence absolue est la définition~\ref{DefVFUIXwU}.
            \item
                Une suite réelle croissante et majorée converge, proposition \ref{LemSuiteCrBorncv}.
            \item
                Toute suite dans un compact admet une sous-suite convergente, théorème \ref{THOooRDYOooJHLfGq}.
            \item
                Pour tout réel, il existe une suite croissante de rationnels qui y converge, proposition \ref{PropSLCUooUFgiSR}.
        \end{enumerate}
    \item[Série] 
        Les séries sont en général dans la section \ref{SECooYCQBooSZNXhd}.
        \begin{enumerate}
    \item
        Quelques séries usuelles en \ref{SUBSECooDTYHooZjXXJf} : série harmonique, géométrique, de Riemann, et la mythique arithmético-géométrique.
        \begin{enumerate}
            \item
                La série harmonique diverge : \( \sum_k\frac{1}{ k }=\infty\), exemple \ref{EXooDVQZooEZGoiG}.
            \item
                La série géométrique : \( \sum_{k=0}^Nq^k=\frac{ 1-q^{N+1} }{ 1-q }\), exemple \ref{ExZMhWtJS}.
            \item
                Une autre cool série : \( \sum_{k=1}^N\frac{ 1 }{ k(k+1) }=\frac{ N }{ N+1 }\), lemme \ref{LEMooKDHPooPlFTIT}.
        \end{enumerate}
    \item
        Critère des séries alternées, théorème \ref{THOooOHANooHYfkII}.
    \item
        Convergence d'une série implique convergence vers zéro du terme général, proposition~\ref{PROPooYDFUooTGnYQg}.
        \end{enumerate}
    \item[Sommes infinies]
        Nous pouvons dire plusieurs choses à propos d'une somme infinie.% cette phrase est là pour le mot-clef ``somme infinie''.
        \begin{enumerate}
            \item
Une somme indexée par un ensemble quelconque est la définition~\ref{DefHYgkkA}.
    \item
        La définition de la somme d'une infinité de termes est donnée par la définition~\ref{DefGFHAaOL}.
  \item
      si la série converge, on peut regrouper ses termes sans modifier la convergence ni la somme (associativité);
    Pour les sommes infinies l'associativité et la commutativité dans une série sont perdues. Néanmoins, il subsiste que
  \begin{enumerate}
  \item
      si la série converge absolument, on peut modifier l'ordre des termes sans modifier la convergence ni la somme (commutativité, proposition~\ref{PopriXWvIY}).
  \end{enumerate}
  \item Permuter une somme infinie avec une application linéaire : \( f(\sum_{i\in I}v_i)=\sum_{i\in I}f(v_i)\), c'est la proposition \ref{PROPooWLEDooJogXpQ}.
        \end{enumerate}
\end{description}
