\InternalLinks{déduire la nullité d'une fonction depuis son intégrale}		\label{THEMEooDASVooWZLOjw}
Des résultats qui disent que si \( \int f=0\) c'est que \( f=0\) dans un sens ou dans un autre.
\begin{enumerate}
	\item
	      Il y a le lemme~\ref{Lemfobnwt} qui dit ça.
	\item
	      Un lemme du genre dans \( L^2\) existe aussi pour \( \int f\varphi=0\) pour tout \( \varphi\). C'est le lemme~\ref{LemDQEKNNf}.
	\item
	      En utilisant le théorème de représentation de Riesz, on peut prouver que \( \int_{\Omega}f\chi=0\) implique \( f=0\) pour tout \( f\in L^p\), proposition \ref{PropUKLZZZh}.
	\item
	      Si \( \int f\chi=0\) pour tout \( \chi\) à support compact alors \( f=0\) presque partout, proposition~\ref{PropAAjSURG}.
	\item
	      Si \( \int_a^xf=0\) pour tout \( x\), alors \( f=0\), proposition \ref{PROPooZOJHooKwZOFW}.
	\item
	      La proposition~\ref{PropRERZooYcEchc} donne \( f=0\) dans \( L^p\) lorsque \( \int fg=0\) pour tout \( g\in L^q\).
	\item
	      Une fonction \( h\in C^{\infty}_c(I)\) admet une primitive dans \(  C^{\infty}_c(I)\) si et seulement si \( \int_Ih=0\). Théorème~\ref{PropHFWNpRb}.
\end{enumerate}

Dans le même ordre d'idées, si \( f>0\) et si \( \mu(X)>0\) alors \( \int_Xf>0\) par le lemme \ref{LEMooLXVGooUDuQzc}.
