\InternalLinks{sommation finie et infinie}          \label{THEMEooMKLBooLGFCdx}
La définition du symbole \( \sum_{i\in I}\) se fait en trois étapes et deux demi, chacune se basant sur la précédente.
\begin{enumerate}
	\item
	      Si \( (A,+)\) est un ensemble avec une loi de composition interne, \( \sum_{i=0}^na_i\) est en \ref{DEFooNEVNooJlmJOC}.
	\item
	      Si \( I\) est fini et si est à valeur dans un groupe commutatif, \( \sum_{i\in I}f(i)\) est \ref{DEFooLNEXooYMQjRo}.
	\item
	      Enfin si \( I\) est un ensemble quelconque, la définition \ref{DefIkoheE} introduit la notion de famille sommable dans un espace vectoriel normé.
	      %TODOooPEXGooQmUzXc prouver que les définitions se réduisent à la précédente comme cas particulier.
	\item
	      Si \( (a_k)\) est une suite dans un espace vectoriel normé, la somme \( \sum_{k=0}^{\infty}a_k\) est avec les sommes partielles dans la définition \ref{DefGFHAaOL}.
	\item
	      La somme au sens de Cesàro est la somme des moyennes partielles, définition \ref{DEFooLVRLooTeowkn}.
\end{enumerate}
Notez que \( \sum_{k=0}^{\infty}a_k\) n'est pas un cas particulier de \( \sum_{k\in \eN}a_k\). Une différence de taille entre les deux est que pour que \( \sum_{k=0}^{\infty}a_k\) existe, il suffit que les \( a_k\) puissent être sommés dans cet ordre. À l'inverse pour que \( \sum_{k\in \eN}a_k\) existe, il faut que l'ordre de sommation puisse être arbitraire.

Si vous voulez sommer des séries encore moins convergentes, vous pouvez avoir envie d'utiliser la supersomme\cite{BIBooUCSPooNKNWEK}, ou la \( \eta\)-régularisation\cite{BIBooLFLMooHFZFHu}.
