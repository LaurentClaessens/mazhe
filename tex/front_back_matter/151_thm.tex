\InternalLinks{arithmétique modulo, théorème de Bézout, théorème chinois} \label{THEMEooNRZHooYuuHyt}
\begin{enumerate}
	\item
	      Pour \( \eZ^*\) c'est le théorème~\ref{ThoBuNjam}.
	\item
	      Existence d'un pgcd et d'un ppcm pour tout partie d'un anneau principal, proposition \ref{PROPooPZXNooNpVZCm}.
	\item
	      Théorème de Bézout dans un anneau principal : proposition \ref{PROPooXQKMooWJlEFq} qui dit que si \( \delta\in \pgcd(a,b) \) alors il existe \( u,v\) tels que \( ua+vb=\delta\)
	\item
	      Théorème de Bézout dans un anneau de polynômes : théorème~\ref{ThoBezoutOuGmLB}.
	\item
	      En parlant des racines de l'unité et des générateurs du groupe unitaire dans le lemme~\ref{LemcFTNMa}. Au passage nous y parlerons de solfège.
	\item
	      La proposition~\ref{PropLAbRSE} qui donne tout entier assez grand comme combinaisons de \( a \) et \( b\) à coefficients positifs est utilisée en chaines de Markov, voir la définition~\ref{DefCxvOaT} et ce qui suit.
	\item
	      PGCD et PPCM sont dans la définition \ref{DefrYwbct}.
	\item
	      Calcul effectif du PGCD puis des coefficients de Bézout : sous-sections~\ref{SUBSECooAEBLooFGJRkg} et~\ref{SUBSECooRHSQooEuBWbd}.
	\item
	      Dans un anneau principal il y a une relation entre idéal engendré et pgcd, ppcm : \( (a)+(b)=\pgcd(a,g)\) et \( (a)\cap(b)=\ppcm(a,b)\). C'est la proposition \ref{PROPooYTMYooEYxuQc}.
	\item
	      Il existe beaucoup de théorèmes chinois. Dans un anneau principal, si \( a=a_1\ldots a_r\) alors on a isomorphisme \( A/(a)\simeq A/(a_1)\times \ldots\times A/(a_r)\) par le théorème \ref{THOooLQDNooZGaVfB}.
	\item
	      Si les \( I_i\) sont des idéaux  solutions à l'équation \( x\in[a_i]_{I_i}\) sont donnés par la proposition \ref{PROPooKWZLooCHhSjs}.
\end{enumerate}
