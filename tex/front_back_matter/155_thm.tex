
\InternalLinks{réduction, diagonalisation}	\label{THEMEooDiagonalisation}
Des résultats qui parlent diagonalisation
\begin{enumerate}
	\item
	      Définition d'un endomorphisme diagonalisable :~\ref{DefCNJqsmo}.
	\item
	      Conditions équivalentes au fait d'être diagonalisable en termes de polynôme minimal, y compris la décomposition en espaces propres : théorème~\ref{ThoDigLEQEXR}.
	\item
	      Diagonalisation simultanée~\ref{PropGqhAMei}, pseudo-diagonalisation simultanée~\ref{CorNHKnLVA}.
	\item
	      Diagonalisation d'exponentielle~\ref{PropCOMNooIErskN} utilisant la décomposition de Dunford.
	\item
	      Décomposition polaire théorème~\ref{ThoLHebUAU}. \( M=SQ\), \( S\) est symétrique, réelle, définie positive, \( Q\) est orthogonale.
	\item
	      Décomposition de Dunford~\ref{ThoRURcpW}. \( u=s+n\) où \( s\) est diagonalisable et \( n\) est nilpotent, \( [s,n]=0\).
	\item
	      Réduction de Jordan (bloc-diagonale)~\ref{ThoGGMYooPzMVpe}.
	\item
	      L'algorithme des facteurs invariants~\ref{PropPDfCqee} donne \( U=PDQ\) avec \( P\) et \( Q\) inversibles, \( D\) diagonale, sans hypothèse sur \( U\). De plus les éléments de \( D\) forment une chaine d'éléments qui se divisent l'un l'autre.
\end{enumerate}
Le théorème spectral et ses variantes :
\begin{enumerate}
	\item
	      Théorème spectral, matrice symétrique, théorème~\ref{ThoeTMXla}. Via le lemme de Schur complexe \ref{LemSchurComplHAftTq}.
	\item
	      Théorème spectral autoadjoint (c'est le même, mais vu sans matrices), théorème~\ref{ThoRSBahHH}
	\item
	      Théorème spectral hermitien, lemme~\ref{LEMooVCEOooIXnTpp}.
	\item
	      Théorème spectral, matrice normales, théorème~\ref{ThogammwA}.
\end{enumerate}
Pour les résultats de décomposition dont une partie est diagonale, voir le thème~\ref{DECooWTAIooNkZAFg} sur les décompositions.
Réduction de quadriques :
\begin{enumerate}
	\item
	      Réduction de Gauss, théorème \ref{THOooOMMFooKxqICS}.
\end{enumerate}

Trigonalisation.
\begin{enumerate}
    \item
        Le lemme de Schur complexe \ref{LemSchurComplHAftTq} dit que toute matrice est unitairement équivalente à une matrice triangulaire supérieure.
\end{enumerate}
