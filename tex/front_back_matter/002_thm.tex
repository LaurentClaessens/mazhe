\InternalLinks{tribu, algèbre de parties, \( \lambda\)-systèmes et co.}  \label{INTooVDSCooHXLLKp}
Il existe des centaines de notions de mesures et de classes de parties.
\begin{enumerate}
	\item
	      Le plus souvent lorsque nous parlons de mesure est que nous parlons de mesure positive, définition~\ref{DefBTsgznn} sur un espace mesuré avec une tribu, définition~\ref{DefjRsGSy}.
	\item
	      Une mesure extérieure est la définition~\ref{DefUMWoolmMaf}
	\item
	      Une algèbre de partie : définition~\ref{DefTCUoogGDud}. Une mesure sur une algèbre de parties : définition~\ref{DefWUPHooEklLmR}. L'intérêt est que si on connait une mesure sur une algèbre de parties, elle se prolonge en une mesure sur la tribu engendrée par le théorème de prolongement de Hahn~\ref{ThoLCQoojiFfZ}.
	\item
	      Un \( \lambda\)-système : définition~\ref{DefRECXooWwYgej}.
	\item
	      Une mesure complexe : définition~\ref{DefGKHLooYjocEt}.
\end{enumerate}

En théorie de l'intégration, si \( X\) est une partie de \( \eR^n\), la convention est de considérer des fonctions
\[
	f\colon \big( X,\Lebesgue(X) \big)\to \big( \eR,\Borelien(\eR) \big).
\]
Voir les points \ref{NORMooNFOMooYnaflN} et \ref{NORMooFZEDooIxSgLe} pour les conventions à ce propos.

À propos d'applications mesurables :
\begin{enumerate}
	\item
	      Définition d'une application mesurable, définition \ref{DefQKjDSeC}.
	\item
	      Une fonction continue est borélienne, théorème \ref{ThoJDOKooKaaiJh}.
	\item
	      Si les \( f_n\) sont mesutables (au sens des boréliens), alors \( \sup_nf_n\) est mesurable, lemme \ref{LemIGKvbNR}.
\end{enumerate}


À propos de tribu induite:
\begin{enumerate}
	\item
	      Définition \ref{DefDHTTooWNoKDP}.
	\item
	      Les boréliens induits sont bien les boréliens de la topologie induite\footnote{Topologie induite, définition \ref{DefVLrgWDB}.} : \( \Borelien(Y)=\Borelien(X)_Y\), théorème \ref{ThoSVTHooChgvYa}.
\end{enumerate}

Tribu engendrée.
\begin{enumerate}
	\item
	      Tribu engendrée par des parties, définition \ref{DEFooJSAKooSDnGzt}.
\end{enumerate}
