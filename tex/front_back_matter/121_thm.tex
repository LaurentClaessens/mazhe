
\InternalLinks{intégration sur des variétés}
\begin{description}
	\item[orientation]
	      La notion d'orientation commence avec l'orientation des bases d'un espace vectoriel et continue jusqu'à orienter des variétés à partir de ses cartes.
	      \begin{enumerate}
		      \item
		            Classe d'orientation sur les bases d'un espace vectoriel, définition \ref{DEFooNVRHooEBHUSu}.
		      \item
		            Orientation sur une surface, définition \ref{DEFooFTQLooXXbtOQ}.
		      \item
		            Variété orientable, définition \ref{DEFooSWREooNdQpdA}.
	      \end{enumerate}
	\item[théorème de Stokes, théorème de Green et compagnie]
	      Tous ces théorèmes sont des conséquences plus ou moins directes de celui de Stokes, et des généralisations du théorème fondamental de l'analyse.
	      \begin{enumerate}
		      \item
		            Forme générale, théorème~\ref{ThoATsPuzF}.
		      \item
		            Rotationnel et circulation, théorème~\ref{THOooIRYTooFEyxif}.
	      \end{enumerate}
	      Le théorème de Stokes peut être utilisé pour montrer le théorème de Brouwer, proposition~\ref{PropDRpYwv}.
\end{description}
