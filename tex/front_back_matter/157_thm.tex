\InternalLinks{déterminant}     \label{THMooUXJMooOroxbI}
\begin{enumerate}
	\item
	      Déterminant d'une matrice : définition \ref{DEFooYCKRooTrajdP}.
	\item
	      Déterminant d'un endomorphisme~\ref{LEMooQTRVooAKzucd}.
	\item
	      Déterminant \( \det_{\varphi}\) d'une application linéaire entre espaces différents, proposition \ref{PROPooQCPJooFSzaPc}.
	\item
	      Le lemme \ref{LEMooEZFIooXyYybe} donne la formule \( \det(f)=\sum_{\sigma\in S_n}\epsilon(\sigma)\prod_{i=1}^n\langle e_{\sigma(i)}, f(e_i)\rangle\).
	\item
	      Principales propriétés algébriques du déterminant : la proposition \ref{PropYQNMooZjlYlA}.
	\item
	      La formule \( \det(AB)=\det(A)\det(B)\) est la proposition \ref{PROPooHQNPooIfPEDH} pour des matrices et la proposition \ref{PropYQNMooZjlYlA}\ref{ItemUPLNooYZMRJy} pour les applications linéaires.
	\item
	      Déterminant et manipulations de lignes et colonnes, section \ref{SUBSECooKMSVooBBHwkH} et les propositions qui précèdent à partir du lemme \ref{LEMooCEQYooYAbctZ} qui dit que \( \det(A)=\det(A^t)\).
	\item
	      Les \( n\)-formes alternées forment un espace de dimension \( 1\), proposition~\ref{ProprbjihK}.
	\item
	      Déterminant d'une famille de vecteurs~\ref{DEFooODDFooSNahPb}.
	\item
	      Calcul d'un déterminant de taille \( 2\times 2\) : équation \eqref{EQooQRGVooChwRMd}.
	\item
	      Interprétations géométriques
	      \begin{enumerate}
		      \item
		            À propos d'orthogonalité, le déterminant est très lié au produit vectoriel en dimension~\( 3\). Et il le généralise en dimension supérieure.
		            \begin{enumerate}
			            \item
			                  Liaison au produit vectoriel (orthogonalité) dans la proposition~\ref{PROPooTUVKooOQXKKl}.
			            \item
			                  En particulier le lemme~\ref{LEMooFRWKooVloCSM} nous dit comment un déterminant donne un vecteur orthogonal à une famille donnée de vecteurs.
		            \end{enumerate}
		      \item
		            Déterminant et aires, volumes
		            \begin{enumerate}
			            \item
			                  Déterminant et mesure de Lebesgue : théorème~\ref{ThoBVIJooMkifod}.
			            \item
			                  Aire du parallélogramme : il y a la formule avec le produit vectoriel dans la proposition~\ref{PropNormeProdVectoabsint}, mais l'aire proprement dite, avec une intégrale est dans la proposition \ref{PROPooAVVNooOOlSzr}.
			            \item
			                  Volume du parallélépipède avec le produit mixte et le déterminant \( 3\times 3\),~\ref{NORMooWWOKooWzScnZ}.
		            \end{enumerate}
	      \end{enumerate}
	      Tant que nous en sommes dans les interprétations géométriques, il faut lier déterminant, produit vectoriel, orthogonalité et mesure en notant que l'élément de volume lors de l'intégration en dimension \( 3\) est donné par \eqref{EQooNYWSooZuvcPe} : \( dS=\| T_u\times T_v \|\) qui est la norme du produit vectoriel des vecteurs tangents au paramétrage.

	      Nous voyons dans l'équation \eqref{EQooARMAooQPhQAL} que l'élément de volume pour une partie de dimension \( n\) dans \( \eR^m\) (à l'occasion d'y intégrer une fonction) est donné par un déterminant mettant en jeu les vecteurs tangents du paramétrage.
	\item
	      Le déterminant de Vandermonde est à la proposition~\ref{PropnuUvtj}. Il est utilisé à divers endroits :
	      \begin{enumerate}
		      \item
		            Pour prouver que \( u\) est nilpotente si et seulement si \( \tr(u^p)=0\) pour tout \( p\) (lemme \ref{LemzgNOjY})
		      \item
		            Pour prouver qu'un endomorphisme possédant \( \dim(E)\) valeurs propres distinctes est cyclique (proposition~\ref{PropooQALUooTluDif}).
	      \end{enumerate}

\end{enumerate}
