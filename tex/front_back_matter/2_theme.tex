
\InternalLinks{formule des accroissements finis}
    Il en existe plusieurs formes :
    \begin{enumerate}
        \item
            Une version adaptée aux espaces de dimension finie est le théorème~\ref{val_medio_2}.
        \item
        L'existence de \( c\in \mathopen] a , b \mathclose[\) tel que
            \begin{equation}
                f'(c)=\frac{ f(b)-f(a) }{ b-a }
            \end{equation}
            est le théorème des accroissements finis proprement dit. C'est le théorème \ref{ThoAccFinis}.
        \item
            Au premier ordre, proposition \ref{PropUTenzfQ}.
        \item
            Pour les fonctions \( \eR\to \eR\) en le théorème~\ref{ThoAccFinis}.
        \item
            Une généralisation pour les intervalles non bornés : théorème~\ref{THOooRIIBooOjkzMa}.
        \item
            Espaces vectoriels normés, théorème~\ref{ThoNAKKght}
    \end{enumerate}
