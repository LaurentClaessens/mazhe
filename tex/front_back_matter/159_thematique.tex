%+++++++++++++++++++++++++++++++++++++++++++++++++++++++++++++++++++++++++++++++++++++++++++++++++++++++++++++++++++++++++++
\section{Conventions sur les matrices et changement de bases}
%+++++++++++++++++++++++++++++++++++++++++++++++++++++++++++++++++++++++++++++++++++++++++++++++++++++++++++++++++++++++++++
\label{SECooBTTTooZZABWA}

%--------------------------------------------------------------------------------------------------------------------------- 
\subsection{Matrices et applications linéaires}
%---------------------------------------------------------------------------------------------------------------------------
\label{SUBSECooAFPDooOzXdGz}

Le lien entre matrice et application linéaire est donné par la définition \ref{DEFooJVOAooUgGKme}. L'application d'une matrice à un vecteur est \eqref{EQooQFVTooMFfzol}. Le lien le plus simple entre l'application linéaire et les éléments de matrice est donné par la proposition \ref{PROPooGXDBooHfKRrv}. Voici les relations :
\begin{subequations}
	\begin{align}
		T_{\alpha i}  & =T(e_i)_{\alpha}                         \\
		T(e_i)        & =\sum_{\alpha}T_{\alpha i}f_{\alpha}     \\
		T(x)          & =\sum_{i\alpha}T_{\alpha i}x_if_{\alpha} \\
		T(x)_{\alpha} & =\sum_{i}T_{\alpha i}x_i.
	\end{align}
\end{subequations}

De la même manière nous utiliserons (rarement) la notation suivante (définition \ref{DEFooRFQCooQrLPVw}) si \( x\in \eR^n\) et \( T\in \eM(n\times n)\):
\begin{equation}
	xT=\sum_{ij}x_iT_{ij}e_j.
\end{equation}
À partir de là, il est possible de parler de vecteur propre à gauche lorsque \( xT=\lambda x\).

Cela définit une application \( \psi\colon \eM(n\times m, \eK)\to \aL(E,F)\) qui a plein de propriétés.
\begin{enumerate}
	\item
	      C'est une bijection, proposition \ref{PROPooGXDBooHfKRrv}\ref{ITEMooHSMLooRJZref}.
	\item
	      C'est un isomorphisme d'algèbre, proposition \ref{PROPooFMBFooEVCLKA}.
	\item
	      C'est un isomorphisme d'espaces vectoriels, proposition \ref{PROPooIYVQooOiuRhX}.
	\item
	      Isomorphisme d'algèbres et d'anneaux, proposition \ref{PROPooFMBFooEVCLKA}.
	\item
	      Isomorphisme d'espaces topologiques, proposition \ref{PROPooFMBFooEVCLKA}.
\end{enumerate}

Lorsque nous avons une base orthonormée\footnote{Définition \ref{DEFooZBWTooIqXwRp}.} nous avons aussi les propositions \ref{PROPooZKWXooWmEzoA} et \ref{PROPooZKWXooWmEzoA} qui donnent des formules avec produit scalaire :
\begin{enumerate}
	\item
	      \( T_{\alpha i}=e_{\alpha}\cdot T(e_i)\)
	\item
	      \( x\cdot Ay=\sum_{kl}A_{kl}x_ky_l\).
\end{enumerate}
où le point est le produit scalaire usuel de \( \eR^n\).

%---------------------------------------------------------------------------------------------------------------------------
\subsection{Le changement de base}
%---------------------------------------------------------------------------------------------------------------------------

Soit un espace vectoriel \( V\) muni de deux bases \( (e_i)_{i=1,\ldots, n}\) et \( (f_{\alpha})_{\alpha=1,\ldots, n}\). Le lemme \ref{LEMooIHZGooOZoYZd} donne le lien entre les vecteurs de base :
\begin{enumerate}
	\item
	      \( f_{\alpha}=\sum_iQ_{i\alpha}e_i\)
	\item
	      \( e_i=\sum_{\alpha}Q^{-1}_{\alpha i}f_{\alpha}\)
\end{enumerate}
La proposition \ref{PROPooNYYOooHqHryX} donne un certain nombre de formules pour les coordonnées des vecteurs :
\begin{enumerate}
	\item
	      \( y_{\alpha}=\sum_iQ^{-1}_{\alpha i}x_i\)
	\item
	      \( x_i=\sum_{\alpha}Q_{i\alpha}y_{\alpha}\).
	\item
	      \( x_i=(Qy)_i\)
	\item
	      \( x=Qy\)
\end{enumerate}

La transformation de la matrice d'une application linéaire lors d'un changement de base est la proposition \ref{PROPooNZBEooWyCXTw}. Soit une application linéaire \( T\colon V\to V\) de matrices \( A\) et \( B\) dans les bases \( \{ e_i \}\) et \( \{ f_{\alpha} \}\). Si les bases sont liées par \( f_{\alpha}=\sum_iQ_{i\alpha}e_i\), alors les matrices \( A\) et \( B\) sont liées par
\begin{equation}
	B=Q^{-1}AQ.
\end{equation}

%---------------------------------------------------------------------------------------------------------------------------
\subsection{Changement de base : matrice d'une forme bilinéaire}
%---------------------------------------------------------------------------------------------------------------------------

La proposition \ref{PROPooLBIOooUpzxXA} fait le changement de matrice d'une forme bilinéaire lors d'un changement de base. Si la matrice de \( q\) dans la base \( \{ e_i \}\) est \( A\) et celle dans la base \( \{ f_{\alpha} \}\) est \( B\), alors
\begin{equation}
	B=Q^tAQ.
\end{equation}
Pour comparaison avec la loi de transformation des matrices des applications linéaires, voir la remarque \ref{REMooNEJLooSqgeih}.

Plus généralement, si \( \phi\) est une application linéaire, la matrice de \( q\circ \phi\) est \( \phi^tq\phi\), proposition \ref{PROPooOKTGooOYukoB}.

%+++++++++++++++++++++++++++++++++++++++++++++++++++++++++++++++++++++++++++++++++++++++++++++++++++++++++++++++++++++++++++ 
\section{Multiindice et liste d'indices}
%+++++++++++++++++++++++++++++++++++++++++++++++++++++++++++++++++++++++++++++++++++++++++++++++++++++++++++++++++++++++++++

\begin{normaltext}      \label{NORMooRRZCooMOKAzY}
	Je crois qu'il y a quelques incohérences de notations/dénominations dans le texte. En principe quand on parle de \( \eR^n\), un \defe{multiindice}{multiindice} \cite{BIBooYDMJooGDtdbo} est une vecteur d'entiers positifs à \( n\) composantes. Si \( \alpha=(2,1)\) alors nous avons la notation
	\begin{equation}
		\partial^{\alpha}f=\frac{ \partial^2  }{ \partial x_1 }\frac{ \partial  }{ \partial x_2 }f
	\end{equation}
	Cette notation pose problème lorsque, par exemple, \( \partial_1^2\partial_2f\neq \partial_1\partial_2\partial_1f\).

	Elle pose également problème lorsque l'on veut faire une récurrence sur l'ordre de dérivation en ajoutant une seule dérivation à la fois.

	C'est pourquoi nous introduisons le concept de \defe{liste d'indices}{liste d'indices}. En parlant de \( \eR^n\), une liste d'indices est un vecteur arbitrairement long (mais fini) d'entiers dans \( \{ 1,\ldots, n \}\). Si, dans \( \eR^7\), \( \alpha=(1,3,1,5)\), alors
	\begin{equation}
		\partial^{\alpha}f=\partial_1\partial_3\partial_1\partial_5f.
	\end{equation}

	Si \( \alpha\) est une liste d'indices de longueur \( p\), une \defe{queue de}{queue de liste d'indices} \( \alpha\) est une liste d'indices de longueur \( 0 < k \leq p\) de la forme \( (\alpha_{p-k+1}, \alpha_{p-k+2},\ldots, \alpha_p)\).
\end{normaltext}


%+++++++++++++++++++++++++++++++++++++++++++++++++++++++++++++++++++++++++++++++++++++++++++++++++++++++++++++++++++++++++++ 
\section{Anglicismes}
%+++++++++++++++++++++++++++++++++++++++++++++++++++++++++++++++++++++++++++++++++++++++++++++++++++++++++++++++++++++++++++
\label{SECooPBZVooCVInFT}

Voici quelque anglicismes dont je ne me souviens jamais.
\begin{enumerate}
	\item
	      Une \( \sigma\)-algèbre est une tribu, définition \ref{DefjRsGSy}.
	\item
	      Le «uniform boundedness principle» est le théorème de Banach-Steinhaus \ref{THOooJHVNooIDDxyT}.
	\item
	      Un anneau est ce qu'on appelle «\emph{ring}» en anglais. Un corps est en anglais «\emph{field}». De plus le mot «\emph{field}» comprend la commutativité. Donc certains utilisent le mot «corps» pour dire «corps commutatif» et parlent alors d'anneau \emph{à division} pour parler de corps non commutatifs.
\end{enumerate}
