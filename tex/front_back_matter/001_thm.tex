\InternalLinks{cardinalité}
Le Frido\footnote{Ici je mets la référence \cite{MonCerveau}; pas parce qu'elle est utile ici, mais parce que je veux être sûr qu'elle soit numéro 1 de façon à être facilement reconnaissable. Elle indique les affirmation à propos desquelles \randomGender{le lecteur}{la lectrice} doit être doublement \randomGender{attentif}{attentive}.} ne définit pas la notion de nombre cardinal; ça nous mènerait trop loin. Au lieu de cela, nous allons nous contenter des notions d'équipotence, surpotence et subpotence, et démontrer un certain nombre de résultats en utilisant sans retenue le lemme de Zorn \ref{LemUEGjJBc}.
\begin{enumerate}
	\item
	      Ensemble infini, définition \ref{DefEOZLooUMCzZR}.
	\item
	      Ensemble dénombrable, définition \ref{DefEnsembleDenombrable}.
	\item
	      Cardinal d'un ensemble fini, définition \ref{PROPooJLGKooDCcnWi}.
	\item
	      Définition d'équipotence, surpotence et subpotence, notations \( A\succ B\) et \( A\approx B\), définition \ref{DEFooXGXZooIgcBCg}.
	\item
	      Toute partie d'un ensemble fini est finie, lemme \ref{LEMooTUIRooEXjfDY}.
	\item
	      Si \( A\) est un ensemble fini ou dénombrable, alors il existe une surjection \( \eN\to A\), lemme \ref{LEMooSRZWooASgEfy}.
	\item
	      Si \( A\) est un ensemble infini et si \( f\colon A\to B\) est une application injective, alors \( f(A)\) est infini, proposition \ref{PROPooWKSIooHcfYPN}.
	\item
	      Toute partie infinie de \( \eN\) est dénombrable, proposition \ref{PROPooOBKMooWEGCvM}
	\item
	      Une bijection \( \eN\to \eN\times \eN\), proposition \ref{PROPooLPKUooAlsYJg}.
	\item
	      Une décomposition de \( \eN\) en une infinité de parties équipotentes à \( \eN\), corolaire \ref{CORooNRPIooZPSmqa}.
	\item
	      Si il existe une surjection \( \eN\to A\), alors \( A\) est fini ou dénombrable, lemme \ref{LEMooDLWFooNAJbbq}.
	\item
	      Une union dénombrable d'ensembles finis ou dénombrables est finie ou dénombrable, proposition \ref{PROPooENTPooSPpmhY}
	\item
	      Tout ensemble infini contient une partie en bijection avec \( \eN\), proposition \ref{PROPooUIPAooCUEFme}.
	\item
	      Toute partie d'un ensemble fini est finie, et toute partie d'un ensemble dénombrable est finie ou dénombrable, proposition \ref{PropQEPoozLqOQ}.
	\item
	      Si \( A\succeq B\) et \( B\succeq A\), alors \( A\approx B\), théorème de Cantor-Schröder-Bernstein \ref{THOooRYZJooQcjlcl}
	\item
	      Le théorème de Cantor \ref{THOooJPNFooWSxUhd} dit qu'il n'existe pas de surjection d'un ensemble vers son ensemble des parties. On en déduit qu'il n'existe pas d'ensemble contenant tous les ensembles (corolaire \ref{CORooZMAOooPfJosM}).
	\item
	      Si \( A\) est infini et si \( A\succeq B\), alors \( A\approx A\cup B\) par le lemme \ref{LEMooXMVDooIWLWis}.
	\item
	      Si \( S\) est un ensemble infini alors il existe une bijection \( \varphi\colon \{ 0,1 \}\times S\to S\), proposition \ref{PropVCSooMzmIX}.
	\item
	      Si \( A\) est infini, alors \( A\times \eN\approx A\), proposition \ref{PROPooFKBEooKXqujV}.
	\item
	      Si \( A\) est infini et si \( B\prec A\), alors \( A\setminus B\approx A\), lemme \ref{LEMooIVCBooHWQiZB}.
	\item
	      Si \( A\) est infini, alors \( A\approx A\times A\), théorème \ref{THOooDGOVooRdURVi}.
\end{enumerate}

Il y a aussi des résultats de cardinalité autour des extensions de corps.
\begin{enumerate}
	\item
	      Si \( \eK\) est un corps infini, alors \( \eK[X]\approx \eK\).
	\item
	      Le théorème de Steinitz \ref{THOooEDQKooLEGlDv} affirme que tout corps admet une unique clôture algébrique. La preuve utilise pas mal de cardinalité ainsi que le lemme de Zorn \ref{LemUEGjJBc}.
\end{enumerate}
