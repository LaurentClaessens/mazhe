
\InternalLinks{produit semi-direct de groupes}
\begin{enumerate}
	\item
	      Définition~\ref{DEFooKWEHooISNQzi}.         % Ce commentaire sert à rendre cette ligne unique ooQLCCooQfibcp
	\item
	      Le corolaire~\ref{CoroGohOZ} donne un critère pour prouver qu'un produit \( NH\) est un produit semi-direct.
	\item
	      L'exemple~\ref{EXooHNYYooUDsKnm} donne le groupe des isométries du carré comme un produit semi-direct.
	\item
	      Le théorème~\ref{ThoLnTMBy} classifie les groupes d'ordre \( pq\) (\( p\), \( q\) premiers distincts) à grands coups de produit semi-directs.
	\item
	      Le théorème~\ref{THOooQJSRooMrqQct} donne les isométries de \( \eR^n\) par \( \Isom(\eR^n)=T(n)\times_{\rho} O(n)\) où \( T(n)\) est le groupe des translations.
	\item
	      La proposition~\ref{PROPooDHYWooXxEXvl} donne une décomposition du groupe orthogonal \( \gO(n)=\SO(n)\times_{\rho} C_2\) où \( C_2=\{ \id,R \}\) où \( R\) est de déterminant \( -1\).
	\item
	      La proposition~\ref{PROPooTPFZooKtFxhg} donne \( \Aff(\eR^n)=T(n)\times_{\rho}\GL(n,\eR)\) où \( \Aff(\eR^n)\) est le groupe des applications affines bijectives de \( \eR^n\).
\end{enumerate}
