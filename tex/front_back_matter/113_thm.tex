
    \InternalLinks{formule des accroissements finis}        \label{INTERNooXFNTooNNaOzP}
    Il en existe plusieurs formes :
    \begin{enumerate}
        \item
            Une version adaptée aux espaces normés de dimension finie, avec hypothèse de différentiabilité, est le théorème~\ref{val_medio_2}. La formule $\|f(b)-f(a)\|_n\leq \sup_{x\in[a,b]}\|df_x\|_{\aL(\eR^m,\eR^n)}\|b-a\|_m$.
        \item
            Une version pour les dérivées partielles est dans le lemme \ref{LEMooNMTAooLgMkgH}. Pour rappel, la définition de la dérivation partielle est \ref{DEFooCATTooTPLtpR}.
        \item
            La formule \( f(a+\epsilon e_i)=f(a)+\epsilon(\partial_if)(a)+\epsilon\alpha(\epsilon)\), proposition \ref{PROPooYYSMooUDxtlB}.
        \item
        L'existence de \( c\in \mathopen] a , b \mathclose[\) tel que
            \begin{equation}
                f'(c)=\frac{ f(b)-f(a) }{ b-a }
            \end{equation}
            est le théorème des accroissements finis proprement dit. C'est le théorème \ref{ThoAccFinis}.
        \item
            Il existe un \( c\) entre \( a\) et \( b\) tel que
            \begin{equation}
                f(b)=f(a)+(b-a)(\partial_{\beta}f)(c)
            \end{equation}
            où \( \beta=b-a\) est la proposition \ref{PROPooCAWBooINcNxj}.
        \item
            La formule \( f(a+h)=f(a)+hf'(a)+\alpha(h)\) pour une fonction \( \eR\to \eR\) en le théorème \ref{PropUTenzfQ}.
        \item
            Une généralisation pour les intervalles non bornés : théorème~\ref{THOooRIIBooOjkzMa}.
        \item
            Espaces vectoriels normés, théorème~\ref{ThoNAKKght}
    \end{enumerate}

