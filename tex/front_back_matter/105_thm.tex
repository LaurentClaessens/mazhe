
\InternalLinks{constructions topologiques}       \label{THEMEooYRIWooDXZnhX}
\begin{description}
    \item[topologie produit] 
        Si \( X\) et \( Y\) sont des espaces topologiques, nous pouvons construire une topologie sur \( X\times Y\).
\begin{enumerate}
	\item
	      La définition de la topologie produit est~\ref{DefIINHooAAjTdY}.
	\item
	      Pour les espaces vectoriels normés, le produit est donné par la définition~\ref{DefFAJgTCE}.
	\item
	      L'équivalence entre la topologie de la norme produit et la topologie produit est le lemme~\ref{DefFAJgTCE}.
	\item
	      Quand \( V\) et \( W\) sont des espaces métriques, la topologie considérée sur \( V\times W\) est celle de la définition \ref{DefFAJgTCE}. C'est à la fois la topologie de la norme produit et la topologie produit.
	\item
	      La convergence dans un espace vectoriel est si et seulement si il y a convergence composante par composante, proposition \ref{PROPooNRRIooCPesgO}.
      \item Dans le cas d'espaces normés, la topologie produit est la même que celle de la norme produit, lemme \ref{LEMooFQMSooLmdIvD}.
\end{enumerate}
\item[topologie induite]
    Si \( X\) est un espace topologique et si \( A\) est une partie de \( X\), nous mettons une topologie sur \( A\).
    \begin{enumerate}
        \item Le topologie induite, définition \ref{DefVLrgWDB}.
        \item Si \( X\) est un espace vectoriel normé, la topologie induite est celle de la norme restreinte : lemme \ref{LEMooKDMYooMIcFRI}.
    \end{enumerate}
\item[topologie quotient]
    Si \( X\) est topologique et si \( \sim\) est une relation d'équivalence, nous définissons une topologie sur \( X/\sim\).
    \begin{enumerate}
        \item La topologie quotient est définie en \ref{DEFooHWSYooZZLXQU}.
        \item
            Si \( X\) est vectoriel normé, la topologie sur \( X/\sim\) est aussi donnée par une norme quotient de la définition \ref{PROPooDUAVooEfrEGI}. La proposition \ref{PROPooKLXSooSOUZkc} donne l'équivalence entre la topologie quotient et la norme quotient.
    \end{enumerate}
\end{description}

