\InternalLinks{intégration}     \label{THEMEooHINHooJaSYQW}

À propos d'intégration.
\begin{description}
	\item[Vocabulaire] La différence entre une intégrale qui existe et une fonction intégrable est dans \ref{NORMooRDHAooKRniYt}.
	\item[L'ordre dans lequel les choses sont faites]
		\begin{itemize}
			\item
			      Nous commençons par considérer des fonctions \( f\colon \Omega\to \mathopen[ 0 , +\infty \mathclose]\) dans la définition \ref{DefTVOooleEst}.
			\item
			      Nous donnerons ensuite quelques propriétés restreintes aux fonctions à valeurs positives, par exemple
			      \begin{enumerate}
				      \item
				            La convergence monotone \ref{ThoRRDooFUvEAN},
				      \item
				            Lemme de Fatou \ref{LemFatouUOQqyk}.
				      \item
				            (presque) linéarité pour les fonctions positives, théorème \ref{ThoooCZCXooVvNcFD}.
			      \end{enumerate}
			\item
			      La définition pour les fonctions à valeurs dans \( \eR\) puis \( \eC\) est \ref{DefTCXooAstMYl}.
			\item
			      Pour les fonctions à valeurs dans un espace vectoriel, c'est la définition \ref{PROPooOFSMooLhqOsc}.
		\end{itemize}
	\item[primitive et intégrale]
		\begin{enumerate}
			\item
			      La définition \ref{DEFooGLJDooFeZBBC} donne \( \int_a^bf=\int_{\mathopen] a ,b \mathclose[}f\) lorsque \( f\) est intégrable sur \( \mathopen] a , b \mathclose[\).
			\item Lorsque \( f\) n'est pas intégrable sur \( \mathopen] a , b \mathclose[\) nous pouvons poser \( \lim_{x\to b} \int_a^bf\) et dire que c'est une intégrale impropre, définition \ref{DEFooINPOooWWObEz}.
			\item
			      Si \( F\) et \( G\) sont primitives de \( f\), alors \( F=G+cst\), corolaire \ref{CorZeroCst}.
		\end{enumerate}
	\item[Quelque résultats]
		\begin{enumerate}
			\item
			      Intégrale associée à une mesure, définition~\ref{DefTVOooleEst}
			\item
			      L'existence d'une primitive pour toute fonction continue est le théorème~\ref{ThoEXXyooCLwgQg}.
			\item
			      La définition d'une primitive est la définition~\ref{DefXVMVooWhsfuI}.
			\item
			      Primitive et intégrale, proposition~\ref{PropEZFRsMj}.
			\item
			      Intégrale impropre, définition~\ref{DEFooINPOooWWObEz}.
		\end{enumerate}
	\item[Intégrale et mesure]
		\begin{enumerate}
			\item
			      L'intégrale de la fonction \( 1\) donne la mesure : \( \int_B1d\mu=\mu(B)\), c'est le lemme \ref{LemooPJLNooVKrBhN}.
			\item
			      Le théorème de Radon-Nikodym \ref{THOooKSISooPAqZcp} donne une densité pour certaines mesures.
			\item
			      Le produit d'une mesure par une fonction donnée par la définition \ref{PropooVXPMooGSkyBo} introduit aussi une densité : \( (w\cdot \mu)(A)=\int_Awd\mu\).
		\end{enumerate}

	\item[Limite dans les bornes]
		\begin{enumerate}
			\item
			      Si \(f \colon \eR\to \eR  \) est intégrable alors \(  \lim_{b \to\infty}\int_{\mathopen[ a,b\mathclose]}f=\int_{\mathopen[ 0,\infty\mathclose[}f\).
			\item
			      Nous avons \( \int_{\eR}f=\lim_{R\to \infty}\int_{-R}^Rf\) par la proposition \ref{PROPooSBKEooOlxqRk}.
		\end{enumerate}

	\item[Autre résultats]
		\begin{enumerate}
			\item
			      Si \( A,B\subset \Omega\) sont des parties disjointes, alors \( \int_{A\cup B}f=\int_Af+\int_Bf\), proposition \ref{PropOPSCooVpzaBt}.
			\item
			      La \( \sigma\)-additivité dénombrable, \( \int_{\bigcup_iA_i}fd\mu=\sum_{i=0}^{\infty}\int_{A_i}fd\mu\) est dans les propositions \ref{PROPooTFOAooJBwmCV} et \ref{PROPooDWYNooWKJmEV}.
		\end{enumerate}
\end{description}
