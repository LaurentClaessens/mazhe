\InternalLinks{classification de groupes}
\begin{enumerate}
	\item Ordre d'un groupe, définition \ref{DEFooKWBCooMlmpCP}.
	\item Structure des groupes d'ordre \( pq\), théorème~\ref{ThoLnTMBy}.
	\item Le groupe alterné est simple, théorème~\ref{ThoURfSUXP}.
	\item Définition~\ref{DEFooPRCHooVZdwST} d'un \( p\)-groupe.
	\item Théorème de Sylow~\ref{ThoUkPDXf}.
	\item Théorème de Burnside sur les sous-groupes d'exposant fini de \( \GL(n,\eC)\), théorème~\ref{ThooJLTit}.
	\item \( (\eZ/p\eZ)^*\simeq \eZ/(p-1)\eZ\), corolaire~\ref{CorpRUndR}.
	\item Le premier théorème d'isomorphisme \ref{ThoPremierthoisomo} dit que si \( \theta\) est un morphisme de groupes, alors  \( G/\ker(\theta)=\Image(\theta)\).
	\item Le deuxième théorème d'isomorphisme \ref{THOooURXUooQJvkjx} dit que si \( N\) est normal dans \( G\), alors \( NH/N=H/(H\cap N)\).
	\item Le troisième théorème d'isomorphisme \ref{ThoezgBep} dit que, avec des hypothèses de normalité, \( (G/M)/(N/M)=G/N\).
\end{enumerate}

À propos d'ordre dans un groupe.
\begin{enumerate}
	\item
	      L'ordre d'un élément divise l'ordre du groupe, corolaire \ref{CorpZItFX}.
	\item
	      Si \( H\) est normal dans \( G\), alors \( | G/H |=| G |/| G |\), théorème de Lagrange \ref{ThoLagrange}.
\end{enumerate}
