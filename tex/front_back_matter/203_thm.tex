\InternalLinks{fonction indicatrice d'Euler}		\label{THMooUDYMooCCXdbw}
\begin{enumerate}
	\item
	      Définition de \(\varphi \colon \eN^*\to \eN^*  \) dans \ref{DEFooZRYMooZCozga}.
	\item
	      Propriétés genre \( \varphi(pq)=\varphi(p)\varphi(q)\), corolaire \ref{CorlvTmsf}.
	\item
	      \( \varphi(n)=\Card(\Delta_n)\), proposition \ref{PROPooFKCHooWdFicM}.
	\item
	      \( n=\sum_{d\divides n}\varphi(d)\), lemme \ref{PROPooYHUDooUROTiN}.
	\item
	      Le théorème de Euler-Fermat \ref{THOooXMBSooXrrfOP} donne \( a^{\varphi(n)}\in[1]_n\) dès que \( a\) et \( n\) sont premiers entre eux.
	\item
	      Si \( A\) et \( B\) sont premiers entre eux, il existe \( p,m\) tels que \( A^p=mB+1\), proposition \ref{THOooXMBSooXrrfOP}.
\end{enumerate}
