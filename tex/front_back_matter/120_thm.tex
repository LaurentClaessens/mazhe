\InternalLinks{points fixes}        \label{THEMEooWAYJooUSnmMh}
\begin{enumerate}
	\item
	      Il y a plusieurs théorèmes de points fixes.
	      \begin{description}
		      \item[Théorème de Picard]~\ref{ThoEPVkCL} donne un point fixe comme limite d'itérés d'une fonction lipschitzienne. Il aura pour conséquence le théorème de Cauchy-Lipschitz~\ref{ThokUUlgU}, l'équation de Fredholm, théorème~\ref{ThoagJPZJ} et le théorème d'inversion locale dans le cas des espaces de Banach~\ref{ThoXWpzqCn}.
		      \item[Théorème de Brouwer] qui donne un point fixe pour une application d'une boule vers elle-même. Nous allons donner plusieurs versions et preuves.
			      \begin{enumerate}
				      \item
				            Dans \( \eR^n\) en version \( C^{\infty}\) via le théorème de Stokes, proposition~\ref{PropDRpYwv}.
				            %TODOooBBGRooSelkgJ. Adapter cette phrase parce que cette proposition n'utilise plus Stokes.
				      \item
				            Dans \( \eR^n\) en version continue, en s'appuyant sur le cas \( C^{\infty}\) et en faisant un passage à la limite, théorème~\ref{ThoRGjGdO}.
				      \item
				            Dans \( \eR^2\) via l'homotopie, théorème~\ref{ThoLVViheK}. Oui, c'est très loin. Et c'est normal parce que ça va utiliser la formule de l'indice qui est de l'analyse complexe\footnote{On aime bien parce que ça ne demande pas Stokes, mais quand même hein, c'est pas gratos non plus.}.
			      \end{enumerate}
			      \item[Théorème de Markov-Kakutani]\ref{ThoeJCdMP} qui donne un point fixe à une application affine continue d'un convexe fermé borné dans lui-même.
		      \item[Théorème de Schauder] C'est une version valable en dimension infinie du théorème de Brouwer. Théorème \ref{ThovHJXIU}
	      \end{description}

	\item Pour les équations différentielles
	      \begin{enumerate}
		      \item
		            Le théorème de Schauder a pour conséquence le théorème de Cauchy-Arzela~\ref{ThoHNBooUipgPX} pour les équations différentielles.
		      \item
		            Le théorème de Schauder~\ref{ThovHJXIU} permet de démontrer une version du théorème de Cauchy-Lipschitz (théorème~\ref{ThokUUlgU}) sans la condition de Lipschitz, mais alors sans unicité de la solution. Notons que de ce point de vue nous sommes dans la même situation que la différence entre le théorème de Brouwer et celui de Picard : hors hypothèse de type «contraction», point d'unicité.
	      \end{enumerate}
	\item
	      En calcul numérique
	      \begin{itemize}
		      \item
		            La convergence d'une méthode de point fixe est donnée par la proposition~\ref{PROPooRPHKooLnPCVJ}.
		      \item
		            La convergence quadratique de la méthode de Newton est donnée par le théorème~\ref{THOooDOVSooWsAFkx}.
		      \item
		            En calcul numérique, section~\ref{SECooWUVTooMhmvaW}
		      \item
		            Méthode de Newton comme méthode de point fixe, sous-section~\ref{SUBSECooIBLNooTujslO}.
	      \end{itemize}

	\item
	      D'autres utilisations de points fixes.
	      \begin{itemize}
		      \item
		            Processus de Galton-Watson, théorème~\ref{ThoJZnAOA}.
		      \item
		            Dans le théorème de Max-Milgram~\ref{THOooLLUXooHyqmVL}, le théorème de Picard est utilisé.
	      \end{itemize}
\end{enumerate}
