\InternalLinks{exponentielle}        \label{THEMEooKXSGooCsQNoY}

Toutes les exponentielles sont définies par la série
\begin{equation*}
	\exp(x)=\sum_{k=0}^{\infty}\frac{ x^k }{ k! },
\end{equation*}
tant que la somme a un sens.

\begin{description}
	\item[Réels]

		Voici le plan que nous suivons dans le Frido :
		\begin{itemize}
			\item L'exponentielle est définie par sa série en~\ref{DEFooSFDUooMNsgZY}.
			\item Nous démontrons qu'elle vérifie l'équation différentielle \( y'=y\), \( y(0)=1\) (théorème \ref{ThoKRYAooAcnTut}).
			\item
			      L'équation différentielle \( y'=\lambda y\) est dans la proposition \ref{PROPooIKJBooLOipUM}.
			\item Nous démontrons l'unicité de la solution à cette équation différentielle.
			\item Nous démontrons qu'elle est égale à \( x\mapsto y(1)^x\). Cela donne la définition du nombre \( e\) comme valant \( y(1)\).
			\item Nous définissons le logarithme comme l'application réciproque de l'exponentielle (définition~\ref{DEFooELGOooGiZQjt}).
			\item Les fonctions trigonométriques (sinus et cosinus) sont définies par leurs séries. Il est alors montré que \( e^{ix}=\cos(x)+i\sin(x)\) (lemme \ref{LEMooHOYZooKQTsXW}).
		\end{itemize}

		\item[Propriétés]\hspace{1cm}
		\begin{itemize}
			\item
			      La formule \( a^{-x}=1/a^x\) est la proposition \ref{PROPooVADRooLCLOzP}\ref{ITEMooSCJBooNVJZah}.
			\item
			      \( \exp(x)= e^{x}\), proposition \ref{PropCELWooLBSYmS}.
			\item
			      Nous avons l'encadrement \( 2.5 < e < 3\), lemme \ref{LEMooXFAXooLVbebl}.
			\item
			      Le nombre \( e\) est irrationnel, proposition \ref{PROPooFRKUooZyhHIC}.
		\end{itemize}

	\item[Complexes]

		\begin{enumerate}
			\item
			      La définition de \( \exp(a+ib)\) est la définition \ref{DEFooSFDUooMNsgZY}.
			\item
			      Les principales propriétés, dont \(  e^{z+w}= e^{z} e^{w}\), sont dans la proposition \ref{PropdDjisy}.
			\item
			      Nous avons \(  e^{ix}= e^{iy}\) si et seulement si il existe \( k\in\eZ\) tel que \( y=x+2k\pi\), corolaire \ref{CORooTFMAooHDRrqi}.
			\item
			      Le fait que \(  e^{i\theta}\) donne tous les nombres complexes de norme \( 1\) est la proposition \ref{PROPooXELTooYKjDav}\ref{ITEMooOHRHooRXvxrL}.
			\item
			      Le groupe des racines de l'unité est donné par l'équation \eqref{EqIEAXooIpvFPe}.
		\end{enumerate}

	\item[Algèbre normée commutative]

		Pour la définition c'est la proposition~\ref{DEFooSFDUooMNsgZY} et pour la régularité \(  C^{\infty}\) c'est la proposition~\ref{PROPooTBDAooQouzSk}.

	\item[Idem non commutatif]

		Il y a une tentative de théorème~\ref{THOooFGTQooZPiVLO}, mais c'est principalement pour les matrices qu'il y a des résultats.

	\item[Matrices]

		De nombreux résultats sont disponibles pour les exponentielles de matrices.

		\begin{enumerate}
			\item
			      \( e^{sA} e^{tA}= e^{(s+t)A}\), proposition \ref{PROPooKDKDooCUpGzE}.
			\item
			      Si \( A\) est une matrice, alors \( (e^{tA})'(u)=Ae^{uA}\), proposition \ref{PROPooSDNNooQtHkhA}.
			\item
			      Les sections \ref{secAOnIwQM} et \ref{SECooBYQBooZifJsg} parlent d'exponentielle de matrices.
			\item
			      L'exponentielle donne lieu à une fonction de classe \(  C^{\infty}\), proposition~\ref{PropXFfOiOb}.
			\item
			      Le lemme à propos d'exponentielle de matrice~\ref{LemQEARooLRXEef} donne :
			      \begin{equation*}
				      \|  e^{tA} \|\leq P\big( | t | \big)\sum_{i=1}^r e^{t\real(\lambda_i)}.
			      \end{equation*}
			\item
			      La proposition~\ref{PropCOMNooIErskN} : si \( A\in \eM(n,\eR)\) a un polynôme caractéristique scindé, alors \( A\) est diagonalisable si et seulement si \( e^A\) est diagonalisable.
			\item
			      La section~\ref{subsecXNcaQfZ} parle des fonctions exponentielle et logarithme pour les matrices. Entre autres la dérivation et les séries.
			\item
			      Pour résoudre des équations différentielles linéaires : sous-section~\ref{SUBSECooMDKIooKaaKlZ}.
			\item
			      La proposition~\ref{PropKKdmnkD} dit que l'exponentielle est surjective sur \( \GL(n,\eC)\).
			\item

			      La proposition~\ref{PropFMqsIE} : si \( u\) est un endomorphisme, alors \( \exp(u)\) est un polynôme en \( u\).
			\item
			      Calcul effectif : sous-section~\ref{SUBSECooGAHVooBRUFub}.
			\item Proposition~\ref{PROPooZUHOooQBwfZq} : si \( A\in\eM(n,\eC)\) alors \( e^{\tr(A)}=\det( e^{A})\).
			\item
			      Les séries entières de matrices sont traitées autour de la proposition~\ref{PropFIPooSSmJDQ}.
		\end{enumerate}

	\item[Paramétrisation du cercle]
		\begin{enumerate}
			\item
			      La proposition \ref{PROPooXELTooYKjDav} parle des propriétés de
			      \begin{equation}
				      \begin{aligned}
					      \varphi\colon \mathopen] 0,2\pi\mathclose[ & \to S^1         \\
					      t                                          & \mapsto e^{it}.
				      \end{aligned}
			      \end{equation}
			      % ici : 3072217048
		\end{enumerate}

\end{description}
