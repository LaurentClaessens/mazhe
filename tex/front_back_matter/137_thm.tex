\InternalLinks{convolution}	\label{THEMEooConvolution}
\begin{enumerate}
	\item
	      Définition \ref{DEFooHHCMooHzfStu}, et principales propriétés sur \( L^1(\eR)\) dans le théorème \ref{THOooMLNMooQfksn}.
	\item
	      Inégalité de normes : si \( f\in L^p\) et \( g\in L^1\), alors \( \| f*g \|_p\leq \| f \|_p\| g \|_1\), proposition~\ref{PROPooDMMCooPTuQuS}.
	\item
	      \( \varphi\in L^1(\eR)\) et \( \psi\in\swS(\eR)\), alors \( \varphi * \psi\in \swS(\eR)\), proposition~\ref{PROPooUNFYooYdbSbJ}.
	\item
	      Les suites régularisantes : \( \lim_{n\to \infty} \rho_n*f=f\) dans la proposition~\ref{PROPooYUVUooMiOktf}.
	\item
	      Convolution d'une distribution par une fonction, définition par l'équation \eqref{EQooOUXKooGHDSzL}.
	\item
	      Nous avons \( (f*g)'=f*g'\), proposition \ref{PropHNbdMQe}.
	\item
              Si \( f\in L^1(\eR^d)\) et \( g\in C^{\infty}(\eR^d)\), alors \( f*g\in C^{\infty}\), corolaire \ref{CORooBSPNooFwYQrc}.
	\item
	      Somme de variables aléatoires : \( f_{x+Y}=f_X*f_Y\), proposition \ref{PROPooBNUEooOUvpdp}.
\end{enumerate}
