\InternalLinks{types d'anneaux}
\begin{enumerate}
	\item
	      Définition d'anneau : définition \ref{DefHXJUooKoovob}.
	\item
	      La définition d'anneau principal est \ref{DEFooGWOZooXzUlhK}; pour un idéal principal, c'est \ref{DEFooMZRKooBPLAWH}.
	\item
	      \( \eZ\) est intègre, lemme \ref{LEMooFUSTooDCcBDb}, principal et euclidien (proposition~\ref{PROPooPJGLooQSrJTU}).
	\item
	      \( \eZ[X]\) n'est pas principal (voir~\ref{ITEMooNQQMooSnuKvW}).
	\item   \label{ITEMooNQQMooSnuKvW}
	      Si \( A\) est un anneau intègre\footnote{Définition \ref{DEFooTAOPooWDPYmd}.} qui n'est pas un corps, alors \( A[X]\) n'est pas principal, lemme~\ref{LEMooDJSUooJWyxCL}.
	\item
	      L'anneau des fonctions holomorphes sur un compact donné est principal, proposition~\ref{PROPooVWRPooGQMenV}.
	\item
	      L'anneau \( \eZ[i\sqrt{ 3 }]\) n'est pas factoriel, exemple~\ref{EXooCWJUooCDJqkr}.
	\item
	      L'anneau \( \eZ[i\sqrt{ 5 }]\) n'est ni factoriel ni principal, exemple~\ref{EXooYCTDooGXAjGC}.
	\item
	      Tous les idéaux de \( \eZ/6\eZ\) sont principaux, mais \( \eZ/6\eZ\) n'est pas principal. Exemple~\ref{EXooCJRPooYkWdyr}.
\end{enumerate}

\begin{description}
	\item[anneau principal]
		\begin{itemize}
			\item
			      Il est intègre, définition \ref{DEFooGWOZooXzUlhK}.
			\item
			      il est factoriel\cite{BIBooENJGooWhQisg}.
		\end{itemize}
	\item[anneau intègre]
		\begin{itemize}
			\item
			      \( A\) est intègre si et seulement si \( A[X]\) est intègre, théorème \ref{ThoBUEDrJ}.
			\item
			      Si un pgcd\footnote{pgcd, définition \ref{DefrYwbct}.} est inversible, tous les pgcd sont inversibles, lemme \ref{LEMooSFHMooQoKsPV}.
		\end{itemize}
	\item[anneau factoriel]
		\begin{itemize}
			\item
			      Il est intègre, définition \ref{DEFooVCATooPJGWKq}.
			\item
			      \( P\) et \( Q\) sont primitifs si et seulement si \( PQ\) est primitif, lemme \ref{LEMooHJECooTeELgN}.
			\item
			      Formule \( c(PQ)=c(P)c(Q)\). Lemme \ref{LEMooHJECooTeELgN}.
		\end{itemize}
	\item[corps]
		\begin{itemize}
			\item
			      Il est un anneau intègre, lemme \ref{LemAnnCorpsnonInterdivzer}.
		\end{itemize}
\end{description}

\begin{description}
	\item[Dans un anneau principal]
		\begin{itemize}
			\item
			      \( p\) est premier si et seulement si il est irréductible si et seulement si l'idéal \( pA\) est maximal.
			\item
			      Tous les idéaux sont principaux, définition \ref{DEFooGWOZooXzUlhK}.
		\end{itemize}
	\item[Dans un anneau factoriel]
		\begin{itemize}
			\item
			      Si \( p\) est irréductible et si \( p\divides ab\), alors \( p\) divise \( a\) ou \( b\), lemme \ref{LEMooLVKMooSLuzao}.
			\item
			      Tout élément irréductible est premier, proposition \ref{PROPooOQSXooYidPQv}.
			\item
			      Si \( p\) est irréductible, \( pA\) est un idéal premier, lemme \ref{LEMooGKOSooRKtfDJ}.
			\item
			      Si \( p\) est irréductible, \( A/pA\) est un anneau intègre, lemme \ref{LEMooGKOSooRKtfDJ} et proposition \ref{PROPooRUQKooIfbnQX}.
		\end{itemize}
\end{description}


\begin{description}
	\item[Un idéal premier]
		\begin{itemize}
			\item Définition \ref{DEFooAQSZooVhvQWv}.
			\item Idéal premier si et seulement si \( A/I\) est anneau intègre, proposition \ref{PROPooRUQKooIfbnQX}.
		\end{itemize}
	\item[Idéal maximal]
		\begin{itemize}
			\item
			      Idéal maximal si et seulement si \( A/I\) est un corps, proposition \ref{PROPooRUQKooIfbnQX}.
		\end{itemize}
\end{description}
