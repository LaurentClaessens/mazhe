
\InternalLinks{types d'anneaux}
\begin{enumerate}
	\item
	      Définition d'anneau : définition \ref{DefHXJUooKoovob}.
	\item
	      La définition d'anneau principal est \ref{DEFooGWOZooXzUlhK}; pour un idéal principal, c'est \ref{DEFooMZRKooBPLAWH}.
	\item
        \( \eZ\) est intègre, lemme \ref{LEMooFUSTooDCcBDb}, principal et euclidien (proposition~\ref{PROPooPJGLooQSrJTU}).
	\item
	      \( \eZ[X]\) n'est pas principal (voir~\ref{ITEMooNQQMooSnuKvW}).
	\item   \label{ITEMooNQQMooSnuKvW}
	      Si \( A\) est un anneau intègre\footnote{Définition \ref{DEFooTAOPooWDPYmd}.} qui n'est pas un corps, alors \( A[X]\) n'est pas principal, lemme~\ref{LEMooDJSUooJWyxCL}.
	\item
	      L'anneau des fonctions holomorphes sur un compact donné est principal, proposition~\ref{PROPooVWRPooGQMenV}.
	\item
	      L'anneau \( \eZ[i\sqrt{ 3 }]\) n'est pas factoriel, exemple~\ref{EXooCWJUooCDJqkr}.
	\item
	      L'anneau \( \eZ[i\sqrt{ 5 }]\) n'est ni factoriel ni principal, exemple~\ref{EXooYCTDooGXAjGC}.
	\item
	      Tous les idéaux de \( \eZ/6\eZ\) sont principaux, mais \( \eZ/6\eZ\) n'est pas principal. Exemple~\ref{EXooCJRPooYkWdyr}.
\end{enumerate}

