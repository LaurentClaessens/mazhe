\InternalLinks{types d'anneaux}	    \label{THEMEooVIQIooOcFAQS}
\begin{enumerate}
	\item
	      Définition d'anneau : définition \ref{DefHXJUooKoovob}.
	\item
	      La définition d'anneau principal est \ref{DEFooGWOZooXzUlhK}; pour un idéal principal, c'est \ref{DEFooMZRKooBPLAWH}.
	\item
	      \( \eZ\) est intègre, lemme \ref{LEMooFUSTooDCcBDb}, principal et euclidien (proposition~\ref{PROPooPJGLooQSrJTU}).
	\item
	      \( \eZ[X]\) n'est pas principal (voir~\ref{ITEMooNQQMooSnuKvW}).
	\item   \label{ITEMooNQQMooSnuKvW}
	      Si \( A\) est un anneau intègre\footnote{Définition \ref{DEFooTAOPooWDPYmd}.} qui n'est pas un corps, alors \( A[X]\) n'est pas principal, lemme~\ref{LEMooDJSUooJWyxCL}.
	\item
	      L'anneau des fonctions holomorphes sur un compact donné est principal, proposition~\ref{PROPooVWRPooGQMenV}.
	\item
	      L'anneau \( \eZ[i\sqrt{ 3 }]\) n'est pas factoriel, exemple~\ref{EXooCWJUooCDJqkr}.
	\item
	      L'anneau \( \eZ[i\sqrt{ 5 }]\) n'est ni factoriel ni principal, exemple~\ref{EXooYCTDooGXAjGC}.
	\item
	      Tous les idéaux de \( \eZ/6\eZ\) sont principaux, mais \( \eZ/6\eZ\) n'est pas principal. Exemple~\ref{EXooCJRPooYkWdyr}.
\end{enumerate}

\begin{description}
	\item[anneau principal]\hspace{1cm}
	\begin{itemize}
		\item
		      Définition \ref{DEFooGWOZooXzUlhK}.
		\item
		      Il est intègre, définition \ref{DEFooGWOZooXzUlhK}.
		\item
		      Il est noetherien, lemme \ref{LEMooHQPVooTfkhRV}.
		\item
		      il est factoriel, théorème \ref{THOooANCAooBChmwp}.
	\end{itemize}
	\item[anneau euclidien]\hspace{1cm}
	\begin{itemize}
		\item
		      Définition \ref{DefAXitWRL}.
		\item
		      Il est principal par \ref{Propkllxnv}
	\end{itemize}
	\item[anneau intègre]\hspace{1cm}
	\begin{itemize}
		\item
		      Définition \ref{DEFooTAOPooWDPYmd}.
	\end{itemize}
	\item[anneau noetherien]\hspace{1cm}
	\begin{itemize}
		\item
		      Définition \ref{DEFooPWMHooCnrQuJ}.
	\end{itemize}
	\item[anneau factoriel]\hspace{1cm}
	\begin{itemize}
		\item
		      Définition \ref{DEFooVCATooPJGWKq}.
		\item
		      Il est intègre, définition \ref{DEFooVCATooPJGWKq}.
	\end{itemize}
	\item[corps]\hspace{1cm}
	\begin{itemize}
		\item
		      Corps, définition \ref{DefTMNooKXHUd}.
		\item
		      Il est un anneau intègre, lemme \ref{LEMooIKNMooMfvQnu}.
		\item
		      Il est un anneau principal, proposition \ref{PROPooPVFOooJvWRIh}.
		\item
		      Il est un anneau factoriel, proposition \ref{PROPooPVFOooJvWRIh}.
	\end{itemize}
\end{description}

\begin{description}
	\item[Dans un anneau intègre]\hspace{1cm}
	\begin{itemize}
		\item
		      Un anneau fini intègre est un corps, proposition \ref{PropanfinintimpCorp}.
		\item
		      \( A\) est intègre si et seulement si \( A[X]\) est intègre, théorème \ref{ThoBUEDrJ}.
		\item
		      Si un pgcd\footnote{pgcd, définition \ref{DefrYwbct}.} est inversible, tous les pgcd sont inversibles, lemme \ref{LEMooSFHMooQoKsPV}.
	\end{itemize}
	\item[Dans un anneau principal]\hspace{1cm}
	\begin{itemize}
		\item
		      \( p\) est premier si et seulement si il est irréductible si et seulement si l'idéal \( pA\) est maximal, proposition \ref{PROPooSGLNooBYKNyo}.
		\item
		      Tous les idéaux sont principaux, définition \ref{DEFooGWOZooXzUlhK}.
		\item
		      Si \( \pgcd(a,b)=1\), \( \pgcd(a,c)=1\), alors \( \pgcd(a,bc)=1\), lemme \ref{LEMooQJGIooEtVnyj}.
	\end{itemize}
	\item[Dans un anneau factoriel]\hspace{1cm}
	\begin{itemize}
		\item
		      Si \( p\) est irréductible et si \( p\divides ab\), alors \( p\) divise \( a\) ou \( b\), lemme \ref{LEMooLVKMooSLuzao}.
		\item
		      Tout élément irréductible est premier, proposition \ref{PROPooOQSXooYidPQv}.
		\item
		      Si \( p\) est irréductible, \( pA\) est un idéal premier, lemme \ref{LEMooGKOSooRKtfDJ}.
		\item
		      Si \( p\) est irréductible, \( A/pA\) est un anneau intègre, lemme \ref{LEMooGKOSooRKtfDJ} et proposition \ref{PROPooRUQKooIfbnQX}.
		\item
		      \( P\) et \( Q\) sont primitifs si et seulement si \( PQ\) est primitif, lemme \ref{LEMooHJECooTeELgN}.
		\item
		      Formule \( c(PQ)=c(P)c(Q)\). Lemme \ref{LEMooHJECooTeELgN}.
		\item
		      L'anneau des polynômes \( \polyP(A)=A[X]\) est factoriel, théorème \ref{THOooVZUBooNSvcBr}.
	\end{itemize}
	\item[Dans un corps commutatif]\hspace{1cm}
	\begin{itemize}
		\item
		      L'anneau \( \eK[X]\) est factoriel, proposition \ref{PropqGZXvr}, euclidien et principal, lemme \ref{LEMooIDSKooQfkeKp}.
		\item
		      Les seuls idéaux de \( \eK\) sont \( \{ 0 \}\) et \( \eK\), lemme \ref{LEMooMAHXooXSowdn}.
		\item
		      L'anneau \( \polyP(\eK)\) est euclidien, lemme \ref{LEMooIDSKooQfkeKp}.
	\end{itemize}
\end{description}

\begin{description}
	\item[élément irréductible]\hspace{1cm}
	\begin{itemize}
		\item Définition \ref{DeirredBDhQfA}
	\end{itemize}
	\item[élément premier]\hspace{1cm}
	\begin{itemize}
		\item Définition \ref{DEFooZCRQooWXRalw}
		\item
		      Si \( A\) est intègre et unitaire, l'élément \( p\) est premier si et seulement si l'idéal \( pA\) est premier, proposition \ref{PROPooZBTIooRhAhvg}.
	\end{itemize}
	\item[éléments associés]\hspace{1cm}
	\begin{itemize}
		\item Définition \ref{DefrXUixs}.
	\end{itemize}
\end{description}

Dans un anneau, un idéal est \ref{DefooQULAooREUIU}.
\begin{description}
	\item[idéal premier]\hspace{1cm}
	\begin{itemize}
		\item
		      Définition \ref{DEFooAQSZooVhvQWv}.
		\item Idéal premier si et seulement si \( A/I\) est anneau intègre, proposition \ref{PROPooRUQKooIfbnQX}.
	\end{itemize}
	\item[idéal maximal]\hspace{1cm}
	\begin{itemize}
		\item
		      Idéal maximal si et seulement si \( A/I\) est un corps, proposition \ref{PROPooRUQKooIfbnQX}.
	\end{itemize}
	\item[idéal principal]\hspace{1cm}
	\begin{itemize}
		\item
		      Définition \ref{DEFooMZRKooBPLAWH}.
		\item
		      Dans un anneau unitaire intègre, \( p\) est irréductible si et seulement si il n'y a pas d'idéal principal \( I\) tel que \( pA\subsetneq I\subsetneq A\), proposition \ref{PROPooZBTIooRhAhvg}.
	\end{itemize}
\end{description}
