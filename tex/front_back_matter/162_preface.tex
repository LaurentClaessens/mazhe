% This is part of (almost) Everything I know in mathematics 
% Copyright (c) 2016-2017
%   Laurent Claessens
% See the file fdl-1.3.txt for copying conditions.

%+++++++++++++++++++++++++++++++++++++++++++++++++++++++++++++++++++++++++++++++++++++++++++++++++++++++++++++++++++++++++++
\section{Préface}
%+++++++++++++++++++++++++++++++++++++++++++++++++++++++++++++++++++++++++++++++++++++++++++++++++++++++++++++++++++++++++++

Ce livre s'adresse principalement au candidat à l'agrégation qui y trouvera de quoi dire pour presque toutes les leçons. La matière de ce livre (sous-entendu : tous volumes ensemble) recouvre l'épreuve d'analyse, d'algèbre et de l'option de modélisation.

Petit avertissement : ce livre ne contient pas beaucoup d'exemples, alors que le titre de presque toutes les leçons contient le mot «exemple». Ce livre peut vous aider à
\begin{itemize}
    \item apprendre la mathématique,
    \item trouver des développements
    \item remplir votre plan de théorèmes.
\end{itemize}
Pour les exemples, il faudra souvent aller chercher ailleurs.

%---------------------------------------------------------------------------------------------------------------------------
\subsection*{Mon modèle économique}
%---------------------------------------------------------------------------------------------------------------------------

Mon modèle économique est le don. Chaque achat sur \href{http://www.thebookedition.com/fr/}{thebookedition.com} me génère exactement zéro euro. Si vous aimez ce livre, vous pouvez faire un don, surtout si vous avez réussi l'agrégation et que vous pensez que ce livre en est pour quelque chose.

Dons en espèces :

% Moralement, vous devriez considérer ce fichier comme une section invariante
% au sens de la licence FDL.
% Autrement dit, je ne serais pas content que ce fichier soit modifié.

\begin{description}
\item[IBAN] FR76 3000 4004 0600 0035 2497 784
\item[BIC] BNPAFRPPBSC
\end{description}


Dons en nature acceptés : rédaction d'un chapitre, un développement, relecture d'une partie, réponse à l'une ou l'autre des questions qui restent en suspens \ldots
