
\InternalLinks{lemme de transfert}      \label{THEMEooJREIooKEdMOl}

Il y a deux résultats qui portent ce nom. Le premier est dans la théorie de Fourier, le résultat \( \hat{f'}=i\xi \hat{f}\).
\begin{enumerate}
	\item
	      Lemme~\ref{LemQPVQjCx} sur \( \swS(\eR^d)\)
	\item
	      Lemme~\ref{LEMooAGBZooWCbPDd} pour \( L^2\).
\end{enumerate}

L'autre lemme de transfert est en théorie des tribus, le résultat \( \sigma\big( f^{-1}(\tribC) \big)=f^{-1}\big( \sigma(\tribC) \big)\) du lemme \ref{LemOQTBooWGYuDU}. Celui-ci est d'ailleurs plutôt nommé «lemme de transport».

Il existe aussi un théorème de transfert \ref{PropintdPintdPXeR} qui parle de variables aléatoires :
	\begin{equation}
		E(f\circ X)=\int_{\Omega}f\big( X(\omega) \big)dP(\omega)=\int_{\eR^d}f(x)dP_X(x)
	\end{equation}

