
\InternalLinks{espaces de fonctions}                \label{THEMooNMYKooVVeGTU}

En ce qui concerne les densités, voir le thème~\ref{THEooPUIIooLDPUuq}.
\begin{enumerate}
	\item
	      \(  C^{\infty}(A)\) est l'ensemble des fonctions de classe \(  C^{\infty}\) sur \( A\). Les éléments de \(  C^{\infty}(A)\) peuvent être à valeurs dans \( \eR\) ou dans \( \eC\) selon le contexte.
	\item
	      \( \swD(A)\) est l'ensemble des fonctions de classe \(  C^{\infty}\) dont le support est un compact dans \( A\).
\end{enumerate}
Pour les ensembles \( \swS(A)\), \( \swD(A)\) et \( L^p(A)\), les fonctions sont à valeurs dans \( \eC\). La raison est que, de toutes façons, le passage à la transformée de Fourier produit en général des fonctions à valeurs dans \( \eC\) même si les fonctions de départ sont à valeurs dans \( \eR\).

\begin{description}
	\item[Topologie]

	      Les espaces de fonctions sont souvent munis de topologies définies par des seminormes.

	      \begin{enumerate}
		      \item
		            La topologie des seminormes est la définition~\ref{DefPNXlwmi}.
		      \item
		            La définition~\ref{DefFGGCooTYgmYf} donne les topologies sur \(  C^{\infty}(\Omega)\), \( \swD(K)\) et \( \swD(\Omega)\).
		      \item
		            La topologie \( *\)-faible sur \( \swD'(\Omega)\) est donnée par la définition~\ref{DefASmjVaT}.
	      \end{enumerate}

	\item[L'espace \( { L^2\big( \mathopen[ 0 , 2\pi \mathclose] \big) } \)]

	      C'est un espace très important, entre autres parce qu'il est de Hilbert et est bien adapté à la transformée de Fourier.

	      \begin{enumerate}
		      \item
		            Un rappel de la construction en \ref{NORMooUEIEooYtlFse}.
		      \item
		            Le produit scalaire\footnote{Produit scalaire, définition \ref{DefVJIeTFj}.} \( \langle f, g\rangle \) est donné en \eqref{EQooBFKDooMkCZOt} et la base trigonométrique est \eqref{EQooKMYOooLZCNap}.
		      \item
		            La densité des polynômes trigonométriques dans \( L^p(S^2)\) est le théorème~\ref{ThoQGPSSJq} ou le théorème~\ref{ThoDPTwimI}, au choix.
		      \item
		            Une conséquence de cette densité est que le système trigonométrique est une base hilbertienne\footnote{Définition \ref{DEFooADQXooFoIhTG}.} de \( L^2\) par le lemme~\ref{LEMooBJDQooLVPczR}.
	      \end{enumerate}

	      L'espace \( L^2\) est discuté en analyse fonctionnelle, dans la section \ref{SECooEVZSooLtLhUm} et les suivantes parce que l'étude de \( L^2\) utilise entre autres l'inégalité de Hölder~\ref{ProptYqspT}.

	      Le fait que \( L^2\) soit un espace de Hilbert est utilisé dans la preuve du théorème de représentation de Riesz~\ref{PropOAVooYZSodR}.
\end{description}
Si \( (\Omega,\tribA,\mu)\) est un espace mesuré, alors \( L^p(\Omega,\tribA,\mu)\) est un espace de Banach; c'est le théorème de Riesz-Fischer \ref{ThoGVmqOro}.
