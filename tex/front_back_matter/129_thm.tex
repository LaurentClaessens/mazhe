
\InternalLinks{équations différentielles}	\label{THEMEooEquaDiff}
L'utilisation des théorèmes de point fixe pour l'existence de solutions à des équations différentielles est fait dans le chapitre sur les points fixes.
\begin{enumerate}
	\item
	      Le théorème de Schauder a pour conséquence le théorème de Cauchy-Arzela~\ref{ThoHNBooUipgPX} pour les équations différentielles.
	\item
	      Le théorème de Schauder~\ref{ThovHJXIU} permet de démontrer une version du théorème de Cauchy-Lipschitz (théorème~\ref{ThokUUlgU}) sans la condition de Lipschitz.
	\item
	      Le théorème de Cauchy-Lipschitz~\ref{ThokUUlgU} est utilisé à plusieurs endroits :
	      \begin{itemize}
		      \item
		            Pour calculer la transformée de Fourier de \(  e^{-x^2/2}\) dans le lemme~\ref{LEMooPAAJooCsoyAJ}.
	      \end{itemize}
	\item
	      Théorème de stabilité de Lyapunov~\ref{ThoBSEJooIcdHYp}.
	\item
	      Le système proie-prédateur de Lotka-Volterra~\ref{ThoJHCLooHjeCvT}.
	\item
	      Équation de Schrödinger, théorème~\ref{ThoLDmNnBR}.
	\item
	      L'équation \( (x-x_0)^{\alpha}u=0\) pour \( u\in\swD'(\eR)\), théorème~\ref{ThoRDUXooQBlLNb}.
	\item
	      La proposition~\ref{PropMYskGa} donne un résultat sur \( y''+qy=0\) à partir d'une hypothèse de croissance.
	\item
	      Équation de Hill \( y''+qy=0\), proposition~\ref{PropGJCZcjR}.
\end{enumerate}
