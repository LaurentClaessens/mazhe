
\InternalLinks{convexité}
\begin{description}
	\item[Fonctions convexes]
	      L'essentiel des résultats sur les fonctions convexes sont dans la section~\ref{SECooVZWWooUjxXYi}. On a surtout :
	      \begin{enumerate}
		      \item
		            Définition des fonctions convexes :~\ref{DefVQXRJQz} et~\ref{DEFooKCFPooLwKAsS} en dimension supérieure.
		      \item
		            En termes de différentielles,~\ref{PROPooYNNHooSHLvHp} pour la différentielle première et~\ref{CORooMBQMooWBAIIH} pour la hessienne.
		      \item
		            Une courbe paramétrée convexe est la définition~\ref{DEFooVQODooJSNYLw}.
		      \item
		            L'enveloppe convexe d'une courbe fermée simple et convexe :~\ref{PROPooWZITooTFiWsi}.
		      \item
		            Courbure et convexité d'une courbe paramétrée : section~\ref{SUBSECooNJOLooYuGRjA}.
		      \item
		            Une courbe paramétrée convexe est localement le graphe d'une fonction convexe par le lemme~\ref{LEMooGEVEooHxPTMO}.
		      \item
		            La convexité est utilisée dans la méthode du gradient à pas optimal de la proposition~\ref{PropSOOooGoMOxG}.
		      \item
		            La fonction \( t\mapsto t^p\) est strictement convexe sur les positifs dans le lemme \ref{LEMooSXTXooZOmtKq}.
	      \end{enumerate}
	\item[Parties convexes]

	      En termes de parties convexes, on a :
	      \begin{enumerate}
		      \item
		            Définition \ref{DEFooQQEOooAFKbcQ} d'une partie convexe d'un espace vectoriel.
		      \item
		            Une boule est convexe, proposition \ref{PROPooUQLUooDQfYLT}.
                \item
                    Un polygone convexe est défini en \ref{PROPooFYRMooTqVDEm}, et les racines de l'unité forment un polygone convexe par la proposition \ref{PROPooUPPTooZBFvPg}.
	      \end{enumerate}
\end{description}

