\InternalLinks{compacts}        \label{THEMEooQQBHooLcqoKB}
\begin{description}

	\item[Propriétés générales]

		Quelques propriétés de compacts.

		\begin{enumerate}
			\item
			      La définition d'un ensemble compact est la définition~\ref{DefJJVsEqs}.
			\item
			      Ne pas confondre le compactifié d'Alexandrov \ref{PROPooHNOZooPSzKIN} avec la droite réelle achevée \ref{DEFooRUyiBSUooALDDOa}.
			\item
			      Une partie est relativement compacte si sa fermeture est compacte, définition \ref{DEFooBODRooEFhzeT}. Une union finie de relativement compacts est relativement compacte, lemme \ref{LEMooJXGUooXDGIXG}.
			\item
			      Si \( M\) est un compact de \( A\times B\), alors \( M\subset K\times L\) où \( K\) est compact de \( A\) et \( L\) de \( B\), proposition \ref{PropGBZUooRKaOxy}.
			\item
			      Un fermé dans un compact est compact, lemme \ref{LemnAeACf}. L'intersection entre un compact et un fermé est compacte, proposition \ref{CORooSSFFooNkNmlS}.
			\item
			      Dans un espace Hausdorff\footnote{Hausdorff, définition \ref{DefYFmfjjm}.}, les compacts sont fermés, \ref{LemnAeACf}\ref{ITEMooAZWVooLyPDeY}.
			\item
			      Un espace est compact si et seulement si toute intersection finie de fermé est non vide, théorème \ref{THOooCQSQooDuasqo}.
			\item
			      Tout compact d'un espace topologique séparé est fermé, lemme \ref{LemnAeACf}\ref{ITEMooAZWVooLyPDeY}.
			\item
			      Dans un espace vectoriel réel de dimension finie, les compacts sont les fermés bornés par le théorème~\ref{ThoXTEooxFmdI}.
			\item
			      Le théorème de Borel-Lebesgue \ref{ThoBOrelLebesgue} dit qu'un intervalle\footnote{Définition \ref{DefEYAooMYYTz}.} de \( \eR\) est compact si et seulement si il est de la forme \( \mathopen[ a , b \mathclose]\).
			\item
			      Théorème des fermés emboîtés dans le cas compact, corolaire \ref{CORooQABLooMPSUBf}. À ne pas confondre avec celui dans le cas des espaces métrique, théorème \ref{ThoCQAcZxX}.
			\item
			      Théorème des compacts emboîtés, proposition \ref{PROPooFXCXooXNeXWl}. Une intersection dénombrable de compacts emboîtés est compacte et non vide.
			\item
			      L'image d'un compact par une fonction continue est un compact, théorème~\ref{ThoImCompCotComp}.
			\item
			      Si \( f\colon K\to X\) est une bijection continue, sa réciproque est continue, lemme \ref{LEMooPLGTooATIGov}.
			\item
			      Suites dans un compact
			      \begin{enumerate}
				      \item
				            Toute suite dans un compact admet une sous-suite convergente, théorème \ref{THOooRDYOooJHLfGq}.
				      \item
				            Dans \( \eR^n\), toute suite dans un compact admet une sous-suite convergente, théorème \ref{ThoBolzanoWeierstrassRn}. La démonstration de ce théorème est non seulement plus compliquée que le cas général, mais utilise en plus le cas dans \( \eR\); lequel cas n'est pas démontré de façon directe dans le Frido.
				      \item
				            Un espace métrique est compact si et seulement si toute suite contient une sous-suite convergente. C'est le théorème de Bolzano-Weierstrass~\ref{ThoBWFTXAZNH}. La démonstration de ce théorème est indépendante.
				      \item
				            Si une suite de fonctions continues converge simplement vers une fonction continue, alors la convergence est uniforme sur tout compact, proposition \ref{PROPooFWVIooCzXojO}.
			      \end{enumerate}
			\item
			      Une fonction continue sur un compact est bornée et atteint ses bornes, théorème~\ref{ThoWeirstrassRn}.
			\item
			      Une fonction continue sur un compact y est uniformément continue, théorème de Heine \ref{PROPooBWUFooYhMlDp}.
			\item
			      Une bijection continue \( f\colon K\to X\) entre un compact et un séparé est un isomorphisme, \ref{LEMooNEEVooSeHYzx}.
		\end{enumerate}

	\item[Produits de compacts]
		Le théorème de Tychonoff dit que tout produit de compact est compact pour la topologie produit. Nous allons en voir diverses versions.
		\begin{itemize}
			\item
			      \( \eR\), lemme~\ref{LemCKBooXkwkte}.
			\item
			      Produit fini d'espaces métriques compacts, théorème~\ref{THOIYmxXuu}.
			\item
			      Produit dénombrable d'espaces métriques compacts, théorème~\ref{ThoKKBooNaZgoO}.
			\item
			      En général, théorème \ref{ThoFWXsQOZ}.
		\end{itemize}
	\item[Composante connexe]
		À propos de composantes connexes et de compacts.
		\begin{enumerate}
			\item
			      Si \( K\) est compact dans \( \eC\), alors la partie \( \eC\setminus K\) possède exactement une composante connexe non bornée. Lemme \ref{LEMooJNPTooScfSvA}.
		\end{enumerate}
\end{description}
