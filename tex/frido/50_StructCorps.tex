% This is part of Mes notes de mathématique
% Copyright (c) 2011-2020
%   Laurent Claessens
% See the file fdl-1.3.txt for copying conditions.

%---------------------------------------------------------------------------------------------------------------------------
\subsection{Corps de décomposition}
%---------------------------------------------------------------------------------------------------------------------------

\begin{definition}      \label{DEFooEKGZooSkvbum}
	Soit \( \eK\) un corps commutatif et \( F=(P_i)_{i\in I}\) une famille d'éléments non constants de \( \eK[X]\). Un \defe{corps de décomposition}{corps!de décomposition}\index{décomposition!corps} de \( F\) est une extension \( \eL\) de \( \eK\) telle que
	\begin{enumerate}
		\item
		      les \( P_i\) sont scindés sur \( \eL\),
		\item
		      \( \eL=\eK(R)\) où \( R=\bigcup_{i\in I}\{ x\in\eL\tq P_i(x)=0 \}\).
	\end{enumerate}
	C'est-à-dire que \( \eL\) étend \( \eK\) par toutes les racines de tous les polynômes de \( F\).
\end{definition}

\begin{proposition}[\cite{ooJTWFooASSMjI}]      \label{PROPooDPOYooFHcqkU}
	Tout polynôme admet un corps de décomposition. Plus précisément, soit un corps \( \eK\) et un polynôme \( P\in \eK[X]\) de degré \( n\). Il existe un corps de décomposition \( \eD\) de la forme \( \eD=\eK(\alpha_1,\ldots,\alpha_n)\) où les \( \alpha_i\) sont des racines de \( P\) dans \( \eD\).
\end{proposition}

Notons que rien dans l'énoncé ne prétend que les \( \alpha_i\) soient tous distincts, ni même que certains (voire tous) ne seraient pas dans \( \eK\).

\begin{proof}
	Soient un corps \( \eK\) et un polynôme \( P\in \eK[X]\). Si le degré de \( P\) est \( 0\) ou \( 1\), alors \( \eK\) est un corps de décomposition pour \( P\). Pour le reste nous faisons une récurrence sur le degré de \( P\).

	Il y a deux possibilités, soit il existe \( \alpha\in \eK\) tel que \( P(\alpha)=0\), soit non.

	\begin{subproof}
		\item[Si racine dans \( \eK\)]
		Alors le corolaire~\ref{CorDIYooEtmztc} nous permet de factoriser \( X-\alpha\) :
		\begin{equation}
			P=(X-\alpha)Q
		\end{equation}
		avec \( \deg(Q)=\deg(P)-1\). Dans ce cas, \( \eK\) est un corps de rupture de \( P\).

		\item[Si pas de racines dans \( \eK\)]

		Nous prenons alors un corps de rupture \( \eL=\eK(\alpha)\) avec \( P(\alpha)=0\) (c'est la proposition~\ref{PROPooUBIIooGZQyeE} qui donne l'existence d'un corps de rupture). Dans \( \eL_1\) nous avons
		\begin{equation}
			P=(X-\alpha)Q
		\end{equation}
		avec \( Q\in \eL_1[X]\) et \( \deg(Q)=\deg(P)-1\).

		\item[Dans les deux cas]

		Dans les deux cas, par hypothèse de récurrence nous avons un corps de décomposition pour \( Q\) qui se présente sous la forme
		\begin{equation}
			\eL=\eK(\alpha_1,\ldots, \alpha_{n-1}).
		\end{equation}
		De plus, \( \eL\) est une extension de \( \eL_1\) parce que c'est une extension du corps sur lequel est \( Q\).

	\end{subproof}
	Pour terminer la preuve nous prouvons que
	\begin{equation}
		\eD=\eK(\alpha_1,\ldots, \alpha_{n-1},\alpha)
	\end{equation}
	est un corps de décomposition de \( P\). Vu que \( \eD\) contient \( \eK(\alpha)\) (comme cas particulier du lemme \ref{LEMooQEJHooAmSNxU}), dans \( \eD\) nous avons l'égalité \( P=(X-\alpha)Q\). Et vu que \( \eD\) contient également \( \eK(\alpha_1,\ldots, \alpha_{n-1})\), toujours dans \( \eD\) nous avons aussi
	\begin{equation}
		Q=(X-\alpha_1)\ldots(X-\alpha_{n-1}).
	\end{equation}
	Donc nous avons dans \( \eD\) l'égalité
	\begin{equation}
		P=(X-\alpha)(X-\alpha_1)\ldots (X-\alpha_{n-1}).
	\end{equation}
\end{proof}

\begin{lemma}[\cite{MonCerveau}]        \label{LEMooJNGWooTXdGre}
	Soit un polynôme \( P\in \eK[X]\), et un corps \( \eL\) dans lequel \( P\) est scindé. Si \( P=P_1\ldots P_r\) est la décomposition de \( P\) en irréductibles dans \( \eK\), alors chacun des \( P_i\) est scindé dans \( \eL\).
\end{lemma}

\begin{proof}
	Juste pour le mentionner, le fait que \( P\) ait une décomposition en irréductibles est le fait que \( \eK[X]\) soit factoriel, c'est-à-dire la proposition~\ref{PropqGZXvr}.

	Le polynôme \( P\) est scindé dans \( \eL\), c'est-à-dire que, en notant \( n\) le degré de \( P\),
	\begin{equation}        \label{EQooNFXSooUkWeQu}
		P=\prod_{i=1}^n(X-\lambda_i)
	\end{equation}
	avec \( \lambda_i\in \eL\).

	Soit \( \eL_1\), une extension de \( \eL\) dans laquelle \( P_1\) est scindé. Ensuite, \( \eL_2\) une extension de \( \eL_1\) dans laquelle \( P_2\) est scindé et ainsi de suite. Nous avons construit \( \eL_r\), une extension de \( \eL\) dans laquelle tous les \( P_i\) sont scindés ainsi que \( P\) lui-même. Dans ce corps nous avons l'égalité
	\begin{equation}        \label{EQooBEFUooBnqSUS}
		P=\prod_{k=1}^n(X-\mu_k)
	\end{equation}
	où les \( \mu_k\) sont des éléments des diverses extensions \( \eL_i\), et sont les racines des \( P_i\). En tout cas, tous sont dans \( \eL_r\).

	Les deux décompositions \eqref{EQooNFXSooUkWeQu} et \eqref{EQooBEFUooBnqSUS} sont des décompositions dans \( \eL_r[X]\) du polynôme \( P\). Vu que ce dernier est factoriel, en réalité les deux décompositions sont identiques (se souvenir de la définition~\ref{DEFooVCATooPJGWKq}), et nous avons \( \mu_k\in \eL\) pour tout \( k\). Toutes les extensions \( \eL_i\) sont en réalité triviales, et nous avons \( \eL_r=\eL\).

	Bref, tout cela pour dire que les \( P_i\) ont toutes leurs racines dans \( \eL\).
\end{proof}

\begin{theorem}[\cite{ooUHHUooONXDDl}]      \label{THOooQVKWooZAAYxK}
	Soient :
	\begin{itemize}
		\item un isomorphisme de corps \( \tau\colon \eK\to \eK'\);
		\item un polynôme non constant \( P\in \eK[X]\) de degré \( n\);
		\item un corps de décomposition \( \eL\) de \( P\) sur \( \eK\);
		\item un corps de décomposition \( \eL'\) de \( P\) sur \( \eK'\);
	\end{itemize}
	Alors \( \tau\) se prolonge en un isomorphisme \( \sigma\colon \eL\to \eL'\).
\end{theorem}

\begin{proof}
	Soit \( m\) le nombre de racines de \( P\) dans \( \eL\setminus \eK\). Nous faisons une récurrence sur \( m\).

	Si \( m=0\) alors \( \eK\) est un corps de rupture de \( P\); nous avons
	\begin{equation}
		P=(X-\lambda_1)\ldots (X-\lambda_n)
	\end{equation}
	avec \( \lambda_i\in \eK\). Dans ce cas nous avons aussi
	\begin{equation}
		\tau(P)=\big( X-\tau(\lambda_1) \big)\ldots \big( X-\tau(\lambda_n) \big)
	\end{equation}
	avec \( \tau(\lambda_i)\in \eK'\). Nous avons donc \( \eL'=\eK'\) et prendre \( \sigma=\tau\) fonctionne.

	Nous supposons à présent que \( m>0\). Plus précisément nous considérons un polynôme possédant exactement \( m\) racines dans \( \eL\setminus \eK\). Soit
	\begin{equation}
		P=P_1\ldots P_r
	\end{equation}
	sa décomposition en irréductibles dans \( \eK[X]\) (notons que \( r\leq n\) parce que chacun des facteurs est de degré au moins \( 1\)). Au moins un des \( P_i\) est de degré plus grand ou égal à \( 2\). Nous savons de la proposition~\ref{PropqGZXvr} que \( \eK[X]\) est un anneau factoriel. Le lemme~\ref{LEMooJNGWooTXdGre} nous assure que les polynômes \( P_i\) sont également scindés dans \( \eL\). Et l'unicité de la décomposition fait en sorte que les racines des \( P_i\) sont celles de \( P\).

	Soit \( \alpha\in \eL\), une racine de \( P_1\). Vu que \( P_1\) est irréductible sur \( \eK\), l'application suivante est un isomorphisme de corps par le lemme~\ref{LEMooHKTMooKEoOuK} :
	\begin{equation}
		\begin{aligned}
			\psi\colon \eK[X]/(P_1) & \to \eK[\alpha]    \\
			\bar Q                  & \mapsto Q(\alpha).
		\end{aligned}
	\end{equation}
	Notons que le lemme parle du quotient par le polynôme minimal, mais ici nous avons un irréductible. Un polynôme annulateur irréductible est multiple du polynôme minimal, et l'idéal engendré est le même.

	Nous avons aussi la décomposition
	\begin{equation}
		\tau(P)=\tau(P_1)\ldots \tau(P_r),
	\end{equation}
	et chacun des \( \tau(P_i)\) a ses racines dans \( \eL'\). Soit \( \beta\), une racine de \( \tau(P_1)\) dans \( \eL'\). Alors nous avons l'isomorphisme
	\begin{equation}
		\psi'\colon \eK'[X]/\big( \tau(P_1) \big)\to \eK'[\beta].
	\end{equation}
	De plus, par le lemme~\ref{LEMooGRIMooPxCXAZ}, nous savons que \( \tau\) passe aux classes :
	\begin{equation}
		\phi_{\tau}\colon \eK[X]/(P_1)\to \eK'[X]/\big( \tau(P_1) \big)
	\end{equation}
	est un isomorphisme d'anneaux. Et enfin, dernier résultat externe à invoquer, la proposition~\ref{PropURZooVtwNXE} nous assure que \( \eK[\alpha]=\eK(\alpha)\) et \( \eK'[\beta]=\eK'(\beta)\). Posons pour l'occasion \( \eK_1=\eK(\alpha)\) et \( \eK'_1=\eK'(\beta)\).

	Nous avons l'enchainement suivant d'isomorphismes de corps\footnote{En réalité il est plus exact de dire «isomorphisme d'anneaux», parce que la structure de corps n'est en réalité aucune nouvelle structure par rapport à l'anneau.} :
	\begin{equation}
		\tau_1=\psi'\circ\phi_{\tau}\circ\psi^{-1} \colon \eK_1\to \eK[X]/(P_1)\to\eK'[X]/\big( \tau(P_1) \big)\to \eK'_1.
	\end{equation}
	Cet isomorphisme \( \tau_1\colon \eK_1\to \eK_2\) prolonge \( \tau\). Si vous aimez les diagrammes, en voici un sur lequel les \( i\) représentent des inclusions et où \( \tau\) et \( \tau_1\) sont des isomorphismes
	\begin{equation}
		\xymatrix{%
		\eK \ar[r]^-{i}\ar[d]_-{\tau}       &   \eK_1   \ar[d]^{\tau_1} \ar[r]^i    &   \eL\\
		\eK' \ar[r]_{i}                     &   \eK'_1  \ar[r]^{i}                  &   \eL'
		}
	\end{equation}
	Le corps \( \eL\) est un corps de décomposition de \( P\) sur \( \eK_1\), et le nombre de racines de \( P\) dans \( \eL\setminus\eK_1\) est strictement plus petit que \( m\) parce qu'il y en a exactement \( m\) dans \( \eL\setminus \eK\) et que \( \eK_1\) en a au moins une de plus que \( \eK\). Même raisonnement pour \( \eK'\), \( \eK'_1\) et \( \eL'\).

	Résumons la situation :
	\begin{itemize}
		\item \( \tau_1\colon \eK_1\to \eK'_1\) est un isomorphisme de corps;
		\item \( P\in \eK_1[X]\) est un polynôme non constant;
		\item \( \eL\) est un corps de décomposition de \( P\) sur \( \eK_1\);
		\item \( \eL'\) est un corps de décomposition de \( P\) sur \( \eK'_1\);
		\item le nombre de racines de \( P\) dans \( \eL\setminus \eK_1\) est strictement inférieur à \( m\).
	\end{itemize}
	Donc, par hypothèse de récurrence sur \( m\), il existe un isomorphisme \( \sigma\colon \eL\to \eL'\) qui prolonge \( \tau_1\). Vu que \( \tau_1\) prolonge \( \tau\), nous avons également \( \sigma\) qui prolonge \( \tau\).
\end{proof}

L'énoncé le plus compact pour l'unicité du corps de décomposition (à isomorphisme près) est le suivant :
\begin{proposition}     \label{PropTMkfyM}
	Soit \( \eK\) un corps et \( P\in\eK[X]\). Soient \( \eL\) et \( \eF\) deux corps de décomposition de \( P\). Alors il existe un isomorphisme \( f\colon \eL\to \eF\) tel que \( f|_{\eK}=\id\).
\end{proposition}
\begin{proof}
	C'est un cas particulier du théorème~\ref{THOooQVKWooZAAYxK}, où nous considérons \( \eK=\eK'\) muni de l'isomorphisme identité.
\end{proof}

Cependant le passage par l'énoncé plus compliqué~\ref{THOooQVKWooZAAYxK} est nécessaire pour les besoins de la récurrence.

\begin{normaltext}
	À propos de terminologie. Lorsque nous disons «\emph{un} corps de décomposition» nous référons à la définition~\ref{DEFooEKGZooSkvbum} et il n'y a pas vraiment unicité. Si nous disons «\emph{le} corps de décomposition» nous référons en général à celui construit dans la proposition~\ref{PROPooDPOYooFHcqkU} qui n'est en réalité même pas très explicite.

	Quoi qu'il en soit, nous nous permettons de dire «le» corps de décomposition lorsque nous parlons de propriétés invariantes par isomorphisme.
\end{normaltext}

\begin{normaltext}
	La construction du corps de décomposition d'un polynôme fonctionne en prenant successivement le corps de rupture des facteurs irréductibles. Nous insistons sur le fait que cette opération se fait sur chaque facteur irréductible séparément.

	L'exemple suivant montre dans quel ordre se passent les choses.

	\begin{example}
		Soit le polynôme \( P=X^4-5X^2+6\). Sa factorisation en irréductibles est :
		\begin{equation}
			P=(X^2-2)(X^2-3).
		\end{equation}
		Ce polynôme n'est pas irréductible sur \( \eQ\) et il ne s'agit donc pas de prendre brutalement un corps de rupture pour \( P\). Il s'agit de poser \( P=P_1P_2\) avec
		\begin{subequations}
			\begin{align}
				P_1 & =X^2-2  \\
				P_2 & =X^2-3,
			\end{align}
		\end{subequations}
		de remarquer que \( P_1\) et \( P_2\) sont irréductibles sur \( \eQ\) et de chercher des corps de rupture pour eux. On commence par \( P_1\). Nous avons le corps de rupture \( \eL_1=\eQ(\sqrt{ 2 })\) et la factorisation
		\begin{equation}
			P_1=(X+\sqrt{ 2 })(X-\sqrt{ 2 }).
		\end{equation}
		Ensuite nous considérons \( P_2\) dans \( \eL_1[X]\). Ce \( P_2\) est encore irréductible. Nous lui cherchons un corps de rupture, et c'est \( \eL_2=\eL_1(\sqrt{ 3 })\) dans lequel nous avons
		\begin{equation}
			P_2=(X-\sqrt{ 3 })(X+\sqrt{ 3 }).
		\end{equation}
		Nous savons (par le lemme~\ref{LEMooTURZooXnjmjT}) que
		\begin{equation}
			\eL_2=\eL_1(\sqrt{ 3 })=\big( \eQ(\sqrt{ 2 }) \big)(\sqrt{ 3 })=\eQ(\sqrt{ 2 },\sqrt{ 3 }).
		\end{equation}
		Nous pouvons donc écrire en toute confiance, dans \( \eQ(\sqrt{ 2 },\sqrt{ 3 })\) la factorisation
		\begin{equation}
			P=(X+\sqrt{ 2 })(X-\sqrt{ 2 })(X+\sqrt{ 3 })(X-\sqrt{ 3 }).
		\end{equation}

		Et nous notons que si nous avions commencé par \( P_2\) au lieu de \( P_1\), nous aurions eu le même résultat final.
	\end{example}
\end{normaltext}

\begin{corollary}[\cite{MonCerveau}]    \label{CORooELAUooPQGLkR}
	Le corps de décomposition d'un polynôme est une extension finie.
\end{corollary}

\begin{proof}
	Soient un corps \( \eK\), un polynôme \( P\in \eK[X]\) et un corps de décomposition \( \eD\) de \( P\) de la forme \( \eD=\eK(\alpha_1,\ldots, \alpha_n)\) où les \( \alpha_i\) sont les racines de \( P\) dans \( \eD\). Cela existe par la proposition~\ref{PROPooDPOYooFHcqkU}.

	Vu que le lemme~\ref{LEMooTURZooXnjmjT} donne
	\begin{equation}
		\eK(\alpha_1,\ldots, \alpha_n)=\big( \eK(\alpha_1,\ldots, \alpha_{n-1}) \big)(\alpha_n),
	\end{equation}
	le corps \( \eD\) se construit comme une pile d'extensions finies. Les degrés se composant par le lemme~\ref{PROPooEGSJooBSocTf}, au final ce corps de décomposition est une extension finie.

	Soit maintenant un corps de décomposition quelconque \( \eL\). La proposition~\ref{PropTMkfyM} donne un isomorphisme de corps\footnote{Un isomorphisme de corps est juste un isomorphisme d'anneaux.} \( f\colon \eL\to \eD\) tel que \( f\) soit l'identité sur \( \eK\).

	Si \( \{ v_i \}_{i\in I}\) est une base de \( \eD\) comme espace vectoriel sur \( \eK\), êtes-vous prêts à parier que \( \{ f(v_i) \}_{i\in I}\) est une base de \( \eL\) comme espace vectoriel sur \( \eK\)\quext{Personnellement, je n'ai pas vérifié. Vérifiez vous-même et écrivez-moi pour dire si c'est bon ou non.} ?
\end{proof}

%---------------------------------------------------------------------------------------------------------------------------
\subsection{Non irréductible ou pas corps ?}
%---------------------------------------------------------------------------------------------------------------------------
\label{SUBSECooEDMJooTXBfOu}

Nous avons déjà mentionné que nous ne définissons le corps de rupture d'un polynôme que dans le cas de polynôme irréductible à coefficients dans un corps.

D'abord si \( P\) n'est pas irréductible, la question de chercher un corps de rupture n'a pas beaucoup de sens.

Si \( A\) est un anneau intègre et si \( P\) est un polynôme irréductible sur \( A\), nous pouvons considérer le corps des fractions de \( A\), dire \( P\in\Frac(A)[X]\) et continuer. Étendre la définition de corps de rupture de cette façon aux polynômes à coefficients dans un anneau intègre n'est pas une grande révolution.

Au lieu de cela, nous pouvons nous demander dans quel cas nous aurions que \( A[X]/(P)\) est directement un corps.

\begin{example}
	Soit le polynôme constant \( P=2\) dans \( \eZ[X]\). Il y est irréductible parce qu'il ne peut pas être écrit comme produit de deux non inversibles. Ce polynôme a deux propriétés ennuyeuses :
	\begin{itemize}
		\item Il n'est plus irréductible sur \( \eQ\),
		\item Il n'existe aucun corps contenant une racine de \( P\) tout en contenant \( \eZ\) comme sous-anneau.
	\end{itemize}
\end{example}

%---------------------------------------------------------------------------------------------------------------------------
\subsection{Clôture algébrique}
%---------------------------------------------------------------------------------------------------------------------------

\begin{theorem}     \label{THOooQFWWooMWXEhT}
	Tout corps \( \eK\) possède une clôture algébrique\footnote{Définition \ref{DEFooREUHooLVwRuw}.} \( \Omega\). De plus si \( \eL\) est une extension de \( \eK\), alors \( \eL\) est \( \eK\)-isomorphe à un sous corps de \( \Omega\).
\end{theorem}
Les deux parties de ce théorème utilisent l'axiome du choix.

Notons en particulier que si \( \Omega'\) est une autre clôture algébrique de \( \eK\), alors \( \Omega\) et \( \Omega'\) sont des sous corps l'un de l'autre et sont donc \( \eK\)-isomorphes.

\begin{lemma}
	Les polynômes \( P,Q\in \eK[X]\) ne sont pas premiers entre eux si et seulement s'ils ont une racine commune dans la clôture algébrique \( \Omega\) de \( \eK\).
\end{lemma}

\begin{proof}
	Soit \( A\) un polynôme non inversible divisant \( P\) et \( Q\). Par définition de \( \Omega\), ce polynôme \( A\) a une racine dans \( \Omega\) qui est alors une racine commune à \( P\) et \( Q\) dans \( \Omega\).

	Pour le sens inverse, si \( \alpha\) est une racine commune de \( P\) et \( Q\), alors le polynôme \( X-\alpha\) divise \( P\) et \( Q\) et donc \( P\) et \( Q \) ne sont pas premiers entre eux.
\end{proof}

\begin{example}     \label{ExfUqQXQ}
	Soit \( p\) un nombre premier. Montrons que le polynôme
	\begin{equation}
		Q(X)=X^p-X+1
	\end{equation}
	est irréductible dans \( \eF_p\).

	Nous supposons qu'il n'est pas irréductible, c'est-à-dire que
	\begin{equation}
		Q(X)=R(X)S(X)
	\end{equation}
	avec \( R\) et \( S\), des polynômes de degrés \( \geq 1\) dans \( \eF_p[X]\)

	Soit \( \bar\eF_p\) une clôture algébrique\footnote{Définition \ref{DEFooREUHooLVwRuw}. Pour l'existence c'est le théorème \ref{THOooQFWWooMWXEhT}.} de \( \eF_p\) et \( \alpha\in \bar \eF_p\) tel que \( R(\alpha)=0\). Pour tout \( a\in \eF_p\), nous avons
	\begin{subequations}
		\begin{align}
			Q(\alpha+a) & =(\alpha+a)^p-(\alpha+a)+1 \\
			            & =\alpha^p+a^p-\alpha-a+1   \\
			            & =\alpha^p-\alpha+1         \\
			            & =Q(\alpha)                 \\
			            & =0
		\end{align}
	\end{subequations}
	où nous avons utilisé le fait que \( a^p=a\) et que \( \alpha\) était une racine de \( Q\). Ce que nous venons de prouver est que l'ensemble des racines de \( Q\) dans \( \bar\eF_p\) est donné par \( \{ \alpha+a\tq a\in \eF_p \}\).

	Les polynômes \( R\) et \( S\) sont donc formés de produits de termes \( X-(\alpha+a)\) avec \( a\in \eF_p\). L'un des deux --disons \( R\) pour fixer les idées-- doit bien en avoir plus que \( 1\). Nous avons alors
	\begin{equation}
		R(X)=\prod_{i=1}^{k}\big( X-(\alpha+a_i) \big)
	\end{equation}
	où les \( a_i\) sont les éléments de \( \eF_p\). En développant un peu,
	\begin{equation}
		R(X)=X^k-\sum_{i=1}^k(\alpha+a_i^{k-1})+\text{termes de degré plus bas en } X.
	\end{equation}
	Le coefficient devant \( X^{k-1}\) n'est autre que \( k\alpha+\sum_ia_i\). Étant donné que \( k\neq 0\) et que \( R\in \eF_p[X]\), nous devons avoir \( \alpha\in \eF_p\). Par conséquent nous avons \( \alpha^p=\alpha\) et une contradiction :
	\begin{equation}
		Q(\alpha)=\alpha^p-\alpha+1=1\neq 0.
	\end{equation}

	Le polynôme \( X^p-X+1\) est donc irréductible sur \( \eF_p\).
\end{example}

%---------------------------------------------------------------------------------------------------------------------------
\subsection{Extensions séparables}
%---------------------------------------------------------------------------------------------------------------------------

Notons que dans ce qui va suivre nous allons parler de \( \eK[X]\), l'ensemble des polynômes sur un corps. Cela ne s'applique donc pas à \( \eZ[X]\) par exemple.

Une des choses intéressantes avec les extensions séparables c'est qu'elles vérifient le théorème de l'élément primitif~\ref{ThoORxgBC}.

\begin{definition}      \label{DEFooLXSBooCHIUFU}
	Soit \( \eK\) un corps. Un polynôme \emph{irréductible} \( P\in \eK[X]\) est \defe{séparable}{séparable!polynôme irréductible}\index{polynôme!irréductible!séparable} sur \( \eK\) si dans un corps de décomposition, ses racines sont distinctes, c'est-à-dire que si \( P\) est de degré \( n\), alors il possède \( n\) racines distinctes dans un corps de décomposition.

	Si \( P\) est un polynôme non constant dont la décomposition en irréductibles est \( P=P_1\ldots P_r\), nous disons qu'il est \defe{séparable}{séparable!polynôme non constant}\index{polynôme!séparable} si tous les \( P_i\) le sont.
\end{definition}

La proposition suivante donne un sens à la définition de polynôme irréductible séparable.
\begin{proposition}
	Soit \( P\) irréductible dans \( \eK[X]\) ayant des racines distinctes dans le corps de décomposition \( \eL\). Si \( \eL'\) est un autre corps de décomposition pour \( P\), alors \( P\) a aussi ses racines distinctes dans \( \eL'\).
\end{proposition}

\begin{proof}
	L'ingrédient est la proposition~\ref{PropTMkfyM} qui donne l'unicité du corps de décomposition à \( \eK\)-isomorphisme près. Soit donc \( \psi\colon \eL\to \eL'\) un isomorphisme laissant invariant les éléments de \( \eK\). D'une part, étant donné que \( P\) est à coefficients dans \( \eK\), nous avons \( \psi(P)=P\). D'autre part dans \( \eL\) le polynôme \( P\) s'écrit
	\begin{equation}
		P=a(X-\alpha_1)\ldots (X-\alpha_n)
	\end{equation}
	avec \( a\in \eK\) et \( \alpha_i\in \eL\). Nous avons donc
	\begin{equation}
		P=\psi(P)=a(X-\psi(\alpha_1))\ldots (X-\psi(\alpha_n)).
	\end{equation}
	Donc les racines de \( P\) dans \( \eL'\) sont les éléments \( \psi(\alpha_i)\) qui sont distincts.
\end{proof}

\begin{example}
	Un polynôme peut être séparable sur un corps, mais non séparable sur un autre. Soit \( \eL=\eF_p(T)\) et \( \eK=\eF_p(T^p)\). Nous considérons le polynôme
	\begin{equation}
		P=X^p-T^p
	\end{equation}
	dans \( \eK[X]\). Par le morphisme de Frobenius nous avons
	\begin{equation}
		P=(X-T)^p
	\end{equation}
	dans \( \eL[X]\). Le polynôme \( P\) est irréductible sur \( \eK[X]\) parce que ses diviseurs sont de la forme \( (X-T)^k\) qui contiennent \( T^k\) qui n'est pas dans \( \eK\) (sauf si \( k=n\) ou \( k=0\)).

	Ce polynôme n'est pas séparable sur \( \eK\) parce que dans le corps de décomposition \( \eL\), la racine \( T\) est multiple. Notons bien le raisonnement : \( P\) étant irréductible, pour savoir si il est séparable, on le regarde dans un corps de décomposition.

	Par contre si nous regardons \( P\) dans \( \eL[X]\) alors \( P\) n'est plus irréductible parce que ses facteurs irréductibles sont \( (X-T)\). N'étant pas irréductible, nous regardons les racines de \emph{ses facteurs irréductibles}. Or chacun des facteurs irréductibles étant \( X-T\), les racines sont simples.
\end{example}

\begin{example}
	Le polynôme \( (X-1)^3\) est séparable sur \( \eQ\) parce que ses facteurs irréductibles dans \( \eQ[X]\) sont \( X-1\) et \(X^2 + X + 1\), et ces deux polynômes ont des racines simples (dans \( \eQ(i)\)).
\end{example}

\begin{example}
	Le polynôme \( (X^2+1)^2\) est séparable dans \( \eQ[X]\). En effet, il a pour facteurs irréductibles le polynôme \( X^2+1\) (en deux exemplaires), et ce polynôme a pour racines \( \pm i\) dans l'extension \( \eQ(i)\), racines qui sont simples pour ce polynôme.
\end{example}

\begin{proposition}[\cite{vgQYwF}]  \label{PropolyeZff}
	Soit \( P\in \eK[X]\) un polynôme non constant. Les propriétés suivantes sont équivalentes.
	\begin{enumerate}
		\item\label{ItemdqPFUi}
		      \( P\) a une racine multiple dans une extension de \( \eK\). C'est-à-dire qu'il existe une extension de \( \eK\) dans laquelle \( P\) a une racine multiple.
		\item\label{ItemdqPFUib}
		      \( P\) a une racine multiple dans tout corps de décomposition .
		\item\label{ItemdqPFUii}
		      \( P\) et \( P'\) ont une racine commune dans une extension de \( \eK\).
		\item\label{ItemdqPFUiii}
		      le degré de \( \pgcd(P,P')\) est \( \geq 1\).
	\end{enumerate}
\end{proposition}
\index{corps!extension}

\begin{proof}
	\begin{subproof}
		\item[\ref{ItemdqPFUi}\( \Rightarrow\)\ref{ItemdqPFUib}] Soit \( a\), une racine multiple de \( P\) dans une extension \( \eL\) de \( \eK\), et \( \eE\), un corps de décomposition de \( P\). Alors nous voulons prouver que \( P\) ait une racine multiple dans \( \eE\).

		Nous pouvons voir \( P\in \eL[X]\), et construire une corps de décomposition \( \eE'\) qui est une extension de \( \eL\). Vu que \( \eE\) et \( \eE'\) sont deux corps de décomposition de \( P\)
		% iDIUoR
		nous avons un isomorphisme \( \psi\colon \eE'\to \eE\). Si \( a\in \eE\) est une racine multiple de \( P\), alors \( \psi(a)\) est une racine multiple de \( P\) dans \( \eE'\) parce que
		\begin{equation}
			P\big( \psi(a) \big)=\psi\big( P(a) \big).
		\end{equation}
		\item[\ref{ItemdqPFUib}\( \Rightarrow\)\ref{ItemdqPFUii}] Soit \( \eL\) un corps de décomposition de \( P\) sur \( \eK\) et \( a\in \eL\), une racine multiple de \( P\). On a alors \( P=(X-a)^2Q\) avec \( Q\in \eL[X]\). En dérivant,
		\begin{equation}
			P'=2(X-a)Q+(X-a)^2Q',
		\end{equation}
		et donc \( a\) est également une racine de \( P'\).
		\item[\ref{ItemdqPFUii}\( \Rightarrow\)\ref{ItemdqPFUiii}] Soit \( D\) un \( \pgcd\) de \( P\) et \( P'\). D'après le théorème de Bézout il existe \( A,B\in \eK[X]\) tels que
		\begin{equation}
			AP+BP'=D.
		\end{equation}
		Si \( a\) est une racine commune de \( P\) et \( P'\) dans une extension \( \eL\), alors c'est aussi une racine de \( D\) et donc \( \deg(D)\geq 1\).
		\item[\ref{ItemdqPFUiii}\(\Rightarrow\)\ref{ItemdqPFUi}] Si le degré de \( D\) est plus grand ou égal à \( 1\), alors nous considérons une racine \( a\) de \( D\) dans \( \eL\) (une extension de \( \eK\)). Étant donné que \( D\) divise \( P\) et \( P'\), l'élément \( a\) est une racine commune de \( P\) et \( P'\). Nous montrons maintenant que \( a\) est alors une racine multiple de \( P\). Vu que \( P(a)=0\) nous avons
		\begin{equation}
			P=(X-a)Q,
		\end{equation}
		et \( P'=Q+(X-a)Q'\). Mais alors \( P'(a)=Q(a)\) et donc \( Q(a)=0\) et donc \( a\) est une racine double de \( P\). Par conséquent \( a\) est une racine multiple de \( P\) dans \( \eK\).
	\end{subproof}
\end{proof}
Notons que si \( P\) est irréductible, cette proposition donne des conditions pour que \( P\) ne soit pas séparable.

\begin{proposition}
	Soit \( P\in \eK[X]\) irréductible. Le polynôme \( P\) est séparable si et seulement si \( P'\neq 0\).
\end{proposition}

\begin{proof}
	Soit \( D=\pgcd(P,P')\) et nous voudrions prouver que \( \deg(D)\geq 1\) si et seulement si \( P'=0\). Si \( P'=0\), alors \( \pgcd(P,P')=P\) est donc \( \deg(D)\geq 1\).

	Dans l'autre sens, si \( P\) est irréductible, il est associé à \( D\) parce qu'il n'a pas d'autres diviseurs que lui-même et le polynôme constant \( 1\). Ainsi, \( D \in \eK \), ou bien \( P=\lambda D\) avec \( \lambda\in \eK\). et donc \( \deg(P)\geq 1\). Dans les deux cas, \( P' \) est nécessairement non-nul.
\end{proof}

\begin{corollary}   \label{CorUjfJSE}
	Si \( \eK\) est de caractéristique nulle, alors tout polynôme de \( \eK[X]\) est séparable.
\end{corollary}

\begin{proof}
	Il suffit de montrer que les irréductibles sont séparables. Soit \( P\) un polynôme irréductible et unitaire de degré \( d\). Le terme de plus haut degré de \( P'\) est alors \( dX^{d-1}\) qui est non nul parce que \( d\neq 0\) en caractéristique nulle. Donc \( P'\neq 0\) et donc \( P\) est séparable par la proposition~\ref{PropolyeZff}.
\end{proof}

\begin{definition}      \label{DEFooKTVHooTydOTM}
	Soit \( \eL\) une extension algébrique de \( \eK\).
	\begin{enumerate}
		\item       \label{ITEMooOFYPooLYkIPr}
		      On dit que l'élément \( a\in \eL\) est \defe{séparable}{séparable!élément d'une extension} sur \( \eK\) si son polynôme minimal dans \( \eK[X]\) est séparable sur \( \eK\) (définition~\ref{DEFooLXSBooCHIUFU}).
		\item
		      L'extension \( \eL\) est \defe{séparable}{séparable!extension de corps} si tous ses éléments sont séparables.
	\end{enumerate}
\end{definition}
\index{extension!séparable}

\begin{proposition} \label{PropUmxJVw}
	Soit \( \eK\) un corps. Les conditions suivantes sont équivalentes :
	\begin{enumerate}
		\item       \label{ITEMooUSKRooDmsGmw}
		      toutes les extensions algébriques de \( \eK\) sont séparables;
		\item       \label{ITEMooJGWLooKInxSG}
		      tout polynôme irréductible de \( \eK[X]\) est séparable.
	\end{enumerate}
	En particulier les extensions algébriques des corps de caractéristique nulle sont toutes séparables.
\end{proposition}

\begin{proof}

	En plusieurs parties.

	\begin{subproof}
		\item[\ref{ITEMooUSKRooDmsGmw} implique~\ref{ITEMooJGWLooKInxSG}]

		Soit un polynôme irréductible \( P\) de \( \eK[X]\), et un corps de décomposition \( \eL\) de \( P\). Cela est une extension algébrique par le corolaire~\ref{CORooELAUooPQGLkR}. Elle est donc séparable par hypothèse.

		Voilà une première chose de dite.

		Maintenant, nous voudrions montrer que \( P\) est un polynôme séparable. Dans \( \eL\) nous avons
		\begin{equation}
			P=\prod_{i=1}^n(X-a_i),
		\end{equation}
		et tout le défi est de prouver que les \( a_i\) sont tous distincts.

		Soient deux racines \( a,b\in \eL\) de \( P\). Nous considérons les polynômes minimaux \( \mu_a\) et \( \mu_b\) dans \( \eK[X]\). Ces deux polynômes divisent \( P\) parce que \( P\) est à la fois dans l'idéal annulateur de \( a\) et de \( b\). Mais comme \( P\) est irréductible, il existe \( k_a,k_n\in \eK\) tels que \( P=k_a\mu_a\) et \( P=k_b\mu_b\). Donc les polynômes \( \mu_a,\mu_b\) et \( P\) sont multiples les uns des autres. Vu que \( \mu_a\) et \( \mu_n\) sont unitaires, \( \mu_a=\mu_b\).

		Nous avons :
		\begin{equation}
			P=k\mu=\prod_{i=1}^n(X-a_i).
		\end{equation}
		Or le polynôme \( \mu\) est irréductible par la proposition~\ref{PropRARooKavaIT}\ref{ItemDOQooYpLvXri}, et l'extension \( \eL\) est séparable, donc \( \mu\) n'a que des racines simples, Donc tous les \( a_i\) sont distincts.

		\item[\ref{ITEMooJGWLooKInxSG} implique~\ref{ITEMooUSKRooDmsGmw}]

		Soit une extension algébrique \( \eL\) de \( \eK\). Soit \( a\in \eL\). Nous devons prouver que le polynôme minimal de \( a\) dans \( \eK\) est séparable, c'est-à-dire qu'il n'a que des racines simples.

		Le polynôme minimal \( \mu_a\in \eK[X]\) de \( a\) est irréductible et donc séparable. Donc \( \eL\) est séparable.

	\end{subproof}

	La dernière phrase est une conséquence du corolaire~\ref{CorUjfJSE}.
\end{proof}

\begin{corollary}  \label{CORooNZZMooIoBYXY}
	Toute les extensions algébriques de \( \eQ\) sont séparables.
\end{corollary}

\begin{proof}
	Le corps \( \eQ\) est de caractéristique nulle (définition~\ref{LEMDEFooVEWZooUrPaDw}). Le corolaire~\ref{CorUjfJSE} dit alors que tout polynôme sur \( \eQ\) est séparable. La proposition~\ref{PropUmxJVw} conclut en disant que toutes les extensions algébriques de \( \eQ\) sont séparables.
\end{proof}

\begin{theorem}[\cite{rqrNyg}]      \label{ThobkwCMm}
	Soit \( \eK\) un corps (pas spécialement fini). Tout sous-groupe fini de \( \eK^*\) est cyclique.
\end{theorem}

\begin{proof}
	Soit \( G\) un sous-groupe fini de \( \eK^*\) et \( \omega\) son exposant (qui est le PPCM des ordres des éléments de \( G\)). Étant donné que \( | G |\) est divisé par tous les ordres, il est divisé par le PPCM des ordres. Bref, nous avons
	\begin{equation}
		x^{\omega}=1
	\end{equation}
	pour tout \( x\in G\). Mais ce polynôme possède au plus \( \omega\) racines dans \( \eK\). Du coup \( | G |\leq \omega\). Et comme on avait déjà vu que \( \omega\divides | G |\), on a \( \omega=| G |\). Il suffit plus que trouver un élément d'ordre effectivement \( \omega\). Cela est fait par le lemme~\ref{LemqAUBYn}.
\end{proof}

\begin{theorem}[Théorème de l'élément primitif\cite{rqrNyg}]   \label{ThoORxgBC}
	Toute extension de corps séparable finie admet un élément primitif\footnote{Définition~\ref{DefZCYIbve}.}.

	Autrement dit, soient des éléments algébriques \( \alpha_1,\ldots, \alpha_n\) séparables\footnote{Définition \ref{DEFooKTVHooTydOTM}\ref{ITEMooOFYPooLYkIPr}.} sur \( \eK\), et soit l'extension engendrée \( \eL=\eK(\alpha_1,\ldots, \alpha_n)\). Alors \( \eL \) admet un élément primitif, c'est-à-dire un élément \( \theta \) tel que \( \eL = \eK(\theta)\).
\end{theorem}
\index{théorème!élément primitif}

\begin{proof}
	Si le corps \( \eK\) est fini, alors \( \eL\) est également fini. Donc \( \eL^*\) est cyclique par le théorème~\ref{ThobkwCMm}. Si \( \theta\) est un générateur de \( \eL^*\), alors \( \eL=\eK(\theta)\).

	Passons au cas où \( \eK\) est infini. Il suffit d'examiner le cas \( n=2\); en effet pour \( n=1\) c'est trivial et si \( n>2\), alors
	\begin{equation}
		\eK(\alpha_1,\ldots, \alpha_n)=\eK(\alpha_1,\ldots, \alpha_{n-1})(\alpha_n),
	\end{equation}
	et donc si \( \eK(\alpha_1,\ldots, \alpha_{n-1})=\eK(\theta)\), nous avons
	\begin{equation}
		\eK(\alpha_1,\ldots, \alpha_n)=\eK(\theta,\alpha_n)
	\end{equation}
	et nous sommes réduit au cas \( n=2\) par récurrence.

	Soit donc \( \eL=\eK(\alpha,\beta)\); soit \( P\) le polynôme minimal de \( \alpha\) sur \( \eK\) et \( Q\) celui de \( \beta\). Nous nommons \( \eE\), un corps de décomposition de \( PQ\). Nous avons \( \eL\subset \eE\). Vu que \( P\) et \( Q\) sont polynômes minimaux d'éléments qui sont par hypothèse séparables, les polynômes \( P\) et \( Q\) sont séparables. Donc dans \( \eE\) les racines de \( P\) sont distinctes parce que \( P\) est irréductible (et idem pour \( Q\)). Soient les racines
	\begin{equation}
		\alpha_1=\alpha,\alpha_2,\ldots, \alpha_r
	\end{equation}
	de \( P\) dans \( \eE\) et les racines
	\begin{equation}
		\beta_1=\beta,\beta_2,\ldots, \beta_s
	\end{equation}
	de \( Q\) dans \( \eE\). Ici \( r\) et \( s\) sont les degrés de \( P\) et \( Q\).

	Si \( s=1\) alors \( Q=X-\beta\) et donc \( \beta\in \eK\) (parce que \( Q\in \eK[X]\)). Du coup nous avons \( \eL=\eK(\alpha)\) et le théorème est démontré. Nous supposons donc maintenant que \( s\geq 2\).

	Pour chaque \( (i,j)\in \llbracket 1,r\rrbracket \times \llbracket 2,s\rrbracket \), l'équation \( \alpha_i+x\beta_k=\alpha_1+x\beta_1\) pour \( x\in \eK\) a au plus\footnote{La solution \eqref{EqWzUFHe} peut être dans \(  \eL\) et non dans \( \eK\). L'équation peut donc très bien ne pas avoir de solutions \( x\in \eK\).} une solution donnée le cas échéant par
	\begin{equation}    \label{EqWzUFHe}
		x=(\alpha_i-\alpha_1)(\beta_1-\beta_k)^{-1}
	\end{equation}
	Notons que cela est de toutes façons dans \( \eL\) et qu'étant donné que \( \beta_1\neq \beta_k\), cette solution a un sens (ici on utilise l'hypothèse de séparabilité). Étant donné que \( \eK\) est infini nous pouvons donc trouver un \( c\in \eK\) qui ne résout aucune des équations \eqref{EqWzUFHe} :
	\begin{equation}\label{EQooIIMVooSmvrjP}
		\alpha_i+c\beta_k\neq \alpha_1+c\beta_1.
	\end{equation}
	Nous posons \( \theta=\alpha_1+c\beta_1\) et nous prétendons que \( \eL=\eK(\theta)\).

	Pour cela, commençons par montrer que \( \beta_1 \in \eK(\theta)\). On considère, dans \( \eK(\theta)[T]\), les polynômes \( Q(T)\) et \( S(T)=P(\theta-cT)\), et on nomme \( R\) le PGCD de ces deux polynômes. Alors, une racine de \( R\) doit être une racine de \( Q\), et est donc un des \( \beta_i\). Or, d'une part, le choix de \( \theta\) fait que \( \beta_1\) est une racine de \( R\) parce que
	\begin{equation}
		S(\beta_1)=P(\theta-c\beta_1)=P(\alpha_1+c\beta_1-c\beta_1)=P(\alpha_1)=0.
	\end{equation}
	D'autre part, si \( k\geq 2\), alors
	\begin{equation}
		S(\beta_k)=P(\alpha_1 + c \beta_1 - c \beta_k) = P\big(\alpha_1 +c(\beta_1-\beta_k)\big)\neq 0
	\end{equation}
	parce que \( \alpha_1 +c(\beta_1 - \beta_k)\) ne vaut ni \( \alpha_1 \) (le second terme est non-nul), ni un autre \( \alpha_i\) (à cause de \eqref{EQooIIMVooSmvrjP}).

	Il s'ensuit que \( Q \) et \(S \) n'ont qu'une racine commune \( \beta_1 = \beta \), qui est donc l'unique racine de \( R\). Ainsi,
	\begin{equation}
		R=X-\beta\in \eK(\theta)[T],
	\end{equation}
	et donc \( \beta\in \eK(\theta)\).

	Dès lors \( \alpha=\alpha_1=\theta-c\beta\) est alors immédiatement dans \( \eK(\theta)\); puisque les deux éléments \( \alpha\) et \( \beta\) sont dans \( \eK(\theta)\), nous avons obtenu \( \eL=\eK(\alpha,\beta)=\eK(\theta)\).

\end{proof}

\begin{example}
	Le théorème de l'élément primitif~\ref{ThoORxgBC} ne tient pas pour les corps non commutatifs. Considérons par exemple le corps \( \eK\) des quaternions\index{quaternion} et le groupe à \( 8\) éléments \( G=\{ \pm 1,\pm i,\pm j,\pm k \}\). Ce dernier groupe n'est pas cyclique alors qu'il est un groupe fini dans \( \eK^*\).
\end{example}

\begin{example}
	Il est aussi possible pour un groupe fini d'avoir \( \omega(G)=| G |\) sans pour autant que \( G\) soit cyclique. Par exemple pour \( G=S_3\), nous avons \( | S_3 |=6\) alors que les éléments de \( S_3\) sont soit d'ordre \( 2\) soit d'ordre \( 3\) et \( \omega(G)=\ppcm(2,3)=6\). Pourtant \( S_3\) n'est pas cyclique.
\end{example}

%+++++++++++++++++++++++++++++++++++++++++++++++++++++++++++++++++++++++++++++++++++++++++++++++++++++++++++++++++++++++++++
\section{Idéal maximum}
%+++++++++++++++++++++++++++++++++++++++++++++++++++++++++++++++++++++++++++++++++++++++++++++++++++++++++++++++++++++++++++

%---------------------------------------------------------------------------------------------------------------------------
\subsection{Idéal maximum}
%---------------------------------------------------------------------------------------------------------------------------

\begin{definition}  \label{DefWHDdTrC}
	Une \( \eK\)-algèbre est de \defe{type fini}{type!fini!en algèbre} si elle est le quotient de \( \eK[X_1,\ldots, X_n]\) par un idéal (pour un certain \( n\)).
\end{definition}

\begin{theorem}[\cite{OorXst}]      \label{ThonoZyKa}
	Soit \( \eK\) un corps et \( B\), une \( \eK\)-algèbre de type fini. Si \( B\) est un corps, alors c'est une extension algébrique finie de \( \eK\).
\end{theorem}
%TODO : faire la démonstration

\begin{theorem}[\cite{OorXst}]  \label{ThowgZYqx}
	Si \( \eK\) est un corps algébriquement clos, les idéaux maximaux de \( \eK[X_1,\ldots, X_n]\) sont de la forme
	\begin{equation}
		(X_1-a_1,\ldots, X_n-a_n)
	\end{equation}
	où les \( a_i\) sont des éléments de \( \eK\).
\end{theorem}

\begin{proof}
	Nous commençons par montrer que
	\begin{equation}
		J=(X_1-a_1,\ldots, X_n-a_n)
	\end{equation}
	est un idéal maximum. Pour cela nous considérons le morphisme surjectif d'anneaux
	\begin{equation}
		\begin{aligned}
			\phi\colon \eK[X_1,\ldots, X_n] & \to \eK                     \\
			P                               & \mapsto P(a_1,\ldots, a_n).
		\end{aligned}
	\end{equation}
	Soit \( P\in\ker(\phi)\); nous écrivons la division euclidienne de \( P\) par \( X-a_1\) puis celle du reste par \( X-a_2\) et ainsi de suite :
	\begin{equation}    \label{EqDAkijH}
		P=(X-a_1)Q_1+\cdots +(X_n-a_n)Q_n+R
	\end{equation}
	où \( R\) doit être une constante parce que le premier reste est de degré zéro en \( X_1\), le second est de degré zéro en \( X_1\) et \( X_2\), etc. Afin d'identifier cette constante, nous appliquons l'égalité \eqref{EqDAkijH} à \( (a_1,\ldots, a_n)\) et en nous rappelant que \( P\in \ker(\phi)\) nous obtenons
	\begin{equation}
		0=P(a_1,\ldots, a_n)=R,
	\end{equation}
	donc \( R=0\) et \( P=(X_1-a_1)Q_1+\cdots +(X_n-a_n)Q_n\), c'est-à-dire \( P\in J\). Nous avons donc \( \ker(\phi)\subset J\). Par ailleurs \( J\subset \ker(\phi)\) est évident, donc \( J=\ker(\phi)\).

	Vu que \( J\) est le noyau de l'application \( \eK[X_1,\ldots, X_n]\to \eK\), nous avons
	\begin{equation}
		\frac{ \eK[X_1,\ldots, X_n] }{ J }=\eK.
	\end{equation}
	Donc \( J\) est un idéal maximal parce que tout polynôme n'étant pas dans \( J\) doit avoir un terme indépendant non nul et donc être dans \( \eK\) vis à vis du quotient \( \eK[X_1,\ldots, X_n]/J\).

	Nous montrons maintenant l'implication inverse. Nous supposons que \( I\) est un idéal maximum et nous montrons qu'il doit être égal à \( J\) (pour un certain choix de \( a_1,\ldots, a_n\)).

	Le quotient
	\begin{equation}
		\frac{ \eK[X_1,\ldots, X_n] }{ I }
	\end{equation}
	est une \( \eK\)-algèbre de type fini (définition~\ref{DefWHDdTrC}). De plus c'est un corps par la proposition~\ref{PROPooSHHWooCyZPPw}. C'est donc une extension algébrique finie de \( \eK\) par le théorème~\ref{ThonoZyKa}. Mais \( \eK\) étant algébriquement clos, il est sa propre et unique extension algébrique; nous en déduisons que
	\begin{equation}
		\frac{ \eK[X_1,\ldots, X_n] }{ I }=\eK.
	\end{equation}
	Donc pour tout \( 1\leq i\leq n\), il existe \( a_i\in \eK\) tel que \( X_i-a_i\in I\), sinon le monôme \( X_i\) ne se projetterait pas sur un élément dans \( \eK\) dans le quotient. Cela prouve que \( J\) est contenu dans \( I\); par maximalité nous avons donc \( I=J\).
\end{proof}

\begin{corollary}
	Soit \( \eK\) un corps algébriquement clos et \( I\), un idéal de \( \eK[X_1,\ldots, X_n]\). Si nous notons
	\begin{equation}
		V(I)=\{ x\in \eK^n\tq P(x_1,\ldots, x_n)=0 \}
	\end{equation}
	l'ensemble des racines communes à tous les éléments de \( I\), on a \( V(I)=\emptyset\) si et seulement si \( I=\eK[X_1,\ldots, X_n]\).
\end{corollary}

\begin{proof}
	Si \( I=\eK[X_1,\ldots, X_n]\) en particulier \( 1\in I\) et nous avons évidemment \( V(I)=\emptyset\). Le sens difficile est l'autre sens.

	Supposons que \( I\neq \eK[X_1,\ldots, X_n]\) et que \( K\) est un idéal maximum contenu dans \( I\). Nous savons déjà par le théorème~\ref{ThowgZYqx} que \( K\) est de la forme \( K=(X_1-a_1,\ldots, X_n-a_n)\). Un élément de \( I\) est dans \( K\), donc si \( P\in I\) nous avons
	\begin{equation}
		P(a_1,\ldots, a_n)=0,
	\end{equation}
	c'est-à-dire que \( (a_1,\ldots, a_n)\in V(I)\) et donc que \( V(I)\neq \emptyset \).
\end{proof}

%+++++++++++++++++++++++++++++++++++++++++++++++++++++++++++++++++++++++++++++++++++++++++++++++++++++++++++++++++++++++++++
\section{Polynômes symétriques et alternés}
%+++++++++++++++++++++++++++++++++++++++++++++++++++++++++++++++++++++++++++++++++++++++++++++++++++++++++++++++++++++++++++

%---------------------------------------------------------------------------------------------------------------------------
\subsection{Polynômes symétriques, alternés ou semi-symétriques}
%---------------------------------------------------------------------------------------------------------------------------


Nous rappelons que le groupe symétrique \( S_n\) agit sur l'anneau des polynômes de \( n\) variables sur l'anneau \( A\) par le lemme \ref{LEMooIRVQooHvoNBq}.

\begin{definition}
	Un polynôme \( Q\) en \( n\) indéterminées est
	\begin{enumerate}
		\item
		      \defe{symétrique}{polynôme!symétrique}\index{symétrique!polynôme} si \( Q=\sigma\cdot Q\) pour tout \( \sigma\in S_n\);
		\item
		      \defe{alterné}{polynôme!alterné}\index{alterné!polynôme} si \( \sigma\cdot Q=\epsilon(\sigma)Q\) pour tout \( \sigma\in S_n\);
		\item
		      \defe{semi-symétrique}{semi-symétrique!polynôme}\index{polynôme!semi-symétrique} si \( \sigma\cdot Q=Q\) pour tout \( \sigma\in A_n\)
	\end{enumerate}
\end{definition}
Le polynôme \( T_1+T_2\) est symétrique; le polynôme \( T_1+T_2^2\) ne l'est pas.

%---------------------------------------------------------------------------------------------------------------------------
\subsection{Polynôme symétrique élémentaire}
%---------------------------------------------------------------------------------------------------------------------------

\begin{definition}  \label{DEFooTREUooZKoXeg}
	Le \( k\)-ième \defe{polynôme symétrique élémentaire}{élémentaire!polynôme symétrique}\index{polynôme!symétrique!élémentaire} à \( n\) inconnues est le polynôme
	\begin{equation}
		\sigma_k(T_1,\ldots, T_n)=\sum_{s\in F_k}\prod_{i=1}^kT_{s(i)}
	\end{equation}
	où \( F_k\) est l'ensemble des fonctions strictement croissantes \( \{ 1,2,\ldots, k \}\to\{ 1,2,\ldots, n \}\).
\end{definition}

Une autre façon de décrire ces polynômes élémentaires est
\begin{equation}
	\sigma_k=\sum_{1\leq i_1<\ldots<i_k\leq n}X_{i_1}\ldots X_{i_k}.
\end{equation}
Par exemple
\begin{subequations}
	\begin{align}
		\sigma_1(T_1,\ldots, T_n) & =T_1+T_2+\cdots +T_n                                            \\
		\sigma_2(T_1,\ldots, T_n) & =T_1T_2+\cdots +T_1T_n+T_2T_3+\cdots +T_2T_n+\cdots +T_{n-1}T_n \\
		\sigma_n(T_1,\ldots, T_n) & =T_1\ldots T_n.
	\end{align}
\end{subequations}
En particulier, \( \sigma_2(x,y,z)=xy+yz+xz\).

\begin{theorem}[\cite{PoloPolSym}]  \label{TholReBiw}
	Si \( Q\) est un polynôme symétrique en \( T_1,\ldots, T_n\), alors il existe un et un seul polynôme \( P\) en \( n\) indéterminées tel que
	\begin{equation}
		Q(T_1,\ldots, T_n)=P\big( \sigma_1(T_1,\ldots, T_n),\ldots, \sigma_n(T_1,\ldots, T_n) \big).
	\end{equation}
\end{theorem}
%TODO : la preuve de ce théorème

\begin{example}
	Nous voulons décomposer \( P(x,y,z)=x^3+y^3+z^3\) en polynômes symétriques élémentaires, c'est-à-dire en
	\begin{subequations}
		\begin{numcases}{}
			\sigma_1=x+y+z\\
			\sigma_2=xy+xz+yz\\
			\sigma_3=xyz.
		\end{numcases}
	\end{subequations}
	Étant donné que \( P\) est de degré \( 3\), les seules combinaisons des \( \sigma_i\) qui peuvent intervenir sont \( \sigma_1^3\), \( \sigma_1\sigma_2\) et \( \sigma_3\). Étant donné que dans \( P\) le coefficient de \( x^3\) est un, il est obligatoire d'avoir un coefficient \( 1\) devant \( \sigma_1^3\). Nous le calculons :
	\begin{verbatim}
----------------------------------------------------------------------
| Sage Version 4.8, Release Date: 2012-01-20                         |
| Type notebook() for the GUI, and license() for information.        |
----------------------------------------------------------------------
sage: var('x,y,z')
(x, y, z)
sage: P=x**3+y**3+z**3
sage: S1=x+y+z
sage: S2=x*y+x*z+y*z
sage: S3=x*y*z
sage: (S1**3).expand()
x^3 + 3*x^2*y + 3*x^2*z + 3*x*y^2 + 6*x*y*z + 3*x*z^2 + y^3 
                + 3*y^2*z + 3*y*z^2 + z^3
sage: (S1**3-P).expand()
3*x^2*y + 3*x^2*z + 3*x*y^2 + 6*x*y*z + 3*x*z^2 + 3*y^2*z + 3*y*z^2
x^3 + 3*x^2*y + 3*x^2*z + 3*x*y^2 + 6*x*y*z + 3*x*z^2 
            + y^3 + 3*y^2*z + 3*y*z^2 + z^3
    \end{verbatim}
	Dans la différence \( \sigma_1^3-P\) nous voyons que le terme en \( xyz\) est \( 6xyz\); par conséquent nous savons que le coefficient de \( \sigma_3\) sera \( -6\). Il nous reste :
	\begin{verbatim}
sage: (S1**3+6*S3-P).expand()
3*x^2*y + 3*x^2*z + 3*x*y^2 + 12*x*y*z + 3*x*z^2 + 3*y^2*z + 3*y*z^2
    \end{verbatim}
	que nous identifions facilement avec \( 3\sigma_1\sigma_2\). Nous avons donc
	\begin{equation}
		P=\sigma_1^3-3\sigma_1\sigma_2+3\sigma_3.
	\end{equation}
\end{example}


\begin{lemma}[\cite{fJhCTE}]    \label{LemSoXCQH}
	Soit \( \eK\) une extension de degré \( \delta\) de \( \eQ\) et \( P\in \eK[T_1,\ldots, T_m]\). Alors il existe \( \bar P\in \eQ[T_1,\ldots, T_m]\) tel que
	\begin{enumerate}
		\item
		      \( \deg\bar P=\delta\deg(P)\)
		\item
		      pour tout \( (z_1,\ldots, z_m)\in \eC^m\) tel que \( P(z_1,\ldots, z_m)=0\), on a \( \bar P(z_1,\ldots, z_m)=0\).
	\end{enumerate}
\end{lemma}
\index{polynôme!symétrique}
\index{polynôme!racines}
\index{extension!de corps}
\index{corps!extension}

\begin{proof}
	En vertu de la proposition~\ref{PropUmxJVw} et du corolaire~\ref{CORooNZZMooIoBYXY}, \( \eK\) est une extension séparable de \( \eQ\), et donc vérifie le théorème de l'élément primitif (\ref{ThoORxgBC}). Il existe \( \theta\in \eK\) tel que \( \eK=\eQ(\theta)\). Soit \( P_{\theta}\in\eQ[X]\) le polynôme minimal de \( \theta\). L'extension \( \eK\) étant de degré \( \delta\), et \( \theta\) étant un générateur, une base de \( \eK\) comme espace vectoriel sur \( \eQ\) est
	\begin{equation}
		\{ 1,\theta,\ldots, \theta^{\delta-1} \}.
	\end{equation}
	Mais par ailleurs la proposition~\ref{PropURZooVtwNXE}\ref{ItemJCMooDgEHajiv} nous indique qu'une base de \( \eQ(\theta)\) sur \( \eQ\) est donnée par
	\begin{equation}
		\{ 1,\theta,\ldots, \theta^{n-1} \}
	\end{equation}
	où \( n\) est le degré de \( P_{\theta}\). Donc \( P_{\theta}\) est de degré \( \delta\). Nous nommons \( \theta_1,\ldots, \theta_{\delta}\) les racines de \( P_{\theta}\) dans un corps de décomposition. Ici nous notons \( \theta=\theta_1\) et nous ne prétendons pas que \( \theta_k\in \eK\). Notons que ces \( \theta_i\) sont toutes des racines simples de \( P_{\theta}\), sinon nous aurions un facteur irréductible \( (X-\theta_k)^2\), et \( P_{\theta}\) ne serait pas irréductible sur \( \eQ\).

	Soit \( \sigma_k\) le morphisme canonique
	\begin{equation}
		\begin{aligned}
			\sigma_k\colon \eQ(\theta) & \to \eQ(\theta_k)           \\
			\sum_i q_i\theta^i         & \mapsto \sum_iq_i\theta_k^i
		\end{aligned}
	\end{equation}
	Nous avons \( \sigma_1\colon \eK\to \eK\) qui est l'identité.

	Notons \( N\) le degré du polynôme \( P\in \eK[T_1,\ldots, T_m]\) dont il est question dans l'énoncé. Nous le décomposons alors en
	\begin{equation}
		P=\sum_{l=0}^N\sum_{i=1}^mc_{il}T_i^l
	\end{equation}
	avec \( c_{il}\in \eK\). Nous voyons \( c_{i,.}\) comme un élément de \( \eK^m\) et donc nous écrivons\footnote{Il me semble qu'il manque la somme sur \( i\) dans \cite{fJhCTE}.}
	\begin{equation}
		P=\sum_{l=0}^N\sum_{i=1}^m c_l(\theta)_iT_i^l
	\end{equation}
	où \( c_l\in \eQ[X]^m\). Nous pouvons choisir \( \deg(c_l)<\delta\) parce que les puissances plus grandes de \( \theta\) ne génèrent rien de nouveau.

	Nous posons aussi
	\begin{equation}
		P^{\sigma_k}=\sum_{l,i} c_l(\theta_k)_iT_i^l\in \eQ(\theta_k)[T_1,\ldots, T_m],
	\end{equation}
	et \( \bar P=PP^{\sigma_2}\ldots P^{\sigma_k}\). Le coefficient de \( T_i^l\) dans \( \bar P\) est
	\begin{equation}
		\bar c_l(\theta_1,\ldots, \theta_{\delta})_i=\sum_{l_1+\cdots +l_{\delta}=l}c_{l_1}(\theta_1)_i\ldots c_{l_{\delta}}(\theta_{\delta})_i.
	\end{equation}
	Ce dernier est un polynôme en les \( \theta_k\) à coefficients dans \( \eQ\). Qui plus est, c'est un polynôme symétrique. En effet un terme contenant \( \theta_k^a\theta_l^b\) provenant de \( c_{l_i}(\theta_k)c_{l_j}(\theta_l)\) a un terme correspondant \( \theta_k^b\theta_l^a\) provenant de \( c_{l_j}(\theta_k)c_{l_i}(\theta_l)\).

	C'est donc le moment d'utiliser le théorème~\ref{TholReBiw} à propos des polynômes symétriques élémentaires qui nous dit que les coefficients de \( \bar P\) sont en réalité des polynômes en ceux de \( P_{\theta}\) qui sont dans \( \eQ\). Donc \( \bar P\in \eQ[T_1,\ldots, T_m]\). Par ailleurs nous avons que
	\begin{equation}
		\deg(\bar P)=\delta \deg(P)
	\end{equation}
	parce que \( \bar P\) est le produit de \( \delta\) «copies»  de \( P\). De plus \( P=P^{\sigma_1}\) divise \( \bar P \) donc on a bien que si \( P(z)=0\) alors \( \bar P(z)=0\). Le polynôme \( \bar P\) est celui que nous cherchions.
\end{proof}

%---------------------------------------------------------------------------------------------------------------------------
\subsection{Relations coefficients racines}
%---------------------------------------------------------------------------------------------------------------------------

\begin{theorem}[Relations coeffitients-racines] \label{ThoOQRgjpl}
	Soit le polynôme \( P=a_nX^n+\cdots +a_1X+a_0\) et \( r_i\) ses \( n\) racines. Alors nous avons pour chaque \( 1\leq k\leq n\) la relation
	\begin{equation}
		\sigma_k(r_1,\ldots, r_n)=(-1)^k\frac{ a_{n-k} }{ a_n }
	\end{equation}
	où \( \sigma_k\) est le \( k\)\ieme polynôme symétrique défini en~\ref{DEFooTREUooZKoXeg}.
\end{theorem}
\index{relations!coefficient-racines}
\index{polynôme!symétrique!élémentaire}

%TODO : citer Wikipédia pour l'exemple suivant.
%TODO : ici aussi il faudra faire référence au théorème sur le fait qu'un polynôme ait toutes ses racines dans \eC.

\begin{example} \label{ExHIfHhBr}
	Soit le polynôme
	\begin{equation}
		P(x)=x^3+2x^2+3x+4
	\end{equation}
	et ses racines que nous nommons \( a,b,c\). Nous voudrions calculer \( a^2+b^2+c^2\). D'abord nous décomposons \( Q(a,b,c)=a^2+b^2+c^2\) en polynômes symétriques élémentaires : \( Q(a,b,c)=\sigma_1(a,b,c)^2-2\sigma_2(a,b,c)\).

	Mais les relations coefficients-racines\footnote{Théorème \ref{ThoOQRgjpl}} nous donnent \( \sigma_1(a,b,c)=-2\) et \( \sigma_2(a,b,c)=3\), donc
	\begin{equation}
		a^2+b^2+c^2=(-2)^2-2\cdot 3=-2.
	\end{equation}

	Cela nous assure déjà qu'au moins une des solutions n'est pas réelle.

	Nous pouvons en avoir une vérification directe en calculant explicitement les racines (ce qui est possible pour le degré \( 3\)) :
	\lstinputlisting{tex/frido/VAYVmNRpolynomeSym.py}

	Notez qu'il faut un peu chipoter pour isoler les solutions depuis la réponse de la fonction \info{solve}.
\end{example}

En suivant le même cheminement que dans l'exemple, si \( P\) est un polynôme de degré \( n\) et si \( r_i\) sont ses racines, il est facile de calculer \( Q(r_1,\ldots, r_n)\) pour n'importe quel polynôme symétrique \( Q\)

\begin{proposition}[Annulation de fonctions polynomiales\cite{WARooZoFOBn}] \label{PropTETooGuBYQf}
	Soit \( \eK\) un corps et \( P\) un polynôme à \( n\) indéterminées. Nous supposons que \(P\) s'annule sur un ensemble de la forme \( A_1\times\cdots\times A_n\) avec \( \Card(A_j)>\deg_{X_j}(P)\) pour tout \( j\). Alors \( P=0\).

	De plus si \( P=0\) alors tous ses coefficients sont nuls\footnote{L'intérêt de cela est qu'un polynôme de \( \eZ[X_1,\ldots, X_n]\) peut s'évaluer sur un élément de n'importe quel corps; il restera le polynôme nul.}.
\end{proposition}

\begin{proof}
	Nous prouvons le résultat par récurrence sur le nombre \( n\) d'indéterminées. Si \( n=1\), cela est le théorème~\ref{ThoLXTooNaUAKR}. Nous classons les monômes du polynôme \( P\) par ordre de puissance de \( X_n\) et nous le factorisons :
	\begin{equation}
		P=\sum_{i=1}^mP_iX_n^i
	\end{equation}
	avec \( P_i\in \eK[X_1,\ldots, X_{n-1}]\). Soit \( (a_1,\ldots, a_{n-1})\in A_1\times \ldots \times A_{n-1}\) et posons
	\begin{equation}
		Q(T)=P(a_1,\ldots, a_{n-1},T)= \sum_{i=1}^mP_i(a_1,\ldots, a_{n-1})T^i.
	\end{equation}
	Le polynôme \( Q\) s'annule sur \( A_n\) avec \( \deg(Q)=\deg_{X_n}(P)<\Card(A_n)\) et le théorème~\ref{ThoLXTooNaUAKR} nous donne \( Q=0\). Or les coefficients des différentes puissances de \( T\) dans \( Q(T) \) sont les \( P_i(a_1,\ldots, a_{n-1})\); ils sont donc nuls.

	Nous avons montré que le polynôme \( P_i\) s'annule pour tout élément de \( A_1\times \ldots \times A_{n-1}\), mais nous avons
	\begin{equation}
		\deg_{X_j}(P_i)\leq \deg_{X_j}P<\Card(A_j),
	\end{equation}
	donc l'hypothèse de récurrence donne \( P_i=0\). Par suite, \( P=0\) également.
\end{proof}

%+++++++++++++++++++++++++++++++++++++++++++++++++++++++++++++++++++++++++++++++++++++++++++++++++++++++++++++++++++++++++++
\section{Minuscule morceau sur la théorie de Galois}
%+++++++++++++++++++++++++++++++++++++++++++++++++++++++++++++++++++++++++++++++++++++++++++++++++++++++++++++++++++++++++++

Vous trouverez des détails et des preuves à propos de la théorie de Galois dans \cite{GalIEl,rqrNyg}.

\begin{definition}
	Soit \( \eK\), un corps.

	Le \defe{groupe de Galois}{groupe!de Galois} d'une extension \( \eL\) de \( \eK\) est le groupe des automorphismes de \( \eL\) laissant \( \eK\) invariant.

	Le groupe de Galois d'un polynôme sur \( \eK\) est le groupe de Galois de son corps de décomposition sur \( \eK\).
\end{definition}

\begin{definition}
	Des éléments \( b_1,\ldots, b_n\) d'une extension de \( \eK\) sont \defe{algébriquement indépendants}{algébriquement!indépendant}\index{indépendance!algébrique} si ils ne satisfont à aucune relation du type
	\begin{equation}
		\sum \alpha_{i_1\ldots i_n}b_1^{i_1}\ldots b_n^{i_n}=0
	\end{equation}
	avec \( \alpha_{i_1\ldots i_n}\in \eK\).
\end{definition}

Nous disons que l'équation
\begin{equation}
	x^n+a_{n-1}x^{n-1}+\cdots+a_1x+a_0=0
\end{equation}
est l'\defe{équation générale}{equation@équation!générale de degré \( n\)} de degré \( n\) si les coefficients \( a_i\) sont algébriquement indépendants sur \( \eK\).

\begin{theorem}
	Le groupe de Galois d'un polynôme de degré \( n\) est isomorphe au groupe symétrique \( S_n\).
\end{theorem}

\begin{corollary}[\cite{FWZHooBLvuCJ}]
	L'équation générale de degré \( n\) est résoluble par radicaux si et seulement si \( n\le 5\).
\end{corollary}
