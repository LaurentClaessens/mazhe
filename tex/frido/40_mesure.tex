% This is part of Mes notes de mathématique
% Copyright (c) 2011-2016,2018-2019,2022-2023
%   Laurent Claessens, Carlotta Donadello
% See the file fdl-1.3.txt for copying conditions.

%+++++++++++++++++++++++++++++++++++++++++++++++++++++++++++++++++++++++++++++++++++++++++++++++++++++++++++++++++++++++++++
\section{Tribus}
%+++++++++++++++++++++++++++++++++++++++++++++++++++++++++++++++++++++++++++++++++++++++++++++++++++++++++++++++++++++++++++

Vous pouvez vous reporter au thème~\ref{INTooVDSCooHXLLKp} pour voir plus vite où sont les définitions associées.

%---------------------------------------------------------------------------------------------------------------------------
\subsection{Généralités}
%---------------------------------------------------------------------------------------------------------------------------

\begin{definition}[Tribu, espace mesurable\cite{ProbaDanielLi}]  \label{DefjRsGSy}
	Si \( \Omega\) est un ensemble, un ensemble \( \tribA\) de sous-ensembles de \( \Omega\) est une \defe{tribu}{tribu} si
	\begin{enumerate}
		\item
		      \( \Omega\in\tribA\);
		\item
		      \( A^c\in \tribA\) pour tout \( A\in\tribA\);
		\item       \label{ItemooPEQNooYiYNtN}
		      si \( (A_i)_{i\in \eN}\) est une suite dénombrable d'éléments de \( \tribA\), alors \( \bigcup_{i\in \eN}A_i\in\tribA\).
	\end{enumerate}
	Le couple \( (\Omega,\tribA)\) est alors un \defe{espace mesurable}{espace!mesurable}.
\end{definition}

\begin{remark}
	Nous trouvons parfois la notation
	\begin{equation}
		\bigcup_{k\in \eN}A_k=\sup_{k\geq 0}A_k.
	\end{equation}
	\index{supremum!d'une suite d'ensembles}
\end{remark}

\begin{lemma}   \label{LemBWNlKfA}
	Opérations ensemblistes sur les tribus.
	\begin{enumerate}
		\item       \label{ITEMooTDXNooFszBzi}
		      Une tribu est stable par intersections au plus dénombrables.
		\item       \label{ItemXQVLooFGBQNj}
		      Une tribu est stable par différence ensembliste.
	\end{enumerate}
\end{lemma}

\begin{proof}
	Soit \( (A_i)_{i\in I}\) une famille au plus dénombrable d'éléments de la tribu \( \tribA\). Nous devons prouver que \( \bigcap_{i\in I}A_i\) est également un élément de \( \tribA\). Pour cela nous passons au complémentaire :
	\begin{equation}
		\complement\left( \bigcap_{i\in I}A_i \right)=\bigcup_{i\in I}\complement A_i.
	\end{equation}
	La définition d'une tribu implique que le membre de droite est un élément de la tribu. Par stabilité d'une tribu par complémentaire, l'ensemble \( \bigcap_{i\in I}A_i\) est également un élément de la tribu.

	La seconde assertion est immédiate à partir de la première parce que \( A\setminus B=A\cap \complement B\).
\end{proof}

Si \( (\tribA_i)_{i\in I}\) est un ensemble de tribus (indexé par un ensemble \( I\) quelconque) alors
\begin{equation}
	\tribA=\bigcap_{i\in I}\tribA_i
\end{equation}
est également une tribu.

%--------------------------------------------------------------------------------------------------------------------------- 
\subsection{Tribu engendrée}
%---------------------------------------------------------------------------------------------------------------------------

\begin{definition}      \label{DEFooJSAKooSDnGzt}
	Soit \( \tribD\) un ensemble de parties de \( \Omega\). La \defe{tribu engendrée}{tribu engendrée} par \( \tribD\) est l'intersection de toutes les tribus de \( \Omega\) contenant \( \tribD\). C'est la plus petite tribu contenant \( \tribD\). Nous la noterons le plus souvent \( \sigma(\tribA)\)\nomenclature[Y]{\( \sigma(\tribA)\)}{tribu engendrée par \( \tribD\)}
\end{definition}


\begin{lemma}       \label{LEMooCAUMooEdpjRs}
    Si \( (\Omega,\tribA)\) est un espace mesurable et si \( A\) est une partie mesurable de \( \Omega\), alors la tribu engendrée\footnote{Tribu engendrée par une partie, définition \ref{DEFooJSAKooSDnGzt}.} par \( A\) est
    \begin{equation}
        \sigma(A)=\{ \emptyset,A,\complement A,\Omega \}.
    \end{equation}
\end{lemma}

\begin{lemmaDef}[Tribu engendrée\cite{TribuLi}]        \label{DefNOJWooLGKhmJ} 
	Soit une application \( f\colon S_1\to S_2\) et \( \tribF\) une tribu de \( S_1\). Alors
	\begin{enumerate}
		\item
		      L'ensemble
		      \begin{equation}
			      \tribF_f=\{  B\subset S_2\tq f^{-1}(B)\in \tribF  \}
		      \end{equation}
		      est une tribu sur \( S_2\).
		\item
		      C'est la plus grande tribu de \( S_2\) pour laquelle \( f\) est mesurable.
	\end{enumerate}
    Cette tribu est la est la \defe{tribu engendrée}{tribu engendrée par une application} par \( f\). Elle sera souvent notée \( \sigma(f)\)\nomenclature[P]{\( \sigma(f)\)}{La tribu engendrée par une variable aléatoire ou une application}.
\end{lemmaDef}

\begin{proof}
    Vérifier les trois propriétés de la définition \ref{DefjRsGSy}.
	\begin{enumerate}
		\item
		      \( S_2\in\tribF\) parce que \(f^{-1}(S_2)=S_1\in \tribF_f\).
		\item
            Si \( B\in \tribF_f\), alors nous prouvons que \( B^c\in \tribF_f\). Nous avons \( f^{-1}(B)\in\tribF\) et donc \( f^{-1}(B)^c\in\tribF\). Pour conclure, il suffit de remarquer que \( f^{-1}(B^c)=f^{-1}(B)^c\).
		\item
            Si \( (B_i)\in\tribF_f\), alors
            \begin{equation}
                f^{-1}\left( \bigcup_iB_i \right)=\bigcup_if^{-1}(B_i)\in \tribF_f.
            \end{equation}
            Donc \( \bigcup_iB_i\in\tribF_f\).
	\end{enumerate}
	En ce qui concerne la maximalité, si \( R\subset S_2\) n'est pas dans \( \tribF_f\) alors \( f^{-1}(R)\) n'est pas dans \( \tribF\) et donc \( f\) ne serait pas mesurable.
\end{proof}

\begin{proposition}[\cite{SFYoobgQUp}]  \label{PropHYLooLgOCy}
	Soit \( S\) un ensemble et \( \tribF\) une tribu de \( S\). Soit une classe \( \tribN\subset\partP(S)\) telle que
	\begin{enumerate}
		\item
		      Si \( A\in\tribN\) alors il existe \( Y\in\tribF\cap\tribN\) tel que \( A\subset Y\).
		\item
		      Si \( A\in\tribN\) et \( B\subset A\) alors \( B\in\tribN\).
		\item
		      La classe \( \tribN\) est stable par union dénombrable.
	\end{enumerate}
	Alors la classe
	\begin{equation}
		\tribT=\{ X\cup A\text{ avec } A\in\tribN\text{ et } X\in\tribF \}
	\end{equation}
	est une tribu.
\end{proposition}

\begin{proof}
	L'ensemble \( \tribT\) est stable par union dénombrable parce que \( \tribF\) et \( \tribN\) le sont. De plus \( S\) et \( \emptyset\) sont dans \( \tribF\) et donc dans \( \tribT\). Nous devons voir que \( \tribT\) est stable par complémentarité.

	Soit donc \( A\in\tribN\) et \( X\in\tribF\); nous savons que \( (A\cup X)^c=A^c\cap X^c\). De plus il existe \( Y\in\tribF\cap\tribN\) tel que \( A\subset Y\) et nous pouvons exprimer \( A^c\) en termes de \( Y\) : \( A^c=Y^c\cup(Y\setminus A)\). Donc
	\begin{equation}
		(A\cup X)^c=\big( Y^c\cup(Y\setminus A) \big)\cap X^c=\underbrace{(Y^c\cap X^c)}_{\in\tribF}\cup\underbrace{\big( (Y\setminus A)\cap X^c \big)}_{\in\tribN}.
	\end{equation}
	Le fait que la seconde partie soit dans \( \tribN\) est due au fait que ce soit une partie de \( Y\in\tribN\). Nous avons donc bien \( (A\cup X)^c\in\tribT\).
\end{proof}

%---------------------------------------------------------------------------------------------------------------------------
\subsection{Tribu induite et engendrée}
%---------------------------------------------------------------------------------------------------------------------------

\begin{propositionDef}[Tribu induite, tribu-trace\cite{BIBooIDTRooMHRyUm}]      \label{DefDHTTooWNoKDP}
	Soit un espace mesurable \( (S,\tribF)  \) et une partie \( R\subset S\). L'ensemble
	\begin{equation}        \label{EQooNOQNooYIdDOz}
		\tribF_R=\{ A\cap R\tq A\in\tribF \}
	\end{equation}
	est une tribu. Elle est la \defe{tribu induite}{tribu!induite}\index{induite!tribu} de \( R\) depuis \( S\). Elle est aussi nommée «tribu trace».
\end{propositionDef}

\begin{proof}
	D'abord \( R\) et \( \emptyset\) dont dans \( \tribF_R\). Si \( C\in\tribF_R \) alors \( C=A\cap R\) pour un certain \( A\in \tribF\) et nous devons prouver que \( R\cap C^c\) est dans \( \tribF_R\) (le complémentaire de \( C\) dans \( R\)). Nous avons
	\begin{equation}
		R\cap C^c=R\cap(A\cap R)^c=R\cap A^c\in \tribF_R
	\end{equation}
	parce que \( A^c\in \tribF\). Enfin si \( C_i\in \tribF_R\) alors \( C_i=R\cap A_i\) pour des \( A_i\) dans \( \tribF\). Nous avons
	\begin{equation}
		\bigcup_{i\in \eN}C_i=\bigcup_{i\in \eN}(R\cap A_i)=R\cap\big( \bigcup_{i\in \eN}A_i \big),
	\end{equation}
	mais \( \bigcup_iA_i \in \tribF\) donc \( \bigcup_iC_i\in \tribF_R\).
\end{proof}

\begin{proposition}     \label{PROPooUNNSooMUQKfp}
	Si \( R\) est mesurable dans \( (\Omega,\tribA)\) alors
	\begin{equation}
		\tribA_R=\{ A\cap R\tq A\in\tribA \}=\{ S\in\tribA\tq S\subset R \},
	\end{equation}
	où \( \tribA_R\) est la tribu induite de \( \tribA\) sur \( R\).
\end{proposition}

\begin{proof}
	La première égalité est simplement la définition \eqref{EQooNOQNooYIdDOz} de la tribu induite.
	Pour le reste, nous notons \( \tribF=\{ S\in \tribA\tq S\subset R \}\), et nous prouvons que \( \tribA_R=\tribF \).

	\begin{subproof}
		\spitem[\( \tribF\subset\tribA_R\)]
		Si \( S\in \tribF\), alors \( S\in \tribA\) et \( S\subset R\). Donc \( S=S\cap R\in\tribA_R\).

		\spitem[\( \tribF_R\subset \tribA_R\)]
		Dans l'autre sens, si \( S\in \tribA_R\), alors il existe \( A\in\tribA\) tel que \( S=A\cap R\). Donc \( S\subset A\) et \( S\in\tribA\) parce que \( R\) et \( A\) sont des éléments de \( \tribA\) (stable par intersection).
	\end{subproof}
\end{proof}


%+++++++++++++++++++++++++++++++++++++++++++++++++++++++++++++++++++++++++++++++++++++++++++++++++++++++++++++++++++++++++++
\section{Théorie de la mesure}
%+++++++++++++++++++++++++++++++++++++++++++++++++++++++++++++++++++++++++++++++++++++++++++++++++++++++++++++++++++++++++++
\label{SecSLOooeMaig}

\begin{definition}[\cite{MesureLebesgueLi}] \label{DefUMWoolmMaf}
	Une \defe{mesure extérieure}{mesure!extérieure} sur un ensemble \( S\) est une application \( m^*\colon \partP(S)\to \mathopen[ 0 , \infty \mathclose]\) telle que
	\begin{enumerate}
		\item
		      \( m^*(\emptyset)=0\),
		\item
		      Si \( A\subset B\) dans \( S\) alors \( m^*(A)\leq m^*(B)\)
		\item   \label{ItemARKooppZfDaiii}
		      Si les \( A_n\) sont des parties de \( S\) alors
		      \begin{equation}    \label{EqZLMooSxvaL}
			      m^*\big( \bigcup_{n\in \eN}A_n \big)\leq \sum_{n\in \eN}m^*(A_n).
		      \end{equation}
	\end{enumerate}
\end{definition}
La différence avec une mesure est que nous ne demandons pas que \eqref{EqZLMooSxvaL} soit une égalité lorsque les \( A_n\) sont disjoints.

%---------------------------------------------------------------------------------------------------------------------------
\subsection{Mesure sur un ensemble de parties}
%---------------------------------------------------------------------------------------------------------------------------

\begin{definition}[Mesure sur un ensemle de parties\cite{ooZFLHooQmmfyS}]       \label{DefWUPHooEklLmR}
	Soient \( S\) un ensemble et \( \tribC\) un ensemble de parties de \( S\) contenant \( \emptyset\). Une \defe{mesure positive}{mesure!sur un ensemble de parties} sur \( (S,\tribC)\) est une application \( \mu\colon \tribC\to \mathopen[ 0 , \infty \mathclose]\) telle que
	\begin{enumerate}
		\item
		      \( \mu(\emptyset)=0\),
		\item
		      Si \( A_n\in\tribC\) sont des éléments deux à deux disjoints dans \( \tribC\) et tels que \( \bigcup_nA_n\in\tribC\) alors
		      \begin{equation}
			      \mu\big( \bigcup_{n=0}^{\infty}A_n \big)=\sum_{n=0}^{\infty}\mu(A_n).
		      \end{equation}
	\end{enumerate}

	La mesure est \defe{finie}{mesure finie} si \( \mu(S)<\infty\) et \defe{\( \sigma\)-finie}{mesure \( \sigma\)-finie} si il existe une suite \( (S_n)\) dans \( \tribC\) telle que \( S=\bigcup_nS_n\) et \( \mu(S_n)<\infty\).
\end{definition}

\begin{remark}
	La condition \( \mu(\emptyset)=0\) est nécessaire. Certes, si \( A\in \tribA\) nous avons
	\begin{equation}        \label{EQooCDSQooZMjgOY}
		\mu(A)=\mu(A\cup \emptyset)=\mu(A)+\mu(\emptyset)
	\end{equation}
	parce que \( A\) et \( \emptyset\) sont disjoints. Cela semble indiquer que \( \mu(\emptyset)=0\), mais pas tout à fait : il est encore possible d'avoir \( \mu(B)=\infty\) pour tout \( B\in\tribA\), y compris \( \mu(\emptyset)=\infty\). À cause de cette exception, la relation \eqref{EQooCDSQooZMjgOY} n'implique pas \( \mu(\emptyset)=0\).
\end{remark}

Sans hypothèse sur l'ensemble de parties considéré, nous ne pouvons pas dire grand chose de plus.

%---------------------------------------------------------------------------------------------------------------------------
\subsection{Mesure sur une algèbre de parties}
%---------------------------------------------------------------------------------------------------------------------------

\begin{definition}[Algèbre de parties\cite{MesureLebesgueLi}]   \label{DefTCUoogGDud}
	Soit \( S\), un ensemble. Une classe \( \tribD\) de parties de \( S\) est une \defe{algèbre de parties}{algèbre!de parties} de \( S\) si
	\begin{enumerate}
		\item
		      \( S\in\tribD\) et \( \emptyset\in\tribD\),
		\item
		      si \( A\in\tribD\) alors \( A^c\in\tribD\),
		\item
		      si \( A,B\in\tribD\) alors \( A\cup B\in\tribD\).
	\end{enumerate}
\end{definition}

\begin{normaltext}
    En anglais, ce sont des \emph{field of sets}\cite{BIBooNBGPooSPkEKX}.
\end{normaltext}

Les algèbres de parties ne sont pas des classes si sauvages que ça; en témoigne le lemme suivant.
\begin{lemma}   \label{LemBFKootqXKl}
	Une algèbre de partie est stable par intersection (finie) et par différence ensembliste.
\end{lemma}

\begin{proof}
	Il suffit de remarquer que \( A\cap B=\big( A^c\cup B^c \big)^c\) et que \( A\setminus B=A\cap B^c\).
\end{proof}

\begin{lemma}[\cite{MesureLebesgueLi}]  \label{LemZQUooMdCpq}
	Si \( \tribD\) est une algèbre de parties de \( S\) et si \( \mu\) est une mesure sur \( (S,\tribD)\) alors
	\begin{enumerate}
		\item
		      si \( A,B\in\tribD\) avec \( A\subset B\) alors \( \mu(A)\leq \mu(B)\)
		\item   \label{ItemMFUooWCPNnii}
		      si \( A_n\in\tribD\) et \( \bigcup_nA_n\in\tribD\) alors
		      \begin{equation}
			      \mu\big( \bigcup_nA_n \big)\leq\sum_n\mu(A_n).
		      \end{equation}
	\end{enumerate}
\end{lemma}
La propriété~\ref{ItemMFUooWCPNnii} est la \( \sigma\)-\defe{sous-additivité}{sous-additivité!sur algèbre de parties}.

\begin{proof}
	Si \( A\subset B\) alors \( B=A\cup(B\setminus A)\) avec \( A\) et \( B\setminus A\) disjoints donc
	\begin{equation}
		\mu(B)=\mu(A)+\mu(B\setminus A)\geq \mu(A).
	\end{equation}

	Pour la seconde, on passe par les compléments deux à deux : nous posons
	\begin{subequations}
		\begin{numcases}{}
			B_0=\emptyset\\
			B_{n}=A_n\setminus \bigcup_{k<n}B_k.
		\end{numcases}
	\end{subequations}
	Ces ensembles sont deux à deux disjoints et \( \bigcup_nB_n=\bigcup_nA_n\in\tribD\), donc
	\begin{equation}
		\mu\big( \bigcup_nA_n \big)=\sum_n\mu\big( A_n\setminus\bigcup_{k<n}B_k \big)\leq \sum_n\mu(A_n),
	\end{equation}
	où nous avons utilisé la première partie du lemme.
\end{proof}

\begin{proposition}[Mesure extérieure à partir d'une algèbre de parties\cite{MesureLebesgueLi}]    \label{PropIUOoobjfIB}
	Soient \( \tribD\) une algèbre de partie sur l'ensemble \( S\) et \( \mu\) une mesure sur \( (S,\tribD)\). Alors l'application
	\begin{equation}    \label{EqRNJooQrcoa}
		\begin{aligned}
			\mu^*\colon \partP(S) & \to \mathopen[ 0 , +\infty \mathclose]                                           \\
			X                     & \mapsto \inf\left\{ \sum_n\mu(A_n)\tq A_n\in\tribD,X\subset\bigcup_nA_n \right\}
		\end{aligned}
	\end{equation}
	est une mesure extérieure\footnote{Définition~\ref{DefUMWoolmMaf}.} sur \( S\) et pour tout \( A\in\tribD\) nous avons \( \mu^*(A)=\mu(A)\).
\end{proposition}

\begin{proof}
	En plusieurs parties.
	\begin{subproof}
		\spitem[La définition est bonne]
		L'ensemble sur lequel l'infimum est pris n'est pas vide : il suffit de prendre \( A_1=S\) et \( A_{n\geq 2}=\emptyset\).
		\spitem[Le vide]
		D'abord \( \mu^*(\emptyset)=0\) parce que \( \emptyset\in\tribD\). Prendre ensuite \( A_n=\emptyset\).
		\spitem[\( \mu^*\) est croissante]
		Soit \( X\subset Y\) dans \( \partP(S)\). Si une suite \( (A_n)\) dans \( \tribD\) vérifie \( Y\subset \bigcup_nA_n\), alors la même suite vérifie \( X\subset\bigcup_nA_n\). Par conséquent nous avons l'inclusion
		\begin{equation}
			\left\{ \sum_n\mu(A_n)\tq A_n\in\tribD,X\subset\bigcup_nA_n \right\} \subset\left\{ \sum_n\mu(A_n)\tq A_n\in\tribD,Y\subset\bigcup_nA_n \right\} ,
		\end{equation}
		et donc l'inégalité \( \mu^*(X)\leq \mu^*(Y)\).
		\spitem[Inégalité par union dénombrable]
		Soit \( (X_n)_{n\in \eN}\) une suite de parties de \( S\). Si il existe \( n_0\) tel que \( \mu^*(X_{n_0})=\infty\) alors nous avons automatiquement \( \sum_n\mu^*(X_n)=\infty\) et l'inégalité demandée est évidente parce que n'importe quel nombre est plus petit ou égal à \( \infty\). Nous supposons donc maintenant que \( \mu^*(X_n)<\infty\) pour tout \( n\).

		Soit \( \epsilon>0\). Pour tout \( n\geq 1\) il existe une suite \( (B_k^{(n)})_{k\in \eN}\) dans \( \tribD\) telle que \( X_n\subset\bigcup_kB_k^{(n)}\) et
		\begin{equation}
			\mu^*(X_n)+\frac{ \epsilon }{ 2^n }\geq \sum_k\mu(B_k^{(n)}).
		\end{equation}
		Étant donné que
		\begin{equation}
			\bigcup_nX_n\subset\bigcup_n\big( \bigcup_kB_k^{(n)} \big),
		\end{equation}
		nous avons\footnote{Nous utilisons la somme \( \sum_{n\in \eN}(1/2^n)=1\), proposition \ref{PROPooWOWQooWbzukS}\ref{ITEMooBJHBooBMEmiG}.}
		\begin{equation}
			\mu^*\big( \bigcup_nX_n \big)\leq \sum_n\sum_k\mu^*(B_k^{(n)})\leq \sum_n\left( \mu^*(X_n)+\frac{ \epsilon }{ 2^n } \right)=\sum_n\mu^*(X_n)+\epsilon.
		\end{equation}
		Cette inégalité étant valable pour tout \( \epsilon\), nous avons bien
		\begin{equation}
			\mu^*\big( \bigcup_nX_n \big)=\sum_n\mu^*(X_n).
		\end{equation}

		\spitem[Restriction]
		Soit \( A\in\tribD\). Nous avons automatiquement \( \mu^*(A)\leq \mu(A)\) parce que \( \mu(A)\) est dans l'ensemble dont nous prenons l'infimum (prendre \( A_1=A\) et \( A_{n\geq 2}=\emptyset\)).

		En ce qui concerne l'inégalité inverse nous considérons une suite \( A_n\) dans \( \tribD\) telle que \( A\subset\bigcup_nA_n\). Étant donné que \( A\in\tribD\) et que \( \tribD\) est une algèbre de parties nous avons \( A\cap A_n\in \tribD\) et \( \bigcup_n(A\cap A_n)=A\in\tribD\). Par conséquent
		\begin{equation}
			\mu(A)=\mu\big( \bigcup_n(A\cap A_n) \big)\leq \sum_n\mu(A\cap A_n)\leq \sum_n\mu(A_n).
		\end{equation}
		Donc tous les éléments de l'ensemble sur lequel nous prenons l'infimum sont plus grands que \( \mu(A)\). Nous en déduisons que \( \mu^*(A)\geq \mu(A)\).
	\end{subproof}
\end{proof}

%---------------------------------------------------------------------------------------------------------------------------
\subsection{Mesure sur une tribu, espace mesuré}
%---------------------------------------------------------------------------------------------------------------------------

La définition suivante est une simple copie de la définition générale \ref{DefWUPHooEklLmR} d'une mesure sur un ensemble de parties. La seule différence est que l'union d'éléments d'une tribu est encore dans la tribu, et la condition \ref{ItemQFjtOjXiii} ne demande pas de le préciser.

\begin{definition}[Espace mesuré\cite{ooZFLHooQmmfyS}]  \label{DefBTsgznn}
	Une \defe{mesure positive}{mesure!positive} sur l'espace mesurable\footnote{Les définitions de tribus et d'espaces mesurables sont en \ref{DefjRsGSy}.} \( (\Omega,\tribA)\) est une application \( \mu\colon \tribA\to \eR\) telle que
	\begin{enumerate}
        \item
            \( \mu(\tribA)\subset \mathopen[ 0 , \infty \mathclose]\)
		\item
		      \( \mu(\emptyset)=0\),
		\item       \label{ItemQFjtOjXiii}
		      \( \mu\left( \bigcup_{i=0}^{\infty}A_i\right)=\sum_{i=0}^{\infty}\mu(A_i)\) si les \( A_i\) sont des éléments de \( \tribA\) deux à deux disjoints.
	\end{enumerate}
	Le triplet \( (\Omega,\tribA,\mu)\) est alors un \defe{espace mesuré}{espace!mesuré}.

	Une mesure est \defe{\( \sigma\)-finie}{mesure!\( \sigma\)-finie} si il existe un recouvrement dénombrable de \( \Omega\) par des ensembles de mesure finie. Si la mesure est \( \sigma\)-finie, nous disons que l'espace \( (\Omega,\tribA,\mu)\) est un espace mesuré \( \sigma\)-fini.

	La mesure \( \mu\) sur \( \Omega\) est \defe{finie}{mesure!finie} si \( \mu(\Omega)<\infty\).
\end{definition}

Si \( (\Omega,\tribA,\mu)\) et \( (S,\tribF,\nu)\) sont deux espaces mesurés, alors nous notons
\begin{equation}
	(\Omega,\tribA,\mu)\subset (S,\tribF,\nu)
\end{equation}
lorsque \( \Omega\subset S\), \( \tribA\subset\tribF\) et pour tout \( A\in\tribA\), \( \mu(A)=\nu(A)\).

\begin{definition}[Ensemble mesurable]\label{DefHGsQxHB}
	Les éléments de \( \tribA\) sont les ensembles \defe{mesurables}{mesurable!ensemble} pour la mesure \( \mu\).
\end{definition}

Si la mesure est \( \sigma\)-finie, nous pouvons choisir le recouvrement croissant pour l'inclusion. En effet si \( (E_n)_{n\in \eN}\) est le recouvrement, il suffit de considérer \( F_n=\bigcup_{k\leq n}E_k\). Ces ensembles \( F_n\) forment tout autant un recouvrement dénombrable, mais ils vont évidemment croissants.

Le lemme suivant complète la propriété~\ref{DefBTsgznn}\ref{ItemQFjtOjXiii} lorsque les ensembles ne sont pas disjoints.
\begin{lemma}[Unions et différences ensemblistes dans un espace mesurable] \label{LemPMprYuC}
	Soit un espace mesuré \( (\Omega,\tribF,\mu)\).
	\begin{enumerate}
		\item       \label{ITEMooSUIRooNDVOoB}
		      Soient \( A,B\in\tribF\) avec \( A\subset B\). Alors
		      \begin{equation}
			      \mu(B\setminus A)=\mu(B)-\mu(A)
		      \end{equation}
		\item       \label{ITEMooLEGKooWnYmlf}
		      Si \( A,B\in \tribF\) avec \( A\subset B\), alors
		      \begin{equation}
			      \mu(A)\leq \mu(B).
		      \end{equation}
		\item       \label{ITEMooMCNBooRGVGqA}
		      Si \( A,B\in \tribF\) avec \( A\subset B\), et si \( \mu(B)<\infty\), alors \( \mu(A)<\infty\).
		\item       \label{ITEMooABPYooFQEzqE}
		      Si \( (M_n)\) est une suite d'éléments de \( \tribF\) pas spécialement disjoints, alors
		      \begin{equation}\label{EqWWFooYPCTt}
			      \mu\big( \bigcup_kM_k \big)\leq \sum_{k}\mu(M_k).
		      \end{equation}
	\end{enumerate}
\end{lemma}

\begin{proof}
	Point par point.
	\begin{subproof}
		\spitem[Pour \ref{ITEMooSUIRooNDVOoB}]
		Nous décomposons \( B=A\cup(B\setminus A)\) en remarquant que l'union est disjointe et que \( B\setminus A\in \tribF\) par le lemme \ref{LemBWNlKfA}. La propriété \ref{ItemQFjtOjXiii} de la définition de mesure nous donne alors
		\begin{equation}
			\mu(B)=\mu\big( (B\setminus A)\cup A \big)=\mu(B\setminus A)+\mu(A)
		\end{equation}
		et donc
		\begin{equation}        \label{EQooTWCWooGGORjZ}
			\mu(B)=\mu(B\setminus A)+\mu(A)
		\end{equation}
		comme demandé.
		\spitem[Pour \ref{ITEMooLEGKooWnYmlf}]
		Il s'agit de reprendre \eqref{EQooTWCWooGGORjZ} :
		\begin{equation}    \label{EQooXGVQooRucvoF}
			\mu(A)=\mu(B)-\mu(B\setminus A)\leq \mu(B).
		\end{equation}
		\spitem[Pour \ref{ITEMooMCNBooRGVGqA}]
		Il s'agit de continuer \eqref{EQooXGVQooRucvoF} :
		\begin{equation}
			\mu(A)\leq \mu(B)<\infty.
		\end{equation}
		\spitem[Pour \ref{ITEMooABPYooFQEzqE}]
		Nous considérons la suite disjointe
		\begin{subequations}
			\begin{numcases}{}
				M'_0=\emptyset\\
				M'_k=M_k\setminus M'_{k-1}.
			\end{numcases}
		\end{subequations}
		Nous avons \( \bigcup_kM'_k=\bigcup_kM_k\). Nous avons alors le calcul suivant :
		\begin{equation}
			\mu\big( \bigcup_kM_k \big)=\mu\big( \bigcup_kM'_k \big)=\sum_{k}\mu(M'_k)=\sum_k\mu(M_k\setminus M'_{k-1})\leq \sum_k\mu(M_k).
		\end{equation}
		La dernière inégalité utilise le point \ref{ITEMooLEGKooWnYmlf}.
	\end{subproof}
\end{proof}

\begin{lemma}[\cite{MonCerveau}]\label{LemAZGByEs}
	Résultats sur les unions croissantes d'ensembles mesurables dans \( (S,\tribA,\mu)\).
	\begin{enumerate}
		\item\label{ItemJWUooRXNPci}
		      Si \( (A_k)\) est une suite croissante d'ensembles \( \mu\)-mesurables dont l'union est mesurable, alors
		      \begin{equation}
			      \lim_{n\to \infty} \mu(A_k)=\mu(\bigcup_kA_k).
		      \end{equation}

		\item\label{ItemJWUooRXNPcii}
		      Soit \( K_n\), une suite emboîtée d'éléments de \( \tribA\) tels que \( K_n\to S\). Si \( A\in\tribA\) alors
		      \begin{equation}
			      \lim_{n\to \infty} \mu(A\cap K_n)=\mu(A).
		      \end{equation}
	\end{enumerate}
\end{lemma}

\begin{proof}
	Pour prouver~\ref{ItemJWUooRXNPci}, nous faisons le coup de l'union télescopique, en posant \( A_0=\emptyset\) :
	\begin{equation}
		\bigcup_{k=1}^{\infty}A_k=\bigcup_{k=1}^{\infty}(A_k\setminus A_{k-1}).
	\end{equation}
	Les ensembles \( A_k\setminus A_{k-1}\) sont deux à deux disjoints, donc la propriété~\ref{ItemQFjtOjXiii} de la définition d'une mesure donne
	\begin{subequations}
		\begin{align}
			\mu(\bigcup_{k=1}^{\infty}A_k) & =\mu\left( \bigcup_{k=1}^{\infty}(A_k\setminus A_{k-1}) \right)              \\
			                               & =\sum_{k=1}^{\infty}\mu(A_k\setminus A_{k-1})                                \\
			                               & =\sum_{k=1}^{\infty}\big( \mu(A_k)-\mu(A_{k-1}) \big)   \label{subEqMDRRorb} \\
			                               & =\lim_{k\to \infty} \mu(A_k)-\mu(A_0)                                        \\
			                               & =\lim_{k\to \infty} \mu(A_k).
		\end{align}
	\end{subequations}
	où pour obtenir~\ref{subEqMDRRorb}, nous avons utilisé le lemme~\ref{LemPMprYuC}.

	Le point~\ref{ItemJWUooRXNPcii} est une application du point~\ref{ItemJWUooRXNPci}.
\end{proof}

\begin{example}
	L'intégration «à la Riemann» n'est pas dans la théorie des espaces mesurés. En effet l'ensemble
	\begin{equation}
		\tribA=\{   A\subset\mathopen[ 0 , 1 \mathclose]\tq   \mtu_A\text{ est intégrable au sens de Riemann}   \}
	\end{equation}
	n'est pas une tribu. Par exemple les singletons en font partie tandis que \( \mathopen[ 0 , 1 \mathclose]\cap \eQ\) n'en fait pas partie bien que ce soit une union dénombrable de singletons.
\end{example}

\begin{definition}
	Si \( \mu\) est une mesure nous disons qu'une propriété est vraie \( \mu\)-\defe{presque partout}{presque!partout} si elle est fausse seulement sur un ensemble de mesure nulle.
\end{definition}

Par exemple la fonction de Dirichlet est presque partout égale à la fonction \( 1\) (pour la mesure de Lebesgue).

\begin{definition}[fonction mesurable]
	Une application entre espace mesurés
	\begin{equation}
		f\colon (\Omega,\tribA)\to (\Omega',\tribA')
	\end{equation}
	est \defe{mesurable}{mesurable!application} si pour tout \( B\in\tribA'\), l'ensemble \( f^{-1}(B)\) est dans \( \tribA\).
\end{definition}
\index{application!mesurable}

\begin{lemma}   \label{LemIDITgAy}
	Une union dénombrable de parties de mesure nulle est de mesure nulle.
\end{lemma}

\begin{proof}
	C'est une conséquence du lemme \ref{LemPMprYuC}\ref{ITEMooABPYooFQEzqE} : si les \( A_i\) sont de mesure nulle,
	\begin{equation}
		\mu\left( \bigcup_{i=1}^{\infty}A_i \right)\leq\sum_{i=1}^{\infty} \mu(A_i)=0
	\end{equation}
\end{proof}

\begin{definition}
	Si \( (A_n)\) est une suite croissante d'ensembles alors la \defe{limite}{limite!d'ensembles} est
	\begin{equation}
		\lim_nA_n=\bigcup_{i=0}^{\infty}A_i.
	\end{equation}
	Si la suite est décroissante alors la limite est
	\begin{equation}
		\lim_nA_n=\bigcap_{i=0}^{\infty}A_i.
	\end{equation}
\end{definition}

\ifbool{isGiulietta}{ Pour une suite ni croissante ni décroissante d'ensembles, il y a la notion de limite inductive\footnote{\emph{direct limit} en anglais.} qui sera un peu traitée à la section~\ref{SecDirectLimit}.  }{}

\begin{proposition}[\cite{RArwFWJ}] \label{PropAFNPSsm}
	Soient \( \mu\) une mesure sur \( \Omega\) et \( (S_n)\) une suite croissante d'ensembles \( \mu\)-mesurables de \( \Omega\). Nous notons
	\begin{equation}
		S=\lim_nS_n.
	\end{equation}
	Alors pour tout ensemble mesurable\footnote{Définition~\ref{DefHGsQxHB}} \( A\subset\Omega\) nous avons
	\begin{equation}
		\mu(A\cap S)=\lim_{n\to \infty} \mu(A\cap S_n).
	\end{equation}
\end{proposition}

\begin{proof}
	L'inégalité \( \lim\mu(A\cap S_n)\leq \mu(A\cap S)\) est simple à prouver. En effet pour tout \( n\) nous avons \( A\cap S_n\subset A\cap S\) et donc par le lemme~\ref{LemPMprYuC} nous avons
	\begin{equation}
		\mu(A\cap S_n)\leq\mu(A\cap S).
	\end{equation}
	En passant à la limite (qui respecte les inégalités) nous avons l'inégalité.

	Nous passons à l'inégalité dans l'autre sens.

	\begin{subproof}
		\spitem[Si \( \infty\)]
		D'abord si \( \mu(A\cap S_n)=\infty\) pour un certain \( n\), alors cela vaut encore \( \infty\) pour tous les \( n\) suivants, et la limite est \( \infty\) sans problème. Donc nous supposons que \( \mu(A\cap S_n)<\infty\) pour tout \( n\in \eN\).

		\spitem[Une petite hypothèse en plus]
		Quitte à renommer les indices, nous supposons que \( S_0=\emptyset\).

		\spitem[\( S\) comme union de différences]
		Nous montrons à présent que \( S=\bigcup_n(S_{n+1}\setminus S_n)\). Soit \( x\in S\). Il existe \( n\geq 0\) tel que \( x\in S_n\). Et vu que \( S_0=\emptyset\), il existe même un \( n\geq 0\) tel que \( x\notin S_n\) et \( x\in S_{n+1}\). Autrement dit, pour tout \( x\in S\), il existe \( n\geq 0\) tel que \( x\in S_{n+1}\setminus S_n\).

		Nous avons donc bien
		\begin{equation}
			S=\bigcup_{n=0}^{\infty}\big( S_{n+1}\setminus S_n \big)
		\end{equation}
		Comme annoncé.
		\spitem[Un bon petit calcul]
		Par conséquent
		\begin{equation}
			A\cap S = A\cap\bigcup_{n=0}^{\infty}(S_{n+1}\setminus S_n)
			= \bigcup_{n=0}^{\infty}A\cap(S_{n+1}\setminus S_n)
		\end{equation}
		Étant donné que les ensembles \( A\cap(S_{n+1}\setminus S_n)\) sont disjoints,
		\begin{subequations}
			\begin{align}
				\mu(A\cap S) & =\sum_{n=0}^{\infty}\mu\big( A\cap(S_{n+1}\setminus S_n) \big)                                 \\
				             & =\sum_{n=0}^{\infty}\mu\Big( (A\cap S_{n+1})\setminus (A\cap S_n) \Big)                        \\
				             & =\sum_{n=0}^{\infty}\big( \mu(A\cap S_{n+1})-\mu(A\cap S_n) \big)                              \\
				             & =\lim_{n\to \infty} \mu(A\cap S_{n+1})-\underbrace{\mu(A\cap S_0)}_{=0}   \label{subeqLTvmTjO} \\
				             & =\lim_{n\to \infty} \mu(A\cap S_n).
			\end{align}
		\end{subequations}
		Dans ce calcul nous avons utilisé plusieurs fois le fait que les \( S_n\) et \( A\) étaient mesurables (et la propriété de tribu qui dit que \( A\cap S_n\) est également mesurable) ainsi que le lemme~\ref{LemPMprYuC}. Nous avons aussi utilisé la série télescopique dans \( \eR\) pour obtenir \eqref{subeqLTvmTjO}.
	\end{subproof}
\end{proof}

\begin{definition}[\( \lambda\)-système\cite{PVWUyDH}]      \label{DefRECXooWwYgej}
	Soit \( E\) un ensemble. Un ensemble \( \tribD\) de parties de \( E\) est un \defe{\( \lambda\)-système}{\( \lambda\)-système} lorsqu'il vérifie les conditions suivantes :
	\begin{enumerate}
		\item
		      si \( A,B\in\tribD\) avec \( A\subset B\) alors \( B\setminus A\in \tribD\),
		\item
		      si \( (A_k)_{k\geq 1}\) est une suite croissante d'éléments de \( \tribD\) alors \( \bigcup_kA_k\in\tribD\).
	\end{enumerate}
\end{definition}
Note : une tribu est un \( \lambda\)-système.

\begin{lemma}[\cite{PVWUyDH}]
	Une intersection quelconque de \( \lambda\)-systèmes dans \( E\) est un \( \lambda\)-système dans \( E\).
\end{lemma}

\begin{proof}
	Soient \( \{ \tribD_l \}_{l\in L}\) des \( \lambda\)-systèmes indicés par un ensemble \( L\). Si \( A,B\in\bigcap_{l\in L}\tribD_l\) alors \( B\setminus A\in\tribD_l\) pour tout \( l\in L\) et donc \( B\setminus A\in\bigcap_{l\in L}\tribD_l\). De la même façon si \( (A_k)\) est une suite croissante dans \( \bigcap_{l\in L}\tribD_l\) alors pour tout \( l\in L\) nous avons \( \bigcup_kA_k\in\tribD_l\). Donc \( \bigcup_kA_k\in\bigcap_l\tribD_l\).
\end{proof}
Ce lemme est ce qui permet de définir le \( \lambda\)-système \defe{engendré}{engendré!\( \lambda\)-système} par une classe \( \tribA\) de parties de \( E\) : c'est l'intersection de tous les \( \lambda\)-systèmes de \( E\) contenant \( \tribA\).

\begin{lemma}[\cite{PVWUyDH}]   \label{LemLUmopaZ}
	Soit \( \tribC\) une classe de parties de \( E\) (contenant \( E\) lui-même) qui soit stable par intersection finie. Alors le \( \lambda\)-système engendré par \( \tribC\) coïncide avec la tribu engendrée par \( \tribC\).
\end{lemma}

\begin{proof}
	Nous notons \( \tribE\) le \( \lambda\)-système engendré par \( \tribC\) et \( \tribF\) la tribu engendrée par \( \tribC\). Étant donné que \( \tribF\) est un \( \lambda\)-système nous avons \( \tribE\subset\tribF\). Pour montrer l'inclusion inverse nous allons prouver que \( \tribE\) est une tribu.

	D'abord pour \( C\in\tribC\) nous posons
	\begin{equation}
		\mG_C=\{ A\subset \tribE\tq A\cap C\in\tribE \}.
	\end{equation}
	et pour \( F\in\tribE\),
	\begin{equation}
		\mH_F=\{ A\in\tribE\tq A\cap F\in\tribE \}.
	\end{equation}
	Nous allons montrer que \( \mG_C\) et \( \mH_F\) sont des \( \lambda\)-systèmes et que \( \mG_C=\mH_F=\tribE\).

	Nous commençons par \( \mG_C\). Si \( A,B\in\mG_C\) avec \( A\subset B\) alors
	\begin{equation}
		(B\setminus A)\cap C=\underbrace{(B\cap C)}_{\in\tribE}\setminus\underbrace{(A\cap C)}_{\in\tribE}.
	\end{equation}
	Puisque \( \tribE\) est un \( \lambda\)-système et que \( (A\cap C)\subset(B\cap C)\), nous avons bien \( (B\setminus A)\cap C\in\tribE\) et donc \( B\setminus A\in\mG_C\). Soit maintenant \( (A_k)\) une suite croissante dans \( \mG_C\). Nous avons
	\begin{equation}
		\big( \bigcup_{k=1}^{\infty}A_k \big)\cap C=\bigcup_{k=1}^{\infty}(A_k\cap C)
	\end{equation}
	qui est une union d'une suite croissante d'éléments de \( \tribE\). Donc \( \bigcup_{k=1}^{\infty}(A_k\cap C)\in\tribE\), ce qui signifie que \( \bigcup_{k=1}^{\infty}A_k\in\mG_C\). Cela termine la preuve du fait que \( \mG_C\) soit un \( \lambda\)-système.

	Étant donné que \( \tribC\) est stable par intersection finie, si \( K\in\tribC\) nous avons \( C\cap K\in\tribC\), ce qui signifie que \( K\in\mG_C\). Nous avons donc \( \tribC\subset\mG_C\). Donc \( \mG_C\) est un \( \lambda\)-système vérifiant \( \tribC\subset\mG_C\subset\tribE\). Mais comme \( \tribE\) est le plus petit \( \lambda\)-système contenant \( \tribC\) nous avons en fait \( \mG_C=\tribE\).

	Nous montrons à présent que \( \mH_F\) est un \( \lambda\)-système. Si \( A,B\in\mH_F\) avec \( A\subset B\) alors \( (B\setminus A)\cap F=(B\cap F)\setminus(A\cap F)\). Comme \( \tribE\) est un \( \lambda\)-système et que \( A\cap F\) et \( B\cap F\) sont dans \( \tribE\) avec \( A\cap F\subset B\cap F\), nous avons
	\begin{equation}
		(B\cap F)\setminus(A\cap F)\in\mH_F.
	\end{equation}
	Soit maintenant \( (A_k)_{k\geq 1}\) une suite croissante dans \( \mH_F\). Pour tout \( k\) nous avons \( A_k\cap F\in\tribE\), ce qui donne
	\begin{equation}
		\big( \bigcap_{k=1}^{\infty}A_k \big)\cap F=\bigcap_{k=1}^{\infty}(A_k\cap F)\in\tribE.
	\end{equation}
	Donc \( \mH_F\) est un \( \lambda\)-système vérifiant \( \tribC\subset\mH_F\subset\tribE\). Nous en concluons que pour tout \( C\in\tribC\) et pour tout \( F\in\tribE\),
	\begin{equation}
		\mG_C=\mH_F=\tribE.
	\end{equation}

	Nous allons maintenant prouver que \( \tribE\) est une tribu\footnote{Définition~\ref{DefjRsGSy}.}.
	\begin{enumerate}
		\item
		      Si \( F\in\tribE\) alors \( E\cap F=F\in\tribE\), ce qui signifie que \( E\in\mH_F=\tribE\).
		\item
		      Si \( A\in \tribE\) alors \( E\setminus A\in\tribE\) parce que \( \tribE\) est un \( \lambda\)-système et \( E\in\tribE\). Donc \( \complement A\in\tribE\).
		\item
		      Montrons que \( \tribE\) est stable par union finie en considérant \( A,B\in\tribE\). Comme \( E\) est également un élément de \( \tribE\) nous avons
		      \begin{equation}
			      E\setminus(A\cup B)=(E\setminus A)\cap(E\setminus B)\in\tribE.
		      \end{equation}
		      Cela prouve que \( \complement( A\cup B)\in \tribE\). Par complémentarité nous avons aussi \( A\cup B\in\tribE\).

		      Soient \( A_k\in\tribE\), et nommons \( B_p=A_1\cup\ldots\cup A_p\). Les ensembles \( B_p\) forment une suite croissante d'éléments de \( \tribE\). L'union est donc dans \( \tribE\) et ce dernier est, au final, stable par union dénombrable.
	\end{enumerate}
	Maintenant que \( \tribE\) est une tribu nous avons \( \tribF\subset\tribE\) parce que \( \tribF\) est la plus petite tribu contenant \( \tribC\). Nous en déduisons que \( \tribE=\tribF\), ce qu'il fallait démontrer.
\end{proof}

Le théorème suivant permet de prouver que la mesure de Lebesgue est l'unique mesure possible ayant les bonnes valeurs sur les intervalles (théorème~\ref{ThoDESooEyDOe}).
\begin{theorem}[Unicité des mesures\cite{PVWUyDH}] \label{ThoJDYlsXu}
	Soient \( \mu\) et \( \nu\), deux mesures sur \( (E,\tribA)\) et un ensemble \( \tribE\) de parties de \( E\) telles que
	\begin{enumerate}
		\item
		      La tribu engendrée par \( \tribE\) soit \( \tribA\).
		\item
		      si \( A,B\in\tribE\) alors \( A\cap B\in\tribE\)
		\item
		      il existe une suite croissante \( (E_n)\) dans \( \tribE\) telle que
		      \begin{enumerate}
			      \item
			            \( E=\lim E_n\),
			      \item
			            \( \mu(E_n)\) et \( \nu(E_n)\) sont finis pout tout \( n\).
		      \end{enumerate}
	\end{enumerate}
	Alors si les mesures \( \mu\) et \( \nu\) coïncident sur \( \tribE\), elles coïncident sur \( \tribA\) en entier.
\end{theorem}
\index{unicité!des mesures}

\begin{proof}
	Soit \( (E_n)\) une suite croissante dans \( \tribE\) telle que \( E=\lim E_n\).

	\begin{subproof}
		\spitem[Des restrictions]
		Nous considérons \( \mu_n\) et \( \nu_n\), les restrictions de \( \mu\) et \( \nu\) à \( E_n\), c'est-à-dire
		\begin{subequations}
			\begin{align}
				\mu_n(A) & =\mu(A\cap E_n)  \\
				\nu_n(A) & =\nu(A\cap E_n).
			\end{align}
		\end{subequations}
		Puisque les \( E_n\) sont dans \( \tribE\subset\tribA\), ils sont mesurables au sens de \( \mu\) et \( \nu\). Par la proposition~\ref{PropAFNPSsm}, pour tout \( A\in \tribE\) nous avons alors
		\begin{subequations}
			\begin{align}
				\lim_{n\to \infty} \mu_n(A) & =\mu(A) \\
				\lim_{n\to \infty} \nu_n(A) & =\nu(A)
			\end{align}
		\end{subequations}
		\spitem[Ce que nous devons prouver]
		Nous devons donc seulement montrer que pour tout \( A\in\tribA\) et pour tout \( n\in\eN\), \( \mu_n(A)=\nu_n(A)\). Pour cela nous nous fixons un \( n\) et nous considérons l'ensemble de parties
		\begin{equation}
			\tribD=\{ A\in\tribA\tq\mu_n(A)=\nu_n(A) \}.
		\end{equation}
		Le but sera de prouver que \( \tribD=\tribA\).
		\spitem[\( \nu_n=\mu_n\) sur \( \tribE\)]
		Soit \( A\in \tribE\). Vu que \( E_n\in \tribE\), par hypothèse \( A\cap E_n\in\tribE\) et donc
		\begin{subequations}
			\begin{align}
				\mu_n(A) & =\mu(A\cap E_n)                                     \\
				         & =\nu(A\cap E_n)        \label{SUBaEQooUJDPooOLSWmn} \\
				         & =\nu_n(A).
			\end{align}
		\end{subequations}
		Pour \eqref{SUBaEQooUJDPooOLSWmn}, nous avons utilisé l'hypothèse comme quoi \( \mu=\nu\) sur \( \tribE\).

		\spitem[Encore d'autres parties]
		Nous définissons \( \tribE'=\tribE\cup\{ E \}\). En particulier \( \tribE'\subset\tribD\).

		\spitem[\( \mu_n=\nu_n\) sur \( \tribE'\)]
		Nous avons déjà vue l'égalité sur \( \tribE\). Il suffit de vérifier l'égalité sur \( E\). Vu que \( E\cap E_n=E_n\in\tribE\), nous avons
		\begin{equation}
			\mu_n(E)=\mu(E\cap E_n)=\mu(E_n)=\nu(E_n)=\nu(E_n\cap E)=\nu_n(E).
		\end{equation}

		\spitem[\( \tribD\) est un \( \lambda\)-système]
		Montrons que \( \tribD\) est un \( \lambda\)-système. Soient \( A,B\in\tribD\) avec \( A\subset B\). Alors, étant donné que les mesures \( \mu_n\) et \( \nu_n\) sont finies, le lemme~\ref{LemPMprYuC} nous donne
		\begin{subequations}
			\begin{align}
				\mu_n(B\setminus A) & =\mu_n(B)-\mu_n(A)  \\
				\nu_n(B\setminus A) & =\nu_n(B)-\nu_n(A).
			\end{align}
		\end{subequations}
		Donc \( \mu_n(B\setminus A)=\nu_n(B\setminus A)\) et \( B\setminus A\in\tribD\).

		Soit par ailleurs une suite croissante \( (A_k)_{k\geq 1}\) d'éléments de \( \tribD\). Le lemme~\ref{LemAZGByEs}\ref{ItemJWUooRXNPci} nous donne
		\begin{equation}
			\mu_n(\bigcup_{k=1}^{\infty}A_k)=\lim_{p\to \infty} \mu_n(A_p).
		\end{equation}
		Mais puisque pour chaque \( p\) nous avons \( \mu_n(A_p)=\nu_n(A_p)\), nous avons aussi
		\begin{equation}
			\mu_n(\bigcup_{p=1}^{\infty}A_p)=\nu_n(\bigcup_{p=1}^{\infty}A_p).
		\end{equation}
		Donc \( \tribD\) est bel et bien un \( \lambda\)-système contenant \( \tribE'\).

		\spitem[Conclusion]
		Par le lemme~\ref{LemLUmopaZ}, le \( \lambda\)-système engendré par \( \tribE'\) est égal à la tribu engendrée par \( \tribE'\), mais par hypothèse la tribu engendrée par \( \tribE\) est \( \tribA\), donc le \( \lambda\)-système engendré par \( \tribE'\) est \( \tribA\). Comme \( \tribD\) est un \( \lambda\)-système contenant \( \tribE'\), nous avons alors \( \tribA\subset\tribD\) et donc \( \tribA=\tribD\), ce qu'il fallait.
	\end{subproof}
\end{proof}

\begin{example}\label{ExDMPoohtNAj}
	La partie \( \tribE\) des intervalles de \( \eR\) de la forme \( \mathopen] a , b \mathclose[\) engendre les boréliens par la proposition~\ref{PropNBSooraAFr}. Par conséquent pour vérifier que deux mesures sont égales sur les boréliens de \( \eR\), il suffit de prouver qu'elles sont égales sur les intervalles ouverts.
\end{example}

%---------------------------------------------------------------------------------------------------------------------------
\subsection{Mesure extérieure}
%---------------------------------------------------------------------------------------------------------------------------

Nous avons déjà défini la notion de mesure extérieure en la définition~\ref{DefUMWoolmMaf}.

\begin{lemma}[\cite{MesureLebesgueLi}]  \label{LemULSooBgZLI}
	Soit \( (S,\tribF,\mu)\) un espace mesuré et \( X\subset S\). Alors
	\begin{equation}
		\inf_{\substack{A\in\tribF\\X\in A}}\mu(A) = \inf\left\{ \sum_n\mu(A_n) \tq A_n\in\tribF,X\subset\bigcup_kA_k \right\}.
	\end{equation}
\end{lemma}

\begin{proof}
	Pour montrer l'inégalité \( \geq\), nous remarquons qu'il y a plus d'éléments dans l'ensemble du second membre que dans le premier. En effet si \( A\in\tribF\) avec \( X\subset A\) alors dans le membre de gauche nous pouvons prendre \( A_1=A\) et \( A_{n\geq 1}=\emptyset\).

	Pour l'inégalité dans l'autre sens, nous montrons que tout élément de
	\begin{equation}    \label{EqZRAooBCPFk}
		\left\{ \sum_n\mu(A_n) \tq A_n\in\tribF,X\subset\bigcup_kA_k \right\}
	\end{equation}
	est plus grand qu'un élément de
	\begin{equation}    \label{EqYNMooNvCtS}
		\{ \mu(A)\tq A\in\tribF,X\subset A \}.
	\end{equation}
	En effet si \( A_n\in\tribF\) avec \( X\subset \bigcup_kA_k\) alors en posant \( A=\bigcup_kA_k\) nous avons \( A\in\tribF\) avec \( X\subset A\) ainsi que \( \mu(A)\leq\sum_n\mu(A_n)\). Cela prouve que l'élément \( \sum_n\mu(A_n)\) de \eqref{EqZRAooBCPFk} est plus grand que l'élément \( \mu(A)\) de \eqref{EqYNMooNvCtS}.
\end{proof}

\begin{normaltext}
	La proposition \ref{PropFDUooVxJaJ} pourrait être vue comme un cas particulier de la proposition \ref{PropIUOoobjfIB} en utilisant~\ref{LemULSooBgZLI}. Nous en donnons cependant une preuve directe, qui est presque identique à celle de~\ref{PropIUOoobjfIB}, mais avec une ou deux simplifications.
\end{normaltext}

\begin{proposition}[\cite{MesureLebesgueLi}]    \label{PropFDUooVxJaJ}
	Soit un espace mesuré \( (S,\tribF,\mu)\) et l'application\nomenclature[Y]{\( \mu^*\)}{La mesure extérieure associée à la mesure \( \mu\)}
	\begin{equation}
		\begin{aligned}
			\mu^*\colon \partP(S) & \to \mathopen[ 0 , \infty \mathclose]              \\
			X                     & \mapsto \inf\{ \mu(A)\tq A\in\tribF,X\subset A \}.
		\end{aligned}
	\end{equation}
	Alors \( \mu^*\) est une mesure extérieure sur \( S\) et sa restriction à \( \tribF\) est égale à \( \mu\).
\end{proposition}

\begin{proof}
	Notons que la définition est bonne parce que l'ensemble sur lequel l'infimum est pris n'est pas vide : considérer \( A=S\).
	\begin{subproof}
		\spitem[Le vide]
		D'abord \( \mu^*(\emptyset)=0\) parce que \( \emptyset\in\tribF\).
		\spitem[\( \mu^*\) est croissante]
		Soit \( X\subset Y\) dans \( \partP(S)\). Si \( Y\subset A\) alors \( X\subset A\), donc
		\begin{equation}
			\inf \{ \mu(A)\tq A\in\tribF,X\subset A \}\leq \inf \{ \mu(A)\tq A\in\tribF,Y\subset A \},
		\end{equation}
		ce qui signifie que \( \mu^*(X)\leq \mu^*(Y)\).
		\spitem[Inégalité par union dénombrable]
		Soit \( (X_n)_{n\in \eN}\) une suite de parties de \( S\). Si il existe \( n_0\) tel que \( \mu^*(X_{n_0})=\infty\) alors nous avons automatiquement \( \sum_n\mu^*(X_n)=\infty\) et l'inégalité demandée est évidente parce que n'importe quel nombre est plus petit ou égal à \( \infty\). Nous supposons donc que \( \mu^*(X_n)<\infty\) pour tout \( n\).

		Soit \( \epsilon>0\) et par définition pour chaque \( n\), il existe un \( A_n\in \tribF\) tel que \( X_n\subset A_n\) et \( \mu(A_n)\leq \mu^*(X_n)+\frac{ \epsilon }{ 2^n }\). Bien entendu nous avons
		\begin{equation}
			\bigcup_nX_n\subset \bigcup_nA_n\in\tribF.
		\end{equation}
		Nous en déduisons que
		\begin{equation}
			\mu^*\big( \bigcup_nX_n \big)\leq\mu\big( \bigcup_nA_n \big).
		\end{equation}
		Mais \( (S,\tribF,\mu)\) étant un espace mesuré,
		\begin{equation}
			\mu\big( \bigcup_nA_n \big)\leq \sum_n\mu(A_n).
		\end{equation}
		Au final nous avons les inégalités
		\begin{subequations}
			\begin{align}
				\mu^*\big( \bigcup_nX_n \big) & \leq  \mu\big( \bigcup_nA_n \big)                                       \\
				                              & \leq  \sum_n\mu(A_n)                                                    \\
				                              & \leq  \sum_n\mu^*(X_n)+\epsilon\underbrace{ \sum_n\frac{1}{ 2^n }}_{=1} \\
				                              & =   \sum_n\mu^*(X_n)+\epsilon.
			\end{align}
		\end{subequations}
		Ceci étant vrai pour tout \( \epsilon\),
		\begin{equation}
			\mu^*\big( \bigcup_nX_n \big)\leq\sum_n\mu^*(X_n),
		\end{equation}
		ce qui prouve que \( \mu^*\) est une mesure extérieure.
		\spitem[Restriction]
		Supposons que \( X\in\tribF\). Alors si \( X\subset A\) nous avons \( \mu(X)\leq \mu(A)\); mais en même temps, \( \mu(X)\) est dans l'infimum qui définit \( \mu^*(X)\) donc
		\begin{equation}
			\mu^*(X)\leq\mu(X)\leq \inf \{ \mu(A)\tq A\in\tribF,X\subset A \}\leq \mu(X)\leq\mu^*(X).
		\end{equation}
		Donc nous avons égalité de tous les éléments de cette chaine d'inégalité.
	\end{subproof}
\end{proof}

\begin{definition}  \label{DefTRBoorvnUY}
	Soit \( S\) un ensemble et \( m^*\) une mesure extérieure sur \( S\). Une partie \( A\subset S\) est \defe{\(  m^*\)-mesurable}{mesurable!au sens de \(  m^*\)} si pour tout \( X\subset S\),
	\begin{equation}
		m^*(X)=m^*(X\cap A)+m^*(X\cap A^c).
	\end{equation}
\end{definition}

\begin{remark}
	L'inégalité
	\begin{equation}
		m^*(X)\leq m^*(X\cap A)+m^*(X\cap A^c)
	\end{equation}
	étant toujours vraie, pour prouver qu'un ensemble est \( m^*\)-mesurable, il est suffisant de prouver l'inégalité inverse :
	\begin{equation}
		m^*(X)\geq m^*(X\cap A)+m^*(X\cap A^c)
	\end{equation}
\end{remark}
La définition~\ref{DefTRBoorvnUY} est motivée par la proposition suivante.

\begin{proposition} \label{PropOJFoozSKAE}
	Soit un espace mesuré \( (S,\tribF,\mu)\) et \( \mu^*\) la mesure extérieure qui va avec. Alors tous les éléments de \( \tribF\) sont \( \mu^*\)-mesurables.

	En d'autres termes, pour tout \( A\in\tribF\) et tout \( X\subset S\) nous avons
	\begin{equation}
		\mu^*(X)=\mu^*(X\cap A)+\mu^*(X\cap A^c).
	\end{equation}
\end{proposition}

\begin{proof}
	Puisque \( X=(X\cap A)\cup(X\cap A^c)\), et que \( \mu^*\) est une mesure extérieure,
	\begin{equation}
		\mu^*(X)\leq \mu^*(X\cap A)+\mu^*(X\cap A^c).
	\end{equation}
	Nous devons montrer l'inégalité inverse.

	Soit \( B\in\tribF\) tel que \( X\subset B\). D'une part nous avons \( X\cap A\subset B\cap A\in\tribF\), donc
	\begin{equation}
		\mu^*(X\cap A)\leq \mu^*(B\cap A)= \mu(B\cap A).
	\end{equation}
	Et d'autre part, \( X\cap A^c\subset B\cap A^c\in\tribF\), donc
	\begin{equation}
		\mu^*(X\cap A^c)\leq \mu(B\cap A^c).
	\end{equation}
	En rassemblant,
	\begin{equation}    \label{EqLSMooTyHLB}
		\mu^*(X\cap A)+\mu^*(X\cap A^c)\leq \mu(B\cap A)+\mu(B\cap A^c)=\mu(B).
	\end{equation}
	La dernière égalité vient du fait que \( B\cap A\) et \( B\cap A^c\) sont disjoints et que \( \mu\) est une mesure. L'inégalité \eqref{EqLSMooTyHLB} étant vraie pour tout \( B\in \tribF\) tel que \( X\subset B\), elle est encore vraie pour l'infimum :
	\begin{equation}
		\mu^*(X\cap A)+\mu^*(X\cap A^c)\leq \inf \{ \mu(B)\tq B\in\tribF,X\subset B \}=\mu^*(X).
	\end{equation}
	Nous avons donc prouvé que
	\begin{equation}
		\mu^*(X\cap A)+\mu^*(X\cap A^c)\leq \mu^*(X).
	\end{equation}
\end{proof}

\begin{remark}
	Notons la duplicité du vocabulaire. Les ensembles \( \mu\)-mesurables sont les éléments de \( \tribF\), qui sont à priori les seuls sur lesquels \( \mu\) est calculable\footnote{«calculable» au sens où \( \mu\) y vaut un nombre bien défini; après, que ce soit facile ou pas à calculer dans la pratique, c'est une autre histoire.}, alors que les \( \mu^*\)-mesurables sont les parties de \( S\) qui vérifient une certaine propriété (et \( \mu^*\) est calculable sur toutes les parties de \( S\)).
\end{remark}
