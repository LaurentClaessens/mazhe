% This is part of Mes notes de mathématique
% Copyright (c) 2022-2024
%   Laurent Claessens
% See the file fdl-1.3.txt for copying conditions.

%+++++++++++++++++++++++++++++++++++++++++++++++++++++++++++++++++++++++++++++++++++++++++++++++++++++++++++++++++++++++++++ 
\section{Connexité par arcs}
%+++++++++++++++++++++++++++++++++++++++++++++++++++++++++++++++++++++++++++++++++++++++++++++++++++++++++++++++++++++++++++

\begin{definition}      \label{DEFooOXVCooBizpgK}
	Une partie \( A\) d'un espace topologique est \defe{connexe par arcs}{connexe par arc} si pour tout \( a,b \in A\), il existe une application continue \( \gamma\colon \mathopen[ 0 , 1 \mathclose]\to A\) telle que \( \gamma(0)=a\) et \( \gamma(1)=b\).
\end{definition}

\begin{normaltext}
	Un exemple d'ensemble connexe mais pas connexe par arcs est donné par la proposition \ref{PROPooVXDNooPZYKPr}. L'idée de cet exemple est de construire un ensemble en deux parties reliées par un chemin de longueur infinie.

	Un espoir fou nous prend alors de croire que nous pouvons produire un exemple plus simple avec \( \eR\cup\{ +\infty \}\) parce que, dans cet ensemble, \( 1\) et \( +\infty\) sont reliés par un chemin de longueur infinie. La proposition \ref{PROPooLOQVooULDhZz} nous montrera que non.
\end{normaltext}


\begin{lemma}		\label{LEMooQPYMooRKVSrv}
	Une partie connexe par arcs est connexe.
\end{lemma}

Beaucoup d'espaces connexe sont connexes par arcs. La proposition suivante couvre entre autres le cas de tout ouvert connexe d'un espace vectoriel normé comme l'espace euclidien.

\begin{proposition}[\cite{BIBooPUQRooLuONhM}]       \label{PROPooYFDBooHbBjzF}
	Tout espace connexe et localement connexe par arcs est connexe par arcs.
\end{proposition}


\begin{lemma}       \label{LEMooTVQMooFxrFaT}
	Soient deux espaces topologiques \( E\) et \( F\), et \( f : E\to F\) un homéomorphisme\footnote{Définition \ref{DEFooYPGQooMAObTO}.}. \( E\) est connexe par arcs\footnote{Définition \ref{DEFooOXVCooBizpgK}} si et
	seulement si \( F\) l'est.
\end{lemma}

\begin{normaltext}
	Voici une idée de la preuve.

	On montre en réalité que l'image d'un connexe par arcs
	par une application continue est un connexe par arcs, ce qui
	implique chaque sens de l'équivalence de l'énoncé.

	Soient \( p\) et \( q\) des points de \( F\). Il existe un chemin reliant
	un antécédent de \( p\) et un antécédent de \( q\) (dans \( E\)). L'image
	de ce chemin est un chemin reliant \( p\) et \( q\) (dans \( F\)) puisque
	composé d'applications continues.
\end{normaltext}

\begin{lemma}       \label{LEMooQFQFooDlxkw}
	Une sphère de \( \eR^n\) est connexe par arcs si \( n >1\)
\end{lemma}

\begin{normaltext}
	Une idée de la preuve.

	On voit qu'un cercle est connexe par arcs car on a un
	paramétrage en sinus et cosinus. Pour une sphère \( S\) de centre
	\( a\) en dimension \( n > 2\), on se donne \( p\) et \( q\) sur \( S\) et on
	définit \( P\) le plan affine passant par \( a\), \( p\) et \( q\). Alors \( P\cap S\) est un cercle, donc on peut relier \( p\) à \( q\) par un chemin
	dans cette intersection.

	Pour voir sur une formule que \( P \cap S\) est un cercle, on peut
	écrire \( x - a = \lambda(a-p) + \mu(a-q)\) l'équation (en \( x\)) du
	plan \( P\), et \( \module{x-a}^2 = R^2\) l'équation (en \( x\)) de la
	sphère. En injectant, on obtient une équation du second degré en
	\( \lambda,\mu\) qui se révèle être l'équation d'un cercle à une
	transformation affine près.
\end{normaltext}

\begin{lemma}       \label{LEMooDYNSooOmJbYq}
	Un ouvert connexe par arcs dans \( \eR^n\) (\( n \geq 2\)) reste connexe par arcs même si on lui enlève un point.
\end{lemma}

\begin{normaltext}
	Une idée de la preuve.

	En effet, soit \( U\) un tel ouvert connexe par arcs, et \( p\) un point
	de \( U\). Soient \( x\) et \( y\) sur \( U\setminus\{p\}\). Il existe un
	chemin \( \gamma\) de \( x\) à \( y\). Si le chemin ne passe pas par \( p\),
	c'est gagné. Si il passe par \( p\), on choisit une boule \( B\) fermée
	(de rayon non-nul) centrée en \( p\) qui ne contient ni \( x\) ni
	\( y\). On note
	\( E = \gamma^{-1}(B) \subset [0;1]\) c'est un ensemble compact
	(fermé, par continuité de \( \gamma\), et borné) dont on regarde le
	maximum \( \bar t\) et le minimum \( \underline t\).

	Il reste enfin à définir un chemin entre \( p\) et \( q\) par morceaux
	\begin{enumerate}
		\item Les points \( p\) et \( \gamma(\underline t)\) sont reliés par
		      \( \gamma\),
		\item Par connexité par arcs, il existe un chemin sur la sphère qui
		      relie \( \gamma(\underline t)\) à \( \gamma(\bar t)\),
		\item et enfin \( \gamma(\bar t)\) et \( q\) sont reliés via \( \gamma\);
	\end{enumerate}
	ce qui achève la construction d'un chemin continu entre \( p\) et \( q\).
\end{normaltext}

Pour conclure l'exercice, par l'absurde, on prend un voisinage
connexe et ouvert \( V\) de \( 0\) dans le cône, homéomorphe à un ouvert
connexe \( U\) de \( \eR^2\). Or \( V\setminus\{0\}\) n'est pas connexe par
arcs, alors que l'ouvert dont on retire un point reste connexe par
arcs. C'est impossible, donc l'homéomorphisme n'existe pas, et le
cône n'est pas une variété de dimension \( 2\).
