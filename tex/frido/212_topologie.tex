

 \emph{Principe.} L'idée est de considérer une propriété topologique
 (invariante par homéomorphisme) et de voir qu'elle est vérifiée par
 les ouverts de $\eR^2$ mais pas ceux du cône.

 \begin{lemma}Si $V$ est un voisinage de $0$ sur le cône $C$, alors
   $V\setminus\{0\}$ n'est pas connexe, donc n'est pas connexe par
   arc.\end{lemma}
 \begin{proof}Le cône $C$ est la réunion de $C^+ = C \cap
   \left(\eR^2\times \eR_0^+\right)$ et $C^- = C \cap \left(\eR^2\times
     \eR_0^-\right)$ car le seul point à cote nulle est la singularité
   $0$. Dès lors, $V$ s'écrit comme l'union disjointe de $V\cap C^+$
   et $V\cap C^-$, qui sont non-vides. Donc $V$ n'est pas
   connexe.\end{proof}

On procède en deux étapes, en montrant d'abord qu'il
   existe des points en « dessous » et au « dessus » de
   $0$, puis en essayant de les relier.
     Comme $V$ est un voisinage de $0$, il existe un ouvert $U$ du
     cône centré en $0$ inclut à $V$. Donc par définition de la
     topologie induite, et puisque les boules forment une base de la
     topologie de $\eR^3$, il existe une boule $B$ centrée en $0$ dont
     $U$ est la trace sur $C$, telle que $0 \in (B \cap C) \subset
     V$. On choisit $p = (p_x,p_y,p_z) \in (B \cap C)$, et en
     considérant $p^\prime = (p_x, p_y, -p_z)$ on a ainsi trouvé deux
     points qui vérifient $p_z > 0$ et $p^\prime_z < 0$ (au besoin,
     on les échange).

   \begin{enumerate}
   \item Supposons que $V\setminus\{0\}$ soit connexe par arc. Donc
     il existe un chemin
     \[\gamma : [0;1] \to V\setminus\{0\} : t \mapsto
     (\gamma_x(t),\gamma_y(t),\gamma_z(t))\] qui relie $p$ à
     $p^\prime$ et qui vérifie $\gamma_z(0) = p_z > 0$ et
     $\gamma_z(1) = -p_z < 0$. Or $\gamma_z(t)$ est une fonction
     continue (car $\gamma$ est continu), donc par le théorème des
     valeurs intermédiaires, il existe $\bar t$ qui vérifie
     $\gamma_z(\bar t) = 0$. Or le seul point de $C$ dont la cote
     (coordonnée en $z$) soit nulle est le sommet $0$ qui n'est pas
     dans $V\setminus\{0\}$, d'où la contradiction.
   \end{enumerate}

 \begin{rem}Soient deux espaces topologiques $E$ et $F$, et $f :
   E\to F$ un homéomorphisme. Pour toute partie $A$ de $E$,
   l'espace $E\setminus A$ est homéomorphe au sous-espace $F\setminus
   f(A)$ via la restriction $f_{\vert E\setminus A}$.\end{rem}

 \begin{lemma}Soient deux espaces topologiques $E$ et $F$, et $f :
   E\to F$ un homéomorphisme. $E$ est connexe par arc si et
   seulement si $F$ l'est.\end{lemma}
 \begin{proof}On montre en réalité que l'image d'un connexe par arc
   par une application continue est un connexe par arc, ce qui
   implique chaque sens de l'équivalence de l'énoncé.

   Soient $p$ et $q$ des points de $F$. Il existe un chemin reliant
   un antécédent de $p$ et un antécédent de $q$ (dans $E$). L'image
   de ce chemin est un chemin reliant $p$ et $q$ (dans $F$) puisque
   composé d'applications continues.
 \end{proof}

 \begin{lemma}Une sphère de $\eR^n$ est connexe par arc si $n >
   1$\end{lemma}
 \begin{proof}On voit qu'un cercle est connexe par arc car on a un
   paramétrage en sinus et cosinus. Pour une sphère $S$ de centre
   $a$ en dimension $n > 2$, on se donne $p$ et $q$ sur $S$ et on
   définit $P$ le plan affine passant par $a$, $p$ et $q$. Alors $P
   \cap S$ est un cercle, donc on peut relier $p$ à $q$ par un chemin
   dans cette intersection.

   Pour voir sur une formule que $P \cap S$ est un cercle, on peut
   écrire $x - a = \lambda(a-p) + \mu(a-q)$ l'équation (en $x$) du
   plan $P$, et $\module{x-a}^2 = R^2$ l'équation (en $x$) de la
   sphère. En injectant, on obtient une équation du second degré en
   $\lambda,\mu$ qui se révèle être l'équation d'un cercle à une
   transformation affine près.
 \end{proof}

 \begin{lemma}Un ouvert connexe par arc dans $\eR^n$ ($n \geq 2$) reste
   connexe par arc même si on lui enlève un point.\end{lemma}
 \begin{proof}
   En effet, soit $U$ un tel ouvert connexe par arc, et $p$ un point
   de $U$. Soient $x$ et $y$ sur $U\setminus\{p\}$. Il existe un
   chemin $\gamma$ de $x$ à $y$. Si le chemin ne passe pas par $p$,
   c'est gagné. Si il passe par $p$, on choisit une boule $B$ fermée
   (de rayon non-nul) centrée en $p$ qui ne contient ni $x$ ni
   $y$. On note
   \[E = \gamma^{-1}(B) \subset [0;1]\] c'est un ensemble compact
   (fermé, par continuité de $\gamma$, et borné) dont on regarde le
   maximum $\bar t$ et le minimum $\underline t$.

   Il reste enfin à définir un chemin entre $p$ et $q$ par morceaux
   \begin{enumerate}
   \item Les points $p$ et $\gamma(\underline t)$ sont reliés par
     $\gamma$,
   \item Par connexité par arc, il existe un chemin sur la sphère qui
     relie $\gamma(\underline t)$ à $\gamma(\bar t)$,
   \item et enfin $\gamma(\bar t)$ et $q$ sont reliés via $\gamma$;
   \end{enumerate}
   ce qui achève la construction d'un chemin continu entre $p$ et
   $q$.
 \end{proof}
 Pour conclure l'exercice, par l'absurde, on prend un voisinage
 connexe et ouvert $V$ de $0$ dans le cône, homéomorphe à un ouvert
 connexe $U$ de $\eR^2$. Or $V\setminus\{0\}$ n'est pas connexe par
 arc, alors que l'ouvert dont on retire un point reste connexe par
 arc. C'est impossible, donc l'homéomorphisme n'existe pas, et le
 cône n'est pas une variété de dimension $2$.
