% This is part of Mes notes de mathématique
% Copyright (c) 2011-2024
%   Laurent Claessens
% See the file fdl-1.3.txt for copying conditions.

%+++++++++++++++++++++++++++++++++++++++++++++++++++++++++++++++++++++++++++++++++++++++++++++++++++++++++++++++++++++++++++
\section{Polynômes}
%+++++++++++++++++++++++++++++++++++++++++++++++++++++++++++++++++++++++++++++++++++++++++++++++++++++++++++++++++++++++++++

%--------------------------------------------------------------------------------------------------------------------------- 
\subsection{Polynômes d'une variable}
%---------------------------------------------------------------------------------------------------------------------------

Et voilà la définition que tout le monde attendait; la définition des anneaux de polynômes. Pour ne pas taper trop fort du premier coup, nous commençons par les polynômes d'une seule variable\footnote{Pour les polynômes à plusieurs variables, voir la définition \ref{DEFooZNHOooCruuwI}.}.


L'ensemble des polynômes sur \( A\) sera simplement \( A^{(\eN)}\) (notation \ref{DEFooBMEPooFsCHgb}). Puisque \( \eN\) est un ensemble bien particulier possédant plein de structure, nous allons pouvoir installer sur \( A^{(\eN)}\) une structure non seulement de \( A\)-module (ça c'est déjà fait), mais en plus d'anneau, ainsi qu'une évaluation.
\begin{definition}      \label{DEFooFYZRooMikwEL}
	L'ensemble des \defe{polynômes}{polynômes} en une indéterminée sur l'anneau \( A\) est l'anneau
	\begin{equation}
		\polyP(A)=A^{(\eN)}
	\end{equation}
	défini en \ref{DEFooBMEPooFsCHgb}.
\end{definition}

\begin{normaltext}
	En ce qui concerne la notation \( A[X]\), voir \ref{SUBSECooLEKVooFBPSJz}. Pour \( \eK(X)\) lorsque \( \eK\) est un corps, voir~\ref{DEFooQPZIooQYiNVh}.
\end{normaltext}

\begin{propositionDef}[\cite{BIBooZDRQooLrUVrb}]  \label{DefDegrePoly}
	Soit \( P\) non nul dans \(\polyP(A)\). Nous notons \( a_n\) la valeur\footnote{Ici il y a une énorme subtilité de terminologie. Formellement, \( P\) est une application \( \eN\to A\). Cela n'a rien à voir avec le fait que \( P\) puisse être évalué sur \( A\) avec des formule du type \( P(x)=\sum_na_nx^n\). D'ailleurs nous n'avons pas encore vu cette évaluation.} de \( P\) en \( n\in \eN\) : \( P=(a_n)_{n\in \eN}\).
	\begin{enumerate}
		\item
		      L'ensemble \( \{ n\in \eN\tq a_n\neq 0 \}\) est fini dans \( \eN\).
		\item
		      Cet ensemble possède un minimum et un maximum.
	\end{enumerate}
	Le \defe{degré}{degré d'un polynôme} de \( P\) est
	\begin{equation}
		\deg(P)=\max\{ n\in \eN\tq a_n\neq 0 \},
	\end{equation}
	et la \defe{valuation}{valuation d'un polynôme} de \( P\) est
	\begin{equation}
		\val(P)=\min\{ n\tq a_n\neq 0 \}.
	\end{equation}
	Nous notons
	\begin{equation}		\label{EQooAULWooDgogWT}
		\polyP_n(A)=\{ P\in \polyP(A)\tq \deg(P)\leq n \},
	\end{equation}
	Dans le cas du polynôme nul, l'ensemble \( \{ n\in \eN\tq a_n\neq 0 \}\) est vide, et les définitions ne s'appliquent pas. Nous convenons que
	\begin{subequations}
		\begin{align}
			\val(0) & =+\infty  \\
			\deg(0) & =-\infty.
		\end{align}
	\end{subequations}
\end{propositionDef}

\begin{proof}
	Le fait que \( P\) soit non nul implique que \( A=\{ n\in \eN\tq a_n\neq 0 \}\) est non vide. De plus cet ensemble est fini parce que \( P\in A^{(\eN)}\). Toute partie finie non vide de \( \eN\) étant majorée et minorée (lemme \ref{LEMooKUWUooPLWelf}), le lemme \ref{LEMooOEJOooOgaxzi} définit correctement le minimum et le maximum de \( A\).
\end{proof}

Vu que \( A^{(\eN)}\) est engendré par les \( e_i\), tout polynôme sur \( A\) s'écrit \( P=\sum_{i=1}^na_ie_i\).

\begin{definition}      \label{DEFooNXKUooLrGeuh}
	Nous ajoutons deux structures à \( A^{(\eN)}\).
	\begin{description}
		\item[L'évaluation] Si \( \alpha\in A\) et si \( P\in A^{(\eN)}\), nous définissons \( P(\alpha)\) par
			\begin{equation}        \label{EQooDJISooTEkMOw}
				P(\alpha)=(\sum_{i=0}^{n}a_ie_i)(\alpha)=\sum_{i=0}^na_i\alpha^i,
			\end{equation}
			étant entendu que \( \alpha^0=1\) dans \( A\).

			Cette définition s'étend immédiatement au cas où \( B\) est un anneau qui étend \( A\). Dans ce cas nous pouvons définir \( P(b)\) pour tout \( P\in A^{(\eN)}\) et \( b\in B\) avec la même formule \eqref{EQooDJISooTEkMOw}.
		\item[Le produit] C'est ici que la structure particulière de \( \eN\) est utilisée. Nous définissons le produit \( A^{\eN}\times A^{(\eN)}\to A^{(\eN)}\) de la façon suivante. Si \( (P_k)_{k\in \eN}\) est la suite (presque partout nulle) d'éléments de \( A\) qui définit \( P\) et si \( (Q_k)_{k\in \eN}\) est celle de \( Q\), nous notons
			\begin{equation}    \label{EQooTNCSooKklisb}
				(PQ)_n=\sum_{k=0}^nP_kQ_{n-k},
			\end{equation}
			et donc \( PQ=\sum_i(PQ)_ie_i\). Plus explicitement,
			\begin{equation}    \label{EQooCIBUooAgpxjL}
				(\sum_{i=0}^na_ie_i)(\sum_{j=0}^mb_je_j)=\sum_{k=0}^{\infty}\Big( \sum_{\substack{  (i,j)\in \eN^2 \\i+j=k}}a_ib_j \Big)e_k.
			\end{equation}
			Notons qu'à droite, la somme sur \( k\) est une somme finie.
	\end{description}
\end{definition}

\begin{proposition}     \label{PROPooGDQCooHziCPH}
	Soit un anneau \( A\). À propos de structure sur \( A^{(\eN)}\).
	\begin{enumerate}
		\item
		      Avec le produit, l'ensemble \( A^{(\eN)}\) devient un anneau.
		\item
		      L'application
		      \begin{equation}
			      \begin{aligned}
				      g\colon A^{(\eN)} & \to A             \\
				      P                 & \mapsto P(\alpha)
			      \end{aligned}
		      \end{equation}
		      est un morphisme d'anneaux\footnote{Définition \ref{DEFooSPHPooCwjzuz}.}. En particulier, \( (PQ)(\alpha)=P(\alpha)Q(\alpha)\).
	\end{enumerate}
\end{proposition}

\begin{proof}
	En plusieurs points
	\begin{subproof}
		\spitem[Anneau]
		L'identité pour le produit dans \( A^{(\eN)}\) est le polynôme donné par \( a_0=1\) et \( a_i=0\) pour \( i\neq 0\). Cela se vérifie en utilisant directement la définition \eqref{EQooCIBUooAgpxjL}. La distributivité aussi\quext{Je n'ai pas fait les calculs, écrivez-moi pour me dire si ça va facilement.}.
		\spitem[Le morphisme]
		Nous notons \( P_k\) les éléments de la suite définissant \( P\) et \( Q_k\) ceux de \( Q\). Alors nous avons
		\begin{equation}
			(P+Q)(\alpha)=\sum_k(P_k+Q_k)\alpha^k=\sum_kP_k\alpha^k+\sum_kQ_k\alpha^k=P(\alpha)+Q(\alpha).
		\end{equation}
		Vous aurez noté que la première égalité était la définition \eqref{EQooODBMooQKLUgd}. De même,
		\begin{subequations}
			\begin{align}
				P(\alpha)Q(\alpha) & =\big( \sum_nP_n\alpha^n \big)\big( \sum_kQ_k\alpha^k \big)=\sum_kQ_k\big( \sum_nP_n\alpha^n \big)\alpha^k=\sum_k\sum_nQ_kP_n\alpha^{n+k} \\
				                   & =\sum_m\big( \sum_{l=0}^mP_lQ_{m-l} \big)\alpha^m=\sum_m(PQ)_m\alpha^m=(PQ)(\alpha).
			\end{align}
		\end{subequations}
	\end{subproof}
\end{proof}

\begin{lemma}       \label{LEMooWVUXooQlaepO}
	Si \( A\) est commutatif, alors \( A^{(\eN)}\) est commutatif.
\end{lemma}
%TODOooRDJYooNAqcYr: préciser si A est un anneau ou autre chose.

\begin{proof}
	Soient \( P,Q\in A^{(\eN)}\); pour rappel, le produit est donné par la définition \ref{EQooTNCSooKklisb}. L'application
	\begin{equation}
		\begin{aligned}
			\varphi\colon \{ 0,\ldots, n \} & \to \{ 0,\ldots, n \} \\
			k                               & \mapsto n-k
		\end{aligned}
	\end{equation}
	est une bijection. Voici maintenant le calcul :
	\begin{subequations}
		\begin{align}
			(PQ)_n & =\sum_{k=0}^nP_kQ_{n-k}                                                    \\
			       & =\sum_{k=0}^nP_{\varphi(k)}Q_{n-\varphi(k)}    \label{SUBEQooISTNooLPvSIy} \\
			       & =\sum_{k=0}^nP_{n-k}Q_{k}                                                  \\
			       & =\sum_{k=0}^nQ_kP_{n-k}      \label{SUBEQooCUMAooFjqqHW}                   \\
			       & =(QP)_n.
		\end{align}
	\end{subequations}
	Justifications
	\begin{itemize}
		\item Pour \eqref{SUBEQooISTNooLPvSIy}. Lemme \ref{DEFooLNEXooYMQjRo} et le fait que \( \varphi\) soit une bijection.
		\item Pour \eqref{SUBEQooCUMAooFjqqHW}. Commutativité de \( A\).
	\end{itemize}
\end{proof}


%--------------------------------------------------------------------------------------------------------------------------- 
\subsection{La notation \texorpdfstring{\(  A[X]\)}{A[X]}}
%---------------------------------------------------------------------------------------------------------------------------
\label{SUBSECooLEKVooFBPSJz}

Si \( A\) est un anneau, nous avons déjà défini les polynômes en une indéterminée sur \( A\) comme étant le module \( A^{(\eN)}\) qui est devenu un anneau par la proposition \ref{PROPooGDQCooHziCPH}.

Le polynôme donné par la suite \( (a_n)_{n\in \eN}\) est souvent notée
\begin{equation}
	\sum_ka_kX^k.
\end{equation}
Par exemple avec \( a=(4,2,8)\) nous avons \( a=8X^2+2X+4\). Nous utiliserons souvent cette notation, qui est très pratique parce qu'elle s'adapte bien aux règles de multiplication et d'addition, en particulier la distributivité.

Il y a (au moins) deux façons de comprendre ce que signifie réellement «\( X\)» dans cette notation.

%///////////////////////////////////////////////////////////////////////////////////////////////////////////////////////////
\subsubsection{Première façon (qui botte en touche)}
%///////////////////////////////////////////////////////////////////////////////////////////////////////////////////////////

La première est de dire qu'il n'a pas de significations, et que \( X^2\) est un simple abus de notations pour écrire \( (0,0,1,0,\cdots)\). Avec cette façon de voir, nous notons l'anneau des polynômes sur \( A\) par «\( A[X]\)» où le \( X\) n'a pas d'autres raisons d'être que d'avertir le lecteur que nous réservons la lettre «\( X\)» pour utiliser la notation pratique des polynômes.

%///////////////////////////////////////////////////////////////////////////////////////////////////////////////////////////
\subsubsection{Seconde façon (la bonne)}
%///////////////////////////////////////////////////////////////////////////////////////////////////////////////////////////
\label{SUBSUBSECooPNBYooWXEHrg}

\begin{normaltext}      \label{NORMooHHIVooSfHlxv}
	La seconde façon de voir le «\( X\)» est de nous rappeler que \( A^{(\eN)}\) a une base en tant que module : les \( e_k\) dont nous avons parlé plus haut. Nous posons \( X=e_1\), et nous prenons la convention \( X^0=1\). Alors nous avons \( e_k=X^k\) et nous notons \( A[X]\)\nomenclature[A]{\( A[X]\)}{tous les polynômes de degré fini à coefficients dans \( A\)} l'anneau \(A^{(\eN)}\) exprimé avec \( X\).

	Dans les deux cas, il n'est pas vraiment légitime d'écrire des égalités comme « \( P(X)=X^2+2X-3\) », et encore moins de dire «Le polynôme \( P\), \emph{évalué} en \( X\) vaut \( X^2+2X-3\)»  : il est plus correct d'écrire « \( P=X^2+2X-3\) ».

	Le lemme suivant montre que ces notations tombent vraiment à point. La véritable difficulté de l'énoncé est de comprendre qu'il n'est pas trivial.

	Nous avons vu dans la définition \ref{DEFooNXKUooLrGeuh} que si \( B\) est un anneau qui étend \( A\), et si \(P\in A[X] \), alors nous avons une définition de \( P(b)\) pour tout \( b\in B\). Nous appliquons cela à \( B=A[X]\), qui est un anneau qui étend \( A\). Autrement dit, si \( P\) et \( Q\) sont des polynômes, ça a un sens d'écrire \( P(Q)\) et le résultat sera un élément de \( A[X]\).
\end{normaltext}

Dans le cas particulier \( Q=X\), nous avons une chouette formule.
\begin{lemma}       \label{LEMooGKWQooVOyeDX}
	Nous avons
	\begin{equation}
		P(X)=P
	\end{equation}
	pour tout \( P\in A[X]\).
\end{lemma}

\begin{proof}
	Si \( P=(a_k)_{k\in \eN}\) alors par définition \( P(\alpha)=\sum_ka_k\alpha^k\) dès que \( \alpha\) est dans un anneau \( B\) qui étend \( A\). Nous considérons le cas particulier \( B=A[X]\) et \( \alpha=X\), c'est-à-dire \( Q=(0,1,0,\ldots)\), l'élément \( P(X)\) de \( A[X]\) vaut
	\begin{equation}        \label{EQooABULooFCEasf}
		\sum_ka_kX^k,
	\end{equation}
	qui est exactement \( P\) lui-même.
\end{proof}

Mais il faut bien comprendre que si \( P\) est le polynôme \( (-3,2,1,0,\ldots)\), noté \( X^2+2X-3\), écrire \( P(X)=X^2+2X-3\) est une pirouette de notations que rien ne justifie par rapport à simplement écrire \( P=X^2+2X-3\).



%--------------------------------------------------------------------------------------------------------------------------- 
\subsection{Action du groupe symétrique}
%---------------------------------------------------------------------------------------------------------------------------

\begin{definition}[Thème~\ref{THEMEooKZHBooRCULcr}]  \label{DefActionGroupe}
	Une \defe{action de groupe}{action}\index{action} \( G\) sur un ensemble \( E\) est la donnée, pour chaque élément \( g \in G\), d'une fonction \(\phi_g : E \to E \), de telle sorte que:
	\begin{gather*}
		\phi_{e}(x) = x, \hspace{2em} \forall x \in E;\\
		\phi_{gh}(x) = \phi_g (\phi_h (x)),  \hspace{2em} \forall g,h \in G, \forall x \in E.
	\end{gather*}
	On dit dans ce cas que \( G \) \defe{agit}{action} sur \( E \).
\end{definition}

Par souci de notations, nous notons \( \Poly_n(A)\) l'anneau des polynômes de \( n\) variables sur \( A\). La propriété universelle de \( \Poly_n(A)=A^{(\eN^n)}\) du théorème \ref{THOooPDZCooJnHbOd} nous donne une application
\begin{equation}
	g\colon \Fun\big(\eN^n,\Poly_n(A)\big)\to \Hom_A\big( \Poly_n(A),\Poly_n(A) \big)
\end{equation}
Avec cela nous pouvons énoncer et démontrer le lemme qui donne l'action de \( S_n\)\footnote{Définition du groupe symétrique \( S_n\) en \ref{DEFooJNPIooMuzIXd}.} sur \( \Poly_n(A)\).

\begin{lemma}[\cite{BIBooFDZDooJQLjlB}]       \label{LEMooIRVQooHvoNBq}
	Pour \( \sigma\in S_n\) nous définissons
	\begin{equation}
		\begin{aligned}
			\phi_{\sigma}\colon \eN^n & \to \Poly_n(A)         \\
			m                         & \mapsto e_{\sigma(m)}.
		\end{aligned}
	\end{equation}
	Alors l'application
	\begin{equation}
		\begin{aligned}
			\rho\colon S_n & \to \Hom_A\big( \Poly_n(A),\Poly_n(A) \big) \\
			\sigma         & \mapsto g(\phi{\sigma})
		\end{aligned}
	\end{equation}
	est une action\footnote{Définition \ref{DefActionGroupe}.}.
\end{lemma}

\begin{proof}
	Nous commençons par donner une expression à notre \( \rho\). Un élément de \( \Poly_n(A)\) est de la forme \( \sum_{m\in \eN^n}a_me_m\), et nous avons\footnote{La somme est définie par \ref{DEFooLNEXooYMQjRo}, et ça va être important. Ah oui, en réalité partout, les sommes sont finies parce que les \( a_m\) (\( m\in \eN^n\)) sont presque tous nuls. Il faudrait écrire sur la somme sur \(\{ m\in \eN^2\tq a_m\neq 0 \}\), mais vous vous imaginez la complication dans la notation.}
	\begin{equation}
		\rho(\sigma)\big( \sum_{m\in \eN^n}a_me_m \big)=\sum_ma_m\phi_{\sigma}(m)=\sum_ma_me_{\sigma(m)}.
	\end{equation}

	Nous avons tout de suite \( \rho(\id)=\id\).

	En ce qui concerne la composition, nous avons d'une part
	\begin{equation}
		\rho(\sigma_1\sigma_2)\big( \sum_ma_me_m \big)=g(\phi_{\sigma_1\sigma_2})\big( \sum_ma_me_m \big)=\sum_ma_me_{\sigma_1\sigma_2(m)},
	\end{equation}
	et d'autre part,
	\begin{subequations}
		\begin{align}
			\rho(\sigma_1)\rho(\sigma_2)\big( \sum_ma_me_m \big) & =\rho(\sigma_1)\big( \sum_ma_me_{\sigma_2(m)} \big)                                    \\
			                                                     & =\rho(\sigma_1)\big( \sum_ma_{\sigma_2^{-1}(m)}e_m \big)   \label{SUBEQooTSCYooCUWiRz} \\
			                                                     & =\sum_ma_{\sigma_2^{-1}(m)}e_{\sigma_1(m)}                                             \\
			                                                     & =\sum_ma_me_{\sigma_1\sigma_2(m)}      \label{SUBEQooQPGPooVvqJdT}
		\end{align}
	\end{subequations}
	La proposition \ref{PROPooJBQVooNqWErk} est utilisée pour \eqref{SUBEQooTSCYooCUWiRz} et pour \eqref{SUBEQooQPGPooVvqJdT}.
\end{proof}


%---------------------------------------------------------------------------------------------------------------------------
\subsection{Corps des fractions}
%---------------------------------------------------------------------------------------------------------------------------

\begin{propositionDef}[\cite{ooGSDHooLgtHCb}]       \label{DEFooGJYXooOiJQvP}
	Soit un anneau commutatif et intègre\footnote{Définition~\ref{DEFooTAOPooWDPYmd}.} \( A\). Nous posons \( E=A\times A\setminus\{ 0 \}\), et nous définissons les deux opérations suivantes sur \( E\) :
	\begin{enumerate}
		\item       \label{ITEMooWBWHooYsXFkO}
		      \( (a,b)+(c,d)=(ad+cb,bd)\);
		\item       \label{ITEMooGOOIooCHqLRl}
		      \( (a,b)(c,d)=(ac,bd)\).
	\end{enumerate}
	Nous notons \( a/b\) la classe de \( (a,b)\). Alors
	\begin{enumerate}
		\item

		      La relation \( (a,b)\sim(c,d)\) si et seulement si \( ad=bc\) est une relation d'équivalence.

		\item
		      Le quotient
		      \begin{equation}
			      \Frac(A)=\big( A\times A\setminus\{ 0 \} \big)/\sim.
		      \end{equation}
		      est un corps.
		\item	\label{ITEMooYUNEooAKZJEh}
		      L'unité dans \( \Frac(A)\) est la classe \( 1/1\).
		\item		\label{ITEMooOGPCooAZGWoF}
		      L'opposé de \( a/b\) est \( (-a)/b\).
		\item		\label{ITEMooECDBooJdJEza}
		      Si \( a\neq 0\), alors l'inverse de \( a/b\) est \( b/a\).
	\end{enumerate}

	Le corps \( \Frac(A)\) est nommé le \defe{corps des fractions}{corps!des fractions} de \( A\).

	Lorsque \( A\) est un anneau de polynômes\footnote{Définition \ref{DEFooFYZRooMikwEL}.}, alors les éléments de \( \Frac(A)\) sont des \defe{fractions rationnelles}{fractions!rationnelles}.
\end{propositionDef}
Le fait que \( A\) soit intègre est important pour être certain que \( bd\neq 0\) sous l'hypothèse que \( b,d\neq 0\).



\begin{proposition}     \label{PROPooUULNooKbwuEw}
	Soit un anneau commutatif \( A\). Nous considérons l'application
	\begin{equation}
		\begin{aligned}
			i\colon A & \to \Frac(A) \\
			a         & \mapsto a/1.
		\end{aligned}
	\end{equation}
	\begin{enumerate}
		\item
		      Elle est injective.
		\item
		      Elle est un morphisme d'anneaux\footnote{Définition \ref{DEFooSPHPooCwjzuz}.}.
	\end{enumerate}
\end{proposition}

\begin{proof}
	En plusieurs parties.
	\begin{subproof}
		\spitem[Injective]
		%-----------------------------------------------------------
		Supposons que \( f(a)=f(b)\), c'est-à-dire que \( a/1=b/1\). En vertu de la relation d'équivalence donnée en \ref{DEFooGJYXooOiJQvP}, nous avons \( a1=b1\), c'est-à-dire \( a=b\).

		\spitem[Morphisme d'anneaux]
		%-----------------------------------------------------------
		D'abord nous avons
		\begin{subequations}
			\begin{align}
				i(a)+i(b) & =(a/1)+(b+1) \\
				          & =(a1+1b)/1   \\
				          & =i(a+b).
			\end{align}
		\end{subequations}
		Ensuite,
		\begin{equation}
			i(a)i(b)=(a/1)(b/1)=ab/1=i(ab).
		\end{equation}
		Et enfin \( i(1)=1/1\) est bien l'unité de \( \Frac(A)\) par \ref{DEFooGJYXooOiJQvP}\ref{ITEMooYUNEooAKZJEh}.
	\end{subproof}
\end{proof}

\begin{normaltext}		\label{NORMooFBKEooBQuATg}
	Si \( a\in A\), nous nous permettrons de noter également \( a\) l'élément \( a/1\) de \( \Frac(A)\). Ce sera toujours clair \href{http://www.madore.org/~david/weblog/d.2006-09-19.1360.html}{dans le contexte}.
\end{normaltext}

\begin{normaltext}		\label{NORMooVYEMooUiUfnE}
	En vertu de la proposition \ref{PROPooUULNooKbwuEw}, à partir de maintenant, nous allons identifier la partie \( i(\eZ)\) à \( \eZ\). Nous nous autorisons donc à dire que \( 4\in \eQ\) ou que \( -7\in \eQ\), et même que \( 0\in \eQ\).
\end{normaltext}

\begin{proposition}[Règle de simplification de fraction\cite{MonCerveau}]	\label{PROPooTXGUooRvViwT}
	Soit un anneau commutatif \( A\). Si \( a,x,y\in A\) avec \( a,y\neq 0\), nous avons
	\begin{enumerate}
		\item
		      \begin{equation}
			      ax/a=x/1.
		      \end{equation}
		      Si vous avez suivi \ref{NORMooFBKEooBQuATg}, cela signifie que \( ax/a=x\) dans \( \Frac(A)\).
		\item
		      \begin{equation}
			      ax/ay=x/y
		      \end{equation}
	\end{enumerate}
	dans \( \Frac(A)\).
\end{proposition}


\begin{lemma}       \label{LEMooBJRCooIZnaid}
	Soient un anneau \( A\) ainsi que \( a,b,x,y\in A\). Si \( b,y\neq 0\), nous avons \( ay=bx\) dans \( A\) si et seulement si nous avons \( ab^{-1}=xy^{-1}\) dans \( \Frac(A)\)\footnote{Rappel de \ref{NORMooVYEMooUiUfnE} : si \( a\in A\), nous nommons \( a\in \Frac(A)\) l'élément \( a/1\), c'est-à-dire la classe de \( (a,1)\)}.
\end{lemma}

\begin{proof}
	Deux parties.
	\begin{subproof}
		\spitem[\( \Rightarrow\)]
		%-----------------------------------------------------------
		Nous supposons avoir \( ay=bx\) dans \( A\). En passant aux classes, cela donne l'égalité \( ay/1=bx/1\) dans \( \Frac(A)\). Nous multiplions les deux membres par \( (b/1)^{-1}(y/1)^{-1}\). En utilisant les différents points de la définition \ref{DEFooGJYXooOiJQvP}, et en particulier \ref{ITEMooECDBooJdJEza}, nous avons\( (b/1)^{-1}(y/1)^{-1}=1/by\). Nous avons donc
		\begin{equation}
			(ay/1)(1/by)=(bx/1)(1/by).
		\end{equation}
		En utilisant les relations de simplification de la proposition \ref{PROPooTXGUooRvViwT}, nous avons \( ay/by=a/b\) à gauche et \( bx/by=x/y\) à droite. Bref, nous avons
		\begin{equation}
			a/b=x/y.
		\end{equation}
		Le membre de gauche peut être récrit \( a/b=(a/1)(b/1)^{-1}\), ce qui peut s'écrire \( ab^{-1}\).
	\end{subproof}

	\spitem[\( \Leftarrow\)]
	%-----------------------------------------------------------
	L'hypothèse s'écrit, sans abus de notations sous la forme
	\begin{equation}
		(a/1)(b/1)^{-1}=(x/1)(y/1)^{-1}.
	\end{equation}
	En multipliant les deux côtés par \( (b/1)(y/1)\) nous trouvons
	\begin{equation}
		ay/1=xb/1.
	\end{equation}
	Nous pouvons écrire cela explicitement avec la relation d'équivalence qui définit \( \Frac(A)\) (définition \ref{DEFooGJYXooOiJQvP}) :
	\begin{equation}
		(ay,1)\sim (xb,1),
	\end{equation}
	c'est-à-dire \( ay1=1xb\), ce que nous voulions.
\end{proof}

La proposition suivante montre encore que le corps des fractions est le plus petit corps que l'on puisse imaginer à partir d'un anneau.
\begin{proposition}[\cite{BIBooZFPUooIiywbk, MonCerveau}]       \label{PROPooIJBEooDjsoHr}
	Soit un anneau commutatif \( A\). Tout corps commutatif contenant un sous-anneau isomorphe\footnote{Morphisme d'anneaux, définition \ref{DEFooSPHPooCwjzuz}.} à \( A\) contient un sous-corps isomorphe à \( \Frac(A)\).
\end{proposition}

\begin{proof}
	Soit un corps \( \eK\) contenant un sous-anneau \( A'\) isomorphe à \( A\). Nous notons \( \sigma\colon A'\to A\) un isomorphisme d'anneaux entre \( A'\) et \( A\).

	\begin{subproof}
		\spitem[Une partie bien choisie]

		Nous considérons la partie suivante de \( \eK\) :
		\begin{equation}
			S=\{ ab^{-1}\tq a,b\in A' \}.
		\end{equation}

		\spitem[\( S\) est un corps]
		Deux éléments arbitraires de \( S\) sont \( ab^{-1}\) et \( xy^{-1}\). Nous devons prouver plusieurs choses.
		\begin{subproof}
			\spitem[Neutres]
			En prenant \( a=b=1\) nous avons \( ab^{-1}=1\in S\). En prenant \( a=0\) et \( b=1\) nous avons \( ab^{-1}=0\in S\).
			\spitem[Somme]
			Il faut remarquer que \( ab^{-1}+xy^{-1}=(ay+xb)(by)^{-1}\). En effet,
			\begin{subequations}
				\begin{align}
					(ay+xb)(by)^{-1} & =(ay+xb)y^{-1}b^{-1}                                           \\
					                 & =ayy^{-1}b^{-1}+xby^{-1}b^{-1}     \label{SUBEQooRGPSooXaBGyx} \\
					                 & =ab^{-1}+xy^{-1}       \label{SUBEQooOHJGooWrfPow}
				\end{align}
			\end{subequations}
			Justifications :
			\begin{itemize}
				\item Pour \eqref{SUBEQooRGPSooXaBGyx}. Distributivité.
				\item Pour \eqref{SUBEQooOHJGooWrfPow}. Commutativité dans \( A\).
			\end{itemize}
			\spitem[Produit]
			Il s'agit du même genre de calculs en utilisant les mêmes propriétés. Nous avons
			\begin{equation}
				(ab^{-1})(xy^{-1})=(ax)(by)^{-1}.
			\end{equation}
		\end{subproof}

		\spitem[Ce qui va être notre isomorphisme]


		Ensuite nous montrons que l'application
		\begin{equation}
			\begin{aligned}
				\varphi\colon S & \to \Frac(A)                \\
				ab^{-1}         & \mapsto \sigma(a)/\sigma(b)
			\end{aligned}
		\end{equation}
		est bien définie et est un isomorphisme de corps.

		\spitem[Bien définie]

		Si \( ab^{-1}=xy^{-1}\) alors \( ay=xb\). Puisque \( \sigma\) est un isomorphisme nous avons aussi \( \sigma(a)\sigma(y)=\sigma(x)\sigma(b)\) et donc \( \sigma(a)/\sigma(b)=\sigma(x)/\sigma(y)\) par définition des classes de \( \Frac(A)\).
		\spitem[Morphisme]
		Deux éléments arbitraires de \( S\) sont \( ab^{-1}\) et \( xy^{-1}\). Calculons un peu :
		\begin{subequations}
			\begin{align}
				\varphi\big( (ab^{-1})(xy^{-1}) \big) & =\varphi(axy^{-1}b^{-1})      \label{SUBEQooRONTooKVTRdZ}                                          \\
				                                      & =\varphi\big( (ax)(by)^{-1} \big)      \label{SUBEQooNOTAooZVJymC}                                 \\
				                                      & =\sigma(ax)/\sigma(by)                                                                             \\
				                                      & =\big(\sigma(a)/\sigma(b)\big)\big(\sigma(x)/\sigma(y)\big)            \label{SUBEQooVQUOooVyVjEU} \\
				                                      & =\varphi(ab^{-1})\varphi(xy^{-1}).
			\end{align}
		\end{subequations}
		Justifications :
		\begin{itemize}
			\item Pour \eqref{SUBEQooRONTooKVTRdZ}. Commutativité dans \( A\).
			\item Pour \eqref{SUBEQooNOTAooZVJymC}. Associativité dans \( A\).
			\item Pour \eqref{SUBEQooVQUOooVyVjEU}. Définition \ref{DEFooGJYXooOiJQvP}\ref{ITEMooGOOIooCHqLRl} de la multiplication de fractions.
		\end{itemize}


		\spitem[Surjectif]

		Tout élément de \( \Frac(A)\) est de la forme \( a'/b'\) avec \( a',b'\in A\), et donc de la forme \( \sigma(a)/\sigma(b)\) avec \( a,b\in A'\). Un tel élément est l'image par \( \varphi\) de \( ab^{-1}\in S\).

		\spitem[Injectif]

		Si \( \varphi(ab^{-1})=\varphi(xy^{-1})\) alors \( \sigma(a)/\sigma(b)=\sigma(x)/\sigma(y)\), et par définition des classes nous avons \( \sigma(a)\sigma(y)=\sigma(b)\sigma(x)\). De là nous avons \( \sigma(ay)=\sigma(bx)\) et donc \( ay=bx\) (parce que \( \sigma\) est un isomorphisme). Nous en déduisons que \( ab^{-1}=xy^{-1}\).
	\end{subproof}
\end{proof}

\begin{normaltext}
	Soit un anneau \( A\) et son anneau des polynômes \( \Poly(A)\). Si \( \alpha\in A\), nous avons la définition \ref{DEFooNXKUooLrGeuh} qui donne l'évaluation \( P(\alpha)\).

	Si par contre \( P\) et \( Q\) sont des polynômes sur \( A\), nous n'avons pas encore défini ce que serait l'évaluation de la fraction rationnelle \( P/Q\) en \( \alpha\). Nous comblons à présent ce manque.
\end{normaltext}

\begin{definition}[Évaluation d'une fraction rationnelle]       \label{DEFooLBIWooCPCaSY}
	Soit un corps \( \eK\) contenant l'anneau \( A\). Si \( R=P/Q\in \Frac(A)\) et si \( \alpha\in \eK\) nous définissons\footnote{Les fractions rationnelles, définition \ref{DEFooGJYXooOiJQvP}.}
	\begin{equation}
		R(\alpha)=(P/Q)(\alpha)=P(\alpha)Q^{-1}(\alpha).
	\end{equation}
	Dans cette formule, les polynômes, l'inverse et le produit sont calculés dans \( \eK\) et non dans \( A\).
\end{definition}

\begin{theoremDef}     \label{ThogbhWgo}
	Soit \( A\) un anneau commutatif intègre.
	\begin{enumerate}
		\item
		      Il existe un couple \( (\eK,\epsilon)\) où \( \eK\) est un corps commutatif et \( \epsilon\colon A\to \eK\) est un morphisme injectif d'anneaux tels que pour tout \( \lambda\in\eK\), il existe \( (a,b)\in A\times A^*\) tels que
		      \begin{equation}
			      \lambda=\epsilon(a)\big( \epsilon(b) \big)^{-1}
		      \end{equation}
		\item
		      Si \( (\eK',\epsilon')\) est un autre couple qui vérifie la propriété, les corps \( \eK\) et \( \eK'\) sont isomorphes.

		      Le corps \( \eK\) associé à l'anneau \( A\) est le \defe{corps des fractions}{corps!des fractions}\index{fractions (corps)} de \( A\), et sera noté \( \Frac(A)\).\nomenclature[A]{\( \Frac(A)\)}{Le corps des fractions de l'anneau \( A\)}

		\item
		      Nous posons
		      \begin{equation}
			      \begin{aligned}
				      \sigma\colon A\times A^* & \to \eK                                          \\
				      (a,b)                    & \mapsto \epsilon(a)\big( \epsilon(b) \big)^{-1}.
			      \end{aligned}
		      \end{equation}
		      Nous avons
		      \begin{equation}
			      \sigma(xa, xb)=\sigma(a,b)
		      \end{equation}
		      pour tout \( a,b,x\in A\).
	\end{enumerate}
\end{theoremDef}


%---------------------------------------------------------------------------------------------------------------------------
\subsection{Corps totalement ordonné}
%---------------------------------------------------------------------------------------------------------------------------

Rappel. Vu qu'un corps est un anneau, nous avons déjà défini la notion d'être totalement ordonné en \ref{DEFooWACWooDWvXKJ} valeur absolue sur un corps en \ref{DEFooJXKVooErANPh}.



\begin{definition}      \label{DefKCGBooLRNdJf}
	Ordre et choses reliées dans un corps.
	\begin{enumerate}

		\item       \label{ItemVXOZooTYpcYN}
		      La suite \( (x_n)\) dans le corps totalement ordonné \( \eK\) est \defe{de Cauchy}{suite!de Cauchy!dans un corps} si pour tout \( \epsilon\in \eK^+\), il existe \( N\in \eN\) tel que si \( p,q\geq N\) alors \( | x_p-x_q |\leq \epsilon\).
		\item       \label{ITEMooDERQooLmJwFR}
		      La suite \( (x_n)\) dans le corps totalement ordonné \( \eK\) est \defe{convergente}{convergence!suite!dans un corps} si il existe \( q\in \eK\) tel que pour tout \( \epsilon\in \eK^+\), il existe \( N\) tel que si \( k\geq N\) alors \( | x_k-q |\leq \epsilon\).

		\item       \label{ITEMooKZZYooDaidGU}
		      Un corps totalement ordonné est \defe{complet}{corps!complet}\index{complet!corps} si toute suite de Cauchy y est convergente.
		\item       \label{ITEMooMWASooEzhVyh}
		      Si \( a,\epsilon\in \eK\) avec \( \epsilon>0\) alors nous définissons la \defe{boule ouverte}{boule dans un corps} de centre \( a\) et de rayon \( \epsilon\) par
		      \begin{equation}
			      B(a,\epsilon)=\{ x\in \eK\tq | a-x |<\epsilon \},
		      \end{equation}
		      et la \defe{boule fermée}{boule dans un corps} par
		      \begin{equation}
			      \overline{ B(a,\epsilon) }=\{ x\in \eK\tq | a-x |\leq \epsilon \}.
		      \end{equation}
	\end{enumerate}
\end{definition}

\begin{lemma}
	Une suite \( (x_k)\) converge vers \( q\) si et seulement si pour tout \( \epsilon>0\), il existe \( N>0\) tel que \( x_k\in B(q,\epsilon)\) pour tout \( k\geq N\).
\end{lemma}

\begin{proof}
	Il s'agit de mettre côte à côte les points~\ref{ITEMooDERQooLmJwFR} et~\ref{ITEMooMWASooEzhVyh} de la définition \ref{DefKCGBooLRNdJf}.
\end{proof}

\begin{normaltext}
	Ces boules prendront une nouvelle force avec le super-théorème~\ref{ThoORdLYUu}.
\end{normaltext}


\begin{proposition}[\cite{MonCerveau}]	\label{PROPooJZIMooHOimlv}
	Soit une suite de Cauchy \( (x_n)\) dans \( \eQ\). Nous supposons que \( (x_n)\) admette une sous-suite convergente vers \( q\in \eQ\). Alors \( x_n\to q\).
\end{proposition}

\begin{proof}
	Soit \( \epsilon>0\). Nous considérons \( N>0\) tel que si \( n,k\geq N\) alors \( | x_k-q |<\epsilon/2\). Soit \( k>N\) tel que \( | x_k-q |<\epsilon/2\). Cela existe; il suffit de prendre \( k\) assez loin dans la sous-suite qui converge vers \( q\). Pour tout \( n\geq k\) nous avons alors
	\begin{equation}
		| x_n-q |\leq | x_n-x_k |+| x_k-q |=\epsilon/2+\epsilon/2=\epsilon.
	\end{equation}
\end{proof}

Parmi ces définitions, celles de suite convergente, de Cauchy et de corps complet seront utilisées dans le cas de \( \eQ\) (et de \( \eR\) pour la complétude). Elles seront prouvées être équivalentes aux définitions topologiques dans le cas particulier de \( \eR\) et \( \eQ\) lorsque la topologie métrique sera définie. Dans cet état d'esprit nous n'allons pas démontrer tout de suite que \( \eR\) est un corps complet. Nous allons directement démontrer que c'est un espace topologique complet.

\begin{lemma}[Règle des signes\cite{ooTKEHooQuaFuD}]        \label{LEMooXJTAooZauchx}
	Soit un corps totalement ordonné \( \eK\) ainsi que \( x,y\in \eK\). Nous avons :
	\begin{enumerate}
		\item
		      Si \( x\leq 0\) et \( y\leq 0\) alors \( xy\geq 0\).
		\item
		      Si \( x\leq 0\) et \( y\geq 0\) alors \( xy\leq 0\).
		\item
		      Si \( x\geq 0\) et \( y\leq 0\) alors \( xy\leq 0\).
		\item       \label{ITEMooRGYAooCUIfss}
		      \( 0\leq 1\).
		\item       \label{ITEMooMRNHooLglPKn}
		      Si \( x\geq 0\) alors \( x^{-1}\geq 0\).
	\end{enumerate}
	%TODOooFJVDooAjbzME. Prouver ça.
\end{lemma}

\begin{lemma}[Propriétés de la valeur absolue]  \label{LemooANTJooYxQZDw}
	Soit \( \eK\) un corps totalement ordonné. Si \( x,y\in \eK\) alors\footnote{La «valeur absolue» est définie en \ref{DEFooJXKVooErANPh}.}
	\begin{enumerate}
		\item       \label{ItemooNVDIooSuiSoB}
		      Si \( x\geq 0\) alors \( -x\leq 0\).
		\item       \label{ITEMooVNAZooSxmtuH}
		      Si \( x\leq 0\) alors \( -x\geq 0\).
		\item       \label{ITEMooSDNHooDnjScE}
		      \( | x |\geq 0\)
		\item       \label{ITEMooLQLTooTJTPVM}
		      \( | x |=0\) si et seulement si \( x=0\)
		\item       \label{ITEMooVJAEooOEatzY}
		      \( | -x |=| x |\).
		      \item\label{ItemooOMKNooRlanvk}
		      \( | x+y |\leq | x |+| y |\).
		      \item\label{ITEMooEFMLooYVCuHD}
		      \( | xy |=| x | |y |\)
	\end{enumerate}
\end{lemma}

\begin{proof}
	Point par point
	\begin{subproof}
		\spitem[\ref{ItemooNVDIooSuiSoB}]
		Nous partons de \( x\geq 0\) et nous ajoutons \( -x\) des deux côtés en profitant de la définition d'un corps totalement ordonné : \( x-x\geq -x\) et donc \( 0\geq-x\), c'est-à-dire \( -x\leq 0\).
		\spitem[\ref{ITEMooVNAZooSxmtuH}]
		Nous partons de \( x\leq 0\) et nous ajoutons \( -x\) des deux côtés.
		\spitem[\ref{ITEMooSDNHooDnjScE}]
		Si \( x\geq 0\) alors c'est vrai. Sinon, \( x\leq 0\) et \( | x |=-x\geq 0\) par le point~\ref{ItemooNVDIooSuiSoB}.
		\spitem[\ref{ITEMooLQLTooTJTPVM}]
		Si \( x=0\) alors \( x=-x\) et \( | x |=0\). Au contraire si \(x\neq 0\) alors \( -x\neq 0\) et que \( x\) soit positif ou négatif, nous aurons toujours \( \pm x\neq 0\).
		\spitem[\ref{ITEMooVJAEooOEatzY}]
		Il faut décomposer en deux cas selon que \( x\geq 0\) et \( x\leq 0\). Supposons \( x\geq 0\). Alors d'une part \( | x |=x\). D'autre part \( -x\leq 0\) par le point \ref{ItemooNVDIooSuiSoB}, de telle sorte que
		\begin{equation}
			| -x |=-(-x)=x.
		\end{equation}
		Nous avons donc \( | x |=| -x |=x\).

		Le même raisonnement tient avec \( x\leq 0\).
		\spitem[\ref{ItemooOMKNooRlanvk}]
		Nous supposons que \( x\leq y\) et nous distinguons divers cas suivant la positivité de \( x\) et \( y\).
		\begin{enumerate}
			\item
			      Si \( x,y\geq 0\). Dans ce cas, \( x+y\geq y\geq 0\), donc \( | x+y |=x+y=| x |+| y |\).
			\item
			      Si \( x,y\leq 0\). Dans ce cas, \( x+y\leq 0\) et nous avons \( | x+y |=-x-y=| x |+| y |\).
			\item
			      Si \( x\leq 0\) et \( y\geq 0\). Nous subdivisons encore en deux cas suivant que \( x+y\) est positif ou négatif. Si \( x+y\geq 0\), alors nous écrivons successivement
			      \begin{subequations}
				      \begin{align}
					      x   & \leq 0                         \\
					      x+y & \leq y\leq y+| x |=| x |+| y |
				      \end{align}
			      \end{subequations}
			      et donc \( | x+y |=x+y\leq | x |+| y |\).

			      Nous supposons à présent que \( x\leq 0\), \( y\geq 0\) et \( x+y\leq 0\). Dans ce cas il suffit d'écrire \( | x+y |=| (-x)+(-y) |\) pour retomber dans le cas précédent à inversion près de \( x\) et \( y\).
		\end{enumerate}
		\spitem[Pour \ref{ITEMooEFMLooYVCuHD}]
		% -------------------------------------------------------------------------------------------- 
		Il suffit de prendre les 4 cas suivant les signes de \( x\) et \( y\), et d'utiliser les règles de signes du lemme \ref{LEMooXJTAooZauchx} dans la définition \ref{EQooNONAooHLSERO}.
	\end{subproof}
\end{proof}

\begin{remark}      \label{RemooJCAUooKkuglX}
	La partie~\ref{ItemooOMKNooRlanvk} est très importante parce que c'est elle qui fera presque toutes les majorations dont nous aurons besoin en analyse. En effet elle donne l'inégalité triangulaire de la façon suivante : si \( x,y,z\in \eK\) nous avons
	\begin{equation}
		| x-y |= |  (x-z)+(z-y) |\leq | x-z |+| z-y |.
	\end{equation}
\end{remark}

\begin{lemma}[À propos de boules]       \label{LEMooQXDCooPEABBm}
	Soient un corps totalement ordonné \( \eK\) et des éléments \( x,y\in \eK\). Soit aussi \( \epsilon>0\) dans \( \eK\). Nous avons :
	\begin{enumerate}
		\item       \label{ITEMooXJGVooSebiip}
		      \( y\in B(x,\epsilon)\) si et seulement si \( x-\epsilon<y<x+\epsilon\).
		\item       \label{ITEMooRUBBooRayiMs}
		      Si \( y\in  \overline{ B(x,\epsilon) }  \) alors \( y\in B(x,\epsilon')\) pour tout \( \epsilon'>\epsilon\).
	\end{enumerate}
\end{lemma}

\begin{proof}
	Pour rappel,
	\begin{equation}
		| x-y |=\begin{cases}
			x-y & \text{si } x-y\geq 0  \\
			y-x & \text{si } x-y\leq 0.
		\end{cases}
	\end{equation}
	Nous pouvons maintenant démontrer nos assertions.
	\begin{subproof}
		\spitem[\ref{ITEMooXJGVooSebiip}]
		En deux parties.
		\begin{subproof}
			\spitem[\( \Rightarrow\)]
			Nous supposons que \( | x-y |<\epsilon\).

			Si \( x-y\geq 0\) alors l'hypothèse signifie \( x-y<\epsilon\), ce qui donne \( y>x-\epsilon\). Mais l'inégalité \( x-y\geq 0\) donne également \( x\geq y\) et donc \( x+\epsilon\geq y+\epsilon>y\). Notez le jeu de l'inégalité non stricte qui se change en inégalité stricte.

			Si \( x-y\leq 0\) nous pouvons faire le même raisonnement.

			\spitem[\( \Leftarrow\)]
			Des inégalités \( x-\epsilon<y\) et \( y<x+\epsilon\) nous tirons \( x-y<\epsilon\) et \( y-x<\epsilon\). Donc quel que soit le signe de \( x-y\) nous avons toujours \( | x-y |<\epsilon\).
		\end{subproof}

		\spitem[\ref{ITEMooRUBBooRayiMs}]

		C'est immédiat parce que
		\begin{equation}
			| x-y |\leq \epsilon<\epsilon'.
		\end{equation}
	\end{subproof}
\end{proof}


\begin{lemma}       \label{LEMooVZNCooRJatKK}
	Tout corps totalement ordonné est de caractéristique nulle.
\end{lemma}

%+++++++++++++++++++++++++++++++++++++++++++++++++++++++++++++++++++++++++++++++++++++++++++++++++++++++++++++++++++++++++++
\section{Les rationnels}
%+++++++++++++++++++++++++++++++++++++++++++++++++++++++++++++++++++++++++++++++++++++++++++++++++++++++++++++++++++++++++++

Note : pour l'existence et l'unicité de l'écriture \( q=k/d\), il faut voir le théorème \ref{THOooWYQVooRBaAAM}.


Une construction très explicite est faite dans \cite{RWWJooJdjxEK}. Ici nous allons prendre plus court :
\begin{definition}
	Le corps des fractions de \( \eZ\) (définition~\ref{DEFooGJYXooOiJQvP}) est noté \( \eQ\) et ses éléments sont les \defe{rationnels}{rationnels}.
\end{definition}


La proposition suivante provient de \ref{PROPooUULNooKbwuEw}.
%TODOooNLYLooSETVJC il faut quand même rédiger une preuve.
\begin{proposition}[\cite{MonCerveau}]	\label{PROPooKUIYooCxMHTX}
	L'application
	\begin{equation}
		\begin{aligned}
			i\colon \eZ & \to \eQ     \\
			a           & \mapsto a/1
		\end{aligned}
	\end{equation}
	est un morphisme injectif.
\end{proposition}




\begin{proposition}     \label{PROPooDHIAooZysvNs}
	L'ensemble des rationnels est infini dénombrable\footnote{Ouais, je sais, dans les définitions prises ici, un ensemble dénombrable est toujours infini. Mais l'excès de précision ne tue pas, loin s'en faut.}.
\end{proposition}

\begin{proof}
	L'ensemble \( \eZ\) est infini\footnote{Lemme \ref{LEMooJNXIooBmdOVi}.} et la proposition \ref{PROPooUULNooKbwuEw} donne une injection \( f\colon \eZ\to \eQ\). Donc \( f(\eZ)\) est infini.

	L'ensemble \( \eQ\) contient une partie infinie. Il est donc infini par le lemme \ref{LEMooTUIRooEXjfDY}.

	L'application
	\begin{equation}
		\begin{aligned}
			g\colon \eZ^2 & \to \eQ     \\
			a,b           & \mapsto a/b
		\end{aligned}
	\end{equation}
	est surjective alors que \( \eZ^2\) est dénombrable. Le lemme \ref{LEMooDTAEooIBdHyo} dit alors que \( \eQ\) est fini ou dénombrable. Vu que nous avons déjà prouvé que \( \eQ\) était infini, nous déduisons que \( \eQ\) est infini dénombrable.
\end{proof}


%--------------------------------------------------------------------------------------------------------------------------- 
\subsection{Relation d'ordre}
%---------------------------------------------------------------------------------------------------------------------------

\begin{proposition}[\cite{MonCerveau}]	\label{PROPooEBCHooYEQkIO}
	Soient \( a,b\in \eZ\) tels que \( ab\geq 0\)\footnote{Définition \ref{PROPooMYYDooOABOdB} pour l'ordre dans \( \eZ\).}. Alors si \( (x,y)\sim(a,b)\), nous avons \( xy\geq 0\).
\end{proposition}

\begin{proof}
	L'hypothèse \( (x,y)\sim (a,b)\) dit que \( ab=ya\). L'astuce est de calculer \( (xy)(bb)\) en nous servant de \( xb=ya\) :
	\begin{equation}
		(xy)b^2=yayb=aby^2.
	\end{equation}
	Vu que \( ab\geq 0\) et \( y^2\geq 0\), le membre de droite est positif (proposition \ref{PROPooYFUBooJUZgwH}). Nous avons donc aussi \( xyb^2\geq 0\). La proposition \ref{PROPooYFUBooJUZgwH}\ref{ITEMooBUHIooLJOSLQ} dit encore que dans ce cas \( xy\geq 0\).
\end{proof}


\begin{proposition}[\cite{MonCerveau}]	\label{PROPooEOZWooLxRPXa}
	La partie
	\begin{equation}
		\eQ^+=\{ a/b\tq ab>0 \}\subset \eQ
	\end{equation}
	est bien définie et est une positivité\footnote{Définition \ref{DEFooZRMFooCtzMov}.}
\end{proposition}

\begin{proof}
	Le fait que ce soit bien défini est la proposition \ref{PROPooEBCHooYEQkIO}.
	\begin{enumerate}
		\item
		      Supposons pour le plaisir de la contradiction que \( a/b\in \eQ^+\) et \( -(a/b)\in \eQ^+\). En utilisant la propriété \ref{DEFooGJYXooOiJQvP}\ref{ITEMooOGPCooAZGWoF} nous avons \( ab>0\) et \( (-a)b>0\). En utilisant \ref{PROPooJMETooHQGwnv}\ref{ITEMooWNYGooMopPcq} nous avons
		      \begin{subequations}
			      \begin{numcases}{}
				      (ab)>0\\
				      -(ab)>0.
			      \end{numcases}
		      \end{subequations}
		      En utilisant \ref{PROPooYFUBooJUZgwH}\ref{ITEMooSAJPooEDSgXJ} nous trouvons \( ab(-ab)>0\). En utilisant encore une des nombreuses propriétés déjà citées, nous trouvons \( -a^2b^2>0\), c'est à dire
		      \begin{equation}
			      a^2b^2<0,
		      \end{equation}
		      ce qui est faux parce que \( a^2b^2\geq 0\).

		\item
		      Supposons que \( a/b\in \eQ^+\) et \( x/y\in \eQ^+\) (\( b\) et \( y\) sont non nuls). Cette hypothèse signifie que \( ab>0\) et \( xy>0\). Nous devons prouver que \( (a/b)+(x/y)\in \eQ^+\). Nous avons
		      \begin{equation}
			      \frac{ a }{ b }+\frac{ x }{ y }=\frac{ ay+bx }{ by }.
		      \end{equation}
		      Nous avons un petit calcul à faire :
		      \begin{equation}		\label{EQooLGUJooYAUXPJ}
			      by(ay+bx)=aby^2+b^2xy.
		      \end{equation}
		      Par hypothèse, \( ab>0\). Nous savons aussi que \( y^2\geq 0\). Mais dans notre cas nous avons \( y^2\neq 0\). Donc \( aby^2>0\). Même raisonnement pour \( b^2xy\). Bref ce que nous avons dans \eqref{EQooLGUJooYAUXPJ} est strictement positif.
	\end{enumerate}
\end{proof}


\begin{propositionDef}[Ordre sur \( \eQ\)\cite{RWWJooJdjxEK}]      \label{DEFooZEXXooUtOhqB}
	Nous considérons l'ensemble
	\begin{equation}
		\eQ^+=\{ a/b\tq ab>0 \}\subset \eQ.
	\end{equation}

	\begin{enumerate}
		\item		\label{ITEMooUSILooIfkVsR}
		      \( \eQ^+\) est une positivité sur \( \eQ\).
		\item		\label{ITEMooSDSVooEOTRLY}
		      Pour \( \alpha,\beta\in \eQ\) nous disons \( \alpha\leq \beta\) si et seulement si
		      \begin{equation}
			      \beta-\alpha\in \eQ^+\cup\{ 0 \}.
		      \end{equation}
		      Cela définit un ordre total\footnote{Définition \ref{DEFooVGYQooUhUZGr}.}.
	\end{enumerate}
\end{propositionDef}

\begin{proof}
	Plusieurs points.
	\begin{subproof}
		\spitem[Pour \ref{ITEMooUSILooIfkVsR}]
		%-----------------------------------------------------------
		C'est la proposition \ref{PROPooEOZWooLxRPXa}.

		\spitem[Pour \ref{ITEMooSDSVooEOTRLY}]
		%-----------------------------------------------------------
		C'est la proposition \ref{PROPooKLOPooBgQqhM}.
	\end{subproof}
\end{proof}

\begin{proposition}[\cite{MonCerveau}]	\label{PROPooNSICooAQczDh}
	Soient \( q_1,q_2,r_1,r_2\in \eQ\) tels que
	\begin{subequations}
		\begin{numcases}{}
			q_1\geq r_1\\
			q_2\geq r_2.
		\end{numcases}
	\end{subequations}
	Alors \( q_1+q_2\geq r_1+r_2\).
	%TODOooKKMCooBXwQDX. Prouver ça.
\end{proposition}

\begin{proposition}[\cite{MonCerveau}]	\label{PROPooCILQooOuyrfI}
	Soit \( q\in \eQ\).
	\begin{enumerate}
		\item
		      Nous avons \( q^2\geq 0\).
		\item
		      Nous avons \( q^2=0\) si et seulement si \( q=0\).
	\end{enumerate}
	%TODOooYEPXooUXPDzH. Prouver ça.
\end{proposition}



\begin{proposition}[\cite{MonCerveau}]	\label{PROPooXNBVooIvBmkW}
	L'application
	\begin{equation}
		\begin{aligned}
			i\colon \eZ & \to \eQ     \\
			a           & \mapsto a/1
		\end{aligned}
	\end{equation}
	est un morphisme injectif qui respecte l'ordre.
\end{proposition}

\begin{proof}
	Le fait que ce soit un morphisme injectif est la proposition \ref{PROPooKUIYooCxMHTX}.

	Le respect de l'ordre doit encore être fait\quext{Rédigez une preuve, faites un scan de votre feuille et envoyez-moi.}.
	%TODOooSPPDooIEkJvJ le faire.
\end{proof}

\begin{normaltext}
	Le lemme suivant \ref{LEMooEBTIooGMoHsj} dit que tout rationnel est majoré par un naturel. Vu que les naturels et les rationnels ne sont pas du tout les mêmes ensembles (les entiers sont des classes de naturels et les rationnels des classes d'entiers), l'énoncé n'a à priori pas de sens.

	La réalité est que nous identifions \( \eN\) à une partie de \( \eQ\) via les propositions \ref{PROPooCHGRooRksGGO} et \ref{PROPooXNBVooIvBmkW}. L'énoncé exact serait que si \( q\in \eQ\), il existe \( n\in \eN\) tel que \( i\big( \iota(n) \big)> q\).
\end{normaltext}

\begin{proposition}[\cite{MonCerveau}]	\label{PROPooTEREooZgblWu}
	Si \( q>0\) dans \( \eQ\), alors il existe \( a,b\in \eN\) tels que \( q=a/b\).
\end{proposition}

\begin{proof}
	Posons \( q=a/b\) avec \( a,b\in \eZ\). Si \( a,b>0\), c'est bon. Si \( a=0\) alors \( q=0=0/1\) c'est bon aussi. Supposons que ce ne soit pas le cas. L'hypothèse \( q>0\) dit que \( q=a/b\) avec \( ab>0\). La proposition \ref{PROPooYFUBooJUZgwH}\ref{ITEMooGDFPooBbpBgn} dit qu'il n'y a que deux possibilités : soit \( a,b>0\), soit \( a,b<0\).

	Supposons que \( a,b<0\). En vertu des classes d'équivalence, nous avons \( a=a/b=(-a)/(-b)\). Vu que nous avons \( -a>0\) et \( -b>0\), c'est bon.
\end{proof}

\begin{proposition}[\cite{MonCerveau}]	\label{PROPooPSGBooLuMXoI}
	Soit \( a>0\) et \( b>1\). Alors \( a>a/b\).
\end{proposition}

\begin{proof}
	Pour prouver que \( a>a/b\), nous devons calculer
	\begin{equation}
		a-\frac{ a }{ b }=\frac{ a }{ 1 }-\frac{ a }{ b }=\frac{ ab-a }{ b }.
	\end{equation}
	Nous devons donc évaluer la quantité \( b(ab-a)\). Vu que \( b>1\), le lemme \ref{LEMooSVDDooWsyxNP} dit que \( ab>a\). Donc \( ab-a>0\). Le produit \( b(ab-a)\) est donc le produit de deux entiers strictement positifs. En vertu de la proposition \ref{PROPooYFUBooJUZgwH}\ref{ITEMooSAJPooEDSgXJ} nous avons donc \( b(ab-a)>0\), et donc \( a>a/b\).
\end{proof}

\begin{lemma} \label{LEMooEBTIooGMoHsj}
	Tout rationnel est strictement majoré par un naturel.
\end{lemma}

\begin{proof}
	Si \( q<0\), alors \( n=0\) fait l'affaire. Si \( q=0\), alors \( n=1\) fonctionne. Supposons donc que \( q=a/b>0\) avec \( ab>0\) ainsi que \( a,b\in \eN\)\footnote{Existe pas la proposition \ref{PROPooTEREooZgblWu}.}. Si \( b=1\), alors \( n=a+1\) fonctionne.

	Si \( b>1\), alors la proposition \ref{PROPooPSGBooLuMXoI} dit que \( a>a/b\), et donc \( n=a\) fonctionne.
\end{proof}

\begin{proposition}     \label{PROPooBTCCooVVvaeL}
	Si \( q<1\) dans \( \eQ\), alors \( ql<l\) pour tout \( l\in \eQ^+\).
\end{proposition}

\begin{proof}
	Si \( q<0\) c'est facile. Supposons que \( 0<q<1\). Vu que \( q-1<0\) et \( l>0\), nous avons
	\begin{equation}
		ql-l=(q-1)l<0.
	\end{equation}
	Mais dire \( ql-l<0\) signifie \( ql<l\).
\end{proof}

\begin{lemma}		\label{LEMooMUYAooDLgDcf}
	Si \( q\in \eQ\), alors il existe \( k\in \eN\) tel que \( kq\in \eZ\).
	%TODOooAGNJooSafbmG. Prouver ça. Et j'en mets plusieurs parce que je suis presque sûr que tout sera fait que j'arriverai ici.
	\index{archimédien!\( \eQ\)}
\end{lemma}



\begin{proposition}     \label{PROPooMXGPooDUkOuv}
	Le corps \( \eQ\) est archimédien\footnote{Définition~\ref{DEFooLCWLooYrToFv}.}.
\end{proposition}

\begin{proof}
	Soit \( x,y\in\eQ\) avec \( x>0\). Si \( y\leq 0\), nous avons directement \( x\geq y\). Sinon, il existe \( m\in \eN\) tel que \( mx\in \eN\) (lemme \ref{LEMooMUYAooDLgDcf}). Vu que \( \eN\) est archimédien (proposition \ref{PROPooCCVNooYUYcqG}), il existe \( n\in \eN\) tel que \( nmx>y\).
\end{proof}


%--------------------------------------------------------------------------------------------------------------------------- 
\subsection{Caractéristique}
%---------------------------------------------------------------------------------------------------------------------------

\begin{lemma}       \label{LEMooYCPUooNxEPhB}
	Le corps \( \eQ\) est de caractéristique\footnote{Définition \ref{LEMDEFooVEWZooUrPaDw}.} nulle.
	%TODOooAGNJooSafbmG. Prouver ça. Et j'en mets plusieurs parce que je suis presque sûr que tout sera fait que j'arriverai ici.
\end{lemma}

%+++++++++++++++++++++++++++++++++++++++++++++++++++++++++++++++++++++++++++++++++++++++++++++++++++++++++++++++++++++++++++ 
\section{Suite de Cauchy dans un corps totalement ordonné}
%+++++++++++++++++++++++++++++++++++++++++++++++++++++++++++++++++++++++++++++++++++++++++++++++++++++++++++++++++++++++++++


\begin{lemma}[\cite{ooIBWOooSjOvXd, MonCerveau}]        \label{LEMooLTBIooSZnvsQ}
	Tout corps commutatif de caractéristique nulle contient un sous-corps isomorphe à \( \eQ\).
\end{lemma}

\begin{proof}
	Soit un corps \( \eK\) de caractéristique nulle. Nous savons du lemme \ref{LEMDEFooVEWZooUrPaDw} que
	\begin{equation}
		\begin{aligned}
			\mu\colon \eZ & \to \eK          \\
			n             & \mapsto n1_{\eK}
		\end{aligned}
	\end{equation}
	est un morphisme d'anneaux vérifiant \( \ker(\mu)=\{ 0 \}\). Nous posons \( Z=\mu(\eZ)\). L'application \( \mu\colon \eZ\to Z\) est un isomorphisme d'anneaux. Prouvons cela :
	\begin{subproof}
		\spitem[Morphisme]
		L'application \( \mu\) est un morphisme par le lemme \ref{LEMDEFooVEWZooUrPaDw}.
		\spitem[Surjectif]
		Par définition les éléments de \( Z\) sont dans l'image de \( \eZ\).
		\spitem[Injectif] Si \( x,y\in \eZ\) vérifient \( \mu(x)=\mu(y)\), alors \( \mu(x-y)=0\) parce que \( \mu\) est un morphisme. Mais \( \eK\) est de caractéristique nulle, c'est-à-dire \( \ker(\mu)=\{ 0 \}\). Donc \( x-y=0\).
	\end{subproof}
	Le corps \( \eK\) contient donc un sous-anneau isomorphe à \( \eZ\). Puisque \( \eZ\) et \( \eK\) sont commutatifs, la proposition \ref{PROPooIJBEooDjsoHr} s'applique et \( \eK\) contient un sous-corps isomorphe à \( \Frac(\eZ)=\eQ\).
\end{proof}

La proposition suivante donne des précisions à propos du lemme \ref{LEMooLTBIooSZnvsQ}.

\begin{proposition}[\cite{MonCerveau}]      \label{PROPooKNROooFdgIeQ}
	Soit un corps totalement ordonné \( \eK\). Nous considérons l'application
	\begin{equation}
		\begin{aligned}
			\mu\colon \eZ & \to \eK                \\
			n             & \mapsto n\cdot 1_{\eK}
		\end{aligned}
	\end{equation}
	et ensuite
	\begin{equation}
		\begin{aligned}
			\sigma\colon \eQ & \to \eK                    \\
			a/b              & \mapsto \mu(a)\mu(b)^{-1}.
		\end{aligned}
	\end{equation}
	Alors
	\begin{enumerate}
		\item
		      L'application \( \sigma\) est bien définie.
		\item
		      L'application \( \sigma\) est un morphisme de corps.
		\item
		      Si \( q\leq q'\) dans \( \eQ\), alors \( \sigma(q)\leq \sigma(q')\).
	\end{enumerate}
\end{proposition}

\begin{proof}
	En plusieurs morceaux.
	\begin{subproof}
		\spitem[\( \sigma\) est bien définie]
		Montrons que \( \sigma\) est bien définie. Pour cela nous considérons \( a,b,x,y\in \eZ\) tels que \( a/b=x/y\) dans \( \eQ\). Par définition des classes (définition \ref{DEFooGJYXooOiJQvP} du corps des fractions), nous avons \( ay=bx\) dans \( \eQ\). Vu que \( \mu\) est un morphisme nous avons alors
		\begin{equation}
			\mu(a)\mu(y)=\mu(b)\mu(x)
		\end{equation}
		et donc \( \mu(a)\mu(b)^{-1}=\mu(x)\mu(y)^{-1}\), c'est-à-dire \( \sigma(a/b)=\sigma(x/y)\). L'application \( \sigma\) est donc bien définie.

		\spitem[Morphisme pour la somme]

		L'application \( \mu\) est un morphisme d'anneaux, comme déjà dit depuis le lemme \ref{LEMDEFooVEWZooUrPaDw}. Notons aussi que, parce que \( \eK\) est commutatif,
		\begin{equation}
			\mu(qy)^{-1}=\mu(q)^{-1}\mu(y)^{-1}.
		\end{equation}

		En utilisant la définition \ref{DEFooGJYXooOiJQvP}\ref{ITEMooWBWHooYsXFkO} de la somme nous avons
		\begin{subequations}
			\begin{align}
				\sigma(p/q+x/y) & =\sigma\big( (py+qx)/qy \big)            \\
				                & =\big[ \mu(py)+\mu(qx) \big]\mu(qy)^{-1} \\
				                & =\mu(py)\mu(qy)^{-1}+\mu(qx)\mu(qy)^{-1} \\
				                & =\mu(p)\mu(q)^{-1}+\mu(x)\mu(y)^{-1}     \\
				                & =\sigma(p/q)+\sigma(x/y).
			\end{align}
		\end{subequations}

		\spitem[Morphisme pour le produit]
		Même genre de calculs que pour la somme.
		\spitem[Croissante]

		Nous savons aussi par le lemme \ref{LEMooXJTAooZauchx}\ref{ITEMooRGYAooCUIfss} que \( 1\geq 0\). Puisque \( \mu\) est un morphisme d'anneaux,
		\begin{equation}
			\mu(n+1)=\mu(n)+\mu(1)=\mu(n)+1
		\end{equation}

		La définition \ref{DefKCGBooLRNdJf}\ref{ITEMooZISJooWNxnBj} dit alors que \( \mu(n)\geq 0\) pour tout \( n\in \eN\). Nous avons pour la même raison que si \( m\geq n\) dans \( \eN\), alors \( \mu(m)\geq\mu(n)\) dans \( \eK\).

		Soient maintenant \( p,q\in \eN\), et prouvons que \( \sigma(p/q)\geq 0\). D'abord
		\begin{equation}
			\sigma(p/q)=\mu(p)\mu(q)^{-1}
		\end{equation}
		où \( \mu(p)\geq 0\) et \( \mu(q)\geq 0\). Ensuite le lemme \ref{LEMooXJTAooZauchx}\ref{ITEMooMRNHooLglPKn} nous indique que \( \mu(q)^{-1}\geq 0\). Enfin la condition \ref{DefKCGBooLRNdJf}\ref{CONDooBYYDooElXgPO} nous permet de conclure que \( \sigma(p/q)\geq 0\).

		Finalement, si \( q_1\geq q_2\) dans \( \eQ\), alors \( q_1-q_2\geq 0\), et nous avons
		\begin{equation}
			\sigma(q_1)=\sigma(q_2+q_1-q_2)=\sigma(q_2)+\sigma(q_1-q_2)\geq \sigma(q_2)
		\end{equation}
		par la condition \ref{DefKCGBooLRNdJf}\ref{ITEMooZISJooWNxnBj}.
	\end{subproof}
\end{proof}


\begin{normaltext}      \label{NORMooJRRZooTwTVYG}
	Si \( \eK\) est un corps totalement ordonné, la proposition \ref{PROPooKNROooFdgIeQ} nous donne un morphisme de corps \( \sigma\colon \eQ\to \eK\) qui respecte l'ordre. Pour \( q\in \eQ\) et \( k\in \eK\) nous notons
	\begin{equation}        \label{EQooERFIooMpZVEs}
		qk=\sigma(q)k.
	\end{equation}
	Nous pourrons donc écrire \( \frac{ k }{2}\) pour \( \sigma(1/2)k\).
\end{normaltext}

Le lemme suivant explique que la notation \eqref{EQooERFIooMpZVEs} n'est pas complètement idiote.
\begin{lemma}       \label{LEMooWIONooGTKfcJ}
	Soit un corps commutatif totalement ordonné \( \eK\). Soit \( k\in \eK\). Nous avons
	\begin{equation}
		k+k=2k.
	\end{equation}
\end{lemma}

\begin{proof}
	Vu que \(  \sigma\colon \eQ\to \eK \) est un morphisme, il vérifie \( \sigma(1)=1\), donc
	\begin{subequations}
		\begin{align}
			k+k & =\sigma(1)k+\sigma(1)k            \\
			    & =\big( \sigma(1)+\sigma(1) \big)k \\
			    & =\sigma(2)k                       \\
			    & =2k.
		\end{align}
	\end{subequations}
\end{proof}


\begin{proposition}     \label{PROPooTFVOooFoSHPg}
	Toute suite convergente dans un corps totalement ordonné est de Cauchy\footnote{Définition \ref{DefKCGBooLRNdJf}\ref{ItemVXOZooTYpcYN}.}.
\end{proposition}

\begin{proof}
	Soit un corps totalement ordonné \( \eK\) et une suite \( x_n\stackrel{\eK}{\longrightarrow}x\). Soit \( \epsilon>0\). Il est important de se rendre compte que \( \epsilon\in \eK\) et que l'inégalité est au sens de l'ordre dans \( \eK\); en particulier ce n'est pas \( \epsilon\in \eR\) ni \( \epsilon\in \eQ\). D'ailleurs nous n'avons pas encore défini \( \eR\).

	Vu que \( (x_n)\) converge vers \( x\), il existe \( N\in \eN\) tel que pour tout \( k>N\),
	\begin{equation}
		| x_k-x |<\epsilon.
	\end{equation}
	Soient \( p,q>N\). Alors en utilisant la majoration du lemme~\ref{LemooANTJooYxQZDw}\ref{ItemooOMKNooRlanvk},
	\begin{equation}        \label{EQooMQYGooLpgEQO}
		| x_p-x_q |=\big| (x_p-x)+(x-x_q) \big|\leq | x_p-x |+| x-x_q |\leq 2\epsilon.
	\end{equation}
	En analyse en général, on s'arrête là et on dit que \( (x_n)\) est de Cauchy parce qu'il n'y a pas vraiment de différence entre réaliser une majoration avec \( \epsilon\) ou avec \( 2\epsilon\). Détaillons toutefois comment ça se passe dans le cas où \( \epsilon\) est un élément d'un corps totalement ordonné.

	Le \( 2\epsilon\) arrivant à la fin de \eqref{EQooMQYGooLpgEQO} est en réalité \( \epsilon+\epsilon=\sigma(2)\epsilon\) en vertu de ce qui est raconté en \ref{NORMooJRRZooTwTVYG} et en vertu du lemme \ref{LEMooWIONooGTKfcJ}.

	Considérons \( \epsilon'=\sigma(1/2)\epsilon\), que nous pouvons noter \( \epsilon'=\epsilon/2\). Vu que \( \epsilon'>0\), il existe un \( N'\) tel que pour tout \( p,q>N'\) nous ayons
	\begin{equation}
		| x_p-x_q |\leq 2\epsilon'=\sigma(2)\sigma(1/2)\epsilon=\sigma(1)\epsilon=\epsilon.
	\end{equation}
	Ce dernier \( \epsilon\) étant bien celui fixé au début de la preuve, nous en déduisons que \( (x_n)\) est de Cauchy.
\end{proof}

%---------------------------------------------------------------------------------------------------------------------------
\subsection{Suites de Cauchy dans les rationnels}
%---------------------------------------------------------------------------------------------------------------------------

\begin{proposition}[\cite{RWWJooJdjxEK}]        \label{PropFFDJooAapQlP}
	Principales propriétés des suites de Cauchy\footnote{Définition \ref{DefKCGBooLRNdJf}\ref{ItemVXOZooTYpcYN}.} dans \( \eQ\).
	\begin{enumerate}
		\item       \label{ItemRKCIooJguHdji}
		      Toute suite convergente est de Cauchy\footnote{Et non la réciproque, qui sera justement la grande innovation des nombres réels.}.
		\item       \label{ItemRKCIooJguHdjii}
		      Toute suite de Cauchy est bornée.
		\item       \label{ItemRKCIooJguHdjiii}
		      Si \( x_n\to 0\) et si \( (y_n)\) est bornée, alors \( x_ny_n\to 0\)
		\item
		      Si \( (x_n)\) et \( (y_n)\) sont de Cauchy alors \( (x_n+y_n)\), \( (x_n-y_n)\) et \( (x_ny_n)\) sont également de Cauchy.
		\item       \label{ITEMooIAFSooAIUpAN}
		      Si il existe \( a,b\in \eQ\) tels que \( x_n\to a \) et \( y_n\to b \) alors \( x_n+y_n\to a+b\), \( x_n-y_n\to a-b\) et \(  x_ny_n\to ab  \).
		\item   \label{ItemRKCIooJguHdjvi}
		      Soit \( (x_n)\) une suite de Cauchy qui ne converge pas vers zéro. Alors il existe \( n_0\) tel que la suite \( \left( \frac{1}{ x_n } \right)_{n\geq n_0}\) soit de Cauchy.
	\end{enumerate}
\end{proposition}

\begin{proof}
	Point par point.
	\begin{enumerate}
		\item
		      C'est la proposition~\ref{PROPooTFVOooFoSHPg}.
		\item
		      Soit \( (x_n)\) une suite de Cauchy dans \( \eQ\). Avec \( \epsilon=1\) dans la définition, si \( q>N\), nous avons
		      \begin{equation}
			      | x_q-x_{N} |\leq 1.
		      \end{equation}
		      Et donc pour tout \( q\) plus grand que \( N\), \( x_N-1\leq x_q\leq x_N+1\), ou encore, pour tout \( n\) :
		      \begin{equation}
			      | x_n |\leq\max\{ | x_1 |,| x_2 |,\ldots,| x_N |,| x_N+1 | \}.
		      \end{equation}
		      La suite est donc bornée.
		\item
		      Soit \(\epsilon>0\). Les hypothèses disent qu'il existe un \( N\) tel que \( | x_n |\leq \epsilon\) dès que \( n\geq N\). Et il existe aussi \( M\geq 0\) tel que \( | y_n |\leq M\) pour tout \( n\). Du coup, lorsque \( n\geq N\) nous avons \( | x_ny_n |\leq M\epsilon\).
		\item
		      En ce qui concerne la somme,
		      \begin{equation}        \label{EqDCNBooAzrrBi}
			      | x_p+y_p-x_q-y_q |\leq | x_p-x_q |+| y_p-y_q |.
		      \end{equation}
		      Soit \( N_1\) tel que si \( p,q\geq N_1\) alors \( | x_p-x_q |\leq \epsilon\) et \( N_2\) de même pour la suite \( (y_n)\). En prenant \( N=\max\{ N_1,N_2 \}\), la somme \eqref{EqDCNBooAzrrBi} est plus petite que \( 2\epsilon\) dès que \( p,q\geq N\).

		      Passons à la démonstration du fait que le produit de deux suites de Cauchy est de Cauchy. Les suites \( (x_n)\) et \( (y_n)\) sont bornées et quitte à prendre le maximum, nous disons qu'elles sont toutes les deux bornées par le nombre \( M\) : pour tout \( n\) nous avons \( | x_n |\leq M\) et \( | y_n |\leq M\). Nous avons :
		      \begin{equation}
			      | x_py_p-x_qy_q |\leq | x_py_p-x_qy_p |+| x_qy_p-x_qy_q |\leq | y_p | |x_p-x_q |+| x_q | |y_p-y_q |.
		      \end{equation}
		      Puisque \( (x_n)\) et \( (y_n)\) sont de Cauchy, si \( p\) et \( q\) sont assez grands, les deux différences sont majorées par \( \epsilon\) et nous avons
		      \begin{equation}
			      | x_py_p-x_qy_q |\leq M\epsilon+M\epsilon=2M\epsilon,
		      \end{equation}
		      ce qui prouve que \( (x_ny_n)\) est de Cauchy.
		\item
		      En ce qui concerne la somme, nous pouvons tout de suite calculer
		      \begin{equation}
			      | x_n+y_n-(a+b) |\leq | x_n-a |+| y_n-b |.
		      \end{equation}
		      Il existe une valeur de \( n\) à partir de laquelle le premier terme est plus petit que \( \epsilon\) et une à partir de laquelle le second terme est plus petit que \( \epsilon\). En prenant le maximum des deux, la somme est plus petite que \( 2\epsilon\).

		      En ce qui concerne le produit,
		      \begin{equation}
			      | x_ny_n-ab |\leq | x_ny_n-ay_n |+| ay_n-ab |\leq | y_n || x_n-a |+| a || y_n-b |.
		      \end{equation}
		      Les suites \( | x_n-a |\) et \( | y_n-b |\) convergent vers zéro; la suite \( (y_n)\) est bornée parce que convergente (combinaison des points~\ref{ItemRKCIooJguHdji} et~\ref{ItemRKCIooJguHdjii})  et \( a\) (la suite constante) est également bornée. Donc par le point~\ref{ItemRKCIooJguHdjiii}, nous avons
		      \begin{equation}
			      y_n| x_n-a |+a| y_n-b |\to 0.
		      \end{equation}
		      Au passage nous avons également utilisé la propriété de la somme que nous venons de démontrer.
		\item Soit \( (x_n)\) une suite de Cauchy dans \( \eQ\) ne convergeant pas vers zéro : il existe \( \alpha>0\) tel que pour tout \( N\in \eN\), il existe \( n\geq N\) tel que \( | x_n |>\alpha\). Mais notre suite est de Cauchy, donc il existe \( n_0\in \eN\) tel que si \( p,q\geq n_0\) alors
		      \begin{equation}
			      | x_p-x_q |\leq \frac{ \alpha }{2}.
		      \end{equation}
		      En fixant \( N = n_0\), on obtient un naturel \( n\geq n_0\) tel que \( | x_n |\geq \alpha\). De plus, comme la suite est de Cauchy, si \( p>n\) nous avons aussi \( | x_n-x_p |\leq \frac{ \alpha }{2}\). Cela implique \( | x_p |\geq \frac{ \alpha }{2}\) et en particulier \( x_p\neq 0\).

		      Nous venons de prouver que la suite ne s'annule plus à partir de l'indice \( n\), et même que \( | x_k |\geq\alpha/2\) pour tout \( k\geq n\). La suite \( (1/x_k)_{k\geq n}\) est donc bien définie.

		      Soit \( \epsilon>0\). Soit \( n_0\) tel que \( | x_p-x_q |<\epsilon\) pour tout \( p,q>n_0\). Soit \( K\) plus grand que \( n_0\) et que \( n\). En prenant \( p,q\geq K\), nous avons \( |  x_p|>\frac{ \alpha }{2}\) et \( | x_q |>\frac{ \alpha }{2}\). Nous en déduisons que
		      \begin{equation}
			      \left| \frac{1}{ x_p }-\frac{1}{ x_q } \right| \leq \frac{ | x_q-x_p | }{ | x_px_q | }\leq \frac{ 4 }{ \alpha^2 }| x_q-x_p |\leq \frac{ 4 }{ \alpha^2 }\epsilon.
		      \end{equation}
		      Donc \( \left( \frac{1}{ x_n } \right)\) est de Cauchy.
	\end{enumerate}
\end{proof}


%+++++++++++++++++++++++++++++++++++++++++++++++++++++++++++++++++++++++++++++++++++++++++++++++++++++++++++++++++++++++++++ 
\section{Insuffisance des rationnels}
%+++++++++++++++++++++++++++++++++++++++++++++++++++++++++++++++++++++++++++++++++++++++++++++++++++++++++++++++++++++++++++

Nous allons voir qu'il n'existe pas de nombres rationnels \( x\) tels que \( x^2=2\), mais que pourtant il existe une infinité de suites de rationnels \( (x_n)\) tels que \(  x_n^2\to 2  \).

\begin{lemma}       \label{LemJPIUooWFHaFM}
	Un entier \( x\) est pair si et seulement si l'entier \( x^2\) est pair.
\end{lemma}

\begin{proof}
	Si \( x\) est un nombre pair, alors il existe un entier \( a\) tel que \( x=2a\) alors \( x^2=4a^2\) est pair.

	Inversement, si \( x\) est impair alors il existe un entier \( a\) tel que \( x=2a+1\) et alors \( x^2=4a^2+4a+1=2(2a^2+2a)+1\) est impair.
\end{proof}

Le théorème~\ref{THOooYXJIooWcbnbm} nous dira que tous les \( \sqrt{n}\) sont irrationnels dès que \( n\) n'est pas un carré parfait. Voici déjà le résultat pour \( n=2\). Le fait que \( \sqrt{ 2 }\) existe dans \( \eR\) sera la proposition \ref{PROPooUHKFooVKmpte}.
\begin{proposition}[Irrationalité de \( \sqrt{2}\)]     \label{PropooRJMSooPrdeJb}
	Il n'existe pas de fractions d'entiers dont le carré soit égal à \( 2\).
\end{proposition}
\index{irrationalité!\( \sqrt{2}\)}

\begin{proof}
	Nous supposons que la fraction d'entiers \( a/b\) est telle que \( a^2/b^2=2\), et nous allons construire une suite d'entiers strictement décroissante et strictement positive, ce qui est impossible.

	Grâce au lemme~\ref{LemJPIUooWFHaFM} nous avons successivement les affirmations suivantes :
	\begin{itemize}
		\item
		      \(\frac{ a^2 }{ b^2 }=2 \)  avec \( a\neq 0\) et \( b\neq 0\).
		\item
		      \( a^2=2b^2\), donc \( a^2\) est pair.
		\item
		      \( a\) est alors pair et \( a^2\) est divisible par \( 4\). Soit \( a^2=4k\).
		\item
		      \( 4k/b^2=2\), donc \( 4k=2b^2\), donc \( b^2=2k\) et \( b^2\) est pair.
		\item
		      Nous déduisons que \( b\) est pair.
	\end{itemize}
	La fraction \( \frac{ a/2 }{ b/2 }\) est alors une nouvelle fraction d'entiers dont le carré vaut \( 2\). En procédant de la même façon, en remplaçant \( a\) par \( a/2\) et \( b\) par \( b/2\), on obtient que la fraction d'entiers \( \frac{ a/4 }{ b/4 }\) a la même propriété.

	En particulier, tous les nombres de la forme \( a/2^n\) sont des entiers.  Ils forment une suite strictement décroissante d'entiers strictement positifs. Impossible, me diriez-vous ? Et vous auriez bien raison : toute partie non vide de \( \eN\) admet un plus petit élément\footnote{Voir \cite{RWWJooJdjxEK}, et attention : ce n'est pas tout à fait évident.}. Il n'y a donc pas de fractions d'entiers dont le carré vaut \( 2\).
\end{proof}

\begin{lemma}[Série géométrique]   \label{LEMooOTVUooImvusn}
	Si \( q\neq 1\) dans \( \eQ\) et \( p\in \eN\) nous avons
	\begin{equation}
		\sum_{k=0}^pq^k=\frac{ 1-q^{p+1} }{ 1-q }.
	\end{equation}
\end{lemma}

\begin{proof}
	En posant \( S_p=1+q+q^2+\cdots +q^{p}\), nous avons \( S_p-qS_p=1-q^{p+1}\) et donc
	\begin{equation}
		S_p=\sum_{k=0}^pq^k=\frac{ 1-q^{p+1} }{ 1-q }.
	\end{equation}
\end{proof}

\begin{proposition}
	La suite donnée par
	\begin{equation}
		x_n=1+\frac{ 1 }{ 1! }+\cdots +\frac{1}{ n! }
	\end{equation}
	est de Cauchy et ne converge pas dans \( \eQ\).
\end{proposition}

\begin{proof}
	Si \( p>q>0\) nous avons
	\begin{subequations}
		\begin{align}
			x_p-x_q & =\sum_{k=q+1}^p\frac{1}{ k! }                                                                            \\
			        & \leq \sum_{k=q+1}^p\frac{1}{ (q+1)! }\frac{1}{ (q+1)^{k-q-1} }  \label{SUBEQooAXILooEAcpVB}              \\
			        & \leq \frac{1}{ (q+1)! }\lim_{p\to \infty} \sum_{k=0}^{p}\frac{1}{ (q+1)^k }  \label{SUBEQooNDPTooDSEYEJ} \\
			        & \leq \frac{1}{ (q+1)! }\frac{1}{ 1-\frac{1}{ q+1 } } \label{SUBEQooEMHJooSnCUiK}                         \\
			        & \leq \frac{1}{ (q+1)! }\frac{q+1}{q}                                                                     \\
			        & \leq \frac{1}{ q!q }.
		\end{align}
	\end{subequations}
	Justifications :
	\begin{itemize}
		\item Pour \eqref{SUBEQooAXILooEAcpVB}, il s'agit de remplacer dans \( k!\) tous les facteurs plus grands que \( (q+1)\) par \( q+1\). Cela rend le dénominateur plus petit.
		\item Pour \eqref{SUBEQooNDPTooDSEYEJ}, il y a une inégalité parce que la suite \( p\mapsto \sum_{k=0}^p1/(q+1)^k\) est une suite strictement croissante.

		\item Pour \eqref{SUBEQooEMHJooSnCUiK}, le lemme~\ref{LEMooOTVUooImvusn} donne la valeur de la somme finie. En ce qui concerne la limite, nous avons demandé \( p>q>0\) et donc \( q+1>1\). Dans ce cas la limite fonctionne.
	\end{itemize}

	Cette inégalité une fois établie nous permet de prouver les assertions. La suite \( (x_n) \) est de Cauchy car, pour tout \( \epsilon\in\eQ\) s'écrivant \( \epsilon=\frac{ a }{ b }\) avec \( a,b\in \eN\), en prenant \( p,q>b\), nous avons
	\begin{equation}
		x_p-x_q\leq \frac{1}{ b!b }<\frac{1}{ b }<\frac{ a }{ b }=\epsilon.
	\end{equation}

	Montrons par l'absurde que cette suite ne converge pas dans \( \eQ\). Pour cela, nous supposons que \( \lim_{n\to \infty} x_n=\frac{ a }{b }\in \eQ\). Pour tout \( p>q\) nous avons établi
	\begin{equation}
		0<x_p-x_q<\frac{1}{ qq! }.
	\end{equation}
	Prenons la limite \( p\to \infty\); par stricte croissance de la suite, les inégalités restent strictes :
	\begin{equation}        \label{EqQLCTooOgQOdh}
		0<\frac{ a }{ b }-x_q<\frac{1}{ qq! }.
	\end{equation}
	Si \( n>b\) alors nous pouvons écrire
	\begin{equation}
		\frac{ a }{ b }-x_n=\frac{ \alpha }{ n! }
	\end{equation}
	avec \( \alpha\in \eZ\) parce que le dénominateur commun entre \( \frac{ a }{ b }\) et \( x_n\) est dans \( n!\). En prenant donc \( q>n\) dans \eqref{EqQLCTooOgQOdh} nous pouvons écrire
	\begin{equation}
		0<\frac{ \alpha }{ q! }<\frac{1}{ qq! },
	\end{equation}
	c'est-à-dire \( 0<\alpha<\frac{1}{ q }\), ce qui est impossible pour \( \alpha\in \eZ\).
\end{proof}

\begin{lemma}   \label{LEMooDTXYooKwmlZh}
	Soit \( A>0\) dans \( \eQ\). Il existe un rationnel \( q>0\) tel que \( q^2<A\).
\end{lemma}

\begin{proof}
	Vu que \( \eQ\) est archimédien (proposition \ref{PROPooMXGPooDUkOuv}), il existe \( n\in \eN\) tel que \( 1<nA\). Pour ce \( n\), nous avons
	\begin{equation}
		\left( \frac{1}{ n } \right)^2<\frac{1}{ n }<A.
	\end{equation}
\end{proof}

La proposition suivante donne une suite de rationnels qui convergerait dans \( \eR\) vers \( \sqrt{ A }\) (non encore défini à ce stade). Il est expliqué dans \cite{BIBooMPXEooQLKhku} que la suite peut être vue comme une forme de méthode de Newton \ref{THOooDOVSooWsAFkx}; voir l'exemple \ref{EXooDLSVooMHPpcl}. Si vous aimez les dessins et les approches géométriques, il y a une explication sur Wikipédia\cite{BIBooVCWCooQcolIq}.
\begin{proposition}[\cite{BIBooMPXEooQLKhku}]       \label{PROPooSTQXooHlIGVf}
	Soient \( A>0\) dans \( \eQ\) et \( x_0\in \eQ\). La suite \( (x_k)\) définie par
	\begin{equation}        \label{EQooOUIVooUqWhXe}
		x_{k+1}=\frac{ 1 }{2}\left( x_k+\frac{ A }{ x_k } \right)
	\end{equation}
	a les propriétés suivantes :
	\begin{enumerate}
		\item
		      La suite \( y_k=x_k^2 \) converge dans \( \eQ\) vers \( A\).
		\item
		      La suite \( (x_k)\) est de Cauchy dans \( \eQ\).
		\item
		      La suite \( (x_k)\) ne converge pas dans \( \eQ\) dans le cas de \( A=2\).
	\end{enumerate}
\end{proposition}

\begin{proof}
	En plusieurs points.
	\begin{subproof}
		\spitem[La suite \( s_k\)]
		En posant \( y_k=x_k^2\) nous calculons que
		\begin{equation}
			y_{k+1}-A=\frac{ (y_k-A)^2 }{ 4y_k }.
		\end{equation}
		Autrement dit, la suite \( s_k=y_k-A\) vérifie
		\begin{equation}
			s_{k+1}=\frac{ s_k^2 }{ 4(A+s_k) }.
		\end{equation}
		Quelle que soit la valeur de \( s_0=x_0^2-A\), nous avons
		\begin{equation}
			s_1=\frac{ s_0^2 }{ 4(A+s_0) }=\frac{ (x_0^2-A)^2 }{ 4(A+x_0^2-A) }=\frac{ (x_0^2-A)^2 }{ 4x_0^2 }>0.
		\end{equation}
		Donc à partir de \( s_1\), tous les éléments sont positifs. Vu que \( A>0\) nous avons aussi
		\begin{equation}
			s_{k+1}<\frac{ s_k^2 }{ 4s_k }=\frac{ s_k }{ 4 }
		\end{equation}
		et donc \( s_k<s_0/4^k\). Donc \( s_k\to 0\).
		\spitem[La suite \( (y_k)\)]
		Nous venons de prouver que si \( y_k=A+s_k\), alors \( s_k\to 0\). Autrement dit, la suite \( y_k\) converge vers \( A\) dans \( \eQ\).

		La suite \( (y_k)\) est donc de Cauchy par la proposition \ref{PropFFDJooAapQlP}\ref{ItemRKCIooJguHdji}.
		\spitem[La suite \( (x_k)\) est de Cauchy]
		Soit \( \epsilon>0\) dans \( \eQ\). Puisque \( (y_k)\) est de Cauchy, il existe \( n_0\in \eN\) tel que
		\begin{equation}
			| x^2_r-x_s^2 |<\epsilon
		\end{equation}
		pour tout \( r,s\geq n_0\).

		Soit par ailleurs \( q\neq 0\) dans \( \eQ\) tel que \( q^2<A\), assuré par le lemme \ref{LEMooDTXYooKwmlZh}. Quitte à augmenter la valeur de \( n_0\), nous supposons que \( x_r,x_s>q\), et en particulier que \( x_r+x_s\neq 0\). Cela permet d'écrire d'abord
		\begin{equation}
			x_r^2-x_s^2=(x_r+x_s)(x_r-x_s)
		\end{equation}
		et ensuite de prendre la valeur absolue et de diviser par \( | x_r+x_s |\) :
		\begin{equation}
			| x_r-x_s |=\frac{ | x_r^2-x_s^2 | }{ | x_r+x_s | }<\frac{ \epsilon }{ 2q }.
		\end{equation}
		Donc \( (x_k)\) est une suite de Cauchy.
		\spitem[Pas de convergence pour \( A=2\)]
		Supposons que \( x_k\to a\in \eQ\). Dans ce cas nous aurions \( x_k^2\to a^2=A=2\) (proposition~\ref{PropFFDJooAapQlP}\ref{ITEMooIAFSooAIUpAN}). Mais nous savons par la proposition~\ref{PropooRJMSooPrdeJb} que \( a^2=2\) est impossible dans \( \eQ\).
	\end{subproof}
\end{proof}

Notons que cette proposition ne présume en rien de l'existence ou de la non-existence dans \( \eQ\) d'un élément qui pourrait décemment être nommé \( \sqrt{ A }\). Il se fait que le théorème \ref{THOooYXJIooWcbnbm} dira que \( \sqrt{ n }\) est soit entier, soit irrationnel.

\begin{normaltext}
	Un petit programme en python pour explorer la suite de la proposition \ref{PROPooSTQXooHlIGVf}.
	\lstinputlisting{tex/frido/codeSnip_4.py}
\end{normaltext}
