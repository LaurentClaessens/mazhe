% This is part of Mes notes de mathématique
% Copyright (c) 2006-2019
%   Laurent Claessens, Carlotta Donadello
% See the file fdl-1.3.txt for copying conditions.

%++++++++++++++++++++++++++++++++++++++++++++++++++++++++++++++++++++++++++++++++++++++++
\section{Fonctions de classe \texorpdfstring{$ C^1$}{C1}}
%++++++++++++++++++++++++++++++++++++++++++++++++++++++++++++++++++++++++++++++++++++++++++++++++++++++++++++++++++++++++++++++

Soit $f$ une fonction différentiable de $U$, ouvert de $\eR^m$, dans $\eR^n$. L'application différentielle de $f$ est une application  de $\eR^m$ dans $\aL(\eR^m, \eR^n)$
\begin{equation}
    \begin{aligned}
        df\colon \eR^m&\to \aL(\eR^m,\eR^n) \\
        a&\mapsto df_a 
    \end{aligned}
\end{equation}
Nous savons que $\aL(\eR^m, \eR^n)$ est un espace vectoriel normé avec la définition~\ref{DefDQRooVGbzSm}. Si $T$ est un élément dans $\aL(\eR^m, \eR^n)$ alors la norme de $T$ est définie par
\[
\|T\|_{\aL(\eR^m, \eR^n)}=\sup_{x\in\eR^m} \frac{\|T(x)\|_n}{\|x\|_m}=\sup_{\begin{subarray}{l}
    x\in\eR^m\\
\|x\|_m\leq 1
  \end{subarray}} \|T(x)\|_n.
\]

Lorsqu'il existe un $M>0$ tel que $\| df(a) \|_{\aL(\eR^m,\eR^n)}<M$ pour tout $a$ dans $U$, nous disons que la différentielle de $f$ est \defe{bornée}{bornée!différentielle} sur $U$.

\begin{definition}
	La fonction $f$ est dite \defe{de classe $\mathcal{C}^1$}{fonction!de classe  $\mathcal{C}^1$} de $U\subset\eR^m$  dans $\eR^n$ si son application différentielle $df$ est continue de $\eR^m$ dans $\aL(\eR^m, \eR^n)$. Nous écrivons $f\in\mathcal{C}^1(U,\eR^n)$\nomenclature{$C^1(U,\eR^n)$}{Les applications une fois continument dérivables}.
\end{definition}

\begin{proposition}		\label{PropDerContCun}
	Une fonction \( f\colon U\to \eR^n\) où \( U\) est ouvert dans \( \eR^m\) est de classe \( C^1\) si et seulement si les dérivées partielles de $f$ existent et sont continues.
\end{proposition}

\begin{proof}
	Supposons que les dérivées partielles de $f$ existent et sont continues. Nous savons alors déjà par la proposition~\ref{Diff_totale} que la fonction $f$ est différentiable et qu'elle s'exprime sous la forme
	\[
		df_a(h)=\sum_{i=1}^{m}\partial_if (a)h_i, \qquad \forall a \in U,\,\forall h\in\eR^m.
	\]
	Pour montrer que $df$ est continue, nous devons montrer que la quantité $\| df(x)-df(a) \|_{\aL(\eR^m,\eR^n)}$ peut être rendue arbitrairement petite si $\| x-a \|_m$ est rendu petit. Nous avons
	\begin{equation}
		\begin{aligned}
			\| df_x-df_a \|_{\aL}&=\sup_{\| h \|=1}\| df_x(h)-df_a(h) \|\\
			&=\sup_{\| h \|_m=1}\left\|\sum_{i=1}^{m}\left(\partial_if (x)-\partial_if (a)\right)h_i\right\|_n\leq\\
			&\leq\sup_{\| h \|_m=1}\sum_{i=1}^{m} \left\|\left(\partial_if (x)-\partial_if (a)\right)\right\|_n|h_i|\leq\\
			&\leq\sup_{\| h \|_m=1} \|h\|_\infty\sum_{i=1}^{m} \left\|\left(\partial_if (x)-\partial_if (a)\right)\right\|_n\\
			&\leq \sum_{i=1}^m\| \partial_if(x)-\partial_if(a) \|.
		\end{aligned}
	\end{equation}
	Dans ce calcul, nous avons utilisé le fait que si $\| h \|_m\leq 1$, alors $\| h \|_{\infty}\leq 1$. Étant donné la continuité de $\partial_if$, la dernière ligne peut être rendue arbitrairement petite lorsque $x$ est proche e $a$.

Supposons maintenant que $f$ soit dans $\mathcal{C}^1(U,\eR^n)$. Alors
\[
\left\|\partial_if (x)-\partial_if (a)\right\|_n= \left\|df(x).e_i-df(a).e_i\right\|_n \leq  \left\|df(x)-df(a)\right\|_{\aL(\eR^m,\eR^n)},
\]
la continuité de $df$ implique donc celle de $\partial_i f$ pour tout $i$ dans $\{1,\ldots,m\}$.
\end{proof}
\begin{proposition}
  Soient $U$ un ouvert de $\eR^m$ et $V$ un ouvert de $\eR^n$. Soient $f: U\to V$  dans $\mathcal{C}^1(U,V)$ et $g: V \to \eR^p$ dans $\mathcal{C}^1(V,\eR^n)$.  Alors la fonction composée $g\circ f: U\to \eR^p $ est dans $\mathcal{C}^1(U,\eR^p)$.
\end{proposition}
\begin{proof} On fixe $a$ dans $U$
  \begin{equation}
    \begin{aligned}
     \big\|d(g\circ f)(x)&-d(g\circ f)(a)\big\|_{\aL(\eR^m,\eR^p)}\\
     &=\left\|dg(f(x))\circ df(x)-dg(f(a))\circ df(a)\right\|_{\aL(\eR^m,\eR^p)}\leq\\
&\leq \left\|\left(dg(f(x))-dg(f(a))\right)\circ df(x)\right\|_{\aL(\eR^m,\eR^p)}+\\
&\quad+ \left\|dg(f(a))\circ \left(df(x)-df(a)\right)\right\|_{\aL(\eR^m,\eR^p)}\leq\\
&\leq \left\|dg(f(x))-dg(f(a))\right\|_{\aL(\eR^n,\eR^p)}\left\| df(x)\right\|_{\aL(\eR^m,\eR^n)}+\\
&\quad+ \left\|dg(f(a))\right\|_{\aL(\eR^n,\eR^p)}\left\| df(x)-df(a)\right\|_{\aL(\eR^n,\eR^p)}.\\
    \end{aligned}
  \end{equation}
On peut conclure en passant à la limite $x\to a$ parce que les fonctions $f$, $g$, $df$ et $dg$ sont continues, de telle sorte que
\begin{equation}
	\begin{aligned}[]
		\lim_{x\to a} dg\big( f(x) \big)=dg\big( f(a) \big)\\
		\lim_{x\to a} df(x)=df(a).
	\end{aligned}
\end{equation}
\end{proof}

\begin{remark}
  On peut prouver le même résultat en utilisant la continuité de l'application bilinéaire
\begin{equation}
  \begin{array}{rccc}
    \circ : & \mathcal{C}^1(U,V)\times\mathcal{C}^1(V,\eR^p)  & \to & \aL(U, \eR^p)\\
& (T,S)& \mapsto & T\circ S.
  \end{array}
\end{equation}
\end{remark}


%+++++++++++++++++++++++++++++++++++++++++++++++++++++++++++++++++++++++++++++++++++++++++++++++++++++++++++++++++++++++++++
\section{Différentielle et dérivée complexe}
%+++++++++++++++++++++++++++++++++++++++++++++++++++++++++++++++++++++++++++++++++++++++++++++++++++++++++++++++++++++++++++
\label{SECooJWNOooOgMiWR}

\begin{normaltext}
    Nous commençons par donner quelques éléments à propos de dérivée et de différentielle pour des fonctions \( \eC\to \eC\) parce que les séries entières vont souvent être des fonctions complexes. Le gros du chapitre sur les fonctions holomorphes est le chapitre~\ref{ChapICHIooXbLccl}.
\end{normaltext}

Nous identifions \( \eR^2\) à \( \eC\) par l'application
\begin{equation}
    \begin{aligned}
        \varphi\colon \eR^2&\to \eC \\
        (x,y)&\mapsto x+iy.
    \end{aligned}
\end{equation}
Dans cette partie, nous désignons par \( \Omega\) un ouvert de \( \eC\).

\begin{definition}      \label{DEFooVJVXooKlnFkh}
    Une fonction \( f\colon \Omega\to \eC\) est \defe{$\eC$-dérivable}{dérivable!au sens complexe} si la limite
    \begin{equation}
        \lim_{\substack{h\to0\\h\in \eC}} \frac{ f(a+h)-f(a) }{ h }
    \end{equation}
    existe. Dans ce cas, cette limite est la dérivée de \( f\) et est notée \( f'\).
\end{definition}

\begin{definition}  \label{DefMMpjJZ}
    Soit \( \Omega\) un ouvert dans \( \eC\). Une fonction \( f\colon \Omega\to \eC\) est \defe{holomorphe}{holomorphe}\index{fonction!holomorphe} si elle est \( \eC\)-dérivable sur \( \Omega\).
\end{definition}

\begin{definition}
    Une matrice de la forme
    \begin{equation}
        \begin{pmatrix}
            \alpha    &   \beta    \\
            -\beta    &   \alpha
        \end{pmatrix}
    \end{equation}
    avec \( \alpha,\beta\in \eR\) est une \defe{similitude}{matrice!de similitude}\index{similitude}.
\end{definition}

\begin{lemma}       \label{LEMooJNFEooZCbJMo}
    En tant qu'application linéaire \( \eC\to \eC\), l'opération de multiplication par \( \alpha+\beta i\) est la matrice
    \begin{equation}
        \begin{pmatrix}
            \alpha    &   -\beta    \\
            \beta    &   \alpha
        \end{pmatrix}.
    \end{equation}
\end{lemma}

\begin{proof}
    Cela est vite remarqué en calculant explicitement \( (\alpha+\beta i)(u_1+iu_2)\).
\end{proof}

\begin{lemma}
    Une application \( A\colon \eC\to \eC\) est \( \eC\)-linéaire si et seulement si elle est une similitude en tant qu'application \( \eR^2\to \eR^2\).

    Dans ce cas, il existe \( z_0\in \eC\) tel que \( A(z)=z_0z\) pour tout \( z\in \eC\).
\end{lemma}

\begin{proof}
    Commençons par considérer l'application \( A\) sur \( \eR^2\). Elle est en particulier une application \( \eR\)-linéaire et par conséquent il existe une matrice \( \begin{pmatrix}
        \alpha    &   \beta    \\
        \gamma    &   \delta
    \end{pmatrix}\) telle que
    \begin{equation}
        A\begin{pmatrix}
            x    \\
            y
        \end{pmatrix}=\begin{pmatrix}
            \alpha    &   \beta    \\
            \gamma    &   \delta
        \end{pmatrix}\begin{pmatrix}
            x    \\
            y
        \end{pmatrix}.
    \end{equation}
    Nous voulons maintenant imposer la \( \eC\)-linéarité, c'est-à-dire que nous voulons
    \begin{equation}
        A\big( (a+bi)(x+iy) \big)=(a+bi)A(x+iy)
    \end{equation}
    pour tout \( a,b,x,y\in \eR\). À gauche nous avons
    \begin{equation}
        A\big( ax-by+i(bx+ay) \big)
    \end{equation}
    et à droite nous avons
    \begin{equation}
        (a+bi)\big( \alpha x+\beta y+i(\gamma x+\delta y) \big).
    \end{equation}
    En égalant les deux expressions nous obtenons les équations
    \begin{subequations}
        \begin{numcases}{}
            \beta b=-b\gamma\\
            -\alpha b+\beta a =a\beta -b\delta\\
            \delta b=b\alpha\\
            -\gamma b+\delta a=b\beta+a\delta,
        \end{numcases}
    \end{subequations}
    dont nous tirons immédiatement que \( \gamma=-b\beta\) et \( \delta=\alpha\). La matrice de \( A\) est donc de la forme demandée.

    Inversement nous devons prouver que la fonction
    \begin{equation}        \label{EqOEWYooMaHCNb}
        f(x+iy)=\alpha x+\beta y+i(-\beta x+\alpha y)
    \end{equation}
    est \( \eC\)-linéaire, c'est-à-dire qu'elle vérifie \( f(z_0z)=z_0f(z)\) pour tout \( z_0,z\in \eC\). Cela est un simple calcul que nous confions à Sage : le code suivant affiche «\( 0\)».
    \lstinputlisting{tex/frido/code_sage3.py}

    Pour conclure, notons que la fonction \eqref{EqOEWYooMaHCNb} est la fonction de multiplication par \( \alpha-i\beta\).
\end{proof}

\begin{normaltext}      \label{NORMooMKNDooBeoGRN}
    Soient une fonction \( f\colon \eC\to \eC\) et l'isomorphisme canonique \( \varphi\colon \eC\to \eR^2\). La fonction \( f\) définit une la fonction
    \begin{equation}
        F=\varphi^{-1} \circ f\circ \varphi\colon \eR^2\to \eR^2.
    \end{equation}
    Cela est la fonction \( \eR^2\to \eR^2\) associée à \( f\). Il serait tentant de croire que tout ce qui est vrai pour \( F\) est également vrai pour \( f\). Eh bien non.

    Par exemple, \( F\) peut être différentiable sans que \( f\) le soit. La proposition suivant donne une condition sur \( dF\) pour que \( f\) soit différentiable.
\end{normaltext}

\begin{proposition}     \label{PropKJUDooJfqgYS}
    Une fonction \( f\colon \Omega\to \eC\) est $\eC$-dérivable en \( a\in\Omega\) si et seulement si elle est différentiable en \( a\) et si \( df_a\) est une similitude.

    Plus précisément avec les notations de~\ref{NORMooMKNDooBeoGRN}, la fonction \( f\) est $\eC$-dérivable (donc holomorphe) au point \( z_0=x_0+iy_0\) si et seulement si la fonction \( F\) est différentiable en \( (x_0,y_0)\) et si la matrice de \( dF\) est de la forme
    \begin{equation}        \label{EQooWZGKooLDEHGr}
        dF=\begin{pmatrix}
            \alpha    &   \beta    \\
            -\beta    &   \alpha
        \end{pmatrix},
    \end{equation}
    c'est-à-dire si \( dF_{(x_0,y_0)}\) fournit une application \( \eC\)-linéaire.

    Dans ce cas, le lien entre \( \eC\)-dérivée et différentielle est donné par
    \begin{equation}        \label{EqPAEFooYNhYpz}
        (df_{z_0})(z)=f'(z_0)z.
    \end{equation}
\end{proposition}

\begin{proof}
    Nous décomposons \( f\) en parties réelles et imaginaires :
    \begin{equation}
        f(x+iy)=P(x,y)+iQ(x,y)
    \end{equation}
    où \( P\) et \( Q\) sont des fonctions réelles. La jacobienne de \( F\) est la matrice
    \begin{equation}
        \begin{pmatrix}
            \frac{ \partial P }{ \partial x }    &   \frac{ \partial P }{ \partial y }    \\
            \frac{ \partial Q }{ \partial x }    &   \frac{ \partial Q }{ \partial y }
        \end{pmatrix},
    \end{equation}
    et la condition dont nous parlons s'écrit comme le système
    \begin{subequations}    \label{EqFDUrXBP}
        \begin{numcases}{}
            \frac{ \partial P }{ \partial x }=\frac{ \partial Q }{ \partial y }\\
            \frac{ \partial P }{ \partial y }=-\frac{ \partial Q }{ \partial x}.
        \end{numcases}
    \end{subequations}
    Si \( F\) est différentiable en \( (x_0,y_0)\) alors nous avons
    \begin{equation}        \label{EqwlVfiR}
        F\big( (x_0,y_0)+(h,k) \big)=F(x_0,y_0)+dF_{(x_0,y_0)}\begin{pmatrix}
            h    \\
            k
        \end{pmatrix}+s(| h |+| k |)
    \end{equation}
    où \( s\) est une fonction vérifiant \( \lim_{t\to 0} \frac{ s(t) }{ t }=0\). Soit
    \begin{equation}
        dF_{(x_0,y_0)}=\begin{pmatrix}
            \alpha    &   \beta    \\
            -\beta    &   \alpha
        \end{pmatrix}.
    \end{equation}
    Si nous posons \( \sigma=\alpha-i\beta\) et \( w=h+ik\), l'équation \eqref{EqwlVfiR} s'écrit dans \( \eC\) sous la forme
    \begin{equation}        \label{EqYFmoiM}
        f(z_0+w)=f(z_0)+\sigma w+s(|w|),
    \end{equation}
    ce qui implique que \( f\) est $\eC$-dérivable en \( z_0\).

    Supposons maintenant que \( f\) soit $\eC$-dérivable en \( z_0\). Alors nous avons
    \begin{equation}
        f'(z_0)=\lim_{w\to 0} \frac{ f(z_0+w)-f(z_0) }{ w }=\sigma\in \eC,
    \end{equation}
    ce qui se récrit sous la forme
    \begin{equation}
        \lim_{w\to 0} \frac{ f(z_0+w)-f(z_0)-\sigma w }{ w }=0.
    \end{equation}
    Si nous posons \( z_0=x_0+iy_0\), \( w=h+ik\) et \( \sigma=\alpha-i\beta\) nous avons
    \begin{equation}
        \lim_{(h,k)\to (0,0)} \left| \frac{ F\big( (x_0,y_0)+(h,k) \big)-F(x_0,y_0)-\begin{pmatrix}
            \alpha    &   \beta    \\
            -\beta    &   \alpha
        \end{pmatrix}\begin{pmatrix}
            h    \\
            k
        \end{pmatrix}}{ | w | } \right| =0,
    \end{equation}
    ce qui signifie que \( F\) est différentiable et que sa différentielle est la matrice
    \begin{equation}    \label{EqMLtbLD}
       \begin{pmatrix}
           \alpha &   \beta    \\
           -\beta &   \alpha
       \end{pmatrix}.
    \end{equation}

    La matrice \eqref{EqMLtbLD} est, vue dans \( \eR^2\), la matrice de multiplication dans \( \eC\) par \( \alpha-i\beta=f'(z_0)\). En d'autre termes, dans \( \eC\) nous avons
    \begin{equation}
        df_{z_0}(z)=f'(z_0)z,
    \end{equation}
    et en particulier la différentielle est donnée par
    \begin{equation}        \label{EqPropZOkfmO}
        df_{z_0}=f'(z_0)dz.
    \end{equation}
\end{proof}

\begin{example}[Une application \( C^{\infty}\) mais pas \( \eC\)-dérivable]
    Nous considérons la fonction
    \begin{equation}
        \begin{aligned}
            f\colon \eC&\to \eC \\
            x+iy&\mapsto x.
        \end{aligned}
    \end{equation}
    Vu que c'est une application linéaire, elle est différentiable une infinité de fois et sa différentielle est elle-même. C'est donc une application \( C^{\infty}\).

    Elle n'est cependant pas \( \eC\)-dérivable. En effet le quotient différentiel est, pour \( \epsilon\in \eC\) :
    \begin{equation}
        \frac{ f(x+iy+\epsilon_x+i\epsilon_y)-f(x+iy) }{ \epsilon }=\frac{ \epsilon_x }{ \epsilon }.
    \end{equation}
    Cela n'a pas de limite lorsque \( \epsilon\to 0\). Pour voir cela nous invoquons la méthode des chemins du corollaire~\ref{CorMethodeChemin} avec les chemins \( \epsilon_1(t)=t\) et \( \epsilon_2(t)=it\). Dans le premier cas, le quotient différentiel vaut \( 1\) pour tout \( t\), tandis que dans le second il vaut zéro pour tout \( t\).
\end{example}

%---------------------------------------------------------------------------------------------------------------------------
\subsection{Quelques règles de calcul}
%---------------------------------------------------------------------------------------------------------------------------

\begin{lemma}       \label{LEMooVDXOooUyFHXZ}
    Si \( f\) et \( g\) sont deux fonctions holomorphes sur un ouvert \( \Omega\subset \eC\) et si \( g\) ne s'annule pas sur \( \Omega\), alors \( f/g\) est holomorphe sur \( \Omega\).
\end{lemma}


%++++++++++++++++++++++++++++++++++++++++++++++++++++++++++++++++++++++++++++++++++++++++++++++++++++++++++++++++++++++
\section{Théorèmes des accroissements finis}		\label{SecThoAccrsFinis}
%++++++++++++++++++++++++++++++++++++++++++++++++++++++++++++++++++++++++++++++++++++++++++++++++++++++++++++++++++++++

Nous avons déjà démontré (lemme~\ref{LemdfaSurLesPartielles}) que si $f$ est différentiable au point $x$ alors  $df_x(u)=\partial_uf(x)$. Une importante conséquence est le théorème des accroissements finis
\begin{theorem}[Accroissements finis, inégalité de la moyenne]\label{val_medio_2}
   Soit $U$ un ouvert dans $\eR^m$ et soit $f:U\to\eR^n$ une fonction différentiable. Soient $a$ et $b$ deux points dans $U$, $a\neq b$, tels que le segment $[a,b]$ soit contenu dans $U$. Alors
   \begin{equation}
        \|f(b)-f(a)\|_n\leq \sup_{x\in[a,b]}\|df(x)\|_{\aL(\eR^m,\eR^n)}\|b-a\|_m.
   \end{equation}
\end{theorem}
\index{application!différentiable}
\index{inégalité!de la moyenne}
\index{théorème!accroissements finis!forme générale}

\begin{proof}
 On utilise le théorème~\ref{val_medio_1} et le fait que
\[
\|\partial_u f(x)\|_n\leq \|df(x)\|_{\aL(\eR^m,\eR^n)}\|u\|_m,
\]
pour tout $u$ dans $\eR^m$.
\end{proof}

La proposition suivante est une application fondamentale du théorème des accroissements finis~\ref{val_medio_2}.
\begin{proposition}		\label{PropAnnulationEtConstance}
	Soit $U$ un ouvert connexe par arcs de $\eR^m$ et une fonction $f\colon U\to \eR^n$. Les conditions suivantes sont équivalentes :
	\begin{enumerate}
		\item\label{ItemPropCstDiffZeroi}
			$f$ est constante;
		\item\label{ItemPropCstDiffZeroii}
			$f$ est différentiable et $df(a)=0$ pour tout $a\in U$;
		\item\label{ItemPropCstDiffZeroiii}
			les dérivées partielles $\partial_1f,\ldots,\partial_mf$ existent et sont nulles sur $U$.
	\end{enumerate}
\end{proposition}
\index{connexité!par arc!fonction différentiable}
\index{différentiabilité}

\begin{proof}
	Nous allons démonter les équivalences en plusieurs étapes. D'abord~\ref{ItemPropCstDiffZeroi} $\Rightarrow$~\ref{ItemPropCstDiffZeroii}, puis~\ref{ItemPropCstDiffZeroii} $\Rightarrow$~\ref{ItemPropCstDiffZeroiii}, ensuite~\ref{ItemPropCstDiffZeroiii} $\Rightarrow$~\ref{ItemPropCstDiffZeroii} et enfin~\ref{ItemPropCstDiffZeroii} $\Rightarrow$~\ref{ItemPropCstDiffZeroi}.

	Commençons par montrer que la condition~\ref{ItemPropCstDiffZeroi} implique la condition~\ref{ItemPropCstDiffZeroii}. Si $f(x)$ est constante, alors la condition \eqref{EqCritereDefDiff} est vite vérifiée en posant $T(h)=0$.

	Afin de voir que la condition~\ref{ItemPropCstDiffZeroii} implique la condition~\ref{ItemPropCstDiffZeroiii}, remarquons d'abord que la différentiabilité de $f$ implique que les dérivées partielles existent (proposition~\ref{diff1}) et que nous avons l'égalité $df(a).u=\sum_iu_i\partial_if(a)$ pour tout $u\in\eR^m$ (lemme~\ref{LemdfaSurLesPartielles}). L'annulation de $\sum_iu_i\partial_if(a)$ pour tout $u$ implique l'annulation des $\partial_if(a)$ pour tout $i$.

	Prouvons maintenant que la propriété~\ref{ItemPropCstDiffZeroiii} implique la propriété~\ref{ItemPropCstDiffZeroii}. D'abord, par la proposition~\ref{Diff_totale}, l'existence et la continuité des dérivées partielles $\partial_if(a)$ implique la différentiabilité de $f$. Ensuite, la formule $df(a).u=\sum_i u_i\partial_if(a)$ implique que $df(a)=0$.


	Il reste à montrer que~\ref{ItemPropCstDiffZeroii} implique la condition~\ref{ItemPropCstDiffZeroi}, c'est-à-dire que l'annulation de la différentielle implique la constance de la fonction. C'est ici que nous allons utiliser le théorème des accroissements finis. En effet, si $a$ et $b$ sont des points de $U$, le théorème~\ref{val_medio_2} nous dit que
	\begin{equation}
		\|f(b)-f(a)\|_n\leq \sup_{x\in[a,b]}\|df(x)\|_{\aL(\eR^m,\eR^n)}\|b-a\|_m.
	\end{equation}
	Mais $\| df(x) \|=0$ pour tout $x\in U$, donc ce supremum est nul et $f(b)=f(a)$, ce qui signifie la constance de la fonction.
\end{proof}

%\begin{proof}
%  \begin{itemize}
%  \item Le théorème~\ref{val_medio_2} nous dit que si la différentielle de $f$ est nulle alors $f$ est constante sur chaque segment contenu dans $U$. Cela nous dit que $f$ est constante sur chaque boule contenue dans $U$, donc $f $ est localement constante. Il est possible de démontrer que toute fonction localement constante sur un connexe est constante.
%\item Si toutes les dérivées partielles $\partial_1 f, \ldots, \partial_m f $ existents et sont identiquement nulles sur $U$ alors $f$ est différentiable et sa différentielle est identiquement nulle. On utilise la première partie de la preuve pour conclure.
%  \end{itemize}
%\end{proof}

%+++++++++++++++++++++++++++++++++++++++++++++++++++++++++++++++++++++++++++++++++++++++++++++++++++++++++++++++++++++++++++
\section{Fonctions Lipschitziennes}
%+++++++++++++++++++++++++++++++++++++++++++++++++++++++++++++++++++++++++++++++++++++++++++++++++++++++++++++++++++++++++++


\begin{definition}      \label{DEFooQHVEooDbYKmz}
    Soient \( (E,d_E)\) et \( (F,d_F)\) deux espaces métriques\footnote{Pour rappel, les espaces métriques sont définis par la définition~\ref{DefMVNVFsX} et le théorème~\ref{ThoORdLYUu}; je précise que nous ne supposons pas que \( E\) soit vectoriel; en particulier il peut être un ouvert de \( \eR^n\).}, \( f\colon E\to F\) une application et un réel \( k\) strictement positif. Nous disons que \( f\) est \defe{Lipschitzienne}{Lipschitzienne} de constante $k$ sur \( E\) si pour tout \( x,y\in E\),
    \begin{equation}
        d_F\big( f(x)-f(y) \big)\leq kd_E(x,y).
    \end{equation}
\end{definition}
%TODO : faire la chasse aux endroits où cette définition devrait être référencée.
Soit \( f\) une fonction \( k\)-Lipschitzienne. Si \( y\in \overline{ B(x,\delta)}\) alors \( \| x-y \|\leq\delta\) et donc \( \big\| f(x)-f(y) \big\|\leq k\delta\). Cela signifie que la condition Lipschitz pour s'énoncer en termes de boules fermées par
\begin{equation}    \label{EqDZvtUbn}
    f\big( \overline{ B(x,\delta) } \big)\subset \overline{  B\big( f(x),k\delta \big) }
\end{equation}
tant que \( \overline{ B(x,\delta) } \) est contenue dans le domaine sur lequel \( f\) est Lipschitz.

\begin{proposition}
  Soit  $U$ un ouvert convexe  de $\eR^m$, et soit $f:U\to \eR^n$ une fonction différentiable. La fonction $f$ est Lipschitzienne sur $U$ si et seulement si $df$ est bornée sur $U$.
\end{proposition}
\begin{proof}
	Le fait que l'application différentielle $df$ soit bornée signifie qu'il existe un $M>0$ dans $\eR$ tel que $\|df_a\|_{\aL(\eR^m,\eR^n)}\leq M$, pour tout $a$ dans $U$. Si cela est le cas, alors le théorème~\ref{val_medio_2} et la convexité\footnote{La convexité de $U$ sert à assurer que la droite reliant $a$ à $b$ est contenue dans $U$; c'est ce que nous utilisons dans la démonstration du théorème~\ref{val_medio_2}.} de $U$ impliquent évidemment que $f$ est de Lipschitz de constante plus petite ou égale à $M$.

	Inversement, si $f$ est Lipschitz de constante $k$, alors pour tout $a$ dans $U$ et $u$ dans $\eR^m$ on a
	\[
		\left\|\frac{f(a+tu)-f(a)}{t}\right\|_n\leq k \|u\|_m,
	\]
	En passant à la limite pour $t\to 0$ on a
	\[
		\|\partial_u f(a)\|_n=\|df_a(u)\|_n\leq k \|u\|_m,
	\]
	donc la norme de $df_a$ est majorée par $k$ pour tout $a$ dans $U$.
\end{proof}

Notez cependant qu'une fonction peut être Lipschitzienne sans être différentiable.

\begin{proposition} \label{PropFZgFTEW}
    Une fonction Lipschitzienne \( f\colon \eR\to \eR\) est continue.
\end{proposition}

\begin{proof}
    Nous utilisons la caractérisation de la continuité donnée par le théorème~\ref{ThoESCaraB}. Prouvons donc la continuité en \( a\in \eR\). Pour tout \( x\) nous avons
    \begin{equation}
        \big| f(x)-f(a) \big|\leq k| x-a |.
    \end{equation}
    Si \( \epsilon>0\) est donné, il suffit de prendre \( \delta<\frac{ \epsilon }{ k }\) pour avoir
    \begin{equation}
        \big| f(x)-f(a) \big|\leq k\frac{ \epsilon }{ k }=\epsilon.
    \end{equation}
    Donc \( f\) est continue en \( a\).
\end{proof}

\begin{definition}      \label{DefJSFFooEOCogV}
    Une fonction
    \begin{equation}
        \begin{aligned}
            f\colon \eR^n\times \eR^m&\to \eR^p \\
            (t,y)&\mapsto f(t,y)
        \end{aligned}
    \end{equation}
    est \defe{localement Lipschitz}{Lipschitz!localement} en \( y\) au point \( (t_0,y_0)\) s'il existe des voisinages \( V\) de \( t_0\) et \( W\) de \( y_0\) et un nombre \( k>0\) tels que pour tout \( (t,y)\in V\times W\) on ait
    \begin{equation}
        \big\| f(t_0,y_0)-f(t,y) \big\|\leq k\| y-y_0 \|.
    \end{equation}
    La fonction est localement Lipschitz sur un ouvert \( U\) de \( \eR^n\times \eR^m\) si elle est localement Lipschitz en chaque point de \( U\).
\end{definition}

\begin{normaltext}      \label{NORMooYNRAooBgobcK}
    Autrement dit, une fonction est localement Lipschitzienne en sa deuxième variable lorsque tout point admet un voisinage sur lequel elle est Lipschitzienne.
\end{normaltext}

\begin{proposition} \label{PROPooVZSAooUneOQK}
    Une application Lipschitz\footnote{Définition~\ref{DEFooQHVEooDbYKmz}.} est uniformément continue.
\end{proposition}

\begin{proposition}     \label{PropGIBZooVsIqfY}
    Si \( f\) et \( g\) sont deux fonctions localement Lipschitz alors \( f+g\) l'est.
\end{proposition}

\begin{proof}
    Il s'agit d'un simple calcul avec une majoration standard :
    \begin{subequations}
        \begin{align}
            \| (f+g)(t_0,y_0)-(f+g)(t,y) \|&\leq \|  f(t_0,y_0)-f(t,y)  \|+\| g(t_0,y_0)-g(t,y) \|\\
            &\leq k_f\| y-y_0 \|+k_g\| y-y_0 \|\\
            &=(k_f+k_g)\| y-y_0 \|.
        \end{align}
    \end{subequations}
\end{proof}

\begin{lemma}   \label{LemCFZUooVqZmpc}
    La fonction donné par
    \begin{equation}
        f(t, (x,y) )=xy
    \end{equation}
    est localement Lipschitz en tout point.
\end{lemma}

\begin{proof}
    Nous avons la majoration classique
    \begin{equation}
        | f\big(t,(x_0,y_0)\big)-f\big( t,(x,y) \big) |=| x_0y_0-xy |\leq| x_0y_0-x_0y |+| x_0y-xy |\leq | x_0 || y_0-y |+| y || x_0-x |.
    \end{equation}
    Vu que nous parlons de fonction \emph{localement Lipschitzienne}, nous pouvons majorer \( | y |\) et \( | x_0 |\) par un même nombre \( k\) dans un voisinage de \( (x_0,y_0)\). Cela donne
    \begin{equation}
        | f\big(t,(x_0,y_0)\big)-f\big( t,(x,y) \big) |\leq k\big( | y_0-y |+| x_0-x | \big)\leq \sqrt{2}k\| \begin{pmatrix}
            x_0-x    \\
            y_0-y
        \end{pmatrix}\|.
    \end{equation}
    Nous avons utilisé l'équivalence de norme de la proposition~\ref{PropLJEJooMOWPNi}\ref{ItemABSGooQODmLNi}.
\end{proof}



%++++++++++++++++++++++++++++++++++++++++++++++++++++++++++++++++++++++++++++++++++++++++
\section{Différentielles d'ordre supérieur}		\label{SecDiffOrdSup}
%++++++++++++++++++++++++++++++++++++++++++++++++++++++++++++++++++++++++++++++++++++++++++++++++++++++++++++++++++++++++++++++
\begin{definition}
	Soit $U$ un ouvert de $\eR^m$ et  $f:U\subset\eR^m\to \eR^n$ une fonction. La fonction $f$ est dite \defe{deux fois différentiable}{différentiable!deux fois} au point $a$ dans $U$,  si $f$ est différentiable dans un voisinage de $a$, et sa différentielle $df$ est différentiable au point $a$ en tant que application de $U$ dans $\aL(\eR^m, \eR^n)$.

La fonction $f$ sera dite deux fois différentiable sur l'ensemble $U$ si elle est deux fois différentiable en chaque point de $U$.

\end{definition}

%---------------------------------------------------------------------------------------------------------------------------
\subsection{Identification des espaces d'applications multilinéaires}
%---------------------------------------------------------------------------------------------------------------------------

La différentielle de la différentielle de $f$ est notée
\[
d(df)(a)=d^2f(a),
\]
et est une application de $U$ dans $\aL(\eR^m,\aL(\eR^m, \eR^n) )$. Comme on a vu dans la proposition~\ref{isom_isom}, l'espace $\aL(\eR^m,\aL(\eR^m, \eR^n) )$ est isométriquement isomorphe à l'espace $\aL(\eR^m\times\eR^m, \eR^n )$. On verra comment cette propriété  est utilisé dans l'exemple~\ref{bilin_2diff}.


Soient \( V\) et \( W\) deux espaces vectoriel normés de dimension finie et \( \mO\) un ouvert autour de \( x\in V\). D'une part l'espace des applications linéaires \( \aL(V,W)\) est lui-même un espace vectoriel normé de dimension finie, et on peut identifier \(  \aL\big( V,\aL^{(k)}(V,W) \big)\)\nomenclature[Y]{\( \aL^{(n)}(V,W)\)}{L'espace des applications \( n\)-linéaires \( V^n\to W\)} avec \( \aL^{(k+1)}(V,W)\), ce qui nous permet de dire que la \( k\)\ieme\ différentielle est une application
\begin{equation}
    d^kf\colon \mO\to \aL^{(k)}(V,W).
\end{equation}
Plus précisément, l'identification se fait de la façon suivante : si \( \omega\in \aL\big( V,\aL^{(k)}(V,W) \big)\), alors \( \omega\) vu dans \( \aL^{(k+1)}(V,W)\) est définie par
\begin{equation}
    \omega(u_1,\ldots, u_{k+1})=\omega(u_1)(u_2,\ldots, u_{k+1}).
\end{equation}

Cela étant posé nous pouvons donner les définitions.

%---------------------------------------------------------------------------------------------------------------------------
\subsection{Fonctions différentiables plusieurs fois}
%---------------------------------------------------------------------------------------------------------------------------

\begin{definition}[\cite{ZCKMFRg}]  \label{DefPNjMGqy}
    La fonction \( f\colon \mO\subset V\to W\) est
    \begin{enumerate}
        \item
            de classe \( C^0\) si elle est continue,
        \item
            de classe \( C^1\) si \( df\colon \mO\to \aL(V,W)\) est continue,
        \item
            de classe \( C^k\) si \( d^kf\colon \mO\to \aL^{(k)}(V,W)\) est continue,
        \item
            de classe \(  C^{\infty}\) si \( f\) est dans \( \bigcap_{k=0}^{\infty}C^k(V,W)\).
    \end{enumerate}
\end{definition}
\index{application!différentiable}
\index{application!de classe \( C^k\)}

\begin{definition}
    Un \defe{\( C^k\)-difféomorphisme}{difféomorphisme!de classe $C^k$} est une application inversible de classe \( C^k\) dont l'inverse est également de classe \( C^k\).
\end{definition}

\begin{example}\label{bilin_2diff}
	Soit $B:\eR^m\times \eR^m\to\eR^n$ une application bilinéaire. On définit $f:\eR^m\to\eR^n$ par $f(x)=B(x,x)$. Le lemme~\ref{bilin_diff} nous dit que $B$ est différentiable. Cela implique la différentiabilité de $f$. Pour trouver la différentielle de la fonction $f$, nous écrivons $f=B\circ s$ où $s\colon \eR^m\to \eR^m\times\eR^m$ est l'application $s(x)=(x,x)$. En utilisant la règle de différentiation de fonctions composées,
	\begin{equation}
		df(a)=dB\big( s(a) \big)\circ ds(a).
	\end{equation}
	Mais $ds(a).u=(u,u)$ parce que $s(a+h)-s(a)-(h,h)=0$. Par conséquent,
	\begin{equation}		\label{EqdBsaExp}
		df(a).u=dB\big( s(a) \big)(u,u)=B(u,a)+B(a,u)
	\end{equation}
	où nous avons utilisé la formule du lemme~\ref{bilin_diff}. La formule \eqref{EqdBsaExp} peut être écrite sous la forme compacte
	\begin{equation}
		df(a)=B(\cdot,a)+B(a,\cdot)
	\end{equation}
    La fonction $df(a)$ ainsi écrite est linéaire par rapport à $a$, donc différentiable. En outre elle coïncide avec sa différentielle, comme on a vu dans le lemme \ref{LEMooZSNMooCfjzOB}, au sens que la différentielle de $df$ au point $a$ sera l'application que à chaque $x$ dans $\eR^m$ associe l'application linéaire $B(x,\cdot)+B(\cdot, x)$. On voit bien que $d^2f$ au point $a$ est une application de $\eR^m$ vers l'espace des applications linéaires $\aL(\eR^m, \eR^n)$. On peut utiliser d'autre part l'isomorphisme des espaces $\aL(\eR^m,\aL(\eR^m, \eR^n) )$ et $\aL(\eR^m\times\eR^m, \eR^n )$ et dire que, une fois que $a$ est fixé, l'application $d^2f(a)$ est une application bilinéaire sur $\eR^m\times\eR^m$. On écrit alors $d^2f(a)(x,y)=B(x,y)+B(y,x)$.
\end{example}

%---------------------------------------------------------------------------------------------------------------------------
\subsection{Différentielle seconde, fonction de classe \texorpdfstring{$ C^2$}{C2}}
%---------------------------------------------------------------------------------------------------------------------------

Une condition nécessaire et suffisante pour l'existence de la différentielle seconde est la suivante
\begin{proposition}
   Soit $U$ un ouvert de $\eR^m$ et  $f:U\subset\eR^m\to \eR^n$ une fonction. La fonction $f$ est deux fois différentiable au point $a$ si et seulement si les dérivées partielles $\partial_1 f, \ldots, \partial_m f $ sont différentiables en $a$.
\end{proposition}
Cela veut dire, en particulier, que $f$ est deux fois différentiable si et seulement si ses dérivées partielles secondes, $\partial_i\partial_j f$, pour tout couple d'indices $i,j$  dans $\{1,\ldots, m\}$, existent et sont continues. Pour les différentielles d'ordre supérieur on a la proposition suivante.

La différentielle seconde dans l'exemple ~\ref{bilin_2diff} est symétrique, c'est-à-dire que $d^2f(a)(x_1,x_2)=d^2f(a)(x_2,x_1)$. En fait toute différentielle seconde est symétrique.


\begin{theorem}[Schwarz]\label{Schwarz}
 Soit $U$ un ouvert de $\eR^m$ et  $f:U\subset\eR^m\to \eR^n$ une fonction de classe $\mathcal{C}^2$. Alors, pour tout couple $i,j$ d'indices dans $\{1,\ldots, m\}$ et pour tout point $a$ dans $U$, on a
\[
\frac{\partial^2 f}{\partial  x_i\partial x_j}(a)=\frac{\partial^2 f}{\partial  x_j\partial x_i}(a).
\]
\end{theorem}
\begin{proof}
  Pour simplifier nous nous limitons ici au cas $m=2$. Soit $(h,g)$ un vecteur fixé dans $\eR^2$. Pour tout  $v=(x,y)$ dans $\eR^2$ on note
  \begin{equation}
    \begin{array}{c}
      \Delta_h f(v)=f(v+he_1) -f(v) = f(x+h,y)-f(x,y),\\
      \Delta_g f(v)=f(v+ge_2) -f(v) = f(x,y+g)-f(x,y),\\
    \end{array}
  \end{equation}
Nous avons
\begin{equation}
  \begin{array}{c}
   \Delta_g   \Delta_h f(v)=\left(f(x+h,y+g)-f(x,y+g)\right)-\left(f(x+h,y)-f(x,y)\right),\\
   \Delta_h   \Delta_g f(v)=\left(f(x+h,y+g)-f(x+h,y)\right)-\left(f(x,y+g)-f(x,y)\right),
  \end{array}
\end{equation}
donc,
\begin{equation}
  \frac{1}{g} \Delta_g  \left(\frac{1}{h} \Delta_h f(v)\right) = \frac{1}{h} \Delta_h \left(\frac{1}{g} \Delta_g f(v)\right).
\end{equation}
On utilise alors le théorème des accroissements finis~\ref{ThoAccFinis}
\begin{equation}
\frac{1}{h} \Delta_h f(v)=\frac{1}{h}\big(f(x+h,y)-f(x,y)\big)=\frac{1}{h}\partial_1f(x+t_1h,y )h=\partial_1f(x+t_1h, y),
\end{equation}
pour un certain $t_1$ dans $]0,1[$. De même on obtient
\[
\frac{1}{g} \Delta_g f(v)= \partial_2 f(x, y+t_2g),
\]
pour un certain $t_2$ dans $]0,1[$. Alors
 \begin{equation}
  \frac{1}{g} \Delta_g  \big(\partial_1f(x+t_1h, y)\big) = \frac{1}{h} \Delta_h \big(\partial_2 f(x, y+t_2g)\big).
\end{equation}
En appliquant encore une fois le théorème des accroissements finis on a
 \begin{equation}
  \partial_2\partial_1f(x+t_1h, y+s_1g) = \partial_1\partial_2 f(x+s_2h, y+t_2g).
\end{equation}
Il suffit maintenant de passer à la limite pour $(h,g) \to (0,0)$ et de se souvenir du fait que $f$ est $\mathcal{C}^2$ seulement si ses dérivées partielles secondes sont continues pour avoir $\partial_2\partial_1f(v)=\partial_1\partial_2 f(v)$.
\end{proof}
Si $f$ est deux fois différentiable $d^2f(a)$ est l'application bilinéaire associée avec la matrice symétrique
\begin{equation}
 H_f(a)= \begin{pmatrix}
    \partial^2_1f(a)& \ldots& \partial_1\partial_m f(a)\\
    \vdots& \ddots& \vdots\\
    \partial_1\partial_m f(a)&\ldots&\partial^2_1f(a),
  \end{pmatrix}
\end{equation}
Cette matrice est dite la matrice \defe{hessienne}{hessienne} de $f$.

\begin{example}
  Montrons qu'il n'existe pas de fonctions $f$ de classe $\mathcal{C}^2$ telles que
  \begin{subequations}
      \begin{numcases}{}
  \partial_xf(x,y)= 5\sin x\\
  \partial_y(x,y)=6x+y.
      \end{numcases}
  \end{subequations}
  Ceci est vite fait en appliquant le théorème de Schwarz,~\ref{Schwarz}; ce que nous trouvons est
\[
\partial_y (\partial_xf)= 0\neq \partial_x(\partial_yf)= 6.
\]
Donc, l'existence d'une fonction $f$ de classe $\mathcal{C}^2$ telle que $\partial_x(x,y)= 5\sin x$ et $\partial_yf(x,y)=6x+y$ serait en contradiction avec le théorème.
\end{example}

Soit une fonction de classe \( C^2\) \( f\colon V\to \eR\) où \( V\) est un espace vectoriel de dimension \( n<\infty\). Nous avons
\begin{subequations}
    \begin{align}
        f&\colon V\to \eR\\
        df&\colon V\to \aL(V,\eR)\\
        d^2f&\colon V\to \aL\Big( V,\aL(V,\eR) \Big),
    \end{align}
\end{subequations}
avec, en suivant les différentes formules du lemme~\ref{LemdfaSurLesPartielles},
\begin{equation}
        df_a(u)=\Dsdd{ f(v+tu) }{t}{0}
\end{equation}
et
\begin{equation}
    (d^2f)_a(u)=\Dsdd{ df_{v+tu} }{t}{0}
\end{equation}
pour tout \( a,u\in V\). Notons que dans le deuxième cas, il s'agit d'une limite dans \( \aL(V,\eR)\). Si \( \dim(V)=n\), alors \( \dim\aL(V,\eR)=n\) et avec un choix de base, nous pouvons trouver une matrice \( n\times n\) pour \( (d^2f)_a\).

Soit une base \( \{ e_i \}\) de \( V\) et la base duale \( \{ e_i^* \}\) de \( \aL(V,\eR)\). Nous allons chercher la matrice de \( (d^2f)_a\) pour ces bases. L'élément de matrice
\begin{equation}
    \big[ (d^2f)_a \big]_{ij}
\end{equation}
est la composante \( e_j^*\) de \( (d^2f)_a\) appliqué à \( e_i\). Trouver cette composante \( e_j^*\) revient à appliquer l'élément \( (d^2f)_ae_i\) de \( \aL(V,\eR)\) à \( e_j\). Le calcul est donc :
\begin{subequations}
    \begin{align}
        \big[ (d^2f)_{a} \big]_{ij}&=\big( (d^2f)_ae_i \big)(e_j)\\
        &=\Dsdd{ df_{a+te_i}(e_j) }{t}{0}       \label{SUBEQooDRZFooAuuaad}\\
        &=\Dsdd{    \Dsdd{ f(a+te_i+se_j) }{s}{0}    }{t}{0}\\
        &=\frac{ \partial^2f }{ \partial x_i\partial x_j }(a).
    \end{align}
\end{subequations}
Attention : le passage à \eqref{SUBEQooDRZFooAuuaad} n'est pas une trivialité. Le fait est que si \( t\mapsto A(t)\) est une application continue \( \eR\to \aL(V,\eR)\) alors
\begin{equation}
    \lim_{t\to 0} \big( A(t)v \big)=\big( \lim_{t\to 0} A(t) \big)v.
\end{equation}

Donc la matrice de \( d^2f  \) est la matrice des dérivées secondes. Il s'agit d'une matrice symétrique par le théorème de Schwarz~\ref{Schwarz}.

\begin{normaltext}      \label{NORMooZAOEooGqjpLH}
    Si \( a\in v\), nous pouvons aussi voir \( (d^2f)_a\) comme une forme bilinéaire sur \( V\) grâce à la proposition~\ref{isom_isom}. Si \( u,v\in V\) nous notons
    \begin{equation}
        (d^2f)_a(u,v)=(d^2f)_a(u)v.
    \end{equation}
    À droite, il s'agit de la définition réelle de \( d^2f\) sans abus de notations, et à gauche, il s'agit d'une notation. Cette application bilinéaire \( (d^2f)_a\in \aL^{(2)}(V,\eR)\) a pour matrice symétrique la matrice des dérivées secondes calculées en \( a\).
\end{normaltext}

\begin{example} \label{ExZHZYcNH}
    Voyons comment la différentielle seconde fonctionne entre deux espaces vectoriels. Soient deux espaces vectoriels de dimension finie \( V\) et \( W\). Pour que les choses soient claires, nous avons :
    \begin{subequations}
        \begin{align}
            f&\colon V\to W\\
            df&\colon V\to \aL(V,W)\\
            d^2f&\colon V\to \aL\Big( V,\aL(V,W) \Big).
        \end{align}
    \end{subequations}
    Si \( a\in V\), alors \( (d^2f)_a\) est une application \( V\to \aL(V,W)\). Il faut donc l'appliquer à \( u\in V\) et ensuite à \( v\in V\) pour obtenir un élément de \( W\) :
    \begin{subequations}
        \begin{align}
            (d^2f)_a(u)v&=\Dsdd{ df_{a+tu} }{t}{0}v\\
            &=\Dsdd{ df_{a+tu}(v) }{t}{0}\\
            &=\Dsdd{ \Dsdd{ f(a+tu+sv) }{s}{0} }{t}{0}\\
            &=\frac{ \partial^2f }{ \partial u\partial v }(a).
        \end{align}
    \end{subequations}


    Par conséquent nous voyons
    \begin{equation}\label{EqQHINNtD}
        \begin{aligned}
            d^2f\colon V&\to \aL^{(2)}(V,W) \\
            d^2f_a(u,v)&=\frac{ \partial^2f  }{ \partial u\partial v }(a).
        \end{aligned}
    \end{equation}

    Dans le cas d'une fonction \( f\colon \eR\to \eR\), nous avons une seule direction et par linéarité de \eqref{EqQHINNtD} par rapport à \( u\) et \( v\), nous avons
    \begin{equation}        \label{EQooSOCGooIiNGmG}
        d^2f_a(u,v)=f''(a)uv
    \end{equation}
    où les produits sont des produits usuels dans \( \eR\) et \( f''\) est la dérivée seconde usuelle.
\end{example}

Tout ceci est un peu résumé dans la proposition suivante.
\begin{proposition}     \label{PROPooFWZYooUQwzjW}
    Soit une fonction \( f\colon \eR^n\to \eR\) de classe \( C^2\). Alors en désignant par \( H_af\) sa matrice hessienne au point \( a\) nous avons
    \begin{equation}
        (d^2f)_a(u,v)=\frac{ \partial^2f }{ \partial u\partial v }(a)=\langle (H_af)u, v\rangle
    \end{equation}
    pour tout \( u,v\in \eR^n\).
\end{proposition}

\begin{proof}
    La première égalité est l'équation \eqref{EQooSOCGooIiNGmG} déjà faite. Pour la seconde, il faut se rappeler du lien entre dérivée partielle et dérivée directionnelle, donné en le lemme~\ref{LemdfaSurLesPartielles}. En particulier ici nous avons
    \begin{equation}
        \frac{ \partial^2f }{ \partial u\partial v }=\sum_{kl}\frac{ \partial^2f }{ \partial x_k\partial x_l  }(a)u_kv_l=\langle (H_af)u, v\rangle .
    \end{equation}
\end{proof}

En particulier, la matrice hessienne \( H_af\) est symétrique et donc diagonalisable (théorème spectral~\ref{ThoeTMXla}). Si \( e_i\) est un vecteur propre unitaire pour la valeur propre \( \lambda_i\) nous avons
\begin{equation}
    (d^2f)_a(e_i,e_i)=\langle (H_af)e_i, e_i\rangle =\lambda_i\langle e_i, e_i\rangle =\lambda.
\end{equation}

Enfin pour celles qui aiment les notations matricielles de tout poil, il y a cette façon-ci d'écrire :
\begin{equation}
    (d^2f)_a(\alpha,\beta)=\begin{pmatrix}
        \alpha    &   \beta
    \end{pmatrix}\begin{pmatrix}
        \partial^2_xf(a)    &   \partial^2_{xy}f(a)    \\
        \partial^2_{xy}f(a)    &   \partial^2_yf(a)
    \end{pmatrix}\begin{pmatrix}
        \alpha    \\
        \beta
    \end{pmatrix}.
\end{equation}

%---------------------------------------------------------------------------------------------------------------------------
\subsection{Ordre supérieur}
%---------------------------------------------------------------------------------------------------------------------------

Intuitivement, une fonction est de classe \( C^p\) si elle est \( p\) fois continûment différentiable. Nous posons la définition suivante.

\begin{definition}
    Une fonction \( f\colon E\to F\) est \defe{de classe \( C^0\)}{classe \( C^0\)} si elle est continue.

    Nous disons que la fonction \( f\colon E\to F\) est \defe{de classe \( C^p\)}{classe \( C^p\)} si elle est différentiable et si sa différentielle \( df\colon E\to \aL(E,F)\) est continue.
\end{definition}

Ce qui est terrible avec les différentielles d'ordre supérieurs, c'est l'emboîtement des espaces. Pour une fonction \( f\colon E\to F\), nous allons souvent poser
\begin{subequations}
    \begin{align}
        V_0&=F\\
        V_{k+1}&=\aL(E,V_k),
    \end{align}
\end{subequations}
de telle sorte à avoir
\begin{equation}
    df\colon E\to \aL(E,F)=V_1
\end{equation}
et 
\begin{equation}
    d^2f\colon E\to \aL(E,V_1)=V_2,
\end{equation}
ce qui donne en général
\begin{equation}
    d^kf\colon E\to \aL(E,v_{k-1})=V_k.
\end{equation}

La proposition suivante lie une bonne fois pour toute dérivée et différentielle dans le cadre de fonctions \( \eR\to \eR\).
\begin{proposition}[\cite{MonCerveau}]      \label{PROPooCNDHooKRwils}
    Une fonction \( f\colon \eR\to \eR\) est de classe \( C^p\) si et seulement si elle est \( p\) fois continûment dérivable.
\end{proposition}

\begin{proof}
    Nous commençons par poser un certain nombre de notations. Comme souvent nous posons \( V_0=\eR\) et
    \begin{equation}
        V_{k+1}=\aL(\eR,V_k).
    \end{equation}
    De plus nous considérons \( M_1\in \aL(\eR,\eR)\) donnée par \( M_1(t)=t\). Et par récurrence
    \begin{equation}
        M_{k+1}(t)=tM_{k}.
    \end{equation}
    Nous avons \( M_1\in V_1\) et \( M_k\colon \eR\to V_{k-1}\), c'est-à-dire \( M_k\in V_k\).

    Les formules que nous allons prouver sont que d'une part,
    \begin{equation}
        df_a=f'(a)M_1.
    \end{equation}
    et que d'autre part, plus généralement,
    \begin{equation}
        (d^kf)_a=f^{(k)}(a)M_k.
    \end{equation}

    En plusieurs parties et par récurrence.
    \begin{subproof}
    \item[Si \( f\) est continûment dérivable, alors \( f\) est \( C^1\) ]
        Le candidat différentielle serait \( df_a(h)=hf'(a)\). Vérifions cela directement dans la définition :
        \begin{equation}        \label{EQooCPWKooWdgbED}
            \frac{ f(a+h)-f(a)-f'(a)h }{ \| h \| }=\frac{ f(a+h)-f(a) }{ \| h \| }-1_h'f(a).
        \end{equation}
        où nous avons noté \( 1_h\) le vecteur unité dans la direction de \( h\), c'est-à-dire \( 1_h=h/\| h \|\). Vu que \( h\in \eR\), c'est simplement
        \begin{equation}
            1_h=\begin{cases}
                1    &   \text{si } h>0\\
                -1    &    \text{si }h<0
            \end{cases}
        \end{equation}
        et nous ne définissons pas \( 1_h\) si \( h=0\).
        
        C'est le moment de prendre la limite de \eqref{EQooCPWKooWdgbED} pour \( h\to 0^+\) et \( h\to 0^-\) séparément. Lorsque \( h\to 0^+\), nous avons \( \| h \|=h\) et \( 1_h=h\). Vu que \( f\) est supposée dérivable, la limite du quotient existe et vaut \( f'(a)\). Donc le tout a une limite nulle :
        \begin{equation}       
            \lim_{h\to 0^+} \frac{ f(a+h)-f(a)-f'(a)h }{ \| h \| }=\lim_{h\to 0^+}\frac{ f(a+h)-f(a) }{  h  }-'f(a)=0.
        \end{equation}
        En ce qui concerne la limite \( h\to 0^-\), nous avons \( \| h \|=-h\) et \( 1_h=-1\), et à nouveau une limite nulle. La proposition \ref{PROPooGDDJooDCmydE} nous permet alors de conclure que la limite existe et est nulle. Les limites à gauche et à droite étant nulles, la limite existe et est nulle par la proposition \ref{PROPooGDDJooDCmydE}.

    \item[Si \( f^{(p)}\) est continue alors \( d^pf\) aussi]
        Nous passons à la récurrence de notre preuve. Sous l'hypothèse que \( f^{(p)}\) existe et est continue, nous allons montrer que \( d^pf\) existe, est continue et vaut
        \begin{equation}
            (d^pf)_a=f^{(p)}(a)M_p.
        \end{equation}
        Soit \( k<p\) tel que ce soit bon (pour \( k=1\) c'est déjà fait). Nous avons \( (d^kf)_a=f^{(k)}(a)M_k\), et pour prouver que \( (d^{k+1}f)_a=f^{(k+1)}(a)M_{k+1}\) nous l'y mettons dans la définition de la différentielle. Nous avons :
        \begin{equation}
            \frac{ (d^kf)_{a+h}-(d^kf)_a-f^{(k+1)}(a)M_{k+1}(h) }{ \| h \| }=\frac{ f^{(k)}(a+h)M_k-f^{(k)}(a)M_k-hf^{(k+1)}(a)M_k }{ \| h \| }.
        \end{equation}
        La limite \( h\to 0\) est une limite dans \( V_k\), et elle se traite comme précédemment. Elle vaut zéro parce que \( f^{(k+1)}\) est la dérivée de \( f^{(k)}\). Cela justifie les faits que \( d^kf\) est différentiable en \( a\) et que la différentielle est donné par la formule voulue.

        Par hypothèse, \( k+1\leq p\), donc \( f^{(k+1)}\) est continue. Par conséquent l'application \( a\mapsto f^{(k+1)}(a)M_{k+1}\) est continue.

    \item[Si \( f\) est de classe \( C^1\) alors \( f'\) existe et est continue]
        Dire que \( f\) est de classe \( C^1\) revient à dire que la différentielle \( df\colon \eR\to \aL(\eR,\eR)\) existe et est continue. Soyons conscient que \( df_a(\epsilon)=\epsilon df_a(1)\) et calculons
        \begin{equation}
            \frac{ f(a+\epsilon)-f(a)-df_a(\epsilon) }{ \epsilon }=\frac{ f(a+\epsilon)-f(a) }{ \epsilon }-df_a(1).
        \end{equation}
        La définition de la différentielle est que la limite de cela pour \( \epsilon\to 0\) soit nulle. Cela implique que la limite suivante existe et vaut
        \begin{equation}
            \lim_{\epsilon\to 0}\frac{ f(a+\epsilon)-f(a) }{ \epsilon }=df_a(1).
        \end{equation}
        Nous avons prouvé que \( f'(a)=df_a(1)\).

        La fonction \( a\mapsto df_a\) est continue. Pouvons-nous en déduire que \( f'\) est continue ? Nous avons
        \begin{equation}
            f'=ev_1\circ df
        \end{equation}
        où \( ev_1\) est l'application d'évaluation dont le lemme \ref{LEMooWFNXooLyTyyX} a déjà donné la continuité. Donc \( f'\) est continue comme composée d'applications continues.
    \item[\( f\) est \( C^p\). Récurrence]
        Nous supposons que \( f\) est de classe \( C^p\), et nous allons montrer par récurrence que \( f^{(k)}\) existe et est continue pour tout \( k\leq p\). Posons exactement l'énoncé de notre récurrence.

        Pour \( k=1\) c'est fait. Nous supposons que la formule soit correcte pour un certain \( k\leq p\) et nous y allons pour \( k+1\). Nous avons
        \begin{subequations}        \label{SUBEQSooUPLAooQhueCl}
            \begin{align}
            \frac{ (d^kf)(a+h)-(d^kf)(a)-f^{(k+1)}(a)M_{k+1}(h) }{ \| h \| }&=\frac{ \big[ f^{(k)}(a+h)-f^k(a)-hf^{(k+1)}(a) \big]M_k  }{ \| h \| }\\
                &=\big[ \frac{ f^{(k)}(a+h)-f^{(k)}(a) }{ \| h \| }-1_hf^{(k+1)}(a) \big]M_k.
            \end{align}
        \end{subequations}
        où nous avons aussi tenu compte que \( M_{k+1}(h)=hM_k\).

        C'est le moment de calculer séparément les limites \( h\to 0^+\) et \( h\to 0^-\). Cela fonctionne comme toutes les autres fois.
    \end{subproof}
\end{proof}

Soit une fonction \( f\colon \eR^n\to \eR\) différentiable \( l\) fois. L'application
\begin{equation}
    d^lf\colon \eR^n\to \aL\Big( \eR^n,\aL\big( \eR^n,\aL(\ldots \big) \Big)
\end{equation}
au point \( x\) appliquée à \( v^{(1)}\) appliquée au point \( v^{(2)}\), \ldots, appliquée à \( v^{(l)}\) est notée
\begin{equation}        \label{EQooITOLooQllUlJ}
    (d^lf)_x(v^{(1)},\ldots ,v^{(l)})\in \eR.
\end{equation}

\begin{proposition}     \label{PROPooQKZIooXTvkIr}
    Soit une fonction \( f\colon \eR^n\to \eR\) différentiable \( l\) fois. Avec la notation \eqref{EQooITOLooQllUlJ} nous avons
    \begin{equation}
        (d^lf)_x(v^{(1)},\ldots v^{(l)})=\sum_{k_1,\ldots, k_l}v^{(1)}_{k_1}\ldots v_{k_l}^{(l)}\frac{ \partial^lf }{ \partial x_{k_1}\ldots \partial x_{k_l} }(x).
    \end{equation}
\end{proposition}

\begin{proof}
    La preuve se fait par récurrence sur \( l\), en sachant que la formule est déjà vraie pour \( l=1\) et \( l=2\). Si la formule est valable pour \( l\), nous avons
    \begin{subequations}
        \begin{align}
            (d^{l+1}f)_x(v^{(1)},\ldots, v^{(l+1)})&=\Dsdd{ (d^l)_{x+tv^{(l+1)}}(v^{(1)},\ldots, v^{(l)}) }{t}{0}\\
            &=\sum_{k_1\ldots k_l}v_{k_1}^{(1)}\ldots v_{k_l}^{(l)}\Dsdd{   \frac{ \partial^lf }{ \partial x_1\ldots \partial x_l }(x+tv^{l+1})   }{t}{0}\\
            &=\sum_{k_1\ldots k_l}v_{k_1}^{(1)}\ldots v_{k_l}^{(l)}\sum_i\frac{ \partial  }{ \partial x_i }\frac{ \partial^lf }{ \partial x_{k_1}\ldots \partial x_k }(x).
        \end{align}
    \end{subequations}
    Cela donne le résultat attendu.
\end{proof}

\begin{normaltext}
    La formule de la proposition~\ref{PROPooQKZIooXTvkIr} nous permet d'écrire de jolies formules comme
    \begin{equation}        \label{EQooXRWWooMoKoOB}
        (d^3f)_x(h,h,h)=\sum_{ijk}h_ih_jh_k(\partial^3_{ijk}f)(x).
    \end{equation}
\end{normaltext}

\begin{proposition}[Dérivées partielles et fonctions \( C^k\)] \label{PropDYKooHvrfGw}
    Soit $U$ un ouvert de $\eR^m$ et  $f:U\subset\eR^m\to \eR^n$. La fonction $f$ est de classe $C^k$ si et seulement si les dérivées partielles $\partial_1 f, \ldots, \partial_m f $ existent et sont de classe $C^{k}$.
\end{proposition}
% TODO : une preuve serait importante.

\begin{proposition}[\cite{MonCerveau}]
    Soient des espaces vectoriels \( E\),  \( V\) et \( W\) de dimension fine, et une fonction \( f\colon E\to V\) de classe \( C^p\). Si \( \varphi\colon V\to W\) est linéaire, alors
    \begin{equation}
        \varphi\circ f\colon E\to W
    \end{equation}
    est de classe \( C^p\).
\end{proposition}

\begin{proof}
    En utilisant le théorème de différentiation de fonctions composées \ref{THOooIHPIooIUyPaf},
    \begin{equation}
        f(\varphi\circ f)_a(u)=d\varphi_{f(a)}df_a(u),
    \end{equation}
    et donc, parce que \( \varphi\) est linéaire,
    \begin{equation}
        d(\varphi\circ f)_a=\varphi\circ df_a.
    \end{equation}
    Nous pouvons exprimer cela de façon un peu différente en posant \( \varphi_1\colon \aL(E,V)\to \aL(E,W)\),
    \begin{equation}
        \varphi_1(\alpha)(a)=(\varphi\circ \alpha)(a).
    \end{equation}
    Cela nous permet d'écrire \( \varphi\circ df_a=(\varphi_1\circ df)(a)\) et donc
    \begin{equation}        \label{EQooUJPWooTzgSJx}
        d(\varphi\circ f)=\varphi_1\circ df
    \end{equation}
    où \( \varphi_1\) est encore une application linéaire. Une récurrence semble possible. Nous posons \( V_0=V\) et \( W_0=W\) puis
    \begin{subequations}
        \begin{align}
            V_{k+1}&=\aL(E,V_k)\\
            W_{k+1}&=\aL(E,W_k)
        \end{align}
    \end{subequations}
    et
    \begin{equation}
        \begin{aligned}
            \varphi_k\colon \aL(E,V_{k-1})&\to \aL(E,W_{k-1}) \\
            g&\mapsto \varphi_{k-1}\circ g.
        \end{aligned}
    \end{equation}
    Avec tout cela, nous prétendons que \( d^k(\varphi\circ f)=\varphi_k\circ d^kf\) avec \( \varphi_k\) linéaire.

    \begin{subproof}
        \item[\( \varphi_k\) est linéaire]
            Soient \( \alpha_1,\alpha_2\in \aL(E,V_{k-1})\), ainsi que \( \lambda,\mu\in \eK\). Nous avons, en utilisant la linéarité de \( \varphi_{k-1}\) :
            \begin{subequations}
                \begin{align}
                    \varphi_k(\lambda\alpha_1+\mu\alpha_2)(a)&=\varphi_{k-1}\big( (\lambda\alpha_1+\mu\alpha_2)(a) \big)\\
                    &=\varphi_{k-1}\big(\lambda \alpha_1(a)\big)+\mu\varphi_{k-1}\big( \alpha_2(a) \big)\\
                    &=\lambda\varphi_k(\alpha_1)a+\mu\varphi_k(\alpha_2)(a).
                \end{align}
            \end{subequations}
            Donc \( \varphi_k\) est linéaire pour tout \( k\).
        \item[La relation]
            La relation 
            \begin{equation}
                d^k(\varphi\circ f)=\varphi_k\circ d^kf
            \end{equation}
            se démontre par récurrence, chaque pas étant justifié de la même manière que \eqref{EQooUJPWooTzgSJx}.
    \end{subproof}
\end{proof}

