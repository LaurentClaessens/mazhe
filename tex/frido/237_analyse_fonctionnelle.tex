% This is part of Mes notes de mathématique
% Copyright (c) 2011-2022
%   Laurent Claessens
% See the file fdl-1.3.txt for copying conditions.


%--------------------------------------------------------------------------------------------------------------------------- 
\subsection{Espace de Banach réflexif}
%---------------------------------------------------------------------------------------------------------------------------

\begin{definition}[\cite{BIBooFDGQooYferue}]        \label{DEFooZGLDooRRarRj}
Soit un espace de Banach\footnote{Définition \ref{DefVKuyYpQ}.} \( E\), et son dual \( E'\) Pour \( \varphi\in E'\) nous notons
\begin{equation}
    \begin{aligned}
        f_{\varphi}\colon E & \to \eC             \\
        f_{\varphi}(x)      & \mapsto \varphi(x).
    \end{aligned}
\end{equation}
La \defe{topologie faible}{topologie faible} sur \( E\), notée \( \sigma(E,E')\) est la plus faible topologie sur \( E\) rendant toutes les applications \( f_{\varphi}\) continues.

L'existence d'une telle topologie est la proposition \ref{PROPooGOEVooZBAOQh}
\end{definition}

\begin{proposition}[\cite{BIBooFDGQooYferue}]       \label{PROPooYARHooOpmztY}
Un convexe dans un espace de Banach est fermé si et seulement si il est faiblement fermé.
\end{proposition}

\begin{proposition}[\cite{BIBooFDGQooYferue}]       \label{PROPooPVVYooMZjQSq}
Une suite bornée dans un espace de Banach réflexif contient une sous-suite faiblement convergente.
\end{proposition}

\begin{theorem}[Kakutami\cite{BIBooFDGQooYferue}]       \label{THOooTFIHooPQjVAr}
Un espace de Banach est réflexif si et seulement si la boule fermée \( \overline{ B(0,1) }\) est compacte pour la topologie faible\footnote{Définition \ref{DEFooZGLDooRRarRj}.}.
\end{theorem}

\begin{proposition}[\cite{BIBooFDGQooYferue}]       \label{PROPooBBNBooGcXDRH}
Un espace de Banach est réflexif si et seulement si son dual est réflexif.
\end{proposition}

%--------------------------------------------------------------------------------------------------------------------------- 
\subsection{Espaces \texorpdfstring{\(  L^{\infty}\)}{Linfinity}}
%---------------------------------------------------------------------------------------------------------------------------

\begin{lemma}[\cite{BIBooFDGQooYferue}]        \label{LEMooMSYAooGEMgoc}
Si \( (\Omega,\tribA,\mu)\) est un espace mesuré \( \sigma\)-fini, l'espace \( L^1(\Omega,\tribA,\mu)\) n'est pas réflexif.
\end{lemma}

\begin{lemma}[\cite{BIBooFDGQooYferue}]     \label{LEMooUSXTooFvpsVd}
Si \( (\Omega,\tribA,\mu)\) est un espace mesuré \( \sigma\)-fini, alors \( L^{\infty}(\Omega,\tribA,\mu)\) n'est pas réflexif\footnote{Définition \ref{PROPooMAQSooCGFBBM}.}.
\end{lemma}

\begin{proof}
Supposons que \( L^{\infty}(\Omega,\tribA,\mu)\) est réflexif. Le théorème de représentation de Riesz \ref{ThoLPQPooPWBXuv}\ref{ITEMooCQGJooOWzjoV} dit que \( (L^1)'\) est en bijection linéaire isométrique avec \( L^{\infty}\); le lemme \ref{PROPooVRQKooLdmajh} dit alors que \( (L^1)'\) est réflexif\footnote{Dans de nombreuses références, par exemple\cite{BIBooFDGQooYferue}, il est simplement dit que \( (L^1)'=L^{\infty}\). C'est un abus de notation qui permet de se passer du lemme \ref{PROPooVRQKooLdmajh}.}. La proposition \ref{PROPooBBNBooGcXDRH} dit alors que \( L^1\) est réflexif.

Or le lemme \ref{LEMooMSYAooGEMgoc} dit que \( L^1\) n'est pas réflexif.
\end{proof}



La proposition suivante est souvent présentée en disant que l'inclusion \( L^1\subset (L^{\infty})\) est stricte, ou, pire, en disant que \( (L^{\infty})'\) est strictement plus grand que \( L^1\). Cette façon de dire est un gros abus de language. D'abord \( L^1\) n'est même pas inclus à \( L^{\infty}\); ce sont deux ensembles qui n'ont rien à voir. Ensuite, ce que signifie réellement cette proposition est seulement que la première injection \( L^1\to (L^{\infty})'\) qui nous tombe sous la main (celle du théorème de représentation de Riesz) n'est pas surjective.
\begin{proposition}     \label{PROPooXXRQooNSBZOi}
Soit \( g\in L^1(\eR^d)\). L'application
\begin{equation}
    \begin{aligned}
        \Phi_g\colon L^{\infty}(\eR^d) & \to \eC                     \\
        f                              & \mapsto \int_{\eR^d}f\bar g
    \end{aligned}
\end{equation}
est linéaire et bien définie.

L'application
\begin{equation}
    \begin{aligned}
        \Phi\colon L^1(\eR^d) & \to L^{\infty}(\eR^d)' \\
        g                     & \mapsto \Phi_g
    \end{aligned}
\end{equation}
n'est pas surjective.
\end{proposition}

\begin{proof}
Prouvons d'abord que \( \Phi_g\) est bien définie. Vu que \( f\in L^{\infty}\), il existe \( M\in \eR\) tel que \( | f(x) |<M\) sur \( \eR^d\setminus A\) où \( A\) est de mesure nulle dans \( \eR^d\). La fonction \( x\mapsto| f(x)\overline{ g(x) } |\) est donc majorée par la fonction intégrable \( x\mapsto | g(x) |\) qui est intégrable (sur \( \eR^d\setminus A\)). L'intégrabilité de \( f\bar g\) n'est donc pas un problème.

Le fait que \( \Phi\) prenne ses valeurs dans \( (L^{\infty})'\) est le théorème \ref{THOooXMVTooBAbyvr}\ref{ITEMooBFFZooNxoHER}.

Le fait qu'elle ne soit pas surjective est la proposition \ref{PROPooXNRRooUdgFPr}.
\end{proof}
