% This is part of Le Frido and Giulietta
% Copyright (c) 2011-2023, 2025
%   Laurent Claessens, Carlotta Donadello
% See the file fdl-1.3.txt for copying conditions.

% ♡2011-2023 by Laurent Claessens. Copying Art is an act of love. Please copy and share.
% https://copyheart.org/manifesto/


%+++++++++++++++++++++++++++++++++++++++++++++++++++++++++++++++++++++++++++++++++++++++++++++++++++++++++++++++++++++++++++ 
\section{Topologie faible}
%+++++++++++++++++++++++++++++++++++++++++++++++++++++++++++++++++++++++++++++++++++++++++++++++++++++++++++++++++++++++++++

\begin{definition}[\cite{BIBooFDGQooYferue,MonCerveau}]        \label{DEFooZGLDooRRarRj}
	Soit un espace de Banach\footnote{Définition \ref{DefVKuyYpQ}.} sur le corps \( \eF\) (\( =\eR\) ou \( \eC\)) \( E\), et son dual \( E'\). La \defe{topologie faible}{topologie faible} sur \( E\), est la plus petite topologie \( \tau\) pour laquelle
	\begin{equation}
		\big( E,\| . \| \big)'=\big( E,\tau \big)'.
	\end{equation}
	Autrement dit, c'est la topologie de la proposition \ref{PROPooGOEVooZBAOQh} rendant continues toutes les applications de \( E'\).

	Elle sera notée \( \tau_w\) ou \( \sigma(E,E')\).
\end{definition}

\begin{lemma}[\cite{MonCerveau}]	\label{LEMooZPJNooJyaeAC}
	Les ouverts faibles sont ouverts.
\end{lemma}

\begin{proof}
	Nous avons \( \tau_w\subset \tau\) pour toute topologie \( \tau \) telle que \( (E,\tau_N)'=(E,\tau)'\) où \( \tau_N\) est la topologie de la norme. En particulier en posant \( \tau=\tau_N\) on a l'égalité. Donc \( \tau_w\subset\tau_N\).
\end{proof}

Vu que, pour rendre continue une application, il suffit que toutes les images inverses des ouverts de \( \eF\) soient des ouverts, la topologie faible sur \( E\) est également la topologie engendrée\footnote{Proposition \ref{DefTopologieEngendree}.} par les parties \( \varphi^{-1}(V)\) avec \( \varphi\in E'\) et \( V\in \tau_{\eF}\).

\begin{remark}
	Il faut noter que la topologie faible n'est pas une topologie métrique. Cela même si la condition \( A_ix\to Ax\), elle, est métrique vu qu'elle est écrite dans \( E\).

	Dans le cas où \( E\) est de dimension infinie, la topologie faible est réellement différente de la topologie forte. Nous verrons à la sous-section~\ref{subsecaeSywF} que dans le cas des projections sur un espace de Hilbert, l'égalité
	\begin{equation}
		\sum_{i=1}^{\infty}\pr_{u_i}=\id
	\end{equation}
	est vraie pour la topologie faible, mais pas pour la topologie forte.
\end{remark}

\begin{lemma}[\cite{BIBooFDGQooYferue}]
	La topologie faible est Hausdorff\footnote{Espace topologique Hausdorff, définition \ref{DefYFmfjjm}.}.
\end{lemma}

\begin{proof}
	Soient un espace de Banach \( E\) ainsi que \( x_1\neq x_2\) dans \( E\). Les parties \( \{ x_1 \}\) et \( \{ x_2 \}\) vérifient les hypothèses de Hahn-Banach (seconde forme géométrique, \ref{ThoACuKgtW}). Il existe donc une fonctionnelle \( \varphi\colon E\to \eF\) telle que \( \varphi(x_1)\neq \varphi(x_2)\). Avec un peu de bonne volonté, nous supposons que \( \real\big( \varphi(x_1) \big)\neq \real\big( \varphi(x_2) \big)\). Soit \( \alpha\in \eR\) tel que
	\begin{equation}
		\real\big( \varphi(x_1) \big)<\alpha <\real\big( \varphi(x_2) \big).
	\end{equation}
	Posons
	\begin{subequations}
		\begin{align}
			\mO_1 & =\{ x\in E\tq\real\big( \varphi(x) \big)< \alpha \}, \\
			\mO_2 & =\{ x\in E\tq\real\big( \varphi(x) \big)> \alpha \}.
		\end{align}
	\end{subequations}
	Nous avons \( \mO_i=\varphi^{-1}(V_i)\) avec\footnote{Adaptez si \( \eF=\eR\) au lieu de \( \eC\).}
	\begin{equation}
		V_1= \mathopen] -\infty , \alpha \mathclose[+i\eR
	\end{equation}
	et quelque chose du même genre pour \( V_2\). Vu que \( V_1\) et \( V_2\) sont ouverts dans \( \eF\), et que \( \varphi\) est continue pour la topologie faible, les parties \( \mO_i\) sont ouvertes dans \( (E,\tau_w)\).

	Nous avons de plus \( \mO_1\cap\mO_2=\emptyset\) et \( x_i\in \mO_i\), de telle sorte que \( x_1\) et \( x_2\) sont correctement séparés.
\end{proof}

%-------------------------------------------------------
\subsection{Bases de topologie}
%----------------------------------------------------


\begin{lemma}[\cite{BIBooFDGQooYferue}]			\label{LEMooEFVXooIWBBdW}
	Soit un espace de Banach \( E\). Soient \( x_0\in E\), \( \varphi_1,\ldots, \varphi_k\in E'\). La partie
	\begin{equation}
		V(\varphi_1,\ldots, \varphi_k,\epsilon)=\{ x\in E\tq | \varphi_i(x-x_0) |<\epsilon \}
	\end{equation}
	est faiblement ouverte.
\end{lemma}

\begin{proof}
	Remarquez que
	\begin{equation}
		V(\varphi_1,\ldots, \varphi_k,\epsilon)=\bigcap_{i=1}^k\varphi_i^{-1}\{ z\in \eC\tq | z-\varphi_i(x_0) | <\epsilon \}.
	\end{equation}
	Vu que \( \varphi_i^{-1}\) transforme un ouvert en un ouvert, \( V(\varphi_1,\ldots, \varphi_k,\epsilon)\) est une intersection d'ouverts, et donc un ouvert. De plus \( x_0\) est dedans parce que \( \varphi_i(x_0-x_0)=\varphi_i(0)=0 <\epsilon\).
\end{proof}

\begin{lemma}[\cite{BIBooFDGQooYferue}]
	Soient un espace de Banach \( E\) ainsi qu'un élément \( x_0\in E\). Les ensembles
	\begin{equation}
		V(\varphi_1,\ldots, \varphi_k,\epsilon)=\{ x\in E\tq | \varphi_i(x-x_0) |<\epsilon\,\forall i \}
	\end{equation}
	forment une base de topologie en \( x_0\).
\end{lemma}

\begin{proof}
	Soit un ouvert faible \( U\) contenant \( x_0\). Nous devons prouver que \( U\) contient une partie de la forme \( V(\varphi_1,\ldots, \varphi_k,\epsilon)\). C'est le moment d'avoir bien en tête la construction de la topologie engendrée donnée en la proposition \ref{DefTopologieEngendree}.

	La topologie faible est engendrée par les parties de la forme \( \varphi^{-1}(V)\) avec \( \varphi\in E'\) et \( V\in \tau_{\eF}\). La partie \( U\) est une union de parties de la forme
	\begin{equation}        \label{EQooKJEPooRjssME}
		\bigcap_{j=1}^k\varphi_j^{-1}(V_j)
	\end{equation}
	où \( V_j\) est ouvert dans \( \eF\) et \( \varphi_j\in E'\). Vu que \( x_0\in U\), il existe une partie de la forme \eqref{EQooKJEPooRjssME} contenant \( x_0\). Nous la notons \( W\) :
	\begin{equation}
		x_0\in W=\bigcap_{j=1}^k\varphi_j^{-1}(V_j)\subset U.
	\end{equation}
	Vu que \( V_j\) est un ouvert de \( \eF\) contenant \( \varphi_j(x_0)\), il existe \( \delta_j>0\) tel que \( B\big( \varphi_j(x_0),\delta_j \big)\subset V_j\). En prenant \( \epsilon=\min_{j=1,\ldots, k}(\delta_j)\), nous avons \( B\big( \varphi_j(x_0),\epsilon \big)\subset V_j\) pour tout \( j\). En particulier,
	\begin{equation}
		\bigcap_{j=1}^k\varphi_j^{-1}\Big( B\big(\varphi_j(x_0),\epsilon\big) \Big)\subset W.
	\end{equation}
	Nous montrons à présent que \( V(\varphi_1,\ldots, \varphi_k,\epsilon)\subset W\). Pour cela nous considérons \( x\in V(\varphi_1,\ldots, \varphi_k,\epsilon)\) et nous montrons que \( x\in\varphi_j^{-1}\Big( B\big( \varphi_j(x_0),\epsilon \big) \Big)\) pour tout \( j=1,\ldots, k\). Nous avons
	\begin{equation}
		| \varphi_j(x)-\varphi_j(x_0) |=| \varphi_j(x-x_0) |<\epsilon,
	\end{equation}
	et donc bien \( \varphi_j(x)\in B\big( \varphi_k(x_0),\epsilon \big)\). Au final nous avons
	\begin{equation}
		V(\varphi_1,\ldots, \varphi_k,\epsilon)\subset\bigcap_{j=1}^k\varphi_j^{-1}\Big( B\big( \varphi_j(x_0),\epsilon \big) \Big)\subset W\subset U.
	\end{equation}
\end{proof}

\begin{proposition}[\cite{BIBooQLUCooRojlrq}]	\label{PROPooMBOJooQcwyuv}
	Soit un espace de Banach \( E\). Soit \( \varphi_0\in E'\). Une base de la topologie \( \sigma(E,E')\) en \( \varphi_0\) est donnée par les parties
	\begin{equation}		\label{EQooTINIooJulDll}
		V(\varphi_0;x_1,\ldots,x_k,\epsilon)=\{ \varphi\in E'\tq | (\varphi-\varphi_0)(x_i) |<\epsilon\,\forall i=1,\ldots,k \}.
	\end{equation}
	%TODOooDGHLooWrJRDr. Prouver ça.
\end{proposition}

%-------------------------------------------------------
\subsection{Convergence}
%----------------------------------------------------


\begin{proposition}[\cite{BIBooFDGQooYferue,BIBooMTUZooVJxnSj}]     \label{PROPooFJBBooKkRwIp}
	Soient un espace de Banach \( E\) ainsi qu'une suite \( (x_n)\) dans \( E\).
	\begin{enumerate}
		\item       \label{ITEMooDMMTooSBINKN}
		      \( x_n\stackrel{w}{\longrightarrow}x\) si et seulement si \( \varphi(x_n)\to\varphi(x)\) pour tout \( \varphi\in E'\).
		\item       \label{ITEMooFZKXooNqFGUb}
		      \( x_n\stackrel{\| . \|}{\longrightarrow}x\) implique \( x_n\stackrel{w}{\longrightarrow}x\).
		\item       \label{ITEMooXPTSooYPwNgU}
		      Si \( x_n\stackrel{w}{\longrightarrow}x\) alors \( \{ \| x_n \| \}\) est borné et
		      \begin{equation}
			      \| x \|\leq \liminf\| x_n \|.
		      \end{equation}
		\item       \label{ITEMooAFRFooOXdsBy}
		      Si \( x_n\stackrel{w}{\longrightarrow}x\) et \( \varphi_n\stackrel{E'}{\longrightarrow}\varphi\) alors \( \varphi_n(x_n)\to\varphi(x)\).
	\end{enumerate}
\end{proposition}

\begin{proof}
	Point par point.
	\begin{subproof}
		\spitem[Pour \ref{ITEMooDMMTooSBINKN}]
		% -------------------------------------------------------------------------------------------- 
		C'est le lemme \ref{LEMooADPLooYylNsj}.
		\spitem[Pour \ref{ITEMooFZKXooNqFGUb}]
		% -------------------------------------------------------------------------------------------- 
		Nous prouvons que \( \varphi(x_n)\to\varphi(x)\) pour tout \( \varphi\in E'\) :
		\begin{equation}
			| \varphi(x_n)-\varphi(x) |=| \varphi(x_n-x) |\leq \| \varphi \|_{E'}\| x_n-x \|_E\to 0
		\end{equation}
		parce que par hypothèse \( \| x_n-x \|\to 0\).
		\spitem[Pour \ref{ITEMooXPTSooYPwNgU}]
		% -------------------------------------------------------------------------------------------- 

		Vu que \( \varphi(x_n)\stackrel{\eF}{\longrightarrow}\varphi(x)\), l'ensemble \( \{ \varphi(x_n) \}\) est borné dans \( \eF\). Considérons les opérateur suivants:
		\begin{equation}
			\begin{aligned}
				T_n\colon E' & \to \eF               \\
				\varphi      & \mapsto \varphi(x_n).
			\end{aligned}
		\end{equation}
		Cet opérateur a la propriété que \( \| T_n \|_{(E')'}=\| x_n \|_E\); en effet, en utilisant la proposition \ref{PROPooFJPXooWrjbuH},
		\begin{equation}
			\| T_n \|=\sup_{\| \varphi \|=1}| T_n(\varphi) |=\sup_{\| \varphi \|=1}| \varphi(x_n) |=\| x_n \|.
		\end{equation}
		Chacun des \( T_n\) est donc un opérateur linéaire borné. Nous vérifions que la famille \( \{ T_n \}_{n\in \eN}\) satisfait aux hypothèses du théorème de Banach-Steinhaus \ref{THOooJHVNooIDDxyT}. D'abord \( E'\) est un Banach par la proposition \ref{PROPooOVGGooNffWJW}. Ensuite nous venons de voir que chacun de  \(  T_n \) est borné dans \( E'\). Et enfin, pour chaque \( \varphi\in E'\) nous avons
		\begin{equation}
			\sup_{n\in \eN}\| T_n(\varphi) \|=\sup_{n\in \eN}| \varphi(x_n) |<\infty.
		\end{equation}
		Le théorème de Banach-Steinhaus nous assure donc que
		\begin{equation}
			\sup_{n\in \eN}\| x_n \|=\sup_{n\in \eN}\| T_n \|<\infty.
		\end{equation}
		Donc l'ensemble \( \{ \| x_n \| \}\) est borné.

		Pour chaque \( \varphi\in E'\) tel que \( \| \varphi \|\leq 1\) nous avons
		\begin{equation}
			| \varphi(x_n) |_{\eF}\leq \| \varphi \|_{E'}\| x_n \|_E\leq \| x_n \|
		\end{equation}
		Étant donnée l'inégalité \( | \varphi(x_n) |\leq \| x_n \|\) pour tout \( n\) et étant donnée la convergence \( | \varphi(x_n) |\to | \varphi(x) |\), nous avons pour tout \( n\) :
		\begin{equation}
			| \varphi(x) |\leq \| x_n \|.
		\end{equation}
		Cette inégalité valable pour tout \( n \) donne la conclusion :
		\begin{equation}
			| \varphi(x) |\leq  \liminf\| x_n \|.
		\end{equation}


		\spitem[Pour \ref{ITEMooAFRFooOXdsBy}]
		%---------------------------------------------------------------------------------- 

		Nous avons
		\begin{subequations}        \label{SUBEQooPHWLooZUiQLO}
			\begin{align}
				| \varphi_n(x_n)-\varphi(x) | & \leq | \varphi_n(x_n)-\varphi(x_n) |+| \varphi(x_n)-\varphi(x) | \\
				                              & \leq \| x_n \|\| \varphi_n-\varphi \|+| \varphi(x_n-x) |
			\end{align}
		\end{subequations}
		Mais par hypothèse \( \| \varphi_n-\varphi \|\to 0\), par le point \ref{ITEMooXPTSooYPwNgU}, \( \| x_n \|\) est borné, et par le point \ref{ITEMooDMMTooSBINKN}, nous avons \( | \varphi(x_n)-\varphi(x) |\to 0\). Tout ça mis ensemble nous permet de prendre la limite dans \eqref{SUBEQooPHWLooZUiQLO} et de voir que \( \| \varphi_n(x_n)-\varphi(x) \|\to 0\).

	\end{subproof}
\end{proof}


\begin{proposition}
	Si \( E\) est un espace de Banach de dimension finie, alors sa topologie normée est la même que sa topologie faible : \( \tau_w=\tau_{\| . \|}\).
\end{proposition}

\begin{proof}
	Vu que \( E'\) est défini comme étant l'ensemble des formes continues pour la topologie \( \| . \|\), et que \( \tau_w\) est la plus petite topologie pour laquelle tous les éléments de \( E'\) sont continus, nous avons \( \tau_w\subset\tau_{\| . \|}\).

	Pour prouver l'inclusion inverse, nous considérons \( U\in\tau_{\| . \|}\) et nous prouvons que \( U\) est également un ouvert faible en montrant que tout élément de $U$ est inclus dans un ouvert faible contenu dans \( U\).


	Soit \( x_0\in U\) ainsi que \( r>0\) tel que \( B(x_0,r)\subset U\). Nous considérons une base \( \{ e_i \}_{i=1,\ldots, n}\) de \( E\) telle que \( \| e_i \|=1\) pour tout \( i\). Vu que tout élément de \( E\) peut être décomposé de façon unique en \( x=\sum_ix_ie_i\), nous considérons les fonctionnelles linéaires
	\begin{equation}
		\begin{aligned}
			\varphi_j\colon E & \to \eF      \\
			x                 & \mapsto x_j.
		\end{aligned}
	\end{equation}

	C'est le moment de ressortir notre ouvert préféré\footnote{Nous avons un \ldots faible pour lui !} autour de \( x_0\) :
	\begin{equation}
		V(x_0,\varphi_1,\ldots, \varphi_n,\epsilon)=\bigcap_{i=1}^n\varphi_i^{-1}\Big( \{  \| z\in \eC\tq | z-\varphi_i(x_0) | \|<\epsilon  \} \Big).
	\end{equation}
	Supposons que \( x\in V(x_0,\varphi_1,\ldots, \varphi_n,\epsilon)\). Il vérifie \( | \varphi_j(x)-\varphi_j(x_0) |\leq \epsilon\) et donc
	\begin{subequations}
		\begin{align}
			\| x-x_0 \| & =\| \sum_{j=1}^n\varphi_j(x-x_0)e_j \|     \\
			            & \leq \sum_j| \varphi_j(x-x_0) |\| e_j \|   \\
			            & = \sum_j| \varphi_j(x-x_0) |               \\
			            & \leq \sum_j| \varphi_j(x)-\varphi_j(x_0) | \\
			            & \leq n\epsilon.
		\end{align}
	\end{subequations}
	En choisissant \( \epsilon<\frac{ r }{ n }\), nous avons \( x\in B(x_0,r)\) et donc
	\begin{equation}
		V(x_0,\ldots, )\subset B(x_0,r).
	\end{equation}
\end{proof}

\begin{lemma}       \label{LEMooMCYAooGMzbbs}
	Soit un espace de Banach \( E\) sur le corps \( \eF\). Nous notons \( \Fun(E,\eF)\) l'ensemble de toutes les applications de \( E\) vers \( \eF\). Pour chaque \( x\in E\) nous considérons l'application
	\begin{equation}
		\begin{aligned}
			f_x\colon \Fun(E,\eF) & \to \eF            \\
			\omega                & \mapsto \omega(x).
		\end{aligned}
	\end{equation}
	Nous considérons sur \( \Fun(E,\eF)\) la plus petite topologie telle que tous les \( f_x\) soient continues.

	La suite \( (\omega_n)\) dans \( \Fun(E,\eF)\) converge vers  \(\omega\in \Fun(E,\eF)\) si et seulement si \( \omega_n(x)\to\omega(x)\) pour tout \( x\in E\).
\end{lemma}


\begin{lemma}       \label{LEMooWXBVooSjafZr}
	Soit un espace de Banach \( E\) sur le corps \( \eF\) (\( =\eR\) ou \( \eC\)). Nous considérons l'ensemble \( \Fun(E,\eF) \) sur lequel nous mettons la topologie minimale qui rend continue les applications
	\begin{equation}
		\begin{aligned}
			f_x\colon \Fun(E,\eF) & \to \eF            \\
			\omega                & \mapsto \omega(x).
		\end{aligned}
	\end{equation}
	Nous notons \( I=\{ f_x\tq x\in E \} \) et \( \tau_I\) la topologie de \( \Fun(E,\eF)\).

	Si \( (\omega_n)\) est une suite dans \( \Fun(E,\eF)\) nous avons \( \omega_n\stackrel{\tau_I}{\longrightarrow}\omega\) si et seulement si \( \omega_n(x)\to \omega(x)\) pour tout \( x\in E\).
\end{lemma}

\begin{theorem}[Banach-Alaogly-Bourbaki\cite{BIBooFDGQooYferue}]       \label{THOooRECTooEVLHSq}
	Soit un espace de Banach \( E\). La boule unité
	\begin{equation}
		B=\{ \varphi\in E'\tq \| \varphi \|=1 \}
	\end{equation}
	est compacte pour la topologie faible.
\end{theorem}

\begin{proof}
	Nous considérons à nouveau \( \Fun(E, \eF)\) muni de la topologie du lemme \ref{LEMooWXBVooSjafZr}. Nous considérons l'inclusion
	\begin{equation}
		\begin{aligned}
			\Phi\colon \big( E',\| . \| \big) & \to \big( \Fun(E,\eF), \tau_I \big) \\
			\omega                            & \mapsto \omega.
		\end{aligned}
	\end{equation}
	En particulier nous notons \( S=\Phi(E')\) et nous allons prouver que \( \Phi\colon E'\to S\) bijective et continue et que \( \Phi^{-1}\colon S\to E'\) est également continue.
\end{proof}


\begin{proposition}[\cite{BIBooFDGQooYferue}]       \label{PROPooYARHooOpmztY}
	Un convexe dans un espace de Banach est fermé si et seulement si il est faiblement\footnote{Topologie faible, définition \ref{DEFooZGLDooRRarRj}.} fermé.
\end{proposition}

\begin{proof}
	En deux parties.
	\begin{subproof}
		\spitem[\( \Rightarrow\)]
		%-----------------------------------------------------------
		Soit un connexe fermé \( C\) de \( E\). Nous devons prouver que \( C\) est faiblement fermé, c'est à dire que \( E\setminus C\) est faiblement ouvert. Soit \( x_0\in E\setminus C\).

		Étant donné que tout espace de Banach est localement convexe (proposition \ref{PROPooBVWIooZocheH}) Nous utilisons le théorème de Hahn-Banach \ref{ThoACuKgtW} pour les partes \( A=\{ x_0 \}\) (qui est compacte) et \( C=C\) (qui est fermé). Il existe donc un hyperplan qui sépare strictement\footnote{Hyperplan qui sépare, définition \ref{DEFooQUDEooMrgcCV}.} \( \{ x_0 \}\) et \( C\). Autrement dit, il existe \( \alpha\in \eR\) et \( f\in E'\) tels que
		\begin{equation}
			f(x_0)<\alpha<f(x)
		\end{equation}
		pour tout \( x\in C\).

		Nous posons \( V=\{ x\in E\tq f(x)<\alpha \}\), qui n'est autre que le \( V(f,\alpha)\) du lemme \ref{LEMooEFVXooIWBBdW}. C'est un ouvert faible contenant \( x_0\) et contenu dans \( E\setminus C\). Cela prouve que \( E\setminus C\) est faiblement ouvert.

		\spitem[\( \Leftarrow\)]
		%-----------------------------------------------------------
		Nous supposons que \( C\) est faiblement fermé. Donc \( E\setminus C\) est faiblement ouvert et donc ouvert (lemme \ref{LEMooZPJNooJyaeAC}). Donc \( C\) est fermé.
	\end{subproof}
\end{proof}


%-------------------------------------------------------
\subsection{Espace de Banach réflexif, Kakutani}
%----------------------------------------------------

\begin{lemma}[Helly\cite{BIBooFDGQooYferue,BIBooJDASooFmDfPf}]	\label{LEMooSBWJooKIfuJj}
	Soit un espace de Banach \( E\) sur \( \eK\) (\( =\eR\) ou \( \eC\)). Nous considérons \( f_1,\ldots,f_k\in E'\) ainsi que des éléments \( (\gamma_1,\ldots,\gamma_k)\in \eK^n\). De plus nous introduisons
	\begin{equation}
		\begin{aligned}
			\varphi\colon E & \to \eK^k                                 \\
			x               & \mapsto \big( f_1(x),\ldots,f_k(x) \big).
		\end{aligned}
	\end{equation}

	Nous avons équivalence entre
	\begin{enumerate}
		\item		\label{ITEMooDMOOooPJtgTE}
		      Pour tout \( \epsilon>0\), il existe \( x_{\epsilon}\in E\) tel que \( \| x_{\epsilon} \|<1\) et \( | f_l(x_{\epsilon})-\gamma_{l} |<\epsilon\).
		\item		\label{ITEMooAQAJooHmSXGW}
		      \( \gamma\in\overline{\varphi\big( B(0,1) \big)}\).
		\item		\label{ITEMooXJKAooFHUsOv}
		      \( | \sum_{l=1}^k\beta_l\gamma_l |\leq \| \sum_{l=1}^k\beta_lf_l \|\) pour tout \( \beta_1,\ldots,\beta_k\in \eK\).
	\end{enumerate}
	Notez que \( | . |\) dénote soit la valeur absolue (si \( \eK=\eR\)) soit le module d'un nombre complexe (si \( \eK=\eC\)). Par ailleurs \( \| . \|\) dénote la norme opérateur.
\end{lemma}

\begin{proof}
	En plusieurs parties.
	\begin{subproof}
		\spitem[\ref{ITEMooDMOOooPJtgTE} \( \Rightarrow\) \ref{ITEMooAQAJooHmSXGW}]
		%-----------------------------------------------------------
		Soit \( \epsilon>0\). Il existe \( x_{\epsilon}\in E\) tel que \( x_{\epsilon}\in \overline{B(0,1)}\), et donc tel que
		\begin{equation}
			\| \varphi(x_{\epsilon})-\gamma \|<k\epsilon
		\end{equation}
		où à gauche, c'est la norme dans \( \eK^k\). Cela prouve que, en utilisant \ref{PROPooXGBEooZRVkSc},
		\begin{equation}
			\gamma\in\varphi\big( \overline{B(0,1)} \big)\subset\overline{\varphi\big( B(0,1) \big)}
		\end{equation}

		\spitem[\ref{ITEMooAQAJooHmSXGW} \( \Rightarrow\) \ref{ITEMooDMOOooPJtgTE}]
		%-----------------------------------------------------------
		Soit \( \epsilon>0\). Étant donné que \( \gamma\in\overline{\varphi\big( B(0,1) \big)}\), il existe \( y_{\epsilon}\in \varphi\big( B(0,1) \big)\) tel que \( \| y_{\epsilon}-\gamma \|<\epsilon\) et donc \( x_{\epsilon}\in B(0,1)\) tel que \( \| \varphi(x_{\epsilon})-\gamma \|<\epsilon\).

		Par inégalité triangulaire, cela implique \( | f_l(x_{\epsilon})-\gamma_l |<\epsilon\).

		\spitem[\ref{ITEMooDMOOooPJtgTE} \( \Rightarrow\) \ref{ITEMooXJKAooFHUsOv}]
		%-----------------------------------------------------------
		Nous posons \( S=\sum_{i=1}^k| \beta_i |\). Soit \( \epsilon>0\). Nous avons
		\begin{subequations}
			\begin{align}
				| \sum_i\beta_if_i(x_{\epsilon})-\sum_i\beta_i\gamma_i | & \leq | \sum_i\beta_i\big( f_i(x_{\epsilon})-\gamma_i \big)| \\
				                                                         & \leq \sum_i| \beta_i || f_i(x_{\epsilon})-\gamma_i |        \\
				                                                         & \leq S\epsilon.
			\end{align}
		\end{subequations}
		Grâce à la formule de la proposition \ref{PROPooVSVMooZrqxdc}, nous en déduisons
		\begin{subequations}
			\begin{align}
				| \sum_i\beta_i\gamma_i | & \leq | \sum_i\beta_if_i(x_{\epsilon}) |+S\epsilon       \\
				                          & \leq \| \sum_i\beta_if_i \|\| x_{\epsilon} \|+S\epsilon \\
				                          & \leq \| \sum_i\beta_if_i \|+S\epsilon.
			\end{align}
		\end{subequations}
		Vu que cette inégalité est valide pour tout \( \epsilon\), nous déduisons que \( | \sum_i\beta_i\gamma_i |\leq \| \sum_i\beta_if_i \|\).

		\spitem[\ref{ITEMooXJKAooFHUsOv} \( \Rightarrow\) \ref{ITEMooDMOOooPJtgTE}]
		%-----------------------------------------------------------
		Nous faisons non\ref{ITEMooDMOOooPJtgTE} \( \Rightarrow\) non\ref{ITEMooXJKAooFHUsOv}. Et comme \ref{ITEMooDMOOooPJtgTE} est équivalent à \ref{ITEMooAQAJooHmSXGW}, nous faisons non\ref{ITEMooAQAJooHmSXGW} \( \Rightarrow\) non\ref{ITEMooXJKAooFHUsOv}.

		Nous avons donc \( \gamma\not\in\overline{\varphi\big( B(0,1) \big)}\). En utilisant le théorème \ref{ThoACuKgtW} ou \ref{THOooEOOYooPdTGLi} selon que \( \eK=\eC\) ou \( \eK=\eC\), nous savons que \( \{ \gamma \}\) et \( \overline{\varphi\big( B(0,1) \big)}\) sont strictement séparés par un hyperplan de \( \eK^k\). Autrement dit, il existe \( \beta\in \eK^k\) et \( \alpha\in \eR\) tels que
		\begin{equation}
			\real\big( \beta\cdot \varphi(x) \big)<\alpha<\real(\beta\cdot\gamma).
		\end{equation}
		Autrement dit,
		\begin{equation}		\label{EQooSEUBooAoyMVa}
			\real(\sum_{i=1}^k\beta_if_i(x))<\alpha<\real(\sum_i\beta_i\gamma_i).
		\end{equation}
		Maintenant on fait deux cas selon que \( \eK=\eC\) ou \( \eK=\eR\).
		\begin{subproof}
			\spitem[Si \( \eK=\eR\)]
			%-----------------------------------------------------------
			Alors nous pouvons enlever le \( \real\) dans \eqref{EQooSEUBooAoyMVa}. Vu que l'inégalité est valide pour tout \( x\in E\) et vu que \( | A(-x)|=| A(x) | |\) lorsque \( A\) est linéaire, nous avons
			\begin{equation}
				| (\sum_i\beta_if_i)(x) |<\alpha<\sum_i\beta_i\gamma_i.
			\end{equation}
			En prenant le suprémum sur \( \| x \|=1\) nous trouvons
			\begin{equation}		\label{EQooMNEGooMhPMWu}
				\| \sum_i\beta_if_i \|\leq\alpha\leq \sum_i\beta_i\gamma_i,
			\end{equation}
			qui contredit \ref{ITEMooXJKAooFHUsOv}.

			\spitem[Si \( \eK=\eC\)]
			%-----------------------------------------------------------
			Nous ne pouvons pas enlever les modules si facilement. C'est d'ailleurs le moment de relire le lemme \ref{LEMooBZHIooSQJSnM}. Nous posons \( \phi=\sum_i\beta_if_i\) et \( g=\real(\phi)\). Ce que dit \eqref{EQooSEUBooAoyMVa} est
			\begin{equation}
				| g(x) |<\alpha<\real(\sum_i\beta_i\gamma_i)\leq | \sum_i\beta_i\gamma_i |.
			\end{equation}
			En prenant le suprémum sur \( \| x \|=1\) et en tenant compte du fait que \( \| \phi \|=\| g \|\), nous obtenons à nouveau une inégalité similaire à \eqref{EQooMNEGooMhPMWu}, et donc une contradiction.
		\end{subproof}
	\end{subproof}
\end{proof}

\begin{theorem}[Théorème de Goldstine\cite{BIBooJDASooFmDfPf}]	\label{THOooATSGooSZZYPz}
	Soit un espace de Banach \( E\) sur \( \eK\) (\( =\eR\) ou \( \eC\)). Nous avons\footnote{Définition \ref{PROPooMAQSooCGFBBM} pour \( J\).}
	\begin{enumerate}
		\item		\label{ITEMooMDBQooSVTyjx}
		      La partie \( J\big( B_E(0,1) \big)\) est dense dans \( B_{E''}(0,1)\) pour la topologie faible\footnote{Définie en \ref{DEFooZGLDooRRarRj}.} \( \sigma(E',E'')\).
		\item		\label{ITEMooXIGCooVpzqGA}
		      La partie \( J(E)\) est dense dans \( E''\).
	\end{enumerate}
\end{theorem}

\begin{proof}
	En deux parties.
	\begin{subproof}
		\spitem[Pour \ref{ITEMooMDBQooSVTyjx}]
		%-----------------------------------------------------------

		Soit \( \xi_0\in B_{E''}(0,1)\), et soit un voisinage \( V\) de \( \xi_0\). Nous devons prouver que \( V\cap J\big( B_E(0,1) \big)\neq \emptyset\). La proposition \ref{PROPooMBOJooQcwyuv} donne une base de la topologie faible. Nous pouvons donc supposer que \( V\) est de la forme \eqref{EQooTINIooJulDll}. Il faut juste ajouter des primes un peu partout parce que nous parlons ici de \( \sigma(E',E'') \) et non de \( \sigma(E,E')\). Bref. Il existe des \( \varphi_i\in E'\) et \( \epsilon>0\) tels que
		\begin{equation}
			V=V(\xi_0;\varphi_1,\ldots,\varphi_k,\epsilon)=\{ \eta\in E''\tq | (\eta-\xi_0)(\varphi_i) |<\epsilon \}.
		\end{equation}
		Nous allons montrer qu'il existe \( x_{\epsilon}\in B_E(0,1)\) tel que \( J(x)\in V\).

		Nous posons \( \gamma_i=\xi_0(\varphi_i)\in \eK\). Soient \( \beta_1,\ldots,\beta_k\in \eK\). Nous avons
		\begin{equation}		\label{EQooNGRVooVPapQa}
			|\sum_{i=1}^k\beta_i\gamma_i|=|\sum_i\beta_i\xi_0(\varphi_i)|=|\xi_0\big( \sum_i\beta_i\varphi_i \big)|\leq \| \sum_i\beta_i\varphi_i \|
		\end{equation}
		parce que \( \| \xi_0 \|\leq 1\). L'inégalité \eqref{EQooNGRVooVPapQa} étant valide pour tout choix de \( \beta_i\), le théorème d'Helly \ref{LEMooSBWJooKIfuJj} implique l'existence de \( x_{\epsilon}\in E\) tel que \( \| x_{\epsilon} \|<1\) et
		\begin{equation}
			| \varphi_i(x_{\epsilon})-\gamma_i |<\epsilon.
		\end{equation}
		Nous avons donc
		\begin{equation}
			\epsilon<| \varphi_i(x_{\epsilon})-\gamma_i |=| \varphi_i(x_{\epsilon})-\xi_0(\varphi_i) |=| \big( J(x_{\epsilon})-\xi_0 \big)(\varphi_i) |,
		\end{equation}
		et donc \( J(x_{\epsilon})\in V\).

		\spitem[Pour \ref{ITEMooXIGCooVpzqGA}]
		%-----------------------------------------------------------
		Soit \( \xi_0\in E''\) ainsi qu'un voisinage \( V\) de \( \xi_0\) de la forme
		\begin{equation}
			V(\xi_0;\varphi_1,\ldots,\varphi_k,\epsilon)=\{ \eta\in E''\tq | (\eta-\xi_0)(\varphi_i) |<\epsilon \}.
		\end{equation}
		Il existe \( \lambda\in \eK\) tel que \( \lambda\xi_0\in B_{E''}(0,1)\). Nous avons
		\begin{subequations}
			\begin{align}
				\lambda V(\xi_0;\varphi_1,\ldots,\varphi_k,\epsilon) & = \{ \lambda\eta\tq | (\eta-\xi_0)(\varphi_i) |<\epsilon \}                     \\
				                                                     & =\{ \lambda\eta\tq | (\lambda\eta-\lambda\xi_0)(\varphi_i) |<\lambda\epsilon \} \\
				                                                     & =\{ \eta\tq | (\eta-\lambda\xi_0)(\varphi_i) |<\lambda\epsilon \},
			\end{align}
		\end{subequations}
		autrement dit
		\begin{equation}		\label{EQooOGQNooIxEnGG}
			\lambda V(\xi_0;\varphi_1,\ldots,\varphi_k,\epsilon)=V(\lambda \xi_0;\varphi_1,\ldots,\varphi_k,\lambda\epsilon).
		\end{equation}
		Étant donné que \( \lambda\xi_0\in B_{E''}(0,1)\), la première partie dit qu'il existe \( x_{\epsilon}\in B_E(0,1)\) tel que
		\begin{equation}
			J(x_{\epsilon})\in V(\lambda\xi_0;\varphi_1,\ldots,\varphi_k,\lambda\epsilon)=\lambda V(\xi_0;\varphi_1,\ldots,\varphi_k,\epsilon).
		\end{equation}
		En reprenant \eqref{EQooOGQNooIxEnGG} (avec \( 1/\lambda\) au lieu de \( \lambda\)), nous avons
		\begin{equation}
			J(x_{\epsilon}/\lambda)\in \frac{1}{ \lambda}V(\lambda\xi_0;\varphi_1,\ldots,\varphi_k,\epsilon)=V(\xi_0;\varphi_1,\ldots,\varphi_k,\epsilon),
		\end{equation}
		ce qui signifie que \( V(\xi_0;\varphi_1,\ldots,\varphi_k,\epsilon) \) contient bien un élément de \( J(E)\).
	\end{subproof}
\end{proof}

\begin{theorem}[Kakutani\cite{BIBooFDGQooYferue}]       \label{THOooTFIHooPQjVAr}
	Un espace de Banach est réflexif si et seulement si la boule fermée \( \overline{ B(0,1) }\) est compacte pour la topologie faible\footnote{Définition \ref{DEFooZGLDooRRarRj}.}.
\end{theorem}

\begin{proposition}[\cite{BIBooFDGQooYferue}]       \label{PROPooPVVYooMZjQSq}
	Une suite bornée dans un espace de Banach réflexif\footnote{Définition \ref{PROPooMAQSooCGFBBM}.} contient une sous-suite faiblement convergente.
\end{proposition}

\begin{proposition}[\cite{BIBooFDGQooYferue}]       \label{PROPooBBNBooGcXDRH}
	Un espace de Banach est réflexif si et seulement si son dual est réflexif.
\end{proposition}

%--------------------------------------------------------------------------------------------------------------------------- 
\subsection{Espaces \texorpdfstring{\(  L^{\infty}\)}{Linfinity}}
%---------------------------------------------------------------------------------------------------------------------------

\begin{lemma}[\cite{BIBooFDGQooYferue}]        \label{LEMooMSYAooGEMgoc}
	Si \( (\Omega,\tribA,\mu)\) est un espace mesuré \( \sigma\)-fini, l'espace \( L^1(\Omega,\tribA,\mu)\) n'est pas réflexif.
\end{lemma}

\begin{lemma}[\cite{BIBooFDGQooYferue}]     \label{LEMooUSXTooFvpsVd}
	Si \( (\Omega,\tribA,\mu)\) est un espace mesuré \( \sigma\)-fini, alors \( L^{\infty}(\Omega,\tribA,\mu)\) n'est pas réflexif\footnote{Définition \ref{PROPooMAQSooCGFBBM}.}.
\end{lemma}

\begin{proof}
	Supposons que \( L^{\infty}(\Omega,\tribA,\mu)\) est réflexif. Le théorème de représentation de Riesz \ref{ThoLPQPooPWBXuv}\ref{ITEMooCQGJooOWzjoV} dit que \( (L^1)'\) est en bijection linéaire isométrique avec \( L^{\infty}\); le lemme \ref{PROPooVRQKooLdmajh} dit alors que \( (L^1)'\) est réflexif\footnote{Dans de nombreuses références, par exemple\cite{BIBooFDGQooYferue}, il est simplement dit que \( (L^1)'=L^{\infty}\). C'est un abus de notation qui permet de se passer du lemme \ref{PROPooVRQKooLdmajh}.}. La proposition \ref{PROPooBBNBooGcXDRH} dit alors que \( L^1\) est réflexif.

	Or le lemme \ref{LEMooMSYAooGEMgoc} dit que \( L^1\) n'est pas réflexif.
\end{proof}

La proposition suivante est souvent présentée en disant que l'inclusion \( L^1\subset (L^{\infty})'\) est stricte, ou, pire, en disant que \( (L^{\infty})'\) est strictement plus grand que \( L^1\). Cette façon de dire est un gros abus de langage. D'abord \( L^1\) n'est même pas inclus dans \( L^{\infty}\); ce sont deux ensembles qui n'ont rien à voir. Ensuite, ce que signifie réellement cette proposition est seulement que la première injection \( L^1\to (L^{\infty})'\) qui nous tombe sous la main (celle du théorème de représentation de Riesz) n'est pas surjective\quext{Si vous savez comment prouver qu'il n'existe pas de surject de \( L^1\) vers $(L^{\infty})'$, écrivez-moi.}.
\begin{proposition}     \label{PROPooXXRQooNSBZOi}
	Soit \( g\in L^1(\eR^d)\). L'application
	\begin{equation}
		\begin{aligned}
			\Phi_g\colon L^{\infty}(\eR^d) & \to \eC                     \\
			f                              & \mapsto \int_{\eR^d}f\bar g
		\end{aligned}
	\end{equation}
	est linéaire et bien définie.

	L'application
	\begin{equation}
		\begin{aligned}
			\Phi\colon L^1(\eR^d) & \to L^{\infty}(\eR^d)' \\
			g                     & \mapsto \Phi_g
		\end{aligned}
	\end{equation}
	n'est pas surjective.
\end{proposition}

\begin{proof}
	Prouvons d'abord que \( \Phi_g\) est bien définie. Vu que \( f\in L^{\infty}\), il existe \( M\in \eR\) tel que \( | f(x) |<M\) sur \( \eR^d\setminus A\) où \( A\) est de mesure nulle dans \( \eR^d\). La fonction \( x\mapsto| f(x)\overline{ g(x) } |\) est donc majorée par la fonction intégrable \( x\mapsto M| g(x) |\) qui est intégrable (sur \( \eR^d\setminus A\)). L'intégrabilité de \( f\bar g\) n'est donc pas un problème.

	Le fait que \( \Phi\) prenne ses valeurs dans \( (L^{\infty})'\) est le théorème \ref{THOooXMVTooBAbyvr}\ref{ITEMooBFFZooNxoHER}.

	Le fait qu'elle ne soit pas surjective est la proposition \ref{PROPooXNRRooUdgFPr}.
\end{proof}
