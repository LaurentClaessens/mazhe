% This is part of Mes notes de mathématique
% Copyright (c) 2011-2022
%   Laurent Claessens
% See the file fdl-1.3.txt for copying conditions.

%+++++++++++++++++++++++++++++++++++++++++++++++++++++++++++++++++++++++++++++++++++++++++++++++++++++++++++++++++++++++++++ 
\section{Topologie faible}
%+++++++++++++++++++++++++++++++++++++++++++++++++++++++++++++++++++++++++++++++++++++++++++++++++++++++++++++++++++++++++++

\begin{definition}[\cite{BIBooFDGQooYferue,MonCerveau}]        \label{DEFooZGLDooRRarRj}
    Soit un espace de Banach\footnote{Définition \ref{DefVKuyYpQ}.} sur le corps \( \eF\) (\( =\eR\) ou \( \eC\)) \( E\), et son dual \( E'\). La \defe{topologie faible}{topologie faible} sur \( E\), est la plus petite topologie \( \tau\) pour laquelle 
    \begin{equation}
        \big( E,\| . \| \big)'=\big( E,\tau \big)'.
    \end{equation}
    Autrement dit, c'est la topologie de la proposition \ref{PROPooGOEVooZBAOQh} rendant continues toutes les applications de \( E'\).

    Elle sera notée \( \tau_w\) ou \( \sigma(E,E')\).
\end{definition}

Vu que, pour rendre continue une application, il suffit que toutes les images inverses des ouverts de \( \eF\) soient des ouverts, la topologie faible sur \( E\) est également la topologie engendrée\footnote{Proposition \ref{DefTopologieEngendree}.} par les parties $\varphi^{-1}(V)$ avec \( \varphi\in E'\) et \( V\in \tau_{\eF}\).

\begin{lemma}[\cite{BIBooFDGQooYferue}]
    La topologie faible est Hausdorff\footnote{Espace topologique Hausdorff, définition \ref{}.}.
\end{lemma}

\begin{proof}
    Soient un espace de Banach \( E\) ainsi que \( x_1\neq x_2\) dans \( E\). Les parties \( \{ x_1 \}\) et \( \{ x_2 \}\) vérifient les hypothèses de Hahn-Banach (seconde forme géométrique, \ref{ThoACuKgtW}). Il existe donc une fonctionnelle \( \varphi\colon E\to \eF\) telle que \( \varphi(x_1)\neq \varphi(x_2)\). Avec un peu de bonne volonté, nous supposons que \( \real\big( \varphi(x_1) \big)\neq \real\big( \varphi(x_2) \big)\). Soit \( \alpha\in \eR\) tel que
    \begin{equation}
        \real\big( \varphi(x_1) \big)<\alpha <\real\big( \varphi(x_2) \big).
    \end{equation}
    Posons
    \begin{subequations}
        \begin{align}
            \mO_1&=\{ x\in E\tq\real\big( \varphi(x) \big)< \alpha \},
            \mO_2&=\{ x\in E\tq\real\big( \varphi(x) \big)> \alpha \}.
        \end{align}
    \end{subequations}
    Nous avons \( \mO_i=\varphi^{-1}(V_i)\) avec\footnote{Adaptez si \( \eF=\eR\) au lieu de \( \eC\).}
    \begin{equation}
        V_1= \mathopen] -\infty , \alpha \mathclose[+i\eR 
    \end{equation}
    et quelque chose du même genre pour \( V_2\). Vu que \( V_1\) et \( V_2\) sont ouverts dans \( \eF\), et que \( \varphi\) est continue pour la topologie faible, les parties \( \mO_i\) sont ouvertes dans \( (E,\tau_w)\).

    Nous avons de plus \( \mO_1\cap\mO_2=\emptyset\) et \( x_i\in \mO_i\), de telle sorte que \( x_1\) et \( x_2\) sont correctement séparés.
\end{proof}

\begin{proposition}[\cite{BIBooFDGQooYferue}]       \label{PROPooYARHooOpmztY}
    Un convexe dans un espace de Banach est fermé si et seulement si il est faiblement fermé.
\end{proposition}

%--------------------------------------------------------------------------------------------------------------------------- 
\subsection{Espace de Banach réflexif}
%---------------------------------------------------------------------------------------------------------------------------

\begin{proposition}[\cite{BIBooFDGQooYferue}]       \label{PROPooPVVYooMZjQSq}
    Une suite bornée dans un espace de Banach réflexif\footnote{Définition \ref{}.} contient une sous-suite faiblement convergente.
\end{proposition}

\begin{theorem}[Kakutami\cite{BIBooFDGQooYferue}]       \label{THOooTFIHooPQjVAr}
Un espace de Banach est réflexif si et seulement si la boule fermée \( \overline{ B(0,1) }\) est compacte pour la topologie faible\footnote{Définition \ref{DEFooZGLDooRRarRj}.}.
\end{theorem}

\begin{proposition}[\cite{BIBooFDGQooYferue}]       \label{PROPooBBNBooGcXDRH}
Un espace de Banach est réflexif si et seulement si son dual est réflexif.
\end{proposition}

%--------------------------------------------------------------------------------------------------------------------------- 
\subsection{Espaces \texorpdfstring{\(  L^{\infty}\)}{Linfinity}}
%---------------------------------------------------------------------------------------------------------------------------

\begin{lemma}[\cite{BIBooFDGQooYferue}]        \label{LEMooMSYAooGEMgoc}
Si \( (\Omega,\tribA,\mu)\) est un espace mesuré \( \sigma\)-fini, l'espace \( L^1(\Omega,\tribA,\mu)\) n'est pas réflexif.
\end{lemma}

\begin{lemma}[\cite{BIBooFDGQooYferue}]     \label{LEMooUSXTooFvpsVd}
    Si \( (\Omega,\tribA,\mu)\) est un espace mesuré \( \sigma\)-fini, alors \( L^{\infty}(\Omega,\tribA,\mu)\) n'est pas réflexif\footnote{Définition \ref{PROPooMAQSooCGFBBM}.}.
\end{lemma}

\begin{proof}
Supposons que \( L^{\infty}(\Omega,\tribA,\mu)\) est réflexif. Le théorème de représentation de Riesz \ref{ThoLPQPooPWBXuv}\ref{ITEMooCQGJooOWzjoV} dit que \( (L^1)'\) est en bijection linéaire isométrique avec \( L^{\infty}\); le lemme \ref{PROPooVRQKooLdmajh} dit alors que \( (L^1)'\) est réflexif\footnote{Dans de nombreuses références, par exemple\cite{BIBooFDGQooYferue}, il est simplement dit que \( (L^1)'=L^{\infty}\). C'est un abus de notation qui permet de se passer du lemme \ref{PROPooVRQKooLdmajh}.}. La proposition \ref{PROPooBBNBooGcXDRH} dit alors que \( L^1\) est réflexif.

Or le lemme \ref{LEMooMSYAooGEMgoc} dit que \( L^1\) n'est pas réflexif.
\end{proof}



La proposition suivante est souvent présentée en disant que l'inclusion \( L^1\subset (L^{\infty})\) est stricte, ou, pire, en disant que \( (L^{\infty})'\) est strictement plus grand que \( L^1\). Cette façon de dire est un gros abus de language. D'abord \( L^1\) n'est même pas inclus à \( L^{\infty}\); ce sont deux ensembles qui n'ont rien à voir. Ensuite, ce que signifie réellement cette proposition est seulement que la première injection \( L^1\to (L^{\infty})'\) qui nous tombe sous la main (celle du théorème de représentation de Riesz) n'est pas surjective.
\begin{proposition}     \label{PROPooXXRQooNSBZOi}
Soit \( g\in L^1(\eR^d)\). L'application
\begin{equation}
    \begin{aligned}
        \Phi_g\colon L^{\infty}(\eR^d) & \to \eC                     \\
        f                              & \mapsto \int_{\eR^d}f\bar g
    \end{aligned}
\end{equation}
est linéaire et bien définie.

L'application
\begin{equation}
    \begin{aligned}
        \Phi\colon L^1(\eR^d) & \to L^{\infty}(\eR^d)' \\
        g                     & \mapsto \Phi_g
    \end{aligned}
\end{equation}
n'est pas surjective.
\end{proposition}

\begin{proof}
Prouvons d'abord que \( \Phi_g\) est bien définie. Vu que \( f\in L^{\infty}\), il existe \( M\in \eR\) tel que \( | f(x) |<M\) sur \( \eR^d\setminus A\) où \( A\) est de mesure nulle dans \( \eR^d\). La fonction \( x\mapsto| f(x)\overline{ g(x) } |\) est donc majorée par la fonction intégrable \( x\mapsto | g(x) |\) qui est intégrable (sur \( \eR^d\setminus A\)). L'intégrabilité de \( f\bar g\) n'est donc pas un problème.

Le fait que \( \Phi\) prenne ses valeurs dans \( (L^{\infty})'\) est le théorème \ref{THOooXMVTooBAbyvr}\ref{ITEMooBFFZooNxoHER}.

Le fait qu'elle ne soit pas surjective est la proposition \ref{PROPooXNRRooUdgFPr}.
\end{proof}
