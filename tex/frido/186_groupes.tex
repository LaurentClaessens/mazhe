% This is part of Mes notes de mathématique
% Copyright (c) 2011-2020, 2022-2025
%   Laurent Claessens
% See the file fdl-1.3.txt for copying conditions.

%+++++++++++++++++++++++++++++++++++++++++++++++++++++++++++++++++++++++++++++++++++++++++++++++++++++++++++++++++++++++++++
\section{Théorèmes de Sylow}
%+++++++++++++++++++++++++++++++++++++++++++++++++++++++++++++++++++++++++++++++++++++++++++++++++++++++++++++++++++++++++++

\begin{lemma}
	Soient \( H\) et \( K\) des sous-groupes finis de \( G\). Alors
	\begin{equation}
		\Card(HK)=\frac{ | H |\cdot | K | }{ | H\cap K | }.
	\end{equation}
\end{lemma}
Attention : dans ce lemme, l'ensemble \( HK\) n'est pas spécialement un groupe. Ce serait le cas si \( H\) normalisait \( K\), c'est-à-dire si nous avions \( hkh^{-1}\in K,\,\forall (h,k)\in H\times K\).

\begin{theorem}[Théorème de Cauchy\cite{ooBZOQooUnlnoI}]\label{ThoCauchyGpFini}
	Soit \( G\) un groupe fini et \( p\) un nombre premier divisant \( | G |\). Alors
	\begin{enumerate}
		\item
		      \( G\) contient un élément d'ordre \( p\).
		\item
		      Si \( G\) est un \( p\)-groupe, il existe un élément central d'ordre \( p\) dans \( G\).
	\end{enumerate}
\end{theorem}
\index{Cauchy!théorème}

\begin{lemma}[Théorème de Cayley]    \label{ThoIfdlEB}   \index{Cayley!théorème}
	Si \( G\) est un groupe d'ordre \( n\) alors il est isomorphe à un sous-groupe du groupe symétrique \( S_n\).
\end{lemma}

\begin{proof}
	L'action à gauche de \( G\) sur lui-même
	\begin{equation}
		\begin{aligned}
			\varphi\colon G & \to S_n    \\
			\varphi(x)g     & \mapsto xg
		\end{aligned}
	\end{equation}
	est une permutation des éléments de \( G\). Cela donne un morphisme injectif parce que si \( \varphi(x)=\varphi(y)\) nous avons \( xg=yg\) pour tout \( g\) et en particulier pour \( g=e\) nous trouvons \( x=y\).
\end{proof}

Pour rappel, lorsque \( p\) est premier, nous notons \( \eF_p=\eZ/p\eZ\).

\begin{lemma}       \label{LemaQxjcm}
	Soit \( p\) un diviseur premier de \( n\). Alors il existe un morphisme injectif du groupe symétrique \( S_n\) dans \( \GL(n,\eF_p)\).
\end{lemma}

\begin{proof}
	Soit \( \{ e_i \}\) la base canonique de \( \eF_p^n\). Par exemple \( e_1=\big( [1]_p,[0]_p,\ldots, [0]_p \big)\). Nous avons le morphisme injectif \( \varphi\colon S_n\to \GL(n,\eF_p)\) donné par \( \varphi(\sigma)e_i=e_{\sigma(i)}\).
\end{proof}

\begin{remark}  \label{RemFzxxst}
	En mettant bout à bout les lemmes~\ref{ThoIfdlEB} et~\ref{LemaQxjcm}, nous trouvons que si \( p\) est un diviseur premier de \( | G |\), alors \( G\) peut être vu comme un sous-groupe de \( \GL(n,\eF_p)\).
\end{remark}

\begin{definition}      \label{DEFooPRCHooVZdwST}
	Soit \( p\) un nombre premier. Un \defe{\( p\)-groupe}{\( p\)-groupe}\index{groupe!\( p\)-groupe} est un groupe dont tous les éléments sont d'ordre \( p^m\) pour un certain \( m\) (dépendant de l'élément).

	Soit \( G\) un groupe fini et \( p\), un diviseur premier de \( | G |\). Un \defe{\(p\)-Sylow}{\( p\)-Sylow}\index{Sylow!\(  p\)-Sylow} dans \( G\) est un \( p\)-sous-groupe d'ordre \( p^n\) où \( p^n\) est la plus grande puissance de \( p\) divisant \( | G |\).
\end{definition}
Notons que si \( p\) est un nombre premier, alors tout groupe d'ordre \( p^m\) est un \( p\)-groupe.

\begin{lemma}
	Soit \( G\) un groupe fini et \( P\), \( Q\) des \( p\)-sous-groupes. Nous supposons que \( Q\) normalise \( P\). Alors \( PQ\) est un \( p\)-sous-groupe de \( G\).
\end{lemma}

Si \( S\) est un \( p\)-Sylow, alors \( p\) ne divise pas le nombre \( | G:S |=| G |/| S |\).

\begin{proposition}     \label{Propvocmon}
	Soit le corps fini \( \eF_p=\eZ/p\eZ\) (\( p\) premier). Soit \( T\) le sous-ensemble de \( \GL_n(\eF_p)\) formé des matrices triangulaires supérieures de rang\footnote{Définition~\ref{DefALUAooSPcmyK}.} \( n\) et dont les éléments diagonaux sont \( 1\). Alors \( T\) est un \( p\)-Sylow de \( \GL_n(\eF_p)\).
\end{proposition}

\begin{proof}
	Nous commençons par étudier le cardinal de \( \GL_n(\eF_p)\). Pour la première colonne, la seule contrainte à vérifier est qu'elle ne soit pas nulle. Il y a donc \( p^n-1\) possibilités. Pour la seconde, il faut ne pas être multiple de la première. Il y a donc \( p^n-p\) possibilités (parce qu'il y a \( p\) multiples possibles de la premières colonne). Pour la \( k\)-ième colonne, il faut éviter toutes les combinaisons linéaires des \( (k-1)\) premières colonnes. Il y a \( p^{k-1}\) telles combinaisons et donc \( p^n-p^{k-1}\) possibilités pour la \( k\)-ième colonne. Nous avons donc
	\begin{subequations}
		\begin{align}
			\Card\big( \GL(n,\eF_{p}) \big) & =(p^n-1)(p^n-p)\ldots(p^n-p^{n-1})                      \\
			                                & =p\cdot p^2\cdots p^{n-1}(p^n-1)(p^{n-1}-1)\ldots (p-1) \\
			                                & =p^{\frac{ n(n-1) }{2}}m
		\end{align}
	\end{subequations}
	où \( m\) est un entier qui ne divise pas \( p\).

	En ce qui concerne le cardinal de \( T\), le calcul est plus simple : pour la première ligne nous avons \( p^{n-1}\) choix (parce qu'il y a un \( 1\) qui est imposé sur la diagonale), pour la seconde \( p^{n-2}\), etc. En tout nous avons alors
	\begin{equation}
		| T |=p^{\frac{ n(n-1) }{2}},
	\end{equation}
	et \( T\) est un \( p\)-Sylow de \( \GL_n(\eF_p)\).
\end{proof}


\begin{proposition}
	Soit \( p\) un nombre premier. Un groupe fini \( G\) est un \( p\)-groupe si et seulement l'ordre de \( G\) est \( p^n\) pour un certain \( n\).
\end{proposition}

\begin{proof}
	Supposons que \( G\) est un \( p\)-groupe. Soit \( q\) un nombre premier divisant \( | G |\). Par le théorème de Cauchy (\ref{ThoCauchyGpFini}), le groupe \( G\) contient un élément d'ordre \( q\), soit \( g\) un tel élément. Étant donné que \( G\) est un \( p\)-groupe, \( g^{p^n}=g^q=e\) pour un certain \( n\). Donc \( q=p^n\) et \( q=p\) parce que \( q\) est premier. Nous venons de prouver que \( p\) est le seul nombre premier qui divise \( | G |\). L'ordre de \( G\) est par conséquent une puissance de \( p\).

	Nous nous intéressons maintenant à l'implication inverse. Nous supposons que \( | G |=p^n\) pour un certain entier \( n\geq 0\). Soit \( g\in G\); nous notons \( r\) l'ordre de \( G\). Le sous-groupe \( \gr(g)\) est d'ordre \( r\), donc \( r\) divise \( | G |\) (par le théorème~\ref{ThoLagrange} de Lagrange). Le nombre \( r\) est alors une puissance de \( p\).
\end{proof}

\begin{lemma}       \label{LemwDYQMg}
	Soit \( G\), un groupe fini de cardinal \( | G |=n\) et \( p\), un diviseur premier de \( n\). Nous notons \( n=p^m\cdot r\) où \( p\) ne divise pas \( r\). Soit \( H\) un sous-groupe de \( G\) et \( S\), un \( p\)-Sylow de \( G\). Alors il existe \( g\in G\) tel que
	\begin{equation}
		gSg^{-1}\cap H
	\end{equation}
	soit un \( p\)-Sylow de \( H\).
\end{lemma}

\begin{proof}
	Nous considérons l'ensemble \( G/S\) sur lequel \( H\) agit. Si \( a\in G\), le stabilisateur de \( [a]\) dans \( G/S\) est
	\begin{subequations}
		\begin{align}
			\Fix\big( [a] \big) & =\{ h\in H\tq [ha]=[a] \}     \\
			                    & =\{ h\in H\tq a^{-1}ha\in S\} \\
			                    & =aSa^{-1}\cap H.
		\end{align}
	\end{subequations}
	Nous cherchons \( a\in G\) tel que l'entier
	\begin{equation}        \label{EqZpUbWx}
		\frac{ \Card(H) }{ \Card\big( aSa^{-1}\cap H \big) }
	\end{equation}
	soit premier avec \( p\). En effet, dans ce cas le groupe \( \Fix([a])\) est un \( p\)-Sylow de \( H\) parce que \( | H:aSa^{-1}\cap H |\) ne divise pas \( p\). La formule des orbites (équation \eqref{EqCewSXT}) nous dit que
	\begin{equation}
		\frac{ | H | }{ | aSa^{-1}\cap H | }=\Card\big( \mO_{[a]} \big).
	\end{equation}
	Supposons que toutes les orbites aient un cardinal divisible par \( p\). Étant donné que \( G/S\) est une réunion disjointe de ses orbites, nous aurions
	\begin{equation}
		p\divides \Card(G/S)=\frac{ | G | }{ | S | }
	\end{equation}
	alors que \( S\) étant un \( p\)-Sylow, \( p\) ne peut pas diviser \( | G |/| S |\). Toutes les orbites n'ont donc pas un cardinal divisible par \( p\), et il existe un \( a\in G\) tel que \eqref{EqZpUbWx} soit vérifiée.
\end{proof}


\begin{theorem}[Théorème de Sylow]  \label{ThoUkPDXf}
	Soit \( G\) un groupe fini et \( p\), un diviseur premier de \( | G |\). Alors
	\begin{enumerate}
		\item       \label{ITEMooETYHooXlUMQZ}
		      \( G\) possède au moins un \( p\)-Sylow\footnote{Définition~\ref{DEFooPRCHooVZdwST}.}.
		\item
		      Tout \( p\)-sous-groupe de \( G\) est contenu dans un \( p\)-Sylow.
		\item   \label{ItemMzNRVf}
		      Les \( p\)-Sylow de \( G\) sont conjugués.
		\item   \label{ItemkYbdzZ}
		      Si \( n_p\) est le nombre de \( p\)-Sylow de \( G\), alors \( n_p\) divise \( | G |\) et \( n_p\in[1]_p\).
	\end{enumerate}
\end{theorem}
\index{groupe!fini}

\begin{proof}
	En plusieurs points.
	\begin{enumerate}
		\item

		      Nous savons de la remarque~\ref{RemFzxxst} que \( G\) est un sous-groupe de \( \GL_n(\eF_p)\) et que ce dernier a un \( p\)-Sylow par la proposition~\ref{Propvocmon}. Par conséquent \( G\) possède un \( p\)-Sylow par le lemme~\ref{LemwDYQMg}.

		\item

		      Soit \( H\) un \( p\)-sous-groupe de \( G\) et \( S\), un \( p\)-Sylow de \( G\) (qui existe par le point précédent). Par le lemme~\ref{LemwDYQMg} il existe \( a\in G\) tel que \( aSa^{-1}\cap H\) soit un \( p\)-Sylow de \( H\). Mais \( H\) est un \(p\)-groupe et un \( p\)-Sylow dans un \( p\)-groupe est automatiquement le groupe entier. Par conséquent,
		      \begin{equation}
			      H=aSa^{-1}\cap H
		      \end{equation}
		      et \( H\subset aSa^{-1}\), ce qui signifie que \( H\) est inclus dans un \( p\)-Sylow.

		\item

		      Soit \( H\) un \( p\)-Sylow. Nous venons de voir que si \( S\) est un \( p\)-Sylow quelconque, alors \( H\) est inclus au \( p\)-Sylow \( aSa^{-1}\) pour un certain \( a\in G\). Donc \( H\) est un \( p\)-Sylow inclus dans le \( p\)-Sylow \( aSa^{-1}\), donc \( H=aSa^{-1}\).

		\item

		      Le fait que \( n_p\) divise \( n\) vient du fait que tous les \( p\)-Sylow ont le même nombre d'éléments (ils sont conjugués) et sont deux à deux disjoints. Donc ils forment une partition de \( G\) et \( | G |=n_p| S |\) si \( S\) est un \( p\)-Sylow quelconque.

		      Montrons maintenant que \( n_p\) est congru à un modulo \( p\). Soit \( E\) l'ensemble des \( p\)-Sylow de \( G\). Le groupe \( G\) agit sur \( E\) par conjugaison. Soit \( S\) un \( p\)-Sylow et considérons l'ensemble
		      \begin{equation}
			      E_S=\{ T\in E\tq s\cdot T=T, \forall s\in S \}.
		      \end{equation}
		      où l'action est celle par conjugaison. C'est l'ensemble des points fixes de \( E\) sous l'action de \( S\). L'ensemble \( E\) est la réunion des orbites sous \( S\) et chacune de ces orbites a un cardinal qui divise \( | S |=p^m\). Par conséquent \( | \mO_T |\) vaut \( 1\) lorsque \( T\in E_S\) et est un multiple de \( p\) sinon. Nous avons donc
		      \begin{equation}
			      | E |\equiv | E_S |\mod p.
		      \end{equation}
		      Nous voulons obtenir \( | E_S |=1\). Évidemment \( S\in E_S\) parce que si \( s\in S\) alors \( sSs^{-1}=S\). Nous voudrions montrer que \( S\) est le seul élément de \( E_S\). Soit \( T\in E_S\), c'est-à-dire que \( T\) est un \( p\)-Sylow de \( G\) tel que
		      \begin{equation}
			      sTs^{-1}=T
		      \end{equation}
		      pour tout \( s\in S\). Soit \( N\) le groupe engendré par \( S\) et \( T\). Montrons que \( T\) est normal dans \( N\). Un élément \( g\) dans \( N\) s'écrit
		      \begin{equation}
			      g=s_1t_1\cdots s_rt_r
		      \end{equation}
		      avec \( s_i\in S\) et \( t_i\in T\). Si \( t\in T\), en utilisant le fait que \( T\) est un groupe et le fait que \( S\) le normalise, nous avons
		      \begin{equation}
			      gtg^{-1}=s_1t_1\ldots s_rt_rtt_r^{-1}s_r^{-1}\ldots t_1^{-1}s_1^{-1}\in T.
		      \end{equation}
		      Donc \( T\) est un sous-groupe normal de \( N\). Mais \( S\) et \( T\) sont conjugués dans \( N\) (parce que ils sont des \( p\)-Sylow de \( N\)), donc il existe un élément \( a\in N\) tel que \( aTa^{-1}=S\). Mais étant donné que \( T\) est normal,
		      \begin{equation}
			      S=aTa^{-1}=T.
		      \end{equation}
		      Ceci achève la démonstration des théorèmes de Sylow.

	\end{enumerate}
\end{proof}

\begin{proposition}
	Si \( S\) est un \( p\)-Sylow dans le groupe \( G\) alors pour tout \( g\in G\), l'ensemble \( gSg^{-1}\) est encore un \( p\)-groupe.
\end{proposition}

\begin{proof}
	Si les éléments de \( S\) sont d'ordre \( p^n\), alors nous avons
	\begin{equation}
		(gsg^{-1})^q=gs^qg^{-1}=e.
	\end{equation}
	Pour avoir \( gs^qg^{-1}=e\), il faut et suffit que \( gs^q=g\), alors \( s^q=e\), c'est-à-dire \( q=p^n\). Donc \( gSg^{-1}\) est encore un \( p\)-Sylow.
\end{proof}

\begin{lemma}[\cite{ooGQNTooEiWtsy}]\label{Lemcmbzum}
	Soit \( G\), un groupe fini et \( p\), un nombre premier. Si \( H\) et \( K\) sont des groupes distincts d'ordre \( p\), alors \( H\cap K=\{ e \}\).
\end{lemma}

\begin{proof}
	L'ensemble \( H\cap K\) est un sous-groupe de \( H\). Par conséquent son ordre divise celui de \( H\) qui est un nombre premier. Par conséquent soit \( | H\cap K |=1\), soit \( | H\cap K |=| H |\). Dans le second cas nous aurions \( H=K\), alors que nous avons supposé que \( H\) et \( K\) étaient distincts.
\end{proof}

\begin{proposition}[\cite{ooGQNTooEiWtsy}] \label{PropyfhTmf}
	Soit \( G\) un groupe fini et \( n\) le nombre de sous-groupes d'ordre \( p\) dans \( G\). Alors le nombre d'éléments d'ordre \( p\) dans \( G\) vaut \( n(p-1)\).
\end{proposition}

\begin{proof}
	Si \( g\) est un élément d'ordre \( p\) dans \( G\), le groupe \( H\) engendré par \( g\) est d'ordre \( p\). Réciproquement si \( H\) est un groupe d'ordre \( p\), tous les éléments de \( H\setminus\{ e \}\) sont d'ordre \( p\) (parce que l'ordre d'un élément divise l'ordre du groupe). Donc l'ensemble des éléments d'ordre \( p\) dans \( G\) est la réunion des ensembles \( H\setminus\{ e \}\) où \( H\) parcourt les sous-groupes d'ordre \( p\) dans \( G\). Chacun de ces ensembles possède \( p-1\) éléments et le lemme~\ref{Lemcmbzum} nous assure qu'ils sont disjoints. Par conséquent nous avons \( n(p-1)\) éléments d'ordre \( p\) dans \( G\).
\end{proof}

\begin{corollary}
	Un groupe d'ordre premier est cyclique.
\end{corollary}

\begin{proof}
	Soit \( p\) l'ordre de \( G\). Le nombre de sous-groupes d'ordre \( p\) est \( n=1\) (et c'est \( G\) lui-même). La proposition~\ref{PropyfhTmf} nous dit alors que le nombre d'éléments d'ordre \( p\) dans \( G\) est \( p-1\). Donc tout élément est générateur.
\end{proof}


%+++++++++++++++++++++++++++++++++++++++++++++++++++++++++++++++++++++++++++++++++++++++++++++++++++++++++++++++++++++++++++
\section{Groupe monogène}
%+++++++++++++++++++++++++++++++++++++++++++++++++++++++++++++++++++++++++++++++++++++++++++++++++++++++++++++++++++++++++++
\label{SECooXIHPooWVSjhT}

Groupe monogène : définition \ref{DEFooWMFVooLDqVxR}; groupe cyclique : définition \ref{DefHFJWooFxkzCF}.

Le théorème suivant donne quelques informations à propos des groupes monogènes. Il impliquera dans le corolaire~\ref{CORooMBLSooMHKmAq} qu'un groupe monogène d'ordre \( n\) possède \( \varphi(n)\) générateurs où \( \varphi\) est la fonction indicatrice d'Euler définie en~\ref{DEFooZRYMooZCozga}.

\begin{theorem}     \label{THOooDOMZooOEYHAe}
	Un groupe monogène est abélien. Plus précisément,
	\begin{enumerate}
		\item
		      un groupe monogène infini est isomorphe à \( \eZ\),
		\item
		      un groupe monogène fini est isomorphe à \( \eZ/n\eZ\) pour un certain \( n\).
	\end{enumerate}
\end{theorem}

\begin{proof}
	Le groupe est abélien parce que \( g=a^n\), \( g'=a^{n'}\) implique \( gg'=q^{n+n'}=g'g\). Nous considérons un générateur \( a\) de \( G\) (qui existe parce que \( G\) est monogène) et le morphisme surjectif
	\begin{equation}
		\begin{aligned}
			f\colon \eZ & \to G        \\
			p           & \mapsto a^p.
		\end{aligned}
	\end{equation}
	Si \( G\) est infini, alors \( f\) est injective parce que si \( a^n=a^{n'}\), alors \( a^{n-n'}=e\), ce qui rendrait \( G\) cyclique et par conséquent non infini. Nous concluons que si \( G\) est infini, alors \( f\) est une bijection et donc un isomorphisme \( \eZ\simeq G\).

	Si \( G\) est fini, alors \( f\) n'est pas injective et a un noyau \( \ker f\). Étant donné que \( \ker f\) est un sous-groupe de \( G\), il existe un (unique) \( n\) tel que \( \ker f=n\eZ\) et le premier théorème d'isomorphisme (théorème~\ref{ThoPremierthoisomo}) nous indique que
	\begin{equation}
		\eZ/\ker f=\eZ/n\eZ=\Image f=G.
	\end{equation}

\end{proof}

Le lemme suivant donne une démonstration alternative, avec une construction plus explicite de l'isomorphisme.

\begin{lemma}[\cite{MonCerveau}]   \label{LemZhxMit}

	À propos de groupes monogènes\footnote{Définition~\ref{DEFooWMFVooLDqVxR}.} et cycliques.

	\begin{enumerate}
		\item

		      Soit un groupe cyclique \( G\) de cardinal \( n\) dont \( g\) est un générateur. Alors il existe un isomorphisme
		      \begin{equation}
			      \phi\colon G\to (\eZ/n\eZ,+)
		      \end{equation}
		      tel que \( \phi(g)=1\).

		\item

		      Si \( G\) est un groupe monogène d'ordre infini et si \( g\) est un générateur, alors il existe un isomorphisme
		      \begin{equation}
			      \phi\colon G\to (\eZ,+)
		      \end{equation}
		      tel que \( \phi(g)=1\).

		\item

		      Soient \( G\) et \( H\) deux groupes monogènes de même ordre. Soient \( g\) un générateur de \( G\) et \( h\), un générateur de \( H\). Il existe un isomorphisme de \( G\) sur \( H\) qui envoie \( g\) sur \( h\).
	\end{enumerate}
\end{lemma}

\begin{proof}

	Commençons par enfoncer une porte ouverte : comme le groupe est monogène, l'ordre du groupe est égal à l'ordre de son générateur. Nous séparons les cas selon que l'ordre soit fini ou non.

	\begin{subproof}
		\spitem[L'ordre de \( G\) est fini et vaut \( n\)]
		Si \( k\in\eZ\), nous notons \( [k]_n\) la classe de \( k\) modulo \( n\), c'est-à-dire l'ensemble \( \{ k+pn\tq p\in \eZ \}\).

		Nous construisons l'isomorphisme \( \phi\colon G\to \eZ/n\eZ\) de la façon suivante:
		\begin{equation}
			\phi(g^m)=[m]_n.
		\end{equation}
		Cela est une bonne définition parce qu'une égalité du type \( g^m=g^{m'}\) implique que \( m\) et \( m'\) soient dans la même classe modulo \( n\). Nous vérifions que cela est un isomorphisme entre \( G\) et \( \eZ/n\eZ\).

		\begin{subproof}
			\spitem[Morphisme]
			Pour l'identité, si \( x=e\) alors \( m=0\) et \( \phi(e)=[0]_n\). Et si \( x=g^k\), \( y=g^l\) alors \( \phi(xy)=\phi(g^{k+l})=[k+l]_n=[k]_n+[l]_n=\phi(x)+\phi(y) \).
			\spitem[Injectif]
			Supposons \( \phi(g^k)=\phi(g^l)\) avec \( k\geq l\). Nous avons \( h^k=h^l\), donc \( h^{k-l}=e\), ce qui donne \( k-l\in [0]_n\) ou encore \( [k]_n=[l]_n\). En particulier \( g^k=g^l\).
			\spitem[Surjectif]
			La classe \( [k]_n\) est l'image de \( g^k\).
		\end{subproof}

		\spitem[L'ordre de \( G\) est infini]

		Si l'ordre de \( G\) est infini alors un élément \( x\in G\) s'écrit de façon unique sous la forme \( x=g^m\) avec \( m\in \eZ\). Dans ce cas nous définissons directement \( \phi(g^m)=m\).

		Le reste de la preuve est alors identique au cas d'ordre fini, mais sans les complications liées au modulo.

	\end{subproof}

	La dernière assertion s'obtient des précédentes par composition d'isomorphismes.

\end{proof}

%+++++++++++++++++++++++++++++++++++++++++++++++++++++++++++++++++++++++++++++++++++++++++++++++++++++++++++++++++++++++++++
\section{Automorphismes du groupe \texorpdfstring{\(  \eZ/n\eZ\)}{Z/nZ}}
%+++++++++++++++++++++++++++++++++++++++++++++++++++++++++++++++++++++++++++++++++++++++++++++++++++++++++++++++++++++++++++

Notons que \( \eZ/n\eZ=\eF_n\) est un groupe pour l'addition tandis que \( (\eZ/n\eZ)^*\) est un groupe pour la multiplication. Il ne peut donc pas y avoir d'équivoque.

\begin{theorem}[\cite{ooUDKTooTWYpzN}]   \label{ThoozyeSn}
	Pour chaque \( x\in (\eZ/n\eZ)^*\) nous considérons l'application
	\begin{equation}
		\begin{aligned}
			\sigma_x\colon \eZ/n\eZ & \to \eZ/n\eZ \\
			y                       & \mapsto xy.
		\end{aligned}
	\end{equation}
	L'application
	\begin{equation}
		\sigma\colon \big( (\eZ/n\eZ)^*,\cdot\big)\to \Aut\big( \eZ/n\eZ,+ \big)
	\end{equation}
	ainsi définie est un isomorphisme de groupes.
\end{theorem}
L'énoncé de ce théorème s'écrit souvent rapidement par
\begin{equation}
	\Aut(\eZ/n\eZ)=(\eZ/n\eZ)^*,
\end{equation}
mais il faut bien garder à l'esprit qu'à gauche on considère le groupe additif et à droite celui multiplicatif.

\begin{proof}
	Nous notons \( [x]\) la classe de \( x\) dans \( \eZ/n\eZ\). Nous avons \( \eZ/n\eZ=[1]\). Soit \( f\) un automorphisme de \( (\eZ/n\eZ,+)\); pour tout \( r\in \eZ\) nous avons
	\begin{equation}
		f([r])=f(r[1])=rf([1])=[r]f([1]).
	\end{equation}
	En particulier, puisque \( f\) est surjective, il existe un \( r\) tel que \( f([r])=[1]\). Pour un tel \( r\) nous avons \( [1]=[r]f([1])\), c'est-à-dire que nous avons montré que \( f([1])\) est inversible dans \(  \big( (\eZ/n\eZ)^*,\cdot\big)\). Nous montrons à présent que\footnote{Le \( \sigma\) donné ici est l'inverse de celui donné dans l'énoncé. Cela ne change évidemment rien à la validité de l'énoncé et de la preuve.}
	\begin{equation}
		\begin{aligned}
			\sigma\colon \Aut( (\eZ/n\eZ,+)) & \to \big( (\eZ/n\eZ)^*,\cdot \big) \\
			f                                & \mapsto f([1])
		\end{aligned}
	\end{equation}
	est un isomorphisme.

	Nous commençons par la surjectivité. Soit \( [a]\in (\eZ/n\eZ)^*\). Les élément \( [a]\) et \( [1]\) étant tous deux des générateurs de \( (\eZ/n\eZ,+)\), il existe un automorphisme de \( \eZ/n\eZ\) qui envoie \( [1]\) sur \( [a]\) par le lemme~\ref{LemZhxMit}. Cela prouve la surjectivité de \( \sigma\).

	En ce qui concerne l'injectivité, considérons des automorphismes \( f_1\) et \( f_2\) de \( (\eZ/n\eZ,+)\) tels que \( f_1([1])=f_2([1])\). Les automorphismes \( f_1\) et \( f_2\) prennent la même valeur sur un générateur et donc sur tout le groupe. Donc \( f_1=f_2\).

	Enfin nous prouvons que \( \sigma\) est un morphisme, c'est-à-dire que \( \sigma(f\circ g)=\sigma(f)\sigma(g)\). Nous avons
	\begin{subequations}
		\begin{align}
			f\big( g([1]) \big) & =f\big( g([1])[1] \big)=g([1])f([1])=\sigma(f)\sigma(g).
		\end{align}
	\end{subequations}
\end{proof}

Ce dernier résultat s'étend aux groupes cycliques.
\begin{proposition}     \label{PROPooBZOMooVOHoYf}
	Si \( G\) est un groupe cyclique\footnote{Définition \ref{DefHFJWooFxkzCF}.} d'ordre \( n\), alors
	\begin{equation}
		\Aut(G)=\big( (\eZ/n\eZ)^*,\cdot\big).
	\end{equation}
\end{proposition}

\begin{proof}
	Vu que \( G\) est cyclique, le lemme \ref{LemZhxMit} nous dit que \( G\) est isomorphe à \( (\eZ/n\eZ),+\). Maintenant le théorème \ref{ThoozyeSn} nous indique que

	Les égalités suivantes sont en réalité des isomorphismes de groupes :
	\begin{subequations}
		\begin{align}
			\Aut(G) & =\Aut\Big( \eZ/n\eZ,+ \Big)  \label{SUBEQooBNGBooFKRZUn}       \\
			        & =\Big( (\eZ/n\eZ)^*,\cdot \Big)    \label{SUBEQooZBMEooBSGPUB}
		\end{align}
	\end{subequations}
	Jusitifications.
	\begin{itemize}
		\item
		      Pour \eqref{SUBEQooBNGBooFKRZUn}. Le lemme \ref{LemZhxMit} nous dit que \( G\) est isomorphe à \( (\eZ/n\eZ, +)\), et le lemme \ref{LEMooDPISooRoAFmt} dit que des groupes isomorphes ont des groupes d'isomorphismes isomorphes.

		\item
		      Pour \eqref{SUBEQooZBMEooBSGPUB}. C'est le théorème \ref{ThoozyeSn}.
	\end{itemize}
\end{proof}

\begin{corollary}       \label{CorwgmoTK}
	Si \( p\) divise \( q-1\) alors \( \Aut(\eF_q)\) possède un unique sous-groupe d'ordre \( p\).
\end{corollary}

\begin{proof}
	Si \( a\) est un générateur de \( \eF_q^*\) alors le groupe
	\begin{equation}    \label{EqAdGiil}
		\gr\left( a^{\frac{ q-1 }{ p }} \right)
	\end{equation}
	est un sous-groupe d'ordre \( p\). En ce qui concerne l'unicité, soit \( S\) un sous-groupe d'ordre \( p\). Il est donc d'indice \( (q-1)/p\) dans \( \eF_q^*\) et le lemme~\ref{PropubeiGX} nous enseigne que le groupe donné en \eqref{EqAdGiil} est contenu dans \( S\). Il est donc égal à \( S\) parce qu'il a l'ordre de \( S\). Le fait que \( S\) soit normal est dû au fait que \( \eF_q^*\) est abélien.
\end{proof}


%+++++++++++++++++++++++++++++++++++++++++++++++++++++++++++++++++++++++++++++++++++++++++++++++++++++++++++++++++++++++++++
\section{Groupes abéliens finis}
%+++++++++++++++++++++++++++++++++++++++++++++++++++++++++++++++++++++++++++++++++++++++++++++++++++++++++++++++++++++++++++

Nous rappelons que l'exposant\index{exposant} d'un groupe fini est le \( \ppcm\) des ordres de ses éléments. Dans le cas des groupes abéliens finis, l'exposant joue un rôle important du fait qu'il existe un élément dont l'ordre est l'exposant. C'est le théorème suivant.

\begin{theorem}[Exposant dans un groupe abélien fini]
	Un groupe abélien fini contient un élément dont l'ordre est l'exposant du groupe.
\end{theorem}

\begin{proof}
	Soit \( G\) un groupe abélien fini et \( x\in G\), un élément d'ordre maximum \( m\). Nous montrons par l'absurde que l'ordre de tous les éléments de \( G\) divise \( m\). Soit donc \( y\in G\), un élément dont l'ordre ne divise pas \( m\); nous notons \( q\) son ordre. Vu que \( q\) ne divise pas \( m\), le nombre \( q\) possède au moins un facteur premier plus de fois que \( m\) : soit \( p\) premier tel que la décomposition de \( q\) contienne \( p^{\beta}\) et celle de \( m\) contienne \( p^{\alpha}\) avec \( \beta>\alpha\). Autrement dit,
	\begin{subequations}
		\begin{align}
			m=p^{\alpha}m' \\
			q=p^{\beta}q'
		\end{align}
	\end{subequations}
	où \( m'\) et \( q'\) ne contiennent plus le facteur \( p\). L'élément \( x\) étant d'ordre \( m\), l'élément \( x^{p^{\alpha}}\) est d'ordre \( m'\). De la même manière, l'élément \( y^{q'}\) est d'ordre \( p^{\beta}\). Étant donné que \( p^{\beta}\) et \( m'\) sont premiers entre eux, l'élément  \( x^{p^{\alpha}}y^{q'}\) est d'ordre \( p^{\alpha}m'>m\). D'où une contradiction avec le fait que \( x\) était d'ordre maximal.

	Par conséquent l'ordre de tous les éléments de \( G\) divise celui de \( x\) qui est alors le \( \ppcm\) des ordres de tous les éléments de \( G\), c'est-à-dire l'exposant de \( G\).
\end{proof}

\begin{proposition} \label{PropfPRVxi}
	Soit \( G\) un groupe abélien fini et \( x\in G\), un élément d'ordre maximum. Alors
	\begin{enumerate}
		\item
		      Il existe un morphisme \( \varphi\colon G\to \gr(x)\) tel que \( \varphi(x)=x\).
		\item   \label{ItemKRYwjU}
		      Il existe un sous-groupe \( K\) de \( G\) tel que \( G=\gr(x)\oplus K\).
	\end{enumerate}
\end{proposition}

\begin{proof}
	Nous notons \( a\) l'ordre de \( x\) qui est également l'exposant du groupe \( G\).

	Nous allons prouver la première partie par récurrence sur l'ordre du groupe. Si \( G=\gr(x)\), alors c'est évident. Soit \( H\) un sous-groupe propre de \( G\) contenant \( x\) et tel que le problème soit déjà résolu pour \( H\) : il existe un morphisme \( \varphi\colon H\to \gr(x)\) tel que \( \varphi(x)=x\). Soit \( y\in G\setminus H\), d'ordre \( b\). Nous allons trouver un morphisme \( \hat\varphi\colon \gr(H,y)\to \gr(x) \) telle que \( \hat\varphi(x)=x\).

	Pour cela nous commençons par construire les applications suivantes :
	\begin{equation}
		\begin{aligned}
			\tilde \varphi\colon \eZ/b\eZ\times H & \to \gr(x)               \\
			(\bar k,h)                            & \mapsto x^{kl}\varphi(h)
		\end{aligned}
	\end{equation}
	où \( l\) est encore à déterminer, et
	\begin{equation}
		\begin{aligned}
			p\colon \eZ/b\eZ\times H & \to \gr(y,H)  \\
			(\bar k,h)               & \mapsto y^kh.
		\end{aligned}
	\end{equation}
	Pour que \( \tilde \varphi\) soit bien définie, il faut que \( a\) divise \( bl\). L'application \( p\) est bien définie parce que \( \bar k\) est pris dans \( \eZ/b\eZ\) et que \( b\) est l'ordre de \( y\).

	Nous allons construire le morphisme \( \hat \varphi\) en considérant le diagramme
	\begin{equation}
		\xymatrix{%
			\ker(p) \ar@{^{(}->}[r]        &   \eZ/b\eZ\times H\ar[d]_{\tilde \varphi}\ar[r]^p&\gr(y,H)\ar[ld]^{\hat \varphi}\\
			&   \gr(x)
		}
	\end{equation}
	que l'on voudra être commutatif. Puisque \( p\) est surjective, les théorèmes d'isomorphismes nous disent que
	\begin{equation}
		\gr(y,H)\simeq\frac{ \eZ/b\eZ\times H }{ \ker p }.
	\end{equation}
	Si \( [\bar k,h]\) est la classe de \( (\bar k,h)\) modulo \( \ker(p)\) alors nous voudrions définir \( \hat \varphi\) par
	\begin{equation}        \label{EqeesVxc}
		\hat\varphi\big( [\bar k,h] \big)=\tilde \varphi(\bar k,h).
	\end{equation}
	Pour que cela soit bien défini, il faut que si \( (\bar r,z)\in \ker p\), alors,
	\begin{equation}
		\hat\varphi\big( [\bar k\bar r,hz] \big)=\hat\varphi\big( [\bar k,h] \big),
	\end{equation}
	c'est-à-dire que \( \tilde \varphi(\bar r,z)=e\). Du coup la définition \eqref{EqeesVxc} n'est bonne que si et seulement si
	\begin{equation}
		\ker(p)\subset\ker(\tilde\varphi ).
	\end{equation}
	Nous pouvons obtenir cela en choisissant bien \( l\).

	Déterminons d'abord le noyau de \( p\). Pour cela nous considérons un nombre \( \beta\) divisant \( b\) tel que \( \gr(y)\cap H=\gr(y^{\beta})\). Nous aurons \( p(\bar k,h)=e\) si et seulement si \( y^h=e\). En particulier \( h=y^{-k}\in\gr(y)\cap H=\gr(y^{\beta})\). Si \( h=(y^{\beta})^m=y^{m\beta}\), alors \( k=-m\beta\) et nous avons
	\begin{equation}
		\ker(p)=\{ (-m\beta,y^{m\beta})\tq m\in \eZ \}.
	\end{equation}
	En plus court : \( \ker(p)=\gr(\beta,y^{-\beta})\). Nous devons donc fixer \( l\) de telle sorte que \( \tilde \varphi(\beta,y^{-\beta})=e\). Étant donné que \( \varphi\) prend ses valeurs dans \( \gr(x)\), il existe un entier \( \alpha\) tel que \( \varphi(y^{-\beta})=x^{\alpha}\); en utilisant cet \( \alpha\), nous écrivons
	\begin{equation}
		\tilde \varphi(\beta,y^{-\beta})=x^{\beta l}\varphi(y^{-\beta})=x^{\beta l+\alpha}.
	\end{equation}
	Par conséquent nous choisissons \( l=-\alpha/\beta\). Nous devons maintenant vérifier que ce choix est légitime, c'est-à-dire que \( a\) divise \( bl\) et que \( \alpha/\beta\) est un entier.

	Étant donné que \( y\) est d'ordre \( b\),
	\begin{equation}
		e=\varphi(y^b)=\varphi(y^{-\beta b/\beta})=\varphi(y^{-\beta})^{b/\beta}=x^{b\beta/\alpha}.
	\end{equation}
	Par conséquent \( a\) divise \( \frac{ b\alpha }{ \beta }=-bl\).

	Pour voir que \( l\) est entier, nous nous rappelons que \( a\) est l'exposant de \( G\) (parce que \( x\) est d'ordre maximum) et que par conséquent \( b\) divise \( a\). Mais \( a\) divise \( \alpha\frac{ b }{ \beta }\). Donc \( \alpha/\beta\) est entier.

	Nous passons maintenant à la seconde partie de la preuve. Nous considérons un morphisme \( \varphi\colon G\to \gr(x)\) tel que \( \varphi(x)=x\). La première partie nous en assure l'existence. Nous montrons que
	\begin{equation}
		\begin{aligned}
			\psi\colon G & \to \gr(x)\oplus \ker(\varphi)                  \\
			g            & \mapsto \big( \varphi(g),g\varphi(g)^{-1} \big)
		\end{aligned}
	\end{equation}
	est un isomorphisme. D'abord \( g\varphi(g)^{-1}\) est dans le noyau de \( \varphi\) parce que \( \varphi(g)^{-1}\) étant dans \( \gr(x)\), et \( \varphi\) étant un morphisme,
	\begin{equation}
		\varphi\big( g\varphi(g)^{-1} \big)=\varphi(g)\varphi(g)^{-1}=e.
	\end{equation}
	L'application \( \psi\) est un morphisme parce que, en utilisant le fait que \( G\) est abélien,
	\begin{subequations}
		\begin{align}
			\psi(g_1g_2) & =\big( \varphi(g_1g_2),g_1g_2\varphi(g_1g_2)^{-1} \big)                        \\
			             & =\big( \varphi(g_1)\varphi(g_2),g_1\varphi(g_1)^{-1}g_2\varphi(g_2)^{-1} \big) \\
			             & =\psi(g_1)\psi(g_2).
		\end{align}
	\end{subequations}
	L'application \( \psi\) est injective parce que si \( \psi(g)=(e,e)\) alors \( \varphi(g)=e\) et \( g\varphi(g)^{-1}=e\), ce qui implique \( g=e\).

	Enfin \( \psi\) est surjective parce qu'elle est injective et que les ensembles de départ et d'arrivée ont même cardinal. En effet par le premier théorème d'isomorphisme (théorème~\ref{ThoPremierthoisomo}) appliqué à \( \varphi\) nous avons
	\begin{equation}
		| G |=| \gr(x) |\cdot | \ker(\varphi) |.
	\end{equation}
\end{proof}

\begin{theorem} \label{ThoRJWVJd}
	Si \( G\) est un groupe abélien fini non trivial, il exise un unique \( r>0\) et une unique liste de naturels \( (d_1,\ldots, d_r)\) tels que
	\begin{enumerate}
		\item
		      $G\simeq \eZ/d_1\eZ\oplus\ldots\oplus \eZ/d_r\eZ$
		\item
		      \( d_1\geq 1\)
		\item
		      \( d_i\) divise \( d_{i+1}\) pour tout \( i=1,\ldots, r-1\).
	\end{enumerate}
\end{theorem}

\begin{proof}
	Soit \( x_1\) un élément d'ordre maximal dans \( G\). Soit \( n_1\) son ordre et
	\begin{equation}
		H_1=\gr(x_1)=\eF_{n_1}.
	\end{equation}
	D'après la proposition~\ref{PropfPRVxi}\ref{ItemKRYwjU}, il existe un supplémentaire \( K_1\) tel que \( G=\eF_{n_1}\oplus K_1\). Si \( K_1=\{ e \}\) on s'arrête et on garde \( G=\eF_{n_1}\). Sinon on continue de la sorte en prenant \( x_2\) d'ordre maximal dans \( K_1\) etc.

	Nous devons maintenant prouver l'unicité de cette décomposition. Supposons deux décompositions avec les nombres \( (d_1,\ldots, d_r)\) et \( (s_1,\ldots, s_q)\) :
	\begin{equation}
		G=\eF_{d_1}\oplus\ldots\oplus \eF_{d_r}=\eF_{s_1}\oplus\ldots\oplus \eF_{s_q}.
	\end{equation}
	L'exposant de \( G\) est \( d_r\) et \( s_q\). Donc \( d_r=s_q\). Les complémentaires étant égaux nous avons
	\begin{equation}
		\eF_{d_1}\oplus\ldots\oplus \eF_{d_{r-1}}=\eF_{s_1}\oplus\ldots\oplus \eF_{s_{q-1}}.
	\end{equation}
	En continuant nous trouvons \( r=q\) et \( d_i=s_i\).
\end{proof}

%+++++++++++++++++++++++++++++++++++++++++++++++++++++++++++++++++++++++++++++++++++++++++++++++++++++++++++++++++++++++++++
\section{Groupes d'ordre \texorpdfstring{\(  pq\)}{pq}}
%+++++++++++++++++++++++++++++++++++++++++++++++++++++++++++++++++++++++++++++++++++++++++++++++++++++++++++++++++++++++++++
\index{quotient!de groupes}\index{sous-groupe!normal}

\begin{lemma}
	Soit \( G\) un groupe d'ordre \( pq\) où \( p\) et \( q\) sont des nombres premiers distincts. Nous supposons que \( p<q\).
	\begin{enumerate}
		\item
		      Le groupe \( G\) possède un unique \( q\)-Sylow.
		\item
		      Cet unique \( q\)-Sylow est normal dans \( G\).
		\item
		      Il n'est ni \( \{ e \}\) ni \( G\).
		\item
		      Le groupe \( G\) n'est pas un groupe simple\footnote{Pas de sous-groupes normaux non triviaux,~\ref{DefGroupeSimple}.}.
	\end{enumerate}
\end{lemma}

\begin{proof}
	Soit \( n_q\) le nombre de \( q\)-Sylow; par le théorème de Sylow~\ref{ThoUkPDXf}\ref{ITEMooETYHooXlUMQZ} le groupe \( G\) possède des \( q\)-Sylow et par~\ref{ThoUkPDXf}\ref{ItemkYbdzZ},
	\begin{equation}
		n_q\in[1]_q.
	\end{equation}
	De plus le nombre \( n_q\) divise \( | G |=pq\). Donc \( n_q\) vaut \( p\), \( q\) ou \( 1\). Avoir \( n_q=p\) n'est pas possible parce que \( n_q\in[1]_q\) et \( p<q\). Avoir \( n_q=q\) n'est pas possible non plus, pour la même raison. Donc \( n_q=1\). Notons \( H\) l'unique \( q\)-Sylow de \( G\).

	Le fait que \( H\) soit normal est une conséquence de~\ref{ThoUkPDXf}\ref{ItemMzNRVf} parce que le conjugué de \( H\) est encore un \( q\)-Sylow alors que \( H\) est l'unique \( q\)-Sylow.

	Vu que
	\begin{equation}
		1<p=| H |<pq=| G |,
	\end{equation}
	le sous-groupe \( H \) n'est ni réduit à l'identité ni le groupe entier.

	Par conséquent \( G\) n'est pas simple parce qu'il contient un sous-groupe normal non trivial.
\end{proof}

Avant de lire le théorème suivant, n'oubliez pas de lire la définition d'un produit semi-direct~\ref{DEFooKWEHooISNQzi}.
\begin{theorem}[\cite{ooUWQNooKHTzdO}] \label{ThoLnTMBy}
	Soient deux nombres premiers distincts\footnote{Le cas \( p=q\) sera traité par la proposition~\ref{PropssttFK}.} \( p\) et \( q\) avec \( q>p\).
	\begin{enumerate}
		\item
		      Si \( p\) ne divise pas \( q-1\) alors tout groupe d'ordre \( pq\) est cyclique et plus précisément le seul groupe (à isomorphisme près) d'ordre \( pq\) est \( \eZ/pq\eZ\).
		\item       \label{ITEMooFQXIooFLAiUD}
		      Si \( p\divides q-1\), alors il n'existe que deux groupes d'ordre \( pq\) :
		      \begin{itemize}
			      \item Le groupe abélien et cyclique \( \eZ/pq\eZ\).
			      \item Le produit semi-direct non abélien
			            \begin{equation}    \label{EqNuuTRE}
				            G=\eZ/q\eZ\times_{\varphi}\eZ/p\eZ
			            \end{equation}
			            où \( \varphi(\bar 1)\) est d'ordre \( p\) dans \( \Aut(\eZ/q\eZ)\).
		      \end{itemize}

		\item

		      Si \( p\) et \( q\) sont premiers entre eux, le produit est direct\quext{Cette affirmation me semble très bizarre. Comment deux nombres premiers distincts pourraient ne pas être premiers entre eux ???}.
	\end{enumerate}
\end{theorem}
\index{sous-groupe!distingué}
\index{groupe!fini}
\index{anneau!\( \eZ/n\eZ\)}
\index{nombre!premier}

\begin{proof}
	Division de la preuve en plusieurs parties.
	\begin{subproof}
		\spitem[Préliminaires avec Sylow]

		Soit un groupe \( G\) d'ordre \( pq\). Soient \( H\), un \( q\)-Sylow et \( K\), un \( p\)-Sylow de \( G\). Ils existent parce que \( p\) et \( q\) sont des diviseurs premiers de \( | G |\) (théorème de Sylow~\ref{ThoUkPDXf}). Si \( n_q\) est le nombre de \( q\)-Sylow dans \( G\) alors \( n_q\) divise \( | G |\) et \( n_q=1\mod q\). Donc d'abord \( n_q\) vaut \( 1\), \( p\) ou \( q\). Ensuite \( n_q=q\) est exclu par la condition \( n_q=1\mod q\); la possibilité \( n_q=p\) est également impossible parce que \( p=1\mod q\) est impossible avec \( p<q\). Donc \( n_q=1\) et \( H\) est normal dans \( G\).

		L'ensemble \( H\cap K\) est un sous-groupe à la fois de \( H\) et de \( K\), ce qui entraine que (théorème de Lagrange~\ref{ThoLagrange}) \( | H\cap K |\) divise à la fois \( p\) et \( q\). Nous en déduisons que \( | H\cap K |=1\) et donc que \( H\cap K=\{ e \}\).

		Étant donné que \( H\) est normal, l'ensemble \( HK\) est un sous-groupe de \( G\). De plus l'application
		\begin{equation}
			\begin{aligned}
				\psi\colon H\times K & \to HK     \\
				(h,k)                & \mapsto hk
			\end{aligned}
		\end{equation}
		est un bijection. Nous ne devons vérifier seulement l'injectivité. Supposons que \( hk=h'k'\). Alors \( e=h^{-1}h'k'k^{-1}\), et donc
		\begin{equation}
			h^{-1} h'=(k'k^{-1})^{-1}\in H\cap K=\{ e \}.
		\end{equation}
		Par conséquent \( | pq |=| H\times K |=| HK |\), et \( HK=G\). Le corolaire~\ref{CoroGohOZ} nous indique que
		\begin{equation}    \label{EqGjQjFN}
			G=H\times_{\varphi}K
		\end{equation}
		où \( \varphi\) est l'action adjointe. Nous devons maintenant identifier cette action. En d'autres termes, nous savons que \( H=\eZ/q\eZ\) et \( K=\eZ/p\eZ\) et que \( \varphi\colon \eZ/p\eZ\to \Aut(\eZ/q\eZ)\) est un morphisme. Nous devons déterminer les possibilités pour \( \varphi\).

		Soit \( n_p\) le nombre de \( p\)-Sylow de \( G\). Comme précédemment, \( n_p\) vaut \( 1\), \( p\) ou \( q\) et la possibilité \( n_p=p\) est exclue. Donc \( n_p\) est \( 1\) ou \( q\).

		\spitem[Si \( p\) ne divise pas \( q-1\)]

		Si \( p\) ne divise pas \( q-1\) alors il n'est pas possible d'avoir \( n_p=q\) parce que \( n_p\in [1]_p\). Or dire \( n_p=q\) demanderait \( q\in [1]_p\), c'est-à-dire \( q=kp+1\), qui impliquerait que \( p\) divise \( q-1\).

		La seule possibilité est que \( n_p=1\). Dans ce cas, \( K\) est également normal dans \( G\). Du coup le produit semi-direct \eqref{EqGjQjFN} est en réalité un produit direct (\( \varphi\) est triviale) et nous avons
		\begin{equation}
			G=\eZ/q\eZ\times \eZ/p\eZ=\eZ/pq\eZ.
		\end{equation}

		\spitem[Si \( p\) divise \( q-1\)]

		Cette fois \( n_p=1\) et \( n_p=q\) sont tous deux possibles. Ce que nous savons est que \( \varphi(\eZ/p\eZ)\) est un sous-groupe de \( \Aut(\eZ/q\eZ)\). Par le premier théorème d'isomorphisme~\ref{ThoPremierthoisomo}, nous avons
		\begin{equation}
			| \varphi(\eZ/p\eZ) |=\frac{ | \eZ/p\eZ | }{ | \ker\varphi | },
		\end{equation}
		ce qui signifie que \( | \varphi(\eZ/p\eZ) |\) divise \( | \eZ/p\eZ |=p\). Par conséquent, \( | \varphi(\eZ/p\eZ) |\) est égal à \( 1\) ou \( p\). Si c'est \( 1\), alors l'action est triviale et le produit est direct.

		Nous supposons que \( | \varphi(\eZ/p\eZ) |=p\). Le corolaire~\ref{CorwgmoTK} nous indique que \( \Aut(\eZ/q\eZ)\) possède un unique sous-groupe d'ordre \( p\) que nous notons \( \Gamma\); c'est-à-dire que \( \Gamma=\Image(\varphi)\). Vu que \( \varphi\colon \eZ/p\eZ\to \Aut(\eZ/q\eZ)\) est un morphisme, \( \Gamma\) est généré par \( \varphi(\bar 1)\) qui est alors un élément d'ordre \( p\), comme annoncé.

		\spitem[Unicité]
		Nous nous attaquons maintenant à l'unicité. Soient \( \varphi\) et \( \varphi'\) deux morphismes non triviaux \( \eZ/p\eZ\to \Aut(\eZ/q\eZ)\). Étant donné que \( \Aut(\eZ/q\eZ)\) ne possède qu'un seul sous-groupe d'ordre \( p\), nous savons que \( \Image(\varphi)=\Image(\varphi')=\Gamma\). Nous pouvons donc parler de \( \varphi'^{-1}\) en tant qu'application de \( \eZ/p\eZ\) dans \( \Gamma\). Nous montrons que
		\begin{equation}
			\begin{aligned}
				f\colon \eZ/q\eZ\times_{\varphi}\eZ/p\eZ & \to \eZ/q\eZ\times_{\varphi'}\eZ/p\eZ \\
				(h,k)                                    & \mapsto (h,\alpha(k))
			\end{aligned}
		\end{equation}
		où \( \alpha=\varphi'^{-1}\circ\varphi\) est un isomorphisme de groupes. Le calcul est immédiat :
		\begin{subequations}
			\begin{align}
				f(h_1,k_1)f(h_2mk_2) & =\big( h_1,\alpha(k_1) \big)(h_2,\alpha(k_2))           \\
				                     & =\big( h_1\varphi'(\alpha(k_1))h_2m\alpha(k_1k_2) \big) \\
				                     & =f\big( h_1\varphi(k_1)h_2,k_1k_2 \big)                 \\
				                     & =f\big( (h_1,k_1),(h_2,k_2) \big).
			\end{align}
		\end{subequations}
		Par conséquent \( \eZ/q\eZ\times_{\varphi}\eZ/p\eZ\simeq \eZ/q\eZ\times_{\varphi'} \eZ/p\eZ\).
	\end{subproof}
\end{proof}

Note : il existe des nombres premiers \( p\) et \( q\) tels que \( q=1\mod p\). Par exemple \( 7=1\mod 3\).

\begin{proposition}[\cite{PDFpersoWanadoo}]     \label{PROPooNSRYooEodtUl}
	Soit \( G\) un groupe fini d'ordre \( pq\) où \( p\) et \( q\) sont deux nombres premiers distincts vérifiant
	\begin{subequations}
		\begin{numcases}{}
			p\neq 1\mod q\\
			q\neq 1\mod p.
		\end{numcases}
	\end{subequations}
	Alors \( G\) est cyclique, abélien et
	\begin{equation}
		G\simeq \eZ/p\eZ\times \eZ/q\eZ.
	\end{equation}
\end{proposition}

\begin{proof}
	Soient \( n_p\) et \( n_q\) les nombres de \( p\)-Sylow et \( q\)-Sylow. Par le théorème de Sylow~\ref{ThoUkPDXf}, \( n_p\) divise \( pq\) et \( n_p=1\mod p\). Le second point empêche \( n_p\) de diviser \( p\). Par conséquent \( n_p\) divise \( q\) et donc \( n_p\) vaut \( 1\) ou \( q\). La possibilité \( n_p=q\) est exclue par l'hypothèse \( q\neq 1\mod p\). Donc \( n_p=1\), et de la même façon nous obtenons \( n_q=1\).

	Soient \( S\) l'unique \( p\)-Sylow et \( T\), l'unique \( q\)-Sylow. Pour les mêmes raisons que celles exposées plus haut, ce sont deux sous-groupes normaux dans \( G\). Étant donné que \( S\) est d'ordre \( p^n\) pour un certain \( n\) et que l'ordre de \( S\) doit diviser celui de \( G\), nous avons \( |S|=p\). De la même façon, \( | T |=q\). Par conséquent \( S\) est un groupe cyclique d'ordre \( p\) et nous considérons \( x\), un de ses générateurs. De la même façon soit \( y\), un générateur de \( T\).

	Nous montrons maintenant que \( x\) et \( y\) commutent, puis que \( xy\) engendre \( G\). Nous savons que \( S\cap T\) est un sous-groupe à la fois de \( S\) et de \( T\), de telle façon que \( | S\cap T |\) divise à la fois \( | S |=p\) et \( | T |=q\). Nous avons donc \( | S\cap T |=1\) et donc \( S\cap T\) se réduit au neutre. Par ailleurs, \( S\) et \( T\) sont normaux, donc
	\begin{subequations}
		\begin{align}
			(xyx^{-1})y^{-1}\in T \\
			x(yx^{-1}y^{-1})\in S,
		\end{align}
	\end{subequations}
	donc \( xyx^{-1}y^{-1}=e\), ce qui montre que \( xy=yx\).

	Montrons que \( xy\) engendre \( G\). Soit \( m>0\) tel que \( (xy)^m=e\). Pour ce \( m\) nous avons \( x^m=y^{-m}\) et \( y^{-m}=x^m\), ce qui signifie que \( x^m\) et \( y^m\) appartiennent à \( S\cap T\) et donc \( x^m=y^m=e\). Les nombres \( p\) et \( q\) divisent donc tous deux \( m\); par conséquent \( \ppcm(p,q)=pq\) divise \( m\). Nous en concluons que \( xy\) est d'ordre \( pq\) (il ne peut pas être plus) et qu'il est alors générateur.

	Pour la suite nous allons d'abord prouver que \( G=ST\) puis que \( G\simeq S\times T\). Nous savons déjà que \( | S\cap T |=1\), ce qui nous amène à dire que \( | ST |=| S | |T |\). En effet si \( s,s'\in S\) et \( t,t'\in t\) et si \( st=s't'\), alors \( t=s^{-1}s't'\), ce qui voudrait dire que \( s^{-1}s'\in T\) et donc que \( s^{-1}s'=e\). Au final nous avons
	\begin{equation}
		| ST |=| S | |T |=pq=| G |.
	\end{equation}
	Par conséquent \( G=ST\). En nous rappelant que \( S\cap T=\{ e \}\) et que \( S\) et \( T\) sont normaux, le lemme~\ref{LemHUkMxp} nous dit que \( G\simeq S\times T\). Le groupe \( S\) étant cyclique d'ordre \( p\) nous avons \( S=\eZ/p\eZ\) et pour \( T\), nous avons la même chose : \( T=\eZ/q\eZ\). Nous concluons que
	\begin{equation}
		G\simeq \eZ/p\eZ\times \eZ/q\eZ.
	\end{equation}
\end{proof}



\begin{theorem}[Théorème de Burnside\cite{FabricegPSFinis}] \label{ThoImkljy}
	Le centre d'un \( p\)-groupe non trivial est non trivial.
\end{theorem}

\begin{proof}
	Soit \( G\) un \( p\)-groupe non trivial. Nous considérons l'action adjointe \( G\) sur lui-même. Les points fixes de cette action sont les éléments du centre :
	\begin{equation}
		\mZ_G=\{ z\in G\tq \sigma_x(z)=z, \forall x\in G \}=\Fix_G(G).
	\end{equation}
	Nous utilisons l'équation aux classes \eqref{PropUyLPdp} pour dire que \( | G |=| \mZ_G |\mod p\). Mais \( | \mZ_G |\) n'est pas vide parce qu'il contient l'identité. Donc \( | \mZ_G |\) est au moins d'ordre \( p\).
\end{proof}

\begin{proposition} \label{PropssttFK}
	Si \( p\) est un nombre premier, tout groupe d'ordre \( p\) ou \( p^2\) est abélien.
\end{proposition}
Rappel : un groupe d'ordre \( p\) ou \( p^2\) est automatiquement un \( p\)-groupe.

\begin{proof}
	Si \( | G |=p\), alors le théorème de Cauchy~\ref{ThoCauchyGpFini} nous donne l'existence d'un élément d'ordre \( p\). Cet élément est alors automatiquement générateur, \( G\) est cyclique et donc abélien.

	Si par contre \( G\) est d'ordre \( p^2\), alors les choses se compliquent (un peu). D'après le théorème de Burnside~\ref{ThoImkljy}, le centre \( \mZ\) n'est pas trivial; il est alors d'ordre \( p\) ou \( p^2\). Supposons qu'il soit d'ordre \( p\) et prenons \( x\in G\setminus\mZ\). Alors le stabilisateur de \( x\) pour l'action adjointe contient au moins \( \mZ\) et \( x\), c'est-à-dire que \( |\Fix_G(x)|\geq p+1\). Étant donné que \( \Fix_G(x)\) est un sous-groupe, son ordre est automatiquement \( 1\), \( p\) ou \( p^2\). En l'occurrence, il doit être \( p^2\) (parce que plus grand que \( p\)), et donc \( x\) doit être central, ce qui est une contradiction.
\end{proof}



%---------------------------------------------------------------------------------------------------------------------------
\subsection{Fonction indicatrice d'Euler}
%---------------------------------------------------------------------------------------------------------------------------

\begin{definition}		\label{DEFooZRYMooZCozga}
	La \defe{fonction indicatrice d'Euler}{indicatrice d'Euler} est l'application
	\begin{equation}
		\begin{aligned}
			\varphi\colon \eN^* & \to \eN^*                                                                    \\
			n                   & \mapsto \Card\Big(   \{ m\in \eN^*\tq 1\leq m\leq n, \pgcd(m,n)=1 \}  \Big).
		\end{aligned}
	\end{equation}
\end{definition}

\begin{normaltext}
	\begin{enumerate}
		\item
		      Voir le thème \ref{THMooUDYMooCCXdbw} pour des formules concernant l'indicatrice d'Euler.
		\item

		      Une vidéo de 3Blue1Brown : \url{https://www.youtube.com/watch?v=EK32jo7i5LQ}.
	\end{enumerate}
\end{normaltext}

\begin{lemma}[\cite{BIBooYVNFooEVOwyw}]		\label{LEMooVGDHooStUaKH}
	L'élément \( [m]_n\) est inversible dans le groupe \( \big( (\eZ/n\eZ)^*,\cdot \big)\) si et seulement si \( \pgcd(m,n)=1\).
\end{lemma}

\begin{proof}
	Dans les deux sens.
	\begin{subproof}
		\spitem[\( \Rightarrow\)]
		%-----------------------------------------------------------
		Si \( [m_n]\) est inversible, il existe \( u\in \eZ\) tel que \( [u]_n[m]_n=[1]_n\). Cela donne \( [um]_n=[1]_n\) ou encore \( um\in [1]_n\), c'est-à-dire \( um=1+vn\). Le théorème de Bézout \ref{ThoBuNjam} conclut que \( \pgcd(m,n)=1\).

		\spitem[\( \Leftarrow\)]
		%-----------------------------------------------------------
		Si \( \pgcd(m,n)=1\), alors le théorème de Bézout \ref{ThoBuNjam} nous dit qu'il existe \( u,v\in \eZ\) tels que \( um+vn=1\). Cela donne directement \( um=1-vn\in [1]_n\) et donc \( [u]_n\) est un inverse de \( [m]_n\) dans le groupe multiplicatif \( \eZ/n\eZ\).
	\end{subproof}
\end{proof}

\begin{lemma}		\label{LEMooCLYEooONhWKs}
	Nous avons
	\begin{equation}
		\varphi(n)=\Card\big( (\eZ/n\eZ)^{\times} \big)
	\end{equation}
	où \( A^{\times}\) est le groupe des inversibles (pour la multiplication) dans l'anneau \( A\).
\end{lemma}

\begin{proof}
	En utilisant le lemme \ref{LEMooVGDHooStUaKH},
	\begin{subequations}
		\begin{align}
			\big( \eZ/n\eZ \big)^{\times} & =\{ [m]_n\tq \pgcd(m,n)=1 \}               \\
			                              & =\{ [m]_n\tq 1\leq m\leq n,\pgcd(m,n)=1 \}
		\end{align}
	\end{subequations}
	Comme deux entiers différents entre \( 1 \) et \( n\) ne peuvent pas être dans la même classe modulo \( n\), il y a bijection entre le dernier ensemble et \( \{ 1\leq m\leq n\tq \pgcd(m,n)=1 \}\). Donc
	\begin{equation}
		\varphi(n)=\Card\big( \{ [m]_n\tq 1\leq m\leq n,\pgcd(m,n)=1 \} \big)=\Card\big( (\eZ/n\eZ)^{\times} \big).
	\end{equation}
\end{proof}

\begin{lemma}[\cite{BIBooHIFAooWOavtO}]		\label{LEMooRGIYooRxgyCO}
	Soit \( n,d\in \eN\) tels que \( d\divides n\). Nous notons \( q=n/d\) et \( G=\{ k[q]_n\tq 0\leq k\leq d-1 \}\). Alors, pour \( r\in \eZ\) nous avons \( d[r]_n=[0]_n\) si et seulement si \( [r]_n\in G\).
\end{lemma}

\begin{proof}
	Dans les deux sens.
	\begin{subproof}
		\spitem[\( \Rightarrow\)]
		%-----------------------------------------------------------
		Nous supposons que \( d[r]_n=[0]_n\). Étant donné que \( n=dq\) nous avons les implications suivantes :
		\begin{equation}
			d[r]_n=[0]_n\Rightarrow d\divides dr\Rightarrow dq\divides dr\Rightarrow q\divides r.
		\end{equation}
		Nous avons donc \( r=kq\) pour un certain \( k\in \eZ\). Par la division euclidienne \ref{ThoDivisEuclide}, il existe \( s,t\in \eN\) tels que \( k=sd+t\) avec \( t<d\). Avec ça, nous avons le calcul
		\begin{equation}
			[r]_n=k[q]_n=sd[q]_n+t[q]_n=\underbrace{s[dq]_n}_{=[0]_n}+t[q]_n\in G.
		\end{equation}

		\spitem[$\Leftarrow$]
		%-----------------------------------------------------------
		Si \( [r]_n\in G\), il existe \( 0\leq k\leq d-1\) tel que \( [r]_n=k[q]_n\). De ce fait,
		\begin{equation}
			d[r]_n=dk[q]_n=k[dq]_n=[0]_n.
		\end{equation}
		Et voilà.
	\end{subproof}
\end{proof}

La proposition suivante est un pas important dans l'algorithme de Shor permettant aux ordinateurs quantiques de factoriser rapidement des grands nombres\cite{BIBooBBRQooJxksHX}.

\begin{theorem}[Euler-Fermat\cite{BIBooPCENooPSwCvv}]		\label{THOooXMBSooXrrfOP}
	Deux énoncés très similaires.
	\begin{enumerate}
		\item
		      Soient \( a,n\) premiers entre eux dans \( \eN\). Alors
		      Nous avons la formule
		      \begin{equation}
			      a^{\varphi(n)}\in [1]_n.
		      \end{equation}
		\item		\label{ITEMooJHZBooXVKMlT}
		      Soient \( A,B\in \eN\) premiers entre eux. Alors il existe \( p,m\in \eN\) tels que
		      \begin{equation}
			      A^p=mB+1.
		      \end{equation}
	\end{enumerate}
\end{theorem}

\begin{proof}
	Vu que \( \pgcd(a,n)=1\), le théorème de Bézout \ref{ThoBuNjam} donne \( ua+vn=1\), c'est-à-dire \( u[a]_n\in [1]_n\). Autrement dit, \( [a]_n\) est inversible dans \( \eZ/n\eZ\). De ce fait l'application
	\begin{equation}
		\begin{aligned}
			f\colon (\eZ/n\eZ)^{\times} & \to (\eZ/n\eZ)^{\times} \\
			x                           & \mapsto [a]_nx
		\end{aligned}
	\end{equation}
	est une bijection.

	Un produit peut être réindexé par une bijection\footnote{Proposition \ref{DEFooLNEXooYMQjRo}. Pour rappel, le produit n'est rien d'autre qu'une somme pour un groupe dont la loi est notée multiplicativement.}. En posant \( P=\prod_{x\in (\eZ/n\eZ)^{\times}}x\), nous avons
	\begin{subequations}
		\begin{align}
			P & =\prod_{x\in (\eZ/n\eZ)^{\times}}x                                                                \\
			  & =  \prod_{x\in (\eZ/n\eZ)^{\times}}f(x)                                                           \\
			  & =\prod_{x\in (\eZ/n\eZ)^{\times}}[a]_nx                                                           \\
			  & =\Big( \prod_{x\in (\eZ/n\eZ)^{\times}} [a]_n \Big)\Big( \prod_{x\in (\eZ/n\eZ)^{\times}} x \Big) \\
			  & =[a]_n^{\Card\big( (\eZ/n\eZ)^{\times} \big)}P
		\end{align}
	\end{subequations}
	En simplifiant par \( P\) dans \( \eZ/n\eZ\) nous trouvons
	\begin{equation}
		[a]_n^{\Card\big( (\eZ/n\eZ)^{\times} \big)}=[1]_n.
	\end{equation}
	Et comme le lemme \ref{LEMooCLYEooONhWKs} donne \( \Card\big( (\eZ/n\eZ)^{\times} \big)=\varphi(n)\), nous avons trouvé \( [a]_n^{\varphi(n)}=[1]_n\). Autrement dit,
	\begin{equation}
		a^{\varphi(n)}\in [1]_n.
	\end{equation}

	Le point \ref{ITEMooJHZBooXVKMlT} n'est qu'une reformulation. Vu que \( A\) et \( B\) sont premiers entre eux, nous venons de voir que \( A^{\varphi(B)}\in [1]_B\). Il existe donc \( m\in \eZ\) tel que \( A^{\varphi(B)}=1+mB\). Vu que \( A\) et \( B\) sont dans \( \eN\), le nombre \( m\) est contraint d'être positif, c'est-à-dire \( m\in \eN\).
\end{proof}



\begin{lemma}[\cite{BIBooHIFAooWOavtO}]		\label{LEMooKPKBooPbrHkI}
	Soient \( d\divides n\) dans \( \eN^*\). Nous considérons le groupe additif \( G_d=\{ k[q]_n\tq 0\leq k\leq d-1 \}\). Les éléments d'ordre\footnote{Ordre d'un élément, définition \ref{DEFooKSTVooOObpgC}.} \( d\) dans \( (\eZ/n\eZ,+)\) sont les générateurs de \( G_d\).
\end{lemma}

\begin{proof}
	Les générateurs de \( G_d\) sont d'ordre \( d\) parce que \( | G_d |=d\). Ça, c'était le sens facile. Dans l'autre sens, si \( [r]_n\) est d'ordre \( d\), alors \( d[r]_n=[0]_n\). D'après le lemme \ref{LEMooRGIYooRxgyCO}, cela prouve que \( [r]_n\in G_d\).

	Comme le groupe engendré par \( [r]_n\) est d'ordre \( d\), il est tout \( G_d\). Donc \( [r]_n\) est générateur de \( G_d\).
\end{proof}


\begin{lemma}[\cite{BIBooDNSJooTXvNqc}]		\label{LEMooQGGLooDkkmcF}
	L'ensemble des générateurs de \( (\eZ/n\eZ,+)\) est \( (\eZ/n\eZ)^{\times}\).
\end{lemma}

\begin{proof}
	Si \( [r]_n\) est générateur de \( \eZ/n\eZ\), il existe \( k\) tel que \( k[r]_n=[1]_n\). Dans ce cas, \( [k]_n\) est l'inverse de \( [r]_n\) pour la multiplication. Donc \( [r]_n\in(\eZ/n\eZ)^{\times}\).

	À l'inverse, si \( [r]_n\) est inversible, alors il existe \( k\) tel que \( k[r]_n=[1]_n\). Dans ce cas, \( [t]_n=kt[r]_n\), ce qui montre que \( [r]_n\) est générateur (pour l'addition).
\end{proof}

\begin{lemma}[\cite{BIBooDNSJooTXvNqc}]		\label{LEMooRMWRooRSjGPL}
	Si \( G\) est un groupe cyclique d'ordre \( n\), alors \( G\) possède \( \varphi(n)\) générateurs.
\end{lemma}

\begin{proof}
	Vu que tous les groupes cycliques d'ordre \( n\) sont isomorphes à \( (\eZ/n\eZ),+\), nous nous contentons de prouver le résultat pour ce groupe. Le lemme \ref{LEMooQGGLooDkkmcF} montre que \( (\eZ/n\eZ,+)\) possède \( \Card\big( (\eZ/n\eZ)^{\times} \big)\) générateurs.

	Mais le lemme \ref{LEMooCLYEooONhWKs} assure que \( \Card\big( (\eZ/n\eZ)^{\times} \big)=\varphi(n)\).
\end{proof}

\begin{proposition}       \label{PROPooYHUDooUROTiN}
	Nous avons la formule
	\begin{equation}        \label{EqTPHqgJ}
		n=\sum_{d\divides n}\varphi(d).
	\end{equation}
\end{proposition}

\begin{proof}
	Nous notons \( H_d\) la partie de \( (\eZ/n\eZ,+)\) composée des éléments d'ordre \( d\). Nous avons vu dans le lemme \ref{LEMooKPKBooPbrHkI} que \( H_d\) sont justement les générateurs de \( G_d\) -- voir le lemme pour la notation. Mais comme \( G_d\) est un groupe cyclique d'ordre \( d\), il contient \( \varphi(d)\) générateurs (lemme \ref{LEMooRMWRooRSjGPL}) : \( \Card(H_d)=\varphi(d)\).

	Vu que tous les éléments de \( \eZ/n\eZ\) ont un ordre qui divise \( n\) (corolaire \ref{CorpZItFX}), nous avons l'union disjointe
	\begin{equation}
		\eZ/n\eZ=\bigcup_{d\divides n}H_d,
	\end{equation}
	et donc au niveau des cardinals,
	\begin{equation}
		n=\Card(\eZ/n\eZ)=\sum_{d\divides n}\Card(H_d)=\sum_{d\divides n}\varphi(d).
	\end{equation}
\end{proof}


\begin{lemma}       \label{LEMooBEJOooDqTirj}
	Si \( p\) est un nombre premier, alors \( \varphi(p^n)=p^n-p^{n-1}\).
\end{lemma}

\begin{proof}
	Les éléments de \( \{ 1,\ldots,p^n \}\) qui ont un \( \pgcd\) différent de \( 1\) avec \( p^n\) sont des nombres qui s'écrivent sous la forme \( qp\) avec \( q\leq p^{n-1}\)\footnote{Corolaire~\ref{CORooQIMHooUzLUJY}.}. Il y a évidemment \( p^{n-1}\) tels nombres.

	Par conséquent le cardinal de \( P_{p^n}\) est \( \varphi(p^{n})=p^n-p^{n-1}\).
	%TODOooWJIYooYtATMi Il faut élucider ce qu'est P_n, voir 2754128708
\end{proof}

\begin{probleme}
	%2754128708
	\( P_n\) n'a pas été défini.

	Définition proposée (et vue par après): \( P_n = \{ m \in \eN \tq \pgcd(m,n) = 1 \}. \) À mettre donc en lien avec \( \Delta_d\).
\end{probleme}

%---------------------------------------------------------------------------------------------------------------------------
\subsection{Générateurs}
%---------------------------------------------------------------------------------------------------------------------------

\begin{proposition}     \label{PropZnmuphiGensn}
	Soit \( n\in\eN\setminus\{ 0 \}\) et le groupe (additif) \( \eZ/n\eZ\). L'élément \( [x]_n\) est un générateur de \( \eZ/n\eZ\) si et seulement si \( x\in P_n\). En particulier \( \eZ/n\eZ\) est un groupe contenant \( \varphi(n)\) générateurs.
\end{proposition}

\begin{proof}
	Nous avons \( \gr\big( [1]_n \big)=\eZ/n\eZ\). L'élément \( [x]_n\) sera générateur si et seulement si il génère \( [1]_n \), c'est-à-dire si il existe \( u\) tel que \( u[x]_n=[1]_n\). Cette dernière égalité étant une égalité de classes dans \( \eZ/n\eZ\), elle sera vraie si et seulement si il existe \( v\) tel que
	\begin{equation}
		ux+vn=1.
	\end{equation}
	Cela signifie entre autres que\footnote{Corolaire~\ref{CorgEMtLj}} \( x\eZ+n\eZ=\eZ\), et aussi que \( \pgcd(x,n)=1\) par le théorème de Bézout~\ref{ThoBuNjam}, et donc que \( x\in P_n\).
\end{proof}

\begin{corollary}\label{CORooMBLSooMHKmAq}
	Un groupe monogène d'ordre \( n\) possède \( \varphi(n)\) générateurs, où \( \varphi\) est la fonction indicatrice d'Euler définie en~\ref{DEFooZRYMooZCozga}.
\end{corollary}

\begin{proof}
	Le théorème~\ref{THOooDOMZooOEYHAe} nous dit qu'un groupe monogène d'ordre \( n\) est isomorphe à \( \eZ/n\eZ\). La proposition~\ref{PropZnmuphiGensn} nous indique que \( \eZ/n\eZ\) possède \( \varphi(n)\) générateurs.
\end{proof}

%---------------------------------------------------------------------------------------------------------------------------
\subsection{Fonction indicatrice d'Euler (propriétés)}
%---------------------------------------------------------------------------------------------------------------------------
\label{subSecKGDFooAbETjs}

\begin{corollary}       \label{CorlvTmsf}
	Deux propriétés.
	\begin{enumerate}
		\item
		      L'indicatrice d'Euler est multiplicative : si \( p\) est premier avec \( q\), alors
		      \begin{equation}
			      \varphi(pq)=\varphi(p)\varphi(q).
		      \end{equation}
		\item
		      Si \( p\) est un nombre premier,
		      \begin{equation}
			      \varphi(p)=(p-1).
		      \end{equation}
	\end{enumerate}
\end{corollary}

\begin{proof}
	Nous savons que si \( p\) et \( q\) sont premiers entre eux, alors le théorème~\ref{ThoLnTMBy} nous donne l'isomorphisme de groupe
	\begin{equation}
		(\eZ/pq\eZ,+)\simeq(\eZ/p\eZ,+)\times(\eZ/q\eZ,+).
	\end{equation}
	Un élément \( (x,y)\) est générateur du produit si et seulement si \( x\) est générateur de \( \eZ/p\eZ\) et \( y\) est générateur de \( \eZ/q\eZ\). Par la proposition~\ref{PropZnmuphiGensn}, il y a \( \varphi(p)\varphi(q)\) tels éléments. Par ailleurs le nombre de générateurs de \( \eZ/pq\eZ\) est \( \varphi(pq)\), d'où l'égalité.

	Si \( p\) est premier, nous avons \( \varphi(p)=p-1\) parce que tous les entiers de \( \{ 1,\ldots, p-1 \}\) sont premiers avec \( p\).
\end{proof}


%+++++++++++++++++++++++++++++++++++++++++++++++++++++++++++++++++++++++++++++++++++++++++++++++++++++++++++++++++++++++++++
\section{Groupe symétrique, groupe alterné}
%+++++++++++++++++++++++++++++++++++++++++++++++++++++++++++++++++++++++++++++++++++++++++++++++++++++++++++++++++++++++++++
\label{SECooZFYQooFfopMa}

La définition des permutations et du groupe symétrique sont~\ref{DEFooJNPIooMuzIXd}. Voir aussi le thème~\ref{THEMEooQEEWooXDhvhv}.

%---------------------------------------------------------------------------------------------------------------------------
\subsection{Le groupe alterné}
%---------------------------------------------------------------------------------------------------------------------------

\begin{definition}      \label{DEFooEIVIooFvVkHH}
	Le groupe \( A_n\)\nomenclature[R]{\( A_n\)}{groupe alterné} des permutations paires\footnote{Définition \ref{PROPooKRHEooAxtmRv}.} dans \( S_n\) est le \defe{groupe alterné}{alterné!groupe}\index{groupe!alterné}.
\end{definition}

\begin{proposition} \label{PROPooCPXOooVxPAij}
	À propos du groupe alterné dans le groupe symétrique.
	\begin{enumerate}
		\item
		      Le groupe alterné \( A_n\) est un sous-groupe caractéristique\footnote{Définition~\ref{DEFooUXXTooCCLmQe}.} de \( S_n\)
		\item   \label{ITEMooWXXUooOWvFgE}
		      Le sous-groupe \( A_n\) est d'indice \( 2\) dans \( S_n\).
		\item       \label{ITEMooGGAHooRYgNqq}
		      Le sous-groupe \( A_n\) est l'unique sous-groupe d'indice\footnote{Définition \ref{DEFooMPIAooIeZNaR}.} \( 2\) de \( S_n\).
	\end{enumerate}
\end{proposition}

\begin{proof}
	Soit \( \alpha\in \Aut(S_n)\). Étant donné que \( \epsilon\circ\alpha\) est un homomorphisme\footnote{Voir \ref{NORMooTXFWooApjnOY}.} surjectif sur \( \{ -1,1 \}\), par unicité de cet homomorphisme, nous avons \( \epsilon\circ\alpha=\epsilon\), et donc \( \alpha(A_n)=A_n\). Par le premier théorème d'isomorphisme~\ref{ThoPremierthoisomo}, il existe un isomorphisme
	\begin{equation}
		f\colon S_n/\ker(\epsilon)\to \Image(\epsilon).
	\end{equation}
	En égalant le nombre d'éléments nous avons \( | S_n:\ker\epsilon |=| S_n:A_n |=2\).

	Nous prouvons maintenant l'unicité. Soit \( H\) un sous-groupe d'indice \( 2\) dans \( S_n\). Par le lemme \ref{LemSkIOOG}, \( H\) est distingué et nous pouvons considérer le groupe \( S_n/H\). Ce dernier ayant \( 2\) éléments, il est isomorphe à \( \{ -1,1 \}\). Soit \( \theta\) l'isomorphisme. On note \( \varphi\) le morphisme canonique \( \varphi\colon S_n\to S_n/H\) :
	\begin{equation}    \label{EqSZBPTH}
		\xymatrix{%
			S_n \ar[r]^{\varphi}        &   S_n/H\ar[r]^{\theta}&\{ -1,1 \}.
		}
	\end{equation}
	La composition \( \theta \circ \varphi\) est alors un homomorphisme\footnote{Voir \ref{NORMooTXFWooApjnOY}.} surjectif de \( S_n\) sur \( \{ -1,1 \}\) et nous avons \( \theta\circ\varphi=\epsilon\) par la proposition~\ref{ProphIuJrC}. L'enchainement \eqref{EqSZBPTH} nous montre que \( H=\ker(\theta\circ\varphi)=\ker(\epsilon)=A_n\).
\end{proof}

\begin{proposition}[\cite{ooFCRKooQdAaqw}]      \label{PROPooPSZVooSmAgPA}
	Le groupe symétrique \( S_n\) peut être écrit comme un produit semi-direct\footnote{Définition~\ref{DEFooKWEHooISNQzi}.} du groupe alterné :
	\begin{equation}
		S_n=A_n\times_{\varphi}\eZ/2\eZ
	\end{equation}
	où l'action de \( \eZ/2\eZ\) sur \( A_n\) est la conjugaison par \( \sigma=(12)\), c'est-à-dire \( \rho(-1)\tau=\sigma\tau\sigma^{-1}\).
\end{proposition}

\begin{proof}
	Nous avons la suite exacte
	\begin{equation}
		1\stackrel{i}{\longrightarrow}A_n\stackrel{i}{\longrightarrow}S_n\stackrel{\epsilon}{\longrightarrow}\{ \pm 1 \}\longrightarrow 1
	\end{equation}
	où les \( i\) représentent des inclusions et \( \epsilon\) est la signature définie en~\ref{DEFooYDUHooKIXGNW}. Grâce à cette suite et au fait que la signature soit un isomorphisme à partir de la partie \( \{ \id,\sigma \}\) (pour \( \sigma\) d'ordre \( 2\), par exemple \( \sigma=(12)\)), le théorème~\ref{THOooZNYTooPhnIdE} nous dit que
	\begin{equation}
		S_n\simeq A_n\times_{\varphi}\{ \id,\sigma \}
	\end{equation}
	où \( \varphi\) est l'action adjointe de \( \{ \id,\sigma \}\) sur \( A_n\).
\end{proof}

\begin{proposition}     \label{PROPooZOWBooIMxxlj}
	Si \( \beta\in S_n\) est une transposition, nous avons les égalités suivantes d'ensembles :
	\begin{equation}
		S_n=A_n\cup A_n\beta=A_n\cup \beta A_n.
	\end{equation}
\end{proposition}

\begin{proof}
	Les parties \( A_n\) et \( \beta A_n\) ont le même nombre d'éléments. En effet, l'application
	\begin{equation}
		\begin{aligned}
			\varphi\colon A_n & \to A_n\beta        \\
			\sigma            & \mapsto \sigma\beta
		\end{aligned}
	\end{equation}
	est une bijection.

	De plus ces deux ensembles sont disjoints à cause de la proposition \ref{ProphIuJrC}. En effet si \( \sigma\in A_n\), alors \( \epsilon(\sigma)=1\). Mais un élément de \( A_n\beta\) est de la forme \( \sigma\beta\) avec \( \sigma\in A_n\). Or \( \epsilon\) est un homomorphisme\footnote{Voir \ref{NORMooTXFWooApjnOY}.}, donc \( \epsilon(\sigma\beta)=\epsilon(\sigma)\epsilon(\beta)=-1\).

	Enfin, la proposition \ref{PROPooCPXOooVxPAij}\ref{ITEMooWXXUooOWvFgE} dit que \( A_n\) est d'indice deux dans \( S_n\). Donc la partie
	\begin{equation}
		A_n\cup A_n\beta
	\end{equation}
	contient \( | S_n |/2+| S_n |/2=| S_n |\) éléments. C'est donc \( S_n\).
\end{proof}

\begin{lemma}   \label{LemiApyfp}   \index{groupe!dérivé!du groupe symétrique}
	Le groupe dérivé du groupe symétrique est le groupe alterné : \( D(S_n)=A_n\).
\end{lemma}

\begin{proof}
	Tout élément de \( D(S_n)\) s'écrit sous la forme \( ghg^{-1}h^{-1}\). Quel que soit le nombre de transpositions dans \( g\) et \( h\), le nombre de transpositions dans \( [g,h]\) est pair.
\end{proof}

\begin{proposition}[\cite{LoFdlw}]     \label{PropsHlmvv}
	Soit \( n\geq 3\). Les \( 3\)-cycles \( c_i=(1,2,i)\) avec \( i=3,\ldots, n\) engendrent le groupe alterné \( A_n\).
\end{proposition}

\begin{proof}
	Soit \( H\), le groupe engendré par les \( c_i\). D'abord nous avons
	\begin{equation}
		c_i=(1,2,i)=(1,2)(2,i),
	\end{equation}
	de telle sorte que \( \epsilon(c_i)=1\). Par conséquent nous avons \( H\subset A_n\). Nous montrons par récurrence que \( A_n\subset H\).

	Pour \( n=3\) il suffit de vérifier que \( A_3=\{ \id,c_3,c_3^2 \}\). Supposons avoir obtenu le résultat pour \(A_{n-1}\), et prouvons le pour \( A_n\). Soit \( s\in A_n\).

	Si \( s(n)=n\), alors \( s\) se décompose de la même manière que sa restriction \( s'\) à \( \{ 1,\ldots, n-1 \}\). Par l'hypothèse de récurrence, cette restriction, appartenant à \( A_{n-1}\),  se décompose en produit des \( c_3,\ldots, c_{n-1}\) et de leurs inverses.

	Si \( s(n)=k\) alors nous considérons l'élément \( c^2_nc_ks\). Cet élément envoie \( n\) sur \( n\) et peut donc être décomposé avec les \( c_i\) (\( i=1,\ldots, n-1\)) en vertu du point précédent.
\end{proof}

\begin{proposition} \label{PropiodtBG}
	Lorsque \( n\geq 5\), tous les \( 3\)-cycles de \( A_n\) sont conjugués. Autrement dit, la classe de conjugaison d'un \( 3\)-cycle est l'ensemble des \( 3\)-cycles.
\end{proposition}

\begin{proof}
	Soient les \( 3\)-cycles \( \sigma=(i_1,i_2,i_3)\) et \( \varphi=(j_1,j_2,j_3)\). Nous considérons une bijection \( \alpha\) de \( \{ 1,\ldots, n \}\) telle que \( \alpha(i_s)=j_s\). Nous avons immédiatement que \( \alpha\in S_n\) et que \( \alpha\sigma\alpha^{-1}=\varphi\). Donc les \( 3\)-cycles sont conjugués dans \( S_n\). Il reste à prouver qu'ils le sont dans \( A_n\).

	Si \( \alpha\) est une permutation paire, la preuve est terminée. Si \( \alpha\) est impaire, alors nous devons un peu la modifier. Comme \( n\geq 5\), nous pouvons prendre \( s\) et \( t\), des éléments distincts dans \( \{ 1,\ldots, n \}\setminus\{ j_1,j_2,j_3 \}\) et poser \( \tau=(st)\). Puisque la signature est un homomorphisme\footnote{Voir \ref{NORMooTXFWooApjnOY}.} et que \( \tau\) et \( \alpha\) sont impairs, l'élément \( \tau\alpha\) est pair (lemme et proposition~\ref{LemhxnkMf} et~\ref{PropPWIJbu}) et est donc dans \( A_n\). Les supports de \( \tau\) et \( \varphi\) étant disjoints, ces derniers commutent et nous avons
	\begin{equation}
		(\tau\alpha)\sigma(\tau\alpha)^{-1}=\tau(\alpha\sigma\alpha^{-1})\tau^{-1}=\tau\varphi\tau^{-1} = \varphi.
	\end{equation}
	Donc \( \sigma\) et \( \varphi\) sont conjugués par \( \tau\alpha\) qui est dans \( A_n\).
\end{proof}

\begin{theorem}[\cite{PDFpersoWanadoo}] \label{ThoURfSUXP}
	Le groupe alterné \( A_n\) est simple\footnote{Pas de sous-groupes normaux non triviaux, définition~\ref{DefGroupeSimple}.} pour \( n\geq 5\).
\end{theorem}
\index{sous-groupe!distingué!dans le groupe alterné}
\index{groupe!fini!alterné}
\index{groupe!partie génératrice}


\begin{proof}
	Soit \( N\), un sous-groupe normal de \( A_n\) non réduit à l'identité. Étant donné que les \( 3\)-cycles engendrent \( A_n\) (proposition~\ref{PropsHlmvv}) et que tous les \( 3\)-cycles sont conjugués dans \( A_n\) (proposition~\ref{PropiodtBG}), il suffit de montrer que \( N\) contient un \( 3\)-cycle. En effet si \( N\) contient un \( 3\)-cycle, le fait qu'il soit normal implique (par conjugaison) qu'il les contienne tous et donc qu'il contient une partie génératrice de \( A_n\).

	Soit donc \( \sigma\in N\) différent de l'identité. Nous prenons \( i\) dans le support de \( \sigma\) et \( j=\sigma(i)\). Nous choisissons ensuite \( k\in\{ 1,\ldots, n \}\setminus\{ i,j,\sigma^{-1}(i) \}\) et \( m=\sigma(k)\). Nous considérons la permutation \( \alpha=(ijk)\). Étant donné que \( N\) est normal, l'élément
	\begin{equation}
		\theta=(\alpha^{-1}\sigma\alpha)\sigma^{-1}
	\end{equation}
	est dans \( N\). De plus en utilisant le lemme~\ref{LemmvZFWP} et le fait que \( \alpha^{-1}=(ikj)\) nous avons
	\begin{equation}
		\theta=(ikj)(j\sigma(j)m).
	\end{equation}
	Cela n'est pas spécialement un \( 3\)-cycle, mais nous allons en construire un. Nous allons déterminer que \( \theta\) est soit un \( 5\)-cycle, soit un \( 3\)-cycle , soit un \( 2\times 2\)-cycle suivant les valeurs de \( \sigma(j)\) et \( m\).

	Souvenons-nous que nous avons :
	\begin{itemize}
		\item
		      \( i \neq j = \sigma(i) \), puisque \( i\) est dans le support de \( \sigma \);
		\item
		      \( k \neq i \) et \( k \neq j \), par définition de \( k\) (rappelons aussi que \( k \neq \sigma^{-1}(i) \));
		\item
		      \( m \neq i \), \( m \neq j \) et \( m \neq  \sigma(j) \) puisque \( m = \sigma(k) \).
	\end{itemize}
	Il ne nous reste alors seulement les deux possibilités suivantes :
	\begin{enumerate}
		\item
		      soit \( m=k\), soit \( m \neq k \), d'une part;
		\item
		      soit \( \sigma(j) = i \), soit \( \sigma(j) = k \), soit \( \sigma(j) \) n'est ni \( i\), ni \( k\), ni \( m\), d'autre part.
	\end{enumerate}

	Supposons dans un premier temps que \( m=k\); alors
	\begin{equation}
		\theta=(ik)(j\sigma(j)).
	\end{equation}
	C'est à priori un \( 2\times 2\)-cycle. Mais si de plus \( \sigma(j) = i \), alors
	\begin{equation}
		\theta=(ijk)
	\end{equation}
	qui est un \( 3\)-cycle; et si \( \sigma(j) = k \), alors
	\begin{equation}
		\theta=(ikj)
	\end{equation}
	qui est un autre \( 3\)-cycle.

	Supposons à présent que \( m \neq k \). Si \( \sigma(j) \) n'est ni \( i\), ni \( k\), ni \( m\), alors \( i\), \( j\), \( k\), \( \sigma(j)\) et \( m\) sont cinq nombres différents, et
	\begin{equation}
		\theta=(i,j,\sigma(j),m,k)
	\end{equation}
	est un \( 5\)-cycle. Si \( \sigma(j) = i\), alors
	\begin{equation}
		\theta=(ikj)(jim) = (imk)
	\end{equation}
	qui est un \( 3\)-cycle. Si \( \sigma(j)=k\), alors
	\begin{equation}
		\theta=(ikj)(jkm) = (ikm)
	\end{equation}
	qui est encore un \( 3\)-cycle.

	Bref nous avons montré que \( \theta\) est soit un \( 3\)-cycle, soit un \( 5\)-cycle, soit un \( 2\times 2\)-cycle. Si \( \theta\) est un \( 3\)-cycle, la preuve est terminée.

	Si \( \theta=(ab)(cd)\), alors on considère \( e\in \{ 1,\ldots, n \}\setminus\{ a,b,c,d \}\) et nous avons
	\begin{equation}
		\underbrace{(abe)^{-1}\theta(abe)}_{\in N}\theta^{-1}=(aeb)(ab)(cd)(abe)(an)(cd)=(abe)\in N.
	\end{equation}

	Si \( \theta\) est le \( 5\)-cycle \( (abcde)\), alors l'élément suivant est dans \( N\) :
	\begin{equation}
		(abc)^{-1}\theta(abc)\theta^{-1}=(acb)(abcde)(abc)(aedcb)=(acd).
	\end{equation}

	Dans tous les cas nous avons trouvé un \( 3\)-cycle dans \( N\) et nous avons par conséquent \( N=A_n\), ce qui fait que \( A_n\) ne contient pas de sous-groupes normaux non triviaux. Le groupe alterné \( A_n\) est donc simple.
\end{proof}

Nous en déduisons immédiatement que si \( n\geq 5\), le groupe dérivé de \( A_n\) est \( A_n\) parce que \( A_n\) ne contient pas d'autres sous-groupes non triviaux.\index{groupe!dérivé!du groupe alterné}

\begin{lemma}       \label{LEMooICEHooGSSpkq}
	Le groupe alterné\footnote{Définition~\ref{DEFooEIVIooFvVkHH}.} \( A_6\) n'accepte pas de sous-groupes normaux d'ordre \( 60\).
\end{lemma}

\begin{proof}
	Soit \( G\) normal dans \( A_6\), et \( a\), un élément d'ordre \( 5\) dans \( G\) (qui existe parce que \( 5\) divise \( 60\)). Soit aussi un élément \( b\) d'ordre \( 5\) dans \( A_6\). Les groupes \( \gr(a)\) et \( \gr(b)\) sont deux \( 5\)-Sylow dans \( A_6\). En effet, \( 5\) est un nombre premier, et est la plus grande puissance de \( 5\) dans la décomposition de \( 60\); donc \( \gr(a)\) est un \( 5\)-Sylow dans \( G\). D'autre part, l'ordre de \( A_6\) (qui est \( \frac{ 1 }{2}\cdot 6!\)) ne possède également que \( 5\) à la puissance \( 1\) dans sa décomposition.

	En vertu du théorème de Sylow~\ref{ThoUkPDXf}\ref{ItemMzNRVf}, les \( 5\)-Sylow \( \gr(a)\) et \( \gr(b)\) sont conjugués et il existe \( \tau\in A_6\) tel que \( b=\tau a\tau^{-1}\). Mais \( G\) étant normal dans \( A_6\), l'élément \( \tau a\tau^{-1}\) est encore dans \( G\), de telle sorte que \( b\in G\). Du coup \( G\) doit contenir tous les éléments d'ordre \( 5\) de \( A_6\).

	Les éléments d'ordre \( 5\) de \( A_6\) doivent fixer un des points de \( \{ 1,2,3,4,5,6 \}\) puis permuter les autres de façon à n'avoir qu'un seul cycle. Un cycle correspond à écrire les nombres \( 1,2,3,4,5\) dans un certain ordre. Ce faisant, le premier n'a pas d'importance parce qu'on considère la permutation cyclique, par exemple \( (3,5,2,1,4)\) est la même chose que \( (5,2,1,4,3)\). Le nombre de cycles sur \( \{ 1,2,3,4,5 \}\) est donc de \( 4!\), et par conséquent le nombre d'éléments d'ordre \( 5\) dans \( A_6\) est \( 6\cdot 4!=144\).

	Le groupe \( G\) doit contenir au moins \( 144\) éléments alors que par hypothèse il en contient \( 60\); contradiction.
\end{proof}

Le théorème suivant montre que tout groupe peut être vu, en agissant sur lui-même, comme une partie du groupe symétrique.
\begin{theorem}
	Un groupe \( G\) est isomorphe à un sous-groupe de son groupe symétrique \( S(G)\).
\end{theorem}

\begin{proof}
	Nous considérons \( \varphi\), la translation à gauche :
	\begin{equation}
		\begin{aligned}
			\varphi\colon G & \to S(G)    \\
			g               & \mapsto t_g
		\end{aligned}
	\end{equation}
	où \( f_g(h)=gh\). Étant donné que
	\begin{equation}
		\varphi(gh)= ghx=g(t_hx)=t_g\circ t_h(x),
	\end{equation}
	l'application \( \varphi\) est un morphisme de groupes. Il est injectif parce que si \( gx=hx\) pour tout \( x\), en particulier pour \( x=e\) nous trouvons \( g=h\).

	De la même manière, \( \varphi(g)x=\varphi(g)y\) implique \( x=y\). Cela montre que l'image est bien dans le groupe symétrique.

	L'ensemble \( \Image(\varphi)\) est donc un sous-groupe de \( S(G)\), et \( \varphi\) est un isomorphisme vers ce groupe.
\end{proof}

\begin{lemma}       \label{LEMooMVUGooRiDaDz}
	Si \( n\geq 3\), alors
	\begin{enumerate}
		\item
		      Le centre de \( S_n\) est trivial.
		\item
		      Le groupe \( S_n\) est non abélien.
	\end{enumerate}
\end{lemma}

\begin{proof}
	Soit \( s\in Z(S_n)\) et trois éléments distincts \( a\),  \( b\) et \( c\) de \( \{ 1,\ldots, n \}\). Nous posons \( \tau=(ab)\) et nous avons \( s\tau=\tau s\). En notant \( a'=s(a)\) et \( b'=s(b)\) nous avons
	\begin{subequations}
		\begin{align}
			a'=s(a)=(\tau s\tau^{-1})(a)=(\tau s)(b)=\tau(b') \\
			b'=s(b)=(\tau s\tau^{-1})(b)=(\tau s)(a)=\tau(a').
		\end{align}
	\end{subequations}
	Donc \( \tau\) permute \( a'\) et \( b'\). Mais comme \( \tau\) ne permute que \( a\) et \( b\), en tant qu'ensembles, \( \{ a,b \}=\{ s(a), s(b) \}\). Le même raisonnement sur \( \{ b,c \}\) donne \( \{ b,c \}=\{ s(b),s(c) \}\). Et puisque \( a\), \( b\) et \( c\) sont distincts,
	\begin{equation}
		\{ b \}=\{ b,c \}\cap\{ a,b \}=\{ s(b) \}.
	\end{equation}
	Cela montre que \( s(b)=b\), et donc que le centre de \( S_n\) est réduit à la permutation identité.

	En ce qui concerne le fait que \( S_n\) est non abélien, si nous avions \( st=ts\) pour tout \( s,t\in S_n\) alors \( s=tst^{-1}\) pour tout \( t\). Alors \( s\) serait dans le centre de \( S_n\). En bref, si \( S_n\) était abélien, son centre serait \( S_n\) et non \( \{ \id \}\).

\end{proof}

\begin{proposition}[\cite{Exo7Sylow,ooGQNTooEiWtsy}]        \label{PROPooUBIWooTrfCat}
	Tout groupe simple\footnote{Définition~\ref{DefGroupeSimple}.} d'ordre \( 60\) est isomorphe au groupe alterné \( A_5\).
\end{proposition}

\begin{proof}
	Nous avons la décomposition en nombres premiers \( 60=2^2\cdot 3\cdot 5\). Déterminons pour commencer le nombre \( n_5\) de \( 5\)-Sylow dans \( G\). Le théorème de Sylow~\ref{ThoUkPDXf}\ref{ItemkYbdzZ} nous renseigne que \( n_5\) doit diviser \( 60\) et doit être égal à \( 1\mod 5\). Les deux seules possibilités sont \( n_5=1\) et \( n_5=6\). Étant donné que tous les \( p\)-Sylow sont conjugués, si \( n_5=1\) alors le \( 5\)-Sylow serait un sous-groupe invariant à l'intérieur de \( G\), ce qui est impossible vu que \( G\) est simple. Donc \( n_5=6\).

	Par le point~\ref{ItemMzNRVf} du théorème de Sylow, le groupe \( G\) agit transitivement sur l'ensemble des \( 5\)-Sylow par l'action adjointe :
	\begin{equation}
		g\cdot S=gSg^{-1}.
	\end{equation}
	Cela donne donc un morphisme \( \theta\colon G\to S_6\). Le noyau de \( \theta\) est un sous-groupe normal. En effet si \( k\in \ker\theta\) et si \( g\in G\) nous avons
	\begin{subequations}
		\begin{align}
			(gkg^{-1})\cdot S & =gkg^{-1} Ggk^{-1}g^{-1} \\
			                  & =gkTk^{-1}g^{-1}         \\
			                  & =gTg^{-1}                \\
			                  & =S
		\end{align}
	\end{subequations}
	où \( T\) est le Sylow \( T=g^{-1}Sg\). Étant donné que \( k\in \ker\theta\) nous avons utilisé \( kTk^{-1}=aT\). Au final \( gkg^{-1}\cdot S=S\), ce qui prouve que \( gkg^{-1} \in\ker\theta\).

	Étant donné que \( \ker\theta\) est normal dans \( G\), soit il est réduit à \( \{ e \}\) soit il vaut \( G\). La seconde possibilité est exclue parce qu'elle reviendrait à dire que \( G\) agit trivialement, ce qui n'est pas correct étant donné qu'il agit transitivement. Nous en déduisons que \( \ker\theta=\{ e \}\), que \( \theta\) est injective et que \( G\) est isomorphe à un sous-groupe de \( S_6\).

	Par ailleurs le groupe dérivé de \( G\) est un sous-groupe normal (et non réduit à l'identité parce que \( G\) est non commutatif). Donc \( D(G)=G\). Étant donné que \( G\subset S_6\), nous avons
	\begin{equation}
		G=D(G)\subset D(S_6)=A_6
	\end{equation}
	parce que le groupe dérivé du groupe symétrique est le groupe alterné (lemme~\ref{LemiApyfp}).

	L'ensemble \( \theta^{-1}(A_6)\) est distingué dans \( G\). En effet si \( \sigma\in A_6\) et si \( g\in G\) nous avons
	\begin{equation}
		\theta\big( g\theta^{-1}(\sigma)g^{-1} \big)=\theta(g)\sigma \theta(g)^{-1}\in A_6.
	\end{equation}
	Nous en déduisons que \( \theta^{-1}(A_6)\) est soit \( G\) entier soit réduit à \( \{ e \}\). Si \( \theta^{-1}(A_6)=\{ e \}\), alors pour tout \( g\in G\) nous aurions \( g^2=e\) parce que \( \theta(g^2)\in A_6\). L'ordre de \( G\) étant \( 60\), il n'est pas possible que tous ses éléments soient d'ordre \( 2\). Nous en déduisons que \( \theta(G)\subset A_6\).

	Nous nommons \( H=\theta(G)\) et nous considérons l'ensemble \( X=A_6/H\) où les classes sont prises à gauche, c'est-à-dire
	\begin{equation}
		[\sigma]=\{ h\sigma\tq h\in H \}.
	\end{equation}
	Évidemment \( A_6\) agit sur \( X\) de façon naturelle. Au niveau de la cardinalité,
	\begin{equation}
		\Card(X)=\frac{ | A_6 | }{ | G | }=\frac{ 360 }{ 60 }=6.
	\end{equation}
	Le groupe \( A_6\) agit sur \( X\) qui a \( 6\) éléments. Nous avons donc une application \( \varphi\colon A_6\to A_6\). Encore une fois, la simplicité de \( A_6\) montre que \( \varphi(A_6)=A_6\).

	Nous étudions maintenant \( \varphi(H)\) agissant sur \( X\). Un élément \( x\in A_6\) fixe la classe de l'unité \( [e]\) si et seulement si \( x\in H\) et par conséquent \( \varphi(H)\) est le fixateur de \( [e]\) dans \( X\). À la renumérotation près, nous pouvons identifier \( \varphi(H)\) au sous-groupe de \( A_6\) agissant sur \( \{ 1,\ldots, 6 \}\) et fixant \( 6\). Nous avons alors \( \varphi(H)=S_5\cap A_6=A_5\). Nous venons de prouver que \( \varphi\) fournit un isomorphisme entre \( A_5\) et \( H\). Étant donné que \( H\) était isomorphe à \( G\), nous concluons que \( G\) est isomorphe à \( A_6\).
\end{proof}

%---------------------------------------------------------------------------------------------------------------------------
\subsection{Sous-groupes normaux}
%---------------------------------------------------------------------------------------------------------------------------

\begin{normaltext}[\cite{ooDZHIooQIYqwZ}]     \label{NORMooQAZTooBQLqDn}
	Soit le groupe \( V_4\) engendré par les doubles transpositions de \( S_4\). Nous savons de l'exemple~\ref{ExVYZPzub}\ref{ITEMooGCMYooKZgFHX} que ce groupe contient exactement \( 3\) éléments non triviaux et l'identité. De plus, comme c'est une classe de conjugaison, \( V_4\) est normal dans \( S_4\).
\end{normaltext}

\begin{lemma}
	Les sous-groupes \( \Fix_{S_n}(a)\) (avec \( a\in\{ 1,\ldots, n \}\)) sont conjugués entre eux.
\end{lemma}

\begin{proof}
	Soit \( \sigma\in \Fix(a)\) et \( s\in S_n\) nous devons prouver que \( s \sigma s^{-1}\) est le fixateur d'un élément de \( \{ 1,\ldots, n \}\). Nous notons \( s(a)=b\). Alors
	\begin{equation}
		(s\sigma s^{-1})(b)=(s\sigma)(a)=s(a)=b.
	\end{equation}
	Donc \( s\Fix(a)s^{-1}\subset \Fix(b)\).

	Dans l'autre sens, si \( \sigma\in \Fix(b)\) alors \( s^{-1} \sigma s\in\Fix(a)\). Mais \( \sigma=s(s^{-1}\sigma s)s^{-1}\), donc \( \sigma\in s\Fix(a)s^{-1}\).
\end{proof}

\begin{proposition}[Sous-groupes normaux de \( S_n\) \cite{ooDZHIooQIYqwZ}]     \label{PROPooOTJAooUbzGZm}
	Les sous-groupes normaux de \( S_n\) ne sont pas légions.
	\begin{enumerate}
		\item
		      Pour \( n=4\), les sous-groupes normaux de \( S_4\) sont \(  \{ \id \}  \), \( V_4\), \( A_4\) et \( S_4\).
		\item
		      Pour \( n\neq 4\), les sous-groupes normaux de \( S_n\) sont \( \{ \id \}\), \( A_n\) et \( S_n\).
	\end{enumerate}
\end{proposition}

\begin{proof}
	Les cas \( n\leq 2\) sont un peu triviaux, donc nous faisons \( n\geq 3\). Soit \( H\) normal dans \( S_n\) et \( s\neq \id\) dans \( H\); par le lemme~\ref{LEMooMVUGooRiDaDz}, \( s\) n'est pas dans le centre de \( S_n\) et il existe \( u\in S_n\) tel que \( us\neq su\). Comme \( u\) est un produit de transpositions (proposition~\ref{PropPWIJbu}), il existe une transposition \( t\) telle que \( st\neq ts\). Le sous-groupe \( H\) est normal et puisque \( s\in H\) nous avons aussi \( ts^{-1}t^{-1}\in H\). Mais en même temps, la combinaison \( sts^{-1}\) est le conjugué d'une transposition et est donc également une transposition (classe de conjugaison de \( S_4\) dans~\ref{ExVYZPzub}). Nous en concluons que \( sts^{-1}t^{-1}\) est un produit de deux transpositions appartenant à \( H\).

	Nous venons de prouver que \( H\) contient au moins un produit de deux transpositions. Et ce produit est différent de \( \id\) parce que \( sts^{-1}t^{-1}=\id\) impliquerait \( st=ts\).

	Soient donc deux transpositions \( t_1,t_2\in H\) telles que \( t_1t_2\neq \id\). Les supports de \( t_1\) et \( t_2\) ont soit \( 1\) soit aucun élément communs.

	\begin{subproof}
		\spitem[Premier cas]

		Supposons \( t_1=(a,b)\), \( t_2=(b,c)\) avec \( a,b,c\) distincts dans \( \{ 1,\ldots, n \}\). Dans ce cas \( t_1t_2=(a,b,c)\) et \( H\) contient un cycle de longueur \( 3\). Puisque \( H\) est normal et que les cycles de longueur trois sont une classe de conjugaison (exemple~\ref{ExVYZPzub}) et que \( A_n\) est engendré par ceux-ci (proposition~\ref{PropsHlmvv}), \( A_n\subset H\). Mais \( A_n\) est d'indice deux dans \( S_n\) (proposition~\ref{ITEMooWXXUooOWvFgE}\ref{ITEMooWXXUooOWvFgE}). Quel nombre plus grand que \( n!/2\) divise \( n!\) ? Seulement \( n\) lui-même. Donc \( H\) est soit \( A_n\) soit \( S_n\).

		\spitem[Second cas]

		Le groupe \( H\) contient un élément de la forme \( (ab)(cd)\) avec \( a,b,c,d\) distincts dans \( \{ 1,\ldots, n \}\).

		\begin{subproof}

			\spitem[Si \( n=3\)]

			Impossible parce que avec \( n=3\) nous n'avons pas quatre éléments distincts.

			\spitem[Si \( n=4\)]

			Le sous-groupe \( H\) de \( S_4\) contient un élément de \( V_4\) qui n'est pas l'identité. Par normalité et classes de conjugaison, \( H\) contient \( V_4\). Nous devons maintenant prouver que si \( H\) n'est pas \( V_4\) alors \( H\) est \( A_4\) ou \( S_4\). Nous avons les inclusions \( V_4\subset H\subset S_4\) et donc les inégalités
			\begin{equation}
				4\leq | H |\leq 24.
			\end{equation}
			Donc le nombre \( | H |\) est un multiple de \( 4\) qui divise \( 24\). Les possibilités sont \( | H |=4,8,12,24\). La possibilité \( | H |=4\) donne \( H=V_4\); si \( |H |=24\) alors \( H=S_4\); si \( | H |=12\) alors \( H\) est d'indice \( 2\) dans \( S_4\) et \( H=A_n\) (proposition~\ref{PROPooCPXOooVxPAij}\ref{ITEMooGGAHooRYgNqq}). Quid de \( | H |=8\) ?

			D'après le corolaire~\ref{CorpZItFX} au théorème de Lagrange, l'ordre d'un élément divise l'ordre du groupe. Soit \( x\) dans \( H\) mais pas dans \( V_4\). L'ordre de \( x\) peut être \( 1\), \( 2\), \( 4\) ou \( 8\). Ordre \( 1\) serait \( x=\id\). Ordre \( 8\), pas possible parce que \( S_4\) n'a pas d'éléments d'ordre \( 8\).
			\begin{subproof}
				\spitem[\( x\) d'ordre \( 2\)]

				Prenons la décomposition de \( x\) en cycles disjoints. Puisqu'on est dans \( S_4\), ces cycles ne peuvent être que des transpositions. Soit il y en a un (alors \( H\) contient une transposition et donc \( H=S_4\)), soit il y en a deux et alors \( x\) est dans \( V_4\).

				\spitem[\( x\) d'ordre \( 4\)]

				L'élément \( x\) est alors un cycle de longueur \( 4\), et \( H\) contient tous les cycles de longueur \( 4\); par exemple, le produit \( (abcd)(bacd)=(adc)\). Le sous-groupe \( H\) contient alors \( A_4\) (parce qu'il contient tous les \( 3\)-cycles).
			\end{subproof}

			\spitem[Si \( n\geq 5\)]

			Soit un élément \( e\)\footnote{\( e\) n'est pas l'élément neutre ici} distinct de \( a,b,c\) et \( d\). Par notre liste préférée des classes de conjugaison (exemple~\ref{ExVYZPzub}\ref{ITEMooGCMYooKZgFHX}), le \( 2\)-cycle \( (c,e)(a,b)\) est conjugué à \( (a,b)(c,d)\) et appartient donc à \( H\). Mais alors le produit suivant est également dans \( H\) :
			\begin{equation}
				(ce)(ab)(ab)(cd)=(ce)(cd)=(ecd).
			\end{equation}
			Donc \( H\) contient un \( 3\)-cycle, et par conséquent tous les \( 3\)-cycles. Encore une fois, cela prouve que \( H\) est soit \( A_n\) soit \( S_n\).

			\spitem[Pourquoi \( n=4\) est spécial ?]

			Dans le premier cas, nous montrons tout de suite que \( H=V_4\) n'est pas possible. Dans le deuxième cas, nous montrons que, grâce à un élément différent de \( a,b,c\) et \( d\), la possibilité \( H=V_4\) est exclue. La possibilité \( H=V_4\) n'existe que pour \( n=4\).

		\end{subproof}
	\end{subproof}

\end{proof}

%---------------------------------------------------------------------------------------------------------------------------
\subsection{Indice}
%---------------------------------------------------------------------------------------------------------------------------

\begin{theorem}     \label{THOooXDRNooIyaGlv}
	Tout sous-groupe d'indice \( n\) dans \( S_n\) est isomorphe à \( S_{n-1}\).
\end{theorem}

\begin{proof}
	Pour \( n=1\), il n'y a pas de sous-groupe. Pour \( n=2\), un sous-groupe d'indice \( 2\) ne peut contenir que \( 1\) élément, qui est donc l'identité. Ok pour que \( \{ \id \}\) soit égal à \( S_1\) ?

	Pour les autres, il y a un peu plus de travail.

	\begin{subproof}
		\spitem[Pour \( n=3\)]

		Nous avons \( | S_3 |=6\). Donc un sous-groupe d'indice \( 3\) dans \( S_3\) contient exactement \( 2\) éléments. Il contient \( \id\) et un autre élément \( \sigma\in S_3\) qui doit vérifier \( \sigma^2=\id\) ou \( \sigma^2=\sigma\). Aucun élément de \( S_3\) ne vérifie \( \sigma^2=\sigma\) (à part l'identité). Donc \( \sigma^2=\id\), ce qui implique que \( \sigma\) est une transposition. Donc
		\begin{equation}
			H=\{ \id,(12) \}
		\end{equation}
		ou l'identité avec \( (23)\), ou avec \( (13)\). Dans tous les cas c'est isomorphe à \( S_2\).

		\spitem[Pour \( n=4\)]

		Nous avons \( | S_4:H |=4\), donc \( | H |=6\). Mais \( 6=2\times 3\) et \( 2\divides 3-1\), donc le théorème~\ref{ThoLnTMBy} nous dit que \( H\) est soit cyclique\footnote{Définition~\ref{DefHFJWooFxkzCF}.} (et donc abélien), soit un produit semi-direct. Vu que \( S_4\) n'a pas d'éléments d'ordre \( 6\), aucun sous-groupe d'ordre \( 6\) ne peut être cyclique. Nous sommes donc dans le cas du produit semi-direct
		\begin{equation}        \label{EQooSHWGooFMNRvf}
			H=\eZ_3\times_{\varphi}\eZ_2
		\end{equation}
		où \( \varphi\colon \eZ_2\to \Aut(\eZ_3)\) et \( \varphi(1)\) est d'ordre \( 2\) dans \( \Aut(\eZ_3)\). Il convient de nous attarder un peu pour être sûr d'avoir bien compris tout ce qui se trouve dans l'identification \eqref{EQooSHWGooFMNRvf}. D'abord un point de notations : ici nous considérons les groupes \( \eZ_p=\eZ/p\eZ\) munis de l'addition. Donc \( 1\) n'est pas le neutre. Ensuite nous savons du théorème~\ref{ThoozyeSn} que \( \Aut(\eZ/3\eZ)=(\eZ/3\eZ)^*\), et que via cette identification, \( \varphi(1)=2\in(\eZ/3\eZ)^*\) au sens où \( \varphi(1)x=2x\). Nous avons alors \( \varphi(1)^2x=4x=x\) dans \( \eZ/3\eZ\). Cela montre bien que \( \varphi(1)\) est d'ordre \( 2\).

		Par rapport à la proposition~\ref{PROPooPSZVooSmAgPA}, ici nous écrivons \( \eZ_2=\big( \{ 0,1 \},+ \big)\) alors que là nous écrivons \( \eZ_2=\big( \{ -1,1 \},\cdot \big)\). Ce sont les mêmes groupes, mais il convient de remarquer que le \( 1\) ici est le \( -1\) là.

		Nous savons par la proposition~\ref{PROPooPSZVooSmAgPA} que \( S_n=A_n\times_{\varphi}\eZ_2\); en comparant avec \eqref{EQooSHWGooFMNRvf} nous voyons qu'il suffit de prouver que \( A_3=\eZ/3\eZ\) pour avoir \( H=S_3\).

		Le groupe \( A_3\) possède \( | S_3 |/2=3\) éléments. Il est vite vu que \( A_3=\{ \id,(12)(31), (12)(32) \}\) : ce sont trois éléments de signature paire dans \( S_3\); donc c'est \( S_3\). La correspondance \( \id\mapsto 0\), \( (12)(13)\mapsto 1\), \( (13)(12)\mapsto 2\) donne un isomorphisme avec \( (\eZ_3,+)\).

		\spitem[Pour \( n\geq 5\)]

		Soit un sous-groupe \( H\) d'indice \( n\) dans \( S_n\) et l'action à gauche de \( S_n\) sur \( E=S_n/H\) (qui n'est à priori pas un groupe) donnée par \( g\cdot [s]=[gs]\).

		\begin{subproof}
			\spitem[Morphisme \( \varphi\colon S_n\to S_E\)]

			Le \( \varphi\) défini par l'action est un morphisme parce que
			\begin{equation}
				\varphi(g_1g_2)[s]=[g_1g_2s]=\varphi(g_1)[g_2s]=\varphi(g_1)\varphi(g_2)[s].
			\end{equation}
			Mais il faut également vérifier que pour chaque \( g\in G\), l'application \( \varphi(g)\colon E\to E\) est bien une permutation. Pour l'injectivité, si \( \varphi(g)[s_1]=\varphi(g)[s_2]\) alors \( [gs_1]=[gs_2]\), donc il existe \( h\in H\) tel que \( gs_1=gs_2h\), ce qui prouve que \( s_1=s_2h\) et donc que \( [s_1]=[s_2]\). Pour la surjectivité, soit \( [t]\in S_n/H\) et résolvons \( \varphi(g)[s]=[t]\) par rapport à \( s\). L'élément \( s=g^{-1} t\) convient.

			\spitem[\( \ker(\varphi)\) est normal]

			Soit \( z\in\ker(\varphi)\), c'est-à-dire que \( \varphi(z)=\id_E\). Alors pour \( \sigma\in S_n\) nous avons \( \varphi(\sigma z\sigma^{-1})=\varphi(\sigma)\varphi(z)\varphi(\sigma^{-1})=\id_E\).

			\spitem[\( \ker(\varphi)=\bigcap_{g\in S_n}gHg^{-1}\)]

			Supposons que \( z\in gHg^{-1}\) pour tout \( g\), et calculons \( \varphi(z)[s]\). D'abord par hypothèse il existe \( h\in H\) tel que \( z=shs^{-1}\), donc
			\begin{equation}
				\varphi(z)[s]=[zs]=[shs^{-1}s]=[sh]=[s],
			\end{equation}
			ce qui prouve que \( \varphi(z)=\id\).

			Dans l'autre sens, soit \( z\in\ker(\varphi)\). Donc \( \varphi(z)[s]=[s]\). Il existe donc \( h\in H\) tel que \( zs=sh\), c'est-à-dire tel que \( z=shs^{-1}\). La formule demandée est donc prouvée.

			\spitem[Questions d'ordre]

			Nous savons que \( | H |=(n-1)!\) alors que \( | A_n |=\frac{ n! }{2}\). Donc \( | H |<| A_n |\) avec une inégalité stricte. En même temps nous avons \( | \ker(\varphi) |\leq | H |\) parce que \( \ker(\varphi)\) est une intersection dont un des termes est \( H\) lui-même. Nous avons alors les inégalités
			\begin{equation}
				| \ker(\varphi) |\leq | H |=(n-1)!<| A_n |.
			\end{equation}
			Mais les seuls sous-groupes normaux de \( S_n\) sont \( A_n\), \( S_n\) et \( \{ \id \}\) (proposition~\ref{PROPooOTJAooUbzGZm}). Donc \( \ker(\varphi)=\id\) et \( \varphi\) est une injection entre deux ensembles finis de même cardinalité. Cela fait de \( \varphi\) une bijection et donc un isomorphisme de groupes
			\begin{equation}
				\varphi\colon S_n\to S_E.
			\end{equation}
			Soit une fonction de numérotation \( \psi\colon E\to \{ 1,\ldots, n \}\). Avec cela nous définissons un isomorphisme de groupes
			\begin{equation}
				\begin{aligned}
					\tilde \psi\colon S_E & \to S_n                      \\
					\sigma                & \mapsto \psi\sigma\psi^{-1}.
				\end{aligned}
			\end{equation}

			\spitem[Fixateur]

			Nous montrons à présent que \( (\tilde \psi\circ\varphi)(H)=\Fix\big( \psi[\id] \big)\) où le stabilisateur est pris dans \( S_n\). Pour la première inclusion, soit \( h\in H\). Nous avons \( (\tilde \psi\circ\varphi)(h)=\psi\circ\varphi(h)\psi^{-1}\), qui nous appliquons à \( \psi[\id]\) :
			\begin{equation}
				(\tilde \psi\circ\varphi)(h)\psi[\id]=\psi\circ\varphi(h)[\id]=\psi[h]=\psi[\id].
			\end{equation}
			Donc \( (\tilde \psi\circ\varphi)(H)\subset\Fix\big( \psi[\id] \big)\).

			Pour l'autre inclusion, soit \( \sigma\in S_n\) tel que \( \sigma\psi[\id]=\psi[\id]\). Puisque \( \sigma\in S_n\) nous avons \( s\in S_E\) tel que \( \sigma=\tilde \psi(s)\). Pour ce \( s\) nous avons donc
			\begin{equation}
				\big( \tilde \psi(s)\circ\psi \big)[\id]=\psi[\id],
			\end{equation}
			d'où nous déduisons \( s[\id]=[\id]\). Cela prouve que \( s\) stabilise \( [\id]\) dans \( S_E\). Donc \( s=\varphi(h)\) pour un certain \( h\in H\), et au final \( \sigma=\tilde \psi\big( \varphi(h) \big)\).

			\spitem[Conclusion]

			L'application \( \tilde \psi\circ\varphi\colon H\to S_n\) est une application dont l'image est le fixateur d'un point. Plus précisément,
			\begin{equation}
				\tilde \psi\circ\varphi\colon H\to \Fix\big( \psi[\id] \big)
			\end{equation}
			est un isomorphisme de groupe. Mais le stabilisateur d'un point dans \( S_n\) est \( S_{n-1}\).
		\end{subproof}
	\end{subproof}
\end{proof}


%+++++++++++++++++++++++++++++++++++++++++++++++++++++++++++++++++++++++++++++++++++++++++++++++++++++++++++++++++++++++++++
\section{Isométries du cube}
%+++++++++++++++++++++++++++++++++++++++++++++++++++++++++++++++++++++++++++++++++++++++++++++++++++++++++++++++++++++++++++
\label{SecPVCmkxM}
\index{isométrie!espace euclidien!isométries du cube}
\index{groupe!et géométrie!isométries du cube}
Les isométries du cube proviennent de \cite{KXjFWKA}.

\begin{wrapfigure}{r}{6.0cm}
	\vspace{-0.5cm}        % à adapter.
	\centering
	\input{auto/pictures_tex/Fig_MCKyvdk.pstricks}
\end{wrapfigure}
Nous considérons le cube centré en l'origine de \( \eR^3\) et \( G\), le groupe des isométries de \( \eR^3\) préservant ce cube. Nous notons aussi \( G^+\) le sous-groupe de \( G\) constitué des éléments de déterminant positif. Nous notons
\begin{equation}
	\mD=\{ D_1,\ldots, D_4 \}
\end{equation}
l'ensemble des grandes diagonales, c'est-à-dire les segments \( [AG]\), \( [EC]\), \( [FD]\), et \( [BH]\). Nous savons que \( G\) préserve les longueurs et que ces segments sont les plus longs possibles à l'intérieur du cube. Donc \( G\) agit sur \( \mD\) parce qu'il ne peut transformer une grande diagonale qu'en une autre grande diagonale. Nous avons donc un morphisme de groupes
\begin{equation}
	\rho\colon G\to S_4.
\end{equation}
Nous montrons que ce morphisme est surjectif en montrant qu'il contient les transpositions. Le groupe \( G\) contient la symétrie axiale passant par le milieu \( M\) de \( [A,E]\) et le milieu \( N\) de \( [C,G]\). Il est facile de voir que cette symétrie permute \( [AG]\) avec \( [EC]\). De plus elle laisse \( [FD]\) inchangée. En effet, aussi incroyable que cela paraisse en regardant le dessin, nous avons \( FD\perp MN\), parce qu'en termes de vecteurs directeurs,
\begin{equation}
	\begin{aligned}[]
		\vect{ ON } & =\begin{pmatrix}
			               1  \\
			               -1 \\
			               0
		               \end{pmatrix} & \vect{ OF } & =\begin{pmatrix}
			                                              1 \\
			                                              1 \\
			                                              -1
		                                              \end{pmatrix}.
	\end{aligned}
\end{equation}

Étudions à présent le noyau \( \ker(\rho)\). Si \( f\in\ker(\rho)\) n'est pas l'identité, alors \( f(D_i)=D_i\) pour tout \( i\), mais au moins pour une des diagonales les sommets sont inversés. Quitte à renommer les sommets du cube nous supposons que la diagonale \( [AG]\) est retournée : \( f(A)=G\) et \( f(G)=A\). Regardons où peut partir \( B\) sous l'effet de \( f\). Étant donné que \( f\) préserve les diagonales, \( f(B)\in\{ B,C \}\), mais étant donné que \( f\) est une isométrie, \( d\big( f(B),f(G) \big)=d(B,G)\), et nous concluons que \( f(B)=H\). Donc la diagonale \( [BH]\) est retournée sous l'effet de \( f\). En raisonnant de même, nous voyons que \( f\) retourne toutes les diagonales. Donc les éléments non triviaux de \( \ker(\rho)\) retournent toutes les diagonales; il n'y en a donc qu'un seul et c'est la symétrie centrale :
\begin{equation}
	\ker(\rho)=\{ \id,s_0 \}.
\end{equation}
Le premier théorème d'isomorphisme~\ref{ThoPremierthoisomo} nous permet d'écrire le quotient de groupes :
\begin{equation}
	\frac{ G }{ \{ \id,s_0 \} }\simeq S_4.
\end{equation}
Une classe d'équivalence modulo \( \ker(\rho)\) dans \( G\) est donc toujours de la forme \( \{ f,f\circ s_0 \}\). Et comme \( s_0\) est de déterminant \( -1\), une classe contient toujours exactement un élément de déterminant \( 1\) et un de déterminant \( -1\).

D'autre part, \( \ker(\rho)\) est normal dans \( G\) parce qu'en tant que matrice, \( s_0=-\mtu\), donc les problèmes de commutativité ne se posent pas. L'application
\begin{equation}
	\begin{aligned}
		\varphi\colon \frac{ G }{ \{ \id,s_0 \} } & \to G^+                                   \\
		[g]                                       & \mapsto \begin{cases}
			                                                    g          & \text{si } \det(g)>0 \\
			                                                    g\circ s_0 & \text{sinon}
		                                                    \end{cases}
	\end{aligned}
\end{equation}
est un isomorphisme de groupes. Et enfin nous pouvons écrire
\begin{equation}
	G^+\simeq S_4.
\end{equation}

Nous allons maintenant utiliser le corolaire~\ref{CoroGohOZ} pour montrer que \( G=G^+\times_{\sigma}\ker(\rho)\). Les conditions sont remplies :
\begin{itemize}
	\item \( \ker(\rho)\) normalise \( G^+\) parce que \( \ker(\rho)\) ne contient que \( \pm\mtu\).
	\item \( \ker(\rho)\cap G^+=\{ \id \}\).
	\item \( \ker(\rho)G^+=G\) parce que les classes d'équivalence de \( G\) modulo \( \ker(\rho)\) sont composées de \( \{ f,f\circ s_0 \}\).
\end{itemize}
Puisque \( G^+\simeq S_4\) et \( \ker(\rho)\simeq \eZ/2\eZ\) nous pouvons écrire de façon plus brillante que
\begin{equation}
	G\simeq S_4\times_{\sigma}\eZ/2\eZ.
\end{equation}
Mais étant donné que la conjugaison par \( s_0\) est triviale, le produit semi-direct est un produit direct :
\begin{equation}
	G\simeq S_4\times\eZ/2\eZ.
\end{equation}
Il est maintenant du meilleur gout de pouvoir identifier géométriquement ces éléments. Les éléments de \( \eZ/2\eZ=\{ \id,s_0 \}\) ne font pas de mystère. Dans \( S_4\) nous avons les classes de conjugaison des éléments \( \id\), \( (12)\), \( (123)\), \( (1234)\) et \( (12)(34)\) déterminées durant l'exemple~\ref{ExVYZPzub}.
\begin{enumerate}
	\item
	      L'élément \( (12)\) consiste à permuter deux diagonales et laisser les autres en place. Nous avons déjà vu que c'était une symétrie axiale passant par les milieux de deux côtés opposés. Cela fait \( 6\) axes d'ordre \( 2\).
	\item
	      L'élément \( (123)\) fixe une des diagonales. C'est donc la symétrie axiale le long de la diagonale fixée. Par exemple la symétrie d'axe \( (AG)\) fait bouger le point \( B\) de la façon suivante :
	      \begin{equation}
		      B\to D\to E\to B.
	      \end{equation}
	      C'est une rotation d'angle \( \frac{ 2\pi }{ 3 }\). En tout, nous avons donc \( 8\) rotations d'ordre \( 3\).

	      Notons à ce propos que la différence entre \( (234)\) et \( (243)\) est que la première réalise une rotation d'angle \( 2\pi/3\) tandis que la seconde, une rotation d'angle \( -2\pi/3\).

	\item
	      L'élément \( (1234)\) ne maintient aucune des diagonales et est d'ordre \( 4\). C'est donc la rotation d'angle \( \pi/2\) ou \( -\pi/2\) autour de l'axe passant par les milieux de deux faces opposées. Il y en a \( 6\) comme ça (\( 3\) paires de faces puis pour chaque il y a \( \pi/2\) et \( -\pi/2\)), et ça tombe bien \( 6\) est justement la taille de la classe de conjugaison de \( (1234)\) dans \( S_4\).

	\item
	      L'élément \( (12)(34)\) est le carré de la précédente\footnote{En fait c'est \( (13)(24)\), le carré de la précédente, mais c'est la même classe de conjugaison.}, c'est-à-dire les rotations d'angle \( \pi\) autour des mêmes axes. Cela fait \( 3\) éléments d'ordre \( 2\).

\end{enumerate}
