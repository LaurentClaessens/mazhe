% This is part of (everything) I know in mathematics
% Copyright (c) 2011-2013,2016-2023
%   Laurent Claessens
% See the file fdl-1.3.txt for copying conditions.

%+++++++++++++++++++++++++++++++++++++++++++++++++++++++++++++++++++++++++++++++++++++++++++++++++++++++++++++++++++++++++++
\section{Trigonométrie}
%+++++++++++++++++++++++++++++++++++++++++++++++++++++++++++++++++++++++++++++++++++++++++++++++++++++++++++++++++++++++++++

%---------------------------------------------------------------------------------------------------------------------------
\subsection{Définitions, périodicité et quelques valeurs remarquables}
%---------------------------------------------------------------------------------------------------------------------------

\begin{propositionDef}[Définition du cosinus et du sinus]       \label{PROPooZXPVooBjONka}
	La série
	\begin{equation}
		\cos(x)=\sum_{n=0}^{\infty}\frac{ (-1)^n }{ (2n)! }x^{2n}
	\end{equation}
	définit une fonction \( \cos\colon \eR\to \eR\) de classe \(  C^{\infty}\). Nous l'appelons \defe{cosinus}{cosinus}.

	La série
	\begin{equation}        \label{EQooCMRFooCTtpge}
		\sin(x)=\sum_{n=0}^{\infty}\frac{ (-1)^n }{ (2n+1)! }x^{2n+1}
	\end{equation}
	définit une fonction \( \sin\colon \eR\to \eR\) de classe \(  C^{\infty}\). Nous l'appelons \defe{sinus}{sinus}.
\end{propositionDef}

\begin{proof}
	La série entière définissant \( \cos(x)\) a pour coefficients
	\begin{equation}
		a_n=\begin{cases}
			0                         & \text{si } n\text{ est impair} \\
			\frac{ (-1)^{n/2} }{ n! } & \text{si } n\text{ est pair}.
		\end{cases}
	\end{equation}
	Nous pouvons la majorer par la série entière donnée par les coefficients
	\begin{equation}
		b_n=\begin{cases}
			1/n!                      & \text{si } n\text{ est impair} \\
			\frac{ (-1)^{n/2} }{ n! } & \text{si } n\text{ est pair}.
		\end{cases}
	\end{equation}
	Quelle que soit la parité de \( k\) nous avons toujours
	\begin{equation}
		\left| \frac{ b_{k+1} }{ b_k } \right|=\frac{1}{ k+1 },
	\end{equation}
	de telle sorte que la formule d'Hadamard \eqref{EqAlphaSerPuissAtern} nous donne \( R=\infty\) pour la série \( \sum_{k=0}^{\infty}b_kx^k\). A fortiori\footnote{Remarque~\ref{REMooYOTEooKvxHSf}.} le rayon de convergence pour la série du cosinus est infini.

	L'assertion concernant le sinus se démontre de même.

	En ce qui concerne le fait que les fonctions \( \sin\) et \( \cos\) sont de classe \(  C^{\infty}\) sur \( \eR\), il faut invoquer le corolaire~\ref{CorCBYHooQhgara}.
\end{proof}

\begin{lemma}[\cite{MonCerveau}]        \label{LEMooZMJTooJPnyfv}
	Nous avons
	\begin{subequations}        \label{SUBEQooTTNNooXzApSM}
		\begin{align}
			\cos(0) & =1  \\
			\sin(0) & =0.
		\end{align}
		ainsi que
		\begin{subequations}        \label{SUBEQSooFOGNooQrBxYc}
			\begin{align}
				\cos(-x) & =\cos(x)  \\
				\sin(-x) & =-\sin(x) \\
			\end{align}
		\end{subequations}
	\end{subequations}
\end{lemma}

\begin{proof}
	Par substitution directe dans les séries.
\end{proof}

\begin{lemma}       \label{LEMooBBCAooHLWmno}
	En ce qui concerne la dérivation, nous avons
	\begin{subequations}
		\begin{align}
			\sin' & =\cos   \\
			\cos' & =-\sin.
		\end{align}
	\end{subequations}
\end{lemma}

\begin{proof}
	Il s'agit de se permettre de dériver terme à terme (proposition~\ref{ProptzOIuG}) les séries qui définissent le sinus et le cosinus.
\end{proof}

\begin{lemma}       \label{LEMooAEFPooGSgOkF}
	Les fonctions sinus et cosinus vérifient
	\begin{equation}        \label{EQooNYCZooApyyRd}
		\cos^2(x)+\sin^2(x)=1
	\end{equation}
	pour tout \( x\in \eR\).
\end{lemma}

\begin{proof}
	Posons \( f(x)=\sin^2(x)+\cos^2(x)\) et dérivons :
	\begin{equation}
		f'(x)=2\sin(x)\cos(x)+2\cos(x)(-)\sin(x)=0.
	\end{equation}
	La fonction \( f\) est donc constante par le corolaire~\ref{CORooEOERooYprteX}. Nous avons donc pour tout \( x\) :
	\begin{equation}
		f(x)=f(0)=\sin^2(0)+\cos^2(0)=1.
	\end{equation}
	Le dernier calcul s'obtient en substituant directement \( x\) par zéro dans les séries : \( \sin(0)=0\) et \( \cos(0)=1\).
\end{proof}

%---------------------------------------------------------------------------------------------------------------------------
\subsection{Fonction puissance (pour les complexes)}
%---------------------------------------------------------------------------------------------------------------------------

La fonction puissance a déjà fait l'objet de nombreuses définitions et extensions. Voir le thème \ref{THEMEooBSBLooWcaQnR}. Nous allons maintenant définir \( a^z\) pour \( a>0\) et \( z\in \eC\).


Soit \( z=x+iy\in \eC\). L'exponentielle \( \exp(x+yi)\) est déjà définie en \ref{DEFooSFDUooMNsgZY}; il suffit donc maintenant de définir les notations \(  e^{z}\) et \( a^z\) pour \( z\in \eC\).

\begin{definition}      \label{DEFooRBTDooNLcWGj}
	Pour le nombre \( e\in \eR\) et le nombre imaginaire pur \( iy\) (\( y\in \eR\)), nous définissons
	\begin{equation}
		e^{iy}=\exp(iy)
	\end{equation}
	où \( \exp\) est la série usuelle de la définition \ref{DEFooSFDUooMNsgZY}. Pour un nombre complexe général \( x+yi\) nous définissons
	\begin{equation}
		e^{x+iy}= e^{x} e^{iy}.
	\end{equation}
	Et enfin, si \( a>0\) et si \( z\in \eC\) nous définissons
	\begin{equation}
		a^z= e^{z\ln(a)},
	\end{equation}
	la fonction logarithme ici étant celle \( \ln\colon \mathopen] 0 , \infty \mathclose[\to \eR\) définie par la proposition \ref{DEFooELGOooGiZQjt}.
\end{definition}

Si \( z\in \eC\) et si \( n\in \eZ\), la définition de \( z^n\) ne pose pas de problème, c'est la définition \ref{DEFooGVSFooFVLtNo}.

\begin{normaltext}  \label{DefJilXoM}
	Soit \( z=x+iy\in \eC\). L'exponentielle \( \exp(x+yi)\) est déjà définie en \ref{DEFooSFDUooMNsgZY}; elle est la fonction donnée par
	\begin{equation}
		\begin{aligned}
			\exp\colon \eC & \to \eC                                        \\
			z              & \mapsto \sum_{n=0}^{\infty}\frac{ z^n }{ n! }.
		\end{aligned}
	\end{equation}
\end{normaltext}

\begin{proposition}     \label{PROPooXEYFooIEaPvU}
	Le rayon de convergence\footnote{Définition \ref{DefZWKOZOl}.} de la série exponentielle est infini.
\end{proposition}

\begin{proof}
	L'exponentielle est la série de puissance dont les coefficients sont donnés par la suite \( (a_k)=1/k!\). Nous utilisons la formule de Hadamard de la proposition \ref{ThoSerPuissRap} :
	\begin{equation}
		\frac{1}{ R }=\lim_{n\to \infty} \left| \frac{ (n+1)! }{ n! } \right| =\lim_{k\to \infty} (n+1)=\infty.
	\end{equation}
	Donc \( R=\infty\).
\end{proof}

\begin{proposition}     \label{PROPooWSDKooJREQGk}
	Pour tout \( z\in \eC\) nous avons
	\begin{equation}
		\exp(z)= e^{z}.
	\end{equation}
\end{proposition}

\begin{proposition}[\cite{RomainBoilEnt}]     \label{PropdDjisy}
	Quelques propriétés de l'exponentielle.
	\begin{enumerate}
		\item
		      Le fonction \( \exp\) est continue.
		\item       \label{ITEMooRLHCooJTuYKV}
		      Nous avons la formule \(  e^{z+w}= e^{z}e^w\) pour tout \( z,w\in \eC\).
		\item
		      \( (e^z)^{-1}= e^{-z}\)
		\item       \label{ITEMooIFYFooUniuKS}
		      \( (\exp(z))^n=\exp(nz)\).
	\end{enumerate}
\end{proposition}

\begin{proof}
    La proposition \ref{PROPooXEYFooIEaPvU} nous enseigne que le rayon de convergence est infini. La fonction ainsi définie est alors continue par la proposition \ref{PropOMBbwst}.

	Les séries \( \exp(z)\) et \( \exp(w)\) ayant un rayon de convergence infini, nous pouvons utiliser le produit de Cauchy (théorème~\ref{ThokPTXYC}) :
	\begin{subequations}
		\begin{align}
			e^{z} e^{w} & =\sum_{n=0}^{\infty}\left( \sum_{i+j=n}\frac{ z^iw^j }{ i!j! } \right)         \\
			            & =\sum_{n=0}^{\infty}\left( \sum_{i=0}^n\frac{ z^iw^{n-i} }{ i!(n-i)! } \right) \\
			            & =\sum_{n=0}^{\infty}\frac{1}{ n! }\sum_{i=0}^{n}{n\choose i}z^iw^{n-i}         \\
			            & =\sum_{n=0}^{\infty}\frac{1}{ n! }(z+w)^{n}                                    \\
			            & =\exp(z+w).
		\end{align}
	\end{subequations}
	Nous avons utilisé la formule du binôme (proposition~\ref{PropBinomFExOiL}).

	Les autres propriétés énoncées sont des corolaires :
	\begin{equation}
		e^{z} e^{-z}= e^{0}=1.
	\end{equation}
\end{proof}

D'autres propriétés de l'exponentielle sur \( \eC\), entre autres l'holomorphie, sont données dans le théorème \ref{THOooNGOIooEECfAv}.

\begin{lemma}[\cite{MonCerveau}]        \label{LEMooTDGKooWdpUTD}
	Soient \( a>0\), \( z\in \eC\) et \( n\in \eZ\). Alors
	\begin{equation}
		(a^z)^n=a^{nz}.
	\end{equation}
\end{lemma}

\begin{proof}
	Il s'agit d'un calcul utilisant les propositions \ref{PropdDjisy}\ref{ITEMooIFYFooUniuKS} et \ref{PROPooWSDKooJREQGk} :
	\begin{subequations}
		\begin{align}
			(a^z)^n & =\big(  e^{z\ln(a)} \big)^n \\
			        & =\exp\big( z\ln(a) \big)^n  \\
			        & =\exp\big( nz\ln(a) \big)   \\
			        & = e^{nz\ln(a)}              \\
			        & =a^{nz}.
		\end{align}
	\end{subequations}
\end{proof}

%---------------------------------------------------------------------------------------------------------------------------
\subsection{Formules de trigonométrie}
%---------------------------------------------------------------------------------------------------------------------------

Le lemme suivant est un premier pas pour le paramétrage du cercle dont nous parlerons dans la proposition \ref{PROPooZEFEooEKMOPT}.
\begin{lemma}       \label{LEMooHOYZooKQTsXW}
	Pour tout \( x\in \eR\) nous avons :
	\begin{enumerate}
		\item
		      \begin{equation}        \label{EQooRVPJooTMwNTU}
			      e^{ix}=\cos(x)+i\sin(x)
		      \end{equation}
		\item
		      \( |  e^{ix} |=1\).
	\end{enumerate}
\end{lemma}

\begin{proof}
	La définition de l'exponentielle sur \( \eC\) est la définition~\ref{DEFooSFDUooMNsgZY}. Cette définition fonctionne parce que \( \eC\) est une algèbre normée, et que \( \eC\) est un \( \eC\)-module vérifiant l'inégalité \(  | zz' |\leq | z | |z' | \) (en l'occurrence, une égalité).

	Nous remarquons que \( i^k\) vaut \( 1\), \( i\), \( -1\), \( -i\). Donc un terme sur deux est imaginaire pur et parmi ceux-là, un sur deux est positif. À bien y regarder, les termes imaginaires purs forment la série du sinus et ceux réels la série du cosinus.

	Si vous aimez les formules,
	\begin{equation}
		e^{iy}=\sum_{n=0}^{\infty}\frac{ (iy)^n }{ n! }
		=\sum_{n=0}^{\infty}(-1)^n\frac{ y^{2n} }{ (2n)! }+i\sum_{n=0}^{\infty}(-1)^n\frac{ y^{2n+1} }{ (2n+1)! }
		=\cos(y)+i\sin(y).
	\end{equation}
	Nous avons utilisé le fait que \( i^{2n}=(-1)^n\) et \( i^{2n+1}=i(-1)^n\).
\end{proof}

\begin{corollary}       \label{CORooWZFIooDTCoRo}
	Le complexe conjugué\footnote{Définition \ref{DEFooQDDVooRYDsAJ}.} de \(  e^{ix}\) est \(  e^{-ix}\).
\end{corollary}

\begin{proof}
	Vu le lemme \ref{LEMooHOYZooKQTsXW}, le complexe conjugué de \(  z=e^{ix}\) est \(\bar z= \cos(x)-i\sin(x)\). En utilisant \eqref{SUBEQSooFOGNooQrBxYc} nous avons également
	\begin{equation}
		\bar z=\cos(x)-i\sin(x)=\cos(-x)+i\sin(-x)= e^{-ix}.
	\end{equation}
\end{proof}

\begin{lemma}       \label{LEMooJAWBooJGfZIL}
	Nous avons les formules d'addition d'angles\footnote{Rien ne nous empêche de donner ce nom à ces formules, mais seriez-vous capable de définir précisément le mot «angle» ?}
	\begin{subequations}        \label{SUBEQSooFSSMooHcYwRc}
		\begin{align}
			\cos(a+b)=\cos(a)\cos(b)-\sin(a)\sin(b) \label{EQooJYEMooQaOMib} \\
			\sin(a+b)=\cos(a)\sin(b)+\sin(a)\cos(b) \label{EQooECAUooQzckDv} \\
			\cos(a-b)=\cos(a)\cos(b)+\sin(a)\sin(b) \label{EQooCVZAooQfocya} \\
			\sin(a-b)=\sin(a)\cos(b)-\cos(a)\sin(b).
		\end{align}
	\end{subequations}
	pour tout \( a\), \( b\) réels.
\end{lemma}

\begin{proof}
	Nous utilisons la formule d'addition dans l'exponentielle, proposition \eqref{EQooVFXUooBfwjJY} et la formule \eqref{EQooRVPJooTMwNTU} avant de séparer les parties réelles et imaginaires :
	\begin{equation}
		e^{i(a+b)}= e^{ia} e^{ib}=\cos(a)\cos(b)-\sin(a)\sin(b)+i\big( \cos(a)\sin(b)+\sin(a)\cos(b) \big).
	\end{equation}
	Cela est également égal à
	\begin{equation}
		\cos(a+b)+i\sin(a+b).
	\end{equation}
	En identifiant les parties réelle et imaginaires, nous obtenons les formules \eqref{EQooJYEMooQaOMib} et \eqref{EQooCVZAooQfocya} annoncées.

	Pour la formule \eqref{EQooCVZAooQfocya}, il suffit de se souvenir que \( \sin(-b)=-\sin(b)\) et \( \cos(-b)=\cos(b)\) (ces deux égalités sont immédiatement visibles sur les développements en série : l'un a uniquement des puissances paires et l'autre impaires) et d'écrire \eqref{EQooJYEMooQaOMib} avec \( -b\) au lieu de \( b\).
\end{proof}

\begin{corollary}       \label{CORooQZDQooWjMXTF}
	Les formules suivantes pour les duplications d'angles s'ensuivent :
	\begin{subequations}
		\begin{align}
			\cos(2a) & =\cos^2(a)-\sin^2(a)                               \\
			\sin(2a) & =2\cos(a)\sin(a).      \label{SUBEQooLRJDooQuFvux}
		\end{align}
	\end{subequations}
\end{corollary}

\begin{proof}
	Poser \( b=a\) dans les relations du lemme~\ref{LEMooJAWBooJGfZIL}.
\end{proof}

\begin{lemma}       \label{LEMooPQWWooMdPWUT}
	Un sous-groupe de \( (\eR,+)\) est soit dense dans \( \eR\) soit de la forme \( p\eZ\) pour un certain réel \( p\neq 0\).
\end{lemma}

\begin{proof}
	Soit \( A\), un sous-groupe de \( (\eR,+)\) qui ne soit pas dense. Soit un intervalle \( \mathopen] a , b \mathclose[\) qui n'intersecte pas \( A\) (si vous voulez frimer, vous noterez ici que nous utilisons le fait que les intervalles ouverts forment une base de la topologie de \( \eR\)). Si \( d=| b-a |\), l'ensemble \( A\) ne contient pas deux éléments séparés par strictement moins de \( d\). Soit \( p\), le plus petit élément strictement positif de \( A\); nous avons \( p\geq d\) (parce que \( 0\in A\) de toutes façons).

	Puisque \( A\) est un groupe, nous avons \( p\eZ\subset A\).

	Pour l'inclusion inverse, si \( x\in A\) est hors de \( p\eZ\), il existe un \( y\in p\eZ\) avec \( | x-y |<p\). Et donc le nombre \( | x-y |\) est dans \( A\) tout en étant plus petit que \( p\). Contradiction.
\end{proof}

\begin{propositionDef}[Périodicité, le nombre \( \pi\)\cite{ooUMDHooHrJpfV}]      \label{PROPooFRVCooKSgYUM}
	Plusieurs choses à propos de la périodicité de la fonction \( \cos\).
	\begin{enumerate}
		\item
		      La fonction \( \cos\) est périodique\footnote{Définition \ref{DEFooHUZAooYyBmwe}.}.
		\item       \label{ITEMooVPMWooBqidZG}
		      Un nombre \( T>0\) est une période si et seulement si \( \cos(T)=1\) et \( \sin(T)=0\).
	\end{enumerate}

	Nous définissons le nombre \( \pi>0\) comme étant la moitié de la période de la fonction \( \cos\) :
	\begin{equation}
		2\pi=\min\{ T>0\tq \cos(x+T)=\cos(x) , \forall x \in \eR\}.
	\end{equation}

\end{propositionDef}

\begin{proof}
	Plusieurs étapes.
	\begin{subproof}
		\spitem[La fonction cosinus n'est pas toujours positive]
		Supposons d'abord que \( \cos(x)>0\) pour tout \( x\in \eR\). Dans ce cas, la fonction \( \sin\) est strictement croissante. Mais les deux fonctions sont bornées par \( 1\) du fait de la formule \( \cos^2(x)+\sin^2(x)=1\). La fonction \( \sin\) étant croissante et bornée, elle est convergente vers un réel par la proposition~\ref{PropMTmBYeU} :
		\begin{equation}
			\lim_{x\to \infty} \sin(x)=\ell
		\end{equation}
		pour un certain \( \ell>0\). Avec ça nous avons aussi (pour cause de dérivée) \( \lim_{x\to \infty} \sin'(x)=0\), c'est-à-dire \( \lim_{x\to \infty} \cos(x)=0\). Mais vu que \( \cos^2(x)+\sin^2(x)=1\), nous en déduisons que \( \lim_{x\to \infty} \sin(x)=1\). Mézalor \( \lim_{x\to \infty} \cos'(x)=-1\), ce qui veut dire que la fonction \( \cos\) n'est pas bornée. Cela est impossible. Nous en déduisons que \( \cos(x)\) n'est pas toujours positive.

		\spitem[Il existe \( T>0\) tel que \( \cos(T)=1\) et \( \sin(T)=0\)]
		Par ce que nous venons de faire, il existe \( r>0\) tel que \( \cos(r)=0\). Pour cette valeur, nous avons aussi obligatoirement \( \sin(r)=\pm 1\). Nous avons aussi, en utilisant les formules \eqref{SUBEQSooFSSMooHcYwRc},
		\begin{subequations}
			\begin{align}
				\cos(2r)=\cos^2(r)-\sin^2(r)=-1 \\
				\sin(2r)=2\cos(r)\sin(r)=0.
			\end{align}
		\end{subequations}
		et par conséquent
		\begin{subequations}
			\begin{align}
				\cos(4r)=\cos^2(2r)-\sin^2(2r)=1 \\
				\sin(4r)=2\cos(2r)\sin(2r)=0.
			\end{align}
		\end{subequations}
		Donc \( T=4r\) fonctionne.

		\spitem[Si \( T\) est une période]
		Nous entrons dans le vif de la preuve. Soit un \( T>0\) tel que \( \cos(x+T)=\cos(x)\) pour tout \( x\in \eR\). Avec la formule d'addition d'angle dans le cosinus nous cherchons un \( T\) tel que
		\begin{equation}
			\cos(x+T)=\cos(x)\cos(T)-\sin(x)\sin(T)=\cos(x)
		\end{equation}
		et donc tel que
		\begin{equation}        \label{EQooELSAooLNtBnm}
			\cos(x)\big( \cos(T)-1 \big)=\sin(x)\sin(T).
		\end{equation}
		Nous dérivons cette équation :
		\begin{equation}        \label{EQooCECFooLpxXaw}
			-\sin(x)\big( \cos(T)-1 \big)=\cos(x)\sin(T).
		\end{equation}
		Nous multiplions chacune des deux équations \eqref{EQooELSAooLNtBnm} et \eqref{EQooCECFooLpxXaw} par \( \sin(x)\) et \( \cos(x)\) pour obtenir les quatre relations suivantes :
		\begin{subequations}
			\begin{align}
				\cos^2(x)\big( \cos(T)-1 \big)-\sin(x)\cos(x)\sin(T)=0      \label{SUBEQooLGQXooIrLMLW} \\
				-\sin(x)\cos(x)\big( \cos(T)-1 \big)-\cos^2(x)\sin(T)=0     \label{SUBEQooCHTDooKwvyZF} \\
				\sin(x)\cos(x)\big( \cos(T)-1 \big)-\sin^2(x)\sin(T)=0      \label{SUBEQooEWPTooTLCUMf} \\
				-\sin^2(x)\big( \cos(T)-1 \big)-\sin(x)\cos(x)\sin(T)=0     \label{SUBEQooGBXTooCFekGJ}
			\end{align}
		\end{subequations}
		En faisant \eqref{SUBEQooLGQXooIrLMLW} moins \eqref{SUBEQooGBXTooCFekGJ} nous trouvons \( \cos(T)=1\). Et en sommant \eqref{SUBEQooCHTDooKwvyZF} avec \eqref{SUBEQooEWPTooTLCUMf} nous avons \( -\sin(T)=0\).

		\spitem[Si \( T>0\) est tel que \( \sin(T)=0\) et \( \cos(T)=1\)]
		Alors les formules d'addition d'angle du lemme \ref{LEMooJAWBooJGfZIL} donnent tout de suite
		\begin{equation}
			\cos(x+T)=\cos(x).
		\end{equation}

	\end{subproof}

	À ce niveau nous croyons avoir prouvé que \( \cos\) était périodique et que la période est donnée par
	\begin{equation}
		\min\{ T>0\tq \sin(T)=0,\cos(T)=1 \}.
	\end{equation}
	Or rien n'est moins sûr parce qu'il pourrait arriver que ce minimum n'existe pas, c'est-à-dire que l'infimum soit zéro. Autrement dit, il peut arriver que l'ensemble des périodes soit dense. Plus précisément, soit \( P\subset \eR\) l'ensemble des périodes de \( \cos\). C'est un sous-groupe de \( (\eR,+)\) et le lemme~\ref{LEMooPQWWooMdPWUT} nous dit que \( P\) est soit dense dans \( \eR\), soit de la forme \( p\eZ\) pour un \( p>0\).

	Si \( P\) est dense, soient \( t\in \eR\) et une suite \( (t_n)\) dans \( P\) telle que \( t_n\to t\). Pour tout \( x\) et tout \( n\) nous avons
	\begin{equation}
		\cos(x+t_n)=\cos(x),
	\end{equation}
	Comme la fonction cosinus est continue, nous pouvons passer à la limite et écrire \( \cos(x+t)=\cos(x)\). Cela étant valable pour tout \( x\) et pour tout \( t\), la fonction cosinus est constante. Or nous savons que ce n'est pas le cas, donc \( P\) n'est pas dense. Donc cosinus est périodique.
\end{proof}

\begin{proposition}     \label{PROPooKNLAooLwQHea}
	La fonction \( \sin\) est périodique de période \( 2\pi\) et
	\begin{equation}
		2\pi=\min\{ T>0\tq \sin(T)=0,\cos(T)=1 \}.
	\end{equation}
\end{proposition}

\begin{proof}
	La proposition \ref{PROPooFRVCooKSgYUM} dit que \( \cos\) est périodique. Puisque \( \sin=-\cos'\) par le lemme \ref{LEMooBBCAooHLWmno}, la fonction \( \sin\) est également périodique par le lemme \ref{LEMooHWQYooXcNLts}. Si \( T\) est une période de \( \cos\), alors \( T\) est une période de \( \sin\).

	Mais \( \sin'=\cos\), de telle sorte que les périodes de \( \sin\) sont périodes de \( \cos\). Bref, \( T\) est une période de \( \sin\) si et seulement si \( T\) est une période de \( \cos\).
\end{proof}

\begin{normaltext}
	Notons que tout ceci ne nous donne pas la plus petite indication d'ordre de grandeur de la valeur de \( \pi\). Cela peut encore être \( 0.1\) autant que \( 500\).
\end{normaltext}

\begin{proposition}[\cite{ooUMDHooHrJpfV,MonCerveau}]      \label{PROPooMWMDooJYIlis}
	Des propriétés à la chaine à propos des sinus, cosinus et de leurs périodes.
	\begin{enumerate}
		\item       \label{ITEMooRJZHooCXcKmM}
		      Nous avons
		      \begin{equation}
			      2\pi=\min\{ x>0\tq \cos(x)=1, \sin(x)=0 \}.
		      \end{equation}
		\item       \label{ITEMooTNHMooUtOjNC}
		      Les fonctions \( \sin\) et \( \cos\) sont périodiques de période \( 2\pi\).
		\item       \label{ITEMooSPZBooIQLUXh}
		      Nous avons \( \cos(\pi)=- 1\) et \( \sin(\pi)=0\).
		\item
		      Pour tout \( a\in \eR\) nous avons
		      \begin{subequations}
			      \begin{align}
				      \cos(a+\pi) & =-\cos(a)  \\
				      \sin(a+\pi) & =-\sin(a).
			      \end{align}
		      \end{subequations}
		\item       \label{ITEMooHDQNooYHVCkg}
		      Nous avons
		      \begin{equation}
			      \pi=\min\{ x>0\tq \cos(x)=-1,\sin(x)=0 \}.
		      \end{equation}
		\item       \label{ITEMooWFNUooYAybDB}
		      Nous avons
		      \begin{subequations}        \label{SUBEQSooBTNPooSvCAHO}
			      \begin{numcases}{}
				      \cos(\pi/2)=0\\
				      \sin(\pi/2)=1.
			      \end{numcases}
		      \end{subequations}
		\item       \label{ITEMooIRALooBMGOXP}
		      Nous avons les formules
		      \begin{subequations}        \label{EQSooRJZGooCFVqbZ}
			      \begin{numcases}{}
				      \cos(x+\pi/2)=-\sin(x)\\
				      \sin(x+\pi/2)=\cos(x)
			      \end{numcases}
		      \end{subequations}
		      pour tout \( x\in \eR\).
		\item       \label{ITEMooMQQPooGwOdbt}
		      Nous avons
		      \begin{equation}
			      \frac{ \pi }{2}=\min\{ x>0 \tq \sin(x)=1,\cos(x)=0 \}.
		      \end{equation}
		\item
		      Nous avons les valeurs
		      \begin{subequations}
			      \begin{numcases}{}
				      \cos(\frac{ 3\pi }{ 2 })=0\\
				      \sin(\frac{ 3\pi }{ 2 })=-1.
			      \end{numcases}
		      \end{subequations}
		\item       \label{ITEMooQKPKooEPeHER}
		      Nous avons
		      \begin{equation}
			      \frac{ \pi }{2}=\min\{ x>0\tq \cos(x)=0 \}.
		      \end{equation}
		\item               \label{ITEMooMEXUooGfSInJ}
		      Pour tout \( x\in \mathopen] 0 , \frac{ \pi }{ 2 } \mathclose[\), nous avons \( \cos(x)>0\) et \( \sin(x)>0\).
	\end{enumerate}
\end{proposition}

\begin{proof}
	C'est parti.
	\begin{enumerate}
		\item
		      Le fond de la proposition~\ref{PROPooFRVCooKSgYUM} est que toutes les périodes \( T>0\) vérifient \( \cos(T)=1\) et \( \sin(T)=0\). La définition de \( 2\pi\) est que c'est la plus petite période.
		\item
		      En utilisant le fait que l'une est la dérivée de l'autre, si \( T\) est une période de \( \cos\) nous avons
		      \begin{subequations}
			      \begin{align}
				      \sin(x+T) & =-\cos'(x+T)                                                            \\
				                & =-\lim_{\epsilon\to 0}\frac{ \cos(x+T+\epsilon)-\cos(x+T) }{\epsilon  } \\
				                & =-\lim_{\epsilon\to 0}\frac{ \cos(x+\epsilon)-\cos(x) }{ \epsilon }     \\
				                & =-\cos'(x)                                                              \\
				                & =\sin(x).
			      \end{align}
		      \end{subequations}
		      Nous déduisons que toute période de \( \cos\) est une période de \( \sin\). De la même façon, nous pouvons prouver l'autre sens : toute période de \( \sin\) est une période de \( \cos\).
		\item
		      D'un côté nous avons
		      \begin{equation}
			      \cos(2\pi)=\cos^2(\pi)-\sin^2(\pi)=1
		      \end{equation}
		      parce que \( \cos(2\pi)=\cos(0)=1\). Puisque \( \cos(\pi)\) et \( \sin(\pi)\) sont bornés par \( -1\) et \( 1\), nous devons avoir \( \sin(\pi)=0\) et \( \cos(\pi)=\pm 1\).

		      Mais d'un autre côté, le nombre \( 2\pi\) est le plus petit \( T\) vérifiant \( \cos(T)=1\), \( \sin(T)=0\). Donc, avoir \( \cos(\pi)=1\) n'est pas possible. Nous concluons
		      \begin{subequations}
			      \begin{numcases}{}
				      \cos(\pi)=-1\\
				      \sin(\pi)=0.
			      \end{numcases}
		      \end{subequations}
		\item
		      Il s'agit d'utiliser les formules d'addition d'angles du lemme~\ref{LEMooJAWBooJGfZIL} pour calculer \( \cos(a+\pi)\) et \( \sin(a+\pi)\) en tenant compte du fait que \( \cos(\pi)=-1\) et \( \sin(\pi)=0\).
		\item
		      Soit \( a\in\mathopen] 0 , \pi \mathclose[\) tel que \( \cos(a)=-1\) et \( \sin(a)=0\). Alors nous avons
			      \begin{subequations}
				      \begin{align}
					      \cos(a+\pi)=-\cos(\pi)=1 \\
					      \sin(a+\pi)=-\sin(\pi)=0,
				      \end{align}
			      \end{subequations}
			      ce qui donnerait \( a+\pi\in\mathopen] \pi , 2\pi \mathclose[\) dont le cosinus est \( 1\) et le sinus est zéro. Mais nous savons déjà que \( 2\pi\) est le minimum pour cette propriété.
		\item
		      Nous avons
		      \begin{equation}
			      -1=\cos(\pi)=\cos^2(\pi/2)-\sin^2(\pi/2),
		      \end{equation}
		      donc \( \cos(\pi/2)=0\) et \( \sin^2(\pi/2)=1\), ce qui donne \( \sin(\pi/2)=\pm 1\).

		      Nous devons départager le \( \pm\). Pour cela nous savons que \( \sin'(0)=\cos(0)=1\) et que \( \sin(0)=0\), donc il existe \( \epsilon>0\) tel que pour tout \( x\in\mathopen] 0 , \epsilon \mathclose[\) nous avons \( 0<\cos(x)<1\) et \( 0<\sin(x)<1\) (nous avons aussi utilisé le lien entre dérivation et croissance de la proposition \ref{PropGFkZMwD}). Nous choisissons \( \epsilon\) plus petit que \( \pi/2\) .

			      Supposons que \( \sin(\pi/2)=-1\). Le théorème des valeurs intermédiaires~\ref{ThoValInter} dit qu'il existe \( x_0\in\mathopen] \epsilon , \pi/2 \mathclose[\) tel que \( \sin(x_0)=0\). Pour cette valeur de \( x_0\) nous devons aussi avoir \( \cos(x_0)=\pm 1\). Mais puisque \( 2\pi\) est minimum pour avoir \( \cos=1\) et \( \sin=0\), nous devons avoir \( \cos(x_0)=-1\). Alors nous avons aussi
		      \begin{subequations}
			      \begin{align}
				      \cos(x_0+\pi) & =\cos(x_0)\cos(\pi)-\sin(x_0)\sin(\pi)  =-\cos(x_0)   =1  \\
				      \sin(x_0+\pi) & =\cos(x_0)\sin(\pi)+\sin(x_0)\cos(\pi)  = \sin(x_0)   =0.
			      \end{align}
		      \end{subequations}
		      Encore une fois par minimalité de \( 2\pi\), cela ne va pas. Conclusion : \( \sin(\pi/2)=1\).
		\item
		      Il s'agit encore d'utiliser les formules d'addition d'angle en tenant compte des valeurs remarquables \( \cos(\pi/2)=0\) et \( \sin(\pi/2)=1\).
		\item
		      Supposons \( x_0\in\mathopen] 0 , \pi/2 \mathclose[\) tel que \( \sin(x_0)=1\) et \( \cos(x_0)=0\). En utilisant les formules \eqref{EQSooRJZGooCFVqbZ} nous avons
		      \begin{subequations}
			      \begin{align}
				      \cos(x_0+\pi/2)=-1 \\
				      \sin(x_0+\pi/2)=0,
			      \end{align}
		      \end{subequations}
		      avec \( x_0+\pi/2<\pi\). Cela contredirait la minimalité de \( \pi\).
		\item
		      Il s'agit d'utiliser les formules \eqref{EQSooRJZGooCFVqbZ} :
		      \begin{subequations}
			      \begin{align}
				      \cos(\frac{ 3\pi }{ 2 }) & =\cos(\pi+\pi/2)   =-\sin(\pi) =0   \\
				      \sin(\frac{ 3\pi }{ 2 }) & =\sin(\pi+\pi/2)   = \cos(\pi) =-1.
			      \end{align}
		      \end{subequations}
		\item
		      Si \( \cos(x_0)=0\) alors \( \sin(x_0)=-1\) (parce que \( \sin(x_0)=1\) est déjà exclu). Alors \( \cos(x_0+\pi/2)=1\) et \( \sin(x_0+\pi/2)=0\), ce qui est également impossible.
		\item
		      La fonction cosinus est continue (proposition \ref{PROPooZXPVooBjONka}) et \( \cos(0)=1\). Le théorème des valeurs intermédiaires implique que si \( \cos(x)\leq 0\), alors il existe \( t\in \mathopen] 0 , x \mathclose]\) avec \( \cos(x)=0\). Cela n'est pas possible pour \( x<\pi/2\), par le point \ref{ITEMooMQQPooGwOdbt}.

		      Le cosinus est positif sur l'intervalle considéré et \( \sin'(x)=\cos(x)\). Donc \( \sin(0)=0\) et la dérivée est positive. La proposition \ref{PropGFkZMwD} conclut que \( \sin\) est strictement croissante et donc, strictement positive.
	\end{enumerate}
\end{proof}

\begin{lemma}[Positivité\cite{MonCerveau}]      \label{LEMooFESYooBoiuol}
	À propos de positivité de la fonction cosinus.
	\begin{enumerate}
		\item       \label{ITEMooIXSDooJyCQyb}
		      \( \cos(0)=1\)
		\item       \label{ITEMooWJEVooGZykbO}
		      \( \cos(x)>0\) pour \( x\in\mathopen[ 0 , \pi/2 \mathclose[\).
		\item       \label{ITEMooANEPooLGmYtc}
		      \( \cos(\pi/2)=0\).
		\item       \label{ITEMooRDWJooZXWyfv}
		      \( \cos(x)<0\) pour \( x\in \mathopen] \pi/2 , 3\pi/2 \mathclose[\)
		\item       \label{ITEMooFKPAooBNlvPU}
		      \( \cos(3\pi/2)=0\).
		\item       \label{ITEMooIDZGooBTDvDF}
		      \( \cos(x)>0\) pour \( x\in\mathopen] 3\pi/2 , 2\pi \mathclose]\).
	\end{enumerate}
\end{lemma}

\begin{proof}
	En plusieurs points.
	\begin{subproof}
		\spitem[Pour \ref{ITEMooIXSDooJyCQyb}]
		C'est déjà fait dans le lemme \ref{LEMooZMJTooJPnyfv}.
		\spitem[Pour \ref{ITEMooWJEVooGZykbO}]
		C'est la proposition \ref{PROPooMWMDooJYIlis}\ref{ITEMooMEXUooGfSInJ}.
		\spitem[Pour \ref{ITEMooANEPooLGmYtc}]
		C'est la proposition \ref{PROPooMWMDooJYIlis}\ref{ITEMooWFNUooYAybDB}.
		\spitem[Pas d'annulation entre \( \pi/2\) et \( \pi\)]
		Nous montrons à présent que \( \cos\) ne s'annule pas entre \( \pi/2\) et \( \pi\). Supposons que \( \cos(\frac{ \pi }{2}+s)=0\) avec \( s\in\mathopen] 0 , \pi/2 \mathclose[\). Comme \( \cos(x)^2+\sin(x)^2=1\) (lemme \ref{LEMooAEFPooGSgOkF}), nous avons
			\begin{subequations}
				\begin{numcases}{}
					\cos(\frac{ \pi }{2}+s)=0\\
					\sin(\frac{ \pi }{2}+s)=\epsilon
				\end{numcases}
			\end{subequations}
			avec \( \epsilon = \pm 1 \). Utilisant trois fois la proposition \ref{PROPooMWMDooJYIlis}\ref{ITEMooIRALooBMGOXP} nous trouvons
			\begin{subequations}
				\begin{numcases}{}
					\cos(x+\frac{ 3\pi }{2})=\sin(x)\\
					\sin(x+\frac{ 3\pi }{2})=-\cos(x)
				\end{numcases}
			\end{subequations}
			pour tout \( x\). Nous appliquons cela à \( x=\frac{ \pi }{2}+s\), en nous souvenant que \( \cos(x+2\pi)=\cos(x)\) et \( \sin(x+2\pi)=\sin(x)\) (par \ref{PROPooMWMDooJYIlis}\ref{ITEMooTNHMooUtOjNC}) :
			\begin{equation}
				\cos(s)=\cos(\frac{ \pi }{2}+s+\frac{ 3\pi }{2})=\sin(\frac{ \pi }{2}+s)=\epsilon
			\end{equation}
			et
			\begin{equation}
				\sin(s)=\sin(\frac{ \pi }{2}+s+\frac{ 3\pi }{2})=-\cos(\frac{ \pi }{2}+s)=0.
			\end{equation}
			Si \( \epsilon=1\), nous avons une contradiction avec \ref{PROPooMWMDooJYIlis}\ref{ITEMooRJZHooCXcKmM}. Si \( \epsilon=-1\), nous avons une contradiction avec \ref{PROPooMWMDooJYIlis}\ref{ITEMooHDQNooYHVCkg}.

			Donc \( \cos(x)\neq 0\) pour \( x\in \mathopen] \frac{ \pi }{2} , \pi \mathclose]\).
		\spitem[\( \cos(x)<0\) sur \( {\mathopen] \pi/2 , \pi \mathclose]}\)]
		Nous savons que \( \cos(\pi)=-1\) (\ref{PROPooMWMDooJYIlis}\ref{ITEMooSPZBooIQLUXh}). Étant donné que la fonction \( \cos\) est continue et qu'elle ne s'annule pas sur \( \mathopen] \pi/2 , \pi \mathclose]\), nous en déduisons qu'elle y est partout strictement négative.
		\spitem[Pour \ref{ITEMooRDWJooZXWyfv}, \ref{ITEMooFKPAooBNlvPU}, \ref{ITEMooIDZGooBTDvDF}]
		Il est directement visible sur le développement de définition que \( \cos(-x)=\cos(x)\). Et comme \( \cos(x+2\pi)=\cos(x)\), nous avons
		\begin{equation}
			\cos(\pi+s)=\cos(-\pi-s)=\cos(\pi-s).
		\end{equation}
		Donc toutes les valeurs (et tous les signes) de \( \cos(x)\) sur \(\mathopen[ \pi , 2\pi \mathclose] \) peuvent être déduits de ceux sur \( \mathopen[ 0 , \pi \mathclose]\).
	\end{subproof}
\end{proof}

\begin{lemma}       \label{LEMooPARBooTXbbiB}
	Soient \( x,y\in \eR\).
	\begin{enumerate}
		\item
		      Nous avons \( \cos(x)=\cos(y)\) si et seulement si
		      \begin{equation}
			      y\in \{ x+2k\pi \}_{k\in \eZ}\cup\{ -x+2k\pi \}_{k\in \eZ}.
		      \end{equation}
		\item
		      Nous avons \( \sin(x)=\sin(y)\) si et seulement si
		      \begin{equation}
			      y\in\{ x+2k\pi \}_{k\in \eZ}\cup\{ -x+2(k+1)\pi \}_{k\in \eZ}.
		      \end{equation}
	\end{enumerate}
\end{lemma}


\begin{proposition}     \label{PROPooZULQooBKWrcv}
	Les nombres \( x,y\in \eR\) vérifient
	\begin{subequations}    \label{SUBEQSooIHUGooAUchjn}
		\begin{numcases}{}
			\cos(x)=\cos(y)\\
			\sin(x)=\sin(y)
		\end{numcases}
	\end{subequations}
	si et seulement si il existe \( k\in \eZ\) tel que \( y=x+2k\pi\).
\end{proposition}

\begin{proof}
	En deux parties.
	\begin{subproof}
		\spitem[\(  \Leftarrow\)]
		Si \( y=x+2k\pi\), le résultat est correct parce que la proposition \ref{PROPooMWMDooJYIlis}\ref{ITEMooTNHMooUtOjNC} dit que \( \sin\) et \( \cos\) sont périodiques de période \( 2\pi\).
		\spitem[\(  \Rightarrow\)]
		Supposons que \( x>y\). Nous calculons \( \sin(x-y)\) et \( \cos(x-y)\) en utilisant les formules du lemme \ref{LEMooJAWBooJGfZIL} et en tenant compte de \eqref{SUBEQSooIHUGooAUchjn}. Cela donne \( \cos(x-y)=1\) et \( \sin(x-y)=0\). La proposition \ref{PROPooFRVCooKSgYUM}\ref{ITEMooVPMWooBqidZG} dit alors que \( x-y\) est une période de la fonction \( \cos\).

		Or \emph{la} période de \( \cos\) est \( 2\pi\) (proposition \ref{PROPooMWMDooJYIlis}\ref{ITEMooTNHMooUtOjNC}). Donc toutes les périodes de \( \cos\) sont les \( 2k\pi\) avec \( k\in \eN\) (lemme \ref{LEMooOGFGooCnTDjO}).
	\end{subproof}
\end{proof}

\begin{corollary}   \label{CORooTFMAooHDRrqi}
	Des nombres \( x,y\in \eR\) vérifient \(  e^{ix}= e^{iy}\) si et seulement si il existe \( k\in\eZ\) tel que \( y=x+2k\pi\).
\end{corollary}

\begin{proof}
	Le lemme \ref{LEMooHOYZooKQTsXW} donne \(  e^{ix}=\cos(x)+i\sin(x)\). Donc l'équation \(  e^{ix}= e^{iy}\) revient au système \eqref{SUBEQSooIHUGooAUchjn} dont les solutions sont bien \( y=x+2k\pi\).
\end{proof}

\begin{lemma}[\cite{MonCerveau}]        \label{LEMooBIPFooQNiTqZ}
	À propos de croissance et décroissance des fonctions trigonométriques.
	\begin{enumerate}
		\item
		      Sur \( \mathopen] 0 , \pi \mathclose[\), la fonction \( \cos\) est décroissante.
		\item
		      Sur \( \mathopen] -\pi , 0 \mathclose[\), la fonction \( \cos\) est croissante.
	\end{enumerate}
\end{lemma}

\begin{proof}
	Nous savons que \( \cos'=\sin\) par le lemme \ref{LEMooBBCAooHLWmno}. La liaison entre dérivée et croissance est la proposition \ref{PropGFkZMwD}. Les signes de la fonction cosinus sont dans le lemme \ref{LEMooFESYooBoiuol}. Les signes de la fonction sinus peuvent être déduits de la proposition \ref{PROPooMWMDooJYIlis}\ref{ITEMooIRALooBMGOXP}.

	Vous avez tout en main.
\end{proof}

Tout cela nous permet de calculer quelques valeurs remarquables de cosinus et sinus ainsi que d'écrire le tableau de variations de sinus et cosinus.

\begin{lemma}       \label{LEMooIGNPooPEctJy}
	Nous avons les valeurs remarquables
	\begin{equation}
		\sin(\frac{ \pi }{ 4 })=\cos(\frac{ \pi }{ 4 })=\frac{ \sqrt{ 2 } }{2}.
	\end{equation}
\end{lemma}

\begin{proof}
	La relation \eqref{SUBEQooLRJDooQuFvux} donne
	\begin{equation}
		0=\cos(\pi/2)=\cos^2(\pi/4)-\sin^2(\pi/4).
	\end{equation}
	Donc \( \cos^2(\pi/4)=\sin^2(\pi/4)\). Mais puisque \( \sin(\pi/4)\) et \( \cos(\pi/4)\) sont positifs, ils sont égaux.

	Nous avons aussi \( \sin^2(\pi/4)+\cos^2(\pi/4)=1\). Donc le nombre \( x=\cos(\pi/4)=\sin(\pi/4)\) vérifie l'équation \( 2x^2=1\), dont l'unique solution positive est \( x=\frac{1}{ \sqrt{ 2 } }=\frac{ \sqrt{ 2 } }{2}\).
\end{proof}

\begin{lemma}       \label{LEMooRMHAooDEAPMw}
	Nous avons la valeur remarquable
	\begin{equation}
		\cos(\frac{ \pi }{ 3 })=\frac{ 1 }{2}.
	\end{equation}
\end{lemma}

\begin{proof}
	Il faut utiliser la formule \eqref{EQooJYEMooQaOMib} avec \( \cos(\pi)=\cos(2\pi/3+\pi/3)\) en sachant que \( \cos(\pi)=-1\). Ensuite \( \cos(2\pi/3)=\cos(\pi/3+\pi/3)\). En décomposant ainsi, nous exprimons \( -1=\cos(\pi)\) en termes de \( \cos(\pi/3)\) et de \( \sin(\pi/3)\). En substituant \( \sin^2(\pi/3)=1-\cos^2(\pi/3)\) nous trouvons que le nombre \( \cos(\pi/3)\) vérifie l'équation
	\begin{equation}
		4x^3-3x+1=0.
	\end{equation}
	Croyez-le ou non, les solutions de cette équation sont \( x=-1\) et \( x=1/2\). Allez. Faisons comme si nous le savions pas. En tout cas, ces deux nombres sont des solutions, et nous avons la factorisation\footnote{Factorisation d'un polynôme en sachant des racines, proposition \ref{PropHSQooASRbeA}.}
	\begin{equation}
		4x^3-3x+1=(2x-1)^2(x+1).
	\end{equation}
	Donc \( 1/2\) est de multiplicité \( 2\) et \( -1\) de multiplicité \( 1\). Le théorème~\ref{ThoSVZooMpNANi} nous dit qu'il n'y a alors pas d'autres racines que ces deux-là\footnote{Nous attirons votre attention sur le fait que cela n'est en aucun cas une trivialité.}.

	Nous en déduisons que la valeur de \( \cos(\pi/3)\) est soit \( 1/2\) soit \( -1\). La proposition~\ref{PROPooMWMDooJYIlis}\ref{ITEMooHDQNooYHVCkg} nous dit qu'il est impossible que \( \cos(\pi/3)\) soit égal à \( -1\) parce que \( \pi/3<\pi\). Donc \( \cos(\pi/3)=1/2\) comme annoncé.
\end{proof}

\begin{remark}
	Vous avez déjà sans doute vu la démonstration de \( \cos(\unit{30}{\degree})=1/2\) à partir de la figure~\ref{LabelFigGVDJooYzMxLW}. Il n'est pas possible de l'utiliser parce que cela n'est en réalité pas loin d'être la définition de l'angle entre deux droites.

	Si vous voulez savoir la définition de l'angle entre deux droites, il faut passer par la définition~\ref{DEFooFLGNooCZUkHY}, laquelle se base sur le lemme~\ref{LEMooHRESooQTrpMz} qui, elle-même, se base sur la proposition~\ref{PROPooKSGXooOqGyZj}.

	Bref, à notre niveau, nous sommes encore loin de pouvoir faire des raisonnements trigonométriques sur base de géométrie dans les triangles.
\end{remark}

\begin{proposition}     \label{PROPooJFAGooYjRJcb}
	Pour tout \( x\in \mathopen[ 0 , \pi/4 \mathclose[\) nous avons \( \cos(x)>\sin(x)\).
\end{proposition}

\begin{proof}
	Nous posons \( f(x)=\cos(x)-\sin(x)\). Elle vérifie \( f(0)=1\). En utilisant les dérivées du lemme \ref{LEMooBBCAooHLWmno}, nous trouvons
	\begin{equation}
		f'(x)=-\big( \sin(x)+\cos(x) \big).
	\end{equation}
	Mais sur \( \mathopen] 0 , \pi/2 \mathclose[\) nous avons \( \cos(x)>0\) et \( \sin(x)>0\) (proposition \ref{PROPooMWMDooJYIlis}\ref{ITEMooMEXUooGfSInJ}). Donc \( f\) est strictement décroissante. Elle ne peut donc passer qu'une seule fois par zéro. Le lemme \ref{LEMooIGNPooPEctJy} nous indique que \( f(\pi/4)=0\). Donc \( f(x)>0\) sur \( \mathopen[ 0 , \pi/4 \mathclose[\).
\end{proof}

\begin{proposition}
	Quelques valeurs trigonométriques.
	\begin{multicols}{2}
		\begin{enumerate}
			\item
			      Pour le sinus :
			      \begin{enumerate}
				      \item
				            \( \sin(0)=0\)
				      \item
				            \( \sin(\pi/6)=1/2\)
				      \item
				            \( \sin(\pi/4)=\sqrt{ 2 }/2\)
				      \item
				            \( \sin(\pi/3)=\sqrt{ 3 }/2\)
				      \item
				            \( \sin(\pi/2)=1\)
			      \end{enumerate}

			\item
			      Pour le cosinus :
			      \begin{enumerate}
				      \item
				            \( \cos(0)=1\)
				      \item
				            \( \cos(\pi/6)=\sqrt{ 3 }/2\)
				      \item
				            \( \cos(\pi/4)=\sqrt{ 2 }/2\)
				      \item
				            \( \cos(\pi/3)=1/2\)
				      \item
				            \( \cos(\pi/2)=0\)
			      \end{enumerate}
			\item
			      Pour la tangente :
			      \begin{enumerate}
				      \item
				            \( \tan(0)=0\)
				      \item
				            \( \tan(\pi/6)=\sqrt{ 3 }/3\)
				      \item
				            \( \tan(\pi/4)=1\)
				      \item
				            \( \tan(\pi/3)=\sqrt{ 3 }\)
				      \item
				            \( \tan(\pi/2)\) est non défini.
			      \end{enumerate}
		\end{enumerate}
	\end{multicols}
\end{proposition}

\begin{proof}
	Plusieurs ont déjà été faites. Les autres ne seront pas démontrées dans l'ordre énoncé.
	\begin{subproof}
		\spitem[\( \sin(0)=0\)]
		Substitution dans la définition \eqref{EQooCMRFooCTtpge}.
		\spitem[\(  \sin(\pi/4)=\sqrt{ 2 }/2\)]
		C'est le lemme \ref{LEMooIGNPooPEctJy}.
		\spitem[\(  \sin(\pi/3)=1/\sqrt{ 2 }\)]
		Nous utilisons la formule \( \sin^2(x)+\cos^2(x)=1\) avec \( x=\pi/3\). Cela donne \( \sin^2(\pi/3)=1/2\). Nous en déduisons que \( \sin(\pi/3)\) vaut \( \pm\frac{1}{ \sqrt{ 2 } }\).

		La proposition \ref{PROPooMWMDooJYIlis}\ref{ITEMooHDQNooYHVCkg} nous dit que \( \sin\) est positive sur \(\mathopen[ 0 , \pi \mathclose]\). Donc c'est bien la possibilité \( 1/\sqrt{ 2 }\) qui est la bonne.
		\spitem[\( \sin(\pi/6)=1/2\) et \( \cos(\pi/6)=\sqrt{ 3 }/2 \)]
		Nous partons de l'équation \eqref{SUBEQooLRJDooQuFvux} pour écrire
		\begin{equation}
			\sin(\pi/3)=2\cos(\pi/6)\sin(\pi/6).
		\end{equation}
		Nous avons déjà vu que \( \sin(\pi/3)=\sqrt{ 3 }/2\). En posant \( x=\sin(\pi/6)\) nous avons également \( \cos(\pi/6)=\sqrt{ 1-x^2 }\) parce que nous savons que la fonction cosinus est positive sur \( \mathopen[ 0 , \pi/2 \mathclose]\) (proposition \ref{PROPooMWMDooJYIlis}\ref{ITEMooMEXUooGfSInJ}). Nous avons donc l'équation
		\begin{equation}
			\frac{ \sqrt{ 3 } }{2}=2x\sqrt{ 1-x^2 }.
		\end{equation}
		Nous passons au carré et posons \( y=x^2\). Après quelque manipulations,
		\begin{equation}
			16y^2-16y+3=0.
		\end{equation}
		Cela donne deux possibilités pour \( y\) : \( \frac{ 3 }{ 4 }\) et \( \frac{1}{ 4 }\). Puisque \( x>0\), nous pouvons simplement passer à la racine carrée : \( x=\sqrt{ 3 }/2\) ou \( x=1/2\).

		Notez que si nous avion posé \( x=\cos(\pi/6)\) au lieu de \( x=\sin(\pi/6)\), nous aurions obtenu le même résultat. Donc \( \sin(\pi/6)\) et \( \cos(\pi/6)\) peuvent tous deux avoir les valeurs \( \sqrt{ 3 }/2\) ou \( 1/2\). Cela fait \( 4\) possibilités.

		Étant donné que \( \sin^2(\pi/6)+\cos^2(\pi/6)=1\), les deux possibilités avec \( \sin(\pi/6)=\cos(\pi/6)\) sont exclues.

		La proposition \ref{PROPooJFAGooYjRJcb} nous dit aussi que \( \cos(\pi/6)>\sin(\pi/6)\). Donc \( \cos(\pi/6)=\sqrt{ 3 }/2\) et \( \sin(\pi/6)=1/2\).
		\spitem[\( \sin(\pi/2)=1 \)] C'est dans \eqref{SUBEQSooBTNPooSvCAHO}.
		\spitem[\( \cos(0)=1 \)] Substitution dans la définition.
		\spitem[\( \cos(\pi/6)=\sqrt{ 3 }/2 \)] Déjà fait avec le sinus de \( \pi/6\).
		\spitem[\( \cos(\pi/4)=\sqrt{ 2 }/2 \)]  Lemme \ref{LEMooIGNPooPEctJy}.
		\spitem[\( \cos(\pi/3)=1/2 \)] Lemme \ref{LEMooRMHAooDEAPMw}.
		\spitem[\( \cos(\pi/2)=0 \)] Dans \eqref{SUBEQSooBTNPooSvCAHO}.
	\end{subproof}
	Toutes les valeurs pour la tangente s'obtiennent maintenant par la définition, en calculant \( \tan(x)=\frac{ \sin(x) }{ \cos(x) }\).
\end{proof}

Voici un tableau qui rappelle les valeurs à retenir pour les fonctions sinus, cosinus et tangente.
\begin{equation}\label{PGooIMQFooTnBdIl}
	\begin{array}[]{|c|c|c|c|}
		\hline
		x     & \sin(x)    & \cos(x)    & \tan(x)     \\
		\hline
		0     & 0          & 1          & 0           \\
		\hline
		\pi/6 & 1/2        & \sqrt{3}/2 & \sqrt{3}/3  \\
		\hline
		\pi/4 & \sqrt{2}/2 & \sqrt{2}/2 & 1           \\
		\hline
		\pi/3 & \sqrt{3}/2 & 1/2        & \sqrt{3}    \\
		\hline
		\pi/2 & 1          & 0          & \text{N.D.} \\
		\hline
	\end{array}
\end{equation}
où «N.D.»  signifie «non défini».

Rappelons le graphe de la fonction sinus :
\begin{center}
	\input{auto/pictures_tex/Fig_TWHooJjXEtS.pstricks}
\end{center}
celui de la fonction cosinus :
\begin{center}
	\input{auto/pictures_tex/Fig_JJAooWpimYW.pstricks}
\end{center}



\begin{example}     \label{developcosenpisur3}
	Développer la fonction \( \cos\) autour de \( x=\frac{ \pi }{ 3 }\). Utiliser la valeur remarquable du lemme \ref{LEMooRMHAooDEAPMw}. Nous développons autour de \( h=0\) la fonction \( \cos(\frac{ \pi }{ 3 }+h)\) :
	\begin{equation}
		\cos\big( \frac{ \pi }{ 3 }+h \big)\sim \cos\big( \frac{ \pi }{ 3 } \big)+h\cos'(\frac{ \pi }{ 3 })+\frac{ h^2 }{2}\cos''\big( \frac{ \pi }{ 3 } \big)=\frac{ 1 }{2}-\frac{ \sqrt{3} }{2}h-\frac{1}{ 4 }h^2.
	\end{equation}
	Il est aussi possible d'écrire cela en notant \( x=x_0+h\), c'est-à-dire en remplaçant \( h\) par \( x-\frac{ \pi }{ 3 }\) :
	\begin{equation}
		\cos(x)\sim\frac{ 1 }{2}-\frac{ \sqrt{3} }{ 2 }(x-\frac{ \pi }{ 3 })-\frac{1}{ 4 }(x-\frac{ \pi }{ 3 })^2.
	\end{equation}
\end{example}

\begin{normaltext}
	Voici un petit dessin pour donner une idée.
	\begin{center}
		\input{auto/pictures_tex/Fig_WJBooMTAhtl.pstricks}
	\end{center}
	\begin{enumerate}
		\item
		      En noir le graphe de \( \cos(x)\).
		\item
		      En rouge, le développement de \( \cos(x)\) à l'ordre \( 4\) autour de \( x=0\).
		\item
		      En bleu, le développement de \( \cos(x)\) à l'ordre \( 4\) autour de \( x=3\pi/4\).
	\end{enumerate}
\end{normaltext}
