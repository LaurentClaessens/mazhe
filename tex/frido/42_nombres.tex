% This is part of Mes notes de mathématique
% Copyright (c) 2011-2022
%   Laurent Claessens
% See the file fdl-1.3.txt for copying conditions.

%+++++++++++++++++++++++++++++++++++++++++++++++++++++++++++++++++++++++++++++++++++++++++++++++++++++++++++++++++++++++++++
\section{Quelques éléments sur les ensembles}
%+++++++++++++++++++++++++++++++++++++++++++++++++++++++++++++++++++++++++++++++++++++++++++++++++++++++++++++++++++++++++++

%---------------------------------------------------------------------------------------------------------------------------
\subsection{Petit mot d'introduction}
%---------------------------------------------------------------------------------------------------------------------------

Le Frido n'est pas supposé être lu dans l'ordre de la première à la dernière page; les matières y sont présentées dans l'ordre logique mathématique, et non dans l'ordre logique pédagogique, et encore moins par ordre de difficulté croissante.

\begin{normaltext}[On saute la théorie des ensembles]
	En mathématique, si on lit une démonstration et que l'on veut vraiment tout justifier, et justifier toutes les étapes de tous les résultats utilisés, on tombe forcément un jour sur les axiomes.

	Or l'axiomatique est un sujet particulièrement difficile. Nous n'allons donc pas «tout justifier» jusque là. Nous n'allons même pas préciser quel système d'axiome est utilisé. En particulier nous n'allons pas donner l'axiomatique des ensembles : nous allons supposer connus les ensembles et leurs principales propriétés.

	Bref. Nous supposons avoir une théorie des ensembles qui tient la route. En particulier nous supposons connues les notions suivantes :
	\begin{enumerate}
		\item
		      ensemble vide,
		\item
		      ensemble, appartenance, intersection, union,
		\item
		      application entre deux ensembles, notation \( f(x)\) pour désigner l'image de \( x\) par \( f\),
		\item
		      produit cartésien de plusieurs ensembles.
	\end{enumerate}
	Ce sont toutes des choses dont la construction à partir des axiomes n'est en aucun cas évidente. En particulier, des «définitions» comme «l'intersection de deux ensembles est l'ensemble contenant exactement les éléments communs aux deux ensembles» ne sont pas correctes parce qu'elles passent à côté de l'existence et de l'unicité d'un tel ensemble.
\end{normaltext}

\begin{normaltext}[On saute la grammaire]
	Nous n'allons pas non plus formaliser la grammaire des expressions mathématiques\footnote{J'utilise ici des mots que je ne comprends pas, juste pour me donner l'air malin.}. Nous supposons que vous êtes capables de lire des expressions comme
	\begin{equation}
		x\in \eN\Rightarrow\{ a\geq x \}\text{ est infini}.
	\end{equation}
\end{normaltext}

Tout cela pour dire que le Frido ne traitera que de la partie facile de la mathématique.

\begin{definition}\label{DefEnsemblesDisjoints}
	Deux ensembles \( A\) et \( B\) sont \defe{disjoints}{ensembles!disjoints} si leur intersection est vide\footnote{Remarquez que les mots «intersection» et «vide» sont de ceux que nous avons décidé de ne pas définir.}; en d'autres termes, si il n'existe aucun élément commun à \( A\) et \( B\).
\end{definition}

\begin{normaltext}
	Remarquez par exemple que la première phrase de l'article de Wikipédia sur la construction de \( \eN\) est «Partant de la théorie des ensembles, on identifie 0 à l'ensemble vide, puis on construit \ldots». Il est bien précisé que l'on part d'une théorie des ensembles.
\end{normaltext}

\begin{normaltext}
	La suite de ce chapitre sera essentiellement sans exemple parce qu'avant d'avoir construit les ensembles de nombres, je ne sais pas très bien quels exemples on peut donner de quoi que ce soit.
\end{normaltext}


%---------------------------------------------------------------------------------------------------------------------------
\subsection{Injection, surjection, bijection}
%---------------------------------------------------------------------------------------------------------------------------

\begin{definition}
	Soient deux ensembles \( E\) et \( F\). Une application \( f\colon E\to F\) est
	\begin{enumerate}
		\item
		      \defe{surjective}{surjection} si pour tout \( y\in F\), il existe \( x\in E\) tel que \( y=f(x)\);
		\item
		      \defe{injective}{injection} si pour tout \( y\in F\), il existe au plus un \(x\in E \) tel que \( y=f(x)\);
		\item
		      \defe{bijective}{bijection} si elle est à la fois injective et surjective.
	\end{enumerate}
\end{definition}
La méthode la plus courante pour démontrer qu'une application \( f\colon E\to F\) est injective est de considérer \( x,y\in E\) tels que \( f(x)=f(y)\), et de prouver à partir de là que \( x=y\). Ou alors de supposer \( x\neq y\) et d'obtenir une contradiction.

La technique de la contradiction est évidemment la plus courante lorsque l'égalité \( f(x)=g(x)\) implique une équation faisant intervenir \( 1/(x-y)\).

\begin{lemma}       \label{LEMooWBYSooFqyqQP}
	Soient deux ensembles \( A\) et \( B\) ainsi qu'une application \( f\colon A\to B\). Nous supposons qu'il existe une application \( g\colon B\to A\) telle que \( f\circ g=\id_B\) et \( g\circ f=\id_A\).

	Alors \( f\) est une bijection.
\end{lemma}

\begin{proof}
	En deux parties.
	\begin{subproof}
		\item[Injection]
		Supposons que \( f(a)=f(b)\). Alors en appliquant \( g\) des deux côtés, et en utilisant le fait que \( g\circ f=\id_A\), nous trouvons \( a=b\).
		\item[Surjection]
		Soit \( x\in B\). Posons \( a=g(x)\). Alors, en utilisant le fait que \( f\circ g=\id_B\) nous avons
		\begin{equation}
			f(a)=(f\circ g)(x)=x.
		\end{equation}
		Donc \( x\) est dans l'image de \( f\) et \( f\) est surjective.
	\end{subproof}
\end{proof}

%---------------------------------------------------------------------------------------------------------------------------
\subsection{Ensemble ordonné}
%---------------------------------------------------------------------------------------------------------------------------

\begin{normaltext}\label{NORooLMBYooYjUoju}
	L'\defe{axiome du choix}{axiome!du choix} que nous acceptons peut s'énoncer comme ceci\cite{ooKLIXooHbpufL} : Étant donné un ensemble \( X\) d'ensembles non vides, il existe une fonction définie sur \( X\), appelée fonction de choix, qui à chacun d'entre eux associe un de ses éléments.
\end{normaltext}

\begin{definition}[\cite{BIBooMYZQooPqLtKR}]        \label{DEFooRFVTooUUuFuE}
	Une \defe{relation binaire}{relation binaire} d'un ensemble \( E\) vers un ensemble \( F\) est une partie de \( E\times F\).

	Si \( G\) est une relation binaire entre \( E\) et \( F\), nous notons \( x\mR_G y\) et nous disons que \( x\) est en relation avec \( y\).
\end{definition}

\begin{definition}      \label{DefooFLYOooRaGYRk}
	Une \defe{relation d'ordre}{ordre} sur un ensemble \( E\) est une relation binaire\footnote{Définition \ref{DEFooRFVTooUUuFuE}.} (notée \( \leq\)) sur \( E\) telle que pour tous \( x,y,z\in E\),
	\begin{description}
		\item[réflexivité] : \( x\leq x\)
		\item[antisymétrie] : \( x\leq y\) et \( y\leq x\) implique \( x=y\)
		\item[transitivité] : \( x\leq y\) et \( y\leq z\) implique \( x\leq z\).
	\end{description}

	Pour suivre les notations de la définition \ref{DEFooRFVTooUUuFuE}, la partie \( G\) de \( E\times E\) est une relation d'ordre lorsque
	\begin{enumerate}
		\item
		      \( (x,x)\in G\) pour tout \( x\in G\),
		\item
		      Si \( (x,y)\in G\) et \( (y,x)\in G\), alors \( x=y\),
		\item
		      Si \( (x,y)\in G\) et \( (y,z)\in G\), alors \( (x,z)\in G\).
	\end{enumerate}
	Dans la suite nous n'allons plus écrire de relations binaires en détaillant l'ensemble sous-jacent.

	Lorsque nous avons un ensemble \( E\) et une relation d'ordre \( \leq\) sur \( E\), nous disons que le couple \( (E,\leq)\) est un \defe{ensemble ordonné}{ensemble ordonné}.
\end{definition}

\begin{definition}      \label{DEFooVGYQooUhUZGr}
	Un ensemble ordonné est \defe{totalement ordonné}{ordre!total} si deux éléments sont toujours comparables : si \( x,y\in E\) alors nous avons soit \( x\leq y\) soit \( y\leq x\). Si les éléments ne sont pas tous comparables, nous disons que l'ordre est \defe{partiel}{ordre!partiel}.
\end{definition}

\begin{definition}[\cite{BIBooROFNooTKJKxG}]        \label{DEFooBZNRooYRPGme}
	Soit un ensemble ordonné \( (E,\leq)\). Un élément \(M\in E\) est un \defe{élément maximal}{élément maximal} de \( E\) si pour tout \( x\in E\) tel que \( M\leq x\), nous avons \( M\leq x\Rightarrow x=M\).

	Nous disons que \( m\in E\) est un \defe{élément minimal}{élément minimal} si pour tout \( x\in E\), nous avons \( x\leq m\Rightarrow x=m\).
\end{definition}

\begin{definition}      \label{DEFooDNWRooTiMAzK}
	Soit un ensemble ordonné \( (E,\leq)\) et une partie \( A\) de \( E\). Nous disons que \( m\in A\) est un \defe{minimum}{minimum!ensemble ordonné} de \( A\) si pour tout \( x\in A\), l'élément \( m\) est comparable à \( x\) et \( m\leq x\).

	Un élément \( p\in E\) est un \defe{minorant}{minorant} de \( A\) si pour tout \( a\in A\), les éléments \( p\) et \( a\) sont comparables et \( p\leq a\).

	Les notions de \defe{maximum}{maximum} et de \defe{majorant}{majorant} sont définies de façon analogue.
\end{definition}

\begin{normaltext}    \label{NORMooVHIBooJAOsou}
	Notons qu'il n'est pas demandé à un élément maximal\footnote{Définition \ref{DEFooBZNRooYRPGme}} d'être comparable à tous les autres éléments. Si \( (E,\leq)\) n'est pas totalement ordonné, un élément maximal peut ne majorer qu'une partie de \( E\).

	Un élément maximal est plus grand que tous les éléments avec lesquels il est comparable.

	Dans un ensemble totalement ordonné, les notions d'élément maximal de \( E\) et de maximum\footnote{Définition \ref{DEFooDNWRooTiMAzK}.} de \( E\) coïncident ; dans le cas d'un ordre partiel, ces notions sont distinctes.

	Il se peut que nous parlions d'un «maximum» ou «élément maximum»  au lieu d'un «élément maximal», en particulier en utilisant le lemme de Zorn. Si vous voyez de telles choses, n'hésitez pas à me le dire.

	Par exemple, quand on applique le lemme de Zorn pour démontrer l'existence d'une base dans un espace vectoriel de dimension infinie, on obtient une famille libre qui est maximale, c'est-à-dire qui est un élément maximal dans l'ensemble, ordonné par inclusion, des familles libres de l'espace vectoriel. C'est un élément maximal, et non un maximum, car l'ensemble des familles libres d'un espace vectoriel non réduit à \( \{ 0 \}\) n'admet pas de maximum (pour l'inclusion). Voir \ref{THOooOQLQooHqEeDK} et \ref{NORMooREVQooEFJWta}.
\end{normaltext}

Lorsqu'une partie possède un minimum, ce dernier est nommé le «plus petit élément» de la partie. Attention : il n'en existe pas toujours. D'innombrables exemples pourront être vus lorsque nous aurons construit \( \eQ\) et \( \eR\). Typiquement les intervalles du type \( \mathopen] a , b \mathclose[\).

\begin{example}[\cite{MonCerveau}]
	Soit un ensemble \( E\) ainsi que \( a\neq b\) dans \( E\). Nous considérons les parties
	\begin{subequations}
		\begin{align}
			A=\{ P\subset E\tq a\in P,b\notin P \} \\
			B=\{ P\subset E\tq b\in P,a\notin P \}.
		\end{align}
	\end{subequations}
	Et enfin nous considérons l'ensemble \( F=A\cup B\), c'est à dire l'ensemble des parties de \( E\) qui contiennent soit \( a\) soit \( b\) mais pas les deux. Nous ordonnons partiellement \( F\) par l'inclusion.

	Dans \( (F,\subset)\), l'élément \( E\setminus\{ b \}\) est un élément maximal, mais pas un majorant.
\end{example}

\begin{definition}   \label{DEFooLJEAooBLGsiS}
	Un ensemble ordonné est \defe{bien ordonné}{bon!ordre}\index{ordre!bon ordre} si toute partie non vide possède un plus petit élément.
\end{definition}

Autrement dit, l'ensemble ordonné \( E\) est bien ordonné si pour toute partie non vide \( A\), il existe \( x\in A\) tel que \( x\leq y\) pour tout \( y\in A\).

\begin{normaltext}
	Quelques remarques.
	\begin{enumerate}
		\item
		      L'inégalité stricte (définie par: \( x<y\) si et seulement si \( x\leq y\) et \( x\neq y\)) n'est pas une relation d'ordre parce qu'elle n'est pas réflexive.
		\item
		      Nous verrons dans la remarque~\ref{REMooXOIOooHjwMvA} que l'intervalle \( \mathopen[ -1 , 1 \mathclose]\) dans \( \eR\) n'est pas bien ordonné.
		\item
		      Un ensemble bien ordonné est forcément totalement ordonné parce que toutes les parties de la forme \( \{ x,y \}\) possèdent un minimum. Par conséquent \( x\) et \( y\) doivent être comparables : \( x\leq y\) ou \( y\leq x\).
	\end{enumerate}
\end{normaltext}

\begin{example}
	Si \( E\) est un ensemble, l'inclusion est un ordre sur l'ensemble des parties de \( E\), mais pas un ordre total parce que si \( X,Y\) sont des parties de \( E\), alors nous n'avons pas automatiquement soit \( X\subset Y\) soit \( Y\subset X\).
\end{example}

La notion d'ordre permet d'introduire la notion d'intervalle.

\begin{definition}  \label{DefEYAooMYYTz}
	Soit un ensemble totalement ordonné \( (E,\leq)\). Un \defe{intervalle}{intervalle} de \( E\) est une partie \( I\) telle que tout élément compris entre deux éléments de \( I \) soit dans \( I \). En language mathématique, la partie \( I \) de \( E\) est un intervalle si
	\[
		\forall a,b\in I,(a\leq x\leq b)\Rightarrow x\in I.
	\]
\end{definition}

%---------------------------------------------------------------------------------------------------------------------------
\subsection{Lemme de Zorn}
%---------------------------------------------------------------------------------------------------------------------------

Nous admettons l'axiome du choix\cite{BIBooBZJMooJDJRgg} qui s'énonce de la façon suivante\cite{BIBooTQMWooFFAQQZ} :
\begin{quote}
	Pour tout ensemble \( X\) d'ensembles non vides, il existe une fonction définie sur \( X\), appelée fonction de choix, qui à chaque ensemble \( A\) appartenant à \( X\) associe un élément de cet ensemble \( A\).
\end{quote}

\begin{definition}[Ensemble inductif\cite{MathAgreg}]  \label{DefGHDfyyz}
	Un ensemble est \defe{inductif}{inductif} si tout sous-ensemble totalement ordonné admet un majorant.
\end{definition}


\begin{lemma}[Lemme de Zorn\cite{BIBooYDIJooWCVynX}]    \label{LemUEGjJBc}
	Tout ensemble ordonné inductif non vide admet au moins un élément maximal.
\end{lemma}
\index{lemme!de Zorn}

\begin{proposition}[\cite{BIBooZFPUooIiywbk}]       \label{PROPooFOETooWYLOeq}
	Soient un ensemble inductif \( (E,\leq)\) et \( b\in E\). Il existe un élément maximal\footnote{Définition \ref{DEFooBZNRooYRPGme}.} \( m\in E\) tel que \( b\leq m\).
\end{proposition}

\begin{proof}
	En plusieurs parties.
	\begin{subproof}
		\item[Un ensemble]
		Nous considérons
		\begin{equation}
			E_b=\{ x\in E\tq b\leq x \}.
		\end{equation}
		\item[\( E_b\) est inductif]
		Soit une partie non vide totalement ordonnée \( A\) de \( E_b\). Puisque \( E\) est inductif, la partie \( A\) admet un majorant \( m\in E\).

		Comme \( A\) est non vide, nous pouvons considérer \( x\in A\). Nous avons \( b\leq x\) parce que \( x\in A\subset E_b\). Mais comme \( m\) est un majorant de \( A\), \( x\leq m\). Bref, nous avons les inégalités
		\begin{equation}
			b\leq x\leq m.
		\end{equation}
		Donc \( m\in E_b\). Nous avons prouvé que \( m\) est un majorant de \( A\) contenu dans \( E_b\). Donc \( E_b\) est inductif.
		\item[Zorn]
		Puisque \( (E_b,\leq)\) est inductif, le lemme de Zorn \ref{LemUEGjJBc} nous indique que \( E_b\) a un élément maximal. Nous le notons \( m\).
		\item[\( m\) est maximal dans \( E\)]
		Supposons avoir un élément \( a\in E\) tel que \( m\leq a\). Nous avons
		\begin{equation}
			b\leq m\leq a,
		\end{equation}
		et donc \( a\in E_b\). Mais \( m\) est maximal dans \( E_b\), donc \( a=m\).
	\end{subproof}
\end{proof}


%---------------------------------------------------------------------------------------------------------------------------
\subsection{Complémentaire}
%---------------------------------------------------------------------------------------------------------------------------
\label{AppComplement}

\begin{definition}
	Soit \( E\), un ensemble et \( A\), une partie de \( E\) (c'est-à-dire un sous-ensemble de \( E\)). Le \defe{complémentaire}{complémentaire} de l'ensemble \(A \), dans \( E\), noté \( \complement A\)\nomenclature[T]{$\complement A$}{Le complémentaire de l'ensemble \( A\)} est l'ensemble des éléments de \( E\) qui ne font pas partie de \( A\) :
	\begin{equation}
		\complement A=E\setminus A=\{ x\in E\tq x\notin A \}.
	\end{equation}
\end{definition}

Nous allons aussi régulièrement noter le complémentaire de \( A\) par \( A^c\)\nomenclature[T]{\( A^c\)}{complémentaire de \( A\)}.

\begin{lemma}[Quelques relations ensemblistes]       \label{LEMooHRKAooRskzQL}
	Soient \( A,B,C\subset X\). Nous avons
	\begin{enumerate}
		\item
		      \( X\setminus (A\cap B)=(X\setminus A)\cup(X\setminus B)\).
		\item       \label{ITEMooQCGUooKnWfBo}
		      \( X\setminus (A\cup B)=(X\setminus A)\cap(X\setminus B)\).
		\item       \label{ITEMooXWKCooUASxlh}
		      \( A\cap(B\setminus C)=(A\cap B)\setminus C\).
	\end{enumerate}
\end{lemma}

\begin{lemma}       \label{LemPropsComplement}
	Quelques propriétés à propos des complémentaires. Si \( E\) est un ensemble et si \( A\) et \( B\) sont des sous-ensembles de \( E\), nous avons
	\begin{enumerate}
		\item
		      \( \complement \complement A =A\), en d'autres termes, \( E\setminus(E\setminus A)=A\),
		\item       \label{ItemLemPropComplementiii}
		      \( A\setminus B=A\cap\complement B\).
		\item       \label{ITEMooNHDUooWtURqQ}
		      \( (A\setminus B)^c=A^c\cup B\).
		\item       \label{ITEMooTBWKooTChOmC}
		      \( A^c\setminus B^c=B\setminus A\).
	\end{enumerate}
\end{lemma}

\begin{proof}
	Plusieurs points.
	\begin{subproof}
		\item[Pour \ref{ItemLemPropComplementiii}]
		\item[Pour \ref{ITEMooNHDUooWtURqQ}]
		Il faut le faire en deux inclusions.
		\begin{subproof}
			\item[\( (A\setminus B)^c\subset A^c\cup B\)]
			Supposons que \( x\in(A\setminus B)^c\). Si \( x\in A^c\), c'est bon. Supposons que \( x\) n'est pas dans \( A^c\), et montrons que \( x\in B\). Le fait que \( x\) ne soit pas dans \( A^c\) signifie que \( x\in A\). Si \( x\) n'était pas dans \( B\), alors \( x\) serait dans \( A\setminus B\), ce qui est contraire à l'hypothèse. Donc \( x\in B\).
			\item[\( A^c\cup B\subset (A\setminus B)^c\)]
			Supposons d'abord que \( x\in A^c\). Comme \( A\setminus B\subset A\), si \( x\in A^c\), alors \( x\in (A\setminus B)^c\).

			Si \( x\in B\), alors \( x\) n'est pas dans \( A\setminus B\) et donc \( x\) est dans \( (A\setminus B)^c\).
		\end{subproof}
		\item[Pour \ref{ITEMooTBWKooTChOmC}]
		Pour cette égalité, nous séparons \( 4\) cas suivant que \( x\) est dans \( A\) ou \( B\) ou non. Bref, nous écrivons la table de vérité :
		\begin{equation}
			\begin{array}{|c|c|c|c|c|}
				\hline%
				A                & 1 & 1 & 0 & 0 \\
				\hline%
				B                & 1 & 0 & 1 & 0 \\
				\hline%
				A^c\setminus B^c & 0 & 0 & 1 & 0 \\
				\hline%
				B\setminus A     & 0 & 0 & 1 & 0 \\
				\hline%
			\end{array}
		\end{equation}
		Les deux dernières lignes étant égales, nous avons l'égalité d'ensembles annoncée.
	\end{subproof}
\end{proof}

\begin{definition}[différence symétrique]    \label{DefBMLooVjlSG}
	Si \( A\) et \( B\) sont des ensembles, l'ensemble \( A\Delta B\)\nomenclature[T]{\( A\Delta B\)}{différence symétrique} est la \defe{différence symétrique}{ensemble!différence symétrique} d'ensembles :
	\begin{equation}
		A\Delta B=(A\cup B)\setminus(A\cap B).
	\end{equation}
\end{definition}
C'est l'ensemble des éléments étant soit dans \( A\) soit dans \( B\) mais pas dans les deux, ni dans aucun des deux. La table de vérité de \( A\Delta B\) est intéressante :
\begin{equation}        \label{EQooOJBOooKkKbYp}
	\begin{array}{|c|c|c|c|c|}
		\hline%
		A         & 1 & 1 & 0 & 0 \\
		\hline%
		B         & 1 & 0 & 1 & 0 \\
		\hline%
		A\Delta B & 0 & 1 & 1 & 0 \\
		\hline%
	\end{array}
\end{equation}
La deuxième colonne signifie que si \( x\in A\) et \( x\in B^c\), alors \( x\in A\Delta B\).

\begin{lemma}[\cite{BIBooRFFSooROjnXs}]   \label{LemCUVoohKpWB}
	Si \( A\) et \( B\) sont des parties d'un ensemble, nous avons
	\begin{enumerate}
		\item       \label{ItemVUCooHAztC}
		      \( A^c\Delta B^c=A\Delta B\).
		\item\label{ItemVUCooHAztCii}
		      \( (A\Delta B)\Delta B=A\).
		\item       \label{ITEMooSPZXooPTgisP}
		      \( (A\Delta B)^c=(A^c\cap B^c)\cup(A\cap B)\).
		\item       \label{ITEMooSMXWooYcWsRC}
		      Associativité : \( A\Delta (B\Delta C)=(A\Delta B)\Delta C\).
	\end{enumerate}
\end{lemma}

\begin{proof}
	En plusieurs points.
	\begin{subproof}
		\item[Pour \ref{ItemVUCooHAztC}]
		Nous rappelons l'égalité \( X^c\setminus Y^c=Y\setminus X\) du lemme \ref{LemPropsComplement}\ref{ITEMooTBWKooTChOmC}. De là nous écrivons
		\begin{equation}
			A^c\Delta B^c=(A^c\cup B^c)\setminus(A^c\cap B^c)=(A^c\cap B^c)^c\setminus(A^c\cup B^c)^c=(A\cup B)\setminus (A\cap B)=A\Delta B.
		\end{equation}
		\item[Pour \ref{ItemVUCooHAztCii}]
		Ici, il faut remarquer que \( (A\Delta B)\cup B=A\cup B\) et que \( (A\Delta B)\cap B=B\setminus A\), donc
		\begin{equation}
			(A\Delta B)\Delta B=(A\cup B)\setminus (B\setminus A)=A.
		\end{equation}

		\item[Pour \ref{ITEMooSPZXooPTgisP}]
		Il s'agit d'utiliser le lemme \ref{LemPropsComplement}\ref{ITEMooNHDUooWtURqQ} :
		\begin{subequations}
			\begin{align}
				(A\Delta B)^c & =\Big( (A\cup B)\setminus (A\cap B) \Big)^c \\
				              & =(A\cup B)^c  \cup(A\cap B)                 \\
				              & =(A^c\cap B^c)\cup(A\cap B).
			\end{align}
		\end{subequations}
		\item[Pour l'associativité \ref{ITEMooSMXWooYcWsRC}]
		Nous écrivons les tables de vérités selon que \( x\) est dans \( A\), \( B\), \( C\) ou non. D'abord
		\begin{equation}
			\begin{array}{|c|c|c|c|c|c|c|c|c|}
				A                   & 1 & 1 & 1 & 1 & 0 & 0 & 0 & 0 \\
				B                   & 1 & 1 & 0 & 0 & 1 & 1 & 0 & 0 \\
				C                   & 1 & 0 & 1 & 0 & 1 & 0 & 1 & 0 \\
				\hline%
				B\Delta C           & 0 & 1 & 1 & 0 & 0 & 1 & 1 & 0 \\
				\hline%
				A\Delta (B\Delta C) & 1 & 0 & 0 & 1 & 0 & 1 & 1 & 0
			\end{array}
		\end{equation}
		La quatrième ligne s'écrit sur le modèle de \eqref{EQooOJBOooKkKbYp} en regardant les deuxièmes et troisièmes lignes. La dernière ligne se fait avec la première et la quatrième.

		L'autre table de vérité se fait de la même manière :
		\begin{equation}
			\begin{array}{|c|c|c|c|c|c|c|c|c|}
				A                   & 1 & 1 & 1 & 1 & 0 & 0 & 0 & 0 \\
				B                   & 1 & 1 & 0 & 0 & 1 & 1 & 0 & 0 \\
				C                   & 1 & 0 & 1 & 0 & 1 & 0 & 1 & 0 \\
				\hline%
				A\Delta B           & 0 & 0 & 1 & 1 & 1 & 1 & 0 & 0 \\
				\hline%
				(A\Delta B)\Delta C & 1 & 0 & 0 & 1 & 0 & 1 & 1 & 0
			\end{array}
		\end{equation}
		Puisque les lignes pour \( A\Delta (B\Delta C)\) et pour \( (A\Delta B)\Delta C\) sont identiques, nous avons égalité.
	\end{subproof}
\end{proof}

%---------------------------------------------------------------------------------------------------------------------------
\subsection{Relations d'équivalence}
%---------------------------------------------------------------------------------------------------------------------------
\label{appEquivalence}

\begin{definition}  \label{DefHoJzMp}
	Si \( E\) est un ensemble, une \defe{relation d'équivalence}{relation d'équivalence} sur \( E\) est une relation binaire\footnote{Définition \ref{DEFooRFVTooUUuFuE}.} \( \sim\) qui est à la fois
	\begin{description}
		\item[réflexive]  \( x\sim x\) pour tout \( x\in E\),
		\item[symétrique] \( x\sim y\) si et seulement si \( y\sim x\);
		\item[transitive] si \( x\sim y\) et \( y\sim z\), alors \( x\sim z\).
	\end{description}
\end{definition}

\begin{definition}      \label{DEFooRHPSooHKBZXl}
	Si \( E\) est un ensemble et si \( \sim\) est une relation d'équivalence sur \( E\), alors nous notons \( E/\sim\) l'\defe{ensemble quotient}{ensemble quotient}, c'est-à-dire l'ensemble des classes d'équivalence dans \( E\). Un élément de \( E/\sim\) est de la forme
	\begin{equation}
		[a]=\{ x\in E\tq x\sim a \}.
	\end{equation}
\end{definition}

\begin{lemma}
	Soit un ensemble \( E\) et une relation d'équivalence \( \sim\). Pour \( a,b\in E\), nous avons \( [a]=[b]\) si et seulement si \( a\sim b\).
\end{lemma}

\begin{proof}
	En deux parties.
	\begin{subproof}
		\item[\( \Rightarrow\)]
		Nous supposons que \( [a]=[b]\). Par réflexivité, \( a\sim a\) et nous avons \( a\in [a]=[b]\). Mais \( a\in [b]\) signifie \( a\sim b\), ce qu'il fallait.
		\item[\( \Leftarrow\)]
		Nous supposons que \( a\sim b\), et nous démontrons que \( [a]\subset [b]\) (pour l'inclusion inverse, vous devriez vous en sortir tout seul). Si \( x\in [a]\), alors \( x\sim a\). Mais \( a\sim b\). Donc \( x\sim a\sim b\), ce qui implique \( x\sim b\) par transitivité. Or dire \( x\sim b\) implique \( x\in [b]\).
	\end{subproof}
\end{proof}

\begin{example}
	Sur l'ensemble de tous les polygones du plan, la relation «a le même nombre de côtés» est une relation d'équivalence. Plus précisément, si \( P\) et \( Q\) sont deux polygones, nous disons que \( P\sim Q\) si et seulement si \( P\) et \( Q\) ont le même nombre de côtés. C'est une relation d'équivalence :
	\begin{itemize}
		\item
		      un polygone \( P\) a toujours le même nombre de côtés que lui-même : \( P\sim P\);
		\item
		      si \( P\) a le même nombre de côtés que \( Q\) (\( P\sim Q\)), alors \( Q\) a le même nombre de côtés que \( P\) (\( Q\sim P\));
		\item
		      si \( P\) a le même nombre de côtés que \( Q\) (\( P\sim Q\)) et que \( Q\) a le même nombre de côtés que \( R\) (\( Q\sim R\)), alors \( P\) a le même nombre de côtés que \( R\) (\( P\sim R\)).
	\end{itemize}
\end{example}

\begin{example}
	Soit \( f\) une application entre deux ensembles \( E\) et \( F\). Nous définissons une relation d'équivalence sur \( E\) par
	\begin{equation}
		x\sim y\Leftrightarrow f(x)=f(y).
	\end{equation}
	Nous notons par \( \pi\colon E\to E/\sim\) la projection canonique. L'application
	\begin{equation}
		\begin{aligned}
			g\colon E/\sim & \to F        \\
			[x]            & \mapsto f(x)
		\end{aligned}
	\end{equation}
	est bien définie et injective. Elle n'est pas surjective tant que \( f\) ne l'est pas. La \defe{décomposition canonique}{canonique!décomposition}\index{décomposition!canonique} de \( f\) est
	\begin{equation}
		f=g\circ\pi.
	\end{equation}
\end{example}

%+++++++++++++++++++++++++++++++++++++++++++++++++++++++++++++++++++++++++++++++++++++++++++++++++++++++++++++++++++++++++++
\section{Quelques structures algébriques}
%+++++++++++++++++++++++++++++++++++++++++++++++++++++++++++++++++++++++++++++++++++++++++++++++++++++++++++++++++++++++++++

Nous collectons ici les définitions des principales structures algébriques.

\begin{definition}[Groupe]      \label{DEFooBMUZooLAfbeM}
	Un \defe{groupe}{groupe} est un ensemble \( G\) muni d'une opération interne \( \cdot\colon G\times G\to G\) telle que
	\begin{enumerate}
		\item
		      pour tous \( g\), \( h\), \( k\in G\), \( g\cdot(h\cdot k)=(g\cdot h)\cdot k\),
		\item
		      il existe un élément \( e\in G\) tel que \( e\cdot g=g\cdot e=g\) pour tout \( g\in G\),
		\item
		      pour tout \( g\in G\), il existe un élément \( h\in  G\) tel que \(g\cdot h=h\cdot g=e \).
	\end{enumerate}
	Un groupe est \defe{commutatif}{groupe commutatif} ou \defe{abélien}{groupe abélien} si \( g\cdot h=h\cdot g\) pour tout \( g,h\in G\).
\end{definition}

Notons que nous avons écrit \( g\cdot h\) et non \( \cdot(g,h)\) comme une notation purement fonctionnelle nous l'aurait suggéré. Dans les exemples concrets, selon les cas, la loi de groupe appliquée à \( g\) et \( h\) sera notée tantôt \( g+h\), tantôt \( g\cdot h\) ou, le plus souvent pour un groupe générique, simplement \( gh\).

\begin{definition}[morphisme, automorphisme]        \label{DEFooBEHTooMeCOTX}
	Soient deux groupes \( G\) et \( H\). Un \defe{morphisme}{morphisme de groupes} entre \( G\) et \( H\) est une application \( \alpha\colon G\to H\) telle que pour tout \( g,h\in G\) nous ayons \( \alpha(gh)=\alpha(g)\alpha(h)\).

	Comme d'habitude, un isomorphisme est un morphisme bijectif. Un \defe{automorphisme}{automorphisme de groupes} de \( G\) est un isomorphisme de \( G\) vers \( G\) lui-même.
\end{definition}

\begin{definition}[Anneau\cite{Tauvel}]     \label{DefHXJUooKoovob}
	Un \defe{anneau}{anneau}\footnote{Nous faisons le choix qu'un anneau admet toujours un neutre pour la multiplication. Certains ouvrages parlent dans ce cas d'anneau unitaire.} est un triplet \( (A,+,\cdot)\) avec les conditions
	\begin{enumerate}
		\item
		      \( (A,+)\) est un groupe\footnote{Groupe, définition \ref{DEFooBMUZooLAfbeM}.} commutatif. Nous notons \( 0\) le neutre.
		\item
		      La multiplication est associative et nous notons \( 1\) le neutre.
		\item       \label{ITEMooGMNOooSTGiXw}
		      La multiplication est distributive par rapport à l'addition.
	\end{enumerate}
	L'anneau \( (A,+,\cdot)\) est \defe{commutatif}{anneau commutatif} si pour tout \( a,b\in A\) nous avons \( a\cdot b=b\cdot a\).
\end{definition}


\begin{definition}[Morphisme d'anneaux\cite{ooZRUJooXyxPqQ}]      \label{DEFooSPHPooCwjzuz}
	Si \( (A,+,\cdot)\) et \( (B,+,\cdot)\) sont des anneaux, un \defe{morphisme d'anneaux}{morphisme!d'anneaux} est une application \( f\colon A\to B\) telle que
	\begin{enumerate}
		\item \( f(a+b)=f(a)+f(b)\)
		\item \( f(a\cdot b)=f(a)\cdot f(b)\)
		\item \( f(1)=1\).
	\end{enumerate}
	Étant bien entendu que les significations de \( 1\), \( +\) et \( \cdot\) sont différentes à gauche et à droite.
\end{definition}

%+++++++++++++++++++++++++++++++++++++++++++++++++++++++++++++++++++++++++++++++++++++++++++++++++++++++++++++++++++++++++++
\section{Les naturels}
%+++++++++++++++++++++++++++++++++++++++++++++++++++++++++++++++++++++++++++++++++++++++++++++++++++++++++++++++++++++++++++
\label{SECooPJSYooNYaIaq}

\begin{definition}[\cite{RWWJooJdjxEK}]     \label{DEFooBJBOooWlblAx}
	Un \defe{triplet naturel}{triplet naturel} est un triplet \( (\mN, o, s)\) où \( \mN\) est un ensemble, \( o\) est un élément de \( \mN\) et \( s\) est une application \( s\colon \mN\to \mN\) satisfaisant les propriétés suivantes :
	\begin{enumerate}
		\item
		      \( s\) est injective,
		\item       \label{ITEMooQAKJooGKdJsM}
		      \( s(\mN)=\mN\setminus \{ o \} \)
		\item       \label{ITEMooXPYEooFajywh}
		      Si \( A\subset \mN\) est tel que \( o\in A\) et \( s(A)\subset A\), alors \( A=\mN\).
	\end{enumerate}
\end{definition}

Le théorème suivant est typiquement de ceux qui vont demander de gratter la théorie axiomatique des ensembles avec une certaine précision\quext{Ou alors il y a quelque chose qui m'échappe. Écrivez-moi si vous connaissez une construction «simple».}.
\begin{theorem}     \label{THOooOXMHooXYgMqb}
	Il existe un\footnote{Nous verrons plus tard que toute partie infinie d'un triplet naturel fournit un nouveau triplet naturel; il en existe donc plusieurs.} triplet naturel.
\end{theorem}

\begin{normaltext}[Définition de \( \eN\)]
	Pour la suite, nous considérons un triplet naturel \( (\mN,o,s)\) et nous notons \( \eN=\mN\). Donc la nature de tous les objets que nous allons considérer à partir de maintenant dépend du choix de triplet naturel que nous faisons à présent. Le théorème \ref{THOooFUXMooJuigHK} nous assurera que peu de choses devraient réellement dépendre de ce choix.

	Nous notons \( 0\) l'élément \( o\) et \( 1\) l'élément \( s(0)\). C'est tout ce dont nous avons besoin dans l'immédiat.
\end{normaltext}

%---------------------------------------------------------------------------------------------------------------------------
\subsection{Applications définies par récurrence}
%---------------------------------------------------------------------------------------------------------------------------

\begin{proposition}[Récurrence\cite{RWWJooJdjxEK}]      \label{PROPooXTRCooKwrWkq}
	Soit un triplet naturel \( (\mN,o,s)\) et une application \( P\colon \mN\to \{ 0,1 \}\) vérifiant\footnote{Les plus \randomGender{pointilleux}{pointilleuses} diront que \( 1\) n'est pas encore défini. Bon j'avoue. Ce qui est important est que \( P\) prenne ses valeurs dans un ensemble contenant deux éléments distincts. Si maintenant vous râlez parce que «deux» est encore moins défini, prenez un ensemble quelconque \( A\) et dites que \( P\) prend ses valeurs dans \( \{ A, \partP(A)\}\). Mais êtes-vous bien \randomGender{certain}{certaine} que \( \partP(A)\neq A\) ?}
	\begin{enumerate}
		\item
		      \( P(o)=1\),
		\item
		      pour tout \( a\in \mN\), si \( P(a)=1\), alors \( P\big( s(a) \big)=1\).
	\end{enumerate}
	Alors \( P(x)=1\) pour tout \( x\in \mN\).
\end{proposition}

\begin{proof}
	Nous posons
	\begin{equation}
		A=\{ x\in\mN\tq P(x)=1 \}.
	\end{equation}
	Cet ensemble vérifie la propriété \ref{DEFooBJBOooWlblAx}\ref{ITEMooXPYEooFajywh}. Donc \( A=\mN\).
\end{proof}

\begin{theorem}[\cite{BIBooZFPUooIiywbk,BIBooMSSFooOOeRKE}]       \label{THOooEJPYooZFVnez}
	Soient \( E\) un ensemble, \( g\) une application de \( E\) dans \( E\) et \( b\) un élément de \( E\).  Alors il existe une unique application \( f\colon \eN\to E\) telle que :
	\begin{enumerate}
		\item
		      \( f(0)=b\)
		\item
		      \( f\big( s(n) \big)=g\big( f(n) \big)\) pour tout \( n\in \eN\setminus\{ 0 \}\).
	\end{enumerate}
\end{theorem}

\begin{proof}
	Nous commençons par l'unicité. Soient \( f_1\) et \( f_2\) deux telles applications. Nous posons
	\begin{equation}
		A=\{ n\in \eN\tq f_1(n)=f_2(n) \}.
	\end{equation}
	Nous avons \( 0\in A\) parce que \( f_1(0)=f_2(0)=b\).

	Supposons que \( f_1(k)=f_2(k)\). Alors nous avons
	\begin{equation}
		f_1\big( s(k) \big)=g\big( f_1(k) \big)=g\big( f_2(k) \big)=f_2\big( s(k) \big).
	\end{equation}
	Nous en déduisons que \( s(k)\in A\). Autrement dit \( s(A)\subset A\). La définition \ref{DEFooBJBOooWlblAx}\ref{ITEMooXPYEooFajywh} nous indique alors que \( A=\eN\), c'est-à-dire que \( f_1=f_2\).

	Nous montrons à présent l'existence en plusieurs étapes.
	\begin{subproof}
		\item[L'ensemble est assez grand]
		Nous considérons l'ensemble \( \mA\) des parties \( A\subset \eN\times E\) telles que
		\begin{enumerate}
			\item
			      \( (0,b)\in A\)
			\item
			      \( (n,x)\in A \Rightarrow \big( s(n),g(x) \big)\in A\).
		\end{enumerate}
		L'ensemble \( \mA\) est non vide parce que \( \eN\times E\in \mA\).
		\item[Le plus petit]
		Nous posons
		\begin{equation}
			G=\bigcap_{A\in \mA}A,
		\end{equation}
		et nous prouvons que \( G\in\mA\). D'abord \( (0,b)\in G\) parce que cet élément est dans chacun des \( A\in G\). Ensuite si \( (n,x)\in G\), alors pour tout \( A\in\mA \) nous avons \( (n,x)\in A\) et donc \( \big( s(n),g(x) \big)\in A\). Par conséquent \( \big( s(n),g(x) \big)\in \bigcup_{A\in\mA}A=G\).

		Pour \( n\in \eN\) nous posons
		\begin{equation}
			G_n=\{ x\in E\tq (n,x)\in G \}.
		\end{equation}
		Nous avons en particulier que \( b\in G_0\) parce que \( (0,b)\in G\).
		\item[\( G\) contient un \( (n,x)\) pour tout \( n\)]
		Nous prouvons que pour tout \( n\in \eN\), il existe \( x\in E\) tel que \( (n,x)\in G\). Nous faisons ça avec la proposition \ref{PROPooXTRCooKwrWkq} en posant
		\begin{equation}
			\begin{aligned}
				P\colon \eN & \to \{ 0,1 \}                      \\
				n           & \mapsto \begin{cases}
					1 & \text{si } G_n\neq\emptyset \\
					0 & \text{sinon.}
				\end{cases}
			\end{aligned}
		\end{equation}
		Puisque \( (0,b)\in G\) nous avons \( P(0)=1\). Supposons que \( P(k)=1\) et montrons que \( P\big( s(k) \big)=1\). Comme \( P(k)=1\), il existe \( x\in E\) tel que \( (k,x)\in G\). De ce fait, \( \big( s(k),g(x) \big)\in G\), ce qui donne \( G_{s(k)}\neq \emptyset\) et \( P\big( s(k) \big)=1\).
		\item[\( G_n\) est un singleton]
		Nous avons vu que \( G_n\) n'est jamais vide. Nous allons montrer que \( G_n\) est un singleton pour tout \( n\). Pour cela nous posons
		\begin{equation}
			\begin{aligned}
				P\colon \eN & \to \{ 0,1 \}                      \\
				n           & \mapsto \begin{cases}
					1 & \text{si }  G_n \text{ est un singleton} \\
					0 & \text{sinon. }
				\end{cases}
			\end{aligned}
		\end{equation}
		Nous prouvons par récurrence que \( P(n)=1\) pour tout \( n\).
		\begin{subproof}
			\item[\( P(0)=1\)]
			Nous commençons par prouver que \( P(0)=1\). Nous savons que \( (0,b)\in G_0\). Supposons \( a\neq b\) tel que \( (0,a)\in G_0\). Alors en posant \( G'=G\setminus\{ (0,a) \}\) nous avons \( G'\in \mA\).

			En effet \( (0,b)\in G'\) parce que \( (0,b)\in G\) et \( (0,b)\neq (0,a)\). De plus si \( (n,x)\in G'\), alors \( \big( s(n),g(x) \big)\in G\). Mais comme \( s(n)\neq 0\) nous avons \( \big( s(n),g(x) \big)\neq (0,a)\) et donc \( \big( s(n),g(x) \big)\in G'\).

			L'ensemble \( G'\) serait un élément de \( \mA\) strictement inclus dans \( G\). Impossible. Donc \( G_0\) est un singleton.

			\item[Récurrence]
			Supposons que \( P(k)=1\), c'est-à-dire que \( G_k\) est un singleton. Soit \( e\) l'unique élément de \( G_k\) : \( (k,e)\in G\). Nous avons alors aussi que \( \big( s(k),g(e) \big)\in G\). Nous devons prouver que si \( y\in G_{s(k)}\), alors \( y=g(e)\).

			Supposons donc \( y\neq g(e)\) soit dans \( G_{s(k)}\). Nous posons
			\begin{equation}
				G'=G\setminus\{ \big( s(k),y \big) \}.
			\end{equation}
			Nous prouvons que \( G'\in\mA\). D'abord \( (0,b)\in G'\) parce que \( s(k)\neq 0\). Soit ensuite \( (m,z)\in G'\). Si \( m=k\), alors \( z=e\) (parce que par hypothèse \( G_k\) est un singleton) et nous savons que \( \big( s(m),g(e) \big)\in G'\). Si par contre \( m\neq k\), comme  \( s\) est injective, nous avons aussi \( s(m)\neq s(k)\). Donc \( \big( s(m),g(z) \big)\neq \big( s(k),y \big)\) et \( \big( s(m),g(z) \big)\in G'\). Donc \( G'\in \mA\) et est strictement plus petit que \( G\). Contradiction.

			Nous concluons que \( G_{s(k)}\) est un singleton, c'est-à-dire que \( P\big( s(k) \big)=1\).
			\item[Conclusion]
			Nous avons prouvé que \( G_n\) est un singleton pour tout \( n\).
		\end{subproof}

		\item[Et enfin]

		Nous définissons \( f(n)\) comme étant l'unique élément de \( G_n\). Puisque \( (0,b)\in G\) nous avons \( G_0=\{ b \}\) et donc \( g(0)=b\).

		Par définition de \( f\), nous avons \( \big( n,f(n) \big)\in G\). Parce que \( G\in\mA\) nous avons alors
		\begin{equation}
			\big( s(n),g\big( f(n) \big) \big)\in G.
		\end{equation}
		Autrement dit, \( G_{s(n)}= \{  g\big( f(n) \big) \}\). Cela montre que
		\begin{equation}
			f\big( s(n) \big)=g\big( f(n) \big),
		\end{equation}
		et donc que \( f\) vérifie les propriétés demandées.
	\end{subproof}
\end{proof}

\begin{corollary}[\cite{BIBooZFPUooIiywbk}]       \label{CORooVNHKooRkKtXf}
	Soient deux ensembles \( X,Y\), une application \( \alpha\colon X\to Y\) et une application \( \beta\colon Y\to Y\). Alors il existe une unique application \( H\colon X\times \eN\to Y\) telle que
	\begin{enumerate}
		\item
		      \( H(x , 0)   = \alpha(x)\)   pour tout élément \( x\in X\);
		\item
		      \( H(x , n+1) = \beta( H( x , n) )\) pour tout élément \( x\in X\) et pour tout \( n\in \eN\).
	\end{enumerate}
\end{corollary}

\begin{proof}
	Pour faire le lien avec les notations du théorème \ref{THOooEJPYooZFVnez}, nous notons \( E=\Fun(X,Y)\), \( b=\alpha\in E\) et
	\begin{equation}
		\begin{aligned}
			g\colon E & \to E                 \\
			s         & \mapsto \beta\circ s.
		\end{aligned}
	\end{equation}
	Le théorème \ref{THOooEJPYooZFVnez} donne alors l'existence d'une application \( f\colon \eN\to E\) telle que
	\begin{enumerate}
		\item
		      \( f(0)=b\)
		\item
		      \( f(n+1)=g\big( f(n) \big)\).
	\end{enumerate}
	Nous définissons alors
	\begin{equation}
		\begin{aligned}
			H\colon X\times \eN & \to Y          \\
			(x,n)               & \mapsto f(n)x,
		\end{aligned}
	\end{equation}
	et nous vérifions qu'elle satisfait aux exigences.

	\begin{enumerate}
		\item
		      D'abord nous avons
		      \begin{equation}
			      H(x,0)=f(0)x=b(x)=\alpha(x).
		      \end{equation}
		\item
		      Ensuite, pour \( x\in X\) et \( n\in \eN\) nous avons :
		      \begin{subequations}
			      \begin{align}
				      H(x,n+1) & =f(x+1)x                      \\
				               & =g\big( f(n) \big)x           \\
				               & =g\big( f(n) \big)x           \\
				               & =\big( \beta\circ f(n) \big)x \\
				               & =\beta\big( f(n)x \big)       \\
				               & =\beta\big( H(x,n) \big).
			      \end{align}
		      \end{subequations}
	\end{enumerate}
	Et voilà.
\end{proof}

%---------------------------------------------------------------------------------------------------------------------------
\subsection{Addition sur les naturels}
%---------------------------------------------------------------------------------------------------------------------------

\begin{definition}[élément régulier\cite{BIBooYNNEooAunyKF}]        \label{DEFooIJIEooZaAdSs}
	Soit un ensemble \( E\) muni d'une opération \( *\colon E\times E\to E\). Un élément \( s\in E\) est \defe{régulier à gauche}{élément régulier à gauche} si pour tout \( x,y\in E\) nous avons
	\begin{equation}
		s*x=s*y\Rightarrow x=y.
	\end{equation}
	L'élément \( s\) est régulier à droite si pour tout \( x,y\in E\) nous avons
	\begin{equation}
		x*s=y*s\Rightarrow x=y.
	\end{equation}
	Il est \defe{régulier}{élément régulier} si il est régulier à gauche et à droite.
\end{definition}

\begin{propositionDef}[\cite{RWWJooJdjxEK,MonCerveau}]      \label{PROPooVFOXooXmwpFh}
	Il existe une unique fonction \( f\colon \eN\times \eN\to \eN\) vérifiant
	\begin{enumerate}
		\item       \label{ITEMooILZSooNYIkYR}
		      \( f(a,0)=a\) pour tout \( a\in \eN\)
		\item       \label{ITEMooZWHQooBAjZyE}
		      \( f\big( a,s(b) \big)=s\big( f(a,b) \big)\) pour tout \( a,b\in \eN\).
	\end{enumerate}
	Pour \( a,b\in \eN\) nous notons \( f(a,b)=a+b\).
\end{propositionDef}

\begin{lemma}[\cite{RWWJooJdjxEK}]      \label{LEMooMJMTooOtUuJT}
	Pour tout \( a\in \eN\) nous avons \( s(a)=a+1\).
\end{lemma}

\begin{proof}
	Nous avons :
	\begin{subequations}
		\begin{align}
			s(a) & =s(a+0)        \label{SUBEQooMNBLooTOruhE} \\
			     & =a+s(0)        \label{SUBEQooAGUSooGijYGj} \\
			     & =a+1.          \label{SUBEQooUZQDooWtNBHO}
		\end{align}
	\end{subequations}
	Justifications.
	\begin{enumerate}
		\item
		      Pour \eqref{SUBEQooMNBLooTOruhE} c'est dans la définition \ref{PROPooVFOXooXmwpFh}\ref{ITEMooILZSooNYIkYR} de la somme.
		\item
		      Pour \eqref{SUBEQooAGUSooGijYGj}, c'est dans la définition \ref{PROPooVFOXooXmwpFh}\ref{ITEMooZWHQooBAjZyE} de la somme.
		\item
		      Pour \eqref{SUBEQooUZQDooWtNBHO}. Le symbole «\( 1\)» désigne l'élément \( s(0)\) dans \( \eN\).
	\end{enumerate}
\end{proof}

\begin{proposition}[\cite{RWWJooJdjxEK}]     \label{PROPooTLTSooGNMTmV}
	En ce qui concerne la somme dans \( \eN\).
	\begin{enumerate}
		\item       \label{ITEMooIFFPooXfftfG}
		      La somme est associative et commutative.
		\item       \label{ITEMooSGRVooPAVFYK}
		      L'élément \( 0\) est neutre.
		\item       \label{ITEMooNUTHooJWWzGv}
		      Tous les éléments de \( \eN\) sont réguliers\footnote{Élément régulier pour une opération, définition \ref{DEFooIJIEooZaAdSs}.} par rapport à l'addition.
	\end{enumerate}
\end{proposition}

\begin{proof}
	En plusieurs parties.
	\begin{subproof}
		\item[Associative]
		Nous devons prouver que \( (a+b)+c=a+(b+c)\) pour tout \( a,b,c\in \eN\). Pour ce faire, nous fixons \( a,b\in\eN\) et nous prouvons l'égalité demandée par récurrence sur \( c\).

		Pour \( c=0\), nous avons \( (a+b)+c=a+b\) et \( a+(b+c)=a+b\). Donc nous sommes d'accord\footnote{Notez que nous n'avons pas utilisé le fait que \( 0\) était neutre des deux côtés -- chose que nous n'avons pas encore démontré. Nous avons seulement utilisé \( a+0=a\), qui est dans la définition de la somme.}.

		Nous vérifions avec \( s(c)\) :
		\begin{subequations}
			\begin{align}
				(a+b)+s(c) & =s\big( (a+b)+c \big)                              \\
				           & =s\big( a+(b+c) \big)  \label{SUBEQooDDUBooAjuuZq} \\
				           & =a+s(b+c)                                          \\
				           & =a+\big( b+s(c) \big).
			\end{align}
		\end{subequations}
		Justifications.
		\begin{itemize}
			\item Pour \eqref{SUBEQooDDUBooAjuuZq}. C'est l'hypothèse de récurrence. À ce stade, je vous conseille d'être capable de rédiger complètement la récurrence et l'appel au théorème \ref{THOooEJPYooZFVnez}.
		\end{itemize}
		\item[Neutre]
		La définition de l'addition contient déjà \( a+0=a\). Nous prouvons par récurrence que \( 0+a=a\) pour tout \( a\in \eN\).

		Pour \( a=0\), l'égalité demandé est correcte : \( 0+0=0\) parce que pour tout \( x\) dans \( \eN\), \( 0+x=x\). Pour \( s(a)\) nous avons
		\begin{equation}
			0+s(a)=s(0+a)=s(a).
		\end{equation}
		La dernière égalité est l'hypothèse de récurrence.

		Nous posons
		\begin{equation}
			A=\{ a\in \eN\tq 0+a=a \}.
		\end{equation}
		Nous avons prouvé que \( 0\in A\) et que \( s(A)\subset A\). Le théorème \ref{THOooEJPYooZFVnez} nous assure alors que \( A=\eN\).
		\item[Commutativité]
		Nous fixons \( a\in \eN\) et nous prouvons par récurrence sur \( b\) que \( a+b=b+a\) pour tout \( b\in \eN\). Cela va être décomposé en plusieurs étapes.
		\item[\( a+0=0+a\)]
		Pour \( b=0\) c'est correct, car \( b+0=0+b=b\) parce que \( 0\) est neutre.
		\item[\( a+1=1+a\)]
		Nous démontrons par récurrence sur \( a\) que \( a+1=1+a\). Avec \( a=0\) c'est déjà fait. Pour les autres,
		\begin{subequations}
			\begin{align}
				s(a)+1 & =(a+1)+1 & \text{lemme \ref{LEMooMJMTooOtUuJT}} \\
				       & =(1+a)+1 & \text{hypothèse récurrence}          \\
				       & =1+(a+1) & \text{associativité}                 \\
				       & =1+s(a).
			\end{align}
		\end{subequations}
		\item[\( a+b=b+a\)]
		Nous y voici. Nous fixons \( a\) et nous prouvons par récurrence que \( a+b=b+a\). Pour \( b=0\) c'est déjà fait. Pour les autres,
		\begin{subequations}
			\begin{align}
				a+s(b) & =a+(b+1)                                    \\
				       & =(a+b)+1                                    \\
				       & =(b+a)+1 & \text{hypothèse récurrence}      \\
				       & =b+(a+1)                                    \\
				       & =b+(1+a) & \text{commutativité avec \( 1\)} \\
				       & =(b+1)+a & \text{associativité}             \\
				       & =s(b)+a.
			\end{align}
		\end{subequations}
		Récurrence terminée.
		\item[Régularité]
		Nous devons prouver que, pour tout \( a,x,y\in \eN\), si \( a+x=a+y\) alors \( x=y\). Nous allons procéder par récurrence en posant
		\begin{equation}
			A=\{ a\in \eN\tq\forall x,y\in \eN, a+x=a+y\Rightarrow x=y\}.
		\end{equation}
		Puisque \( 0+x=x\) et \( 0+y=y\), nous avons \( 0\in A\). Supposons à présent que \( a\in A\) et montrons que \( s(a)\in A\). Soient \( x,y\in \eN\) tels que \( a+x=a+y\). Nous avons :
		\begin{subequations}
			\begin{align}
				            & \quad s(a)+x=s(a)+y                                \\
				\Rightarrow & \quad s(a+x)=s(a+y)    \label{SUBEQooNJHZooSIKPxN} \\
				\Rightarrow & \quad a+x=a+y          \label{SUBEQooWDJLooJhzIFe} \\
				\Rightarrow & \quad x=y              \label{SUBEQooTLYZooFsMaJD}
			\end{align}
		\end{subequations}
		Justifications.
		\begin{itemize}
			\item Pour \eqref{SUBEQooNJHZooSIKPxN}. En utilisant la définition de l'addition et la commutativité, nous avons \( s(a)+x=s(a+x)\).
			\item Pour \eqref{SUBEQooWDJLooJhzIFe}. Parce que \( s\) est injective; c'est dans la définition \ref{DEFooBJBOooWlblAx} d'un triplet naturel.
			\item Pour \eqref{SUBEQooTLYZooFsMaJD}. Parce que \( a\in A\).
		\end{itemize}
		Nous avons prouvé que \( s(A)\subset A\), et donc que \( A=\eN\).
	\end{subproof}
\end{proof}

\begin{lemma}       \label{LEMooCOMSooEWrumL}
	Nous avons \( 0\neq 1\).
\end{lemma}

\begin{proof}
	Par définition \( 1=s(0)\). Comme \( s\) est à valeurs dans \( \eN\setminus\{ 0 \}\), nous ne pouvons pas avoir \( s(0)=0\).
\end{proof}

\begin{lemma}[\cite{MonCerveau}]       \label{LEMooQBHFooCuCusQ}
	Si \( a+b=0\), alors \( a=b=0\).
\end{lemma}

\begin{proof}
	Soient \( a,b\in \eN\) tels que \( a+b=0\), et supposons que \( b\neq 0\). Par la définition \ref{DEFooBJBOooWlblAx}\ref{ITEMooQAKJooGKdJsM}, nous avons \( b=s(c)\) pour un certain \( c\in \eN\).

	Dans ce cas nous avons \( a+b=a+s(c)=s(a+c)\neq 0\) parce que l'image de \( s\) ne contient pas \( 0\). Hélas, par hypothèse nous avons \( a+b=0\). Nous avons obtenu une contradiction, et nous déduisons que \( b=0\).

	Maintenant que nous savons que \( b=0\), il reste \( 0=a+b=a+0=a\).
\end{proof}

%---------------------------------------------------------------------------------------------------------------------------
\subsection{Ordre sur les naturels}
%---------------------------------------------------------------------------------------------------------------------------

\begin{definition}[\cite{RWWJooJdjxEK}]     \label{DEFooAXZSooTEMjlV}
	Pour \( a,b\in \eN\), nous notons \( a\leq b\) si il existe \( x\in \eN\) tel que \( a+x=b\).

	Nous notons également \( a<b \) si \( a\leq b\) et \( a\neq b\).
\end{definition}

\begin{lemma}       \label{LEMooWMYPooLTMyWb}
	Nous avons \( a\leq s(a)\) pour tout \( a\in \eN\).
\end{lemma}

\begin{proof}
	Cela est une conséquence du lemme \ref{LEMooMJMTooOtUuJT} : \( s(a)=a+1\).
\end{proof}

\begin{proposition}     \label{PROPooVXBBooZcghrA}
	La relation \( \leq\) est une relation d'ordre compatible avec l'addition.
\end{proposition}

\begin{proof}
	Plusieurs choses à vérifier.
	\begin{subproof}
		\item[Réflexive]
		Nous avons \( a\leq a\) parce que \( a+0=a\).
		\item[Antisymétrique]
		Soient \( a,b\in \eN\) tels que \( a\leq b\) et \( b\leq a\). Il existe \( x,y\in \eN\) tels que
		\begin{subequations}
			\begin{align}
				b & =a+x  \\
				a & =b+y.
			\end{align}
		\end{subequations}
		En substituant la seconde équation dans la première, \( b=(b+y)+x\) que nous récrivons, en utilisant l'associativité\footnote{Proposition \ref{PROPooTLTSooGNMTmV}\ref{ITEMooIFFPooXfftfG}.},
		\begin{equation}
			0+b=b+(x+y).
		\end{equation}
		En utilisant la régularité, \( 0=x+y\) et donc \( x=y=0\) par le lemme \ref{LEMooQBHFooCuCusQ}. Cela donne alors \( a=b\).
		\item[Transitive]
		Si \( a\leq b\) et \( b\leq c\), nous avons \( n,p\in \eN\) tels que \( b=a+n\) et \( c=b+p\). Donc
		\begin{equation}
			c=(a+n)+p=a+(n+p),
		\end{equation}
		ce qui signifie que \( a\leq c\). Notez l'utilisation de l'associativité de la somme, démontrée en la proposition \ref{PROPooTLTSooGNMTmV}\ref{ITEMooIFFPooXfftfG}.
		\item[Compatibilité]
		Soient \( a,b,n\in \eN\) tels que \( a\leq b\). Nous devons montrer que \( a+n\leq b+n\). Puisque \( a\leq b\), il existe \( x\in \eN\) tel que \( b=a+x\). Par conséquent,
		\begin{equation}
			b+n=a+x+n=(a+n)+x,
		\end{equation}
		qui signifie bien que \( a+n\leq b+n\).
	\end{subproof}
\end{proof}

\begin{lemma}       \label{LEMooPVRQooXPMKTt}
	À propos d'ordre et de stricte inégalité.
	\begin{enumerate}
		\item       \label{ITEMooGWWFooYGPCZw}
		      Si \( x\leq a\) et \( b\neq 0\), alors \( x<a+b\).
		\item       \label{ITEMooRWGWooAfkrri}
		      Si \( x\leq a\), alors \( x<s(a)\).
		\item       \label{ITEMooWCOIooMWrCag}
		      Si \( x<a\), alors \( s(x)\leq a\).
	\end{enumerate}
\end{lemma}

\begin{proof}
	En plusieurs parties.
	\begin{subproof}
		\item[Pour \ref{ITEMooGWWFooYGPCZw}]
		Si \( x\leq a\), il existe \( d\in \eN\) tel que \( x+d=a\). Nous avons alors aussi
		\begin{equation}
			x+d+b=a+b,
		\end{equation}
		ce qui signifie que \( x\leq a+b\). Mais si \( x\) était égal à \( a+b\), nous aurions \( d+b=0\), ce qui impliquerait\footnote{Par le lemme \ref{LEMooQBHFooCuCusQ}.} \( d=b=0\), alors que l'hypothèse stipule que \( b\neq 0\). Donc \( x\neq a+b\).
		\item[Pour \ref{ITEMooRWGWooAfkrri}]
		Il s'agit seulement d'utiliser la point \ref{ITEMooGWWFooYGPCZw} avec \( b=1\) et le fait que \( s(a)=a+1\) par le lemme \ref{LEMooMJMTooOtUuJT}.
		\item[Pour \ref{ITEMooWCOIooMWrCag}]
		Par hypothèse, il existe \( b\neq 0\) tel que \( x+b=a\). Puisque \( b\neq 0\), il existe \( c\in \eN\) tel que \( b=s(c)\) et donc, tel que
		\begin{equation}
			x+s(c)=a.
		\end{equation}
		En utilisant le fait que \( s(c)=c+1\) ainsi que l'associativité et la commutativité de l'addition (proposition \ref{PROPooTLTSooGNMTmV}\ref{ITEMooIFFPooXfftfG}) nous avons
		\begin{equation}
			a=x+s(c)=s(x)+c,
		\end{equation}
		ce qui prouve que \( s(x)\leq a\).
	\end{subproof}
\end{proof}

\begin{lemma}       \label{LEMooCSIXooHeuWEd}
	L'élément \( 0\) est l'unique plus petit élément de \( \eN\).
\end{lemma}

\begin{proof}
	Puisque \( 0\) est neutre pour l'addition\footnote{Proposition \ref{PROPooTLTSooGNMTmV}\ref{ITEMooSGRVooPAVFYK}.}, nous avons \( a+0=a\) pour tout \( a\in \eN\) et donc \( 0\leq a\) pour tout \( a\). Cela veut dire que \( 0\) est plus petit que tout élément de \( \eN\).

	En ce qui concerne l'unicité, soit \( z\in \eN\) tel que \( z\leq a\) pour tout \( a\in \eN\). Si \( z\neq 0\), il existe \( x\in \eN\) tel que \( z=s(x)\). Nous avons donc \( z\geq x\) en même temps que \(x\leq z\). Cela implique \( z=x\) (parce qu'une relation d'ordre est symétrique) et donc \( z=z+1\). En utilisant la régularité de \( z\) pour l'addition nous en déduisons que \( 0=1\), ce qui est impossible par le lemme \ref{LEMooCOMSooEWrumL}.
\end{proof}

\begin{lemma}       \label{LEMooJRZKooOMhOkH}
	Si \( a\leq b\) et \( b\leq a\), alors \( a=b\).
\end{lemma}

\begin{proof}
	L'inégalité \( a\leq b\) dit qu'il existe \( x\in \eN\) tel que \( a+x=b\). En mettant cela dans l'inégalité \( b\leq a\) nous trouvons \( a+x\leq a\) qui donne, via la proposition \ref{PROPooVXBBooZcghrA} : \( x\leq 0\). Nous en déduisons que \( x=0\) parce que zéro est l'unique minimum de \( \eN\) par le lemme \ref{LEMooCSIXooHeuWEd}.
\end{proof}

\begin{proposition}     \label{PROPooGCCRooFBYrlo}
	Le couple \( (\eN,\leq)\) est totalement ordonné\footnote{Définition \ref{DEFooVGYQooUhUZGr}.}.
\end{proposition}

\begin{proof}
	Soit \( a\in \eN\). Nous devons prouver que pour tout \( x\in \eN\) nous avons \( x\leq a\) ou \( a\leq x\) (non exclusifs). Nous posons
	\begin{subequations}
		\begin{align}
			A & =\{ x\in \eN\tq x\leq a \}  \\
			B & =\{ x\in \eN\tq a\leq x \},
		\end{align}
	\end{subequations}
	et nous prouvons que \( A\cup B=\eN\) en montrant que \( 0\in A\cup B\) et que \( s(A\cup B)\subset A\cup B\).

	Nous avons \( 0\in A\subset A\cup B\) par le lemme \ref{LEMooCSIXooHeuWEd}.

	Pour étudier \( s(A\cup B)\), nous considérons \( x\in A\cup B\) et nous subdivisons en deux cas selon que \( x\in A\) ou \( x\in B\).

	\begin{subproof}
		\item[Si \( x\in B\)]
		Si \( x\in B\), alors \( a\leq x\leq s(x)\) parce que \( x\leq s(x)\) par le lemme \ref{LEMooWMYPooLTMyWb}. Donc \( s(x)\in B\subset A\cup B\).
		\item[Si \( x\in A\)]
		Si \( x\in A\), il y a deux possibilités : \( x=a\) et \( x\neq a\). Si \( x=a\), alors \( a\leq s(x)\) et donc \( s(x)\in A\subset A\cup B\).

		Si \( x\neq a\), alors le lemme \ref{LEMooPVRQooXPMKTt}\ref{ITEMooWCOIooMWrCag} nous indique que \( s(x)\leq a\) et donc \( s(x)\in A\subset A\cup B\).
	\end{subproof}
	Nous avons donc prouvé que \( s(A\cup B)\subset A\cup B\), et donc que \( A\cup B=\eN\).
\end{proof}

\begin{proposition}[\cite{RWWJooJdjxEK,MonCerveau}]     \label{PROPooMZOWooHmsXzI}
	L'ensemble ordonné \( (\eN,\leq)\) vérifie les propriétés suivantes.
	\begin{enumerate}
		\item       \label{ITEMooJLAHooDKukfH}
		      L'élément \( 0\) est l'unique minimum de \( \eN\).
		\item       \label{ITEMooYAJIooEFmOpB}
		      Toute partie non vide a un unique plus petit élément.
		\item       \label{ITEMooSRGOooNYJJHY}
		      L'ensemble \( \eN\) n'a pas de plus grand élément.
		\item       \label{ITEMooKIHZooDRTCdx}
		      Toute partie non vide majorée a un unique plus grand élément.
	\end{enumerate}
\end{proposition}

\begin{proof}
	Point par point.
	\begin{subproof}
		\item[Pour \ref{ITEMooJLAHooDKukfH}]
		C'est le lemme \ref{LEMooCSIXooHeuWEd}.

		\item[Pour \ref{ITEMooYAJIooEFmOpB}]
		Soit une partie \( A\) non vide dans \( \eN\). Si \( 0\in A\), nous avons fini.
		\begin{subproof}
			\item[L'ensemble \( B\)]
			Nous supposons donc que \( A\) ne contient pas zéro et nous définissons
			\begin{equation}
				B=\{ n\in \eN\setminus A\tq n\leq a, \forall a\in A \}.
			\end{equation}
			\item[Un élément particulier dans \( B\)]
			L'ensemble \( B\) vérifie :
			\begin{itemize}
				\item \( 0\in B\)
				\item \( B\neq \eN\) parce que \( A\) est non vide.
			\end{itemize}
			La contraposée de la condition \ref{ITEMooXPYEooFajywh} de la définition \ref{DEFooBJBOooWlblAx} d'un triplet naturel implique que \( s(B)\nsubset B\). Autrement dit, il existe \( b\in B\) tel que \( s(b)\notin B\).

			\item[Deux fonctions sur \( A\)]
			Puisque \( b\in B\), nous avons une application \( c\colon A\to \eN\) telle que \( b+c(a)=a\). Nous avons \( c(a)\neq 0\) parce que \( c(a)=0\) signifierait \( b=a\), ce qui est impossible parce que \( a\in A\) et \( b\in B\).

			Comme pour tout \( a\in A\), l'élément \( c(a)\) est non nul, il existe une fonction \( d\colon A\to \eN\) telle que \( c(a)=d(a)+1\).
			\item[\( s(b)\in A\)]
			Supposons que \( s(b)\notin A\). Alors il existe \( a\in A\) tel que \( s(b)\leq a\) est faux. Puisque l'ordre est total (proposition \ref{PROPooGCCRooFBYrlo}), nous avons
			\begin{equation}        \label{EQooYQSFooPSPJMt}
				a\leq s(b).
			\end{equation}
			Comme \( b\in B\) nous avons aussi
			\begin{equation}        \label{EQooIPAWooDBSJEa}
				b\leq a.
			\end{equation}
			Et enfin nous avons
			\begin{equation}        \label{EQooWFHRooGDSBFD}
				b\neq a
			\end{equation}
			parce que \( a\in A\) et \( b\in B\).

			Les conditions \eqref{EQooIPAWooDBSJEa} et \eqref{EQooWFHRooGDSBFD} se résument en \( b<a\). Le lemme \ref{LEMooPVRQooXPMKTt}\ref{ITEMooWCOIooMWrCag} nous indique alors que \( s(b)\leq a\). Cela mis à côté de \eqref{EQooYQSFooPSPJMt} conclut que \( a=s(b)\), et donc que \( s(b)\) est dans \( A\). Contradiction. Nous en concluons que \( s(b)\in A\).
			\item[\( s(b)\) est un minimum de \( A\)]
			En utilisant la commutativité et l'associativité de la somme nous avons, pour tout \( a\in A\) :
			\begin{equation}
				a=b+c(a)=b+\big( d(a)+1 \big)=(b+1)+d(a)=s(b)+d(a).
			\end{equation}
			Donc \( s(b)\leq a\) pour tout \( a\in A\). Mais comme \( s(b)\in A\), l'élément \( s(b)\) est bien un minimum de \( A\).
			\item[Unicité]
			Si \( a\) et \( a'\) sont des minimums de \( A\), alors \( a\leq a'\) et \( a'\leq a\). Nous en déduisons que \( a=a'\).
		\end{subproof}
		\item[Pour \ref{ITEMooSRGOooNYJJHY}]
		Si \( M\in \eN\) majore tous les éléments de \( \eN\), alors en particulier \( M\geq s(M)\). Mais le lemme \ref{LEMooWMYPooLTMyWb} nous indique que \( M\leq s(M)\). Nous avons donc \( s(M)=M\), c'est-à-dire \( M=M+1\). En utilisant la régularité de \( M\)\footnote{Dit plus simplement : en simplifiant par \( M\).}, nous trouvons \( 0=1\), ce qui est impossible par le lemme \ref{LEMooCOMSooEWrumL}.
		\item[Pour \ref{ITEMooKIHZooDRTCdx}]
		Soit \( A\) une partie non vide et majorée de \( \eN\).
		\begin{subproof}
			\item[L'ensemble \( B\)]
			Nous posons
			\begin{equation}
				B=\{ n\in \eN\setminus A\tq a\leq n, \forall a \in A \}.
			\end{equation}
			\item[\( B\) est non vide]
			Soit un majorant \( M\) de \( A\) : pour tout \( a\in A\) nous avons \( a\leq M\). Nous avons \( s(M)\notin A\), parce que si \( s(M)\) était dans \( A\), ce serait un élément de \( A\) strictement plus grand que tout \( a\in A\). Donc \( B\) est non vide parce qu'il contient \( s(M)\).
		\end{subproof}
		\item[Minimum]
		Puisque \( B\) est non vide, il possède un plus petit élément que nous notons \( b\). Nous savons que \( b\neq 0\) parce que sinon \( A\) serait vide. Il existe donc \( c\in \eN\) tel que \( b=s(c)\).

		\item[\( a\leq c\) pour tout \( a\in A\)]
		Comme \( s(c)\in B\) nous avons \( a< s(c)\) pour tout \( a\in A\). Donc, par le lemme \ref{LEMooPVRQooXPMKTt}\ref{ITEMooWCOIooMWrCag} nous avons \( s(a)\leq s(c)\), c'est-à-dire \( a+1\leq c+1\). Par régularité nous avons \( a\leq c\).

		Nous avons prouvé que \( a\leq c\) pour tout \( a\in A\).
		\item[\( c\in A\)]
		Si \( c\) n'est pas dans \( A\), alors il est dans \( B\) et il contredit la minimalité de \( b\). Donc \( c\) est dans \( A\).
		\item[Conclusion]
		L'élément \( c\) est dans \( A\) tout en étant plus petit que tout élément de \( A\).
		\item[Unicité]
		Si \( x\) est un élément minium de \( A\), alors nous avons \( x\leq c\) parce que \( x\) est minimum et \( c\leq x\) parce que \( c\) est minimum, et donc \( x=c\).
	\end{subproof}
\end{proof}

\begin{lemma}       \label{LEMooKUWUooPLWelf}
	Toute partie finie non vide de \( \eN\) est majorée et minorée.
\end{lemma}

\begin{lemmaDef}        \label{LEMooOEJOooOgaxzi}
	Si \( A\) est une partie de \( \eN\), il existe un unique élément \( m\in \eN\) tel que
	\begin{subequations}
		\begin{numcases}{}
			m\in A\\
			m\leq a\,\forall a\in A.
		\end{numcases}
	\end{subequations}
	Cet élément est noté \( \min(A)\) et nommé \defe{minimum de \( A\)}{minimum}.

	Si \( A\) est majoré, il existe un unique élément \( M\in \eN\) tel que
	\begin{subequations}
		\begin{numcases}{}
			M\in A\\
			M\geq a\,\forall a\in A.
		\end{numcases}
	\end{subequations}
	Cet élément est noté \( \max(A)\) et nommé \defe{maximum de \( A\)}{maximum}.
\end{lemmaDef}

\begin{lemma}[\cite{MonCerveau}]       \label{LEMooYMRJooYIAhBb}
	Quelques affirmations sur l'ordre dans \( \eN\).
	\begin{enumerate}
		\item       \label{ITEMooTLOIooTWNtod}
		      Il n'existe pas de \( n\in \eN\) tel que \( n<0\).
		\item       \label{ITEMooPJKQooGfLCUM}
		      Si \( a,b\in \eN\) vérifient \( a>b\), alors il n'existe pas de \( x\) dans \( \eN\) tel que \( a+x=b\).
	\end{enumerate}
\end{lemma}

\begin{proof}
	En deux parties.
	\begin{subproof}
		\item[Pour \ref{ITEMooTLOIooTWNtod}]
		Nous savons par la proposition \ref{PROPooMZOWooHmsXzI}\ref{ITEMooJLAHooDKukfH} que \( 0\) est l'unique minimum de \( \eN\). Nous avons donc forcément \( 0\leq n\). Si \( n\) vérifie de plus \( n\leq 0\) alors nous avons \( n=0\) par symétrie de la relation d'ordre \( \leq\). Il n'est donc pas possible d'avoir \( n\neq 0\).
		\item[Pour \ref{ITEMooPJKQooGfLCUM}]
		Si \( b\leq a\) il existe \( c\in \eN\) tel que \( b+c=a\). Et comme \( a\neq b\), \( c\) n'est pas nul et il existe \( y\in \eN\) tel que \( c=s(y)\). Bref, nous avons
		\begin{equation}
			b+s(y)=a.
		\end{equation}
		Si de plus il existe \( x\in \eN\) tel que \( a+x=b\) nous aurions
		\begin{equation}
			a+x+s(y)=a.
		\end{equation}
		Comme \( a\) est régulier pour l'addition\footnote{Proposition \ref{PROPooTLTSooGNMTmV}\ref{ITEMooNUTHooJWWzGv}.}, nous avons
		\begin{equation}
			x+s(y)=0,
		\end{equation}
		ce qui signifie, par le lemme \ref{LEMooQBHFooCuCusQ} que \( x=s(y)=0\). Puisque \( s\) prend ses valeurs dans \( \eN\setminus\{ 0 \}\), cela est impossible.
	\end{subproof}
\end{proof}

\begin{definition}      \label{DEFooKBUFooLvMHrf}
	Soient \( a,b\in \eN\) tels que \( a\leq b\). Nous notons par \( \{ a,\ldots, b \}\) l'ensemble
	\begin{equation}
		\{ x\in \eN\tq a\leq x\leq b \}.
	\end{equation}
\end{definition}

\begin{proposition}     \label{PROPooFYMJooWihvhk}
	Toute application \( \eN\to \eN\) strictement croissante est injective.
\end{proposition}

\begin{proof}
	Soit une application strictement croissante \( f\colon \eN\to \eN\). Soient \( a,b\in \eN\) tels que \( f(a)=f(b)\). Puisque l'ordre est total\footnote{Proposition \ref{PROPooGCCRooFBYrlo}.}, nous supposons que \( a\leq b\). Si \( a=b\) nous avons terminé. Nous supposons donc que \( a\neq b\), c'est-à-dire que \( a<b\). Par stricte croissance nous avons alors \( f(a)<f(b)\) qui signifie \( f(a)\leq f(b)\) et \( f(a)\neq f(b)\). Contradiction. Il n'existe donc pas de \( a\neq b\) tels que \( f(a)=f(b)\). L'application \( f\) est donc injective.
\end{proof}

\begin{lemma}[\cite{MonCerveau}]        \label{LEMooFKLPooPrmeUU}
	Si \( S\) n'est pas majoré dans \( \eN\), alors il existe une bijection \( \eN\to S\).
\end{lemma}

\begin{proof}
	Nous considérons l'application suivante :
	\begin{equation}
		\begin{aligned}
			g\colon S & \to S                            \\
			n         & \mapsto \min\{ x\in S\tq x>n \}.
		\end{aligned}
	\end{equation}
	Cette application est bien définie parce que tout partie non vide de \( \eN\) a un plus petit élément\footnote{Proposition \ref{PROPooMZOWooHmsXzI}\ref{ITEMooYAJIooEFmOpB}.}. Maintenant nous définissons \( f\colon \eN\to S\) par
	\begin{subequations}
		\begin{numcases}{}
			f(0)=\min(S)\\
			f(n+1)=g\big( f(n) \big).
		\end{numcases}
	\end{subequations}
	C'est le théorème \ref{THOooEJPYooZFVnez} qui nous permet de le faire. Nous montrons que \( f\) est bijective.
	\begin{subproof}
		\item[Injective]
		Nous avons
		\begin{equation}
			f(n+1)\in\{ x\in S\tq x>f(n) \}.
		\end{equation}
		Donc \( f\) est strictement croissante. Elle est donc injective.
		\item[Surjective]
		Soit \( a\in S\). Nous allons voir que \( a\) est dans l'image de \( f\). Pour cela nous posons
		\begin{equation}
			A=\{ x\in \eN\tq f(x)<a \}.
		\end{equation}
		Cet ensemble est majoré par \( a\). En effet si \( x\in A\) nous avons \( x\leq f(x)<a\). La partie \( A\) de \( \eN\) possède un maximum. Nous notons \( M=\max(A)\). Ce \( M\) a deux propriétés intéressantes.
		\begin{subproof}
			\item[D'abord]
			Puisque \( M\in A\), nous avons \( f(M)<a\). Une autre façon de dire cela est de dire que
			\begin{equation}
				a\in \{ x\in S\tq x>f(M) \}.
			\end{equation}
			Or \( f(M+1)=\min\{ x\in S\tq x>f(M) \}\). Donc \( f(M+1)\leq a\).
			\item[Ensuite]
			Puisque \( M\) est le maximum de \( A\), \( M+1\) majore \( A\), c'est-à-dire que \( f(M+1)\geq a\).
			\item[Les deux ensemble]
			Nous avons prouvé que \( f(M+1)\leq a\) et \( f(M+1)\geq a\). Nous en déduisons, par le lemme \ref{LEMooJRZKooOMhOkH}, que \( f(M+1)=a\).
		\end{subproof}
	\end{subproof}
\end{proof}

\begin{normaltext}      \label{NORMooQXASooMXqhjI}
	Durant la preuve du lemme \ref{LEMooFKLPooPrmeUU}, nous n'avons pas été loin de prouver que
	\begin{equation}
		\big( \min(S),S,g \big)
	\end{equation}
	est un triplet naturel.

	Toute partie non bornée de \( \eN\) donne lieu à un triplet naturel.
\end{normaltext}

\begin{definition}[\cite{MonCerveau}]      \label{DEFooAZAYooVjNzmy}
	Soit un ensemble muni d'une loi de composition interne \( (A,+)\). Soit \( n\in \eN\) et \( a\in A\). Nous définissons \( n\times A\) par
	\begin{subequations}
		\begin{numcases}{}
			0\times a=0\\
			(n+1)\times a= n\times a+a.
		\end{numcases}
	\end{subequations}
\end{definition}

\begin{definition}      \label{DEFooLCWLooYrToFv}
	Un ensemble totalement ordonné muni d'une loi de composition interne \( (A,+, \leq)\) est \defe{archimédien}{ensemble!archimédien}\index{archimédien} si pour tout \( x,y\in A\) avec \( x>0\), il existe \( n\in \eN\) tel que \( n\times x\geq y\) (voir la définition \ref{DEFooAZAYooVjNzmy}).
\end{definition}

%---------------------------------------------------------------------------------------------------------------------------
\subsection{Multiplication dans les naturels}
%---------------------------------------------------------------------------------------------------------------------------

\begin{propositionDef}      \label{PROPooBBQPooRgPOjf}
	Il existe une unique fonction \( f\colon \eN\times \eN\to \eN\) telle que
	\begin{enumerate}
		\item       \label{ITEMooNTUUooDAUVsV}
		      \( f(a,0)=0\) pour tout \( a\in \eN\)
		\item       \label{ITEMooPPZZooQQabwn}
		      \( f\big( a,s(b) \big)=f(a,b)+a\) pour tout \( a,b\in \eN\).
	\end{enumerate}
	Cette fonction est la \defe{multiplication}{multiplication} et nous notons \( f(a,b)=a\times b\), voire \( ab\) quand il n'y a pas d'ambigüité. Le nombre \( a\times b\) est nommé le \defe{produit}{produit} de \( a\) par \( b\).
\end{propositionDef}

\begin{proof}
	En deux parties.
	\begin{subproof}
		\item[Fonctions définies par récurrence]
		Soit \( a\in \eN\). Par le théorème \ref{THOooEJPYooZFVnez}, il existe une unique application \( f_a\colon \eN\to \eN\) telle que
		\begin{subequations}        \label{EQSooWVCTooNTVjKU}
			\begin{numcases}{}
				f_a(0)=0\\
				f_a\big( s(b) \big)=f_a(b)+a
			\end{numcases}
		\end{subequations}
		\item[Existence]
		Nous considérons, pour chaque \( a\in \eN\) la fonction \( f_a\) définie par les conditions \eqref{EQSooWVCTooNTVjKU}. En posant \( f(a,b)=f_a(b)\), nous avons une application qui vérifie toutes les conditions.
		\item[Unicité]
		Soient des applications \( f\) et \( g\) vérifiant les propriétés demandées. Soit \( a\in \eN\). Nous pouvons définir \( f_a\colon \eN\to \eN\) et \( g_a\colon \eN\to \eN\) par \( f_a(n)=f(a,b)\) et \( g_a(b)=g(a,b)\).

		Les applications \( f_a\) et \( g_a\) vérifient toutes deux les conditions \eqref{EQSooWVCTooNTVjKU}, et sont donc égales : \( f_a=g_a\) pour tout \( a\). Donc \( f=g\).
	\end{subproof}
\end{proof}

\begin{normaltext}
	Nous supposons que \randomGender{le lecteur}{la lectrice} connait déjà la priorité des opérations. \randomGender{Il}{Elle} saura donc interpréter des expressions comme \( a\times b+c\) comme voulant dire \( (a\times b)+c\) sans que nous ayons à ajouter des parenthèses.
\end{normaltext}

\begin{proposition}[\cite{RWWJooJdjxEK,MonCerveau}]     \label{PROPooGHDOooFYRmon}
	La multiplication a les propriétés suivantes.
	\begin{enumerate}
		\item       \label{ITEMooHFWRooDCEpjj}
		      \( n\times 1=n\) pour tout \( n\in \eN\).
		\item       \label{ITEMooRSYMooSUrRsl}
		      \( 1\times n=n\) pour tout \( n\in \eN\).
		\item       \label{ITEMooWJPOooRUYjwQ}
		      La multiplication est commutative.
		\item       \label{ITEMooNBYKooXnGRrf}
		      \( 0\times n=0\) pour tout \( n\in \eN\)
		\item      \label{ITEMooLJQBooVpUxUv}
		      L'élément \( 1\) est neutre pour la multiplication.
		\item       \label{ITEMooDYLIooETIBEL}
		      La multiplication est distributive par rapport à l'addition.
		\item       \label{ITEMooQBFSooWGDQYX}
		      La multiplication est associative.
	\end{enumerate}
\end{proposition}

\begin{proof}
	Point par point.
	\begin{subproof}
		\item[Pour \ref{ITEMooHFWRooDCEpjj}]
		Nous avons \( n\times 1=n\times s(0)=(n\times 0)+n=0+n=n\). Donc \( n\times 1=n\).
		\item[Pour \ref{ITEMooRSYMooSUrRsl}]
		Nous le faisons par récurrence. Par définition c'est vrai pour \( n=0\). En ce qui concerne la récurrence, nous supposons que \( 1\times n=n\), et nous prouvons que \( 1\times s(n)=s(n)\) :
		\begin{equation}
			1\times s(n)=(1\times n)+1=n+1=s(n).
		\end{equation}
		\item[Pour \ref{ITEMooWJPOooRUYjwQ}]
		Soit \( a\in \eN\). Nous prouvons par récurrence sur \( b\in \eN\) que \( a\times b=b\times a\). Pour \( b=0\) c'est bon. Pour la récurrence, nous supposons que \( a\times b=b\times a\) et nous prouvons que \( a\times s(b)=s(b)\times a\) :
		\begin{subequations}
			\begin{align}
				a\times s(b) & =a\times b+a                                                   \\
				             & =b\times a+a             & \text{récurrence}                   \\
				             & =(b\times a)+(1\times a) & \text{par \ref{ITEMooRSYMooSUrRsl}} \\
				             & =(b+1)\times a           & \text{distributivité}               \\
				             & =s(b)\times a.
			\end{align}
		\end{subequations}
		\item[Pour \ref{ITEMooNBYKooXnGRrf}]
		C'est vrai pour \( n=0\) par la définition \ref{PROPooBBQPooRgPOjf}\ref{ITEMooNTUUooDAUVsV}. En ce qui concerne \( s(n)\), nous avons
		\begin{equation}
			0\times s(n)=(0\times n)+0=0.
		\end{equation}
		\item[Pour \ref{ITEMooLJQBooVpUxUv}]
		C'est la combinaison de \ref{ITEMooHFWRooDCEpjj} et \ref{ITEMooRSYMooSUrRsl}.
		\item[Pour \ref{ITEMooDYLIooETIBEL}]
		Soient \( a,b\in \eN\). Nous prouvons par récurrence sur \( c\in \eN\) que\footnote{Nous n'écrivons pas toutes les parenthèses parce que les règles de priorité des opérations sont supposées connues. J'invite cependant \randomGender{le lecteur}{la lectrice} à remarquer qu'une formalisation de ces règles n'est probablement pas facile. Pour que tout soit rigoureux, il faudrait un algorithme qui parcourt une suite de caractères et l'interprète en ajoutant correctement les parenthèses.}
		\begin{equation}
			(a+b)\times c=a\times c+b\times c.
		\end{equation}
		Pour \( c=0\), nous avons \( (a+b)\times 0=0\) ainsi que \( a\times 0=b\times 0=0\) en vertu des points précédents sur la multiplication par zéro. Pour la récurrence nous utilisons associativité et commutativité de la somme :
		\begin{subequations}
			\begin{align}
				(a+b)\times s(c) & =(a+b)\times c+(a+b)                                \\
				                 & =a\times c+b\times c+a+b                            \\
				                 & =(a\times c +a)+(b\times c+b)                       \\
				                 & =\big( a\times s(c) \big)+\big( b\times s(c) \big).
			\end{align}
		\end{subequations}
		\item[Pour \ref{ITEMooQBFSooWGDQYX}]
		Soient \( a,b\in \eN\). Nous démontrons par récurrence sur \( c\in \eN\) que \( (a\times b)\times c=a\times (b\times c)\). Pour \( c=0\) l'égalité est triviale. Nous supposons que l'égalité est correcte pour \( c\), et nous la prouvons pour \( s(c)\) :
		\begin{subequations}
			\begin{align}
				(a\times b)\times s(c) & =\big( (a\times b)\times c \big)+a\times b                                                                      \\
				                       & =\big( a\times (b\times c) \big)+a\times b & \text{récurrence}                                                  \\
				                       & =\big( (b\times c)\times a \big)+b\times a & \text{commutativité}                                               \\
				                       & =(b\times c+b)\times a                     & \text{distributivité}                                              \\
				                       & =\big( b\times s(c) \big)\times a          & \text{définition \ref{PROPooBBQPooRgPOjf}\ref{ITEMooPPZZooQQabwn}} \\
				                       & =a\times \big( b\times s(c) \big)          & \text{commutativité}.
			\end{align}
		\end{subequations}
	\end{subproof}
\end{proof}

\begin{lemma}[\cite{MonCerveau}]
	La multiplication est compatible avec l'ordre :
	\begin{equation}
		a\leq b\Rightarrow a\times n\leq b\times n
	\end{equation}
	pour tout \( n\in \eN\).
\end{lemma}

\begin{proof}
	Par la définition \ref{DEFooAXZSooTEMjlV} de l'ordre, si \( a\leq b\), il existe \( c\in \eN\) tel que \( b=a+c\). En utilisant la distributivité\footnote{Proposition \ref{PROPooGHDOooFYRmon}\ref{ITEMooDYLIooETIBEL}.}, nous avons
	\begin{equation}
		b\times n=(a+c)\times n=a\times n+c\times n.
	\end{equation}
	Nous en déduisons que \( a\times n\leq b\times n\) parce que \( c\times n \in \eN\).
\end{proof}

\begin{lemma}       \label{LEMooEHYEooLDudfn}
	Si \( a\times b=0\), alors \( a=0\) ou \( b=0\) (ou les deux).
\end{lemma}

\begin{proof}
	Supposons que \( a\neq 0\). Alors il existe \( c\in \eN\) tel que \( a=s(c)\). Nous avons
	\begin{equation}
		0=a\times b=s(c)\times b=c\times b+b.
	\end{equation}
	Le lemme \ref{LEMooQBHFooCuCusQ} nous dit alors que \( c\times b=b=0\).
\end{proof}

\begin{lemma}[\cite{MonCerveau}]        \label{LEMooGUXGooBcKJdS}
	Si \( a< b\)  et si \( n\neq 0\), alors
	\begin{equation}
		a\times n<b\times n.
	\end{equation}
\end{lemma}

\begin{proof}
	L'hypothèse \( a\leq b\) implique qu'il existe \( c\in \eN\) tel que \( a+c=b\). De plus \( c\neq 0\) parce que \( a\neq b\). En utilisant la distributivité\footnote{Proposition \ref{PROPooGHDOooFYRmon}\ref{ITEMooDYLIooETIBEL}.}, nous avons
	\begin{equation}
		b\times n=(a+c)\times n=(a\times n)+(c\times n).
	\end{equation}
	Cela prouve que \( a\times n\leq b\times n\). Et comme \( c\) et \( n\) ne sont pas nuls, nous avons même\footnote{Lemme \ref{LEMooEHYEooLDudfn}.} \( c\times n\neq 0\) et donc \( a\times n<b\times n\).
\end{proof}

Une version dans \( \eZ\) sera le lemme \ref{LEMooSVDDooWsyxNP}.
\begin{lemma}[\cite{MonCerveau}]        \label{LEMooSFUKooBNAple}
	Soient \( a\neq 0\) et \( b>1\) dans \( \eN\). Alors
	\begin{equation}
		ab>a.
	\end{equation}
\end{lemma}

\begin{proof}
	Il s'agit d'une application du lemme \ref{LEMooGUXGooBcKJdS} en partant de l'inégalité \( 1<b\) et en la «multipliant» par \( a\).
\end{proof}

\begin{proposition}[\cite{MonCerveau}]
	Tous les naturels non nuls sont réguliers par rapport à la multiplication. Autrement dit, si \( a\neq 0\), alors nous avons
	\begin{equation}
		a\times x=a\times y\Rightarrow x=y.
	\end{equation}
\end{proposition}

\begin{proof}
	Soit \( a\neq 0\) dans \( \eN\). Nous supposons que \( a\times x=a\times y\). Puisque l'ordre sur \( \eN\) est total (proposition \ref{PROPooGCCRooFBYrlo}), nous pouvons supposer que \( y\geq x\); sinon il suffit de permuter les rôles de \( x\) et \( y\) dans tout ce qui suit.

	Il existe \( d\in \eN\) tel que \( y=x+d\). En utilisant l'hypothèse \( a\times y=a\times x\) et la distributivité\footnote{Proposition \ref{PROPooGHDOooFYRmon}\ref{ITEMooDYLIooETIBEL}.},
	\begin{equation}
		a\times x=a\times y=a\times (x+d)=(a\times x)+(a\times d).
	\end{equation}
	Puisque \( (a\times x)\) est régulier pour la somme\footnote{Proposition \ref{PROPooTLTSooGNMTmV}\ref{ITEMooNUTHooJWWzGv}.} nous en déduisons que
	\begin{equation}
		0=a\times d.
	\end{equation}
	Le lemme \ref{LEMooEHYEooLDudfn} dit alors que \( a=0\) ou que \( d=0\). Étant donné que \( a\neq 0\) par hypothèse, nous déduisons que \( d=0\), c'est-à-dire que \( x=y\).
\end{proof}

%---------------------------------------------------------------------------------------------------------------------------
\subsection{Presque unicité des triplets naturels}
%---------------------------------------------------------------------------------------------------------------------------

Il existe de nombreux triplets naturels; l'existence d'un triplet naturel est un théorème de la théorie des ensembles que nous avons accepté. Nous avons déjà à peu près montré que toute partie non bornée de \( \eN\) donne lieu à un nouveau triplet naturel. Voir \ref{NORMooQXASooMXqhjI}.

Nous voyons maintenant que tous les triplets naturels sont équivalents au moins pour l'ordre.

\begin{theorem}[\cite{MonCerveau}]     \label{THOooFUXMooJuigHK}
	Soient des triplets naturels \( (\mN_1,o_1,s_1)\) et \( (\mN_2,o_2,s_2)\). Alors
	\begin{enumerate}
		\item
		      il existe une unique application \( f\colon \mN_1\to \mN_2\) telle que
		      \begin{enumerate}
			      \item
			            \( f(o_1)=o_2\)
			      \item
			            \( f\circ s_1=s_2\circ f\).
		      \end{enumerate}
		\item
		      Une telle application est une bijection croissante.
	\end{enumerate}
\end{theorem}

\begin{proof}
	En plusieurs points.
	\begin{subproof}
		\item[Existence]
		Nous voyons \( (\mN_1, o_1, s_1)\) comme un triplet naturel, et \( \mN_2\) comme un simple ensemble. Nous pouvons appliquer le théorème \ref{THOooEJPYooZFVnez} à \( (\mN_1, o_1, s_1)\). L'élément \( o_1\) va jouer le rôle de \( 0\) alors que \( o_2\) va jouer le rôle de \( b\). L'application \( g\) est \( s_2\). Bref, il existe une unique application \( f\colon \mN_1\to \mN_2\) telle que
		\begin{enumerate}
			\item
			      \( f(o_1)=o_2\)
			\item
			      \( f\big( s_1(n) \big)=s_2\big( f(n) \big)\)
		\end{enumerate}
		pour tout \( n\in \mN_1\).
		\item[Unicité]
		Le théorème \ref{THOooEJPYooZFVnez} donne déjà l'unicité. Nous la faisons quand même, juste pour vous faire plaisir. Soit \( g\), une autre application vérifiant les mêmes conditions. Pour faire la récurrence de façon très explicite, nous posons
		\begin{equation}
			\begin{aligned}
				P\colon \mN_1 & \to \{ 0,1 \}                       \\
				x             & \mapsto \begin{cases}
					1 & \text{si } g(x)=f(x) \\
					0 & \text{sinon. }
				\end{cases}
			\end{aligned}
		\end{equation}
		Notre but est de prouver que \( P(x)=1\) pour tout \( x\in \mN_1\), en utilisant la récurrence telle que décrite dans la proposition \ref{PROPooXTRCooKwrWkq}.

		Nous avons \( f(o_1)=o_2=g(o_1)\). Donc \( P(o_1)=1\). Nous supposons que, pour un certain \( a\in \mN_1\), nous ayons \( P(a)=1\), et nous prouvons que \( P\big( s_1(a) \big)=1\).

		Nous avons \( g(a)=f(a)\), et nous prenons \( s_2\) des deux côtés, nous avons succéssivement
		\begin{subequations}
			\begin{align}
				(s_2\circ g)(a)     & =(s_2\circ f)(a)      \\
				(g\circ s_1)(a)     & =(f\circ s_1)(a)      \\
				g\big( s_1(a) \big) & =f\big( s_1(a) \big).
			\end{align}
		\end{subequations}
		La dernière égalité signifie que \( P\big( s_1(a) \big)=1\). La proposition \ref{PROPooXTRCooKwrWkq} implique que \( P(x)=1\) pour tout \( x\in \mN_1\).
		\item[Bijection, définir l'inverse]
		Nous allons trouver un inverse et le lemme \ref{LEMooWBYSooFqyqQP} nous dit que c'est suffisant. La partie «existence», en inversant les rôles de \( \mN_1\) et \( \mN_2\) nous donne une application \( g\colon \mN_2\to \mN_1\) telle que
		\begin{enumerate}
			\item
			      \( g(o_2)=o_1\)
			\item
			      \( g\circ s_2=s_1\circ g\).
		\end{enumerate}
		Nous allons prouver que \( g\) est un inverse de \( f\).
		\item[\( f\circ g=\id\)]
		Nous posons \( A=\{ x\in \mN_2\tq (f\circ g)(x)=x \}\). Nous avons
		\begin{equation}
			f\big( g(o_2) \big)=f(o_1)=o_2,
		\end{equation}
		et donc \( o_2\in A\).

		Supposons que \( x\in A\). Alors
		\begin{subequations}
			\begin{align}
				(f\circ g)\big( s_2(x) \big) & =(f\circ \underbrace{g\circ s_2}_{s_1\circ g})(x)  \\
				                             & =(\underbrace{f\circ s_1}_{=s_2\circ f}\circ g)(x) \\
				                             & =(s_2\circ f\circ g)(x)                            \\
				                             & =s_2\big( (f\circ g)(x) \big)                      \\
				                             & =s_2(x)
			\end{align}
		\end{subequations}
		Donc \( s_2(x)\in A\). Nous en déduisons que \( A=\mN_2\) par le point \ref{ITEMooXPYEooFajywh} de la définition \ref{DEFooBJBOooWlblAx} d'un triplet naturel.
		\item[\( g\circ f=\id\)]
		J'imagine que c'est la même chose que dans l'autre sens (ci-dessus)\quext{Je n'ai pas fait les calculs; écrivez-moi si ça pose un problème.}.
	\end{subproof}
\end{proof}

\begin{proposition}     \label{PROPooCCVNooYUYcqG}
	L'ensemble structuré \( (\eN,+,\times, \leq)\) est archimédien\footnote{Définition \ref{DEFooLCWLooYrToFv}.}. En d'autres termes, pour tout \( a,b\in \eN\setminus\{ 0 \}\), il existe \( n\in \eN\) tel que
	\begin{equation}
		b<n\times a.
	\end{equation}
\end{proposition}

\begin{proof}
	Soient \( a,b\in \eN\) avec \( a\neq 0\).

	Si \( a>b\), nous avons le résultat avec \( n=1\).

	Si \( a=b\), en prenant \( n=s(1)\) nous avons le résultat. En effet \( s(1)\times a=a+a\). Puisque \( a\neq 0\), nous avons \( a+a\geq a\) et \( a+a\neq a\), donc \( s(1)\times a>a\).

	La vraie vie est avec \( a<b\). Nous posons
	\begin{equation}
		X=\{ x\in \eN\tq 1\leq x\times a\leq b \}
	\end{equation}
	et
	\begin{equation}
		B=\{ x\times a\tq x\in X \}.
	\end{equation}
	L'ensemble \( X\) est non vide parce que \( 1\in X\). L'ensemble \( B\) est alors également non vide, et majoré par \( b\). La proposition \ref{PROPooMZOWooHmsXzI}\ref{ITEMooKIHZooDRTCdx} nous indique alors que \( B\) possède un plus grand élément que nous allons noter \( x_0\times a\) (\( x_0\in X\)).

	Nous posons \( n=s(x_0)\), et nous avons
	\begin{equation}
		x_0\times a< x_0\times a +a=s(x_0)\times a=n\times a.
	\end{equation}
	Nous en déduisons que \( n\times a\) n'est pas dans \( B\) parce que \( x_0\times a\) est le plus grand élément de \( B\). Donc \( x_0\) n'est pas dans \( X\); nous n'avons donc pas les inégalités
	\begin{equation}
		1\leq n\times a\leq b.
	\end{equation}
	Laquelle des deux inégalités est fausse ? Puisque \( n=s(x_0)\geq 1\) et que \( a\geq 1\), nous avons \( 1\leq n\times a\). Donc c'est la seconde inégalité qui est fausse. Nous avons donc \( n\times a>b\).
\end{proof}

\begin{definition}      \label{DEFooNEVNooJlmJOC}
	Soit \( A\) un ensemble muni d'une loi de composition interne\footnote{Peut-être un anneau, mais comme nous avons l'intention, dans les propositions \ref{PROPooXXGHooLafGsI} et suivantes, de faire des sommes vers \( (\eN,+)\), plutôt un monoïde.} notée \( +\). Si nous avons une application \( \alpha\colon \eN\to A\), alors nous définissons la notation \( \sum_{i=0}^N\alpha(i)\) par récurrence de la façon suivante :
	\begin{enumerate}
		\item
		      \( \sum_{i=0}^0\alpha(i)=\alpha(0)\),
		\item
		      \( \sum_{i=0}^{k}\alpha(i)=\sum_{i=0}^{k-1}\alpha(i)+\alpha(k)\).
	\end{enumerate}
\end{definition}

\begin{normaltext}      \label{NORMooKERZooGWhWwo}
	Si vous êtes \randomGender{attentif}{attentive}, vous remarquerez que la définition \ref{DEFooNEVNooJlmJOC} a besoin du théorème \ref{THOooEJPYooZFVnez} pour s'assurer que \( \sum_{i=0}^N\) est bien définie pour tout \( N\in \eN\).
\end{normaltext}

\begin{proposition}[La multiplication est une somme itérée\cite{RWWJooJdjxEK}]        \label{PROPooXXGHooLafGsI}
	Pour tout \( a,b\in \eN\), nous avons
	\begin{equation}
		\sum_{i=1}^na=a\times n.
	\end{equation}
\end{proposition}

\begin{proof}
	Nous le faisons par récurrence en partant de \( n=1\). Avec \( n=1\) nous avons \( \sum_{i=1}^1a=a\), et \( a\times 1=a\). Donc c'est bon.

	Pour la récurrence nous avons :
	\begin{equation}
		a\times s(n)=a\times n+a=\sum_{i=1}^na+a=\sum_{i=1}^{n+1}a=\sum_{i=1}^{s(n)}a.
	\end{equation}
\end{proof}

\begin{lemma}       \label{LEMooIETGooMyrilW}
	Soit \( a>1\) dans \( \eN\). Pour tout \( n\geq 1\) nous avons \( na\leq a^n\).
\end{lemma}

\begin{proof}
	Par récurrence. Avec \( n=1\) nous avons bien \( a\leq a\); pas de problème. Supposons que \( na\leq a^n\), et montrons le pas de récurrence. Nous avons :
	\begin{subequations}
		\begin{align}
			(n+1)a & =   na+a                                   \\
			       & \leq  na+na  & \text{parce que }  a\leq na \\
			       & =   2na                                    \\
			       & \leq  2a^n   & \text{récurrence}           \\
			       & \leq  aa^n   & \text{parce que } a\geq 2   \\
			       & =   a^{n+1}.
		\end{align}
	\end{subequations}
\end{proof}

\begin{proposition}[\cite{RWWJooJdjxEK}]
	Soit \( a>1\). Alors
	\begin{enumerate}
		\item
		      l'application \( n\mapsto a^n\) est strictement croissante;
		\item
		      l'ensemble \( \{ a^n\tq n\in \eN \}\) n'est pas majoré.
	\end{enumerate}
\end{proposition}

\begin{proof}
	Nous avons
	\begin{subequations}
		\begin{align}
			a^{n+1} & =a^n\times a                                        \\
			        & >a^n\times 1 & \text{lemme \ref{LEMooSFUKooBNAple}} \\
			        & =a^n.
		\end{align}
	\end{subequations}
	Cela prouve le premier point.

	Pour le second point, soit \( m\in \eN\). Nous devons trouver \( N\in \eN\) tel que \( a^N\geq m\). Puisque \( \eN\) est archimédien\footnote{Proposition \ref{PROPooCCVNooYUYcqG}.}, nous pouvons considérer \( N\) tel que \( Na>m\). Le lemme \ref{LEMooQBHFooCuCusQ} nous assure alors que
	\begin{equation}
		m<Na\leq a^N.
	\end{equation}
\end{proof}

\begin{theorem}[division euclidienne \cite{RWWJooJdjxEK}]       \label{THOooKDJVooRIJRHP}
	Pour tout \( a\in \eN\), pour tout \( b\in \eN\setminus\{ 0 \}\), il existe un unique couple \( (q,r)\in \eN^2\) tel que \( a=bq+r\) avec \( 0\leq r<b\).

	Si \( r=0\), nous disons que \( a\) est \defe{divisible}{divisible} par \( b\).
\end{theorem}

\begin{proof}
	Existence puis unicité.
	\begin{subproof}
		\item[Existence]

		Nous posons
		\begin{equation}
			A=\{ bx\tq x\in \eN,bx\leq a \}.
		\end{equation}
		L'ensemble \( A\) contient \( 0\) (avec \( x=0\)) et est majoré par \( a\). Donc il possède un plus grand élément que nous notons \( bq\). Puisque \( bq\in A\), nous avons \( bq\leq a\) et donc il existe \( r\in \eN\) tel que
		\begin{equation}        \label{EQooIUICooLenNBP}
			bq+r=a.
		\end{equation}
		Il reste à montrer que \( r<b\). Supposons que \( r\geq b\). Il existerait alors un \( x\) tel que \( b+x=r\). En mettant ça dans \eqref{EQooIUICooLenNBP},
		\begin{equation}
			bq+b+x=a,
		\end{equation}
		c'est-à-dire \( b(q+1)+x=a\), qui signifierait \( b(q+1)\leq a\), ce qui est faux parce que \( bq\) est le plus grand élément de \( A\).
		\item[Unicité]
		Supposons que nous ayons
		\begin{equation}
			a=bq+r=bq'+r'
		\end{equation}
		avec \( 0\leq r<b\) et \( 0\leq r'<b\). Il y a trois possibilités : \( q'<q\), \( q'=q\) et \( q'>q\).
		\begin{subproof}
			\item[Si \( q'<q\)]
			Alors il existe \( x\in \eN\) tel que \( q'+x=q\), et nous avons
			\begin{equation}
				b(q'+x)+r=bq'+bx+r,
			\end{equation}
			ce qui, après distribution et simplification, donne \( r'=bx+r\). Puisque nous avons \( x\geq 1\), il vient
			\begin{equation}
				r'=bx+r\geq b+r\geq b.
			\end{equation}
			Cela n'est pas possible parce que \( r'<b\). Le cas \( q'<q\) n'est pas possible.
			\item[Si \( q'=q\)]
			Nous avons alors immédiatement \( bq+r=bq+r'\) et donc \( r=r'\). Unicité.
			\item[Si \( q'>q\)]
			En posant \( q+x=q'\) nous trouvons la même impossibilité que dans le cas \( q'<q\).
		\end{subproof}
	\end{subproof}
\end{proof}

%---------------------------------------------------------------------------------------------------------------------------
\subsection{Écriture d'un naturel dans une base}
%---------------------------------------------------------------------------------------------------------------------------

\begin{normaltext}
	Nous avons déjà donné la notation \( 1=s(0)\). Nous continuons avec \( 2=s(1)\), \( 3=s(2)\), \( 4=s(3)\), \( 5=s(4)\), \( 6=s(5)\), \( 7=s(6)\), \( 8=s(7)\) et \( 9=s(8)\).

	Nous allons maintenant voir comment écrire des nombres plus grands.
\end{normaltext}

Si \( b>1\) et \( N\in \eN\) sont donnés, nous notons
\begin{equation}
	C_{b,N}=\big\{  u\in \{ 0,\ldots, b-1 \}^{N+1}\tq u_N\neq 0  \big\}.
\end{equation}
où les \( u_i\) sont numérotés à partir de \( 0\); donc dire \( u_N\neq 0\) revient à dire que le \emph{dernier} est non nul, et non l'avant dernier. Nous définissons\footnote{Le symbole de sommation est défini par \ref{DEFooNEVNooJlmJOC}.}
\begin{equation}        \label{EQooWWTUooHAnSEv}
	\begin{aligned}
		\varphi_{b,N}\colon C_{b,N} & \to \eN                     \\
		u                           & \mapsto \sum_{i=0}^Nu_ib^i.
	\end{aligned}
\end{equation}
Cette application \( \varphi_{b,N}\) sera encore bien étudiée pour la partie décimale d'un réel. Voir la définition \ref{EqXXXooOTsCK}.

\begin{lemma}[\cite{RWWJooJdjxEK}]       \label{LEMooJUGKooGsbrhi}
	Soient \( b>1\), \( N\geq 0\) ainsi que \( u\in C_{b,N}\). Alors
	\begin{equation}        \label{EQooYHTLooNwqIIq}
		b^N\leq \varphi_{b,N}(u)<b^{N+1}.
	\end{equation}
\end{lemma}

\begin{proof}
	En séparant la somme nous avons
	\begin{equation}
		\varphi_{b,N}(u)=u_Nb^N+\sum_{i=0}^{N-1}u_ib^i.
	\end{equation}
	Puisque \( u_N\geq 1\) nous avons \( b^N\leq u_Nb^N\), et donc
	\begin{equation}
		b^N\leq u_Nb^N\leq \varphi_{b,N}(u).
	\end{equation}
	Voilà qui prouve la première inégalité de \eqref{EQooYHTLooNwqIIq}.

	Pour prouver que \( \varphi_{b,N}(u)<b^{N+1}\), nous faisons une récurrence sur \( N\).
	\begin{subproof}
		\item[Pour \( N=0\)]
		Nous devons prouver que \( \varphi_{b,0}(u)<b\). Par définition \( \varphi_{b,N}(u)=u_0b^0\). Puisque \( u\in\{ 0,\ldots, b-1 \}^{N+1}\), nous avons \( u_0\leq b-1<b\).
		\item[Récurrence]
		Nous supposons que pour tout \( u\in C_{b,N}\) nous avons \( \varphi_{b,N}(u)<b^{N+1}\). Et nous devons montrer que pour tout \( v\in C_{b,N+1}\) nous avons \( \varphi_{b,N+1}(v)<b^{N+2}\).

		Nous posons \( u=(v_0,\ldots, v_N)\); nous avons alors
		\begin{subequations}
			\begin{align}
				\varphi_{b,N+1}(v) & =   v_{N+1}b^{N+1}+\sum_{i=0}^Nv_ib^i                                    \\
				                   & =   v_{N+1}b^{N+1}+\varphi_{b,N}(u)                                      \\
				                   & <   v_{N+1}b^{N+1}+b^{N+1}            & \text{récurrence}                \\
				                   & =   (v_{N+1}+1)b^{N+1}                                                   \\
				                   & \leq  bb^{N+1}                        & \text{parce que }v_{N+1}\leq b-1 \\
				                   & =   b^{N+2}.
			\end{align}
		\end{subequations}
	\end{subproof}
\end{proof}

\begin{lemma}[\cite{MonCerveau}]        \label{LEMooKDKJooSkhcJS}
	Soient \( x\in \eN\) ainsi que \( b\geq 2\). Nous posons
	\begin{equation}
		N=\max\{ k\in \eN\tq b^k\leq x \}.
	\end{equation}
	Alors
	\begin{enumerate}
		\item
		      Si \( n>N\) alors \( \varphi_{b,n}(u)>x\) pour tout \( u\in C_{b,n}\).
		\item
		      Si \( n<N\) alors \( \varphi_{b,n}(u)<x\) pour tout \( u\in C_{b,n}\).
	\end{enumerate}
\end{lemma}

\begin{proof}
	En deux parties.
	\begin{subproof}
		\item[Si \( n>N\)]
		Nous avons, par définition de \( C_{b,n}\) que \( u_n\neq 0\), de telle sorte que
		\begin{equation}
			\varphi_{b,n}(u)\geq u_nb^n\geq b^n>x .
		\end{equation}
		La dernière inégalité est due au fait que \( n\notin \{ k\in \eN\tq b^k\leq x \}\).
		\item[Si \( n<N\)]
		Nous avons
		\begin{equation}
			x\geq b^N> \varphi_{b,n}(u).
		\end{equation}
		Le seconde inégalité est une conséquence du lemme \ref{LEMooJUGKooGsbrhi}.
	\end{subproof}
\end{proof}

\begin{theorem}[\cite{RWWJooJdjxEK}]
	Soit \( b\geq 2\). Si \( x\in \eN\), alors il existe un unique \( N\in \eN\) et un unique \( u\in C_{b,N} \) tels que
	\begin{equation}
		x=\varphi_{b,N}(u).
	\end{equation}
\end{theorem}

\begin{proof}
	Nous commençons par \( x<b\). Dans ce cas, \( N=0\) parce que si \( u_k\neq 0\) avec \( k\neq 0\), nous avons
	\begin{equation}
		\sum_{i=0}^Nu_ib^i\geq u_kb^k\geq b>x.
	\end{equation}
	Donc \( x=x_0b^0=u_0\). Bref, dans le cas \( x<b\) nous avons obligatoirement \( N=0\) et \( u_0=x\).

	Nous étudions à présent le cas \( x\geq b\) que nous subdivisons en plusieurs étapes.
	\begin{subproof}
		\item[\( N\geq 1\)]
		Si \( N=0\), alors \( \varphi_{b,0}(u)=u_0<b\leq x\). Donc \( N\geq 1\).

		Notons incidemment que nous pouvons parler de \( N-1\) à partir de maintenant.
		\item[Unicité, préambule]
		Le lemme \ref{LEMooKDKJooSkhcJS} nous indique que si \( x=\varphi_{b,N}(u)\) pour un certain \( N\in \eN\) et un certain \( u\in C_{b,N}\), alors
		\begin{equation}
			N=\max\{ k\in \eN\tq b^k\leq x \}.
		\end{equation}
		Nous posons
		\begin{equation}
			X_k=\sum_{i=k}^Nu_ib^{i-k},
		\end{equation}
		et nous allons montrer que le couple \( (X_{k+1}, u_k)\) est le résultat de la division euclidienne\footnote{Théorème \ref{THOooKDJVooRIJRHP}.} de \( X_k\) par \(b\).

		D'abord, \( u_k<b\), donc ça a bien la tête d'un reste. Ensuite, pour le quotient,
		\begin{subequations}
			\begin{align}
				bX_{k+1}+u_k & =b\sum_{i=k+1}^Nu_ib^{i-(k+1)}+u_k \\
				             & =\sum_{i=k+1}^Nu_ib^{i-k}+u_k      \\
				             & =\sum_{i=k}^Nu_ib^{i-k}            \\
				             & =X_k.
			\end{align}
		\end{subequations}
		\item[Unicité]
		En quoi cela fait-il avancer la choucroute ? Supposons que \( \varphi_{b,N}(u)=\varphi_{b,M}(v)\). Alors nous avons déjà prouvé que
		\begin{equation}
			M=N=\max\{ k\in \eN\tq b^k\leq x \}.
		\end{equation}
		Ensuite nous devons montrer que \( u=v\). Nous posons \( X_k=\sum_{i=k}^Nu_ib^{i-k}\) et \( Y_k=\sum_{i=k}^Nv_ib^{i-k}\). Notez que
		\begin{equation}
			X_0=Y_0=x.
		\end{equation}
		Si \( X_k=Y_k\), alors par unicité de la division euclidienne nous avons \( X_{k+1}=Y_{k+1}\) et \( u_k=v_k\). Par récurrence nous avons \( X_k=Y_k\) et \( u_k=v_k\) pour tout \( k\).

		\item[Existence]
		Soit \( x\in \eN\). Nous posons \( y_0=x\) et
		\begin{equation}
			y_k=by_{k+1}+u_k
		\end{equation}
		avec \( u_k<b\). Vus l'unicité dans la division euclidienne et le théorème\footnote{Nous ne citerons pas toujours ce théorème à chaque fois que nous définissons quelque chose par récurrence.} \ref{THOooEJPYooZFVnez} permettant la définition par récurrence, ces conditions définissent deux suites \( (u_k)\) et \( (y_k)\) dans \( \eN\).

		Montrons qu'il existe un \( N\in \eN\) tel que \( y_n=0\) pour tout \( n\geq N+1\). Nous avons :
		\begin{subequations}
			\begin{align}
				2y_{k+1} & \leq b y_{k+1} & \text{parce que } b\geq 2    \\
				         & \leq y_k       & \text{pcq }by_{k+1}+u_k=y_k.
			\end{align}
		\end{subequations}
		Bref : \( 2y_{k+1}\leq y_k\). Par récurrence\footnote{Faut-il citer la proposition \ref{PROPooXTRCooKwrWkq} et donner explicitement la fonction \( P\) ?} nous trouvons que
		\begin{equation}
			2^ky_k\leq x
		\end{equation}
		parce que \( y_0=x\). Par le lemme \ref{LEMooIETGooMyrilW}, si \( k\) est assez grand,
		\begin{equation}
			2ky_k\leq 2^ky_k\leq x.
		\end{equation}
		Puisque \( \eN\) est archimédien\footnote{Proposition \ref{PROPooCCVNooYUYcqG}.}, nous pouvons considérer \( s\in \eN\) tel que \( 2s>x\). À ce moment nous avons
		\begin{equation}
			y_n=0
		\end{equation}
		pour tout \( n\geq s\). Nous posons
		\begin{equation}
			N=\max\{ k\tq y_k\neq 0 \}.
		\end{equation}
		Prouvons par récurrence sur \( l\) que
		\begin{equation}        \label{EQooZBKQooFqcckr}
			y_{N-l}=\sum_{i=N-l}^Nu_ib^{(i+l)-N}.
		\end{equation}
		Notez que \( i+l\geq N-l+l=N\), donc \( (i+l)-N\) a un sens.
		\begin{subproof}
			\item[Pour \( l=0\)]
			Avec \( l=0\) nous avons \( \sum_{i=N-l}^Nu_ib^{(i+l)-N}=u_N\). Il faut donc voir que \( y_N=u_N\). Nous avons
			\begin{equation}
				y_N=by_{N+1}+u_N.
			\end{equation}
			En se rappelant que \( y_{N+1}=0\), nous avons le résultat.
			\item[Pour \( l+1\)]
			Pour la récurrence nous avons le calcul suivant :
			\begin{subequations}
				\begin{align}
					y_{N-l-1} & =by_{N-l}+u_{N-l-1}                       \\
					          & =b\sum_{i=N-l}^Nu_ib^{(i+l)-N}+u_{N-l-1}  \\
					          & =\sum_{i=N-l}^Nu_ib^{(i+l)-N+1}+u_{N-l-1} \\
					          & =\sum_{i=N-l-1}^Nu_ib^{(i+l+1)-N}.
				\end{align}
			\end{subequations}
			La récurrence est prouvée. L'égalité \eqref{EQooZBKQooFqcckr} est validée pour tout \( l\).
		\end{subproof}
		En posant \( l=N\) dans \eqref{EQooZBKQooFqcckr} nous trouvons
		\begin{equation}
			y_0=\sum_{i=0}^Nu_ib^i.
		\end{equation}
		Mais la définition de la suite \( (y_k)\) contient \( y_0=x\). Donc nous avons prouvé que
		\begin{equation}
			x=\sum_{i=0}^Nu_ib^i=\varphi_{b,N}(u).
		\end{equation}
	\end{subproof}
\end{proof}

\begin{example}
	Comment écrire le nombre \( b\) en base \( b\) ? Nous devons trouver un \( N\) et une suite \( (u_i)\) tels que
	\begin{equation}
		b=\sum_{i=0}^Nu_ib^i.
	\end{equation}
	Il est facile de voir que le choix \( N=1\) et \( u=(0,1)\) fonctionne bien : \( b=1\times b^1+0\). Nous avons donc
	\begin{equation}
		b=\varphi_{b,1}(1,0).
	\end{equation}
	Nous écrivons cela plus sobrement \( b=10\).
\end{example}

\begin{normaltext}
	À part des cas très exceptionnels, nous utilisons toujours la base \( b=s(9)=s^9(0)\). Nous nous permettons donc d'écrire «64» le nombre \( \varphi_{s(9), 2}(6,4)\). Vous saviez que tout groupe simple d'ordre \( \varphi_{s(9), 2}(6,0)\) est isomorphe au groupe alterné \( A_{\varphi_{s(9),0}(5)}\) ? C'est la proposition \ref{PROPooUBIWooTrfCat}.
\end{normaltext}

La proposition suivante dit que le nombre qui a le plus de chiffres est le plus grand.
\begin{proposition}[\cite{RWWJooJdjxEK}]
	Si \( u\in C_{b,N}\) et \( v\in C_{b,M}\) avec \( M>N\) alors \( \varphi_{b,N}(u)<\varphi_{b,M}(v)\).
\end{proposition}

\begin{proof}
	Le lemme  \ref{LEMooJUGKooGsbrhi} nous dit que
	\begin{equation}
		b^N\leq \varphi_{b,N}(u)< b^{N+1}
	\end{equation}
	et
	\begin{equation}
		b^M\leq \varphi_{b,M}(v)< b^{M+1}.
	\end{equation}
	Puisque \( M>N\) nous avons \( b^{N+1}\leq b^M\) et donc
	\begin{equation}
		\varphi_{b,N}(u)< b^{N+1}\leq b^M\leq \varphi_{b,M}(v).
	\end{equation}
\end{proof}

La proposition suivante dit que si deux nombres s'écrivent avec le même nombre de chiffres, le plus grand est celui dont le premier chiffre différent est le plus grand. Autrement dit, les nombres en écriture de position se classent par ordre lexicographique.

\begin{proposition}
	Soient \( u,v\in C_{b,N}\) tels que \( u_i=v_i\) pour \( i=r+1,\ldots, N\). Si \( u_r>v_r\) alors \( \varphi_{b,N}(u)>\varphi_{b,N}(v)\).
\end{proposition}

\begin{proof}
	En découpant les sommes nous avons
	\begin{equation}
		\varphi_{b,N}(u)=\sum_{i=r+1}^Nu_ib^i+u_rb^r+\sum_{i=0}^{r-1}u_ib^i
	\end{equation}
	et
	\begin{equation}
		\varphi_{b,N}(v)=\sum_{i=r+1}^Nu_ib^i+v_rb^r+\sum_{i=0}^{r-1}v_ib^i.
	\end{equation}
	Puisque \( b^r>\sum_{i=0}^{r-1}u_ib^i\) (lemme \ref{LEMooJUGKooGsbrhi}), nous avons aussi
	\begin{equation}        \label{EQooTZPBooTeauhX}
		b^r+\sum_{i=0}^{r-1}u_ib^i>\sum_{i=0}^{r-1}v_ib^i.
	\end{equation}
	Et le calcul final :
	\begin{subequations}
		\begin{align}
			\varphi_{b,N}(v) & <   \sum_{i=r+1}^Nv_ib^i+v_rb^r+b^r+\sum_{i=0}^{r-1}u_ib^i & \text{pcq \eqref{EQooTZPBooTeauhX}} \\
			                 & =   \sum_{i=r+1}^Nu_ib^i+(v_r+1)b^r+\sum_{i=0}^{r-1}u_ib^i                                       \\
			                 & \leq  \sum_{i=r+1}^Nu_ib^i+u_rb^r+\sum_{i=0}^{r-1}u_ib^i   & \text{pcq } u_r\geq v_r+1           \\
			                 & =   \sum_{i=0}^Nu_ib^i                                                                           \\
			                 & =   \varphi_{b,N}(u).
		\end{align}
	\end{subequations}
	Et voilà.
\end{proof}

%+++++++++++++++++++++++++++++++++++++++++++++++++++++++++++++++++++++++++++++++++++++++++++++++++++++++++++++++++++++++++++
\section{Les entiers}
%+++++++++++++++++++++++++++++++++++++++++++++++++++++++++++++++++++++++++++++++++++++++++++++++++++++++++++++++++++++++++++

\begin{propositionDef}[\cite{RWWJooJdjxEK}]     \label{PROPooFIKUooVHlvTt}
	Soient \( a,b,a',b'\in \eN\). Nous disons que \( (a,b)\sim(a',b')\) si et seulement si
	\begin{equation}
		a+b'=b+a'
	\end{equation}
	\begin{enumerate}
		\item
		      \( \sim\) est une relation d'équivalence sur \( \eN^2\).
		\item       \label{ITEMooZQSHooSDfdvK}
		      Si \( (a,b)\sim (a',b')\) et \( (x,y)\sim (x',y')\) alors
		      \begin{equation}
			      (a+x,b+y)\sim(a'+x',b'+y').
		      \end{equation}
	\end{enumerate}
	L'ensemble des \defe{entiers}{entier} est
	\begin{equation}
		\eZ=(\eN\times \eN)/\sim,
	\end{equation}
	et nous notons \( \overline{ a,b }\in \eZ\) la classe de \( (a,b)\in \eN\times \eN\).
\end{propositionDef}

\begin{proof}
	En plusieurs points.
	\begin{subproof}
		\item[Symétrie]
		C'est la commutativité de la somme dans \( \eN\), proposition \ref{PROPooTLTSooGNMTmV}\ref{ITEMooIFFPooXfftfG}.
		\item[Réflexive]
		Immédiat.
		\item[Transitive]
		Nous supposons que \( (a,b)\sim(u,v)\) et que \( (u,v)\sim(x,y)\). Alors nous avons
		\begin{subequations}
			\begin{align}
				a+v & =u+b  \\
				u+y & =v+x.
			\end{align}
		\end{subequations}
		En additionnant membre à membre,
		\begin{equation}
			a+v+u+y=u+b+v+x.
		\end{equation}
		La commutativité nous permet de mettre \( u\) et \( v\) à droite dans chacun des deux membres. Ensuite la proposition \ref{PROPooTLTSooGNMTmV}\ref{ITEMooNUTHooJWWzGv} nous permet de simplifier par \( u+v\). Il reste \( a+y=b+x\), qui signifie \( (a,b)\sim(x,y)\).
		\item[Pour \ref{ITEMooZQSHooSDfdvK}]
		L'hypothèse donne les égalités
		\begin{subequations}
			\begin{align}
				a+b' & =b+a' \\
				x+y' & =y+x'
			\end{align}
		\end{subequations}
		En sommant, et en utilisant l'associativité,
		\begin{equation}
			(a+x)+(b'+y')=(b+y)+(a'+x').
		\end{equation}
		Cela signifie bien que \( (a+x,b+y)\sim(a'+x',b'+y')\).
	\end{subproof}
\end{proof}

\begin{lemma}
	Soient \( a,b\in \eN\). Nous avons \( (a,b)\sim (0,0)\) si et seulement si \( a=b\).
\end{lemma}

\begin{proof}
	Dire que \( (a,b)\sim (0,0)\) est équivalent à dire que \( a+0=b+0\), ou encore que \( a=b\).
\end{proof}

\begin{propositionDef}[\cite{RWWJooJdjxEK}]
	Soient \( a,b,x,y\in \eN\). L'application
	\begin{equation}
		\begin{aligned}
			f\colon \overline{ (a,b)\times \overline{ (x,y) } } & \to \eZ               \\
			\big( (a',b'),(x',y') \big)                         & \mapsto (a'+x',b'+y')
		\end{aligned}
	\end{equation}
	est constante.

	Nous nommons sa valeur \( \overline{ (a,b) }+\overline{ (x,y) }\).
\end{propositionDef}

\begin{proof}
	Cela est une conséquence de la proposition \ref{PROPooFIKUooVHlvTt}\ref{ITEMooZQSHooSDfdvK}.
\end{proof}

\begin{proposition}
	La paire \( (\eZ,+)\) est un groupe commutatif.
\end{proposition}

\begin{proof}
	En plusieurs points.
	\begin{subproof}
		\item[Neutre]
		Le neutre est \( e=\overline{ (0,0) }\). En effet,
		\begin{equation}
			\overline{ (a,b) }+\overline{ (0,0) }=\overline{ (a+0,b+0) }=\overline{ (a,b) }.
		\end{equation}
		De même \( e+\overline{ (a,b) }=\overline{ (a,b) }\) par commutativité de la somme dans \( \eN\).
		\item[Inverse]
		Il est facile de vérifier que \( \overline{ (b,a) }\) est l'inverse de \( \overline{ (a,b) }\).
		\item[Associativité]
		Calcul direct en utilisant l'associativité dans \( \eN\).
	\end{subproof}
\end{proof}

\begin{proposition}
	L'application
	\begin{equation}
		\begin{aligned}
			\iota\colon \eN & \to \eZ                    \\
			n               & \mapsto \overline{ (n,0) }
		\end{aligned}
	\end{equation}
	est un morphisme\footnote{Certes \( \eN\) n'est pas un groupe, donc le mot «morphisme» est un peu abusé, mais vous voyez ce que je veux dire.} injectif.
\end{proposition}

\begin{proof}
	Le fait que ce soit un morphisme est le calcul
	\begin{equation}
		\iota(a+b)=\overline{ (a+b,0) }=\overline{ (a,0) }+\overline{ (b,0) }=\iota(a)+\iota(b).
	\end{equation}

	Pour l'injectivité, supposons que \( \iota(a)=\iota(b)\). Alors \( \overline{ (a,0) }=\overline{ (b,0) }\), c'est-à-dire \( a+0=b+0\). Donc \( a=b\).
\end{proof}

%---------------------------------------------------------------------------------------------------------------------------
\subsection{Opposé}
%---------------------------------------------------------------------------------------------------------------------------

\begin{lemma}       \label{LEMooSABNooZZDIes}
	Tout élément de \( \eZ\) a un représentant de la forme \( (a,0)\) ou \( (0,b)\).
\end{lemma}

\begin{proof}
	Soient \( a,b\in \eN\). Si \( b\leq a\), alors nous avons
	\begin{equation}
		(a,b)\sim(a-b,0)
	\end{equation}
	où la différence est calculée dans \( \eN\) et a un sens parce que nous avons supposé \( b\leq a\). Si par contre \( a\leq b\) alors
	\begin{equation}
		(a,b)\sim(0,b-a).
	\end{equation}
	Puisque l'ordre sur \( \eN\) est total\footnote{Proposition \ref{PROPooGCCRooFBYrlo}.}, tous les cas sont couverts.
\end{proof}

\begin{lemmaDef}
	Soit \( z\in \eZ\). L'application\footnote{Pour rappel, \( z\) est une classe d'équivalence dans \( \eN\times \eN\), c'est-à-dire une partie de \( \eN\times \eN\). Ça a un sens de prendre \( z\) comme ensemble sur lequel on définit une fonction.}
	\begin{equation}
		\begin{aligned}
			f\colon z & \to \eZ                    \\
			(a,b)     & \mapsto \overline{ (b,a) }
		\end{aligned}
	\end{equation}
	est constante.

	Nous nommons \( -z\) sa valeur.
\end{lemmaDef}

\begin{proof}
	Soient \( (a,b)\) et \( (x,y)\) dans \( z\). Nous avons successivement :
	\begin{itemize}
		\item
		      \( (a,b)\sim (x,y)\).
		\item
		      \( a+y=b+x\).
		\item
		      \( (b,a)\sim (y,x)\)
		\item
		      \( \overline{ (b,a) }=\overline{ (y,x) }\)
		\item
		      \( f(a,b)=f(x,y)\).
	\end{itemize}
	D'où la constance de \( f\).
\end{proof}

\begin{lemma}
	Nous avons
	\begin{enumerate}
		\item       \label{ITEMooSQFGooQPgIMu}
		      \( \eZ=\iota(\eN)\cup -\iota(\eN)\)
		\item       \label{ITEMooHQUQooJeqULl}
		      \( \iota(\eN)\cap -\iota(\eN)=\{ 0 \}\).
	\end{enumerate}
\end{lemma}

\begin{proof}
	Nous avons
	\begin{subequations}
		\begin{align}
			\iota(\eN)  & =\{ \overline{ (n,0) }\tq n\in \eN \}    \label{SUBEQooVJGVooCUxtvk} \\
			-\iota(\eN) & =\{ \overline{ (0,n) }\tq n\in \eN \}.   \label{SUBEQooAPGRooOCkYRr}
		\end{align}
	\end{subequations}
	Montrons à présent les deux points.
	\begin{subproof}
		\item[Pour \ref{ITEMooSQFGooQPgIMu}]
		Nous savons par le lemme \ref{LEMooSABNooZZDIes} que tous les éléments de \( \eZ\) sont de la forme \eqref{SUBEQooVJGVooCUxtvk} ou \eqref{SUBEQooAPGRooOCkYRr}.
		\item[Pour \ref{ITEMooHQUQooJeqULl}]
		Si \( z\in \iota(\eN)\cap -\iota(\eN)\), il existe \( n,m\in \eN\) tels que \( \overline{ (n,0) }=\overline{ (0,m) }\), ce qui signifie en particulier que \( (n,0)\sim(0,m)\) ou encore que \( n+m=0\). Le lemme \ref{LEMooQBHFooCuCusQ} dit alors que \( n=m=0\).

		Nous avons donc \( z= \overline{ (0,0) }=0\).
	\end{subproof}
\end{proof}

%---------------------------------------------------------------------------------------------------------------------------
\subsection{Ordre sur \( \eZ\)}
%---------------------------------------------------------------------------------------------------------------------------

Si \( z\in \eZ\), nous disons que \( z\in \eN\) lorsque \( z\in \iota(\eN)\). C'est un abus de notation qu'il est difficile de ne pas faire.

\begin{propositionDef}[Relation d'ordre \cite{RWWJooJdjxEK} ]       \label{PROPooMYYDooOABOdB}
	Nous disons que \( x\leq y\) si et seulement si \( y-x\in \eN\).

	L'ensemble \( (\eZ, \leq)\) est totalement ordonné.
\end{propositionDef}

Une version dans \( \eR\) sera le lemme \ref{LEMooKAXFooIPyzJC}.
\begin{lemma}       \label{LEMooSVDDooWsyxNP}
	Soient \( a>0\) et \( b>1\) dans \( \eZ\). Nous avons
	\begin{equation}
		ab>a.
	\end{equation}
\end{lemma}

\begin{lemma}       \label{LEMooMYEIooNFwNVI}
	Toute partie bornée de \( \eZ\) possède un plus grand élément.
\end{lemma}

\begin{proposition} \label{PROPooYJBMooZrzkNX}
	Soit \( a,b\in \eZ\) tels que \( a\) divise \( b\). Alors \( | a |\leq | b |\).
	%TODOooEJAGooHKEIFZ définir ``divise'' dans le cas de Z.
\end{proposition}

\begin{lemma}       \label{LEMooJNXIooBmdOVi}
	L'ensemble \( \eZ\) est infini dénombrable.
\end{lemma}
