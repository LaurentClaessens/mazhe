% This is part of Mes notes de mathématique
% Copyright (c) 2011-2025
%   Laurent Claessens
% See the file fdl-1.3.txt for copying conditions.

%+++++++++++++++++++++++++++++++++++++++++++++++++++++++++++++++++++++++++++++++++++++++++++++++++++++++++++++++++++++++++++
\section{Les bases}
\label{SECooIntroBases}
%+++++++++++++++++++++++++++++++++++++++++++++++++++++++++++++++++++++++++++++++++++++++++++++++++++++++++++++++++++++++++++

%---------------------------------------------------------------------------------------------------------------------------
\subsection{Axiomatique et grammaire}
\label{SUBooTheorieEnsembles}
%---------------------------------------------------------------------------------------------------------------------------
\begin{normaltext}\label{NorooFridoIntro1}
	Le Frido n'est pas supposé être lu dans l'ordre de la première à la dernière page; les matières y sont présentées dans l'ordre logique mathématique, et non dans l'ordre logique pédagogique, et encore moins par ordre de difficulté croissante.

	En mathématique, si on lit une démonstration et que l'on veut vraiment tout justifier, et justifier toutes les étapes de tous les résultats utilisés, on tombe forcément un jour sur les axiomes.

	Or l'axiomatique est un sujet particulièrement difficile. Nous n'allons donc pas «tout justifier» jusque là. Nous n'allons même pas préciser quel système d'axiome est utilisé. En particulier nous n'allons pas donner l'axiomatique des ensembles : nous allons supposer connus les ensembles et leurs principales propriétés.
\end{normaltext}

\begin{normaltext}\label{NorooFridoGrammaire}
	Nous n'allons pas non plus formaliser la grammaire des expressions mathématiques\footnote{J'utilise ici des mots que je ne comprends pas, juste pour me donner l'air malin.}. Nous supposons que vous êtes capables de lire des expressions comme
	\begin{equation}
		x\in \eN\Rightarrow\{ a\geq x \}\text{ est infini}.
	\end{equation}
\end{normaltext}

%---------------------------------------------------------------------------------------------------------------------------
\subsection{Les théorèmes encapsulants}
\label{SUBooThmEncapsulants}
%---------------------------------------------------------------------------------------------------------------------------

\begin{normaltext}\label{NorooThmEncapsulants}
	Je propose d'introduire le concept de «théorème encapsulant». Un théorème est encapsulant lorsque les techniques et concepts utiles à sa démonstration ne sont pas utiles pour énoncer et utiliser le théorème.

	Sauter la démonstration d'un théorème encapsulant permet donc de sauter des parties entières de la mathématique.

	Un exemple typique d'un théorème qui n'est pas encapsulant est le théorème de Rolle \ref{ThoRolle}. % \futureok
 	Les ingrédients de la démonstration (topologie, dérivée) sont présents dans l'énoncé\footnote{C'est à dire que pour énoncer le théorème de Rolle, il faut déjà connaître pratiquement toutes les notions nécessaires pour le démontrer.}. D'habitude, quand on utilise le théorème de Rolle on est dans un contexte où on est en train de parler de dérivée, de topologie, et de théorème des valeurs intermédiaires.

	Notez que le théorème de Stokes (théorème \ref{THOooIRYTooFEyxif},  % \futureok
 	plus généralement théorème \ref{ThoATsPuzF}) % \futureok
 	n'est pas non plus encapsulant parce que pour énoncer le théorème, il faut quand même définir les intégrales, les différentielles, les bords, etc. La démonstration de Stokes ne fait pas intervenir de concepts réellement nouveaux par rapport à ce qu'il faut savoir pour énoncer Stokes.

	Le Frido va accepter sans démonstrations deux théorèmes encapsulants qui évitent de devoir parler de pans entiers de la théorie des ensembles.

	\begin{enumerate}
		\item
		      Le lemme de Zorn \ref{LemUEGjJBc}. Il cachera toute la complexité de l'axiome du choix et la moitié de ce qu'apporte cet axiome par rapport à ZF.

		\item
		      L'existence d'un ensemble des naturels (théorème \ref{THOooOXMHooXYgMqb}). Même si nous proposerons un moyen d'en construire un, cela cachera toute la complexité d'axiomatisation, des différentes définitions d'ensembles finis, d'existence de maximum, etc. Voir \cite{BIBooGVVDooDoQSaM,BIBooFGIWooMkmaKH}.
	\end{enumerate}

	Notez que les énoncés de ces théorèmes ne demandent pas grand chose : relation d'ordre sur un ensemble pour le lemme de Zorn; ensemble et application injective pour les naturels.

	Je me demande d'ailleurs si il existe beaucoup d'autres théorèmes encapsulants.
\end{normaltext}

%+++++++++++++++++++++++++++++++++++++++++++++++++++++++++++++++++++++++++++++++++++++++++++++++++++++++++++++++++++++++++++
\section{Ensembles, relations, fonctions}
\label{SECooEltsEnsembles}
%+++++++++++++++++++++++++++++++++++++++++++++++++++++++++++++++++++++++++++++++++++++++++++++++++++++++++++++++++++++++++++

\begin{normaltext}	\label{NORMooFridoMathsFaciles}
	Tout cela pour dire que le Frido ne traitera que de la partie facile de la mathématique.
\end{normaltext}

%---------------------------------------------------------------------------------------------------------------------------
\subsection{Opérations ensemblistes}
\label{SUBooOperationsEnsemblistes}
%---------------------------------------------------------------------------------------------------------------------------

\begin{normaltext}\label{NorooFridoIntro2}
	Nous supposons avoir une théorie des ensembles qui tient la route. En particulier nous supposons que les notions suivantes sont connues :
	\begin{enumerate}
		\item
		      ensemble vide,
		\item
		      ensemble, appartenance, intersection, union,
		\item
		      produit cartésien de plusieurs ensembles\footnote{Nous en proposons une définition dans un cadre général -- définition \ref{DEFooTYIMooZUsYJw} --, dont les fondements devraient être analysés de très près.}.
		      % TODO réf futur OK.
	\end{enumerate}
	Ce sont toutes des choses dont la construction à partir des axiomes n'est en aucun cas évidente. En particulier, des «définitions» comme «l'intersection de deux ensembles est l'ensemble contenant exactement les éléments communs aux deux ensembles» ne sont pas correctes parce qu'elles passent à côté de l'existence et de l'unicité d'un tel ensemble.
\end{normaltext}

\begin{definition}		\label{DEFooJLBVooMyCQMO}
	Soit \( E\), un ensemble et \( A\), une partie de \( E\) (c'est-à-dire un sous-ensemble de \( E\)). Le \defe{complémentaire}{complémentaire} de l'ensemble \(A \), dans \( E\), noté \( \complement A\)\nomenclature[T]{\( \complement A\)}{Le complémentaire de l'ensemble \( A\)} est l'ensemble des éléments de \( E\) qui ne font pas partie de \( A\) :
	\begin{equation}
		\complement A=E\setminus A=\{ x\in E\tq x\notin A \}.
	\end{equation}
	Nous allons aussi régulièrement noter le complémentaire de \( A\) par \( A^c\)\nomenclature[T]{\( A^c\)}{complémentaire de \( A\)}.
\end{definition}

\begin{definition}\label{DefEnsemblesDisjoints}
	Deux ensembles \( A\) et \( B\) sont \defe{disjoints}{ensembles!disjoints} si leur intersection est vide\footnote{Remarquez que les mots «intersection» et «vide» sont de ceux que nous avons décidé de ne pas définir.}; en d'autres termes, si il n'existe aucun élément commun à \( A\) et \( B\).
\end{definition}

\begin{definition}    \label{DefPartitionEnsembles}
	Soit \( E \) un ensemble, et \( E_\alpha \) des parties de \( E \) indexées sur un ensemble \( A \). Les \( E_\alpha \) forment une \defe{partition}{partition d'ensembles} si:
	\begin{itemize}
		\item pour tous \( \alpha, \beta \in A \), on a soit \( E_\alpha =  E_\beta \), soit \( E_\alpha \cap E_\beta = \varnothing \);
		\item la réunion des \( E_\alpha \) est \( E \).
	\end{itemize}
	On dit aussi que \( E \) est partitionné par les \(E_\alpha\).
\end{definition}

\begin{normaltext}\label{NorooDepartTheoEns}
	Remarquez par exemple que la première phrase de l'article de Wikipédia sur la construction de \( \eN\) est «Partant de la théorie des ensembles, on identifie 0 à l'ensemble vide, puis on construit \ldots». Il est bien précisé que l'on part d'une théorie des ensembles.

	La suite de ce chapitre sera essentiellement sans exemple parce qu'avant d'avoir construit les ensembles de nombres, je ne sais pas très bien quels exemples on peut donner de quoi que ce soit.
\end{normaltext}




%---------------------------------------------------------------------------------------------------------------------------
\subsection{Des relations aux bijections}
\label{SUBooRelFctBijections}
%---------------------------------------------------------------------------------------------------------------------------

\begin{definition}[\cite{BIBooMYZQooPqLtKR}]        \label{DEFooRFVTooUUuFuE}
	Une \defe{relation binaire}{relation binaire} d'un ensemble \( E\) vers un ensemble \( F\) est une partie de \( E\times F\).

	Si \( G\) est une relation binaire entre \( E\) et \( F\), et si \( (x; y) \in G \), nous notons \( x\mR_G y\) et nous disons que \( x\) est \emph{en relation avec} \( y\).

	On définit aussi une \defe{relation unaire}{relation unaire} d'un ensemble \( E \) comme une partie de \( E \), et une \defe{relation ternaire}{relation ternaire} de trois ensembles \( E_1,\ E_2 \) et \( E_3 \) comme une partie de \( E_1 \times E_2 \times E_3 \).
\end{definition}

\begin{definition}\label{DefooApplicFct}
	Soit \(\ G\) une relation binaire entre deux ensembles \(E\) et \( F\). Si, pour tout \(x \in E\), et tous \(y,\ y' \in F\), les relations \(x\mR_G y\) et \(x\mR_G y'\) entraînent \( y = y'\), alors on peut associer à chaque \( x \in E\) apparaissant comme première composante dans \( G \), l'élément unique \(y \in F\) qui est en relation avec \(x\) dans \(G\). Cette association définit une \defe{fonction}{fonction} de \(E\) dans \(F\).

	Le plus souvent, les fonctions entre deux ensembles sont désignées par \(f \), ou selon le contexte, par une autre lettre, souvent minuscule. On notera presque toujours les ensembles concernés par la fonction, et on adopte alors la notation \( f: E \to F \).

	Pour marquer l'association, à \(x \in E \), du \(y \in F\) correspondant à travers \( f \), on note cet élément \( f(x) \): ainsi, \( f(x) \in F \), et il est appelé \defe{image de \( x \) par \( f \)}{image!par une fonction, cas général}.

	L'ensemble des \( x \in E \) pour lesquels il existe \( y \in F\) tel que \(x\mR_G y\) est noté \(D_f \), ou bien \( \Dom f \); on l'appelle \defe{domaine de définition de \( f \)}{domaine! de définition d'une fonction} \index{fonction!domaine de définition}, ou encore \defe{ensemble de définition de \( f \)}{fonction!ensemble de définition}. Lorsque \(\Dom(f) = E\), on dit que \( f \) est une \defe{application}{application}.
\end{definition}

\begin{remark}[Lien entre fonction et relation]\label{RemLienFctRelation}
	Si \( f: E \to F \) est une fonction de relation sous-jacente \( G \), alors:
	\begin{itemize}
		\item \( \Dom f = \{ x \in E \tq \exists y \in F: (x, y) \in G\} \);
		\item \( \Im f = \{ y \in F \tq \exists x \in E: (x, y) \in G\} \);
		\item \( \Graph f = G \).
	\end{itemize}
\end{remark}

\begin{remark}	\label{REMooFonctionDomaineEntier}
	Dans le Frido, on prend comme convention que, quand on écrit \( f: E \to F \), alors \( \Dom f = E \). C'est un choix qui n'est peut-être pas standard, mais qui évite des complications sur ce qu'est réellement le domaine de définition. En particulier, dans le Frido, on ne verra que très rarement \( f \) \og tout seul\fg.

	En revanche, on ne sait rien sur \( \Im f \) a priori. Il ne faut pas déduire de la notation que l'ensemble image \( \Im f \) serait \( F \) tout entier.
\end{remark}

\begin{normaltext}\label{NORooLMBYooYjUoju}
	L'\defe{axiome du choix}{axiome!du choix} que nous acceptons peut s'énoncer comme ceci\cite{ooKLIXooHbpufL} : Étant donné un ensemble \( X\) d'ensembles non vides, il existe une fonction définie sur \( X\), appelée fonction de choix, qui à chacun d'entre eux associe un de ses éléments.
\end{normaltext}

\begin{definition}\label{DefooInjSurjBij}
	Soient deux ensembles \( E\) et \( F\). Une fonction \( f\colon E\to F\) est
	\begin{enumerate}
		\item
		      \defe{surjective}{surjection}\index{fonction!surjective} si pour tout \( y\in F\), il existe \( x\in E\) tel que \( y=f(x)\);
		\item
		      \defe{injective}{injection}\index{fonction!injective} si pour tout \( y\in F\), il existe au plus un \(x\in E \) tel que \( y=f(x)\);
		\item
		      \defe{bijective}{bijection}\index{fonction!bijective} si elle est à la fois injective et surjective.
	\end{enumerate}
\end{definition}


\begin{normaltext}\label{NORMooMethodePreuveInj}
	La méthode la plus courante pour démontrer qu'une application \( f\colon E\to F\) est injective est de considérer \( x,y\in E\) tels que \( f(x)=f(y)\), et de prouver à partir de là que \( x=y\). Ou alors de supposer \( x\neq y\) et d'obtenir une contradiction.

	La technique de la contradiction est évidemment la plus courante lorsque l'égalité \( f(x)=g(x)\) implique une équation faisant intervenir \( 1/(x-y)\).

	La surjection et l'injection sont des propriétés bien différentes qu'il convient de prouver séparément. De plus une même «formule» peut définir une application injective, surjective, bijective ou non selon les ensembles sur lesquels nous la considérons.
\end{normaltext}

\begin{definition}  \label{DEFcomposeeFonctions}
	Soit \( f: X \to Y \) et \( g: Y \to Z \) deux fonctions définies respectivement sur les ensembles \( X \) et \( Y \). On définit alors la \defe{composée}{fonction!composée de fonctions} de \( f \) avec \( g \) comme la fonction \( g \circ f \) de l'ensemble \( X \) vers l'ensemble \( Z \) telle que \( (g \circ f) (x) = g\bigl(f(x)\bigr) \).

	On remarquera que le domaine de définition de \( g \circ f \) est l'ensemble \( \{ x \in \Dom f \tq f(x) \in \Dom g\}\), sous-ensemble de \( \Dom f \).
\end{definition}

\begin{lemma}       \label{LEMooWBYSooFqyqQP}
	Soient deux ensembles \( A\) et \( B\) ainsi qu'une application \( f\colon A\to B\). Nous supposons qu'il existe une application \( g\colon B\to A\) telle que \( f\circ g=\id_B\) et \( g\circ f=\id_A\).

	Alors \( f\) est une bijection.
\end{lemma}

\begin{proof}
	En deux parties.
	\begin{subproof}
		\spitem[Injection]
		Supposons que \( f(a)=f(b)\). Alors en appliquant \( g\) des deux côtés, et en utilisant le fait que \( g\circ f=\id_A\), nous trouvons \( a=b\).
		\spitem[Surjection]
		Soit \( x\in B\). Posons \( a=g(x)\). Alors, en utilisant le fait que \( f\circ g=\id_B\) nous avons
		\begin{equation}
			f(a)=(f\circ g)(x)=x.
		\end{equation}
		Donc \( x\) est dans l'image de \( f\) et \( f\) est surjective.
	\end{subproof}
\end{proof}

\begin{definition}      \label{DEFooTRGYooRxORpY}
	Soit \( f\colon A\to B\) une bijection. L'\defe{application réciproque}{application réciproque} de \( f\) est la fonction
	\begin{equation}
		\begin{aligned}
			f^{-1}\colon B & \to A                                             \\
			y              & \mapsto \text{le } x\in A\text{ tel que } f(x)=y.
		\end{aligned}
	\end{equation}

	Plus généralement si \( f\colon X\to Y\) est une fonction quelconque et si \( S\subset Y\), nous notons
	\begin{equation}
		f^{-1}(S)=\{ x\in \Dom f \tq f(x)\in S \},
	\end{equation}
	et dans le cas où \( S\) est réduit à un unique élément \( y\), nous notons \( f^{-1}(y)\) au lieu de \( f^{-1}\big( \{ y \} \big)\). Si de plus \( f^{-1}(S)\) est un singleton \( x\), nous noterons \( f^{-1}(S)=x\) et non \( f^{-1}(S)=\{ x \}\).
\end{definition}

\begin{remark}	\label{REMooDifferenceElementSingleton}
	Les plus \randomGender{acharnés}{acharnées} parmi les \randomGender{lecteurs}{lectrices} se rendront compte de la différence ontologique fondamentale entre \( x\) et \( \{ x \}\), et donc des abus de notations explicitées dans la définition précédente.
\end{remark}

\begin{proposition}[\cite{MonCerveau}]		\label{PROPooBXVSooZXmwKC}
	Soit \(f \colon A\to B  \) une fonction. Si \( X\) et \( Y\) sont des parties de \( A\), alors:
	\begin{enumerate}
		\item \label{ITEMooFctInclusionInter}
		      on a l'inclusion \( f(X\cap Y) \subseteq f(X)\cap f(Y)\);
		\item \label{ITEMooFctInjEgaliteInter}
		      si de plus, \( f \) est  injective\footnote{Définition~ \ref{DefooInjSurjBij}.}, alors \( f(X\cap Y)=f(X)\cap f(Y)\);
		\item \label{ITEMooFctInjDisjonction}
		      si enfin, \( f \) est injective et \( X\) et \( Y\) sont disjoints, alors \( f(X)\cap f(Y) = \varnothing \) (autrement dit:  \( f(X) \) et \( f(Y) \) sont disjoints aussi).
	\end{enumerate}
\end{proposition}

\begin{proof}
	Prouvons d'abord \( f(X\cap Y) \subseteq f(X)\cap f(Y)\), ensuite \( f(X)\cap f(Y)\subseteq f(X\cap Y)\) dans le cas où \( f \) est une injection, et enfin le cas d'ensembles disjoints.
	\begin{subproof}
		\spitem[Première inclusion]
		%-----------------------------------------------------------
		Si \( \xi\in f(X\cap Y)\), alors il existe \( s\in X\cap Y\) tel que \( \xi=f(s)\). Étant donné que \( s\in X\), nous avons \( \xi=f(s)\in f(X)\). Et comme \( s\in Y\) nous avons aussi \( \xi=f(s)\in f(Y)\). Au final \( \xi=f(x)\in f(X)\cap f(Y)\).
		\spitem[Seconde inclusion]
		%-----------------------------------------------------------
		Si \( \xi\in f(X)\cap f(Y)\), il existe \( x\in X\) et \( y\in Y\) tels que \( \xi=f(x)\) et \( \xi=f(y)\). Par injectivité de \( f\), nous avons \( x=y\) et donc \( \xi\in f(X\cap Y)\).
		\spitem[Ensembles disjoints]
		%-----------------------------------------------------------
		Si \( f \) est injective, et si \( X\cap Y = \varnothing \), alors il n'y a aucun élément dans \( f(X\cap Y)=f(X)\cap f(Y) \).
	\end{subproof}
\end{proof}

\begin{definition}[Produit cartésien infini\cite{BIBooRNUTooNwYWtS}]	\label{DEFooTYIMooZUsYJw}
	Soit un ensemble \( I\) ainsi qu'une collection d'ensembles \( \{ A_i \}_{i\in I}\). Nous notons \( \prod_{i\in I}A_i\) l'ensemble des applications \(f \colon I\to \bigcup A_i  \) telles que \( f(i)\in A_i\).
\end{definition}

%---------------------------------------------------------------------------------------------------------------------------
\subsection{Ensemble ordonné}
\label{SUBooEnsembleOrdonne}
%---------------------------------------------------------------------------------------------------------------------------


\begin{definition}      \label{DefooFLYOooRaGYRk}
	Une \defe{relation d'ordre}{ordre}\index{relation!d'ordre} sur un ensemble \( E\) est une relation binaire\footnote{Définition \ref{DEFooRFVTooUUuFuE}.} (notée \( \leq\)) sur \( E\) telle que pour tous \( x,y,z\in E\),
	\begin{enumerate}
		\item
		      Réflexivité : \( x\leq x\)
		\item \label{ITEMooQTXOooTZXEnu}
		      Antisymétrie : \( x\leq y\) et \( y\leq x\) implique \( x=y\)
		\item
		      Transitivité : \( x\leq y\) et \( y\leq z\) implique \( x\leq z\).
	\end{enumerate}

	Pour suivre les notations de la définition \ref{DEFooRFVTooUUuFuE}, la partie \( G\) de \( E\times E\) est une relation d'ordre lorsque
	\begin{enumerate}
		\item
		      \( (x,x)\in G\) pour tout \( x\in G\),
		\item
		      Si \( (x,y)\in G\) et \( (y,x)\in G\), alors \( x=y\),
		\item
		      Si \( (x,y)\in G\) et \( (y,z)\in G\), alors \( (x,z)\in G\).
	\end{enumerate}
	Dans la suite nous n'allons plus écrire de relations binaires en détaillant l'ensemble sous-jacent.

	Lorsque nous avons un ensemble \( E\) et une relation d'ordre \( \leq\) sur \( E\), nous disons que le couple \( (E,\leq)\) est un \defe{ensemble ordonné}{ensemble ordonné}.
\end{definition}

\begin{definition}      \label{DEFooVGYQooUhUZGr}
	Un ensemble ordonné est \defe{totalement ordonné}{ordre!total}\index{ensemble ordonné!totalement ordonné} si deux éléments quelconques si deux éléments sont toujours comparables : si \( x,y\in E\) alors nous avons soit \( x\leq y\) soit \( y\leq x\). Si les éléments ne sont pas tous comparables, nous disons que l'ordre est \defe{partiel}{ordre!partiel}.
\end{definition}

\begin{example}  \label{ExeooOrdreInclusion}
	Si \( E\) est un ensemble, l'inclusion est un ordre sur l'ensemble des parties de \( E\), mais pas un ordre total parce que si \( X,Y\) sont des parties de \( E\), alors nous n'avons pas automatiquement soit \( X\subset Y\) soit \( Y\subset X\).
\end{example}

\begin{normaltext}\label{NORMooIntroIntervalleOrdonne}
	On convient assez naturellement de la notation suivante: écrire \( x \leq y \leq z \) revient à écrire \( (x \leq y)\text{ et } (y \leq z) \). Cela s'étend à un nombre (fini) quelconque d'éléments, et \randomGender{le lecteur}{la lectrice} saura interpréter correctement, nous l'espérons, des inégalités du genre \( x_1 \leq x_2 \leq \dots \leq x_n \) dans les prochains développements.

	La notion d'ordre permet d'introduire la notion d'intervalle, qui s'écrit assez simplement une fois la convention prise.
\end{normaltext}

\begin{definition}  \label{DefEYAooMYYTz}
	Soit un ensemble totalement ordonné \( (E,\leq)\). Un \defe{intervalle}{intervalle!d'un ensemble ordonné} de \( E\) est une partie \( I\) telle que tout élément compris entre deux éléments de \( I \) soit dans \( I \). En language mathématique, la partie \( I \) de \( E\) est un intervalle si
	\[
		\forall a,b\in I,\, \forall x \in E, (a\leq x\leq b \Rightarrow x\in I).
	\]
\end{definition}

\begin{definition}      \label{DEFooDNWRooTiMAzK}
	Soit un ensemble ordonné \( (E,\leq)\) et une partie \( A\) de \( E\). Un élément \( p\in E\) est un \defe{minorant}{minorant} de \( A\) si pour tout \( a\in A\), les éléments \( p\) et \( a\) sont comparables et \( p\leq a\). Si de plus, \( p \in A \), on dira que \( p \) est un \defe{minimum}{minimum!ensemble ordonné} de \( A\).

	Les notions de \defe{maximum}{maximum} et de \defe{majorant}{majorant} sont définies de façon analogue (en remplaçant \( p\leq a\) par \( a\leq p\)).
\end{definition}

\begin{proposition}\label{PROPooUniciteMinMax}
	Soit \( (E, \leq) \) un ensemble ordonné, et \( A \) un sous-ensemble de \( E \) admettant un minimum. Alors ce minimum est unique.
\end{proposition}

\begin{proof}
	Soit \( m \) le premier élément minimum de \( A \). Quel que soit \(x \in A \), on sait que \( x \) et \( m \) sont comparables, et que \( m \leq x \).

	Supposons que \( p \in A \) soit un (autre) élément minimum de \( A \). Alors, par ce fait, on a \( p \leq m \); mais par définition de \( m \), on a aussi \( m \leq p \). Ainsi, par antisymétrie\footnote{Point \ref{ITEMooQTXOooTZXEnu} de la définition \ref{DefooFLYOooRaGYRk}.}, on a \( m = p\).
\end{proof}

\begin{normaltext}\label{NORMooPlusPetitElement}
	On peut ainsi parler, lorsqu'une partie possède un minimum, de «plus petit élément» de la partie. Attention : il n'en existe pas toujours. D'innombrables exemples pourront être vus lorsque nous aurons construit \( \eQ\) et \( \eR\). Typiquement les intervalles du type \( \mathopen] a , b \mathclose[\).
\end{normaltext}


\begin{definition}[\cite{BIBooROFNooTKJKxG}]        \label{DEFooBZNRooYRPGme}
	Soit un ensemble ordonné \( (E,\leq)\). Un élément \(M\in E\) est un \defe{élément maximal}{élément maximal} de \( E\) si pour tout \( x\in E\), dès que \( M\leq x\), on a \(x=M\).

	De même, nous disons que \( m\in E\) est un \defe{élément minimal}{élément minimal} de \( E\) si pour tout \( x\in E\), nous avons \( x\leq m\Rightarrow x=m\).
\end{definition}

\begin{normaltext}    \label{NORMooVHIBooJAOsou}
	Notons qu'il n'est pas demandé à un élément maximal\footnote{Définition \ref{DEFooBZNRooYRPGme}} d'être comparable à tous les autres éléments. Si \( (E,\leq)\) n'est pas totalement ordonné, un élément maximal peut ne majorer qu'une partie de \( E\).

	Un élément maximal est plus grand que tous les éléments avec lesquels il est comparable.

	Dans un ensemble totalement ordonné, les notions d'élément maximal de \( E\) et de maximum\footnote{Définition \ref{DEFooDNWRooTiMAzK}.} de \( E\) coïncident ; dans le cas d'un ordre partiel, ces notions sont distinctes.

	Il se peut que nous parlions d'un «maximum» ou «élément maximum»  au lieu d'un «élément maximal», en particulier en utilisant le lemme de Zorn. Si vous voyez de telles choses, n'hésitez pas à me le dire.

	Par exemple, quand on applique le lemme de Zorn pour démontrer l'existence d'une base dans un espace vectoriel de dimension infinie, on obtient une famille libre qui est maximale, c'est-à-dire qui est un élément maximal dans l'ensemble, ordonné par inclusion, des familles libres de l'espace vectoriel. C'est un élément maximal, et non un maximum, car l'ensemble des familles libres d'un espace vectoriel non réduit à \( \{ 0 \}\) n'admet pas de maximum (pour l'inclusion).

	Voir \ref{THOooOQLQooHqEeDK} et \ref{NORMooREVQooEFJWta}.		% ooXKRAooNMnxHp Juste pour rendre la ligne unique.
\end{normaltext}

\begin{example}[\cite{MonCerveau}] \label{ExeooMaximalPasMajorant}
	Soit un ensemble \( E\) ainsi que \( a\neq b\) dans \( E\). Nous considérons les parties
	\begin{subequations}
		\begin{align}
			A=\{ P\subset E\tq a\in P,b\notin P \} \\
			B=\{ P\subset E\tq b\in P,a\notin P \}.
		\end{align}
	\end{subequations}
	Et enfin nous considérons l'ensemble \( F=A\cup B\), c'est-à-dire l'ensemble des parties de \( E\) qui contiennent soit \( a\) soit \( b\) mais pas les deux. Nous ordonnons partiellement \( F\) par l'inclusion.

	Dans \( (F,\subset)\), l'élément \( E\setminus\{ b \}\) est un élément maximal, mais pas un majorant.
\end{example}

\begin{definition}   \label{DEFooLJEAooBLGsiS}
	Un ensemble ordonné est \defe{bien ordonné}{bon!ordre}\index{ordre!bon ordre} si toute partie non vide possède un plus petit élément.

	Autrement dit, l'ensemble ordonné \( E\) est bien ordonné si pour toute partie non vide \( A\), il existe \( x\in A\) tel que \( x\leq y\) pour tout \( y\in A\).
\end{definition}

\begin{normaltext}  \label{NorooOrdreBienOrdonne}
	Quelques remarques.
	\begin{enumerate}
		\item
		      L'inégalité stricte (définie par: \( x<y\) si et seulement si \( x\leq y\) et \( x\neq y\)) n'est pas une relation d'ordre parce qu'elle n'est pas réflexive.
		\item
		      Nous verrons dans la remarque~\ref{REMooXOIOooHjwMvA}
		      que l'intervalle \( \mathopen[ -1 , 1 \mathclose]\) dans \( \eR\) n'est pas bien ordonné.
		\item
		      Un ensemble bien ordonné est forcément totalement ordonné parce que toutes les parties de la forme \( \{ x,y \}\) possèdent un minimum. Par conséquent \( x\) et \( y\) doivent être comparables : \( x\leq y\) ou \( y\leq x\).
	\end{enumerate}
\end{normaltext}



%---------------------------------------------------------------------------------------------------------------------------
\subsection{Lemme de Zorn}
\label{SUBooLemmeZorn}
%---------------------------------------------------------------------------------------------------------------------------



\begin{definition}[Ensemble inductif\cite{MathAgreg}]  \label{DefGHDfyyz}
	Un ensemble est \defe{inductif}{inductif} si toute partie totalement ordonnée admet un majorant.
\end{definition}


\begin{lemma}[Lemme de Zorn\cite{BIBooYDIJooWCVynX}]    \label{LemUEGjJBc}
	Tout ensemble ordonné inductif non vide admet au moins un élément maximal.
\end{lemma}
\index{lemme!de Zorn}

\begin{proposition}[\cite{BIBooZFPUooIiywbk}]       \label{PROPooFOETooWYLOeq}
	Soient un ensemble inductif \( (E,\leq)\) et \( b\in E\). Il existe un élément maximal\footnote{Définition \ref{DEFooBZNRooYRPGme}.} \( m\in E\) tel que \( b\leq m\).
\end{proposition}

\begin{proof}
	En plusieurs parties.
	\begin{subproof}
		\spitem[Un ensemble]
		Nous considérons
		\begin{equation}
			E_b=\{ x\in E\tq b\leq x \}.
		\end{equation}
		\spitem[\( E_b\) est inductif]
		Soit une partie non vide totalement ordonnée \( A \subset E_b\); démontrons que \( A \) admet un majorant dans \( E_b \). Puisque \( E_b \subset E\), par transitivité de l'inclusion, \( A \subset E \) et puisque \( E\) est inductif, la partie \( A\) admet un majorant \( m'\in E\): justifions maintenant que \( m' \in E_b \).

		Comme \( A\) est non vide, nous pouvons considérer \( x\in A\). Nous avons \( b\leq x\) parce que \( x\in A\subset E_b\). Mais comme \( m'\) est un majorant de \( A\), \( x\leq m'\). Bref, nous avons les inégalités
		\begin{equation}
			b\leq x\leq m'.
		\end{equation}
		Par transitivité de l'ordre, \( b \leq m' \) et donc \( m'\in E_b\).

		Nous avons ainsi prouvé que \( m'\) est un majorant de \( A\) contenu dans \( E_b\). Donc \( E_b\) est inductif.
		\spitem[Zorn]
		Puisque \( (E_b,\leq)\) est inductif, le lemme de Zorn \ref{LemUEGjJBc} nous indique que \( E_b\) a un élément maximal. Nous le notons \( m\).
		\spitem[\( m\) est maximal dans \( E\)]
		Supposons avoir un élément \( a\in E\) tel que \( m\leq a\). Comme \( m \in E_b \), nous avons les inégalités
		\begin{equation}
			b\leq m\leq a,
		\end{equation}
		et donc \( a\in E_b\) par transitivité. Mais \( m\) est maximal dans \( E_b\), donc \( a \leq m \), d'où \( a=m\) par antisymétrie.
	\end{subproof}
\end{proof}


\begin{proposition}[\cite{MonCerveau}]	\label{PROPooRHNFooHFUOEx}
	Soit \( A \) et \( B \) deux ensembles non vides, et une application surjective \(f \colon A\to B  \). Il existe une application \(\alpha \colon B\to A  \) telle que
	\begin{equation}
		\alpha(b)\in f^{-1}(b)
	\end{equation}
	pour tout \( b\in B\).
\end{proposition}

\begin{proof}
	Nous considérons l'ensemble
	\begin{equation}
		\mA=\{ (X,\alpha)\tq  X\subset B, \alpha \colon X\to A  \tq \alpha(x)\in f^{-1}(x)\forall x\in X \}
	\end{equation}
	que nous ordonnons par inclusion pour \( X\) et par extension pour \( \alpha\).

	D'abord, \( \mA\) est non vide: considérons un élément \( a\in A\) et posons \( b=f(a)\); en posant \( X=\{ b \}\) et \(\alpha \colon \{ b \}\to A  \) donné par \( \alpha(b)=a\), le couple \( (X,\alpha)\) est bien un élément de \( \mA\).

	Ensuite, \( \mA\) est inductif. En effet, soit une partie \( \mF=\{ (X_i,\alpha_i) \}_{i\in I}\) totalement ordonnée de \( \mA\). Nous posons \( X=\bigcup_iX_i \) et \(\alpha \colon X\to A  \) par \( \alpha(b)=\alpha_i(b)\) dès que \( b\in X_i\). L'élément \( (X,\alpha)\) est dans \( \mA\) et majore \( \mF\).

	Nous pouvons donc utiliser le lemme de Zorn\footnote{Lemme \ref{LemUEGjJBc}.}, et affirmer que l'ensemble \( \mA\) contient un maximum \( (Y,\beta)\). Montrons alors par l'absurde que \( Y=B\).

	Supposons que \( Y\neq B\). Alors il existe \( b\in B\setminus Y\). Comme \(f \colon A\to B  \) est surjective, il existe \( a\in A\) tel que \( f(a)=b\)\footnote{Toute la subtilité de l'axiome du choix. Si on a un élément \( b\), on a le droit de considérer un élément \( a\in f^{-1}(b)\), mais on n'a pas le droit de considérer un tel élément de \( A\) pour tout élément de \( B\).}. Nous posons alors \( Y'=Y\cup\{ b \}\) et
	\begin{equation}
		\beta'(x)=\begin{cases}
			\beta(x) & \text{si } x\in B \\
			a        & \text{si }x=b.
		\end{cases}
	\end{equation}
	Le couple \( (Y',\beta')\) majore \( (Y,\beta)\) et nous avons une contradiction avec la maximalité de \( Y\).
\end{proof}



%-------------------------------------------------------
\subsection{Quelques relations ensemblistes}
\label{SUBooRelEnsemblistes}
%----------------------------------------------------

\begin{lemma}[\cite{MonCerveau}]	\label{LEMooIJAMooDQfGtM}
	Si \( B\subset A\), alors \( X\setminus A\subset X\setminus B\).
\end{lemma}

\begin{proof}
	Si \( x\in X\setminus A\), alors \( x\) n'est pas dans \( A\). Si \( x\) était dans \( B\), nous aurions \( x\in B\subset A\). Donc \( x\) n'est pas dans \( B\) et donc \( x\in X\setminus B\).
\end{proof}

\begin{lemma}[Quelques relations ensemblistes]       \label{LEMooHRKAooRskzQL}
	Soient \( A,B,C\subset X\). Nous avons
	\begin{enumerate}
		\item	\label{ITEMooIXUAooXDPtWj}
		      \( X\setminus (A\cap B)=(X\setminus A)\cup(X\setminus B)\).
		\item       \label{ITEMooQCGUooKnWfBo}
		      \( X\setminus (A\cup B)=(X\setminus A)\cap(X\setminus B)\).
		\item       \label{ITEMooXWKCooUASxlh}
		      \( A\cap(B\setminus C)=(A\cap B)\setminus C\).
		\item		\label{ITEMooRXJOooUYPaCp}
		      \( A\cap(X\setminus B)=A\setminus B\).
		\item		\label{ITEMooYYWAooLnADjR}
		      Si \( I\) est un ensemble et que \( A_i\subset X\) pour tout \( i\in I\) alors
		      \begin{equation}
			      X\setminus\big( \bigcap_{i\in I}A_i \big)=\bigcup_{i\in I}(X\setminus A_i).
		      \end{equation}
	\end{enumerate}
\end{lemma}

\begin{proof}
	En plusieurs parties.
	\begin{subproof}
		\spitem[Pour \ref{ITEMooIXUAooXDPtWj}]
		%-----------------------------------------------------------
		Deux inclusions.
		\begin{subproof}
			\spitem[Dans un sens]
			%-----------------------------------------------------------
			Si \( x\in X\setminus(A\cap B)\). Il y a deux possibilités : soit \( x\in A\), soit \( x\not\in A\). Si \( x\in A\), alors \( x\not\in B\), sinon nous aurions \( x\in A\cap B\). Donc \( x\in X\setminus B\). Le cas \( x\not\in A\) est similaire.

			\spitem[Dans l'autre sens]
			%-----------------------------------------------------------
			Soit \( x\in X\setminus A\). Vu que \( A\cap B\subset A\), le lemme \ref{LEMooIJAMooDQfGtM} dit que \( x\in X\setminus (A\cap B)\).
		\end{subproof}
		\spitem[Pour \ref{ITEMooQCGUooKnWfBo}]
		%-----------------------------------------------------------
		Deux inclusions.
		\begin{subproof}
			\spitem[Premier sens]
			%-----------------------------------------------------------
			Soit \( x\in X\setminus(A\cup B)\). Si \( x\) était dans \( A\), il serait dans \( A\cup B\). Or il n'est pas dans \( A\cup B\) donc \( x\) n'est pas dans \( A\). On montre de même que \( x\in X\setminus B\).

			\spitem[Dans l'autre sens]
			%-----------------------------------------------------------
			Soit \( x\in (X\setminus A)\cap(X\setminus B)\). Supposons que \( x\in A\cup B\), et disons \( x\in A\) pour fixer les idées. Nous savons que \( x\in X\setminus A\); contradiction. Donc \( x\not\in A\). Nous obtenons la même contradiction si \( x\in B\).
		\end{subproof}
		\spitem[Pour \ref{ITEMooXWKCooUASxlh}]
		%-----------------------------------------------------------
		Deux inclusions.
		\begin{subproof}
			\spitem[Premier sens]
			%-----------------------------------------------------------
			Supposons que \( x\in A\cap(B\setminus C)\). D'une part \( x\in A\). Or \( x\in B\setminus C\), donc \( x\in B\). Nous avons déjà \( x\in A\cap B\). Mais \( x\in B\setminus C\) implique \( x\not\in C\). Donc \( x\in (A\cap B)\setminus C\).

			\spitem[Deuxième sens]
			%-----------------------------------------------------------
			Nous avons :
			\begin{enumerate}
				\item		\label{ITEMooMGGEooVRxrB}
				      \( x\in A\cap B\) donc \( x\in A\).
				\item		\label{ITEMooOIQCooUAwng}
				      \( x\in A\cap B\) donc \( x\in B\).
				\item	\label{ITEMooCTHOooBLcsZv}
				      \( x\in (A\cap B)\setminus C\) donc \( x\not\in C\).
			\end{enumerate}
			Les points \ref{ITEMooOIQCooUAwng} et \ref{ITEMooCTHOooBLcsZv} font \( x\in B\setminus C\). En ajoutant \ref{ITEMooMGGEooVRxrB} nous avons \( x\in A\cap(B\setminus C)\).
		\end{subproof}

		\spitem[Pour \ref{ITEMooRXJOooUYPaCp}]
		%-----------------------------------------------------------
		C'est un cas particulier de \ref{ITEMooXWKCooUASxlh} avec \( B=X\).

		\spitem[Pour \ref{ITEMooYYWAooLnADjR}]
		%-----------------------------------------------------------
		En deux inclusions.
		\begin{subproof}
			\spitem[Dans un sens]
			%-----------------------------------------------------------
			Soit \( x\in X\setminus(\bigcap_{i\in I}A_i\). Vu que \( x\) n'est pas dans \( \bigcap_{i\in I}A_i\), il existe \( j\in I\) tel que \( x\not\in A_j\), et donc tel que \( x\in X\setminus A_j\).
			\spitem[Dans l'autre sens]
			%-----------------------------------------------------------
			Si \( x\in\bigcup_{i\in I}(X\setminus A_i)\), alors il existe \( j\in I\) tel que \( x\in X\setminus A_j\). Donc \( x\) n'est pas dans \( A_j\) et à fortiori pas dans \( \bigcap_{i\in I}A_i\).
		\end{subproof}
	\end{subproof}
\end{proof}

\begin{lemma}[\cite{MonCerveau}]	\label{LEMooBGQRooCVLGsh}
	Si \( A\subset B\) et si \(f \colon B\to B   \) est injective, alors
	\begin{equation}		\label{EQooHLEHooADUbJO}
		f(A)=f(B)\setminus f(B\setminus A).
	\end{equation}
\end{lemma}

\begin{proof}
	Les deux inclusions.
	\begin{subproof}
		\spitem[\( \subset\)]
		%-----------------------------------------------------------
		Soit \( a\in A\). Alors \( a\in A\subset B\), de telle sorte que \( f(a)\in f(B)\). Mais \( a\) n'est pas dans \( B\setminus A\), disjoint de \( A \), et comme \( f\) est injective, grâce à la propriété \ref{PROPooBXVSooZXmwKC}\ref{ITEMooFctInjDisjonction}, \( f(A)\cap f(B\setminus A)=\emptyset\). Comme \( f(a)\in f(A)\), nous avons \( f(a)\notin f(B\setminus A)\).

		\spitem[Dans l'autre sens]
		%-----------------------------------------------------------
		Suppose \( y\in f(B)\setminus f(B\setminus A)\). Vu que \( y\in f(B)\), il existe \( b\in B\) tel que \( y=f(b)\). Mais comme \( y\notin f(B\setminus A)\), nous avons \( b\notin B\setminus A\). Donc, \( b\in B \cap A\), d'où \(b \in A \) et par suite \( f(b)\in f(A)\).
	\end{subproof}
\end{proof}

\begin{lemma}[\cite{BIBooQHFRooOYlEiR}]		\label{LEMooHPUNooPmViwi}
	Pour tous ensembles \( A\), \( B\), \( C\) nous avons
	\begin{gather}
		A\cap(B\cup C)=(A\cap B)\cup (A\cap C);\\ \text{et }
		A\cup(B\cap C)=(A\cup B)\cap (A\cup C).
	\end{gather}
\end{lemma}

\begin{proof}
	Nous démontrons la première égalité d'ensembles par double inclusion.
	\begin{subproof}
		\spitem[Inclusion dans un sens]
		%-----------------------------------------------------------
		Nous supposons que \( x\in A\cap(B\cup C)\). Il y a deux possibilités : soit \( x\) est dans \( A\cap B\) soit non. Si \( x\in A\cap B\) c'est bon. Sinon, nous allons prouver que \( x\in A\cap C\).

		Vu que \( x\in A\) et \( x\not\in A\cap B\), nous avons \( x\not\in B\). Mais \( x\in B\cup C\). Donc \( x\in C\). Nous avons prouvé que \( x\in A\cap C\).

		\spitem[Inclusion inverse]
		%-----------------------------------------------------------
		Soit \( x\in (A\cap B)\cup(A\cap C)\). Nous allons prouver que \( x\in A\cap(B\cup C)\).
		\begin{subproof}
			\spitem[Si \( x\in A\cap B\)]
			%-----------------------------------------------------------
			Nous supposons que \( x\in A\cap B\). Directement \( x\in A\). Et comme \( x\in B\), nous avons aussi \( x\in B\cup C\). Donc \( x\in A\cap(B\cup C)\).
			\spitem[Si \( x\in A\cap C\)]
			%-----------------------------------------------------------
			Même raisonnement.
		\end{subproof}
	\end{subproof}
	L'autre égalité d'ensembles se démontre de manière similaire (quoique plus simple).
	\begin{subproof}
		\spitem[Inclusion dans un sens]
		%-----------------------------------------------------------
		Si \( x\in A\cup(B\cap C)\), alors, il est dans \( A \), ou bien il est à la fois dans \( B \) et dans \( C \).
		\begin{itemize}
			\item
			      Si \( x \in A\), alors il est aussi dans  \( A\cup B\), et dans \( A\cup C\); donc il se trouve dans \( (A\cup B)\cap (A\cup C) \).
			\item
			      Si \( x \in B \), alors il est dans  \( A\cup B\). Si, en même temps, \( x \in C \), alors il est dans \( A\cup C\). Donc il se trouve dans l'intersection \( (A\cup B)\cap (A\cup C) \).
		\end{itemize}
		\spitem[Inclusion inverse]
		%-----------------------------------------------------------
		Soit \( x\in (A\cup B)\cap(A\cup C)\). Il est donc à la fois dans \( A\cup B \) et dans \(A\cup C \).
		\begin{itemize}
			\item
			      Si  \( x \in A\), alors il est bien dans \( A\cup(B\cap C)\).
			\item
			      Sinon, il doit se trouver à la fois dans \( B \) et dans \( C \), car dans le cas contraire il ne pourrait pas se trouver dans les unions ci-dessus. Ainsi \( x \in B \cap C \), et donc  \( x\in A\cup(B\cap C)\).
		\end{itemize}
	\end{subproof}
\end{proof}


\begin{lemma}       \label{LemPropsComplement}
	Quelques propriétés à propos des complémentaires\footnote{Définition \ref{DEFooJLBVooMyCQMO}.}. Si \( E\) est un ensemble et si \( A\) et \( B\) sont des sous-ensembles de \( E\), nous avons
	\begin{enumerate}
		\item       \label{ITEMooComplementDisjoint}
		      \(A \cap ( E \setminus A) = \varnothing \).
		\item       \label{ITEMooComplementPartition}
		      \( A \) et \( E \setminus A \) forment une partition\footnote{Définition~\ref{DefPartitionEnsembles}.} de \( E \).
		\item       \label{ITEMooWIYBooKMLVnZ}
		      \( \complement \complement A =A\), en d'autres termes, \( E\setminus(E\setminus A)=A\).
		\item       \label{ItemLemPropComplementiii}
		      \( A\setminus B=A\cap\complement B\).
		\item       \label{ITEMooNHDUooWtURqQ}
		      \( (A\setminus B)^c=A^c\cup B\).
		\item       \label{ITEMooTBWKooTChOmC}
		      \( A^c\setminus B^c=B\setminus A\).
	\end{enumerate}
\end{lemma}

\begin{proof}
	Point par point.
	\begin{subproof}
		\spitem[Pour \ref{ITEMooComplementDisjoint}]
		%-----------------------------------------------------------
		Par définition même du complémentaire, on ne peut avoir en même temps \( x \in A \) et \(x \notin A \).

		\spitem[Pour \ref{ITEMooComplementPartition}]
		%-----------------------------------------------------------
		On a clairement \( A \subset E \) et \( E \setminus A \subset E \). Le point précédent nous justifie que l'intersection est vide. On a aussi \( A \cup ( E \setminus A) \subset E \) : prouvons l'inclusion inverse.

		Si \( x \in E \), alors ou bien \( x \in A \), ou bien \( x \not\in A\), et donc par définition du complémentaire, \( x \in E \setminus A \).

		\spitem[Pour \ref{ITEMooWIYBooKMLVnZ}]
		%-----------------------------------------------------------
		On peut faire une preuve avec une double inclusion.
		\begin{subproof}
			\spitem[Si \( x\in E\setminus(E\setminus A)\)]
			%-----------------------------------------------------------
			Alors  \( x\not\in E\setminus A \). Par l'absurde, si \( x\not\in A\), alors \( x \in E \setminus A \), ce qui est contradictoire. Ainsi, la possibilité \( x\not\in A\) est exclue et donc \( x\in A\).

			\spitem[Si \( x\in A\)]
			%-----------------------------------------------------------
			Alors \( x\not\in (E\setminus A) \).  c'est à dire \( x\in E\setminus B=E\setminus(E\setminus A )\).
		\end{subproof}

		Une autre preuve consiste à utiliser le point \ref{ITEMooComplementPartition} d'abord avec \( A \), ensuite avec \( E \setminus A\). Dans les deux cas, on partitionne \( E \) en deux; l'un des ensembles est \( E \setminus A \), l'autre est respectivement \( A \) et \( E \setminus (E \setminus A) \), qui sont dès lors nécessairement égaux.

		\spitem[Pour \ref{ItemLemPropComplementiii}]
		En deux parties.
		\begin{subproof}
			\spitem[Si \( x\in A\setminus B\)]
			%-----------------------------------------------------------
			Alors deux choses :
			\begin{enumerate}
				\item
				      \( x\in A\).
				\item
				      \( x\not\in B\), c'est à dire \( x\in E\setminus B\).
			\end{enumerate}
			Donc \( x\in A\cap(E\setminus B)\).

			\spitem[Si \( x\in A\cap(E\setminus B)\)]
			%-----------------------------------------------------------
			Alors deux choses :
			\begin{enumerate}
				\item
				      \( x\in A\).
				\item
				      \( x\in E\setminus B\), ce qui implique\footnote{Je vous laisse disserter sur le fait que ce soit une implication ou juste une reformulation, vu que sous-entendu \( x\in E\).} que \( x\not\in B\).
			\end{enumerate}
			Donc \( x\in A\cap (E\setminus B)\).
		\end{subproof}
		\spitem[Pour \ref{ITEMooNHDUooWtURqQ}]
		Il faut le faire en deux inclusions.
		\begin{subproof}
			\spitem[\( (A\setminus B)^c\subset A^c\cup B\)]
			Supposons que \( x\in(A\setminus B)^c\). Si \( x\in A^c\), c'est bon. Supposons que \( x\) n'est pas dans \( A^c\), et montrons que \( x\in B\). Le fait que \( x\) ne soit pas dans \( A^c\) signifie que \( x\in A\). Si \( x\) n'était pas dans \( B\), alors \( x\) serait dans \( A\setminus B\), ce qui est contraire à l'hypothèse. Donc \( x\in B\).
			\spitem[\( A^c\cup B\subset (A\setminus B)^c\)]
			Supposons d'abord que \( x\in A^c\). Comme \( A\setminus B\subset A\), si \( x\in A^c\), alors \( x\in (A\setminus B)^c\).

			Si \( x\in B\), alors \( x\) n'est pas dans \( A\setminus B\) et donc \( x\) est dans \( (A\setminus B)^c\).
		\end{subproof}
		\spitem[Pour \ref{ITEMooTBWKooTChOmC}]
		Pour cette égalité, nous séparons \( 4\) cas suivant que \( x\) est dans \( A\) ou \( B\) ou non. Bref, nous écrivons la table de vérité :
		\begin{equation}
			\begin{array}{|c|c|c|c|c|}
				\hline%
				A                & 1 & 1 & 0 & 0 \\
				\hline%
				B                & 1 & 0 & 1 & 0 \\
				\hline%
				A^c\setminus B^c & 0 & 0 & 1 & 0 \\
				\hline%
				B\setminus A     & 0 & 0 & 1 & 0 \\
				\hline%
			\end{array}
		\end{equation}
		Les deux dernières lignes étant égales, nous avons l'égalité d'ensembles annoncée.
	\end{subproof}
\end{proof}

\begin{lemma}[\cite{MonCerveau}]	\label{LEMooMDTPooAhmYqv}
	Soient des parties \( (A_j)_{j\in I}\) de \( X\). Nous avons
	\begin{equation}
		\bigcup_{j\in I}A_j=X\setminus\Big( \bigcap_{j\in I}(X\setminus A_j) \Big).
	\end{equation}
\end{lemma}

\begin{proof}
	En deux inclusions.
	\begin{subproof}
		\spitem[Sens direct]
		%-----------------------------------------------------------
		Si \( x\in A_i\), alors \( x\not\in (X\setminus A_i)\). Or \( \bigcap_{j\in I}(X\setminus A_j)\subset (X\setminus A_i)\), donc
		\begin{equation}
			x\in  X\setminus\Big( \bigcap_{j\in I}(X\setminus A_j) \Big).
		\end{equation}

		\spitem[Dans l'autre sens]
		%-----------------------------------------------------------
		Dire que \( x\in  X\setminus\Big( \bigcap_{j\in I}(X\setminus A_j) \Big)\), c'est dire que \( x\) n'est pas dans l'intersection des \( X\setminus A_i\). Il existe donc un \( i\in I\) tel que \( x\notin (X\setminus A_i)\), c'est à dire tel que \( x\in A_i\). Donc \( x\in\bigcup_{i\in I}A_i\).
	\end{subproof}
\end{proof}

\begin{lemma}[\cite{MonCerveau}]	\label{LEMooPWUHooMBKquL}
	Soient des ensembles \( \{ A_i \}_{i\in I}\) et \( \{ B_j \}_{j\in J}\). Alors
	\begin{equation}
		\Big( \bigcup_{i\in I}A_i \Big)\cap \Big( \bigcup_{k\in J}B_j \Big)=\bigcup_{(i,j)\in I\times J}(A_i\cap B_j).
	\end{equation}
\end{lemma}

\begin{proof}
	L'inclusion dans un sens. Si \( x\in (\bigcup_iA_i)\cap (\bigcup_{j}B_j)\), alors il existe \( i\in I\) et \( j\in J\) tels que \( x\in A_i\) et \( x\in B_j\), c'est à dire \( x\in A_i\cap B_j\).

	Pour l'autre sens, si \( x\in A_{i_0}\cap B_{j_0}\), alors \( x\in \bigcup_{i\in I}A_i \) et \( x\in \bigcup_{j\in J}B_j\).
\end{proof}

\begin{definition}[différence symétrique]    \label{DefBMLooVjlSG}
	Si \( A\) et \( B\) sont des ensembles, l'ensemble \( A\Delta B\)\nomenclature[T]{\( A\Delta B\)}{différence symétrique} est la \defe{différence symétrique}{ensemble!différence symétrique} d'ensembles :
	\begin{equation}
		A\Delta B=(A\cup B)\setminus(A\cap B).
	\end{equation}
	C'est l'ensemble des éléments étant soit dans \( A\) soit dans \( B\) mais pas dans les deux, ni dans aucun des deux.
\end{definition}

\begin{remark}\label{REMooTableVeriteDiffSym}
	La table de vérité de \( A\Delta B\) est intéressante :
	\begin{equation}        \label{EQooOJBOooKkKbYp}
		\begin{array}{|c|c|c|c|c|}
			\hline%
			A         & 1 & 1 & 0 & 0 \\
			\hline%
			B         & 1 & 0 & 1 & 0 \\
			\hline%
			A\Delta B & 0 & 1 & 1 & 0 \\
			\hline%
		\end{array}
	\end{equation}
	La deuxième colonne signifie que si \( x\in A\) et \( x\in B^c\), alors \( x\in A\Delta B\).
\end{remark}


\begin{lemma}[\cite{BIBooRFFSooROjnXs}]   \label{LemCUVoohKpWB}
	Si \( A\) et \( B\) sont des parties d'un ensemble, nous avons
	\begin{equation}		\label{EQooMWTNooKXfFvU}
		A\Delta B=(A\setminus B)\cup (B\setminus A),
	\end{equation}
	et aussi
	\begin{enumerate}
		\item       \label{ItemVUCooHAztC}
		      \( A^c\Delta B^c=A\Delta B\).
		      \item\label{ItemVUCooHAztCii}
		      \( (A\Delta B)\Delta B=A\).
		\item       \label{ITEMooSPZXooPTgisP}
		      \( (A\Delta B)^c=(A^c\cap B^c)\cup(A\cap B)\).
		\item       \label{ITEMooSMXWooYcWsRC}
		      Associativité : \( A\Delta (B\Delta C)=(A\Delta B)\Delta C\).
	\end{enumerate}
\end{lemma}

\begin{proof}
	Pour \eqref{EQooMWTNooKXfFvU}, il suffit de réfléchir un peu: \( A\Delta B \) contient précisément les éléments de \( A \) qui ne sont pas dans \( B \), ainsi que les éléments de \( B \) qui ne sont pas dans \( A \). On peut d'ailleurs voir que la table de vérité de \( (A\setminus B)\cup (B\setminus A)\) est la même que \eqref{EQooOJBOooKkKbYp}.

	Et pour le reste, c'est parti.
	\begin{subproof}
		\spitem[Pour \ref{ItemVUCooHAztC}]
		Nous rappelons l'égalité \( X^c\setminus Y^c=Y\setminus X\) du lemme \ref{LemPropsComplement}\ref{ITEMooTBWKooTChOmC}. De là nous écrivons
		\begin{equation}
			A^c\Delta B^c=(A^c\cup B^c)\setminus(A^c\cap B^c)=(A^c\cap B^c)^c\setminus(A^c\cup B^c)^c=(A\cup B)\setminus (A\cap B)=A\Delta B.
		\end{equation}
		\spitem[Pour \ref{ItemVUCooHAztCii}]
		Ici, il faut remarquer que \( (A\Delta B)\cup B=A\cup B\) et que \( (A\Delta B)\cap B=B\setminus A\), donc
		\begin{align}
			(A\Delta B)\Delta B & = ((A\Delta B)\cup B) \setminus ((A\Delta B)\cap B) \\
			                    & = (A\cup B)\setminus (B\setminus A).
		\end{align}
		Donc tous les éléments de \( (A\Delta B)\Delta B \) sont dans \( A \cup B \) mais pas dans \( B \setminus A \).
		\begin{itemize}
			\item
			      Un élément de \( A \cup B \) est dans \( A \) ou bien dans \( B \); s'il n'est pas dans \( B \setminus A \), il se trouve nécessairement dans \( A \).
			\item
			      Réciproquement, tous les éléments de \( A \) sont dans \( A \cup B \) et ne sont pas dans \( B \setminus A \).
		\end{itemize}
		Ainsi, on a bien \( (A\Delta B)\Delta B = A \).
		\spitem[Pour \ref{ITEMooSPZXooPTgisP}]
		Il s'agit d'utiliser le lemme \ref{LemPropsComplement}\ref{ITEMooNHDUooWtURqQ} :
		\begin{subequations}
			\begin{align}
				(A\Delta B)^c & =\Big( (A\cup B)\setminus (A\cap B) \Big)^c \\
				              & =(A\cup B)^c  \cup(A\cap B)                 \\
				              & =(A^c\cap B^c)\cup(A\cap B).
			\end{align}
		\end{subequations}
		\spitem[Pour l'associativité \ref{ITEMooSMXWooYcWsRC}]
		Nous écrivons les tables de vérités selon que \( x\) est dans \( A\), \( B\), \( C\) ou non. D'abord
		\begin{equation}
			\begin{array}{|c|c|c|c|c|c|c|c|c|}
				A                   & 1 & 1 & 1 & 1 & 0 & 0 & 0 & 0 \\
				B                   & 1 & 1 & 0 & 0 & 1 & 1 & 0 & 0 \\
				C                   & 1 & 0 & 1 & 0 & 1 & 0 & 1 & 0 \\
				\hline%
				B\Delta C           & 0 & 1 & 1 & 0 & 0 & 1 & 1 & 0 \\
				\hline%
				A\Delta (B\Delta C) & 1 & 0 & 0 & 1 & 0 & 1 & 1 & 0
			\end{array}
		\end{equation}
		La quatrième ligne s'écrit sur le modèle de \eqref{EQooOJBOooKkKbYp} en regardant les deuxièmes et troisièmes lignes. La dernière ligne se fait avec la première et la quatrième.

		L'autre table de vérité se fait de la même manière :
		\begin{equation}
			\begin{array}{|c|c|c|c|c|c|c|c|c|}
				A                   & 1 & 1 & 1 & 1 & 0 & 0 & 0 & 0 \\
				B                   & 1 & 1 & 0 & 0 & 1 & 1 & 0 & 0 \\
				C                   & 1 & 0 & 1 & 0 & 1 & 0 & 1 & 0 \\
				\hline%
				A\Delta B           & 0 & 0 & 1 & 1 & 1 & 1 & 0 & 0 \\
				\hline%
				(A\Delta B)\Delta C & 1 & 0 & 0 & 1 & 0 & 1 & 1 & 0
			\end{array}
		\end{equation}
		Puisque les lignes pour \( A\Delta (B\Delta C)\) et pour \( (A\Delta B)\Delta C\) sont identiques, nous avons égalité.
	\end{subproof}
\end{proof}

%---------------------------------------------------------------------------------------------------------------------------
\subsection{Relations d'équivalence}
\label{appEquivalence}
%---------------------------------------------------------------------------------------------------------------------------

\begin{definition}  \label{DefHoJzMp}
	Si \( E\) est un ensemble, une \defe{relation d'équivalence}{relation d'équivalence} sur \( E\) est une relation binaire\footnote{Définition \ref{DEFooRFVTooUUuFuE}.} \( \sim\) qui est à la fois
	\begin{description}
		\item[réflexive]  \( x\sim x\) pour tout \( x\in E\),
		\item[symétrique] \( x\sim y\) si et seulement si \( y\sim x\);
		\item[transitive] si \( x\sim y\) et \( y\sim z\), alors \( x\sim z\).
	\end{description}
\end{definition}

\begin{definition}      \label{DEFooRHPSooHKBZXl}
	Si \( E\) est un ensemble et si \( \sim\) est une relation d'équivalence sur \( E\), alors nous notons \( E/\sim\) l'\defe{ensemble quotient}{ensemble quotient}, c'est-à-dire l'ensemble des classes d'équivalence dans \( E\). Un élément de \( E/\sim\) est de la forme
	\begin{equation}
		[a]=\{ x\in E\tq x\sim a \}.
	\end{equation}
	Les éléments de \( E/\sim\) sont donc des sous-ensembles de \( E \).
\end{definition}

\begin{lemma}		\label{LEMooEgaliteClassesGen}
	Soit un ensemble \( E\) et une relation d'équivalence \( \sim\). Pour \( a,b\in E\), nous avons \( [a]=[b]\) si et seulement si \( a\sim b\).
\end{lemma}

\begin{proof}
	En deux parties.
	\begin{subproof}
		\spitem[\( \Rightarrow\)]
		Nous supposons que \( [a]=[b]\). Par réflexivité, \( a\sim a\) et nous avons \( a\in [a]=[b]\). Mais \( a\in [b]\) signifie \( a\sim b\), ce qu'il fallait.
		\spitem[\( \Leftarrow\)]
		Nous supposons que \( a\sim b\), et nous démontrons que \( [a]\subset [b]\). Si \( x\in [a]\), alors \( x\sim a\). Mais \( a\sim b\). Donc \( x\sim a\sim b\), ce qui implique \( x\sim b\) par transitivité. Or dire \( x\sim b\) implique \( x\in [b]\). Pour l'inclusion \( [b]\subset [a]\), il suffit de réécrire: on vous laisse faire.
	\end{subproof}
\end{proof}

\begin{example}     \label{EXooYDRVooHsANlC}
	Sur l'ensemble de tous les polygones du plan, la relation «a le même nombre de côtés» est une relation d'équivalence. Plus précisément, si \( P\) et \( Q\) sont deux polygones, nous disons que \( P\sim Q\) si et seulement si \( P\) et \( Q\) ont le même nombre de côtés. C'est une relation d'équivalence :
	\begin{itemize}
		\item
		      un polygone \( P\) a toujours le même nombre de côtés que lui-même : \( P\sim P\);
		\item
		      si \( P\) a le même nombre de côtés que \( Q\) (\( P\sim Q\)), alors \( Q\) a le même nombre de côtés que \( P\) (\( Q\sim P\));
		\item
		      si \( P\) a le même nombre de côtés que \( Q\) (\( P\sim Q\)) et que \( Q\) a le même nombre de côtés que \( R\) (\( Q\sim R\)), alors \( P\) a le même nombre de côtés que \( R\) (\( P\sim R\)).
	\end{itemize}
\end{example}

\begin{example}\label{EXooProjCanoniqueEquivAppl}
	Soit \( f\) une application entre deux ensembles \( E\) et \( F\). Nous définissons une relation d'équivalence sur \( E\) par
	\begin{equation}
		x\sim y\Leftrightarrow f(x)=f(y).
	\end{equation}
	Nous notons par \( \pi\colon E\to E/\sim\) la projection canonique. L'application
	\begin{equation}
		\begin{aligned}
			g\colon E/\sim & \to F        \\
			[x]            & \mapsto f(x)
		\end{aligned}
	\end{equation}
	est bien définie et injective. Elle n'est pas surjective tant que \( f\) ne l'est pas. La \defe{décomposition canonique}{canonique!décomposition}\index{décomposition!canonique} de \( f\) est
	\begin{equation}
		f=g\circ\pi.
	\end{equation}
\end{example}

%-----------------------------------------------------------
\subsection{Relations ternaires, opérations}
\label{SUBooRelationsTernaires}
%-----------------------------------------------------------

\begin{normaltext}	\label{NORMooRelationsTernaires}
	Comme pour les fonctions construites sur base des relations binaires, on peut en construire pour les relations ternaires\footnote{Voire des relations avec plus d'ensembles, mais nous n'en avons pas (encore) besoin.}.
\end{normaltext}

\begin{normaltext}	\label{NORMooFonctionDeuxVariables}
	Soit \( G \) une relation ternaire sur les ensembles  \( E_1,\ E_2, \ E_3 \). Supposons que, étant donnés \(x_1 \in E_1 \) et \(x_2 \in E_2 \), il existe un unique \(x_3 \in E_3 \) tel que \( (x_1, x_2, x_3) \in G \). On définit alors la fonction
	\begin{equation}
		\begin{aligned}
			f\colon E_1 \times E_2 & \to E_3                                          \\
			(x_1, x_2)             & \mapsto \text{le }x_3\text{ dans la relation }G.
		\end{aligned}
	\end{equation}
\end{normaltext}

\begin{definition}  \label{DEFooOperation}
	On appelle \defe{opération}{opération sur un ensemble} sur l'ensemble \( E \) une fonction définie sur \( E \times E \) et à valeurs dans  \( E \). Cette opération peut être notée sous forme préfixe (\( f(x, y) \) ) ou sous forme infixe, c'est-à-dire entre les deux éléments que l'on opère ( \( x * y \) ).
 
	Parfois, on dit aussi que \( * \) est une \defe{loi de composition interne}{composition, loi interne} sur \( E \), ou encore que \( E \), muni de l'opération \( * \), est un \defe{magma}{magma}.
\end{definition}

\begin{definition}[neutres dans un magma]	\label{DEFooMagmaNeutre}
	Soit \( E \) un ensemble muni d'une opération \( *\colon E\times E\to E\). On dit d'un élément \( e \in E \) qu'il est:
 	\begin{enumerate}
 		\item
   			\defe{neutre à gauche}{magma!élément neutre!à gauche} si, pour tout \( x \in E \), on ait \( e*x = x \);
      		\item
			\defe{neutre à droite}{magma!élément neutre!à droite} si, pour tout \( x \in E \), on ait \( x*e = x \);
   		\item
     			\defe{neutre}{magma!élément neutre} s'il est neutre à gauche et à droite.
	\end{enumerate}
\end{definition}

\begin{propositionDef}		\label{PROPooDEFMagmaUnifere}
	Soit \( (E, *) \) un magma admettant un neutre à gauche \( e \in E \) et un neutre à droite \( n \in E \). Alors:
 	\begin{enumerate}
  		\item
    			on a \( e = n \);
       		\item
	 		il n'y a pas d'autre neutre (à gauche ou à droite) que \( e \).
	\end{enumerate}
 	Ainsi, l'élément \( e \) est appelé \defe{neutre}{magma!élément neutre} du magma \( (E, *) \) et on dit que  \( E \) est un \defe{magma unifère}{magma!unifère}.
\end{propositionDef}

\begin{proof}
	On sait que \( e*n \) vaut \( n \), car \( e \) est neutre à gauche, mais vaut aussi \( e \) car \( n \) est neutre à droite. Bref, on a
 	\[
  		n = e*n = e.
	\]
	Si \( e' \) était un autre neutre à gauche, alors \( e' = e' * n = n = e \): la première égalité venant du fait que \( n \) est neutre à droite, la deuxième que \( e' \) est neutre à gauche, la troisième de l'égalité ci-dessus. De la même façon, si \( n' \) était un autre neutre à droite, alors \( n' = e*n' = e = n \).
\end{proof}

\begin{definition}[magma commutatif]	\label{DEFooMagmaCommutatif}
	Soit \( E \) un ensemble muni d'une opération \( *\colon E\times E\to E\). On dit que \( E \) est un \defe{magma commutatif}{magma!commutatif} si pour tout \( x, y \in E \), on a \( x * y = y * x\).
\end{definition}

\begin{definition}[élément régulier\cite{BIBooYNNEooAunyKF}]        \label{DEFooIJIEooZaAdSs}
	Soit un ensemble \( E\) muni d'une opération \( *\colon E\times E\to E\). Un élément \( s\in E\) est \defe{régulier à gauche}{élément régulier!à gauche} si pour tout \( x,y\in E\) nous avons
	\begin{equation}
		s*x=s*y\Rightarrow x=y.
	\end{equation}
	L'élément \( s\) est \defe{régulier à droite}{élément régulier!à droite} si pour tout \( x,y\in E\) nous avons
	\begin{equation}
		x*s=y*s\Rightarrow x=y.
	\end{equation}
	Il est \defe{régulier}{élément régulier}\index{magma!élément régulier} si il est régulier à gauche et à droite.
\end{definition}

\begin{remark}	\label{REMooMagmasEltsReguliers}
	Dans un magma \( E \), si \( e \) est un neutre à gauche (respectivement à droite), alors il est forcément régulier à gauche (resp. à droite).

  	On verra plus tard la notion de groupe\footnote{Définition~\ref{DEFooBMUZooLAfbeM}.}% \futureok
	, qui étend celle de magma: nous verrons que tous les éléments d'un groupe sont réguliers, et pour cause: ils admettent tous un inverse qui permet de simplifier.

   	On aura plus tard la notion d'anneau\footnote{Définition~\ref{DefHXJUooKoovob}.}% \futureok
    	Dans ce cadre, la régularité des éléments vis-à-vis de l'opération de multiplication sera moins évidente; en particulier, elle sera loin d'être automatique.
\end{remark}


%+++++++++++++++++++++++++++++++++++++++++++++++++++++++
\section{Cardinalité}
\label{SECooCardinalite}
%+++++++++++++++++++++++++++++++++++++++++++++++++++++++


%--------------------------------------------------------------------------------------------------------------------------- 
\subsection{Équipotence, surpotence, subpotence}
\label{SUBooEquiSurSubpotence}
%---------------------------------------------------------------------------------------------------------------------------

\begin{normaltext}	\label{NORMooEquiSurSubpotence}
	Les notions d'équipotence, surpotence et de subpotence permettent de comparer les «tailles» des ensembles sans avoir besoin de la théorie des ordinaux. Tout ceci ne sera pas très souvent utile par la suite. Un exemple d'utilisation de ces notions est le théorème de Steinitz \ref{THOooEDQKooLEGlDv} qui démontre l'existence de clôture algébrique pour tout corps.
\end{normaltext}

\begin{definition}[\cite{BIBooAKHUooProFGE,BIBooWNKRooETlebF}]      \label{DEFooXGXZooIgcBCg}
	Soient deux ensembles \( A\) et \( B\).
	\begin{enumerate}
		\item
		      Les ensembles \( A\) et \( B\) sont \defe{équipotents}{équipotent} si il existe une bijection entre \( A\) et \( B\). Nous notons \( A\approx B\).
		\item
		      L'ensemble \( A\) est \defe{surpotent}{surpotent} à \( B\) si il existe une surjection de \( A\) vers \( B\). Nous notons \( A\succeq B\).
		\item
		      L'ensemble \( A\) est \defe{subpotent}{subpotent} à \( B\) si il existe une injection de \( A\) vers \( B\). Nous notons \( A\preceq B\).
	\end{enumerate}
	Nous disons également «strictement» surpotent quand il y a surpotence mais pas équipotence, et de même pour la subpotence. Les symboles \( \succ\) et \( \prec\) sont alors utilisés.
\end{definition}

\begin{proposition}[\cite{MonCerveau,BIBooZFPUooIiywbk}]      \label{PROPooWSXTooMQPcNG}
	L'ensemble \( A\) est subpotent à \( B\) si et seulement si \( B\) est surpotent à \( A\).
\end{proposition}

\begin{proof}
	En deux parties.
	\begin{subproof}
		\spitem[\( \Rightarrow\)]
		Nous supposons que \( A\) est subpotent à \( B\). Il existe une injection \( \varphi\colon A\to B\). Nous définissons \( f\colon B\to A\) par
		\begin{equation}
			f(x)=\begin{cases}
				\varphi^{-1}(x) & \text{si } x\in\varphi(A) \\
				a               & \text{sinon }
			\end{cases}
		\end{equation}
		où \( a\) est un élément quelconque de \( A\). Cette application est bien définie parce que \( \varphi\) est injective, de telle sorte que \( \varphi^{-1}\) est bien définie. Puisque \( \varphi\) est définie sur tout \( A\), l'application \( f\) est une surjection.
		\spitem[\( \Leftarrow\)]
		Nous supposons que \( B\) est surpotent à \( A\). Il existe donc une surjection \( \varphi\colon B\to A\). Pour chaque \( x\in A\) nous considérons un élément \( b_x\in \varphi^{-1}(x)\), qui existe parce que \( \varphi\) est surjective. Nous considérons ensuite l'application
		\begin{equation}
			\begin{aligned}
				f\colon A & \to B        \\
				x         & \mapsto b_x.
			\end{aligned}
		\end{equation}
		Nous prouvons que \( f\) est une injection. Supposons que \( x,y\in A\) soient tels que \( f(x)=f(y)\). Nous avons \( b_x=b_y\). Donc
		\begin{equation}
			x=\varphi(b_x)=\varphi(b_y)=y.
		\end{equation}
		Nous avons prouvé que \( x=y\), et donc que \( f\) est injective.
	\end{subproof}
\end{proof}

%--------------------------------------------------------------------------------------------------------------------------- 
\subsection{Un peu d'infinité}
\label{SUBooCardinalInfini}
%---------------------------------------------------------------------------------------------------------------------------

\begin{normaltext}\label{NORooPasDeClasses}
	Vu que l'ensemble des ensembles n'existe pas\footnote{Voir le corolaire \ref{CORooZMAOooPfJosM}.}, nous n'allons pas énoncer le fait que ces notions donnent une relation d'ordre sur les ensembles; il faudrait parler de classes et nous ne nous en sortirions pas. Nous allons toutefois énoncer quelques résultats qui vont dans ce sens. Pour en savoir plus, vous pouvez lire les différentes pages de Wikipédia sur les nombres cardinaux.
\end{normaltext}

\begin{definition}[ensemble Dedekind infini]      \label{DefEOZLooUMCzZR}
	Un ensemble est \defe{infini}{ensemble!infini} si il peut être mis en bijection avec un de ses sous-ensembles propres (c'est-à-dire différent de lui-même).

	Un ensemble est \defe{fini}{ensemble fini} si il n'est pas infini.
\end{definition}

\begin{normaltext}\label{NORooQuelleDefEnsInfini}
	Nous adoptons les notions d'ensembles finis et infinis au sens de Dedekind. De nombreuses sources (dont wikipédia \cite{BIBooJYONooLCnCtQ,BIBooPARTooZDteDq}) définissent un ensemble fini comme étant un ensemble en bijection avec une partie de \( \eN\) de la forme \( \{ 0,\ldots, N \}\). Alors un ensemble est infini si il n'est pas fini.

	Cependant, d'une part \emph{les deux définitions d'ensembles infinis ne sont pas équivalentes}, mais d'autre part, elles sont équivalentes si on accepte l'axiome du choix\footnote{Et même seulement l'axiome du choix dénombrable; si vous voulez en savoir plus, lisez la page wikipédia \cite{BIBooSPDRooHTpBqh}.}. Or le Frido accepte l'axiome du choix sans vergogne et sous toutes ses formes. Nous démontrerons donc, en utilisant le lemme de Zorn, qu'un ensemble \( A\) est fini (définition \ref{DefEOZLooUMCzZR}) si et seulement si il existe une bijection \( \{ 0,\ldots, N \}\to A\) pour un certain \( N\in \eN\). Ce sera le théorème  \ref{PROPooJLGKooDCcnWi}.
\end{normaltext}


\begin{lemma}       \label{LEMooTUIRooEXjfDY}
	Toute partie d'un ensemble fini est finie.
\end{lemma}

\begin{proof}
	Nous allons prouver la contraposée : si un ensemble contient une partie infinie, alors il est infini. Soit \( A\subset B\) où \( A\) est infini. Nous allons prouver que \( B\) est infini. En vertu de la définition \ref{DefEOZLooUMCzZR}, il existe une partie \( A'\subsetneq A\) et une bijection \( \sigma\colon A'\to A\).

	Nous considérons la partie \( B'=A'\cup(B\setminus A)\), qui est une partie stricte de \( B\). Puisque \( A'\cap (B\setminus A)=\emptyset\), nous pouvons définir
	\begin{equation}
		\begin{aligned}
			\varphi\colon B' & \to B                                             \\
			x                & \mapsto \begin{cases}
				                           \sigma(x) & \text{si } x\in A'            \\
				                           x         & \text{si } x\in B\setminus A.
			                           \end{cases}
		\end{aligned}
	\end{equation}
	Montrons que \( \varphi\) est une bijection.
	\begin{subproof}
		\spitem[Surjectif]
		Nous avons \( \varphi(A')=A\) et \( \varphi(B\setminus A)=B\setminus A\). Donc
		\begin{equation}
			\varphi(B')=\varphi(A')\cup\varphi(B\setminus A)=A\cup (B\setminus A)=B.
		\end{equation}
		\spitem[Injectif]
		Si \( \varphi(x)=\varphi(y)\), nous avons \( 4\) possibilités suivant que \( x\) et \( y\) sont dans \( A'\) ou \( B\setminus A\).

		Si \( x,y\in A'\), alors \( \varphi(x)=\varphi(y)\) implique \( \sigma(x)=\sigma(y)\) et donc \( x=y\) parce que \( \sigma\) est injective.

		Si \( x\in A'\) et \( y\in B\setminus A\) alors \( \varphi(x)=\sigma(x)\in A\) et \( \varphi(y)=y\in B\setminus A\). Ce cas n'est pas possible. Le cas \( x\in B\setminus A\) et \( y\in A'\) n'est pas possible non plus.

		Si \( x,y\in B\setminus A\), alors \( \varphi(x)=\varphi(y)\) implique immédiatement \( x=y\).
	\end{subproof}
	Nous avons une bijection entre \( B'\) et \( B\) alors que \( B'\) est un sous-ensemble strict de \( B\). Donc \( B\) est infini.
\end{proof}

\begin{proposition}[\cite{MonCerveau,BIBooZFPUooIiywbk}]      \label{PROPooVOKDooOStPzU}
	Si \( A\) est fini et si \( \omega\notin A\), alors \( A\cup\{ \omega \}\) est fini.
\end{proposition}

\begin{proof}
	Supposons que \( A\cup\{ \omega \}\) est infini. Il existe un sous-ensemble strict de \( A\cup\{ \omega \}\) en bijection avec \( A\cup\{ \omega \}\). Soient donc \( B\subsetneq A\cup\{ \omega \}\) et \( \sigma\colon B\to A\cup\{ \omega \}\) une bijection.

	Il y a deux possibilités : soit \( \omega\) est dans \( B\), soit non.

	\begin{subproof}
		\spitem[\( \omega\notin B\)]
		Alors \( B\subset A\), et il existe \( x\in B\) tel que \( \sigma(x)=\omega\). Considérons \( B'=B\setminus\{ x \}\); cela est une partie propre de \( A\). Ensuite nous définissons
		\begin{equation}
			\begin{aligned}
				\varphi\colon B\setminus\{ x \} & \to A              \\
				a                               & \mapsto \sigma(a).
			\end{aligned}
		\end{equation}
		C'est injectif parce que \( \sigma\) est injective, et c'est surjectif parce que
		\begin{equation}
			\varphi\big( B\setminus\{ x \} \big)=\sigma(B)\setminus\sigma(\{ \omega \})=\sigma(B)\setminus\{ \omega \}=(A\cup\{ \omega \})\setminus\{ \omega \}=A.
		\end{equation}
		Pour la dernière égalité nous avons utilisé le fait que \( \omega\) n'est pas dans \( A\).
		\spitem[Si \( \omega\in B\)]
		%---------------------------------------------------
		Plusieurs choses à dire.
		\begin{subproof}
			\spitem[\( \sigma(\omega)\neq \omega\)]
			%---------------------------------------

			Si \( \omega\in B\), nous commençons par prouver que \( \sigma(\omega)\neq \omega\). En effet si \( \sigma(\omega)=\omega\), alors nous pouvons poser
			\begin{equation}
				\begin{aligned}
					\sigma'\colon B\setminus\{ \omega \} & \to A              \\
					x                                    & \mapsto \sigma(x).
				\end{aligned}
			\end{equation}
			Ce \( \sigma'\) est une bijection entre \( A\) et la partie propre \( B\setminus \{ A \}\). Cela est impossible parce que nous avons supposé que \( A\) est fini.

			\spitem[Definition de \( \varphi\)]
			%-----------------------------------------------------------
			Puisque \( B\) est une partie propre de \( A\cup\{ \omega \}\), il existe \( x\in A\setminus B\). Nous considérons \( B'=\big( B\setminus\{ \omega \} \big)\cup\{ x \}\) et nous définissons
			\begin{equation}
				\begin{aligned}
					\varphi\colon B' & \to A                                       \\
					b                & \mapsto \begin{cases}
						                           \sigma(b)      & \text{si } b\neq x \\
						                           \sigma(\omega) & \text{si } b=x.
					                           \end{cases}
				\end{aligned}
			\end{equation}
			Nous montrons à présent que \( \varphi\) est une bijection.
			\spitem[Injectif]
			Soient \( u,v\in B'\) tels que \( \varphi(u)=\varphi(v)\). Il y a \( 4\) possibilités suivant que \( u\) ou \( v\) est égal à \( x\).
			\begin{itemize}

				\item
				      Si \( u=v=x\) on est bon.
				\item

				      Si \( u=x\) et \( v\neq x\). D'une part \( \varphi(u)=\varphi(v)=\sigma(v)\) parce que \( v\neq x\). Mais d'autre part \( \varphi(u)=\varphi(x)=\sigma(\omega)\). Donc \( \sigma(v)=\sigma(\omega)\) et \( v=\omega\) par injectivité de \( \sigma\). Or cela est impossible parce que \( v\in B'\) alors que \( \omega\not\in B'\). Ici nous utilisons le fait que \( \omega\) ne peut pas être égal à \( x\) parce que \( x\in A\) alors que \( \omega\) est hors de \( A\).

				      Bref, ce cas est impossible.
				\item
				      Si \( u\neq x\) et \( v=x\), alors c'est le même cas que le précédent avec quelque adaptations.
				\item
				      Si \( u\neq x\) et \( v\neq x\), alors \( \varphi(u)=\sigma(u)\) et \( \varphi(v)=\sigma(v)\), de telle sorte que \( \sigma(u)=\sigma(v)\) et donc \( u=v\) par injectivité de \( \sigma\).
			\end{itemize}
			\spitem[Surjective]
			Soit \( y\in A\). Vu que \( \sigma\colon B\to A\cup\{ \omega \}\) est surjective, il existe \( b\in B\) tel que \(\sigma(b)=y\). Notez que \( b\neq x\) parce que \( x\in A\setminus B\).

			Si \( b\neq \omega\) alors \( b\in B\setminus\{ \omega \}\subset B'\). Nous pouvons calculer \( \varphi(b)\). Étant donné que \( b\neq x\), nous avons \( \varphi(b)=\sigma(b)=y\).

			Si au contraire \( b=\omega\), alors \( \varphi(x)=\sigma(\omega)=y\). Dans les deux cas, \( y\) est dans l'image de \( \varphi\).
		\end{subproof}
	\end{subproof}
	Dans tous les cas nous avons construit une bijection entre une partie propre \( B'\subsetneq A\) et \( A\), ce qui est absurde parce que \( A\) est un ensemble fini.
\end{proof}

\begin{proposition}[\cite{MonCerveau}]    \label{PROPooWKSIooHcfYPN}
	Si \( A\) est infini et si \( \sigma\colon A\to B\) est injective, alors \( B\) est infini.
\end{proposition}

\begin{proof}
	Nous allons prouver que \( \sigma(A)\) est une partie infinie de \( B\). Puisque \( A\) est infini, nous pouvons considérer une partie \( A'\subsetneq A\) et une bijection \( \varphi_A\colon A'\to A\). Nous définissons
	\begin{equation}
		\begin{aligned}
			\varphi_B\colon \sigma(A') & \to \sigma(A)                                                   \\
			y                          & \mapsto \sigma\Big( \varphi_A\big( \sigma^{-1}(y) \big)  \Big).
		\end{aligned}
	\end{equation}
	Cette définition a un sens parce que si \( y\in \sigma(A')\), alors il existe un unique \( x\in A'\) tel que \( \sigma(x)=y\) parce que \( \sigma\) est injective. De là, \( \varphi_A(x)\in A\) et nous pouvons lui appliquer \( \sigma\).

	Nous montrons que \( \varphi_B\) est une bijection.
	\begin{subproof}
		\spitem[Injective]
		Supposons que \( \varphi_B(a)=\varphi_B(b)\), c'est-à-dire que
		\begin{equation}
			(\sigma\circ\varphi_A\circ\sigma^{-1})(a)=(\sigma\circ\varphi_A\circ\sigma^{-1})(b).
		\end{equation}
		Étant donné que \( \sigma\) et \( \varphi_A\) sont injectives, nous avons \( \sigma^{-1}(a)=\sigma^{-1}(b)\). En appliquant \( \sigma\) des deux côtés, nous trouvons \( a=b\).
		\spitem[surjective]
		Soit \( y\in \sigma(A)\). En prenant \( x\in(\sigma\circ\varphi_A^{-1}\circ\sigma^{-1})(y)\) nous avons \( \varphi_B(x)=y\).
	\end{subproof}
	Donc \( B\) contient une partie infinie (\( \sigma(A)\)). Le lemme \ref{LEMooTUIRooEXjfDY} conclut que \( B\) est infini.
\end{proof}

\begin{lemma}       \label{LEMooPGPVooZzlFvf}
	À propos d'applications entre ensembles finis.
	\begin{enumerate}
		\item       \label{ITEMooNCCUooBGrtdn}
		      Si \( \sigma\colon A\to B\) est une application quelconque et si \( A\) est fini, alors \( \sigma(A)\) est une partie finie de \( B\).
		\item       \label{ITEMooKQMFooSzmXrd}
		      Si \( \sigma\colon A\to B\) est surjective et si \( A\) est fini, alors \( B\) est fini.
	\end{enumerate}
\end{lemma}

\begin{proof}
	Nous allons utiliser le lemme de Zorn. Nous considérons l'ensemble
	\begin{equation}
		\mA=\{ X\subset A\tq \sigma\colon X\to \sigma(A) \text{ est injective} \}
	\end{equation}
	que nous ordonnons (partiellement) par l'inclusion.
	\begin{subproof}
		\spitem[\( \mA\) est inductif]
		Soit une partie totalement ordonnée \( \mF\) de \( \mA\). Nous considérons \( Y=\bigcup_{X\in \mF}X\), et nous prouvons que \( Y\) est un majorant de \( \mF\).

		Pour cela nous commençons par prouver que \( Y\in \mA\). Soient \( a,b\in Y\) tels que \( \sigma(a)=\sigma(b)\). Il existe \( X_1,X_2\in \mF\) tels que \( a\in X_1\) et \( b\in X_2\). Supposons pour fixer les idées que \( X_1\leq X_2\) (\( \mF\) étant totalement ordonné nous avons toujours \( X_1\leq X_2\) ou \( X_2\leq X_1\)). Puisque l'ordre est l'inclusion, cela signifie que \( X_1\subset X_2\). Nous avons donc \( a,b\in X_2\), alors que \( \sigma\) est injective sur \( X_2\). Donc \(\sigma(a)=\sigma(b)\) implique \( a=b\), et \( \sigma\) est injective sur \( Y\). Nous avons donc prouvé que \( Y\in \mA\).

		Puisque pour tout \( X\in\mF\) nous avons \( X\subset Y\), nous avons \( X\leq Y\) (dans \( \mA\)) pour tout \( X\in \tribF\). Bref, \( Y\) est un majorant de \( \mF\) dans \( \mA\).

		Toute partie totalement ordonnée de \( \mA\) est majorée. Cela signifie que \( \mA\) est inductif\footnote{Plus précisément c'est l'ensemble ordonné \( (\mA,\subset)\) qui est inductif.}.
		\spitem[Zorn]
		L'ensemble \( \mA\) étant inductif et non vide (les singletons dans \( A\) sont dans \( \mA\)), il possède un élément maximal\footnote{Définition \ref{DEFooBZNRooYRPGme}; voir aussi \ref{NORMooVHIBooJAOsou}.} par le lemme de Zorn \ref{LemUEGjJBc}. Nous nommons \( A'\) un élément maximal dans \( \mA\).
		\spitem[Bijective]
		L'application \( \sigma\colon A'\to \sigma(A)\) est injective parce que \( A'\in \mA\). Nous devons prouver qu'elle est surjective.

		Supposons que \( y\in\sigma(A)\setminus\sigma(A')\). Alors il existe \( a\in A\setminus A'\) tel que \( \sigma(a)=y\). Dans ce cas, la partie \( A'\cup\{ a \}\) est un majorant de \( A'\) dans \( \mA\), ce qui est impossible.

		Donc \( \sigma\colon A'\to \sigma(A)\) est bijective.
		\spitem[Conclusion]
		L'ensemble \( A'\) est fini en tant que partie de l'ensemble fini \( A\) (lemme \ref{LEMooTUIRooEXjfDY}). L'application \( \sigma\) étant injective, la proposition \ref{PROPooWKSIooHcfYPN} conclut que \( \sigma(A')\) est fini. Et comme \( \sigma(A')\) n'est autre que \( \sigma(A)\) nous avons fini.
	\end{subproof}
	La partie \ref{ITEMooNCCUooBGrtdn} est prouvée. La partie \ref{ITEMooKQMFooSzmXrd} est maintenant facile. La partie \ref{ITEMooNCCUooBGrtdn} dit que \( \sigma(A)\) est une partie finie de \( B\), mais si \( \sigma\) est surjective, alors \( \sigma(A)=B\).
\end{proof}


%--------------------------------------------------------------------------------------------------------------------------- 
\subsection{Théorème de Cantor-Schröder-Bernstein}
\label{SUBooCantorSchroderBernstein}
%---------------------------------------------------------------------------------------------------------------------------

\begin{lemma}[\cite{BIBooECJMooGPxBem}]     \label{LEMooTNMHooBpdzab}
	Soient un ensemble \( A\) et une partie \( B\) de \( A\). Si il existe une injection \( f\colon A\to B\) alors il existe une bijection \( \alpha\colon A\to B\).
\end{lemma}

Nous donnons deux preuves de ce lemme.

\begin{proof}[Première preuve de \ref{LEMooTNMHooBpdzab}]
	Nous posons \( Y=A\setminus B\) et nous décomposons la preuve en étapes.
	\begin{subproof}
		\spitem[Les \( f^k(Y)\) sont disjoints]
		Vu que \( f\) prend ses valeurs dans \( B\), nous avons \( f^k(Y)\subset B\) pour tout \( k\). Et vu que \( Y=A\setminus B\), nous avons
		\begin{equation}        \label{EQooDNHJooFJBrDq}
			f^k(Y)\cap Y=\emptyset
		\end{equation}
		pour tout \( k\). L'application \( f\) étant injective, elle vérifie \( f(C\cap D)=f(C)\cap f(D)\). Nous appliquons \( f^m\) des deux côtés de \eqref{EQooDNHJooFJBrDq} :
		\begin{equation}
			f^{k+m}(Y)\cap f^m(Y)=\emptyset
		\end{equation}
		pour tout \( k,m\in \eN\).
		\spitem[Une décomposition]
		Nous posons
		\begin{equation}
			X=\bigcup_{k\in \eN}f^k(Y)=Y\cup\bigcup_{k=1}^{\infty}f^k(Y).
		\end{equation}
		Vu que \( f(X)\subset B\) nous avons l'égalité
		\begin{equation}
			B=f(X)\cup\big(B\setminus f(X)\big).
		\end{equation}
		\spitem[\( A\setminus X=B\setminus f(X)\)]
		Supposons \( x\in A\setminus X\). Vu que \( Y=A\setminus B\) est dans \( X\), l'élément \( x\) n'est pas dans \( A\setminus B\) et donc est dans \( B\) parce qu'il est dans \( A\). Mais \( x\) n'est pas dans \( X\) et en particulier pas dans \( f(X)\) parce que \( f(X)\subset X\). Donc \( x\) est dans \( B\setminus f(X)\).

		Dans l'autre sens, nous supposons que \( x\in B\setminus f(X)\). Vu que \( B\subset A\) nous avons \( x\in A\). Comme \( x\) est hors de \( f(X)\), il est hors des \( f^k(Y)\) pour \( k\geq 1\). Mais \( x\in B\), donc \( x\) est hors de \( A\setminus B=f^0(Y)\). Donc \( x\) est hors de \( f^k(Y)\) pour tout \( k\geq 0\). Donc \( x\) est hors de \( X\).

		\spitem[La bijection]
		Nous considérons l'application
		\begin{equation}
			\begin{aligned}
				\alpha\colon A & \to B                                        \\
				x              & \mapsto \begin{cases}
					                         f(x) & \text{si } x\in X             \\
					                         x    & \text{si } x\in A\setminus X.
				                         \end{cases}
			\end{aligned}
		\end{equation}
		Nous démontrons dans les points suivants que \( \alpha\) est bijective.
		\spitem[Injective]
		Nous supposons \( \alpha(x)=\alpha(y)\). Il y a 4 possibilités suivant que \( x\) et \( y\) soient dans \( X\) ou \( A\setminus X\).

		Si \( x,y\in X\) alors \( f(x)=f(y)\) et donc \( x=y\) parce que \( f\) est injective.

		Si \( x\in X\) et \( y\in A\setminus X\), alors \( f(x)=y\). Mais \( f(x)\in f(X)\) et \( y\in A\setminus X=B\setminus f(X)\). Donc l'élément \( f(x)=y\) est dans \( f(X)\cap \big( B\setminus f(X) \big)=\emptyset\). Il n'est donc pas possible d'avoir \( \alpha(x)=\alpha(y)\) avec \( x\in X\) et \( y\in A\setminus X\).

		Si \( x\in A\setminus X\) et \( y\in X\), c'est la même chose.

		Si \( x,y\in A\setminus X\), alors \( x=\alpha(x)=\alpha(y)=y\).

		\spitem[Surjective]
		Soit \( y\in B\). Il y a deux possibilités : \( y\in X\) et \( y\in A\setminus X\). La première se divise en deux : \( y\in Y\) et \( y\in \bigcup_{k=1}^{\infty}f^k(Y)\). On y va.

		\begin{subproof}
			\spitem[\( y\in Y\)]
			Ce cas n'est pas possible parce que \( y\in B\) alors que \( Y=A\setminus B\).
			\spitem[\( y\in f^k(Y)\) avec \( k\geq 1\)]
			Nous avons
			\begin{equation}
				y\in f\big( f^{k-1}(Y) \big)\subset f(X)\subset \alpha(A).
			\end{equation}
			\spitem[\( y\in A\setminus X\)]
			Alors \( y=\alpha(y)\).
		\end{subproof}
	\end{subproof}
\end{proof}


\begin{proof}[Deuxième preuve de \ref{LEMooTNMHooBpdzab}\cite{BIBooZFPUooIiywbk}]
	Nous posons \( Y=A\setminus B\) et
	\begin{equation}
		\mM=\{ M\subset A\tq Y\cup f(M)\subset M \}.
	\end{equation}
	Nous allons dire de nombreuses choses à propos de ce \( \mM\).
	\begin{subproof}
		\spitem[\( \mM\) est non vide]
		Nous avons \( A\in \mM\) parce que \( Y\) et \( f(A)\) sont dans \( A\).
		\spitem[Si \( M\in\mM\) alors \( Y\subset M\)]
		C'est dans la définition de \( \mM\).
		\spitem[Encore un ensemble]
		Vu que \( \mM\) est non vide, nous pouvons poser
		\begin{equation}
			X=\bigcap_{M\in \mM}M
		\end{equation}
		sans nous poser trop de questions. Cela étant fait, nous pouvons passer aux choses sérieuses.
		\spitem[\( f(X)\subset M\) pour tout \( M\in \mM\)]
		Soit \( M\in \mM\). Nous avons
		\begin{subequations}
			\begin{align}
				f(X) & \subset f(M)       \label{SUBEQooRJKWooQFoLZp}     \\
				     & \subset Y\cup f(M)                                 \\
				     & \subset M              \label{SUBEQooSAVZooPXYXqC}
			\end{align}
		\end{subequations}
		Justifications.
		\begin{itemize}
			\item Pour \eqref{SUBEQooRJKWooQFoLZp}. Parce que \( X\subset \mM\).
			\item Pour \eqref{SUBEQooSAVZooPXYXqC}. Parce que \( M\in \mM\).
		\end{itemize}
		\spitem[\( X\in\mM\)]
		Vu que \( Y\subset M\) pour tout \( M\) dans \( \mM\), nous avons \( Y\subset \bigcap_{M\in \mM}M=X\). Nous devons prouver \( f(X)\subset X\). Nous venons de prouver que \( f(X)\subset M\) pour tout \( M\in \mM\), donc
		\begin{equation}
			f(X)\subset \bigcap_{M\in\mM}M=X.
		\end{equation}
		L'ensemble \( X\) est le plus petit élément de \( \mM\) pour l'inclusion.
		\spitem[\( Y\cup f(X)\subset X\)]
		Ça fait partie de \( X\in \mM\). Mais c'est bien de le dire explicitement parce que nous allons l'utiliser quelques fois dans la suite.
		\spitem[\( Y\cup f(X)\in\mM\)]
		Vu que \( X\in\mM\), nous savons déjà que \( Y\cup f(X)\subset X\). En appliquant \( f\) des deux côtés,
		\begin{equation}
			f\big( Y\cup f(X) \big)\subset f(X).
		\end{equation}
		En ajoutant \( Y\) des deux côtés,
		\begin{equation}
			Y\cup f\big( Y\cup f(X) \big)\subset Y\cup f(X),
		\end{equation}
		et donc \( Y\cup f(X)\in \mM\).
		\spitem[\( Y\cup f(X)=X\)]
		Nous savons déjà que \( Y\cup f(X)\in \mM\). Vu que \( X\) est le plus petit élément de \( \mM\), nous avons
		\begin{equation}
			X\subset Y\cup f(X).
		\end{equation}
		L'inclusion inverse étant déjà faite, nous avons l'égalité.
		\spitem[\( B\setminus f(X)=A\setminus X\)]
		Nous avons :
		\begin{subequations}
			\begin{align}
				A\setminus X & =A\setminus\big( Y\cup f(X) \big)                                               \\
				             & =(A\setminus Y)\cap \big( A\setminus f(X) \big)     \label{SUBEQooBVZCooAEUJtT} \\
				             & =B\cap \big( A\setminus f(X) \big)         \label{SUBEQooUATVooPErmHT}          \\
				             & =(B\cap A)\setminus f(X)       \label{SUBEQooKWVWooOMLzrv}                      \\
				             & =B\setminus f(X).      \label{SUBEQooQTTSooIFSFKo}
			\end{align}
		\end{subequations}
		Justifications :
		\begin{itemize}
			\item Pour \eqref{SUBEQooBVZCooAEUJtT}. Complémentaire de réunion, lemme \ref{LEMooHRKAooRskzQL}\ref{ITEMooQCGUooKnWfBo}.
			\item Pour \eqref{SUBEQooUATVooPErmHT}. Parce que \( A\setminus Y=B\) du fait que \( Y=A\setminus B\).
			\item Pour \eqref{SUBEQooKWVWooOMLzrv}. Lemme \ref{LEMooHRKAooRskzQL}\ref{ITEMooXWKCooUASxlh}.
			\item Pour \eqref{SUBEQooQTTSooIFSFKo}. Parce que \( B\subset A\).
		\end{itemize}
		\spitem[Notre bijection]
		Nous voulons définir
		\begin{equation}
			\begin{aligned}
				\alpha\colon A & \to B                                        \\
				x              & \mapsto \begin{cases}
					                         f(x) & \text{si } x\in X             \\
					                         x    & \text{si } x\in A\setminus X.
				                         \end{cases}
			\end{aligned}
		\end{equation}
		Pour y parvenir, nous devons prouver que \( \alpha\) prend effectivement ses valeurs dans \( B\). Ensuite nous prouverons que \( \alpha\) est une bijection.
		\spitem[\( \alpha\) prend ses valeurs dans \( B\)]
		Si \( x\in X\) nous avons \( \alpha(x)=f(x)\in B\). Si au contraire \( x\in A\setminus X\) nous avons
		\begin{equation}
			\alpha(x)=x\in A\setminus X=B\setminus f(X)\subset B.
		\end{equation}
		\spitem[\( \alpha\) est injective]
		Soient \( x,y\in A\) tels que \( \alpha(x)=\alpha(y)\). Il y a quatre possibilités suivant que \( x\) et \( y\) sont dans \( X\) ou dans \( A\setminus X\).
		\begin{enumerate}
			\spitem[\( x\in X\), \( y\in X\)]
			Alors \( f(x)=\alpha(x)=\alpha(y)=f(y)\). Vu que \( f\) est injective, nous trouvons que \( x=y\).
			\spitem[\( x\in X\), \( y\in A\setminus X\)]
			Nous avons \( \alpha(x)=f(x)\in f(X)\) et \( \alpha(y)=y\in A\setminus X=B\setminus f(X)\). Il n'est donc pas possible d'avoir \( \alpha(x)=\alpha(y)\) dans ce cas.
			\spitem[\( x\in A\setminus X\), \( y\in X\)]
			Idem.
			\spitem[\( x\in A\setminus X\), \( y\in A\setminus X\)]
			Dans ce cas nous avons \( \alpha(x)=x\) et \( \alpha(y)=y\). Donc \( x=y\).
		\end{enumerate}
		\spitem[\( \alpha\) est surjective]
		Soit \( b\in B\). Il y a deux possibilités : \( b\in f(X)\) ou \( b\notin f(X)\).
		\begin{subproof}
			\spitem[Si \( b\in f(X)\)]
			Soit \( x\in X\) tel que \( f(x)=b\). Alors \( \alpha(x)=f(x)=b\).
			\spitem[Si \( b\notin f(X)\)]
			Alors \( b\in B\setminus f(X)=A\setminus X\), et donc \( \alpha(b)=b\).
		\end{subproof}
	\end{subproof}
\end{proof}


\begin{theorem}[Cantor-Schröder-Bernstein]      \label{THOooRYZJooQcjlcl}
	Soient deux ensembles \( A\) et \( B\) pour lesquels il existe des injections \( f\colon A\to B\) et \( g\colon B\to A\). Alors il existe une bijection entre \( A\) et \( B\).
\end{theorem}

\begin{proof}
	La composée \( g\circ f\colon A\to A\) est injective et prend ses valeurs dans \( g\big( f(A) \big)\subset g(B)\subset A\). Bref, l'application \( g\circ f\colon A \to g(B)\) est injective. Le lemme \ref{LEMooTNMHooBpdzab} donne alors une bijection \( \varphi\colon A\to g(B)\).

	Nous montrons que \( g^{-1}\circ\varphi\colon A\to B\) est une bijection.

	\begin{subproof}
		\spitem[Injective]
		Nous supposons \( x,y\in A\) tels que
		\begin{equation}
			g^{-1}\big( \varphi(x) \big)=g^{-1}\big( \varphi(y) \big).
		\end{equation}
		Nous appliquons \( g\) des deux côtés : \( \varphi(x)=\varphi(y)\). Puisque \( \varphi\) est une bijection, cela entraîne \( x=y\).
		\spitem[Surjective]
		Soit \( b\in B\). En posant \( a=\varphi^{-1}\big( g(b) \big)\) nous avons bien \( (g^{-1}\circ \varphi)(a)=b\).
	\end{subproof}
\end{proof}

%--------------------------------------------------------------------------------------------------------------------------- 
\subsection{Comparabilité cardinale}
\label{SUBooComparabiliteCardinale}
%---------------------------------------------------------------------------------------------------------------------------

\begin{normaltext}\label{NORooThmComparCardIntro}
	Le théorème de comparabilité cardinale énonce que si \( A\) et \( B\) sont des ensembles, alors nous avons toujours \( A\succeq B\) ou \( A\preceq B\) (ou les deux; dans ce cas \( A\approx B\) par Cantor-Schröder-Bernstein).
\end{normaltext}

\begin{theorem}[Théorème de comparabilité cardinale\cite{MonCerveau,BIBooTSOKooCWxMwj,BIBooZZNRooZytRuC}]     \label{THOooCBSKooCmzfUf}
	Entre deux ensembles, il existe forcément une injection de l'un dans l'autre.
\end{theorem}

\begin{proof}
	Nous allons montrer que le graphe d'une injection de \( A\) dans \( B\) ou de \( B\) dans \( A\) est donné par un élément maximal (au sens de l'inclusion) de l'ensemble (inductif) des graphes d'injections d'une partie de \( A\) dans une partie de \( B\).

	Nous allons utiliser le lemme de Zorn \ref{LemUEGjJBc} à l'ensemble\footnote{Attention : dans le Frido, la notation \( f\colon A\to B\), signifie que \( f\) est définie sur tout l'ensemble \( A\), mais pas qu'elle est surjective sur \( B\).}
	\begin{equation}
		\mA=\Big\{  (X,Y,\varphi)  \tq
		\begin{cases}
			X\subset A \\
			Y\subset B \\
			\varphi\colon X\to Y\text{ est injective.}
		\end{cases}
		\Big\}
	\end{equation}
	que nous ordonnons par l'inclusion, c'est-à-dire par \( (X_1,Y_1,\varphi_1)<(X_2,Y_2,\varphi_2)\) lorsque \( X_1\subset X_2\), \( Y_{1}\subset Y_2\) et \( \varphi_2|_{X_1}=\varphi_1\).

	Nous passons rapidement sur le fait que cet ensemble est inductif, et nous considérons tout de suite un élément maximal \( (X,Y,\varphi)\). Il y a deux possibilités : soit \( \varphi(X)=B\), soit \( \varphi(X)\neq B\).
	\begin{subproof}
		\spitem[Si \( \varphi(X)=B\)]
		Dans ce cas, \( \varphi\colon X\to B\) est une surjection. L'ensemble \( A\) est donc surpotent à \( B\). La proposition \ref{PROPooWSXTooMQPcNG} conclut que \( B\) est subpotent à \( A\).
		\spitem[Si \( \varphi(X)\neq B\)]
		Nous subdivisons en deux nouveaux cas : soit \( X=A\), soit \( X\neq A\).
		\begin{subproof}
			\spitem[Si \( X=A\)]
			Alors nous avons une injection \( \varphi\colon A\to B\), et c'est bon.
			\spitem[Si \( X\neq A\)]
			Nous sommes dans le cas \( X\neq A\) et \( \varphi(X)\neq B\). Soient \( a\in A\setminus X\) et \( b\in B\setminus \varphi(X)\). Nous considérons l'application
			\begin{equation}
				\begin{aligned}
					\psi\colon X\cup\{ a \} & \to Y\cup\{ b \}                       \\
					x                       & \mapsto \begin{cases}
						                                  \varphi(x) & \text{si } x\in X \\
						                                  b          & \text{si }x=a.
					                                  \end{cases}
				\end{aligned}
			\end{equation}
			C'est une application injective. Donc le triplet \( (X\cup \{ a \}, Y\cup\{ b \},\psi)\) majore \( (X,Y,\varphi)\). Nous avons une contradiction et ce cas n'est pas possible.
		\end{subproof}
	\end{subproof}
\end{proof}

\begin{normaltext}	\label{NORMooOrdreSurEnsembles}
	Le théorème de comparabilité cardinale couplé au théorème de Cantor-Schröder-Bernstein nous indique que pour tout ensembles \( A\) et \( B\), nous avons soit \( A\preceq B\), soit \( B\preceq A\). Et si \( A\preceq B\preceq A\), alors \( A\approx B\).

	Nous ne sommes pas loin de dire que la relation \( \preceq\) donne un ordre total sur l'ensemble des ensembles. C'est très beau sauf que l'ensemble des ensembles n'existe pas\footnote{Corolaire \ref{CORooZMAOooPfJosM}.}. Il faudrait parler de \emph{classe} des ensembles, mais ça nous mènerait trop loin. Toujours est-il que ces deux théorèmes montrent qu'on n'est pas loin d'avoir un ordre sur les ensembles, et que cela est une des bases possibles pour développer les nombres cardinaux.
\end{normaltext}

%--------------------------------------------------------------------------------------------------------------------------- 
\subsection{Théorème de Cantor, ensemble des ensembles}
\label{SUBooThmCantor}
%---------------------------------------------------------------------------------------------------------------------------

\begin{theorem}[Cantor\cite{BIBooXHFNooSmqUar}]     \label{THOooJPNFooWSxUhd}
	Un ensemble est toujours strictement subpotent à son ensemble des parties.
\end{theorem}

\begin{proof}
	Soit un ensemble \( E\) et son ensemble des parties \( \mP(E)\). Nous commençons par prouver qu'il n'existe pas de surjection \( E\to \mP(E)\). Soit en effet une application \( f\colon E\to \mP(E)\). Nous posons
	\begin{equation}
		D=\{ x\in E\tq x\notin f(x) \}.
	\end{equation}
	Nous prouvons que \( D\) ne peut pas être dans l'image de \( f\). Supposons que \( y\in E\) soit tel que \( f(y)=D\).
	\begin{subproof}
		\spitem[Si \( y\in D\)]
		Alors par définition de \( D\), nous avons \( y\notin f(y)=D\). Contradiction.
		\spitem[Si \( y\notin D\)]
		Alors \( y\in f(y)=D\), contradiction.
	\end{subproof}
	Donc aucune surjection \( f\colon E\to \mP(E)\) n'existe. En particulier pas de bijections.

	Par ailleurs, l'application \( g\colon \mP(E)\to E\) qui fait \( g(\{ a \})=a\) (et n'importe quoi d'autre sur les autres éléments de \( \mP(E)\)) est une surjection \( \mP(E)\to E\).

	Donc \( \mP(E)\) est toujours strictement surpotent à \( E\).
\end{proof}

\begin{normaltext}\label{NORooSensEnsSurpotent}
	Le théorème de Cantor implique en particulier qu'il existe (au moins) une infinité dénombrable d'ensembles infinis de cardinalité différentes (plus évidemment une infinité dénombrable d'ensembles finis de cardinalité différentes).

	Pour tout ensemble \( A\), il est donc possible de dire «soit \( E\), un ensemble strictement surpotent à \( A\)».
\end{normaltext}

\begin{corollary}       \label{CORooZMAOooPfJosM}
	Il n'existe pas d'ensemble contenant tous les ensembles.
\end{corollary}

\begin{proof}
	Si \( E\) était un tel ensemble, nous aurions \( \mP(E)\subset E\) parce que les éléments de \( \mP(E)\) sont des ensembles. Or cela donnerait une surjection \( E\to \mP(E)\) alors que cela est impossible par le théorème de Cantor \ref{THOooJPNFooWSxUhd}.
\end{proof}




%+++++++++++++++++++++++++++++++++++++++++++++++++++++++++++++++++++++++++++++++++++++++++++++++++++++++++++++++++++++++++++
\section{Quelques structures algébriques}
\label{SECooStructuresAlgebriques}
%+++++++++++++++++++++++++++++++++++++++++++++++++++++++++++++++++++++++++++++++++++++++++++++++++++++++++++++++++++++++++++

\begin{normaltext}\label{NORMooIntroStructuresAlgebriques}
	Nous collectons ici les définitions des principales structures algébriques.
\end{normaltext}

\begin{definition}[Groupe]      \label{DEFooBMUZooLAfbeM}
	Un \defe{groupe}{groupe} est un ensemble \( G\) muni d'une opération interne \( \cdot\colon G\times G\to G\) telle que
	\begin{enumerate}
		\item
		      pour tous \( g\), \( h\), \( k\in G\), \( g\cdot(h\cdot k)=(g\cdot h)\cdot k\),
		\item
		      il existe un élément \( e\in G\) tel que \( e\cdot g=g\cdot e=g\) pour tout \( g\in G\),
		\item
		      pour tout \( g\in G\), il existe un élément \( h\in  G\) tel que \(g\cdot h=h\cdot g=e \).
	\end{enumerate}
	Un groupe est \defe{commutatif}{groupe commutatif} ou \defe{abélien}{groupe abélien} si \( g\cdot h=h\cdot g\) pour tout \( g,h\in G\).
\end{definition}

\begin{remark}\label{REMooNotationOperationGroupe}
	Remarquez que nous avons écrit \( g\cdot h\) et non \( \cdot(g,h)\) comme une notation purement fonctionnelle nous l'aurait suggéré. Dans les exemples concrets, selon les cas, la loi de groupe appliquée à \( g\) et \( h\) sera notée tantôt \( g+h\), tantôt \( g\cdot h\) ou, le plus souvent pour un groupe générique, simplement \( gh\).
\end{remark}

\begin{definition}[Sous-groupe]    \label{DEFooSousGroupe}
	Soit \( G \) un groupe et \( H \) un sous-ensemble de \( G \). On dit que \( H \) est un \defe{sous-groupe}{groupe!sous-groupe} de \( G \) si \( H \), muni de l'opération héritée de \(G \), est un groupe. Plus précisément, \( H \) est un sous-groupe de \( G \) si :
	\begin{enumerate}
		\item
		      l'opération interne \( \cdot\colon G\times G\to G\) restreinte à \( H \times H \) a ses images dans \( H \);
		\item
		      l'élément \( e \) est un élément de \( H \);
		\item
		      lorsque \( g  \in H \), l'élément \( h \) tel que \(g\cdot h=h\cdot g=e \) est en réalité dans \( H \).
	\end{enumerate}
\end{definition}

\begin{definition}[morphisme, automorphisme]        \label{DEFooBEHTooMeCOTX}
	Soient deux groupes \( G\) et \( H\). Un \defe{morphisme}{morphisme de groupes} entre \( G\) et \( H\) est une application \( \alpha\colon G\to H\) telle que pour tous \( g,h\in G\) nous ayons \( \alpha(g\cdot_Gh)=\alpha(g)\cdot_H\alpha(h)\).

	Comme d'habitude, un isomorphisme est un morphisme bijectif.

	Un \defe{automorphisme}{automorphisme de groupes} de \( G\) est un isomorphisme de \( G\) vers \( G\) lui-même. L'ensemble des automorphismes de \( G\) est noté \( \Aut(G)\).
\end{definition}

\begin{proposition}		\label{PROPooJHHPooIpciPA}
	Si \(\alpha \colon G\to H  \) est un morphisme de groupes, alors \( \alpha(G)\) est un sous-groupe de \( H\).
\end{proposition}

\begin{proof}
	Point par point vis-à-vis de la définition~\ref{DEFooSousGroupe}. On notera \( \cdot \) l'opération de \( H \) et \( \cdot_G \) celle de \( G \).
	\begin{subproof}
		\item[Opération interne]
		Si \( h \) et \( h' \) sont deux éléments de \( \alpha(G)\), il existe deux éléments \( g \) et \( g'\) de \( G \) tels que \( h = \alpha(g) \) et \( h' = \alpha(g') \). Alors
		\[
			h \cdot h' = \alpha(g) \cdot \alpha(g') = \alpha(g\cdot_G g'),
		\]
		puisque \( \alpha \) est un morphisme de groupes. Ainsi \( h \cdot h' \in \alpha(G) \) car il est l'image de \( g\cdot_G g' \).
		\item[Neutre]
		On a \( e_H = \alpha(e_G) \) quel que soit le morphisme \( \alpha \).
		\item[Inverse]
		L'inverse d'un élément \( h = \alpha(g) \) est l'élément \( \alpha(g^{-1}) \).
	\end{subproof}
\end{proof}

\begin{lemma}       \label{LEMooKJMWooQeIwgF}
	Si \( G\) est un groupe, alors \( \Aut(G)\) est un groupe pour la composition.
\end{lemma}

\begin{proof}
	Soient \( \alpha\) et \( \beta\) des automorphismes de \( G\). Alors nous prouvons que \( \alpha\circ\beta\) est un automorphisme de \( G\) :
	\begin{equation}
		(\alpha\circ\beta)(gh)=\alpha\big( \beta(g)\beta(h) \big)=\alpha\big( \beta(g) \big)\alpha\big( \beta(h) \big)=(\alpha\circ\beta)(g)(\alpha\circ\beta)(h).
	\end{equation}
\end{proof}

\begin{lemma}       \label{LEMooDPISooRoAFmt}
	Si \( G\) et \( H\) sont des groupes isomorphes, alors les groupes \( \Aut(G)\) et \( \Aut(H)\) sont isomorphes.
\end{lemma}

\begin{proof}
	Soit un isomorphisme \( f\colon G\to H\). D'abord pour tout \( \alpha\in \Aut(G)\), l'application \( f\circ\alpha\circ f^{-1}\) est un automorphisme de \( H\). Cela est rapidement vérifié parce que \( f\), \( \alpha\) et \( f^{-1}\) sont des bijections et des morphismes.

	Nous pouvons donc considérer l'application
	\begin{equation}
		\begin{aligned}
			\psi\colon \Aut(G) & \to \Aut(H)                        \\
			\alpha             & \mapsto f\circ \alpha\circ f^{-1}.
		\end{aligned}
	\end{equation}
	\begin{subproof}
		\spitem[\( \psi\) est un morphisme]
		Soient \( \alpha,\beta\in\Aut(G)\). Vu que \( f\) est une bijection, nous pouvons introduire \( f^{-1}\circ f\) partout où ça nous plaît :
		\begin{equation}
			\psi(\alpha\beta)=f\circ\alpha\circ\beta\circ f^{-1}=f\circ\alpha\circ f^{-1}\circ f\circ\beta\circ f^{-1}=\psi(\alpha)\circ\psi(\beta).
		\end{equation}
		\spitem[\( \psi\) est injective]
		% -------------------------------------------------------------------------------------------- 
		Si \( \alpha\beta\in \Aut(G)\) sont tels que \( \psi(\alpha)=\psi(\beta)\), alors
		\begin{equation}
			f\circ\alpha\circ f^{-1}=f\circ \beta\circ f^{-1}.
		\end{equation}
		Comme \( f\) est une bijection, cela implique que \( \alpha=\beta\).
		\spitem[\( \psi\) est surjective]
		% -------------------------------------------------------------------------------------------- 
		Si \( \alpha\colon H\to H\) est un automorphisme, alors \( \sigma=\psi(f^{-1}\circ\sigma\circ f)\) où il est facile de vérifier que \( f^{-1}\circ\sigma\circ f\in\Aut(G)\).
	\end{subproof}
\end{proof}

\begin{definition}[Anneau\cite{Tauvel}]     \label{DefHXJUooKoovob}
	Un \defe{anneau}{anneau}\footnote{Nous faisons le choix qu'un anneau admet toujours un neutre pour la multiplication. Certains ouvrages parlent dans ce cas d'anneau unitaire.} est un triplet \( (A,+,\cdot)\) avec les conditions
	\begin{enumerate}
		\item
		      \( (A,+)\) est un groupe\footnote{Groupe, définition \ref{DEFooBMUZooLAfbeM}.} commutatif. Nous notons \( 0\) le neutre.
		\item
		      La multiplication est associative et admet un neutre, que nous notons \( 1\).
		\item       \label{ITEMooGMNOooSTGiXw}
		      La multiplication est distributive (à gauche et à droite) par rapport à l'addition. Cela signifie que, pour tous \( a \), \( b \) et \( c \) de \( A \), nous avons les deux égalités
		      \begin{equation}
			      a(b+c) = ab + ac \qquad\text{et}\qquad (a+b)c = ac + bc.
		      \end{equation}
	\end{enumerate}
	L'anneau \( (A,+,\cdot)\) est \defe{commutatif}{anneau commutatif} si pour tout \( a,b\in A\) nous avons \( a\cdot b=b\cdot a\).
\end{definition}

\begin{lemma}       \label{LEMooVUSMooWisQpD}
	Pour tout élément \( a\) d'un anneau nous avons \( a\cdot 0=0\) et \( 0\cdot a = 0\)
\end{lemma}

\begin{proof}
	L'élément \( 0\) est le neutre de l'addition. Il peut être écrit \( 1+(-1)\), et en utilisant la distributivité (à gauche),
	\begin{equation}
		a\cdot 0=a\cdot (1+(-1))=a+(-a)=0.
	\end{equation}
 	L'autre égalité se prouve de même avec la distributivité à droite.
\end{proof}

\begin{proposition}     \label{PROPooNCCGooXjVyVt}
	Dans un anneau\footnote{Définition \ref{DefHXJUooKoovob}.} non nul, le neutre pour l'addition est distinct du neutre pour la multiplication.
\end{proposition}
\begin{proof}
	Supposons par contraposée que dans un anneau \( A\), nous ayons \( 1 = 0 \). Alors, pour tout \( a \in A \), on a \( a = 1a = 0a = 0 \): les égalités étant justifées respectivement par la définition de \( 1 \), l'hypothèse \( 1 = 0 \) , et le lemme \ref{LEMooVUSMooWisQpD}. Ainsi, \( a = 0\) pour tout \( a\in A\) et l'anneau est nul.
\end{proof}

\begin{lemma}[\cite{MonCerveau}]        \label{LEMooLTERooVKgqjn}
	Un peu d'arithmétique. Soit un anneau \( A\) et deux éléments \( a, b \in A\). Nous avons les égalités suivantes:
	\begin{enumerate}
		\item	    \label{ITEMooWNPVooFeMQAp}
		      Si \( a\neq 0\) alors \( -a\neq 0\).
		\item       \label{ITEMooUGHCooOPgoeR}
		      \( 1\times 1=1\).
		\item       \label{ITEMooJMBSooVgvVwg}
		      \( (-1)\times a=-a\).
		\item       \label{ITEMooXJGMooKNLlHU}
		      \( -(-a)=a\).
		\item       \label{ITEMooYMRKooHVYYKU}
		      \((-1)\times (-1)=1\).
		\item	    \label{ITEMooAnneauArith5}
		      \( (-a)b=-ab\)
		\item	    \label{ITEMooHUJFooKkylVt}
		      \( (-a)(-b)=ab\)
	\end{enumerate}
\end{lemma}

\begin{proof}
	En plusieurs parties.
	\begin{subproof}
		\spitem[Pour \ref{ITEMooWNPVooFeMQAp}]
		Si \( -a=0\) alors en ajoutant \( a\) deux côtés, nous avons \( 0=a\) qui est contraire à l'hypothèse.
		\spitem[Pour \ref{ITEMooUGHCooOPgoeR}]
		La définition de \( 1\) est que \( 1\times a=a\) pour tout \( a\). En particulier pour \( a=1\) nous avons le résultat.
		\spitem[Pour \ref{ITEMooJMBSooVgvVwg}]
		Nous avons
		\begin{equation}
			(-1)\times a + a= a\times \big( (-1)+1 \big)=a\times 0=0.
		\end{equation}
		Nous avons utilisé le fait que la multiplication était distributive et que le zéro était absorbant (lemme \ref{LEMooVUSMooWisQpD}).

		\spitem[Pour \ref{ITEMooXJGMooKNLlHU}]
		Nous avons \( -a+a=0\) par définition de la notation \( -a\). Donc \( a\) est bien l'inverse de \( -a\) pour l'addition.

		\spitem[Pour \ref{ITEMooYMRKooHVYYKU}]
		En utilisant les points \ref{ITEMooJMBSooVgvVwg} et \ref{ITEMooXJGMooKNLlHU} nous avons
		\begin{equation}
			(-1)\times (-1)=-(-1)=1.
		\end{equation}
		\spitem[Pour \ref{ITEMooAnneauArith5} et \ref{ITEMooHUJFooKkylVt}]
		Par propriété de distributivité\footnote{Définition~\ref{DefHXJUooKoovob}, point~\ref{ITEMooGMNOooSTGiXw}.}, on a \( ab + (-a)b  = (a+(-a))b = 0b = 0 \); donc l'opposé de \( ab \) s'écrit \( -ab \) ou bien \( (-a)b \).

		Par la même propriété (avec l'autre égalité), on a \( (-a) b + (-a)(-b) = (-a)(b+(-b)) = (-a) 0 = 0 \), donc \( (-a) b \) a pour opposé \( (-a)(-b)=ab\).
	\end{subproof}
\end{proof}

\begin{normaltext}	\label{NORooDistributiviteSoustraction}
	La distributivité de la partie \ref{ITEMooGMNOooSTGiXw} de la définition \ref{DefHXJUooKoovob} ne traite que de l'addition; pas de la soustraction. Voici un lemme qui dit que la distriutivité fonctionne aussi avec les soustractions.
\end{normaltext}

\begin{lemma}[\cite{BIBooZFPUooIiywbk}]     \label{LEMooVPYUooRzexke}
	Soient un anneau \( A\) ainsi que \( a,b,c\in A\). Alors
	\begin{equation}
		a(b-c)=ab-ac.
	\end{equation}
\end{lemma}

\begin{proof}
	Nous avons le calcul suivant :
	\begin{subequations}
		\begin{align}
			a(b-c)+ac & =a\big( (b-c)+c \big)     \label{SUBEQooKCOWooFeOHUM} \\
			          & =ab.       \label{SUBEQooMLLOooNRmIYM}
		\end{align}
	\end{subequations}
	Justifications :
	\begin{itemize}
		\item Pour \ref{SUBEQooKCOWooFeOHUM}. Distributivité.
		\item Pour \ref{SUBEQooMLLOooNRmIYM}. Parce que \( (b-c)+c=b\).
	\end{itemize}
	Nous avons donc \( a(b-c)+ac=ab\) et donc l'égalité demandée en ajoutant \( -ac\) des deux côtés.
\end{proof}


\begin{definition}[Morphisme d'anneaux\cite{ooZRUJooXyxPqQ}]      \label{DEFooSPHPooCwjzuz}
	Si \( (A,+,\cdot)\) et \( (B,+,\cdot)\) sont des anneaux\footnote{Définition \ref{DefHXJUooKoovob}.}, un \defe{morphisme d'anneaux}{morphisme!d'anneaux} est une application \( f\colon A\to B\) telle que
	\begin{enumerate}
		\item \( f(a+b)=f(a)+f(b)\)
		\item \( f(a\cdot b)=f(a)\cdot f(b)\)
		\item \( f(1)=1\).
	\end{enumerate}
	Étant bien entendu que les significations de \( 1\), \( +\) et \( \cdot\) sont différentes à gauche et à droite.
\end{definition}


%-------------------------------------------------------
\subsection{Positivité}
\label{SUBooPositiviteAnneaux}
%----------------------------------------------------

\begin{definition}[\cite{MonCerveau}]	\label{DEFooZRMFooCtzMov}
	Soit un anneau\footnote{Définition \ref{DefHXJUooKoovob}.} \( A\). Une \defe{positivité}{positivité} sur \( A\) est une partie \( A^+\subset A\) telle que
	\begin{enumerate}
		\item
		      Pour tout \( a\neq 0\) dans \( A\), nous avons\footnote{\emph{xor} est le ou exclusif. L'un ou bien l'autre est vrai, mais on ne peut avoir les deux en même temps.} \( a\in A^+ \text{ xor } -a\in A^+\).
		\item
		      Pour tout \( a,b\in A^+\), nous avons \( a+b\in A^+\).
	\end{enumerate}
	Oui, il faudrait dire une \emph{stricte} positivité.
\end{definition}

\begin{proposition}[\cite{MonCerveau}]	\label{PROPooKLOPooBgQqhM}
	Soit \( A^+\) est une positivité\footnote{Définition \ref{DEFooZRMFooCtzMov}.} sur \( A\). Pour \( a, b \in A \), on définit la relation \( a\leq b\) si et seulement si \( b-a\in A^+\cup\{ 0 \}\). Cette relation est une relation d'ordre\footnote{Définition \ref{DefooFLYOooRaGYRk}.} sur \( A \).
\end{proposition}

\begin{proof}
	Nous vérifions les conditions de la définition \ref{DefooFLYOooRaGYRk}.
	\begin{subproof}
		\spitem[Réflexivité]
		%-----------------------------------------------------------
		Vu que \( a-a=0\) et que \( 0\in A^\cup\{ 0 \}\), nous avons \( a\leq a\).
		\spitem[Antisymétrie]
		%-----------------------------------------------------------
		Si \( a\leq b\) et \( b\leq a\), nous avons
		\begin{subequations}
			\begin{numcases}{}
				b-a\in A^+\cup\{ 0 \}\\
				a-b\in A^+\cup\{ 0 \}.
			\end{numcases}
		\end{subequations}
  		Par définition de la positivité, on ne peut avoir \( a-b\) et \( b-a\) tous les deux dans \( A^+\). Donc soit \( a-b=0\), soit \( b-a=0\) (ou les deux en même temps, ce qui est le cas). Quoi qu'il en soit, on en déduit \( a = b \).
		\spitem[Transitivité]
		%-----------------------------------------------------------
		Supposons que \( a\leq b\) et \( b\leq c\). Nous avons alors
		\begin{subequations}
			\begin{numcases}{}
				b-a\in A^+\cup\{ 0 \}\\
				c-b\in A^+\cup\{ 0 \}.
			\end{numcases}
		\end{subequations}
		Si l'un des deux est nul, c'est facile. Nous supposons qu'ils sont tous les deux dans \( A^+\). Alors leur somme est dans \( A^+\) : \( (b-a)+(c-b)\in A^+\). Cela donne tout de suite \( c-a\in A^+\) et donc \( a\leq c\) comme demandé.
	\end{subproof}
\end{proof}

%-------------------------------------------------------
\subsection{Diviseurs, pgcd et ppcm}
\label{SUBooDiviseursPGCDPPCM}
%----------------------------------------------------

\begin{definition}[Diviseurs dans un anneau]\label{DiviseursAnneau}
	Soit un anneau\footnote{Anneau,définition \ref{DEFooSPHPooCwjzuz}.} \( A\). Soient \( a, b \in A \). On dit que \( a\) divise \( b\), ou que \( a\) est un \defe{diviseur (à gauche)}{diviseur!dans un anneau} de \( b\) si il existe \( c \in A \) tel que \( ac = b \). On dit que c'est un diviseur de \( b\) à droite si \( ca = b \) pour un certain \( c \in A \).

	Dans un anneau commutatif, nous disons que \( a\) \defe{divise}{divise} \( b\) si il existe \( x\in A\) tel que \( ax=b\). Dans ce cas nous écrivons \( a\divides b\).
\end{definition}

\begin{normaltext}	\label{NORMooAnneauIntegreIntro}
	Un cas particulier est le cas des diviseurs de zéro. L'absence de tels diviseurs dans un anneau est une propriété intéressante: on dit dans ce cas que l'anneau est \emph{intègre}. Nous définirons cela proprement en la définition \ref{DEFooTAOPooWDPYmd}, grâce à des propriétés équivalentes.
\end{normaltext}

\begin{definition}\label{DEFooAnneauEltRegulier}
	Un élément \( a\in A\) est \defe{régulier à droite}{anneau!élément régulier} si, dès lors qu'un élément \( b \in a \) est tel que \( ba=0\), alors \( b=0\). Il est régulier à gauche si \( ab=0\) implique \( b=0\).
\end{definition}

\begin{proposition}	\label{PROPooRegulierMagmaAnneau}
	Soit \( A \) un anneau. La notion de régularité dans un anneau est équivalente à la notion de régularité\footnote{Définition~\ref{DEFooIJIEooZaAdSs}} dans le magma \( A \) muni de l'opération de multiplication.
\end{proposition}

\begin{proof}
	En deux étapes. On ne va faire que la régularité à gauche, il suffit d'adapter pour la régularité à droite.
 	\begin{subproof}
  		\item[Régulier dans le magma implique régulier dans l'anneau]
    			Soit \( a \in A \). On sait, par régularité de \( A \) dans le magma, que pour tous \( x \) et \( y \) de \( A \), dès lors que \( ax = ay \), alors \( x = y \).

       			Soit \( b \in A \). Supposons que \( ab = 0 \). Le lemme \ref{LEMooVUSMooWisQpD} permet d'écrire \( ab = a0 \). En prenant \( x = b \) et \( y = 0 \), on en déduit \( b = 0 \), soit la régularité de \( a \) dans l'anneau.
	  	\item[Régulier dans l'anneau implique régulier dans le magma]
    			Soit \( a \in A \). On sait que l'égalité \( ab=0\) implique \( b=0\).

       			Soit \( x, y \in A \). Supposons que \( ax = ay \); alors \( ax - ay = 0 \) et par distributivité soustractive\footnote{Lemme \ref{LEMooVPYUooRzexke}.}, \( a(x-y) = 0\). Mais la régularité d'anneau dit alors que \( x - y = 0 \), et donc que \( x = y\). Donc \( a \) est régulier dans le magma \( A \).
	  \end{subproof}
\end{proof}

\begin{definition}          \label{DefrYwbct}
	Soient \( A\) un anneau commutatif et \( S\subset A\). Nous disons que \( \delta\in A\) est un \defe{PGCD}{pgcd!dans un anneau intègre} de \( S\) si
	\begin{enumerate}
		\item
		      \( \delta\) divise\footnote{Définition \ref{DiviseursAnneau}.} tous les éléments de \( S\).
		\item       \label{ITEMooVCKGooWDXZOj}
		      si \( d\) divise également tous les éléments de \( S\), alors \( d\) divise \( \delta\).
	\end{enumerate}
	Nous disons que \( \mu\in A\) est un \defe{PPCM}{ppcm!dans un anneau intègre} de \( S\) si
	\begin{enumerate}
		\item
		      \( S\divides \mu\),
		\item
		      si \( S\divides m\), alors \( \mu\divides m\).
	\end{enumerate}
	Si \( P\) et \( Q\) sont des polynômes, ce que nous notons \( \pgcd(P,Q)\) est l'unique polynôme unitaire dans \( \pgcd\big( \{ P,Q \} \big)\). Voir \ref{NORMooUJDJooWfijxT}.
\end{definition}

\begin{remark}\label{REMooExistenceUnicitePGCD}
	On parle d'existence de pgcd dans \ref{PROPooPZXNooNpVZCm}.

	Au sens de la définition \ref{DefrYwbct}, le pgcd n'est pas unique. Dans \( \eZ\) par exemple les nombres \( 4\) et \( -4\) sont tous deux pgcd de \( \{4,16  \}\).

	Dans \( \eZ\) cependant, nous modifions implicitement la définition et nous n'acceptons que les positifs, de telle sorte que l'unique pgcd soit effectivement le plus grand pour l'ordre usuel sur \( \eZ\).

	Pour l'unicité dans \( \eZ\), voir \ref{LEMooBJVJooFyuFeN}.
\end{remark}


%-------------------------------------------------------
\subsection{Règle des signes dans un anneau totalement ordonné}
\label{SUBooSignesAnnTotOrdonne}
%----------------------------------------------------

\begin{definition}[Anneau totalement ordonné\cite{MonCerveau}]	\label{DEFooWACWooDWvXKJ}
	Un anneau \( A\) est \defe{totalement ordonné}{ordre!dans un anneau}\index{corps!ordonné} si, sur \( A \),  il existe une relation d'ordre total\footnote{Définition~\ref{DEFooVGYQooUhUZGr}.} compatible avec la structure d'anneau, c'est-à-dire tel que
	\begin{enumerate}
		\item       \label{ITEMooZISJooWNxnBj}
		      \( x\leq y\) implique \( x+z\leq y+z\) pour tout \( x,y,z\in A\)
		\item   \label{CONDooBYYDooElXgPO}
		      \( x\geq 0\) et \( y\geq 0\) implique \( xy\geq 0\).
	\end{enumerate}
\end{definition}


\begin{proposition}[\cite{MonCerveau}]	\label{PROPooELXCooCYzEVD}
	Soit un anneau totalement ordonné \( A\).
	\begin{enumerate}
		\item   \label{ITEMooRRUGooRUpYwt}
		      Si \( a\leq 0\) alors \( -a\geq 0\).
		\item	\label{ITEMooNKYCooCmqHsO}
		      Si \( a\geq 0\) alors \( -a\leq 0\)
		\item	\label{ITEMooXOTGooYxKubR}
		      Si \( a,b\geq 0\) alors \( a+b\geq 0\)
	\end{enumerate}
\end{proposition}

\begin{proof}
	Point par point.
	\begin{subproof}
		\spitem[Pour \ref{ITEMooRRUGooRUpYwt}]
		%-----------------------------------------------------------
		En vertu de \ref{DEFooWACWooDWvXKJ}\ref{ITEMooZISJooWNxnBj} nous pouvons ajouter \( -a\) est deux côtés de \( a\leq 0\) : \( a-a\leq 0-a\) et donc \( 0\leq -a\).

		\spitem[Pour \ref{ITEMooNKYCooCmqHsO}]
		%-----------------------------------------------------------
		On écrit \( a\geq 0\) et nous faisons \( -a\) des deux côtés pour obtenir \( a-a\geq 0-a\), c'est à dire \( 0\geq -a\).

		\spitem[Pour \ref{ITEMooXOTGooYxKubR}]
		%-----------------------------------------------------------
		Nous avons \( a\geq 0\). Nous ajoutons \( b\) des deux côtés pour avoir \( a+b\geq b\geq 0\).
	\end{subproof}
\end{proof}


\begin{lemma}[Règle des signes\cite{ooTKEHooQuaFuD}]        \label{LEMooXJTAooZauchx}
	Soit un anneau totalement ordonné \( A\) ainsi que \( x,y\in A\). Nous avons :
	\begin{enumerate}
		\item		\label{ITEMooHESRooKrDult}
		      Si \( x,y\geq 0\) alors \( xy\geq 0\).
		\item		\label{ITEMooNLAQooHtERZJ}
		      Si \( a\geq 0\) et \( b\leq 0\) alors \( ab\leq 0\).
		\item		\label{ITEMooDZUJooVWuLbD}
		      Si \( a,b\leq 0\) alors \( ab\geq 0\)
		\item       	\label{ITEMooRGYAooCUIfss}
		      \( 0<1\).
		\item       	\label{ITEMooMRNHooLglPKn}
		      Si \( x\geq 0\) et \( x \) est inversible dans \( A \), alors \( x^{-1}\geq 0\).
	\end{enumerate}
\end{lemma}

\begin{proof}
	Plusieurs points.
	\begin{subproof}
		\spitem[Pour \ref{ITEMooHESRooKrDult}]
		%-----------------------------------------------------------
		C'est directement la définition \ref{DEFooWACWooDWvXKJ}\ref{CONDooBYYDooElXgPO}.
		\spitem[Pour \ref{ITEMooNLAQooHtERZJ}]
		%-----------------------------------------------------------
		En posant \( x = a \) et \( y=-b \geq 0\)  (par la proposition \ref{PROPooELXCooCYzEVD}\ref{ITEMooRRUGooRUpYwt}), nous avons \( xy\geq 0\) par la définition \ref{DEFooWACWooDWvXKJ}\ref{CONDooBYYDooElXgPO}. Nous avons alors \( a(-b) = -ab \geq 0\) par \ref{ITEMooHESRooKrDult}, et donc \( ab \leq 0 \).

		\spitem[Pour \ref{ITEMooDZUJooVWuLbD}]
		%-----------------------------------------------------------
		Nous posons \( a'=-a\) et \( b'=-b\). Par \ref{PROPooELXCooCYzEVD}\ref{ITEMooRRUGooRUpYwt} nous avons \( a',b'\geq 0\). Donc \( a'b'\geq 0\) par le point \ref{ITEMooNLAQooHtERZJ}. Mais \( a'b'=(-a)(-b)=ab\) par le lemme \ref{LEMooLTERooVKgqjn}\ref{ITEMooHUJFooKkylVt}.

		\spitem[Pour \ref{ITEMooRGYAooCUIfss}]
		%-----------------------------------------------------------
		Supposons par l'absurde que \( 1\leq 0\). L'utilisation de \ref{ITEMooDZUJooVWuLbD} avec \( a = b = 1 \) donne \( ab = 1\cdot 1 = 1 \geq 0 \), ce qui nous contredit. Ainsi, \( 1> 0\).

		\spitem[Pour \ref{ITEMooMRNHooLglPKn}]
		%-----------------------------------------------------------
		Si \( x^{-1}<0\), alors \( 1=xx^{-1}<0\). Mais comme nous venons de voir que \( 1\geq 0\) nous déduisons que \( x^{-1}\geq 0\).
	\end{subproof}
\end{proof}


%-------------------------------------------------------
\subsection{Valeur absolue, corps valué}
\label{SUBooValAbsCorpsValue}
%----------------------------------------------------

\begin{definition}[Valeur absolue, corps valué\cite{BIBooVKGHooSPijZp,BIBooNRMUooJmwzpn}]       \label{DEFooBWXXooAkBBRS}
	Un \defe{anneau valué}{anneau valué}\index{corps valué} \( A\) est un annau totalement ordonné muni d'une application \( | . |\colon A\to \eK\) telle que
	\begin{enumerate}
		\item
		      \( | x |\geq 0\) pour tout \( x\in \eK\),
		\item
		      \( | x |=0\) si et seulement si \( x=0\),
		\item
		      \( | x+y |\leq | x |+| y |\)
		\item
		      \( | xy |\leq | x | | y |\).
	\end{enumerate}
	Une telle application est une \defe{valeur absolue}{valeur absolue}.
\end{definition}

\begin{normaltext}\label{NORooLienCorpsValueEspaceMetrique}
	Un corps valué sera un espace topologique métrique dans la définition \ref{PROPooAWAKooKRmbGT}. Dans le cas d'un corps totalement ordonné, nous avons une valeur absolue donnée par \ref{DEFooJXKVooErANPh} et les principales propriétés dans le lemme \ref{LemooANTJooYxQZDw}.
\end{normaltext}

\begin{normaltext}\label{NORooValAbsGenerale}
	La plupart du temps, la valeur absolue \( | . |\) sera à valeurs dans \( \eR^+\). Cependant nous avons besoin d'une définition générale parce que nous allons devoir définir ce qu'est une suite de Cauchy (définition \ref{DefKCGBooLRNdJf}\ref{ItemVXOZooTYpcYN}) dans un corps avant la définition des réels; les réels étant définis comme étant des suites de Cauchy dans \( \eQ\).
\end{normaltext}


\begin{propositionDef}[Valeur absolue\cite{MonCerveau}]	\label{DEFooJXKVooErANPh}
	Si \( A\) est un anneau totalement ordonné\footnote{Définition \ref{DEFooWACWooDWvXKJ}.}, alors l'application
	\begin{equation}      \label{EQooNONAooHLSERO}
		| x |=\begin{cases}
			-x & \text{si }x< 0  \\
			0  & \text{si }x= 0  \\
			x  & \text{si } x>0.
		\end{cases}
	\end{equation}
	est une valeur absolue\footnote{Définition \ref{DEFooBWXXooAkBBRS}.}. Elle vérifie de plus
	\begin{equation}		\label{EQooORNAooBYFeFk}
		| xy |=| x || y |.
	\end{equation}

	Par défaut, lorsque nous parlons de la valeur absolue sur \( A\), c'est celle-ci.
\end{propositionDef}

\begin{proof}
	Il faut vérifier les conditions de la définition \ref{DEFooBWXXooAkBBRS}.
	\begin{subproof}
		\spitem[\( | x |=0\) si et seulement si \( x=0\)]
		%-----------------------------------------------------------
		C'est la définition et le fait que \( -x\neq 0\) lorsque \( x\neq 0\) par le lemme \ref{LEMooLTERooVKgqjn}\ref{ITEMooWNPVooFeMQAp}.

		\spitem[\( | x+y |\leq | x |+| y |\)]
		%-----------------------------------------------------------
		Il y a 4 possibilités suivant la positivité de \( x\) et \( y\).
		\begin{subproof}
			\spitem[Si \( x,y\geq 0\)]
			%-----------------------------------------------------------
			Alors \( x+y\geq 0\) par la proposition \ref{PROPooELXCooCYzEVD}\ref{ITEMooXOTGooYxKubR}. Donc \( | x |=x\), \( | y |=y\) et \( | x+y |=x+y\).

			\spitem[Si \( x\geq 0\) et \( y<0\)]
			%-----------------------------------------------------------
			Nous posons \( y'=-y\), et nous avons deux possibilités : \( y'\geq x\) et \( y'<x\).
			\begin{subproof}
				\spitem[Si \( y'<x\)]
				%-----------------------------------------------------------
				Alors en posant \( a=x-y'\) nous avons \( a>0\). Nous avons ensuite :
				\begin{equation}
					| x+y |=| y'+a+y |=| y'+a-y' |=| a |=a.
				\end{equation}
				Mais
				\begin{equation}
					| x |+| y |=x+y'=y'+a+y'=a+2y'\geq a=| x+y |.
				\end{equation}
				\spitem[Si \( y'\geq x\)]
				%-----------------------------------------------------------
				Alors en posant \( a=y'-x\) nous avons \( a\geq 0\) et
				\begin{equation}
					| x+y |=| x-y' |=| x-(x+a) |=| -a |=a.
				\end{equation}
				Mais aussi
				\begin{equation}
					| x |+| y |=x+y'=x+x+y=2x+a\geq a.
				\end{equation}
			\end{subproof}
			Les autres cas sont du même style.
		\end{subproof}

		\spitem[\( | xy |=| x || y |\)]
		%-----------------------------------------------------------

		Il y a 4 possibilités.
		\begin{subproof}
			\spitem[Si \( x,y\geq 0\)]
			%-----------------------------------------------------------
			Nous avons directement \( | x |=x\), \( | y |=y\). Par ailleurs \( A\) est un anneau totalement ordonné; la définition \ref{DEFooWACWooDWvXKJ}\ref{CONDooBYYDooElXgPO} donne \( xy\geq 0\) et donc \( | xy |=xy\).

			\spitem[Si \( x\geq 0\) et \( y\leq 0\)]
			%-----------------------------------------------------------
			La définition de la valeur absolue donne \( | x |=x\) et \( | y |=-y\). Par ailleurs la proposition \ref{LEMooXJTAooZauchx}\ref{ITEMooNLAQooHtERZJ} dit que \( xy\leq 0\) de telle sorte que \( | xy |=-xy\).

			\spitem[Si \( x\leq 0\) et \( y\geq 0\)]
			%-----------------------------------------------------------
			Même chose.

			\spitem[Si \( x,y\leq 0\)]
			%-----------------------------------------------------------
			Même chose, en utilisant \ref{LEMooXJTAooZauchx}\ref{ITEMooDZUJooVWuLbD}.
		\end{subproof}
	\end{subproof}
\end{proof}



\begin{lemma}[Propriétés de la valeur absolue]  \label{LemooANTJooYxQZDw}
	Soit \( A \) un anneau totalement ordonné muni de la valeur absolue usuelle (définition \ref{DEFooJXKVooErANPh}). Si \( x,y\in A\),  alors :
	\begin{enumerate}
		\item       \label{ITEMooSDNHooDnjScE}
		      \( | x |\geq 0\)
		\item       \label{ITEMooLQLTooTJTPVM}
		      \( | x |=0\) si et seulement si \( x=0\)
		\item       \label{ITEMooVJAEooOEatzY}
		      \( | -x |=| x |\).
		\item			\label{ItemooOMKNooRlanvk}
		      \( | x+y |\leq | x |+| y |\).
		\item		\label{ITEMooISWFooMeOtjZ}
		      \( | x-y |\leq | x-z |+| z-y |\).
		\item		\label{ITEMooEFMLooYVCuHD}
		      \( | xy |=| x | |y |\)
	\end{enumerate}
\end{lemma}

\begin{proof}
	Point par point
	\begin{subproof}
		\spitem[\ref{ITEMooSDNHooDnjScE}]
		Si \( x\geq 0\) alors c'est vrai. Sinon, \( x\leq 0\) et \( | x |=-x\geq 0\) par le point~\ref{ITEMooRRUGooRUpYwt} de la propostion \ref{PROPooELXCooCYzEVD}.
		\spitem[\ref{ITEMooLQLTooTJTPVM}]
		Si \( x=0\) alors \( x=-x\) et \( | x |=0\).  Si \(x\neq 0\) alors \( -x\neq 0\) (par le lemme \ref{LEMooLTERooVKgqjn} point \ref{ITEMooWNPVooFeMQAp}).  Que l'on ait \( x\) positif ou \( x\) négatif, nous aurons toujours \( | x | \pm x\neq 0\).
		\spitem[\ref{ITEMooVJAEooOEatzY}]
		Il faut décomposer en deux cas selon que \( x\geq 0\) ou  \( x\leq 0\).

		Lorsque \( x\leq 0\), on a \( | x |= - x\) par définition, et par ailleurs, \( -x\geq 0\) grâce au point \ref{ITEMooNKYCooCmqHsO} du lemme \ref{LEMooLTERooVKgqjn}, donc \( | -x |= - x = | x |\).

		Supposons \( x\geq 0\). Alors d'une part \( | x |=x\). D'autre part \( -x\leq 0\) par le point~\ref{ITEMooRRUGooRUpYwt} de la propostion \ref{PROPooELXCooCYzEVD}, de telle sorte que
		\begin{equation}
			| -x |=-(-x)=x,
		\end{equation}
		cette dernière égalité étant justifiée par le point \ref{ITEMooXJGMooKNLlHU} du lemme \ref{LEMooLTERooVKgqjn}.

		Nous avons donc bien \( | x |=| -x |\) quel que soit le cas considéré.
		\spitem[\ref{ItemooOMKNooRlanvk}]
		Nous supposons que \( x\leq y\) et nous distinguons divers cas suivant la positivité de \( x\) et \( y\).
		\begin{enumerate}
			\item
			      Si \( x,y\geq 0\). Dans ce cas, \( x+y\geq y\geq 0\), donc \( | x+y |=x+y=| x |+| y |\).
			\item
			      Si \( x,y\leq 0\). Dans ce cas, \( x+y\leq 0\) et nous avons \( | x+y |=-x-y=| x |+| y |\).
			\item
			      Si \( x\leq 0\) et \( y\geq 0\). Nous subdivisons encore en deux cas suivant que \( x+y\) est positif ou négatif. Si \( x+y\geq 0\), alors nous écrivons successivement
			      \begin{subequations}
				      \begin{align}
					      x   & \leq 0                         \\
					      x+y & \leq y\leq y+| x |=| x |+| y |
				      \end{align}
			      \end{subequations}
			      et donc \( | x+y |=x+y\leq | x |+| y |\).

			      Nous supposons à présent que \( x\leq 0\), \( y\geq 0\) et \( x+y\leq 0\). Dans ce cas il suffit d'écrire \( | x+y |=| (-x)+(-y) |\) pour retomber dans le cas précédent à inversion près de \( x\) et \( y\).
		\end{enumerate}

		\spitem[Pour \ref{ITEMooISWFooMeOtjZ}]
		%-----------------------------------------------------------
		Soient \( x,y,z\in \eK\). En utilisant \ref{ItemooOMKNooRlanvk} nous avons
		\begin{equation}
			| x-y |= |  (x-z)+(z-y) |\leq | x-z |+| z-y |.
		\end{equation}

		\spitem[Pour \ref{ITEMooEFMLooYVCuHD}]
		% -------------------------------------------------------------------------------------------- 
		Il suffit de prendre les 4 cas suivant les signes de \( x\) et \( y\), et d'utiliser les règles de signes du lemme \ref{LEMooXJTAooZauchx} dans la définition \eqref{EQooNONAooHLSERO}.
	\end{subproof}
\end{proof}




%+++++++++++++++++++++++++++++++++++++++++++++++++++++++++++++++++++++++++++++++++++++++++++++++++++++++++++++++++++++++++++
\section{Les naturels}
\label{SECooPJSYooNYaIaq}
%+++++++++++++++++++++++++++++++++++++++++++++++++++++++++++++++++++++++++++++++++++++++++++++++++++++++++++++++++++++++++++

\begin{definition}[\cite{RWWJooJdjxEK}]     \label{DEFooBJBOooWlblAx}
	Un \defe{triplet naturel}{triplet naturel} est un triplet \( (\mN, o, s)\) où \( \mN\) est un ensemble, \( o\) est un élément de \( \mN\) et \( s\) est une application \( s\colon \mN\to \mN\) satisfaisant les propriétés suivantes :
	\begin{enumerate}
		\item
		      \( s\) est injective,
		\item       \label{ITEMooQAKJooGKdJsM}
		      \( s(\mN)=\mN\setminus \{ o \} \)
		\item       \label{ITEMooXPYEooFajywh}
		      Si \( \mA\subset \mN\) est tel que \( o\in \mA\) et \( s(\mA)\subset \mA\), alors \( \mA=\mN\).
	\end{enumerate}
\end{definition}

\begin{normaltext}	\label{NORMooTripletNaturelGratterTheoEns}
	Le théorème suivant est typiquement de ceux qui vont demander de gratter la théorie axiomatique des ensembles avec une certaine précision. Comme nous l'avons dit en introduction \ref{NorooFridoIntro1}, nous n'avons pas l'intention de préciser nos axiomes, mais notre principe de construction est connu et dû à John Von Neumann.
 	% TODOooCiteBiblio à indiquer dans la biblio la réf vers Wikipedia par exemple: https://fr.wikipedia.org/wiki/Construction_des_entiers_naturels , et aussi https://fr.wikipedia.org/wiki/Axiome_de_l%27infini

	Nous verrons plus tard, en \ref{NORMooQXASooMXqhjI}, que toute partie infinie d'un triplet naturel fournit un nouveau triplet naturel; il en existe donc plusieurs.
\end{normaltext}

\begin{theorem}     \label{THOooOXMHooXYgMqb}
	Il existe un triplet naturel.
\end{theorem}

\begin{proof}
	Nous effectuons une construction de plusieurs ensembles, puis vérifier que l'ensemble final, muni du nécessaire, forme un triplet naturel.

 	\begin{subproof}
		\item[Le premier élément]
  		%-----------------------------
    			On pose \( o = \varnothing \). Dans une axiomatisation raisonnable, l'ensemble vide existe toujours.

       		\item[Le principe de construction]
	 	%-----------------------------
   			On construit \( A_1 = o \cup \{ o \} \) et on pose \( s(o) = A_1 \).
			Puis on définit \( A_2 = A_1 \cup \{ A_1 \} \), et on pose \( s(A_1) = A_2 \).

      			De manière générale, une fois un ensemble \( A_i \) construit sur ce principe, on ajoute l'ensemble \( A_i \cup \{ A_i \} \) à notre construction, et on pose \( s(A_i) = A_i \cup \{ A_i \} \).

  			Finalement, l'ensemble \( \mN \) est la réunion de tous les ensembles \( A_i \) ainsi construits, et l'application \( s \) est définie par  \( s(A) = A \cup \{ A \} \).
     
     		\item[Une remarque intéressante]
	 	%-----------------------------
   			En observant bien, on remarque que \( o \in A_1 \) mais aussi \( o \subset A_1 \). De même, \( A_1 \subset A_2 \) et donc \( o \in A_2 \) tout comme \( A_1 \in A_2 \). On a aussi \( o \subset A_2 \)

      			En fait, dans tous les ensembles que l'on construit, on aura toujours un pricipe d'inclusion, et aussi d'appartenance. Bien que l'intuition nous fait penser que ça fonctionne, la preuve n'est pas évidente.

     		\item[On a un triplet naturel]
	 	%-----------------------------
   			Montrons que  \( (\mN, o, s)\) est bien un triplet naturel.
      			\begin{enumerate}
				\item \( s \) est-elle injective? Si \( A \) et \( B \) sont deux éléments de \( \mN \) avec \( A \neq B \), alors par la remarque ci-dessus, et quitte à inverser \( A \) et \( B \), on a \( A \subset B \). On en déduit que \( B \) est construit après \( A \), et donc \( A \cup \{ A \} \subset B \). Mais, toujours par construction, \( \{ B \} \)  ne peut être dans \( A \), ni être égal au singleton \( \{ A \} \): ainsi \( A \cup \{ A \} \subset B \cup \{ B \} \) mais \( A \cup \{ A \} \neq B \cup \{ B \} \), soit \( s(A) \neq s(B) \).

    				\item A-t-on \( s( \mN ) = \mN \setminus \{ o \} \)? Par construction, un élément \( B \in  \mN \setminus \{ o \} \) est de la forme \( A \cup \{ A \} \) pour un certain ensemble \( A \), lui-même dans \( \mN \).
				\item Soit \( \mA \) une partie de \( \mN \) tel que \( o \in \mA \) et \( s(\mA) \subset \mA \). Cette dernière hypothèse nous dit que, dès que \( A \in \mA \), alors \( A \cup \{ A \} \in \mA \). Comme \( \mA \) contient \( o = \emptyset \), il contient aussi \( A_1 \), donc aussi \( A_2 \), et par suite \( \mN \) tout entier. Finalement, \( \mA \subset \mN \) et \( \mN \subset \mA \), donc l'égalité.
			\end{enumerate}
  	\end{subproof}
\end{proof}

\begin{normaltext}[Définition de \( \eN\)]    \label{NORooDefinitionNChoixTriplet}
	Pour la suite, nous considérons un triplet naturel \( (\mN,o,s)\) et nous notons \( \eN=\mN\). Donc la nature de tous les objets que nous allons considérer à partir de maintenant dépend du choix de triplet naturel que nous faisons à présent. Le théorème \ref{THOooFUXMooJuigHK} % \futureok
	nous assurera que peu de choses devraient réellement dépendre de ce choix.

	Nous notons \( 0\) l'élément \( o\) et \( 1\) l'élément \( s(0)\). C'est tout ce dont nous avons besoin dans l'immédiat.
\end{normaltext}

%---------------------------------------------------------------------------------------------------------------------------
\subsection{Applications définies par récurrence}
\label{SUBooApplicationRecurrence}
%---------------------------------------------------------------------------------------------------------------------------

\begin{proposition}[Récurrence\cite{RWWJooJdjxEK}]      \label{PROPooXTRCooKwrWkq}
	Soit un triplet naturel \( (\mN,o,s)\) et une application \( P\colon \mN\to \{ 0,1 \}\) vérifiant\footnote{Les plus \randomGender{pointilleux}{pointilleuses} diront que \( 1\) n'est pas encore défini. Bon j'avoue. Ce qui est important est que \( P\) prenne ses valeurs dans un ensemble contenant deux éléments distincts. Si maintenant vous râlez parce que «deux» est encore moins défini, prenez un ensemble quelconque \( A\) et dites que \( P\) prend ses valeurs dans \( \{ A, \partP(A)\}\). Mais êtes-vous bien \randomGender{certain}{certaine} que \( \partP(A)\neq A\) ?}
	\begin{enumerate}
		\item
		      \( P(o)=1\),
		\item
		      pour tout \( a\in \mN\), si \( P(a)=1\), alors \( P\big( s(a) \big)=1\).
	\end{enumerate}
	Alors \( P(x)=1\) pour tout \( x\in \mN\).
\end{proposition}

\begin{proof}
	Nous posons
	\begin{equation}
		A=\{ x\in\mN\tq P(x)=1 \}.
	\end{equation}
	Cet ensemble vérifie la propriété \ref{DEFooBJBOooWlblAx}\ref{ITEMooXPYEooFajywh}. Donc \( A=\mN\).
\end{proof}

\begin{theorem}[\cite{BIBooZFPUooIiywbk,BIBooMSSFooOOeRKE}]       \label{THOooEJPYooZFVnez}
	Soient \( E\) un ensemble, \( g\) une application de \( E\) dans \( E\) et \( b\) un élément de \( E\).  Alors il existe une unique application \( f\colon \eN\to E\) telle que :
	\begin{enumerate}
		\item
		      \( f(0)=b\)
		\item
		      \( f\big( s(n) \big)=g\big( f(n) \big)\) pour tout \( n\in \eN\setminus\{ 0 \}\).
	\end{enumerate}
\end{theorem}

\begin{proof}
	Nous commençons par l'unicité. Soient \( f_1\) et \( f_2\) deux telles applications. Nous posons
	\begin{equation}
		A=\{ n\in \eN\tq f_1(n)=f_2(n) \}.
	\end{equation}
	Nous avons \( 0\in A\) parce que \( f_1(0)=f_2(0)=b\).

	Supposons que \( f_1(k)=f_2(k)\). Alors nous avons
	\begin{equation}
		f_1\big( s(k) \big)=g\big( f_1(k) \big)=g\big( f_2(k) \big)=f_2\big( s(k) \big).
	\end{equation}
	Nous en déduisons que \( s(k)\in A\). Autrement dit \( s(A)\subset A\). La définition \ref{DEFooBJBOooWlblAx}\ref{ITEMooXPYEooFajywh} nous indique alors que \( A=\eN\), c'est-à-dire que \( f_1=f_2\).

	Nous montrons à présent l'existence en plusieurs étapes.
	\begin{subproof}
		\spitem[L'ensemble est assez grand]
		Nous considérons l'ensemble \( \mA\) des parties \( A\subset \eN\times E\) telles que
		\begin{enumerate}
			\item
			      \( (0,b)\in A\)
			\item
			      \( (n,x)\in A \Rightarrow \big( s(n),g(x) \big)\in A\).
		\end{enumerate}
		L'ensemble \( \mA\) est non vide parce que \( \eN\times E\in \mA\).
		\spitem[Le plus petit]
		Nous posons
		\begin{equation}
			G=\bigcap_{A\in \mA}A,
		\end{equation}
		et nous prouvons que \( G\in\mA\). D'abord \( (0,b)\in G\) parce que cet élément est dans chacun des \( A\in G\). Ensuite si \( (n,x)\in G\), alors pour tout \( A\in\mA \) nous avons \( (n,x)\in A\) et donc \( \big( s(n),g(x) \big)\in A\). Par conséquent \( \big( s(n),g(x) \big)\in \bigcup_{A\in\mA}A=G\).

		Pour \( n\in \eN\) nous posons
		\begin{equation}
			G_n=\{ x\in E\tq (n,x)\in G \}.
		\end{equation}
		Nous avons en particulier que \( b\in G_0\) parce que \( (0,b)\in G\).
		\spitem[\( G\) contient un \( (n,x)\) pour tout \( n\)]
		Nous prouvons que pour tout \( n\in \eN\), il existe \( x\in E\) tel que \( (n,x)\in G\). Nous faisons ça avec la proposition \ref{PROPooXTRCooKwrWkq} en posant
		\begin{equation}
			\begin{aligned}
				P\colon \eN & \to \{ 0,1 \}                           \\
				n           & \mapsto \begin{cases}
					                      1 & \text{si } G_n\neq\emptyset \\
					                      0 & \text{sinon.}
				                      \end{cases}
			\end{aligned}
		\end{equation}
		Puisque \( (0,b)\in G\) nous avons \( P(0)=1\). Supposons que \( P(k)=1\) et montrons que \( P\big( s(k) \big)=1\). Comme \( P(k)=1\), il existe \( x\in E\) tel que \( (k,x)\in G\). De ce fait, \( \big( s(k),g(x) \big)\in G\), ce qui donne \( G_{s(k)}\neq \emptyset\) et \( P\big( s(k) \big)=1\).
		\spitem[\( G_n\) est un singleton]
		Nous avons vu que \( G_n\) n'est jamais vide. Nous allons montrer que \( G_n\) est un singleton pour tout \( n\). Pour cela nous posons
		\begin{equation}
			\begin{aligned}
				P\colon \eN & \to \{ 0,1 \}                                        \\
				n           & \mapsto \begin{cases}
					                      1 & \text{si }  G_n \text{ est un singleton} \\
					                      0 & \text{sinon. }
				                      \end{cases}
			\end{aligned}
		\end{equation}
		Nous prouvons par récurrence que \( P(n)=1\) pour tout \( n\).
		\begin{subproof}
			\spitem[\( P(0)=1\)]
			Nous commençons par prouver que \( P(0)=1\). Nous savons que \( (0,b)\in G_0\). Supposons \( a\neq b\) tel que \( (0,a)\in G_0\). Alors en posant \( G'=G\setminus\{ (0,a) \}\) nous avons \( G'\in \mA\).

			En effet \( (0,b)\in G'\) parce que \( (0,b)\in G\) et \( (0,b)\neq (0,a)\). De plus si \( (n,x)\in G'\), alors \( \big( s(n),g(x) \big)\in G\). Mais comme \( s(n)\neq 0\) nous avons \( \big( s(n),g(x) \big)\neq (0,a)\) et donc \( \big( s(n),g(x) \big)\in G'\).

			L'ensemble \( G'\) serait un élément de \( \mA\) strictement inclus dans \( G\). Impossible. Donc \( G_0\) est un singleton.

			\spitem[Récurrence]
			Supposons que \( P(k)=1\), c'est-à-dire que \( G_k\) est un singleton. Soit \( e\) l'unique élément de \( G_k\) : \( (k,e)\in G\). Nous avons alors aussi que \( \big( s(k),g(e) \big)\in G\). Nous devons prouver que si \( y\in G_{s(k)}\), alors \( y=g(e)\).

			Supposons donc \( y\neq g(e)\) soit dans \( G_{s(k)}\). Nous posons
			\begin{equation}
				G'=G\setminus\{ \big( s(k),y \big) \}.
			\end{equation}
			Nous prouvons que \( G'\in\mA\). D'abord \( (0,b)\in G'\) parce que \( s(k)\neq 0\). Soit ensuite \( (m,z)\in G'\). Si \( m=k\), alors \( z=e\) (parce que par hypothèse \( G_k\) est un singleton) et nous savons que \( \big( s(m),g(e) \big)\in G'\). Si par contre \( m\neq k\), comme  \( s\) est injective, nous avons aussi \( s(m)\neq s(k)\). Donc \( \big( s(m),g(z) \big)\neq \big( s(k),y \big)\) et \( \big( s(m),g(z) \big)\in G'\). Donc \( G'\in \mA\) et est strictement plus petit que \( G\). Contradiction.

			Nous concluons que \( G_{s(k)}\) est un singleton, c'est-à-dire que \( P\big( s(k) \big)=1\).
			\spitem[Conclusion]
			Nous avons prouvé que \( G_n\) est un singleton pour tout \( n\).
		\end{subproof}

		\spitem[Et enfin]

		Nous définissons \( f(n)\) comme étant l'unique élément de \( G_n\). Puisque \( (0,b)\in G\) nous avons \( G_0=\{ b \}\) et donc \( g(0)=b\).

		Par définition de \( f\), nous avons \( \big( n,f(n) \big)\in G\). Parce que \( G\in\mA\) nous avons alors
		\begin{equation}
			\big( s(n),g\big( f(n) \big) \big)\in G.
		\end{equation}
		Autrement dit, \( G_{s(n)}= \{  g\big( f(n) \big) \}\). Cela montre que
		\begin{equation}
			f\big( s(n) \big)=g\big( f(n) \big),
		\end{equation}
		et donc que \( f\) vérifie les propriétés demandées.
	\end{subproof}
\end{proof}

\begin{normaltext}[\cite{MonCerveau}]      \label{NORMooLNXMooWIblPf}
	Le théorème \ref{THOooEJPYooZFVnez} nous permet de définir une suite \( (x_n)_{n\in \eN}\) par récurrence en posant par exemple
	\begin{subequations}
		\begin{numcases}{}
			x_1=b\\
			x_{n+1}=f(x_n).
		\end{numcases}
	\end{subequations}

	Parfois nous pouvons avoir envie de définir une suite par récurrence de telle sorte que chaque nouveau terme dépende de tous les précédents :
	\begin{subequations}
		\begin{numcases}{}
			x_1=b\\
			x_{n+1}=g(x_1,\ldots,x_{n-1}).
		\end{numcases}
	\end{subequations}
	Pour ce faire,  il faut un peu adapter.

	Posons \( E'=E\cup (E\times E)\cup (E\times E\times E)\cup\ldots\). Nous supposons avoir une application \(g \colon E'\to E'  \) qui ne prend ses valeurs que dans \( E\) : \( g(E')\subset E\). De plus nous considérons l'application
	\begin{equation}
		\begin{aligned}
			a\colon \eN & \to E'                                \\
			n           & \mapsto \big( f(0),\ldots,f(n) \big).
		\end{aligned}
	\end{equation}

	Le théorème \ref{THOooEJPYooZFVnez} appliqué à \( g\circ a\) nous donne l'existence et l'unicité d'une application \(f \colon \eN\to E'  \) telle que
	\begin{subequations}
		\begin{numcases}{}
			f(0)=b\\
			f(n+1)=g\big( f(0),\ldots,f(n) \big).
		\end{numcases}
	\end{subequations}
\end{normaltext}


\begin{corollary}[\cite{BIBooZFPUooIiywbk}]       \label{CORooVNHKooRkKtXf}
	Soient deux ensembles \( X,Y\), une application \( \alpha\colon X\to Y\) et une application \( \beta\colon Y\to Y\). Alors il existe une unique application \( H\colon X\times \eN\to Y\) telle que
	\begin{enumerate}
		\item
		      \( H(x , 0)   = \alpha(x)\)   pour tout élément \( x\in X\);
		\item
		      \( H(x , n+1) = \beta( H( x , n) )\) pour tout élément \( x\in X\) et pour tout \( n\in \eN\).
	\end{enumerate}
\end{corollary}

\begin{proof}
	Pour faire le lien avec les notations du théorème \ref{THOooEJPYooZFVnez}, nous notons \( E=\Fun(X,Y)\), \( b=\alpha\in E\) et
	\begin{equation}
		\begin{aligned}
			g\colon E & \to E                 \\
			s         & \mapsto \beta\circ s.
		\end{aligned}
	\end{equation}
	Le théorème \ref{THOooEJPYooZFVnez} donne alors l'existence d'une application \( f\colon \eN\to E\) telle que
	\begin{enumerate}
		\item
		      \( f(0)=b\)
		\item
		      \( f(n+1)=g\big( f(n) \big)\).
	\end{enumerate}
	Nous définissons alors
	\begin{equation}
		\begin{aligned}
			H\colon X\times \eN & \to Y          \\
			(x,n)               & \mapsto f(n)x,
		\end{aligned}
	\end{equation}
	et nous vérifions qu'elle satisfait aux exigences.

	\begin{enumerate}
		\item
		      D'abord nous avons
		      \begin{equation}
			      H(x,0)=f(0)x=b(x)=\alpha(x).
		      \end{equation}
		\item
		      Ensuite, pour \( x\in X\) et \( n\in \eN\) nous avons :
		      \begin{subequations}
			      \begin{align}
				      H(x,n+1) & =f(x+1)x                      \\
				               & =g\big( f(n) \big)x           \\
				               & =g\big( f(n) \big)x           \\
				               & =\big( \beta\circ f(n) \big)x \\
				               & =\beta\big( f(n)x \big)       \\
				               & =\beta\big( H(x,n) \big).
			      \end{align}
		      \end{subequations}
	\end{enumerate}
	Et voilà.
\end{proof}

%---------------------------------------------------------------------------------------------------------------------------
\subsection{Addition sur les naturels}
\label{SUBooAdditionNaturels}
%---------------------------------------------------------------------------------------------------------------------------

\begin{propositionDef}[\cite{RWWJooJdjxEK,MonCerveau}]      \label{PROPooVFOXooXmwpFh}
	Il existe une unique fonction \( f\colon \eN\times \eN\to \eN\) vérifiant
	\begin{enumerate}
		\item       \label{ITEMooILZSooNYIkYR}
		      \( f(a,0)=a\) pour tout \( a\in \eN\)
		\item       \label{ITEMooZWHQooBAjZyE}
		      \( f\big( a,s(b) \big)=s\big( f(a,b) \big)\) pour tout \( a,b\in \eN\).
	\end{enumerate}
	Pour \( a,b\in \eN\) nous notons \( f(a,b)=a+b\).
\end{propositionDef}

\begin{proof}
	Soit \( a\in \eN\). Le théorème \ref{THOooEJPYooZFVnez} dit qu'il existe une unique application \(f_a \colon \eN\to \eN  \) telle que pour tout \( x\in \eN\),
	\begin{subequations}		\label{SUBEQSooWUGPooSxezFi}
		\begin{numcases}{}
			f_a(0)=a		\label{EQooVOCZooOCHZLn}\\
			f_a\big( s(x) \big)=s\big( f_a(x) \big).	\label{EQooMEYWooEsWofl}
		\end{numcases}
	\end{subequations}
	Montrons qu'une telle application vérifie \( f_0(a)=a\) pour tout \( a\in \eN\). Pour cela nous utilisons la récurrence de la proposition \ref{PROPooXTRCooKwrWkq} en posant
	\begin{equation}
		\begin{aligned}
			P\colon \eN & \to \{ 0,1 \}                   \\
			n           & \mapsto \begin{cases}
				                      1 & \text{si } f_0(n)=n \\
				                      0 & \text{sinon. }
			                      \end{cases}
		\end{aligned}
	\end{equation}
	Nous avons \( f_0(0)=0\) en posant \( a=0\) dans \eqref{EQooVOCZooOCHZLn}. Supposons ensuite que \( P(a)=1\). Nous avons
	\begin{subequations}
		\begin{align}
			f_0\big( s(a) \big) & =s\big( f_0(a) \big)                      \\
			                    & =s(a)                & \text{pcq }P(a)=1.
		\end{align}
	\end{subequations}
	Donc \( P\big( s(a) \big)=s(a)\). Nous en déduisons que \( P(x)=1\) pour tout \( x\in \eN\) et donc que
	\begin{equation}		\label{EQooIKGVooHQMQqX}
		f_0(x)=x
	\end{equation}
	pour tout \( x\).


	Avec ça, nous prouvons ce qu'il faut.
	\begin{subproof}
		\spitem[Existence]
		%-----------------------------------------------------------
		Nous posons \( f(a,b)=f_a(b)\), et nous prouvons les deux conditions.
		\begin{subproof}
			\spitem[Condition \ref{ITEMooILZSooNYIkYR}]
			%-----------------------------------------------------------
			Nous avons \( f(a,0)=f_a(0)=a \) par	\eqref{EQooIKGVooHQMQqX}.
			\spitem[Condition \ref{ITEMooZWHQooBAjZyE}]
			%-----------------------------------------------------------
			Nous avons :
			\begin{equation}
				f\big( a,s(b) \big)=f_a\big( s(b) \big)=s\big( f_a(b) \big)=s\big( f(a,b) \big).
			\end{equation}
		\end{subproof}

		\spitem[Unicité]
		%-----------------------------------------------------------
		Soient \( f\) et \( g\) deux applications vérifiant les conditions. Nous posons
		\begin{equation}
			\begin{aligned}
				f_a\colon \eN & \to \eN        \\
				x             & \mapsto f(a,x)
			\end{aligned}
		\end{equation}
		et \(g_a \colon \eN\to \eN  \) de la même manière. Nous prouvons que \( f_a\) et \( g_a\) vérifient tous deux les conditions \eqref{SUBEQSooWUGPooSxezFi}. Cela donnera \( f_a=g_a\) et donc \( f=g\).

		D'abord
		\begin{equation}
			f_a\big( s(b) \big)=f\big( a,s(b) \big)=s\big( f(a,b) \big)=s\big( f_a(b) \big).
		\end{equation}
		Ensuite nous prouvons par récurrence que \( f_a(0)=a\) pour tout \( a\in \eN\). D'abord \( f(0,0)=0\), ensuite
		\begin{equation}
			f\big( 0,s(a) \big)=s\big( f(0,a) \big)=s(a).
		\end{equation}
		De même \( g_a\) vérifie les conditions.
	\end{subproof}
\end{proof}

\begin{lemma}[\cite{RWWJooJdjxEK}]      \label{LEMooMJMTooOtUuJT}
	Pour tout \( a\in \eN\) nous avons \( s(a)=a+1\).
\end{lemma}

\begin{proof}
	Nous avons :
	\begin{subequations}
		\begin{align}
			s(a) & =s(a+0)        \label{SUBEQooMNBLooTOruhE} \\
			     & =a+s(0)        \label{SUBEQooAGUSooGijYGj} \\
			     & =a+1.          \label{SUBEQooUZQDooWtNBHO}
		\end{align}
	\end{subequations}
	Justifications.
	\begin{enumerate}
		\item
		      Pour \eqref{SUBEQooMNBLooTOruhE} c'est dans la définition \ref{PROPooVFOXooXmwpFh}\ref{ITEMooILZSooNYIkYR} de la somme.
		\item
		      Pour \eqref{SUBEQooAGUSooGijYGj}, c'est dans la définition \ref{PROPooVFOXooXmwpFh}\ref{ITEMooZWHQooBAjZyE} de la somme.
		\item
		      Pour \eqref{SUBEQooUZQDooWtNBHO}. Le symbole «\( 1\)» désigne l'élément \( s(0)\) dans \( \eN\).
	\end{enumerate}
\end{proof}

\begin{proposition}[\cite{RWWJooJdjxEK}]     \label{PROPooTLTSooGNMTmV}
	En ce qui concerne la somme dans \( \eN\).
	\begin{enumerate}
		\item       \label{ITEMooIFFPooXfftfG}
		      La somme est associative et commutative.
		\item       \label{ITEMooSGRVooPAVFYK}
		      L'élément \( 0\) est neutre.
		\item       \label{ITEMooNUTHooJWWzGv}
		      Tous les éléments de \( \eN\) sont réguliers\footnote{Élément régulier pour une opération, définition \ref{DEFooIJIEooZaAdSs}.} par rapport à l'addition.
	\end{enumerate}
\end{proposition}

\begin{proof}
	En plusieurs parties.
	\begin{subproof}
		\spitem[Associative]
		Nous devons prouver que \( (a+b)+c=a+(b+c)\) pour tout \( a,b,c\in \eN\). Pour ce faire, nous fixons \( a,b\in\eN\) et nous prouvons l'égalité demandée par récurrence sur \( c\).

		Pour \( c=0\), nous avons \( (a+b)+c=a+b\) et \( a+(b+c)=a+b\). Donc nous sommes d'accord\footnote{Notez que nous n'avons pas utilisé le fait que \( 0\) était neutre des deux côtés -- chose que nous n'avons pas encore démontré. Nous avons seulement utilisé \( a+0=a\), qui est dans la définition de la somme.}.

		Nous vérifions avec \( s(c)\) :
		\begin{subequations}
			\begin{align}
				(a+b)+s(c) & =s\big( (a+b)+c \big)                              \\
				           & =s\big( a+(b+c) \big)  \label{SUBEQooDDUBooAjuuZq} \\
				           & =a+s(b+c)                                          \\
				           & =a+\big( b+s(c) \big).
			\end{align}
		\end{subequations}
		Justifications.
		\begin{itemize}
			\item Pour \eqref{SUBEQooDDUBooAjuuZq}. C'est l'hypothèse de récurrence. À ce stade, je vous conseille d'être capable de rédiger complètement la récurrence et l'appel au théorème \ref{THOooEJPYooZFVnez}.
		\end{itemize}
		\spitem[Neutre]
		La définition de l'addition contient déjà \( a+0=a\). Nous prouvons par récurrence que \( 0+a=a\) pour tout \( a\in \eN\).

		Pour \( a=0\), l'égalité demandé est correcte : \( 0+0=0\) parce que pour tout \( x\) dans \( \eN\), \( 0+x=x\). Pour \( s(a)\) nous avons
		\begin{equation}
			0+s(a)=s(0+a)=s(a).
		\end{equation}
		La dernière égalité est l'hypothèse de récurrence.

		Nous posons
		\begin{equation}
			A=\{ a\in \eN\tq 0+a=a \}.
		\end{equation}
		Nous avons prouvé que \( 0\in A\) et que \( s(A)\subset A\). Le théorème \ref{THOooEJPYooZFVnez} nous assure alors que \( A=\eN\).
		\spitem[Commutativité]
		Nous fixons \( a\in \eN\) et nous prouvons par récurrence sur \( b\) que \( a+b=b+a\) pour tout \( b\in \eN\). Cela va être décomposé en plusieurs étapes.
		\spitem[\( a+0=0+a\)]
		Pour \( b=0\) c'est correct, car \( b+0=0+b=b\) parce que \( 0\) est neutre.
		\spitem[\( a+1=1+a\)]
		Nous démontrons par récurrence sur \( a\) que \( a+1=1+a\). Avec \( a=0\) c'est déjà fait. Pour les autres,
		\begin{subequations}
			\begin{align}
				s(a)+1 & =(a+1)+1 & \text{lemme \ref{LEMooMJMTooOtUuJT}} \\
				       & =(1+a)+1 & \text{hypothèse récurrence}          \\
				       & =1+(a+1) & \text{associativité}                 \\
				       & =1+s(a).
			\end{align}
		\end{subequations}
		\spitem[\( a+b=b+a\)]
		Nous y voici. Nous fixons \( a\) et nous prouvons par récurrence que \( a+b=b+a\). Pour \( b=0\) c'est déjà fait. Pour les autres,
		\begin{subequations}
			\begin{align}
				a+s(b) & =a+(b+1)                                    \\
				       & =(a+b)+1                                    \\
				       & =(b+a)+1 & \text{hypothèse récurrence}      \\
				       & =b+(a+1)                                    \\
				       & =b+(1+a) & \text{commutativité avec \( 1\)} \\
				       & =(b+1)+a & \text{associativité}             \\
				       & =s(b)+a.
			\end{align}
		\end{subequations}
		Récurrence terminée.
		\spitem[Régularité]
		Nous devons prouver que, pour tout \( a,x,y\in \eN\), si \( a+x=a+y\) alors \( x=y\). Nous allons procéder par récurrence en posant
		\begin{equation}
			A=\{ a\in \eN\tq\forall x,y\in \eN, a+x=a+y\Rightarrow x=y\}.
		\end{equation}
		Puisque \( 0+x=x\) et \( 0+y=y\), nous avons \( 0\in A\). Supposons à présent que \( a\in A\) et montrons que \( s(a)\in A\). Soient \( x,y\in \eN\) tels que \( a+x=a+y\). Nous avons :
		\begin{subequations}
			\begin{align}
				            & \quad s(a)+x=s(a)+y                                \\
				\Rightarrow & \quad s(a+x)=s(a+y)    \label{SUBEQooNJHZooSIKPxN} \\
				\Rightarrow & \quad a+x=a+y          \label{SUBEQooWDJLooJhzIFe} \\
				\Rightarrow & \quad x=y              \label{SUBEQooTLYZooFsMaJD}
			\end{align}
		\end{subequations}
		Justifications.
		\begin{itemize}
			\item Pour \eqref{SUBEQooNJHZooSIKPxN}. En utilisant la définition de l'addition et la commutativité, nous avons \( s(a)+x=s(a+x)\).
			\item Pour \eqref{SUBEQooWDJLooJhzIFe}. Parce que \( s\) est injective; c'est dans la définition \ref{DEFooBJBOooWlblAx} d'un triplet naturel.
			\item Pour \eqref{SUBEQooTLYZooFsMaJD}. Parce que \( a\in A\).
		\end{itemize}
		Nous avons prouvé que \( s(A)\subset A\), et donc que \( A=\eN\).
	\end{subproof}
\end{proof}

\begin{lemma}       \label{LEMooCOMSooEWrumL}
	Nous avons \( 0\neq 1\).
\end{lemma}

\begin{proof}
	Par définition \( 1=s(0)\). Comme \( s\) est à valeurs dans \( \eN\setminus\{ 0 \}\), nous ne pouvons pas avoir \( s(0)=0\).
\end{proof}

\begin{lemma}[\cite{MonCerveau}]       \label{LEMooQBHFooCuCusQ}
	Si \( a+b=0\), alors \( a=b=0\).
\end{lemma}

\begin{proof}
	Soient \( a,b\in \eN\) tels que \( a+b=0\), et supposons que \( b\neq 0\). Par la définition \ref{DEFooBJBOooWlblAx}\ref{ITEMooQAKJooGKdJsM}, nous avons \( b=s(c)\) pour un certain \( c\in \eN\).

	Dans ce cas nous avons \( a+b=a+s(c)=s(a+c)\neq 0\) parce que l'image de \( s\) ne contient pas \( 0\). Hélas, par hypothèse nous avons \( a+b=0\). Nous avons obtenu une contradiction, et nous déduisons que \( b=0\).

	Maintenant que nous savons que \( b=0\), il reste \( 0=a+b=a+0=a\).
\end{proof}

%---------------------------------------------------------------------------------------------------------------------------
\subsection{Ordre sur les naturels}
\label{SUBooOrdreNaturels}
%---------------------------------------------------------------------------------------------------------------------------

\begin{definition}[\cite{RWWJooJdjxEK}]     \label{DEFooAXZSooTEMjlV}
	Pour \( a,b\in \eN\), nous notons \( a\leq b\) si il existe \( x\in \eN\) tel que \( a+x=b\).

	Nous notons également \( a<b \) si \( a\leq b\) et \( a\neq b\).
\end{definition}

\begin{lemma}       \label{LEMooWMYPooLTMyWb}
	Nous avons \( a\leq s(a)\) pour tout \( a\in \eN\).
\end{lemma}

\begin{proof}
	Cela est une conséquence du lemme \ref{LEMooMJMTooOtUuJT} : \( s(a)=a+1\).
\end{proof}

\begin{proposition}     \label{PROPooVXBBooZcghrA}
	La relation \( \leq\) est une relation d'ordre compatible avec l'addition.
\end{proposition}

\begin{proof}
	Plusieurs choses à vérifier.
	\begin{subproof}
		\spitem[Réflexive]
		Nous avons \( a\leq a\) parce que \( a+0=a\).
		\spitem[Antisymétrique]
		Soient \( a,b\in \eN\) tels que \( a\leq b\) et \( b\leq a\). Il existe \( x,y\in \eN\) tels que
		\begin{subequations}
			\begin{align}
				b & =a+x  \\
				a & =b+y.
			\end{align}
		\end{subequations}
		En substituant la seconde équation dans la première, \( b=(b+y)+x\) que nous récrivons, en utilisant l'associativité\footnote{Proposition \ref{PROPooTLTSooGNMTmV}\ref{ITEMooIFFPooXfftfG}.},
		\begin{equation}
			0+b=b+(x+y).
		\end{equation}
		En utilisant la régularité, \( 0=x+y\) et donc \( x=y=0\) par le lemme \ref{LEMooQBHFooCuCusQ}. Cela donne alors \( a=b\).
		\spitem[Transitive]
		Si \( a\leq b\) et \( b\leq c\), nous avons \( n,p\in \eN\) tels que \( b=a+n\) et \( c=b+p\). Donc
		\begin{equation}
			c=(a+n)+p=a+(n+p),
		\end{equation}
		ce qui signifie que \( a\leq c\). Notez l'utilisation de l'associativité de la somme, démontrée en la proposition \ref{PROPooTLTSooGNMTmV}\ref{ITEMooIFFPooXfftfG}.
		\spitem[Compatibilité]
		Soient \( a,b,n\in \eN\) tels que \( a\leq b\). Nous devons montrer que \( a+n\leq b+n\). Puisque \( a\leq b\), il existe \( x\in \eN\) tel que \( b=a+x\). Par conséquent,
		\begin{equation}
			b+n=a+x+n=(a+n)+x,
		\end{equation}
		qui signifie bien que \( a+n\leq b+n\).
	\end{subproof}
\end{proof}

\begin{lemma}       \label{LEMooPVRQooXPMKTt}
	À propos d'ordre et de stricte inégalité.
	\begin{enumerate}
		\item       \label{ITEMooGWWFooYGPCZw}
		      Si \( x\leq a\) et \( b\neq 0\), alors \( x<a+b\).
		\item       \label{ITEMooRWGWooAfkrri}
		      Si \( x\leq a\), alors \( x<s(a)\).
		\item       \label{ITEMooWCOIooMWrCag}
		      Si \( x<a\), alors \( s(x)\leq a\).
	\end{enumerate}
\end{lemma}

\begin{proof}
	En plusieurs parties.
	\begin{subproof}
		\spitem[Pour \ref{ITEMooGWWFooYGPCZw}]
		Si \( x\leq a\), il existe \( d\in \eN\) tel que \( x+d=a\). Nous avons alors aussi
		\begin{equation}
			x+d+b=a+b,
		\end{equation}
		ce qui signifie que \( x\leq a+b\). Mais si \( x\) était égal à \( a+b\), nous aurions \( d+b=0\), ce qui impliquerait\footnote{Par le lemme \ref{LEMooQBHFooCuCusQ}.} \( d=b=0\), alors que l'hypothèse stipule que \( b\neq 0\). Donc \( x\neq a+b\).
		\spitem[Pour \ref{ITEMooRWGWooAfkrri}]
		Il s'agit seulement d'utiliser le point \ref{ITEMooGWWFooYGPCZw} avec \( b=1\) et le fait que \( s(a)=a+1\) par le lemme \ref{LEMooMJMTooOtUuJT}.
		\spitem[Pour \ref{ITEMooWCOIooMWrCag}]
		Par hypothèse, il existe \( b\neq 0\) tel que \( x+b=a\). Puisque \( b\neq 0\), il existe \( c\in \eN\) tel que \( b=s(c)\) et donc, tel que
		\begin{equation}
			x+s(c)=a.
		\end{equation}
		En utilisant le fait que \( s(c)=c+1\) ainsi que l'associativité et la commutativité de l'addition (proposition \ref{PROPooTLTSooGNMTmV}\ref{ITEMooIFFPooXfftfG}) nous avons
		\begin{equation}
			a=x+s(c)=s(x)+c,
		\end{equation}
		ce qui prouve que \( s(x)\leq a\).
	\end{subproof}
\end{proof}

\begin{lemma}       \label{LEMooCSIXooHeuWEd}
	L'élément \( 0\) est l'unique plus petit élément de \( \eN\).
\end{lemma}

\begin{proof}
	Puisque \( 0\) est neutre pour l'addition\footnote{Proposition \ref{PROPooTLTSooGNMTmV}\ref{ITEMooSGRVooPAVFYK}.}, nous avons \( a+0=a\) pour tout \( a\in \eN\) et donc \( 0\leq a\) pour tout \( a\). Cela veut dire que \( 0\) est plus petit que tout élément de \( \eN\).

	En ce qui concerne l'unicité, soit \( z\in \eN\) tel que \( z\leq a\) pour tout \( a\in \eN\). Si \( z\neq 0\), il existe \( x\in \eN\) tel que \( z=s(x)\). Nous avons donc \( z\geq x\) en même temps que \(x\leq z\). Cela implique \( z=x\) (parce qu'une relation d'ordre est symétrique) et donc \( z=z+1\). En utilisant la régularité de \( z\) pour l'addition nous en déduisons que \( 0=1\), ce qui est impossible par le lemme \ref{LEMooCOMSooEWrumL}.
\end{proof}

\begin{lemma}       \label{LEMooJRZKooOMhOkH}
	Si \( a\leq b\) et \( b\leq a\), alors \( a=b\).
\end{lemma}

\begin{proof}
	L'inégalité \( a\leq b\) dit qu'il existe \( x\in \eN\) tel que \( a+x=b\). En mettant cela dans l'inégalité \( b\leq a\) nous trouvons \( a+x\leq a\) qui donne, via la proposition \ref{PROPooVXBBooZcghrA} : \( x\leq 0\). Nous en déduisons que \( x=0\) parce que zéro est l'unique minimum de \( \eN\) par le lemme \ref{LEMooCSIXooHeuWEd}.
\end{proof}

\begin{proposition}     \label{PROPooGCCRooFBYrlo}
	Le couple \( (\eN,\leq)\) est totalement ordonné\footnote{Définition \ref{DEFooVGYQooUhUZGr}.}.
\end{proposition}

\begin{proof}
	Soit \( a\in \eN\). Nous devons prouver que pour tout \( x\in \eN\) nous avons \( x\leq a\) ou \( a\leq x\) (non exclusifs). Nous posons
	\begin{subequations}
		\begin{align}
			A & =\{ x\in \eN\tq x\leq a \}  \\
			B & =\{ x\in \eN\tq a\leq x \},
		\end{align}
	\end{subequations}
	et nous prouvons que \( A\cup B=\eN\) en montrant que \( 0\in A\cup B\) et que \( s(A\cup B)\subset A\cup B\).

	Nous avons \( 0\in A\subset A\cup B\) par le lemme \ref{LEMooCSIXooHeuWEd}.

	Pour étudier \( s(A\cup B)\), nous considérons \( x\in A\cup B\) et nous subdivisons en deux cas selon que \( x\in A\) ou \( x\in B\).

	\begin{subproof}
		\spitem[Si \( x\in B\)]
		Si \( x\in B\), alors \( a\leq x\leq s(x)\) parce que \( x\leq s(x)\) par le lemme \ref{LEMooWMYPooLTMyWb}. Donc \( s(x)\in B\subset A\cup B\).
		\spitem[Si \( x\in A\)]
		Si \( x\in A\), il y a deux possibilités : \( x=a\) et \( x\neq a\). Si \( x=a\), alors \( a\leq s(x)\) et donc \( s(x)\in A\subset A\cup B\).

		Si \( x\neq a\), alors le lemme \ref{LEMooPVRQooXPMKTt}\ref{ITEMooWCOIooMWrCag} nous indique que \( s(x)\leq a\) et donc \( s(x)\in A\subset A\cup B\).
	\end{subproof}
	Nous avons donc prouvé que \( s(A\cup B)\subset A\cup B\), et donc que \( A\cup B=\eN\).
\end{proof}

\begin{proposition}[\cite{RWWJooJdjxEK,MonCerveau}]     \label{PROPooMZOWooHmsXzI}
	L'ensemble ordonné \( (\eN,\leq)\) vérifie les propriétés suivantes.
	\begin{enumerate}
		\item       \label{ITEMooJLAHooDKukfH}
		      L'élément \( 0\) est l'unique minimum de \( \eN\).
		\item       \label{ITEMooYAJIooEFmOpB}
		      Toute partie non vide a un unique plus petit élément.
		\item       \label{ITEMooSRGOooNYJJHY}
		      L'ensemble \( \eN\) n'a pas de plus grand élément.
		\item       \label{ITEMooKIHZooDRTCdx}
		      Toute partie non vide majorée a un unique plus grand élément.
	\end{enumerate}
\end{proposition}

\begin{proof}
	Point par point.
	\begin{subproof}
		\spitem[Pour \ref{ITEMooJLAHooDKukfH}]
		C'est le lemme \ref{LEMooCSIXooHeuWEd}.

		\spitem[Pour \ref{ITEMooYAJIooEFmOpB}]
		Soit une partie \( A\) non vide dans \( \eN\). Si \( 0\in A\), nous avons fini.
		\begin{subproof}
			\spitem[L'ensemble \( B\)]
			Nous supposons donc que \( A\) ne contient pas zéro et nous définissons
			\begin{equation}
				B=\{ n\in \eN\setminus A\tq n\leq a, \forall a\in A \}.
			\end{equation}
			\spitem[Un élément particulier dans \( B\)]
			L'ensemble \( B\) vérifie :
			\begin{itemize}
				\item \( 0\in B\)
				\item \( B\neq \eN\) parce que \( A\) est non vide.
			\end{itemize}
			La contraposée de la condition \ref{ITEMooXPYEooFajywh} de la définition \ref{DEFooBJBOooWlblAx} d'un triplet naturel implique que \( s(B)\nsubset B\). Autrement dit, il existe \( b\in B\) tel que \( s(b)\notin B\).

			\spitem[Deux fonctions sur \( A\)]
			Puisque \( b\in B\), nous avons une application \( c\colon A\to \eN\) telle que \( b+c(a)=a\). Nous avons \( c(a)\neq 0\) parce que \( c(a)=0\) signifierait \( b=a\), ce qui est impossible parce que \( a\in A\) et \( b\in B\).

			Comme pour tout \( a\in A\), l'élément \( c(a)\) est non nul, il existe une fonction \( d\colon A\to \eN\) telle que \( c(a)=d(a)+1\).
			\spitem[\( s(b)\in A\)]
			Supposons que \( s(b)\notin A\). Alors il existe \( a\in A\) tel que \( s(b)\leq a\) est faux. Puisque l'ordre est total (proposition \ref{PROPooGCCRooFBYrlo}), nous avons
			\begin{equation}        \label{EQooYQSFooPSPJMt}
				a\leq s(b).
			\end{equation}
			Comme \( b\in B\) nous avons aussi
			\begin{equation}        \label{EQooIPAWooDBSJEa}
				b\leq a.
			\end{equation}
			Et enfin nous avons
			\begin{equation}        \label{EQooWFHRooGDSBFD}
				b\neq a
			\end{equation}
			parce que \( a\in A\) et \( b\in B\).

			Les conditions \eqref{EQooIPAWooDBSJEa} et \eqref{EQooWFHRooGDSBFD} se résument en \( b<a\). Le lemme \ref{LEMooPVRQooXPMKTt}\ref{ITEMooWCOIooMWrCag} nous indique alors que \( s(b)\leq a\). Cela mis à côté de \eqref{EQooYQSFooPSPJMt} conclut que \( a=s(b)\), et donc que \( s(b)\) est dans \( A\). Contradiction. Nous en concluons que \( s(b)\in A\).
			\spitem[\( s(b)\) est un minimum de \( A\)]
			En utilisant la commutativité et l'associativité de la somme nous avons, pour tout \( a\in A\) :
			\begin{equation}
				a=b+c(a)=b+\big( d(a)+1 \big)=(b+1)+d(a)=s(b)+d(a).
			\end{equation}
			Donc \( s(b)\leq a\) pour tout \( a\in A\). Mais comme \( s(b)\in A\), l'élément \( s(b)\) est bien un minimum de \( A\).
			\spitem[Unicité]
			Si \( a\) et \( a'\) sont des minimums de \( A\), alors \( a\leq a'\) et \( a'\leq a\). Nous en déduisons que \( a=a'\).
		\end{subproof}
		\spitem[Pour \ref{ITEMooSRGOooNYJJHY}]
		Si \( M\in \eN\) majore tous les éléments de \( \eN\), alors en particulier \( M\geq s(M)\). Mais le lemme \ref{LEMooWMYPooLTMyWb} nous indique que \( M\leq s(M)\). Nous avons donc \( s(M)=M\), c'est-à-dire \( M=M+1\). En utilisant la régularité de \( M\)\footnote{Dit plus simplement : en simplifiant par \( M\).}, nous trouvons \( 0=1\), ce qui est impossible par le lemme \ref{LEMooCOMSooEWrumL}.
		\spitem[Pour \ref{ITEMooKIHZooDRTCdx}]
		Soit \( A\) une partie non vide et majorée de \( \eN\).
		\begin{subproof}
			\spitem[L'ensemble \( B\)]
			Nous posons
			\begin{equation}
				B=\{ n\in \eN\setminus A\tq a\leq n, \forall a \in A \}.
			\end{equation}
			\spitem[\( B\) est non vide]
			Soit un majorant \( M\) de \( A\) : pour tout \( a\in A\) nous avons \( a\leq M\). Nous avons \( s(M)\notin A\), parce que si \( s(M)\) était dans \( A\), ce serait un élément de \( A\) strictement plus grand que tout \( a\in A\). Donc \( B\) est non vide parce qu'il contient \( s(M)\).
		\end{subproof}
		\spitem[Minimum]
		Puisque \( B\) est non vide, il possède un plus petit élément que nous notons \( b\). Nous savons que \( b\neq 0\) parce que sinon \( A\) serait vide. Il existe donc \( c\in \eN\) tel que \( b=s(c)\).

		\spitem[\( a\leq c\) pour tout \( a\in A\)]
		Comme \( s(c)\in B\) nous avons \( a< s(c)\) pour tout \( a\in A\). Donc, par le lemme \ref{LEMooPVRQooXPMKTt}\ref{ITEMooWCOIooMWrCag} nous avons \( s(a)\leq s(c)\), c'est-à-dire \( a+1\leq c+1\). Par régularité nous avons \( a\leq c\).

		Nous avons prouvé que \( a\leq c\) pour tout \( a\in A\).
		\spitem[\( c\in A\)]
		Si \( c\) n'est pas dans \( A\), alors il est dans \( B\) et il contredit la minimalité de \( b\). Donc \( c\) est dans \( A\).
		\spitem[Conclusion]
		L'élément \( c\) est dans \( A\) tout en étant plus petit que tout élément de \( A\).
		\spitem[Unicité]
		Si \( x\) est un élément minium de \( A\), alors nous avons \( x\leq c\) parce que \( x\) est minimum et \( c\leq x\) parce que \( c\) est minimum, et donc \( x=c\).
	\end{subproof}
\end{proof}


\begin{probleme}		\label{PROBooUCQFooTugfDl}
	Je ne suis pas certain qu'il y ait moyen de prouver le lemme \ref{LEMooPCRFooXRGrUr} sans d'abord définir les naturels. Une raison est que la véracité de ce lemme est à peu près équivalente au théorème \ref{THOooEJPYooZFVnez} qui donne l'existence de \( \eN\).

	J'ai essayé un certain nombre de choses avec le lemme de Zorn sans succès.

	Si vous avez une idée, n'hésitez pas à me la partager, ou à l'écrire ici :
	\begin{center}
		\url{https://math.stackexchange.com/questions/5019157/every-non-empty-dedekind-finite-totally-ordered-set-has-a-maximum-using-zorns}
	\end{center}
	%TODOooODPBooJZFYau. Voir si il y a une réponse. C'est déjà dans ma liste.

	Dans la réponse de Asaf Karagila, il y a seulement le fait que \(f \colon D\to D  \) que je ne comprends pas. C'est d'ailleurs ma dernière question dans les commentaires.
\end{probleme}



\begin{lemmaDef}        \label{LEMooOEJOooOgaxzi}
	Si \( A\) est une partie de \( \eN\), il existe un unique élément \( m\in \eN\) tel que
	\begin{subequations}
		\begin{numcases}{}
			m\in A\\
			m\leq a\,\forall a\in A.
		\end{numcases}
	\end{subequations}
	Cet élément est noté \( \min(A)\) et nommé \defe{minimum de \( A\)}{minimum}.

	Si \( A\) est majoré, il existe un unique élément \( M\in \eN\) tel que
	\begin{subequations}
		\begin{numcases}{}
			M\in A\\
			M\geq a\,\forall a\in A.
		\end{numcases}
	\end{subequations}
	Cet élément est noté \( \max(A)\) et nommé \defe{maximum de \( A\)}{maximum}.
	%TODOooGKSOooLXwZsi. Prouver ça.
	% c'est un corollaire de LEMooPCRFooXRGrUr. Tellement que j'en mets deux avec le même.
\end{lemmaDef}
Nous verrons dans le lemme \ref{LEMooGQUWooYJQfJB} qu'une partie de \( \eN\) admet un maximum si et seulement si elle est finie.

\begin{lemma}[\cite{MonCerveau}]       \label{LEMooYMRJooYIAhBb}
	Quelques affirmations sur l'ordre dans \( \eN\).
	\begin{enumerate}
		\item       \label{ITEMooTLOIooTWNtod}
		      Il n'existe pas de \( n\in \eN\) tel que \( n<0\).
		\item       \label{ITEMooPJKQooGfLCUM}
		      Si \( a,b\in \eN\) vérifient \( a>b\), alors il n'existe pas de \( x\) dans \( \eN\) tel que \( a+x=b\).
	\end{enumerate}
\end{lemma}

\begin{proof}
	En deux parties.
	\begin{subproof}
		\spitem[Pour \ref{ITEMooTLOIooTWNtod}]
		Nous savons par la proposition \ref{PROPooMZOWooHmsXzI}\ref{ITEMooJLAHooDKukfH} que \( 0\) est l'unique minimum de \( \eN\). Nous avons donc forcément \( 0\leq n\). Si \( n\) vérifie de plus \( n\leq 0\) alors nous avons \( n=0\) par symétrie de la relation d'ordre \( \leq\). Il n'est donc pas possible d'avoir \( n\neq 0\).
		\spitem[Pour \ref{ITEMooPJKQooGfLCUM}]
		Si \( b\leq a\) il existe \( c\in \eN\) tel que \( b+c=a\). Et comme \( a\neq b\), \( c\) n'est pas nul et il existe \( y\in \eN\) tel que \( c=s(y)\). Bref, nous avons
		\begin{equation}
			b+s(y)=a.
		\end{equation}
		Si de plus il existe \( x\in \eN\) tel que \( a+x=b\) nous aurions
		\begin{equation}
			a+x+s(y)=a.
		\end{equation}
		Comme \( a\) est régulier pour l'addition\footnote{Proposition \ref{PROPooTLTSooGNMTmV}\ref{ITEMooNUTHooJWWzGv}.}, nous avons
		\begin{equation}
			x+s(y)=0,
		\end{equation}
		ce qui signifie, par le lemme \ref{LEMooQBHFooCuCusQ} que \( x=s(y)=0\). Puisque \( s\) prend ses valeurs dans \( \eN\setminus\{ 0 \}\), cela est impossible.
	\end{subproof}
\end{proof}

\begin{definition}      \label{DEFooKBUFooLvMHrf}
	Soient \( a,b\in \eN\) tels que \( a\leq b\). Nous notons par \( \{ a,\ldots, b \}\) l'ensemble
	\begin{equation}
		\{ x\in \eN\tq a\leq x\leq b \}.
	\end{equation}
\end{definition}

\begin{proposition}     \label{PROPooFYMJooWihvhk}
	Toute application \( \eN\to \eN\) strictement croissante est injective.
\end{proposition}

\begin{proof}
	Soit une application strictement croissante \( f\colon \eN\to \eN\). Soient \( a,b\in \eN\) tels que \( f(a)=f(b)\). Puisque l'ordre est total\footnote{Proposition \ref{PROPooGCCRooFBYrlo}.}, nous supposons que \( a\leq b\). Si \( a=b\) nous avons terminé. Nous supposons donc que \( a\neq b\), c'est-à-dire que \( a<b\). Par stricte croissance nous avons alors \( f(a)<f(b)\) qui signifie \( f(a)\leq f(b)\) et \( f(a)\neq f(b)\). Contradiction. Il n'existe donc pas de \( a\neq b\) tels que \( f(a)=f(b)\). L'application \( f\) est donc injective.
\end{proof}

\begin{lemma}[\cite{MonCerveau}]        \label{LEMooFKLPooPrmeUU}
	Si \( S\) n'est pas majoré dans \( \eN\), alors il existe une bijection \( \eN\to S\).
\end{lemma}

\begin{proof}
	Nous considérons l'application suivante :
	\begin{equation}
		\begin{aligned}
			g\colon S & \to S                            \\
			n         & \mapsto \min\{ x\in S\tq x>n \}.
		\end{aligned}
	\end{equation}
	Cette application est bien définie parce que tout partie non vide de \( \eN\) a un plus petit élément\footnote{Proposition \ref{PROPooMZOWooHmsXzI}\ref{ITEMooYAJIooEFmOpB}.}. Maintenant nous définissons \( f\colon \eN\to S\) par
	\begin{subequations}
		\begin{numcases}{}
			f(0)=\min(S)\\
			f(n+1)=g\big( f(n) \big).
		\end{numcases}
	\end{subequations}
	C'est le théorème \ref{THOooEJPYooZFVnez} qui nous permet de le faire. Nous montrons que \( f\) est bijective.
	\begin{subproof}
		\spitem[Injective]
		Nous avons
		\begin{equation}
			f(n+1)\in\{ x\in S\tq x>f(n) \}.
		\end{equation}
		Donc \( f\) est strictement croissante. Elle est donc injective.
		\spitem[Surjective]
		Soit \( a\in S\). Nous allons voir que \( a\) est dans l'image de \( f\). Pour cela nous posons
		\begin{equation}
			A=\{ x\in \eN\tq f(x)<a \}.
		\end{equation}
		Cet ensemble est majoré par \( a\). En effet si \( x\in A\) nous avons \( x\leq f(x)<a\). La partie \( A\) de \( \eN\) possède un maximum. Nous notons \( M=\max(A)\). Ce \( M\) a deux propriétés intéressantes.
		\begin{subproof}
			\spitem[D'abord]
			Puisque \( M\in A\), nous avons \( f(M)<a\). Une autre façon de dire cela est de dire que
			\begin{equation}
				a\in \{ x\in S\tq x>f(M) \}.
			\end{equation}
			Or \( f(M+1)=\min\{ x\in S\tq x>f(M) \}\). Donc \( f(M+1)\leq a\).
			\spitem[Ensuite]
			Puisque \( M\) est le maximum de \( A\), \( M+1\) majore \( A\), c'est-à-dire que \( f(M+1)\geq a\).
			\spitem[Les deux ensemble]
			Nous avons prouvé que \( f(M+1)\leq a\) et \( f(M+1)\geq a\). Nous en déduisons, par le lemme \ref{LEMooJRZKooOMhOkH}, que \( f(M+1)=a\).
		\end{subproof}
	\end{subproof}
\end{proof}

\begin{normaltext}      \label{NORMooQXASooMXqhjI}
	Durant la preuve du lemme \ref{LEMooFKLPooPrmeUU}, nous n'avons pas été loin de prouver que
	\begin{equation}
		\big( \min(S),S,g \big)
	\end{equation}
	est un triplet naturel.

	Toute partie non bornée de \( \eN\) donne lieu à un triplet naturel.
\end{normaltext}


%---------------------------------------------------------------------------------------------------------------------------
\subsection{Multiplication dans les naturels}
\label{SUBooMultiplicationNaturels}
%---------------------------------------------------------------------------------------------------------------------------

\begin{definition}[\cite{MonCerveau}]      \label{DEFooAZAYooVjNzmy}
	Soit un ensemble muni d'une loi de composition interne \( (A,+)\). Soit \( n\in \eN\) et \( a\in A\). Nous définissons \( n\times A\) par
	\begin{subequations}
		\begin{numcases}{}
			0\times a=0\\
			(n+1)\times a= n\times a+a.
		\end{numcases}
	\end{subequations}
\end{definition}

\begin{definition}      \label{DEFooLCWLooYrToFv}
	Un ensemble totalement ordonné muni d'une loi de composition interne \( (A,+, \leq)\) est \defe{archimédien}{ensemble!archimédien}\index{archimédien} si pour tout \( x,y\in A\) avec \( x>0\), il existe \( n\in \eN\) tel que \( n\times x\geq y\) (voir la définition \ref{DEFooAZAYooVjNzmy}).
\end{definition}


\begin{lemma}       \label{LEMooGQUWooYJQfJB}
	Une partie de \( \eN\) admet un maximum si et seulement si elle est finie.
	%TODOooRRGMooVNRcLR. Prouver ça.
\end{lemma}

\begin{propositionDef}      \label{PROPooBBQPooRgPOjf}
	Il existe une unique fonction \( f\colon \eN\times \eN\to \eN\) telle que
	\begin{enumerate}
		\item       \label{ITEMooNTUUooDAUVsV}
		      \( f(a,0)=0\) pour tout \( a\in \eN\)
		\item       \label{ITEMooPPZZooQQabwn}
		      \( f\big( a,s(b) \big)=f(a,b)+a\) pour tout \( a,b\in \eN\).
	\end{enumerate}
	Cette fonction est la \defe{multiplication}{multiplication} et nous notons \( f(a,b)=a\times b\), voire \( ab\) quand il n'y a pas d'ambigüité. Le nombre \( a\times b\) est nommé le \defe{produit}{produit} de \( a\) par \( b\).
\end{propositionDef}

\begin{proof}
	En deux parties.
	\begin{subproof}
		\spitem[Fonctions définies par récurrence]
		Soit \( a\in \eN\). Par le théorème \ref{THOooEJPYooZFVnez}, il existe une unique application \( f_a\colon \eN\to \eN\) telle que
		\begin{subequations}        \label{EQSooWVCTooNTVjKU}
			\begin{numcases}{}
				f_a(0)=0\\
				f_a\big( s(b) \big)=f_a(b)+a
			\end{numcases}
		\end{subequations}
		\spitem[Existence]
		Nous considérons, pour chaque \( a\in \eN\) la fonction \( f_a\) définie par les conditions \eqref{EQSooWVCTooNTVjKU}. En posant \( f(a,b)=f_a(b)\), nous avons une application qui vérifie toutes les conditions.
		\spitem[Unicité]
		Soient des applications \( f\) et \( g\) vérifiant les propriétés demandées. Soit \( a\in \eN\). Nous pouvons définir \( f_a\colon \eN\to \eN\) et \( g_a\colon \eN\to \eN\) par \( f_a(n)=f(a,b)\) et \( g_a(b)=g(a,b)\).

		Les applications \( f_a\) et \( g_a\) vérifient toutes deux les conditions \eqref{EQSooWVCTooNTVjKU}, et sont donc égales : \( f_a=g_a\) pour tout \( a\). Donc \( f=g\).
	\end{subproof}
\end{proof}

\begin{normaltext}	\label{NORMooPrioriteOperations}
	Nous supposons que \randomGender{le lecteur}{la lectrice} connait déjà la priorité des opérations. \randomGender{Il}{Elle} saura donc interpréter des expressions comme \( a\times b+c\) comme voulant dire \( (a\times b)+c\) sans que nous ayons à ajouter des parenthèses.
\end{normaltext}

\begin{proposition}[\cite{RWWJooJdjxEK,MonCerveau}]     \label{PROPooGHDOooFYRmon}
	La multiplication dans \( \eN \) possède les propriétés suivantes.
	\begin{enumerate}
		\item       \label{ITEMooHFWRooDCEpjj}
		      \( n\times 1=n\) pour tout \( n\in \eN\).
		\item       \label{ITEMooRSYMooSUrRsl}
		      \( 1\times n=n\) pour tout \( n\in \eN\).
		\item       \label{ITEMooWJPOooRUYjwQ}
		      La multiplication est commutative.
		\item       \label{ITEMooNBYKooXnGRrf}
		      \( 0\times n=0\) pour tout \( n\in \eN\)
		\item      \label{ITEMooLJQBooVpUxUv}
		      L'élément \( 1\) est neutre pour la multiplication.
		\item       \label{ITEMooDYLIooETIBEL}
		      La multiplication est distributive par rapport à l'addition.
		\item       \label{ITEMooQBFSooWGDQYX}
		      La multiplication est associative.
	\end{enumerate}
\end{proposition}

\begin{proof}
	Point par point.
	\begin{subproof}
		\spitem[Pour \ref{ITEMooHFWRooDCEpjj}]
		Nous avons \( n\times 1=n\times s(0)=(n\times 0)+n=0+n=n\). Donc \( n\times 1=n\).
		\spitem[Pour \ref{ITEMooRSYMooSUrRsl}]
		Nous le faisons par récurrence. Par définition c'est vrai pour \( n=0\). En ce qui concerne la récurrence, nous supposons que \( 1\times n=n\), et nous prouvons que \( 1\times s(n)=s(n)\) :
		\begin{equation}
			1\times s(n)=(1\times n)+1=n+1=s(n).
		\end{equation}
		\spitem[Pour \ref{ITEMooWJPOooRUYjwQ}]
		Soit \( a\in \eN\). Nous prouvons par récurrence sur \( b\in \eN\) que \( a\times b=b\times a\). Pour \( b=0\) c'est bon. Pour la récurrence, nous supposons que \( a\times b=b\times a\) et nous prouvons que \( a\times s(b)=s(b)\times a\) :
		\begin{subequations}
			\begin{align}
				a\times s(b) & =a\times b+a                                                   \\
				             & =b\times a+a             & \text{récurrence}                   \\
				             & =(b\times a)+(1\times a) & \text{par \ref{ITEMooRSYMooSUrRsl}} \\
				             & =(b+1)\times a           & \text{distributivité}               \\
				             & =s(b)\times a.
			\end{align}
		\end{subequations}
		\spitem[Pour \ref{ITEMooNBYKooXnGRrf}]
		C'est vrai pour \( n=0\) par la définition \ref{PROPooBBQPooRgPOjf}\ref{ITEMooNTUUooDAUVsV}. En ce qui concerne \( s(n)\), nous avons
		\begin{equation}
			0\times s(n)=(0\times n)+0=0.
		\end{equation}
		\spitem[Pour \ref{ITEMooLJQBooVpUxUv}]
		C'est la combinaison de \ref{ITEMooHFWRooDCEpjj} et \ref{ITEMooRSYMooSUrRsl}.
		\spitem[Pour \ref{ITEMooDYLIooETIBEL}]
		Soient \( a,b\in \eN\). Nous prouvons par récurrence sur \( c\in \eN\) que\footnote{Nous n'écrivons pas toutes les parenthèses parce que les règles de priorité des opérations sont supposées connues. J'invite cependant \randomGender{le lecteur}{la lectrice} à remarquer qu'une formalisation de ces règles n'est probablement pas facile. Pour que tout soit rigoureux, il faudrait un algorithme qui parcourt une suite de caractères et l'interprète en ajoutant correctement les parenthèses.}
		\begin{equation}
			(a+b)\times c=a\times c+b\times c.
		\end{equation}
		Pour \( c=0\), nous avons \( (a+b)\times 0=0\) ainsi que \( a\times 0=b\times 0=0\) en vertu des points précédents sur la multiplication par zéro. Pour la récurrence nous utilisons associativité et commutativité de la somme :
		\begin{subequations}
			\begin{align}
				(a+b)\times s(c) & =(a+b)\times c+(a+b)                                \\
				                 & =a\times c+b\times c+a+b                            \\
				                 & =(a\times c +a)+(b\times c+b)                       \\
				                 & =\big( a\times s(c) \big)+\big( b\times s(c) \big).
			\end{align}
		\end{subequations}
		On devrait aussi prouver que \( c(a+b) = ca + cb \): c'est peu ou prou la même chose.
		\spitem[Pour \ref{ITEMooQBFSooWGDQYX}]
		Soient \( a,b\in \eN\). Nous démontrons par récurrence sur \( c\in \eN\) que \( (a\times b)\times c=a\times (b\times c)\). Pour \( c=0\) l'égalité est triviale. Nous supposons que l'égalité est correcte pour \( c\), et nous la prouvons pour \( s(c)\) :
		\begin{subequations}
			\begin{align}
				(a\times b)\times s(c) & =\big( (a\times b)\times c \big)+a\times b                                                                      \\
				                       & =\big( a\times (b\times c) \big)+a\times b & \text{récurrence}                                                  \\
				                       & =\big( (b\times c)\times a \big)+b\times a & \text{commutativité}                                               \\
				                       & =(b\times c+b)\times a                     & \text{distributivité}                                              \\
				                       & =\big( b\times s(c) \big)\times a          & \text{définition \ref{PROPooBBQPooRgPOjf}\ref{ITEMooPPZZooQQabwn}} \\
				                       & =a\times \big( b\times s(c) \big)          & \text{commutativité}.
			\end{align}
		\end{subequations}
	\end{subproof}
\end{proof}

\begin{lemma}[\cite{MonCerveau}]	\label{LEMooCompatMultiplicationOrdre}
	La multiplication est compatible avec l'ordre :
	\begin{equation}
		a\leq b\Rightarrow a\times n\leq b\times n
	\end{equation}
	pour tout \( n\in \eN\).
\end{lemma}

\begin{proof}
	Par la définition \ref{DEFooAXZSooTEMjlV} de l'ordre, si \( a\leq b\), il existe \( c\in \eN\) tel que \( b=a+c\). En utilisant la distributivité\footnote{Proposition \ref{PROPooGHDOooFYRmon}\ref{ITEMooDYLIooETIBEL}.}, nous avons
	\begin{equation}
		b\times n=(a+c)\times n=a\times n+c\times n.
	\end{equation}
	Nous en déduisons que \( a\times n\leq b\times n\) parce que \( c\times n \in \eN\).
\end{proof}

\begin{lemma}       \label{LEMooEHYEooLDudfn}
	Si \( a\times b=0\), alors \( a=0\) ou \( b=0\) (ou les deux).
\end{lemma}

\begin{proof}
	Supposons que \( a\neq 0\). Alors il existe \( c\in \eN\) tel que \( a=s(c)\). Nous avons
	\begin{equation}
		0=a\times b=s(c)\times b=c\times b+b.
	\end{equation}
	Le lemme \ref{LEMooQBHFooCuCusQ} nous dit alors que \( c\times b=b=0\).
\end{proof}

\begin{lemma}[\cite{MonCerveau}]        \label{LEMooGUXGooBcKJdS}
	Soit \(a, b, n\) trois entiers naturels. Si \( a< b\) et si \( n\neq 0\), alors
	\begin{equation}
		a\times n<b\times n.
	\end{equation}
\end{lemma}

\begin{proof}
	L'hypothèse \( a\leq b\) implique qu'il existe \( c\in \eN\) tel que \( a+c=b\). De plus \( c\neq 0\) parce que \( a\neq b\). En utilisant la distributivité\footnote{Proposition \ref{PROPooGHDOooFYRmon}\ref{ITEMooDYLIooETIBEL}.}, nous avons
	\begin{equation}
		b\times n=(a+c)\times n=(a\times n)+(c\times n).
	\end{equation}
	Cela prouve que \( a\times n\leq b\times n\). Et comme \( c\) et \( n\) ne sont pas nuls, nous avons même\footnote{Lemme \ref{LEMooEHYEooLDudfn}.} \( c\times n\neq 0\) et donc \( a\times n<b\times n\).
\end{proof}

\begin{remark}	\label{REMooultiplicationNaturelsOrdreStrictZ}
	Une version du lemme \ref{LEMooGUXGooBcKJdS} dans \( \eZ\) sera le lemme \ref{LEMooSVDDooWsyxNP}.
\end{remark}

\begin{lemma}[\cite{MonCerveau}]        \label{LEMooSFUKooBNAple}
	Soient \( a\neq 0\) et \( b>1\) dans \( \eN\). Alors
	\begin{equation}
		ab>a.
	\end{equation}
\end{lemma}

\begin{proof}
	Il s'agit d'une application du lemme \ref{LEMooGUXGooBcKJdS} en partant de l'inégalité \( 1<b\) et en la «multipliant» par \( a\).
\end{proof}

\begin{proposition}[\cite{MonCerveau}]		\label{PROPooNaturelsReguliersMultiplication}
	Tous les naturels non nuls sont réguliers par rapport à la multiplication. Autrement dit, si \( a\neq 0\), alors nous avons
	\begin{equation}
		a\times x=a\times y\Rightarrow x=y.
	\end{equation}
\end{proposition}

\begin{proof}
	Soit \( a\neq 0\) dans \( \eN\). Nous supposons que \( a\times x=a\times y\). Puisque l'ordre sur \( \eN\) est total (proposition \ref{PROPooGCCRooFBYrlo}), nous pouvons supposer que \( y\geq x\); sinon il suffit de permuter les rôles de \( x\) et \( y\) dans tout ce qui suit.

	Il existe \( d\in \eN\) tel que \( y=x+d\). En utilisant l'hypothèse \( a\times y=a\times x\) et la distributivité\footnote{Proposition \ref{PROPooGHDOooFYRmon}\ref{ITEMooDYLIooETIBEL}.},
	\begin{equation}
		a\times x=a\times y=a\times (x+d)=(a\times x)+(a\times d).
	\end{equation}
	Puisque \( (a\times x)\) est régulier pour la somme\footnote{Proposition \ref{PROPooTLTSooGNMTmV}\ref{ITEMooNUTHooJWWzGv}.} nous en déduisons que
	\begin{equation}
		0=a\times d.
	\end{equation}
	Le lemme \ref{LEMooEHYEooLDudfn} dit alors que \( a=0\) ou que \( d=0\). Étant donné que \( a\neq 0\) par hypothèse, nous déduisons que \( d=0\), c'est-à-dire que \( x=y\).
\end{proof}

%---------------------------------------------------------------------------------------------------------------------------
\subsection{Presque unicité des triplets naturels}
\label{SUBooUniciteTripletsNaturels}
%---------------------------------------------------------------------------------------------------------------------------

\begin{normaltext}	\label{NORooIntroUniciteTripletsNaturels}
	Il existe de nombreux triplets naturels; l'existence d'un triplet naturel est un théorème de la théorie des ensembles que nous avons accepté. Nous avons déjà à peu près montré que toute partie non bornée de \( \eN\) donne lieu à un nouveau triplet naturel. Voir \ref{NORMooQXASooMXqhjI}.

	Nous voyons maintenant que tous les triplets naturels sont équivalents au moins pour l'ordre.
\end{normaltext}

\begin{theorem}[\cite{MonCerveau}]     \label{THOooFUXMooJuigHK}
	Soient des triplets naturels \( (\mN_1,o_1,s_1)\) et \( (\mN_2,o_2,s_2)\). Alors
	\begin{enumerate}
		\item
		      il existe une unique application \( f\colon \mN_1\to \mN_2\) telle que
		      \begin{enumerate}
			      \item
			            \( f(o_1)=o_2\)
			      \item
			            \( f\circ s_1=s_2\circ f\).
		      \end{enumerate}
		\item
		      Une telle application est une bijection croissante.
	\end{enumerate}
\end{theorem}

\begin{proof}
	En plusieurs points.
	\begin{subproof}
		\spitem[Existence]
		Nous voyons \( (\mN_1, o_1, s_1)\) comme un triplet naturel, et \( \mN_2\) comme un simple ensemble. Nous pouvons appliquer le théorème \ref{THOooEJPYooZFVnez} à \( (\mN_1, o_1, s_1)\). L'élément \( o_1\) va jouer le rôle de \( 0\) alors que \( o_2\) va jouer le rôle de \( b\). L'application \( g\) est \( s_2\). Bref, il existe une unique application \( f\colon \mN_1\to \mN_2\) telle que
		\begin{enumerate}
			\item
			      \( f(o_1)=o_2\)
			\item
			      \( f\big( s_1(n) \big)=s_2\big( f(n) \big)\)
		\end{enumerate}
		pour tout \( n\in \mN_1\).
		\spitem[Unicité]
		Le théorème \ref{THOooEJPYooZFVnez} donne déjà l'unicité. Nous la faisons quand même, juste pour vous faire plaisir. Soit \( g\), une autre application vérifiant les mêmes conditions. Pour faire la récurrence de façon très explicite, nous posons
		\begin{equation}
			\begin{aligned}
				P\colon \mN_1 & \to \{ 0,1 \}                    \\
				x             & \mapsto \begin{cases}
					                        1 & \text{si } g(x)=f(x) \\
					                        0 & \text{sinon. }
				                        \end{cases}
			\end{aligned}
		\end{equation}
		Notre but est de prouver que \( P(x)=1\) pour tout \( x\in \mN_1\), en utilisant la récurrence telle que décrite dans la proposition \ref{PROPooXTRCooKwrWkq}.

		Nous avons \( f(o_1)=o_2=g(o_1)\). Donc \( P(o_1)=1\). Nous supposons que, pour un certain \( a\in \mN_1\), nous ayons \( P(a)=1\), et nous prouvons que \( P\big( s_1(a) \big)=1\).

		Nous avons \( g(a)=f(a)\), et nous prenons \( s_2\) des deux côtés, nous avons succéssivement
		\begin{subequations}
			\begin{align}
				(s_2\circ g)(a)     & =(s_2\circ f)(a)      \\
				(g\circ s_1)(a)     & =(f\circ s_1)(a)      \\
				g\big( s_1(a) \big) & =f\big( s_1(a) \big).
			\end{align}
		\end{subequations}
		La dernière égalité signifie que \( P\big( s_1(a) \big)=1\). La proposition \ref{PROPooXTRCooKwrWkq} implique que \( P(x)=1\) pour tout \( x\in \mN_1\).
		\spitem[Bijection, définir l'inverse]
		Nous allons trouver un inverse et le lemme \ref{LEMooWBYSooFqyqQP} nous dit que c'est suffisant. La partie «existence», en inversant les rôles de \( \mN_1\) et \( \mN_2\) nous donne une application \( g\colon \mN_2\to \mN_1\) telle que
		\begin{enumerate}
			\item
			      \( g(o_2)=o_1\)
			\item
			      \( g\circ s_2=s_1\circ g\).
		\end{enumerate}
		Nous allons prouver que \( g\) est un inverse de \( f\).
		\spitem[\( f\circ g=\id\)]
		Nous posons \( A=\{ x\in \mN_2\tq (f\circ g)(x)=x \}\). Nous avons
		\begin{equation}
			f\big( g(o_2) \big)=f(o_1)=o_2,
		\end{equation}
		et donc \( o_2\in A\).

		Supposons que \( x\in A\). Alors
		\begin{subequations}
			\begin{align}
				(f\circ g)\big( s_2(x) \big) & =(f\circ \underbrace{g\circ s_2}_{s_1\circ g})(x)  \\
				                             & =(\underbrace{f\circ s_1}_{=s_2\circ f}\circ g)(x) \\
				                             & =(s_2\circ f\circ g)(x)                            \\
				                             & =s_2\big( (f\circ g)(x) \big)                      \\
				                             & =s_2(x)
			\end{align}
		\end{subequations}
		Donc \( s_2(x)\in A\). Nous en déduisons que \( A=\mN_2\) par le point \ref{ITEMooXPYEooFajywh} de la définition \ref{DEFooBJBOooWlblAx} d'un triplet naturel.
		\spitem[\( g\circ f=\id\)]
		J'imagine que c'est la même chose que dans l'autre sens (ci-dessus)\quext{Je n'ai pas fait les calculs; écrivez-moi si ça pose un problème.}.
	\end{subproof}
\end{proof}

\begin{proposition}     \label{PROPooCCVNooYUYcqG}
	L'ensemble structuré \( (\eN,+,\times, \leq)\) est archimédien\footnote{Définition \ref{DEFooLCWLooYrToFv}.}. En d'autres termes, pour tout \( a,b\in \eN\setminus\{ 0 \}\), il existe \( n\in \eN\) tel que
	\begin{equation}
		b<n\times a.
	\end{equation}
	\index{archimédien!\( \eN\)}
\end{proposition}

\begin{proof}
	Soient \( a,b\in \eN\) avec \( a\neq 0\).

	Si \( a>b\), nous avons le résultat avec \( n=1\).

	Si \( a=b\), en prenant \( n=s(1)\) nous avons le résultat. En effet \( s(1)\times a=a+a\). Puisque \( a\neq 0\), nous avons \( a+a\geq a\) et \( a+a\neq a\), donc \( s(1)\times a>a\).

	La vraie vie est avec \( a<b\). Nous posons
	\begin{equation}
		X=\{ x\in \eN\tq 1\leq x\times a\leq b \}
	\end{equation}
	et
	\begin{equation}
		B=\{ x\times a\tq x\in X \}.
	\end{equation}
	L'ensemble \( X\) est non vide parce que \( 1\in X\). L'ensemble \( B\) est alors également non vide, et majoré par \( b\). La proposition \ref{PROPooMZOWooHmsXzI}\ref{ITEMooKIHZooDRTCdx} nous indique alors que \( B\) possède un plus grand élément que nous allons noter \( x_0\times a\) (\( x_0\in X\)).

	Nous posons \( n=s(x_0)\), et nous avons
	\begin{equation}
		x_0\times a< x_0\times a +a=s(x_0)\times a=n\times a.
	\end{equation}
	Nous en déduisons que \( n\times a\) n'est pas dans \( B\) parce que \( x_0\times a\) est le plus grand élément de \( B\). Donc \( x_0\) n'est pas dans \( X\); nous n'avons donc pas les inégalités
	\begin{equation}
		1\leq n\times a\leq b.
	\end{equation}
	Laquelle des deux inégalités est fausse ? Puisque \( n=s(x_0)\geq 1\) et que \( a\geq 1\), nous avons \( 1\leq n\times a\). Donc c'est la seconde inégalité qui est fausse. Nous avons donc \( n\times a>b\).
\end{proof}

\begin{propositionDef}      \label{DEFooNEVNooJlmJOC}
	Soit \( A\) un ensemble muni d'une loi de composition interne\footnote{Peut-être un anneau, mais comme nous avons l'intention, dans les propositions \ref{PROPooXXGHooLafGsI} et suivantes, de faire des sommes vers \( (\eN,+)\), plutôt un monoïde.} notée \( +\). Si nous avons une application \( \alpha\colon \eN\to A\), alors il existe une unique application \(f \colon \eN\to A  \) telle que
	\begin{subequations}		\label{SUBEQSooKJOFooOOxDKH}
		\begin{numcases}{}
			f(0)=\alpha(0)\\
			f(k)=f(k-1)+\alpha(k).
		\end{numcases}
	\end{subequations}

	La valeur de \( f(k)\) est notée \( \sum_{i=0}^k\alpha(i)\).
\end{propositionDef}

\begin{proof}
	Application du théorème \ref{THOooEJPYooZFVnez}.
\end{proof}

\begin{normaltext}	\label{NORooNotationSommeNaturels}
	Les équations de définition \eqref{SUBEQSooKJOFooOOxDKH} signifient que nous définissons la notation \( \sum_{i=0}^N\alpha(i)\) par récurrence par les conditions suivantes :
	\begin{enumerate}
		\item       \label{ITEMooIPDTooEhOxea}
		      \( \sum_{i=0}^0\alpha(i)=\alpha(0)\),
		\item	\label{ITEMooSommeNaturelsSucc}
		      \( \sum_{i=0}^{k}\alpha(i)=\sum_{i=0}^{k-1}\alpha(i)+\alpha(k)\).
	\end{enumerate}
\end{normaltext}


\begin{proposition}[La multiplication est une somme itérée\cite{RWWJooJdjxEK}]        \label{PROPooXXGHooLafGsI}
	Pour tout \( a,b\in \eN\), nous avons
	\begin{equation}
		\sum_{i=1}^na=a\times n.
	\end{equation}
\end{proposition}

\begin{proof}
	Nous le faisons par récurrence en partant de \( n=1\). Avec \( n=1\) nous avons \( \sum_{i=1}^1a=a\), et \( a\times 1=a\). Donc c'est bon.

	Pour la récurrence nous avons :
	\begin{equation}
		a\times s(n)=a\times n+a=\sum_{i=1}^na+a=\sum_{i=1}^{n+1}a=\sum_{i=1}^{s(n)}a.
	\end{equation}
\end{proof}

\begin{lemma}       \label{LEMooIETGooMyrilW}
	Soit \( a>1\) dans \( \eN\). Pour tout \( n\geq 1\) nous avons \( na\leq a^n\).
\end{lemma}

\begin{proof}
	Par récurrence. Avec \( n=1\) nous avons bien \( a\leq a\); pas de problème. Supposons que \( na\leq a^n\), et montrons le pas de récurrence. Nous avons :
	\begin{subequations}
		\begin{align}
			(n+1)a & =   na+a                                   \\
			       & \leq  na+na  & \text{parce que }  a\leq na \\
			       & =   2na                                    \\
			       & \leq  2a^n   & \text{récurrence}           \\
			       & \leq  aa^n   & \text{parce que } a\geq 2   \\
			       & =   a^{n+1}.
		\end{align}
	\end{subequations}
\end{proof}

\begin{proposition}[\cite{RWWJooJdjxEK}]	\label{PROPooSuiteGeometriqueCroissante}
	Soit \( a>1\). Alors
	\begin{enumerate}
		\item
		      l'application \( n\mapsto a^n\) est strictement croissante;
		\item
		      l'ensemble \( \{ a^n\tq n\in \eN \}\) n'est pas majoré.
	\end{enumerate}
\end{proposition}

\begin{proof}
	Nous avons
	\begin{subequations}
		\begin{align}
			a^{n+1} & =a^n\times a                                        \\
			        & >a^n\times 1 & \text{lemme \ref{LEMooSFUKooBNAple}} \\
			        & =a^n.
		\end{align}
	\end{subequations}
	Cela prouve le premier point.

	Pour le second point, soit \( m\in \eN\). Nous devons trouver \( N\in \eN\) tel que \( a^N\geq m\). Puisque \( \eN\) est archimédien\footnote{Proposition \ref{PROPooCCVNooYUYcqG}.}, nous pouvons considérer \( N\) tel que \( Na>m\). Le lemme \ref{LEMooQBHFooCuCusQ} nous assure alors que
	\begin{equation}
		m<Na\leq a^N.
	\end{equation}
\end{proof}

\begin{theorem}[division euclidienne \cite{RWWJooJdjxEK}]       \label{THOooKDJVooRIJRHP}
	Pour tout \( a\in \eN\), pour tout \( b\in \eN\setminus\{ 0 \}\), il existe un unique couple \( (q,r)\in \eN^2\) tel que \( a=bq+r\) avec \( 0\leq r<b\).
\end{theorem}

\begin{proof}
	Existence puis unicité.
	\begin{subproof}
		\spitem[Existence]

		Nous posons
		\begin{equation}
			A=\{ bx\tq x\in \eN,bx\leq a \}.
		\end{equation}
		L'ensemble \( A\) contient \( 0\) (avec \( x=0\)) et est majoré par \( a\). Donc il possède un plus grand élément que nous notons \( bq\). Puisque \( bq\in A\), nous avons \( bq\leq a\) et donc il existe \( r\in \eN\) tel que
		\begin{equation}        \label{EQooIUICooLenNBP}
			bq+r=a.
		\end{equation}
		Il reste à montrer que \( r<b\). Supposons que \( r\geq b\). Il existerait alors un \( x\) tel que \( b+x=r\). En mettant ça dans \eqref{EQooIUICooLenNBP},
		\begin{equation}
			bq+b+x=a,
		\end{equation}
		c'est-à-dire \( b(q+1)+x=a\), qui signifierait \( b(q+1)\leq a\), ce qui est faux parce que \( bq\) est le plus grand élément de \( A\).
		\spitem[Unicité]
		Supposons que nous ayons
		\begin{equation}
			a=bq+r=bq'+r'
		\end{equation}
		avec \( 0\leq r<b\) et \( 0\leq r'<b\). Il y a trois possibilités : \( q'<q\), \( q'=q\) et \( q'>q\).
		\begin{subproof}
			\spitem[Si \( q'<q\)]
			Alors il existe \( x\in \eN\) tel que \( q'+x=q\), et nous avons
			\begin{equation}
				b(q'+x)+r=bq'+bx+r,
			\end{equation}
			ce qui, après distribution et simplification, donne \( r'=bx+r\). Puisque nous avons \( x\geq 1\), il vient
			\begin{equation}
				r'=bx+r\geq b+r\geq b.
			\end{equation}
			Cela n'est pas possible parce que \( r'<b\). Le cas \( q'<q\) n'est pas possible.
			\spitem[Si \( q'=q\)]
			Nous avons alors immédiatement \( bq+r=bq+r'\) et donc \( r=r'\). Unicité.
			\spitem[Si \( q'>q\)]
			En posant \( q+x=q'\) nous trouvons la même impossibilité que dans le cas \( q'<q\).
		\end{subproof}
	\end{subproof}
\end{proof}

\begin{definition}	\label{DEFooDivisibleNombrePairImpair}
	Quelques définitions associées au théorème \ref{THOooKDJVooRIJRHP}. Soit \( a \) et \( b \) deux entiers naturels.
	\begin{enumerate}
		\item
		      Dans le couple \( (q,r)\in \eN^2\) tel que \( a=bq+r\) avec \( 0\leq r<b\), fourni par ce théorème, \( q \) est appelé \defe{quotient}{division euclidienne!quotient dans \( \eN \)} et \( r \) est appelé \defe{reste}{division euclidienne!reste dans \( \eN \)}.
		\item
		      Si \( r=0\) (et donc s'il existe un \( q \in \eN \) tel que \( a = bq \), nous disons que \( a\) est \defe{divisible}{nombre entier!divisible} par \( b\).
		\item
		      Si \( n\) est divisible par \( 2\), alors nous disons qu'il est \defe{pair}{nombre entier!pair}.
		      \item\label{ITEMooDLHHooTmWSnN}
		      Si \( n\) n'est pas divisible par \( 2\), alors nous disons qu'il est \defe{impair}{nombre entier!impair}.
	\end{enumerate}
\end{definition}

%---------------------------------------------------------------------------------------------------------------------------
\subsection{Écriture d'un naturel dans une base}
\label{SUBooEcritureNaturels}
%---------------------------------------------------------------------------------------------------------------------------

\begin{normaltext}	\label{NORooEcritureChiffres}
	Nous avons déjà donné la notation \( 1=s(0)\). Nous continuons avec \( 2=s(1)\), \( 3=s(2)\), \( 4=s(3)\), \( 5=s(4)\), \( 6=s(5)\), \( 7=s(6)\), \( 8=s(7)\) et \( 9=s(8)\).

	Nous allons maintenant voir comment écrire des nombres plus grands.
\end{normaltext}

\begin{normaltext}	\label{NORooTuplesChiffres}
	Si \( b>1\) et \( N\in \eN\) sont donnés, nous notons
	\begin{equation}
		C_{b,N}=\big\{  u\in \{ 0,\ldots, b-1 \}^{N+1}\tq u_N\neq 0  \big\}.
	\end{equation}
	où les \( u_i\) sont numérotés à partir de \( 0\); donc dire \( u_N\neq 0\) revient à dire que le \emph{dernier} est non nul, et non l'avant dernier. Nous définissons\footnote{Le symbole de sommation est défini par \ref{DEFooNEVNooJlmJOC}.}
	\begin{equation}        \label{EQooWWTUooHAnSEv}
		\begin{aligned}
			\varphi_{b,N}\colon C_{b,N} & \to \eN                     \\
			u                           & \mapsto \sum_{i=0}^Nu_ib^i.
		\end{aligned}
	\end{equation}
	Cette application \( \varphi_{b,N}\) sera encore bien étudiée pour la partie décimale d'un réel. Voir la définition \ref{EqXXXooOTsCK}.
\end{normaltext}

\begin{lemma}[\cite{RWWJooJdjxEK}]       \label{LEMooJUGKooGsbrhi}
	Soient \( b>1\), \( N\geq 0\) ainsi que \( u\in C_{b,N}\). Alors
	\begin{equation}        \label{EQooYHTLooNwqIIq}
		b^N\leq \varphi_{b,N}(u)<b^{N+1}.
	\end{equation}
\end{lemma}

\begin{proof}
	En séparant la somme nous avons
	\begin{equation}
		\varphi_{b,N}(u)=u_Nb^N+\sum_{i=0}^{N-1}u_ib^i.
	\end{equation}
	Puisque \( u_N\geq 1\) nous avons \( b^N\leq u_Nb^N\), et donc
	\begin{equation}
		b^N\leq u_Nb^N\leq \varphi_{b,N}(u).
	\end{equation}
	Voilà qui prouve la première inégalité de \eqref{EQooYHTLooNwqIIq}.

	Pour prouver que \( \varphi_{b,N}(u)<b^{N+1}\), nous faisons une récurrence sur \( N\).
	\begin{subproof}
		\spitem[Pour \( N=0\)]
		Nous devons prouver que \( \varphi_{b,0}(u)<b\). Par définition \( \varphi_{b,N}(u)=u_0b^0\). Puisque \( u\in\{ 0,\ldots, b-1 \}^{N+1}\), nous avons \( u_0\leq b-1<b\).
		\spitem[Récurrence]
		Nous supposons que pour tout \( u\in C_{b,N}\) nous avons \( \varphi_{b,N}(u)<b^{N+1}\). Et nous devons montrer que pour tout \( v\in C_{b,N+1}\) nous avons \( \varphi_{b,N+1}(v)<b^{N+2}\).

		Nous posons \( u=(v_0,\ldots, v_N)\); nous avons alors
		\begin{subequations}
			\begin{align}
				\varphi_{b,N+1}(v) & =   v_{N+1}b^{N+1}+\sum_{i=0}^Nv_ib^i                                    \\
				                   & =   v_{N+1}b^{N+1}+\varphi_{b,N}(u)                                      \\
				                   & <   v_{N+1}b^{N+1}+b^{N+1}            & \text{récurrence}                \\
				                   & =   (v_{N+1}+1)b^{N+1}                                                   \\
				                   & \leq  bb^{N+1}                        & \text{parce que }v_{N+1}\leq b-1 \\
				                   & =   b^{N+2}.
			\end{align}
		\end{subequations}
	\end{subproof}
\end{proof}

\begin{lemma}[\cite{MonCerveau}]        \label{LEMooKDKJooSkhcJS}
	Soient \( x\in \eN\) ainsi que \( b\geq 2\). Nous posons
	\begin{equation}
		N=\max\{ k\in \eN\tq b^k\leq x \}.
	\end{equation}
	Alors
	\begin{enumerate}
		\item
		      Si \( n>N\) alors \( \varphi_{b,n}(u)>x\) pour tout \( u\in C_{b,n}\).
		\item
		      Si \( n<N\) alors \( \varphi_{b,n}(u)<x\) pour tout \( u\in C_{b,n}\).
	\end{enumerate}
\end{lemma}

\begin{proof}
	En deux parties.
	\begin{subproof}
		\spitem[Si \( n>N\)]
		Nous avons, par définition de \( C_{b,n}\) que \( u_n\neq 0\), de telle sorte que
		\begin{equation}
			\varphi_{b,n}(u)\geq u_nb^n\geq b^n>x .
		\end{equation}
		La dernière inégalité est due au fait que \( n\notin \{ k\in \eN\tq b^k\leq x \}\).
		\spitem[Si \( n<N\)]
		Nous avons
		\begin{equation}
			x\geq b^N> \varphi_{b,n}(u).
		\end{equation}
		Le seconde inégalité est une conséquence du lemme \ref{LEMooJUGKooGsbrhi}.
	\end{subproof}
\end{proof}

\begin{theorem}[\cite{RWWJooJdjxEK}]	\label{THMooUniciteEcritureChiffres}
	Soit \( b\geq 2\). Si \( x\in \eN\), alors il existe un unique \( N\in \eN\) et un unique \( u\in C_{b,N} \) tels que
	\begin{equation}
		x=\varphi_{b,N}(u).
	\end{equation}
\end{theorem}

\begin{proof}
	Nous commençons par \( x<b\). Dans ce cas, \( N=0\) parce que si \( u_k\neq 0\) avec \( k\neq 0\), nous avons
	\begin{equation}
		\sum_{i=0}^Nu_ib^i\geq u_kb^k\geq b>x.
	\end{equation}
	Donc \( x=x_0b^0=u_0\). Bref, dans le cas \( x<b\) nous avons obligatoirement \( N=0\) et \( u_0=x\).

	Nous étudions à présent le cas \( x\geq b\) que nous subdivisons en plusieurs étapes.
	\begin{subproof}
		\spitem[\( N\geq 1\)]
		Si \( N=0\), alors \( \varphi_{b,0}(u)=u_0<b\leq x\). Donc \( N\geq 1\).

		Notons incidemment que nous pouvons parler de \( N-1\) à partir de maintenant.
		\spitem[Unicité, préambule]
		Le lemme \ref{LEMooKDKJooSkhcJS} nous indique que si \( x=\varphi_{b,N}(u)\) pour un certain \( N\in \eN\) et un certain \( u\in C_{b,N}\), alors
		\begin{equation}
			N=\max\{ k\in \eN\tq b^k\leq x \}.
		\end{equation}
		Nous posons
		\begin{equation}
			X_k=\sum_{i=k}^Nu_ib^{i-k},
		\end{equation}
		et nous allons montrer que le couple \( (X_{k+1}, u_k)\) est le résultat de la division euclidienne\footnote{Théorème \ref{THOooKDJVooRIJRHP}.} de \( X_k\) par \(b\).

		D'abord, \( u_k<b\), donc ça a bien la tête d'un reste. Ensuite, pour le quotient,
		\begin{subequations}
			\begin{align}
				bX_{k+1}+u_k & =b\sum_{i=k+1}^Nu_ib^{i-(k+1)}+u_k \\
				             & =\sum_{i=k+1}^Nu_ib^{i-k}+u_k      \\
				             & =\sum_{i=k}^Nu_ib^{i-k}            \\
				             & =X_k.
			\end{align}
		\end{subequations}
		\spitem[Unicité]
		En quoi cela fait-il avancer la choucroute ? Supposons que \( \varphi_{b,N}(u)=\varphi_{b,M}(v)\). Alors nous avons déjà prouvé que
		\begin{equation}
			M=N=\max\{ k\in \eN\tq b^k\leq x \}.
		\end{equation}
		Ensuite nous devons montrer que \( u=v\). Nous posons \( X_k=\sum_{i=k}^Nu_ib^{i-k}\) et \( Y_k=\sum_{i=k}^Nv_ib^{i-k}\). Notez que
		\begin{equation}
			X_0=Y_0=x.
		\end{equation}
		Si \( X_k=Y_k\), alors par unicité de la division euclidienne nous avons \( X_{k+1}=Y_{k+1}\) et \( u_k=v_k\). Par récurrence nous avons \( X_k=Y_k\) et \( u_k=v_k\) pour tout \( k\).

		\spitem[Existence]
		Soit \( x\in \eN\). Nous posons \( y_0=x\) et
		\begin{equation}
			y_k=by_{k+1}+u_k
		\end{equation}
		avec \( u_k<b\). Vus l'unicité dans la division euclidienne et le théorème\footnote{Nous ne citerons pas toujours ce théorème à chaque fois que nous définissons quelque chose par récurrence.} \ref{THOooEJPYooZFVnez} permettant la définition par récurrence, ces conditions définissent deux suites \( (u_k)\) et \( (y_k)\) dans \( \eN\).

		Montrons qu'il existe un \( N\in \eN\) tel que \( y_n=0\) pour tout \( n\geq N+1\). Nous avons :
		\begin{subequations}
			\begin{align}
				2y_{k+1} & \leq b y_{k+1} & \text{parce que } b\geq 2    \\
				         & \leq y_k       & \text{pcq }by_{k+1}+u_k=y_k.
			\end{align}
		\end{subequations}
		Bref : \( 2y_{k+1}\leq y_k\). Par récurrence\footnote{Faut-il citer la proposition \ref{PROPooXTRCooKwrWkq} et donner explicitement la fonction \( P\) ?} nous trouvons que
		\begin{equation}
			2^ky_k\leq x
		\end{equation}
		parce que \( y_0=x\). Par le lemme \ref{LEMooIETGooMyrilW}, si \( k\) est assez grand,
		\begin{equation}
			2ky_k\leq 2^ky_k\leq x.
		\end{equation}
		Puisque \( \eN\) est archimédien\footnote{Proposition \ref{PROPooCCVNooYUYcqG}.}, nous pouvons considérer \( s\in \eN\) tel que \( 2s>x\). À ce moment nous avons
		\begin{equation}
			y_n=0
		\end{equation}
		pour tout \( n\geq s\). Nous posons
		\begin{equation}
			N=\max\{ k\tq y_k\neq 0 \}.
		\end{equation}
		Prouvons par récurrence sur \( l\) que
		\begin{equation}        \label{EQooZBKQooFqcckr}
			y_{N-l}=\sum_{i=N-l}^Nu_ib^{(i+l)-N}.
		\end{equation}
		Notez que \( i+l\geq N-l+l=N\), donc \( (i+l)-N\) a un sens.
		\begin{subproof}
			\spitem[Pour \( l=0\)]
			Avec \( l=0\) nous avons \( \sum_{i=N-l}^Nu_ib^{(i+l)-N}=u_N\). Il faut donc voir que \( y_N=u_N\). Nous avons
			\begin{equation}
				y_N=by_{N+1}+u_N.
			\end{equation}
			En se rappelant que \( y_{N+1}=0\), nous avons le résultat.
			\spitem[Pour \( l+1\)]
			Pour la récurrence nous avons le calcul suivant :
			\begin{subequations}
				\begin{align}
					y_{N-l-1} & =by_{N-l}+u_{N-l-1}                       \\
					          & =b\sum_{i=N-l}^Nu_ib^{(i+l)-N}+u_{N-l-1}  \\
					          & =\sum_{i=N-l}^Nu_ib^{(i+l)-N+1}+u_{N-l-1} \\
					          & =\sum_{i=N-l-1}^Nu_ib^{(i+l+1)-N}.
				\end{align}
			\end{subequations}
			La récurrence est prouvée. L'égalité \eqref{EQooZBKQooFqcckr} est validée pour tout \( l\).
		\end{subproof}
		En posant \( l=N\) dans \eqref{EQooZBKQooFqcckr} nous trouvons
		\begin{equation}
			y_0=\sum_{i=0}^Nu_ib^i.
		\end{equation}
		Mais la définition de la suite \( (y_k)\) contient \( y_0=x\). Donc nous avons prouvé que
		\begin{equation}
			x=\sum_{i=0}^Nu_ib^i=\varphi_{b,N}(u).
		\end{equation}
	\end{subproof}
\end{proof}

\begin{example}		\label{EXooEcritureDix}
	Comment écrire le nombre \( b\) en base \( b\) ? Nous devons trouver un \( N\) et une suite \( (u_i)\) tels que
	\begin{equation}
		b=\sum_{i=0}^Nu_ib^i.
	\end{equation}
	Il est facile de voir que le choix \( N=1\) et \( u=(0,1)\) fonctionne bien : \( b=1\times b^1+0\). Nous avons donc
	\begin{equation}
		b=\varphi_{b,1}(1,0).
	\end{equation}
	Nous écrivons cela plus sobrement \( b=10\).
\end{example}

\begin{normaltext}	\label{NORooBaseStandardDix}
	À part des cas très exceptionnels, nous utilisons toujours la base \( b=s(9)=s^9(0)\). Nous nous permettons donc d'écrire «64» le nombre \( \varphi_{s(9), 2}(6,4)\). Vous saviez que tout groupe simple d'ordre \( \varphi_{s(9), 2}(6,0)\) est isomorphe au groupe alterné \( A_{\varphi_{s(9),0}(5)}\) ? C'est la proposition \ref{PROPooUBIWooTrfCat}.
\end{normaltext}

\begin{proposition}[\cite{RWWJooJdjxEK}]	\label{PROPooComparaisonNaturelsParChiffres}
	Soit \( b > 2 \) un entier naturel fixé, et \( M>N\) deux entiers naturels. Si \( u\in C_{b,N}\) et \( v\in C_{b,M}\), alors \( \varphi_{b,N}(u)<\varphi_{b,M}(v)\).

	De manière plus intuitive: à base fixée, le nombre qui a le plus de chiffres est le plus grand.
\end{proposition}

\begin{proof}
	Le lemme  \ref{LEMooJUGKooGsbrhi} nous dit que
	\begin{equation}
		b^N\leq \varphi_{b,N}(u)< b^{N+1}
	\end{equation}
	et
	\begin{equation}
		b^M\leq \varphi_{b,M}(v)< b^{M+1}.
	\end{equation}
	Puisque \( M>N\) nous avons \( b^{N+1}\leq b^M\) et donc
	\begin{equation}
		\varphi_{b,N}(u)< b^{N+1}\leq b^M\leq \varphi_{b,M}(v).
	\end{equation}
\end{proof}

\begin{proposition}	\label{PROPooComparaisonNaturelsParChiffresIdentiques}
	Soient \( u,v\in C_{b,N}\) tels que \( u_i=v_i\) pour \( i=r+1,\ldots, N\). Si \( u_r>v_r\) alors \( \varphi_{b,N}(u)>\varphi_{b,N}(v)\).

	De manière intuitive: avec le même nombre de chiffres, le plus grand est celui dont le premier chiffre différent est le plus grand.
\end{proposition}

Autrement dit, les nombres en écriture de position se classent par ordre lexicographique.

\begin{proof}
	En découpant les sommes nous avons
	\begin{equation}
		\varphi_{b,N}(u)=\sum_{i=r+1}^Nu_ib^i+u_rb^r+\sum_{i=0}^{r-1}u_ib^i
	\end{equation}
	et
	\begin{equation}
		\varphi_{b,N}(v)=\sum_{i=r+1}^Nu_ib^i+v_rb^r+\sum_{i=0}^{r-1}v_ib^i.
	\end{equation}
	Puisque \( b^r>\sum_{i=0}^{r-1}u_ib^i\) (lemme \ref{LEMooJUGKooGsbrhi}), nous avons aussi
	\begin{equation}        \label{EQooTZPBooTeauhX}
		b^r+\sum_{i=0}^{r-1}u_ib^i>\sum_{i=0}^{r-1}v_ib^i.
	\end{equation}
	Et le calcul final :
	\begin{subequations}
		\begin{align}
			\varphi_{b,N}(v) & <   \sum_{i=r+1}^Nv_ib^i+v_rb^r+b^r+\sum_{i=0}^{r-1}u_ib^i & \text{pcq \eqref{EQooTZPBooTeauhX}} \\
			                 & =   \sum_{i=r+1}^Nu_ib^i+(v_r+1)b^r+\sum_{i=0}^{r-1}u_ib^i                                       \\
			                 & \leq  \sum_{i=r+1}^Nu_ib^i+u_rb^r+\sum_{i=0}^{r-1}u_ib^i   & \text{pcq } u_r\geq v_r+1           \\
			                 & =   \sum_{i=0}^Nu_ib^i                                                                           \\
			                 & =   \varphi_{b,N}(u).
		\end{align}
	\end{subequations}
	Et voilà.
\end{proof}

%+++++++++++++++++++++++++++++++++++++++++++++++++++++++++++++++++++++++++++++++++++++++++++++++++++++++++++++++++++++++++++
\section{Les entiers}
\label{SECooNombresEntiers}
%+++++++++++++++++++++++++++++++++++++++++++++++++++++++++++++++++++++++++++++++++++++++++++++++++++++++++++++++++++++++++++

\begin{propositionDef}[\cite{RWWJooJdjxEK}]     \label{PROPooFIKUooVHlvTt}
	Soient \( a,b,a',b'\in \eN\). Nous disons que \( (a,b)\sim(a',b')\) si et seulement si
	\begin{equation}
		a+b'=b+a'
	\end{equation}
	\begin{enumerate}
		\item
		      \( \sim\) est une relation d'équivalence sur \( \eN^2\).
		\item       \label{ITEMooZQSHooSDfdvK}
		      Si \( (a,b)\sim (a',b')\) et \( (x,y)\sim (x',y')\) alors
		      \begin{equation}
			      (a+x,b+y)\sim(a'+x',b'+y').
		      \end{equation}
	\end{enumerate}
	L'ensemble des \defe{entiers}{entier} est
	\begin{equation}
		\eZ=(\eN\times \eN)/\sim,
	\end{equation}
	et nous notons \( \overline{ a,b }\in \eZ\) la classe de \( (a,b)\in \eN\times \eN\).
\end{propositionDef}

\begin{proof}
	En plusieurs points.
	\begin{subproof}
		\spitem[Symétrie]
		C'est la commutativité de la somme dans \( \eN\), proposition \ref{PROPooTLTSooGNMTmV}\ref{ITEMooIFFPooXfftfG}.
		\spitem[Réflexive]
		Immédiat.
		\spitem[Transitive]
		Nous supposons que \( (a,b)\sim(u,v)\) et que \( (u,v)\sim(x,y)\). Alors nous avons
		\begin{subequations}
			\begin{align}
				a+v & =u+b  \\
				u+y & =v+x.
			\end{align}
		\end{subequations}
		En additionnant membre à membre,
		\begin{equation}
			a+v+u+y=u+b+v+x.
		\end{equation}
		La commutativité nous permet de mettre \( u\) et \( v\) à droite dans chacun des deux membres. Ensuite la proposition \ref{PROPooTLTSooGNMTmV}\ref{ITEMooNUTHooJWWzGv} nous permet de simplifier par \( u+v\). Il reste \( a+y=b+x\), qui signifie \( (a,b)\sim(x,y)\).
		\spitem[Pour \ref{ITEMooZQSHooSDfdvK}]
		L'hypothèse donne les égalités
		\begin{subequations}
			\begin{align}
				a+b' & =b+a' \\
				x+y' & =y+x'
			\end{align}
		\end{subequations}
		En sommant, et en utilisant l'associativité,
		\begin{equation}
			(a+x)+(b'+y')=(b+y)+(a'+x').
		\end{equation}
		Cela signifie bien que \( (a+x,b+y)\sim(a'+x',b'+y')\).
	\end{subproof}
\end{proof}

\begin{lemma}	\label{LEMooEntiersEquivalentsZero}
	Soient \( a,b\in \eN\). Nous avons \( (a,b)\sim (0,0)\) si et seulement si \( a=b\).
\end{lemma}

\begin{proof}
	Dire que \( (a,b)\sim (0,0)\) est équivalent à dire que \( a+0=b+0\), ou encore que \( a=b\).
\end{proof}

\begin{propositionDef}[\cite{RWWJooJdjxEK}]	\label{PROPooDEFAdditionEntiersRelatifs}
	Soient \( a,b,x,y\in \eN\). L'application
	\begin{equation}
		\begin{aligned}
			f\colon \overline{ (a,b)\times \overline{ (x,y) } } & \to \eZ               \\
			\big( (a',b'),(x',y') \big)                         & \mapsto (a'+x',b'+y')
		\end{aligned}
	\end{equation}
	est constante.

	Nous nommons sa valeur \( \overline{ (a,b) }+\overline{ (x,y) }\).
\end{propositionDef}

\begin{proof}
	Cela est une conséquence de la proposition \ref{PROPooFIKUooVHlvTt}\ref{ITEMooZQSHooSDfdvK}.
\end{proof}

\begin{proposition}	\label{PROPooGroupeEntiersRelatifs}
	L'ensemble \( \eZ \), muni de l'addition telle que définie par la proposition \ref{PROPooDEFAdditionEntiersRelatifs}, est un groupe commutatif admettant pour neutre l'élément \( e = \overline{(0,0)} \).
\end{proposition}

\begin{proof}
	En plusieurs points.
	\begin{subproof}
		\spitem[Neutre]
		Pour tous entiers naturels \( a,\ b \), on a
		\begin{equation}
			\overline{ (a,b) }+\overline{ (0,0) }=\overline{ (a+0,b+0) }=\overline{ (a,b) }.
		\end{equation}
		De même \( e+\overline{ (a,b) }=\overline{ (a,b) }\) par commutativité de la somme dans \( \eN\).
		\spitem[Inverse]
		Il faut vérifier que \( \overline{ (b,a) }\) est l'inverse de \( \overline{ (a,b) }\).
		% TODOooPreuveOpposeRelatifs à faire.
		\spitem[Associativité]
		Calcul direct en utilisant l'associativité dans \( \eN\).
	\end{subproof}
\end{proof}

\begin{proposition}		\label{PROPooVMGYooUWsbSr}
	L'application
	\begin{equation}
		\begin{aligned}
			\iota\colon \eN & \to \eZ                    \\
			n               & \mapsto \overline{ (n,0) }
		\end{aligned}
	\end{equation}
	est un morphisme\footnote{Certes \( \eN\) n'est pas un groupe, donc le mot «morphisme» est un peu abusé, mais vous voyez ce que je veux dire.} injectif.
\end{proposition}

\begin{proof}
	Le fait que ce soit un morphisme est le calcul
	\begin{equation}
		\iota(a+b)=\overline{ (a+b,0) }=\overline{ (a,0) }+\overline{ (b,0) }=\iota(a)+\iota(b).
	\end{equation}

	Pour l'injectivité, supposons que \( \iota(a)=\iota(b)\). Alors \( \overline{ (a,0) }=\overline{ (b,0) }\), c'est-à-dire \( a+0=b+0\). Donc \( a=b\).
\end{proof}

%---------------------------------------------------------------------------------------------------------------------------
\subsection{Opposé}
\label{SUBooEntiersRelatifsOpposes}
%---------------------------------------------------------------------------------------------------------------------------

\begin{lemma}       \label{LEMooSABNooZZDIes}
	Tout élément de \( \eZ\) a un représentant de la forme \( (a,0)\) ou \( (0,b)\).
\end{lemma}

\begin{proof}
	Soient \( a,b\in \eN\). Si \( b\leq a\), alors nous avons
	\begin{equation}
		(a,b)\sim(a-b,0)
	\end{equation}
	où la différence est calculée dans \( \eN\) et a un sens parce que nous avons supposé \( b\leq a\). Si par contre \( a\leq b\) alors
	\begin{equation}
		(a,b)\sim(0,b-a).
	\end{equation}
	Puisque l'ordre sur \( \eN\) est total\footnote{Proposition \ref{PROPooGCCRooFBYrlo}.}, tous les cas sont couverts.
\end{proof}

\begin{lemmaDef}	\label{LEMooDEFRelatifOppose}
	Soit \( z\in \eZ\). L'application\footnote{Pour rappel, \( z\) est une classe d'équivalence dans \( \eN\times \eN\), c'est-à-dire une partie de \( \eN\times \eN\). Ça a un sens de prendre \( z\) comme ensemble sur lequel on définit une fonction.}
	\begin{equation}
		\begin{aligned}
			f\colon z & \to \eZ                    \\
			(a,b)     & \mapsto \overline{ (b,a) }
		\end{aligned}
	\end{equation}
	est constante.

	Nous nommons \( -z\) sa valeur.
\end{lemmaDef}

\begin{proof}
	Soient \( (a,b)\) et \( (x,y)\) dans \( z\). Nous avons successivement :
	\begin{itemize}
		\item
		      \( (a,b)\sim (x,y)\).
		\item
		      \( a+y=b+x\).
		\item
		      \( (b,a)\sim (y,x)\)
		\item
		      \( \overline{ (b,a) }=\overline{ (y,x) }\)
		\item
		      \( f(a,b)=f(x,y)\).
	\end{itemize}
	D'où la constance de \( f\).
\end{proof}

\begin{proposition}[\cite{MonCerveau}]	\label{PROPooJMETooHQGwnv}
	Soient \( a,b\in \eZ\). Nous avons
	\begin{enumerate}
		\item		\label{ITEMooWNYGooMopPcq}
		      \( (-a)b=-(ab)\)
	\end{enumerate}
\end{proposition}


\begin{lemma}	\label{LEMooEntiersRelatifsSigne}
	Nous avons
	\begin{enumerate}
		\item       \label{ITEMooSQFGooQPgIMu}
		      \( \eZ=\iota(\eN)\cup -\iota(\eN)\)
		\item       \label{ITEMooHQUQooJeqULl}
		      \( \iota(\eN)\cap -\iota(\eN)=\{ 0 \}\).
	\end{enumerate}
\end{lemma}

\begin{proof}
	Nous avons
	\begin{subequations}
		\begin{align}
			\iota(\eN)  & =\{ \overline{ (n,0) }\tq n\in \eN \}    \label{SUBEQooVJGVooCUxtvk} \\
			-\iota(\eN) & =\{ \overline{ (0,n) }\tq n\in \eN \}.   \label{SUBEQooAPGRooOCkYRr}
		\end{align}
	\end{subequations}
	Montrons à présent les deux points.
	\begin{subproof}
		\spitem[Pour \ref{ITEMooSQFGooQPgIMu}]
		Nous savons par le lemme \ref{LEMooSABNooZZDIes} que tous les éléments de \( \eZ\) sont de la forme \eqref{SUBEQooVJGVooCUxtvk} ou \eqref{SUBEQooAPGRooOCkYRr}.
		\spitem[Pour \ref{ITEMooHQUQooJeqULl}]
		Si \( z\in \iota(\eN)\cap -\iota(\eN)\), il existe \( n,m\in \eN\) tels que \( \overline{ (n,0) }=\overline{ (0,m) }\), ce qui signifie en particulier que \( (n,0)\sim(0,m)\) ou encore que \( n+m=0\). Le lemme \ref{LEMooQBHFooCuCusQ} dit alors que \( n=m=0\).

		Nous avons donc \( z= \overline{ (0,0) }=0\).
	\end{subproof}
\end{proof}


\begin{proposition}		\label{PROPooCHGRooRksGGO}
	L'application
	\begin{equation}
		\begin{aligned}
			\iota\colon \eN & \to \eZ                    \\
			n               & \mapsto \overline{ (n,0) }
		\end{aligned}
	\end{equation}
	est un morphisme injectif qui respecte l'ordre.
\end{proposition}

\begin{proof}
	Le fait que ce soit un morphisme injectif est la proposition \ref{PROPooVMGYooUWsbSr}.

	Il faut encore prouver que ça respecte l'ordre\quext{Faites-le en envoyez-moi la preuve.}
	%TODOooKQPQooAUkOlS il faut le faire.
\end{proof}


\begin{proposition}[\cite{MonCerveau}]	\label{PROPooORABooXRbVoz}
	Soit \( G\), un sous-groupe de \( \eZ\) non réduit à \( \{ 0 \}\). Alors
	\begin{enumerate}
		\item		\label{ITEMooUZBJooUeVmmz}
		      La partie \( \{ x>0\tq x\in G \}\) est non vide.
		\item	\label{ITEMooIXXFooOWkbPb}
		      En posant
		      \begin{equation}
			      p=\min\{ x>0\in G \},
		      \end{equation}
		      nous avons \( G=p\eZ\).
	\end{enumerate}
\end{proposition}

\begin{proof}
	Vu que \( G\) est un sous-groupe de \( (\eZ,+)\), si \( x,y\in G\) nous avons \( x+y\in G\), \( -x\in G\), etc.
	\begin{subproof}
		\spitem[Pour \ref{ITEMooUZBJooUeVmmz}]
		%-----------------------------------------------------------
		Soit \( x\neq 0\) dans \( G\). Si \( x>0\) c'est bon. Sinon \( -x\in G\) et \( -x>0\).
		\spitem[Pour \ref{ITEMooIXXFooOWkbPb}]
		%-----------------------------------------------------------
		Nous posons
		\begin{equation}
			p=\min\{ x\in G\tq x>0 \}.
		\end{equation}
		Vu que \( p\in G\) nous avons \( p\eZ\subset G\). Nous devons montrer l'inclusion inverse.

		Soit \( y\in G\). Par la division euclidienne \ref{THOooKDJVooRIJRHP}, il existe \( q\in \eZ\) et \( 0\leq r<p\) tels que \( y=qp+r\). Vu que \( y\) et \( qp\) sont dans \( G\) nous avons \( r=y-qp\in G\). Par minimalité de \( p\), nous devons avoir \( r=0\), et donc \( y=qp\in p\eZ\).
	\end{subproof}
\end{proof}

%---------------------------------------------------------------------------------------------------------------------------
\subsection{Ordre sur \( \eZ\)}
\label{SUBooEntiersRelatifsOrdre}
%---------------------------------------------------------------------------------------------------------------------------

\begin{normaltext}	\label{NORMooNotationInclusionEntiers}
	Si \( z\in \eZ\), nous disons que \( z\in \eN\) lorsque \( z\in \iota(\eN)\). C'est un abus de notation qu'il est difficile de ne pas faire. Le même abus sera fait pour passer de \( \eZ\) à \( \eQ\) puis \( \eR\) puis \( \eC\).
\end{normaltext}

% TODOooKCJMooIHGrZN donner une notion de positivité sur Z, et modifier la définition PROPooMYYDooOABOdB.
% Fusionner 
% - DEFooYMRHooCyEKtr
% - DEFooRelatifsPositifs
% - PROPooMYYDooOABOdB
% - PROPooMYYDooOABOdB
% ceci est déjà dans ma liste

\begin{definition}[Positivité et ordre dans \( \eZ\)\cite{MonCerveau}]	\label{DEFooYMRHooCyEKtr}
	La positivité\footnote{Définition \ref{DEFooZRMFooCtzMov}.} que nous considérons sur \( \eZ\) est \( \eN\setminus\{ 0 \}\). Nous considérons alors la relation d'ordre\footnote{De la proposition \ref{PROPooKLOPooBgQqhM}.} \( \leq\) donnée par \( a\leq b\) si et seulement si \( b-a\in \eN\).

	Nous disons que
	\begin{enumerate}
		\item
		      L'élément \( z\in \eZ\) est \defe{positif}{entier relatif!positif} si \( z\geq 0\).
		\item
		      L'élément \( z\in \eZ\) est \defe{strictement positif}{entier relatif!strictement positif} si \( z\geq 0\) et \( z\neq 0\). Nous notons \( z>0\).
		\item
		      L'élément \( z\in \eZ\) est \defe{négatif}{entier relatif!négatif} si \( z\leq 0\).
		\item
		      L'élément \( z\in \eZ\) est \defe{strictement négatif}{entier relatif!strictement négatif} si \( z\leq 0\) et \( z\neq 0\). Nous notons \( z<0\).
	\end{enumerate}
\end{definition}

\begin{definition}[Éléments positifs de \( \eZ \)]	\label{DEFooRelatifsPositifs}
	On dit qu'un élément \( z \in \eZ \) est \defe{positif}{entiers relatifs!positifs} si \( z \in \eN \). On note dans ce cas \( z \geq 0\).

	On dira qu'il est \defe{négatif}{entiers relatifs!négatifs} quand \( -z \in \eN \). On note ainsi \( z \leq 0 \).

	On dit qu'un entier relatif \( z \) est \defe{strictement positif}{} s'il est positif et non nul; on note \( z > 0 \). Il est \defe{strictement négatif} lorsqu'il est négatif et non nul, et cela est noté \( z < 0 \).
\end{definition}

\begin{propositionDef}[Relation d'ordre \cite{RWWJooJdjxEK}]       \label{PROPooMYYDooOABOdB}
	La paire\footnote{Voir la définition \ref{DEFooYMRHooCyEKtr} pour \( \leq\).} \( (\eZ, \leq)\) est un ensemble totalement ordonné.
\end{propositionDef}

\begin{normaltext}
	%\label{NORMooOrdreEntiersRelatifsExtension}
	% Ce label me semble dangereux parce que si on le cite dans une démonstration entre ici et LEMooKAXFooIPyzJC, le détecteur de référence vers le futur
	% ne préviendra pas du problème.
	Une version de la proposition \ref{PROPooMYYDooOABOdB} dans \( \eR\) sera le lemme \ref{LEMooKAXFooIPyzJC}.
\end{normaltext}

\begin{probleme}	\label{TODOooDefinirMultiplicationRelatifs}
	Définir ici la multiplication dans \( \eZ \) comme extension de celle de \( \eN \). Montrer que ça en fait un anneau totalement ordonné.
\end{probleme}

\begin{proposition}[\cite{MonCerveau}]	\label{PROPooADGIooTioqdu}
	L'anneau \( \eZ\) est totalement ordonné\footnote{Définition \ref{DEFooWACWooDWvXKJ}.}
	%TODOooBRCUooRjkxmG. Prouver ça.
\end{proposition}

\begin{lemma}       \label{LEMooSVDDooWsyxNP}
	Soient \( a>0\) et \( b \geq 1\) (resp. \( b > 1 \) dans \( \eZ\). Nous avons
	\begin{equation}
		ab \geq a \qquad \text{(resp. }ab>a\text{)}.
	\end{equation}
\end{lemma}

\begin{proof}
	On la fait seulement pour l'inégalité stricte: il suffit de recopier pour l'inégalité large. Par récurrence sur \( a \) qui se trouve être dans \( \eN \). Lorsque \( a = 1 \), on lit simplement \( 1b > 1 \) qui est correct.

	Si la propriété est vraie pour un certain \( a \), alors
	\begin{align}
		(a+1) b & = ab + b                          &  & \text{(distributivité)} \\
		        & > a + b\text{(hyp. de réc.)}                                   \\
		        & > a + 1\text{(hyp. sur \( b \)).}
	\end{align}
	Et le principe de récurrence permet de conclure.
\end{proof}

\begin{normaltext}	\label{NORMooLienRelatifsAnnTotOrdonne}
	Cette proposition est en partie similaire à \ref{PROPooELXCooCYzEVD} qui parle d'anneau totalement ordonné.
\end{normaltext}

\begin{proposition}[\cite{MonCerveau}]	\label{PROPooYFUBooJUZgwH}
	Quelques propriétés de l'ordre dans \( \eZ\). Soit \( a, b \in \eZ\).
	\begin{enumerate}
		\item		\label{ITEMooULKWooLmKBse}
		      Si \( a,b\geq 0\), alors \( ab\geq 0\).
		\item	\label{ITEMooSAJPooEDSgXJ}
		      Si \( a,b>0\), alors \( ab>0\).
		\item		\label{ITEMooINGNooOgSGiF}
		      Si \( a,b\leq 0\) alors \( ab\geq 0\).
		\item	\label{ITEMooOGFXooDvLKAE}
		      Si \( a\geq 0\) et \( b\leq 0\) alors \( ab\leq 0\).
		\item	\label{ITEMooRelatifsOrdreOppose}
		      Si \( a>0\), alors \( -a<0\).
		\item	\label{ITEMooRelatifsOrdreCarrePositif}
		      Nous avons \( a^2\geq 0\) et \( a^2=0\) si et seulement si \( a=0\).
		\item		\label{ITEMooBUHIooLJOSLQ}
		      Si \( ab\geq 0\) et si \( a\geq 0\) alors \( b\geq 0\).
		\item		\label{ITEMooGDFPooBbpBgn}
		      Si \( ab>0\) alors soit \( a,b>0\), soit \( a,b<0\).
	\end{enumerate}
	%TODOooEDWPooRDujTs. Prouver ça. J'en mets deux parce que je parie que ce sera fait.
\end{proposition}

\begin{lemma}       \label{LEMooMYEIooNFwNVI}
	Toute partie bornée de \( \eZ\) possède un plus grand élément.
	%TODOooEDWPooRDujTs. Prouver ça. J'en mets deux parce que je parie que ce sera fait.
\end{lemma}

\begin{proposition}	\label{PROPooRelatifsValAbs}
	Si \( z \in \eZ \), alors :
	\begin{enumerate}
		\item
		      \( |z| \) est bien définie et \( |z| \in \eN \);
		\item
		      nous avons \( | ab |=| a || b | \).
	\end{enumerate}
\end{proposition}

\begin{proof}
	C'est le fait que \( \eZ\) est un anneau totalement ordonné\footnote{Proposition \ref{PROPooADGIooTioqdu}.}: donc la proposition-définition \ref{DEFooJXKVooErANPh} s'applique, y compris l'équation \eqref{EQooORNAooBYFeFk}.
\end{proof}


\begin{definition}	\label{DEFooRelatifsDiviseurParite}
	Soit \( a, b \in \eZ \). On étend les notions de divisibilité et de parité de la définition \ref{DEFooDivisibleNombrePairImpair} en disant que:
	\begin{enumerate}
		\item
		      \( a \) est divisible par \( b \) si \( |a| \) est divisible par \( |b| \);
		\item
		      \( a \) est pair si \( |a| \) est pair;
		\item
		      \( b \) est impair si \( |b| \) est impair.
	\end{enumerate}
\end{definition}

\begin{proposition} \label{PROPooYJBMooZrzkNX}
	Soit \( a,b\in \eZ\) non nuls, tels que \( a\) soit divisible par \( b\). Alors \( | a |\geq | b |\).
\end{proposition}

\begin{proof}
	\( a\) est divisible \( b\) dans \( \eZ \) revient à dire que \( |a| \) est divisible \( |b| \) dans \( \eN \), donc par la définition \ref{DEFooDivisibleNombrePairImpair}, il existe \( q \in \eN \) tel que \( |a| = q\, |b| \). Comme \( a \) et \( b \) sont non nuls, \( |a| \) et \( |b| \) sont aussi non nuls, et donc \( q \neq 0 \): en effet, sinon, nous aurions \( q\, |b| = 0 = |a| \) à cause de la proposition \ref{PROPooGHDOooFYRmon} point \ref{ITEMooNBYKooXnGRrf}. Ainsi, \( q \geq 1 \), et grâce au lemme \ref{LEMooSVDDooWsyxNP}, on a \(|a| = q\, |b| \geq |b| \).
\end{proof}

\begin{lemma}[\cite{MonCerveau}]	\label{LEMooCMOEooSoeclk}
	Soit \( n\in \eZ\).
	\begin{enumerate}
		\item
		      Si \( n\) est impair\footnote{Définition \ref{DEFooRelatifsDiviseurParite}.}, alors \( n-1\) et \( n+1\) sont pairs.
		\item
		      Si \( n\) est pair, alors \( n-1\) et \( n+1\) sont impairs.
	\end{enumerate}
\end{lemma}

\begin{proof}
	Selon que \( n \geq 0 \) ou que \( n \leq 0 \), on effectue la division euclidienne de \( n \) ou de \( -n \) par 2.
	\begin{subproof}
		\item[Cas \( n \geq 0 \)]: il existe \( q \in \eN \) et \( r \in \{ 0, 1\} \) tel que \( n = 2q+r \).
		\begin{itemize}
			\item
			      Quand \( n \) est pair, on a \( r = 0 \). Alors \( n+1 = 2q + 1 \) et \( n - 1 = 2(q-1) + 2 - 1 =  2(q-1) + 1 \) sont impairs.
			\item
			      Quand \( n \) est impair, on a \( r = 1 \). Alors \( n+1 = 2q + 2 = 2(q+1) \) et \( n - 1 = 2q + 1 - 1 =  2q \) sont pairs.
		\end{itemize}
		\item[Cas \( n \leq 0 \)]: il existe \( q \in \eN \) et \( r \in \{ 0, 1\} \) tel que \( -n = 2q+r \).
		\begin{itemize}
			\item
			      Quand \( n \) est pair, on a \( r = 0 \). Alors \( -n+1 = 2q + 1 \) est impair, et \( n-1 = -(-n+1) \) l'est aussi. De même, \( -n - 1 = 2(q-1) + 2 - 1 =  2(q-1) + 1 \), et donc \( n+1 = -(-n-1) \), sont impairs.
			\item
			      Quand \( n \) est impair, on a \( r = 1 \). Alors \( -n+1 = 2q + 2 = 2(q+1) \) est pair, et \( n-1 = -(-n+1) \) l'est aussi. De même, \( -n - 1 = 2q + 1 - 1 =  2q \), et donc \( n+1 = -(-n-1) \), sont pairs.
		\end{itemize}
	\end{subproof}
\end{proof}

\begin{lemma}       \label{LEMooJNXIooBmdOVi}
	L'ensemble \( \eZ\) est infini dénombrable.
\end{lemma}

\begin{proof}
	Nous considérons l'application
	\begin{equation}
		\begin{aligned}
			f\colon \eN & \to \eZ                                         \\
			n           & \mapsto \begin{cases}
				                      n/2      & \text{si \( n\) est pair}    \\
				                      -(n+1)/2 & \text{si \( n\) est impair,}
			                      \end{cases}
		\end{aligned}
	\end{equation}
	et nous montrons qu'elle est bijective.
	\begin{subproof}
		\spitem[\( f\) est injective]
		%-----------------------------------------------------------
		Soient \( x,y\in \eZ\) tels que \( f(x)=f(y)\). Il y a trois possibilités : \( f(x)<0\), \( f(x)=0\) et \( f(x)>0\).
		\begin{subproof}
			\spitem[Si \( f(x)=0\)]
			%-----------------------------------------------------------
			Nous supposons que \( f(x)=0\). Si \( x \) est pair, c'est que \( x/2=0\) et donc que \( x=0\). Si \( x\) est impair, c'est que \( x+1=0\) et donc que \( x=-1\) qui n'est pas possible parce que \( x\in \eN\). Bref, la seule possibilité pour \( f(x)=0\) est \( x=0\).

			\spitem[Si \( f(x)<0\)]
			%-----------------------------------------------------------
			Alors \( x\) est pair et \( x=2f(x)\). Même raisonnement pour \( y\) : \( y=2f(y)=2f(x)=x\).

			\spitem[Si \( f(x)<0\)]
			%-----------------------------------------------------------
			Alors \( x\) est impair et
			\begin{equation}
				f(x)=-\frac{ x+1 }{ 2 }=f(y)=-\frac{ y+1 }{ 2 },
			\end{equation}
			et donc \( x=y\).
		\end{subproof}

		\spitem[\( f\) est surjective]
		%-----------------------------------------------------------
		Soit \( n\in \eZ\). Si \( n>0\), alors \( n=f(2n)\). Si \( n<0\) alors \( n=f(-2n-1)\). Notez que \( -2n-1>0\), donc c'est valide.
	\end{subproof}
\end{proof}

\begin{lemma}[\cite{MonCerveau}]	\label{LEMooILYLooTDRtYj}
	Soit un anneau\footnote{Définition \ref{DefHXJUooKoovob}.} \( A\). Si une application \(f \colon \eZ\to A  \) vérifie
	\begin{enumerate}
		\item
		      \( f(x+y)=f(x)+f(y)\) pour tout \( x,y\in \eZ\),
		\item
		      \( f(1)=1\),
	\end{enumerate}
	alors
	\begin{enumerate}
		\item		\label{ITEMooUQKWooByNDTK}
		      \( f(0)=0\)
		\item	\label{ITEMooYOEPooKOQQPH}
		      \( f(-x)=-f(x)\)
		\item	\label{ITEMooKTSDooExAJHd}
		      \( f\) est un morphisme d'anneaux, c'est à dire que
		      \begin{equation}
			      f(xy)=f(x)f(y).
		      \end{equation}
	\end{enumerate}
\end{lemma}

\begin{proof}
	En deux parties.
	\begin{subproof}
		\spitem[Pour \ref{ITEMooUQKWooByNDTK}]
		%-----------------------------------------------------------
		Prenons un \( x\) quelconque. Nous avons \( f(x)= f(x+0)=f(x)+f(0)\). Donc \( f(0)=0\).

		\spitem[Pour \ref{ITEMooYOEPooKOQQPH}]
		%-----------------------------------------------------------
		L'inverse pour la somme de \( f(-x)\) est \( -f(x)\). En effet
		\begin{equation}
			f(-x)+f(x)=f(-x+x)=f(0)=0.
		\end{equation}
		\spitem[Pour \ref{ITEMooKTSDooExAJHd}, si \( y>0\)]
		%-----------------------------------------------------------
		Nous prouvons que \( f(xy)=f(x)f(y)\) par récurrence sur \( y\). D'abord pour \( y=1\) nous avons bien \( f(xy)=f(x)f(y)\) parce que \( f(1)=1\). Ensuite nous avons :
		\begin{subequations}
			\begin{align}
				f\big( x(y+1) \big) & =f(xy+x)                                         \\
				                    & =f(xy)+f(x)                & \text{morph pour +} \\
				                    & =f(x)f(y)+f(x)             & \text{hyp. rec.}    \\
				                    & =f(x)\big( f(y)+1 \big)                          \\
				                    & =f(x)\big( f(y)+f(1) \big)                       \\
				                    & =f(x)f(y+1)
			\end{align}
		\end{subequations}
		\spitem[Si \( y<0\)]
		%-----------------------------------------------------------
		Nous posons \( y'=-y\). Vu que \( y'>0\) nous avons
		\begin{equation}
			f(xy)=f(-xy')=-f(xy')=-f(x)f(y')=f(x)(-1)f(y')=f(x)f(-y')=f(x)f(y).
		\end{equation}
	\end{subproof}
\end{proof}
