% This is part of Mes notes de mathématique
% Copyright (c) 2011-2021
%   Laurent Claessens
% See the file fdl-1.3.txt for copying conditions.

%+++++++++++++++++++++++++++++++++++++++++++++++++++++++++++++++++++++++++++++++++++++++++++++++++++++++++++++++++++++++++++
\section{Quelques éléments sur les ensembles}
%+++++++++++++++++++++++++++++++++++++++++++++++++++++++++++++++++++++++++++++++++++++++++++++++++++++++++++++++++++++++++++

%---------------------------------------------------------------------------------------------------------------------------
\subsection{Petit mot d'introduction}
%---------------------------------------------------------------------------------------------------------------------------

\begin{normaltext}

Le Frido n'est pas supposé être lu dans l'ordre de la première à la dernière page; les matières y sont présentées dans l'ordre logique mathématique, et non dans l'ordre logique pédagogique, et encore moins par ordre de difficulté croissante.

En mathématique, si on lit une démonstration et que l'on veut vraiment tout justifier, et justifier toutes les étapes de tous les résultats utilisés, on tombe forcément un jour sur les axiomes.

Or l'axiomatique est un sujet particulièrement difficile. Nous n'allons donc pas «tout justifier» jusque là. Nous n'allons même pas préciser quel système d'axiome est utilisé. En particulier nous n'allons pas donner l'axiomatique des ensembles : nous allons supposer connus les ensembles et leurs principales propriétés.

Bref. Nous supposons avoir une théorie des ensembles qui tient la route. En particulier nous supposons connues les notions suivantes :
\begin{enumerate}
    \item
        ensemble vide,
    \item
        ensemble, appartenance, intersection, union,
    \item
        application entre deux ensembles, notation \( f(x)\) pour désigner l'image de \( x\) par \( f\),
    \item
        produit cartésien de plusieurs ensembles.
\end{enumerate}
Ce sont toutes des choses dont la construction à partir des axiomes n'est en aucun cas évidente. En particulier, des «définitions» comme «l'intersection de deux ensembles est l'ensemble contenant exactement les éléments communs aux deux ensembles» ne sont pas correctes parce qu'elles passent à côté de l'existence et de l'unicité d'un tel ensemble.
\end{normaltext}


\begin{lemma}[Quelque relations ensemblistes]       \label{LEMooHRKAooRskzQL}
    Soient \( A,B\subset X\). Nous avons
    \begin{equation}
        X\setminus(A\cap B)=(X\setminus A)\cup(X\setminus B).
    \end{equation}
\end{lemma}

\begin{definition}\label{DefEnsemblesDisjoints}
    Deux ensembles $A$ et $B$ sont \defe{disjoints}{ensembles!disjoints} si leur intersection est vide\footnote{Remarquez que les mots «intersection» et «vide» sont de ceux que nous avons décidé de ne pas définir.}; en d'autres termes, s'il n'existe aucun élément commun à $A$ et $B$.
\end{definition}

\begin{normaltext}
    Remarquez par exemple que la première phrase de l'article de Wikipédia sur la construction de \( \eN\) est «Partant de la théorie des ensembles, on identifie 0 à l'ensemble vide, puis on construit \ldots». Il est bien précisé que l'on part d'une théorie des ensembles.
\end{normaltext}

\begin{normaltext}
    La suite de ce chapitre sera essentiellement sans exemples parce qu'avant d'avoir construit les ensembles de nombres, je ne sais pas très bien quels exemples on peut donner de quoi que ce soit.
\end{normaltext}


%--------------------------------------------------------------------------------------------------------------------------- 
\subsection{Injection, surjection, bijection}
%---------------------------------------------------------------------------------------------------------------------------

\begin{definition}
    Soient deux ensembles \( E\) et \( F\). Une application \( f\colon E\to F\) est
    \begin{enumerate}
        \item
            \defe{surjective}{surjection} si pour tout \( y\in F\), il existe \( x\in E\) tel que \( y=f(x)\);
        \item
            \defe{injective}{injection} si pour tout \( y\in F\), il existe au plus un \(x\in E \) tel que \( y=f(x)\);
        \item
            \defe{bijective}{bijection} si elle est à la fois injective et surjective.
    \end{enumerate}
\end{definition}
La méthode la plus courante pour démontrer qu'une application \( f\colon E\to F\) est injective est de considérer \( x,y\in E\) tels que \( f(x)=f(y)\), et de prouver à partir de là que \( x=y\). Ou alors de supposer \( x\neq y\) et obtenir une contradiction.

La technique de la contradiction est évidemment la plus courante lorsque l'égalité \( f(x)=g(x)\) implique une équation faisant intervenir \( 1/(x-y)\).

\begin{lemma}       \label{LEMooWBYSooFqyqQP}
    Soient deux ensembles \( A\) et \( B\) ainsi qu'une application \( f\colon A\to B\). Nous supposons qu'il existe une application \( g\colon B\to A\) telle que \( f\circ g=\id_B\) et \( g\circ f=\id_A\).

    Alors \( f\) est une bijection.
\end{lemma}

\begin{proof}
    En deux parties.
    \begin{subproof}
    \item[Injection]
        Supposons que \( f(a)=f(b)\). Alors en appliquant \( g\) est deux côtés, et en utilisant le fait que \( g\circ f=\id_A\), nous trouvons \( a=b\).
    \item[Surjection]
        Soit \( x\in B\). Posons \( a=g(x)\). Alors, en utilisant le fait que \( f\circ g=\id_B\) nous avons
        \begin{equation}
            f(a)=(f\circ g)(x)=x.
        \end{equation}
        Donc \( x\) est dans l'image de \( f\) et \( f\) est surjective.
    \end{subproof}
\end{proof}

%---------------------------------------------------------------------------------------------------------------------------
\subsection{Ensemble ordonné}
%---------------------------------------------------------------------------------------------------------------------------

\begin{normaltext}\label{NORooLMBYooYjUoju}
L'\defe{axiome du choix}{axiome!du choix} que nous acceptons peut s'énoncer comme ceci\cite{ooKLIXooHbpufL} : Étant donné un ensemble X d'ensembles non vides, il existe une fonction définie sur X, appelée fonction de choix, qui à chacun d'entre eux associe un de ses éléments.
\end{normaltext}

\begin{definition}      \label{DefooFLYOooRaGYRk}
    Une \defe{relation d'ordre}{ordre} sur un ensemble \( E\) est une relation binaire (notée \( \leq\)) sur \( E\) telle que pour tous \( x,y,z\in E\),
    \begin{description}
        \item[réflexivité] : \( x\leq x\)
         \item[antisymétrie] : \( x\leq y\) et \( y\leq x\) implique \( x=y\)
         \item[transitivité] : \( x\leq y\) et \( y\leq z\) implique \( x\leq z\).
    \end{description}
\end{definition}

\begin{definition}      \label{DEFooVGYQooUhUZGr}
    Un ensemble ordonné est \defe{totalement ordonné}{ordre!total} si deux éléments sont toujours comparables : si \( x,y\in E\) alors nous avons soit \( x\leq y\) soit \( y\leq x\). Si les éléments ne sont pas tous comparables, nous disons que l'ordre est \defe{partiel}{ordre!partiel}.
\end{definition}




\begin{definition}
    Soit un ensemble ordonné \( (E,\leq)\) et une partie \( A\) de \( E\). Nous disons que \( m\in A\) est un \defe{minimum}{minimum!ensemble ordonné} de \( A\) si pour tout \( x\in A\), l'élément \( m\) est comparable à \( x\) et \( m\leq x\).

    Un élément \( p\in E\) est un \defe{minorant}{minorant} de \( A\) si pour tout \( a\in A\), l'élément \( p\) et \( a\) sont comparables et \( p\leq a\).

    Les notions de \defe{maximum}{maximum} et de \defe{majorant}{majorant} sont définies de façon analogue.
\end{definition}

Lorsqu'une partie possède un minimum, ce dernier est nommé le «plus petit élément» de la partie. Attention : il n'en existe pas toujours. D'innombrables exemples pourront être vus lorsque nous aurons construits \( \eQ\) et \( \eR\). Typiquement les intervalles du type \( \mathopen] a , b \mathclose[\).

\begin{definition}   \label{DEFooLJEAooBLGsiS}
    Un ensemble ordonné est \defe{bien ordonné}{bon!ordre}\index{ordre!bon ordre} si toute partie non vide possède un plus petit élément.
\end{definition}

Autrement dit, l'ensemble ordonné \( E\) est bien ordonné si pour toute partie non vide \( A\), il existe \( x\in A\) tel que \( x\leq y\) pour tout \( y\in A\).

\begin{normaltext}
    Quelques remarques.
    \begin{enumerate}
        \item
            L'inégalité stricte (définie par: \( x<y\) si et seulement si \( x\leq y\) et \( x\neq y\)) n'est pas une relation d'ordre parce qu'elle n'est pas réflexive.
        \item
            Nous verrons dans la remarque~\ref{REMooXOIOooHjwMvA} que l'intervalle \( \mathopen[ -1 , 1 \mathclose]\) dans \( \eR\) n'est pas bien ordonné.
        \item
            Un ensemble bien ordonné est forcément totalement ordonné parce que toutes les parties de la forme \( \{ x,y \}\) possèdent un minimum. Par conséquent \( x\) et \( y\) doivent être comparables : \( x\leq y\) ou \( y\leq x\).
    \end{enumerate}
\end{normaltext}

\begin{example}
    Si \( E\) est un ensemble, l'inclusion est un ordre sur l'ensemble des parties de \( E\), mais pas un ordre total parce que si \( X,Y\) sont des parties de \( E\), alors nous n'avons pas automatiquement soit \( X\subset Y\) ou \( Y\subset X\).
\end{example}

La notion d'ordre permet d'introduire la notion d'intervalle.

\begin{definition}  \label{DefEYAooMYYTz}
    Soit un ensemble totalement ordonné \( (E,\leq)\). Un \defe{intervalle}{intervalle} de \( E\) est une partie \( I\) telle que tout élément compris entre deux éléments de \( I \) soit dans \( I \). En formule, la partie \( I \) de \( E\) est un intervalle si
    \[
      \forall a,b\in I,(a\leq x\leq b)\Rightarrow x\in I.
    \]
\end{definition}

%---------------------------------------------------------------------------------------------------------------------------
\subsection{Lemme de Zorn}
%---------------------------------------------------------------------------------------------------------------------------

Nous admettons l'axiome du choix\cite{BIBooBZJMooJDJRgg} qui s'énonce de la façon suivante\cite{BIBooTQMWooFFAQQZ} :
\begin{quote}
    Pour tout ensemble $X$ d'ensembles non vides, il existe une fonction définie sur $X$, appelée fonction de choix, qui à chaque ensemble $A$ appartenant à $X$ associe un élément de cet ensemble $A$.
\end{quote}

\begin{definition}[Ensemble inductif\cite{MathAgreg}]  \label{DefGHDfyyz}
    Un ensemble est \defe{inductif}{inductif} si tout sous-ensemble totalement ordonné admet un majorant.
\end{definition}


\begin{lemma}[Lemme de Zorn\cite{BIBooYDIJooWCVynX}]    \label{LemUEGjJBc}
    Tout ensemble ordonné inductif non vide admet au moins un élément maximal.
\end{lemma}
\index{lemme!de Zorn}


%---------------------------------------------------------------------------------------------------------------------------
\subsection{Complémentaire}
%---------------------------------------------------------------------------------------------------------------------------
\label{AppComplement}

\begin{definition}
    Soit $E$, un ensemble et $A$, une partie de $E$ (c'est-à-dire un sous-ensemble de $E$). Le \defe{complémentaire}{complémentaire} de l'ensemble $A$, dans $E$, noté $\complement A$\nomenclature[T]{$\complement A$}{Le complémentaire de l'ensemble $A$} est l'ensemble des éléments de $E$ qui ne font pas partie de $A$ :
    \begin{equation}
	    \complement A=E\setminus A=\{ x\in E\tq x\notin A \}.
    \end{equation}
\end{definition}

Nous allons aussi régulièrement noter le complémentaire de \( A\) par \( A^c\)\nomenclature[T]{\( A^c\)}{complémentaire de \( A\)}.

\begin{lemma}		\label{LemPropsComplement}
	Quelques propriétés à propos des complémentaires. Si $E$ est un ensemble et si $A$ et $B$ sont des sous-ensembles de $E$, nous avons
	\begin{enumerate}
		\item
			$\complement \complement A =A $, en d'autres termes, $E\setminus(E\setminus A)=A$,
		\item
			$\complement(A\cap B)=\complement A\cup\complement B$,
		\item
			$\complement(A\cup B)=\complement A\cap\complement B$,
		\item	\label{ItemLemPropComplementiii}
			$A\setminus B=A\cap\complement B$.
        \item       \label{ITEMooNHDUooWtURqQ}
            \( (A\setminus B)^c=A^c\cup B\).
        \item       \label{ITEMooTBWKooTChOmC}
            \( A^c\setminus B^c=B\setminus A\).
	\end{enumerate}
\end{lemma}

\begin{proof}
    Plusieurs points.
    \begin{subproof}
    \item[Pour \ref{ItemLemPropComplementiii}]
    \item[Pour \ref{ITEMooNHDUooWtURqQ}]
        Il faut le faire en deux inclusions.
        \begin{subproof}
        \item[\( (A\setminus B)^c\subset A^c\cup B\)]
            Supposons que \( x\in(A\setminus B)^c\). Si \( x\in A^c\) on est bon. Sinon, si \( x\in A\) alors nous devons montrer que \( x\in B\). Si \( x\) n'est pas dans \( B\), alors \( x\in A\setminus B\), et n'est donc pas dans \( (A\setminus B)^c\). Donc oui, \( x\in B\).
        \item[\( A^c\cup B\subset (A\setminus B)^c\)]
            Supposons d'abord que \( x\in A^c\). Vu que \( A\setminus B\subset A\), si \( x\in A^c\), alors \( x\in (A\setminus B)^c\).

            Si \( x\in B\), alors \( x\) n'est pas dans \( A\setminus B\) et donc \( x\) est dans \( (A\setminus B)^c\).
        \end{subproof}
    \item[Pour \ref{ITEMooTBWKooTChOmC}]
        Pour cette égalité, nous faisons les cas suivant que \( x\) est dans \( A\) ou \( B\) ou non. Bref, nous écrivons la table de vérité :
\begin{equation}       
    \begin{array}{|c|c|c|c|c|}
        \hline%
        A   &   1   & 1 & 0 & 0\\
        \hline%
        B&1&0&1&0\\
        \hline%
        A^c\setminus B^c & 0 & 0 &1&0\\
        \hline%
        B\setminus A & 0 & 0 &1&0\\
    \end{array}
\end{equation}
Les deux dernières lignes étant égales, nous avons l'égalité d'ensembles annoncée.
    \end{subproof}
\end{proof}

\begin{definition}[différence symétrique]    \label{DefBMLooVjlSG}
    Si \( A\) et \( B\) sont des ensembles, l'ensemble \( A\Delta B\)\nomenclature[T]{\( A\Delta B\)}{différence symétrique} est la \defe{différence symétrique}{ensemble!différence symétrique} d'ensembles :
    \begin{equation}
        A\Delta B=(A\cup B)\setminus(A\cap B).
    \end{equation}
\end{definition}
C'est l'ensemble des éléments étant soit dans \( A\) soit dans \( B\) mais pas dans les deux, ni dans aucun des deux. La table de vérité de \( A\Delta B\) est intéressante :
\begin{equation}        \label{EQooOJBOooKkKbYp}
    \begin{array}{|c|c|c|c|c|}
        \hline%
        A   &   1   & 1 & 0 & 0\\
        \hline%
        B&1&0&1&0\\
        \hline%
      A\Delta B&0&1&1&0  
    \end{array}
\end{equation}
La deuxième colonne signifie que si \( x\in A\) et \( x\in B^c\), alors \( x\in A\Delta B\).

\begin{lemma}[\cite{BIBooRFFSooROjnXs}]   \label{LemCUVoohKpWB}
    Si \( A\) et \( B\) sont des parties d'un ensemble, nous avons
    \begin{enumerate}
        \item\label{ItemVUCooHAztC}
            \( A^c\Delta B^c=A\Delta B\).
        \item\label{ItemVUCooHAztCii}
            \( (A\Delta B)\Delta B=A\).
        \item       \label{ITEMooSPZXooPTgisP}
            \( (A\Delta B)^c=(A^c\cap B^c)\cup(A\cap B)\).
        \item       \label{ITEMooSMXWooYcWsRC}
            Associativité : \( A\Delta (B\Delta C)=(A\Delta B)\Delta C\).
    \end{enumerate}
\end{lemma}

\begin{proof}

    Nous rappelons l'égalité \( X^c\setminus Y^c=Y\setminus X\) du lemme \ref{LemPropsComplement}\ref{ITEMooTBWKooTChOmC}. De là nous prouvons.
    \begin{subproof}
    \item[Pour \ref{ItemVUCooHAztC}]
    De là nous avons la première assertion :
    \begin{equation}
        A^c\Delta B^c=(A^c\cup B^c)\setminus(A^c\cap B^c)=(A\cap B)^c\setminus(A\cup B)^c=(A\cup B)\setminus (A\cap B)=A\Delta B.
    \end{equation}
\item[Pour \ref{ItemVUCooHAztCii}]
    Pour la seconde assertion, il faut remarquer que \( (A\Delta B)\cup B=A\cup B\) et que \( (A\Delta B)\cap B=B\setminus A\), donc
    \begin{equation}
        (A\Delta B)\Delta B=(A\cup B)\setminus (B\setminus A)=A.
    \end{equation}

\item[Pour \ref{ITEMooSPZXooPTgisP}]
    Il s'agit d'utiliser le lemme \ref{LemPropsComplement}\ref{ITEMooNHDUooWtURqQ} :
    \begin{subequations}
        \begin{align}
            (A\Delta B)^c&=\Big( (A\cup B)\setminus (A\cap B) \Big)^c\\
            &=(A\cup B)^c\cup(A\cap B)\\
            &=(A^c\cap B^c)\cup(A\cap B).
        \end{align}
    \end{subequations}
\item[Pour l'associativité \ref{ITEMooSMXWooYcWsRC}]
           Nous écrivons les tables de vérités selon que \( x\) est dans \( A\), \( B\), \( C\) ou non. D'abord
           \begin{equation}
               \begin{array}{|c|c|c|c|c|c|c|c|c|}
                   A&1&1&1&1&0&0&0&0\\
                   B&1&1&0&0&1&1&0&0\\
                   C&1&0&1&0&1&0&1&0\\
                   \hline%
           B\Delta C&0&1&1&0&0&1&1&0\\
           \hline%
A\Delta (B\Delta C) &1&0&0&1&0&1&1&0     
               \end{array}
           \end{equation}
           La quatrième ligne s'écrit sur le modèle de \eqref{EQooOJBOooKkKbYp} en regardant les deuxièmes et troisièmes lignes. La dernière ligne se fait avec la première et la quatrième.

           L'autre table de vérité se fait de la même manière :
           \begin{equation}
               \begin{array}{|c|c|c|c|c|c|c|c|c|}
                   A&1&1&1&1&0&0&0&0\\
                   B&1&1&0&0&1&1&0&0\\
                   C&1&0&1&0&1&0&1&0\\
                   \hline%
           A\Delta B&0&0&1&1&1&1&0&0\\
           \hline%
(A\Delta B)\Delta C &1&0&0&1&0&1&1&0     
               \end{array}
           \end{equation}
           Vu que les lignes pour \( A\Delta (B\Delta C)\) et pour \( (A\Delta B)\Delta C\) sont identiques, nous avons égalité.
    \end{subproof}
\end{proof}

%---------------------------------------------------------------------------------------------------------------------------
\subsection{Relations d'équivalence}
%---------------------------------------------------------------------------------------------------------------------------
\label{appEquivalence}

\begin{definition}  \label{DefHoJzMp}
Si $E$ est un ensemble, une \defe{relation d'équivalence}{relation d'équivalence} sur $E$ est une relation $\sim$ qui est à la fois
\begin{description}
	\item[réflexive] $x\sim x$ pour tout $x\in E$,
	\item[symétrique] $x\sim y$ si et seulement si $y\sim x$;
	\item[transitive] si $x\sim y$ et $y\sim z$, alors $x\sim z$.
\end{description}
\end{definition}

\begin{definition}      \label{DEFooRHPSooHKBZXl}
    Si \( E\) est un ensemble et si \( \sim\) est une relation d'équivalence sur \( E\), alors nous notons \( E/\sim\) l'\defe{ensemble quotient}{ensemble quotient}, c'est-à-dire l'ensemble des classes d'équivalence dans \( E\). Un élément de \( E/\sim\) est de la forme
    \begin{equation}
        [a]=\{ x\in E\tq x\sim a \}.
    \end{equation}
\end{definition}

\begin{lemma}
    Soit un ensemble \( E\) et une relation d'équivalence \( \sim\). Pour \( a,b\in E\), nous avons \( [a]=[b]\) si et seulement si \( a\sim b\).
\end{lemma}

\begin{proof}
    En deux parties.
    \begin{subproof}
        \item[\( \Rightarrow\)]
            Nous supposons que \( [a]=[b]\). Vu que \( a\sim a\), nous avons \( a\in [a]=[b]\). Mais \( a\in [b]\) signifie \( a\sim b\), ce qu'il fallait.
        \item[\( \Leftarrow\)]
            Nous supposons que \( a\sim b\), et nous démontrons que \( [a]\subset [b]\) (pour l'inclusion inverse, vous devriez vous en sortir par tout seul). Si \( x\in [a]\), alors \( x\sim a\). Mais \( a\sim b\). Donc \( x\sim a\sim b\), ce qui implique \( x\sim b\) par transitivité. Or dire \( x\sim b\) implique \( x\in [b]\).
    \end{subproof}
\end{proof}

\begin{example}
    Sur l'ensemble de tous les polygones du plan, la relation «a le même nombre de côtés» est une relation d'équivalence. Plus précisément, si $P$ et $Q$ sont deux polygones, nous disons que $P\sim Q$ si et seulement si $P$ et $Q$ ont le même nombre de côtés. Cela est une relation d'équivalence :
    \begin{itemize}
        \item
            un polygone $P$ a toujours le même nombre de côtés que lui-même : $P\sim P$;
        \item
            si $P$ a le même nombre de côtés que $Q$ ($P\sim Q$), alors $Q$ a le même nombre de côtés que $P$ ($Q\sim P$);
        \item
            si $P$ a le même nombre de côtés que $Q$ ($P\sim Q$) et que $Q$ a le même nombre de côtés que $R$ ($Q\sim R$), alors $P$ a le même nombre de côtés que $R$ ($P\sim R$).
    \end{itemize}
\end{example}

\begin{example}
Soit \( f\) une application entre deux ensembles \( E\) et \( F\). Nous définissons une relation d'équivalence sur \( E\) par
\begin{equation}
    x\sim y\Leftrightarrow f(x)=f(y).
\end{equation}
Nous notons par \( \pi\colon E\to E/\sim\) la projection canonique. L'application
\begin{equation}
    \begin{aligned}
        g\colon E/\sim&\to F \\
        [x]&\mapsto f(x)
    \end{aligned}
\end{equation}
est bien définie et injective. Elle n'est pas surjective tant que \( f\) ne l'est pas. La \defe{décomposition canonique}{canonique!décomposition}\index{décomposition!canonique} de \( f\) est
\begin{equation}
    f=g\circ\pi.
\end{equation}
\end{example}

%+++++++++++++++++++++++++++++++++++++++++++++++++++++++++++++++++++++++++++++++++++++++++++++++++++++++++++++++++++++++++++
\section{Les naturels}
%+++++++++++++++++++++++++++++++++++++++++++++++++++++++++++++++++++++++++++++++++++++++++++++++++++++++++++++++++++++++++++
\label{SECooPJSYooNYaIaq}
% Lorsque ce chapitre est fait, changer la phrase qui le référentie dans la partie sur la constante de Weiner.


\begin{definition}[\cite{RWWJooJdjxEK}]     \label{DEFooBJBOooWlblAx}
    Un \defe{triplet naturel}{triplet naturel} est une triple \( (\mN, o, s)\) où \( \mN\) est un ensemble, \( o\) est un élément de \( \mN\) et \( s\) est une application \( s\colon \mN\to \mN\) satisfaisant les propriétés suivantes :
    \begin{enumerate}
        \item
            \( s\) est injective,
        \item
            \( s(\mN)=\mN\setminus \{ o \} \)
        \item       \label{ITEMooXPYEooFajywh}
            Si \( A\subset \mN\) est tel que \( o\in A\) et \( s(A)\subset A\), alors \( A=\mN\).
    \end{enumerate}
\end{definition}

Le théorème suivant est typiquement de ceux qui vont demander de gratter la théorie axiomatique des ensembles avec une certaine précision\quext{Ou alors il y a quelque chose qui m'échappe. Si vous connaissez une construction «simple», faits-me le savoir.}.
\begin{theorem}     \label{THOooOXMHooXYgMqb}
    Il existe des triples naturels.
\end{theorem}

\begin{proposition}[Récurrence\cite{RWWJooJdjxEK}]
    Soit un triplet naturel \( (\mN,o,s)\) et une application \( P\colon \mN\to \{ 0,1 \}\) vérifiant
    \begin{enumerate}
        \item
            \( P(o)=1\),
        \item
            pour tout \( a\in \mN\), si \( P(a)=1\), alors \( P\big( s(a) \big)=1\).
    \end{enumerate}
    Alors \( P(x)=1\) pour tout \( x\in \mN\).
\end{proposition}

\begin{proof}
    Nous posons 
    \begin{equation}
        A=\{ x\in\mN\tq P(x)=1 \}.
    \end{equation}
    Cet ensemble vérifie la propriété \ref{DEFooBJBOooWlblAx}\ref{ITEMooXPYEooFajywh}. Donc \( A=\mN\).
\end{proof}

\begin{probleme}
    Il me semble très douteux que la partie «existence» du théorème suivant puisse être faite sans être au raz d'utiliser les axiomes de la théorie des ensembles.

    Cela fait donc partie des résultats que je n'ai pas l'intention de démontrer dans le Frido, tout comme le théorème \ref{THOooOXMHooXYgMqb}.

    Mais si vous connaissez une façon de le prouver, écrivez moi.
\end{probleme}

\begin{theorem}     \label{THOooFUXMooJuigHK}
    Soient des triplets naturels \( (\mN_1, o_1,s_1)\) et \( (\mN_2,o_2,s_2)\). Alors
    \begin{enumerate}
        \item
            il existe une unique application \( f\colon \mN_1\to \mN_2\) telle que
            \begin{enumerate}
                \item
                    \( f(o_1)=o_2\)
                \item
                    \( f\circ s_1=s_2\circ f\).
            \end{enumerate}
        \item
            Une telle application est une bijection.
    \end{enumerate}
\end{theorem}

\begin{proof}
    En plusieurs points.
    \begin{subproof}
    \item[Existence]
        À mon avis, cette partie est trop compliquée.
    \item[Unicité]
        Soit \( g\), une autre application vérifiant les mêmes conditions. Nous posons
        \begin{equation}
            A=\{ x\in \mN_1\tq f(x)=g(x) \}.
        \end{equation}
        Nous avons \( f(o_1)=o_2=g(o_1)\). Donc \( o_1\in A\). Supposons que \( P(a)=1\), c'est à dire que \( g(a)=f(a)\). Prenons \( s_2\) des deux côtés :
        \begin{equation}
            (s_2\circ g)(a)=(s_2\circ f)(a),
        \end{equation}
        ce qui donne \( (g\circ s_1)(a)=(f\circ s_1)(a)\). Donc \( s_1(a)\in A\).

        Nous en déduisons que \( A=\mN_1\).
    \item[Bijection, définir l'inverse]
        Nous allons trouver un inverse et le lemme \ref{LEMooWBYSooFqyqQP} nous dit que c'est suffisant. La partie «existence», en inversant les rôles de \( \mN_1\) et \( \mN_2\) nous donne une application \( g\colon \mN_2\to \mN_1\) telle que
        \begin{enumerate}
            \item
                \( g(o_2)=o_1\)
            \item
                \( g\circ s_2=s_1\circ g\).
        \end{enumerate}
        Nous allons prouver que \( g\) est un inverse de \( f\).
    \item[\( f\circ g=\id\)]
        Nous posons \( A=\{ x\in \mN_2\tq (f\circ g)(x)=x \}\). Nous avons
        \begin{equation}
            f\big( g(o_2) \big)=f(o_1)=o_2,
        \end{equation}
        et donc \( o_2\in A\).

        Supposons que \( x\in A\). Alors
        \begin{subequations}
            \begin{align}
                (f\circ g)\big( s_2(x) \big)&=(f\circ \underbrace{g\circ s_2}_{s_1\circ g})(x)\\
                &=(\underbrace{f\circ s_1}_{=s_2\circ f}\circ g)(x)\\
                &=(s_2\circ f\circ g)(x)\\
                &=s_2\big( (f\circ g)(x) \big)\\
                &=s_2(x)
            \end{align}
        \end{subequations}
        Donc \( s_2(x)\in A\). Nous en déduisons que \( A=\mN_2\) par le point \ref{ITEMooXPYEooFajywh} de la définition \ref{DEFooBJBOooWlblAx} de triple naturel.
    \item[\( g\circ f=\id\)]
        J'imagine que c'est la même chose que l'autre\quext{J'ai pas fait les calculs; écrivez-moi si ça pose un problème.}.
    \end{subproof}
\end{proof}

\begin{normaltext}[Définition de \( \eN\)]
Pour la suite, nous considérons un triple naturel \( (\mN,o,s)\) et nous notons \( \eN=\mN\) ainsi que \( 0=o\). Donc la nature de tous les objects que nous allons considérer à partir de maintenant dépend du choix de triple naturel que nous faisons à présent. Le théorème \ref{THOooFUXMooJuigHK} nous assure que peu de choses devraient réellement dépendre de ce choix.
\end{normaltext}


nous admettons la construction, à partir de la théorie des ensembles de
\begin{itemize}
    \item Les opérations \( +\), \( \times\) et leurs inverses là où ils existent
    \item L'ordre.
\end{itemize}

Les résultats énoncés ici sont utilisés plus bas et servent de guide à un éventuel contributeur qui voudrait écrire une partie sur la construction de \( \eN\). Nous espérons que des preuves se trouvent dans \cite{RWWJooJdjxEK}. En tout cas, le lecteur est invité à ne pas les prendre pour évidents.

Nous supposons en particulier que \( \eN\) est construit avec sa relation d'ordre. Voici quelque affirmations que nous admettons.

\begin{lemma}       \label{LEMooYMRJooYIAhBb}
    Quelques affirmations sur l'ordre dans \( \eN\).
    \begin{enumerate}
        \item
            Il n'existe pas de \( n\in \eN\) tel que \( n<0\).
        \item
            Si \( a,b\in \eN\) vérifient \( a>b\), alors il n'existe pas de \( x\) dans \( \eN\) tel que \( a+x=b\).
    \end{enumerate}
\end{lemma}

\begin{lemma}       \label{LEMooFHEOooSHPGgU}
    Toute partie non vide de \( \eN\) possède un unique minimum.
\end{lemma}

\begin{definition}      \label{DEFooKBUFooLvMHrf}
    Pour \( N\in \eN\), nous notons \( \{ 0,\ldots, N \}\) l'ensemble des naturels \( x\) vérifiant \( 0\leq x\leq N\).
\end{definition}

\begin{proposition}
    L'ensemble \( \eN\) est totalement ordonné.
\end{proposition}

\begin{proposition}     \label{PROPooFYMJooWihvhk}
    Une application \( \eN\to \eN\) strictement croissante est injective.
\end{proposition}

%--------------------------------------------------------------------------------------------------------------------------- 
\subsection{Applications définies par récurrence}
%---------------------------------------------------------------------------------------------------------------------------

Voici deux théorèmes qui permettent de définir des applications par récurrence. Comme d'habitude dans ce chapitre, n'imaginez même pas que les démonstrations sont évidentes.

\begin{theorem}[\cite{BIBooZFPUooIiywbk}]       \label{THOooEJPYooZFVnez}
    Soient $E$ un ensemble, $g$ une application de $E$ dans $E $et $b$ un élément de $E$.  Alors il existe une unique application \( f\colon \eN\to E\) telle que :
    \begin{enumerate}
        \item
            \( f(0)=b\)
        \item
            \( f(n+1)=g\big( f(n) \big)\) pour tout \( n\in \eN\setminus\{ 0 \}\).
    \end{enumerate}
\end{theorem}

 
\begin{corollary}[\cite{BIBooZFPUooIiywbk}]       \label{CORooVNHKooRkKtXf}
    Soient deux ensembles \( X,Y\), une application \( \alpha\colon X\to Y\) et une application \( \beta\colon Y\to Y\). Alors il existe une unique application \( H\colon X\times \eN\to Y\) telle que
\begin{enumerate}
    \item
  $H(x , 0) = \alpha(x)$   pour tout élément $x\in X$;
  \item
 $H(x , n+1 ) = \beta( H( x , n) )$ pour tout élément \( x\in X\) et pour tout \( n\in \eN\).
\end{enumerate}
\end{corollary}

\begin{proof}
    Nous posons \( E=\Fun(X,Y)\), \( b=\alpha\in E\) et
    \begin{equation}
        \begin{aligned}
            g\colon E&\to E \\
            s&\mapsto \beta\circ s. 
        \end{aligned}
    \end{equation}
    Le théorème \ref{CORooVNHKooRkKtXf} donne l'existence d'une application \( f\colon \eN\to E\) telle que
    \begin{enumerate}
        \item
            \( f(0)=b\)
        \item
            \( f(n+1)=g\big( f(n) \big)\).
    \end{enumerate}
    Nous définissons alors
    \begin{equation}
        \begin{aligned}
            H\colon X\times \eN&\to Y \\
            (x,n)&\mapsto f(n)x, 
        \end{aligned}
    \end{equation}
    et nous vérifions qu'elle satisfait aux exigences.

    \begin{enumerate}
        \item
            D'abord nous avons
            \begin{equation}
                H(x,0)=f(0)x=b(x)=\alpha(x).
            \end{equation}
        \item
            Ensuite,
            \begin{equation}
                H(x,n+1)=f(n+1)x=g\big( f(n) \big)x=\big( h\circ f(n) \big)x=h\big( f(n)x \big)=h\big( H(x,n) \big).
            \end{equation}
    \end{enumerate}
    Et voila.
\end{proof}

%+++++++++++++++++++++++++++++++++++++++++++++++++++++++++++++++++++++++++++++++++++++++++++++++++++++++++++++++++++++++++++
\section{Les entiers}
%+++++++++++++++++++++++++++++++++++++++++++++++++++++++++++++++++++++++++++++++++++++++++++++++++++++++++++++++++++++++++++


On peut définir les enties comme l'union disjointe de deux copies de \( \eN\), en faisant attention à ne pas prendre deux fois le zéro.
%TODOooTVPUooYNYsVG le faire

\begin{lemma}       \label{LEMooMYEIooNFwNVI}
    Toute partie bornée de \( \eZ\) possède un plus grand élément.
\end{lemma}

\begin{proposition}     \label{PROPooYJBMooZrzkNX}
    Soit \( a,b\in \eZ\) tels que \( a\) divise \( b\). Alors \( | a |\leq | b |\).
\end{proposition}



%+++++++++++++++++++++++++++++++++++++++++++++++++++++++++++++++++++++++++++++++++++++++++++++++++++++++++++++++++++++++++++ 
\section{Quelques résultats de cardinalité}
%+++++++++++++++++++++++++++++++++++++++++++++++++++++++++++++++++++++++++++++++++++++++++++++++++++++++++++++++++++++++++++

Je vous conseille fortement de ne pas considérer les résultats qui viennent comme évidents avant d'avoir lu quelques articles de Wikipédia sur la construction des naturels en théorie des ensembles.

%--------------------------------------------------------------------------------------------------------------------------- 
\subsection{Équipotence, surpotence, subpotence}
%---------------------------------------------------------------------------------------------------------------------------

Les notions d'équipotence, surpotence et de subpotence permettent de comparer les «tailles» des ensembles sans avoir besoin de la théorie des ordinaux. Tout ceci ne sera pas très souvent utile par la suite. Un exemple d'utilisation de ces notions est le théorème de Steinitz \ref{THOooEDQKooLEGlDv} qui démontre l'existence de clôture algébrique pour tout corps.

\begin{definition}[\cite{BIBooAKHUooProFGE,BIBooWNKRooETlebF}]      \label{DEFooXGXZooIgcBCg}
    Soient deux ensembles \( A\) et \( B\).
    \begin{enumerate}
        \item
            Les ensembles \( A\) et \( B\) sont \defe{équipotents}{équipotent} si il existe une bijection entre \( A\) et \( B\). Nous notons \( A\approx B\).
        \item
            L'ensemble \( A\) est \defe{surpotent}{surpotent} à \( B\) si il existe une surjection de \( A\) vers \( B\). Nous notons \( A\succeq B\).
        \item
            L'ensemble \( A\) est \defe{subpotent}{subpotent} à \( B\) si il existe une injection de \( A\) vers \( B\). Nous notons \( A\preceq B\).
    \end{enumerate}
    Nous disons également «strictement» surpotent quand il y a surpotence mais pas équipotence, et de même pour la subpotence. Les symboles \( \succ\) et \( \prec\) sont alors utilisés.
\end{definition}

\begin{proposition}[\cite{MonCerveau,BIBooZFPUooIiywbk}]      \label{PROPooWSXTooMQPcNG}
    L'ensemble \( A\) est subpotent à \( B\) si et seulement si \( B\) est surpotent à \( A\).
\end{proposition}

\begin{proof}
    En deux parties.
    \begin{subproof}
        \item[\( \Rightarrow\)]
            Nous supposons que \( A\) est subpotent à \( B\). Il existe une injection \( \varphi\colon A\to B\). Nous définissons \( f\colon B\to A\) par
            \begin{equation}
                f(x)=\begin{cases}
                    \varphi^{-1}(x)    &   \text{si } x\in\varphi(A)\\
                    a    &    \text{sinon } 
                \end{cases}
            \end{equation}
            où \( a\) est un élément quelconque de \( A\). Cette application est bien définie parce que \( \varphi\) est injective, de telle sorte que \( \varphi^{-1}\) est bien définie. Vu que \( \varphi\) est définie sur tout \( a\), l'application \( f\) est une surjection.
        \item[\( \Leftarrow\)]
            Nous supposons que \( B\) est surpotent à \( A\). Il existe donc une surjection \( \varphi\colon B\to A\). Pour chaque \( x\in A\) nous considérons un élément \( b_x\in \varphi^{-1}(x)\), qui existe parce que \( \varphi\) est surjective. Nous considérons ensuite l'application
            \begin{equation}
                \begin{aligned}
                    f\colon A&\to B \\
                    x&\mapsto b_x. 
                \end{aligned}
            \end{equation}
            Nous prouvons que \( f\) est une injection. Supposons que \( x,y\in A\) soient tels que \( f(x)=f(y)\). Nous avons \( b_x=b_y\). Donc
            \begin{equation}
                x=\varphi(b_x)=\varphi(b_y)=y.
            \end{equation}
            Nous avons prouvé que \( x=y\), et donc que \( f\) est injective.
    \end{subproof}
\end{proof}

Vu que l'ensemble des ensembles n'existe pas\footnote{Voir le corolaire \ref{CORooZMAOooPfJosM}.}, nous n'allons pas énoncer le fait que ces notions donnent une relation d'ordre sur les ensembles; il faudrait parler de classes et nous ne nous en sortirions pas. Nous allons toutefois énoncer quelque résultats qui vont dans ce sens. Pour en savoir plus, vous pouvez lire les différentes pages de Wikipédia sur les nombres cardinaux.

%--------------------------------------------------------------------------------------------------------------------------- 
\subsection{Un peu d'infinité}
%---------------------------------------------------------------------------------------------------------------------------

\begin{definition}      \label{DefEOZLooUMCzZR}
    Un ensemble est \defe{infini}{ensemble!infini} s'il peut être mis en bijection avec un de ses sous-ensembles propres (c'est-à-dire différent de lui-même).
\end{definition}

\begin{lemma}[\cite{MonCerveau}\quext{Écrivez-moi si vous connaissez une preuve de ceci.}]       \label{LEMooYHGCooAwsVQN}
    Infinis et ses parties.
    \begin{enumerate}
        \item
            Si \( I\) et \( J\) sont deux ensembles finis, alors \( I\cup J\) est fini.
        \item
            Si \( I\) ou \( J\) est infini, alors \( I \cup J\) est infini.
    \end{enumerate}
\end{lemma}

\begin{proposition}     \label{PROPooBYKCooGDkfWy}
    L'ensemble \( \eN\) est infini\footnote{Définition \ref{DefEOZLooUMCzZR}.}.
\end{proposition}

Cette proposition est à peu près prise comme définition d'un ensemble fini dans \cite{ooVAYLooJxVYex} qui donne également une preuve de l'équivalence avec notre définition. Rien de tout cela n'est évident\footnote{Surtout que la définition \ref{DEFooKBUFooLvMHrf} de \( \{ 0,\ldots, N \}\) est très légère et pas encore formalisée.}
\begin{propositionDef}     \label{PROPooJLGKooDCcnWi}
    Si \( I\) est un ensemble fini, il existe un unique \( N\in \eN\) tel que \( I\) soit en bijection avec \( \{ 0,\ldots, N \}\).

    Dans ce cas, le nombre \( N+1\) est le \defe{cardinal}{cardinal} de \( I\), et est noté \( \Card(I)\).
\end{propositionDef}
\nomenclature{$\eN_0$}{les naturels non nuls : $\eN_0=\eN\setminus\{ 0 \}$}

Nous ne définissons pas ce qu'est le cardinal d'un ensemble infini; c'est très compliqué et ça ne nous servira pas.

\begin{lemma}       \label{LEMooIAMKooLDucJc}
    Si \( A\) et \( B\) sont des ensembles, alors
    \begin{equation}
        \Card(A\cup B)=\Card(A)+\Card(B)-\Card(A\cap B).
    \end{equation}
    Si les \( \{ A_i \}_{i=1,\ldots, n}\), alors
    \begin{equation}
        \Card\big( \bigcup_{i=1}^nA_i \big)=\sum_{i=1}^n\Card(A_i).
    \end{equation}
\end{lemma}

\begin{definition}\label{DefEnsembleDenombrable}
    Un ensemble est \defe{dénombrable}{dénombrable} s'il peut être mis en bijection avec \( \eN\). Il est \defe{non dénombrable}{non dénombrable} s'il est infini et ne peut pas être mis en bijection avec \( \eN\).
\end{definition}
Une chose vraiment amusante avec cette définition que l'on met en rapport avec la définition~\ref{DefEOZLooUMCzZR}, c'est qu'un ensemble fini n'est ni dénombrable ni non dénombrable\footnote{Beaucoup de sources disent qu'un ensemble est dénombrable lorsqu'il est en bijection avec une partie de \( \eN\). Cela laisse la porte ouverte aux ensembles finis. Par exemple Wikipédia\cite{ooLMVKooUiQUtb}.}.

\begin{lemma}       \label{LEMooGRGFooSWDeMA}
    Si \( A\) est un ensemble finie si \( \sigma\colon A\to B\) est une application quelconque, alors \( \sigma(A)\) est un ensemble fini.
\end{lemma}

\begin{lemma}       \label{LEMooTUIRooEXjfDY}
    Toute partie d'un ensemble fini est finie.
\end{lemma}

\begin{lemma}       \label{LEMooSRZWooASgEfy}
    Si \( A\) est un ensemble fini ou dénombrable, alors il existe une surjection \( \eN\to A\).
\end{lemma}

\begin{lemma}[\cite{MonCerveau}]        \label{LEMooXPSQooRaSrxv}
    Si \( A\) est un ensemble infini et si \( f\colon A\to B\) est une application injective, alors \( f(A)\) est infini.
\end{lemma}

\begin{proof}
    Vu que \( A\) est infini, il existe \( A'\) strictement inclus à \( A\) et une bijection \( \sigma\colon A'\to A\). Nous allons prouver que la partie \( f(A')\) est en bijection avec \( f(A)\) tout en étant un sous-ensemble strict de \( f(A)\).

    \begin{subproof}
        \item[Stricte inclusion]
            Si \( a\in A\setminus A'\), alors par injectivité de \( f\), nous avons aussi \( f(a)\in f(A)\setminus f(A')\). Autrement dit, \( f(a)\) est un élément de \( f(A)\) qui n'est pas dans \( f(A')\).
        \item[La candidate bijection]
            Vu que \( f\) est une injection nous pouvons considérer \( f^{-1}\) sur \( f(A)\) est poser
            \begin{equation}
                \begin{aligned}
                    \varphi\colon f(A')&\to f(A) \\
                    x&\mapsto  f\Big( \sigma\big( f^{-1}(x) \big) \Big).
                \end{aligned}
            \end{equation}
        \item[Surjection]
            Une élément de \( f(A)\) est de la forme \( y=f(a)\) avec \( a\in A\). L'application \( \sigma\) est une bijection, donc nous pouvons poser \( b=\sigma^{-1}(a)\). Il est alors facile de vérifier que \( x=f(b)\) satisfait \( \varphi(x)=f(a)\). En effet :
            \begin{equation}
                \varphi(x)=(\varphi f\sigma^{-1})(a)=(f\sigma f^{-1}f\sigma^{-1})(a)=f(a).
            \end{equation}
            Cela prouve que \( \varphi\) est surjective.
        \item[Injection]
            Soient \( a,b\in A'\) tels que \( \varphi\big( f(a) \big)=\varphi\big( f(b) \big)\). Nous devons prouver que \( f(a)=f(b)\). Nous avons
            \begin{equation}
                (\varphi f)(a)=(f\sigma f^{-1} f)(a)=(f\sigma)(a).
            \end{equation}
            Donc l'hypothèse dit que \( (f\sigma)(a)=(f\sigma)(b)\), c'est à dire \( f\big( \sigma(a) \big)=f\big( \sigma(b) \big)\). Vu que \( f\) est injective, cela implique \( \sigma(a)=\sigma(b)\), et donc \( a=b\) parce que \( \sigma\) est également injective. Partant, \( f(a)=f(b)\).
    \end{subproof}
\end{proof}

%--------------------------------------------------------------------------------------------------------------------------- 
\subsection{Dénombrabilité et ensemble des naturels}
%---------------------------------------------------------------------------------------------------------------------------

\begin{proposition}[\cite{MonCerveau, BIBooZFPUooIiywbk}]      \label{PROPooOBKMooWEGCvM}
    Toute partie infinie de \( \eN\) est dénombrable.
\end{proposition}

\begin{proof}
    Soit \( A\), une partie infinie de \( \eN\). 
    \begin{subproof}
    \item[Définition de \( \sigma\)]
    Nous voulons construire une application \( \sigma\colon \eN\to A\) telle que
    \begin{subequations}
        \begin{numcases}{}
            \sigma(0)=\min(A)   \label{SUBEQooEIEMooZcTOWT}\\
            \sigma(k+1)=\min\Big( A\setminus \sigma\big( \{ 0,\ldots, k \} \big)\Big)      \label{SUBEQooWWOAooAEfrPx}
        \end{numcases}
    \end{subequations}
    Les lâches, par prudence, diront juste que c'est défini par récurrence et n'insisteront pas. Nous, nous insistons.

    Nous allons définir \( \sigma(n)\) à l'aide du théorème \ref{THOooEJPYooZFVnez}. Pour cela nous posons \( E=\mP(A)\), \( b=\emptyset\) et
    \begin{equation}
        \begin{aligned}
            g\colon E&\to E \\
            Z&\mapsto \begin{cases}
                A    &   \text{si } Z=A\\
                Z\cup\{ \min(A\setminus Z) \}    &    \text{sinon. }
            \end{cases}
        \end{aligned}
    \end{equation}
    Notons que le lemme \ref{LEMooFHEOooSHPGgU} nous indique que toute partie non vide de \( \eN\) possède un minimum; la définition de \( g\) a donc un sens. Le théorème \ref{THOooEJPYooZFVnez} donne alors une application \( f\colon \eN\to E\) telle que
    \begin{enumerate}
        \item
            \( f(0)=b=\emptyset\)
        \item
            \( f(n+1)=g\big( f(n) \big)\) pour tout \( n\geq 0\).
    \end{enumerate}
    Prouvons par récurrence que \( f(n)\) est un ensemble fini pour tout \( n\). D'abord \( f(0)=\emptyset\). Ensuite, si \( n\geq 0\) est tel que \( f(n)\) est fini, alors en particulier \( f(n)\neq A\) et nous avons
    \begin{equation}
        f(n+1)=g\big( f(n) \big)=f(n)\cup\{ \min\big( A\setminus f(n) \big)\}.
    \end{equation}
    Dans ce cas, \( f(n+1)\) est également fini comme union de deux ensembles finis.

    Nous posons
    \begin{equation}        \label{EQooGHQHooRnXDdo}
        \sigma(n)=\min\big( A\setminus f(n) \big).
    \end{equation}
    Avec \( n=0\), nous avons \( \sigma(0)=\min\big( A\setminus \emptyset \big)=\min(A)\). La condition \eqref{SUBEQooEIEMooZcTOWT} est donc déjà satisfaite.

    Nous devons encore prouver \eqref{SUBEQooWWOAooAEfrPx}.  Pour tout \( n\), la relation entre \( f(n)\) et \( \sigma(n)\) est donnée par
    \begin{subequations}
        \begin{numcases}{}
            f(0)=\emptyset\\
            f(n+1)=f(n)\cup \sigma(n).
        \end{numcases}
    \end{subequations}
    Par récurrence nous avons alors
    \begin{equation}        \label{EQooPXFEooYzhtBe}
        f(n)=\bigcup_{k=0}^{n-1}\{ \sigma(k) \}=\sigma\big( \{ 0,\ldots, n-1 \} \big)
    \end{equation}
    pour tout \( n\geq 1\). Nous avons alors la condition \ref{SUBEQooWWOAooAEfrPx} en substituant \eqref{EQooPXFEooYzhtBe} dans la définition \eqref{EQooGHQHooRnXDdo} écrite avec \( n+1\) :
    \begin{equation}
        \sigma(n+1)=\min\big( A\setminus f(n+1) \big)=\min\Big( A\setminus \sigma\big( \{ 0,\ldots, n \} \big) \Big).
    \end{equation}

        \item[\( \sigma\) est strictement croissante]
            Vu que \( A\setminus\sigma\{ 0,\ldots, k \}\subset A\setminus\sigma\{ 0,\ldots, k-1 \}\), le minimum est plus grand ou égal : \( \sigma(k+1)\geq \sigma(k)\). Mais \( \sigma(k+1)\) est sélectionné dans l'ensemble \( A\setminus\sigma\{ 0,\ldots, k \}\), qui ne contient justement pas \( \sigma(k)\). Donc \( \sigma(k+1)\neq \sigma(k)\).
        \item[\( \sigma\) est définie sur \( \eN\)]
            Il faut montrer que pour tout \( k\), l'ensemble \( A\setminus\sigma\{ 0,\ldots, k \}\) est non vide. Si il l'était, cela signifierait que \( A\subset \sigma\{ 0,\ldots, k \}\). Par le lemme \ref{LEMooGRGFooSWDeMA}, la partie \( \sigma\{ 0,\ldots, k \}\) est finie dans \( \eN\). Le lemme \ref{LEMooTUIRooEXjfDY} dit alors qu'en tant que partie de \( \sigma\{ 0,\ldots, k \}\), l'ensemble \( A\) est fini. Mais comme les hypothèses disent que \( A\) est infini, nous avons une contradiction et nous concluons que \( \sigma\) est bien définie sur tout \( \eN\).
        \item[\( \sigma\) est injective]
            Une application \( \eN\to \eN\) strictement croissante est injective par la proposition \ref{PROPooFYMJooWihvhk}.
        \item[\( \sigma\) est surjective]
            Soit \( a\in A\). Vu que \( \sigma\) est strictement croissante et que \( \sigma(0)\geq 0\), nous avons \( \sigma(a)\geq a\). Si \( \sigma(a)=a\) nous avons terminé. Supposons \( \sigma(a)>a\). Alors
            \begin{equation}        \label{EQooNHTBooQexzwV}
                \min\big( A\setminus\sigma\{ 0,\ldots, a \} \big)>a.
            \end{equation}
            Si \( \sigma\big( \{ 0,\ldots, a \} \big)\) ne contenait pas \( a\), alors \( A\setminus \sigma(\{ 0,\ldots, a \})\) le contiendrait et nous n'aurions pas l'inégalité \eqref{EQooNHTBooQexzwV}. Donc \( a\in \sigma\big( \{ 0,\ldots, a \} \big)\) et \( a\) est bien dans l'image de \( \sigma\).
    \end{subproof}
\end{proof}

\begin{normaltext}
    La proposition \ref{PROPooOBKMooWEGCvM} pourrait être prouvée plus facilement en acceptant le théorème de Cantor-Schröder-Bernstein \ref{THOooRYZJooQcjlcl}. Il existe une injection \( A\to \eN\) parce que \( A\) est une partie de \( \eN\). Mais vu que \( A\) est infini, il possède une partie dénombrable. Cela donne une surjection \( A\to \eN\) et donc une injection \( \eN\to A\). Le théorème de Cantor-Schröder-Bernstein conclu.

    Cela dit, une telle preuve demanderait des outils plus complexes.
\end{normaltext}


\begin{normaltext}
    La proposition suivante donne une bijection explicite entre \( \eN\) et \( \eN\times \eN\). Elle n'a rien de transcendante, mais je ne résiste pas à la donner ici parce qu'elle est utilisée dans l'article \emph{Un peu de programmation transfinie} de David Madore\footnote{Et comme j'aime beaucoup cet article, il me fallait une excuse pour le placer ici.\\ \url{http://www.madore.org/~david/weblog/d.2017-08-18.2460.html}.}. Son utilité est de pouvoir créer un langage de programmation pouvant traiter des paires d'entiers rien qu'en traitant des entiers.
\end{normaltext}
\begin{proposition}[Une bijection \( \eN\times \eN\to \eN\)]        \label{PROPooLPKUooAlsYJg}
    La fonction
    \begin{equation}
        \begin{aligned}
            f\colon \eN\times \eN&\to \eN \\
            (x,y)&\mapsto \begin{cases}
                y^2+x    &   \text{si } x<y\\
                x^2+x+y    &    \text{si } y\leq x.
            \end{cases}
        \end{aligned}
    \end{equation}
    est une bijection.
\end{proposition}

\begin{proof}
    Il s'agit de prouver qu'elle est injective et surjective. Dans la suite, tous les nombres sont des entiers positifs.
    \begin{subproof}
        \item[\( f\) est injective]

            Pour \( k\in \eN\) donné, nous allons prouver que
            \begin{enumerate}
                \item
                    l'équation \( f(x,y)=k\) possède au maximum une solution avec \( x<y\),
                \item
                    l'équation \( f(x,y)=k\) possède au maximum une solution avec \( y\leq x\),
                \item
                    si \(   k=y^2+x \) avec \( x<y\) alors il est impossible que \( k=x'^2+x'+y'\) avec \( y'\leq x'\).
            \end{enumerate}
            On y va.
            \begin{enumerate}
                \item
                    Nous supposons \( y^2+x=t^2+z\) avec \( x<y\) et \( z<t\). Pour fixer les idées, nous supposons \( t>y\) et nous posons \( t=y+s\) (\( s\geq 1\)). En substituant, et en isolant \( z\),
                    \begin{subequations}
                        \begin{align}
                            z&=x-2sy-s^2\\
                            &<x-2sy\\
                            &<x-2sx\\
                            &=x(1-2s)\\
                            &<0.
                        \end{align}
                    \end{subequations}
                    Impossible parce que \( z\geq 0\).
                \item
                    De même nous supposons \( x^2+x+y=z^2+z+t\) avec \( y\leq x\) et \( t\leq z\). Nous posons \( z=x+s\), et nous déballons le même genre de calculs en isolant \( t\).
                \item
                    Enfin nous supposons \( y^2+x=z^2+z+t\) avec \( x<y\) et \( t\leq z\). Les plus courageux diviseront en trois cas : \( y<z\), \( y=z\) et \( y>z\) et feront les calculs. Par exemple, pour le cas \( y>z\) nous posons \( y=z+s\) et nous substituons :
                    \begin{equation}
                        (y+s)^2+x=z^2+z+t
                    \end{equation}
                    qui donne
                    \begin{equation}
                        x=z+t-2zs-s^2<2z-2zs-s^2=2z(1-s)-s^2\leq -s<0
                    \end{equation}
                    parce que \( s\geq 1\), donc \( 1-s\leq 0\).
            \end{enumerate}

        \item[\( f\) est surjective]

            Nous devons prouver que tous les éléments de \( \eN\) sont dans l'image de \( \eN\times \eN\) par \( f\). En premier lieu, \( 0=f(0,0)\). C'est un bon début. Soit \( a\in \eN\) non nul; nous montrons que tous les nombres de \( a^2\) à \( (a+1)^2\) sont des images de \( f\). D'abord \( a^2=f(0,a)\), ensuite les nombres
            \begin{equation}
                f(1,a),f(2,a),\ldots, f(a-1,a)
            \end{equation}
            prennent les valeurs \( a^2+1\), \ldots, \( a^2+a-1\). Enfin nous avons \( f(a,0)=a^2+a\) et les nombres \( f(a,1),\ldots, f(a,a)\) prennent les valeurs de \( a^2+a+1\) à \( a^2+2a=(a+1)^2-1\).
    \end{subproof}
\end{proof}
Sachez que cette fonction s'étend aux ordinaux (mais là ce n'est plus pour rigoler).

\begin{corollary}       \label{CORooNRPIooZPSmqa}
    Il existe des parties \( \{ \eN_i \}_{i\in \eN}\) telles que \( \bigcup_{i\in \eN}\eN_i=\eN\) et que chaque \( \eN_i\) soit en bijection avec \( \eN\)
\end{corollary}

\begin{proof}
    Nous considérons la bijection \( f\colon \eN\to \eN\times \eN\) donnée par (l'inverse de celle donnée) par la proposition \ref{PROPooLPKUooAlsYJg}, et nous posons
    \begin{equation}
        \eN_i=f^{-1}(i,\eN).
    \end{equation}
    L'application
    \begin{equation}
        f\colon \eN_i\to \{ (i,k) \}_{k\in \eN}
    \end{equation}
    est une bijection. Or l'ensemble \( \{ (i,k) \}_{k\in \eN}\) est évidemment en bijection avec \( \eN\). Par composition nous avons le résultat.
\end{proof}

\begin{lemma}[\cite{MonCerveau}]        \label{LEMooDLWFooNAJbbq}
    Si il existe une surjection \( \eN\to A\), alors \( A\) est fini ou dénombrable.
\end{lemma}

\begin{proof}
    Pour chaque \( a\in A\), l'ensemble \( f^{-1}(a)\) est une partie de \( \eN\). 
    \begin{subproof}
        \item[Une application]
            Le lemme \ref{LEMooFHEOooSHPGgU} nous permet de poser
            \begin{equation}
                \begin{aligned}
                    \sigma\colon A&\to \eN \\
                    a&\mapsto \min\big( f^{-1}(a) \big). 
                \end{aligned}
            \end{equation}
        \item[\( \sigma\) est injective]
            Supposons que \( \sigma(a)=\sigma(b)\). Nous appelons \( x\) ce nombre :
            \begin{equation}
                x=\min\big( f^{-1}(a) \big)=\min\big( f^{-1}(b) \big).
            \end{equation}
            Nous avons \( x\in f^{-1}(a)\cap f^{-1}(b)\), ce qui implique que \( f(x)=a\) et que \( f(x)=b\); donc \( a=b\).

            Donc \( \sigma\) est une injection.
        \item[\( A\) est infini]
            Si \( A\) est fini, le lemme est prouvé. Donc à partir de maintenant nous supposons que \( A\) est infini. Le but est de prouver qu'il est dénombrable, c'est à dire de construire une bijection \( A\to \eN\).
        \item[\( \sigma(A)\) est dénombrable]
            Vu que \( \sigma\colon A\to  \eN\) est injective et que \( A\) est infini, le lemme \ref{LEMooXPSQooRaSrxv} dit que \( \sigma(A)\) est infini dans \( \eN\). La proposition \ref{PROPooOBKMooWEGCvM} nous dit alors que \( \sigma(A)\) est dénombrable.

            Soit une bijection \( \varphi\colon \sigma(A)\to \eN\).
        \item[La candidate bijection]
            Nous posons
            \begin{equation}
                f=\varphi\circ \sigma\colon A\to \eN
            \end{equation}
            et nous allons prouver que c'est une bijection.
        \item[Injective]
            Vu que \( \varphi\) et \( \sigma\) sont injectives, l'égalité \( (\varphi\sigma)(a)=(\varphi\sigma)(b)\) implique immédiatement \( a=b\).
        \item[Surjective]
            Soit \( k\in \eN\). Vu que \( \varphi\) et \( \sigma\) sont des injections, nous pouvons poser \( a=(\sigma^{-1}\varphi^{-1})(k)\). Il est alors immédiat que \( f(a)=k\).
    \end{subproof}
\end{proof}

\begin{proposition}[\cite{MonCerveau,ooLMVKooUiQUtb}]     \label{PROPooENTPooSPpmhY}
    Une union dénombrable d'ensembles finis ou dénombrables est finie ou dénombrable.
\end{proposition}

\begin{proof}
    Soient \( A_i\) des ensembles finis ou dénombrables. Nous posons \( A=\bigcup_{i\in \eN}A_i\), et nous considérons les parties \( \eN_i\) du corolaire \ref{CORooNRPIooZPSmqa}. Vu que \( A_i\) est dénombrable ou fini et que \( \eN_i\) est dénombrable, il existe une surjection \( \varphi_i\colon \eN_i\to A_i\).

    Nous définissons \( s\colon \eN\to \eN\) par \( n\in \eN_{s(n)}\), et nous posons enfin
    \begin{equation}
        \begin{aligned}
            \varphi\colon \eN&\to A \\
            n&\mapsto \varphi_{s(n)}(n). 
        \end{aligned}
    \end{equation}
    Nous prouvons que \( \varphi\) est surjective.

    Soit \( a\in A_i\). Il existe \( n\in \eN_i\) tel que \( a=\varphi_i(n)\). Mais comme \( n\in \eN_i\), nous avons \( s(n)=i\). Donc
    \begin{equation}
        a=\varphi_i(n)=\varphi_{s(n)}(n)=\varphi(n).
    \end{equation}
    Donc \( \varphi\colon \eN\to A\) est surjective.

    Le lemme \ref{LEMooDLWFooNAJbbq} conclu que \( A\) est fini ou dénombrable.
\end{proof}

\begin{lemma}[\cite{MonCerveau}]        \label{LEMooRXSRooBUWOyb}
    Si \( N\) est un ensemble dénombrable, alors il existe une bijection \( g\colon \{ 1,2 \}\times N\to N\).
\end{lemma}

\begin{proof}
    D'abord nous définissons une bijection \( \varphi\colon \{ 0,1 \}\times \eN\to \eN\) par
    \begin{equation}
        \begin{aligned}
            \varphi\colon \{ 0,1 \}\times \eN&\to \eN \\
            (n,k)&\mapsto 2k+n. 
        \end{aligned}
    \end{equation}
    Ensuite si \( f\colon \eN\to N\) est une bijection, il suffit de poser \( g(n,k)=f\big( \varphi(n,k) \big)\).
\end{proof}

\begin{proposition}[\cite{ooLMVKooUiQUtb}]     \label{PROPooDMZHooXouDrQ}
    Si \( N\) est un ensemble dénombrable, alors pour tout \( n\in \eN\), l'ensemble \( N^n\) est dénombrable.
\end{proposition}

Les ensembles dénombrables sont les plus petits ensembles infinis possibles, comme en témoigne la proposition suivante.
\begin{proposition}      \label{PROPooUIPAooCUEFme}
    Tout ensemble infini contient une partie en bijection avec \( \eN\).
\end{proposition}

\begin{proof}
    Soient un ensemble infini \( E_0\) et une partie propre \( E_1\) en bijection avec \( E_0\). Nous notons \( \varphi\colon E_0\to E_1\) une bijection.

    Soit \( x_0\in E_0\setminus E_1\) (axiome du choix et tout ça). Nous définissons
    \begin{equation}
        \begin{aligned}
            \psi\colon \eN&\to \{\varphi^k(x_0)\} \\
            n&\mapsto \varphi^n(x_0)
        \end{aligned}
    \end{equation}
    et nous allons prouver que c'est une bijection. Que ce soit surjectif est immédiat. Pour l'injectivité, soit \( \varphi^k(x_0)=\varphi^l(x_0)\) avec \( k\neq l\). Supposons pour fixer les notations que \( k>l\). Alors, vu que \( \varphi\) est inversible nous pouvons écrire
    \begin{equation}
        x_0=\varphi^{k-l}(x_0)=\varphi\big( \varphi^{k-l-1}(x_0) \big)
    \end{equation}
    où il est entendu que \( \varphi^0(x_0)=x_0\). Cela signifie que \( x_0\) est dans l'image de \( \varphi\), c'est-à-dire dans $E_1$, ce que nous avons exclu par choix de \( x_0\) dans \( E_0\setminus E_1\). Donc en réalité \( \varphi^k(x_0)\neq \varphi^l(x_0)\) dès que \( k\neq l\).
\end{proof}

\begin{proposition} \label{PropQEPoozLqOQ}
    Toute partie d'un ensemble fini est finie, et toute partie d'un ensemble dénombrable est finie ou dénombrable.
\end{proposition}

\begin{lemma}   \label{LEMooGTOTooFbpvzU}
    Soit un ensemble \( E\) non dénombrable ainsi qu'une application \( f\colon E\to F\) où \( F\) est un ensemble quelconque. Si \( f(E)\) est dénombrable (ou fini), alors il existe \( y\in f(E)\) tel que \( f^{-1}(y)\) est indénombrable.
\end{lemma}

\begin{proof}
    Nous avons
    \begin{equation}
        E=\bigcup_{y\in F}f^{-1}(y).
    \end{equation}
    Si tous les \( f^{-1}(y)\) sont dénombrables, alors \( E\) est une union dénombrable (\( F\) est dénombrable) d'ensembles dénombrables. Il serait donc dénombrable (proposition \ref{PROPooENTPooSPpmhY}), ce qui est contraire à l'hypothèse.
\end{proof}

%--------------------------------------------------------------------------------------------------------------------------- 
\subsection{Théorème de Cantor-Schröder-Bernstein}
%---------------------------------------------------------------------------------------------------------------------------

\begin{lemma}[\cite{BIBooECJMooGPxBem}]     \label{LEMooTNMHooBpdzab}
    Soient un ensemble \( A\) et une partie \( B\) de \( A\). Si il existe une injection \( f\colon A\to B\) alors il existe une bijection \( \alpha\colon A\to B\).
\end{lemma}

\begin{proof}
    Nous posons \( Y=A\setminus B\) et nous décomposons la preuve en étapes.
    \begin{subproof}
        \item[Les \( f^k(Y)\) sont disjoints]
            Vu que \( f\) prend ses valeurs dans \( B\), nous avons \( f^k(Y)\subset B\) pour tout \( k\). Et vu que \( Y=A\setminus B\), nous avons
            \begin{equation}        \label{EQooDNHJooFJBrDq}
                f^k(Y)\cap Y=\emptyset
            \end{equation}
            pour tout \( k\). L'application \( f\) étant injective, elle vérifie \( f(C\cap D)=f(C)\cap f(D)\). Nous appliquons \( f^m\) des deux côtés de \eqref{EQooDNHJooFJBrDq} :
            \begin{equation}
                f^{k+m}(Y)\cap f^m(Y)=\emptyset
            \end{equation}
            pour tout \( k,m\in \eN\).
        \item[Une décomposition] 
            Nous posons
            \begin{equation}
                X=\bigcup_{k\in \eN}f^k(Y)=Y\cup\bigcup_{k=1}^{\infty}f^k(Y).
            \end{equation}
            Vu que \( f(X)\subset B\) nous avons l'égalité
            \begin{equation}
                B=f(X)\cup\big(B\setminus f(X)\big).
            \end{equation}
        \item[\( A\setminus X=B\setminus f(X)\)]
            Supposons \( x\in A\setminus X\). Vu que \( Y=A\setminus B\) est dans \( X\), l'élément \( x\) n'est pas dans \( A\setminus B\) et donc est dans \( B\) parce qu'il est dans \( A\). Mais \( x\) n'est pas dans \( X\) et en particulier pas dans \( f(X)\) parce que \( f(X)\subset X\). Donc \( x\) est dans \( B\setminus f(X)\).

            Dans l'autre sens, nous supposons que \( x\in B\setminus f(X)\). Vu que \( B\subset A\) nous avons \( x\in A\). Comme \( x\) est hors de \( f(X)\), il est hors des \( f^k(Y)\) pour \( k\geq 1\). Mais \( x\in B\), donc \( x\) est hors de \( A\setminus B=f^0(Y)\). Donc \( x\) est hors de \( f^k(Y)\) pour tout \( k\geq 0\). Donc \( x\) est hors de \( X\).

        \item[La bijection]
            Nous considérons l'application
            \begin{equation}
                \begin{aligned}
                    \alpha\colon A&\to B \\
                    x&\mapsto \begin{cases}
                        f(x)    &   \text{si } x\in X\\
                        x    &    \text{si } x\in A\setminus X.
                    \end{cases}
                \end{aligned}
            \end{equation}
            Nous démontrons dans les points suivants que \( \alpha\) est bijective.
        \item[Injective]
            Nous supposons \( \alpha(x)=\alpha(y)\). Il y a 4 possibilités suivant que \( x\) et \( y\) soient dans \( X\) ou \( A\setminus X\).

            Si \( x,y\in X\) alors \( f(x)=f(y)\) et donc \( x=y\) parce que \( f\) est injective.

            Si \( x\in X\) et \( y\in A\setminus X\), alors \( f(x)=y\). Mais \( f(x)\in f(X)\) et \( y\in A\setminus X=B\setminus f(X)\). Donc l'élément \( f(x)=y\) est dans \( f(X)\cap \big( B\setminus f(X) \big)=\emptyset\). Il n'est donc pas possible d'avoir \( \alpha(x)=\alpha(y)\) avec \( x\in X\) et \( y\in A\setminus X\).

            Si \( x\in A\setminus X\) et \( y\in X\), c'est la même chose.

            Si \( x,y\in A\setminus X\), alors \( x=\alpha(x)=\alpha(y)=y\).

        \item[Injective]
            Soit \( y\in B\). Il y a deux possibilités : \( y\in X\) et \( y\in A\setminus X\). La première se divise en deux : \( y\in Y\) et \( y\in \bigcup_{k=1}^{\infty}f^k(Y)\). On y va.

            \begin{subproof}
                \item[\( y\in Y\)]
                    Ce cas n'est pas possible parce que \( y\in B\) alors que \( Y=A\setminus B\).
                \item[\( y\in f^k(Y)\) avec \( k\geq 1\)]
                    Nous avons
                    \begin{equation}
                         y\in f\big( f^{k-1}(Y) \big)\subset f(X)\subset \alpha(A).
                    \end{equation}
                \item[\( y\in A\setminus X\)]
                    Alors \( y=\alpha(y)\).
            \end{subproof}
    \end{subproof}
\end{proof}

\begin{theorem}[Cantor-Schröder-Bernstein]      \label{THOooRYZJooQcjlcl}
    Soient deux ensembles \( A\) et \( B\) pour lesquels il existe des injections \( f\colon A\to B\) et \( g\colon B\to A\). Alors il existe une bijection entre \( A\) et \( B\).
\end{theorem}

\begin{proof}
    La composée \( g\circ f\colon A\to A\) est injective et prend ses valeurs dans \( g\big( f(A) \big)\subset g(B)\subset A\). Bref, l'application \( g\circ f\colon A \to g(B)\) est injective. Le lemme \ref{LEMooTNMHooBpdzab} donne alors une bijection \( \varphi\colon A\to g(B)\).

    Nous montrons que \( g^{-1}\circ\varphi\colon A\to B\) est une bijection.

    \begin{subproof}
        \item[Injective]
            Nous supposons \( x,y\in A\) tels que
            \begin{equation}
                g^{-1}\big( \varphi(x) \big)=g^{-1}\big( \varphi(y) \big).
            \end{equation}
            Nous appliquons \( g\) des deux côtés : \( \varphi(x)=\varphi(y)\). Vu que \( \varphi\) est une bijection, cela entraîne \( x=y\).
        \item[Surjective]
            Soit \( b\in B\). En posant \( a=\varphi^{-1}\big( g(b) \big)\) nous avons bien \( (g^{-1}\circ \varphi)(a)=b\).
    \end{subproof}
\end{proof}

%--------------------------------------------------------------------------------------------------------------------------- 
\subsection{Comparabilité cardinale}
%---------------------------------------------------------------------------------------------------------------------------

Le théorème de comparabilité cardinale énonce que si \( A\) et \( B\) sont des ensemble, alors nous avons toujours \( A\succeq B\) ou \( A\preceq B\) (ou les deux; dans ce cas \( A\approx B\) par Cantor-Schröder-Bernstein).
\begin{theorem}[Théorème de comparabilité cardinale\cite{MonCerveau,BIBooTSOKooCWxMwj,BIBooZZNRooZytRuC}]     \label{THOooCBSKooCmzfUf}
    Entre deux ensembles, il existe forcément une injection de l'un dans l'autre.
\end{theorem}

\begin{proof}
    Nous allons montrer que le graphe d'une injection de $A$ dans $B$ ou de $B$ dans $A$ est donné par un élément maximal (au sens de l'inclusion) de l'ensemble (inductif) des graphes d'injections d'une partie de $A$ dans une partie de $B$.

    Nous allons utiliser le lemme de Zorn \ref{LemUEGjJBc} à l'ensemble\footnote{Attention : dans le Frido, la notation \( f\colon A\to B\), signifie que \( f\) est définie sur tout l'ensemble \( A\), mais pas qu'elle est surjective sur \( B\).}
    \begin{equation}
       \mA=\Big\{  (X,Y,\varphi)  \tq
        \begin{cases}
            X\subset A\\
            Y\subset B\\
            \varphi\colon X\to Y\text{ est injective.}
        \end{cases}
    \Big\}
    \end{equation}
    que nous ordonnons par, l'inclusion, c'est à dire par \( (X_1,Y_1,\varphi_1)<(X_2,Y_2,\varphi_2)\) lorsque \( X_1\subset X_2\), \( Y_{1}\subset Y_2\) et \( \varphi_2|_{X_1}=\varphi_1\).

    Nous passons rapidement sur le fait que cet ensemble est inductif, et nous considérons tout de suite un élément maximal \( (X,Y,\varphi)\). Il y a deux possibilités : soit \( \varphi(X)=B\), soit \( \varphi(X)\neq B\).
    \begin{subproof}
        \item[Si \( \varphi(X)=B\)]
            Dans ce cas, \( \varphi\colon X\to B\) est une surjection. L'ensemble \( A\) est donc surpotent à \( B\). La proposition \ref{PROPooWSXTooMQPcNG} conclu que \( B\) est subpotent à \( A\).
        \item[Si \( \varphi(X)\neq B\)]
            Nous subdivisons en deux nouveaux cas : soi \( X=A\), soit \( X\neq A\).
            \begin{subproof}
                \item[Si \( X=A\)]
                    Alors nous avons une injection \( \varphi\colon A\to B\), et c'est bon.
                \item[Si \( X\neq A\)]
                    Nous sommes dans le cas \( X\neq A\) et \( \varphi(X)\neq B\). Soient \( a\in A\setminus X\) et \( b\in B\setminus \varphi(X)\). Nous considérons l'application
                    \begin{equation}
                        \begin{aligned}
                            \psi\colon X\cup\{ a \}&\to Y\cup\{ b \} \\
                            x&\mapsto \begin{cases}
                                \varphi(x)    &   \text{si } x\in X\\
                                b    &    \text{si }x=a.
                            \end{cases}
                        \end{aligned}
                    \end{equation}
                    Cela est une application injective. Donc le triple \( (X\cup \{ a \}, Y\cup\{ b \},\psi)\) majore \( (X,Y,\varphi)\). Nous avons une contradiction et ce cas n'est pas possible.
            \end{subproof}
    \end{subproof}
\end{proof}

\begin{normaltext}
    Le théorème de comparabilité cardinale couplé au théorème de Cantor-Schröder-Bernstein nous indique que pour tout ensembles \( A\) et \( B\), nous avons soit \( A\preceq B\), soit \( B\preceq A\). Et si \( A\preceq B\preceq A\), alors \( A\approx B\).

    Nous ne sommes pas loin de dire que la relation \( \preceq\) donne un ordre total sur l'ensemble des ensembles. C'est très beau sauf que l'ensemble des ensembles n'existe pas\footnote{Corolaire \ref{CORooZMAOooPfJosM}.}. Il faudrait parler de \emph{classe} des ensembles, mais ça nous mènerait trop loin. Toujours est-il que ces deux théorèmes montrent qu'on n'est pas loin d'avoir un ordre sur les ensembles, et que cela est une des bases possibles pour développer les nombres cardinaux.
\end{normaltext}

%--------------------------------------------------------------------------------------------------------------------------- 
\subsection{Théorème de Cantor, ensemble des ensembles}
%---------------------------------------------------------------------------------------------------------------------------

\begin{theorem}[Cantor\cite{BIBooXHFNooSmqUar}]     \label{THOooJPNFooWSxUhd}
    Un ensemble est toujours strictement subpotent à son ensemble des parties.
\end{theorem}

\begin{proof}
    Soit un ensemble \( E\) et son ensemble des parties \( \mP(E)\). Nous commençons par prouver qu'il n'existe pas de surjection \( E\to \mP(E)\). Soit en effet une application \( f\colon E\to \mP(E)\). Nous posons
    \begin{equation}
        D=\{ x\in E\tq x\notin f(x) \}.
    \end{equation}
    Nous prouvons que \( D\) ne peut pas être dans l'image de \( f\). Supposons que \( y\in E\) soit tel que \( f(y)=D\).
    \begin{subproof}
        \item[Si \( y\in D\)]
            Alors par définition de \( D\), nous avons \( y\notin f(y)=D\). Contradiction.
        \item[Si \( y\notin D\)]
            Alors \( y\in f(y)=D\), contradiction.
    \end{subproof}
    Donc aucune surjection \( f\colon E\to \mP(E)\) n'existe. En particulier par de bijections.

    Par ailleurs, l'application \( g\colon \mP(E)\to E\) qui fait \( g(\{ a \})=a\) (et n'importe quoi d'autre sur les autres éléments de \( \mP(E)\)) est une surjection \( \mP(E)\to E\).

    Donc \( \mP(E)\) est toujours strictement surpotent à \( E\).
\end{proof}

\begin{normaltext}
    Le théorème de Cantor implique en particulier qu'il existe (au moins) une infinité dénombrable d'ensemble infinis de cardinalité différentes (plus évidemment une infinité dénombrable d'ensembles finis de cardinalité différentes).

    Pour tout ensemble \( A\), il est donc possible de dire «soit \( E\), un ensemble strictement surpotent à \( A\)».
\end{normaltext}

\begin{corollary}       \label{CORooZMAOooPfJosM}
    Il n'existe pas d'ensemble contenant tous les ensembles.
\end{corollary}

\begin{proof}
    Si \( E\) était un tel ensemble, nous aurions \( \mP(E)\subset E\) parce que les éléments de \( \mP(E)\) sont des ensembles. Or cela donnerait une surjection \( E\to \mP(E)\) alors que cela est impossible par le théorème de Cantor \ref{THOooJPNFooWSxUhd}.
\end{proof}

%--------------------------------------------------------------------------------------------------------------------------- 
\subsection{Ajouter ou soustraire des cardinalités}
%---------------------------------------------------------------------------------------------------------------------------

Nous allons prouver une série de résultats que nous pourrions résumer en  «ajouter ou retrancher des parties de cardinalité plus petite ne change pas la cardinalité».

\begin{lemma}[\cite{MonCerveau}]        \label{LEMooUFCAooSyZtZj}
    Si \( A\) est infini et \( B\) est fini, alors \( A\cup B\approx A\).
\end{lemma}

\begin{proof}
    Nous supposons que \( A\) et \( B\) sont disjoints\footnote{Adaptez la démonstration au cas où l'intersection n'est pas vide.}. La proposition \ref{PROPooJLGKooDCcnWi} nous permet de considérer une bijection \( \psi\colon \{ 1,\ldots, n \}\to B\).
    
    Vu que \( A\) est infini, la proposition \ref{PROPooUIPAooCUEFme} nous permet de considérer \( N\subset A\) et une bijection \( \varphi\colon \eN\to N\).

    Maintenant, il s'agit seulement d'insérer \( B\) dans \( A\) en le mettant «au début» de \( N\) et en décalant les autres. La bijection est
    \begin{equation}
        \begin{aligned}
            f\colon A\cup B&\to A \\
            x&\mapsto \begin{cases}
                x    &   \text{si } x\in A\setminus N\\
                \varphi\big( \varphi^{-1}(x)+n \big)    &   \text{si } x\in N\\
                \varphi\big( \psi^{-1}(x) \big)    &    \text{si }x\in B.
            \end{cases}
        \end{aligned}
    \end{equation}
\end{proof}

\begin{lemma}       \label{LEMooXMVDooIWLWis}
    Si \( A\) est infini et si \( A\) est surpotent à \( B\), alors \( A\approx A\cup B\).
\end{lemma}

\begin{proof}
    Il existe évidemment une injection \( A\to A\cup B\). Donc le théorème de Cantor-Schröder-Bernstein \ref{THOooRYZJooQcjlcl} nous indique que trouver une injection \( A\cup B\to A\) suffira pour la peine.

    Nous allons utiliser le lemme de Zorn \ref{LemUEGjJBc} avec l'ensemble
    \begin{equation}
       \mA=\Big\{  (X,\varphi_X)  \tq
        \begin{cases}
            X\subset B\\
            \varphi_X\colon A\cup X\to A\text{ est injective.}
        \end{cases}
    \Big\}
    \end{equation}
    muni de l'ordre de l'inclusion : \( (X,\varphi_X)<(Y,\varphi_Y)\) si \( X\subset Y\) et \( \varphi_Y(x)=\varphi_X(x)\) pour tout \( x\in A\cup X\).
    
    \begin{subproof}
        \item[\( \mA\) est inductif]
            Soit une famille \( \mF=\{ (X_i,\varphi_i) \}_{i\in I}\) complètement ordonnée indexée par l'ensemble \( I\). En posant \( X=\bigcup_{i\in I}X_i\) et \( \varphi(x)=\varphi_i(x)\) dès que \( x\in A\cup X_i\), l'élément \( (X,\varphi)\) majore \( \mF\).
        \item[Un maximum]
            Le lemme de Zorn nous assure que \( \mA\) possède (au moins) un élément maximum. Soit un tel élément maximum \( (X,\varphi)\). 
        \item[\( X\approx B\)]
            Ah oui, vous auriez aimé avoir \( X=B\). Mais non; il n'y a pas de garanties. Nous allons montrer que \( X\approx B\), et ça suffira.

            Vu que \( X\subset B\), si \( X\) n'est pas équipotent à \( B\), il est strictement inclus à \( B\). Nous pouvons donc considérer
            \begin{subequations}
                \begin{align}
                    b&\in B\setminus X\\
                    a&\in A\setminus \varphi(X).
                \end{align}
            \end{subequations}
            Nous considérons alors l'élément \( (Y,\psi)\in \mA\) défini par
            \begin{subequations}
                \begin{align}
                    Y&=X\cup\{ b \}\\
                    \psi(x)&=\begin{cases}
                        a    &   \text{si } x=b\\
                        \varphi(x)    &    \text{sinon }.
                    \end{cases}
                \end{align}
            \end{subequations}
            Cet élément majore \( (X,\varphi)\).

            Donc \( X\approx B\).
        \item[Résumé de la situation]
            Nous avons \( A\approx A\cup X\) ainsi que une injection \( \varphi\colon A\cup X\to A\) et une bijection \( \psi\colon B\to X\).
        \item[Conclusion si \( A\) est disjoint de \( B\)]
            Si \( A\) et \( B\) sont disjoints, nous avons une bijection
            \begin{equation}
                \begin{aligned}
                    l\colon A\cup B&\to A \\
                    x&\mapsto \begin{cases}
                        \varphi(x)    &   \text{si }  x\in A\\
                        \varphi\big( \psi(x) \big)    &    \text{si } x\in B.
                    \end{cases}
                \end{aligned}
            \end{equation}
        \item[Conclusion si \( A\) n'est pas disjoint de \( B\)]
            Il suffit de poser \( C=B\setminus A\) et avons
            \begin{equation}
                A\cup B=[A\cup (A\cap B)]\cup C.
            \end{equation}
            Cette union est disjointe, \( A\cup(A\cap B)\) est surpotent à \( A\) et \( C\) est subpotent à $B$. La conclusion est donc encore valable.
    \end{subproof}
\end{proof}

La proposition suivante sera utilisée en théorie de la mesure, dans l'exemple~\ref{ExOIXoosScTC}.
\begin{proposition}[\cite{ooFAOQooACUugI,BIBooZFPUooIiywbk}] \label{PropVCSooMzmIX}
    Si \( S\) est un ensemble infini alors il existe une bijection \( \varphi\colon \{ 0,1 \}\times S\to S\).
\end{proposition}
% La preuve de cette proposition a été grandement simplifiée le 6 juin 2021. Ce commentaire sert à retrouver l'ancienne preuve plus
% facilement dans l'historique git. Le premier commit a avoir changé la preuve est celui qui a introduit aussi ce commentaire. ooWADNooUkhTWI

\begin{proof}
    Nous posons \( A=\{ 0 \}\times S\) et \( B=\{ 1 \}\times S\). L'ensemble \( A\) est infini et surpotent à \( B\) (pas strictement, mais quand même).

    Donc \( A\) est idempotent à \( A\cup B\) par le lemme \ref{LEMooUFCAooSyZtZj}. Mais \( A\) est idempotent à \( S\), donc
    \begin{equation}
        S\approx A \approx A\cup B=\{ 0,1 \}\times S.
    \end{equation}
\end{proof}

\begin{corollary}       \label{CORooJCSIooOeOICJ}
    Si \( A\) est un ensemble infini, alors \( A\) possède deux sous-ensembles disjoints \( A_1\) et \( A_2\) qui sont tout deux en bijection avec \( A\).
\end{corollary}

\begin{proof}
    La proposition \ref{PropVCSooMzmIX} donne une bijection \( \varphi\colon \{ 1,2 \}\times A\to A\). Il suffit de poser \( A_1=\varphi(1,A)\) et \( A_2=\varphi(2,A)\).
\end{proof}

Maintenant que nous pouvons mettre dans \( A\) deux copies disjointes de \( A\), il n'est pas très étonnant que nous puissions en mettre une infinité dénombrable. C'est en substance ce que signifie la proposition suivante.
\begin{proposition} \label{PROPooFKBEooKXqujV}
    Si \( A\) est infini, alors \( A\times \eN\approx A\).
\end{proposition}

\begin{proof}
    La démonstration se base sur le fait qu'à l'intérieur de \( A\), nous pouvons construire autant de copies de \( A\) deux à deux disjointes que nous le voulons. La \( k\)\ieme\ «copie» sera naturellement l'image de \( k\times A\).

    Voyons tout cela en détail.
    \begin{subproof}
        \item[Ce que nous allons faire]
            Nous allons construire, pour tout \( i\geq 1 \) des parties \( A_i,B_i\subset A\) telles que
            \begin{itemize}
                \item \( A_i\cap B_i=\emptyset\),
                \item \( A_i,B_i\subset B_{i-1}\),
                \item \( A_i\approx B_i\approx A\),
                \item \( A_i\cap A_j=\emptyset\) si \( i\neq j\)
            \end{itemize}
        \item[La construction]
            Nous commençons à zéro en utilisant le corolaire \ref{CORooJCSIooOeOICJ} pour construire des parties disjointes \( A_0\) et \( B_0\) de \( A\) telles que \( A_0\approx B_0\approx A\).

            Ensuite, vu que \( B_0\approx A\), il existe \( A_1\) et \( B_1\) dans \( B_0\) tels que \(  A_1\cap B_1=\emptyset\) et \( A_1\approx B_1\approx B_0\approx A\). Cela est notre construction pour \( i=1\).

            Pour la récurrence, vu que \( A_i\approx B_i\approx A\), nous considérons \( A_{i+1}\) et \( B_{i+1}\) dans \( B_i\) tels que \( A_{i+1}\cap B_{i+1}=\emptyset\) et \( A_{i+1}\approx B_{i+1}\approx B_i\approx A\). C'est encore le corolaire \ref{CORooJCSIooOeOICJ} qui fait le travail.

        \item[Les propriétés]
            Nous avons \( A_i\cap A_{i+1}=\emptyset\) parce que \( A_i\cap A_{i+1}\subset A_i\cap B_i=\emptyset\).

            Nous devons encore montrer que \( A_i\cap A_j=\emptyset\) dès que \( i\neq j\). Supposons que \( j>i\). Nous avons les inclusions
            \begin{equation}
                A_j\subset B_{j-1}\subset B_{j-2}\subset \ldots \subset B_i.
            \end{equation}
            Donc \( A_j\cap A_i\subset B_i\cap A_i=\emptyset\).
        \item[Une injection]
            Nous pouvons à présent écrire une injection qui termine presque la preuve. Pour cela nous considérons pour tout \( i\), une bijection \( \psi_i\colon A\to A_i\). Ensuite nous posons
            \begin{equation}
                \begin{aligned}
                    \varphi\colon A\times \eN&\to A \\
                    (a,k)&\mapsto \psi_k(a). 
                \end{aligned}
            \end{equation}
                Si \( \varphi(a,k)=\varphi(b,l)\), alors \( \psi_k(a)=\psi_l(b)\). L'élément \( \psi_k(a)\) est donc dans \( A_k\cap A_l\); ce n'est possible que si \( k=l\). Donc \( \psi_l(a)=\psi_l(b)\). Cette dernière égalité n'est possible que si \( a=b\) parce que \( \psi_l\) est une bijection.
            
                Donc \( \varphi\) est une injection, et nous avons prouvé que \( A\times \eN\preceq A\).
            \item[La bijection]
                Nous venons de prouver que \( A\times \eN\preceq A\). La surpotence \( A\times \eN\succeq A\) étant évidente, le théorème de Cantor-Schröder-Bernstein \ref{THOooRYZJooQcjlcl} conclu que \( A\times \eN\approx A\).
    \end{subproof}
\end{proof}

\begin{lemma}[\cite{MonCerveau}]        \label{LEMooDHWSooFqhano}
    Sois un ensemble \( A\) muni de deux sous-ensembles \( B\) et \( B'\) équipotents et disjoints. Alors \( A\setminus B\) est équipotent à \( A\setminus B'\).
\end{lemma}

\begin{proof}
    Soit une bijection \( \psi\colon B'\to B\). L'application
    \begin{equation}
        \begin{aligned}
            \varphi\colon A\setminus B&\to A\setminus B' \\
            x&\mapsto \begin{cases}
                x    &   \text{si } x\notin B'\\
                \psi(x)    &    \text{si } x\in B'.
            \end{cases}
        \end{aligned}
    \end{equation}
    fait la bijection.
\end{proof}

\begin{lemma}[\cite{BIBooYBGLooUZuTrc}]       \label{LEMooIVCBooHWQiZB}
    Si \( A\) est un ensemble infini et si \( B\prec A\), alors \( A\approx A\setminus B\).
\end{lemma}

\begin{proof}
    Nous pouvons écrire
    \begin{equation}
        A=(A\setminus B)\cup B.
    \end{equation}
    Le théorème de comparabilité cardinale \ref{THOooCBSKooCmzfUf} nous indique que soit \( A\setminus B\preceq B\), soit \( A\setminus B\succeq B\). Nous allons voir les deux cas.
    \begin{subproof}
        \item[Si \( A\setminus B\succeq B\)] 
            Dans ce cas, \( (A\setminus B)\cup B\approx A\setminus B\) par le lemme \ref{LEMooXMVDooIWLWis}. Dans ce cas, notre résultat est prouvé parce que \( A=(A\setminus B)\cup B\approx A\setminus B\).
        \item[Si \( A\setminus B\preceq B\)] 
            Dans ce cas, le lemme \ref{LEMooXMVDooIWLWis} nous indique que \( A=(A\setminus B)\cup B\approx B\). Mais \( A\approx B\) est exclu par l'hypothèse. Ce cas est donc impossible.
    \end{subproof}
\end{proof}

\begin{lemma}[\cite{MonCerveau}]        \label{LEMooMRVQooUZSSyL}
    Si \( A\) est infini et si \( B\) est une partie strictement subpotente de \( A\), alors il existe \( U\subset A\) disjoint de \( B\) et équipotent à \( B\).
\end{lemma}

\begin{proof}
    Le lemme \ref{LEMooIVCBooHWQiZB} nous donne une bijection \( \varphi\colon A\to A\setminus B\). Il suffit alors de poser \( U=\varphi(B)\). Cette partie est disjointe de \( B\) parce que \( \varphi\) prend ses valeurs dans \( A\setminus B\).
\end{proof}

\begin{lemma}[\cite{BIBooDLDFooFwXSGV}]
    Soit un ensemble infini \( A\) ainsi qu'un sous-ensemble \( B\subset A\). Nous supposons l'existence d'une fonction surjective \( f\colon B\to B\times B\).

    Alors \( B\preceq B\times B\preceq B\preceq A\).
\end{lemma}

\begin{proof}
    La première est l'hypothèse sur \( f\). La seconde est l'existence (évidente) d'une surjection \( B\times B\to B\). La troisième est le fait que \( B\) soit inclus à \( A\).
\end{proof}

\begin{lemma}[\cite{BIBooDLDFooFwXSGV}]     \label{LEMooPOEFooXaifhT}
    Soit un ensemble infini \( A\) ainsi qu'un sous-ensemble strictement subpotent \( B\subset A\). Nous supposons l'existence d'une fonction surjective \( f\colon B\to B\times B\).

    Alors \( f\) peut être étendue en une injection \( f\colon D\to D\times D\) où \( D\subset A\) contient strictement \( B\).
\end{lemma}

\begin{proof}
    Par le lemme \ref{LEMooMRVQooUZSSyL}, nous considérons une partie \( U\subset A\) disjointe de \( B\) et équipotent à \( B\). Nous pouvons faire le développement
    \begin{equation}
        (B\cup U)\times (B\cup U)=(B\times B)\cup(B\times U)\cup (U\times B)\cup (U\times U).
    \end{equation}
    Nous savons que \( B\) est surpotent à \( U\) (il est même équipotent); donc le lemme \ref{LEMooXMVDooIWLWis} nous dit que \( B\cup U\approx B\). De plus il existe une bijection \( B\to U\), donc 
    \begin{equation}
        B\approx B\times B\approx B\times U\approx U\times B\approx U\times U.
    \end{equation}
    Et enfin, la réunion d'ensembles équipotents est équipotent. Bref, nous avons une bijection
    \begin{equation}
        \varphi\colon U\to (U\times B)\cup (B\times U)\cup (U\times U).
    \end{equation}
    Et enfin nous définissons
    \begin{equation}
        \begin{aligned}
            g\colon B\cup U&\to (B\times U)\times (B\cup U) \\
            x&\mapsto \begin{cases}
                f(x)    &   \text{si }  x\in B\\
                \varphi(x)    &    \text{si } x\in U.
            \end{cases}
        \end{aligned}
    \end{equation}
    Cette définition est bonne parce que \( U\) et \( B\) sont disjoints, et \( g\) est injective.
\end{proof}

Le théorème suivant est une généralisation de la proposition \ref{PropVCSooMzmIX}. Elle implique, entre autres choses, qu'il existe une bijection entre \( \eR\) et \( \eR\times \eR\). Pour le cas de \( \eN\times \eN\approx \eN\), il y a la proposition \ref{PROPooLPKUooAlsYJg} qui donne une bijection explicite et donc sans axiome du choix et sans lemme de Zorn.
\begin{theorem}     \label{THOooDGOVooRdURVi}
    Si \( A\) est infini, alors \( A\approx A\times A\).
\end{theorem}

\begin{proof}
    Une fois de plus, ce sera le lemme de Zorn qui va s'y coller. Soit l'ensemble
    \begin{equation}
       \mA=\Big\{  (X,\varphi)  \tq
        \begin{cases}
            X\subset A\\
            \varphi\colon X\to X\times X\text{ est surjective.}
        \end{cases}
    \Big\}
    \end{equation}
    Cet ensemble est non vide parce que \( A\) est infini; il contient donc une partie dénombrables \( N\), et nous connaissons la surjection \( \eN\to \eN\times \eN\) du lemme \ref{PROPooLPKUooAlsYJg}.

    Nous ordonnons \( \mA\) par l'inclusion : \( (X,\varphi)\leq (Y,\phi)\) lorsque \( X\subset Y\) et \( \phi|_X=\varphi\). La tambouille usuelle montre que \( \mA\) est un ensemble inductif et le lemme de Zorn \ref{LemUEGjJBc} donne l'existence d'un élément maximum que nous notons \( (B,\varphi)\).

    Vu que \( B\) est subpotent à \( A\) (parce qu'il est inclus), soit il est strictement subpotent soit il est équipotent. Nous commençons par montrer que \( B\) ne peut pas être strictement subpotent à \( A\).

    En effet, si nous avions une surjection \( B\to B\times B\), alors que \( B\) est strictement subpotent à \( A\). Le lemme \ref{LEMooPOEFooXaifhT} nous dit alors que \( \varphi\) peut être étendue, ce qui contredirait la maximalité de \( (B,\varphi)\).

    Donc la partie \( B\) est équipotente à \( A\) : il existe une bijection \( g\colon A\to B\). Mais nous avons une surjection \( B\to B\times B\) et donc aussi une injection \( B\times B\to B\). Vu que nous avons par ailleurs une injection \( B\to B\times B\), le théorème de Cantor-Schröder-Bernstein \ref{THOooRYZJooQcjlcl} nous donne une bijection \( \phi\colon B\times B\to B\). Avec ça, l'application
    \begin{equation}
        \begin{aligned}
            f\colon A\times A&\to A \\
            (a,b)&\mapsto \phi\big( g(a),g(b) \big) 
        \end{aligned}
    \end{equation}
    est une bijection. Donc les ensembles \( A\) et \( A\times A\) sont équipotents.
\end{proof}

\begin{lemma}       \label{LEMooNKKDooUvSYPO}
    Si \( A\) est infini, et si pour tout \( i\in \eN\) nous avons \( A_i\approx A\), alors
    \begin{equation}
        \bigcup_{i\in \eN}A_i\approx A.
    \end{equation}
\end{lemma}

\begin{proof}
    Pour chaque \( i\in \eN\) nous avons une bijection \( \varphi_i\colon A_i\to A\). Nous posons
    \begin{equation}        \label{EQooCHJAooRpHypV}
        \begin{aligned}
            \varphi\colon A\times \eN&\to \bigcup_{i=0}^{\infty}A_i \\
            (a,i)&\mapsto \varphi_i(a). 
        \end{aligned}
    \end{equation}
    Cette application est surjective mais peut-être pas injective parce que les \( A_i\) peuvent avoir des intersections non vides. Nous avons alors le calcul
    \begin{subequations}
        \begin{align}
            A&\approx A\times \eN        \label{SUBEQooICFEooTLuFHZ}\\
            &\succeq \bigcup_{i=0}^{\infty}A_i      \label{SUBEQooRVPRooJJevkv}\\
            &\succeq A      \label{SUBEQooFJRGooJnervy}
        \end{align}
    \end{subequations}
    Justifications :
    \begin{itemize}
        \item Pour \eqref{SUBEQooICFEooTLuFHZ}, c'est la proposition \ref{PROPooFKBEooKXqujV}.
        \item Pour \eqref{SUBEQooRVPRooJJevkv}, c'est la surjection \eqref{EQooCHJAooRpHypV}.
        \item Pour \eqref{SUBEQooFJRGooJnervy}, c'est le fait que seulement \( A_0\) possède déjà une surjection vers \( A\).
    \end{itemize}
    Donc \( \bigcup_iA_i\) est à la fois surpotent et subpotent à \( A\). Il est donc équipotent par le théorème \ref{THOooRYZJooQcjlcl}.
\end{proof}

%+++++++++++++++++++++++++++++++++++++++++++++++++++++++++++++++++++++++++++++++++++++++++++++++++++++++++++++++++++++++++++ 
\section{Groupes}
%+++++++++++++++++++++++++++++++++++++++++++++++++++++++++++++++++++++++++++++++++++++++++++++++++++++++++++++++++++++++++++

%---------------------------------------------------------------------------------------------------------------------------
\subsection{Définition, unicité du neutre}
%---------------------------------------------------------------------------------------------------------------------------

\begin{definition}[Groupe]      \label{DEFooBMUZooLAfbeM}
    Un \defe{groupe}{groupe} est un ensemble \( G\) muni d'une opération interne \( \cdot\colon G\times G\to G\) telle que
    \begin{enumerate}
        \item
            pour tous \( g\), \( h\), \( k\in G\), \( g\cdot(h\cdot k)=(g\cdot h)\cdot k\),
        \item
            il existe un élément \( e\in G\) tel que \( e\cdot g=g\cdot e=g\) pour tout \( g\in G\),
        \item
            pour tout \( g\in G\), il existe un élément \( h\in  G\) tel que \(g\cdot h=h\cdot g=e \).
    \end{enumerate}
\end{definition}

    Notons que nous avons écrit \( g\cdot h\) et non \( \cdot(g,h)\) comme une notation purement fonctionnelle nous l'aurait suggéré. Dans les exemples concrets, selon les cas, le loi de groupe appliquée à \( g\) et \( h\) sera notée tantôt \( g+h\), tantôt \( g\cdot h\) ou, le plus souvent pour un groupe générique, simplement \( gh\).

\begin{lemmaDef}[Unicités]  \label{LEMooECDMooCkWxXf}
    L'inverse et le neutre sont uniques, c'est-à-dire :
    \begin{enumerate}
        \item
            il existe un unique élément \( e\in G\) tel que \( e g=g e=g\) pour tout \( g\in G\),
        \item       \label{ITEMooOIWTooYqmMPP}
            pour tout \( g\in G\), il existe un unique élément \( h\in  G\) tel que \(g h=h g=e \).
    \end{enumerate}
    Le \( e\) ainsi défini est nommé \defe{neutre}{neutre!dans un groupe} de \( G\). Le \( h\) tel que \( g h=h g=e\) est nommé l'\defe{inverse}{inverse!dans un groupe} de \( g\) et est noté \( g^{-1}\).
\end{lemmaDef}

\begin{proof}
    Chaque point séparément.
    \begin{enumerate}
        \item
            Supposons que \( e_1\) et \( e_2\) vérifient la propriété. Nous avons pour tout \( g\in G\) : \( e_1g=ge_1=g\). En particulier pour \( g=e_2\) nous écrivons \( e_1e_2=e_2e_1=e_2\). Mais en partant dans l'autre sens : \( e_2g=ge_2=g\) avec \( g=e_1\) nous avons \( e_2e_1=e_1e_2=e_1\). En égalant ces deux valeurs de \( e_2e_1\) nous avons \( e_1=e_2\).

            Pour la suite de la preuve nous écrivons \( e\) l'unique neutre de \( G\).

        \item
            Supposons que \( k_1\) et \( k_2\) soient deux inverses de \( g\). On considère alors le produit \( k_1 g k_2 \). Puisque \(k_1 g = e \), on a \( k_1 g k_2 = e k_2 = k_2 \); mais, comme \(g k_2 = e \), on a aussi \( k_1 g k_2 = k_1 e = k_1 \). Le produit est donc à la fois égal à \( k_1 \) et à \( k_2 \), et donc \( k_1 = k_2 \).
    \end{enumerate}
\end{proof}

\begin{definition}
    Un groupe \( G\) est \defe{abélien}{abélien!groupe} ou \defe{commutatif}{commutatif!groupe} si pour tous \( g\) et \( h\) dans \( G\), \( gh=hg\).
\end{definition}

\begin{lemma}       \label{LEMooBIBFooBHxFYC}
    Si \( G\) est un groupe et si \( h\in G\), alors les applications
    \begin{equation}
        \begin{aligned}
            L_h\colon G&\to G \\
            g&\mapsto hg 
        \end{aligned}
    \end{equation}
    et
    \begin{equation}
        \begin{aligned}
            R_h\colon G&\to G \\
            g&\mapsto gh 
        \end{aligned}
    \end{equation}
    sont des bijections.
\end{lemma}

\begin{proof}
    D'abord si \( L_h(g_1)=L_h(g_2)\), alors \( hg_1=hg_2\) et en multipliant à gauche par \( h^{-1}\) nous avons \( g_1=g_2\); donc \( L_h\) est injective. Ensuite \( L_h\) est surjective parce que si \( g\in G\), alors \( g=L_h(h^{-1} g)\).

    Pour l'application \( R_h\), la preuve est une simple adaptation.
\end{proof}

%--------------------------------------------------------------------------------------------------------------------------- 
\subsection{Permutations, groupe symétrique}
%---------------------------------------------------------------------------------------------------------------------------

Nous donnons ici quelques éléments à propos du groupe symétrique. Beaucoup de choses supplémentaires sont reportées à la section \ref{SECooZFYQooFfopMa}. Voir aussi le thème \ref{THEMEooQEEWooXDhvhv}.

\begin{definition}      \label{DEFooJNPIooMuzIXd}
    Soit un ensemble \( E\). Une \defe{permutation}{permutation} de l'ensemble \( E\) est une bijection \( E\to E\). Le \defe{groupe symétrique}{groupe!symétrique} de \( E\) le groupe des bijections \( E\mapsto E\); il est noté \( S_E\).

    Le \defe{groupe symétrique}{groupe!symétrique} \( S_n\)\nomenclature[R]{\( S_n\)}{le groupe symétrique} est le groupe des permutations de l'ensemble \( \{ 1,\ldots,n \}\). C'est donc l'ensemble des bijections \( \{ 1,\ldots, n \}\to\{ 1,\ldots, n \}\).
\end{definition}


\begin{definition}      \label{DEFooXNAFooGTbTTJ}
    Une \defe{transposition}{transposition} est une permutation\footnote{Une permutation est une bijection, définition \ref{DEFooJNPIooMuzIXd}.} qui inverse deux éléments. Plus précisément, une bijection \( \sigma\colon E\to E\) est une transposition si il existe \( a,b\in E\) tels que
    \begin{equation}
        \sigma(x)=\begin{cases}
              a  &   \text{si } x=b\\
            b    &   \text{si } x=a\\
            x    &    \text{sinon. }
        \end{cases}
    \end{equation}
\end{definition}

\begin{lemma}[\cite{ooJBHPooToyYYI}]        \label{LEMooSGWKooKFIDyT}
    Le groupe symétrique \( S_n\) est un ensemble fini contenant \( n!\) éléments.
\end{lemma}

\begin{lemma}[\cite{ooJFLYooKMbycW}]        \label{LEMooUPBOooWbwMTx}
    Deux résultats.
    \begin{enumerate}
        \item
            Tout groupe est isomorphe à un sous-groupe d'un groupe symétrique.
        \item
            Tout groupe fini d'ordre \( n\) est isomorphe à un sous-groupe de \( S_n\).
    \end{enumerate}
\end{lemma}

\begin{proof}
    Soit, pour \( g\in G\) donné, l'application
    \begin{equation}
        \begin{aligned}
            \tau_g\colon G&\to G \\
            x&\mapsto gx.
        \end{aligned}
    \end{equation}
    Cela est une bijection de \( G\). En effet, d'une part, \( \tau_g(x)=y\) pour \( x=g^{-1} y\) (surjection) et, d'autre part, \( \tau_g(x)=\tau_g(y)\) implique \( gx=gy\) et donc \( x=y\) (injection).

    Nous avons donc \( \tau_g\in S_G\). De plus l'application
    \begin{equation}
        \begin{aligned}
            \varphi\colon G&\to S_G \\
            g&\mapsto \tau_g
        \end{aligned}
    \end{equation}
    est un morphisme de groupe. Il est injectif parce que si \( \tau_g=\tau_h\) alors \( gx=hx\) pour tout \( x\). En particulier \( g=h\).

    Donc \( \varphi\colon G\to \Image(\varphi)\) est un isomorphisme entre \( G\) et un sous-groupe de \( S_G\).

    Un groupe fini de cardinal \( n\) est isomorphe à un sous-groupe de \( S_G\); or \( S_G\) est isomorphe à un des \( S_n\).
\end{proof}
%---------------------------------------------------------------------------------------------------------------------------
\subsection{Décomposition en cycles}
%---------------------------------------------------------------------------------------------------------------------------

\begin{definition}      \label{DEFooSupportPermutation}
    Le \defe{support}{support!d'une permutation} d'une permutation $\sigma$ est l'ensemble constitué des éléments modifiés par $\sigma$:
    \begin{equation*}
        \supp \sigma = \{ i \in \{1,\ldots,n \} \tq \sigma(i) \neq i\}.
    \end{equation*}
\end{definition}

\begin{definition}  \label{DEFooWPYSooPWuwWO}
    Nous disons qu'un élément \( \sigma \in S_n\) \defe{inverse}{inversion!dans le groupe symétrique} les nombres \( i<j\) si \( \sigma(i)>\sigma(j)\). Soit \( N_\sigma\) le nombre d'inversions que \( \sigma\in S_n\) possède (c'est le nombre de couples \( (i,j)\) avec \( i<j\) tels que \( \sigma(i)>\sigma(j)\)). L'entier
    \begin{equation}
        \epsilon(\sigma)=(-1)^{N_\sigma}
    \end{equation}
    est la \defe{signature}{signature!d'une permutation} de \( \sigma\).
\end{definition}

Un \wikipedia{fr}{Permutation}{élément du groupe symétrique} \( S_n\) peut être décomposé en produit de cycles de supports disjoints de la façon suivante. Pour \( \sigma \in S_n \), nous écrivons d'abord le cycle qui correspond à l'orbite de \( 1\). Ce sera le cycle
\begin{equation}
    (1,\sigma(1),\sigma^2(1),\ldots, \sigma^k(1))
\end{equation}
avec \( \sigma^{k+1}(1)=1\). Ensuite nous recommençons avec le plus petit élément de \( \{ 1,\ldots, n \}\) à ne pas être dans ce cycle, et puis le suivant, etc. La \emph{structure} d'une telle décomposition est la donnée des nombres \( k_i\) donnant le nombre de cycles de longueur \( i\).

\begin{lemma}[\cite{Combes}]        \label{LemmvZFWP}
    Soit \( \sigma=(i_1,\ldots, i_k)\in S_n\), un cycle de longueur \( k\) et \( \theta\in S_n\). Alors
    \begin{equation}
        \theta\sigma\theta^{-1}=\big( \theta(i_1),\ldots, \theta(i_k) \big).
    \end{equation}
    Tous les cycles de longueur \( k\) sont conjugués entre eux.
\end{lemma}

\begin{proposition}[Classes de conjugaison et structure en cycles\cite{UXMTXxl}] \label{PropEAHWXwe}
    Une classe de conjugaison dans \( S_n\) est formée des permutations ayant une décomposition en cycles disjoints de même structure. Autrement dit, deux permutations \( \sigma\) et \( \sigma'\) sont conjuguées si et seulement si le nombre \( k_i\) de cycles de longueur \( i\) dans \( \sigma\) est le même que le nombre \( k'_i\) de cycles de longueur \( i\) dans \( \sigma'\).
\end{proposition}

\begin{proof}
    Soit \( \sigma=c_1\ldots c_m\) la décomposition de \( \sigma\) en cycles de supports disjoints. Les \( c_i\) sont des cycles de supports disjoints. Si \( \tau\) est une permutation, alors
    \begin{equation}
        \sigma'=\tau\sigma\tau^{-1}=(\tau c_1\tau^{-1})\ldots (\tau c_m\tau^{-1}),
    \end{equation}
    mais \( \tau c_i\tau^{-1}\) est un cycle de même longueur que \( c\), puisque le lemme~\ref{LemmvZFWP} nous dit que si \( \sigma=(a_1,\ldots, a_k)\), alors \( \tau c\tau^{-1}=\big( \tau(a_1),\ldots, \tau(a_k) \big)\). Notons encore que les cycles \( \tau c_i\tau^{-1}\) restent à support disjoints.

    Donc tous les éléments de la classe de conjugaison de \( \sigma\) sont des permutations de même structure de \( \sigma\).

    Réciproquement, si \( \sigma'=c'_1\ldots c'_m\) est une décomposition de \( \sigma'\) en cycles disjoints tels que la longueur de \( c_i\) est la même que la longueur de \( c'_i\), alors il suffit de construire des permutations \( \tau_i\) telles que \( \tau_i c_i\tau_i^{-1}=c_i'\), à travers le lemme~\ref{LemmvZFWP}. Comme les supports des $c_i$ et des $c'_i$ sont disjoints, la permutation \( \tau_1\ldots \tau_m\) conjugue \( \sigma\) et \( \sigma'\).
\end{proof}

\begin{example}     \label{EXooQAXRooBsPURs}
    Voyons les classes de conjugaison de \( S_3\). Étant donné que ce groupe agit par définition sur un ensemble à \( 3\) éléments, aucun élément de \( S_3\) ne possède un cycle de plus de \( 3\) éléments. Il y a donc seulement des cycles de longueur deux ou trois (à part les triviaux). Aucun élément de \( S_3\) n'a une décomposition en cycles disjoints contenant deux cycles de deux ou un cycle de deux et un de trois.

    En résumé il y a trois classes de conjugaison dans \( S_3\). La première est celle contenant seulement l'identité. La seconde est celle contenant les cycles de longueur deux et la troisième contient les cycles de longueur \( 3\).

    Ce sont donc
    \begin{subequations}
        \begin{align}
            C_1&=\{ \id \}\\
            C_2&=\{ (1,2),(1,3),(2,3) \}\\
            C_3&=\{ (1,2,3),(2,1,3) \}.
        \end{align}
    \end{subequations}
\end{example}

\begin{example} \label{ExVYZPzub}
    Les classes de conjugaison de \( S_4\). Nous savons que les classes de conjugaison dans \( S_4\) sont caractérisées par la structure des décompositions en cycles (proposition~\ref{PropEAHWXwe}). Le groupe symétrique \( S_4\) possède dont les classes de conjugaison suivantes.
\begin{enumerate}
    \item
        Le cycle vide qui représente la classe constituée de l'identité seule.
    \item
        Les transpositions (de type \( (a,b)\)) qui sont au nombre de \( 6\).
    \item
        Les \( 3\)-cycles. Pour savoir \href{http://www.toujourspret.com/techniques/expression/chants/C/cantique_des_etoiles.php}{quel est leur nombre} nous commençons par remarquer qu'il y a \( 4\) façons de prendre \( 3\) nombres parmi \( 4\) et ensuite \( 2\) façons de les arranger. Il y a donc \( 8\) éléments dans cette classe de conjugaison.
    \item
        Les \( 4\)-cycles. Le premier est arbitraire (parce que c'est cyclique). Pour le second il y a \( 3\) possibilités, et deux possibilités pour le troisième; le quatrième est alors automatique. Cette classe de conjugaison contient donc \( 6\) éléments.
    \item       \label{ITEMooGCMYooKZgFHX}
        Les doubles transpositions, du type \( (a,b)(c,d)\). Dans ce cas, tous les nombres sont permutés, et l'image de $1$ détermine la double transposition. Il y a $3$ images possibles, et donc \( 3\) éléments dans cette classe.
\end{enumerate}
\index{classe de conjugaison!dans $S_4$ }
\end{example}

\begin{proposition} \label{PropPWIJbu}
    Tout élément de \( S_n\) peut être écrit sous la forme d'un produit fini de transpositions.
\end{proposition}
Cette décomposition n'est pas à confondre avec celle en cycles de support disjoints. Par exemple \( (1,2,3)=(1,3)(1,2)\).

\begin{propositionDef}\label{PROPooKRHEooAxtmRv}
    Si une permutation peut être écrire sous forme d'un produit d'un nombre pair de permutations, alors toute décomposition en permutations sera en quantité paire.

    Une telle permutation est une \defe{permutation paire}{permutation paire}.
\end{propositionDef}

\begin{lemma}[\cite{PDFpersoWanadoo}]       \label{LemhxnkMf}
    Un \( k\)-cycle est une permutation impaire si \( k\) est pair et paire si \( k\) est impair.
\end{lemma}

\begin{proposition}[\cite{Combes}]  \label{ProphIuJrC}
    Soit \( S_n\) le groupe symétrique.
    \begin{enumerate}
        \item       \label{ITEMooBQKUooFTkvSu}
            L'application \( \epsilon\colon S_n\to \{ 1,-1 \}\) est l'unique homomorphisme surjectif de \( S_n\) sur \( \{ -1,1 \}\).
        \item
            Si \( s=t_1\cdots t_k\) est le produit de \( k\) transpositions, alors \( \epsilon(s)=(-1)^k\).
    \end{enumerate}
\end{proposition}

\begin{proof}
    Soit \( \sigma,\theta \in S_n\). Afin de montrer que \( \epsilon(\sigma\theta )=\epsilon(\sigma)\epsilon(\theta )\), nous divisons les couples \( (i,j)\) tels que \( i\leq j\) en \( 4\) groupes suivant que \( \theta(i)\gtrless \theta(j)\) et \( \sigma\big( \theta(i) \big)\gtrless \sigma\big( \theta(j) \big)\). Nous notons \( N_1\), \( N_2\), \( N_3\) et \( N_4\) le nombre de couples dans chacun des quatre groupes :
    \begin{center}
    \begin{tabular}{c|c|c}
        $ (i,j)$&   \(\sigma\big( \theta(i) \big)<\sigma\big( \theta(j) \big)\)    &   \(\sigma\big( \theta(i) \big)>\sigma\big( \theta(j) \big)\)\\
        \hline
        \( \theta(i)<\theta(j)\)& \( N_1\)&\( N_2\)\\
        \hline
        \( \theta(i)>\theta(j)\)&\( N_3\)&\( N_4\)
    \end{tabular}
    \end{center}
    Nous avons immédiatement \( N_\theta=N_3+N_4\) et \( N_{\sigma\theta}=N_2+N_4\). Les éléments qui participent à \( N_\sigma\) sont ceux où \( \theta(i)\) et \(\theta(j)\) sont dans l'ordre inverse de \( \sigma\big( \theta(i) \big)\) et \( \sigma\big( \theta(j) \big)\) (parce que \(  \theta\) est une bijection). Donc \( N_\sigma=N_2+N_3\). Par conséquent nous avons
    \begin{equation}
        \epsilon(\sigma)\epsilon(\theta)=(-1)^{N_2+N_3}(-1)^{N_3+N_4}=(-1)^{N_2+N_4}=(-1)^{N_{\sigma\theta}}=\epsilon(\sigma\theta).
    \end{equation}
    Nous avons prouvé que \( \epsilon\) est un homomorphisme. Pour montrer que \( \epsilon\) est surjectif sur \( \{ -1,1 \}\) nous devons trouver un élément \( \tau\in S_n\) tel que \( \epsilon(\tau)=-1\). Si \( \tau\) est la transposition \( 1\leftrightarrow 2\) alors le couple \( (1,2)\) est le seul à être inversé par \( \tau\) et nous avons \( \epsilon(\tau)=-1\).

    Avant de montrer l'unicité, nous montrons que si \( \sigma=t_1\ldots t_k\) alors \( \epsilon(\sigma)=(-1)^k\). Pour cela il faut montrer que \( \epsilon(\tau)=-1\) dès que \( \tau\) est une transposition. Soit \( \tau_{ij}\), la transposition \( (i,j)\) et \( \theta=(i,i+1,\ldots, j-1)\) alors le lemme~\ref{LemmvZFWP} dit que
    \begin{equation}
        \tau_{ij}=\theta\tau_{j-1,j}\theta^{-1}.
    \end{equation}
    La signature étant un homomorphisme,
    \begin{equation}
        \epsilon(\tau_{ij})=\epsilon(\theta)\epsilon(\tau_{j-1,j})\epsilon(\theta)^{-1}=\epsilon(\tau_{j-1,j})=-1.
    \end{equation}

    Nous passons maintenant à la partie unicité de la proposition. Soit un homomorphisme surjectif \( \varphi\colon S_n\to \{ -1,1 \}\) et \( \tau\), une transposition telle que \( \varphi(\tau)=-1\) (qui existe parce que sinon \( \varphi\) ne serait pas surjectif\footnote{Nous utilisons ici le fait que tous les éléments de \( S_n\) sont des produits de transpositions, proposition~\ref{PropPWIJbu}.}). Si \( \tau'\) est une autre transposition, il existe \( \sigma\in S_n\) tel que \( \tau'=\sigma\tau\sigma^{-1}\) (lemme~\ref{LemmvZFWP}). Dans ce cas, \( \varphi(\tau')=\varphi(\tau)=-1\), et si \( \sigma=\tau_1\ldots \tau_k) \),
    \begin{equation}
         \varphi(\sigma)=(-1)^k=\epsilon(\sigma).
    \end{equation}
\end{proof}

\begin{corollary}       \label{CORooZLUKooBOhUPG}
    Si \( \sigma\in S_n\), alors
    \begin{equation}
        \epsilon(\sigma)=\epsilon(\sigma^{-1}).
    \end{equation}
\end{corollary}

\begin{proof}
    Comme dit par la proposition \ref{ProphIuJrC}, \( \epsilon\) est un homomorphisme, donc
    \begin{equation}
        \epsilon(\sigma)\epsilon(\sigma^{-1})=\epsilon(\sigma\sigma^{-1})=\epsilon(\id)=1.
    \end{equation}
    Vu que \( \epsilon(\sigma)\) et \( \epsilon(\sigma^{-1})\) ne peuvent être que \( \pm1\), ils doivent être tous les deux \( 1\) ou tous les deux \( -1\) pour que le produit soit \( 1\).
\end{proof}

%+++++++++++++++++++++++++++++++++++++++++++++++++++++++++++++++++++++++++++++++++++++++++++++++++++++++++++++++++++++++++++ 
\section{Anneaux}
%+++++++++++++++++++++++++++++++++++++++++++++++++++++++++++++++++++++++++++++++++++++++++++++++++++++++++++++++++++++++++++

\begin{definition}[Anneau\cite{Tauvel}]     \label{DefHXJUooKoovob}
    Un \defe{anneau}{anneau}\footnote{Nous faisons le choix qu'un anneau admet toujours un neutre pour la multiplication. Certains ouvrages parlent dans ce cas d'anneau unitaire.} est un triplet \( (A,+,\cdot)\) avec les conditions
    \begin{enumerate}
        \item
            \( (A,+)\) est un groupe abélien. Nous notons \( 0\) le neutre.
        \item
            La multiplication est associative et nous notons \( 1\) le neutre.
        \item       \label{ITEMooGMNOooSTGiXw}
            La multiplication est distributive par rapport à l'addition.
    \end{enumerate}
    L'anneau \( (A,+,\cdot)\) est \defe{commutatif}{anneau commutatif} si pour tout \( a,b\in A\) nous avons \( a\cdot b=b\cdot a\).
\end{definition}

\begin{definition}[Morphisme d'anneaux\cite{ooZRUJooXyxPqQ}]      \label{DEFooSPHPooCwjzuz}
    Si \( (A,+,\cdot)\) et \( (B,+,\cdot)\) sont des anneaux, un \defe{morphisme d'anneaux}{morphisme!d'anneaux} est une application \( f\colon A\to B\) telle que
    \begin{enumerate}
        \item \( f(a+b)=f(a)+f(b)\)
        \item
            \( f(a\cdot b)=f(a)\cdot f(b)\)
        \item
            \( f(1)=1\).
    \end{enumerate}
    Étant bien entendu que les significations de \( 1\), $+$ et \( \cdot\) sont différentes à gauche et à droite.
\end{definition}

\begin{definition}      \label{DEFooKWKGooIOwGTA}
    Un \defe{isomorphisme d'anneaux}{isomorphisme!d'anneaux} est un morphisme bijectif.
\end{definition}

La distributivité de la partie \ref{ITEMooGMNOooSTGiXw} de la définition \ref{DefHXJUooKoovob} ne traite que de l'addition; pas de la soustraction. Voici une lemme qui dit que ça fonctionne quand même.
\begin{lemma}[\cite{BIBooZFPUooIiywbk}]     \label{LEMooVPYUooRzexke}
    Soient un anneau \( A\) ainsi que \( a,b,c\in A\). Alors
    \begin{equation}
        a(b-c)=ab-ac.
    \end{equation}
\end{lemma}

\begin{proof}
    Nous avons le calcul suivant :
    \begin{subequations}
        \begin{align}
            a(b-c)+ac&=a\big( (b-c)+c \big)     \label{SUBEQooKCOWooFeOHUM}\\
            &=ab.       \label{SUBEQooMLLOooNRmIYM}
        \end{align}
    \end{subequations}
    Justifications :
    \begin{itemize}
        \item Pour \ref{SUBEQooKCOWooFeOHUM}. Distributivité.
        \item Pour \ref{SUBEQooMLLOooNRmIYM}. Parce que \( (b-c)+c=b\).
    \end{itemize}
    Nous avons donc \( a(b-c)+ac=ab\) et donc l'égalité demandée en ajoutant \( -ac\) des deux côtés.
\end{proof}

\begin{lemma}       \label{LEMooVUSMooWisQpD}
    Pour tout élément \( a\) d'un anneau nous avons \( a\cdot 0=0\).
\end{lemma}

\begin{proof}
    L'élément \( 0\) est le neutre de l'addition. Il peut être écrit \( 1-1\), et en utilisant la distributivité sous la forme du lemme \ref{LEMooVPYUooRzexke},
    \begin{equation}
        a\cdot 0=a\cdot (1-1)=a-a=0.
    \end{equation}
    Notons que la dernière égalité s'écrit en détail \( a-a=a+(-a)\) qui donne le neutre de l'addition.
\end{proof}

\begin{proposition}     \label{PROPooNCCGooXjVyVt}
    Dans un anneau\footnote{Définition \ref{DefHXJUooKoovob}.} non nul, le neutre pour l'addition est distinct du neutre pour la multiplication.
\end{proposition}
\begin{proof}
    Supposons par contraposée que dans un anneau $A$, \( 1 = 0 \). Alors, pour tout \( a \in A \), on a \( a = 1a = 0a = (1 - 1)a = a - a=0 \), d'où l'on déduit \( -a = 0  \) et par suite, \( a = 0. \)
\end{proof}

\begin{lemma}[\cite{MonCerveau}]        \label{LEMooLTERooVKgqjn}
    Un peu d'arithmétique. Soit un anneau \( A\) et un élément \( a\in A\).
    \begin{enumerate}
        \item       \label{ITEMooUGHCooOPgoeR}
            \( 1\times 1=1\).
        \item       \label{ITEMooJMBSooVgvVwg}
            \( (-1)\times a=-a\).
        \item       \label{ITEMooXJGMooKNLlHU}
            \( -(-a)=a\).
        \item       \label{ITEMooYMRKooHVYYKU}
            \( (-1)\times (-1)=1\).
    \end{enumerate}
\end{lemma}

\begin{proof}
    En plusieurs parties.
    \begin{subproof}
    \item[Pour \ref{ITEMooUGHCooOPgoeR}]
        La définition de \( 1\) est que \( 1\times a=a\) pour tout \( a\). En particulier pour \( a=1\) nous avons le résultat.
    \item[Pour \ref{ITEMooJMBSooVgvVwg}]
    Nous avons
    \begin{equation}
        (-1)\times a + a= a\times \big( (-1)+1 \big)=a\times 0=0.
    \end{equation}
    Nous avons utilisé le fait que la multiplication était distributive et que le zéro était absorbant (lemme \ref{LEMooVUSMooWisQpD}).

\item[Pour \ref{ITEMooXJGMooKNLlHU}]
    Nous avons \( -a+a=0\) par définition de la notation \( -a\). Donc \( a\) est bien l'inverse de \( -a\) pour l'addition.

\item[Pour \ref{ITEMooYMRKooHVYYKU}]
    En utilisant les points \ref{ITEMooJMBSooVgvVwg} et \ref{ITEMooXJGMooKNLlHU} nous avons
    \begin{equation}
        (-1)\times (-1)=-(-1)=1.
    \end{equation}
    \end{subproof}
\end{proof}

Soit \( X\) un ensemble et un anneau $(A, +, \times)$. Nous considérons \( \Fun(X,A)\)\nomenclature[A]{\( \Fun(X,Y)\)}{les applications de \( X\) vers \( Y\)} l'ensemble des applications \( X\to A\). Cet ensemble devient un anneau avec les définitions
\begin{subequations}
    \begin{align}
        (f+g)(x)=f(x)+g(x)\\
        (fg)(x)=f(x)g(x).
    \end{align}
\end{subequations}
Cela est la \defe{structure canonique}{structure d'anneau canonique} d'anneau sur \( \Fun(X,A)\).

\begin{definition}
    Le \defe{centralisateur}{centralisateur} de \( x\in A\) dans \( A\) est l'ensemble
    \begin{equation}
        \{ y\in A\tq xy=yx \},
    \end{equation}
    le \defe{centre}{centre!d'un anneau} de \( A\) est
    \begin{equation}
        \{ y\in A\tq xy=yx,\forall x\in A \}.
    \end{equation}
\end{definition}

\begin{definition}[Idéal dans un anneau]  \label{DefooQULAooREUIU}
    Un sous-ensemble \( I\subset A\) est un \defe{idéal à gauche}{idéal!dans un anneau} à gauche si
    \begin{enumerate}
        \item
            \( I\) est un sous-groupe pour l'addition,
        \item
            pour tout \( a\in A\), \( aI\subset I\).
    \end{enumerate}

    Lorsqu'un ensemble est idéal à gauche et à droite, nous disons que c'est un \defe{idéal bilatère}{idéal!bilatère}. Lorsque nous parlons d'idéal sans précisions, nous parlons d'idéal bilatère.
\end{definition}

\begin{propositionDef}      \label{PROPooGXMRooTcUGbi}
    Soit \( A\), un anneau, \( I\) un idéal bilatère\footnote{Définition~\ref{DefooQULAooREUIU}.} de \( A\). Nous considérons la relation d'équivalence \( x\sim y\) si et seulement si \( x-y\in I\). Sur le quotient\footnote{Définition \ref{DEFooRHPSooHKBZXl}.}
    \begin{equation}
        A/\sim=A/I,
    \end{equation}
    nous mettons les opérations
    \begin{enumerate}
        \item
            \( [x]+[y]=[x+y]\)
        \item
            \( [x][y]=[xy]\).
    \end{enumerate}
    Nous avons alors les résultats suivants :
    \begin{enumerate}
        \item       \label{ITEMooEJPEooRKAqmS}
            Les opérations sont bien définies,
        \item       \label{ITEMooYBEGooTlHgNz}
            l'ensemble \( A/I\), muni de ces opérations, est un anneau
        \item       \label{ITEMooLNRLooMkoWXZ}
            la surjection canonique \( \pi\colon A\to A/I\) est un morphisme.
    \end{enumerate}
    Cet anneau est appelé \defe{anneau quotient}{anneau!quotient par un idéal}.
\end{propositionDef}

\begin{proof}
    En plusieurs parties.
    \begin{subproof}
        \item[Pour \ref{ITEMooEJPEooRKAqmS}]
            Nous savons que, par définition,
            \begin{equation}
                \bar x=\{ x+i\tq i\in I \}.
            \end{equation}
            Calculons le produit de représentants génériques de \( \bar x\) et de \( \bar y\) :
            \begin{equation}
                (x+i_1)(y+i_2)=xy+xi_2+yi_1+i_1i_2.
            \end{equation}
            Vu que \( I\) est un idéal nous avons \( xi_2+yi_1+i_1i_2\in I\) et donc bien
            \begin{equation}
                (x+i_1)(y+i_2)\in \overline{ xy }.
            \end{equation}
        \item[Pour \ref{ITEMooYBEGooTlHgNz}]
            Il s'agit de vérifier les conditions de la définition \ref{DefHXJUooKoovob}.

            Vu que \( (A,+)\) est un groupe abélien de neutre \( 0\), nous avons
            \begin{equation}
                [a]+[b]=[a+b]=[b+a]=[b+a],
            \end{equation}
            ainsi que
            \begin{equation}
                [a]+[0]=[a+0]=[a]
            \end{equation}
            et
            \begin{equation}
                [a]+([b]+[c])=[a]+[b+c]=[a+b+c]=[a+b]+[c].
            \end{equation}
            Donc \( (A/I,+)\) est un groupe abélien de neutre \( [0]\).

            L'associativité de \( A\) donne l'associativité dans \( A/I\) :
            \begin{equation}
                \big( [a][b] \big)[c]=[ab][c]=[abc]=[a][bc]=[a]\big( [b][c] \big).
            \end{equation}
            Et enfin pour la distributivité,
            \begin{equation}
                [a]\big( [b]+[c] \big)=[a][b+c]=[a(b+c)]=[ab+ac]=[ab]+[ac]=[a][b]+[a][c].
            \end{equation}
            Nous avons prouvé que \( A/I\) est un anneau de neutre \( [0]\) et d'unité \( [1]\).
        \item[Pour \ref{ITEMooLNRLooMkoWXZ}]
            Nous devons vérifier les trois conditions de la définition \ref{DEFooSPHPooCwjzuz}. Cela est immédiat parce que \( \pi(x)=[x]\).
    \end{subproof}
\end{proof}


\begin{definition}\label{DefrYwbct}
    Soient \( A\) un anneau commutatif et \( S\subset A\). Nous disons que \( \delta\in A\) est un \defe{PGCD}{pgcd!dans un anneau intègre} de \( S\) si
    \begin{enumerate}
        \item
            \( \delta\) divise tous les éléments de \( S\).
        \item
            si \( d\) divise également tous les éléments de \( S\), alors \( d\) divise \( \delta\).
    \end{enumerate}
    Nous disons que \( \mu\in A\) est un \defe{PPCM}{ppcm!dans un anneau intègre} de \( S\) si
    \begin{enumerate}
        \item
            \( S\divides \mu\),
        \item
            si \( S\divides m\), alors \( \mu\divides m\).
    \end{enumerate}
\end{definition}

\begin{remark}
    Au sens de la définition \ref{DefrYwbct}, le pgcd n'est pas unique. Dans \( \eZ\) par exemple les nombres \( 4\) et \( -4\) sont tout deux pgcd de \( \{4,16  \}\).

    Dans \( \eZ\) cependant, nous modifions implicitement la définition et nous n'acceptons que les positifs, de telle sorte à ce que l'unique pgcd soit effectivement le plus grand pour l'ordre usuel sur \( \eZ\).

    Pour l'unicité dans \( \eZ\), voir \ref{LEMooBJVJooFyuFeN}.
\end{remark}

%---------------------------------------------------------------------------------------------------------------------------
\subsection{Anneau intègre}
%---------------------------------------------------------------------------------------------------------------------------

\begin{definition}[Diviseurs dans un anneau]\label{DiviseursAnneau}
	Soient \( a, b \in A \). On dit que $a$ divise $b$, ou que $a$ est un \defe{diviseur (à gauche)}{diviseur!dans un anneau} de $b$ s'il existe \( c \in A \) tel que \( ac = b \). On dit que c'est un diviseur de $b$ à droite si \( ca = b \) pour un certain \( c \in A \).
\end{definition}
Un cas particulier est le cas des diviseurs de zéro. L'absence de tels diviseurs dans un anneau est une propriété intéressante: on dit dans ce cas que l'anneau est intègre. Nous étudions ces anneaux plus en détail en section~\ref{SECAnneauxIntegres}.

Un élément \( a\in A\) est \defe{régulier à droite}{régulier à droite} si \( ba=0\) implique \( b=0\). Il est régulier à gauche si \( ab=0\) implique \( b=0\).

\begin{definition}[Éléments nilpotents, unipotents et inversibles]
	On dit que \( a \in A \) est \defe{nilpotent}{nilpotent} s'il existe \( n \in \eN \) tel que \( a^n = 0 \). Il est dit \defe{unipotent}{unipotent} si \( a-1\) est nilpotent, c'est-à-dire si \( (a-1)^n =0\) pour un certain \( n \in \eN \).

	Un élément \( a \in A \) est dit \defe{inversible}{élément!inversible!dans un anneau} s'il existe \( b \in A \) tel que \( ab = 1 \).
\end{definition}

L'ensemble \( U(A)\)\nomenclature[A]{\( U(A)\)}{ensemble des inversibles} des éléments inversibles de \( A\) est un groupe pour la multiplication. Nous notons \( A^*=A\setminus\{ 0 \}\).

Conformément à la définition \ref{DiviseursAnneau} de diviseur, nous posons la définition suivante pour les diviseurs de zéro.
\begin{definition}[\cite{ooTNKJooSCSCZQ}]
    Un élément \( a\neq 0\) est un \defe{diviseur de zéro à gauche}{diviseur!de zéro} s'il existe \( x\neq 0\) tel que $xa=0$. L'élément \( a\) est un diviseur de zéro \defe{à droite}{diviseur!de zéro à droite} s'il existe \( y\neq 0\) tel que \( ay=0\).

    Nous disons que \( a\) est un \defe{diviseur de zéro}{diviseur de zéro} si il est un diviseur de zéro à gauche ou à droite.
\end{definition}

\begin{propositionDef}[Anneau intègre\cite{MonCerveau}]           \label{DEFooTAOPooWDPYmd}
    Soit $A$ un anneau non réduit à \( \{ 0 \}\). Les assertions suivantes sont équivalentes:
    \begin{enumerate}
        \item       \label{ITEMooMXMKooXMYpkN}
            \( A\) ne possède pas de diviseurs de zéro.
        \item       \label{ITEMooLAJCooFwxXrV}
            La règle du produit nul s'applique dans $A$: pour tous \( a, b \in A \), si \( ab=0\), alors \( a = 0\) ou \( b = 0\).
            \index{règle du produit nul}
        \item       \label{ITEMooQNTFooSRrVPK}
            On peut simplifier par un même élément non-nul, deux expressions produit dans $A$ qui sont égales: pour tous \( a, b, c \in A \) avec \( a \neq 0 \), si \( ab = ac \), alors \( b = c \).
    \end{enumerate}
    Un anneau non réduit à \( \{ 0 \}\) qui vérifie ces propriétés est dit \defe{intègre}{anneau intègre}.
\end{propositionDef}

\begin{proof}
    En trois implications.
    \begin{subproof}
        \item[\ref{ITEMooMXMKooXMYpkN} implique \ref{ITEMooLAJCooFwxXrV}]

            Si \( ab=0\) avec \( a,b\neq 0\) alors \( a\) est un diviseur de zéro. Vu que nous supposons que \( A\) n'a pas de diviseurs de zéros, soit \( a\) soit \( b\) est nul.
        \item[\ref{ITEMooLAJCooFwxXrV} implique \ref{ITEMooQNTFooSRrVPK}]

            Si \( ab=ac\), alors \( a(b-c)=0\) et l'hypothèse dit que soit \( a=0\), soit \( b-c=0\). Donc si \( a\neq 0\), alors \( b-c=0\).
        \item[\ref{ITEMooQNTFooSRrVPK} implique \ref{ITEMooMXMKooXMYpkN}]
            Si \( A=\{ 0 \}\), le point \ref{ITEMooQNTFooSRrVPK} n'est pas applicable.

            Si \( a\neq 0\) et \( xa=0\), alors nous avons aussi \( xa=0\times a\). Par propriété de simplification, \( x=0\). Donc \( a\) n'est pas un diviseur de zéro à gauche. Nous prouvons de la même façon qu'il n'y a pas de diviseurs de zéro à droite.
    \end{subproof}
\end{proof}


%--------------------------------------------------------------------------------------------------------------------------- 
\subsection{Fonction puissance}
%---------------------------------------------------------------------------------------------------------------------------

Voici une première définition de la fonction puissance. Il y en aura d'autres, de plus en plus générales. Voir le thème \ref{THEMEooBSBLooWcaQnR}.
\begin{definition}\label{DEFooGVSFooFVLtNo}
    Si \( A\) est un anneau, si \( a\in A\) et si \( n\in \eN\), nous définissons \( a^n\) par récurrence :
    \begin{enumerate}
        \item
            \( a^0=1\) (l'unité pour la multiplication dans \( A\)),
        \item       \label{ITEMooOUIPooGjAgQb}
            \( a^{k+1}=a\cdot a^{k}\).
    \end{enumerate}
\end{definition}

Le lemme suivant dit que le point \ref{ITEMooOUIPooGjAgQb} de la définition \ref{DEFooGVSFooFVLtNo} aurait pu être écrit \( a^k\cdot a\) au lieu de \( a\cdot a^k\).
\begin{lemma}[\cite{MonCerveau}]        \label{LEMooWPARooYLZlzr}
    Si \( A\) est un anneau, si \( a\in A\) et si \( n\in \eN\), alors
    \begin{equation}
        a^n=a\cdot a^{n-1}=a^{n-1}\cdot a.
    \end{equation}
\end{lemma}

\begin{proof}
    Cela se prouve par récurrence. Pour \( n=1\) c'est l'égalité \( a=a^0a\) qui est correcte parce que par définition \( a^0=1\).

    Supposons que le résultat soit bon pour \( n\) et voyons ce que ça donne pour \( n+1\) :
    \begin{subequations}
        \begin{align}
            a^{n+1}&=aa^n      &\text{Définition de } a^{n+1}     \\
            &=a(a^{n-1}a)       &\text{hypothèse de récurrence pour } a^n\\
            &=(aa^{n-1})a       &\text{associativité}\\
            &=a^na          &\text{Définition de } a^n.
        \end{align}
    \end{subequations}
\end{proof}

%+++++++++++++++++++++++++++++++++++++++++++++++++++++++++++++++++++++++++++++++++++++++++++++++++++++++++++++++++++++++++++
\section{Le groupe et anneau des entiers}
%+++++++++++++++++++++++++++++++++++++++++++++++++++++++++++++++++++++++++++++++++++++++++++++++++++++++++++++++++++++++++++

Certes \( (\eZ,+)\) est un groupe mais en ajoutant la multiplication, \( (\eZ,+,\times)\) devient un anneau\footnote{Définition~\ref{DefHXJUooKoovob}.} 

%---------------------------------------------------------------------------------------------------------------------------
\subsection{Division euclidienne}
%---------------------------------------------------------------------------------------------------------------------------

\begin{theoremDef}[Division euclidienne\cite{ooRBKHooQJqglH}]     \label{ThoDivisEuclide}
    Soient \( a\in\eZ\) et \( b\in\eN^*\). Il existe un unique couple \( (q,r)\in\eZ\times\eN\), avec \( 0\leq r<b\), tel que
    \begin{equation}
        a=bq+r.
    \end{equation}
    L'opération \( (a,b)\mapsto(q,r)\) ainsi définie est la \defe{division euclidienne}{division!euclidienne}. Le nombre \( q\) est le \defe{quotient}{quotient} et \( r\) est le \defe{reste}{reste} de la division de \( a\) par \( b\).
\end{theoremDef}

\begin{proof}
    Remarquons que \( r = a - bq \), et donc, une fois l'existence et l'unicité de $q$ établie, celle de $r$ suivra.

    \begin{subproof}
        \item[Unicité]
            Nous supposons avoir \( (q,r)\in \eZ\times \eN\) tels que
            \begin{subequations}
                \begin{numcases}{}
                    0\leq r<b\\
                    a=qb+r.
                \end{numcases}
            \end{subequations}
            Ce système implique que
            \begin{equation}
                0\leq a-qb<b.
            \end{equation}
            En ajoutant \( qb\) dans les trois membres de cette inégalité,
            \begin{equation}
                qb\leq a<(q+1)b.
            \end{equation}
            Cela implique que
            \begin{equation}
                q=\max\{ k\in \eZ\tq kb\leq a \}.
            \end{equation}
            Donc \( q\) est unique et la relation \( a=bq+r\) implique que \( r\) est également unique.
    \begin{equation*}
        E = \{ q \in \eZ  | bq \leq a \}.
    \end{equation*}
    C'est un sous-ensemble d'entiers non-vide (il contient \( -|a| \) ) et admet \( |a| \) comme majorant; il admet donc un maximum $q$ par le lemme \ref{LEMooMYEIooNFwNVI}. Ce maximum vérifie
     \begin{equation}
         bq\leq a<b(q+1).
     \end{equation}
     Cela donne \( 0\leq a-bq<b\) et le résultat en posant \( r=a-qb\).
    \end{subproof}
\end{proof}


% TODO : À propos de restes, il n'est peut-être pas mal de parler d'algorithme de calcul de la date de pâques.
% L'algorithme de Gauss, Meeus utilise des arrondis.
% http://fr.wikipedia.org/wiki/Calcul_de_la_date_de_Pâques


%---------------------------------------------------------------------------------------------------------------------------
\subsection{PGCD, PPCM et Bézout}
%---------------------------------------------------------------------------------------------------------------------------

Vu que \( \eZ\) est un anneau intègre, nous avons la définition \ref{DefrYwbct} de pgcd et de ppcm.
\begin{proposition}[PPCM et PGCD]       \label{PROPooAVRGooUfhjwF}
    Soient \( p,q\in\eZ^*\). 
    \begin{enumerate}
        \item
            Le pgcd de \( p\) et \( q\) est le plus grand diviseur commun de \( p\) et \( q\). 
        \item
            Le ppcm de \( p\) et \( q\) est leur plus petit multiple commun.
    \end{enumerate}
\end{proposition}

\begin{proof}
    Démontrons le premier point. Notons \( \delta\) le pgcd de \( p\) et \( q\). Si \( d\) est un diviseur commun de \( p\) et \( q\), alors \( d\) divise \( \delta\). Dans \( \eZ\), \( d\divides \delta\) implique \( d\leq\delta\) (proposition \ref{PROPooYJBMooZrzkNX}).
\end{proof}

\begin{lemma}
    Soient \( p,q\in\eZ^*\). Les entiers \( \ppcm(p,q)\) et \( \pgcd(p,q)\) fournissent les isomorphismes de groupes suivants :
\begin{subequations}
    \begin{align}
        p\eZ\cap q\eZ&=\ppcm(p,q)\eZ\\
        p\eZ + q\eZ&=\pgcd(p,q)\eZ.
    \end{align}
\end{subequations}
\end{lemma}

\begin{definition}  \label{DefZHRXooNeWIcB}
    Si \( \pgcd(p,q)=1\), nous disons que \( p\) et \( q\) sont \defe{premiers entre eux}{nombre!premier!deux nombres entre eux}. Si nous avons un ensemble d'entiers \( a_i\), nous disons qu'ils sont premiers \defe{dans leur ensemble}{nombre!premier!dans leur ensemble} si \( 1\) est le PGCD de tous les \( a_i\) ensemble.
\end{definition}

Les nombres \( 2\), \( 4\) et \( 7\) ne sont pas premiers deux à deux (à cause de \( 2\) et \( 4\)), mais ils sont premiers dans leur ensemble parce qu'il n'y a pas de diviseurs communs à tout le monde.

\begin{theorem}[Théorème de Bézout\footnote{Il y a une super application ici : \url{https://perso.univ-rennes1.fr/matthieu.romagny/agreg/dvt/mauvais_prix.pdf}.}\cite{LSAmvR}, thème~\ref{THEMEooNRZHooYuuHyt}] \label{ThoBuNjam}
    Deux entiers non nuls \( a,b\in\eZ^*\) sont premiers entre eux si et seulement s'il existe \( u,v\in\eZ\) tels que
    \begin{equation}
        au+bv=1
    \end{equation}
\end{theorem}
\index{Bézout!nombres entiers}

\begin{proof}
    Soit \( d=\pgcd(a,b)\) et des nombres \( u,v\) tels que \( au+bv=1\). Le PGCD \( d\) divise à la fois \( a\) et \( b\), et donc divise \( au+bv\). Nous en déduisons que \( d\) divise \( 1\) et est par conséquent égal à \( 1\).

    Nous supposons maintenant que \( \pgcd(a,b)=1\) et nous considérons l'ensemble
    \begin{equation}
        E=\{ au+bv\tq u,v\in \eZ \}\cap \eN^*.
    \end{equation}
    C'est-à-dire l'ensemble des nombres strictement positifs pouvant s'écrire sous la forme \( au+bv\). Cet ensemble est non vide parce qu'il contient par exemple soit \( a\) soit \( -a\). Soit \( m\) le plus petit élément de \( E\) et écrivons
    \begin{equation}    \label{EqMBsfrP}
        m=au_1+bv_1.
    \end{equation}
    Par le théorème de division euclidienne\footnote{Théorème~\ref{ThoDivisEuclide}.} (avec \( a\) et \( m\)), il existe des entiers uniques $q$ et $r$ tels que
    \begin{equation}
        a=mq+r
    \end{equation}
    avec \( 0\leq r<m\). En remplaçant \( m\) par sa valeur \eqref{EqMBsfrP}, \( a=(au_1+bv_1)q+r\) et
    \begin{equation}
        r=a(1-u_1q)-bv_1q,
    \end{equation}
    c'est-à-dire que \( r\in \eZ a+\eZ b\) en même temps que \( 0\leq r<m\). Si \( r\) était strictement positif, il serait dans \( E\). Mais cela est impossible par minimalité de \( m\). Donc \( r=0\) et \( a\) est divisible par \( m\).

    De la même façon nous prouvons que \( b\) est divisible par \( m\). Vu que \( m\) divise à la fois \( a\) et \( b\) nous avons \( m=1\).
\end{proof}

Une généralisation de Bézout \ref{ThoBuNjam} à plus de \( 2\) variables.
\begin{proposition}     \label{PROPooWSMTooMdfqse}
    Si \( \{ a_i \}_{i=1,\ldots, N}\) sont des entiers tels que \( \pgcd(a_1,\ldots, a_N)=1\), alors il existe des entiers \( \{ u_i \}_{i=1,\ldots, N}\) tels que
    \begin{equation}
        \sum_ia_iu_i=1.
    \end{equation}
\end{proposition}

\begin{corollary}       \label{CorgEMtLj}
    Soient \( p\) et \( q\) deux entiers premiers entre eux. Alors
    \begin{equation}
        p\eZ+q\eZ=\eZ;
    \end{equation}
    en particulier, pour tout \( x \in \eZ \), il existe \( u_x, v_x \) entiers tels que \(u_x p + v_x q = x \).
\end{corollary}

Notons que l'application \( p\eZ+q\eZ\) vers \( \eZ\) n'est évidemment pas injective: les $u_x$ et $v_x$ ne sont pas uniques à $x$ fixé.

\begin{proof}
    Soit \( x\in \eZ\). Le théorème de Bézout nous donne \( k\) et \( l\) tels que \( kp+lq=1\). Du coup, \( (xk)p+(xl)q=x\).
\end{proof}

La proposition suivante établit que si \( x\) est assez grand, alors il peut même être écrit comme une combinaison de \( p\) et \( q\) à coefficients positifs. Elle sera utilisée pour démontrer que les états apériodiques d'une chaine de Markov peuvent être atteints à tout moment (assez grand), voir la définition~\ref{DefCxvOaT} et ce qui suit.

\begin{proposition}     \label{PropLAbRSE}
    Soient \( a\) et \( b\) deux éléments de \( \eN\) premiers entre eux. Il existe \( N>0\) tel que tout \( x>N\) appartient à \( a\eN+b\eN\).
\end{proposition}

\begin{proof}
    Soient \( a\) et \( b\), premiers entre eux, et \( x\in \eN\). Disons tout de suite, pour éviter les cas triviaux et pénibles, que \( x\), \( a\) et \( b\) sont strictement positifs.

    \begin{subproof}
    \item[Une décomposition pour \( x\)]

    On applique le théorème~\ref{ThoDivisEuclide} de division euclidienne à $x$ et \( a + b \): il existe des entiers \( p_x, r_x \), uniques, tels que
    \begin{subequations}
        \begin{numcases}{}
       x = (p_x-1)(a+b) + r_x\\
       0 \leq r_x < a+b.
        \end{numcases}
    \end{subequations}
    En d'autres termes, \( p_x(a+b)\) est le premier multiple de \( a+b\) supérieur ou égal à $x$. De plus, $p_x$ est strictement positif car $x$ l'est. Il existe alors des entiers $u$ et $v$ tels que
    \begin{equation}    \label{EQooXYSZooJqxPui}
        ua + vb = p_x(a+b) - x
    \end{equation}
    par le corolaire~\ref{CorgEMtLj}. Ainsi, $x$ peut s'écrire
    \begin{equation}
        x = (p_x - u) a + (p_x - v) b.
    \end{equation}

\item[Des maximums]

    Il s'agit maintenant de savoir si nous pouvons être assuré d'avoir \( p_x > u\) et \( p_x > v\) dès que \( x\) est assez grand. Pour cela, grâce au corolaire~\ref{CorgEMtLj}, nous considérons les nombres \( u_i\) et \( v_i\) définis par
    \begin{equation}
        u_ia+v_ib=i
    \end{equation}
    pour \( i=1,\ldots, a+b\). Nous posons \( u^*=\max\{ u_i \}\), \( v^*=\max\{ v_i   \}\), et \( p^*=\max\{ u^*,v^* \}\).  Nous posons alors \( N = p^*(a+b)\), et considérons \( x>N \).

\item[Nouvelle décomposition pour \( x\)]

    Nous voulons écrire
    \begin{equation}        \label{EQooIKNWooBKItYz}
        x= (p_x - u_k) a + (p_x - v_k) b
    \end{equation}
    pour un certain \( k\). Cela demande \( u_ka+v_kb=ua+vb=p_x(a+b)-x\) par l'équation \eqref{EQooXYSZooJqxPui}. Vu que \( p_x(a+b)-x>0\), les nombres \( u_k\) et \( v_k\) existent : il suffit de prendre \( k=p_x(a+b)-x\).

\item[Conclusion]

    Avec tous ces choix, nous avons d'abord \( x>p^*(a+b)\) et donc
    \begin{equation}
        x=(p_x-1)(a+b)+r_x>p^*(a+b),
    \end{equation}
    ce qui donne
    \begin{equation}
        (p_x-1)(a+b)>p^*(a+b)-r_x>(p-1)(a+b).
    \end{equation}
    ou encore \( p_x>p^*\). Nous avons finalement
    \begin{equation}
       p_x \geq p^* \geq u^* \geq u_k
    \end{equation}
    et
    \begin{equation}
       p_x \geq p^* \geq v^* \geq v_k.
    \end{equation}
    De ce fait, la décomposition \eqref{EQooIKNWooBKItYz} est celle que nous voulions.
    \end{subproof}
\end{proof}


%\begin{proof}
    %Soit \( x\in \eN\) et \( k_1,l_1\in \eN\) les plus petits entiers tels que \( k_1p\geq x/2\) et \( l_1q\geq x/2\). Nous avons alors
    %\begin{equation}
        %x\leq k_1p+l_1q<x+(p+q).
    %\end{equation}
    %Nous posons \( \delta=k_1p+l_1q-x\).
   %
    %Soient des entiers \( a_i,b_i\) tels que \( a_ip+b_iq=i\). Nous notons
    %\begin{subequations}
        %\begin{align}
            %A=\max\{ a_i\tq i=1,\ldots, k+p \}\\
            %B=\max\{ b_i\tq i=1,\ldots, k+p \}
        %\end{align}
    %\end{subequations}
    %Notons que \( A\) et \( B\) sont donnés uniquement en termes de \( p\) et \( q\). Ils ne sont en aucun cas dépendants de \( x\).
   %
    %Nous avons
    %\begin{equation}
        %x=k_1p+lq-\delta=(k_1-a_{\delta})p+(l_1+b_{\delta})q
    %\end{equation}
    %avec \( k_1-a_{\delta}\geq k_1-A\) et \( l_1-b_{\delta}\geq l_1-B\). Si \( x\) est suffisamment grand pour avoir \( k_1>A\) et \( l_1>B\), alors la décomposition souhaitée est trouvée.
%
    %Une borne pour \( x\) est donnée par
    %\begin{equation}    \label{EqjQpURG}
        %x>\max\{ 2pA,2qB \}.
    %\end{equation}
%\end{proof}

\begin{normaltext}
    Une méthode pour obtenir les entiers naturels $u$ et $v$ qui permettent la décomposition \(x = au + bv \) est d'abord de choisir $u_0$ et $v_0$ tels que \( au_0 \) et \( bv_0 \) soient les plus proches possibles de $x/2$, puis de décomposer le nombre (relativement petit) \( x - au_0 - bv_0 \) en \( au_1 + bv_1 \). Deux nombres $u$ et $v$ qui fonctionnent sont alors $u = u_0 + u_1$ et $v = v_0 + v_1$.
\end{normaltext}

\begin{example}
    Écrivons \( 1000=u\cdot 7+v\cdot 5\) avec \( u,v\in \eN\). D'abord \( 72\cdot 7=504\) et \( 100\cdot 5=500\). Nous avons donc
    \begin{equation}
        1004=72\cdot 7+100\cdot 5.
    \end{equation}
    Ensuite \( 4=25-21=-3\cdot 7+5\cdot 5\). Au final,
    \begin{equation}
        1000=75\cdot 7+95\cdot 5.
    \end{equation}
\end{example}

%---------------------------------------------------------------------------------------------------------------------------
\subsection{Sous-groupes de \texorpdfstring{$(\eZ,+)$}{(Z,+)}}
%---------------------------------------------------------------------------------------------------------------------------

\begin{proposition} \label{PropSsgpZestnZ}
    Une partie \( H\) du groupe \( (\eZ,+)\) est un sous-groupe si et seulement s'il existe \( n\in\eN\) tel que \( H=n\eZ\).
\end{proposition}

\begin{proof}
    Soit \( H\neq\{ 0 \}\) un sous-groupe de \( \eZ\). L'ensemble \( H\cap\eN^*\) contient un élément minimum que nous notons \( n\). Nous avons certainement \( n\eZ\subset H\) parce que \( H\) est un groupe (donc \( n+n\) et \( -n\) sont dans \( H\) dès que \( n\) est dans \( H\)). Nous devons prouver que \( H\subset n\eZ\).

    Si \( x\in H\), par le théorème de division euclidienne~\ref{ThoDivisEuclide}, il existe \( q\in\eZ\) et \( r\in\eN \), uniques, tels que \( x=nq+r\) et \(0 \leq r < n \). Nous savons déjà que \( nq\in H\), donc \( r = x - nq \in H \). Le nombre \( r\) est donc un élément de \( H\) strictement plus petit que \( n\). Mais nous avions décidé que \( n\) serait le plus petit élément de \( H\cap\eN^*\). Par conséquent \( r=0\) et \( x=nq\in n\eZ\).
\end{proof}


Notons que si un sous-groupe \( H\) de \( \eZ\) est donné, alors le nombre \( n\) tel que \( H=n\eZ\) est unique. En effet si \( n\eZ=m\eZ\) nous avons que \( n\) divise \( m\) (parce que \( m\in m\eZ\subset n\eZ\)) et que \( m\) divise \( n\) parce que \( n\in m\eZ\). Par conséquent \( n=m\).

%---------------------------------------------------------------------------------------------------------------------------
\subsection{Résultats supplémentaires sur l'anneau des entiers}
%---------------------------------------------------------------------------------------------------------------------------

\begin{corollary}       \label{CORooLINXooBlUKPG}
    Les quotients de \( \eZ\) sont \( \eZ/n\eZ\).
\end{corollary}

\begin{proof}
    Tous les idéaux de \( \eZ\) sont de la forme \( n\eZ\). En effet en vertu de la proposition~\ref{PropSsgpZestnZ}, les seuls sous-groupes de \( \eZ\) (en tant que groupe additif) sont les \( n\eZ\). Tous les idéaux sont donc de cette forme. De plus les \( n\eZ\) sont effectivement tous des idéaux : si \( a\in n\eZ\) et si \( k\in \eZ\) alors \( ak\in n\eZ\). Cela est donc un idéal.
\end{proof}

\begin{proposition}     \label{PropZpintssiprempUzn}
    Soient \( n\geq 2\) un entier et \( \phi\colon \eZ\to \eZ/n\eZ\) la surjection canonique. Nous noterons \( \tilde a=\phi(a)\). Alors l'ensemble des inversibles de \( \eZ/n\eZ\) est donné par
    \begin{equation}
        U(\eZ/n\eZ)=\phi(P_n)=\{ \tilde x\tq 0\leq x\leq n\tq\pgcd(x,n)=1 \}.
    \end{equation}
    où \( P_n\) est l'ensemble $P_n=\{ x\in\{ 0,\ldots,n-1 \}\tq\pgcd(x,n)=1 \}$.

    De plus,
    \begin{equation}
        \Card\big( U(\eZ/n\eZ) \big)=\varphi(n).
    \end{equation}
\end{proposition}

\begin{proof}
    Soit \( 0\leq x\leq n\) tel que \( \pgcd(x,n)=1\). Il existe donc\footnote{Théorème de Bézout~\ref{ThoBuNjam}} \( u,v\in\eZ\) tels que \( ux+vn=1\). En passant aux classes,
    \begin{equation}
        \tilde u\tilde x=\tilde 1,
    \end{equation}
    donc \( \tilde u\) est l'inverse de \( \tilde x\). Cela prouve que \( \phi(P_n)\subset U(\eZ/n\eZ)\).

    Nous prouvons maintenant l'inclusion inverse. Soient \( \tilde x\) et \( \tilde y\) inverses l'un de l'autre : $\tilde x\tilde y=\tilde 1$. Il existe donc \( q\in\eZ\) tel que \( xy-qn=1\), ce qui prouve\footnote{À nouveau avec le Théorème de Bézout.} que \( \pgcd(x,n)=1\).
\end{proof}

%+++++++++++++++++++++++++++++++++++++++++++++++++++++++++++++++++++++++++++++++++++++++++++++++++++++++++++++++++++++++++++
\section{Corps}
%+++++++++++++++++++++++++++++++++++++++++++++++++++++++++++++++++++++++++++++++++++++++++++++++++++++++++++++++++++++++++++

%---------------------------------------------------------------------------------------------------------------------------
\subsection{Définitions, morphismes}
%---------------------------------------------------------------------------------------------------------------------------

\begin{definition}[\cite{ooLKFGooTUrnhx}]  \label{DefTMNooKXHUd}
    Un \defe{corps}{corps} est une anneau\footnote{Définition \ref{DefHXJUooKoovob}.} \( (A, +,\times)\) dans lequel tout élément non nul est inversible pour l'opération \( \times\) (pour l'opération \( +\), tous les éléments sont inversibles parce que \( (A,+)\) est un groupe).
\end{definition}

\begin{remark}      \label{REMooYRNUooYgBBKF}
    Un anneau est ce qu'on appelle «\emph{ring}» en anglais. Un corps est en anglais «\emph{field}». De plus le mot «\emph{field}» comprend la commutativité. Donc certains utilisent le mot «corps» pour dire «corps commutatif» et parlent alors d'anneau \emph{à division} pour parler de corps non commutatifs.
\end{remark}

La proposition suivante donne une caractérisation d'un corps, en disant un tout petit peu plus que la définition~\ref{DefTMNooKXHUd}.
\begin{proposition}
    L'anneau $A$ est un corps si et seulement si \( U(A) = A^* \).
\end{proposition}

\begin{proof}
    En deux parties.
    \begin{subproof}
        \item[Sens direct]
            Nous supposons que \( A\) est un corps. D'une part tous les éléments non nuls sont inversibles, c'est-à-dire \( A^*\subset U(A)\).
            
            Pour l'inclusion inverse, nous montrons qu'une élément inversible ne peut pas être nul. Cela n'est autre que le lemme~\ref{LEMooVUSMooWisQpD} couplé à la proposition~\ref{PROPooNCCGooXjVyVt} : \( a\cdot 0=0\neq 1\) pour tout \( a\).
        \item[Sens inverse]
            Si \( U(A)=A^*\), nous avons immédiatement que tous les éléments non nuls sont inversibles et donc que \( A\) est un corps.
    \end{subproof}
\end{proof}

\begin{lemma}       \label{LEMooJNIBooAURhrt}
    Si \( \eK\) est un corps et si \( a\in \eK\) vérifie \( a^2=1\), alors \( a=\pm 1\).
\end{lemma}

\begin{lemma}       \label{LemAnnCorpsnonInterdivzer}
    Un corps non nul est un anneau intègre\footnote{Définition \ref{DEFooTAOPooWDPYmd}.}.
\end{lemma}

\begin{proof}
    Soit un produit nul \( ab=0\). Si \( a\neq 0\), alors il est inversible et nous multiplions \( ab=0\) par \( a^{-1}\). Nous trouvons \( b=0\) parce que \( 0a^{-1}=0\).
\end{proof}
Conséquence : dans un corps nous avons toujours la règle du produit nul, et l'élément nul n'est jamais inversible.

\begin{definition}[Morphisme de corps]
    Un corps étant un anneau sans plus de structure, un \defe{morphisme de corps}{morphisme!de corps}\index{isomorphisme!de corps} n'est qu'un morphisme des anneaux\footnote{Définition \ref{DEFooSPHPooCwjzuz}.}.
\end{definition}

Le lemme suivant montre que définir un morphisme de corps comme étant simplement un morphisme des anneaux est une bonne idée.
\begin{lemma}       \label{LEMooWBOPooZnsZgQ}
    Si \( \phi\colon \eK\to \eK'\) est un morphisme de corps, alors
    \begin{enumerate}
        \item
            pour tout \( a\in \eK\) nous avons \( \varphi(a^{-1})=\varphi(a)^{-1}\);
        \item
            le morphisme \( \varphi\) est injectif.
    \end{enumerate}
\end{lemma}

\begin{proof}
    Vu que \( \varphi(1)=1\), nous avons aussi
    \begin{equation}
        1=\varphi(aa^{-1})=\varphi(a)\varphi(a^{-1}).
    \end{equation}
    Donc, par unicité de l'inverse\footnote{Lemme~\ref{LEMooECDMooCkWxXf}\,\ref{ITEMooOIWTooYqmMPP}.}, \( \varphi(a^{-1})=\varphi(a)^{-1}\).

    Pour l'injectivité nous supposons \( \varphi(a)=\varphi(b)\). Étant donné que \( \eK'\) est un corps, nous pouvons multiplier par \( \varphi(b)^{-1}\) :
    \begin{equation}
        \varphi(a)\varphi(b)^{-1}=1.
    \end{equation}
    En utilisant le premier point nous avons \( 1=\varphi(a)\varphi(b^{-1})\), puis le morphisme d'anneaux : \( 1=\varphi(ab^{-1})\), et encore le morphisme d'anneaux nous permet de déduire \( ab^{-1}=1\) et donc \(a=b\).
\end{proof}

%+++++++++++++++++++++++++++++++++++++++++++++++++++++++++++++++++++++++++++++++++++++++++++++++++++++++++++++++++++++++++++
\section{Anneau intègre}
%+++++++++++++++++++++++++++++++++++++++++++++++++++++++++++++++++++++++++++++++++++++++++++++++++++++++++++++++++++++++++++
\label{SECAnneauxIntegres}

La définition d'un anneau intègre est la définition~\ref{DEFooTAOPooWDPYmd}.

\begin{lemma}     \label{LEMooZSMEooUmSXWZ}
    Un corps\footnote{Définition~\ref{DefTMNooKXHUd}.} est un anneau intègre.
\end{lemma}

\begin{proof}
    En effet, soient un corps \( \eK\) et deux éléments \( x,y\in \eK\) tels que \( xy=0\). Si \( y\) est inversible, alors nous pouvons multiplier par \( y^{-1}\) pour trouver \( x=0\). Cela prouve que \( \eK\) est un anneau intègre.
\end{proof}

\begin{example}     \label{EXooLDXRooSxUAXs}
    L'ensemble \( \eZ\) avec les opérations usuelles est un anneau intègre\footnote{Anneau intègre, définition \ref{DEFooTAOPooWDPYmd}.}.
\end{example}

\begin{example}
    L'anneau \( \eZ/6\eZ\) n'est pas intègre parce que \( 3\cdot 2=0\) alors que ni \( 3\) ni \( 2\) ne sont nuls.
\end{example}

Nous verrons au théorème~\ref{ThoBUEDrJ} que l'anneau \( A\) est intègre si et seulement si \( A[X]\) est intègre.

\begin{corollary}   \label{CorZnInternprem}
    L'anneau \( \eZ/n\eZ\) est intègre si et seulement si \( n\) est premier.
\end{corollary}

\begin{proof}
    Supposons que \( n\) soit premier. La proposition \ref{PropZpintssiprempUzn} donne les inversibles de \( \eZ/n\eZ\) par
    \begin{equation}
        U(\eZ/n\eZ)=\{ \tilde x\tq 0\leq x\leq n\tq\pgcd(x,n)=1 \}.
    \end{equation}
    Mais comme \( n\) est premier, \( \pgcd(x,n)=1\) pour tout \( x\), et donc tous les éléments de \( \eZ/n\eZ\) sont inversibles. Donc \( \eZ/n\eZ\) est intègre.

    Si \( n\) n'est pas premier, alors \( n=pq\) avec \( 1<p\leq q<n\). Alors
    \begin{equation}
        [p]_n[q]_n=[pq]_n=[0]_n.
    \end{equation}
    Donc lorsque \( n\) n'est pas premier,  l'anneau \( \eZ/n\eZ\) possède des diviseurs de zéro et n'est alors pas intègre.
\end{proof}


%--------------------------------------------------------------------------------------------------------------------------- 
\subsection{Élément premier}
%---------------------------------------------------------------------------------------------------------------------------

\begin{definition}[\cite{ooWBLYooLYwALS}]       \label{DEFooZCRQooWXRalw}
    Soit un anneau commutatif \( A\). Un élément \( p\in A\) est \defe{premier}{élément premier} si il est
    \begin{enumerate}
        \item
            non nul,
        \item
            non inversible,
        \item       \label{ITEMooPMTTooCVHPIm}
            si \( p\) divise un produit \( ab\), alors il divise soit \( a\) soit \( b\) (ou le deux).
    \end{enumerate}
\end{definition}


Le lemme suivant est souvent pris pour la définition d'un nombre premier lorsqu'on parle de \( \eN\) ou \( \eZ\).
\begin{lemma}[\cite{frwiki179832418, MonCerveau}]
    Dans \( \eN\), un nombre est premier si et seulement si il admet exactement deux diviseurs entiers distincts.
\end{lemma}

\begin{proof}
    En deux parties.
    \begin{subproof}
    \item[\( \Rightarrow\)]
        Soit un élément premier \( p\in \eN\). Il y a trois possibilités : \( p=0\), \( p=1\) et \( p>1\).

    Le nombre \( p=0\) n'est pas premier parce qu'il est nul. Le nombre \( p=1\) n'est pas premier parce qu'il est inversible. Donc nous savons que si \( p\) est premier, alors \( p>1\).

    Un élément \( p>1\) dans \( \eN\) a toujours au moins deux diviseurs distincts : \( 1\) et \( p\). Soit un diviseur \( k\) de \( p\). Il existe \( l\in \eN\) tel que \( p=kl\). Vu que \( p\) est premier et divise le produit \( kl\), il divise \( k\) ou \( l\). Disons que \( p\) divise \( k\). De cette façon \( p\) divise \( k\) et \( k\) divise \( p\).

    Il existe donc \( n\in \eN\) tel que \( k=np\). En y substituant \( p=kl\), on trouve \( k=np=nkl\). En simplifiant par \( k\), il vient
    \begin{equation}
        1=nl,
    \end{equation}
    ce qui prouve que \( n=l=1\) et donc que \( k=p\) et donc que \( p\) n'a pas d'autres diviseurs que \( 1\) et \( p\).
        
    \item[\( \Leftarrow\)]
        Nous supposons que \( p\in \eN\) ait exactement deux diviseurs entiers distincts. Nous vérifions que \( p\) vérifie les trois conditions de la définition \ref{DEFooZCRQooWXRalw}.

        \begin{enumerate}
            \item
                \( p\neq 0\) parce que \( 0\) a nettement plus que deux diviseurs distincts.
            \item
                \( p\neq 1\) parce que \( 1\) a exactement un diviseur. Donc \( p\) n'est pas inversible dans \( \eN\).
            \item
                Soit \( p\) admettant exactement deux diviseurs distincts. Soit \( p\) divisant le produit \( ab'\) pour certains \( a\) et \( b'\) dans \( \eN\). Nous supposons que \( p\) ne divise pas \( a\), et nous allons prouver que \( p\) divise \( b'\) en supposant d'abord que \( p\) ne divise pas \( b'\). 

                \begin{subproof}
                \item[Un ensemble]
                Pour cela nous posons
                \begin{equation}
                    E=\{ x\in \eN\tq p\divides ax, p\notdivides x  \}.
                \end{equation}
                Nous posons \( b=\min(E)\). Nous avons pour hypothèse que \( E\) est non vide; en particulier \( 0<b\).
            \item[\( b<p\)]
                On vérifie que si \( p+k\in E\) alors \( k\in E\). Donc \( b\) ne peut pas être plus grand que \( p\). Vu que \( p\) lui-même n'est pas dans \( E\), nous avons \( b<p\).
            \item[Division euclidienne]
                Nous effectuons la division euclidienne du théorème \ref{ThoDivisEuclide} :
                \begin{equation}
                    p=mb+r.
                \end{equation}
                En multipliant par \( a\), \( ar=ap-mab\). Vu que \( ab\) est un multiple de \( p\) \( ap-mab\) est un multiple de \( p\). En particulier \( ar\) est divisible en \( p\). 
            \item[La contradiction]
                Nous avons donc \( r\in E\), alors que \( r<b\). Impossible.
                \end{subproof}
        \end{enumerate}
    \end{subproof}
\end{proof}

\begin{proposition}[\cite{ooTGPAooQTbamu}]     \label{PROPooWMNPooZdvOBt}
    Dans un anneau intègre\footnote{Si pas intègre, voir l'exemple \ref{EXooEIUEooCZCPMC}.} tout élément premier est irréductible\footnote{Toutes les définitions dans le thème \ref{THEMEooVIQIooOcFAQS}.}.
\end{proposition}
    
\begin{proof}
    Soit \( p\), un élément premier dans un anneau intègre \( A\).
    \begin{subproof}
        \item[\( p\) n'est pas inversible]
            Cela fait partie de la définition d'un élément premier.
        \item[\( p\) n'est pas un produit d'inversibles]
            Soient \( a,b\in A\) tels que \( p=ab\). Par le point \ref{ITEMooPMTTooCVHPIm} de la définition \ref{DEFooZCRQooWXRalw}, \( p\) divise soit \( a\) soit \( b\). Supposons que \( p\) divise \( a\). Alors il existe \( x\in A\) tel que \( a=px\). En remettant dans \( p=ab\) nous avons :
            \begin{equation}        \label{EQooPYBGooLFHMJZ}
                p=pxb.
            \end{equation}
            Mais l'anneau est intègre et permet donc des simplifications par tout élément non nul. La relation \ref{EQooPYBGooLFHMJZ} donne donc 
            \begin{equation}
                1=xb,
            \end{equation}
            ce qui signifie que \( b\) est inversible.

            Un travail similaire montre que \( a\) est inversible si \( p\) divise \( b\).
    \end{subproof}
\end{proof}

\begin{example}
    Si nous avons l'égalité \( 7=ab\) dans \( \eZ\), alors soit \( a\) soit \( b\) vaut \( 1\) et est donc inversible.
\end{example}

Sur un anneau non intègre, la notion d'élément premier n'est pas aussi intéressante que sur un anneau intègre. Par exemple la proposition \ref{PROPooWMNPooZdvOBt} devient fausse.

\begin{example}     \label{EXooEIUEooCZCPMC}
    Soit l'anneau \( \eZ^2\). L'élément \( (1,0)\) est premier mais pas irréductible.
    \begin{subproof}
        \item[\( (1,0)\) est premier]
            L'élément \( (1,0)\) est non nul; ça c'est pas cher. Pour qu'il soit inversible, il faudrait \( (1,0)(x,y)=(1,1)\). Entre autres, \( 0\times y=1\), ce qui est impossible. Donc il n'est pas inversible.

            Supposons que \( (1,0)\) divise le produit \( (a,b)(c,d)=(ac,b)\). Alors il existe \( (x,y)\) tel que \( (1,0)(x,y)=(ac,bd)\). Cela signifie que \( x=ac\) et \( 0\times y=bd\). En particulier, soit \( b=0\) soit \( d=0\). Si \( b=0\), nous avons \( (a,b)=(a,0)\) et effectivement, \( (1,0)\) le divise.
        \item[\( (1,0)\) n'est pas irréductible]
            Nous avons \( (1,0)=(1,0)(1,0)\). Donc l'élément \( (1,0)\) est le produit de deux éléments non inversibles.
    \end{subproof}
\end{example}

%+++++++++++++++++++++++++++++++++++++++++++++++++++++++++++++++++++++++++++++++++++++++++++++++++++++++++++++++++++++++++++ 
\section{Symbole de sommation}
%+++++++++++++++++++++++++++++++++++++++++++++++++++++++++++++++++++++++++++++++++++++++++++++++++++++++++++++++++++++++++++

%--------------------------------------------------------------------------------------------------------------------------- 
\subsection{Somme à valeurs dans un anneau}
%---------------------------------------------------------------------------------------------------------------------------

\begin{definition}      \label{DEFooNEVNooJlmJOC}
    Si \( f\colon \{ 0,\ldots, N \}\to A\) est une application vers un anneau \( A\), alors nous définissons la notation \( \sum_{i=0}^Nf(i)\) par récurrence de la façon suivante :
    \begin{enumerate}
        \item
            \( \sum_{i=0}^0f(i)=f(0)\),
        \item
            \( \sum_{i=0}^{k}f(i)=\sum_{i=0}^{k-1}f(i)+f(k)\).
    \end{enumerate}
\end{definition}

%--------------------------------------------------------------------------------------------------------------------------- 
\subsection{Somme à valeurs dans un groupe abélien}
%---------------------------------------------------------------------------------------------------------------------------

Si \( S\) est un ensemble fini, nous savons de la proposition \ref{PROPooJLGKooDCcnWi} qu'il existe un unique \( N\in \eN\) pour lequel il existe une bijection \( \varphi\colon \{ 0,\ldots, N \}\to S\). Cette bijection n'est à priori pas unique.

\begin{lemmaDef}[\cite{MonCerveau}]       \label{DEFooLNEXooYMQjRo}
    Soient un groupe abélien \( (G,+)\) ainsi qu'un ensemble fini \( I\) contenant \( n\) éléments. Soit une application \( f\colon I\to G \). Si \( \sigma_1,\sigma_2\colon \{1,\ldots, n \}\to I\) sont deux bijections, alors\footnote{Pour rappel, le symbole \( \sum_{i=1}^n\) est défini par \ref{DEFooNEVNooJlmJOC}.}
    \begin{equation}
        \sum_{i=1}^nf\big( \sigma_1(i) \big)=\sum_{i=1}^nf\big( \sigma_2(i) \big).
    \end{equation}
    La valeur commune est notée
    \begin{equation}
        \sum_{i\in I}f(i)
    \end{equation}
\end{lemmaDef}

\begin{proof}
    Nous commençons par considérer une transposition \( \sigma\) (qui permute \( k\) et \( l\) avec \( k<l\)). Nous avons
    \begin{subequations}
        \begin{align}
            \sum_{i=1}^nf(i)&=\sum_{i=1}^{k-1}f(i)+f(k)+\sum_{i=k+1}^{l-1}f(i)+f(l)+\sum_{i=l+1}^nf(i)\\
            &=\sum_{i=1}^{k-1}f(i)+f(l)+\sum_{i=k+1}^{l-1}f(i)+f(k)+\sum_{i=l+1}^nf(i)\\
            &=\sum_{i=1}^nf\big( \sigma(i) \big).
        \end{align}
    \end{subequations}
    Pour cela nous avons utilisé le fait que \( G\) est abélien pour permuter \( f(l)\in G\) et \( f(k)\in G\) avec \( \sum_{i=k+1}^{l-1}f(i)\in G\).

    Une permutation quelconque est un produit de telles transpositions (proposition \ref{PropPWIJbu}). Donc pour toute permutation \( \sigma\) nous avons
    \begin{equation}
        \sum_{i=1}^nf\big( \sigma(i) \big)=\sum_{i=1}^nf(i).
    \end{equation}
\end{proof}

La définition \ref{DEFooLNEXooYMQjRo} donne lieu à un certain nombre de remarques.
\begin{enumerate}
    \item
        Elle donne la somme sur un ensemble fini. Un problème avec les ensembles infinis (outre la convergence) est l'ordre de sommation. Si vous voulez sommer sur \( \eZ\), dans quel ordre le faire ?
    \item
        Pour aller plus loin, et sommer sur des ensembles infinis, il faut regarder la définition \ref{DefHYgkkA}. 
\end{enumerate}

\begin{proposition}     \label{PROPooJBQVooNqWErk}
    Soient un groupe abélien \( (G,+)\), un ensemble fini \( I\) est un ensemble fini, une application \( f\colon I\to G\) et une bijection \( \sigma\colon I\to I\). Alors
    \begin{equation}
        \sum_{i\in I}f(i)=\sum_{i\in I}f\big( \sigma(i) \big).
    \end{equation}
\end{proposition}

Si nous avons une application \( L\colon S\to S\), nous notons
\begin{equation}
    \sum_{s\in S}f\big( L(s) \big)=\sum_{s\in S}(f\circ L)(s).
\end{equation}
Cette façon d'écrire donne une interprétation pour la notation \( \sum_{g\in G}f(hg)\) qui arrive dans la proposition \ref{PROPooWJQQooFINSEc}. Il s'agit de considérer l'application \( L_h\) du lemme \ref{LEMooBIBFooBHxFYC}, de considérer\footnote{Le fait que \( L_h\) soit une bijection n'a pas d'importance ici.}
\begin{equation}        \label{EQooQQBEooFDOBVG}
    \sum_{g\in G}f(hg)=\sum_{g\in G}(f\circ L_h)(g)
\end{equation}
et de faire tourner la définition \ref{DEFooLNEXooYMQjRo}. La même chose tient pour définir \( \sum_{g\in G}(gh)\) à l'aide de \( R_h\).


\begin{lemma}
    Soit un ensemble \( A\) fini pouvant être écrit comme une union disjointe \( A=\bigcup_{k=1}^nA_k\); nous supposons que les \( A_i\) sont non vides. Soient un groupe abélien \( (G,+)\) et une application \( f\colon A\to G\). Alors
    \begin{equation}
        \sum_{a\in A}f(a)=\sum_{k=1}^n\sum_{a\in A_k}f(a).
    \end{equation}
\end{lemma}


\begin{proof}
    Le lemme \ref{LEMooTUIRooEXjfDY} nous indique que les parties \( A_k\) sont des ensembles finis. Nous notons
    \begin{enumerate}
        \item
            \( N_0=0\), et \( N_k=\Card(A_k)\),
        \item
            \( S_k=\sum_{i=1}^kN_k\).
        \item
            \( \varphi_k\colon \{ 1,\ldots, N_k \}\to A_k\), une bijection (l'existence est dans la proposition \ref{PROPooJLGKooDCcnWi}).
    \end{enumerate}
    Nous avons que \( \Card(A)=S_n\) par le lemme \ref{LEMooIAMKooLDucJc}. Nous définissons une belle bijection comme il faut :
    \begin{equation}
        \begin{aligned}
            \alpha\colon \{ 1,\ldots, S_n \}&\to A \\
            i&\mapsto \varphi_{k+1}(i-S_k) 
        \end{aligned}
    \end{equation}
    pour \( i\in\mathopen] S_k , S_{k+1} \mathclose]\).

    \begin{subproof}
        \item[\( \alpha\) est bien définie]
            Vu que \( i>S_k\) et \( i\leq S_{k+1}\) nous avons \( i-S_k\in \{ 1,\ldots, N_{k+1} \}\), et donc \( \varphi_{k+1}\) s'applique bien à \( i-S_k\).
        \item[\( \alpha\) est injective]
        Supposons que \( \alpha(i)=\alpha(j)\). Si \( i\in \mathopen] S_k , S_{k+1} \mathclose]\) et \( j\in \mathopen] S_l , S_{l+1} \mathclose]\), alors \( \alpha(i)=\varphi_{k+1}(i-S_k)\in A_{k+1}\) et \( \alpha(j)=\varphi_{l+1}(j-S_l)\in A_{l+1}\). Vu que les \( A_i\) sont disjoints, nous avons \( k=l\), et donc
        \begin{equation}
            \varphi_{k+1}(u-S_k)=\varphi_{k+1}(j-S_k).
        \end{equation}
        Étant donné que \( \varphi_{k+1}\) est injective, nous avons \( i-S_k=j-S_k\), ce qui montre que \( i=j\).
    \item[\( \alpha\) est surjective]
    Soit \( a\in A\). Il existe \( k\) tel que \( a\in A_k\). Nous avons donc un \( s\in\{ 1,\ldots, N_k \}\) tel que \( a=\varphi_k(s)\). En posant \( i=s+S_k\), nous avons bien \( a=\alpha(s+S_k)\) parce que \( s+S_k\in \mathopen] S_{k-1} , S_k \mathclose]\).
    \end{subproof}
    Vu que \( \alpha\) est une bijection, nous avons l'égalité
    \begin{equation}
        \sum_{a\in A}f(a)=\sum_{i=1}^{S_n}(f\circ \alpha)(i).
    \end{equation}
    
    Nous avons encore besoin d'introduire une bijection. Nous posons
    \begin{equation}
        \begin{aligned}
        \beta_k\colon \mathopen] S_{k-1} , S_k \mathclose]&\to A_k \\
        i&\mapsto \varphi_k(i-S_{k-1}). 
        \end{aligned}
    \end{equation}
    Cela est une bijection parce que \( \varphi_k\) en est une, et que \( i\mapsto i-S_{k-1}\) est une bijection de \( \mathopen] S_{k_1} , S_k \mathclose]\).

    Nous pouvons maintenant terminer :
    \begin{subequations}
        \begin{align}
            \sum_{a\in A}f(a)&=\sum_{i=1}^{S_n}(f\circ \alpha)(i)\\
            &=\sum_{k=1}^n\left( \sum_{i=S_{k-1}-1}^{S_k}(f\circ \alpha)(i) \right)        \label{SUBEQooNVKWooZqBAau}\\
        &=\sum_{k=1}^n\left( \sum_{i\in \mathopen] S_{k-1} , S_k \mathclose]}f\big( \varphi_k(i-S_{k-1}) \big)  \right)\\
    &=\sum_{k=1}^n\left( \sum_{i\in \mathopen] S_{k-1} , S_k \mathclose]}f\big( \beta_k(i) \big) \right)\\
    &=\sum_{i=1}^n\left( \sum_{a\in A_k}f(a) \right).
        \end{align}
    \end{subequations}
    Justifications :
    \begin{itemize}
        \item Pour \eqref{SUBEQooNVKWooZqBAau}. Associativité de la somme.
    \end{itemize}
\end{proof}


\begin{proposition}[\cite{MonCerveau}]     \label{PROPooWJQQooFINSEc}
    Soient un groupe fini \( G\) et une fonction \( f\colon G\to A\) où \( A\) est un anneau commutatif. Alors
    \begin{equation}
        \sum_{g\in G}f(g)=\sum_{g\in G}f(gh)=\sum_{g\in G}f(hg)
    \end{equation}
    pour tout \( h\in G\).
\end{proposition}

\begin{proof}
    Nous avons une bijection \( \varphi\colon \{ 0,\ldots,  N \}\to G\) garantie par la proposition \ref{PROPooJLGKooDCcnWi}. La définition est que
    \begin{equation}
        \sum_{g\in G}f(g)=\sum_{i=0}^Nf\big( \varphi(i) \big).
    \end{equation}
    Par ailleurs, le lemme \ref{LEMooBIBFooBHxFYC} donne une bijection \( L_h\colon G\to G\) et permet de considérer la composée
    \begin{equation}
        \begin{aligned}
            \varphi'\colon \{ 0,\ldots,  N \}&\to G \\
            \varphi'=L_h\circ \varphi.
        \end{aligned}
    \end{equation}
    La proposition \ref{DEFooLNEXooYMQjRo} nous permet d'utiliser la bijection \( \varphi'\) au lieu de \( \varphi\) pour exprimer la somme \( \sum_{g\in G}\). Ensuite un jeu de notation utilisant \eqref{EQooQQBEooFDOBVG} donne
    \begin{equation}
        \begin{aligned}[]
            &\sum_{g\in G}f(g)=\sum_{i=0}^Nf\big( \varphi(i) \big)=\sum_{i=0}^Nf\big( \varphi'(i) \big)=\sum_{i=0}^N(f\circ L_h\circ \varphi)(i)\\
            &\quad=\sum_{i=0}^N(f\circ L_h)\big( \varphi(i) \big)=\sum_{g\in G}(f\circ L_h)(g)=\sum_{g\in G}f(hg).
        \end{aligned}
    \end{equation}
    En ce qui concerne \( \sum_{g\in G}f(gh)\), c'est la même chose, en utilisant \( R_h\) au lieu de \( L_h\).
\end{proof}

Tout cela nous permet de faire une somme sympathique et bien connue.
\begin{lemma}
    Soit \( n\in \eN\). Nous avons
    \begin{equation}
        \sum_{k=0}^nk=\frac{ n(n+1) }{ 2 }.
    \end{equation}
\end{lemma}

\begin{proof}
    La preuve est pratiquement immédiate par récurrence. Nous allons donner une preuve plus «constructive», qui formalise l'idée classique d'écrire la somme à l'endroit et à l'envers.


    Nous notons \( S\) la somme \( \sum_{k=0}^nk\). Le lemme \ref{DEFooLNEXooYMQjRo} dit que si \( \sigma_i\colon \{ 0,\ldots, n \}\to \{ 0,\ldots, n \}\) sont deux bijections, alors \( \sum_{k=0}^nf\big( \sigma_1(k) \big)=\sum_{k=0}^nf\big( \sigma_2(k) \big)\). Nous sommes intéressé au cas \( f(i)=i\).

    En prenant \( \sigma_1(k)=k\) et \( \sigma_2(k)=n-k\), nous avons
    \begin{equation}
        S=\sum_{k=0}^nk=\sum_{k=0}^n(n-k).
    \end{equation}
    Donc
    \begin{equation}
        2S=\sum_{k=0}^n\big( k+(n-k) \big)=\sum_{k=0}^nn=n\sum_{k=0}^n1=n(n+1).
    \end{equation}
    En divisant par deux, nous obtenons le résultat annoncé.
\end{proof}

\begin{proposition}     \label{PROPooQMUDooQQVRIe}
    Si \( E\) est un ensemble fini et si \( G\) est un groupe abélien, alors pour toute fonction \( f\colon E\to G\) et pour toute permutation\footnote{Une permutation est une bijection, définition \ref{DEFooJNPIooMuzIXd}.} \( \sigma\) de \( E\),
    \begin{equation}
        \prod_{i\in E}f(i)=\prod_{i\in E}f\big( \sigma(i) \big)
    \end{equation}
\end{proposition}

\begin{proof}
    C'est exactement la proposition \ref{DEFooLNEXooYMQjRo}, sauf qu'ici la loi de groupe est notée multiplicativement au lieu de additivement.
\end{proof}

%+++++++++++++++++++++++++++++++++++++++++++++++++++++++++++++++++++++++++++++++++++++++++++++++++++++++++++++++++++++++++++
\section{Module sur un anneau}
%+++++++++++++++++++++++++++++++++++++++++++++++++++++++++++++++++++++++++++++++++++++++++++++++++++++++++++++++++++++++++++

\begin{definition}[module sur un anneau\cite{ooJGVOooSjQBVh}]       \label{DEFooHXITooBFvzrR}
    Soit un anneau \( A\). Un \defe{module à gauche}{module!à gauche} sur \( A\) est la donnée d'un triple \( (M,+,\cdot)\) où
    \begin{enumerate}
        \item
            \( +\) est une loi de composition interne à \( M\), c'est-à-dire \( +\colon M\times M\to M\),
        \item
            \( \cdot\) est une loi de composition externe, c'est-à-dire \( \cdot\colon A\times M\to M\)
    \end{enumerate}
    telles que
    \begin{enumerate}
        \item
            \( (M,+)\) est un groupe\footnote{Nous verrons dans la proposition~\ref{PROPooGARGooDiMqtN} qu'il est forcément commutatif.}.
        \item
            \( a\cdot(x+y)=a\cdot x+a\cdot y\),
        \item
            \( (a+b)\cdot x=a\cdot x+b\cdot x\),
        \item
            \( (ab)\cdot x=a\cdot(b\cdot x)\)
        \item
            \( 1\cdot x=x\).
    \end{enumerate}
    pour tout \( a,b\in A\) et \( x,y\in M\).
\end{definition}

\begin{proposition}\label{PROPooGARGooDiMqtN}
    Si \( M\) est un module sur un anneau, alors \( (M,+)\) est un groupe commutatif.
\end{proposition}

\begin{proof}
    Il suffit de calculer \( (1+1)\cdot (x+y)\) de deux façons différentes :
    \begin{equation}
        (1+1)\cdot (x+y)=1\cdot (x+y)+1\cdot (x+y)=x+y+x+y
    \end{equation}
    d'une part et
    \begin{equation}
        (1+1)\cdot (x+y)=(1+1)\cdot x+(1+1)\cdot y=x+x+y+y,
    \end{equation}
    d'autre part. En égalant les deux expressions, il vient
    \begin{equation}
        x+y+x+y=x+x+y+y,
    \end{equation}
    qui se simplifie (nous sommes dans un groupe) en \( y+x=x+y\).
\end{proof}

\begin{definition}\label{DEFooKHWZooIfxdNc}
    Un \defe{espace vectoriel}{espace!vectoriel} est un module\footnote{Définition \ref{DEFooHXITooBFvzrR}.} sur un corps commutatif\footnote{La condition de commutativité n'est pas indispensable, mais comme nous ne parlerons que de corps commutatifs\ldots}.
\end{definition}

\begin{definition}[\cite{BIBooSTWWooItiMUp}]        \label{DEFooRUKVooLnXxdS}
    Soient un \( A\)-module \( M\) et un ensemble \( I\). Une famille \( (m_i)_{i\in I}\) est \defe{libre}{partie libre!module} si ils sont \defe{linéairement indépendants}{linéairement indépendant!module}, c'est-à-dire si pour tout choix d'une partie finie \( J\) dans \( I\) et d'éléments \( (a_j)_{j\in J}\) dans \( A\), si nous avons
    \begin{equation}
        \sum_{j\in J}a_jm_j=0,
    \end{equation}
    alors \( a_j=0\) pour tout \( j\).
\end{definition}

\begin{definition}[\cite{BIBooNKWVooYfrwSd}]        \label{DEFooWBOBooJNyyBF}
    Soit \( S\), une partie du \( A\)-module \( M\). Le \defe{sous-module engendré}{sous-module engendré} par \( S\) est l'ensemble des éléments de \( M\) qui sont des combinaisons linéaires finies d'éléments de \( S\), c'est-à-dire de sommes de la forme
    \begin{equation}
        \sum_{t\in T}a_tt
    \end{equation}
    où \( T\) est fini dans \( S\) et \( a_t\in A\).
\end{definition}

%--------------------------------------------------------------------------------------------------------------------------- 
\subsection{Module produit}
%---------------------------------------------------------------------------------------------------------------------------

\begin{lemmaDef}[\cite{BIBooSTWWooItiMUp}]        \label{DEFooLCJEooBvVmkV}
    Soient un anneau \( A\) et un ensemble \( I\). Le \( A\)-\defe{module produit}{module produit} \( A^I\) est l'ensemble des applications \( I\to A\).

    En termes de notations, nous écrivons ceci :
    \begin{equation}
        A^I=\{ (a_i)_{i\in I},a_i\in A \}.
    \end{equation}
    L'ensemble \( A^I\) devient un module par les définition, pour \( x,y\in A^I\) et \( a\in A\) :
    \begin{subequations}
        \begin{align}
            ax&=(ax_i)_{i\in I}\\
            x+y&=(x_i+y_i)_{i\in I}     \label{EQooODBMooQKLUgd}.
        \end{align}
    \end{subequations}
    En d'autres termes, \( A^I=\Fun(I,A)\).
\end{lemmaDef}

\begin{lemma}
    Pour chaque \( i\in I\) nous considérons l'élément \( e_i\in A^I\) donné par
    \begin{equation}
        e_i=(\delta_{ij})_{j\in I}.
    \end{equation}
    La famille \( \{ e_i \}_{i\in I}\) est libre\footnote{Définition \ref{DEFooRUKVooLnXxdS}.} dans \( A^I\).
\end{lemma}

\begin{proof}
    Soient \( J\) fini dans \( I\) ainsi que des éléments \( a_j\in A\) (\( j\in J\)). Nous supposons que\footnote{Pour rappel, les sommes finies sont définies par \ref{DEFooLNEXooYMQjRo}.} \( \sum_{j\in J}a_je_j=0\). Calculons un peu :
    \begin{equation}
        \sum_{j\in J}a_je_j=\sum_{j\in J}(a_j\delta_{ji})_{i\in I}=\left( \sum_{j\in J}a_j\delta_{ji} \right)_{i\in I}.
    \end{equation}
    Pour que le tout soit nul dans \( A^I\), il faut que
    \begin{equation}
        \sum_{j\in J}a_j\delta_{ji}
    \end{equation}
    soit nul pour tout \( i\in I\). Si nous fixons \( i\in I\), la somme sur \( j\) possède un seul terme non annulé par \( \delta_{ji}\), et c'est le terme \( j=i\). Nous avons donc \( a_i=0\).
\end{proof}

\begin{definition}      \label{DEFooBMEPooFsCHgb}
    Nous notons \( A^{(I)}\) le sous-module de \( A^I\) engendré\footnote{Définition \ref{DEFooWBOBooJNyyBF}.} par les \( e_i\).
\end{definition}

\begin{theorem}[Propriété universelle de \( A^{(I)}\)\cite{BIBooSTWWooItiMUp}]      \label{THOooPDZCooJnHbOd}
    Soient un anneau \( A\) ainsi qu'un \( A\)-module \( P\). Pour \( \phi\in\Hom_A(A^{(I)}, P)\), nous considérons
    \begin{equation}
        \begin{aligned}
            \phi|_I\colon I&\to P \\
            i&\mapsto \phi(e_i). 
        \end{aligned}
    \end{equation}
    \begin{enumerate}
        \item
            
    L'application
    \begin{equation}
        \begin{aligned}
            f\colon \Hom_A(A^{(I)},P)&\to \Fun(I,P) \\
            \phi&\mapsto \phi|_I 
        \end{aligned}
    \end{equation}
    est une bijection.
\item
    L'application inverse est \( g\colon \Fun(I,P)\to \Hom_A(A^{(I)},P) \) donnée par
    \begin{equation}
        g(\psi)\big( \sum_{j\in J}a_je_j \big)=\sum_{j\in J}a_j\psi(j)
    \end{equation}
    pour tout \( J\) fini dans \( I\) et choix de \( a_j\in A\).
    \end{enumerate}
\end{theorem}

\begin{proof}
    Nous allons montrer que \( g\big( f(\phi) \big)=\phi\) et que \( f\big( g(\psi) \big)=\psi\) pour tout \( \phi\in\Hom_A(A^{(I)},P)\) et pour tout \( \psi\in \Fun(I,P)\).

    Dans un premier sens nous avons :
    \begin{subequations}
        \begin{align}
            g\big( f(\phi) \big)\big( \sum_ja_je_j \big)&=\sum_ja_jf(\phi)(j)\\
            &=\sum_ja_j\phi(e_j)\label{SUBALIGNooBWPLooHeIaQg}\\
            &=\phi(\sum_ja_je_j)        \label{SUBALIGNooUOQPooCwLgZo}.
        \end{align}
    \end{subequations}
    Justifications :
    \begin{itemize}
        \item 
            Pour \eqref{SUBALIGNooBWPLooHeIaQg}, nous avons utilisé le fait que \( f(\phi)(i)=\phi|_I(i)=\phi(e_i)\).
        \item
            Pour \eqref{SUBALIGNooUOQPooCwLgZo}, nous utilisons le fait que \( \phi\) est un morphisme de modules.
    \end{itemize}
    Et pour l'autre sens,
    \begin{equation}
        f\big( g(\psi) \big)(i)=g(\psi)(e_i)=\psi(i).
    \end{equation}
\end{proof}

%--------------------------------------------------------------------------------------------------------------------------- 
\subsection{Sous-module}
%---------------------------------------------------------------------------------------------------------------------------

Soient \( M\) un \( A\)-module et \( x=(x_i)_{i\in I}\) une famille d'éléments de \( M\) paramétrée par l'ensemble \( I\). Nous considérons l'application
\begin{equation}
    \begin{aligned}
        \mu_x\colon A^{(I)}&\to M \\
        (a_i)_{i\in I}&\mapsto \sum_{i\in I}a_ix_i.
    \end{aligned}
\end{equation}
Ici \( A^{(I)}\) désigne l'ensemble de toutes les applications \( I\to A\) de support fini.

\begin{definition}      \label{DefBasePouyKj}
    À l'instar des espaces vectoriels, les modules ont une notion de partie libre, génératrice et de bases :
    \begin{enumerate}
        \item
            Si \( \mu_x\) est surjective, nous disons que \( x\) est une partie \defe{génératrice}{génératrice!partie d'un module}.
        \item
            Si \( \mu_x\) est injective, nous disons que la partie \( x\) est \defe{libre}{libre!partie d'un module}.
        \item
            Si \( \mu_x\) est bijective, nous disons que la partie \( x\) est une \defe{base}{base!d'un module}.
    \end{enumerate}
\end{definition}

\begin{definition}
  Un sous-ensemble \( N\subset M\) est un \defe{sous-module}{sous-module} si \( (N,+)\) est un sous-groupe de \( (M,+)\) et si \( a\cdot x\in N\) pour tout \( x\in N\) et pour tout \( a\in A\).
\end{definition}

\begin{example}
    Un anneau \( A\) est lui-même un \( A\)-module et ses sous-modules sont les idéaux.
\end{example}

\begin{definition}
    Soit \( M\) un module sur un anneau commutatif \( A\). Un \defe{projecteur}{projecteur!dans un module} est une application linéaire \( p\colon M\to M\) telle que \( p^2=p\).

    Une famille \( (p_i)_{i\in I}\) sur \( M\) est \defe{orthogonale}{orthogonal!famille de projecteurs} si \( p_i\circ p_j=0\) pour tout \( i\neq j\). La famille est \defe{complète}{complète!famille de projecteurs} si \( \sum_{i\in I}p_i=\mtu\).
\end{definition}

\begin{theorem}     \label{ThoProjModpAlsUR}
    Soient des sous modules \( M_1,\ldots,M_n\) du module \( M \) tels que \( M=M_1\oplus\ldots\oplus M_n\). Les applications \( p_i\) définies par
    \begin{equation}
        p_i(x_1+\ldots+x_n)=x_i
    \end{equation}
    forment une famille orthogonale de projecteurs et \( p_1+\cdots +p_n=\id\).

    Inversement, si \( (p_1,\ldots, p_n)\) est une famille orthogonale de projecteurs dans un module \( \modE\) tel que \( \sum_{i=1}^np_i=\id\), alors
    \begin{equation}
        M=\bigoplus_{i=1}^np_i(M).
    \end{equation}
\end{theorem}

\begin{definition}
    Un module est \defe{simple}{simple!module}\index{module!simple} ou \defe{irréductible}{irréductible!module}\index{module!irréductible} s'il n'a pas d'autres sous-modules que \( \{ 0 \}\) et lui-même. Un module est \defe{indécomposable}{indécomposable!module}\index{module!indécomposable} s'il ne peut pas être écrit comme somme directe de sous-modules.
\end{definition}

Un module simple est a fortiori indécomposable. L'inverse n'est pas vrai comme le montre l'exemple suivant.

\begin{example}
    Soit \( \modE=\eC[X]/(X^2)\) vu comme \( \eC[X]\)-module. C'est le \( \eC[X]\)-module des polynômes de la forme \( aX+b\) avec \( a,b\in \eC\). L'ensemble des polynômes de la forme \( aX\) est un sous-module. Le module \( \modE\) n'est donc pas simple. Il est cependant indécomposable parce que \( \{ aX \}\) est le seul sous-module non trivial. En effet si \( \modF\) est un sous-module de \( \modE\) contenant \( aX+b\) avec \( b\neq 0\), alors \( \modF\) contient \( X(aX+b)=bX\) et donc contient tout \( \modE\).
\end{example}

\begin{definition}[Algèbre\cite{ZSyHmiy}]   \label{DefAEbnJqI}
    Si \( \eK\) est un corps commutatif\footnote{Définition~\ref{DefTMNooKXHUd}}, une \( \eK\)-\defe{algèbre}{algèbre} \( A\) est un espace vectoriel\footnote{Définition~\ref{DEFooKHWZooIfxdNc}.} muni d'une opération bilinéaire \( \times\colon A\times A\to A\), c'est-à-dire telle que pour tout \( x,y,z\in A\) et pour tout \( \alpha,\beta\in\eK\),
    \begin{enumerate}
        \item
            \( (x+y)\times z=x\times z+y\times z\)
        \item
            \( x\times (y+z)=x\times y+x\times z\)
        \item
            \( (\alpha x)\times (\beta y)=(\alpha\beta)(x\times y)\).
    \end{enumerate}
    Si \( A\) et \( B\) sont deux \( \eK\)-algèbres, une application \( f\colon A\to B\) est un \defe{morphisme d'algèbres}{morphisme!d'algèbres} entre \( A\) et \( B\) si pour tout \( x,y\in A\) et pour tout \( \alpha\in \eK\),
    \begin{enumerate}
        \item
            \( f(xy)=f(x)f(y)\)
        \item
            \( f(x+\alpha y)=f(x)+\alpha f(y)\)
    \end{enumerate}
    où nous avons noté \( xy\) pour \( x\times y\).
\end{definition}

\begin{lemma}[\cite{MonCerveau}]   \label{LEMooVKLKooSAHmpZ}
    Soient une algèbre \( A\) et une famille \( (X_i)_{i\in I}\) de sous-algèbres de \( A\) (ici \( I\) est un ensemble quelconque). Alors la partie \( X=\bigcap_{i\in I}X_i\) est une sous-algèbre de \( A\).
\end{lemma}

\begin{proof}
    Nous devons prouver que si \( x\) et \( y\) sont dans \( X\) et \( \lambda\in \eK\), alors \( xy\), \( x+y\) et \( \lambda x\) sont dans \( X\). Pour tout \( i\in I\) nous avons \( x,y\in X_i\) et donc \( xy\in X_i\), \( x+y\in X_i\) et \( \lambda x\in X_i\) (parce que \( X_i\) est une algèbre). Donc \( xy\),\( x+y\) et \( \lambda x\) sont dans \( X_i\) pour tout \( I\), et donc dans \( X\).
\end{proof}

\begin{definition}\label{DefkAXaWY}
    L'\defe{algèbre engendrée}{algèbre!engendrée} par \( X\) est l'intersection de toutes les sous-algèbres de \( A\) contenant \( X\) (qui est une algèbre par le lemme~\ref{LEMooVKLKooSAHmpZ}).
\end{definition}



%+++++++++++++++++++++++++++++++++++++++++++++++++++++++++++++++++++++++++++++++++++++++++++++++++++++++++++++++++++++++++++
\section{Caractéristique d'un anneau}
%+++++++++++++++++++++++++++++++++++++++++++++++++++++++++++++++++++++++++++++++++++++++++++++++++++++++++++++++++++++++++++

\begin{lemmaDef}        \label{LEMDEFooVEWZooUrPaDw}
    Soit l'application
    \begin{equation}
        \begin{aligned}
            \mu\colon \eZ&\to A \\
            n&\mapsto n\cdot 1_A
        \end{aligned}
    \end{equation}
    où \( n\cdot 1_A\) signifie \( \sum_{k=1}^n1_A\).
    \begin{enumerate}
        \item
            C'est un morphisme d'anneaux.
        \item
            Le noyau est un sous-groupe de \( \eZ\)
        \item
            Il existe un unique \( p\in \eZ\) tel que \( \ker(\mu)=p\eZ\).
    \end{enumerate}
    Ce \( p\) est la \defe{caractéristique}{caractéristique!d'un anneau} de \( A\).
\end{lemmaDef}

Par exemple la caractéristique que \( \eQ\) est zéro parce qu'aucun multiple de l'unité n'est nul.

À propos de diagonalisation en caractéristique \( 2\), voir l'exemple~\ref{ExewINgYo}.

\begin{lemma}
    Si \( A\) est de caractéristique nulle, alors \( A\) est infini.
\end{lemma}

\begin{proof}
    En effet, \( \ker\mu=\{0\} \) implique que \( n1_A \neq  m1_A\) dès que \(n \neq m \) et par conséquent \( A\) contient \(\eZ 1_A \), et  est infini.
\end{proof}

\begin{lemma}       \label{LemHmDaYH}
    Si \( p\) est la caractéristique de l'anneau \( A\), alors nous avons l'isomorphisme d'anneaux
    \begin{equation}
         \eZ 1_A\simeq\eZ/p\eZ.
    \end{equation}
\end{lemma}

\begin{proof}
    L'isomorphisme est donné par l'application \( n1_A\mapsto \phi(n)\) si \( \phi\) est la projection canonique \( \eZ\to \eZ/p\eZ\).
\end{proof}

\begin{proposition}     \label{PropGExaUK}
    La caractéristique d'un anneau fini divise son cardinal.
\end{proposition}

\begin{proof}
    Si \( A\) est un anneau, le groupe \( \eZ\) agit sur \( A\) par
    \begin{equation}
        n\cdot a=a+n1_A.
    \end{equation}
    Chaque orbite de cette action est de la forme
    \begin{equation}
        \mO_a=\{ a+n1_A\tq n=0,\ldots, p-1 \}
    \end{equation}
    où \( p\) est la caractéristique de \( A\). Les orbites ont \( p\) éléments et forment une partition de \( A\), donc le cardinal de \( A\) est un multiple de \( p\).
\end{proof}

\begin{lemma}[\cite{ooIBWOooSjOvXd}]        \label{LEMooJQIKooQgukqn}
    Un anneau totalement ordonné est de caractéristique nulle.
\end{lemma}

\begin{proof}
    Le morphisme \( \mu\colon \eZ\to A\), \( n\mapsto n 1_A\) est strictement croissant, en particulier \( \mu(x)\neq \mu(y)\) dès que \( x\neq y\). Donc \( \ker(\mu)=\{ 0 \}\).
\end{proof}

L'ensemble typique de caractéristique \( p\) est \( \eF_p=\eZ/p\eZ\).

\begin{proposition} \label{PropFrobHAMkTY}
    Soit \( A\) un anneau commutatif unitaire de caractéristique \( p\). L'application
    \begin{equation}
        \begin{aligned}
            \Frob_A\colon A&\to A \\
            x&\mapsto x^p
        \end{aligned}
    \end{equation}
    est un automorphisme d'anneau unitaire.
\end{proposition}
Nous le nommons le \defe{morphisme de Frobenius}{morphisme!Frobenius}\index{Frobenius!morphisme}. Nous utiliserons aussi les itérés du morphisme de Frobenius : \( \Frob^k\colon x\mapsto x^{p^k}\).

\begin{example}
    Soit à factoriser \( X^p-1\) dans \( \eF_p\). Grâce au morphisme de Frobenius, nous avons immédiatement
    \begin{equation}
        X^p-1=(X-1)^p.
    \end{equation}
\end{example}


\begin{lemma}       \label{LemCaractIntergernbrcartpre}
    La caractéristique\footnote{Définition~\ref{LEMDEFooVEWZooUrPaDw}.} d'un anneau intègre est zéro ou un élément premier\footnote{Définition \ref{DEFooZCRQooWXRalw}.}.
\end{lemma}

\begin{proof}
    Si \( A\) est intègre, alors \( \eZ 1_A\) est a fortiori intègre. Notons \( p \) la caractéristique de \( A \). Si \( p = 0 \), la preuve est finie; supposons donc que \( p \neq 0 \). Alors, l'anneau \( \eZ/p\eZ\) est isomorphe à \( \eZ 1_A\), et est donc intègre. Or, la proposition~\ref{CorZnInternprem} dit que \( \eZ/p\eZ\) est intègre si et seulement si \( p\) est premier, ce qui conclut la preuve.
\end{proof}

\begin{example}
    Il existe des corps dont la caractéristique n'est pas égale au cardinal (contrairement à ce que laisserait penser l'exemple des \( \eZ/p\eZ\)). En effet les matrices \( n\times n\) inversibles sur \( \eF_{3}\) forment un corps qui n'est pas de cardinal trois alors que la caractéristique est \( 3\) :
    \begin{equation}
        \begin{pmatrix}
            1    &       \\
                &   1
            \end{pmatrix}+\begin{pmatrix}
                1    &       \\
                    &   1
                \end{pmatrix}+\begin{pmatrix}
                    1    &       \\
                        &   1
                \end{pmatrix}=0.
    \end{equation}
\end{example}

\begin{example}
    Si \( \eK\) est un corps de caractéristique \( 2\), alors l'égalité \( x=-x\) n'implique pas \( x=0\), vu que \( 2x=0\) est vérifiée pour tout \( x\). Cela se répercute sur un certain nombre de résultats. Par exemple, en caractéristique deux, une forme antisymétrique n'est pas toujours alternée: voir le lemme~\ref{LemHiHNey}.
\end{example}



%+++++++++++++++++++++++++++++++++++++++++++++++++++++++++++++++++++++++++++++++++++++++++++++++++++++++++++++++++++++++++++
\section{Polynômes}
%+++++++++++++++++++++++++++++++++++++++++++++++++++++++++++++++++++++++++++++++++++++++++++++++++++++++++++++++++++++++++++

%--------------------------------------------------------------------------------------------------------------------------- 
\subsection{Polynômes d'une variable}
%---------------------------------------------------------------------------------------------------------------------------

Et voila la définition que tout le monde attendait; la définition des anneaux de polynômes. Pour ne pas taper trop fort du premier coup, nous commençons par les polynômes d'une seule variable\footnote{Les polynômes à plusieurs variables seront la définition \ref{DEFooZNHOooCruuwI}.}.

Nous allons définir et étudier ici l'anneau des polynômes sur un anneau \( A\), c'est-à-dire ce qui sera noté \( A[X]\). Pour \( \eK(X)\) lorsque \( \eK\) est un corps, voir~\ref{DEFooQPZIooQYiNVh}.

L'ensemble des polynômes sur \( A\) sera simplement \( A^{(\eN)}\). Vu que \( \eN\) est un ensemble bien particulier possédant plein de structure, nous allons pouvoir mettre sur \( A^{(\eN)}\) une structure non seulement de \( A\)-module (ça c'est déjà fait), mais en plus d'anneau ainsi qu'une évaluation.
\begin{definition}      \label{DEFooFYZRooMikwEL}
    L'ensemble des \defe{polynômes}{polynômes} en une indéterminée sur l'anneau \( A\) est l'anneau \( A^{(\eN)}\) que nous avons défini en \ref{DEFooBMEPooFsCHgb}. Il sera noté \( \polyP(A)\).

    En ce qui concerne la notation \( A[X]\), voir \ref{SUBSECooLEKVooFBPSJz}.
\end{definition}

\begin{definition}  \label{DefDegrePoly}
    Soit \( P \in \polyP\), \( P \neq 0 \). On appelle \defe{degré}{degré!d'un polynôme} de $P$ le plus grand nombre naturel $n$ pour lequel le coefficient correspondant est non-nul. Ce naturel est noté \( \deg(P) \). L'ensemble des polynômes de degré \( n\) sur \( A\) sera noté \( \polyP_n(A)\).
\end{definition}

Notez que nous n'avons pas encore donné la notation \( A[X]\); nous verrons plus tard comment elle arrive. 

Vu que \( A^{(\eN)}\) est engendré par les \( e_i\), tout polynôme sur \( A\) s'écrit \( P=\sum_{i=1}^na_ie_i\).

\begin{definition}      \label{DEFooNXKUooLrGeuh}
    Nous ajoutons deux structures à \( A^{(\eN)}\).
    \begin{description}
        \item[L'évaluation] Si \( \alpha\in A\) et si \( P\in A^{(\eN)}\), nous définissons \( P(\alpha)\) par
            \begin{equation}        \label{EQooDJISooTEkMOw}
                P(\alpha)=(\sum_{i=0}^{n}a_ie_i)(\alpha)=\sum_{i=0}^na_i\alpha^i,
            \end{equation}
            étant entendu que \( \alpha^0=1\) dans \( A\).

            Cette définition s'étend immédiatement au cas où \( B\) est un anneau qui étend \( A\). Dans ce cas nous pouvons définir \( P(b)\) pour tout \( P\in \eA^{(\eN)}\) et \( b\in B\) avec la même formule \eqref{EQooDJISooTEkMOw}.
        \item[Le produit] C'est ici que la structure particulière de \( \eN\) est utilisée. Nous définissons le produit \( A^{\eN}\times A^{(\eN)}\to A^{(\eN)}\) de la façon suivante. Si \( (P_k)_{k\in \eN}\) est la suite (presque partout nulle) d'éléments de \( A\) qui définit \( P\) et si \( (Q_k)_{k\in \eN}\) est celle de \( Q\), nous notons
        \begin{equation}    \label{EQooTNCSooKklisb}
            (PQ)_n=\sum_{k=0}^nP_kQ_{n-k},
        \end{equation}
        et donc \( PQ=\sum_i(PQ)_ie_i\). Plus explicitement,
        \begin{equation}    \label{EQooCIBUooAgpxjL}
            (\sum_{i=0}^na_ie_i)(\sum_{j=0}^mb_je_j)=\sum_{k=0}^{\infty}\Big( \sum_{\substack{  (i,j)\in \eN^2 \\i+j=k}}a_ib_j \Big)e_k.
        \end{equation}
        Notons qu'à droite, la somme sur \( k\) est une somme finie.
    \end{description}
\end{definition}

\begin{proposition}     \label{PROPooGDQCooHziCPH}
    Soit un anneau \( A\). À propos de structure sur \( A^{(\eN)}\).
    \begin{enumerate}
        \item
            Avec le produit, l'ensemble \( A^{(\eN)}\) devient un anneau.
        \item
    L'application
    \begin{equation}
        \begin{aligned}
            g\colon A^{(\eN)}&\to A \\
            P&\mapsto P(\alpha)
        \end{aligned}
    \end{equation}
    est un morphisme d'anneaux\footnote{Définition \ref{DEFooSPHPooCwjzuz}.}. En particulier, \( (PQ)(\alpha)=P(\alpha)Q(\alpha)\).
    \end{enumerate}
\end{proposition}

\begin{proof}
    En plusieurs points
    \begin{subproof}
        \item[Anneau]
            L'identité pour le produit dans \( A^{(\eN)}\) est le polynôme donné par \( a_0=1\) et \( a_i=0\) pour \( i\neq 0\). Cela se vérifie en utilisant directement la définition \eqref{EQooCIBUooAgpxjL}. La distributivité aussi\quext{Je n'ai pas fait les calculs, écrivez-moi pour me dire si ça va facilement.}.
        \item[Le morphisme]
    Nous notons \( P_k\) les éléments de la suite définissant \( P\) et \( Q_k\) ceux de \( Q\). Alors nous avons
    \begin{equation}
        (P+Q)(\alpha)=\sum_k(P_k+Q_k)\alpha^k=\sum_kP_k\alpha^k+\sum_kQ_k\alpha^k=P(\alpha)+Q(\alpha).
    \end{equation}
    Vous aurez noté que la première égalité était la définition \eqref{EQooODBMooQKLUgd}. De même,
    \begin{subequations}
        \begin{align}
            P(\alpha)Q(\alpha)&=\big( \sum_nP_n\alpha^n \big)\big( \sum_kQ_k\alpha^k \big)=\sum_kQ_k\big( \sum_nP_n\alpha^n \big)\alpha^k=\sum_k\sum_nQ_kP_n\alpha^{n+k}\\
            &=\sum_m\big( \sum_{l=0}^mP_lQ_{m-l} \big)\alpha^m=\sum_m(PQ)_m\alpha^m=(PQ)(\alpha).
        \end{align}
    \end{subequations}
    \end{subproof}
\end{proof}

\begin{lemma}       \label{LEMooWVUXooQlaepO}
    Si \( \eA\) est commutatif, alors \( \eA^{(\eN)}\) est commutatif.
\end{lemma}

\begin{proof}
    Soient \( P,Q\in \eA^{(\eN)}\); pour rappel, le produit est donné par la définition \ref{EQooTNCSooKklisb}. L'application
    \begin{equation}
        \begin{aligned}
            \varphi\colon \{ 0,\ldots, n \}&\to \{ 0,\ldots, n \} \\
            k&\mapsto n-k 
        \end{aligned}
    \end{equation}
    est une bijection. Voici maintenant le calcul :
    \begin{subequations}
        \begin{align}
            (PQ)_n&=\sum_{k=0}^nP_kQ_{n-k}\\
            &=\sum_{k=0}^nP_{\varphi(k)}Q_{n-\varphi(k)}    \label{SUBEQooISTNooLPvSIy} \\
            &=\sum_{k=0}^nP_{n-k}Q_{k}\\
            &=\sum_{k=0}^nQ_lP_{n-k}      \label{SUBEQooCUMAooFjqqHW}\\
            &=(QP)_n.
        \end{align}
    \end{subequations}
    Justifications
    \begin{itemize}
        \item Pour \eqref{SUBEQooISTNooLPvSIy}. Lemme \ref{DEFooLNEXooYMQjRo} et le fait que \( \varphi\) soit une bijection.
        \item Pour \eqref{SUBEQooCUMAooFjqqHW}. Commutativité de \( \eA\).
    \end{itemize}
\end{proof}


%--------------------------------------------------------------------------------------------------------------------------- 
\subsection{La notation \texorpdfstring{$ A[X]$}{A[X]}}
%---------------------------------------------------------------------------------------------------------------------------
\label{SUBSECooLEKVooFBPSJz}

Si \( A\) est un anneau, nous avons déjà défini les polynômes en une indéterminée sur \( A\) comme étant le module \( A^{(\eN)}\) qui est devenu un anneau par la proposition \ref{PROPooGDQCooHziCPH}.

Le polynôme donné par la suite \( (a_n)_{n\in \eN}\) est souvent notée
\begin{equation}
    \sum_ka_kX^k.
\end{equation}
Par exemple avec \( a=(4,2,8)\) nous avons \( a=8X^2+2X+4\). Nous utiliserons souvent cette notation, qui est très pratique parce qu'elle s'adapte bien aux règles de multiplication et d'addition, en particulier la distributivité.

Il y a (au moins) deux façons de comprendre ce que signifie réellement «\( X\)» dans cette notation.

%///////////////////////////////////////////////////////////////////////////////////////////////////////////////////////////
\subsubsection{Première façon (qui botte en touche)}
%///////////////////////////////////////////////////////////////////////////////////////////////////////////////////////////

La première est de dire qu'il n'a pas de significations, et que \( X^2\) est un simple abus de notations pour écrire \( (0,0,1,0,\cdots)\). Avec cette façon de voir, nous notons l'anneau des polynômes sur \( A\) par «\( A[X]\)» où le \( X\) n'a pas d'autres raisons d'être que d'avertir le lecteur que nous réservons la lettre «\( X\)» pour utiliser la notation pratique des polynômes.

%///////////////////////////////////////////////////////////////////////////////////////////////////////////////////////////
\subsubsection{Seconde façon (la bonne)}
%///////////////////////////////////////////////////////////////////////////////////////////////////////////////////////////
\label{SUBSUBSECooPNBYooWXEHrg}

La seconde façon de voir le «\( X\)» est de nous rappeler que \( A^{(\eN)}\) a une base en tant de que module : les \( e_k\) dont nous avons parlé plus haut. Nous posons \( X=e_1\), et nous prenons la convention \( X^0=1\). Alors nous avons \( e_k=X^k\) et nous notons \( A[X]\)\nomenclature[A]{\( A[X]\)}{tous les polynômes de degré fini à coefficients dans \( A\)} l'anneau \(A^{(\eN)}\) exprimé avec \( X\).

Dans les deux cas, il n'est pas vraiment légitime d'écrire des égalités comme « \( P(X)=X^2+2X-3\) », et encore moins de dire «Le polynôme \( P\), \emph{évalué} en \( X\) vaut \( X^2+2X-3\)»  : il est plus correct d'écrire « \( P=X^2+2X-3\) ».

Le lemme suivant montre que ces notations tombent vraiment à point. La véritable difficulté de l'énoncé est de comprendre qu'il n'est pas trivial.

Nous avons vu dans la définition \ref{DEFooNXKUooLrGeuh} que si \( B\) est un anneau qui étant \( A\), et si \(P\in A[X] \), alors nous avons une définition de \( P(b)\) pour tout \( b\in B\). Nous appliquons cela à \( B=A[X]\), qui est un anneau qui étend \( A\). Autrement dit, si \( P\) et \( Q\) sont des polynômes, ça a un sens d'écrire \( P(Q)\) et le résultat sera un élément de \( A[X]\). 

Dans le cas particulier \( Q=X\), nous avons une chouette formule.
\begin{lemma}       \label{LEMooGKWQooVOyeDX}
    Nous avons
    \begin{equation}
        P(X)=P
    \end{equation}
    pour tout \( P\in A[X]\).
\end{lemma}

\begin{proof}
    Si \( P=(a_k)_{k\in \eN}\) alors par définition \( P(\alpha)=\sum_ka_k\alpha^k\) dès que \( \alpha\) est dans un anneau \( B\) qui étend \( A\). Nous considérons le cas particulier \( B=\eA[X]\) et \( \alpha=X\), c'est-à-dire \( Q=(0,1,0,\ldots)\), l'élément \( P(X)\) de \( A[X]\) vaut
    \begin{equation}        \label{EQooABULooFCEasf}
        \sum_ka_kX^k,
    \end{equation}
    qui est exactement \( P\) lui-même.
\end{proof}

Mais il faut bien comprendre que si \( P\) est le polynôme \( (-3,2,1,0,\ldots)\), noté \( X^2+2X-3\), écrire \( P(X)=X^2+2X-3\) est une pirouette de notations que rien ne justifie par rapport à simplement écrire \( P=X^2+2X-3\).



%--------------------------------------------------------------------------------------------------------------------------- 
\subsection{Action du groupe symétrique}
%---------------------------------------------------------------------------------------------------------------------------

\begin{definition}[Thème~\ref{THEMEooKZHBooRCULcr}]  \label{DefActionGroupe}
    Une \defe{action de groupe}{action}\index{action} \( G\) sur un ensemble \( E\) est la donnée, pour chaque élément \( g \in G\), d'une fonction \(\phi_g : E \to E \), de telle sorte que:
    \begin{gather*}
        \phi_{e}(x) = x, \hspace{2em} \forall x \in E;\\
        \phi_{gh}(x) = \phi_g (\phi_h (x)),  \hspace{2em} \forall g,h \in G, \forall x \in E.
     \end{gather*}
     On dit dans ce cas que \( G \) \defe{agit}{action} sur \( E \).
\end{definition}

Par souci de notations, nous notons \( \Poly_n(A)\) l'anneau des polynômes de \( n\) variables sur \( A\). La propriété universelle de \( \Poly_n(A)=A^{(\eN^n)}\) du théorème \ref{THOooPDZCooJnHbOd} nous donne une application
\begin{equation}
    g\colon \Fun\big(\eN^n,\Poly_n(A)\big)\to \Hom_A\big( \Poly_n(A),\Poly_n(A) \big)
\end{equation}
Avec cela nous pouvons énoncer et démontrer le lemme qui donne l'action de \( S_n\)\footnote{Définition du groupe symétrique \( S_n\) en \ref{DEFooJNPIooMuzIXd}.} sur \( \Poly_n(A)\).

\begin{lemma}[\cite{BIBooFDZDooJQLjlB}]       \label{LEMooIRVQooHvoNBq}
    Pour \( \sigma\in S_n\) nous définissons 
    \begin{equation}
        \begin{aligned}
            \phi_{\sigma}\colon \eN^n&\to \Poly_n(A) \\
            m&\mapsto e_{\sigma(m)}. 
        \end{aligned}
    \end{equation}
    Alors l'application
    \begin{equation}
        \begin{aligned}
            \rho\colon S_n&\to \Hom_A\big( \Poly_n(A),\Poly_n(A) \big) \\
            \sigma&\mapsto g(\phi{\sigma}) 
        \end{aligned}
    \end{equation}
    est une action\footnote{Définition \ref{DefActionGroupe}.}.
\end{lemma}

\begin{proof}
    Nous commençons par donner une expression à notre \( \rho\). Un élément de \( \Poly_n(A)\) est de la forme \( \sum_{m\in \eN^n}a_me_m\), et nous avons\footnote{La somme est définie par \ref{DEFooLNEXooYMQjRo}, et ça va être important. Ah oui, en réalité partout, les sommes sont finies parce que les \( a_m\) (\( m\in \eN^n\)) sont presque tous nuls. Il faudrait écrire sur la somme sur \(\{ m\in \eN^2\tq a_m\neq 0 \}\), mais vous vous imaginez la complication dans la notation.}
    \begin{equation}
        \rho(\sigma)\big( \sum_{m\in \eN^n}a_me_m \big)=\sum_ma_m\phi_{\sigma}(m)=\sum_ma_me_{\sigma(m)}.
    \end{equation}
    
    Nous avons tout de suite \( \rho(\id)=\id\).

    En ce qui concerne la composition, nous avons d'une part
    \begin{equation}
        \rho(\sigma_1\sigma_2)\big( \sum_ma_me_m \big)=g(\phi_{\sigma_1\sigma_2})\big( \sum_ma_me_m \big)=\sum_ma_me_{\sigma_1\sigma_2(m)},
    \end{equation}
    et d'autre part,
    \begin{subequations}
        \begin{align}
            \rho(\sigma_1)\rho(\sigma_2)\big( \sum_ma_me_m \big)&=\rho(\sigma_1)\big( \sum_ma_me_{\sigma_2(m)} \big)\\
            &=\rho(\sigma_1)\big( \sum_ma_{\sigma_2^{-1}(m)}e_m \big)   \label{SUBEQooTSCYooCUWiRz}\\
            &=\sum_ma_{\sigma_2^{-1}(m)}e_{\sigma_1(m)}\\
            &=\sum_ma_me_{\sigma_1\sigma_2(m)}      \label{SUBEQooQPGPooVvqJdT}
        \end{align}
    \end{subequations}
    La proposition \ref{PROPooJBQVooNqWErk} est utilisée pour \eqref{SUBEQooTSCYooCUWiRz} et pour \eqref{SUBEQooQPGPooVvqJdT}.
\end{proof}


%---------------------------------------------------------------------------------------------------------------------------
\subsection{Corps des fractions}
%---------------------------------------------------------------------------------------------------------------------------

\begin{definition}[\cite{ooGSDHooLgtHCb}]       \label{DEFooGJYXooOiJQvP}
    Soit un anneau commutatif et intègre\footnote{Définition~\ref{DEFooTAOPooWDPYmd}.} \( A\). Nous posons \( E=A\times A\setminus\{ 0 \}\), et nous définissons les deux opérations suivantes sur \( E\) :
    \begin{enumerate}
        \item       \label{ITEMooWBWHooYsXFkO}
            \( (a,b)+(c,d)=(ad+cb,bd)\);
        \item       \label{ITEMooGOOIooCHqLRl}
            \( (a,b)(c,d)=(ac,bd)\).
    \end{enumerate}
    Et aussi la relation d'équivalence \( (a,b)\sim(c,d)\) si et seulement si \( ad=bc\).

    Le \defe{corps des fractions}{corps!des fractions} de \( A\) est le quotient
    \begin{equation}
        \Frac(A)=\big( A\times A\setminus\{ 0 \} \big)/\sim.
    \end{equation}
    Nous notons \( a/b\) la classe de \( (a,b)\).

    Lorsque \( A\) est un anneau de polynômes\footnote{Définition \ref{DEFooFYZRooMikwEL}.}, alors les éléments de \( \Frac(A)\) sont des \defe{fractions rationnelles}{fractions!rationnelles}.
\end{definition}
Le fait que \( A\) soit intègre est important pour être certain que \( bd\neq 0\) sous l'hypothèse que \( b,d\neq 0\).

La proposition suivante montre encore que le corps des fractions est le plus petit corps que l'on puisse imaginer à partir d'un anneau.
\begin{proposition}[\cite{BIBooZFPUooIiywbk, MonCerveau}]       \label{PROPooIJBEooDjsoHr}
    Soit un anneau commutatif \( A\). Tout corps commutatif contenant un sous-anneau isomorphe\footnote{Morphisme d'anneaux, définition \ref{DEFooSPHPooCwjzuz}.} à \( A\) contient un sous-corps isomorphe à \( \Frac(A)\).
\end{proposition}

\begin{proof}
    Soit un corps \( \eK\) contenant un sous-anneau \( A'\) isomorphe à \( A\). Nous notons \( \sigma\colon A'\to A\) un isomorphisme d'anneaux entre \( A'\) et \( A\). 

    \begin{subproof}
    \item[Une partie bien choisié]

    Nous considérons la partie suivante de \( \eK\) :
    \begin{equation}
        S=\{ ab^{-1}\tq a,b\in A' \}.
    \end{equation}

\item[\( S\) est un corps]
Deux éléments arbitraires de \( S\) sont \( ab^{-1}\) et \( xy^{-1}\). Nous devons prouver plusieurs choses.
\begin{subproof}
\item[Neutres]
    En prenant \( a=b=1\) nous avons \( ab^{-1}=1\in S\). En prenant \( a=0\) et \( b=1\) nous avons \( ab^{-1}=0\in S\).
\item[Somme]
    Il faut remarquer que \( ab^{-1}+xy^{-1}=(ay+xb)(by)^{-1}\). En effet,
    \begin{subequations}
        \begin{align}
            (ay+xb)(xb)^{-1}&=(ay+xb)y^{-1}b^{-1}\\
            &=ayy^{-1}b^{-1}+xby^{-1}b^{-1}     \label{SUBEQooRGPSooXaBGyx}\\
            &=ab^{-1}+xy^{-1}       \label{SUBEQooOHJGooWrfPow}
        \end{align}
    \end{subequations}
    Justifications :
    \begin{itemize}
        \item Pour \eqref{SUBEQooRGPSooXaBGyx}. Distributivité.
        \item Pour \eqref{SUBEQooOHJGooWrfPow}. Commutativité dans \( A\).
    \end{itemize}
\item[Produit]
    Il s'agit du même genre de calculs en utilisant les mêmes propriétés. Nous avons que
    \begin{equation}
        (ab^{-1})(xy^{-1})=(ax)(by)^{-1}.
    \end{equation}
\end{subproof}

\item[Ce qui va être notre isomorphisme]


    Ensuite nous montrons que l'application
    \begin{equation}
        \begin{aligned}
            \varphi\colon S&\to \Frac(A) \\
            ab^{-1}&\mapsto \sigma(a)/\sigma(b)
        \end{aligned}
    \end{equation}
    est bien définie et est un isomorphisme de corps.

        \item[Bien définie]

            Si \( ab^{-1}=xy^{-1}\) alors \( ay=xb\). Vu que \( \sigma\) est un isomorphisme nous avons aussi \( \sigma(a)\sigma(y)=\sigma(x)\sigma(b)\) et donc \( \sigma(a)/\sigma(b)=\sigma(x)/\sigma(y)\) par définition des classes de \( \Frac(A)\).
        \item[Morphisme]
            Deux éléments arbitraires de \( S\) sont \( ab^{-1}\) et \( xy^{-1}\). Calculons un peu :
            \begin{subequations}
                \begin{align}
                    \varphi\big( (ab^{-1})(xy^{-1}) \big)&=\varphi(axy^{-1}b^{-1})      \label{SUBEQooRONTooKVTRdZ}\\
                    &=\varphi\big( (ax)(by)^{-1} \big)      \label{SUBEQooNOTAooZVJymC}\\
                    &=\sigma(ax)/\sigma(by)\\
                    &=\big(\sigma(a)/\sigma(b)\big)\big(\sigma(x)/\sigma(y)\big)            \label{SUBEQooVQUOooVyVjEU} \\
                    &=\varphi(ab^{-1})\varphi(xy^{-1}).
                \end{align}
            \end{subequations}
            Justifications :
            \begin{itemize}
                \item Pour \eqref{SUBEQooRONTooKVTRdZ}. Commutativité dans \( A\).
                \item Pour \eqref{SUBEQooNOTAooZVJymC}. Associativité dans \( A\).
                \item Pour \eqref{SUBEQooVQUOooVyVjEU}. Définition \ref{DEFooGJYXooOiJQvP}\ref{ITEMooGOOIooCHqLRl} de la multiplication de fractions.
            \end{itemize}


        \item[Surjectif]

            Tout élément de \( \Frac(A)\) est de la forme \( a'/b'\) avec \( a',b'\in A\), et donc de la forme \( \sigma(a)/\sigma(b)\) avec \( a,b\in A'\). Un tel élément est l'image par \( \varphi\) de \( ab^{-1}\in S\).

        \item[Injectif]

            Si \( \varphi(ab^{-1})=\varphi(xy^{-1})\) alors \( \sigma(a)/\sigma(b)=\sigma(x)/\sigma(y)\), et par définition des classes nous avons \( \sigma(a)\sigma(y)=\sigma(b)\sigma(x)\). De là nous avons \( \sigma(ay)=\sigma(bx)\) et donc \( ay=bx\) (parce que \( \sigma\) est un isomorphisme). Nous en déduisons que \( ab^{-1}=xy^{-1}\).
    \end{subproof}
\end{proof}

\begin{normaltext}
    Soit un anneau \( A\) et son anneau des polynômes \( \Poly(A)\). Si \( \alpha\in A\), nous avons la définition \ref{DEFooNXKUooLrGeuh} qui donne l'évaluation \( P(\alpha)\).

    Si par contre \( P\) et \( Q\) sont des polynômes sur \( A\), nous n'avons pas encore définit ce que serait l'évaluation de la fraction rationnelle \( P/Q\) en \( \alpha\). Nous comblons à présent ce manque.
\end{normaltext}

\begin{definition}[Évaluation d'une fraction rationnelle]       \label{DEFooLBIWooCPCaSY}
    Soit un corps \( \eK\) contenant l'anneau \( A\). Si \( R=P/Q\in \Frac(A)\) et si \( \alpha\in \eK\) nous définissons\footnote{Les fractions rationnelles, définition \ref{DEFooGJYXooOiJQvP}.}
    \begin{equation}
        R(\alpha)=(P/Q)(\alpha)=P(\alpha)Q^{-1}(\alpha).
    \end{equation}
    Dans cette formule, les polynômes, l'inverse et le produit sont calculés dans \( \eK\) et non dans \( A\).
\end{definition}

\begin{theoremDef}     \label{ThogbhWgo}
    Soit \( \eA\) un anneau commutatif intègre.

    \begin{enumerate}
        \item
    Il existe un couple \( (\eK,\epsilon)\) où \( \eK\) est un corps commutatif et \( \epsilon\colon \eA\to \eK\) est un morphisme injectf d'anneaux tels que pour tout \( \lambda\in\eK\), il existe \( (a,b)\in \eA\times \eA^*\) tels que
    \begin{equation}
        \lambda=\epsilon(a)\big( \epsilon(b) \big)^{-1}
    \end{equation}
\item
    Si \( (\eK',\epsilon')\) est un autre couple qui vérifie la propriété, les corps \( \eK\) et \( \eK'\) sont isomorphes.

    Le corps \( \eK\) associé à l'anneau \( \eA\) est le \defe{corps des fractions}{corps!des fractions}\index{fractions (corps)} de \( \eA\), et sera noté \( \Frac(\eA)\).\nomenclature[A]{\( \Frac(\eA)\)}{Le corps des fractions de l'anneau \( \eA\)}

\item
    Nous posons
    \begin{equation}
        \begin{aligned}
            \sigma\colon \eA\times \eA^*&\to \eK \\
            (a,b)&\mapsto \epsilon(a)\big( \epsilon(b) \big)^{-1}. 
        \end{aligned}
    \end{equation}
    Nous avons
    \begin{equation}
        \sigma(xa, xb)=\sigma(a,b)
    \end{equation}
    pour tout \( a,b,x\in \eA\).
    \end{enumerate}
\end{theoremDef}

%---------------------------------------------------------------------------------------------------------------------------
\subsection{Corps totalement ordonné}
%---------------------------------------------------------------------------------------------------------------------------

\begin{definition}      \label{DefKCGBooLRNdJf}
    Ordre et choses reliées dans un corps.
    \begin{enumerate}
        \item \label{ITEMooOOOVooJWwIQr}
            Un corps \( \eK\) est \defe{totalement ordonné}{ordre!dans un corps}\index{corps!ordonné} s'il existe une relation d'ordre total\footnote{Définition~\ref{DEFooVGYQooUhUZGr}.} tel que
            \begin{enumerate}
                \item       \label{ITEMooZISJooWNxnBj}
                    \( x\leq y\) implique \( x+z\leq y+z\) pour tout \( x,y,z\in \eK\)
                \item   \label{CONDooBYYDooElXgPO}
                    \( x\geq 0\) et \( y\geq 0\) implique \( xy\geq 0\).
            \end{enumerate}
        \item       \label{ItemooWUGSooRSRvYC}
            Si \( \eK\) est un corps totalement ordonné, nous y définissons la valeur absolue par
            \begin{equation}     
                | x |=\begin{cases}
                    x    &   \text{si }x\geq 0\\
                    -x    &    \text{si } x\leq 0.
                \end{cases}
            \end{equation}
        \item       \label{ItemVXOZooTYpcYN}
    La suite \( (x_n)\) dans le corps totalement ordonné \( \eK\) est \defe{de Cauchy}{suite!de Cauchy!dans un corps} si pour tout \( \epsilon\in \eK^+\), il existe \( N\in \eN\) tel que si \( p,q\geq N\) alors \( | x_p-x_q |\leq \epsilon\).
\item       \label{ITEMooDERQooLmJwFR}
    La suite \( (x_n)\) dans le corps totalement ordonné \( \eK\) est \defe{convergente}{convergence!suite!dans un corps} s'il existe \( q\in \eK\) tel que pour tout \( \epsilon\in \eK^+\), il existe \( N\) tel que si \( k\geq N\) alors \( | x_k-q |\leq \epsilon\).
\item   \label{ItemooDZQKooPsqeRf}
            Un corps \( \eK\) est \defe{archimédien}{corps!archimédien}\index{archimédien} s'il est totalement ordonné et si pour tout \( x,y\in \eK\) avec \( x>0\), il existe \( n\in \eN\) tel que \( nx\geq y\).
        \item       \label{ITEMooKZZYooDaidGU}
            Un corps totalement ordonné est \defe{complet}{corps!complet}\index{complet!corps} si toute suite de Cauchy y est convergente.
        \item       \label{ITEMooMWASooEzhVyh}
            Si \( a,\epsilon\in \eK\) avec \( \epsilon>0\) alors nous définissons la \defe{boule ouverte}{boule dans un corps} de centre \( a\) et de rayon \( \epsilon\) par
            \begin{equation}
                B(a,\epsilon)=\{ x\in \eK\tq | a-x |<\epsilon \},
            \end{equation}
            et la \defe{boule fermée}{boule dans un corps} par
            \begin{equation}
                \overline{ B(a,\epsilon) }=\{ x\in \eK\tq | a-x |\leq \epsilon \}.
            \end{equation}

    \end{enumerate}
\end{definition}

\begin{lemma}
    Une suite \( (x_k)\) converge vers \( q\) si et seulement si pour tout \( \epsilon>0\), il existe \( N>0\) tel que \( x_k\in B(q,\epsilon)\) pour tout \( k\geq N\).
\end{lemma}

\begin{proof}
    Il s'agit de mettre côte à côte les points~\ref{ITEMooDERQooLmJwFR} et~\ref{ITEMooMWASooEzhVyh} de la définition \ref{DefKCGBooLRNdJf}.
\end{proof}

\begin{normaltext}
    Ces boules prendront une nouvelle force avec le super-théorème~\ref{ThoORdLYUu}.
\end{normaltext}

Parmi ces définitions, celles de suite convergente, de Cauchy et de corps complet seront utilisées dans le cas de \( \eQ\) (et de \( \eR\) pour la complétude). Elles seront prouvées être équivalentes aux définitions topologiques dans le cas particulier de \( \eR\) et \( \eQ\) lorsque la topologie métrique sera définie. Dans cet état d'esprit nous n'allons pas démontrer tout de suite que \( \eR\) est un corps complet. Nous allons directement démontrer que c'est un espace topologique complet.

\begin{lemma}[Règle des signes\cite{ooTKEHooQuaFuD}]        \label{LEMooXJTAooZauchx}
    Soit un corps totalement ordonné \( \eK\) ainsi que \( x,y\in \eK\). Nous avons :
    \begin{enumerate}
        \item
            Si \( x\leq 0\) et \( y\leq 0\) alors \( xy\geq 0\).
            \item
                Si \( x\leq 0\) et \( y\geq 0\) alors \( xy\leq 0\).
\item
    Si \( x\geq 0\) et \( y\leq 0\) alors \( xy\leq 0\).
\item       \label{ITEMooRGYAooCUIfss}
            \( 0\leq 1\).
        \item       \label{ITEMooMRNHooLglPKn}
            Si \( x\geq 0\) alors \( x^{-1}\geq 0\).
    \end{enumerate}
\end{lemma}

\begin{lemma}[Propriétés de la valeur absolue]  \label{LemooANTJooYxQZDw}
    Soit \( \eK\) un corps totalement ordonné. Si \( x,y\in \eK\) alors\footnote{La «valeur absolue» est définie en \eqref{DefKCGBooLRNdJf}\ref{ItemooWUGSooRSRvYC}.}
    \begin{enumerate}
        \item       \label{ItemooNVDIooSuiSoB}
            Si \( x\geq 0\) alors \( -x\leq 0\).
        \item       \label{ITEMooVNAZooSxmtuH}
            Si \( x\leq 0\) alors \( -x\geq 0\).
        \item       \label{ITEMooSDNHooDnjScE}
            \( | x |\geq 0\)
        \item       \label{ITEMooLQLTooTJTPVM}
            \( | x |=0\) si et seulement si \( x=0\)
        \item       \label{ITEMooVJAEooOEatzY}
            \( | -x |=| x |\).
        \item\label{ItemooOMKNooRlanvk}
            \( | x+y |\leq | x |+| y |\).
    \end{enumerate}
\end{lemma}

\begin{proof}
    Point par point
    \begin{subproof}
    \item[\ref{ItemooNVDIooSuiSoB}]
            Nous partons de \( x\geq 0\) et nous ajoutons \( -x\) des deux côtés en profitant de la définition d'un corps totalement ordonné : \( x-x\geq -x\) et donc \( 0\geq-x\), c'est-à-dire \( -x\leq 0\).
        \item[\ref{ITEMooVNAZooSxmtuH}]
            Nous partons de \( x\leq 0\) et nous ajoutons \( -x\) des deux côtés.
        \item[\ref{ITEMooSDNHooDnjScE}]
            Si \( x\geq 0\) alors c'est vrai. Sinon, \( x\leq 0\) et \( | x |=-x\geq 0\) par le point~\ref{ItemooNVDIooSuiSoB}.
        \item[\ref{ITEMooLQLTooTJTPVM}]
            Si \( x=0\) alors \( x=-x\) et \( | x |=0\). Au contraire si \(x\neq 0\) alors \( -x\neq 0\) et que \( x\) soit positif ou négatif, nous aurons toujours \( \pm x\neq 0\).
        \item[\ref{ITEMooVJAEooOEatzY}]
            Il faut décomposer en deux cas selon que \( x\geq 0\) et \( x\leq 0\). Supposons \( x\geq 0\). Alors d'une part \( | x |=x\). D'autre part \( -x\leq 0\) par le point \ref{ItemooNVDIooSuiSoB}, de telle sorte que
            \begin{equation}
                | -x |=-(-x)=x.
            \end{equation}
            Nous avons donc \( | x |=| -x |=x\).

            Le même raisonnement tient avec \( x\leq 0\).
        \item[\ref{ItemooOMKNooRlanvk}]
            Nous supposons que \( x\leq y\) et nous distinguons divers cas suivant la positivité de \( x\) et \( y\).
            \begin{enumerate}
                \item
                    Si \( x,y\geq 0\). Dans ce cas, \( x+y\geq y\geq 0\), donc \( | x+y |=x+y=| x |+| y |\).
                \item
                    Si \( x,y\leq 0\). Dans ce cas, \( x+y\leq 0\) et nous avons \( | x+y |=-x-y=| x |+| y |\).
                \item
                    Si \( x\leq 0\) et \( y\geq 0\). Nous subdivisons encore en deux cas suivant que \( x+y\) est positif ou négatif. Si \( x+y\geq 0\), alors nous écrivons successivement
                    \begin{subequations}
                        \begin{align}
                            x&\leq 0\\
                            x+y&\leq y\leq y+| x |=| x |+| y |
                        \end{align}
                    \end{subequations}
                    et donc \( | x+y |=x+y\leq | x |+| y |\).

                    Nous supposons à présent que \( x\leq 0\), \( y\geq 0\) et \( x+y\leq 0\). Dans ce cas il suffit d'écrire \( | x+y |=| (-x)+(-y) |\) pour retomber dans le cas précédent à inversion près de \( x\) et \( y\).
            \end{enumerate}
    \end{subproof}
\end{proof}

\begin{remark}      \label{RemooJCAUooKkuglX}
    La partie~\ref{ItemooOMKNooRlanvk} est très importante parce que c'est elle qui fera presque toutes les majorations dont nous aurons besoin en analyse. En effet elle donne l'inégalité triangulaire de la façon suivante : si \( x,y,z\in \eK\) nous avons
    \begin{equation}
        | x-y |= |  (x-z)+(z-y) |\leq | x-z |+| z-y |.
    \end{equation}
\end{remark}

\begin{lemma}[À propos de boules]
    Soient un corps totalement ordonné \( \eK\) et des éléments \( x,y\in \eK\). Soit aussi \( \epsilon>0\) dans \( \eK\). Nous avons :
    \begin{enumerate}
        \item       \label{ITEMooXJGVooSebiip}
            Nous avons \( y\in B(x,\epsilon)\) si et seulement si \( x-\epsilon<y<x+\epsilon\).
        \item       \label{ITEMooRUBBooRayiMs}
            Si \( y\in  \overline{ B(x,\epsilon) }  \) alors \( y\in B(x,\epsilon')\) pour tout \( \epsilon'>\epsilon\).
    \end{enumerate}
\end{lemma}

\begin{proof}
    Pour rappel,
    \begin{equation}
        | x-y |=\begin{cases}
               x-y    &     \text{si } x-y\geq 0 \\
                    y-x    &    \text{si } x-y\leq 0.
               \end{cases}
    \end{equation}
    Nous pouvons maintenant démontrer nos choses.
    \begin{subproof}
        \item[\ref{ITEMooXJGVooSebiip}]
            En deux parties.
            \begin{subproof}
            \item[\( \Rightarrow\)]
            Nous supposons que \( | x-y |<\epsilon\).

            Si \( x-y\geq 0\) alors l'hypothèse signifie \( x-y<\epsilon\), ce qui donne \( y>x-\epsilon\). Mais l'inégalité \( x-y\geq 0\) donne également \( x\geq y\) et donc \( x+\epsilon\geq y+\epsilon>y\). Notez le jeu de l'inégalité non stricte qui se change en inégalité stricte.

            Si \( x-y\leq 0\) nous pouvons faire le même raisonnement.

            \item[\( \Leftarrow\)]
            Des inégalités \( x-\epsilon<y\) et \( y<x+\epsilon\) nous tirons \( x-y<\epsilon\) et \( y-x<\epsilon\). Donc quel que soit le signe de \( x-y\) nous avons toujours \( | x-y |<\epsilon\).
            \end{subproof}

        \item[\ref{ITEMooRUBBooRayiMs}]

            C'est immédiat parce que
            \begin{equation}
                | x-y |\leq \epsilon<\epsilon'.
            \end{equation}
    \end{subproof}
\end{proof}


\begin{lemma}       \label{LEMooVZNCooRJatKK}
    Tout corps totalement ordonné est de caractéristique nulle.
\end{lemma}

%+++++++++++++++++++++++++++++++++++++++++++++++++++++++++++++++++++++++++++++++++++++++++++++++++++++++++++++++++++++++++++
\section{Les rationnels}
%+++++++++++++++++++++++++++++++++++++++++++++++++++++++++++++++++++++++++++++++++++++++++++++++++++++++++++++++++++++++++++

Une construction très explicite est faite dans \cite{RWWJooJdjxEK}. Ici nous allons prendre plus court :
\begin{definition}
    Le corps des fractions de \( \eZ\) (définition~\ref{DEFooGJYXooOiJQvP}) est noté \( \eQ\) et ses éléments sont les \defe{rationnels}{rationnels}.
\end{definition}

\begin{normaltext}
    Les résultats énoncés ici sont utilisés plus bas et servent de guide à \randomGender{un éventuel contributeur}{une éventuelle contributrice} qui voudrait écrire une partie dédiée à \( \eQ\) et ses propriétés de base\quext{Par exemple, définir une relation d'ordre sur \( \eQ\) et expliciter l'inclusion de \( \eZ\) dans \( \eQ\).}. Nous espérons que des preuves se trouvent dans \cite{RWWJooJdjxEK}. En tout cas, \randomGender{il est invité}{elle est invitée} à ne rien prendre comme évident.
\end{normaltext}

\begin{lemma} \label{LEMooEBTIooGMoHsj}
    Tout rationnel est majoré par un naturel.
\end{lemma}

\begin{proposition}     \label{PROPooDHIAooZysvNs}
    L'ensemble des rationnels est dénombrable.
\end{proposition}

\begin{proposition}     \label{PROPooBTCCooVVvaeL}
    Si \( q<1\), alors \( qx<x\) pour tout \( x\in \eQ^+\).
\end{proposition}

\begin{proposition}     \label{PROPooMXGPooDUkOuv}
    Le corps \( \eQ\) est archimédien\footnote{Définition~\ref{DefKCGBooLRNdJf}\ref{ItemooDZQKooPsqeRf}.}.
\end{proposition}

%--------------------------------------------------------------------------------------------------------------------------- 
\subsection{Caractéristique}
%---------------------------------------------------------------------------------------------------------------------------

\begin{lemma}       \label{LEMooYCPUooNxEPhB}
    Le corps \( \eQ\) est de caractéristique\footnote{Définition \ref{LEMDEFooVEWZooUrPaDw}.} nulle.
\end{lemma}

%--------------------------------------------------------------------------------------------------------------------------- 
\subsection{Suite de Cauchy dans un corps totalement ordonné}
%---------------------------------------------------------------------------------------------------------------------------

\begin{lemma}[\cite{ooIBWOooSjOvXd, MonCerveau}]        \label{LEMooLTBIooSZnvsQ}
    Tout corps commutatif de caractéristique nulle contient un sous-corps isomorphe à \( \eQ\).
\end{lemma}

\begin{proof}
    Soit un corps \( \eK\) de caractéristique nulle. Nous savons du lemme \ref{LEMDEFooVEWZooUrPaDw} que
    \begin{equation}
        \begin{aligned}
            \mu\colon \eZ&\to \eK \\
            n&\mapsto n1_{\eK} 
        \end{aligned}
    \end{equation}
    est un morphisme d'anneaux vérifiant \( \ker(\mu)=\{ 0 \}\). Nous posons \( Z=\mu(\eZ)\). L'application \( \mu\colon \eZ\to Z\) est un isomorphisme d'anneaux. Prouvons cela :
    \begin{subproof}
    \item[Morphisme]
        L'application \( \mu\) est un morphisme par le lemme \ref{LEMDEFooVEWZooUrPaDw}.
    \item[Surjectif]
        Par définition les éléments de \( Z\) sont dans l'image de \( \eZ\).
    \item[Injectif] Si \( x,y\in \eZ\) vérifient \( \mu(x)=\mu(y)\), alors \( \mu(x-y)=0\) parce que \( \mu\) est un morphisme. Mais \( \eK\) est de caractéristique nulle, c'est à dire \( \ker(\mu)=\{ 0 \}\). Donc \( x-y=0\).
    \end{subproof}
    Le corps \( \eK\) contient donc un sous-anneau isomorphe à \( \eZ\). Vu que \( \eZ\) et \( \eK\) sont commutatifs, la proposition \ref{PROPooIJBEooDjsoHr} s'applique et \( \eK\) contient un sous-corps isomorphe à \( \Frac(\eZ)=\eQ\).
\end{proof}

La proposition suivante donne des précisions à propos du lemme \ref{LEMooLTBIooSZnvsQ}.

\begin{proposition}[\cite{MonCerveau}]      \label{PROPooKNROooFdgIeQ}
    Soit un corps totalement ordonné \( \eK\). Nous considérons l'application
    \begin{equation}
        \begin{aligned}
            \mu\colon \eZ&\to \eK \\
            n&\mapsto n\cdot 1_{\eK} 
        \end{aligned}
    \end{equation}
    et ensuite
    \begin{equation}
        \begin{aligned}
            \sigma\colon \eQ&\to \eK \\
            a/b&\mapsto \mu(a)\mu(b)^{-1}. 
        \end{aligned}
    \end{equation}
    Alors
    \begin{enumerate}
        \item
            L'application \( \sigma\) est bien définie.
        \item
            L'application \( \sigma\) est un morphisme de corps.
        \item
            Si \( q\leq q'\) dans \( \eQ\), alors \( \sigma(q)\leq \sigma(q')\).
    \end{enumerate}
\end{proposition}

\begin{proof}
    En plusieurs morceaux.
    \begin{subproof}
    \item[\( \sigma\) est bien définie]
    Montrons que \( \sigma\) est bien définie. Pour cela nous considérons \( a,b,x,y\in \eZ\) tels que \( a/b=x/y\) dans \( \eQ\). Par définition des classes (définition \ref{DEFooGJYXooOiJQvP} du corps des fractions), nous avons \( ay=bx\) dans \( \eQ\). Vu que \( \mu\) est un morphisme nous avons alors
    \begin{equation}
        \mu(a)\mu(y)=\mu(b)\mu(x)
    \end{equation}
    et donc \( \mu(a)\mu(b)^{-1}=\mu(x)\mu(y)^{-1}\), c'est à dire \( \sigma(a/b)=\sigma(x/y)\). L'application \( \sigma\) est donc bien définie.

\item[Morphisme pour la somme]

    L'application \( \mu\) est un morphisme d'anneaux, comme déjà dit depuis le lemme \ref{LEMDEFooVEWZooUrPaDw}. Notons aussi que, parce que \( \eK\) est commutatif,
    \begin{equation}
        \mu(qy)^{-1}=\mu(q)^{-1}\mu(y)^{-1}.
    \end{equation}

    En utilisant la définition \ref{DEFooGJYXooOiJQvP}\ref{ITEMooWBWHooYsXFkO} de la somme nous avons
    \begin{subequations}
        \begin{align}
            \sigma(p/q+x/y)&=\sigma\big( (py+qx)/qy \big)\\
            &=\big[ \mu(py)+\mu(qx) \big]\mu(qy)^{-1}\\
            &=\mu(py)\mu(qy)^{-1}+\mu(qx)\mu(qy)^{-1}\\
            &=\mu(p)\mu(q)^{-1}+\mu(x)\mu(y)^{-1}\\
            &=\sigma(p/q)+\sigma(x/y).
        \end{align}
    \end{subequations}

\item[Morphisme pour le produit]
    Même genre de calculs que pour la somme.
\item[Croissante]

    Nous savons aussi par lemme \ref{LEMooXJTAooZauchx}\ref{ITEMooRGYAooCUIfss} que \( 1\geq 0\). Vu que \( \mu\) est un morphisme d'anneaux,
    \begin{equation}
        \mu(n+1)=\mu(n)+\mu(1)=\mu(n)+1
    \end{equation}

    La définition \ref{DefKCGBooLRNdJf}\ref{ITEMooZISJooWNxnBj} dit alors que \( \mu(n)\geq 0\) pour tout \( n\in \eN\). Nous avons pour la même raison que si \( m\geq n\) dans \( \eN\), alors \( \mu(m)\geq\mu(n)\) dans \( \eK\).

        Soient maintenant \( p,q\in \eN\), et prouvons que \( \sigma(p/q)\geq 0\). D'abord
        \begin{equation}
            \sigma(p/q)=\mu(p)\mu(q)^{-1}
        \end{equation}
        où \( \mu(p)\geq 0\) et \( \mu(q)\geq 0\). Ensuite le lemme \ref{LEMooXJTAooZauchx}\ref{ITEMooMRNHooLglPKn} nous indique que \( \mu(q)^{-1}\geq 0\). Enfin la condition \ref{DefKCGBooLRNdJf}\ref{CONDooBYYDooElXgPO} nous permet de conclure que \( \sigma(p/q)\geq 0\).

        Finalement, si \( q_1\geq q_2\) dans \( \eQ\), alors \( q_1-q_2\geq 0\), et nous avons
        \begin{equation}
            \sigma(q_1)=\sigma(q_2+q_1-q_2)=\sigma(q_2)+\sigma(q_1-q_2)\geq \sigma(q_2)
        \end{equation}
        par la condition \ref{DefKCGBooLRNdJf}\ref{ITEMooZISJooWNxnBj}.
    \end{subproof}
\end{proof}


\begin{normaltext}      \label{NORMooJRRZooTwTVYG}
    Si \( \eK\) est un corps totalement ordonné, la proposition \ref{PROPooKNROooFdgIeQ} nous donne un morphisme de corps \( \sigma\colon \eQ\to \eK\) qui respecte l'ordre. Pour \( q\in \eQ\) et \( k\in \eK\) nous notons
    \begin{equation}        \label{EQooERFIooMpZVEs}
        qk=\sigma(q)k.
    \end{equation}
    Nous pourrons donc écrire \( \frac{ k }{2}\) pour \( \sigma(1/2)k\).
\end{normaltext}

Le lemme suivant explique que la notation \eqref{EQooERFIooMpZVEs} n'est pas complètement idiote.
\begin{lemma}       \label{LEMooWIONooGTKfcJ}
    Soit un corps commutatif totalement ordonné \( \eK\). Soit \( k\in \eK\). Nous avons
    \begin{equation}
        k+k=2k.
    \end{equation}
\end{lemma}

\begin{proof}
    Vu que \(  \sigma\colon \eQ\to \eK \) est un morphisme, il vérifie \( \sigma(1)=1\), donc
    \begin{subequations}
        \begin{align}
            k+k&=\sigma(1)k+\sigma(1)k\\
            &=\big( \sigma(1)+\sigma(1) \big)k\\
            &=\sigma(2)k\\
            &=2k.
        \end{align}
    \end{subequations}
\end{proof}


\begin{proposition}     \label{PROPooTFVOooFoSHPg}
    Toute suite convergente dans un corps totalement ordonné est de Cauchy.
\end{proposition}

\begin{proof}
    Soit un corps totalement ordonné \( \eK\) et une suite \( x_n\stackrel{\eK}{\longrightarrow}x\). Soit \( \epsilon>0\). Il est important de se rendre compte que \( \epsilon\in \eK\) et que l'inégalité est au sens de l'ordre dans \( \eK\); en particulier ce n'est pas \( \epsilon\in \eR\) ni \( \epsilon\in \eQ\). D'ailleurs nous n'avons pas encore défini \( \eR\).

    Vu que \( (x_n)\) converge vers \( x\), il existe \( N\in \eN\) tel que pour tout \( k>N\),
    \begin{equation}
        | x_k-x |<\epsilon.
    \end{equation}
    Soient \( p,q>N\). Alors en utilisant la majoration du lemme~\ref{LemooANTJooYxQZDw}\ref{ItemooOMKNooRlanvk},
    \begin{equation}        \label{EQooMQYGooLpgEQO}
        | x_p-x_q |=\big| (x_p-x)+(x-x_q) \big|\leq | x_p-x |+| x-x_q |\leq 2\epsilon.
    \end{equation}
    En analyse en général, on s'arrête là et on dit que \( (x_n)\) est de Cauchy parce qu'il n'y a pas vraiment de différence entre réaliser une majoration avec \( \epsilon\) ou avec \( 2\epsilon\). Détaillons toutefois comment ça se passe dans le cas où \( \epsilon\) est un élément d'un corps totalement ordonné.

    Le \( 2\epsilon\) arrivant à la fin de \eqref{EQooMQYGooLpgEQO} est en réalité \( \epsilon+\epsilon=\sigma(2)\epsilon\) en vertu de ce qui est raconté en \ref{NORMooJRRZooTwTVYG} et en vertu du lemme \ref{LEMooWIONooGTKfcJ}.

    Considérons \( \epsilon'=\sigma(1/2)\epsilon\), que nous pouvons noter \( \epsilon'=\epsilon/2\). Vu que \( \epsilon'>0\), il existe un \( N'\) tel que pour tout \( p,q>N'\) nous ayons
    \begin{equation}
        | x_p-x_q |\leq 2\epsilon'=\sigma(2)\sigma(1/2)\epsilon=\sigma(1)\epsilon=\epsilon.
    \end{equation}
    Ce dernier \( \epsilon\) étant bien celui fixé au début de la preuve, nous en déduisons que \( (x_n)\) est de Cauchy.
\end{proof}

%---------------------------------------------------------------------------------------------------------------------------
\subsection{Suites de Cauchy dans les rationnels}
%---------------------------------------------------------------------------------------------------------------------------

\begin{proposition}[\cite{RWWJooJdjxEK}]        \label{PropFFDJooAapQlP}
    Principales propriétés des suites de Cauchy dans \( \eQ\).
    \begin{enumerate}
        \item       \label{ItemRKCIooJguHdji}
            Toute suite convergente est de Cauchy\footnote{Et non la réciproque, qui sera justement la grande innovation des nombres réels.}.
        \item       \label{ItemRKCIooJguHdjii}
            Toute suite de Cauchy est bornée.
        \item       \label{ItemRKCIooJguHdjiii}
            Si \( x_n\to 0\) et si \( (y_n)\) est bornée, alors \( x_ny_n\to 0\)
        \item
            Si \( (x_n)\) et \( (y_n)\) sont de Cauchy alors \( (x_n+y_n)\), \( (x_n-y_n)\) et \( (x_ny_n)\) sont également de Cauchy.
        \item       \label{ITEMooIAFSooAIUpAN}
            Si il existe \( a,b\in \eQ\) tels que \( x_n\to a \) et \( y_n\to b \) alors \( x_n+y_n\to a+b\), \( x_n-y_n\to a-b\) et \(  x_ny_n\to ab  \).
        \item   \label{ItemRKCIooJguHdjvi}
            Soit \( (x_n)\) une suite de Cauchy qui ne converge pas vers zéro. Alors il existe \( n_0\) tel que la suite \( \left( \frac{1}{ x_n } \right)_{n\geq n_0}\) soit de Cauchy.
    \end{enumerate}
\end{proposition}

\begin{proof}
    Point par point.
    \begin{enumerate}
        \item
            C'est la proposition~\ref{PROPooTFVOooFoSHPg}.
        \item
            Soit \( (x_n)\) une suite de Cauchy dans \( \eQ\). Avec \( \epsilon=1\) dans la définition, si \( q>N_1\), nous avons
            \begin{equation}
                | x_q-x_{N_1} |\leq 1.
            \end{equation}
            Et donc pour tout \( q\) plus grand que \( N_1\), \( x_N-1\leq x_q\leq x_N+1\), ou encore, pour tout \( n\) :
            \begin{equation}
                | x_n |\leq\max\{ | x_1 |,| x_2 |,\ldots,| x_N |,| x_N+1 | \}.
            \end{equation}
            La suite est donc bornée.
        \item
            Soit \(\epsilon>0\). Les hypothèses disent qu'il existe un \( N\) tel que \( | x_n |\leq \epsilon\) dès que \( n\geq N\). Et il existe aussi \( M\geq 0\) tel que \( | y_n |\leq M\) pour tout \( n\). Du coup, lorsque \( n\geq N\) nous avons \( | x_ny_n |\leq M\epsilon\).
        \item
            En ce qui concerne la somme,
            \begin{equation}        \label{EqDCNBooAzrrBi}
                | x_p+y_p-x_q-y_q |\leq | x_p-x_q |+| y_p-y_q |.
            \end{equation}
            Soit \( N_1\) tel que si \( p,q\geq N_1\) alors \( | x_p-x_q |\leq \epsilon\) et \( N_2\) de même pour la suite \( (y_n)\). En prenant \( N=\max\{ N_1,N_2 \}\), la somme \eqref{EqDCNBooAzrrBi} est plus petite que \( 2\epsilon\) dès que \( p,q\geq N\).

            Passons à la démonstration du fait que le produit de deux suites de Cauchy est de Cauchy. Les suites \( (x_n)\) et \( (y_n)\) sont bornées et quitte à prendre le maximum, nous disons qu'elles sont toutes les deux bornées par le nombre \( M\) : pour tout \( n\) nous avons \( | x_n |\leq M\) et \( | y_n |\leq M\). Nous avons :
            \begin{equation}
                | x_py_p-x_qy_q |\leq | x_py_p-x_qy_p |+| x_qy_p-x_qy_q |\leq | y_p | |x_p-x_q |+| x_q | |y_p-y_q |.
            \end{equation}
            Vu que \( (x_n)\) et \( (y_n)\) sont de Cauchy, si \( p\) et \( q\) sont assez grands, les deux différences sont majorées par \( \epsilon\) et nous avons
            \begin{equation}
                | x_py_p-x_qy_q |\leq M\epsilon+M\epsilon=2M\epsilon,
            \end{equation}
            ce qui prouve que \( (x_ny_n)\) est de Cauchy.
        \item
            En ce qui concerne la somme, nous pouvons tout de suite calculer
            \begin{equation}
                | x_n+y_n-(a+b) |\leq | x_n-a |+| y_n-b |.
            \end{equation}
            Il existe une valeur de \( n\) à partir de laquelle le premier terme est plus petit que \( \epsilon\) et une à partir de laquelle le second terme est plus petit que \( \epsilon\). En prenant le maximum des deux, la somme est plus petite que \( 2\epsilon\).

            En ce qui concerne le produit,
            \begin{equation}
                | x_ny_n-ab |\leq | x_ny_n-ay_n |+| ay_n-ab |\leq | y_n || x_n-a |+| a || y_n-b |.
            \end{equation}
            Les suites \( | x_n-a |\) et \( | y_n-b |\) convergent vers zéro; la suite \( (y_n)\) est bornée parce que convergente (combinaison des points~\ref{ItemRKCIooJguHdji} et~\ref{ItemRKCIooJguHdjii})  et \( a\) (la suite constante) est également bornée. Donc par le point~\ref{ItemRKCIooJguHdjiii}, nous avons
            \begin{equation}
                y_n| x_n-a |+a| y_n-b |\to 0.
            \end{equation}
            Au passage nous avons également utilisé la propriété de la somme que nous venons de démontrer.
        \item Soit \( (x_n)\) une suite de Cauchy dans \( \eQ\) ne convergeant pas vers zéro : il existe \( \alpha>0\) tel que pour tout \( N\in \eN\), il existe \( n\geq N\) tel que \( | x_n |>\alpha\). Mais notre suite est de Cauchy, donc il existe \( n_0\in \eN\) tel que si \( p,q\geq n_0\) alors
            \begin{equation}
                | x_p-x_q |\leq \frac{ \alpha }{2}.
            \end{equation}
            En fixant \( N = n_0\), on obtient un naturel \( n\geq n_0\) tel que \( | x_n |\geq \alpha\). De plus, comme la suite est de Cauchy, si \( p>n\) nous avons aussi \( | x_n-x_p |\leq \frac{ \alpha }{2}\). Cela implique \( | x_p |\geq \frac{ \alpha }{2}\) et en particulier \( x_p\neq 0\).

            Nous venons de prouver que la suite ne s'annule plus à partir de l'indice \( n\), et même que \( | x_k |\geq\alpha/2\) pour tout \( k\geq n\). La suite \( (1/x_k)_{k\geq n}\) est donc bien définie.

            Soit \( \epsilon>0\). Soit \( n_0\) tel que \( | x_p-x_q |<\epsilon\) pour tout \( p,q>n_0\). Soit \( K\) plus grand que \( n_0\) et que \( n\). En prenant \( p,q\geq K\), nous avons \( |  x_p|>\frac{ \alpha }{2}\) et \( | x_q |>\frac{ \alpha }{2}\). Nous en déduisons que
            \begin{equation}
                \left| \frac{1}{ x_p }-\frac{1}{ x_q } \right| \leq \frac{ | x_q-x_p | }{ | x_px_q | }\leq \frac{ 4 }{ \alpha^2 }| x_q-x_p |\leq \frac{ 4 }{ \alpha^2 }\epsilon.
            \end{equation}
            Donc \( \left( \frac{1}{ x_n } \right)\) est de Cauchy.
    \end{enumerate}
\end{proof}


%+++++++++++++++++++++++++++++++++++++++++++++++++++++++++++++++++++++++++++++++++++++++++++++++++++++++++++++++++++++++++++ 
\section{Insuffisance des rationnels}
%+++++++++++++++++++++++++++++++++++++++++++++++++++++++++++++++++++++++++++++++++++++++++++++++++++++++++++++++++++++++++++

Nous allons voir qu'il n'existe pas de nombres rationnels \( x\) tels que \( x^2=2\), mais que pourtant il existe une infinité de suites de rationnels \( (x_n)\) tels que \(  x_n^2\to 2  \).

\begin{lemma}       \label{LemJPIUooWFHaFM}
    Un entier \( x\) est pair si et seulement si l'entier \( x^2\) est pair.
\end{lemma}

\begin{proof}
    Si \( x\) est un nombre pair, alors il existe un entier \( a\) tel que \( x=2a\) alors \( x^2=4a^2\) est pair.

    Inversement, si \( x\) est impair alors il existe un entier \( a\) tel que \( x=2a+1\) et alors \( x^2=4a^2+4a+1=2(2a^2+2a)+1\) est impair.
\end{proof}

Le théorème~\ref{THOooYXJIooWcbnbm} nous dira que tous les \( \sqrt{n}\) sont irrationnels dès que \( n\) n'est pas un carré parfait. Voici déjà le résultat pour \( n=2\). Le fait que \( \sqrt{ 2 }\) existe dans \( \eR\) sera la proposition \ref{PROPooUHKFooVKmpte}.
\begin{proposition}[Irrationalité de \( \sqrt{2}\)]     \label{PropooRJMSooPrdeJb}
    Il n'existe pas de fractions d'entiers dont le carré soit égal à \( 2\).
\end{proposition}
\index{irrationalité!\( \sqrt{2}\)}

\begin{proof}
    Nous supposons que la fraction d'entiers \( a/b\) est telle que \( a^2/b^2=2\), et nous allons construire une suite d'entiers strictement décroissante et strictement positive, ce qui est impossible.

    Grâce au lemme~\ref{LemJPIUooWFHaFM} nous avons successivement les affirmations suivantes :
    \begin{itemize}
        \item
        \(\frac{ a^2 }{ b^2 }=2 \)  avec \( a\neq 0\) et \( b\neq 0\).
    \item
        \( a^2=2b^2\), donc \( a^2\) est pair.
    \item
        \( a\) est alors pair et \( a^2\) est divisible par \( 4\). Soit \( a^2=4k\).
    \item
        \( 4k/b^2=2\), donc \( 4k=2b^2\), donc \( b^2=2k\) et \( b^2\) est pair.
    \item
        Nous déduisons que \( b\) est pair.
    \end{itemize}
    La fraction \( \frac{ a/2 }{ b/2 }\) est alors une nouvelle fraction d'entiers dont le carré vaut $2$. En procédant de la même façon, en remplaçant \( a\) par \( a/2\) et \( b\) par \( b/2\), on obtient que la fraction d'entiers \( \frac{ a/4 }{ b/4 }\) a la même propriété.

    En particulier, tous les nombres de la forme \( a/2^n\) sont des entiers.  Ils forment une suite strictement décroissante d'entiers strictement positifs. Impossible, me diriez-vous ? Et vous auriez bien raison : toute partie non vide de \( \eN\) admet un plus petit élément\footnote{Voir \cite{RWWJooJdjxEK}, et attention : ce n'est pas tout à fait évident.}. Il n'y a donc pas de fractions d'entiers dont le carré vaut \( 2\).
\end{proof}

\begin{lemma}[Série géométrique]   \label{LEMooOTVUooImvusn}
    Si \( q\neq 1\) dans \( \eQ\) et \( p\in \eN\) nous avons
    \begin{equation}
        \sum_{k=0}^pq^k=\frac{ 1-q^{p+1} }{ 1-q }.
    \end{equation}
\end{lemma}

\begin{proof}
    En posant \( S_p=1+q+q^2+\cdots +q^{p}\), nous avons $S_p-qS_p=1-q^{p+1}$ et donc
    \begin{equation}
        S_p=\sum_{k=0}^pq^k=\frac{ 1-q^{p+1} }{ 1-q }.
    \end{equation}
\end{proof}

\begin{proposition}
    La suite donnée par
    \begin{equation}
        x_n=1+\frac{ 1 }{ 1! }+\cdots +\frac{1}{ n! }
    \end{equation}
    est de Cauchy et ne converge pas dans \( \eQ\).
\end{proposition}

\begin{proof}
    Si \( p>q>0\) nous avons
    \begin{subequations}
        \begin{align}
            x_p-x_q&=\sum_{k=q+1}^p\frac{1}{ k! }\\
            &\leq \sum_{k=q+1}^p\frac{1}{ (q+1)! }\frac{1}{ (q+1)^{k-q-1} }  \label{SUBEQooAXILooEAcpVB}\\
            &\leq \frac{1}{ (q+1)! }\lim_{p\to \infty} \sum_{k=0}^{p}\frac{1}{ (q+1)^k }  \label{SUBEQooNDPTooDSEYEJ}\\
            &=\frac{1}{ (q+1)! }\frac{1}{ 1-\frac{1}{ q+1 } } \label{SUBEQooEMHJooSnCUiK}  \\
            &=\frac{1}{ (q+1)! }\frac{q+1}{q}\\
            &=\frac{1}{ q!q }.
        \end{align}
    \end{subequations}
    Justifications :
    \begin{itemize}
        \item Pour \eqref{SUBEQooAXILooEAcpVB}, il s'agit de remplacer dans \( k!\) tous les facteurs plus grands que \( (q+1)\) par \( q+1\). Cela rend le dénominateur plus petit.
        \item Pour \eqref{SUBEQooNDPTooDSEYEJ}, il y a une inégalité parce que la suite \( p\mapsto \sum_{k=0}^p1/(q+1)^k\) est une suite strictement croissante.

        \item Pour \eqref{SUBEQooEMHJooSnCUiK}, le lemme~\ref{LEMooOTVUooImvusn} donne la valeur de la somme finie. En ce qui concerne la limite, nous avons demandé \( p>q>0\) et donc \( q+1>1\). Dans ce cas la limite fonctionne.
    \end{itemize}

    Cette inégalité une fois établie nous permet de prouver les assertions. La suite \( (x_n) \) est de Cauchy car, pour tout \( \epsilon\in\eQ\) s'écrivant \( \epsilon=\frac{ a }{ b }\) avec \( a,b\in \eN\), en prenant \( p,q>b\), nous avons
    \begin{equation}
        x_p-x_q\leq \frac{1}{ b!b }<\frac{1}{ b }<\frac{ a }{ b }=\epsilon.
    \end{equation}

    Montrons par l'absurde que cette suite ne converge pas dans \( \eQ\). Pour cela, nous supposons que \( \lim_{n\to \infty} x_n=\frac{ a }{b }\in \eQ\). Pour tout \( p>q\) nous avons établi
    \begin{equation}
        0<x_p-x_q<\frac{1}{ qq! }.
    \end{equation}
    Prenons la limite \( p\to \infty\); par stricte croissance de la suite, les inégalités restent strictes :
    \begin{equation}        \label{EqQLCTooOgQOdh}
        0<\frac{ a }{ b }-x_q<\frac{1}{ qq! }.
    \end{equation}
    Si \( n>b\) alors nous pouvons écrire
    \begin{equation}
        \frac{ a }{ b }-x_n=\frac{ \alpha }{ n! }
    \end{equation}
    avec \( \alpha\in \eZ\) parce que le dénominateur commun entre \( \frac{ a }{ b }\) et \( x_n\) est dans \( n!\). En prenant donc \( q>n\) dans \eqref{EqQLCTooOgQOdh} nous pouvons écrire
    \begin{equation}
        0<\frac{ \alpha }{ q! }<\frac{1}{ qq! },
    \end{equation}
    c'est-à-dire \( 0<\alpha<\frac{1}{ q }\), ce qui est impossible pour \( \alpha\in \eZ\).
\end{proof}

\begin{lemma}   \label{LEMooDTXYooKwmlZh}
    Soit \( A>0\) dans \( \eQ\). Il existe un rationnel \( q>0\) tel que \( q^2<A\).
\end{lemma}

\begin{proof}
    Vu que \( \eQ\) est archimédien (proposition \ref{PROPooMXGPooDUkOuv}), il existe \( n\in \eN\) tel que \( 1<nA\). Pour ce \( n\), nous avons
    \begin{equation}
        \left( \frac{1}{ n } \right)^2<\frac{1}{ n }<A.
    \end{equation}
\end{proof}

La proposition suivante donne une suite de rationnels qui convergerait dans \( \eR\) vers \( \sqrt{ A }\) (non encore défini à ce stade). Il est expliqué dans \cite{BIBooMPXEooQLKhku} que la suite peut être vue comme une forme de méthode de Newton \ref{THOooDOVSooWsAFkx}. Si vous aimez les dessins et les approches géométriques, il y a une explication sur Wikipédia\cite{BIBooVCWCooQcolIq}.
\begin{proposition}[\cite{BIBooMPXEooQLKhku}]       \label{PROPooSTQXooHlIGVf}
    Soient \( A>0\) dans \( \eQ\) et \( x_0\in \eQ\). La suite \( (x_k)\) définie par
    \begin{equation}
        x_{k+1}=\frac{ 1 }{2}\left( x_k+\frac{ A }{ x_k } \right)
    \end{equation}
    a les propriétés suivantes :
    \begin{enumerate}
        \item
            La suite \( y_k=x_k^2 \) converge dans \( \eQ\) vers \( A\).
        \item
            La suite \( (x_k)\) est de Cauchy dans \( \eQ\).
        \item
            La suite \( (x_k)\) ne converge pas dans \( \eQ\) dans le cas de \( A=2\).
    \end{enumerate}
\end{proposition}

\begin{proof}
    En plusieurs points.
    \begin{subproof}
        \item[La suite \( s_k\)]
            En posant \( y_k=x_k^2\) nous calculons que
            \begin{equation}
                y_{k+1}-A=\frac{ (y_k-A)^2 }{ 4y_k }.
            \end{equation}
            Autrement dit, la suite \( s_k=y_k-A\) vérifie
            \begin{equation}
                s_{k+1}=\frac{ s_k^2 }{ 4(A+s_k) }.
            \end{equation}
            Quelle que soit la valeur de \( s_0=x_0^2-A\), nous avons
            \begin{equation}
                s_1=\frac{ s_0^2 }{ 4(A+s_0) }=\frac{ (x_0^2-A)^2 }{ 4(A+x_0^2-A) }=\frac{ (x_0^2-A)^2 }{ 4x_0^2 }>0.
            \end{equation}
            Donc à partir de \( s_1\), tous les éléments sont positifs. Vu que \( A>0\) nous avons aussi
            \begin{equation}
                s_{k+1}<\frac{ s_k^2 }{ 4s_k }=\frac{ s_k }{ 4 }
            \end{equation}
            et donc \( s_k<s_0/4^k\). Donc \( s_k\to 0\).
        \item[La suite \( (y_k)\)]
            Nous venons de prouver que si \( y_k=A+s_k\), alors \( s_k\to 0\). Autrement dit, la suite \( y_k\) converge vers \( A\) dans \( \eQ\). 

            La suite \( (y_k)\) est donc de Cauchy par la proposition \ref{PropFFDJooAapQlP}\ref{ItemRKCIooJguHdji}.
        \item[La suite \( (x_k)\) est de Cauchy]
            Soit \( \epsilon>0\) dans \( \eQ\). Vu que \( (y_k)\) est de Cauchy, il existe \( n_0\in \eN\) tel que 
            \begin{equation}
                | x^2_r-x_s^2 |<\epsilon
            \end{equation}
            pour tout \( r,s\geq n_0\).

            Soit par ailleurs \( q\neq 0\) dans \( \eQ\) tel que \( q^2<A\), assuré par le lemme \ref{LEMooDTXYooKwmlZh}. Quitte augmenter la valeur de \( n_0\), nous supposons que \( x_r,x_s>q\), et en particulier que \( x_r+x_s\neq 0\). Cela permet d'écrire d'abord
            \begin{equation}
                x_r^2-x_s^2=(x_r+x_s)(x_r-x_s)
            \end{equation}
            et ensuite de prendre la valeur absolue et de diviser par \( | x_r+x_s |\) :
            \begin{equation}
                | x_r-x_s |=\frac{ | x_r^2-x_s^2 | }{ | x_r+x_s | }<\frac{ \epsilon }{ 2q }.
            \end{equation}
            Donc \( (x_k)\) est une suite de Cauchy.
        \item[Pas de convergence pour \( A=2\)]
            Supposons que \( x_k\to a\in \eQ\). Dans ce cas nous aurions \( x_k^2\to a^2=A=2\) (proposition~\ref{PropFFDJooAapQlP}\ref{ITEMooIAFSooAIUpAN}). Mais nous savons par la proposition~\ref{PropooRJMSooPrdeJb} que \( a^2=2\) est impossible dans \( \eQ\).
    \end{subproof}
\end{proof}

Notons que cette proposition ne présume en rien de l'existence ou de la non-existence dans \( \eQ\) d'un élément qui pourrait décemment être nommé \( \sqrt{ A }\). Il se fait que le théorème \ref{THOooYXJIooWcbnbm} dira que \( \sqrt{ n }\) est soit entier soit irrationnel.

\begin{normaltext}
    Un petit programme en python explorer la suite de la proposition \ref{PROPooSTQXooHlIGVf}.
    \lstinputlisting{tex/frido/codeSnip_4.py}
\end{normaltext}

%+++++++++++++++++++++++++++++++++++++++++++++++++++++++++++++++++++++++++++++++++++++++++++++++++++++++++++++++++++++++++++
\section{Les réels}
%+++++++++++++++++++++++++++++++++++++++++++++++++++++++++++++++++++++++++++++++++++++++++++++++++++++++++++++++++++++++++++

Une construction des réels via les coupures de Dedekind est donnée dans \cite{PaulinTopGmVegN}.

\begin{normaltext}      \label{NormooHRDZooRGGtCd}
    La construction des réels va nécessiter un petit «\wikipedia{fr}{bootstrap}{bootstrap}» au niveau de la topologie. En effet la notion de suite de Cauchy est une notion topologique (définition~\ref{DefZSnlbPc}) alors que la topologie métrique (celle entre autres de \( \eQ\)) ne sera définie que par le théorème~\ref{ThoORdLYUu}. Nous avons donc dû définir en la définition~\ref{DefKCGBooLRNdJf} \emph{ex nihilo} les notions de
\begin{itemize}
    \item
        suite de Cauchy
    \item
        suite convergente
    \item
        complétude
\end{itemize}
Nous allons ensuite construire \( \eR\) comme ensemble de classes d'équivalence de suites de Cauchy dans \( \eQ\). Ce ne sera que plus tard, après avoir défini la notion d'espace métrique que nous allons voir que sur \( \eR\), ces trois notions coïncident avec celles topologiques\footnote{Proposition~\ref{PropooUEEOooLeIImr}.}. Et par conséquent que \( \eR\) sera un espace métrique complet\footnote{Théorème~\ref{THOooUFVJooYJlieh} pour la complétude en tant que corps et théorème~\ref{PROPooTFVOooFoSHPg} pour la complétude en tant que espace métrique.}.
% position 11144-30436
% position 13984-18006

Dans cette optique, il est intéressant de lire ce que dit Wikipédia à propos des suites de Cauchy dans l'article consacré à la construction des nombres réels\cite{BIBooPIGUooHzurMI}.
\end{normaltext}

%---------------------------------------------------------------------------------------------------------------------------
\subsection{L'ensemble}
%---------------------------------------------------------------------------------------------------------------------------

Soit \( \modE\) l'ensemble des suites de Cauchy\footnote{Définition~\ref{DefKCGBooLRNdJf}\ref{ItemVXOZooTYpcYN}} dans \( \eQ\). Soit aussi l'ensemble \( \modE_0\) constituée des suites qui convergent vers zéro\footnote{Nous rappelons qu'à ce niveau nous n'avons pas encore prouvé que toutes les suites de Cauchy convergent.}.

En posant
\begin{equation}
    x+y=(x_n+y_n)
\end{equation}
et
\begin{equation}
    xy=(x_ny_n),
\end{equation}
l'ensemble \( \modE\) devient un anneau\footnote{Définition~\ref{DefHXJUooKoovob}.} commutatif dont le neutre de l'addition est la suite constante \( x_n=0\) et le neutre pour la multiplication est la suite constante \( x_n=1\).

\begin{proposition}     \label{PROPooNUQVooAAkicK}
    La partie \( \modE_0\) est un idéal\footnote{Définition~\ref{DefooQULAooREUIU}.} de l'anneau \( \modE\).
\end{proposition}

\begin{proof}
    Nous savons par la proposition~\ref{PropFFDJooAapQlP}\ref{ItemRKCIooJguHdji} que les suites convergentes sont de Cauchy; par conséquent \( \modE_0\subset\modE\).

    L'ensemble structuré \( (\modE_0,+)\) est un sous-groupe de \( \modE\) par les propriétés de la proposition~\ref{PropFFDJooAapQlP} (il s'agit du fait que la somme de deux suites convergent vers zéro est une suite convergente vers zéro).

    En ce qui concerne la propriété fondamentale des idéaux, si \( x\in\modE_0\) et \( y\in\modE\) nous devons prouver que \( xy\in \modE_0\). Vu que \( (\modE_0,\cdot)\) est commutatif, cela suffira pour être un idéal bilatère. Vu que \( y\) est une suite de Cauchy, elle est bornée; et étant donné que \( x\to 0\) nous avons alors \( xy\to 0\) (par la proposition~\ref{PropFFDJooAapQlP}\ref{ItemRKCIooJguHdjiii}).
\end{proof}

\begin{theoremDef}[L'anneau des réels\cite{RWWJooJdjxEK}]       \label{DefooFKYKooOngSCB}
    Sur l'ensemble quotient \( \modE/\modE_0\), les opérations
    \begin{enumerate}
        \item
            \( \bar u+\bar v=\overline{ u+v }\)
        \item
            \( \bar u\cdot \bar v=\overline{ uv }\)
    \end{enumerate}
    sont bien définies et donnent à \( \modE/\modE_0\) une structure de corps commutatif appelé \defe{corps des réels}{réel} et noté \( \eR\)\nomenclature[Y]{\( \eR\)}{l'ensemble des réels}
\end{theoremDef}
\index{construction!des réels}

\begin{proof}
    Nous divisons la preuve en plusieurs parties.
    \begin{subproof}
    \item[Les opérations sont bien définies]
        La partie \( \modE_0\) est un idéal par la proposition \ref{PROPooNUQVooAAkicK}. Le quotient est donc bien défini et est un anneau par la proposition \ref{PROPooGXMRooTcUGbi}\ref{ITEMooYBEGooTlHgNz}.
    \item[Caractérisation des classes]
        Soit \( q\in \eQ\) et une suite \( x\) convergente vers \( q\). Cette suite est de Cauchy comme toute suite convergente. Montrons que
        \begin{equation}
            \bar x=\{ \text{suites qui convergent vers } q \}.
        \end{equation}
        Si \( y\in\bar x\) alors \( y=x+h\) avec \( h\in \modE_0\), et comme \( h_n\to 0\), on a \( y_n\to q\). Réciproquement, si \( y_n\to q\) alors pour chaque \( n\) nous avons
        \begin{equation}
            y_n=x_n+(y_n-x_n),
        \end{equation}
        mais \( y_n-x_n\to 0\). Donc la suite \( y-x\in\modE_0\) ce qui signifie que \( y\in\bar x\).
    \item[Neutre et unité]
        Il est vite vérifié que \( \bar 0\), la classe de la suite constante égale à zéro est neutre pour l'addition. De même, \( \bar 1\), est un neutre pour la multiplication.
    \item[Corps]
        Nous devons prouver que tout élément non nul est inversible. C'est-à-dire que si \( x\in\modE\) ne converge pas vers zéro\footnote{\( x\in\modE\) peut soit ne pas converger du tout, soit converger vers autre chose que zéro.} alors il existe \( y\in \modE\) tel que \( xy\in\bar 1\).

        Nous savons par la proposition~\ref{PropFFDJooAapQlP}\ref{ItemRKCIooJguHdjvi} que \( x\) étant une suite de Cauchy dans \( \eQ\), il existe \( n_0\in \eN\) tel que \( \left( \frac{1}{ x_n } \right)_{n\geq n_0}\) est une suite de Cauchy. Nous posons alors
        \begin{equation}
            y_n=\begin{cases}
                0    &   \text{si } n\leq n_0\\
                \frac{1}{ x_n }    &    \text{si } n>n_0.
            \end{cases}
        \end{equation}
        Nous avons alors
        \begin{equation}
            (xy)_n=\begin{cases}
                0    &   \text{si } n\leq n_0\\
                1    &    \text{si } n>n_0
            \end{cases}
        \end{equation}
        et donc \( xy\in\bar 1\).
    \end{subproof}
\end{proof}

\begin{normaltext}[Quelques notations entre \( \eQ\) et \( \eR\)]      \label{NORMooWBYNooBQaPPk}
    Si \( k\mapsto x_k\) est une suite, nous notons \( (x_k)\) avec des parenthèses la suite elle-même. Le \( k\) dans \( (x_k)\) est un indice muet, et dans les cas où il peut y avoir une ambiguïté, nous pouvons noter \( (x_k)_{k\in \eN}\). Cette dernière notation est plus lourde, mais plus exacte.
    
    Le mieux est d'écrire simplement \( x\) la suite, mais alors il faut être prudent et ne pas noter \( x\) la limite. Nous éviterons donc d'écrire \( x_k\to x\).

    Si \( (x_k)\) est une suite de Cauchy dans \( \eQ\), nous notons \( \bar x\) l'élément de \( \eR\) qui lui correspond. En fait \( \bar x=(x_k)\) : \( \bar x\) est la suite-elle même, mais pour nous souvenir de l'origine nous allons adopter cette notation.

    D'autre part nous définissons
    \begin{equation}
        \begin{aligned}
            \varphi\colon \eQ&\to \eR \\
            q&\mapsto \overline{ [k\mapsto q]},
        \end{aligned}
    \end{equation}
    c'est-à-dire que \( \varphi(q)\) est la classe de la suite constante \( x_k=q\).
\end{normaltext}

\begin{proposition}     \label{PropooEPFCooMtDOfP}
    Soit l'application
    \begin{equation}
        \begin{aligned}
            \varphi\colon \eQ&\to \eR \\
            q&\mapsto \bar q .
        \end{aligned}
    \end{equation}
    où par \( \bar q\) nous entendons la classe de la suite constante égale à \( q\) (qui est de Cauchy).
    \begin{enumerate}
        \item
            C'est un homomorphisme de corps injectif.
        \item
            \( \Image(\varphi)\) est un sous-corps de \( \eR\)
        \item
            \( \varphi\colon \eQ\to \Image(\varphi)\) est un isomorphisme de corps.
    \end{enumerate}
\end{proposition}

\begin{proof}
    Le fait que ce soit un homomorphisme est simplement
    \begin{itemize}
        \item \( \varphi(q+q')=\overline{ q+q' }=\bar q+\overline{ q' }\)
        \item \( \varphi(qq')=\overline{ qq' }=\overline{ q }\overline{ q' }\).
    \end{itemize}
    En ce qui concerne l'injectivité, si \( q\) est tel que \( \varphi(q)=\bar 0=\modE_0\), c'est que
    \begin{equation}
        \varphi(q)=\{ \text{suites de Cauchy qui convergent vers zéro} \}
    \end{equation}
    Mais nous savons aussi que\footnote{Voir dans la démonstration du théorème~\ref{DefooFKYKooOngSCB}.}
    \begin{equation}
        \varphi(q)=\bar q=\{ \text{suites de Cauchy qui convergent vers } q \}
    \end{equation}
    Nous en déduisons que \( q=0\).
\end{proof}
Lorsque dans la suite nous parlerons d'un élément de \( \eQ\) comme étant un réel, nous aurons en tête l'image de cet élément par \( \varphi\).

\begin{lemma}       \label{LEMooYLQBooFistHs}
    Soient \( q,l\in \eQ\) tels que \( \bar q=\bar l\). Alors \( q=l\) dans \( \eQ\).
\end{lemma}

\begin{proof}
    La suite constante \( x_n=q\) est un représentant de \( \bar q\), et la suite constante \( y_n=l\) est représentant de \( \bar l\). Dire que \( \bar l=\bar q\) signifie qu'il existe une suite \( z\in \modE_0\) tel que 
    \begin{equation}
        x=y+z.
    \end{equation}
    Pour tout \( n\) nous avons donc \( x_n=y_n+z_n\), ou encore
    \begin{equation}
        z_n=q-l
    \end{equation}
    pour tout \( n\). Vu que \( z\) est une suite constante, elle en peut appartenir à \( \modE_0\) que si elle est la suite constante nulle, c'est à dire si \( q=l\).
\end{proof}

%---------------------------------------------------------------------------------------------------------------------------
\subsection{Relation d'ordre}
%---------------------------------------------------------------------------------------------------------------------------

\begin{definition}
    Nous définissons les parties \( \modE^+\) et \( \modE^-\) de \( \modE\) par
    \begin{enumerate}
        \item
            \( x\in  \modE^+\) si et seulement si pour tout \( \epsilon>0\) (\( \epsilon\) est dans \( \eQ\)), il existe \( N_{\epsilon}\) tel que \( n>N_{\epsilon}\) implique \( x_n>-\epsilon\).
        \item
            \( x\in  \modE^-\) si et seulement si pour tout \( \epsilon>0\), il existe \( N_{\epsilon}\) tel que \( n>N_{\epsilon}\) implique \( x_n<\epsilon\).
    \end{enumerate}
    Nous notons aussi \( \modE^{++}=\modE^+\setminus\modE_0\).
\end{definition}


Dans le lemme suivant, le point \ref{ITEMooRCIZooHUIymE} peut sembler perturbant. Il s'agit de dire que si \( x\) est la classe de la suite constante \( 0\), alors il est le neutre pour l'addition dans \( \eR\).
\begin{lemma}[À propos du zéro]       \label{LEMooJOYQooCDlhHW}
    Nous avons
    \begin{enumerate}
        \item   \label{ITEMooKRBYooZGhhch}
    \( \modE^+\cap\modE^-=\{ \bar 0 \}\).
\item       \label{ITEMooRCIZooHUIymE}
    Si \( x=\bar 0\) alors \( x=0\).
    \end{enumerate}
\end{lemma}

\begin{lemma}       \label{LEMooSWYXooMKMLYI}
    Les parties \( \modE^+\) et \( \modE^-\) partitionnent \( \modE\) de la façon suivante :
    \begin{enumerate}
        \item
            \( \modE^+\cap\modE^-=\modE_0\)
        \item       \label{ITEMooZRVXooANHspZ}
            \( \modE^+\cup\modE^-=\modE\)
    \end{enumerate}
\end{lemma}

\begin{proof}
    On prouve d'abord que \( \modE^+\cap\modE^-\subset\modE_0\), l'inclusion inverse est évidente. Soit \( \epsilon>0\) et \( x\in \modE^+\cap\modE^-\). Il existe un \( N\in \eN\) tel que \( x_n>-\epsilon\) et \( x_n<\epsilon\) pour tout \( n\geq N\). Par conséquent, \( | x_n |\leq \epsilon\) pour tout \( n\geq N\) et la suite \( x\) converge vers zéro, c'est-à-dire \( x\in\modE_0\).

    Pour prouver le second point, soit \( x\in \modE\setminus\modE^-\), et prouvons que \( x\in\modE^+\). La condition \( x\notin \modE^-\) donne qu'il existe un \( \alpha>0\) (dans \( \eQ\)) tel que pour tout \( n\), il existe \( p>n\) avec \( x_p>\alpha\). Mais \( x\) est une suite de Cauchy, donc nous avons un \( n_0\) tel que si \( n,p\geq n_0\) alors \( | x_n-x_p |\leq \frac{ \alpha }{2}\). En particulier, si \( n\geq n_0\), et si \( p>n\) est tel que \( x_p>\alpha\), on obtient
    \begin{equation}
        x_n>\frac{ \alpha }{2}>0
    \end{equation}
    Par conséquent \( x\in\modE^+\) parce que \( x\in\modE\) et les \( x_n \) sont tous positifs à partir d'un certain rang.
\end{proof}

\begin{lemma}[\cite{MonCerveau}]        \label{LEMooXNWSooHbNcAV}
    Si \( x\in \modE^{++}\), alors il existe \( N\) tel que \( x_n>0\) pour tout \( n>N\).
\end{lemma}

\begin{proof}
    La suite \( x\) ne tend pas vers zéro. Donc il existe \( \delta>0\) tel que pour tout \( N>0\) il existe \( n>N\) vérifiant \( x_n>\delta\).

    Mais la suite \( x\) est également de Cauchy. Écrivons cette condition pour \( \delta/2\). Il existe \( N_2>0\) tel que \( p,q>N_2\) implique \( | x_p-x_q |<\delta/2\).

    Nous fixons \( n>N_2\) tel que \( x_n>\delta\). Alors pour tout \( p>N_2\) nous avons aussi
    \begin{equation}
        | x_p-x_n |<\frac{ \delta }{2}.
    \end{equation}
    Cela implique que \( x_p>\delta/2>0\) pour tout \( p>N_2\).
\end{proof}


\begin{lemma}[\cite{RWWJooJdjxEK}]      \label{LEMooRKSXooFsIohe}
    Quelques propriétés du partitionnement.
    \begin{enumerate}
        \item       \label{ITEMooRQVKooCnwWOY}
            \( x\in\modE^-\) si et seulement si \( (-x)\in\modE^+\)
        \item       \label{ITEMooJUPOooOBubqA}
            \( x\in\modE^+\) et \( y\in\modE^+\) implique \( x+y\in\modE^+\)
        \item       \label{ITEMooDQLJooPViuVC}
            \( x\in\modE^+\) et \( y\in\modE^+\) implique \( xy\in\modE^+\)
        \item
            Si \( x,y\in\modE\) sont tels que \( x-y\in\modE_0\) alors soit \( x,y\in\modE^+\) soit \( x,y\in\modE^-\).
    \end{enumerate}
\end{lemma}

\begin{proof}
    Point par point.
    \begin{enumerate}
        \item
            Définition de \( \modE^+\) et \( \modE^-\).
        \item
            Pour \( n\geq N_{\epsilon/2}\) nous avons \( x_n>-\epsilon/2\) et \( y_n>-\epsilon/2\). Donc \( x_n+y_n>-\epsilon\).
        \item
            Si \( x\) ou \( y\) est dans \( \modE_0\) alors \( xy\in\modE_0\) et c'est bon. Si par contre \( x,y\in\modE^{++}\) alors le lemme \ref{LEMooXNWSooHbNcAV} nous indique que pour \( n\) suffisamment grand, \( x_n>0\) et \( y_n>0\). Et dans ce cas, \( (xy)_n> 0\), c'est-à-dire \( xy\in\modE^+\).
        \item
            Supposons que \( x-y\in\modE_0\) avec \( x\in\modE^+\) et prouvons qu'alors \( y\in\modE^+\). Soit donc \( \epsilon>0\); il existe \( n_1\) tel que \( x_n>-\frac{ \epsilon }{2}\) dès que \( n\geq n_1\). Mais \( x-y\in\modE_0\), donc il existe \( n_2\) tel que \( | x_n-y_n |<\frac{ \epsilon }{2}\) dès que \( n\geq n_2\). En prenant \( n\) plus grand que \( n_1\) et \( n_2\), nous avons en même temps
            \begin{subequations}
                \begin{numcases}{}
                    x_n>-\frac{ \epsilon }{2}\\
                    | x_n-y_n |<\frac{ \epsilon }{2}.
                \end{numcases}
            \end{subequations}
            Cela implique que \( y_n>-\epsilon\) et donc que \( y\in\modE^+\).

            Nous pouvons de même prouver que si \( x\in\modE^-\) alors \( y\in\modE^-\).
    \end{enumerate}
\end{proof}

\begin{definition}[Positivité dans \( \eR\)]        \label{DefooLMQIooTgzZXd}
    Vocabulaire et notations.
    \begin{enumerate}
        \item
            Nous notons \( \eR=\modE/\modE_0\).
        \item
            Nous notons \( \eR^+=\bar\modE^+\).\nomenclature[Y]{\( \eR^+\)}{les réels positifs ou nuls}
        \item
            Nous notons \( \eR^-=\bar\modE^-\).
        \item
            Un élément de \( \eR\) est \defe{positif}{positif} s'il est la classe d'une suite de Cauchy appartenant à \( \modE^+\).
        \item
            Un élément de \( \eR\) est \defe{négatif}{négatif} s'il est la classe d'une suite de Cauchy appartenant à \( \modE^-\).
        \item
            Lorsque nous parlons de nombres réels, le symbole «\( 0\)» signifie \( \modE_0\) ou plus précisément la classe d'un élément de \( \modE_0\) modulo \( \modE_0\).
    \end{enumerate}
\end{definition}

\begin{normaltext}\label{REMooOCXLooKQrDoq}
    Avec les conventions de la définition~\ref{DefooLMQIooTgzZXd}, et en anticipant sur nos connaissances à propos des réels,
    \begin{enumerate}
        \item
            zéro est positif et négatif.
        \item
            L'intersection entre \( \eR^+\) et \( \eR^-\) est le singleton \( \{ 0 \}\).
        \item
            L'ensemble des nombres \emph{strictement} positifs est noté \( (\eR^+)^*\) ou \( \eR^+\setminus\{ 0 \}\).
        \item
            Le mot «positif» signifie «positif ou nul»; le mot «négatif» signifie «négatif ou nul». Cela sont des conventions qui sont également celles de Wikipédia\cite{ooSBSSooTlnuKi}.
    \end{enumerate}

\end{normaltext}

\begin{definition}[Ordre sur \( \eR\)]      \label{DEFooBXHJooOEYPRI}
    Si \( a,b\in \eR\) nous notons \( a\leq b\) si et seulement si \( b-a\in\overline{ \modE^+ }\).
\end{definition}

\begin{proposition} \label{PROPooYMJVooNAsXae}  
    Le couple \( (\eR,\leq)\) est un corps totalement ordonné\footnote{Corps totalement ordonné, définition \ref{DefKCGBooLRNdJf}.}
\end{proposition}

\begin{proof}
    Il s'agit de vérifier, dans l'ordre, les définitions \ref{DefooFLYOooRaGYRk}, \ref{DEFooVGYQooUhUZGr} et \ref{DefKCGBooLRNdJf}\ref{ITEMooOOOVooJWwIQr}. Pour la suite nous considérons \( x,y,z\in \eR\) et des suites de Cauchy \( a,b,c\) telles que \( x=\bar a\), \( y=\bar b\) et \( z=\bar c\).
    \begin{subproof}
    \item[Réflexivité]
        Pour savoir si \( x\geq x\), nous devons nous demander si \( x-x\in\overline{ \modE^+ }\). Nous avons \( x-x=\bar a-\bar a=\bar 0=0\).
    \item[antisymétrie]
        Nous supposons que \( x\geq y\) et \( y\geq x\). Du côté des suites de Cauchy, cela signifie que \( a-b\in\modE^+\) et \( b-a\in \modE^+\). Le lemme \ref{LEMooRKSXooFsIohe}\ref{ITEMooRQVKooCnwWOY} nous indique alors que \( a-b=-(b-a)\in \modE^-\). Donc
        \begin{equation}
            a-b\in\modE^+\cap\modE^-=\{ \bar 0 \}.
        \end{equation}
        Donc le réel \( x-y\) est la classe de la suite constante \( 0\). Le lemme \ref{LEMooJOYQooCDlhHW}\ref{ITEMooRCIZooHUIymE} dit alors que \( x-y=0\) ou encore que \( x=y\).
    \item[transitivité]
        Nous supposons que \( x\leq y\) et \( y\leq z\). Alors nous avons
        \begin{subequations}
            \begin{align}
                c-a&=c-b+b-a\\
                &=(c-b)+(b-a)\\
                &\in \modE^++\modE^+\\
                &\subset \modE^+
            \end{align}
        \end{subequations}
        où nous avons utilisé le lemme \ref{LEMooRKSXooFsIohe}\ref{ITEMooJUPOooOBubqA}. Vu que \( c-a\in\modE^+\), nous avons \( z-x=\overline{ c-a }\geq 0\).
    \item[Ordre total]
        Nous devons prouver que pour \( x,y\in \eR\) nous avons toujours \( x\leq y\) ou \( y\leq x\). Supposons que nous n'ayons pas \( x\leq y\), c'est à dire \( \overline{ b-a }\notin \modE^+\). Vu le lemme \ref{LEMooSWYXooMKMLYI}\ref{ITEMooZRVXooANHspZ} nous avons \( \overline{ b-a }\in \modE^-\), ce qui donne, par le lemme \ref{LEMooRKSXooFsIohe}\ref{ITEMooRQVKooCnwWOY} que \( \overline{ a-b }\in \modE^+\), c'est à dire \( y\leq x\).
    \item[Corps ordonné]
        Enfin nous devons vérifier les deux conditions de la définition \ref{DefKCGBooLRNdJf}\ref{ITEMooOOOVooJWwIQr}.

        Pour la première condition, nous supposons \( x\leq y\), c'est à dire \( b-a\in \modE^+\). Nous avons donc
        \begin{equation}
            (b+c)-(a+c)=b+x-a-c=b-a\in \modE^+,
        \end{equation}
        donc \( \overline{ a+c }\leq \overline{ b+c }\), c'est à dire \( x+z\leq y+z\).

        Pour la seconde condition, c'est le lemme \ref{LEMooRKSXooFsIohe}\ref{ITEMooDQLJooPViuVC}.
    \end{subproof}
\end{proof}


\begin{definition}
    Vu que \( \eR\) est un corps totalement ordonné (proposition \ref{PROPooYMJVooNAsXae}), si \( x\in \eR\), nous définissons \( | x |\) conformément à \ref{DefKCGBooLRNdJf}\ref{ItemooWUGSooRSRvYC}.
\end{definition}

\begin{lemma}       \label{LEMooTJAXooKEqPCG}
    L'application
    \begin{equation}
        \begin{aligned}
            \varphi\colon \eQ&\to \eR \\
            q&\mapsto \bar q
        \end{aligned}
    \end{equation}
    dont nous avons déjà parlé dans la proposition~\ref{PropooEPFCooMtDOfP} est strictement croissante.
\end{lemma}

\begin{proof}
    Nous supposons \( q< l\) dans \( \eQ\). Nous devons montrer que \( \bar q\leq \bar l\) dans \( \eR\), c'est à dire que \( \bar q\leq \bar l\) et \( \bar q\neq \bar l\).

    Considérons la suite constante \( x_n=l-q\in \eQ\). Pour tout \( \epsilon>0\) dans \( \eQ\) nous avons
    \begin{equation}
        x_n=l-q>0>-\epsilon,
    \end{equation}
    et donc \( x_n\in\modE^+\). Donc \( \overline{ l-q }\geq 0\). Cela signifie \( \bar q\leq \bar l\).

    D'autre part le lemme \ref{LEMooYLQBooFistHs} dit que \( \bar q=\bar l\) que si \( q=l\), ce qui est exclu parce que \( q<l\). Donc \( \bar q\neq\bar l\).
\end{proof}

\begin{remark}
    Comme déjà mentionné plus haut, à chaque fois que nous parlerons d'un élément de \( \eQ\) comme étant un élément de \( \eR\), nous considérons la classe de la suite constante.
\end{remark}

\begin{lemma}       \label{LemooYNOVooOwoRwD}
    Si \( x,y,z\in \eR\) avec \( x>0\) sont tels que \( z>y/x\) alors \( zx>y\).
\end{lemma}

\begin{proof}
    Nous savons que
    \begin{equation}
        z-\frac{ y }{ x }\in \modE^+\setminus\{ 0 \}=\modE^{++}.
    \end{equation}
    Vu que \( x\in\modE^{++}\), multiplier par \( x\) fait rester dans \( \modE^{++}\) :
    \begin{equation}
        zx-x\frac{ y }{ x }\in \modE^{++}.
    \end{equation}
    Un représentant de \( x\frac{ y }{ x }\) est la suite \( n\mapsto x_n\frac{ y_n }{ x_n }=y_n\). Donc \( x\frac{ y }{ x }=y\). Cela signifie que \( zx-y\in\modE^{++}\) et donc que \( zx>y\).
\end{proof}

\begin{lemma}       \label{LemooMWOUooVWgaEi}
    Pour tout \( a\in \eR\), il existe \( p\in \eN\) tel que \( p>a\).
\end{lemma}

\begin{proof}
    Nous allons donner deux preuves différentes de ce lemme.
    \begin{subproof}
    \item[Première façon]

        L'élément \( a\) de \( \eR\) admet un représentant \( (a_n)\) qui est une suite de Cauchy dans \( \eQ\). C'est donc une suite bornée, c'est-à-dire qu'il existe \( m,q\in \eN\) tels que \( | a_n |\leq m/q\) pour tout \( n\) (proposition~\ref{PropFFDJooAapQlP}\ref{ItemRKCIooJguHdjii}). Soit \( M\) un naturel strictement plus grand que \( m/q\)\footnote{Lemme~\ref{LEMooEBTIooGMoHsj}.}.

    La suite de Cauchy \( (M-a_n)_{n\in \eN}\) est constituée de rationnels positifs et est donc dans \( \modE^+\). La classe de \( M-a\) est donc un réel positif\footnote{Et nous allons d'ailleurs arrêter de toujours préciser «la classe de» lorsque ce n'est pas nécessaire.}. Par définition de la relation d'ordre, \( M\geq a\).
\item[Seconde façon]

    La suite \( (a_n)\) est majorée par \( \frac{ m }{ q }\), donc on a dans \( \eQ\) et pour tout \( n\) :
    \begin{equation}
        a_n\leq \frac{ m }{ q }=M\leq qM.
    \end{equation}
    L'application \( \varphi\colon \eQ\to \eR\) est croissante, donc
    \begin{equation}
        \varphi\big( (a_n) \big)\leq \varphi(qM).
    \end{equation}
    \end{subproof}
\end{proof}

En corolaire, nous avons
\begin{lemma}      \label{LEMooMWOUooVWgbFi}
    Pour tout \( x\in \eR\), il existe \( q\in \eZ\) tel que \( q < x\).
\end{lemma}
\begin{proof}
    Utilisation du lemme précédent avec \( a = -x \): on prend \( q = -p \).
\end{proof}

\begin{theorem}[\cite{RWWJooJdjxEK}]        \label{ThoooKJTTooCaxEny}
    Le corps \( \eR\) est archimédien\footnote{Définition~\ref{DefKCGBooLRNdJf}\ref{ItemooDZQKooPsqeRf}.}.
\end{theorem}

\begin{proof}
    Le lemme~\ref{} dit que \( \eR\) est totalement ordonné. Soient \( x,y\in \eR\) avec \( x>0\); posons \( a=\frac{ y }{ x }\). Le lemme~\ref{LemooMWOUooVWgaEi} nous donne un \( p \in \eN\) tel que \(p > a\).Nous concluons alors avec le lemme~\ref{LemooYNOVooOwoRwD} :
    \begin{equation}
        px>ax=\frac{ y }{ x }x=y.
    \end{equation}
\end{proof}

Le lemme suivant n'est pas loin de dire que \( \eQ\) est dense dans \( \eR\), à part que nous n'avons pas encore donné de topologie sur \( \eR\).
\begin{lemma}       \label{LemooHLHTooTyCZYL}
    Si \( x,y\in \eR\) sont tels que \( x<y\), alors il existe \( s\in \eQ\) tel que \( x<s<y\).
\end{lemma}

\begin{proof}
    Nous avons par hypothèse que \( y-x>0\) et donc le fait que \( \eR\) soit archimédien (théorème~\ref{ThoooKJTTooCaxEny}) nous donne \( q\in \eN\) tel que \( q(y-x)>1\). Soit
    \begin{equation}
        E=\{ n\in \eZ\tq \frac{ n }{ q }\leq x \}.
    \end{equation}
    Cet ensemble n'est pas vide à cause du lemme~\ref{LEMooMWOUooVWgbFi}; de plus, comme \( |x|q \leq n_0\) pour un certain \( n_0 \) (à cause du lemme~\ref{LemooMWOUooVWgaEi}), l'ensemble \( E\) est majoré par \( n_0\). Donc \( E\) possède un plus grand élément\footnote{Lemme~\ref{LEMooMYEIooNFwNVI}.} \( p\) qui vérifie
    \begin{equation}
        \frac{ p }{ q }\leq x<\frac{ p+1 }{ q }.
    \end{equation}
    De plus \( (p+1)/q<y\). En effet nous avons
    \begin{equation}
        \frac{ p+1 }{ q }=\frac{ p }{ q }+\frac{1}{ q }\leq x+\frac{1}{ q }<x+y-x=y
    \end{equation}
    où nous avons utilisé l'inégalité stricte \( y-x>\frac{1}{ q }\).

    Nous avons donc
    \begin{equation}
        x<\frac{ p+1 }{ q }<y,
    \end{equation}
    et le nombre \( (p+1)/q\) convient comme \( s\).
\end{proof}

\begin{remark}      \label{REMooXOIOooHjwMvA}
    Le lemme~\ref{LemooHLHTooTyCZYL} a également pour conséquence que des ensembles comme \( \mathopen[ -1 , 1 \mathclose]\) ne sont pas bien ordonnés (définition~\ref{DEFooVGYQooUhUZGr}). En effet la partie \( \mathopen] 0 , 1 \mathclose[\) ne possède pas de minimum parce que si \( x\in \mathopen] 0 , 1 \mathclose[\) alors \( 0<x\) et il existe \( s\in \eQ\) (a fortiori \( s\in \eR\)) tel que \( 0<s<x\), c'est-à-dire que \( x\) n'est pas un minimum de \( \mathopen] 0 , 1 \mathclose[\).
\end{remark}

Tant que nous y sommes dans les encadrements de réels\dots
\begin{normaltext}
  Soit \(q_0 \in \eQ \) tel que \( 0 \leq q_0 < 1 \). On définit alors \( d_1 \in \{0, 1\} \) comme valant \( 1 \) si \( 2 q_0 \geq 1 \) et \(0 \) sinon. Puis on pose \( q_1 = 2 q_0 - d_1 \).

  Poursuivant de la sorte, on crée une suite \( (d_n)_{n\geq 1} \): c'est le \defe{développement dyadique}{dyadique!développement} de \( q_0 \).
\end{normaltext}

\begin{lemma}[\cite{MonCerveau}]        \label{LEMooRSLIooVrZMxM}
  Soit \( q,\ q' \) deux rationnels tels que \( 0 \leq q < q' < 1 \). Il existe deux entiers naturels \( a \) et \( N \) tels que \( q < \frac a {2^N} < q' \).
\end{lemma}
\begin{proof}
  On crée les développements dyadiques de \( q \) et \( q' \), que l'on note respectivement \( (d_n)_{n\geq 1} \) et \( (d'_n)_{n\geq 1} \). Notons
  \begin{equation}
    E = \{ n \in \eN \tq d_n \neq d'_n \}.
  \end{equation}
Comme \( q \neq q' \), les développements dyadiques sont différents\quext{À vérifier tout de même\dots}, l'ensemble \(E\) est non-vide, et il admet un plus petit élément \(N \). Or, \( q < q' \), et donc nécessairement \( d_N < d'_N \). On construit alors \( a = \sum_{i=1}^{N} d_i 2^i \). 
\end{proof}

\begin{corollary}\label{CorDensiteDyadiques}
  Pour tous réels \(x,\ y\) tels que \( 0 \leq x < y \leq 1 \), il existe un nombre de la forme \( d = a / 2^n \), avec \( n \in \eN \) et \( a \in \eN,\ a \leq 2^n\), tel que \( x < d < y \).
\end{corollary}

\begin{lemma}[\cite{MonCerveau}]        \label{LEMooEGYLooCGrDrl}
    Soient des réels \( a,b,x,y\) tels que
    \begin{equation}
        a\leq x\leq b
    \end{equation}
    et
    \begin{equation}
        a\leq y\leq b,
    \end{equation}
    alors \( | x-y |\leq | b-a |\).
\end{lemma}

\begin{lemma}       \label{LEMooTPLUooXiCZHJ}
    Soient deux réels \( a,b\) tels que
    \begin{enumerate}
        \item
            \( a\geq 0\),
        \item
            \( b\geq 0\)
        \item
            \( a+b=0\).
    \end{enumerate}
    Alors \( a=0\) et \( b=0\).
\end{lemma}

\begin{lemma}       \label{LEMooNLGSooSGdvAo}
    Si \( a\in \eR\), alors \( a^2\geq 0\) et \( a^2=0\) si et seulement si \( a=0\).
\end{lemma}

%---------------------------------------------------------------------------------------------------------------------------
\subsection{Complétude}
%---------------------------------------------------------------------------------------------------------------------------

Le théorème \ref{ThoKHTQJXZ} donne une complétion de tout espace métrique en un espace complet. Il serait tentant de l'utiliser ici pour définir \( \eR\) à partir de \( \eQ\). Cette méthode ne fonctionne cependant pas parce que la démonstration de \ref{ThoKHTQJXZ} utilise le fait que \( \eR\) est complet.

\begin{lemma}       \label{LEMooXCVRooOSZYWv}
    Nous avons
    \begin{equation}
        | \varphi(q) |=| \varphi(q) |
    \end{equation}
    pour tout \( q\in \eQ\).
\end{lemma}

\begin{proof}
    Soit \( q\in \eQ\). Si \( q\geq 0\) alors nous avons d'une part \( | q |=q\) dans \( \eQ\), et d'autre part \( \varphi(q)\geq 0\) dans \( \eR\). Donc au final
    \begin{equation}
        | \varphi(q) |=\varphi(q)=\varphi(| q |).
    \end{equation}
    
    Supposons au contraire que \( q<0\). Alors \( | q |=-q\) dans \( \eQ\), mais aussi \( \varphi(q)\leq 0\). Donc
    \begin{equation}
        | \varphi(q) |=-\varphi(q)=\varphi(-q)=\varphi(| q |).
    \end{equation}
\end{proof}

\begin{lemma}[\cite{RWWJooJdjxEK, MonCerveau}]      \label{LemooRTGFooYVstwS}
    Toute suite de Cauchy dans \( \eQ\) converge dans \( \eR\) vers le réel qu'elle représente.

    Plus précisément, en suivant les notations de \ref{NORMooWBYNooBQaPPk}, si \( (x_k)\) est une suite de Cauchy dans \( \eQ\), alors 
    \begin{enumerate}
        \item
            \( \varphi(x_k)\) est une suite de Cauchy dans \( \eR\).
        \item
            \( \varphi(x_k)\stackrel{\eR}{\longrightarrow}\bar x\).
    \end{enumerate}
    Ici \( \bar x\) est la classe de la suite \( x\). C'est donc un élément de \( \eR\).
\end{lemma}

\begin{proof}
    Soit \( (x_n)\) une suite de Cauchy de \( \eQ\), c'est-à-dire que \( x_k\in \eQ\) pour tout \( k\) et qu'elle est de Cauchy. Elle représente un réel \( \bar x\in \eR\), et nous voulons prouver que pour la topologie de \( \eR\) nous avons \( \lim_{n\to \infty} \varphi(x_n)=\bar x\). 

    \begin{subproof}
    \item[\( \varphi(x_k)\) est de Cauchy]

        Soit \( \epsilon>0\) dans \( \eR\). Nous considérons \( \epsilon'\in \eQ\) tel que \( 0<\epsilon'<\epsilon\). Plus précisément tel que
        \begin{equation}
            0<\varphi(\epsilon')<\epsilon.
        \end{equation}
        Soit \( N>0\) tel que \( | x_p-x_q |<\epsilon'\) pour tout \( p,q\geq N\). Cela existe parce que \( (x_k)\) est dans Cauchy dans \( \eQ\). Nous avons alors
        \begin{subequations}
            \begin{align}
                | \varphi(x_p)-\varphi(x_q) |&=| \varphi(x_p-x_q) |\\
                &=\varphi\big( | x_p-w_q | \big)        \label{SUBEQooZPTIooYsiEqh}\\
                &\leq \varphi(\epsilon')            \label{SUBEQooDOFUooUmCZpp} \\
                &\leq \epsilon.
            \end{align}
        \end{subequations}
        Justifications :
        \begin{itemize}
            \item Pour \eqref{SUBEQooZPTIooYsiEqh}. C'est le lemme \ref{LEMooXCVRooOSZYWv}.
            \item Pour \eqref{SUBEQooDOFUooUmCZpp}. L'application \( \varphi\) est croissante, lemme \ref{LEMooTJAXooKEqPCG}.
        \end{itemize}
        Donc la suite \(\varphi(x_k)\) est de Cauchy.

    \item[\( \varphi(x_k)\stackrel{\eR}{\longrightarrow}\bar x\)]
        Nous devons prouver que pour tout \( \epsilon\in \eR\), il existe \( N\) tel que \( n>N\) implique \( \varphi(x_n)\in B\big( \bar x,\epsilon \big)\). Nous allons faire ça en deux parties. D'abord \( \epsilon\in \eQ\) et ensuite \( \epsilon\in \eR\).
        \begin{subproof}
        \item[Avec \( \epsilon\in \eQ\)]
            Soit \( \epsilon\in \eQ\). Vu que \( x\) est une suite de Cauchy dans \( \eQ\), il existe \( N\) tel que si \( p,n>N\) nous avons
            \begin{equation}        \label{EQooJURNooOoSzDZ}
                x_p-\epsilon<x_n<x_p+\epsilon.
            \end{equation}
            Ces inégalités sont dans \( \eQ\). Nous fixons \( p\) et nous commençons par écrire plus en détail la première inéquation :
            \begin{equation}
                x_p-\epsilon-x_n<0.
            \end{equation}
            Autrement dit, pour tout \( n\) nous avons
            \begin{equation}
                (\overline{ x_p-\epsilon })_n<x_n.
            \end{equation}
            Pour rappel, la suite \(  \overline{ x_p-\epsilon } \) est la suite constante dans \( \eQ\). La suite
            \begin{equation}
                n\mapsto x_n-(\overline{ x_p-\epsilon })_n
            \end{equation}
            est dans \( \modE^+\). Donc, en vertu de la définition \ref{DEFooBXHJooOEYPRI} nous avons
            \begin{equation}
                \bar x-\overline{ x_p-\epsilon }\geq 0.
            \end{equation}
            Nous pouvons aussi bien écrire
            \begin{equation}
                \bar x\geq \varphi(x_p)-\varphi(\epsilon).
            \end{equation}
            En prenant l'autre inégalité de \eqref{EQooJURNooOoSzDZ} nous trouvons de la même manière que
            \begin{equation}
                \bar x\leq \varphi(x_p)+\varphi(\epsilon).
            \end{equation}
            Ces deux inégalités ensemble montrent que
            \begin{equation}
                \varphi(x_p)\in B\big( \bar x,\varphi(\epsilon) \big).
            \end{equation}
        \item[Avec \( \epsilon\in \eR\)]
            Nous considérons \( \epsilon\in \eR\) et \( \epsilon'\in \eQ\) tel que \( \varphi(\epsilon')<\epsilon\). Par le point précédent, il existe \( N\) tel que \( p>N\) implique
            \begin{equation}
                \varphi(x_p)\in B\big( \bar x,\varphi(\epsilon') \big).
            \end{equation}
            Étant donné que \( \varphi(\epsilon')<\epsilon\) nous avons 
            \begin{equation}
                \varphi(x_p)\in B\big( \bar x,\varphi(\epsilon') \big)\subset B\big( \bar x,\epsilon \big).
            \end{equation}
        \end{subproof}
    \end{subproof}
\end{proof}

\begin{proposition}     \label{PROPooZSQYooWRKNGY}
    Soit une suite convergente \( x_k\stackrel{\eQ}{\longrightarrow}q\). Alors
    \begin{equation}
        \varphi(x_k)\stackrel{\eR}{\longrightarrow}\varphi(q)
    \end{equation}
    où \( \varphi\) est la fonction qui à un rationnel fait correspondre la classe de la suite constante correspondante\footnote{Voir les notations en \ref{NORMooWBYNooBQaPPk}.}. 
\end{proposition}

\begin{proof}
    Le fait d'avoir une convergence \( x_k\to q\) dans \( \eQ\) implique que la suite \( (x_k)\) est de Cauchy, par la proposition \ref{PropFFDJooAapQlP}\ref{ItemRKCIooJguHdji}.
    
    Le lemme \ref{LemooRTGFooYVstwS} nous indique que \( \varphi(x_k)\) est une suite dans \( \eR\) qui converge vers \( \bar q\), la classe de la suite \( (x_k)\).

    À prouver : \( \varphi(x)=\bar q\). Autrement dit, nous devons prouver que la classe de la suite constante \( a_k=q\) et la classe de la suite \( x\) sont les mêmes.

    La suite \( (x_k-q)\) est de Cauchy dans \( \eQ\) et converge vers zéro par hypothèse. Donc les suites \(x\) et \( (q)\) sont dans la même classe.
\end{proof}

\begin{proposition}[\cite{MonCerveau}]     \label{PROPooFGBOooHiZqbs}
    Deux choses à propos de suites de rationnels convergeant vers un réel.
    \begin{enumerate}
        \item       \label{ITEMooMAVYooKFtqlx}
    Soit un réel \( x\). Il existe une suite de rationnels strictement croissante qui converge vers \( x\).
\item       \label{ITEMooVOVYooFUbccG}
    Si de plus \( x>0\), alors la suite (toujours strictement croissante) peut être choisie parmi les rationnels strictement positifs.
    \end{enumerate}
\end{proposition}

\begin{proof}
    Le lemme \ref{LemooHLHTooTyCZYL} nous sera d'une grande aide. Soit \( x\in \eR\). Il existe \( q_0\in \eQ\) tel que \( x-1<q_0<x\). Ensuite nous construisons la suite par récurrence : \( q_k\) est choisi tel que \( q_{k-1}<q_k<x\). Cela règle le point \ref{ITEMooMAVYooKFtqlx}.

    Pour \ref{ITEMooVOVYooFUbccG}. Il suffit de faire la même chose, en partant de \( 0<q_0<x\).
\end{proof}

\begin{theorem}[Complétude de \( \eR\), critère de Cauchy\cite{RWWJooJdjxEK}] \label{THOooUFVJooYJlieh}
    Nous avons :
    \begin{enumerate}
        \item
            Le corps \( \eR\) est un corps complet (définition~\ref{DefKCGBooLRNdJf}\ref{ITEMooKZZYooDaidGU})
        \item
            Une suite dans \( \eR\) est convergente (définition~\ref{DefKCGBooLRNdJf}\ref{ITEMooDERQooLmJwFR}) si et seulement si elle est de Cauchy (définition~\ref{DefKCGBooLRNdJf}\ref{ItemVXOZooTYpcYN}).
    \end{enumerate}
\end{theorem}
\index{complet!$\eR$!corps}
\index{critère!de Cauchy}
Notez la grande similitude entre ce théorème et le théorème~\ref{THOooNULFooYUqQYo}. Ils ne sont pas équivalents, ne parlent pas exactement du même objet «\( \eR\)», ni des mêmes notions de suites de Cauchy et de complétude.

\begin{proof}
    Soit \( (x_n)\) une suite de Cauchy dans \( \eR\). Pour chaque \( n\), il existe par le lemme~\ref{LemooHLHTooTyCZYL} un \( y_n\in \eQ\) tel que
    \begin{equation}
        x_n-\frac{1}{ n }<y_n<x_n+\frac{1}{ n }.
    \end{equation}
    \begin{subproof}
        \item[\( (y_n)\) est une suite de Cauchy dans \( \eQ\)]
            Nous prouvons que \( (y_n)\) est une suite de Cauchy dans \( \eQ\) (définition~\ref{DefKCGBooLRNdJf}\ref{ItemVXOZooTYpcYN}). Vu que \( (x_n)\) est de Cauchy pour le corps \( \eR\), si \( \epsilon>0\) dans \( \eR\) est donné, il existe \( n_{\epsilon}\) tel que si \( p,q\geq n_{\epsilon}\), alors \( | x_p-x_q |<\epsilon\).

        Nous avons :
        \begin{equation}
            | y_p-y_q |\leq | y_p-x_p |+| x_p-x_q |+| x_q-y_q |<\frac{1}{ p }+\epsilon+\frac{1}{ q }.
        \end{equation}
        En choisissant \( N_{\epsilon}>\max\{ n_{\epsilon},\frac{1}{ \epsilon } \}\) (ce qui est possible par le lemme~\ref{LemooMWOUooVWgaEi}), et en prenant \( p,q>N_{\epsilon}\), nous avons
        \begin{equation}
            | y_p-y_q |\leq 3\epsilon,
        \end{equation}
        ce qui prouve que \( (y_p)\) est une suite de Cauchy dans \( \eQ\), pour la notion de suite de Cauchy dans \( \eQ\).

    \item[Le réel représenté]

        Vu que \( (y_p)\) est de Cauchy dans \( \eQ\), elle représente un réel que nous notons \( \bar y\).

    \item[Convergence de \( (x_n)\)]

        Nous prouvons que \(     x_n\stackrel{\eR}{\longrightarrow}\bar y \).

        Nous savons qu'une suite de Cauchy de rationnels converge dans \( \eR\) vers le réel qu'elle représente, c'est-à-dire : \( y_n\stackrel{\eR}{\longrightarrow}\bar y\) où chaque \( y_n\in \eQ\) est vu comme la suite constante (cela est le lemme~\ref{LemooRTGFooYVstwS}). Autrement dit, pour \( \epsilon>0\), il existe un \( N_{\epsilon}\in \eN\) tel que si \( p>N_{\epsilon}\) alors \( | \bar y-y_p |<\epsilon\). Pour un tel \( p\) nous avons
        \begin{equation}
            | \bar y-x_p |\leq| \bar y-y_p |+| y_p-x_p |\leq \epsilon+\frac{1}{ p }.
        \end{equation}
        Donc dès que \( p\) est plus grand que \( \max\{ N_{\epsilon},\frac{1}{ \epsilon } \}\), nous avons \( | \bar y-x_p |<2\epsilon\), ce qui signifie que la suite \( (x_n) \) converge vers \( \bar y\) dans \( \eR\).

        Ceci achève de prouver que \( \eR\) est un corps complet.
        \end{subproof}

        En ce qui concerne l'équivalence entre les suites convergentes et de Cauchy, nous venons de prouver que toute suite de Cauchy dans \( \eR\) est convergente. La réciproque est la proposition~\ref{PROPooTFVOooFoSHPg}.

\end{proof}

Nous avons terminé avec la construction des réels. Les propriétés topologiques arrivent en la section~\ref{SECooGKHYooMwHQaD}. En particulier le théorème~\ref{THOooNULFooYUqQYo} pour la complétude de \( \eR\) en tant qu'espace métrique.

%--------------------------------------------------------------------------------------------------------------------------- 
\subsection{Intervalles}
%---------------------------------------------------------------------------------------------------------------------------

Nous avons déjà défini la notion d'intervalle pour un espace totalement ordonné en \ref{DefEYAooMYYTz}. Nous posons quelques notations dans \( \eR\).

\begin{definition}  \label{DEFooAQBUooKLChOW}
    Soient \( a\neq b\) dans \( \eR\). Nous définissons les parties suivantes de \( \eR\) :
    \begin{enumerate}
        \item
            \( \mathopen] a , b \mathclose[=\{ x\in \eR\tq a<x<b \}\)
        \item
            \( \mathopen[ a , b \mathclose[=\{ x\in \eR\tq a\leq x<b \}\)
            \item
            \( \mathopen] a , b \mathclose]=\{ x\in \eR\tq a<x\leq b \}\)
        \item
            \( \mathopen[ a , b \mathclose]=\{ x\in \eR\tq a\leq x\leq b \}\)
        \item
        \( \mathopen]-\infty , a \mathclose]=\{ x\in \eR\tq x\leq a \}\)
        \item
        \( \mathopen]-\infty , a \mathclose[=\{ x\in \eR\tq x< a \}\)
        \item
        \( \mathopen] a , \infty \mathclose[=\{ x\in \eR\tq x>a \}\)
        \item
            \( \mathopen[ a , \infty \mathclose[=\{ x\in \eR\tq x\geq a \}\).
            \item
            \( \mathopen] -\infty , \infty \mathclose[=\eR\).
    \end{enumerate}
    La proposition \ref{PROPooHPMWooQJXCAS} nous dira que tous les intervalles de \( \eR\) sont d'une de ces formes.
\end{definition}

%---------------------------------------------------------------------------------------------------------------------------
\subsection{Maximum, supremum et compagnie}
%---------------------------------------------------------------------------------------------------------------------------

Ce n'est un secret pour personne que $\eR$ est un ensemble totalement ordonné\footnote{Lemme \ref{PROPooYMJVooNAsXae}.} : il y a des éléments plus grands que d'autres, et mieux : à chaque fois que je prends deux éléments différents dans $\eR$, il y en a un des deux qui est plus grand que l'autre. Il n'y a pas d'\emph{ex æquo} dans $\eR$.

\begin{definition}
    Soit \( A\), une partie de \( \eR\). 
    \begin{enumerate}
        \item
            Un nombre \( M\) est un \defe{majorant}{majorant} de \( A\) si \( M\) est plus grand que tous les éléments de \( A\) : pour tout \( x\in A\), \( M\geq x\).
        \item
            Un nombre \( m\) est un \defe{minorant}{minorant} de \( A\) si \( m\) est plus petit que tous les éléments de \( A\) : pour tout \( x\in A\), \( m\leq x\).
    \end{enumerate}
    Nous parlons de majorant ou de minorants \emph{stricts} lorsque les inégalités sont strictes.
\end{definition}

Nous insistons sur le fait que l'inégalité n'est pas stricte. Ainsi, $1$ est un majorant de $[0,1]$. Dès qu'un ensemble a un majorant, il en a plein. Si $s$ majore l'ensemble $A$, alors $s+1$, $s+4$, et \( s+\frac{ 3 }{ 7 }\) majorent également $A$.

\begin{example}
Une petite galerie d'exemples de majorants.
\begin{itemize}
\item L'intervalle fermé $[4,8]$ admet entre autres $8$ et $130$ comme majorants,
\item l'intervalle ouvert $]4,8[$ admet également $8$ et $130$ comme majorants,
\item $7$ n'est pas un majorant de $[1,5]\cup]8,32]$,
\item $10/10$ majore les notes qu'on peut obtenir à un devoir.
\item l'intervalle $[4,\infty[$ n'a pas de majorants.
\end{itemize}
\end{example}

\begin{propositionDef}[Least-upper-bound property\cite{BIBooRRUXooKWzcFo}]		\label{DefSupeA}
    Soit $A$ une partie majorée de $\eR$. Il existe un unique élément \( M\in \eR\) tel que
    \begin{enumerate}
        \item
            $M\geq x$ pour tout $x\in A$,
        \item
            pour tout $\varepsilon$, le nombre $M-\varepsilon$ n'est pas un majorant de $a$, c'est-à-dire qu'il existe un élément $x\in A$ tel que $x>M-\varepsilon$.
    \end{enumerate}

    Cet élément est nommé \defe{supremum}{supremum} de $A$ et est noté \( \sup(A)\). De la même façon, \defe{l'infimum}{infimum} de $A$, noté $\inf A$, est le plus grand de ses minorants.
\end{propositionDef}

Par convention, si la partie n'est pas bornée vers le haut, nous dirons que son supremum n'existe pas, ou bien qu'il est égal à $+\infty$, suivant les contextes. Pour votre culture générale, sachez toutefois que $\infty\notin\eR$.

\begin{proof}
    Nous faisons la preuve pour l'infimum.

    \begin{subproof}
    \item[Unicité]

    En ce qui concerne l'unicité, soient \( m_1\) et \( m_2\), deux infimums de \( A\). Supposons \( m_1>m_2\). Alors il existe \( \epsilon>0\) tel que \( m_2<m_2+\epsilon<m_1\) (c'est le lemme~\ref{LemooHLHTooTyCZYL}). Cela prouve que \( m_2+\epsilon\) est un minorant de \( A\) et donc que \( m_2\) n'est pas un infimum.

\item[Existence]

	Soit $A$, une partie de $\eR$. Nous allons trouver son infimum en suivant une méthode de dichotomie. Pour cela nous allons construire trois suites en même temps de la façon suivante. D'abord nous choisissons un point $x_0$ de $A$ et un point $x_1$ qui minore $A$ (qui existe par hypothèse) :
	\begin{equation}
		\begin{aligned}[]
			x_0&\text{ est un élément de }A,\\
			x_1&\text{ est un minorant de }A,\\
			a_0&=x_0\\
			b_0&=x_1\\
			b_1&=x_1.
		\end{aligned}
	\end{equation}
	Ensuite, nous faisons la récurrence suivante :
	\begin{equation}
		\begin{aligned}[]
			x_{n+1}&=\frac{ a_n+b_n }{2},\\
			a_{n+1}&=\begin{cases}
                a_{n}	&	\text{si }x_{n+1} \text{ minore } A \\
				x_{n+1}	&	 \text{sinon},
			\end{cases}\\
			b_{n+1}&=\begin{cases}
                x_{n+1}	&	\text{si }x_{n+1} \text{ minore } A\\
				b_n	&	 \text{sinon}.
			\end{cases}
		\end{aligned}
	\end{equation}
    Nous allons montrer que \( (a_n)\) et \( (b_n)\) sont des suites convergentes de même limite et que cette limite est l'infimum de \( A\).

	Soit $n\in\eN$; il y a deux possibilités. Soit $a_n=a_{n-1}$ et $b_n=x_n$, soit $a_n=x_n$ et $b_n=b_{n-1}$. Supposons que nous soyons dans le premier cas (le second se traite de façon similaire). Alors nous avons
	\begin{equation}
		\begin{aligned}[]
			| a_n-b_n |&=| a_{n-1}-x_n |\\
			&=\left| a_{n-1}-\frac{ a_{n-1}+b_{n-1} }{2} \right| \\
			&=\frac{ 1 }{2}| a_{n-1}-b_{n-1} |,
		\end{aligned}
	\end{equation}
	ce qui prouve que $| a_n-b_n |\to 0$. Nous montrons maintenant que la suite \( (a_n)\) est de Cauchy. En effet nous avons
    \begin{equation}
        | a_n-a_{n-1} |=\begin{cases}
          0\\
          \left| \frac{ a_n -b_n}{ 2} \right|
      \end{cases}\leq \frac{1}{ 2n }.
    \end{equation}
    Il en est de même pour la suite \( (b_n)\). Ce sont deux suites de Cauchy (donc convergentes par la proposition~\ref{PROPooTFVOooFoSHPg}) qui convergent vers la même limite. Soit \( \ell\) cette limite.

	Le nombre $\ell$ minore $A$. En effet si $a\in A$ est plus petit que $\ell$, les éléments $b_n$ tels que $| b_n-\ell |<| a-\ell |$ ne peuvent pas minorer $A$. D'autre part, pour tout $\epsilon$, le nombre $\ell+\epsilon$ ne peut pas minorer $A$. En effet, $\ell$ est la limite de la suite décroissante $(a_n)$, donc il existe $a_n$ entre $\ell$ et $\ell+\epsilon$. Mais $a_n$ ne minore pas $A$, donc $\ell+\epsilon$ ne minore pas non plus $A$.

	Nous avons prouvé que toute partie minorée de $\eR$ possède un infimum.
    \end{subproof}

    La preuve que toute partie majorée possède un supremum se fait de la même façon.
\end{proof}

\begin{lemma}       \label{LEMooSSVKooDPhSkq}
    Soit une partie \( A\) de \( \eR\). Si \( M\) est un majorant de \( A\), alors \( M\geq \sup(A)\).
\end{lemma}

\begin{proof}
    Si \( M<\sup(A)\), alors en posant \( \epsilon=\sup(A)-M\), le nombre \( \sup(A)-\epsilon\) est encore un majorant de \( A\), ce qui est impossible par définition d'un supremum.
\end{proof}

%///////////////////////////////////////////////////////////////////////////////////////////////////////////////////////////
\subsubsection{Intervalles}
%///////////////////////////////////////////////////////////////////////////////////////////////////////////////////////////

\begin{proposition}     \label{PROPooHPMWooQJXCAS}
    Tous les intervalles de \( \eR\) sont d'une des formes listées dans la définition \ref{DEFooAQBUooKLChOW}.
\end{proposition}

%///////////////////////////////////////////////////////////////////////////
\subsubsection{Quelques exemples}
%///////////////////////////////////////////////////////////////////////////

En matière de notations, le maximum de l'ensemble $A$ est noté $\max A$, le supremum est noté $\sup A$. Le minimum et l'infimum sont notés $\min A$ et $\inf A$.

\begin{example}
Exemples de différence entre majorant, supremum et maximum.
\begin{itemize}
\item Le nombre $10$ est un supremum, majorant et maximum de l'intervalle fermé $[0,10]$,
\item Le nombre $10$ est un majorant et un supremum, mais pas un maximum de l'intervalle ouvert $]0,10[$,
\item Le nombre $136$ est un majorant, mais ni un maximum ni un supremum de l'intervalle $[0,10]$.
\end{itemize}
\end{example}

En utilisant les notations concises, ces différents cas s'écrivent ainsi :
\begin{equation}
    \begin{aligned}[]
10&=\max[0,10]=\sup[0,10]	& 10&=\sup[0,10[
    \end{aligned}
\end{equation}


\begin{example}
Si on dit que un pont s'effondre à partir d'une charge de $10$ tonnes, alors $10$ tonnes est un \emph{supremum} des charges que le pont peut supporter : si on met $9,999999$ tonnes dessus, il tient encore le coup, mais si on ajoute un gramme, alors il s'effondre (on sort de l'ensemble des charges acceptables).
\end{example}

\begin{example}
Si on dit qu'un pont résiste jusqu'à $10$ tonnes, alors $10$ tonnes est un \emph{maximum} de la charge acceptable. Sur ce pont-ci, on peut ajouter le dernier gramme. Mais à partir de là, le moindre truc qu'on ajoute, il s'effondre.
\end{example}

\begin{lemma}       \label{LEMooWCUXooFqTwDK}
    À propos de bornes d'un intervalle.
    \begin{enumerate}
        \item
	        La borne inférieure d'un intervalle est son infimum, 
        \item
            la borme supérieure est le supremum. 
        \item
            Si de plus l'intervalle est fermé, l'infimum est un minimum et le supremum est un maximum.
    \end{enumerate}
\end{lemma}


\begin{example}
    Quelques exemples dans les intervalles.
	\begin{enumerate}
		\item
			$A=\mathopen[ 1 , 2 \mathclose]$. Tous les nombres plus petits ou égaux à $1$ sont minorants, $1$ est infimum et minimum. Le nombre $2$ est un majorant, le maximum et le supremum.
		\item
			$B=\mathopen] 3 , \pi \mathclose[$. Le nombre $\pi$ est le supremum et est un majorant, mais n'est pas le maximum (parce que $\pi\notin B$). L'ensemble $B$ n'a pas de maximum. Bien entendu, $-1000$ est un minorant.
	\end{enumerate}
    Dans les deux cas, le nombre $53$ est un majorant.
\end{example}

Il existe évidemment de nombreux exemples plus vicieux.

\begin{example}
	Prenons $E=\{ \frac{1}{ n }\tq n\in\eN_0 \}$, dont les premiers points sont indiqués sur la figure~\ref{LabelFigSuiteUnSurn}. Cet ensemble est constitué des nombres $1$, $\frac{ 1 }{2}$, $\frac{1}{ 3 }$, \ldots Le plus grand d'entre eux est $1$ parce que tous les nombres de la forme $\frac{1}{ n }$ avec $n\geq 1$ sont plus petits ou égaux à $1$. Le nombre $1$ est donc maximum de $E$.

	L'ensemble $E$ n'a par contre pas de minimum parce que tout élément de $E$ s'écrit $\frac{1}{ n }$ pour un certain $n$ et est plus grand que $\frac{1}{ n+1 }$ qui est également dans $E$.

	Prouvons que zéro est l'infimum de $E$. D'abord, tous les éléments de $E$ sont strictement positifs, donc zéro est certainement un minorant de $E$. Ensuite, nous savons que pour tout $\varepsilon>0$, il existe un $n$ tel que $\frac{1}{ n }$ est plus petit que $\varepsilon$. L'ensemble $E$ possède donc un élément plus petit que $0+\varepsilon$, et zéro est bien l'infimum.
\end{example}

\newcommand{\CaptionFigSuiteUnSurn}{Les premiers points du type $x_n=1/n$.}
\input{auto/pictures_tex/Fig_SuiteUnSurn.pstricks}

L'exemple suivant est une source classique d'erreurs en ce qui concerne l'infimum. Il sera à relire après avoir vu la définition de limite (définition~\ref{PropLimiteSuiteNum}).

\begin{example}
	Les premiers points de l'ensemble $F=\{ \frac{ (-1)^n }{ n }\tq n\in\eN_0 \}$ sont représentés à la figure~\ref{LabelFigSuiteInverseAlterne}. Bien que (comme nous le verrons plus tard) la limite de la suite $x_n=(-1)^n/n$ soit zéro, il n'est pas correct de dire que zéro est l'infimum de l'ensemble $F$. Le dessin, au contraire, montre bien que $-1$ est le minium (aucun point est plus bas que $-1$), tandis que le maximum est $1/2$.

	Nous reviendrons avec cet exemple dans la suite. Pour l'instant, ayez bien en tête que zéro n'est rien de spécial pour l'ensemble $F$ en ce qui concerne les notions de maximum, minimum et compagnie.
\end{example}
\newcommand{\CaptionFigSuiteInverseAlterne}{Les quelques premiers points du type $(-1)^n/n$.}
\input{auto/pictures_tex/Fig_SuiteInverseAlterne.pstricks}

%--------------------------------------------------------------------------------------------------------------------------- 
\subsection{Racines}
%---------------------------------------------------------------------------------------------------------------------------
\label{SUBSECooMBCNooEqjjTY}

Dans cette section, nous définissons \( \sqrt{ x }\) pour \( x\in\eR^+\). Vous notez que c'est fait de façon assez algébrique\footnote{Discutable parce que des limites sont utilisées.}, ou en tout cas, en restant proche des définitions. Des définitions plus technologiques utilisant la continuité de \( x\mapsto x^n\) et qui prouvent que c'est bijectif sur un domaine choisi avec prudence existent, et c'est fait dans la définition \ref{DEFooJWQLooWkOBxQ}. Il est même expliqué dans \cite{BIBooMPXEooQLKhku} que la méthode décrite ici permet de définir \( \sqrt[n]{ x }\) pour tout \( n\) entier, et pas seulement pour \( n=2\).

\begin{proposition}     \label{PROPooUHKFooVKmpte}
    Soit \( q\in \eQ^+\). Il existe un unique \( r\in \eR\) tel que \( r^2=q\).
    
    Plus précisément, en termes des notations de \ref{NORMooWBYNooBQaPPk}, pour tout \( q\in \eQ^+\), il existe un unique \( r\in \eR^2\) tel que \( r^2=\varphi(q)\).
\end{proposition}

\begin{proof}
    En deux parties : d'abord l'existence et ensuite l'unicité.
    \begin{subproof}
        \item[Existence]
            Si \( q=0\), c'est \( r=0\). Nous supposons \( q>0\). La suite \( (x_k)\) de la proposition \ref{PROPooSTQXooHlIGVf} a la propriété d'être de Cauchy dans \( \eQ\). Donc il existe un réel \( r\) qui est la classe de cette suite. Nous posons donc
            \begin{equation}
                r=\bar x.
            \end{equation}
            
            En ce qui concerne \( r^2\), nous avons, par définition du produit dans \( \eR\),    
            \begin{equation}        \label{EQooPHLFooAZhebM}
                r^2=\bar x^2=\overline{ (x_k^2) },
            \end{equation}
            c'est la classe de la suite de Cauchy donnée par les \( x_k^2\). Posons \( y_k=x_k\); la relation \eqref{EQooPHLFooAZhebM} s'écrit
            \begin{equation}
                r^2=\bar y.
            \end{equation}
            
            La proposition \ref{PROPooSTQXooHlIGVf} nous dit également que \( y\) est une suite de Cauchy et que
            \begin{equation}
                y_k\stackrel{\eQ}{\longrightarrow}q
            \end{equation}
            La proposition \ref{PROPooZSQYooWRKNGY} donne alors \( \bar y=\bar q\), et finalement
            \begin{equation}
                r^2=\bar q=\varphi(q).
            \end{equation}
            Ici tout n'est pas encore terminé avec l'existence parce qu'il faut nous assurer que \( r\geq 0\). Ce n'est pas très compliqué : si \( r<0\), alors nous pouvons faire le choix \( -r\) qui convient tout aussi bien : \( (-r)^2=r^2\).
        \item[Unicité]
            Supposons \( r_1,r_2\in \eR\) tels que \( r_1^2=r_2^2\). Le lemme \ref{PROPooYMJVooNAsXae} dit que \( \eR\) est totalement ordonné; disons pour fixer les idées que \( r1\leq r_2\). Cela signifie, par définition de l'ordre sur \( \eR\), que \( r_2-r_1\geq 0\). En posant \( s=r_2-r_1\) nous avons \( r_2=r_1+s\). Passons au carré; la distribution dans le calcul suivant provient du fait que \( \eR\) est un corps :
            \begin{equation}
                r_2^2=(r_1+s)^2=r_1^2+2r_1s+s^2.
            \end{equation}
            Vu que \( r_1^2=q=r_2^2\), nous avons \( 2r_1s+s^2=0\) ou encore
            \begin{equation}
                s(2r_1+s)=0.
            \end{equation}
            Vu que \( \eR\) est un corps, il est un anneau intègre\footnote{Lemme \ref{LemAnnCorpsnonInterdivzer}.} et la règle du produit nul s'applique : soit \( s=0\), soit \( 2r_2+s=0\). Vu que \( r_2>0\) et que \( s\geq 0\), nous avons \( 2r_2+s>0\) et donc \( s=0\). 

            Nous en déduisons que \( r_1=r_2\).
    \end{subproof}
\end{proof}


%--------------------------------------------------------------------------------------------------------------------------- 
\subsection{Corps valué}
%---------------------------------------------------------------------------------------------------------------------------

\begin{definition}[Valeur absolue, corps valué\cite{BIBooVKGHooSPijZp,BIBooNRMUooJmwzpn}]       \label{DEFooBWXXooAkBBRS}
    Soit un corps \( \eK\). Une \defe{valeur absolue}{valeur absolue} sur \(\eK\) est une application \( | . |\colon \eK\to \eR^+\) telle que
    \begin{enumerate}
        \item
            \( | x |=0\) si et seulement si \( x=0\),
        \item
            \( | x+y |\leq | x |+| y |\)
        \item
            \( | xy |\leq | x | | y |\).
    \end{enumerate}
    Un corps muni d'une valeur absolue est un \defe{corps valué}{corps valué}.
\end{definition}


%+++++++++++++++++++++++++++++++++++++++++++++++++++++++++++++++++++++++++++++++++++++++++++++++++++++++++++++++++++++++++++
\section{Les complexes}
%+++++++++++++++++++++++++++++++++++++++++++++++++++++++++++++++++++++++++++++++++++++++++++++++++++++++++++++++++++++++++++

\begin{definition}[Nombres complexes\cite{BIBooBSMSooTkhjce}]
    L'ensemble des \defe{nombres complexes}{nombres complexes} \( \eC\) est l'ensemble \( \eR^2\) muni des opérations suivantes :
    \begin{enumerate}
        \item
            \begin{equation}        \label{EQooIJWOooZBiKEW}
                \begin{aligned}
                    \times_{\eC}\colon \eC\times \eC&\to \eC \\
                    \big( (x,y),(x',y') \big)&\mapsto (xx'-yy', xy'+yx') 
                \end{aligned}
            \end{equation}
        \item
            \begin{equation}
                \begin{aligned}
                    +_{\eC}\colon \eC\times \eC&\to \eC \\
                    \big( (x,y),(x',y') \big)&\mapsto (x+x', y+y') 
                \end{aligned}
            \end{equation}
        \item
            \begin{equation}
                \begin{aligned}
                    \cdot\colon \eR\times \eC&\to \eC \\
                    \big( \lambda,(x,y) \big)&\mapsto (\lambda x, \lambda y). 
                \end{aligned}
            \end{equation}
    \end{enumerate}
\end{definition}

\begin{lemma}
    Le triple \( (\eC,+_{\eC}, \times_{\eC})\) est un anneau\footnote{Définition \ref{DefHXJUooKoovob}.} commutatif dont le neutre pour l'addition est \( (0,0)\) et le neutre pour la multiplication est \( (1,0)\).
\end{lemma}

\begin{proof}
    Ce ne sont que des calculs. Juste pour vous montrer, voici la première partie pour l'associativité :
    \begin{subequations}
        \begin{align}
            \big( (a,b)(x,y) \big)(s,t)&=(ax-by,ay+bx)(s,t)\\
            &=(axs-bys-ayt-bxt,axt-byt+ays+bxs).
        \end{align}
    \end{subequations}
    Nous avons utilisé la distributivité sur \( \eR\), provenant du fait que \( \eR\) est un corps par le théorème \ref{DefooFKYKooOngSCB}.
\end{proof}

\begin{lemma}
    L'anneau \( \eC\) est un corps.
\end{lemma}

\begin{proof}
    % TODOooBRFPooClucQE: justifiet la stricte inégalité a^2+b^2>0
    Il suffit de trouver un inverse pour chaque élément non nul. Soit un élément non nul \( (a,b)\in \eC\). En combinant les lemmes \ref{LEMooTPLUooXiCZHJ} et \ref{LEMooNLGSooSGdvAo} nous savons que \( a^2+b^2>0\). En particulier, cet élément est inversible dans \( \eR\), et nous pouvons considérer l'élément suivant de \( \eC\) :
    \begin{equation}
        z=\big( \frac{ a }{ a^2+b^2 }, -\frac{ b }{ a^2+b^2 } \big).
    \end{equation}
    Prouver que \( z(a,b)=(1,0)\) est maintenant juste un calcul.
\end{proof}

\begin{lemma}
    L'application
    \begin{equation}
        \begin{aligned}
            \varphi\colon \eR&\to \eC \\
            x&\mapsto (x,0) 
        \end{aligned}
    \end{equation}
    est un morphisme d'anneaux\footnote{Définition \ref{DEFooSPHPooCwjzuz}.}.
\end{lemma}

\begin{proof}
    Simples calculs. Par exemple
    \begin{equation}
        \varphi(xx')=(xx',0)=(x,0)(x',0)=\varphi(x)\varphi(x').
    \end{equation}
\end{proof}

\begin{normaltext}
    Admirez \ldots
    \begin{itemize}
        \item Un nombre complexe est un couple de réels.
        \item Un réel est une classe d'équivalence de suites de Cauchy de rationnels.
        \item Une suite de Cauchy de rationnels est une application \( \eN\to \eQ\) vérifiant certaines propriétés.
        \item Un rationnel est une classe d'équivalences d'éléments de \( \eZ\).
        \item Un élément de \( \eZ\) n'est pas encore définit, mais ça va être une classe d'équivalence de couples de naturels.
        \item Un naturel sera \ldots là c'est plus compliqué. Une construction vraiment rigoureuse des naturels risque d'être en dehors du cadre du Frido.
    \end{itemize}
    Bref, les objets que nous manipulons sont d'une effroyable complexité. 
\end{normaltext}

\begin{normaltext}
    À partir de maintenant, lorsque nous parlons de \( \eR\), nous parons en réalité de \( \varphi(\eR)\subset \eC\).
\end{normaltext}

\begin{lemma}
    Nous avons \( (0,1)^2=(-1,0)\). Nous notons \( i=(0,1)\).
\end{lemma}

\begin{proof}
    Calcul direct à partir de la définition \ref{EQooIJWOooZBiKEW}.
\end{proof}

\begin{proposition}[\cite{MonCerveau}]     \label{PROPooKQHLooMFxNLe}
    L'application
    \begin{equation}
        \begin{aligned}
            f\colon \eR^2&\to \eC \\
            (a,b)&\mapsto a\varphi(1)+bi
        \end{aligned}
    \end{equation}
    est un isomorphisme de \( \eR\)-module\footnote{Module, définition \ref{DEFooHXITooBFvzrR}.}.
\end{proposition}


Cette proposition permet d'écrire tout nombre complexe sous la forme \( a+bi\) pour des réels \( a\) et \( b\). 

% Quand ce sera fait, il sera du meilleur effet de laisser un commentaire ici:
% http://www.les-mathematiques.net/phorum/read.php?11,2093686,2194966#msg-2194966




L'étude de la série géométrique est reportée à (beaucoup) plus tard, à la proposition \ref{PROPooWOWQooWbzukS}. Dans l'immédiat il nous est possible de calculer la somme partielle.
\begin{lemma}[Somme partielle de la série géométrique]      \label{LEMooAFSCooWEVlvp}
    Soit \( q\in \eC\). Nous avons
    \begin{equation}
        \sum_{n=0}^Nq^n=\frac{ 1-q^{N+1} }{ 1-q }.
    \end{equation}
\end{lemma}

\begin{proof}
    Posons \( S_N=1+q+\ldots+q^N\). Nous avons évidemment $S_N-qS_N=1-q^{N+1}$ et donc
    \begin{equation}    \label{EqASYTiCK}
        S_N=\sum_{n=0}^Nq^n=\frac{ 1-q^{N+1} }{ 1-q }.
    \end{equation}
\end{proof}

