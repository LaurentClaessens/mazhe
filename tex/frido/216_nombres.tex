% This is part of Mes notes de mathématique
% Copyright (c) 2011-2024
%   Laurent Claessens
% See the file fdl-1.3.txt for copying conditions.


%+++++++++++++++++++++++++++++++++++++++++++++++++++++++++++++++++++++++++++++++++++++++++++++++++++++++++++++++++++++++++++
\section{Corps}
%+++++++++++++++++++++++++++++++++++++++++++++++++++++++++++++++++++++++++++++++++++++++++++++++++++++++++++++++++++++++++++

La définition d'un corps est \ref{DefTMNooKXHUd}.

%---------------------------------------------------------------------------------------------------------------------------
\subsection{Définitions, morphismes}
%---------------------------------------------------------------------------------------------------------------------------

La proposition suivante donne une caractérisation d'un corps, en disant un tout petit peu plus que la définition~\ref{DefTMNooKXHUd}.
\begin{proposition}
	L'anneau \( A\) est un corps si et seulement si \( U(A) = A^* \).
\end{proposition}

\begin{proof}
	En deux parties.
	\begin{subproof}
		\spitem[Sens direct]
		Nous supposons que \( A\) est un corps. D'une part tous les éléments non nuls sont inversibles, c'est-à-dire \( A^*\subset U(A)\).

		Pour l'inclusion inverse, nous montrons qu'une élément inversible ne peut pas être nul. Cela n'est autre que le lemme~\ref{LEMooVUSMooWisQpD} couplé à la proposition~\ref{PROPooNCCGooXjVyVt} : \( a\cdot 0=0\neq 1\) pour tout \( a\).
		\spitem[Sens inverse]
		Si \( U(A)=A^*\), nous avons immédiatement que tous les éléments non nuls sont inversibles et donc que \( A\) est un corps.
	\end{subproof}
\end{proof}

\begin{lemma}       \label{LEMooJNIBooAURhrt}
	Si \( \eK\) est un corps et si \( a\in \eK\) vérifie \( a^2=1\), alors \( a=\pm 1\).
\end{lemma}

\begin{proof}
	Soit \( a\in \eK\) tel que \( a^2=1\). Alors \( a^2-1=0\). Il est facile de vérifier\footnote{Parce que tous les corps du Frido sont commutatifs.} que \( (a+1)(a-1)=a^2-1\). Bref, on a le produit nul
	\begin{equation}
		(a+1)(a-1)=0.
	\end{equation}
	Le lemme \ref{LEMooIKNMooMfvQnu} dit que tout corps respecte la règle du produit nul. Donc soit \( a+1=0\) soit \( a-1=0\). Donc \( a\in\{ 1,-1 \}\).
\end{proof}

\begin{definition}[Morphisme de corps]		\label{DEFooWGKFooJaakjf}
	Un corps étant un anneau sans plus de structure, un \defe{morphisme de corps}{morphisme!de corps}\index{isomorphisme!de corps} n'est qu'un morphisme des anneaux\footnote{Définition \ref{DEFooSPHPooCwjzuz}.}.
\end{definition}

Le lemme suivant montre que définir un morphisme de corps comme étant simplement un morphisme des anneaux est une bonne idée.
\begin{lemma}       \label{LEMooWBOPooZnsZgQ}
	Si \( \varphi\colon \eK\to \eK'\) est un morphisme de corps, alors
	\begin{enumerate}
		\item
		      pour tout \( a\in \eK\) nous avons \( \varphi(a^{-1})=\varphi(a)^{-1}\);
		\item
		      le morphisme \( \varphi\) est injectif.
	\end{enumerate}
\end{lemma}

\begin{proof}
	Vu que \( \varphi(1)=1\), nous avons aussi
	\begin{equation}
		1=\varphi(aa^{-1})=\varphi(a)\varphi(a^{-1}).
	\end{equation}
	Donc, par unicité de l'inverse\footnote{Lemme~\ref{LEMooECDMooCkWxXf}\,\ref{ITEMooOIWTooYqmMPP}.}, \( \varphi(a^{-1})=\varphi(a)^{-1}\).

	Pour l'injectivité nous supposons \( \varphi(a)=\varphi(b)\). Étant donné que \( \eK'\) est un corps, nous pouvons multiplier par \( \varphi(b)^{-1}\) :
	\begin{equation}
		\varphi(a)\varphi(b)^{-1}=1.
	\end{equation}
	En utilisant le premier point nous avons \( 1=\varphi(a)\varphi(b^{-1})\), puis le morphisme d'anneaux : \( 1=\varphi(ab^{-1})\), et encore le morphisme d'anneaux nous permet de déduire \( ab^{-1}=1\) et donc \(a=b\).
\end{proof}




%--------------------------------------------------------------------------------------------------------------------------- 
\subsection{Sous-groupe engendré}
%---------------------------------------------------------------------------------------------------------------------------


\begin{definition}[Sous-groupe engendré]          \label{DefooRDRXooEhVxxu}
	Soit \( A\) une partie du groupe \( G\). Le sous-groupe \defe{engendré}{sous-groupe!engendré}\index{engendré!sous-groupe} par \( A\) est l'intersection de tous les sous-groupes de \( G\) contenant \( A\). Nous notons ce groupe \( \gr_G(A)\)\nomenclature[R]{\( \gr_G\)}{groupe engendré}.

	Lorsque \( A \) est fini (disons \( A = \{a_1, \dots, a_n\} \)), on note aussi le sous-groupe engendré \( \langle a_1, \dots, a_n \rangle \).
\end{definition}

\begin{normaltext}
	Un sous-groupe engendré n'est jamais vide parce qu'il contient toujours au moins le neutre (parce que c'est un sous-groupe). Si \( G\) est un groupe, le sous-groupe \( \gr_G(\emptyset)\) lui-même contient \( e\)\footnote{Demandez-vous si il est possible que \( \gr(\emptyset)\) contienne d'autres éléments que \( e\).}.
\end{normaltext}

\begin{normaltext}
	Dans de nombreux cas, le groupe «ambiant» \( G\) est entendu par le contexte et nous noterons \( \gr(A)\) au lieu de \( \gr_G(A)\).

	Si par exemple \( A\) est la matrice \( \begin{pmatrix}
		4 & 5 \\
		6 & 7
	\end{pmatrix}\), le groupe \( \gr(A)\) est à comprendre dans \( \GL(2,\eR)\). Il faudrait être fou pour avoir en tête un autre groupe que \( \GL(2,\eR)\) sans le préciser.

	D'ailleurs, connaissez-vous un groupe contenant la matrice \( A\) et n'étant pas un sous-groupe de \( \GL(2,\eC)\) ?
\end{normaltext}

\begin{lemma}
	Si \( G\) est un groupe et \( A\) une partie de \( G\), alors \( \gr(A)\) est un sous-groupe de \( G\).
\end{lemma}

Le sous-groupe engendré par \( A \) est le plus petit (pour l'inclusion) groupe de \( G\) contenant \( A\). Plus formellement, nous avons le résultat suivant :
\begin{lemma}
	Tout sous-groupe de \( G\) contenant \( A\) contient \( \gr(A)\).
\end{lemma}

\begin{proof}
	Si \( H\) est un sous-groupe de \( G\) contenant \( A\), alors \( \gr(A)\) est l'intersection de \( H\) avec tous les autres sous-groupes de \( G\) contenant \( A\). Il contient donc \( \gr(A)\).
\end{proof}

\begin{lemma}[\cite{BIBooERNQooTXQPvD}]   \label{LemFUIZooBZTCiy}
	Si \( A\) est une partie du groupe \( G\), alors le sous-groupe \( \gr(A)\) engendré\footnote{Définition~\ref{DefooRDRXooEhVxxu}.} par \( A\) est l'ensemble de tous les produits finis d'éléments de \( A\) et de \( A^{-1}\) (l'identité est le produit à zéro éléments).

	C'est-à-dire que tout élément de \( \gr(A)\) peut être écrit sous la forme\footnote{Les \( a_i\) négatifs correspondent aux inverses. Notons que si \( g\in A\), il n'y a pas de garanties que \( g^{-1}\) soit également dans \( A\).}
	\begin{equation}
		\prod_{i=1}^ng_i^{a_i}
	\end{equation}
	où \( a_i\in \eZ\) et \( g\colon \eN\to A\) n'est pas spécialement injective : il peut arriver que \( g_i=g_j\).
\end{lemma}

\begin{proof}
	Puisqu'un produit vide est égal à l'identité\footnote{Voir \ref{NORMooDBOFooQCwbOY}.}, le lemme est vrai (un peu trivialement) dans le cas où \( A=\emptyset\). À partir de maintenant, nous supposons que \( A\) est non vide.

	Nous nommons \( \gr(A)\) le groupe engendré par \( A\) et \( H\), l'ensemble
	\begin{equation}
		H=\{ g_1\ldots g_n\tq g_i\in A\cup A^{-1} \}.
	\end{equation}
	Nous commençons par prouver que \( H\) est un groupe.
	\begin{itemize}
		\item Puisque \( A\) est non vide, nous considérons \( a\in A\). Dans ce cas, \( e=aa^{-1}\in H\). Donc \( e\in H\).
		\item L'inverse de \( g_1\ldots g_n\) est \( g_n^{-1}\ldots g_1^{-1}\) qui est également dans \( H\).
		\item Le produit de \( g_1\ldots g_n\) par \( h_1\ldots h_n\), tous éléments de \( H\), est également dans \( H\)\footnote{Et c'est ici qu'on se rend compte que la décomposition n'est probablement que rarement unique.}.
	\end{itemize}
	Comme \( H\) est un groupe contenant \( A\), nous avons \( \gr(A)\subset H\) parce que \( \gr(A)\) est une intersection dont un des éléments est \( H\).

	Par ailleurs tout groupe contenant \( A\) doit contenir les inverses et les produits finis, donc \( H\subset \gr(A)\).

	Au final, \( H=\gr(A)\), ce qu'il fallait.
\end{proof}

\begin{lemma}       \label{LEMooCFTVooKvmyKN}
	Soit un groupe \( G\) et un sous-groupe \( H=\gr(h_1,\ldots, h_n)\). Si \( \alpha\in G\), alors
	\begin{equation}
		\alpha H\alpha^{-1}=\gr(\alpha h_1\alpha^{-1},\ldots, \alpha h_n\alpha^{-1}).
	\end{equation}
\end{lemma}

\begin{proof}
	Il s'agit d'une conséquence du lemme \ref{LemFUIZooBZTCiy}. Un élément de \( \gr(\alpha h_1\alpha^{-1},\ldots, \alpha h_n\alpha^{-1})\) est un produit d'éléments de \( G\) de la forme \( \alpha h_i\alpha^{-1}\) ou \( (\alpha h_j\alpha^{-1})^{-1}=\alpha h_j^{-1}\alpha^{-1}\). Or nous avons
	\begin{equation}
		\alpha h_i\alpha^{-1}\alpha h_j\alpha^{-1}=\alpha h_ih_j\alpha^{-1}\in \alpha H\alpha^{-1}.
	\end{equation}
	Donc
	\begin{equation}
		\gr(\alpha h_1\alpha^{-1},\ldots, \alpha h_n\alpha^{-1})\subset \alpha H\alpha^{-1}.
	\end{equation}
	L'inclusion dans l'autre sens est du même tonneau.
\end{proof}

\begin{definition}[Partie génératrice, groupe monogène]  \label{DEFooWMFVooLDqVxR}
	Soient un groupe \( G\), et une partie \( A\subset G\). Si \( \gr(A)=G\), alors nous disons que \( A\) est une \defe{partie génératrice}{partie!génératrice} du groupe \( G\).

	Un groupe est \defe{monogène}{monogène} si il a une partie génératrice réduite à un seul élément.
\end{definition}

\begin{definition}[Groupe cyclique]     \label{DefHFJWooFxkzCF}
	Un élément \( a\in G\) est un \defe{générateur}{générateur} de \( G\) si tous les éléments de \( G\) s'écrivent sous la forme \( a^n\) pour un certain \( n\in\eZ\). Un groupe fini et monogène est dit \defe{cyclique}{groupe cyclique}.
\end{definition}

\begin{normaltext}
	La différence entre un groupe monogène et un groupe cyclique est qu'un groupe cyclique est fini. Dans un groupe cyclique, à force d'itérer le générateur, nous finissons par tourner en rond -- d'où le nom.
\end{normaltext}

\begin{example}
	Soit le groupe \( \big( \eZ/10\eZ,+ \big)\). L'élément \( [2]_{10}\) n'est pas générateur parce que ses puissances\footnote{Attention aux notations; en général on écrit la loi de groupe de façon multiplicative et on parle des puissances d'un élément, mais ici on écrit la loi de groupe additivement, donc les «puissances» sont en réalité les multiples.} sont
	\begin{equation}
		\gr([2]_{10})=\{ [2]_{10},[4]_{10},[6]_{10},[8]_{10},[0]_{10} \}.
	\end{equation}
	Par contre l'élément \( [3]_{10}\) est générateur : ses puissances sont dans l'ordre
	\begin{equation}
		[3]_{10}, [6]_{10}, [9]_{10}, [2]_{10}, [5]_{10}, [8]_{10},[1]_{10},[4]_{10},[7]_{10},[0]_{10}.
	\end{equation}
\end{example}

Un exemple presque identique, mais un peu masqué sera l'exemple \ref{EXooOXAAooZMdDfP}.


\begin{lemma}[\cite{MonCerveau}]        \label{LEMooTHBIooJPjBlp}
	Si \( n=dr\), alors
	\begin{enumerate}
		\item
		      \( r\eZ/n\eZ\) est un groupe.
		\item   \label{LEMooSAAXooMezYui}
		      Le groupe \( r\eZ/n\eZ\) est cyclique\footnote{Définition \ref{DefHFJWooFxkzCF}.}.
		\item       \label{ITEMooGCGQooIfZULt}
		      \( \Card(r\eZ/n\eZ)=d=n/r\).
	\end{enumerate}
	%TODOooBKXYooOeWGbQ. Prouver ça.
\end{lemma}


%+++++++++++++++++++++++++++++++++++++++++++++++++++++++++++++++++++++++++++++++++++++++++++++++++++++++++++++++++++++++++++
\section{Module sur un anneau}
%+++++++++++++++++++++++++++++++++++++++++++++++++++++++++++++++++++++++++++++++++++++++++++++++++++++++++++++++++++++++++++

\begin{definition}[module sur un anneau\cite{ooJGVOooSjQBVh}]       \label{DEFooHXITooBFvzrR}
	Soit un anneau \( A\). Un \defe{module à gauche}{module!à gauche} sur \( A\) est la donnée d'un triplet \( (M,+,\cdot)\) où
	\begin{enumerate}
		\item
		      \( +\) est une loi de composition interne à \( M\), c'est-à-dire \( +\colon M\times M\to M\),
		\item
		      \( \cdot\) est une loi de composition externe, c'est-à-dire \( \cdot\colon A\times M\to M\)
	\end{enumerate}
	telles que
	\begin{enumerate}
		\item
		      \( (M,+)\) est un groupe\footnote{Nous verrons dans la proposition~\ref{PROPooGARGooDiMqtN} qu'il est forcément commutatif.}.
		\item
		      \( a\cdot(x+y)=a\cdot x+a\cdot y\),
		\item
		      \( (a+b)\cdot x=a\cdot x+b\cdot x\),
		\item
		      \( (ab)\cdot x=a\cdot(b\cdot x)\)
		\item
		      \( 1\cdot x=x\).
	\end{enumerate}
	pour tout \( a,b\in A\) et \( x,y\in M\).

	Si \( M\) et \( N\) sont des \( A\)-modules, un \defe{morphisme}{morphisme de modules} de \( M\) vers \( N\) est une application \( f\colon M\to N\) qui
	\begin{enumerate}
		\item
		      est un morphisme de groupes entre \( (M,+)\) et \( (N,+)\)
		\item
		      vérifie \( f(a\cdot x)=a\cdot f(x)\) pour tout \( a\in A\), \( x\in M\).
	\end{enumerate}
	L'ensemble des morphismes entre \( M\) et \( N\) est noté \( \Hom_A(M,N)\). Si \( B\) est un sous-anneau de \( A\),  nous parlons de \( \Hom_B(M,N)\) pour parler des morphismes de groupes qui ne vérifient \( f(a\cdot x)=a\cdot f(x)\) que pour \( a\in B\).
\end{definition}

\begin{proposition}\label{PROPooGARGooDiMqtN}
	Si \( M\) est un module sur un anneau, alors \( (M,+)\) est un groupe commutatif.
\end{proposition}

\begin{proof}
	Il suffit de calculer \( (1+1)\cdot (x+y)\) de deux façons différentes :
	\begin{equation}
		(1+1)\cdot (x+y)=1\cdot (x+y)+1\cdot (x+y)=x+y+x+y
	\end{equation}
	d'une part et
	\begin{equation}
		(1+1)\cdot (x+y)=(1+1)\cdot x+(1+1)\cdot y=x+x+y+y,
	\end{equation}
	d'autre part. En égalant les deux expressions, il vient
	\begin{equation}
		x+y+x+y=x+x+y+y,
	\end{equation}
	qui se simplifie (nous sommes dans un groupe) en \( y+x=x+y\).
\end{proof}

\begin{definition}\label{DEFooKHWZooIfxdNc}
	Un \defe{espace vectoriel}{espace!vectoriel} est un module\footnote{Définition \ref{DEFooHXITooBFvzrR}.} sur un corps commutatif\footnote{La condition de commutativité n'est pas indispensable, mais comme nous ne parlerons que de corps commutatifs\ldots}.
\end{definition}

\begin{definition}[\cite{BIBooSTWWooItiMUp}]        \label{DEFooRUKVooLnXxdS}
	Soient un \( A\)-module \( M\) et un ensemble \( I\). Une famille \( \{ m_i \}_{i\in I}\) est \defe{libre}{partie libre!module} si les \( m_i\) sont \defe{linéairement indépendants}{linéairement indépendant!module}, c'est-à-dire si pour tout choix d'une partie finie \( J\) dans \( I\) et d'éléments \( (a_j)_{j\in J}\) dans \( A\), si nous avons
	\begin{equation}
		\sum_{j\in J}a_jm_j=0,
	\end{equation}
	alors \( a_j=0\) pour tout \( j\).
\end{definition}

\begin{definition}[\cite{BIBooNKWVooYfrwSd}]        \label{DEFooWBOBooJNyyBF}
	Soit \( S\), une partie du \( A\)-module \( M\). Le \defe{sous-module engendré}{sous-module engendré} par \( S\) est l'ensemble des éléments de \( M\) qui sont des combinaisons linéaires finies d'éléments de \( S\), c'est-à-dire de sommes de la forme
	\begin{equation}
		\sum_{t\in T}a_tt
	\end{equation}
	où \( T\) est fini dans \( S\) et \( a_t\in A\).
\end{definition}

%--------------------------------------------------------------------------------------------------------------------------- 
\subsection{Module produit}
%---------------------------------------------------------------------------------------------------------------------------

\begin{lemmaDef}[\cite{BIBooSTWWooItiMUp}]        \label{DEFooLCJEooBvVmkV}
	Soient un anneau \( A\) et un ensemble \( I\). Le \( A\)-\defe{module produit}{module produit} \( A^I\) est l'ensemble des applications \( I\to A\).

	En termes de notations, nous écrivons ceci :
	\begin{equation}
		A^I=\{ (a_i)_{i\in I},a_i\in A \}.
	\end{equation}
	L'ensemble \( A^I\) devient un module par les définitions, pour \( x,y\in A^I\) et \( a\in A\) :
	\begin{subequations}
		\begin{align}
			ax  & =(ax_i)_{i\in I}        \label{EQooPLBSooFuQXrj}  \\
			x+y & =(x_i+y_i)_{i\in I}     \label{EQooODBMooQKLUgd}.
		\end{align}
	\end{subequations}
	En d'autres termes, \( A^I=\Fun(I,A)\).
\end{lemmaDef}

\begin{proof}
	Il faut vérifier toutes les conditions de la définition \ref{DEFooHXITooBFvzrR}. En guise d'exemple, nous vérifions la distributivité. Soient \( a\in A\) et \( x,y\in A^I\). Nous avons
	\begin{subequations}
		\begin{align}
			\big[  a\cdot(x+y)  \big]_i & =a(x+y)_i   & \text{\eqref{EQooPLBSooFuQXrj}}  \\
			                            & =a(x_i+y_i) & \text{\eqref{EQooODBMooQKLUgd}}  \\
			                            & =ax_i+ay_i  & \text{distrib. dans \( A\)}      \\
			                            & =(ax+ay)_i  & \text{\eqref{EQooODBMooQKLUgd}}.
		\end{align}
	\end{subequations}
\end{proof}

\begin{lemma}
	Pour chaque \( i\in I\) nous considérons l'élément \( e_i\in A^I\) donné par
	\begin{equation}
		\begin{aligned}
			e_i\colon I & \to A                      \\
			j           & \mapsto \begin{cases}
				                      1 & \text{si } j=i \\
				                      0 & \text{sinon. }
			                      \end{cases}
		\end{aligned}
	\end{equation}
	La famille \( \{ e_i \}_{i\in I}\) est libre\footnote{Définition \ref{DEFooRUKVooLnXxdS}.} dans \( A^I\).
\end{lemma}

\begin{proof}
	Soient \( J\) fini dans \( I\) ainsi que des éléments \( a_j\in A\) (\( j\in J\)). Nous supposons que\footnote{Pour rappel, les sommes finies sont définies par \ref{DEFooLNEXooYMQjRo}.} \( \sum_{j\in J}a_je_j=0\). Calculons un peu :
	\begin{equation}
		\sum_{j\in J}a_je_j=\sum_{j\in J}(a_j\delta_{ji})_{i\in I}=\left( \sum_{j\in J}a_j\delta_{ji} \right)_{i\in I}.
	\end{equation}
	Pour que le tout soit nul dans \( A^I\), il faut que
	\begin{equation}
		\sum_{j\in J}a_j\delta_{ji}
	\end{equation}
	soit nul pour tout \( i\in I\). Si nous fixons \( i\in I\), la somme sur \( j\) possède un seul terme non annulé par \( \delta_{ji}\), et c'est le terme \( j=i\). Nous avons donc \( a_i=0\).
\end{proof}

\begin{definition}      \label{DEFooBMEPooFsCHgb}
	Nous notons \( A^{(I)}\) le sous-module de \( A^I\) engendré\footnote{Définition \ref{DEFooWBOBooJNyyBF}.} par les \( e_i\).
\end{definition}

\begin{lemma}[\cite{MonCerveau}]
	L'ensemble \( A^{(I)}\) est l'ensemble des applications \( I\to A\) de support fini.
\end{lemma}

\begin{proof}
	En deux sens.
	\begin{subproof}
		\spitem[Si \( x\in A^{(I)}\)]
		Pour rappel, la définition \ref{DEFooLCJEooBvVmkV} nous dit que \( x\) est une application \( I\to A\). Vu que \( x\) est dans le sous-module engendré par les \( e_i\), il existe une partie finie \( J\subset I\) telle que
		\begin{equation}
			x=\sum_{j\in J}x_je_j.
		\end{equation}
		Pour \( i\in I\) nous avons
		\begin{equation}
			x(i)=\sum_{j\in J}x_j\delta_{ij}=\begin{cases}
				x_i & \text{si } i\in J \\
				0   & \text{sinon. }
			\end{cases}
		\end{equation}
		Donc le support de \( x\) est dans \( J\) qui est fini. Vu que toute partie d'un ensemble fini est fini (lemme \ref{LEMooTUIRooEXjfDY}), le support de \( x\) est fini.
		\spitem[Si \( x\) est de support fini]
		Supposons que le support de \( x\colon I\to A\) soit la partie finie \( J\subset I\). En notant \( x_j=x(j)\) pour tout \( j\in J\), nous avons
		\begin{equation}
			x=\sum_{j\in J}x_je_j.
		\end{equation}
	\end{subproof}
\end{proof}

\begin{theorem}[Propriété universelle de \( A^{(I)}\)\cite{BIBooSTWWooItiMUp}]      \label{THOooPDZCooJnHbOd}
	Soient un anneau \( A\) ainsi qu'un \( A\)-module \( P\). Pour \( \phi\in\Hom_A(A^{(I)}, P)\), nous considérons
	\begin{equation}
		\begin{aligned}
			\phi|_I\colon I & \to P              \\
			i               & \mapsto \phi(e_i).
		\end{aligned}
	\end{equation}
	\begin{enumerate}
		\item

		      L'application
		      \begin{equation}
			      \begin{aligned}
				      f\colon \Hom_A(A^{(I)},P) & \to \Fun(I,P)   \\
				      \phi                      & \mapsto \phi|_I
			      \end{aligned}
		      \end{equation}
		      est une bijection.
		\item
		      L'application inverse est \( g\colon \Fun(I,P)\to \Hom_A(A^{(I)},P) \) donnée par
		      \begin{equation}
			      g(\psi)\big( \sum_{j\in J}a_je_j \big)=\sum_{j\in J}a_j\psi(j)
		      \end{equation}
		      pour tout \( J\) fini dans \( I\) et choix de \( a_j\in A\).
	\end{enumerate}
\end{theorem}

\begin{proof}
	Nous allons montrer que \( g\big( f(\phi) \big)=\phi\) et que \( f\big( g(\psi) \big)=\psi\) pour tout \( \phi\in\Hom_A(A^{(I)},P)\) et pour tout \( \psi\in \Fun(I,P)\).

	Dans un premier sens nous avons :
	\begin{subequations}
		\begin{align}
			g\big( f(\phi) \big)\big( \sum_ja_je_j \big) & =\sum_ja_jf(\phi)(j)                                       \\
			                                             & =\sum_ja_j\phi(e_j)\label{SUBALIGNooBWPLooHeIaQg}          \\
			                                             & =\phi(\sum_ja_je_j)        \label{SUBALIGNooUOQPooCwLgZo}.
		\end{align}
	\end{subequations}
	Justifications :
	\begin{itemize}
		\item
		      Pour \eqref{SUBALIGNooBWPLooHeIaQg}, nous avons utilisé le fait que \( f(\phi)(i)=\phi|_I(i)=\phi(e_i)\).
		\item
		      Pour \eqref{SUBALIGNooUOQPooCwLgZo}, nous utilisons le fait que \( \phi\) est un morphisme de modules.
	\end{itemize}
	Et pour l'autre sens,
	\begin{equation}
		f\big( g(\psi) \big)(i)=g(\psi)(e_i)=\psi(i).
	\end{equation}
	Vérifions que cela est suffisant pour que \( f\) soit une bijection.
	\begin{subproof}
		\spitem[Surjectif]
		Soit \( \psi\in \Fun(I,P)\). Nous avons \( f\big( g(\psi) \big)=\psi\), ce qui prouve que \( \psi\) est dans l'image de \( f\).
		\spitem[Injectif]
		Supposons que \( f(\phi_1)=f(\phi_2)\). Alors en appliquant \( g\) des deux côtés, il vient \( \phi_1=\phi_2\).
	\end{subproof}
\end{proof}

%--------------------------------------------------------------------------------------------------------------------------- 
\subsection{Sous-module}
%---------------------------------------------------------------------------------------------------------------------------

Soient \( M\) un \( A\)-module et \( x=(x_i)_{i\in I}\) une famille d'éléments de \( M\) paramétrée par l'ensemble \( I\). Nous considérons l'application
\begin{equation}
	\begin{aligned}
		\mu_x\colon A^{(I)} & \to M                        \\
		(a_i)_{i\in I}      & \mapsto \sum_{i\in I}a_ix_i.
	\end{aligned}
\end{equation}
Ici \( A^{(I)}\) désigne l'ensemble de toutes les applications \( I\to A\) de support fini (définition \ref{DEFooBMEPooFsCHgb}).

\begin{definition}      \label{DefBasePouyKj}
	À l'instar des espaces vectoriels, les modules ont une notion de partie libre, génératrice et de bases :
	\begin{enumerate}
		\item
		      Si \( \mu_x\) est surjective, nous disons que \( x\) est une partie \defe{génératrice}{génératrice!partie d'un module}.
		\item
		      Si \( \mu_x\) est injective, nous disons que la partie \( x\) est \defe{libre}{libre!partie d'un module}.
		\item
		      Si \( \mu_x\) est bijective, nous disons que la partie \( x\) est une \defe{base}{base!d'un module}.
	\end{enumerate}
\end{definition}

\begin{definition}
	Un sous-ensemble \( N\subset M\) est un \defe{sous-module}{sous-module} si \( (N,+)\) est un sous-groupe de \( (M,+)\) et si \( a\cdot x\in N\) pour tout \( x\in N\) et pour tout \( a\in A\).
\end{definition}

\begin{example}
	Un anneau \( A\) est lui-même un \( A\)-module et ses sous-modules sont les idéaux.
\end{example}

\begin{definition}
	Soit \( M\) un module sur un anneau commutatif \( A\). Un \defe{projecteur}{projecteur!dans un module} est une application linéaire \( p\colon M\to M\) telle que \( p^2=p\).

	Une famille \( (p_i)_{i\in I}\) sur \( M\) est \defe{orthogonale}{orthogonal!famille de projecteurs} si \( p_i\circ p_j=0\) pour tout \( i\neq j\). La famille est \defe{complète}{complète!famille de projecteurs} si \( \sum_{i\in I}p_i=\mtu\).
\end{definition}

\begin{theorem}     \label{ThoProjModpAlsUR}
	Soient des sous-modules \( M_1,\ldots,M_n\) du module \( M \) tels que \( M=M_1\oplus\ldots\oplus M_n\). Les applications \( p_i\) définies par
	\begin{equation}
		p_i(x_1+\ldots+x_n)=x_i
	\end{equation}
	forment une famille orthogonale de projecteurs et \( p_1+\cdots +p_n=\id\).

	Inversement, si \( (p_1,\ldots, p_n)\) est une famille orthogonale de projecteurs dans un module \( \modE\) tel que \( \sum_{i=1}^np_i=\id\), alors
	\begin{equation}
		M=\bigoplus_{i=1}^np_i(M).
	\end{equation}
\end{theorem}

\begin{definition}
	Un module est \defe{simple}{simple!module}\index{module!simple} ou \defe{irréductible}{irréductible!module}\index{module!irréductible} si il n'a pas d'autres sous-modules que \( \{ 0 \}\) et lui-même. Un module est \defe{indécomposable}{indécomposable!module}\index{module!indécomposable} si il ne peut pas être écrit comme somme directe de sous-modules.
\end{definition}

Un module simple est a fortiori indécomposable. L'inverse n'est pas vrai comme le montre l'exemple suivant.

\begin{example}
	Soit \( \modE=\eC[X]/(X^2)\) vu comme \( \eC[X]\)-module. C'est le \( \eC[X]\)-module des polynômes de la forme \( aX+b\) avec \( a,b\in \eC\). L'ensemble des polynômes de la forme \( aX\) est un sous-module. Le module \( \modE\) n'est donc pas simple. Il est cependant indécomposable parce que \( \{ aX \}\) est le seul sous-module non trivial. En effet si \( \modF\) est un sous-module de \( \modE\) contenant \( aX+b\) avec \( b\neq 0\), alors \( \modF\) contient \( X(aX+b)=bX\) et donc contient tout \( \modE\).
\end{example}

\begin{definition}[Algèbre\cite{ZSyHmiy}]   \label{DefAEbnJqI}
	Si \( \eK\) est un corps commutatif\footnote{Définition~\ref{DefTMNooKXHUd}}, une \( \eK\)-\defe{algèbre}{algèbre} \( A\) est un espace vectoriel\footnote{Définition~\ref{DEFooKHWZooIfxdNc}.} muni d'une opération bilinéaire \( \times\colon A\times A\to A\), c'est-à-dire telle que pour tout \( x,y,z\in A\) et pour tout \( \alpha,\beta\in\eK\),
	\begin{enumerate}
		\item
		      \( (x+y)\times z=x\times z+y\times z\)
		\item
		      \( x\times (y+z)=x\times y+x\times z\)
		\item
		      \( (\alpha x)\times (\beta y)=(\alpha\beta)(x\times y)\).
	\end{enumerate}
	Si \( A\) et \( B\) sont deux \( \eK\)-algèbres, une application \( f\colon A\to B\) est un \defe{morphisme d'algèbres}{morphisme!d'algèbres} entre \( A\) et \( B\) si pour tout \( x,y\in A\) et pour tout \( \alpha\in \eK\),
	\begin{enumerate}
		\item
		      \( f(xy)=f(x)f(y)\)
		\item
		      \( f(x+\alpha y)=f(x)+\alpha f(y)\)
	\end{enumerate}
	où nous avons noté \( xy\) pour \( x\times y\).
\end{definition}

\begin{lemma}[\cite{MonCerveau}]   \label{LEMooVKLKooSAHmpZ}
	Soient une algèbre \( A\) et une famille \( (X_i)_{i\in I}\) de sous-algèbres de \( A\) (ici \( I\) est un ensemble quelconque). Alors la partie \( X=\bigcap_{i\in I}X_i\) est une sous-algèbre de \( A\).
\end{lemma}

\begin{proof}
	Nous devons prouver que si \( x\) et \( y\) sont dans \( X\) et \( \lambda\in \eK\), alors \( xy\), \( x+y\) et \( \lambda x\) sont dans \( X\). Pour tout \( i\in I\) nous avons \( x,y\in X_i\) et donc \( xy\in X_i\), \( x+y\in X_i\) et \( \lambda x\in X_i\) (parce que \( X_i\) est une algèbre). Donc \( xy\), \( x+y\) et \( \lambda x\) sont dans \( X_i\) pour tout \( I\), et donc dans \( X\).
\end{proof}

\begin{definition}\label{DefkAXaWY}
	L'\defe{algèbre engendrée}{algèbre!engendrée} par \( X\) est l'intersection de toutes les sous-algèbres de \( A\) contenant \( X\) (qui est une algèbre par le lemme~\ref{LEMooVKLKooSAHmpZ}).
\end{definition}


%+++++++++++++++++++++++++++++++++++++++++++++++++++++++++++++++++++++++++++++++++++++++++++++++++++++++++++++++++++++++++++
\section{Caractéristique d'un anneau}
%+++++++++++++++++++++++++++++++++++++++++++++++++++++++++++++++++++++++++++++++++++++++++++++++++++++++++++++++++++++++++++


\begin{lemmaDef}        \label{LEMDEFooVEWZooUrPaDw}
	Soit l'application
	\begin{equation}
		\begin{aligned}
			\mu\colon \eZ & \to A              \\
			n             & \mapsto n\cdot 1_A
		\end{aligned}
	\end{equation}
	où \( n\cdot 1_A\) est expliqué dans \ref{NORMooROXXooXOybXN}.
	\begin{enumerate}
		\item	\label{ITEMooPBWYooAWaCKJ}
		      C'est un morphisme d'anneaux.
		\item	\label{ITEMooOQWMooFYcmFf}
		      Le noyau est un sous-groupe de \( \eZ\)
		\item	\label{ITEMooWDWEooMSMxfN}
		      Il existe un unique \( p\in \eZ\) tel que \( \ker(\mu)=p\eZ\).
	\end{enumerate}
	Ce \( p\) est la \defe{caractéristique}{caractéristique!d'un anneau} de \( A\).
\end{lemmaDef}

\begin{proof}
	En plusieurs parties.
	\begin{subproof}
		\spitem[Pour \ref{ITEMooPBWYooAWaCKJ}]
		%-----------------------------------------------------------
		Nous avons \( \mu(1)=1\) et \( \mu(0)=0\). De plus
		\begin{equation}
			\mu(n+m)=\sum_{i=1}^{n+m}1_A=\sum_{i=1}^n1_A+\sum_{i=n+1}^{n+m}1_A=\mu(n)+\sum_{i=1}^m1_A=\mu(n)+\mu(m).
		\end{equation}
		Notez le passage délicat de \( \sum_{i=n+1}^{n+m}1_A=\sum_{i=1}^m1_A\). Vu que ce qui est dans la somme sur \( i\) ne dépend pas de \( i\), seul le cardinal de l'ensemble parcouru par \( i\) compte.

		Nous avons donc \( \mu(m+n)=\mu(n)+\mu(m)\). Notez que c'est également valide pour les négatifs grâce aux différentes conventions prises.

		Le lemme \ref{LEMooILYLooTDRtYj} conclu que \( \mu\) est un morphisme.
		\spitem[Pour \ref{ITEMooOQWMooFYcmFf}]
		%-----------------------------------------------------------
		Nous avons \( \mu(0)=0\) et donc \( 0\in \ker(\mu)\). Si \( k,l\in \ker(\mu)\), alors
		\begin{equation}
			\mu(k+l)=\mu(k)+\mu(l)=0+0=0,
		\end{equation}
		donc \( k+l\in \ker(\mu)\).
		\spitem[Pour \ref{ITEMooWDWEooMSMxfN}]
		%-----------------------------------------------------------
		La proposition \ref{PROPooORABooXRbVoz} montre que tout sous-groupe de \( \eZ\) est de la forme \( p\eZ\).
	\end{subproof}
\end{proof}

Par exemple la caractéristique de \( \eQ\) est zéro parce qu'aucun multiple de l'unité n'est nul.

À propos de diagonalisation en caractéristique \( 2\), voir l'exemple~\ref{ExewINgYo}.

\begin{lemma}
	Si \( A\) est de caractéristique nulle, alors \( A\) est infini.
\end{lemma}

\begin{proof}
	En effet, \( \ker\mu=\{0\} \) implique que \( n1_A \neq  m1_A\) dès que \(n \neq m \) et par conséquent \( A\) contient \(\eZ 1_A \), et  est infini.
\end{proof}

\begin{lemma}       \label{LemHmDaYH}
	Soit un anneau \( A\) de caractéristique \( p\).
	\begin{enumerate}
		\item       \label{ITEMooPKSEooPKChGM}
		      Si \( p>0\), alors nous avons l'isomorphisme d'anneaux
		      \begin{equation}
			      \eZ 1_A\simeq\eZ/p\eZ.
		      \end{equation}
		\item        \label{ITEMooBTUIooYzOycc}
		      Si \( p=0\), alors nous avons l'isomorphisme d'anneaux
		      \begin{equation}
			      \eZ 1_A\simeq \eZ
		      \end{equation}
		      %TODOooZCJXooPsEHbI: démontrer cette seconde partie.
	\end{enumerate}
\end{lemma}

\begin{proof}
	Pour \ref{ITEMooPKSEooPKChGM}, l'isomorphisme est donné par l'application \( n1_A\mapsto \phi(n)\) si \( \phi\) est la projection canonique \( \eZ\to \eZ/p\eZ\).

	Pour \ref{ITEMooBTUIooYzOycc}, la preuve est encore à faire. Écrivez-moi.
\end{proof}

\begin{proposition}     \label{PropGExaUK}
	La caractéristique d'un anneau fini divise son cardinal.
\end{proposition}

\begin{proof}
	Si \( A\) est un anneau, le groupe \( \eZ\) agit sur \( A\) par
	\begin{equation}
		n\cdot a=a+n1_A.
	\end{equation}
	Chaque orbite de cette action est de la forme
	\begin{equation}
		\mO_a=\{ a+n1_A\tq n=0,\ldots, p-1 \}
	\end{equation}
	où \( p\) est la caractéristique de \( A\). Les orbites ont \( p\) éléments et forment une partition de \( A\), donc le cardinal de \( A\) est un multiple de \( p\).
\end{proof}

\begin{lemma}[\cite{ooIBWOooSjOvXd}]        \label{LEMooJQIKooQgukqn}
	Un anneau totalement ordonné est de caractéristique nulle.
\end{lemma}

\begin{proof}
	Le morphisme \( \mu\colon \eZ\to A\), \( n\mapsto n 1_A\) est strictement croissant, en particulier \( \mu(x)\neq \mu(y)\) dès que \( x\neq y\). Donc \( \ker(\mu)=\{ 0 \}\).
\end{proof}

L'ensemble typique de caractéristique \( p\) est \( \eF_p=\eZ/p\eZ\).


%-------------------------------------------------------
\subsection{Endomorphisme de Frobenius}
%----------------------------------------------------


\begin{proposition}[Endomorphisme de Frobenius\cite{BIBooXTCOooOBSlLQ}] \label{PropFrobHAMkTY}
	Soit \( A\) un anneau commutatif unitaire dont la caractéristique est un nombre premier \( p\). L'application
	\begin{equation}
		\begin{aligned}
			\Frob_A\colon A & \to A       \\
			x               & \mapsto x^p
		\end{aligned}
	\end{equation}
	est un automorphisme d'anneau unitaire.

	En particulier nous avons
	\begin{equation}
		(a+b)^p=a^p+b^p.
	\end{equation}

	Cet automorphisme est nommé \defe{morphisme de Frobenius}{morphisme!Frobenius}\index{Frobenius!morphisme}. Nous utiliserons aussi les itérés du morphisme de Frobenius : \( \Frob^k\colon x\mapsto x^{p^k}\).
\end{proposition}

\begin{proof}
	Nous utilisons la formule du binôme de la proposition~\ref{PropBinomFExOiL} :
	\begin{equation}
		\Frob_A(x,y)=(x+y)^p=\sum_{k=0}^p\binom{ p }{ k }x^{p-k}b^k.
	\end{equation}
	Nous savons par le lemme \ref{LEMDEFooVEWZooUrPaDw} que si \( \lambda\in p\eZ\) et \( x\in A\), alors \( \lambda x=0\) (attention aux notations, c'est expliqué dans \ref{NORMooROXXooXOybXN}). La proposition \ref{PROPooVPOYooNNugWU} dit que tous les termes de la somme (sauf le premier et le dernier) sont de la forme \( \lambda x\) avec \( \lambda\in p\eZ\) et \( x\in a\). Bref, tous ce coefficients sont divisibles par \( p\) et donc les éléments de \( A\) correspondants sont nuls :
	\begin{equation}
		\Frob_A(x,y)=x^p+y^p=\Frob_A(x)+\Frob_A(y)
	\end{equation}
	Le lemme \ref{LEMDEFooVEWZooUrPaDw} conclu que \( \Frob_A\) est un morphisme.
\end{proof}


\begin{example}
	Soit à factoriser \( X^p-1\) dans \( \eF_p\). Grâce au morphisme de Frobenius, nous avons immédiatement
	\begin{equation}
		X^p-1=(X-1)^p.
	\end{equation}
\end{example}


\begin{lemma}       \label{LemCaractIntergernbrcartpre}
	La caractéristique\footnote{Définition~\ref{LEMDEFooVEWZooUrPaDw}.} d'un anneau intègre est zéro ou un élément premier\footnote{Définition \ref{DEFooZCRQooWXRalw}.}.
\end{lemma}

\begin{proof}
	Si \( A\) est intègre, alors \( \eZ 1_A\) est a fortiori intègre. Notons \( p \) la caractéristique de \( A \). Si \( p = 0 \), la preuve est finie; supposons donc que \( p \neq 0 \). Alors, l'anneau \( \eZ/p\eZ\) est isomorphe à \( \eZ 1_A\), et est donc intègre. Or, la proposition~\ref{CorZnInternprem} dit que \( \eZ/p\eZ\) est intègre si et seulement si \( p\) est premier, ce qui conclut la preuve.
\end{proof}

\begin{example}
	Il existe des corps dont la caractéristique n'est pas égale au cardinal (contrairement à ce que laisserait penser l'exemple des \( \eZ/p\eZ\)). En effet les matrices \( n\times n\) inversibles sur \( \eF_{3}\) forment un corps qui n'est pas de cardinal trois alors que la caractéristique est \( 3\) :
	\begin{equation}
		\begin{pmatrix}
			1 &   \\
			  & 1
		\end{pmatrix}+\begin{pmatrix}
			1 &   \\
			  & 1
		\end{pmatrix}+\begin{pmatrix}
			1 &   \\
			  & 1
		\end{pmatrix}=0.
	\end{equation}
\end{example}


%--------------------------------------------------------------------------------------------------------------------------- 
\subsection{Caractéristique deux}
%---------------------------------------------------------------------------------------------------------------------------

Beaucoup de résultats demandent une caractéristique différente de deux. Qu'a donc de particulier la caractéristique deux ?

Si \( \eK\) est un corps de caractéristique \( 2\), alors l'égalité \( x=-x\) n'implique pas \( x=0\), puisque \( 2x=0\) est vérifiée pour tout \( x\). Cela se répercute sur un certain nombre de résultats. Par exemple, en caractéristique deux, une forme antisymétrique n'est pas toujours alternée: voir le lemme~\ref{LemHiHNey}.
