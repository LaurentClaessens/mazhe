% This is part of Mes notes de mathématique
% Copyright (c) 2011-2023
%   Laurent Claessens
% See the file fdl-1.3.txt for copying conditions.



%+++++++++++++++++++++++++++++++++++++++++++++++++++++++++++++++++++++++++++++++++++++++++++++++++++++++++++++++++++++++++++
\section{Idéal dans un anneau}
%+++++++++++++++++++++++++++++++++++++++++++++++++++++++++++++++++++++++++++++++++++++++++++++++++++++++++++++++++++++++++++

La définition d'un idéal dans un anneau est la définition~\ref{DefooQULAooREUIU}.

\begin{definition}[\cite{ooLKFGooTUrnhx}]  \label{DefTMNooKXHUd}
	Un \defe{corps}{corps} est un anneau\footnote{Définition \ref{DefHXJUooKoovob}.} \( (A, +,\times)\) dans lequel tout élément non nul est inversible pour l'opération \( \times\) (pour l'opération \( +\), tous les éléments sont inversibles parce que \( (A,+)\) est un groupe).
\end{definition}

\begin{normaltext}
	Dans le Frido, nous ne parlons que de corps commutatifs; nous ne le répéterons pas toujours.
\end{normaltext}

\begin{normaltext}
	Pour savoir ce qu'est un «\emph{ring}» ou «\emph{field}» en anglais, voir \ref{SECooPBZVooCVInFT}.
\end{normaltext}

\begin{definition}[Idéal engendré par un élément]  \label{DefSKTooOTauAR}
	Si \( p\) est un élément d'un anneau \( A\) alors nous notons \( (p)\)\nomenclature[A]{\( (p)\)}{idéal engendré par \( p\)}\index{engendré!idéal dans un anneau} l'idéal dans \( A\) \defe{engendré}{engendré} par \( p\), c'est-à-dire \( pA\).
\end{definition}

\begin{definition}  \label{DefAJVTPxb}
	Un sous-ensemble \( B\subset A\) d'un anneau est un \defe{sous anneau}{sous-anneau} si
	\begin{enumerate}
		\item
		      \( 1\in B\)
		\item
		      \( B\) est un sous-groupe pour l'addition
		\item
		      \( B\) est stable pour la multiplication.
	\end{enumerate}
\end{definition}

\begin{remark}
	Un idéal n'est pas toujours un anneau parce que l'identité pourrait manquer. Un idéal qui contient l'identité est l'anneau complet.
\end{remark}

\begin{lemma}       \label{LEMooQAYSooCYJXkC}
	L'ensemble \( 2\eZ\) est un idéal de \( \eZ\). On peut aussi le noter \( (2) \).
\end{lemma}

\begin{proposition}[Premier théorème d'isomorphisme pour les anneaux]   \label{PROPooJALPooHFIObB}
	Soient \( A\) et \( B\) des anneaux et un homomorphisme \( f\colon A\to B\). Nous considérons l'injection canonique \( j\colon f(A)\to B\) et la surjection canonique \( \phi\colon A\to A/\ker f\). Alors il existe un unique isomorphisme
	\begin{equation}
		\tilde f \colon A/\ker f\to f(A)
	\end{equation}
	tel que \( f=j\circ\tilde f\circ\phi\).

	\begin{equation}
		\xymatrix{%
			A \ar[r]^{f}\ar[d]_{\phi}       &   B                         \\
			A/\ker f    \ar[r]_{\tilde f}   &   f(A)\subset B \ar[u]^{j}
		}
	\end{equation}
\end{proposition}
\index{théorème!isomorphisme!premier!pour les anneaux}

\begin{proposition}     \label{PropIJJIdsousphi}
	Soient \( I\) un idéal de \( A\) et la projection canonique
	\begin{equation}
		\phi\colon A\to A/I.
	\end{equation}
	C'est une bijection entre les idéaux de \( A\) contenant \( I\) et les idéaux de \( A/I\).

	Dit de façon imagée :
	\begin{equation}        \label{EqKbrizu}
		\{ \text{idéaux de } A\text{ contenant } I\}\simeq\{ \text{idéaux de } A/I \}.
	\end{equation}
\end{proposition}

\begin{proof}
	Si \( I\subset J\) et si \( J \) est un idéal de \( A\), alors \( \phi(J)\) est un idéal dans \( A/I\). En effet un élément de \( \phi(J)\) est de la forme \( \phi(j)\) et un élément de \( A/I\) est de la forme \( \phi(i)\). Leur produit vaut
	\begin{equation}
		\phi(i)\phi(j)=\phi(ij)\in\phi(J).
	\end{equation}

	Soit maintenant \( K\) un idéal dans \( A/I\) et soit \( J=\phi^{-1}(K)\). Étant donné qu'un idéal doit contenir \( 0\) (parce qu'un idéal est un groupe pour l'addition), \( [0]\in K\) et par conséquent \( I\subset\phi^{-1}(K)\).
\end{proof}

\begin{proposition}[\cite{MonCerveau}]     \label{AnnCorpsIdeal}\label{PROPooUOCVooZGAVVk}
	Si \( A\) est un anneau, nous avons les équivalences
	\begin{enumerate}
		\item       \label{ITEMooLAAVooXhTcMe}
		      \( A\) est un corps\footnote{Définition \ref{DefTMNooKXHUd}.}.
		\item       \label{ITEMooDGZIooRopYGx}
		      \( A\) est non nul et ses seuls\footnote{Je vous laisse vous poser de grandes questions sur le fait que le vide est un idéal ou non.} idéaux à gauche sont \( \{ 0 \}\) et \( A\).
		\item       \label{ITEMooLPWHooDJpTbR}
		      \( A\) est non nul et ses seuls idéaux à droite sont \( \{ 0 \}\) et \( A\).
	\end{enumerate}
\end{proposition}

\begin{proof}
	Nous allons montrer que le point \ref{ITEMooLAAVooXhTcMe} est équivalent aux deux autres.
	\begin{subproof}
		\spitem[\ref{ITEMooLAAVooXhTcMe} implique \ref{ITEMooDGZIooRopYGx}]
		Si \( I\) est un idéal à gauche différent de \( \{ 0 \}\), alors il contient un certain \( a\neq 0\). Puisque \( A\) est un corps, il contient un inverse \( a^{-1}\), et comme \( I\) est un idéal, \( a^{-1} I\subset I\). En particulier \( a^{-1}a\in I\). Donc \( 1\in I\) et \( I=A\).
		\spitem[\ref{ITEMooDGZIooRopYGx} implique \ref{ITEMooLAAVooXhTcMe}]
		Supposons que les seuls idéaux de \( A\) soient \( \{ 0 \}\) et \( A\). Soit \( a\in A\). Si \( a\) est non nul, alors \( aA\) est un idéal de \( a\). Vu qu'il contient \( a\neq 0\), nous avons \( aA=A\) (par hypothèse, un idéal qui n'est pas \( \{ 1 \}\) est \( A\)). En particulier, \( 1\in aA\), c'est-à-dire qu'il existe \( b\in A\) tel que \( ab=1\). L'élément \( a\) est donc inversible.
		\spitem[\ref{ITEMooLAAVooXhTcMe} implique \ref{ITEMooLPWHooDJpTbR}]
		% -------------------------------------------------------------------------------------------- 
		Comme pour \ref{ITEMooLAAVooXhTcMe} implique \ref{ITEMooDGZIooRopYGx}.
		\spitem[\ref{ITEMooLPWHooDJpTbR} implique \ref{ITEMooLAAVooXhTcMe}]
		% -------------------------------------------------------------------------------------------- 
		Comme pour \ref{ITEMooDGZIooRopYGx} implique \ref{ITEMooLAAVooXhTcMe}.
	\end{subproof}
	Notez que je n'ai pas vérifié les deux derniers points. Donc vous devriez le vérifier et m'écrire si il y a un problème.
\end{proof}

\begin{definition}[\cite{ooWEUDooQybvIx}]      \label{DEFIdealMax}
	Soit un anneau \( A\). Un idéal \( I\neq A\) est dit \defe{idéal maximal}{idéal maximal} si il n'existe pas d'idéal \( J\neq A\) contenant strictement \( I\).
\end{definition}

\begin{proposition}[Thème~\ref{THEMEooZYKFooQXhiPD}]     \label{PROPooSHHWooCyZPPw}
	Un idéal \( I\) dans un anneau \( A \) est maximum si et seulement si \( A/I\) est un corps.
\end{proposition}

\begin{proof}
	Soit un idéal maximum \( I\subset A\). Alors les idéaux contenant \( I\) sont \( A\) et \( I\). L'application \( \phi\) de la proposition~\ref{PropIJJIdsousphi} est une bijection, donc l'ensemble des idéaux de \( A/I\) ne contient que deux éléments. Les seuls idéaux de \( A/I\) sont donc \( \{ 0 \}\) et \( A/I\); donc \( A/I\) est un corps par la proposition~\ref{PROPooUOCVooZGAVVk}.

	Dans l'autre sens, c'est la même chose : si \( A/I\) est un corps, il possède exactement deux idéaux, donc \( A\) ne contient que deux idéaux contenant \( I\). Donc \( I\) est un idéal maximum.
\end{proof}

\begin{theorem}[Théorème de Krull\cite{BIBooAVBIooUFSvVv}]      \label{THOooFWYLooOofaPa}
	Pour tout idéal propre \( I\) d'un anneau commutatif \( A\), il existe au moins un idéal maximal de \( A\) contenant \( I\).
\end{theorem}


%--------------------------------------------------------------------------------------------------------------------------- 
\subsection{Élément premier}
%---------------------------------------------------------------------------------------------------------------------------

\begin{definition}[\cite{ooWBLYooLYwALS}]       \label{DEFooZCRQooWXRalw}
	Soit un anneau commutatif \( A\). Un élément \( p\in A\) est \defe{premier}{élément premier} si il est
	\begin{enumerate}
		\item
		      non nul,
		\item
		      non inversible,
		\item       \label{ITEMooPMTTooCVHPIm}
		      si \( p\) divise un produit \( ab\), alors il divise soit \( a\) soit \( b\) (ou le deux).
	\end{enumerate}
\end{definition}

%---------------------------------------------------------------------------------------------------------------------------
\subsection{Division euclidienne}
%---------------------------------------------------------------------------------------------------------------------------

\begin{theoremDef}[Division euclidienne\cite{ooRBKHooQJqglH}]     \label{ThoDivisEuclide}
	Soient \( a\in\eZ\) et \( b\in\eN^*\). Il existe un unique couple \( (q,r)\in\eZ\times\eN\), avec \( 0\leq r<b\), tel que
	\begin{equation}
		a=bq+r.
	\end{equation}
	L'opération \( (a,b)\mapsto(q,r)\) ainsi définie est la \defe{division euclidienne}{division!euclidienne}. Le nombre \( q\) est le \defe{quotient}{quotient} et \( r\) est le \defe{reste}{reste} de la division de \( a\) par \( b\).
\end{theoremDef}

\begin{proof}
	Remarquons que \( r = a - bq \), et donc, une fois l'existence et l'unicité de \( q\) établie, celle de \( r\) suivra.

	\begin{subproof}
		\spitem[Unicité]
		Nous supposons avoir \( (q,r)\in \eZ\times \eN\) tels que
		\begin{subequations}
			\begin{numcases}{}
				0\leq r<b\\
				a=qb+r.
			\end{numcases}
		\end{subequations}
		Ce système implique que
		\begin{equation}
			0\leq a-qb<b.
		\end{equation}
		En ajoutant \( qb\) dans les trois membres de cette inégalité,
		\begin{equation}
			qb\leq a<(q+1)b.
		\end{equation}
		Cela implique que
		\begin{equation}
			q=\max\{ k\in \eZ\tq kb\leq a \}.
		\end{equation}
		Donc \( q\) est unique et la relation \( a=bq+r\) implique que \( r\) est également unique.

		Soit
		\begin{equation*}
			E = \{ q \in \eZ  | bq \leq a \}.
		\end{equation*}
		La partie \( E\) est non vide (parce qu'elle contient \( -|a| \)) et admet un majorant : l'élément \( |a| \).  Elle admet donc un maximum \( q\) par le lemme \ref{LEMooMYEIooNFwNVI}. Ce maximum vérifie
		\begin{equation}
			bq\leq a<b(q+1).
		\end{equation}
		Cela donne \( 0\leq a-bq<b\) et le résultat, en posant \( r=a-qb\).
	\end{subproof}
\end{proof}


% TODO : À propos de restes, il n'est peut-être pas mal de parler d'algorithme de calcul de la date de pâques.
% L'algorithme de Gauss, Meeus utilise des arrondis.
% http://fr.wikipedia.org/wiki/Calcul_de_la_date_de_Pâques

Le lemme suivant est souvent pris pour la définition d'un nombre premier lorsqu'on parle de \( \eN\) ou \( \eZ\).
\begin{lemma}[\cite{frwiki179832418, MonCerveau}]
	Dans \( \eN\), un nombre est premier si et seulement si il admet exactement deux diviseurs entiers distincts.
\end{lemma}

\begin{proof}
	En deux parties.
	\begin{subproof}
		\spitem[\( \Rightarrow\)]
		Soit un élément premier \( p\in \eN\). Il y a trois possibilités : \( p=0\), \( p=1\) et \( p>1\).

		Le nombre \( p=0\) n'est pas premier parce qu'il est nul. Le nombre \( p=1\) n'est pas premier parce qu'il est inversible. Donc nous savons que si \( p\) est premier, alors \( p>1\).

		Un élément \( p>1\) dans \( \eN\) a toujours au moins deux diviseurs distincts : \( 1\) et \( p\). Soit un diviseur \( k\) de \( p\). Il existe \( l\in \eN\) tel que \( p=kl\). Vu que \( p\) est premier et divise le produit \( kl\), il divise \( k\) ou \( l\). Disons que \( p\) divise \( k\). De cette façon \( p\) divise \( k\) et \( k\) divise \( p\).

		Il existe donc \( n\in \eN\) tel que \( k=np\). En y substituant \( p=kl\), on trouve \( k=np=nkl\). En simplifiant par \( k\), il vient
		\begin{equation}
			1=nl,
		\end{equation}
		ce qui prouve que \( n=l=1\) et donc que \( k=p\) et donc que \( p\) n'a pas d'autres diviseurs que \( 1\) et \( p\).

		\spitem[\( \Leftarrow\)]
		Nous supposons que \( p\in \eN\) ait exactement deux diviseurs entiers distincts. Nous vérifions que \( p\) vérifie les trois conditions de la définition \ref{DEFooZCRQooWXRalw}.

		\begin{enumerate}
			\item
			      \( p\neq 0\) parce que \( 0\) a nettement plus que deux diviseurs distincts.
			\item
			      \( p\neq 1\) parce que \( 1\) a exactement un diviseur. Donc \( p\) n'est pas inversible dans \( \eN\).
			\item
			      Soit \( p\) admettant exactement deux diviseurs distincts. Soit \( p\) divisant le produit \( ab'\) pour certains \( a\) et \( b'\) dans \( \eN\). Nous supposons que \( p\) ne divise pas \( a\), et nous allons prouver que \( p\) divise \( b'\) en supposant d'abord que \( p\) ne divise pas \( b'\).

			      \begin{subproof}
				      \spitem[Un ensemble]
				      Pour cela nous posons
				      \begin{equation}
					      E=\{ x\in \eN\tq p\divides ax, p\notdivides x  \}.
				      \end{equation}
				      Nous posons \( b=\min(E)\). Nous avons pour hypothèse que \( E\) est non vide; en particulier \( 0<b\).
				      \spitem[\( b<p\)]
				      On vérifie que si \( p+k\in E\) alors \( k\in E\). Donc \( b\) ne peut pas être plus grand que \( p\). Vu que \( p\) lui-même n'est pas dans \( E\), nous avons \( b<p\).
				      \spitem[Division euclidienne]
				      Nous effectuons la division euclidienne du théorème \ref{ThoDivisEuclide} :
				      \begin{equation}
					      p=mb+r.
				      \end{equation}
				      En multipliant par \( a\), \( ar=ap-mab\). Vu que \( ab\) est un multiple de \( p\) \( ap-mab\) est un multiple de \( p\). En particulier \( ar\) est divisible en \( p\).
				      \spitem[La contradiction]
				      Nous avons donc \( r\in E\), alors que \( r<b\). Impossible.
			      \end{subproof}
		\end{enumerate}
	\end{subproof}
\end{proof}

\begin{proposition}[\cite{ooTGPAooQTbamu}]     \label{PROPooWMNPooZdvOBt}
	Dans un anneau intègre\footnote{Si pas intègre, voir l'exemple \ref{EXooEIUEooCZCPMC}.} tout élément premier est irréductible\footnote{Toutes les définitions dans le thème \ref{THEMEooVIQIooOcFAQS}.}.
\end{proposition}

\begin{proof}
	Soit \( p\), un élément premier dans un anneau intègre \( A\).
	\begin{subproof}
		\spitem[\( p\) n'est pas inversible]
		Cela fait partie de la définition d'un élément premier.
		\spitem[\( p\) n'est pas un produit d'inversibles]
		Soient \( a,b\in A\) tels que \( p=ab\). Par le point \ref{ITEMooPMTTooCVHPIm} de la définition \ref{DEFooZCRQooWXRalw}, \( p\) divise soit \( a\) soit \( b\). Supposons que \( p\) divise \( a\). Alors il existe \( x\in A\) tel que \( a=px\). En remettant dans \( p=ab\) nous avons :
		\begin{equation}        \label{EQooPYBGooLFHMJZ}
			p=pxb.
		\end{equation}
		Mais l'anneau est intègre et permet donc des simplifications par tout élément non nul. La relation \ref{EQooPYBGooLFHMJZ} donne donc
		\begin{equation}
			1=xb,
		\end{equation}
		ce qui signifie que \( b\) est inversible.

		Un travail similaire montre que \( a\) est inversible si \( p\) divise \( b\).
	\end{subproof}
\end{proof}

\begin{example}
	Si nous avons l'égalité \( 7=ab\) dans \( \eZ\), alors soit \( a\) soit \( b\) vaut \( 1\). Mettons \( a=1\). Dans ce cas, \( b=7\) et n'est donc pas inversible.
\end{example}

Sur un anneau non intègre, la notion d'élément premier n'est pas aussi intéressante que sur un anneau intègre. Par exemple la proposition \ref{PROPooWMNPooZdvOBt} devient fausse.

\begin{example}     \label{EXooEIUEooCZCPMC}
	Soit l'anneau \( \eZ^2\). L'élément \( (1,0)\) est premier mais pas irréductible.
	\begin{subproof}
		\spitem[\( (1,0)\) est premier]
		L'élément \( (1,0)\) est non nul; ça c'est pas cher. Pour qu'il soit inversible, il faudrait \( (1,0)(x,y)=(1,1)\). Entre autres, \( 0\times y=1\), ce qui est impossible. Donc il n'est pas inversible.

		Supposons que \( (1,0)\) divise le produit \( (a,b)(c,d)=(ac,bd)\). Alors il existe \( (x,y)\) tel que \( (1,0)(x,y)=(ac,bd)\). Cela signifie que \( x=ac\) et \( 0\times y=bd\). En particulier, soit \( b=0\) soit \( d=0\). Si \( b=0\), nous avons \( (a,b)=(a,0)\) et effectivement, \( (1,0)\) le divise.
		\spitem[\( (1,0)\) n'est pas irréductible]
		Nous avons \( (1,0)=(1,0)(1,0)\). Donc l'élément \( (1,0)\) est le produit de deux éléments non inversibles.
	\end{subproof}
\end{example}



%---------------------------------------------------------------------------------------------------------------------------
\subsection{Élément irréductible}
%---------------------------------------------------------------------------------------------------------------------------

\begin{definition}[Élément irréductible\cite{ooWUNIooXKxRya}]  \label{DeirredBDhQfA}
	Un élément d'un anneau commutatif est \defe{irréductible}{irréductible!dans un anneau} si il n'est ni inversible, ni le produit de deux éléments non inversibles. \index{polynôme irréductible}
\end{definition}

\begin{normaltext}
	Nous allons voir dans la section \ref{SECooSWGKooEeOZTO} que le concept d'élément irréductible n'est vraiment utile que dans le cas des anneaux intègres.
\end{normaltext}

\begin{example}
	Un corps n'a pas d'élément irréductible parce qu'à part zéro, tous les éléments sont inversibles. Mais \( 0\) n'est pas irréductible parce qu'il peut être écrit comme produit d'éléments non inversibles : \( 0=0\cdot 0\).
\end{example}

\begin{proposition}     \label{PROPooKDWQooTtScrN}
	Les éléments irréductibles de l'anneau \( \eZ\) sont les nombres premiers\footnote{Nombre premier, définition \ref{DEFooZCRQooWXRalw}.}.
\end{proposition}

\begin{proof}
	Les seuls inversibles de \( \eZ\) sont \( \pm 1\).

	Si \( p\) est premier et \( p=ab\) avec \( a,b\in \eZ\), alors nous avons soit \( a=\pm 1\) soit \( b=\pm 1\). Donc \( p\) n'est pas le produit de deux éléments non inversibles.

	Dans le sens inverse, supposons que \( p\) soit irréductible dans \( \eZ\). D'abord \( p\) ne peut pas être \( \pm 1\) parce que \( \pm 1\) sont inversibles. Ensuite supposons que \( p=ab\). Vu que \( p\) est irréductible, nous avons \( a=\pm1\) ou \( b=\pm1\). Autrement dit, dans \( p=ab\), soit \( a\) soit \( b\) est un inversible.
\end{proof}




%+++++++++++++++++++++++++++++++++++++++++++++++++++++++++++++++++++++++++++++++++++++++++++++++++++++++++++++++++++++++++++
\section{Anneau principal et idéal premier}
%+++++++++++++++++++++++++++++++++++++++++++++++++++++++++++++++++++++++++++++++++++++++++++++++++++++++++++++++++++++++++++

\begin{definition}      \label{DEFooMZRKooBPLAWH}
	Un idéal \( I\) dans \( A\) est \defe{principal à gauche}{idéal!principal!à gauche} si il existe \( a\in I\) tel que \( I= A a\). Il est \defe{principal à droite}{idéal!principal!à droite} si il existe \( a\in I\) tel que \( I=a A\). Nous disons qu'il est \defe{principal}{principal!idéal} si il est principal à gauche et à droite.
\end{definition}

\begin{definition}          \label{DEFooGWOZooXzUlhK}
	Un anneau est \defe{principal}{principal!anneau} si
	\begin{enumerate}
		\item
		      il est commutatif et intègre
		\item
		      tous ses idéaux sont principaux.
	\end{enumerate}
\end{definition}

Souvent pour prouver qu'un anneau est principal, nous prouvons qu'il est euclidien (définition~\ref{DefAXitWRL}) et nous utilisons la proposition~\ref{Propkllxnv} qui dit qu'un anneau euclidien est principal.

Une manière de prouver qu'un anneau n'est pas principal est de prouver qu'il n'est pas factoriel, théorème~\ref{THOooANCAooBChmwp}.

\begin{definition}      \label{DEFooAQSZooVhvQWv}
	Nous disons qu'un idéal \( I\) dans \( A\) est \defe{premier}{premier!idéal} si \( I\) est strictement inclus dans \( A\) et si pour tout \( a,b\in A\) tels que \( ab\in I\) nous avons \( a\in I\) ou \( b\in I\).
\end{definition}

\begin{lemma}       \label{LEMooYRPBooYxXXsi}
	L'idéal nul (réduit à \( \{ 0 \}\)) est premier si et seulement si \( A\) est intègre.
\end{lemma}

\begin{proof}
	En deux sens.
	\begin{subproof}
		\spitem[Si \( \{ 0 \}\) est premier]
		Alors \( A\neq \{ 0 \}\) parce que \( I=\{ 0 \}\) est propre (définition d'idéal premier).

		De plus, si \( ab=0\), alors \( ab\in I\) qui est un idéal premier. Donc soit \( a\) soit \( b\) est dans \( I\), c'est-à-dire que soit \( a\) soit \( b\) est nul. Donc \( A\) est intègre.

		\spitem[Si \( A\) est intègre]
		Alors \( A\neq \{ 0 \}\) et l'idéal \( I=\{ 0 \}\) est strictement inclus dans \( A\). Si \( ab\in I\), alors \( ab=0\) et comme \( A\) est intègre, soit \( a\) soit \( b\) est nul, c'est-à-dire appartient à \( I\).
	\end{subproof}
\end{proof}


\begin{lemma}       \label{LemAnnCorpsnonInterdivzer}
	Un corps non nul est un anneau intègre\footnote{Définition \ref{DEFooTAOPooWDPYmd}.}.
\end{lemma}

\begin{proof}
	Soit un produit nul \( ab=0\). Si \( a\neq 0\), alors il est inversible et nous multiplions \( ab=0\) par \( a^{-1}\). Nous trouvons \( b=0\) parce que \( 0a^{-1}=0\).
\end{proof}
Conséquence : dans un corps nous avons toujours la règle du produit nul, et l'élément nul n'est jamais inversible.

\begin{proposition}[\cite{ooWEUDooQybvIx}]      \label{PROPooRUQKooIfbnQX}
	Soit un anneau commutatif\footnote{Tous les anneaux du Frido sont commutatifs} et un idéal \( I\) dans \( A\).
	\begin{enumerate}
		\item       \label{ITEMooUGBTooOGrnWl}
		      \( I\) est un idéal premier si et seulement si \( A/I\) est un anneau intègre.
		\item   \label{ITEMooGLXSooUjINqR}
		      \( I\) est un idéal maximal si et seulement si \( A/I\) est un corps.
		\item       \label{ITEMooTFFQooOUajFw}
		      Tout idéal maximal propre est premier.
	\end{enumerate}
\end{proposition}

\begin{proof}
	En plein d'étapes.
	\begin{subproof}
		\spitem[\ref{ITEMooUGBTooOGrnWl}, \( \Rightarrow\)]
		Évacuons le cas trivial pour être sûr. Si \( I=\{ 0 \}\) alors \( A\) est intègre par le lemme \ref{LEMooYRPBooYxXXsi}. Donc \( A/I=A/\{ 0 \}=A\) est intègre également.

		Soient \( a,b\in A\) tels que \( [a][b]=[0]\). Donc \( [ab]=[0]\), c'est-à-dire \( ab\in I\). Puisque \( I\) est un idéal premier nous avons \( a\in I\) ou \( b\in I\), c'est-à-dire \( [a]=0\) ou \( [b]=0\); nous en déduisons que \( A/I\) est un anneau intègre.
		\spitem[\ref{ITEMooUGBTooOGrnWl}, \( \Leftarrow\)]

		Soit \( ab\in I\). Alors \( [ab]=0\), ce qui signifie que \( [a][b]=0\) donc que \( [a]=0\) ou que \( [b]=0\) parce que \( A/I\) est intègre. Mais la condition \( [a]=0\) signifie \( a\in I\), et \( [b]=0\) signifie \( b\in I\). Nous avons donc prouvé que soit \( a\) soit \( b\) est dans \( I\), c'est-à-dire que \( I\) est premier.
		\spitem[\ref{ITEMooGLXSooUjINqR}, \( \Rightarrow\)]

		Nous devons montrer que tout élément non nul de \( A/I\) est inversible. Un élément non nul de \( A/I\) est \( [x]\) avec \( x\in A\setminus I\).

		Nous considérons \( J=Ax+I\), qui est un idéal parce que pour tout \( a\in A\), \( aAx+aI\in Ax+I\). Mais comme \( I\) est maximal, \( J=I\) ou \( J=A\).

		Si \( J=I\), nous aurions que pour tout \( a\in A\) et pour tout \( i\in I\), \( ax+i\in I\). En particulier pour \( a=1\) et \( i=0\) nous aurions \( x\in I\), ce qui est contraire à l'hypothèse faite sur \( x\).

		Donc \( J=A\). En particulier, \( 1\in J\), c'est-à-dire qu'il existe \( a\in A\) et \( i\in I\) tels que \( ax+i=1\). En passant aux classes, \( [ax]=1\), c'est-à-dire \( [a][x]=1\) qui signifie que \( [a]\) est un inverse de \( [x]\) dans \( A/I\).

		Nous avons prouvé que \( A/I\) est un corps.

		\spitem[\ref{ITEMooGLXSooUjINqR}, \( \Leftarrow\)]
		Si \( x\in A\setminus I\), il faut prouver que tout idéal contenant \( I\) et \( x\) est \( A\).

		Un idéal contenant \( I\) et \( x\) doit contenir l'idéal \( J=Ax+I\). Comme \( x\notin I\), nous avons \( [x]\neq 0\) dans \( A/I\). Donc \( [x] \) est inversible et il existe \( a\in A\) tel que \( [ax]=[1]\). C'est-à-dire que \( ax-1\in I\). Nous avons alors
		\begin{equation}
			1=ax+\underbrace{(1-ax)}_{\in I}.
		\end{equation}
		C'est-à-dire que \( 1\in Ax+I\) et donc \( Ax+I=A\).
	\end{subproof}
	Enfin nous prouvons que tout idéal maximal propre est premier.

	Si \( I\) est maximal, \( A/I\) est un corps par le point \ref{ITEMooGLXSooUjINqR}, et vu que \( I\) est propre, le corps \( A/I\) n'est pas réduit à \( \{ 0 \}\). Donc le lemme \ref{LemAnnCorpsnonInterdivzer} dit que \( A/I\) est un anneau intègre. Le point \ref{ITEMooUGBTooOGrnWl} dit alors que \( I\) est un idéal premier.
\end{proof}

\begin{remark}
	Puisqu'un corps peut être réduit à \( \{0\}\), dans \ref{ITEMooGLXSooUjINqR}, l'idéal peut être \( A\). Mais pas dans \ref{ITEMooTFFQooOUajFw}, parce qu'un idéal premier est propre, ça fait partie de la définition \ref{DEFooAQSZooVhvQWv}.
\end{remark}

\begin{proposition}[\cite{ooOYKZooOJBDHS}]     \label{PROPooHABIooBZZQMj}
	Si \( A\) est un anneau commutatif intègre, alors un idéal \( I\) dans \( A\) est premier si et seulement si \( A/I\) est intègre.
\end{proposition}

\begin{proof}
	Supposons que \( I\) soit un idéal premier. Si \( [a],[b] \in A/I\)  sont tels que \( [a][b]=0\), alors \( [ab]=0\), ce qui signifie que \( ab\in I\). Mais alors, puisque \( I\) est premier, soit \( a\) soit \( b\) est dans \( I\). Cela signifie que soit \( [a]\) soit \( [b]\) est nul dans \( A/I\). Cela prouve que \( A/I\) est un anneau intègre.

	Dans l'autre sens, nous supposons que \( A/I\) est intègre. Cela implique immédiatement que \( I\neq A\) parce que \( A/A\) n'est pas un anneau intègre (tout le monde est évidemment diviseur de zéro).

	Soient donc \( a,b\in A\) tels que \( ab\in I\). Alors \( [a][b]=[ab]=0\) dans \( A/I\), mais comme \( A/I\) est intègre, cela implique que soit \( [a]\) soit \( [b]\) est nul. Autrement dit, soit \( a\) soit \( b\) est dans \( I\).
\end{proof}


%+++++++++++++++++++++++++++++++++++++++++++++++++++++++++++++++++++++++++++++++++++++++++++++++++++++++++++++++++++++++++++
\section{Anneau intègre}
%+++++++++++++++++++++++++++++++++++++++++++++++++++++++++++++++++++++++++++++++++++++++++++++++++++++++++++++++++++++++++++
\label{SECAnneauxIntegres}

%---------------------------------------------------------------------------------------------------------------------------
\subsection{Sous-groupes de \texorpdfstring{\( (\eZ,+)\)}{(Z,+)}}
%---------------------------------------------------------------------------------------------------------------------------

\begin{proposition}[liste des sous groupes de \( \eZ\)] \label{PropSsgpZestnZ}
	À propos de sous-groupes de \( \eZ\).
	\begin{enumerate}
		\item
		      Une partie \( H\) du groupe \( (\eZ,+)\) est un sous-groupe si et seulement si il existe \( n\in\eN\) tel que \( H=n\eZ\).
		\item       \label{ITEMooOWNZooUsYRok}
		      Si \( H\) est une sous-groupe de \( (\eZ,+)\), il existe un unique \( n\) tel que \( H=n\eZ\).
	\end{enumerate}
\end{proposition}

\begin{proof}
	Soit \( H\neq\{ 0 \}\) un sous-groupe de \( \eZ\). L'ensemble \( H\cap\eN^*\) contient un élément minimum que nous notons \( n\). Nous avons certainement \( n\eZ\subset H\) parce que \( H\) est un groupe (donc \( n+n\) et \( -n\) sont dans \( H\) dès que \( n\) est dans \( H\)). Nous devons prouver que \( H\subset n\eZ\).

	Si \( x\in H\), par le théorème de division euclidienne~\ref{ThoDivisEuclide}, il existe \( q\in\eZ\) et \( r\in\eN \), uniques, tels que \( x=nq+r\) et \(0 \leq r < n \). Nous savons déjà que \( nq\in H\), donc \( r = x - nq \in H \). Le nombre \( r\) est donc un élément de \( H\) strictement plus petit que \( n\). Mais nous avions décidé que \( n\) serait le plus petit élément de \( H\cap\eN^*\). Par conséquent \( r=0\) et \( x=nq\in n\eZ\).


	En ce qui concerne l'unicité, supposons que \( n\eZ=m\eZ\). Le nombre \( n\) divise \( m\) (parce que \( m\in m\eZ\subset n\eZ\)) et le nombre \( m\) divise \( n\) parce que \( n\in m\eZ\). Par conséquent \( n=m\).
\end{proof}



%--------------------------------------------------------------------------------------------------------------------------- 
\subsection{Théorème de Bézout}
%---------------------------------------------------------------------------------------------------------------------------


\begin{theorem}[Théorème de Bézout\footnote{Il y a une super application ici : \url{https://perso.univ-rennes1.fr/matthieu.romagny/agreg/dvt/mauvais_prix.pdf}.}\cite{LSAmvR}, thème~\ref{THEMEooNRZHooYuuHyt}] \label{ThoBuNjam}
	Deux entiers non nuls \( a,b\in\eZ^*\) sont premiers entre eux si et seulement si il existe \( u,v\in\eZ\) tels que
	\begin{equation}
		au+bv=1
	\end{equation}
\end{theorem}
\index{Bézout!nombres entiers}

\begin{proof}
	Soit \( d=\pgcd(a,b)\) et des nombres \( u,v\) tels que \( au+bv=1\). Le PGCD \( d\) divise à la fois \( a\) et \( b\), et donc divise \( au+bv\). Nous en déduisons que \( d\) divise \( 1\) et est par conséquent égal à \( 1\).

	Nous supposons maintenant que \( \pgcd(a,b)=1\) et nous considérons l'ensemble
	\begin{equation}
		E=\{ au+bv\tq u,v\in \eZ \}\cap \eN^*.
	\end{equation}
	C'est-à-dire l'ensemble des nombres strictement positifs pouvant s'écrire sous la forme \( au+bv\). Cet ensemble est non vide parce qu'il contient par exemple soit \( a\) soit \( -a\). Soit \( m\) le plus petit élément de \( E\) et écrivons
	\begin{equation}    \label{EqMBsfrP}
		m=au_1+bv_1.
	\end{equation}
	Par le théorème de division euclidienne\footnote{Théorème~\ref{ThoDivisEuclide}.} (avec \( a\) et \( m\)), il existe des entiers uniques \( q\) et \( r\) tels que
	\begin{equation}
		a=mq+r
	\end{equation}
	avec \( 0\leq r<m\). En remplaçant \( m\) par sa valeur \eqref{EqMBsfrP}, \( a=(au_1+bv_1)q+r\) et
	\begin{equation}
		r=a(1-u_1q)-bv_1q,
	\end{equation}
	c'est-à-dire que \( r\in \eZ a+\eZ b\) en même temps que \( 0\leq r<m\). Si \( r\) était strictement positif, il serait dans \( E\). Mais cela est impossible par minimalité de \( m\). Donc \( r=0\) et \( a\) est divisible par \( m\).

	De la même façon nous prouvons que \( b\) est divisible par \( m\). Puisque \( m\) divise à la fois \( a\) et \( b\) nous avons \( m=1\).
\end{proof}

Une généralisation de Bézout \ref{ThoBuNjam} à plus de \( 2\) variables.
\begin{proposition}     \label{PROPooWSMTooMdfqse}
	Si \( \{ a_i \}_{i=1,\ldots, N}\) sont des entiers tels que \( \pgcd(a_1,\ldots, a_N)=1\), alors il existe des entiers \( \{ u_i \}_{i=1,\ldots, N}\) tels que
	\begin{equation}
		\sum_ia_iu_i=1.
	\end{equation}
\end{proposition}

\begin{corollary}       \label{CorgEMtLj}
	Soient \( p\) et \( q\) deux entiers premiers entre eux. Alors
	\begin{equation}
		p\eZ+q\eZ=\eZ;
	\end{equation}
	en particulier, pour tout \( x \in \eZ \), il existe \( u_x, v_x \) entiers tels que \(u_x p + v_x q = x \).
\end{corollary}

Notons que l'application \( p\eZ+q\eZ\) vers \( \eZ\) n'est évidemment pas injective: les \( u_x\) et \( v_x\) ne sont pas uniques à \( x\) fixé.

\begin{proof}
	Soit \( x\in \eZ\). Le théorème de Bézout nous donne \( k\) et \( l\) tels que \( kp+lq=1\). Alors, \( (xk)p+(xl)q=x\).
\end{proof}

%---------------------------------------------------------------------------------------------------------------------------
\subsection{Fonction indicatrice d'Euler}
%---------------------------------------------------------------------------------------------------------------------------

\begin{definition}		\label{DEFooZRYMooZCozga}
	La \defe{fonction indicatrice d'Euler}{indicatrice d'Euler} est l'application
	\begin{equation}
		\begin{aligned}
			\varphi\colon \eN^* & \to \eN^*                                                                    \\
			n                   & \mapsto \Card\Big(   \{ m\in \eN^*\tq 1\leq m\leq n, \pgcd(m,n)=1 \}  \Big).
		\end{aligned}
	\end{equation}
\end{definition}
Note : voir le thème \ref{THMooUDYMooCCXdbw} pour des formules concernant l'indicatrice d'Euler.

\begin{lemma}[\cite{BIBooYVNFooEVOwyw}]		\label{LEMooVGDHooStUaKH}
	L'élément \( [m]_n\) est inversible dans le groupe \( \big( (\eZ/n\eZ)^*,\cdot \big)\) si et seulement si \( \pgcd(m,n)=1\).
\end{lemma}

\begin{proof}
	Dans les deux sens.
	\begin{subproof}
		\spitem[\( \Rightarrow\)]
		%-----------------------------------------------------------
		Si \( [m_n]\) est inversible, il existe \( u\in \eZ\) tel que \( [u]_n[m]_n=[1]_n\). Cela donne \( [um]_n=[1]_n\) ou encore \( um\in [1]_n\), c'est à dire \( um=1+vn\). Le théorème de Bézout \ref{ThoBuNjam} conclu que \( \pgcd(m,n)=1\).

		\spitem[\( \Leftarrow\)]
		%-----------------------------------------------------------
		Si \( \pgcd(m,n)=1\), alors le théorème de Bézout \ref{ThoBuNjam} nous dit qu'il existe \( u,v\in \eZ\) tels que \( um+vn=1\). Cela donne directemen t \( um=1-vn\in [1]_n\) et donc \( [u]_n\) est un inverse de \( [m]_n\) dans le groupe multiplicatif \( \eZ/n\eZ\).
	\end{subproof}
\end{proof}

\begin{lemma}		\label{LEMooCLYEooONhWKs}
	Nous avons
	\begin{equation}
		\varphi(n)=\Card\big( (\eZ/n\eZ)^{\times} \big)
	\end{equation}
	où \( A^{\times}\) est le groupe des inversibles (pour la multiplication) dans l'anneau \( A\).
\end{lemma}

\begin{proof}
	En utilisant le lemme \ref{LEMooVGDHooStUaKH},
	\begin{subequations}
		\begin{align}
			\big( \eZ/n\eZ \big)^{\times} & =\{ [m]_n\tq \pgcd(m,n)=1 \}               \\
			                              & =\{ [m]_n\tq 1\leq m\leq n,\pgcd(m,n)=1 \}
		\end{align}
	\end{subequations}
	Comme deux entiers différents entre \( 1 \) et \( n\) ne peuvent pas être dans la même classe modulo \( n\), il y a bijection entre le dernier ensemble et \( \{ 1\leq m\leq n\tq \pgcd(m,n)=1 \}\). Donc
	\begin{equation}
		\varphi(n)=\Card\big( \{ [m]_n\tq 1\leq m\leq n,\pgcd(m,n)=1 \} \big)=\Card\big( (\eZ/n\eZ)^{\times} \big).
	\end{equation}
\end{proof}

\begin{lemma}[\cite{BIBooHIFAooWOavtO}]		\label{LEMooRGIYooRxgyCO}
	Soit \( n,d\in \eN\) tels que \( d\divides n\). Nous notons \( q=n/d\) et \( G=\{ k[q]_n\tq 0\leq k\leq d-1 \}\). Alors, pour \( r\in \eZ\) nous avons \( d[r]_n=[0]_n\) si et seulement si \( [r]_n\in G\).
\end{lemma}

\begin{proof}
	Dans les deux sens.
	\begin{subproof}
		\spitem[\( \Rightarrow\)]
		%-----------------------------------------------------------
		Nous supposons que \( d[r]_n=[0]_n\). Étant donné que \( n=dq\) nous avons les implications suivantes :
		\begin{equation}
			d[r]_n=[0]_n\Rightarrow d\divides dr\Rightarrow dq\divides dr\Rightarrow q\divides r.
		\end{equation}
		Nous avons donc \( r=kq\) pour un certain \( k\in \eZ\). Par la division euclidienne \ref{ThoDivisEuclide}, il existe \( s,t\in \eN\) tels que \( k=sd+t\) avec \( t<d\). Avec ça, nous avons le calcul
		\begin{equation}
			[r]_n=k[q]_n=sd[q]_n+t[q]_n=\underbrace{s[dq]_n}_{=[0]_n}+t[q]_n\in G.
		\end{equation}

		\spitem[$\Leftarrow$]
		%-----------------------------------------------------------
		Si \( [r]_n\in G\), il existe \( 0\leq k\leq d-1\) tel que \( [r]_n=k[q]_n\). De ce fait,
		\begin{equation}
			d[r]_n=dk[q]_n=k[dq]_n=[0]_n.
		\end{equation}
		Et voila.
	\end{subproof}
\end{proof}


\begin{lemma}[\cite{BIBooHIFAooWOavtO}]		\label{LEMooKPKBooPbrHkI}
	Soient \( d\divides n\) dans \( \eN^*\). Nous considérons le groupe additif \( G_d=\{ k[q]_n\tq 0\leq k\leq d-1 \}\). Les éléments d'ordre\footnote{Ordre d'un élément, définition \ref{DEFooKSTVooOObpgC}.} \( d\) dans \( (\eZ/n\eZ,+)\) sont les générateurs de \( G_d\).
\end{lemma}

\begin{proof}
	Les générateurs de \( G_d\) sont d'ordre \( d\) parce que \( | G_d |=d\). Ça, c'était le sens facile. Dans l'autre sens, si \( [r]_n\) est d'ordre \( d\), alors \( d[r]_n=[0]_n\). D'après le lemme \ref{LEMooRGIYooRxgyCO}, cela prouve que \( [r]_n\in G_d\).

	Comme le groupe engendré par \( [r]_n\) est d'ordre \( d\), il est tout \( G_d\). Donc \( [r]_n\) est générateur de \( G_d\).
\end{proof}


\begin{lemma}[\cite{BIBooDNSJooTXvNqc}]		\label{LEMooQGGLooDkkmcF}
	L'ensemble des générateurs de \( (\eZ/n\eZ,+)\) est \( (\eZ/n\eZ)^{\times}\).
\end{lemma}

\begin{proof}
	Si \( [r]_n\) est générateur de \( \eZ/n\eZ\), il existe \( k\) tel que \( k[r]_n=[1]_n\). Dans ce cas, \( [k]_n\) est l'inverse de \( [r]_n\) pour la multiplication. Donc \( [r]_n\in(\eZ/n\eZ)^{\times}\).

	À l'inverse, si \( [r]_n\) est inversible, alors il existe \( k\) tel que \( k[r]_n=[1]_n\). Dans ce cas, \( [t]_n=kt[r]_n\), ce qui montre que \( [r]_n\) est générateur (pour l'addition).
\end{proof}

\begin{lemma}[\cite{BIBooDNSJooTXvNqc}]		\label{LEMooRMWRooRSjGPL}
	Si \( G\) est un groupe cyclique d'ordre \( n\), alors \( G\) possède \( \varphi(n)\) générateurs.
\end{lemma}

\begin{proof}
	Vu que tous les groupes cycliques d'ordre \( n\) sont isomorphes à \( (\eZ/n\eZ),+\), nous nous contentons de prouver le résultat pour ce groupe. Le lemme \ref{LEMooQGGLooDkkmcF} montre que \( (\eZ/n\eZ,+)\) possède \( \Card\big( (\eZ/n\eZ)^{\times} \big)\) générateurs.

	Mais le lemme \ref{LEMooCLYEooONhWKs} assure que \( \Card\big( (\eZ/n\eZ)^{\times} \big)=\varphi(n)\).
\end{proof}

\begin{proposition}       \label{PROPooYHUDooUROTiN}
	Nous avons la formule
	\begin{equation}        \label{EqTPHqgJ}
		n=\sum_{d\divides n}\varphi(d).
	\end{equation}
\end{proposition}

\begin{proof}
	Nous notons \( H_d\) la partie de \( (\eZ/n\eZ,+)\) composée des éléments d'ordre \( d\). Nous avons vu dans le lemme \ref{LEMooKPKBooPbrHkI} que \( H_d\) sont justement les générateurs de \( G_d\) -- voir le lemme pour la notation. Mais comme \( G_d\) est un groupe cyclique d'ordre \( d\), il contient \( \varphi(d)\) générateurs (lemme \ref{LEMooRMWRooRSjGPL}) : \( \Card(H_d)=\varphi(d)\).

	Vu que tous les éléments de \( \eZ/n\eZ\) ont un ordre qui divise \( n\) (corolaire \ref{CorpZItFX}), nous avons l'union disjointe
	\begin{equation}
		\eZ/n\eZ=\bigcup_{d\divides n}H_d,
	\end{equation}
	et donc au niveau des cardinals,
	\begin{equation}
		n=\Card(\eZ/n\eZ)=\sum_{d\divides n}\Card(H_d)=\sum_{d\divides n}\varphi(d).
	\end{equation}
\end{proof}


\begin{lemma}       \label{LEMooBEJOooDqTirj}
	Si \( p\) est un nombre premier, alors \( \varphi(p^n)=p^n-p^{n-1}\).
\end{lemma}

\begin{proof}
	Les éléments de \( \{ 1,\ldots,p^n \}\) qui ont un \( \pgcd\) différent de \( 1\) avec \( p^n\) sont des nombres qui s'écrivent sous la forme \( qp\) avec \( q\leq p^{n-1}\)\footnote{Corolaire~\ref{CORooQIMHooUzLUJY}.}. Il y a évidemment \( p^{n-1}\) tels nombres.

	Par conséquent le cardinal de \( P_{p^n}\) est \( \varphi(p^{n})=p^n-p^{n-1}\).
	%TODOooWJIYooYtATMi Il faut élucider ce qu'est P_n, voir 2754128708
\end{proof}

\begin{probleme}
	%2754128708
	\( P_n\) n'a pas été défini.

	Définition proposée (et vue par après): \( P_n = \{ m \in \eN \tq \pgcd(m,n) = 1 \}. \) À mettre donc en lien avec \( \Delta_d\).
\end{probleme}

%---------------------------------------------------------------------------------------------------------------------------
\subsection{Générateurs}
%---------------------------------------------------------------------------------------------------------------------------

\begin{proposition}     \label{PropZnmuphiGensn}
	Soit \( n\in\eN\setminus\{ 0 \}\) et le groupe (additif) \( \eZ/n\eZ\). L'élément \( [x]_n\) est un générateur de \( \eZ/n\eZ\) si et seulement si \( x\in P_n\). En particulier \( \eZ/n\eZ\) est un groupe contenant \( \varphi(n)\) générateurs.
\end{proposition}

\begin{proof}
	Nous avons \( \gr\big( [1]_n \big)=\eZ/n\eZ\). L'élément \( [x]_n\) sera générateur si et seulement si il génère \( [1]_n \), c'est-à-dire si il existe \( u\) tel que \( u[x]_n=[1]_n\). Cette dernière égalité étant une égalité de classes dans \( \eZ/n\eZ\), elle sera vraie si et seulement si il existe \( v\) tel que
	\begin{equation}
		ux+vn=1.
	\end{equation}
	Cela signifie entre autres que\footnote{Corolaire~\ref{CorgEMtLj}} \( x\eZ+n\eZ=\eZ\), et aussi que \( \pgcd(x,n)=1\) par le théorème de Bézout~\ref{ThoBuNjam}, et donc que \( x\in P_n\).
\end{proof}

\begin{corollary}\label{CORooMBLSooMHKmAq}
	Un groupe monogène d'ordre \( n\) possède \( \varphi(n)\) générateurs, où \( \varphi\) est la fonction indicatrice d'Euler définie en~\ref{DEFooZRYMooZCozga}.
\end{corollary}

\begin{proof}
	Le théorème~\ref{THOooDOMZooOEYHAe} nous dit qu'un groupe monogène d'ordre \( n\) est isomorphe à \( \eZ/n\eZ\). La proposition~\ref{PropZnmuphiGensn} nous indique que \( \eZ/n\eZ\) possède \( \varphi(n)\) générateurs.
\end{proof}

%---------------------------------------------------------------------------------------------------------------------------
\subsection{Fonction indicatrice d'Euler (propriétés)}
%---------------------------------------------------------------------------------------------------------------------------
\label{subSecKGDFooAbETjs}

\begin{corollary}       \label{CorlvTmsf}
	Deux propriétés.
	\begin{enumerate}
		\item
		      L'indicatrice d'Euler est multiplicative : si \( p\) est premier avec \( q\), alors
		      \begin{equation}
			      \varphi(pq)=\varphi(p)\varphi(q).
		      \end{equation}
		\item
		      Si \( p\) est un nombre premier,
		      \begin{equation}
			      \varphi(p)=(p-1).
		      \end{equation}
	\end{enumerate}
\end{corollary}

\begin{proof}
	Nous savons que si \( p\) et \( q\) sont premiers entre eux, alors le théorème~\ref{ThoLnTMBy} nous donne l'isomorphisme de groupe
	\begin{equation}
		(\eZ/pq\eZ,+)\simeq(\eZ/p\eZ,+)\times(\eZ/q\eZ,+).
	\end{equation}
	Un élément \( (x,y)\) est générateur du produit si et seulement si \( x\) est générateur de \( \eZ/p\eZ\) et \( y\) est générateur de \( \eZ/q\eZ\). Par la proposition~\ref{PropZnmuphiGensn}, il y a \( \varphi(p)\varphi(q)\) tels éléments. Par ailleurs le nombre de générateurs de \( \eZ/pq\eZ\) est \( \varphi(pq)\), d'où l'égalité.

	Si \( p\) est premier, nous avons \( \varphi(p)=p-1\) parce que tous les entiers de \( \{ 1,\ldots, p-1 \}\) sont premiers avec \( p\).
\end{proof}


\begin{corollary}       \label{CORooLINXooBlUKPG}
	Les quotients de \( \eZ\) sont \( \eZ/n\eZ\).
\end{corollary}
%TODOooKQDLooFnggOd: il faut préciser si ça veut dire quelque chose comme les seuls quotients de (Z,+) qui sont encore des groupes.

\begin{proof}
	Tous les idéaux de \( \eZ\) sont de la forme \( n\eZ\). En effet en vertu de la proposition~\ref{PropSsgpZestnZ}, les seuls sous-groupes de \( \eZ\) (en tant que groupe additif) sont les \( n\eZ\). Tous les idéaux sont donc de cette forme. De plus les \( n\eZ\) sont effectivement tous des idéaux\footnote{Définition \ref{DefooQULAooREUIU}.} : si \( a\in n\eZ\) et si \( k\in \eZ\) alors \( ak\in n\eZ\).
\end{proof}

\begin{proposition}     \label{PropZpintssiprempUzn}
	Soient \( n\geq 2\) un entier et \( \phi\colon \eZ\to \eZ/n\eZ\) la surjection canonique. Nous noterons \( \overline a=\phi(a)\). Alors l'ensemble des inversibles de \( \eZ/n\eZ\) est donné par
	\begin{equation}
		U(\eZ/n\eZ)=\phi(P_n)=\{ \overline x\tq 0\leq x\leq n\tq\pgcd(x,n)=1 \}.
	\end{equation}
	où \( P_n\) est l'ensemble \( P_n=\{ x\in\{ 0,\ldots,n-1 \}\tq\pgcd(x,n)=1 \}\).

	De plus,
	\begin{equation}
		\Card\big( U(\eZ/n\eZ) \big)=\phi(n).
	\end{equation}
\end{proposition}

\begin{proof}
	Soit \( 0\leq x\leq n\) tel que \( \pgcd(x,n)=1\). Il existe donc\footnote{Théorème de Bézout~\ref{ThoBuNjam}} \( u,v\in\eZ\) tels que \( ux+vn=1\). En passant aux classes,
	\begin{equation}
		\overline u\overline x=\overline 1,
	\end{equation}
	donc \( \overline u\) est l'inverse de \( \overline x\). Cela prouve que \( \phi(P_n)\subset U(\eZ/n\eZ)\).

	Nous prouvons maintenant l'inclusion inverse. Soient \( \overline x\) et \( \overline y\) inverses l'un de l'autre : \( \overline x\overline y=\overline 1\). Il existe donc \( q\in\eZ\) tel que \( xy-qn=1\), ce qui prouve\footnote{À nouveau avec le Théorème de Bézout.} que \( \pgcd(x,n)=1\).
\end{proof}

\begin{lemma}     \label{LEMooZSMEooUmSXWZ}
	Un corps\footnote{Définition~\ref{DefTMNooKXHUd}.} est un anneau intègre.
\end{lemma}

\begin{proof}
	En effet, soient un corps \( \eK\) et deux éléments \( x,y\in \eK\) tels que \( xy=0\). Si \( y\) est inversible, alors nous pouvons multiplier par \( y^{-1}\) pour trouver \( x=0\). Cela prouve que \( \eK\) est un anneau intègre.
\end{proof}


\begin{example}
	L'anneau \( \eZ/6\eZ\) n'est pas intègre parce que \( 3\cdot 2=0\) alors que ni \( 3\) ni \( 2\) ne sont nuls.
\end{example}

Nous verrons au théorème~\ref{ThoBUEDrJ} que l'anneau \( A\) est intègre si et seulement si \( A[X]\) est intègre.

\begin{corollary}   \label{CorZnInternprem}
	L'anneau \( \eZ/n\eZ\) est intègre si et seulement si \( n\) est premier.
\end{corollary}

\begin{proof}
	Supposons que \( n\) soit premier. La proposition \ref{PropZpintssiprempUzn} donne les inversibles de \( \eZ/n\eZ\) par
	\begin{equation}
		U(\eZ/n\eZ)=\{ \overline x\tq 0\leq x\leq n\tq\pgcd(x,n)=1 \}.
	\end{equation}
	Mais comme \( n\) est premier, \( \pgcd(x,n)=1\) pour tout \( x\), et donc tous les éléments de \( \eZ/n\eZ\) sont inversibles. Donc \( \eZ/n\eZ\) est intègre.

	Si \( n\) n'est pas premier, alors \( n=pq\) avec \( 1<p\leq q<n\). Alors
	\begin{equation}
		[p]_n[q]_n=[pq]_n=[0]_n.
	\end{equation}
	Donc lorsque \( n\) n'est pas premier,  l'anneau \( \eZ/n\eZ\) possède des diviseurs de zéro et n'est alors pas intègre.
\end{proof}



\begin{proposition}[Thème~\ref{THEMEooZYKFooQXhiPD}, \cite{MonCerveau}] \label{PropomqcGe}
	Soit \( A\) un anneau principal\footnote{Définition \ref{DEFooGWOZooXzUlhK}.} qui n'est pas un corps. Pour un idéal propre \( I\) de \( A\), les conditions suivantes sont équivalentes :
	\begin{enumerate}
		\item       \label{ITEMooNOVFooEHtcwE}
		      \( I\) est un idéal maximal\footnote{Définition \ref{DEFIdealMax}.};
		\item       \label{ITEMooMQWVooNocVEU}
		      \( I\) est un idéal premier non nul\footnote{Définition \ref{DEFooAQSZooVhvQWv}.};
		\item       \label{ITEMooJBXGooEISNuW}
		      il existe \( p\) irréductible\footnote{Définition \ref{DeirredBDhQfA}.} dans \( A\) tel que \( I=(p)\).
	\end{enumerate}
\end{proposition}

\begin{proof}
	En plusieurs implications.
	\begin{subproof}
		\spitem[\ref{ITEMooNOVFooEHtcwE} implique~\ref{ITEMooMQWVooNocVEU}]
		Par hypothèse, \( I\) est un idéal propre, de plus il n'est pas égal à \( \{ 0 \}\), parce que lorsque \( A\) et \( \{ 0 \} \) sont les seuls idéaux, nous avons un corps (proposition~\ref{PROPooUOCVooZGAVVk}). Étant donné que \( I\) est un idéal maximal, le quotient \( A/I\) est un corps par la proposition~\ref{PROPooSHHWooCyZPPw}.

		Soient maintenant, pour entrer dans le vif du sujet, des éléments \( a,b\in A\) tels que \( ab\in I\). Dans le corps \( A/I\) nous avons \( [ab]=0\), et par définition du produit dans le quotient, \( [a][b]=0\). Par intégrité de l'anneau \( A/I\) (un corps est un anneau intègre, lemme \ref{LEMooZSMEooUmSXWZ}) nous avons soit \( [a]=0\), soit \( [b]=0\), soit les deux en même temps. Dans tous les cas, soit \( a\) soit \( b\) est dans \( I\).

		\spitem[\ref{ITEMooMQWVooNocVEU} implique~\ref{ITEMooJBXGooEISNuW}]
		Maintenant \( I\) est un idéal premier non réduit à \( \{ 0 \}\). Puisque \( A\) est un anneau principal, il existe \( x\in A\) tel que \( I=(x)\). Nous devons prouver que \( x\) peut être choisi irréductible; et nous allons faire plus : nous allons prouver que \( x\) ne peut être que irréductible\quext{ça me semble un peu trop facile. Lisez attentivement, et écrivez-moi pour dire si vous êtes d'accord ou pas.}.

		Supposons que \( x\) ne soit pas irréductible. Alors il existe \( a,b\in A\) non inversibles tels que \( x=ab\). Si \( a\in (x)\) alors il existe \( k\in A\) tel que \( a=xk\), et en particulier, \( a=abk\), c'est-à-dire \( 1=bk\) (parce que \( A\) est principal et donc intègre). Cela signifie que \( b\) est inversible alors que nous avions dit qu'il ne l'était pas. Nous en déduisons que \( a\) n'est pas dans \( (x)\). On montre de manière similaire que \( b\) n'est pas dans \( (x)\) non plus.

		Nous nous retrouvons donc avec \( a,b\in A\) tel que \( ab\in I\) sans que ni \( a\) ni \( b\) ne soient dans \( I\). Cela contredit le fait que \( I\) soit un idéal premier. En conclusion, \( x\) est irréductible.

		\spitem[\ref{ITEMooJBXGooEISNuW} implique~\ref{ITEMooNOVFooEHtcwE}]
		Nous avons \( I=(p)\) avec \( p\) irréductible dans \( A\). Supposons que \( J\) est un idéal différent de \( A\) contenant \( I\). Comme \( A\) est principal, il existe \( y\in A\) tel que \( J=(y)\). En particulier \( p\in J\), donc \( p=ay\) pour un certain \( a\in A\). Mais \( p\) est irréductible, donc soit \( a\) est inversible, soit \( y\) est inversible. Si \( y\) est inversible, alors \( J=A\), ce qui est exclu. Si \( a\) est inversible, alors \( (y)=(p)\), et \( I=J\).
	\end{subproof}
\end{proof}

\begin{normaltext}
	Dans la proposition \ref{PropomqcGe}, l'hypothèse d'idéal propre est importante. En effet dans le cas \( I=A\), nous avons évidemment que \( I\) est un idéal maximum. Mais \( A\) n'est d'abord pas un idéal premier parce qu'un idéal premier doit être strictement inclus dans l'anneau. Et ensuite, \( A\) est en général loin d'être garanti d'être égal à \( (p)\) pour un de ses éléments \( p\).
\end{normaltext}

\begin{proposition}     \label{PropoTMMXCx}
	Soit \( A \) un anneau principal, et soit \( p \in A \) un élément irréductible. Alors
	\begin{enumerate}
		\item
		      \( (p)\) est un idéal maximum.
		\item       \label{ITEMooKPJQooWuPZXS}
		      \( A/(p)\) est un corps.
	\end{enumerate}
\end{proposition}

\begin{proof}
	Nous notons \( I=(p)\). Soit un idéal \( J\) contenant \( I\). Comme \( A\) est principal, \( J\) est monogène : \( J=(q)\). Mais comme \( p\) est dans \( I\) qui est dans \( J\), il existe \( a\in A\) tel que \( p=qa\).

	Puisque \( p\) est irréductible, soit \( q\), soit \( a\) est inversible. Si \( q\) est inversible, alors \( J=A\). Si \( a\) est inversible, alors nous avons \( p=qa\), donc \( q=pa^{-1}\), ce qui signifie que \( q\in(p)\) et donc que \( J=I\).

	Cela prouve que \( (p)\) est un idéal maximum.

	Le fait que \( A/(p)\) soit un corps est maintenant la proposition~\ref{PROPooSHHWooCyZPPw}.
\end{proof}

\begin{example}
	L'anneau \( \eZ\) est principal parce qu'il est intègre et que ses seuls idéaux sont les \( n\eZ\) qui sont principaux : \( n\eZ\) est engendré par \( n\).
\end{example}

\begin{example}[Les idéaux de \( \eZ/n\eZ\)]       \label{EXooCJRPooYkWdyr}

	Les idéaux de \( \eZ/n\eZ\) sont principaux, mais l'anneau \( \eZ/n\eZ\) n'est pas principal lorsque \( n\) n'est pas premier. Nous allons voir ça.

	\begin{subproof}
		\spitem[Les idéaux de \( \eZ/n\eZ\) sont principaux]
		Soit un idéal \( S\) dans \( \eZ/n\eZ\). Nous considérons la projection canonique \( \phi\colon \eZ\to \eZ/n\eZ\). La proposition~\ref{PropIJJIdsousphi} dit que  \( S=\phi(J)\) où \( J\) est un idéal de \( \eZ\) contenant \( n\eZ\). Mais le corolaire~\ref{CORooLINXooBlUKPG} nous dit qu'alors \( J=m\eZ\) pour un certain \( m\). Pour que \( m\eZ\) contienne \( n\eZ\), il faut que \( m\) divise \( n\).

		Bref, \( S=\phi(m\eZ)\) avec \( m\divides n\). Nous montrons maintenant que \( S\) est engendré par \( [m]_n\). D'abord, l'élément \( [m]_n\) est bien dans \( \phi(m\eZ)\). Ensuite un élément de \( \phi(m\eZ)\) est de la forme
		\begin{equation}
			[km]_n=k[m]_n\in ([m]_n).
		\end{equation}
		Donc \( S\subset ([m]_n)\). Et l'inclusion dans l'autre sens est tout aussi immédiate : un élément de \( ([m]_n)\) est de la forme
		\begin{equation}
			k[m]_n=[km]_n=\phi(km)\in \phi(m\eZ).
		\end{equation}

		\spitem[Si \( n\) n'est pas premier, \( \eZ/n\eZ\) n'est pas principal]
		Le fait est que lorsque \( n\) n'est pas premier, \( \eZ/n\eZ\) n'est pas intègre (corolaire~\ref{CorZnInternprem}).

		\spitem[Moralité]
		Un anneau comme \( \eZ/6\eZ\) est un anneau dont tous les idéaux sont principaux, mais qui n'est pas principal.
	\end{subproof}
\end{example}

\begin{example}
	Nous verrons dans la proposition~\ref{PROPooVWRPooGQMenV} que l'anneau des fonctions holomorphes sur un compact de \( \eC\) est principal.
\end{example}


\begin{theorem}\index{théorème!chinois!anneau principal}        \label{ThofPXwiM}
	Si \( A\) est un anneau principal et si \( p\) et \( q\) sont premiers entre eux dans \( A\), alors on a l'isomorphisme d'anneaux
	\begin{equation}
		A/pqA\simeq A/pA\times A/qA.
	\end{equation}
\end{theorem}


%+++++++++++++++++++++++++++++++++++++++++++++++++++++++++++++++++++++++++++++++++++++++++++++++++++++++++++++++++++++++++++
\section{Anneau euclidien}
%+++++++++++++++++++++++++++++++++++++++++++++++++++++++++++++++++++++++++++++++++++++++++++++++++++++++++++++++++++++++++++

\begin{definition}[\wikipedia{fr}{Anneau_euclidien}{Wikipédia}] \label{DefAXitWRL}
	Soit \( A\) un anneau intègre\footnote{Défnition \ref{DEFooTAOPooWDPYmd}.}. Un \defe{stathme euclidien}{stathme euclidien} sur \( A\) est une application \( \alpha\colon A\setminus\{ 0 \}\to \eN\) tel que
	\begin{enumerate}
		\item       \label{ITEMooLVJAooLpjgEz}
		      \( \forall a,b\in A\setminus\{ 0 \}\), il existe \( q,r\in A\) tel que
		      \begin{equation}
			      a=qb+r
		      \end{equation}
		      et \( \alpha(r)<\alpha(b)\).
		\item
		      Pour tout \( a,b\in A\setminus\{ 0 \}\), \( \alpha(b)\leq \alpha(ab)\).
	\end{enumerate}
	Un anneau est \defe{euclidien}{euclidien!anneau} si il accepte un stathme euclidien.
\end{definition}
Le stathme est la fonction qui donne le «degré» à utiliser dans la division euclidienne. La contrainte est que le degré du reste soit plus petit que le degré du dividende.

\begin{lemma}       \label{LEMooFUSTooDCcBDb}
	L'ensemble \( \eZ\) avec les opérations usuelles est un anneau intègre\footnote{Anneau intègre, définition \ref{DEFooTAOPooWDPYmd}.}.
\end{lemma}

\begin{example} \label{ExwqlCwvV}
	Le stathme de \( \eN\) pour la division euclidienne usuelle est \( \alpha(n)=n\). Si \( a,b\in \eN\) nous écrivons
	\begin{equation}
		a=qb+r
	\end{equation}
	où \( q\) est l'entier le plus proche \emph{inférieur} à \( a/b\) (on veut que le reste soit positif) et \( r=a-qb\). Nous avons donc
	\begin{equation}
		r-b=a-b(q+1)<a-b\frac{ a }{ b }=0,
	\end{equation}
	ce qui montre que \( r<b\).
\end{example}

Cet exemple ne fonctionne pas avec \( \eZ\) au lieu de \( \eN\) parce que le stathme doit avoir des valeurs dans \( \eN\). Cela ne veut cependant pas dire qu'il n'existe pas de stathme sur \( \eZ\); cela veut seulement dire que \( \alpha(x)=x\) ne fonctionne pas.

\begin{proposition}[\cite{ooELVSooZIZCRn}]\label{Propkllxnv}
	Tout anneau euclidien\footnote{Euclidien, définition \ref{DefAXitWRL}.} est principal\footnote{Principal, définition \ref{DEFooMZRKooBPLAWH}}.
\end{proposition}

\begin{proof}
	Soit \( A\) un anneau euclidien et \( \alpha\) un stathme sur \( A\). Nous considérons un idéal \( I\) non nul de \( A\). Nous devons montrer que \( I\) est généré par un élément. En l'occurrence nous allons montrer qu'un élément \( a\in I\setminus\{ 0 \}\) qui minimise \( \alpha(a)\) va générer\footnote{Un tel élément existe\dots} \( I\).

	Soit \( x\in I\). Par construction, il existe \( q,r\in A\) tels que \( x=aq+r\) avec \( r=0\) ou \( \alpha(r)<\alpha(a)\). Étant donné que \( x,a\in I\), \( r\in I\). Si \( r\neq 0\), alors \( r\) contredirait la minimalité de \( \alpha(a)\). Donc \( r=0\) et \( x=aq\), ce qui signifie que \( I\) est principal.
\end{proof}

\begin{proposition}     \label{PROPooPJGLooQSrJTU}
	L'anneau \( \eZ\) est principal et euclidien.
\end{proposition}

\begin{proof}
	Nous allons seulement montrer que \( \alpha(x)=| x |\) est un stathme euclidien. Ainsi \( \eZ\) sera euclidien et donc principal par la proposition~\ref{Propkllxnv}.

	D'abord \( \eZ\) est intègre, c'est le lemme \ref{LEMooFUSTooDCcBDb}.

	La condition \( \alpha(b)\leq \alpha(ab)\) est immédiate : \( | a |\leq | ab |\) pour tout \( a,b\in \eZ\).

	Soient maintenant \( a,b\in \eZ\). Nous définissons \( q_0,r_0\in \eN\) tels que
	\begin{equation}
		| a |=q_0| b |+r_0
	\end{equation}
	avec \( r_0<| b |\). Cela existe parce que \( \alpha(x)=x\) est un stathme sur \( \eN\) par l'exemple~\ref{ExwqlCwvV}.

	\begin{subproof}
		\spitem[Si \( a>0\) et \( b>0\)]

		Alors \( a=q_0b+r_0\) et le couple \( (q_0,r_0)\) vérifie les conditions de la définition~\ref{DefAXitWRL}\ref{ITEMooLVJAooLpjgEz}.

		\spitem[Si \( a>0\) et \( b<0\)]

		Alors \( a=-q_0b+r_0\), et le couple \( (-q_0,r_0)\) vérifie les conditions de la définition~\ref{DefAXitWRL}\ref{ITEMooLVJAooLpjgEz}.


		\spitem[Si \( a<0\) et \( b>0\)]
		Alors \( a=-q_0b-r_0\), et le couple \( (-q_0,-r_0)\) vérifie les conditions de la définition~\ref{DefAXitWRL}\ref{ITEMooLVJAooLpjgEz} parce que
		\begin{equation}
			\alpha(-r_0)=r_0<| b |=\alpha(b).
		\end{equation}
		\spitem[Si \( a<0\) et \( b<0\)]
		Alors \( a=q_0b-r_0\), et le couple \( (q_0,-r_0)\) vérifie les conditions de la définition~\ref{DefAXitWRL}\ref{ITEMooLVJAooLpjgEz}.

	\end{subproof}
\end{proof}

Nous venons de voir que \( \eZ\) est principal; le lemme suivant nous dit que \( \eZ[X]\) n'est lui, pas principal.
\begin{lemma}[\cite{ooRQHSooEBZpKe}]        \label{LEMooDJSUooJWyxCL}
	Si \( A\) est un anneau intègre\footnote{Définition \ref{DEFooTAOPooWDPYmd}.} qui n'est pas un corps, alors \( A[X]\) n'est pas principal.
\end{lemma}

\begin{proof}
	Soit un élément non nul \( a\in A\).
	\begin{subproof}
		\newcommand{\foo}{A[X]}
		\spitem[Un idéal principal contenant \( a\) et \( X\) est {A[X]}]

		Soit \( (P)\) un idéal principal contenant \( a\) et \( X\). Puisque \( a\in(P)\), il existe \( Q\) tel que \( a=QP\). Donc \( P\) divise \( a\) dans \( \eZ[X]\). L'égalité des degrés indique que \( P\) est un polynôme constant, c'est-à-dire en réalité un élément de \( A\). Soit \( P=k\in A\).

		Comme \( P\) divise \( X\), nous avons aussi \( X=kQ\) pour un certain \( Q\in \eZ[X]\). L'égalité des degrés dit qu'il existe \( k'\in A\) tel que \( Q=k'X\) et donc \( Q=k'X=k'kQ\), ce qui implique que \( kk'=1\). L'idéal engendré par \( k\) contient donc en particulier \( kk'=1\) et donc contient \( A[X]\) en entier :
		\begin{equation}
			1=k'k\in k'(P)=(P).
		\end{equation}

		\spitem[Si \( (a,X)=\foo\) alors \( a\) est inversible]

		Si \( (a,X)=A[X]\), en particulier, \( 1\in (a,X)\), ce qui signifie qu'il existe des polynômes \( U,V\in A[X]\) tels que
		\begin{equation}
			1=UX+Va.
		\end{equation}
		Nous évaluons cette égalité en \( 0\) : comme \( (UX)(0)=0\), nous avons \( 1=V(0)a\), ce qui signifie que \( V(0)\) est un inverse de \( A\). Donc \( a\) est inversible.

		\spitem[Si \( a\) n'est pas inversible alors \( (a,X)\) n'est pas principal]

		Si \( (a,X)\) était principal, alors nous aurions, par ce qui est dit plus haut, \( (a,X)=A[X]\). Mais cette dernière égalité impliquerait que \( a\) soit inversible.
	\end{subproof}
	En conclusion, si \( A\) n'est pas un corps, il possède un élément ni nul ni inversible. Dans ce cas, l'idéal \( (a,X)\) n'est pas principal dans \( A[X]\) et nous en déduisons que \( A[X]\) n'est pas un anneau principal.
\end{proof}

Nous verrons dans le lemme~\ref{LEMooIDSKooQfkeKp} que si \( \eK\) est un corps, alors \( \eK[X]\) est principal.

%+++++++++++++++++++++++++++++++++++++++++++++++++++++++++++++++++++++++++++++++++++++++++++++++++++++++++++++++++++++++++++
\section{Le groupe et anneau des entiers}
%+++++++++++++++++++++++++++++++++++++++++++++++++++++++++++++++++++++++++++++++++++++++++++++++++++++++++++++++++++++++++++

Certes \( (\eZ,+)\) est un groupe mais en ajoutant la multiplication, \( (\eZ,+,\times)\) devient un anneau\footnote{Définition~\ref{DefHXJUooKoovob}.}.


%---------------------------------------------------------------------------------------------------------------------------
\subsection{PGCD, PPCM et Bézout}
%---------------------------------------------------------------------------------------------------------------------------

Puisque \( \eZ\) est un anneau intègre, nous avons la définition \ref{DefrYwbct} de pgcd et de ppcm.
\begin{proposition}[PPCM et PGCD]       \label{PROPooAVRGooUfhjwF}
	Soient \( p,q\in\eZ^*\).
	\begin{enumerate}
		\item
		      Le pgcd de \( p\) et \( q\) est le plus grand diviseur commun de \( p\) et \( q\).
		\item
		      Le ppcm de \( p\) et \( q\) est leur plus petit multiple commun.
	\end{enumerate}
\end{proposition}

\begin{proof}
	Démontrons le premier point. Notons \( \delta\) le pgcd de \( p\) et \( q\). Si \( d\) est un diviseur commun de \( p\) et \( q\), alors \( d\) divise \( \delta\). Dans \( \eZ\), \( d\divides \delta\) implique \( d\leq\delta\) (proposition \ref{PROPooYJBMooZrzkNX}).
\end{proof}

\begin{lemma}
	Soient \( p,q\in\eZ^*\). Les entiers \( \ppcm(p,q)\) et \( \pgcd(p,q)\) fournissent les isomorphismes de groupes suivants :
	\begin{subequations}
		\begin{align}
			p\eZ\cap q\eZ & =\ppcm(p,q)\eZ  \\
			p\eZ + q\eZ   & =\pgcd(p,q)\eZ.
		\end{align}
	\end{subequations}
\end{lemma}

\begin{definition}      \label{DEFooXSPFooPumQSy}
	Nous disons que deux éléments d'un anneau principal\footnote{Anneau principal, définition \ref{DEFooGWOZooXzUlhK}.} sont \defe{premiers entre eux}{premier!deux éléments d'un anneau principal} si leurs diviseurs communs sont inversibles.
\end{definition}

Vu que \( \eZ\) est un anneau principal (proposition \ref{PROPooPJGLooQSrJTU}), la définition \ref{DEFooXSPFooPumQSy} d'éléments premiers entre eux s'applique.

\begin{lemma}       \label{LEMooLKLTooXUdsgB}
	Dans \( \eZ\), les nombres \( p\) et \( q\) sont premiers entre eux si et seulement si \( \pgcd(p,q)=1\).
\end{lemma}


\begin{definition}  \label{DefZHRXooNeWIcB}
	Si nous avons un ensemble d'entiers \( a_i\), nous disons qu'ils sont premiers \defe{dans leur ensemble}{nombre!premier!dans leur ensemble} si \( 1\) est le PGCD de tous les \( a_i\) ensemble.
\end{definition}

Les nombres \( 2\), \( 4\) et \( 7\) ne sont pas premiers deux à deux (à cause de \( 2\) et \( 4\)), mais ils sont premiers dans leur ensemble parce qu'il n'y a pas de diviseurs communs plus grand que \( 1\), au triplet \( (2, 4, 7)\).

La proposition suivante établit que si \( x\) est assez grand, alors il peut même être écrit comme une combinaison de \( p\) et \( q\) à coefficients positifs. Elle sera utilisée pour démontrer que les états apériodiques d'une chaine de Markov peuvent être atteints à tout moment (assez grand), voir la définition~\ref{DefCxvOaT} et ce qui suit.

\begin{proposition}     \label{PropLAbRSE}
	Soient \( a\) et \( b\) deux éléments de \( \eN\) premiers entre eux. Il existe \( N>0\) tel que tout \( x>N\) appartient à \( a\eN+b\eN\).
\end{proposition}

\begin{proof}
	Soient \( a\) et \( b\), premiers entre eux, et \( x\in \eN\). Disons tout de suite, pour éviter les cas triviaux et pénibles, que \( x\), \( a\) et \( b\) sont strictement positifs.

	\begin{subproof}
		\spitem[Une décomposition pour \( x\)]

		On applique le théorème~\ref{ThoDivisEuclide} de division euclidienne à \( x\) et \( a + b \): il existe des entiers \( p_x, r_x \), uniques, tels que
		\begin{subequations}
			\begin{numcases}{}
				x = (p_x-1)(a+b) + r_x\\
				0 \leq r_x < a+b.
			\end{numcases}
		\end{subequations}
		En d'autres termes, \( p_x(a+b)\) est le premier multiple de \( a+b\) supérieur ou égal à \( x\). De plus, \( p_x\) est strictement positif car \( x\) l'est. Il existe alors des entiers \( u\) et \( v\) tels que
		\begin{equation}    \label{EQooXYSZooJqxPui}
			ua + vb = p_x(a+b) - x
		\end{equation}
		par le corolaire~\ref{CorgEMtLj}. Ainsi, \( x\) peut s'écrire
		\begin{equation}
			x = (p_x - u) a + (p_x - v) b.
		\end{equation}

		\spitem[Des maximums]

		Il s'agit maintenant de savoir si nous pouvons être assuré d'avoir \( p_x > u\) et \( p_x > v\) dès que \( x\) est assez grand. Pour cela, grâce au corolaire~\ref{CorgEMtLj}, nous considérons les nombres \( u_i\) et \( v_i\) définis par
		\begin{equation}
			u_ia+v_ib=i
		\end{equation}
		pour \( i=1,\ldots, a+b\). Nous posons \( u^*=\max\{ u_i \}\), \( v^*=\max\{ v_i   \}\), et \( p^*=\max\{ u^*,v^* \}\).  Nous posons alors \( N = p^*(a+b)\), et considérons \( x>N \).

		\spitem[Nouvelle décomposition pour \( x\)]

		Nous voulons écrire
		\begin{equation}        \label{EQooIKNWooBKItYz}
			x= (p_x - u_k) a + (p_x - v_k) b
		\end{equation}
		pour un certain \( k\). Cela demande \( u_ka+v_kb=ua+vb=p_x(a+b)-x\) par l'équation \eqref{EQooXYSZooJqxPui}. Vu que \( p_x(a+b)-x>0\), les nombres \( u_k\) et \( v_k\) existent : il suffit de prendre \( k=p_x(a+b)-x\).

		\spitem[Conclusion]

		Avec tous ces choix, nous avons d'abord \( x>p^*(a+b)\) et donc
		\begin{equation}
			x=(p_x-1)(a+b)+r_x>p^*(a+b),
		\end{equation}
		ce qui donne
		\begin{equation}
			(p_x-1)(a+b)>p^*(a+b)-r_x>(p-1)(a+b).
		\end{equation}
		ou encore \( p_x>p^*\). Nous avons finalement
		\begin{equation}
			p_x \geq p^* \geq u^* \geq u_k
		\end{equation}
		et
		\begin{equation}
			p_x \geq p^* \geq v^* \geq v_k.
		\end{equation}
		De ce fait, la décomposition \eqref{EQooIKNWooBKItYz} est celle que nous voulions.
	\end{subproof}
\end{proof}


%\begin{proof}
%Soit \( x\in \eN\) et \( k_1,l_1\in \eN\) les plus petits entiers tels que \( k_1p\geq x/2\) et \( l_1q\geq x/2\). Nous avons alors
%\begin{equation}
%x\leq k_1p+l_1q<x+(p+q).
%\end{equation}
%Nous posons \( \delta=k_1p+l_1q-x\).
%
%Soient des entiers \( a_i,b_i\) tels que \( a_ip+b_iq=i\). Nous notons
%\begin{subequations}
%\begin{align}
%A=\max\{ a_i\tq i=1,\ldots, k+p \}\\
%B=\max\{ b_i\tq i=1,\ldots, k+p \}
%\end{align}
%\end{subequations}
%Notons que \( A\) et \( B\) sont donnés uniquement en termes de \( p\) et \( q\). Ils ne sont en aucun cas dépendants de \( x\).
%
%Nous avons
%\begin{equation}
%x=k_1p+lq-\delta=(k_1-a_{\delta})p+(l_1+b_{\delta})q
%\end{equation}
%avec \( k_1-a_{\delta}\geq k_1-A\) et \( l_1-b_{\delta}\geq l_1-B\). Si \( x\) est suffisamment grand pour avoir \( k_1>A\) et \( l_1>B\), alors la décomposition souhaitée est trouvée.
%
%Une borne pour \( x\) est donnée par
%\begin{equation}    \label{EqjQpURG}
%x>\max\{ 2pA,2qB \}.
%\end{equation}
%\end{proof}

\begin{normaltext}
	Une méthode pour obtenir les entiers naturels \( u\) et \( v\) qui permettent la décomposition \(x = au + bv \) est d'abord de choisir \( u_0\) et \( v_0\) tels que \( au_0 \) et \( bv_0 \) soient les plus proches possibles de \( x/2\), puis de décomposer le nombre (relativement petit) \( x - au_0 - bv_0 \) en \( au_1 + bv_1 \). Deux nombres \( u\) et \( v\) qui fonctionnent sont alors \( u = u_0 + u_1\) et \( v = v_0 + v_1\).
\end{normaltext}

\begin{example}
	Écrivons \( 1000=u\cdot 7+v\cdot 5\) avec \( u,v\in \eN\). D'abord \( 72\cdot 7=504\) et \( 100\cdot 5=500\). Nous avons donc
	\begin{equation}
		1004=72\cdot 7+100\cdot 5.
	\end{equation}
	Ensuite \( 4=25-21=-3\cdot 7+5\cdot 5\). Au final,
	\begin{equation}
		1000=75\cdot 7+95\cdot 5.
	\end{equation}
\end{example}


%+++++++++++++++++++++++++++++++++++++++++++++++++++++++++++++++++++++++++++++++++++++++++++++++++++++++++++++++++++++++++++
\section{Corps}
%+++++++++++++++++++++++++++++++++++++++++++++++++++++++++++++++++++++++++++++++++++++++++++++++++++++++++++++++++++++++++++

La définition d'un corps est \ref{DefTMNooKXHUd}.

%---------------------------------------------------------------------------------------------------------------------------
\subsection{Définitions, morphismes}
%---------------------------------------------------------------------------------------------------------------------------

La proposition suivante donne une caractérisation d'un corps, en disant un tout petit peu plus que la définition~\ref{DefTMNooKXHUd}.
\begin{proposition}
	L'anneau \( A\) est un corps si et seulement si \( U(A) = A^* \).
\end{proposition}

\begin{proof}
	En deux parties.
	\begin{subproof}
		\spitem[Sens direct]
		Nous supposons que \( A\) est un corps. D'une part tous les éléments non nuls sont inversibles, c'est-à-dire \( A^*\subset U(A)\).

		Pour l'inclusion inverse, nous montrons qu'une élément inversible ne peut pas être nul. Cela n'est autre que le lemme~\ref{LEMooVUSMooWisQpD} couplé à la proposition~\ref{PROPooNCCGooXjVyVt} : \( a\cdot 0=0\neq 1\) pour tout \( a\).
		\spitem[Sens inverse]
		Si \( U(A)=A^*\), nous avons immédiatement que tous les éléments non nuls sont inversibles et donc que \( A\) est un corps.
	\end{subproof}
\end{proof}

\begin{lemma}       \label{LEMooJNIBooAURhrt}
	Si \( \eK\) est un corps et si \( a\in \eK\) vérifie \( a^2=1\), alors \( a=\pm 1\).
\end{lemma}

\begin{definition}[Morphisme de corps]
	Un corps étant un anneau sans plus de structure, un \defe{morphisme de corps}{morphisme!de corps}\index{isomorphisme!de corps} n'est qu'un morphisme des anneaux\footnote{Définition \ref{DEFooSPHPooCwjzuz}.}.
\end{definition}

Le lemme suivant montre que définir un morphisme de corps comme étant simplement un morphisme des anneaux est une bonne idée.
\begin{lemma}       \label{LEMooWBOPooZnsZgQ}
	Si \( \varphi\colon \eK\to \eK'\) est un morphisme de corps, alors
	\begin{enumerate}
		\item
		      pour tout \( a\in \eK\) nous avons \( \varphi(a^{-1})=\varphi(a)^{-1}\);
		\item
		      le morphisme \( \varphi\) est injectif.
	\end{enumerate}
\end{lemma}

\begin{proof}
	Vu que \( \varphi(1)=1\), nous avons aussi
	\begin{equation}
		1=\varphi(aa^{-1})=\varphi(a)\varphi(a^{-1}).
	\end{equation}
	Donc, par unicité de l'inverse\footnote{Lemme~\ref{LEMooECDMooCkWxXf}\,\ref{ITEMooOIWTooYqmMPP}.}, \( \varphi(a^{-1})=\varphi(a)^{-1}\).

	Pour l'injectivité nous supposons \( \varphi(a)=\varphi(b)\). Étant donné que \( \eK'\) est un corps, nous pouvons multiplier par \( \varphi(b)^{-1}\) :
	\begin{equation}
		\varphi(a)\varphi(b)^{-1}=1.
	\end{equation}
	En utilisant le premier point nous avons \( 1=\varphi(a)\varphi(b^{-1})\), puis le morphisme d'anneaux : \( 1=\varphi(ab^{-1})\), et encore le morphisme d'anneaux nous permet de déduire \( ab^{-1}=1\) et donc \(a=b\).
\end{proof}




%+++++++++++++++++++++++++++++++++++++++++++++++++++++++++++++++++++++++++++++++++++++++++++++++++++++++++++++++++++++++++++ 
\section{Symbole de sommation}
%+++++++++++++++++++++++++++++++++++++++++++++++++++++++++++++++++++++++++++++++++++++++++++++++++++++++++++++++++++++++++++

%--------------------------------------------------------------------------------------------------------------------------- 
\subsection{Somme à valeurs dans un groupe commutatif}
%---------------------------------------------------------------------------------------------------------------------------

Si \( S\) est un ensemble fini, nous savons de la proposition \ref{PROPooJLGKooDCcnWi} qu'il existe un unique \( N\in \eN\) pour lequel il existe une bijection \( \varphi\colon \{ 0,\ldots, N \}\to S\). Cette bijection n'est à priori pas unique.

\begin{lemmaDef}[\cite{MonCerveau}]       \label{DEFooLNEXooYMQjRo}
	Soient un groupe commutatif \( (G,+)\) ainsi qu'un ensemble fini \( I\) contenant \( n\) éléments. Soit une application \( f\colon I\to G \). Si \( \sigma_1,\sigma_2\colon \{1,\ldots, n \}\to I\) sont deux bijections, alors\footnote{Pour rappel, le symbole \( \sum_{i=1}^n\) est défini par \ref{DEFooNEVNooJlmJOC}.}
	\begin{equation}
		\sum_{i=1}^nf\big( \sigma_1(i) \big)=\sum_{i=1}^nf\big( \sigma_2(i) \big).
	\end{equation}
	La valeur commune est notée
	\begin{equation}
		\sum_{i\in I}f(i)
	\end{equation}
\end{lemmaDef}

\begin{proof}
	Nous commençons par considérer une transposition \( \sigma\) (qui permute \( k\) et \( l\) avec \( k<l\)). Nous avons
	\begin{subequations}
		\begin{align}
			\sum_{i=1}^nf(i) & =\sum_{i=1}^{k-1}f(i)+f(k)+\sum_{i=k+1}^{l-1}f(i)+f(l)+\sum_{i=l+1}^nf(i) \\
			                 & =\sum_{i=1}^{k-1}f(i)+f(l)+\sum_{i=k+1}^{l-1}f(i)+f(k)+\sum_{i=l+1}^nf(i) \\
			                 & =\sum_{i=1}^nf\big( \sigma(i) \big).
		\end{align}
	\end{subequations}
	Pour cela nous avons utilisé le fait que \( G\) est commutatif pour permuter \( f(l)\in G\) et \( f(k)\in G\) avec \( \sum_{i=k+1}^{l-1}f(i)\in G\).

	Une permutation quelconque est un produit de telles transpositions (proposition \ref{PropPWIJbu}). Donc pour toute permutation \( \sigma\) nous avons
	\begin{equation}
		\sum_{i=1}^nf\big( \sigma(i) \big)=\sum_{i=1}^nf(i).
	\end{equation}
\end{proof}

La définition \ref{DEFooLNEXooYMQjRo} donne lieu à un certain nombre de remarques.
\begin{enumerate}
	\item
	      Elle donne la somme sur un ensemble fini. Un problème avec les ensembles infinis (outre la convergence) est l'ordre de sommation. Si vous voulez sommer sur \( \eZ\), dans quel ordre le faire ?
	\item
	      Pour aller plus loin, et sommer sur des ensembles infinis, rendez-vous dans le thème \ref{THEMEooMKLBooLGFCdx}.
\end{enumerate}

\begin{proposition}     \label{PROPooJBQVooNqWErk}
	Soient un groupe commutatif \( (G,+)\), un ensemble fini \( I\), une application \( f\colon I\to G\) et une bijection \( \sigma\colon I\to I\). Alors
	\begin{equation}
		\sum_{i\in I}f(i)=\sum_{i\in I}f\big( \sigma(i) \big).
	\end{equation}
\end{proposition}

Si nous avons une application \( L\colon S\to S\), nous notons
\begin{equation}
	\sum_{s\in S}f\big( L(s) \big)=\sum_{s\in S}(f\circ L)(s).
\end{equation}
Cette façon d'écrire donne une interprétation pour la notation \( \sum_{g\in G}f(hg)\) qui arrive dans la proposition \ref{PROPooWJQQooFINSEc}. Il s'agit de considérer l'application \( L_h\) du lemme \ref{LEMooBIBFooBHxFYC}, de considérer\footnote{Le fait que \( L_h\) soit une bijection n'a pas d'importance ici.}
\begin{equation}        \label{EQooQQBEooFDOBVG}
	\sum_{g\in G}f(hg)=\sum_{g\in G}(f\circ L_h)(g)
\end{equation}
et de faire tourner la définition \ref{DEFooLNEXooYMQjRo}. La même chose tient pour définir \( \sum_{g\in G}(gh)\) à l'aide de \( R_h\).


\begin{lemma}[Changement de variables dans une somme\cite{MonCerveau}]		\label{LEMooGAMAooOAFhrc}
	Soient deux ensembles finis \( I,J\) ainsi qu'une bijection \(\varphi \colon I\to J  \). Soient un groupe abélien \( G\) et une application \(f \colon I\to G  \). Alors
	\begin{equation}
		\sum_{i\in I}f(i)=\sum_{j\in J}f\big( \varphi^{-1}(j) \big).
	\end{equation}
\end{lemma}

\begin{lemma}
	Soit un ensemble \( A\) fini pouvant être écrit comme une union disjointe \( A=\bigcup_{k=1}^nA_k\); nous supposons que les \( A_i\) sont non vides. Soient un groupe commutatif \( (G,+)\) et une application \( f\colon A\to G\). Alors
	\begin{equation}
		\sum_{a\in A}f(a)=\sum_{k=1}^n\sum_{a\in A_k}f(a).
	\end{equation}
\end{lemma}


\begin{proof}
	Le lemme \ref{LEMooTUIRooEXjfDY} nous indique que les parties \( A_k\) sont des ensembles finis. Nous notons
	\begin{enumerate}
		\item
		      \( N_0=0\), et \( N_k=\Card(A_k)\),
		\item
		      \( S_k=\sum_{i=1}^kN_k\).
		\item
		      \( \varphi_k\colon \{ 1,\ldots, N_k \}\to A_k\), une bijection (l'existence est dans la proposition \ref{PROPooJLGKooDCcnWi}).
	\end{enumerate}
	Nous avons \( \Card(A)=S_n\) par le lemme \ref{LEMooVFPNooVmdUXY}\ref{ITEMooSWJCooEpBVkG}. Nous définissons une belle bijection comme il faut :
	\begin{equation}
		\begin{aligned}
			\alpha\colon \{ 1,\ldots, S_n \} & \to A                        \\
			i                                & \mapsto \varphi_{k+1}(i-S_k)
		\end{aligned}
	\end{equation}
	pour \( i\in\mathopen] S_k , S_{k+1} \mathclose]\).

	\begin{subproof}
		\spitem[\( \alpha\) est bien définie]
		Puisque \( i>S_k\) et \( i\leq S_{k+1}\) nous avons \( i-S_k\in \{ 1,\ldots, N_{k+1} \}\), et donc \( \varphi_{k+1}\) s'applique bien à \( i-S_k\).
		\spitem[\( \alpha\) est injective]
		Supposons que \( \alpha(i)=\alpha(j)\). Si \( i\in \mathopen] S_k , S_{k+1} \mathclose]\) et \( j\in \mathopen] S_l , S_{l+1} \mathclose]\), alors \( \alpha(i)=\varphi_{k+1}(i-S_k)\in A_{k+1}\) et \( \alpha(j)=\varphi_{l+1}(j-S_l)\in A_{l+1}\). Vu que les \( A_i\) sont disjoints, nous avons \( k=l\), et donc
		\begin{equation}
			\varphi_{k+1}(i-S_k)=\varphi_{k+1}(j-S_k).
		\end{equation}
		Étant donné que \( \varphi_{k+1}\) est injective, nous avons \( i-S_k=j-S_k\), ce qui montre que \( i=j\).
		\spitem[\( \alpha\) est surjective]
		Soit \( a\in A\). Il existe \( k\) tel que \( a\in A_k\). Nous avons donc un \( s\in\{ 1,\ldots, N_k \}\) tel que \( a=\varphi_k(s)\). En posant \( i=s+S_k\), nous avons bien \( a=\alpha(s+S_k)\) parce que \( s+S_k\in \mathopen] S_{k-1} , S_k \mathclose]\).
	\end{subproof}
	Vu que \( \alpha\) est une bijection, nous avons l'égalité
	\begin{equation}
		\sum_{a\in A}f(a)=\sum_{i=1}^{S_n}(f\circ \alpha)(i).
	\end{equation}

	Nous avons encore besoin d'introduire une bijection. Nous posons
	\begin{equation}
		\begin{aligned}
			\beta_k\colon \mathopen] S_{k-1} , S_k \mathclose] & \to A_k                       \\
			i                                                  & \mapsto \varphi_k(i-S_{k-1}).
		\end{aligned}
	\end{equation}
	C'est une bijection parce que \( \varphi_k\) en est une, et que \( i\mapsto i-S_{k-1}\) est une bijection de \( \mathopen] S_{k-1} , S_k \mathclose]\).

	Nous pouvons maintenant terminer :
	\begin{subequations}
		\begin{align}
			\sum_{a\in A}f(a) & =\sum_{i=1}^{S_n}(f\circ \alpha)(i)                                                                            \\
			                  & =\sum_{k=1}^n\left( \sum_{i=S_{k-1}-1}^{S_k}(f\circ \alpha)(i) \right)        \label{SUBEQooNVKWooZqBAau}      \\
			                  & =\sum_{k=1}^n\left( \sum_{i\in \mathopen] S_{k-1} , S_k \mathclose]}f\big( \varphi_k(i-S_{k-1}) \big)  \right) \\
			                  & =\sum_{k=1}^n\left( \sum_{i\in \mathopen] S_{k-1} , S_k \mathclose]}f\big( \beta_k(i) \big) \right)            \\
			                  & =\sum_{i=1}^n\left( \sum_{a\in A_k}f(a) \right).
		\end{align}
	\end{subequations}
	Justifications :
	\begin{itemize}
		\item Pour \eqref{SUBEQooNVKWooZqBAau}. Associativité de la somme.
	\end{itemize}
\end{proof}


\begin{proposition}[\cite{MonCerveau}]     \label{PROPooWJQQooFINSEc}
	Soient un groupe fini \( G\) et une fonction \( f\colon G\to A\) où \( A\) est un anneau commutatif. Alors
	\begin{equation}
		\sum_{g\in G}f(g)=\sum_{g\in G}f(gh)=\sum_{g\in G}f(hg)
	\end{equation}
	pour tout \( h\in G\).
\end{proposition}

\begin{proof}
	Nous avons une bijection \( \varphi\colon \{ 0,\ldots,  N \}\to G\) garantie par la proposition \ref{PROPooJLGKooDCcnWi}. Sa définition est
	\begin{equation}
		\sum_{g\in G}f(g)=\sum_{i=0}^Nf\big( \varphi(i) \big).
	\end{equation}
	Par ailleurs, le lemme \ref{LEMooBIBFooBHxFYC} donne une bijection \( L_h\colon G\to G\) et permet de considérer la composée
	\begin{equation}
		\begin{aligned}
			\varphi'\colon \{ 0,\ldots,  N \} & \to G \\
			\varphi'=L_h\circ \varphi.
		\end{aligned}
	\end{equation}
	La proposition \ref{DEFooLNEXooYMQjRo} nous permet d'utiliser la bijection \( \varphi'\) au lieu de \( \varphi\) pour exprimer la somme \( \sum_{g\in G}\). Ensuite un jeu de notation utilisant \eqref{EQooQQBEooFDOBVG} donne
	\begin{equation}
		\begin{aligned}[]
			 & \sum_{g\in G}f(g)=\sum_{i=0}^Nf\big( \varphi(i) \big)=\sum_{i=0}^Nf\big( \varphi'(i) \big)=\sum_{i=0}^N(f\circ L_h\circ \varphi)(i) \\
			 & \quad=\sum_{i=0}^N(f\circ L_h)\big( \varphi(i) \big)=\sum_{g\in G}(f\circ L_h)(g)=\sum_{g\in G}f(hg).
		\end{aligned}
	\end{equation}
	En ce qui concerne \( \sum_{g\in G}f(gh)\), c'est la même chose, en utilisant \( R_h\) au lieu de \( L_h\).
\end{proof}

\begin{lemma}       \label{LEMooKSVWooIFsfwm}
	Soit un groupe totalement ordonné\footnote{Définition \ref{DEFooEUHFooYvhnLQ}.} \( (A,+,\leq)\). Soient deux suites \( (a_i)\) et \( (b_i)\) dans \( G\) telles que \( a_i\leq b_i\) pour tout \( i\). Alors pour tout \( n\) nous avons
	\begin{equation}
		\sum_{i=0}^na_i\leq \sum_{i=0}^nb_i.
	\end{equation}
\end{lemma}

Tout cela nous permet de définir une somme sympathique et bien connue.
\begin{lemma}
	Soit \( n\in \eN\). Nous avons
	\begin{equation}
		\sum_{k=0}^nk=\frac{ n(n+1) }{ 2 }.
	\end{equation}
\end{lemma}

\begin{proof}
	La preuve est pratiquement immédiate par récurrence. Nous allons donner une preuve plus «constructive», qui formalise l'idée classique d'écrire la somme à l'endroit et à l'envers.


	Nous notons \( S\) la somme \( \sum_{k=0}^nk\). Le lemme \ref{DEFooLNEXooYMQjRo} dit que si les \( \sigma_i\colon \{ 0,\ldots, n \}\to \{ 0,\ldots, n \}\) sont des bijections, alors \( \sum_{k=0}^nf\big( \sigma_1(k) \big)=\sum_{k=0}^nf\big( \sigma_2(k) \big)\). Nous sommes intéressé au cas \( f(i)=i\).

	En prenant \( \sigma_1(k)=k\) et \( \sigma_2(k)=n-k\), nous avons
	\begin{equation}
		S=\sum_{k=0}^nk=\sum_{k=0}^n(n-k).
	\end{equation}
	Donc
	\begin{equation}
		2S=\sum_{k=0}^n\big( k+(n-k) \big)=\sum_{k=0}^nn=n\sum_{k=0}^n1=n(n+1).
	\end{equation}
	En divisant par deux, nous obtenons le résultat annoncé.
\end{proof}


%+++++++++++++++++++++++++++++++++++++++++++++++++++++++++++++++++++++++++++++++++++++++++++++++++++++++++++++++++++++++++++ 
\section{Symbole de produit}
%+++++++++++++++++++++++++++++++++++++++++++++++++++++++++++++++++++++++++++++++++++++++++++++++++++++++++++++++++++++++++++

\begin{normaltext}      \label{NORMooDBOFooQCwbOY}
	Si \( (G,\cdot)\) est un groupe et si \( H\subset G\), nous notons le produit des éléments de \( H\) par
	\begin{equation}
		\prod_{g\in H }g=\sum_{g\in H}g
	\end{equation}
	où à droite, c'est la somme déjà définie. La différence entre \( \prod\) et \( \sum\) est que nous utilisons \( \prod\) pour les groupes notés «multiplicativement» comme \( (G,\cdot)\) alors que nous utilisons \( \sum\) lorsque le groupe est noté «additivement» comme \( (G,+)\).

	Dans le cas d'un anneau \( (A,+,\cdot)\), la distinction est importante pour savoir quelle opération est sous-entendue.

	La définition \ref{DEFooNEVNooJlmJOC}\ref{ITEMooIPDTooEhOxea} signifie qu'une somme vide vaut zéro : \( \sum_{x\in \emptyset}x=0\). Vu que zéro est la façon usuelle de noter le neutre pour une opération notée «\( +\)», lorsque l'opération est notée \( \cdot\) nous avons
	\begin{equation}        \label{EQooCSDSooTxdfzO}
		\prod_{x\in\emptyset}x=1
	\end{equation}
	parce que \( 1\) est la façon usuelle de noter le neutre d'une opération notée «\( \cdot\)».

	Notez que \eqref{EQooCSDSooTxdfzO} n'est pas une nouvelle définition ou une nouvelle convention. C'est seulement l'égalité \( \sum_{x\in\emptyset x}x=0\), avec des notations adaptées à un groupe dont l'opération est notée multiplicativement.
\end{normaltext}

\begin{proposition}     \label{PROPooQMUDooQQVRIe}
	Si \( E\) est un ensemble fini et si \( G\) est un groupe commutatif, alors pour toute fonction \( f\colon E\to G\) et pour toute permutation\footnote{Une permutation est une bijection, définition \ref{DEFooJNPIooMuzIXd}.} \( \sigma\) de \( E\),
	\begin{equation}
		\prod_{i\in E}f(i)=\prod_{i\in E}f\big( \sigma(i) \big)
	\end{equation}
\end{proposition}

\begin{proof}
	C'est exactement la proposition \ref{DEFooLNEXooYMQjRo}, sauf qu'ici la loi de groupe est notée multiplicativement au lieu d'additivement.
\end{proof}

%--------------------------------------------------------------------------------------------------------------------------- 
\subsection{Sous-groupe engendré}
%---------------------------------------------------------------------------------------------------------------------------


\begin{definition}[Sous-groupe engendré]          \label{DefooRDRXooEhVxxu}
	Soit \( A\) une partie du groupe \( G\). Le sous-groupe \defe{engendré}{sous-groupe!engendré}\index{engendré!sous-groupe} par \( A\) est l'intersection de tous les sous-groupes de \( G\) contenant \( A\). Nous notons ce groupe \( \gr_G(A)\)\nomenclature[R]{\( \gr_G\)}{groupe engendré}.

	Lorsque \( A \) est fini (disons \( A = \{a_1, \dots, a_n\} \)), on note aussi le sous-groupe engendré \( \langle a_1, \dots, a_n \rangle \).
\end{definition}

\begin{normaltext}
	Un sous-groupe engendré n'est jamais vide parce qu'il contient toujours au moins le neutre (parce que c'est un sous-groupe). Si \( G\) est un groupe, le sous-groupe \( \gr_G(\emptyset)\) lui-même contient \( e\)\footnote{Demandez-vous si il est possible que \( \gr(\emptyset)\) contienne d'autres éléments que \( e\).}.
\end{normaltext}

\begin{normaltext}
	Dans de nombreux cas, le groupe «ambiant» \( G\) est entendu par le contexte et nous noterons \( \gr(A)\) au lieu de \( \gr_G(A)\).

	Si par exemple \( A\) est la matrice \( \begin{pmatrix}
		4 & 5 \\
		6 & 7
	\end{pmatrix}\), le groupe \( \gr(A)\) est à comprendre dans \( \GL(2,\eR)\). Il faudrait être fou pour avoir en tête un autre groupe que \( \GL(2,\eR)\) sans le préciser.

	D'ailleurs, connaissez-vous un groupe contenant la matrice \( A\) et n'étant pas un sous-groupe de \( \GL(2,\eC)\) ?
\end{normaltext}

\begin{lemma}
	Si \( G\) est un groupe et \( A\) une partie de \( G\), alors \( \gr(A)\) est un sous-groupe de \( G\).
\end{lemma}

Le sous-groupe engendré par \( A \) est le plus petit (pour l'inclusion) groupe de \( G\) contenant \( A\). Plus formellement, nous avons le résultat suivant :
\begin{lemma}
	Tout sous-groupe de \( G\) contenant \( A\) contient \( \gr(A)\).
\end{lemma}

\begin{proof}
	Si \( H\) est un sous-groupe de \( G\) contenant \( A\), alors \( \gr(A)\) est l'intersection de \( H\) avec tous les autres sous-groupes de \( G\) contenant \( A\). Il contient donc \( \gr(A)\).
\end{proof}

\begin{lemma}[\cite{BIBooERNQooTXQPvD}]   \label{LemFUIZooBZTCiy}
	Si \( A\) est une partie du groupe \( G\), alors le sous-groupe \( \gr(A)\) engendré\footnote{Définition~\ref{DefooRDRXooEhVxxu}.} par \( A\) est l'ensemble de tous les produits finis d'éléments de \( A\) et de \( A^{-1}\) (l'identité est le produit à zéro éléments).

	C'est-à-dire que tout élément de \( \gr(A)\) peut être écrit sous la forme\footnote{Les \( a_i\) négatifs correspondent aux inverses. Notons que si \( g\in A\), il n'y a pas de garanties que \( g^{-1}\) soit également dans \( A\).}
	\begin{equation}
		\prod_{i=1}^ng_i^{a_i}
	\end{equation}
	où \( a_i\in \eZ\) et \( g\colon \eN\to A\) n'est pas spécialement injective : il peut arriver que \( g_i=g_j\).
\end{lemma}

\begin{proof}
	Puisqu'un produit vide est égal à l'identité\footnote{Voir \ref{NORMooDBOFooQCwbOY}.}, le lemme est vrai (un peu trivialement) dans le cas où \( A=\emptyset\). À partir de maintenant, nous supposons que \( A\) est non vide.

	Nous nommons \( \gr(A)\) le groupe engendré par \( A\) et \( H\), l'ensemble
	\begin{equation}
		H=\{ g_1\ldots g_n\tq g_i\in A\cup A^{-1} \}.
	\end{equation}
	Nous commençons par prouver que \( H\) est un groupe.
	\begin{itemize}
		\item Puisque \( A\) est non vide, nous considérons \( a\in A\). Dans ce cas, \( e=aa^{-1}\in H\). Donc \( e\in H\).
		\item L'inverse de \( g_1\ldots g_n\) est \( g_n^{-1}\ldots g_1^{-1}\) qui est également dans \( H\).
		\item Le produit de \( g_1\ldots g_n\) par \( h_1\ldots h_n\), tous éléments de \( H\), est également dans \( H\)\footnote{Et c'est ici qu'on se rend compte que la décomposition n'est probablement que rarement unique.}.
	\end{itemize}
	Comme \( H\) est un groupe contenant \( A\), nous avons \( \gr(A)\subset H\) parce que \( \gr(A)\) est une intersection dont un des éléments est \( H\).

	Par ailleurs tout groupe contenant \( A\) doit contenir les inverses et les produits finis, donc \( H\subset \gr(A)\).

	Au final, \( H=\gr(A)\), ce qu'il fallait.
\end{proof}

\begin{lemma}       \label{LEMooCFTVooKvmyKN}
	Soit un groupe \( G\) et un sous-groupe \( H=\gr(h_1,\ldots, h_n)\). Si \( \alpha\in G\), alors
	\begin{equation}
		\alpha H\alpha^{-1}=\gr(\alpha h_1\alpha^{-1},\ldots, \alpha h_n\alpha^{-1}).
	\end{equation}
\end{lemma}

\begin{proof}
	Il s'agit d'une conséquence du lemme \ref{LemFUIZooBZTCiy}. Un élément de \( \gr(\alpha h_1\alpha^{-1},\ldots, \alpha h_n\alpha^{-1})\) est un produit d'éléments de \( G\) de la forme \( \alpha h_i\alpha^{-1}\) ou \( (\alpha h_j\alpha^{-1})^{-1}=\alpha h_j^{-1}\alpha^{-1}\). Or nous avons
	\begin{equation}
		\alpha h_i\alpha^{-1}\alpha h_j\alpha^{-1}=\alpha h_ih_j\alpha^{-1}\in \alpha H\alpha^{-1}.
	\end{equation}
	Donc
	\begin{equation}
		\gr(\alpha h_1\alpha^{-1},\ldots, \alpha h_n\alpha^{-1})\subset \alpha H\alpha^{-1}.
	\end{equation}
	L'inclusion dans l'autre sens est du même tonneau.
\end{proof}

\begin{definition}[Partie génératrice, groupe monogène]  \label{DEFooWMFVooLDqVxR}
	Soient un groupe \( G\), et une partie \( A\subset G\). Si \( \gr(A)=G\), alors nous disons que \( A\) est une \defe{partie génératrice}{partie!génératrice} du groupe \( G\).

	Un groupe est \defe{monogène}{monogène} si il a une partie génératrice réduite à un seul élément.
\end{definition}

\begin{definition}[Groupe cyclique]     \label{DefHFJWooFxkzCF}
	Un élément \( a\in G\) est un \defe{générateur}{générateur} de \( G\) si tous les éléments de \( G\) s'écrivent sous la forme \( a^n\) pour un certain \( n\in\eZ\). Un groupe fini et monogène est dit \defe{cyclique}{groupe cyclique}.
\end{definition}

\begin{normaltext}
	La différence entre un groupe monogène et un groupe cyclique est qu'un groupe cyclique est fini. Dans un groupe cyclique, à force d'itérer le générateur, nous finissons par tourner en rond -- d'où le nom.
\end{normaltext}

\begin{example}
	Soit le groupe \( \big( \eZ/10\eZ,+ \big)\). L'élément \( [2]_{10}\) n'est pas générateur parce que ses puissances\footnote{Attention aux notations; en général on écrit la loi de groupe de façon multiplicative et on parle des puissances d'un élément, mais ici on écrit la loi de groupe additivement, donc les «puissances» sont en réalité les multiples.} sont
	\begin{equation}
		\gr([2]_{10})=\{ [2]_{10},[4]_{10},[6]_{10},[8]_{10},[0]_{10} \}.
	\end{equation}
	Par contre l'élément \( [3]_{10}\) est générateur : ses puissances sont dans l'ordre
	\begin{equation}
		[3]_{10}, [6]_{10}, [9]_{10}, [2]_{10}, [5]_{10}, [8]_{10},[1]_{10},[4]_{10},[7]_{10},[0]_{10}.
	\end{equation}
\end{example}

Un exemple presque identique, mais un peu masqué sera l'exemple \ref{EXooOXAAooZMdDfP}.


\begin{lemma}[\cite{MonCerveau}]        \label{LEMooTHBIooJPjBlp}
	Si \( n=dr\), alors
	\begin{enumerate}
		\item
		      \( r\eZ/n\eZ\) est un groupe.
		\item   \label{LEMooSAAXooMezYui}
		      Le groupe \( r\eZ/n\eZ\) est cyclique\footnote{Définition \ref{DefHFJWooFxkzCF}.}.
		\item       \label{ITEMooGCGQooIfZULt}
		      \( \Card(r\eZ/n\eZ)=d=n/r\).
	\end{enumerate}
\end{lemma}

%+++++++++++++++++++++++++++++++++++++++++++++++++++++++++++++++++++++++++++++++++++++++++++++++++++++++++++++++++++++++++++
\section{Module sur un anneau}
%+++++++++++++++++++++++++++++++++++++++++++++++++++++++++++++++++++++++++++++++++++++++++++++++++++++++++++++++++++++++++++

\begin{definition}[module sur un anneau\cite{ooJGVOooSjQBVh}]       \label{DEFooHXITooBFvzrR}
	Soit un anneau \( A\). Un \defe{module à gauche}{module!à gauche} sur \( A\) est la donnée d'un triplet \( (M,+,\cdot)\) où
	\begin{enumerate}
		\item
		      \( +\) est une loi de composition interne à \( M\), c'est-à-dire \( +\colon M\times M\to M\),
		\item
		      \( \cdot\) est une loi de composition externe, c'est-à-dire \( \cdot\colon A\times M\to M\)
	\end{enumerate}
	telles que
	\begin{enumerate}
		\item
		      \( (M,+)\) est un groupe\footnote{Nous verrons dans la proposition~\ref{PROPooGARGooDiMqtN} qu'il est forcément commutatif.}.
		\item
		      \( a\cdot(x+y)=a\cdot x+a\cdot y\),
		\item
		      \( (a+b)\cdot x=a\cdot x+b\cdot x\),
		\item
		      \( (ab)\cdot x=a\cdot(b\cdot x)\)
		\item
		      \( 1\cdot x=x\).
	\end{enumerate}
	pour tout \( a,b\in A\) et \( x,y\in M\).

	Si \( M\) et \( N\) sont des \( A\)-modules, un \defe{morphisme}{morphisme de modules} de \( M\) vers \( N\) est une application \( f\colon M\to N\) qui
	\begin{enumerate}
		\item
		      est un morphisme de groupes entre \( (M,+)\) et \( (N,+)\)
		\item
		      vérifie \( f(a\cdot x)=a\cdot f(x)\) pour tout \( a\in A\), \( x\in M\).
	\end{enumerate}
	L'ensemble des morphismes entre \( M\) et \( N\) est noté \( \Hom_A(M,N)\). Si \( B\) st une sous-anneau de \( A\),  nous parlons de \( \Hom_B(M,N)\) pour parler des morphismes de groupes qui ne vérifient \( f(a\cdot x)=a\cdot f(x)\) que pour \( a\in B\).
\end{definition}

\begin{proposition}\label{PROPooGARGooDiMqtN}
	Si \( M\) est un module sur un anneau, alors \( (M,+)\) est un groupe commutatif.
\end{proposition}

\begin{proof}
	Il suffit de calculer \( (1+1)\cdot (x+y)\) de deux façons différentes :
	\begin{equation}
		(1+1)\cdot (x+y)=1\cdot (x+y)+1\cdot (x+y)=x+y+x+y
	\end{equation}
	d'une part et
	\begin{equation}
		(1+1)\cdot (x+y)=(1+1)\cdot x+(1+1)\cdot y=x+x+y+y,
	\end{equation}
	d'autre part. En égalant les deux expressions, il vient
	\begin{equation}
		x+y+x+y=x+x+y+y,
	\end{equation}
	qui se simplifie (nous sommes dans un groupe) en \( y+x=x+y\).
\end{proof}

\begin{definition}\label{DEFooKHWZooIfxdNc}
	Un \defe{espace vectoriel}{espace!vectoriel} est un module\footnote{Définition \ref{DEFooHXITooBFvzrR}.} sur un corps commutatif\footnote{La condition de commutativité n'est pas indispensable, mais comme nous ne parlerons que de corps commutatifs\ldots}.
\end{definition}

\begin{definition}[\cite{BIBooSTWWooItiMUp}]        \label{DEFooRUKVooLnXxdS}
	Soient un \( A\)-module \( M\) et un ensemble \( I\). Une famille \( \{ m_i \}_{i\in I}\) est \defe{libre}{partie libre!module} si les \( m_i\) sont \defe{linéairement indépendants}{linéairement indépendant!module}, c'est-à-dire si pour tout choix d'une partie finie \( J\) dans \( I\) et d'éléments \( (a_j)_{j\in J}\) dans \( A\), si nous avons
	\begin{equation}
		\sum_{j\in J}a_jm_j=0,
	\end{equation}
	alors \( a_j=0\) pour tout \( j\).
\end{definition}

\begin{definition}[\cite{BIBooNKWVooYfrwSd}]        \label{DEFooWBOBooJNyyBF}
	Soit \( S\), une partie du \( A\)-module \( M\). Le \defe{sous-module engendré}{sous-module engendré} par \( S\) est l'ensemble des éléments de \( M\) qui sont des combinaisons linéaires finies d'éléments de \( S\), c'est-à-dire de sommes de la forme
	\begin{equation}
		\sum_{t\in T}a_tt
	\end{equation}
	où \( T\) est fini dans \( S\) et \( a_t\in A\).
\end{definition}

%--------------------------------------------------------------------------------------------------------------------------- 
\subsection{Module produit}
%---------------------------------------------------------------------------------------------------------------------------

\begin{lemmaDef}[\cite{BIBooSTWWooItiMUp}]        \label{DEFooLCJEooBvVmkV}
	Soient un anneau \( A\) et un ensemble \( I\). Le \( A\)-\defe{module produit}{module produit} \( A^I\) est l'ensemble des applications \( I\to A\).

	En termes de notations, nous écrivons ceci :
	\begin{equation}
		A^I=\{ (a_i)_{i\in I},a_i\in A \}.
	\end{equation}
	L'ensemble \( A^I\) devient un module par les définitions, pour \( x,y\in A^I\) et \( a\in A\) :
	\begin{subequations}
		\begin{align}
			ax  & =(ax_i)_{i\in I}                                  \\
			x+y & =(x_i+y_i)_{i\in I}     \label{EQooODBMooQKLUgd}.
		\end{align}
	\end{subequations}
	En d'autres termes, \( A^I=\Fun(I,A)\).
\end{lemmaDef}

\begin{lemma}
	Pour chaque \( i\in I\) nous considérons l'élément \( e_i\in A^I\) donné par
	\begin{equation}
		\begin{aligned}
			e_i\colon I & \to A                      \\
			j           & \mapsto \begin{cases}
				                      1 & \text{si } j=i \\
				                      0 & \text{sinon. }
			                      \end{cases}
		\end{aligned}
	\end{equation}
	La famille \( \{ e_i \}_{i\in I}\) est libre\footnote{Définition \ref{DEFooRUKVooLnXxdS}.} dans \( A^I\).
\end{lemma}

\begin{proof}
	Soient \( J\) fini dans \( I\) ainsi que des éléments \( a_j\in A\) (\( j\in J\)). Nous supposons que\footnote{Pour rappel, les sommes finies sont définies par \ref{DEFooLNEXooYMQjRo}.} \( \sum_{j\in J}a_je_j=0\). Calculons un peu :
	\begin{equation}
		\sum_{j\in J}a_je_j=\sum_{j\in J}(a_j\delta_{ji})_{i\in I}=\left( \sum_{j\in J}a_j\delta_{ji} \right)_{i\in I}.
	\end{equation}
	Pour que le tout soit nul dans \( A^I\), il faut que
	\begin{equation}
		\sum_{j\in J}a_j\delta_{ji}
	\end{equation}
	soit nul pour tout \( i\in I\). Si nous fixons \( i\in I\), la somme sur \( j\) possède un seul terme non annulé par \( \delta_{ji}\), et c'est le terme \( j=i\). Nous avons donc \( a_i=0\).
\end{proof}

\begin{definition}      \label{DEFooBMEPooFsCHgb}
	Nous notons \( A^{(I)}\) le sous-module de \( A^I\) engendré\footnote{Définition \ref{DEFooWBOBooJNyyBF}.} par les \( e_i\).
\end{definition}

\begin{lemma}[\cite{MonCerveau}]
	L'ensemble \( A^{(I)}\) est l'ensemble des applications \( I\to A\) de support fini.
\end{lemma}

\begin{proof}
	En deux sens.
	\begin{subproof}
		\spitem[Si \( x\in A^{(I)}\)]
		Pour rappel, la définition \ref{DEFooLCJEooBvVmkV} nous dit que \( x\) est une application \( I\to A\). Vu que \( x\) est dans le sous-module engendré par les \( e_i\), il existe une partie finie \( J\subset I\) telle que
		\begin{equation}
			x=\sum_{j\in J}x_je_j.
		\end{equation}
		Pour \( i\in I\) nous avons
		\begin{equation}
			x(i)=\sum_{j\in J}x_j\delta_{ij}=\begin{cases}
				x_i & \text{si } i\in J \\
				0   & \text{sinon. }
			\end{cases}
		\end{equation}
		Donc le support de \( x\) est dans \( J\) qui est fini. Vu que toute partie d'un ensemble fini est fini (lemme \ref{LEMooTUIRooEXjfDY}), le support de \( x\) est fini.
		\spitem[Si \( x\) est de support fini]
		Supposons que le support de \( x\colon I\to A\) soit la partie finie \( J\subset I\). En notant \( x_j=x(j)\) pour tout \( j\in J\), nous avons
		\begin{equation}
			x=\sum_{j\in J}x_je_j.
		\end{equation}
	\end{subproof}
\end{proof}

\begin{theorem}[Propriété universelle de \( A^{(I)}\)\cite{BIBooSTWWooItiMUp}]      \label{THOooPDZCooJnHbOd}
	Soient un anneau \( A\) ainsi qu'un \( A\)-module \( P\). Pour \( \phi\in\Hom_A(A^{(I)}, P)\), nous considérons
	\begin{equation}
		\begin{aligned}
			\phi|_I\colon I & \to P              \\
			i               & \mapsto \phi(e_i).
		\end{aligned}
	\end{equation}
	\begin{enumerate}
		\item

		      L'application
		      \begin{equation}
			      \begin{aligned}
				      f\colon \Hom_A(A^{(I)},P) & \to \Fun(I,P)   \\
				      \phi                      & \mapsto \phi|_I
			      \end{aligned}
		      \end{equation}
		      est une bijection.
		\item
		      L'application inverse est \( g\colon \Fun(I,P)\to \Hom_A(A^{(I)},P) \) donnée par
		      \begin{equation}
			      g(\psi)\big( \sum_{j\in J}a_je_j \big)=\sum_{j\in J}a_j\psi(j)
		      \end{equation}
		      pour tout \( J\) fini dans \( I\) et choix de \( a_j\in A\).
	\end{enumerate}
\end{theorem}

\begin{proof}
	Nous allons montrer que \( g\big( f(\phi) \big)=\phi\) et que \( f\big( g(\psi) \big)=\psi\) pour tout \( \phi\in\Hom_A(A^{(I)},P)\) et pour tout \( \psi\in \Fun(I,P)\).

	Dans un premier sens nous avons :
	\begin{subequations}
		\begin{align}
			g\big( f(\phi) \big)\big( \sum_ja_je_j \big) & =\sum_ja_jf(\phi)(j)                                       \\
			                                             & =\sum_ja_j\phi(e_j)\label{SUBALIGNooBWPLooHeIaQg}          \\
			                                             & =\phi(\sum_ja_je_j)        \label{SUBALIGNooUOQPooCwLgZo}.
		\end{align}
	\end{subequations}
	Justifications :
	\begin{itemize}
		\item
		      Pour \eqref{SUBALIGNooBWPLooHeIaQg}, nous avons utilisé le fait que \( f(\phi)(i)=\phi|_I(i)=\phi(e_i)\).
		\item
		      Pour \eqref{SUBALIGNooUOQPooCwLgZo}, nous utilisons le fait que \( \phi\) est un morphisme de modules.
	\end{itemize}
	Et pour l'autre sens,
	\begin{equation}
		f\big( g(\psi) \big)(i)=g(\psi)(e_i)=\psi(i).
	\end{equation}
	Vérifions que cela est suffisant pour que \( f\) soit une bijection.
	\begin{subproof}
		\spitem[Surjectif]
		Soit \( \psi\in \Fun(I,P)\). Nous avons \( f\big( g(\psi) \big)=\psi\), ce qui prouve que \( \psi\) est dans l'image de \( f\).
		\spitem[Injectif]
		Supposons que \( f(\phi_1)=f(\phi_2)\). Alors en appliquant \( g\) des deux côtés, il vient \( \phi_1=\phi_2\).
	\end{subproof}
\end{proof}

%--------------------------------------------------------------------------------------------------------------------------- 
\subsection{Sous-module}
%---------------------------------------------------------------------------------------------------------------------------

Soient \( M\) un \( A\)-module et \( x=(x_i)_{i\in I}\) une famille d'éléments de \( M\) paramétrée par l'ensemble \( I\). Nous considérons l'application
\begin{equation}
	\begin{aligned}
		\mu_x\colon A^{(I)} & \to M                        \\
		(a_i)_{i\in I}      & \mapsto \sum_{i\in I}a_ix_i.
	\end{aligned}
\end{equation}
Ici \( A^{(I)}\) désigne l'ensemble de toutes les applications \( I\to A\) de support fini (définition \ref{DEFooBMEPooFsCHgb}).

\begin{definition}      \label{DefBasePouyKj}
	À l'instar des espaces vectoriels, les modules ont une notion de partie libre, génératrice et de bases :
	\begin{enumerate}
		\item
		      Si \( \mu_x\) est surjective, nous disons que \( x\) est une partie \defe{génératrice}{génératrice!partie d'un module}.
		\item
		      Si \( \mu_x\) est injective, nous disons que la partie \( x\) est \defe{libre}{libre!partie d'un module}.
		\item
		      Si \( \mu_x\) est bijective, nous disons que la partie \( x\) est une \defe{base}{base!d'un module}.
	\end{enumerate}
\end{definition}

\begin{definition}
	Un sous-ensemble \( N\subset M\) est un \defe{sous-module}{sous-module} si \( (N,+)\) est un sous-groupe de \( (M,+)\) et si \( a\cdot x\in N\) pour tout \( x\in N\) et pour tout \( a\in A\).
\end{definition}

\begin{example}
	Un anneau \( A\) est lui-même un \( A\)-module et ses sous-modules sont les idéaux.
\end{example}

\begin{definition}
	Soit \( M\) un module sur un anneau commutatif \( A\). Un \defe{projecteur}{projecteur!dans un module} est une application linéaire \( p\colon M\to M\) telle que \( p^2=p\).

	Une famille \( (p_i)_{i\in I}\) sur \( M\) est \defe{orthogonale}{orthogonal!famille de projecteurs} si \( p_i\circ p_j=0\) pour tout \( i\neq j\). La famille est \defe{complète}{complète!famille de projecteurs} si \( \sum_{i\in I}p_i=\mtu\).
\end{definition}

\begin{theorem}     \label{ThoProjModpAlsUR}
	Soient des sous-modules \( M_1,\ldots,M_n\) du module \( M \) tels que \( M=M_1\oplus\ldots\oplus M_n\). Les applications \( p_i\) définies par
	\begin{equation}
		p_i(x_1+\ldots+x_n)=x_i
	\end{equation}
	forment une famille orthogonale de projecteurs et \( p_1+\cdots +p_n=\id\).

	Inversement, si \( (p_1,\ldots, p_n)\) est une famille orthogonale de projecteurs dans un module \( \modE\) tel que \( \sum_{i=1}^np_i=\id\), alors
	\begin{equation}
		M=\bigoplus_{i=1}^np_i(M).
	\end{equation}
\end{theorem}

\begin{definition}
	Un module est \defe{simple}{simple!module}\index{module!simple} ou \defe{irréductible}{irréductible!module}\index{module!irréductible} si il n'a pas d'autres sous-modules que \( \{ 0 \}\) et lui-même. Un module est \defe{indécomposable}{indécomposable!module}\index{module!indécomposable} si il ne peut pas être écrit comme somme directe de sous-modules.
\end{definition}

Un module simple est a fortiori indécomposable. L'inverse n'est pas vrai comme le montre l'exemple suivant.

\begin{example}
	Soit \( \modE=\eC[X]/(X^2)\) vu comme \( \eC[X]\)-module. C'est le \( \eC[X]\)-module des polynômes de la forme \( aX+b\) avec \( a,b\in \eC\). L'ensemble des polynômes de la forme \( aX\) est un sous-module. Le module \( \modE\) n'est donc pas simple. Il est cependant indécomposable parce que \( \{ aX \}\) est le seul sous-module non trivial. En effet si \( \modF\) est un sous-module de \( \modE\) contenant \( aX+b\) avec \( b\neq 0\), alors \( \modF\) contient \( X(aX+b)=bX\) et donc contient tout \( \modE\).
\end{example}

\begin{definition}[Algèbre\cite{ZSyHmiy}]   \label{DefAEbnJqI}
	Si \( \eK\) est un corps commutatif\footnote{Définition~\ref{DefTMNooKXHUd}}, une \( \eK\)-\defe{algèbre}{algèbre} \( A\) est un espace vectoriel\footnote{Définition~\ref{DEFooKHWZooIfxdNc}.} muni d'une opération bilinéaire \( \times\colon A\times A\to A\), c'est-à-dire telle que pour tout \( x,y,z\in A\) et pour tout \( \alpha,\beta\in\eK\),
	\begin{enumerate}
		\item
		      \( (x+y)\times z=x\times z+y\times z\)
		\item
		      \( x\times (y+z)=x\times y+x\times z\)
		\item
		      \( (\alpha x)\times (\beta y)=(\alpha\beta)(x\times y)\).
	\end{enumerate}
	Si \( A\) et \( B\) sont deux \( \eK\)-algèbres, une application \( f\colon A\to B\) est un \defe{morphisme d'algèbres}{morphisme!d'algèbres} entre \( A\) et \( B\) si pour tout \( x,y\in A\) et pour tout \( \alpha\in \eK\),
	\begin{enumerate}
		\item
		      \( f(xy)=f(x)f(y)\)
		\item
		      \( f(x+\alpha y)=f(x)+\alpha f(y)\)
	\end{enumerate}
	où nous avons noté \( xy\) pour \( x\times y\).
\end{definition}

\begin{lemma}[\cite{MonCerveau}]   \label{LEMooVKLKooSAHmpZ}
	Soient une algèbre \( A\) et une famille \( (X_i)_{i\in I}\) de sous-algèbres de \( A\) (ici \( I\) est un ensemble quelconque). Alors la partie \( X=\bigcap_{i\in I}X_i\) est une sous-algèbre de \( A\).
\end{lemma}

\begin{proof}
	Nous devons prouver que si \( x\) et \( y\) sont dans \( X\) et \( \lambda\in \eK\), alors \( xy\), \( x+y\) et \( \lambda x\) sont dans \( X\). Pour tout \( i\in I\) nous avons \( x,y\in X_i\) et donc \( xy\in X_i\), \( x+y\in X_i\) et \( \lambda x\in X_i\) (parce que \( X_i\) est une algèbre). Donc \( xy\), \( x+y\) et \( \lambda x\) sont dans \( X_i\) pour tout \( I\), et donc dans \( X\).
\end{proof}

\begin{definition}\label{DefkAXaWY}
	L'\defe{algèbre engendrée}{algèbre!engendrée} par \( X\) est l'intersection de toutes les sous-algèbres de \( A\) contenant \( X\) (qui est une algèbre par le lemme~\ref{LEMooVKLKooSAHmpZ}).
\end{definition}



%+++++++++++++++++++++++++++++++++++++++++++++++++++++++++++++++++++++++++++++++++++++++++++++++++++++++++++++++++++++++++++
\section{Caractéristique d'un anneau}
%+++++++++++++++++++++++++++++++++++++++++++++++++++++++++++++++++++++++++++++++++++++++++++++++++++++++++++++++++++++++++++

\begin{lemmaDef}        \label{LEMDEFooVEWZooUrPaDw}
	Soit l'application
	\begin{equation}
		\begin{aligned}
			\mu\colon \eZ & \to A              \\
			n             & \mapsto n\cdot 1_A
		\end{aligned}
	\end{equation}
	où \( n\cdot 1_A\) signifie \( \sum_{k=1}^n1_A\).
	\begin{enumerate}
		\item
		      C'est un morphisme d'anneaux.
		\item
		      Le noyau est un sous-groupe de \( \eZ\)
		\item
		      Il existe un unique \( p\in \eZ\) tel que \( \ker(\mu)=p\eZ\).
	\end{enumerate}
	Ce \( p\) est la \defe{caractéristique}{caractéristique!d'un anneau} de \( A\).
\end{lemmaDef}

Par exemple la caractéristique de \( \eQ\) est zéro parce qu'aucun multiple de l'unité n'est nul.

À propos de diagonalisation en caractéristique \( 2\), voir l'exemple~\ref{ExewINgYo}.

\begin{lemma}
	Si \( A\) est de caractéristique nulle, alors \( A\) est infini.
\end{lemma}

\begin{proof}
	En effet, \( \ker\mu=\{0\} \) implique que \( n1_A \neq  m1_A\) dès que \(n \neq m \) et par conséquent \( A\) contient \(\eZ 1_A \), et  est infini.
\end{proof}

\begin{lemma}       \label{LemHmDaYH}
	Soit un anneau \( A\) de caractéristique \( p\).
	\begin{enumerate}
		\item       \label{ITEMooPKSEooPKChGM}
		      Si \( p>0\), alors nous avons l'isomorphisme d'anneaux
		      \begin{equation}
			      \eZ 1_A\simeq\eZ/p\eZ.
		      \end{equation}
		\item        \label{ITEMooBTUIooYzOycc}
		      Si \( p=0\), alors nous avons l'isomorphisme d'anneaux
		      \begin{equation}
			      \eZ 1_A\simeq \eZ
		      \end{equation}
		      %TODOooZCJXooPsEHbI: démontrer cette seconde partie.
	\end{enumerate}
\end{lemma}

\begin{proof}
	Pour \ref{ITEMooPKSEooPKChGM}, l'isomorphisme est donné par l'application \( n1_A\mapsto \phi(n)\) si \( \phi\) est la projection canonique \( \eZ\to \eZ/p\eZ\).

	Pour \ref{ITEMooBTUIooYzOycc}, la preuve est encore à faire. Écrivez-moi.
\end{proof}

\begin{proposition}     \label{PropGExaUK}
	La caractéristique d'un anneau fini divise son cardinal.
\end{proposition}

\begin{proof}
	Si \( A\) est un anneau, le groupe \( \eZ\) agit sur \( A\) par
	\begin{equation}
		n\cdot a=a+n1_A.
	\end{equation}
	Chaque orbite de cette action est de la forme
	\begin{equation}
		\mO_a=\{ a+n1_A\tq n=0,\ldots, p-1 \}
	\end{equation}
	où \( p\) est la caractéristique de \( A\). Les orbites ont \( p\) éléments et forment une partition de \( A\), donc le cardinal de \( A\) est un multiple de \( p\).
\end{proof}

\begin{lemma}[\cite{ooIBWOooSjOvXd}]        \label{LEMooJQIKooQgukqn}
	Un anneau totalement ordonné est de caractéristique nulle.
\end{lemma}

\begin{proof}
	Le morphisme \( \mu\colon \eZ\to A\), \( n\mapsto n 1_A\) est strictement croissant, en particulier \( \mu(x)\neq \mu(y)\) dès que \( x\neq y\). Donc \( \ker(\mu)=\{ 0 \}\).
\end{proof}

L'ensemble typique de caractéristique \( p\) est \( \eF_p=\eZ/p\eZ\).

\begin{proposition} \label{PropFrobHAMkTY}
	Soit \( A\) un anneau commutatif unitaire de caractéristique \( p\). L'application
	\begin{equation}
		\begin{aligned}
			\Frob_A\colon A & \to A       \\
			x               & \mapsto x^p
		\end{aligned}
	\end{equation}
	est un automorphisme d'anneau unitaire.
\end{proposition}
Nous le nommons le \defe{morphisme de Frobenius}{morphisme!Frobenius}\index{Frobenius!morphisme}. Nous utiliserons aussi les itérés du morphisme de Frobenius : \( \Frob^k\colon x\mapsto x^{p^k}\).

\begin{example}
	Soit à factoriser \( X^p-1\) dans \( \eF_p\). Grâce au morphisme de Frobenius, nous avons immédiatement
	\begin{equation}
		X^p-1=(X-1)^p.
	\end{equation}
\end{example}


\begin{lemma}       \label{LemCaractIntergernbrcartpre}
	La caractéristique\footnote{Définition~\ref{LEMDEFooVEWZooUrPaDw}.} d'un anneau intègre est zéro ou un élément premier\footnote{Définition \ref{DEFooZCRQooWXRalw}.}.
\end{lemma}

\begin{proof}
	Si \( A\) est intègre, alors \( \eZ 1_A\) est a fortiori intègre. Notons \( p \) la caractéristique de \( A \). Si \( p = 0 \), la preuve est finie; supposons donc que \( p \neq 0 \). Alors, l'anneau \( \eZ/p\eZ\) est isomorphe à \( \eZ 1_A\), et est donc intègre. Or, la proposition~\ref{CorZnInternprem} dit que \( \eZ/p\eZ\) est intègre si et seulement si \( p\) est premier, ce qui conclut la preuve.
\end{proof}

\begin{example}
	Il existe des corps dont la caractéristique n'est pas égale au cardinal (contrairement à ce que laisserait penser l'exemple des \( \eZ/p\eZ\)). En effet les matrices \( n\times n\) inversibles sur \( \eF_{3}\) forment un corps qui n'est pas de cardinal trois alors que la caractéristique est \( 3\) :
	\begin{equation}
		\begin{pmatrix}
			1 &   \\
			  & 1
		\end{pmatrix}+\begin{pmatrix}
			1 &   \\
			  & 1
		\end{pmatrix}+\begin{pmatrix}
			1 &   \\
			  & 1
		\end{pmatrix}=0.
	\end{equation}
\end{example}


%--------------------------------------------------------------------------------------------------------------------------- 
\subsection{Caractéristique deux}
%---------------------------------------------------------------------------------------------------------------------------

Beaucoup de résultats demandent une caractéristique différente de deux. Qu'a donc de particulier la caractéristique deux ?

Si \( \eK\) est un corps de caractéristique \( 2\), alors l'égalité \( x=-x\) n'implique pas \( x=0\), puisque \( 2x=0\) est vérifiée pour tout \( x\). Cela se répercute sur un certain nombre de résultats. Par exemple, en caractéristique deux, une forme antisymétrique n'est pas toujours alternée: voir le lemme~\ref{LemHiHNey}.
