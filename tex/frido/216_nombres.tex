% This is part of Mes notes de mathématique
% Copyright (c) 2011-2022
%   Laurent Claessens
% See the file fdl-1.3.txt for copying conditions.

%+++++++++++++++++++++++++++++++++++++++++++++++++++++++++++++++++++++++++++++++++++++++++++++++++++++++++++++++++++++++++++
\section{Le groupe et anneau des entiers}
%+++++++++++++++++++++++++++++++++++++++++++++++++++++++++++++++++++++++++++++++++++++++++++++++++++++++++++++++++++++++++++

Certes \( (\eZ,+)\) est un groupe mais en ajoutant la multiplication, \( (\eZ,+,\times)\) devient un anneau\footnote{Définition~\ref{DefHXJUooKoovob}.}.

%---------------------------------------------------------------------------------------------------------------------------
\subsection{Division euclidienne}
%---------------------------------------------------------------------------------------------------------------------------

\begin{theoremDef}[Division euclidienne\cite{ooRBKHooQJqglH}]     \label{ThoDivisEuclide}
    Soient \( a\in\eZ\) et \( b\in\eN^*\). Il existe un unique couple \( (q,r)\in\eZ\times\eN\), avec \( 0\leq r<b\), tel que
    \begin{equation}
        a=bq+r.
    \end{equation}
    L'opération \( (a,b)\mapsto(q,r)\) ainsi définie est la \defe{division euclidienne}{division!euclidienne}. Le nombre \( q\) est le \defe{quotient}{quotient} et \( r\) est le \defe{reste}{reste} de la division de \( a\) par \( b\).
\end{theoremDef}

\begin{proof}
    Remarquons que \( r = a - bq \), et donc, une fois l'existence et l'unicité de $q$ établie, celle de $r$ suivra.

    \begin{subproof}
        \item[Unicité]
            Nous supposons avoir \( (q,r)\in \eZ\times \eN\) tels que
            \begin{subequations}
                \begin{numcases}{}
                    0\leq r<b\\
                    a=qb+r.
                \end{numcases}
            \end{subequations}
            Ce système implique que
            \begin{equation}
                0\leq a-qb<b.
            \end{equation}
            En ajoutant \( qb\) dans les trois membres de cette inégalité,
            \begin{equation}
                qb\leq a<(q+1)b.
            \end{equation}
            Cela implique que
            \begin{equation}
                q=\max\{ k\in \eZ\tq kb\leq a \}.
            \end{equation}
            Donc \( q\) est unique et la relation \( a=bq+r\) implique que \( r\) est également unique.

    Soit
    \begin{equation*}
        E = \{ q \in \eZ  | bq \leq a \}.
    \end{equation*}
    La partie \( E\) est non vide (parce qu'elle contient \( -|a| \)) et admet un majorant : l'élément \( |a| \).  Elle admet donc un maximum $q$ par le lemme \ref{LEMooMYEIooNFwNVI}. Ce maximum vérifie
     \begin{equation}
         bq\leq a<b(q+1).
     \end{equation}
     Cela donne \( 0\leq a-bq<b\) et le résultat, en posant \( r=a-qb\).
    \end{subproof}
\end{proof}


% TODO : À propos de restes, il n'est peut-être pas mal de parler d'algorithme de calcul de la date de pâques.
% L'algorithme de Gauss, Meeus utilise des arrondis.
% http://fr.wikipedia.org/wiki/Calcul_de_la_date_de_Pâques


%---------------------------------------------------------------------------------------------------------------------------
\subsection{PGCD, PPCM et Bézout}
%---------------------------------------------------------------------------------------------------------------------------

Puisque \( \eZ\) est un anneau intègre, nous avons la définition \ref{DefrYwbct} de pgcd et de ppcm.
\begin{proposition}[PPCM et PGCD]       \label{PROPooAVRGooUfhjwF}
    Soient \( p,q\in\eZ^*\). 
    \begin{enumerate}
        \item
            Le pgcd de \( p\) et \( q\) est le plus grand diviseur commun de \( p\) et \( q\). 
        \item
            Le ppcm de \( p\) et \( q\) est leur plus petit multiple commun.
    \end{enumerate}
\end{proposition}

\begin{proof}
    Démontrons le premier point. Notons \( \delta\) le pgcd de \( p\) et \( q\). Si \( d\) est un diviseur commun de \( p\) et \( q\), alors \( d\) divise \( \delta\). Dans \( \eZ\), \( d\divides \delta\) implique \( d\leq\delta\) (proposition \ref{PROPooYJBMooZrzkNX}).
\end{proof}

\begin{lemma}
    Soient \( p,q\in\eZ^*\). Les entiers \( \ppcm(p,q)\) et \( \pgcd(p,q)\) fournissent les isomorphismes de groupes suivants :
\begin{subequations}
    \begin{align}
        p\eZ\cap q\eZ&=\ppcm(p,q)\eZ\\
        p\eZ + q\eZ&=\pgcd(p,q)\eZ.
    \end{align}
\end{subequations}
\end{lemma}

\begin{definition}  \label{DefZHRXooNeWIcB}
    Si \( \pgcd(p,q)=1\), nous disons que \( p\) et \( q\) sont \defe{premiers entre eux}{nombre!premier!deux nombres entre eux}. Si nous avons un ensemble d'entiers \( a_i\), nous disons qu'ils sont premiers \defe{dans leur ensemble}{nombre!premier!dans leur ensemble} si \( 1\) est le PGCD de tous les \( a_i\) ensemble.
\end{definition}

Les nombres \( 2\), \( 4\) et \( 7\) ne sont pas premiers deux à deux (à cause de \( 2\) et \( 4\)), mais ils sont premiers dans leur ensemble parce qu'il n'y a pas de diviseurs communs plus grand que \( 1\), au triplet \( (2, 4, 7)\).

\begin{theorem}[Théorème de Bézout\footnote{Il y a une super application ici : \url{https://perso.univ-rennes1.fr/matthieu.romagny/agreg/dvt/mauvais_prix.pdf}.}\cite{LSAmvR}, thème~\ref{THEMEooNRZHooYuuHyt}] \label{ThoBuNjam}
    Deux entiers non nuls \( a,b\in\eZ^*\) sont premiers entre eux si et seulement si il existe \( u,v\in\eZ\) tels que
    \begin{equation}
        au+bv=1
    \end{equation}
\end{theorem}
\index{Bézout!nombres entiers}

\begin{proof}
    Soit \( d=\pgcd(a,b)\) et des nombres \( u,v\) tels que \( au+bv=1\). Le PGCD \( d\) divise à la fois \( a\) et \( b\), et donc divise \( au+bv\). Nous en déduisons que \( d\) divise \( 1\) et est par conséquent égal à \( 1\).

    Nous supposons maintenant que \( \pgcd(a,b)=1\) et nous considérons l'ensemble
    \begin{equation}
        E=\{ au+bv\tq u,v\in \eZ \}\cap \eN^*.
    \end{equation}
    C'est-à-dire l'ensemble des nombres strictement positifs pouvant s'écrire sous la forme \( au+bv\). Cet ensemble est non vide parce qu'il contient par exemple soit \( a\) soit \( -a\). Soit \( m\) le plus petit élément de \( E\) et écrivons
    \begin{equation}    \label{EqMBsfrP}
        m=au_1+bv_1.
    \end{equation}
    Par le théorème de division euclidienne\footnote{Théorème~\ref{ThoDivisEuclide}.} (avec \( a\) et \( m\)), il existe des entiers uniques $q$ et $r$ tels que
    \begin{equation}
        a=mq+r
    \end{equation}
    avec \( 0\leq r<m\). En remplaçant \( m\) par sa valeur \eqref{EqMBsfrP}, \( a=(au_1+bv_1)q+r\) et
    \begin{equation}
        r=a(1-u_1q)-bv_1q,
    \end{equation}
    c'est-à-dire que \( r\in \eZ a+\eZ b\) en même temps que \( 0\leq r<m\). Si \( r\) était strictement positif, il serait dans \( E\). Mais cela est impossible par minimalité de \( m\). Donc \( r=0\) et \( a\) est divisible par \( m\).

    De la même façon nous prouvons que \( b\) est divisible par \( m\). Puisque \( m\) divise à la fois \( a\) et \( b\) nous avons \( m=1\).
\end{proof}

Une généralisation de Bézout \ref{ThoBuNjam} à plus de \( 2\) variables.
\begin{proposition}     \label{PROPooWSMTooMdfqse}
    Si \( \{ a_i \}_{i=1,\ldots, N}\) sont des entiers tels que \( \pgcd(a_1,\ldots, a_N)=1\), alors il existe des entiers \( \{ u_i \}_{i=1,\ldots, N}\) tels que
    \begin{equation}
        \sum_ia_iu_i=1.
    \end{equation}
\end{proposition}

\begin{corollary}       \label{CorgEMtLj}
    Soient \( p\) et \( q\) deux entiers premiers entre eux. Alors
    \begin{equation}
        p\eZ+q\eZ=\eZ;
    \end{equation}
    en particulier, pour tout \( x \in \eZ \), il existe \( u_x, v_x \) entiers tels que \(u_x p + v_x q = x \).
\end{corollary}

Notons que l'application \( p\eZ+q\eZ\) vers \( \eZ\) n'est évidemment pas injective: les $u_x$ et $v_x$ ne sont pas uniques à $x$ fixé.

\begin{proof}
    Soit \( x\in \eZ\). Le théorème de Bézout nous donne \( k\) et \( l\) tels que \( kp+lq=1\). Alors, \( (xk)p+(xl)q=x\).
\end{proof}

La proposition suivante établit que si \( x\) est assez grand, alors il peut même être écrit comme une combinaison de \( p\) et \( q\) à coefficients positifs. Elle sera utilisée pour démontrer que les états apériodiques d'une chaine de Markov peuvent être atteints à tout moment (assez grand), voir la définition~\ref{DefCxvOaT} et ce qui suit.

\begin{proposition}     \label{PropLAbRSE}
    Soient \( a\) et \( b\) deux éléments de \( \eN\) premiers entre eux. Il existe \( N>0\) tel que tout \( x>N\) appartient à \( a\eN+b\eN\).
\end{proposition}

\begin{proof}
    Soient \( a\) et \( b\), premiers entre eux, et \( x\in \eN\). Disons tout de suite, pour éviter les cas triviaux et pénibles, que \( x\), \( a\) et \( b\) sont strictement positifs.

    \begin{subproof}
    \item[Une décomposition pour \( x\)]

    On applique le théorème~\ref{ThoDivisEuclide} de division euclidienne à $x$ et \( a + b \): il existe des entiers \( p_x, r_x \), uniques, tels que
    \begin{subequations}
        \begin{numcases}{}
       x = (p_x-1)(a+b) + r_x\\
       0 \leq r_x < a+b.
        \end{numcases}
    \end{subequations}
    En d'autres termes, \( p_x(a+b)\) est le premier multiple de \( a+b\) supérieur ou égal à $x$. De plus, $p_x$ est strictement positif car $x$ l'est. Il existe alors des entiers $u$ et $v$ tels que
    \begin{equation}    \label{EQooXYSZooJqxPui}
        ua + vb = p_x(a+b) - x
    \end{equation}
    par le corolaire~\ref{CorgEMtLj}. Ainsi, $x$ peut s'écrire
    \begin{equation}
        x = (p_x - u) a + (p_x - v) b.
    \end{equation}

\item[Des maximums]

    Il s'agit maintenant de savoir si nous pouvons être assuré d'avoir \( p_x > u\) et \( p_x > v\) dès que \( x\) est assez grand. Pour cela, grâce au corolaire~\ref{CorgEMtLj}, nous considérons les nombres \( u_i\) et \( v_i\) définis par
    \begin{equation}
        u_ia+v_ib=i
    \end{equation}
    pour \( i=1,\ldots, a+b\). Nous posons \( u^*=\max\{ u_i \}\), \( v^*=\max\{ v_i   \}\), et \( p^*=\max\{ u^*,v^* \}\).  Nous posons alors \( N = p^*(a+b)\), et considérons \( x>N \).

\item[Nouvelle décomposition pour \( x\)]

    Nous voulons écrire
    \begin{equation}        \label{EQooIKNWooBKItYz}
        x= (p_x - u_k) a + (p_x - v_k) b
    \end{equation}
    pour un certain \( k\). Cela demande \( u_ka+v_kb=ua+vb=p_x(a+b)-x\) par l'équation \eqref{EQooXYSZooJqxPui}. Vu que \( p_x(a+b)-x>0\), les nombres \( u_k\) et \( v_k\) existent : il suffit de prendre \( k=p_x(a+b)-x\).

\item[Conclusion]

    Avec tous ces choix, nous avons d'abord \( x>p^*(a+b)\) et donc
    \begin{equation}
        x=(p_x-1)(a+b)+r_x>p^*(a+b),
    \end{equation}
    ce qui donne
    \begin{equation}
        (p_x-1)(a+b)>p^*(a+b)-r_x>(p-1)(a+b).
    \end{equation}
    ou encore \( p_x>p^*\). Nous avons finalement
    \begin{equation}
       p_x \geq p^* \geq u^* \geq u_k
    \end{equation}
    et
    \begin{equation}
       p_x \geq p^* \geq v^* \geq v_k.
    \end{equation}
    De ce fait, la décomposition \eqref{EQooIKNWooBKItYz} est celle que nous voulions.
    \end{subproof}
\end{proof}


%\begin{proof}
    %Soit \( x\in \eN\) et \( k_1,l_1\in \eN\) les plus petits entiers tels que \( k_1p\geq x/2\) et \( l_1q\geq x/2\). Nous avons alors
    %\begin{equation}
        %x\leq k_1p+l_1q<x+(p+q).
    %\end{equation}
    %Nous posons \( \delta=k_1p+l_1q-x\).
   %
    %Soient des entiers \( a_i,b_i\) tels que \( a_ip+b_iq=i\). Nous notons
    %\begin{subequations}
        %\begin{align}
            %A=\max\{ a_i\tq i=1,\ldots, k+p \}\\
            %B=\max\{ b_i\tq i=1,\ldots, k+p \}
        %\end{align}
    %\end{subequations}
    %Notons que \( A\) et \( B\) sont donnés uniquement en termes de \( p\) et \( q\). Ils ne sont en aucun cas dépendants de \( x\).
   %
    %Nous avons
    %\begin{equation}
        %x=k_1p+lq-\delta=(k_1-a_{\delta})p+(l_1+b_{\delta})q
    %\end{equation}
    %avec \( k_1-a_{\delta}\geq k_1-A\) et \( l_1-b_{\delta}\geq l_1-B\). Si \( x\) est suffisamment grand pour avoir \( k_1>A\) et \( l_1>B\), alors la décomposition souhaitée est trouvée.
%
    %Une borne pour \( x\) est donnée par
    %\begin{equation}    \label{EqjQpURG}
        %x>\max\{ 2pA,2qB \}.
    %\end{equation}
%\end{proof}

\begin{normaltext}
    Une méthode pour obtenir les entiers naturels $u$ et $v$ qui permettent la décomposition \(x = au + bv \) est d'abord de choisir $u_0$ et $v_0$ tels que \( au_0 \) et \( bv_0 \) soient les plus proches possibles de $x/2$, puis de décomposer le nombre (relativement petit) \( x - au_0 - bv_0 \) en \( au_1 + bv_1 \). Deux nombres $u$ et $v$ qui fonctionnent sont alors $u = u_0 + u_1$ et $v = v_0 + v_1$.
\end{normaltext}

\begin{example}
    Écrivons \( 1000=u\cdot 7+v\cdot 5\) avec \( u,v\in \eN\). D'abord \( 72\cdot 7=504\) et \( 100\cdot 5=500\). Nous avons donc
    \begin{equation}
        1004=72\cdot 7+100\cdot 5.
    \end{equation}
    Ensuite \( 4=25-21=-3\cdot 7+5\cdot 5\). Au final,
    \begin{equation}
        1000=75\cdot 7+95\cdot 5.
    \end{equation}
\end{example}

%---------------------------------------------------------------------------------------------------------------------------
\subsection{Sous-groupes de \texorpdfstring{$(\eZ,+)$}{(Z,+)}}
%---------------------------------------------------------------------------------------------------------------------------

\begin{proposition} \label{PropSsgpZestnZ}
    Une partie \( H\) du groupe \( (\eZ,+)\) est un sous-groupe si et seulement si il existe \( n\in\eN\) tel que \( H=n\eZ\).
\end{proposition}

\begin{proof}
    Soit \( H\neq\{ 0 \}\) un sous-groupe de \( \eZ\). L'ensemble \( H\cap\eN^*\) contient un élément minimum que nous notons \( n\). Nous avons certainement \( n\eZ\subset H\) parce que \( H\) est un groupe (donc \( n+n\) et \( -n\) sont dans \( H\) dès que \( n\) est dans \( H\)). Nous devons prouver que \( H\subset n\eZ\).

    Si \( x\in H\), par le théorème de division euclidienne~\ref{ThoDivisEuclide}, il existe \( q\in\eZ\) et \( r\in\eN \), uniques, tels que \( x=nq+r\) et \(0 \leq r < n \). Nous savons déjà que \( nq\in H\), donc \( r = x - nq \in H \). Le nombre \( r\) est donc un élément de \( H\) strictement plus petit que \( n\). Mais nous avions décidé que \( n\) serait le plus petit élément de \( H\cap\eN^*\). Par conséquent \( r=0\) et \( x=nq\in n\eZ\).
\end{proof}


Notons que si un sous-groupe \( H\) de \( \eZ\) est donné, alors le nombre \( n\) tel que \( H=n\eZ\) est unique. En effet si \( n\eZ=m\eZ\) nous avons que \( n\) divise \( m\) (parce que \( m\in m\eZ\subset n\eZ\)) et que \( m\) divise \( n\) parce que \( n\in m\eZ\). Par conséquent \( n=m\).

%---------------------------------------------------------------------------------------------------------------------------
\subsection{Résultats supplémentaires sur l'anneau des entiers}
%---------------------------------------------------------------------------------------------------------------------------

\begin{corollary}       \label{CORooLINXooBlUKPG}
    Les quotients de \( \eZ\) sont \( \eZ/n\eZ\).
\end{corollary}
%TODOooKQDLooFnggOd: il faut préciser si ça veut dire quelque chose comme les seuls quotients de (Z,+) qui sont encore des groupes.

\begin{proof}
    Tous les idéaux de \( \eZ\) sont de la forme \( n\eZ\). En effet en vertu de la proposition~\ref{PropSsgpZestnZ}, les seuls sous-groupes de \( \eZ\) (en tant que groupe additif) sont les \( n\eZ\). Tous les idéaux sont donc de cette forme. De plus les \( n\eZ\) sont effectivement tous des idéaux\footnote{Définition \ref{DefooQULAooREUIU}.} : si \( a\in n\eZ\) et si \( k\in \eZ\) alors \( ak\in n\eZ\).
\end{proof}

\begin{proposition}     \label{PropZpintssiprempUzn}
    Soient \( n\geq 2\) un entier et \( \phi\colon \eZ\to \eZ/n\eZ\) la surjection canonique. Nous noterons \( \overline a=\phi(a)\). Alors l'ensemble des inversibles de \( \eZ/n\eZ\) est donné par
    \begin{equation}
        U(\eZ/n\eZ)=\phi(P_n)=\{ \overline x\tq 0\leq x\leq n\tq\pgcd(x,n)=1 \}.
    \end{equation}
    où \( P_n\) est l'ensemble $P_n=\{ x\in\{ 0,\ldots,n-1 \}\tq\pgcd(x,n)=1 \}$.

    De plus,
    \begin{equation}
        \Card\big( U(\eZ/n\eZ) \big)=\phi(n).
    \end{equation}
\end{proposition}

\begin{proof}
    Soit \( 0\leq x\leq n\) tel que \( \pgcd(x,n)=1\). Il existe donc\footnote{Théorème de Bézout~\ref{ThoBuNjam}} \( u,v\in\eZ\) tels que \( ux+vn=1\). En passant aux classes,
    \begin{equation}
        \overline u\overline x=\overline 1,
    \end{equation}
    donc \( \overline u\) est l'inverse de \( \overline x\). Cela prouve que \( \phi(P_n)\subset U(\eZ/n\eZ)\).

    Nous prouvons maintenant l'inclusion inverse. Soient \( \overline x\) et \( \overline y\) inverses l'un de l'autre : $\overline x\overline y=\overline 1$. Il existe donc \( q\in\eZ\) tel que \( xy-qn=1\), ce qui prouve\footnote{À nouveau avec le Théorème de Bézout.} que \( \pgcd(x,n)=1\).
\end{proof}

%+++++++++++++++++++++++++++++++++++++++++++++++++++++++++++++++++++++++++++++++++++++++++++++++++++++++++++++++++++++++++++
\section{Corps}
%+++++++++++++++++++++++++++++++++++++++++++++++++++++++++++++++++++++++++++++++++++++++++++++++++++++++++++++++++++++++++++

%---------------------------------------------------------------------------------------------------------------------------
\subsection{Définitions, morphismes}
%---------------------------------------------------------------------------------------------------------------------------

\begin{definition}[\cite{ooLKFGooTUrnhx}]  \label{DefTMNooKXHUd}
    Un \defe{corps}{corps} est un anneau\footnote{Définition \ref{DefHXJUooKoovob}.} \( (A, +,\times)\) dans lequel tout élément non nul est inversible pour l'opération \( \times\) (pour l'opération \( +\), tous les éléments sont inversibles parce que \( (A,+)\) est un groupe).
\end{definition}

\begin{remark}      \label{REMooYRNUooYgBBKF}
    Un anneau est ce qu'on appelle «\emph{ring}» en anglais. Un corps est en anglais «\emph{field}». De plus le mot «\emph{field}» comprend la commutativité. Donc certains utilisent le mot «corps» pour dire «corps commutatif» et parlent alors d'anneau \emph{à division} pour parler de corps non commutatifs.
\end{remark}

La proposition suivante donne une caractérisation d'un corps, en disant un tout petit peu plus que la définition~\ref{DefTMNooKXHUd}.
\begin{proposition}
    L'anneau $A$ est un corps si et seulement si \( U(A) = A^* \).
\end{proposition}

\begin{proof}
    En deux parties.
    \begin{subproof}
        \item[Sens direct]
            Nous supposons que \( A\) est un corps. D'une part tous les éléments non nuls sont inversibles, c'est-à-dire \( A^*\subset U(A)\).
            
            Pour l'inclusion inverse, nous montrons qu'une élément inversible ne peut pas être nul. Cela n'est autre que le lemme~\ref{LEMooVUSMooWisQpD} couplé à la proposition~\ref{PROPooNCCGooXjVyVt} : \( a\cdot 0=0\neq 1\) pour tout \( a\).
        \item[Sens inverse]
            Si \( U(A)=A^*\), nous avons immédiatement que tous les éléments non nuls sont inversibles et donc que \( A\) est un corps.
    \end{subproof}
\end{proof}

\begin{lemma}       \label{LEMooJNIBooAURhrt}
    Si \( \eK\) est un corps et si \( a\in \eK\) vérifie \( a^2=1\), alors \( a=\pm 1\).
\end{lemma}

\begin{lemma}       \label{LemAnnCorpsnonInterdivzer}
    Un corps non nul est un anneau intègre\footnote{Définition \ref{DEFooTAOPooWDPYmd}.}.
\end{lemma}

\begin{proof}
    Soit un produit nul \( ab=0\). Si \( a\neq 0\), alors il est inversible et nous multiplions \( ab=0\) par \( a^{-1}\). Nous trouvons \( b=0\) parce que \( 0a^{-1}=0\).
\end{proof}
Conséquence : dans un corps nous avons toujours la règle du produit nul, et l'élément nul n'est jamais inversible.

\begin{definition}[Morphisme de corps]
    Un corps étant un anneau sans plus de structure, un \defe{morphisme de corps}{morphisme!de corps}\index{isomorphisme!de corps} n'est qu'un morphisme des anneaux\footnote{Définition \ref{DEFooSPHPooCwjzuz}.}.
\end{definition}

Le lemme suivant montre que définir un morphisme de corps comme étant simplement un morphisme des anneaux est une bonne idée.
\begin{lemma}       \label{LEMooWBOPooZnsZgQ}
    Si \( \varphi\colon \eK\to \eK'\) est un morphisme de corps, alors
    \begin{enumerate}
        \item
            pour tout \( a\in \eK\) nous avons \( \varphi(a^{-1})=\varphi(a)^{-1}\);
        \item
            le morphisme \( \varphi\) est injectif.
    \end{enumerate}
\end{lemma}

\begin{proof}
    Vu que \( \varphi(1)=1\), nous avons aussi
    \begin{equation}
        1=\varphi(aa^{-1})=\varphi(a)\varphi(a^{-1}).
    \end{equation}
    Donc, par unicité de l'inverse\footnote{Lemme~\ref{LEMooECDMooCkWxXf}\,\ref{ITEMooOIWTooYqmMPP}.}, \( \varphi(a^{-1})=\varphi(a)^{-1}\).

    Pour l'injectivité nous supposons \( \varphi(a)=\varphi(b)\). Étant donné que \( \eK'\) est un corps, nous pouvons multiplier par \( \varphi(b)^{-1}\) :
    \begin{equation}
        \varphi(a)\varphi(b)^{-1}=1.
    \end{equation}
    En utilisant le premier point nous avons \( 1=\varphi(a)\varphi(b^{-1})\), puis le morphisme d'anneaux : \( 1=\varphi(ab^{-1})\), et encore le morphisme d'anneaux nous permet de déduire \( ab^{-1}=1\) et donc \(a=b\).
\end{proof}

%+++++++++++++++++++++++++++++++++++++++++++++++++++++++++++++++++++++++++++++++++++++++++++++++++++++++++++++++++++++++++++
\section{Anneau intègre}
%+++++++++++++++++++++++++++++++++++++++++++++++++++++++++++++++++++++++++++++++++++++++++++++++++++++++++++++++++++++++++++
\label{SECAnneauxIntegres}

La définition d'un anneau intègre est la définition~\ref{DEFooTAOPooWDPYmd}.

\begin{lemma}     \label{LEMooZSMEooUmSXWZ}
    Un corps\footnote{Définition~\ref{DefTMNooKXHUd}.} est un anneau intègre.
\end{lemma}

\begin{proof}
    En effet, soient un corps \( \eK\) et deux éléments \( x,y\in \eK\) tels que \( xy=0\). Si \( y\) est inversible, alors nous pouvons multiplier par \( y^{-1}\) pour trouver \( x=0\). Cela prouve que \( \eK\) est un anneau intègre.
\end{proof}

\begin{example}     \label{EXooLDXRooSxUAXs}
    L'ensemble \( \eZ\) avec les opérations usuelles est un anneau intègre\footnote{Anneau intègre, définition \ref{DEFooTAOPooWDPYmd}.}.
\end{example}

\begin{example}
    L'anneau \( \eZ/6\eZ\) n'est pas intègre parce que \( 3\cdot 2=0\) alors que ni \( 3\) ni \( 2\) ne sont nuls.
\end{example}

Nous verrons au théorème~\ref{ThoBUEDrJ} que l'anneau \( A\) est intègre si et seulement si \( A[X]\) est intègre.

\begin{corollary}   \label{CorZnInternprem}
    L'anneau \( \eZ/n\eZ\) est intègre si et seulement si \( n\) est premier.
\end{corollary}

\begin{proof}
    Supposons que \( n\) soit premier. La proposition \ref{PropZpintssiprempUzn} donne les inversibles de \( \eZ/n\eZ\) par
    \begin{equation}
        U(\eZ/n\eZ)=\{ \overline x\tq 0\leq x\leq n\tq\pgcd(x,n)=1 \}.
    \end{equation}
    Mais comme \( n\) est premier, \( \pgcd(x,n)=1\) pour tout \( x\), et donc tous les éléments de \( \eZ/n\eZ\) sont inversibles. Donc \( \eZ/n\eZ\) est intègre.

    Si \( n\) n'est pas premier, alors \( n=pq\) avec \( 1<p\leq q<n\). Alors
    \begin{equation}
        [p]_n[q]_n=[pq]_n=[0]_n.
    \end{equation}
    Donc lorsque \( n\) n'est pas premier,  l'anneau \( \eZ/n\eZ\) possède des diviseurs de zéro et n'est alors pas intègre.
\end{proof}


%--------------------------------------------------------------------------------------------------------------------------- 
\subsection{Élément premier}
%---------------------------------------------------------------------------------------------------------------------------

\begin{definition}[\cite{ooWBLYooLYwALS}]       \label{DEFooZCRQooWXRalw}
    Soit un anneau commutatif \( A\). Un élément \( p\in A\) est \defe{premier}{élément premier} si il est
    \begin{enumerate}
        \item
            non nul,
        \item
            non inversible,
        \item       \label{ITEMooPMTTooCVHPIm}
            si \( p\) divise un produit \( ab\), alors il divise soit \( a\) soit \( b\) (ou le deux).
    \end{enumerate}
\end{definition}


Le lemme suivant est souvent pris pour la définition d'un nombre premier lorsqu'on parle de \( \eN\) ou \( \eZ\).
\begin{lemma}[\cite{frwiki179832418, MonCerveau}]
    Dans \( \eN\), un nombre est premier si et seulement si il admet exactement deux diviseurs entiers distincts.
\end{lemma}

\begin{proof}
    En deux parties.
    \begin{subproof}
    \item[\( \Rightarrow\)]
        Soit un élément premier \( p\in \eN\). Il y a trois possibilités : \( p=0\), \( p=1\) et \( p>1\).

    Le nombre \( p=0\) n'est pas premier parce qu'il est nul. Le nombre \( p=1\) n'est pas premier parce qu'il est inversible. Donc nous savons que si \( p\) est premier, alors \( p>1\).

    Un élément \( p>1\) dans \( \eN\) a toujours au moins deux diviseurs distincts : \( 1\) et \( p\). Soit un diviseur \( k\) de \( p\). Il existe \( l\in \eN\) tel que \( p=kl\). Vu que \( p\) est premier et divise le produit \( kl\), il divise \( k\) ou \( l\). Disons que \( p\) divise \( k\). De cette façon \( p\) divise \( k\) et \( k\) divise \( p\).

    Il existe donc \( n\in \eN\) tel que \( k=np\). En y substituant \( p=kl\), on trouve \( k=np=nkl\). En simplifiant par \( k\), il vient
    \begin{equation}
        1=nl,
    \end{equation}
    ce qui prouve que \( n=l=1\) et donc que \( k=p\) et donc que \( p\) n'a pas d'autres diviseurs que \( 1\) et \( p\).
        
    \item[\( \Leftarrow\)]
        Nous supposons que \( p\in \eN\) ait exactement deux diviseurs entiers distincts. Nous vérifions que \( p\) vérifie les trois conditions de la définition \ref{DEFooZCRQooWXRalw}.

        \begin{enumerate}
            \item
                \( p\neq 0\) parce que \( 0\) a nettement plus que deux diviseurs distincts.
            \item
                \( p\neq 1\) parce que \( 1\) a exactement un diviseur. Donc \( p\) n'est pas inversible dans \( \eN\).
            \item
                Soit \( p\) admettant exactement deux diviseurs distincts. Soit \( p\) divisant le produit \( ab'\) pour certains \( a\) et \( b'\) dans \( \eN\). Nous supposons que \( p\) ne divise pas \( a\), et nous allons prouver que \( p\) divise \( b'\) en supposant d'abord que \( p\) ne divise pas \( b'\). 

                \begin{subproof}
                \item[Un ensemble]
                Pour cela nous posons
                \begin{equation}
                    E=\{ x\in \eN\tq p\divides ax, p\notdivides x  \}.
                \end{equation}
                Nous posons \( b=\min(E)\). Nous avons pour hypothèse que \( E\) est non vide; en particulier \( 0<b\).
            \item[\( b<p\)]
                On vérifie que si \( p+k\in E\) alors \( k\in E\). Donc \( b\) ne peut pas être plus grand que \( p\). Vu que \( p\) lui-même n'est pas dans \( E\), nous avons \( b<p\).
            \item[Division euclidienne]
                Nous effectuons la division euclidienne du théorème \ref{ThoDivisEuclide} :
                \begin{equation}
                    p=mb+r.
                \end{equation}
                En multipliant par \( a\), \( ar=ap-mab\). Vu que \( ab\) est un multiple de \( p\) \( ap-mab\) est un multiple de \( p\). En particulier \( ar\) est divisible en \( p\). 
            \item[La contradiction]
                Nous avons donc \( r\in E\), alors que \( r<b\). Impossible.
                \end{subproof}
        \end{enumerate}
    \end{subproof}
\end{proof}

\begin{proposition}[\cite{ooTGPAooQTbamu}]     \label{PROPooWMNPooZdvOBt}
    Dans un anneau intègre\footnote{Si pas intègre, voir l'exemple \ref{EXooEIUEooCZCPMC}.} tout élément premier est irréductible\footnote{Toutes les définitions dans le thème \ref{THEMEooVIQIooOcFAQS}.}.
\end{proposition}
    
\begin{proof}
    Soit \( p\), un élément premier dans un anneau intègre \( A\).
    \begin{subproof}
        \item[\( p\) n'est pas inversible]
            Cela fait partie de la définition d'un élément premier.
        \item[\( p\) n'est pas un produit d'inversibles]
            Soient \( a,b\in A\) tels que \( p=ab\). Par le point \ref{ITEMooPMTTooCVHPIm} de la définition \ref{DEFooZCRQooWXRalw}, \( p\) divise soit \( a\) soit \( b\). Supposons que \( p\) divise \( a\). Alors il existe \( x\in A\) tel que \( a=px\). En remettant dans \( p=ab\) nous avons :
            \begin{equation}        \label{EQooPYBGooLFHMJZ}
                p=pxb.
            \end{equation}
            Mais l'anneau est intègre et permet donc des simplifications par tout élément non nul. La relation \ref{EQooPYBGooLFHMJZ} donne donc 
            \begin{equation}
                1=xb,
            \end{equation}
            ce qui signifie que \( b\) est inversible.

            Un travail similaire montre que \( a\) est inversible si \( p\) divise \( b\).
    \end{subproof}
\end{proof}

\begin{example}
    Si nous avons l'égalité \( 7=ab\) dans \( \eZ\), alors soit \( a\) soit \( b\) vaut \( 1\). Mettons \( a=1\). Dans ce cas, \( b=7\) et n'est donc pas inversible.
\end{example}

Sur un anneau non intègre, la notion d'élément premier n'est pas aussi intéressante que sur un anneau intègre. Par exemple la proposition \ref{PROPooWMNPooZdvOBt} devient fausse.

\begin{example}     \label{EXooEIUEooCZCPMC}
    Soit l'anneau \( \eZ^2\). L'élément \( (1,0)\) est premier mais pas irréductible.
    \begin{subproof}
        \item[\( (1,0)\) est premier]
            L'élément \( (1,0)\) est non nul; ça c'est pas cher. Pour qu'il soit inversible, il faudrait \( (1,0)(x,y)=(1,1)\). Entre autres, \( 0\times y=1\), ce qui est impossible. Donc il n'est pas inversible.

            Supposons que \( (1,0)\) divise le produit \( (a,b)(c,d)=(ac,bd)\). Alors il existe \( (x,y)\) tel que \( (1,0)(x,y)=(ac,bd)\). Cela signifie que \( x=ac\) et \( 0\times y=bd\). En particulier, soit \( b=0\) soit \( d=0\). Si \( b=0\), nous avons \( (a,b)=(a,0)\) et effectivement, \( (1,0)\) le divise.
        \item[\( (1,0)\) n'est pas irréductible]
            Nous avons \( (1,0)=(1,0)(1,0)\). Donc l'élément \( (1,0)\) est le produit de deux éléments non inversibles.
    \end{subproof}
\end{example}

%+++++++++++++++++++++++++++++++++++++++++++++++++++++++++++++++++++++++++++++++++++++++++++++++++++++++++++++++++++++++++++ 
\section{Symbole de sommation}
%+++++++++++++++++++++++++++++++++++++++++++++++++++++++++++++++++++++++++++++++++++++++++++++++++++++++++++++++++++++++++++

%--------------------------------------------------------------------------------------------------------------------------- 
\subsection{Somme à valeurs dans un groupe commutatif}
%---------------------------------------------------------------------------------------------------------------------------

Si \( S\) est un ensemble fini, nous savons de la proposition \ref{PROPooJLGKooDCcnWi} qu'il existe un unique \( N\in \eN\) pour lequel il existe une bijection \( \varphi\colon \{ 0,\ldots, N \}\to S\). Cette bijection n'est à priori pas unique.

\begin{lemmaDef}[\cite{MonCerveau}]       \label{DEFooLNEXooYMQjRo}
    Soient un groupe commutatif \( (G,+)\) ainsi qu'un ensemble fini \( I\) contenant \( n\) éléments. Soit une application \( f\colon I\to G \). Si \( \sigma_1,\sigma_2\colon \{1,\ldots, n \}\to I\) sont deux bijections, alors\footnote{Pour rappel, le symbole \( \sum_{i=1}^n\) est défini par \ref{DEFooNEVNooJlmJOC}.}
    \begin{equation}
        \sum_{i=1}^nf\big( \sigma_1(i) \big)=\sum_{i=1}^nf\big( \sigma_2(i) \big).
    \end{equation}
    La valeur commune est notée
    \begin{equation}
        \sum_{i\in I}f(i)
    \end{equation}
\end{lemmaDef}

\begin{proof}
    Nous commençons par considérer une transposition \( \sigma\) (qui permute \( k\) et \( l\) avec \( k<l\)). Nous avons
    \begin{subequations}
        \begin{align}
            \sum_{i=1}^nf(i)&=\sum_{i=1}^{k-1}f(i)+f(k)+\sum_{i=k+1}^{l-1}f(i)+f(l)+\sum_{i=l+1}^nf(i)\\
            &=\sum_{i=1}^{k-1}f(i)+f(l)+\sum_{i=k+1}^{l-1}f(i)+f(k)+\sum_{i=l+1}^nf(i)\\
            &=\sum_{i=1}^nf\big( \sigma(i) \big).
        \end{align}
    \end{subequations}
    Pour cela nous avons utilisé le fait que \( G\) est commutatif pour permuter \( f(l)\in G\) et \( f(k)\in G\) avec \( \sum_{i=k+1}^{l-1}f(i)\in G\).

    Une permutation quelconque est un produit de telles transpositions (proposition \ref{PropPWIJbu}). Donc pour toute permutation \( \sigma\) nous avons
    \begin{equation}
        \sum_{i=1}^nf\big( \sigma(i) \big)=\sum_{i=1}^nf(i).
    \end{equation}
\end{proof}

La définition \ref{DEFooLNEXooYMQjRo} donne lieu à un certain nombre de remarques.
\begin{enumerate}
    \item
        Elle donne la somme sur un ensemble fini. Un problème avec les ensembles infinis (outre la convergence) est l'ordre de sommation. Si vous voulez sommer sur \( \eZ\), dans quel ordre le faire ?
    \item
        Pour aller plus loin, et sommer sur des ensembles infinis, il faut regarder la définition \ref{DefHYgkkA}. 
\end{enumerate}

\begin{proposition}     \label{PROPooJBQVooNqWErk}
    Soient un groupe commutatif \( (G,+)\), un ensemble fini \( I\), une application \( f\colon I\to G\) et une bijection \( \sigma\colon I\to I\). Alors
    \begin{equation}
        \sum_{i\in I}f(i)=\sum_{i\in I}f\big( \sigma(i) \big).
    \end{equation}
\end{proposition}

Si nous avons une application \( L\colon S\to S\), nous notons
\begin{equation}
    \sum_{s\in S}f\big( L(s) \big)=\sum_{s\in S}(f\circ L)(s).
\end{equation}
Cette façon d'écrire donne une interprétation pour la notation \( \sum_{g\in G}f(hg)\) qui arrive dans la proposition \ref{PROPooWJQQooFINSEc}. Il s'agit de considérer l'application \( L_h\) du lemme \ref{LEMooBIBFooBHxFYC}, de considérer\footnote{Le fait que \( L_h\) soit une bijection n'a pas d'importance ici.}
\begin{equation}        \label{EQooQQBEooFDOBVG}
    \sum_{g\in G}f(hg)=\sum_{g\in G}(f\circ L_h)(g)
\end{equation}
et de faire tourner la définition \ref{DEFooLNEXooYMQjRo}. La même chose tient pour définir \( \sum_{g\in G}(gh)\) à l'aide de \( R_h\).


\begin{lemma}
    Soit un ensemble \( A\) fini pouvant être écrit comme une union disjointe \( A=\bigcup_{k=1}^nA_k\); nous supposons que les \( A_i\) sont non vides. Soient un groupe commutatif \( (G,+)\) et une application \( f\colon A\to G\). Alors
    \begin{equation}
        \sum_{a\in A}f(a)=\sum_{k=1}^n\sum_{a\in A_k}f(a).
    \end{equation}
\end{lemma}


\begin{proof}
    Le lemme \ref{LEMooTUIRooEXjfDY} nous indique que les parties \( A_k\) sont des ensembles finis. Nous notons
    \begin{enumerate}
        \item
            \( N_0=0\), et \( N_k=\Card(A_k)\),
        \item
            \( S_k=\sum_{i=1}^kN_k\).
        \item
            \( \varphi_k\colon \{ 1,\ldots, N_k \}\to A_k\), une bijection (l'existence est dans la proposition \ref{PROPooJLGKooDCcnWi}).
    \end{enumerate}
    Nous avons \( \Card(A)=S_n\) par le lemme \ref{LEMooVFPNooVmdUXY}\ref{ITEMooSWJCooEpBVkG}. Nous définissons une belle bijection comme il faut :
    \begin{equation}
        \begin{aligned}
            \alpha\colon \{ 1,\ldots, S_n \}&\to A \\
            i&\mapsto \varphi_{k+1}(i-S_k) 
        \end{aligned}
    \end{equation}
    pour \( i\in\mathopen] S_k , S_{k+1} \mathclose]\).

    \begin{subproof}
        \item[\( \alpha\) est bien définie]
            Puisque \( i>S_k\) et \( i\leq S_{k+1}\) nous avons \( i-S_k\in \{ 1,\ldots, N_{k+1} \}\), et donc \( \varphi_{k+1}\) s'applique bien à \( i-S_k\).
        \item[\( \alpha\) est injective]
        Supposons que \( \alpha(i)=\alpha(j)\). Si \( i\in \mathopen] S_k , S_{k+1} \mathclose]\) et \( j\in \mathopen] S_l , S_{l+1} \mathclose]\), alors \( \alpha(i)=\varphi_{k+1}(i-S_k)\in A_{k+1}\) et \( \alpha(j)=\varphi_{l+1}(j-S_l)\in A_{l+1}\). Vu que les \( A_i\) sont disjoints, nous avons \( k=l\), et donc
        \begin{equation}
            \varphi_{k+1}(u-S_k)=\varphi_{k+1}(j-S_k).
        \end{equation}
        Étant donné que \( \varphi_{k+1}\) est injective, nous avons \( i-S_k=j-S_k\), ce qui montre que \( i=j\).
    \item[\( \alpha\) est surjective]
    Soit \( a\in A\). Il existe \( k\) tel que \( a\in A_k\). Nous avons donc un \( s\in\{ 1,\ldots, N_k \}\) tel que \( a=\varphi_k(s)\). En posant \( i=s+S_k\), nous avons bien \( a=\alpha(s+S_k)\) parce que \( s+S_k\in \mathopen] S_{k-1} , S_k \mathclose]\).
    \end{subproof}
    Vu que \( \alpha\) est une bijection, nous avons l'égalité
    \begin{equation}
        \sum_{a\in A}f(a)=\sum_{i=1}^{S_n}(f\circ \alpha)(i).
    \end{equation}
    
    Nous avons encore besoin d'introduire une bijection. Nous posons
    \begin{equation}
        \begin{aligned}
        \beta_k\colon \mathopen] S_{k-1} , S_k \mathclose]&\to A_k \\
        i&\mapsto \varphi_k(i-S_{k-1}). 
        \end{aligned}
    \end{equation}
    C'est une bijection parce que \( \varphi_k\) en est une, et que \( i\mapsto i-S_{k-1}\) est une bijection de \( \mathopen] S_{k-1} , S_k \mathclose]\).

    Nous pouvons maintenant terminer :
    \begin{subequations}
        \begin{align}
            \sum_{a\in A}f(a)&=\sum_{i=1}^{S_n}(f\circ \alpha)(i)\\
            &=\sum_{k=1}^n\left( \sum_{i=S_{k-1}-1}^{S_k}(f\circ \alpha)(i) \right)        \label{SUBEQooNVKWooZqBAau}\\
        &=\sum_{k=1}^n\left( \sum_{i\in \mathopen] S_{k-1} , S_k \mathclose]}f\big( \varphi_k(i-S_{k-1}) \big)  \right)\\
    &=\sum_{k=1}^n\left( \sum_{i\in \mathopen] S_{k-1} , S_k \mathclose]}f\big( \beta_k(i) \big) \right)\\
    &=\sum_{i=1}^n\left( \sum_{a\in A_k}f(a) \right).
        \end{align}
    \end{subequations}
    Justifications :
    \begin{itemize}
        \item Pour \eqref{SUBEQooNVKWooZqBAau}. Associativité de la somme.
    \end{itemize}
\end{proof}


\begin{proposition}[\cite{MonCerveau}]     \label{PROPooWJQQooFINSEc}
    Soient un groupe fini \( G\) et une fonction \( f\colon G\to A\) où \( A\) est un anneau commutatif. Alors
    \begin{equation}
        \sum_{g\in G}f(g)=\sum_{g\in G}f(gh)=\sum_{g\in G}f(hg)
    \end{equation}
    pour tout \( h\in G\).
\end{proposition}

\begin{proof}
    Nous avons une bijection \( \varphi\colon \{ 0,\ldots,  N \}\to G\) garantie par la proposition \ref{PROPooJLGKooDCcnWi}. Sa définition est
    \begin{equation}
        \sum_{g\in G}f(g)=\sum_{i=0}^Nf\big( \varphi(i) \big).
    \end{equation}
    Par ailleurs, le lemme \ref{LEMooBIBFooBHxFYC} donne une bijection \( L_h\colon G\to G\) et permet de considérer la composée
    \begin{equation}
        \begin{aligned}
            \varphi'\colon \{ 0,\ldots,  N \}&\to G \\
            \varphi'=L_h\circ \varphi.
        \end{aligned}
    \end{equation}
    La proposition \ref{DEFooLNEXooYMQjRo} nous permet d'utiliser la bijection \( \varphi'\) au lieu de \( \varphi\) pour exprimer la somme \( \sum_{g\in G}\). Ensuite un jeu de notation utilisant \eqref{EQooQQBEooFDOBVG} donne
    \begin{equation}
        \begin{aligned}[]
            &\sum_{g\in G}f(g)=\sum_{i=0}^Nf\big( \varphi(i) \big)=\sum_{i=0}^Nf\big( \varphi'(i) \big)=\sum_{i=0}^N(f\circ L_h\circ \varphi)(i)\\
            &\quad=\sum_{i=0}^N(f\circ L_h)\big( \varphi(i) \big)=\sum_{g\in G}(f\circ L_h)(g)=\sum_{g\in G}f(hg).
        \end{aligned}
    \end{equation}
    En ce qui concerne \( \sum_{g\in G}f(gh)\), c'est la même chose, en utilisant \( R_h\) au lieu de \( L_h\).
\end{proof}

Tout cela nous permet de définir une somme sympathique et bien connue.
\begin{lemma}
    Soit \( n\in \eN\). Nous avons
    \begin{equation}
        \sum_{k=0}^nk=\frac{ n(n+1) }{ 2 }.
    \end{equation}
\end{lemma}

\begin{proof}
    La preuve est pratiquement immédiate par récurrence. Nous allons donner une preuve plus «constructive», qui formalise l'idée classique d'écrire la somme à l'endroit et à l'envers.


    Nous notons \( S\) la somme \( \sum_{k=0}^nk\). Le lemme \ref{DEFooLNEXooYMQjRo} dit que si les \( \sigma_i\colon \{ 0,\ldots, n \}\to \{ 0,\ldots, n \}\) sont des bijections, alors \( \sum_{k=0}^nf\big( \sigma_1(k) \big)=\sum_{k=0}^nf\big( \sigma_2(k) \big)\). Nous sommes intéressé au cas \( f(i)=i\).

    En prenant \( \sigma_1(k)=k\) et \( \sigma_2(k)=n-k\), nous avons
    \begin{equation}
        S=\sum_{k=0}^nk=\sum_{k=0}^n(n-k).
    \end{equation}
    Donc
    \begin{equation}
        2S=\sum_{k=0}^n\big( k+(n-k) \big)=\sum_{k=0}^nn=n\sum_{k=0}^n1=n(n+1).
    \end{equation}
    En divisant par deux, nous obtenons le résultat annoncé.
\end{proof}

\begin{proposition}     \label{PROPooQMUDooQQVRIe}
    Si \( E\) est un ensemble fini et si \( G\) est un groupe commutatif, alors pour toute fonction \( f\colon E\to G\) et pour toute permutation\footnote{Une permutation est une bijection, définition \ref{DEFooJNPIooMuzIXd}.} \( \sigma\) de \( E\),
    \begin{equation}
        \prod_{i\in E}f(i)=\prod_{i\in E}f\big( \sigma(i) \big)
    \end{equation}
\end{proposition}

\begin{proof}
    C'est exactement la proposition \ref{DEFooLNEXooYMQjRo}, sauf qu'ici la loi de groupe est notée multiplicativement au lieu d'additivement.
\end{proof}

%+++++++++++++++++++++++++++++++++++++++++++++++++++++++++++++++++++++++++++++++++++++++++++++++++++++++++++++++++++++++++++
\section{Module sur un anneau}
%+++++++++++++++++++++++++++++++++++++++++++++++++++++++++++++++++++++++++++++++++++++++++++++++++++++++++++++++++++++++++++

\begin{definition}[module sur un anneau\cite{ooJGVOooSjQBVh}]       \label{DEFooHXITooBFvzrR}
    Soit un anneau \( A\). Un \defe{module à gauche}{module!à gauche} sur \( A\) est la donnée d'un triplet \( (M,+,\cdot)\) où
    \begin{enumerate}
        \item
            \( +\) est une loi de composition interne à \( M\), c'est-à-dire \( +\colon M\times M\to M\),
        \item
            \( \cdot\) est une loi de composition externe, c'est-à-dire \( \cdot\colon A\times M\to M\)
    \end{enumerate}
    telles que
    \begin{enumerate}
        \item
            \( (M,+)\) est un groupe\footnote{Nous verrons dans la proposition~\ref{PROPooGARGooDiMqtN} qu'il est forcément commutatif.}.
        \item
            \( a\cdot(x+y)=a\cdot x+a\cdot y\),
        \item
            \( (a+b)\cdot x=a\cdot x+b\cdot x\),
        \item
            \( (ab)\cdot x=a\cdot(b\cdot x)\)
        \item
            \( 1\cdot x=x\).
    \end{enumerate}
    pour tout \( a,b\in A\) et \( x,y\in M\).

    Si \( M\) et \( N\) sont des \( A\)-modules, un \defe{morphisme}{morphisme de modules} de \( M\) vers \( N\) est une application \( f\colon M\to N\) qui 
    \begin{enumerate}
        \item
            est un morphisme de groupes entre \( (M,+)\) et \( (N,+)\) 
        \item
            vérifie \( f(a\cdot x)=a\cdot f(x)\) pour tout \( a\in A\), \( x\in M\).
    \end{enumerate}
    L'ensemble des morphismes entre \( M\) et \( N\) est noté \( \Hom_A(M,N)\). Si \( B\) st une sous-anneau de \( A\),  nous parlons de \( \Hom_B(M,N)\) pour parler des morphismes de groupes qui ne vérifient \( f(a\cdot x)=a\cdot f(x)\) que pour \( a\in B\).
\end{definition}

\begin{proposition}\label{PROPooGARGooDiMqtN}
    Si \( M\) est un module sur un anneau, alors \( (M,+)\) est un groupe commutatif.
\end{proposition}

\begin{proof}
    Il suffit de calculer \( (1+1)\cdot (x+y)\) de deux façons différentes :
    \begin{equation}
        (1+1)\cdot (x+y)=1\cdot (x+y)+1\cdot (x+y)=x+y+x+y
    \end{equation}
    d'une part et
    \begin{equation}
        (1+1)\cdot (x+y)=(1+1)\cdot x+(1+1)\cdot y=x+x+y+y,
    \end{equation}
    d'autre part. En égalant les deux expressions, il vient
    \begin{equation}
        x+y+x+y=x+x+y+y,
    \end{equation}
    qui se simplifie (nous sommes dans un groupe) en \( y+x=x+y\).
\end{proof}

\begin{definition}\label{DEFooKHWZooIfxdNc}
    Un \defe{espace vectoriel}{espace!vectoriel} est un module\footnote{Définition \ref{DEFooHXITooBFvzrR}.} sur un corps commutatif\footnote{La condition de commutativité n'est pas indispensable, mais comme nous ne parlerons que de corps commutatifs\ldots}.
\end{definition}

\begin{definition}[\cite{BIBooSTWWooItiMUp}]        \label{DEFooRUKVooLnXxdS}
    Soient un \( A\)-module \( M\) et un ensemble \( I\). Une famille \( \{ m_i \}_{i\in I}\) est \defe{libre}{partie libre!module} si les \( m_i\) sont \defe{linéairement indépendants}{linéairement indépendant!module}, c'est-à-dire si pour tout choix d'une partie finie \( J\) dans \( I\) et d'éléments \( (a_j)_{j\in J}\) dans \( A\), si nous avons
    \begin{equation}
        \sum_{j\in J}a_jm_j=0,
    \end{equation}
    alors \( a_j=0\) pour tout \( j\).
\end{definition}

\begin{definition}[\cite{BIBooNKWVooYfrwSd}]        \label{DEFooWBOBooJNyyBF}
    Soit \( S\), une partie du \( A\)-module \( M\). Le \defe{sous-module engendré}{sous-module engendré} par \( S\) est l'ensemble des éléments de \( M\) qui sont des combinaisons linéaires finies d'éléments de \( S\), c'est-à-dire de sommes de la forme
    \begin{equation}
        \sum_{t\in T}a_tt
    \end{equation}
    où \( T\) est fini dans \( S\) et \( a_t\in A\).
\end{definition}

%--------------------------------------------------------------------------------------------------------------------------- 
\subsection{Module produit}
%---------------------------------------------------------------------------------------------------------------------------

\begin{lemmaDef}[\cite{BIBooSTWWooItiMUp}]        \label{DEFooLCJEooBvVmkV}
    Soient un anneau \( A\) et un ensemble \( I\). Le \( A\)-\defe{module produit}{module produit} \( A^I\) est l'ensemble des applications \( I\to A\).

    En termes de notations, nous écrivons ceci :
    \begin{equation}
        A^I=\{ (a_i)_{i\in I},a_i\in A \}.
    \end{equation}
    L'ensemble \( A^I\) devient un module par les définitions, pour \( x,y\in A^I\) et \( a\in A\) :
    \begin{subequations}
        \begin{align}
            ax&=(ax_i)_{i\in I}\\
            x+y&=(x_i+y_i)_{i\in I}     \label{EQooODBMooQKLUgd}.
        \end{align}
    \end{subequations}
    En d'autres termes, \( A^I=\Fun(I,A)\).
\end{lemmaDef}

\begin{lemma}
    Pour chaque \( i\in I\) nous considérons l'élément \( e_i\in A^I\) donné par
    \begin{equation}
        \begin{aligned}
            e_i\colon I&\to A \\
            j&\mapsto \begin{cases}
                1    &   \text{si } j=i\\
                0    &    \text{sinon. }
            \end{cases}
        \end{aligned}
    \end{equation}
    La famille \( \{ e_i \}_{i\in I}\) est libre\footnote{Définition \ref{DEFooRUKVooLnXxdS}.} dans \( A^I\).
\end{lemma}

\begin{proof}
    Soient \( J\) fini dans \( I\) ainsi que des éléments \( a_j\in A\) (\( j\in J\)). Nous supposons que\footnote{Pour rappel, les sommes finies sont définies par \ref{DEFooLNEXooYMQjRo}.} \( \sum_{j\in J}a_je_j=0\). Calculons un peu :
    \begin{equation}
        \sum_{j\in J}a_je_j=\sum_{j\in J}(a_j\delta_{ji})_{i\in I}=\left( \sum_{j\in J}a_j\delta_{ji} \right)_{i\in I}.
    \end{equation}
    Pour que le tout soit nul dans \( A^I\), il faut que
    \begin{equation}
        \sum_{j\in J}a_j\delta_{ji}
    \end{equation}
    soit nul pour tout \( i\in I\). Si nous fixons \( i\in I\), la somme sur \( j\) possède un seul terme non annulé par \( \delta_{ji}\), et c'est le terme \( j=i\). Nous avons donc \( a_i=0\).
\end{proof}

\begin{definition}      \label{DEFooBMEPooFsCHgb}
    Nous notons \( A^{(I)}\) le sous-module de \( A^I\) engendré\footnote{Définition \ref{DEFooWBOBooJNyyBF}.} par les \( e_i\).
\end{definition}

\begin{lemma}[\cite{MonCerveau}]
    L'ensemble \( A^{(I)}\) est l'ensemble des applications \( I\to A\) de support fini.
\end{lemma}

\begin{proof}
    En deux sens.
    \begin{subproof}
    \item[Si \( x\in A^{(I)}\)]
        Pour rappel, la définition \ref{DEFooLCJEooBvVmkV} nous dit que \( x\) est une application \( I\to A\). Vu que \( x\) est dans le sous-module engendré par les \( e_i\), il existe une partie finie \( J\subset I\) telle que
        \begin{equation}
            x=\sum_{j\in J}x_je_j.
        \end{equation}
        Pour \( i\in I\) nous avons
        \begin{equation}
            x(i)=\sum_{j\in J}x_j\delta_{ij}=\begin{cases}
                x_i    &   \text{si } i\in J\\
                0    &    \text{sinon. }
            \end{cases}
        \end{equation}
        Donc le support de \( x\) est dans \( J\) qui est fini. Vu que toute partie d'un ensemble fini est fini (lemme \ref{LEMooTUIRooEXjfDY}), le support de \( x\) est fini.
    \item[Si \( x\) est de support fini]
        Supposons que le support de \( x\colon I\to A\) soit la partie finie \( J\subset I\). En notant \( x_j=x(j)\) pour tout \( j\in J\), nous avons
        \begin{equation}
            x=\sum_{j\in J}x_je_j.
        \end{equation}
    \end{subproof}
\end{proof}

\begin{theorem}[Propriété universelle de \( A^{(I)}\)\cite{BIBooSTWWooItiMUp}]      \label{THOooPDZCooJnHbOd}
    Soient un anneau \( A\) ainsi qu'un \( A\)-module \( P\). Pour \( \phi\in\Hom_A(A^{(I)}, P)\), nous considérons
    \begin{equation}
        \begin{aligned}
            \phi|_I\colon I&\to P \\
            i&\mapsto \phi(e_i). 
        \end{aligned}
    \end{equation}
    \begin{enumerate}
        \item
            
    L'application
    \begin{equation}
        \begin{aligned}
            f\colon \Hom_A(A^{(I)},P)&\to \Fun(I,P) \\
            \phi&\mapsto \phi|_I 
        \end{aligned}
    \end{equation}
    est une bijection.
\item
    L'application inverse est \( g\colon \Fun(I,P)\to \Hom_A(A^{(I)},P) \) donnée par
    \begin{equation}
        g(\psi)\big( \sum_{j\in J}a_je_j \big)=\sum_{j\in J}a_j\psi(j)
    \end{equation}
    pour tout \( J\) fini dans \( I\) et choix de \( a_j\in A\).
    \end{enumerate}
\end{theorem}

\begin{proof}
    Nous allons montrer que \( g\big( f(\phi) \big)=\phi\) et que \( f\big( g(\psi) \big)=\psi\) pour tout \( \phi\in\Hom_A(A^{(I)},P)\) et pour tout \( \psi\in \Fun(I,P)\).

    Dans un premier sens nous avons :
    \begin{subequations}
        \begin{align}
            g\big( f(\phi) \big)\big( \sum_ja_je_j \big)&=\sum_ja_jf(\phi)(j)\\
            &=\sum_ja_j\phi(e_j)\label{SUBALIGNooBWPLooHeIaQg}\\
            &=\phi(\sum_ja_je_j)        \label{SUBALIGNooUOQPooCwLgZo}.
        \end{align}
    \end{subequations}
    Justifications :
    \begin{itemize}
        \item 
            Pour \eqref{SUBALIGNooBWPLooHeIaQg}, nous avons utilisé le fait que \( f(\phi)(i)=\phi|_I(i)=\phi(e_i)\).
        \item
            Pour \eqref{SUBALIGNooUOQPooCwLgZo}, nous utilisons le fait que \( \phi\) est un morphisme de modules.
    \end{itemize}
    Et pour l'autre sens,
    \begin{equation}
        f\big( g(\psi) \big)(i)=g(\psi)(e_i)=\psi(i).
    \end{equation}
    Vérifions que cela est suffisant pour que \( f\) soit une bijection.
    \begin{subproof}
    \item[Surjectif]
        Soit \( \psi\in \Fun(I,P)\). Nous avons \( f\big( g(\psi) \big)=\psi\), ce qui prouve que \( \psi\) est dans l'image de \( f\).
    \item[Injectif]
        Supposons que \( f(\phi_1)=f(\phi_2)\). Alors en appliquant \( g\) des deux côtés, il vient \( \phi_1=\phi_2\).
    \end{subproof}
\end{proof}

%--------------------------------------------------------------------------------------------------------------------------- 
\subsection{Sous-module}
%---------------------------------------------------------------------------------------------------------------------------

Soient \( M\) un \( A\)-module et \( x=(x_i)_{i\in I}\) une famille d'éléments de \( M\) paramétrée par l'ensemble \( I\). Nous considérons l'application
\begin{equation}
    \begin{aligned}
        \mu_x\colon A^{(I)}&\to M \\
        (a_i)_{i\in I}&\mapsto \sum_{i\in I}a_ix_i.
    \end{aligned}
\end{equation}
Ici \( A^{(I)}\) désigne l'ensemble de toutes les applications \( I\to A\) de support fini (définition \ref{DEFooBMEPooFsCHgb}).

\begin{definition}      \label{DefBasePouyKj}
    À l'instar des espaces vectoriels, les modules ont une notion de partie libre, génératrice et de bases :
    \begin{enumerate}
        \item
            Si \( \mu_x\) est surjective, nous disons que \( x\) est une partie \defe{génératrice}{génératrice!partie d'un module}.
        \item
            Si \( \mu_x\) est injective, nous disons que la partie \( x\) est \defe{libre}{libre!partie d'un module}.
        \item
            Si \( \mu_x\) est bijective, nous disons que la partie \( x\) est une \defe{base}{base!d'un module}.
    \end{enumerate}
\end{definition}

\begin{definition}
  Un sous-ensemble \( N\subset M\) est un \defe{sous-module}{sous-module} si \( (N,+)\) est un sous-groupe de \( (M,+)\) et si \( a\cdot x\in N\) pour tout \( x\in N\) et pour tout \( a\in A\).
\end{definition}

\begin{example}
    Un anneau \( A\) est lui-même un \( A\)-module et ses sous-modules sont les idéaux.
\end{example}

\begin{definition}
    Soit \( M\) un module sur un anneau commutatif \( A\). Un \defe{projecteur}{projecteur!dans un module} est une application linéaire \( p\colon M\to M\) telle que \( p^2=p\).

    Une famille \( (p_i)_{i\in I}\) sur \( M\) est \defe{orthogonale}{orthogonal!famille de projecteurs} si \( p_i\circ p_j=0\) pour tout \( i\neq j\). La famille est \defe{complète}{complète!famille de projecteurs} si \( \sum_{i\in I}p_i=\mtu\).
\end{definition}

\begin{theorem}     \label{ThoProjModpAlsUR}
    Soient des sous-modules \( M_1,\ldots,M_n\) du module \( M \) tels que \( M=M_1\oplus\ldots\oplus M_n\). Les applications \( p_i\) définies par
    \begin{equation}
        p_i(x_1+\ldots+x_n)=x_i
    \end{equation}
    forment une famille orthogonale de projecteurs et \( p_1+\cdots +p_n=\id\).

    Inversement, si \( (p_1,\ldots, p_n)\) est une famille orthogonale de projecteurs dans un module \( \modE\) tel que \( \sum_{i=1}^np_i=\id\), alors
    \begin{equation}
        M=\bigoplus_{i=1}^np_i(M).
    \end{equation}
\end{theorem}

\begin{definition}
    Un module est \defe{simple}{simple!module}\index{module!simple} ou \defe{irréductible}{irréductible!module}\index{module!irréductible} si il n'a pas d'autres sous-modules que \( \{ 0 \}\) et lui-même. Un module est \defe{indécomposable}{indécomposable!module}\index{module!indécomposable} si il ne peut pas être écrit comme somme directe de sous-modules.
\end{definition}

Un module simple est a fortiori indécomposable. L'inverse n'est pas vrai comme le montre l'exemple suivant.

\begin{example}
    Soit \( \modE=\eC[X]/(X^2)\) vu comme \( \eC[X]\)-module. C'est le \( \eC[X]\)-module des polynômes de la forme \( aX+b\) avec \( a,b\in \eC\). L'ensemble des polynômes de la forme \( aX\) est un sous-module. Le module \( \modE\) n'est donc pas simple. Il est cependant indécomposable parce que \( \{ aX \}\) est le seul sous-module non trivial. En effet si \( \modF\) est un sous-module de \( \modE\) contenant \( aX+b\) avec \( b\neq 0\), alors \( \modF\) contient \( X(aX+b)=bX\) et donc contient tout \( \modE\).
\end{example}

\begin{definition}[Algèbre\cite{ZSyHmiy}]   \label{DefAEbnJqI}
    Si \( \eK\) est un corps commutatif\footnote{Définition~\ref{DefTMNooKXHUd}}, une \( \eK\)-\defe{algèbre}{algèbre} \( A\) est un espace vectoriel\footnote{Définition~\ref{DEFooKHWZooIfxdNc}.} muni d'une opération bilinéaire \( \times\colon A\times A\to A\), c'est-à-dire telle que pour tout \( x,y,z\in A\) et pour tout \( \alpha,\beta\in\eK\),
    \begin{enumerate}
        \item
            \( (x+y)\times z=x\times z+y\times z\)
        \item
            \( x\times (y+z)=x\times y+x\times z\)
        \item
            \( (\alpha x)\times (\beta y)=(\alpha\beta)(x\times y)\).
    \end{enumerate}
    Si \( A\) et \( B\) sont deux \( \eK\)-algèbres, une application \( f\colon A\to B\) est un \defe{morphisme d'algèbres}{morphisme!d'algèbres} entre \( A\) et \( B\) si pour tout \( x,y\in A\) et pour tout \( \alpha\in \eK\),
    \begin{enumerate}
        \item
            \( f(xy)=f(x)f(y)\)
        \item
            \( f(x+\alpha y)=f(x)+\alpha f(y)\)
    \end{enumerate}
    où nous avons noté \( xy\) pour \( x\times y\).
\end{definition}

\begin{lemma}[\cite{MonCerveau}]   \label{LEMooVKLKooSAHmpZ}
    Soient une algèbre \( A\) et une famille \( (X_i)_{i\in I}\) de sous-algèbres de \( A\) (ici \( I\) est un ensemble quelconque). Alors la partie \( X=\bigcap_{i\in I}X_i\) est une sous-algèbre de \( A\).
\end{lemma}

\begin{proof}
    Nous devons prouver que si \( x\) et \( y\) sont dans \( X\) et \( \lambda\in \eK\), alors \( xy\), \( x+y\) et \( \lambda x\) sont dans \( X\). Pour tout \( i\in I\) nous avons \( x,y\in X_i\) et donc \( xy\in X_i\), \( x+y\in X_i\) et \( \lambda x\in X_i\) (parce que \( X_i\) est une algèbre). Donc \( xy\), \( x+y\) et \( \lambda x\) sont dans \( X_i\) pour tout \( I\), et donc dans \( X\).
\end{proof}

\begin{definition}\label{DefkAXaWY}
    L'\defe{algèbre engendrée}{algèbre!engendrée} par \( X\) est l'intersection de toutes les sous-algèbres de \( A\) contenant \( X\) (qui est une algèbre par le lemme~\ref{LEMooVKLKooSAHmpZ}).
\end{definition}



%+++++++++++++++++++++++++++++++++++++++++++++++++++++++++++++++++++++++++++++++++++++++++++++++++++++++++++++++++++++++++++
\section{Caractéristique d'un anneau}
%+++++++++++++++++++++++++++++++++++++++++++++++++++++++++++++++++++++++++++++++++++++++++++++++++++++++++++++++++++++++++++

\begin{lemmaDef}        \label{LEMDEFooVEWZooUrPaDw}
    Soit l'application
    \begin{equation}
        \begin{aligned}
            \mu\colon \eZ&\to A \\
            n&\mapsto n\cdot 1_A
        \end{aligned}
    \end{equation}
    où \( n\cdot 1_A\) signifie \( \sum_{k=1}^n1_A\).
    \begin{enumerate}
        \item
            C'est un morphisme d'anneaux.
        \item
            Le noyau est un sous-groupe de \( \eZ\)
        \item
            Il existe un unique \( p\in \eZ\) tel que \( \ker(\mu)=p\eZ\).
    \end{enumerate}
    Ce \( p\) est la \defe{caractéristique}{caractéristique!d'un anneau} de \( A\).
\end{lemmaDef}

Par exemple la caractéristique de \( \eQ\) est zéro parce qu'aucun multiple de l'unité n'est nul.

À propos de diagonalisation en caractéristique \( 2\), voir l'exemple~\ref{ExewINgYo}.

\begin{lemma}
    Si \( A\) est de caractéristique nulle, alors \( A\) est infini.
\end{lemma}

\begin{proof}
    En effet, \( \ker\mu=\{0\} \) implique que \( n1_A \neq  m1_A\) dès que \(n \neq m \) et par conséquent \( A\) contient \(\eZ 1_A \), et  est infini.
\end{proof}

\begin{lemma}       \label{LemHmDaYH}
    Si \( p\) est la caractéristique de l'anneau \( A\), alors nous avons l'isomorphisme d'anneaux
    \begin{equation}
         \eZ 1_A\simeq\eZ/p\eZ.
    \end{equation}
\end{lemma}

\begin{proof}
    L'isomorphisme est donné par l'application \( n1_A\mapsto \phi(n)\) si \( \phi\) est la projection canonique \( \eZ\to \eZ/p\eZ\).
\end{proof}

\begin{proposition}     \label{PropGExaUK}
    La caractéristique d'un anneau fini divise son cardinal.
\end{proposition}

\begin{proof}
    Si \( A\) est un anneau, le groupe \( \eZ\) agit sur \( A\) par
    \begin{equation}
        n\cdot a=a+n1_A.
    \end{equation}
    Chaque orbite de cette action est de la forme
    \begin{equation}
        \mO_a=\{ a+n1_A\tq n=0,\ldots, p-1 \}
    \end{equation}
    où \( p\) est la caractéristique de \( A\). Les orbites ont \( p\) éléments et forment une partition de \( A\), donc le cardinal de \( A\) est un multiple de \( p\).
\end{proof}

\begin{lemma}[\cite{ooIBWOooSjOvXd}]        \label{LEMooJQIKooQgukqn}
    Un anneau totalement ordonné est de caractéristique nulle.
\end{lemma}

\begin{proof}
    Le morphisme \( \mu\colon \eZ\to A\), \( n\mapsto n 1_A\) est strictement croissant, en particulier \( \mu(x)\neq \mu(y)\) dès que \( x\neq y\). Donc \( \ker(\mu)=\{ 0 \}\).
\end{proof}

L'ensemble typique de caractéristique \( p\) est \( \eF_p=\eZ/p\eZ\).

\begin{proposition} \label{PropFrobHAMkTY}
    Soit \( A\) un anneau commutatif unitaire de caractéristique \( p\). L'application
    \begin{equation}
        \begin{aligned}
            \Frob_A\colon A&\to A \\
            x&\mapsto x^p
        \end{aligned}
    \end{equation}
    est un automorphisme d'anneau unitaire.
\end{proposition}
Nous le nommons le \defe{morphisme de Frobenius}{morphisme!Frobenius}\index{Frobenius!morphisme}. Nous utiliserons aussi les itérés du morphisme de Frobenius : \( \Frob^k\colon x\mapsto x^{p^k}\).

\begin{example}
    Soit à factoriser \( X^p-1\) dans \( \eF_p\). Grâce au morphisme de Frobenius, nous avons immédiatement
    \begin{equation}
        X^p-1=(X-1)^p.
    \end{equation}
\end{example}


\begin{lemma}       \label{LemCaractIntergernbrcartpre}
    La caractéristique\footnote{Définition~\ref{LEMDEFooVEWZooUrPaDw}.} d'un anneau intègre est zéro ou un élément premier\footnote{Définition \ref{DEFooZCRQooWXRalw}.}.
\end{lemma}

\begin{proof}
    Si \( A\) est intègre, alors \( \eZ 1_A\) est a fortiori intègre. Notons \( p \) la caractéristique de \( A \). Si \( p = 0 \), la preuve est finie; supposons donc que \( p \neq 0 \). Alors, l'anneau \( \eZ/p\eZ\) est isomorphe à \( \eZ 1_A\), et est donc intègre. Or, la proposition~\ref{CorZnInternprem} dit que \( \eZ/p\eZ\) est intègre si et seulement si \( p\) est premier, ce qui conclut la preuve.
\end{proof}

\begin{example}
    Il existe des corps dont la caractéristique n'est pas égale au cardinal (contrairement à ce que laisserait penser l'exemple des \( \eZ/p\eZ\)). En effet les matrices \( n\times n\) inversibles sur \( \eF_{3}\) forment un corps qui n'est pas de cardinal trois alors que la caractéristique est \( 3\) :
    \begin{equation}
        \begin{pmatrix}
            1    &       \\
                &   1
            \end{pmatrix}+\begin{pmatrix}
                1    &       \\
                    &   1
                \end{pmatrix}+\begin{pmatrix}
                    1    &       \\
                        &   1
                \end{pmatrix}=0.
    \end{equation}
\end{example}


%--------------------------------------------------------------------------------------------------------------------------- 
\subsection{Caractéristique deux}
%---------------------------------------------------------------------------------------------------------------------------

Beaucoup de résultats demandent une caractéristique différente de deux. Qu'a donc de particulier la caractéristique deux ?

Si \( \eK\) est un corps de caractéristique \( 2\), alors l'égalité \( x=-x\) n'implique pas \( x=0\), puisque \( 2x=0\) est vérifiée pour tout \( x\). Cela se répercute sur un certain nombre de résultats. Par exemple, en caractéristique deux, une forme antisymétrique n'est pas toujours alternée: voir le lemme~\ref{LemHiHNey}.

%+++++++++++++++++++++++++++++++++++++++++++++++++++++++++++++++++++++++++++++++++++++++++++++++++++++++++++++++++++++++++++
\section{Polynômes}
%+++++++++++++++++++++++++++++++++++++++++++++++++++++++++++++++++++++++++++++++++++++++++++++++++++++++++++++++++++++++++++

%--------------------------------------------------------------------------------------------------------------------------- 
\subsection{Polynômes d'une variable}
%---------------------------------------------------------------------------------------------------------------------------

Et voilà la définition que tout le monde attendait; la définition des anneaux de polynômes. Pour ne pas taper trop fort du premier coup, nous commençons par les polynômes d'une seule variable\footnote{Pour les polynômes à plusieurs variables, voir la définition \ref{DEFooZNHOooCruuwI}.}.


L'ensemble des polynômes sur \( A\) sera simplement \( A^{(\eN)}\) (notation \ref{DEFooBMEPooFsCHgb}). Puisque \( \eN\) est un ensemble bien particulier possédant plein de structure, nous allons pouvoir installer sur \( A^{(\eN)}\) une structure non seulement de \( A\)-module (ça c'est déjà fait), mais en plus d'anneau, ainsi qu'une évaluation.
\begin{definition}      \label{DEFooFYZRooMikwEL}
    L'ensemble des \defe{polynômes}{polynômes} en une indéterminée sur l'anneau \( A\) est l'anneau 
    \begin{equation}
        \polyP(A)=A^{(\eN)}
    \end{equation}
    défini en \ref{DEFooBMEPooFsCHgb}.
\end{definition}

\begin{normaltext}
    En ce qui concerne la notation \( A[X]\), voir \ref{SUBSECooLEKVooFBPSJz}. Pour \( \eK(X)\) lorsque \( \eK\) est un corps, voir~\ref{DEFooQPZIooQYiNVh}.
\end{normaltext}

\begin{propositionDef}[\cite{BIBooZDRQooLrUVrb}]  \label{DefDegrePoly}
    Soit \( P\) non nul dans \(\polyP(A)\). Nous notons \( a_n\) la valeur\footnote{Ici il y a une énorme subtilité de terminologie. Formellement, \( P\) est une application \( \eN\to A\). Cela n'a rien à voir avec le fait que \( P\) puisse être évalué sur \( A\) avec des formule du type \( P(x)=\sum_na_nx^n\). D'ailleurs nous n'avons pas encore vu cette évaluation.} de \( P\) en \( n\in \eN\) : \( P=(a_n)_{n\in \eN}\).
    \begin{enumerate}
        \item
            L'ensemble \( \{ n\in \eN\tq a_n\neq 0 \}\) est fini dans \( \eN\).
        \item
            Cet ensemble possède un minimum et un maximum.
    \end{enumerate}
    Le \defe{degré}{degré d'un polynôme} de $P$ est
    \begin{equation}
        \deg(P)=\max\{ n\in \eN\tq a_n\neq 0 \},
    \end{equation}
    et la \defe{valuation}{valuation d'un polynôme} de \( P\) est
    \begin{equation}
        \val(P)=\min\{ n\tq a_n\neq 0 \}.
    \end{equation}
    Dans le cas du polynôme nul, l'ensemble \( \{ n\in \eN\tq a_n\neq 0 \}\) est vide, et les définitions ne s'appliquent pas. Nous convenons que
    \begin{subequations}
        \begin{align}
            \val(0)&=+\infty\\           
            \deg(0)&=-\infty.
        \end{align}
    \end{subequations}
\end{propositionDef}

\begin{proof}
    Le fait que \( P\) soit non nul implique que \( A=\{ n\in \eN\tq a_n\neq 0 \}\) est non vide. De plus cet ensemble est fini parce que \( P\in A^{(\eN)}\). Toute partie finie non vide de \( \eN\) étant majorée et minorée (lemme \ref{LEMooKUWUooPLWelf}), le lemme \ref{LEMooOEJOooOgaxzi} définit correctement le minimum et le maximum de \( A\).
\end{proof}

Vu que \( A^{(\eN)}\) est engendré par les \( e_i\), tout polynôme sur \( A\) s'écrit \( P=\sum_{i=1}^na_ie_i\).

\begin{definition}      \label{DEFooNXKUooLrGeuh}
    Nous ajoutons deux structures à \( A^{(\eN)}\).
    \begin{description}
        \item[L'évaluation] Si \( \alpha\in A\) et si \( P\in A^{(\eN)}\), nous définissons \( P(\alpha)\) par
            \begin{equation}        \label{EQooDJISooTEkMOw}
                P(\alpha)=(\sum_{i=0}^{n}a_ie_i)(\alpha)=\sum_{i=0}^na_i\alpha^i,
            \end{equation}
            étant entendu que \( \alpha^0=1\) dans \( A\).

            Cette définition s'étend immédiatement au cas où \( B\) est un anneau qui étend \( A\). Dans ce cas nous pouvons définir \( P(b)\) pour tout \( P\in A^{(\eN)}\) et \( b\in B\) avec la même formule \eqref{EQooDJISooTEkMOw}.
        \item[Le produit] C'est ici que la structure particulière de \( \eN\) est utilisée. Nous définissons le produit \( A^{\eN}\times A^{(\eN)}\to A^{(\eN)}\) de la façon suivante. Si \( (P_k)_{k\in \eN}\) est la suite (presque partout nulle) d'éléments de \( A\) qui définit \( P\) et si \( (Q_k)_{k\in \eN}\) est celle de \( Q\), nous notons
        \begin{equation}    \label{EQooTNCSooKklisb}
            (PQ)_n=\sum_{k=0}^nP_kQ_{n-k},
        \end{equation}
        et donc \( PQ=\sum_i(PQ)_ie_i\). Plus explicitement,
        \begin{equation}    \label{EQooCIBUooAgpxjL}
            (\sum_{i=0}^na_ie_i)(\sum_{j=0}^mb_je_j)=\sum_{k=0}^{\infty}\Big( \sum_{\substack{  (i,j)\in \eN^2 \\i+j=k}}a_ib_j \Big)e_k.
        \end{equation}
        Notons qu'à droite, la somme sur \( k\) est une somme finie.
    \end{description}
\end{definition}

\begin{proposition}     \label{PROPooGDQCooHziCPH}
    Soit un anneau \( A\). À propos de structure sur \( A^{(\eN)}\).
    \begin{enumerate}
        \item
            Avec le produit, l'ensemble \( A^{(\eN)}\) devient un anneau.
        \item
    L'application
    \begin{equation}
        \begin{aligned}
            g\colon A^{(\eN)}&\to A \\
            P&\mapsto P(\alpha)
        \end{aligned}
    \end{equation}
    est un morphisme d'anneaux\footnote{Définition \ref{DEFooSPHPooCwjzuz}.}. En particulier, \( (PQ)(\alpha)=P(\alpha)Q(\alpha)\).
    \end{enumerate}
\end{proposition}

\begin{proof}
    En plusieurs points
    \begin{subproof}
        \item[Anneau]
            L'identité pour le produit dans \( A^{(\eN)}\) est le polynôme donné par \( a_0=1\) et \( a_i=0\) pour \( i\neq 0\). Cela se vérifie en utilisant directement la définition \eqref{EQooCIBUooAgpxjL}. La distributivité aussi\quext{Je n'ai pas fait les calculs, écrivez-moi pour me dire si ça va facilement.}.
        \item[Le morphisme]
    Nous notons \( P_k\) les éléments de la suite définissant \( P\) et \( Q_k\) ceux de \( Q\). Alors nous avons
    \begin{equation}
        (P+Q)(\alpha)=\sum_k(P_k+Q_k)\alpha^k=\sum_kP_k\alpha^k+\sum_kQ_k\alpha^k=P(\alpha)+Q(\alpha).
    \end{equation}
    Vous aurez noté que la première égalité était la définition \eqref{EQooODBMooQKLUgd}. De même,
    \begin{subequations}
        \begin{align}
            P(\alpha)Q(\alpha)&=\big( \sum_nP_n\alpha^n \big)\big( \sum_kQ_k\alpha^k \big)=\sum_kQ_k\big( \sum_nP_n\alpha^n \big)\alpha^k=\sum_k\sum_nQ_kP_n\alpha^{n+k}\\
            &=\sum_m\big( \sum_{l=0}^mP_lQ_{m-l} \big)\alpha^m=\sum_m(PQ)_m\alpha^m=(PQ)(\alpha).
        \end{align}
    \end{subequations}
    \end{subproof}
\end{proof}

\begin{lemma}       \label{LEMooWVUXooQlaepO}
    Si \( A\) est commutatif, alors \( A^{(\eN)}\) est commutatif.
\end{lemma}
%TODOooRDJYooNAqcYr: préciser si A est un anneau ou autre chose.

\begin{proof}
    Soient \( P,Q\in A^{(\eN)}\); pour rappel, le produit est donné par la définition \ref{EQooTNCSooKklisb}. L'application
    \begin{equation}
        \begin{aligned}
            \varphi\colon \{ 0,\ldots, n \}&\to \{ 0,\ldots, n \} \\
            k&\mapsto n-k 
        \end{aligned}
    \end{equation}
    est une bijection. Voici maintenant le calcul :
    \begin{subequations}
        \begin{align}
            (PQ)_n&=\sum_{k=0}^nP_kQ_{n-k}\\
            &=\sum_{k=0}^nP_{\varphi(k)}Q_{n-\varphi(k)}    \label{SUBEQooISTNooLPvSIy} \\
            &=\sum_{k=0}^nP_{n-k}Q_{k}\\
            &=\sum_{k=0}^nQ_lP_{n-k}      \label{SUBEQooCUMAooFjqqHW}\\
            &=(QP)_n.
        \end{align}
    \end{subequations}
    Justifications
    \begin{itemize}
        \item Pour \eqref{SUBEQooISTNooLPvSIy}. Lemme \ref{DEFooLNEXooYMQjRo} et le fait que \( \varphi\) soit une bijection.
        \item Pour \eqref{SUBEQooCUMAooFjqqHW}. Commutativité de \( A\).
    \end{itemize}
\end{proof}


%--------------------------------------------------------------------------------------------------------------------------- 
\subsection{La notation \texorpdfstring{$ A[X]$}{A[X]}}
%---------------------------------------------------------------------------------------------------------------------------
\label{SUBSECooLEKVooFBPSJz}

Si \( A\) est un anneau, nous avons déjà défini les polynômes en une indéterminée sur \( A\) comme étant le module \( A^{(\eN)}\) qui est devenu un anneau par la proposition \ref{PROPooGDQCooHziCPH}.

Le polynôme donné par la suite \( (a_n)_{n\in \eN}\) est souvent notée
\begin{equation}
    \sum_ka_kX^k.
\end{equation}
Par exemple avec \( a=(4,2,8)\) nous avons \( a=8X^2+2X+4\). Nous utiliserons souvent cette notation, qui est très pratique parce qu'elle s'adapte bien aux règles de multiplication et d'addition, en particulier la distributivité.

Il y a (au moins) deux façons de comprendre ce que signifie réellement «\( X\)» dans cette notation.

%///////////////////////////////////////////////////////////////////////////////////////////////////////////////////////////
\subsubsection{Première façon (qui botte en touche)}
%///////////////////////////////////////////////////////////////////////////////////////////////////////////////////////////

La première est de dire qu'il n'a pas de significations, et que \( X^2\) est un simple abus de notations pour écrire \( (0,0,1,0,\cdots)\). Avec cette façon de voir, nous notons l'anneau des polynômes sur \( A\) par «\( A[X]\)» où le \( X\) n'a pas d'autres raisons d'être que d'avertir le lecteur que nous réservons la lettre «\( X\)» pour utiliser la notation pratique des polynômes.

%///////////////////////////////////////////////////////////////////////////////////////////////////////////////////////////
\subsubsection{Seconde façon (la bonne)}
%///////////////////////////////////////////////////////////////////////////////////////////////////////////////////////////
\label{SUBSUBSECooPNBYooWXEHrg}

\begin{normaltext}      \label{NORMooHHIVooSfHlxv}
    La seconde façon de voir le «\( X\)» est de nous rappeler que \( A^{(\eN)}\) a une base en tant que module : les \( e_k\) dont nous avons parlé plus haut. Nous posons \( X=e_1\), et nous prenons la convention \( X^0=1\). Alors nous avons \( e_k=X^k\) et nous notons \( A[X]\)\nomenclature[A]{\( A[X]\)}{tous les polynômes de degré fini à coefficients dans \( A\)} l'anneau \(A^{(\eN)}\) exprimé avec \( X\).

    Dans les deux cas, il n'est pas vraiment légitime d'écrire des égalités comme « \( P(X)=X^2+2X-3\) », et encore moins de dire «Le polynôme \( P\), \emph{évalué} en \( X\) vaut \( X^2+2X-3\)»  : il est plus correct d'écrire « \( P=X^2+2X-3\) ».

    Le lemme suivant montre que ces notations tombent vraiment à point. La véritable difficulté de l'énoncé est de comprendre qu'il n'est pas trivial.

    Nous avons vu dans la définition \ref{DEFooNXKUooLrGeuh} que si \( B\) est un anneau qui étend \( A\), et si \(P\in A[X] \), alors nous avons une définition de \( P(b)\) pour tout \( b\in B\). Nous appliquons cela à \( B=A[X]\), qui est un anneau qui étend \( A\). Autrement dit, si \( P\) et \( Q\) sont des polynômes, ça a un sens d'écrire \( P(Q)\) et le résultat sera un élément de \( A[X]\). 
\end{normaltext}

Dans le cas particulier \( Q=X\), nous avons une chouette formule.
\begin{lemma}       \label{LEMooGKWQooVOyeDX}
    Nous avons
    \begin{equation}
        P(X)=P
    \end{equation}
    pour tout \( P\in A[X]\).
\end{lemma}

\begin{proof}
    Si \( P=(a_k)_{k\in \eN}\) alors par définition \( P(\alpha)=\sum_ka_k\alpha^k\) dès que \( \alpha\) est dans un anneau \( B\) qui étend \( A\). Nous considérons le cas particulier \( B=A[X]\) et \( \alpha=X\), c'est-à-dire \( Q=(0,1,0,\ldots)\), l'élément \( P(X)\) de \( A[X]\) vaut
    \begin{equation}        \label{EQooABULooFCEasf}
        \sum_ka_kX^k,
    \end{equation}
    qui est exactement \( P\) lui-même.
\end{proof}

Mais il faut bien comprendre que si \( P\) est le polynôme \( (-3,2,1,0,\ldots)\), noté \( X^2+2X-3\), écrire \( P(X)=X^2+2X-3\) est une pirouette de notations que rien ne justifie par rapport à simplement écrire \( P=X^2+2X-3\).



%--------------------------------------------------------------------------------------------------------------------------- 
\subsection{Action du groupe symétrique}
%---------------------------------------------------------------------------------------------------------------------------

\begin{definition}[Thème~\ref{THEMEooKZHBooRCULcr}]  \label{DefActionGroupe}
    Une \defe{action de groupe}{action}\index{action} \( G\) sur un ensemble \( E\) est la donnée, pour chaque élément \( g \in G\), d'une fonction \(\phi_g : E \to E \), de telle sorte que:
    \begin{gather*}
        \phi_{e}(x) = x, \hspace{2em} \forall x \in E;\\
        \phi_{gh}(x) = \phi_g (\phi_h (x)),  \hspace{2em} \forall g,h \in G, \forall x \in E.
     \end{gather*}
     On dit dans ce cas que \( G \) \defe{agit}{action} sur \( E \).
\end{definition}

Par souci de notations, nous notons \( \Poly_n(A)\) l'anneau des polynômes de \( n\) variables sur \( A\). La propriété universelle de \( \Poly_n(A)=A^{(\eN^n)}\) du théorème \ref{THOooPDZCooJnHbOd} nous donne une application
\begin{equation}
    g\colon \Fun\big(\eN^n,\Poly_n(A)\big)\to \Hom_A\big( \Poly_n(A),\Poly_n(A) \big)
\end{equation}
Avec cela nous pouvons énoncer et démontrer le lemme qui donne l'action de \( S_n\)\footnote{Définition du groupe symétrique \( S_n\) en \ref{DEFooJNPIooMuzIXd}.} sur \( \Poly_n(A)\).

\begin{lemma}[\cite{BIBooFDZDooJQLjlB}]       \label{LEMooIRVQooHvoNBq}
    Pour \( \sigma\in S_n\) nous définissons 
    \begin{equation}
        \begin{aligned}
            \phi_{\sigma}\colon \eN^n&\to \Poly_n(A) \\
            m&\mapsto e_{\sigma(m)}. 
        \end{aligned}
    \end{equation}
    Alors l'application
    \begin{equation}
        \begin{aligned}
            \rho\colon S_n&\to \Hom_A\big( \Poly_n(A),\Poly_n(A) \big) \\
            \sigma&\mapsto g(\phi{\sigma}) 
        \end{aligned}
    \end{equation}
    est une action\footnote{Définition \ref{DefActionGroupe}.}.
\end{lemma}

\begin{proof}
    Nous commençons par donner une expression à notre \( \rho\). Un élément de \( \Poly_n(A)\) est de la forme \( \sum_{m\in \eN^n}a_me_m\), et nous avons\footnote{La somme est définie par \ref{DEFooLNEXooYMQjRo}, et ça va être important. Ah oui, en réalité partout, les sommes sont finies parce que les \( a_m\) (\( m\in \eN^n\)) sont presque tous nuls. Il faudrait écrire sur la somme sur \(\{ m\in \eN^2\tq a_m\neq 0 \}\), mais vous vous imaginez la complication dans la notation.}
    \begin{equation}
        \rho(\sigma)\big( \sum_{m\in \eN^n}a_me_m \big)=\sum_ma_m\phi_{\sigma}(m)=\sum_ma_me_{\sigma(m)}.
    \end{equation}
    
    Nous avons tout de suite \( \rho(\id)=\id\).

    En ce qui concerne la composition, nous avons d'une part
    \begin{equation}
        \rho(\sigma_1\sigma_2)\big( \sum_ma_me_m \big)=g(\phi_{\sigma_1\sigma_2})\big( \sum_ma_me_m \big)=\sum_ma_me_{\sigma_1\sigma_2(m)},
    \end{equation}
    et d'autre part,
    \begin{subequations}
        \begin{align}
            \rho(\sigma_1)\rho(\sigma_2)\big( \sum_ma_me_m \big)&=\rho(\sigma_1)\big( \sum_ma_me_{\sigma_2(m)} \big)\\
            &=\rho(\sigma_1)\big( \sum_ma_{\sigma_2^{-1}(m)}e_m \big)   \label{SUBEQooTSCYooCUWiRz}\\
            &=\sum_ma_{\sigma_2^{-1}(m)}e_{\sigma_1(m)}\\
            &=\sum_ma_me_{\sigma_1\sigma_2(m)}      \label{SUBEQooQPGPooVvqJdT}
        \end{align}
    \end{subequations}
    La proposition \ref{PROPooJBQVooNqWErk} est utilisée pour \eqref{SUBEQooTSCYooCUWiRz} et pour \eqref{SUBEQooQPGPooVvqJdT}.
\end{proof}


%---------------------------------------------------------------------------------------------------------------------------
\subsection{Corps des fractions}
%---------------------------------------------------------------------------------------------------------------------------

\begin{definition}[\cite{ooGSDHooLgtHCb}]       \label{DEFooGJYXooOiJQvP}
    Soit un anneau commutatif et intègre\footnote{Définition~\ref{DEFooTAOPooWDPYmd}.} \( A\). Nous posons \( E=A\times A\setminus\{ 0 \}\), et nous définissons les deux opérations suivantes sur \( E\) :
    \begin{enumerate}
        \item       \label{ITEMooWBWHooYsXFkO}
            \( (a,b)+(c,d)=(ad+cb,bd)\);
        \item       \label{ITEMooGOOIooCHqLRl}
            \( (a,b)(c,d)=(ac,bd)\).
    \end{enumerate}
    Et aussi la relation d'équivalence \( (a,b)\sim(c,d)\) si et seulement si \( ad=bc\).

    Le \defe{corps des fractions}{corps!des fractions} de \( A\) est le quotient
    \begin{equation}
        \Frac(A)=\big( A\times A\setminus\{ 0 \} \big)/\sim.
    \end{equation}
    Nous notons \( a/b\) la classe de \( (a,b)\).

    Lorsque \( A\) est un anneau de polynômes\footnote{Définition \ref{DEFooFYZRooMikwEL}.}, alors les éléments de \( \Frac(A)\) sont des \defe{fractions rationnelles}{fractions!rationnelles}.
\end{definition}
Le fait que \( A\) soit intègre est important pour être certain que \( bd\neq 0\) sous l'hypothèse que \( b,d\neq 0\).

La proposition suivante montre encore que le corps des fractions est le plus petit corps que l'on puisse imaginer à partir d'un anneau.
\begin{proposition}[\cite{BIBooZFPUooIiywbk, MonCerveau}]       \label{PROPooIJBEooDjsoHr}
    Soit un anneau commutatif \( A\). Tout corps commutatif contenant un sous-anneau isomorphe\footnote{Morphisme d'anneaux, définition \ref{DEFooSPHPooCwjzuz}.} à \( A\) contient un sous-corps isomorphe à \( \Frac(A)\).
\end{proposition}

\begin{proof}
    Soit un corps \( \eK\) contenant un sous-anneau \( A'\) isomorphe à \( A\). Nous notons \( \sigma\colon A'\to A\) un isomorphisme d'anneaux entre \( A'\) et \( A\). 

    \begin{subproof}
    \item[Une partie bien choisie]

    Nous considérons la partie suivante de \( \eK\) :
    \begin{equation}
        S=\{ ab^{-1}\tq a,b\in A' \}.
    \end{equation}

\item[\( S\) est un corps]
Deux éléments arbitraires de \( S\) sont \( ab^{-1}\) et \( xy^{-1}\). Nous devons prouver plusieurs choses.
\begin{subproof}
\item[Neutres]
    En prenant \( a=b=1\) nous avons \( ab^{-1}=1\in S\). En prenant \( a=0\) et \( b=1\) nous avons \( ab^{-1}=0\in S\).
\item[Somme]
    Il faut remarquer que \( ab^{-1}+xy^{-1}=(ay+xb)(by)^{-1}\). En effet,
    \begin{subequations}
        \begin{align}
            (ay+xb)(xb)^{-1}&=(ay+xb)y^{-1}b^{-1}\\
            &=ayy^{-1}b^{-1}+xby^{-1}b^{-1}     \label{SUBEQooRGPSooXaBGyx}\\
            &=ab^{-1}+xy^{-1}       \label{SUBEQooOHJGooWrfPow}
        \end{align}
    \end{subequations}
    Justifications :
    \begin{itemize}
        \item Pour \eqref{SUBEQooRGPSooXaBGyx}. Distributivité.
        \item Pour \eqref{SUBEQooOHJGooWrfPow}. Commutativité dans \( A\).
    \end{itemize}
\item[Produit]
    Il s'agit du même genre de calculs en utilisant les mêmes propriétés. Nous avons
    \begin{equation}
        (ab^{-1})(xy^{-1})=(ax)(by)^{-1}.
    \end{equation}
\end{subproof}

\item[Ce qui va être notre isomorphisme]


    Ensuite nous montrons que l'application
    \begin{equation}
        \begin{aligned}
            \varphi\colon S&\to \Frac(A) \\
            ab^{-1}&\mapsto \sigma(a)/\sigma(b)
        \end{aligned}
    \end{equation}
    est bien définie et est un isomorphisme de corps.

        \item[Bien définie]

            Si \( ab^{-1}=xy^{-1}\) alors \( ay=xb\). Puisque \( \sigma\) est un isomorphisme nous avons aussi \( \sigma(a)\sigma(y)=\sigma(x)\sigma(b)\) et donc \( \sigma(a)/\sigma(b)=\sigma(x)/\sigma(y)\) par définition des classes de \( \Frac(A)\).
        \item[Morphisme]
            Deux éléments arbitraires de \( S\) sont \( ab^{-1}\) et \( xy^{-1}\). Calculons un peu :
            \begin{subequations}
                \begin{align}
                    \varphi\big( (ab^{-1})(xy^{-1}) \big)&=\varphi(axy^{-1}b^{-1})      \label{SUBEQooRONTooKVTRdZ}\\
                    &=\varphi\big( (ax)(by)^{-1} \big)      \label{SUBEQooNOTAooZVJymC}\\
                    &=\sigma(ax)/\sigma(by)\\
                    &=\big(\sigma(a)/\sigma(b)\big)\big(\sigma(x)/\sigma(y)\big)            \label{SUBEQooVQUOooVyVjEU} \\
                    &=\varphi(ab^{-1})\varphi(xy^{-1}).
                \end{align}
            \end{subequations}
            Justifications :
            \begin{itemize}
                \item Pour \eqref{SUBEQooRONTooKVTRdZ}. Commutativité dans \( A\).
                \item Pour \eqref{SUBEQooNOTAooZVJymC}. Associativité dans \( A\).
                \item Pour \eqref{SUBEQooVQUOooVyVjEU}. Définition \ref{DEFooGJYXooOiJQvP}\ref{ITEMooGOOIooCHqLRl} de la multiplication de fractions.
            \end{itemize}


        \item[Surjectif]

            Tout élément de \( \Frac(A)\) est de la forme \( a'/b'\) avec \( a',b'\in A\), et donc de la forme \( \sigma(a)/\sigma(b)\) avec \( a,b\in A'\). Un tel élément est l'image par \( \varphi\) de \( ab^{-1}\in S\).

        \item[Injectif]

            Si \( \varphi(ab^{-1})=\varphi(xy^{-1})\) alors \( \sigma(a)/\sigma(b)=\sigma(x)/\sigma(y)\), et par définition des classes nous avons \( \sigma(a)\sigma(y)=\sigma(b)\sigma(x)\). De là nous avons \( \sigma(ay)=\sigma(bx)\) et donc \( ay=bx\) (parce que \( \sigma\) est un isomorphisme). Nous en déduisons que \( ab^{-1}=xy^{-1}\).
    \end{subproof}
\end{proof}

\begin{normaltext}
    Soit un anneau \( A\) et son anneau des polynômes \( \Poly(A)\). Si \( \alpha\in A\), nous avons la définition \ref{DEFooNXKUooLrGeuh} qui donne l'évaluation \( P(\alpha)\).

    Si par contre \( P\) et \( Q\) sont des polynômes sur \( A\), nous n'avons pas encore défini ce que serait l'évaluation de la fraction rationnelle \( P/Q\) en \( \alpha\). Nous comblons à présent ce manque.
\end{normaltext}

\begin{definition}[Évaluation d'une fraction rationnelle]       \label{DEFooLBIWooCPCaSY}
    Soit un corps \( \eK\) contenant l'anneau \( A\). Si \( R=P/Q\in \Frac(A)\) et si \( \alpha\in \eK\) nous définissons\footnote{Les fractions rationnelles, définition \ref{DEFooGJYXooOiJQvP}.}
    \begin{equation}
        R(\alpha)=(P/Q)(\alpha)=P(\alpha)Q^{-1}(\alpha).
    \end{equation}
    Dans cette formule, les polynômes, l'inverse et le produit sont calculés dans \( \eK\) et non dans \( A\).
\end{definition}

\begin{theoremDef}     \label{ThogbhWgo}
    Soit \( A\) un anneau commutatif intègre.

    \begin{enumerate}
        \item
    Il existe un couple \( (\eK,\epsilon)\) où \( \eK\) est un corps commutatif et \( \epsilon\colon A\to \eK\) est un morphisme injectif d'anneaux tels que pour tout \( \lambda\in\eK\), il existe \( (a,b)\in A\times A^*\) tels que
    \begin{equation}
        \lambda=\epsilon(a)\big( \epsilon(b) \big)^{-1}
    \end{equation}
\item
    Si \( (\eK',\epsilon')\) est un autre couple qui vérifie la propriété, les corps \( \eK\) et \( \eK'\) sont isomorphes.

    Le corps \( \eK\) associé à l'anneau \( A\) est le \defe{corps des fractions}{corps!des fractions}\index{fractions (corps)} de \( A\), et sera noté \( \Frac(A)\).\nomenclature[A]{\( \Frac(A)\)}{Le corps des fractions de l'anneau \( A\)}

\item
    Nous posons
    \begin{equation}
        \begin{aligned}
            \sigma\colon A\times A^*&\to \eK \\
            (a,b)&\mapsto \epsilon(a)\big( \epsilon(b) \big)^{-1}. 
        \end{aligned}
    \end{equation}
    Nous avons
    \begin{equation}
        \sigma(xa, xb)=\sigma(a,b)
    \end{equation}
    pour tout \( a,b,x\in A\).
    \end{enumerate}
\end{theoremDef}

%---------------------------------------------------------------------------------------------------------------------------
\subsection{Corps totalement ordonné}
%---------------------------------------------------------------------------------------------------------------------------

\begin{definition}      \label{DefKCGBooLRNdJf}
    Ordre et choses reliées dans un corps.
    \begin{enumerate}
        \item \label{ITEMooOOOVooJWwIQr}
            Un corps \( \eK\) est \defe{totalement ordonné}{ordre!dans un corps}\index{corps!ordonné} si il existe une relation d'ordre total\footnote{Définition~\ref{DEFooVGYQooUhUZGr}.} tel que
            \begin{enumerate}
                \item       \label{ITEMooZISJooWNxnBj}
                    \( x\leq y\) implique \( x+z\leq y+z\) pour tout \( x,y,z\in \eK\)
                \item   \label{CONDooBYYDooElXgPO}
                    \( x\geq 0\) et \( y\geq 0\) implique \( xy\geq 0\).
            \end{enumerate}
        \item       \label{ItemooWUGSooRSRvYC}
            Si \( \eK\) est un corps totalement ordonné, nous y définissons la valeur absolue par
            \begin{equation}     
                | x |=\begin{cases}
                    x    &   \text{si }x\geq 0\\
                    -x    &    \text{si } x\leq 0.
                \end{cases}
            \end{equation}
        \item       \label{ItemVXOZooTYpcYN}
    La suite \( (x_n)\) dans le corps totalement ordonné \( \eK\) est \defe{de Cauchy}{suite!de Cauchy!dans un corps} si pour tout \( \epsilon\in \eK^+\), il existe \( N\in \eN\) tel que si \( p,q\geq N\) alors \( | x_p-x_q |\leq \epsilon\).
\item       \label{ITEMooDERQooLmJwFR}
    La suite \( (x_n)\) dans le corps totalement ordonné \( \eK\) est \defe{convergente}{convergence!suite!dans un corps} si il existe \( q\in \eK\) tel que pour tout \( \epsilon\in \eK^+\), il existe \( N\) tel que si \( k\geq N\) alors \( | x_k-q |\leq \epsilon\).

        \item       \label{ITEMooKZZYooDaidGU}
            Un corps totalement ordonné est \defe{complet}{corps!complet}\index{complet!corps} si toute suite de Cauchy y est convergente.
        \item       \label{ITEMooMWASooEzhVyh}
            Si \( a,\epsilon\in \eK\) avec \( \epsilon>0\) alors nous définissons la \defe{boule ouverte}{boule dans un corps} de centre \( a\) et de rayon \( \epsilon\) par
            \begin{equation}
                B(a,\epsilon)=\{ x\in \eK\tq | a-x |<\epsilon \},
            \end{equation}
            et la \defe{boule fermée}{boule dans un corps} par
            \begin{equation}
                \overline{ B(a,\epsilon) }=\{ x\in \eK\tq | a-x |\leq \epsilon \}.
            \end{equation}
    \end{enumerate}
\end{definition}

\begin{lemma}
    Une suite \( (x_k)\) converge vers \( q\) si et seulement si pour tout \( \epsilon>0\), il existe \( N>0\) tel que \( x_k\in B(q,\epsilon)\) pour tout \( k\geq N\).
\end{lemma}

\begin{proof}
    Il s'agit de mettre côte à côte les points~\ref{ITEMooDERQooLmJwFR} et~\ref{ITEMooMWASooEzhVyh} de la définition \ref{DefKCGBooLRNdJf}.
\end{proof}

\begin{normaltext}
    Ces boules prendront une nouvelle force avec le super-théorème~\ref{ThoORdLYUu}.
\end{normaltext}

Parmi ces définitions, celles de suite convergente, de Cauchy et de corps complet seront utilisées dans le cas de \( \eQ\) (et de \( \eR\) pour la complétude). Elles seront prouvées être équivalentes aux définitions topologiques dans le cas particulier de \( \eR\) et \( \eQ\) lorsque la topologie métrique sera définie. Dans cet état d'esprit nous n'allons pas démontrer tout de suite que \( \eR\) est un corps complet. Nous allons directement démontrer que c'est un espace topologique complet.

\begin{lemma}[Règle des signes\cite{ooTKEHooQuaFuD}]        \label{LEMooXJTAooZauchx}
    Soit un corps totalement ordonné \( \eK\) ainsi que \( x,y\in \eK\). Nous avons :
    \begin{enumerate}
        \item
            Si \( x\leq 0\) et \( y\leq 0\) alors \( xy\geq 0\).
            \item
                Si \( x\leq 0\) et \( y\geq 0\) alors \( xy\leq 0\).
\item
    Si \( x\geq 0\) et \( y\leq 0\) alors \( xy\leq 0\).
\item       \label{ITEMooRGYAooCUIfss}
            \( 0\leq 1\).
        \item       \label{ITEMooMRNHooLglPKn}
            Si \( x\geq 0\) alors \( x^{-1}\geq 0\).
    \end{enumerate}
\end{lemma}

\begin{lemma}[Propriétés de la valeur absolue]  \label{LemooANTJooYxQZDw}
    Soit \( \eK\) un corps totalement ordonné. Si \( x,y\in \eK\) alors\footnote{La «valeur absolue» est définie en \eqref{DefKCGBooLRNdJf}\ref{ItemooWUGSooRSRvYC}.}
    \begin{enumerate}
        \item       \label{ItemooNVDIooSuiSoB}
            Si \( x\geq 0\) alors \( -x\leq 0\).
        \item       \label{ITEMooVNAZooSxmtuH}
            Si \( x\leq 0\) alors \( -x\geq 0\).
        \item       \label{ITEMooSDNHooDnjScE}
            \( | x |\geq 0\)
        \item       \label{ITEMooLQLTooTJTPVM}
            \( | x |=0\) si et seulement si \( x=0\)
        \item       \label{ITEMooVJAEooOEatzY}
            \( | -x |=| x |\).
        \item\label{ItemooOMKNooRlanvk}
            \( | x+y |\leq | x |+| y |\).
    \end{enumerate}
\end{lemma}

\begin{proof}
    Point par point
    \begin{subproof}
    \item[\ref{ItemooNVDIooSuiSoB}]
            Nous partons de \( x\geq 0\) et nous ajoutons \( -x\) des deux côtés en profitant de la définition d'un corps totalement ordonné : \( x-x\geq -x\) et donc \( 0\geq-x\), c'est-à-dire \( -x\leq 0\).
        \item[\ref{ITEMooVNAZooSxmtuH}]
            Nous partons de \( x\leq 0\) et nous ajoutons \( -x\) des deux côtés.
        \item[\ref{ITEMooSDNHooDnjScE}]
            Si \( x\geq 0\) alors c'est vrai. Sinon, \( x\leq 0\) et \( | x |=-x\geq 0\) par le point~\ref{ItemooNVDIooSuiSoB}.
        \item[\ref{ITEMooLQLTooTJTPVM}]
            Si \( x=0\) alors \( x=-x\) et \( | x |=0\). Au contraire si \(x\neq 0\) alors \( -x\neq 0\) et que \( x\) soit positif ou négatif, nous aurons toujours \( \pm x\neq 0\).
        \item[\ref{ITEMooVJAEooOEatzY}]
            Il faut décomposer en deux cas selon que \( x\geq 0\) et \( x\leq 0\). Supposons \( x\geq 0\). Alors d'une part \( | x |=x\). D'autre part \( -x\leq 0\) par le point \ref{ItemooNVDIooSuiSoB}, de telle sorte que
            \begin{equation}
                | -x |=-(-x)=x.
            \end{equation}
            Nous avons donc \( | x |=| -x |=x\).

            Le même raisonnement tient avec \( x\leq 0\).
        \item[\ref{ItemooOMKNooRlanvk}]
            Nous supposons que \( x\leq y\) et nous distinguons divers cas suivant la positivité de \( x\) et \( y\).
            \begin{enumerate}
                \item
                    Si \( x,y\geq 0\). Dans ce cas, \( x+y\geq y\geq 0\), donc \( | x+y |=x+y=| x |+| y |\).
                \item
                    Si \( x,y\leq 0\). Dans ce cas, \( x+y\leq 0\) et nous avons \( | x+y |=-x-y=| x |+| y |\).
                \item
                    Si \( x\leq 0\) et \( y\geq 0\). Nous subdivisons encore en deux cas suivant que \( x+y\) est positif ou négatif. Si \( x+y\geq 0\), alors nous écrivons successivement
                    \begin{subequations}
                        \begin{align}
                            x&\leq 0\\
                            x+y&\leq y\leq y+| x |=| x |+| y |
                        \end{align}
                    \end{subequations}
                    et donc \( | x+y |=x+y\leq | x |+| y |\).

                    Nous supposons à présent que \( x\leq 0\), \( y\geq 0\) et \( x+y\leq 0\). Dans ce cas il suffit d'écrire \( | x+y |=| (-x)+(-y) |\) pour retomber dans le cas précédent à inversion près de \( x\) et \( y\).
            \end{enumerate}
    \end{subproof}
\end{proof}

\begin{remark}      \label{RemooJCAUooKkuglX}
    La partie~\ref{ItemooOMKNooRlanvk} est très importante parce que c'est elle qui fera presque toutes les majorations dont nous aurons besoin en analyse. En effet elle donne l'inégalité triangulaire de la façon suivante : si \( x,y,z\in \eK\) nous avons
    \begin{equation}
        | x-y |= |  (x-z)+(z-y) |\leq | x-z |+| z-y |.
    \end{equation}
\end{remark}

\begin{lemma}[À propos de boules]
    Soient un corps totalement ordonné \( \eK\) et des éléments \( x,y\in \eK\). Soit aussi \( \epsilon>0\) dans \( \eK\). Nous avons :
    \begin{enumerate}
        \item       \label{ITEMooXJGVooSebiip}
            \( y\in B(x,\epsilon)\) si et seulement si \( x-\epsilon<y<x+\epsilon\).
        \item       \label{ITEMooRUBBooRayiMs}
            Si \( y\in  \overline{ B(x,\epsilon) }  \) alors \( y\in B(x,\epsilon')\) pour tout \( \epsilon'>\epsilon\).
    \end{enumerate}
\end{lemma}

\begin{proof}
    Pour rappel,
    \begin{equation}
        | x-y |=\begin{cases}
               x-y    &     \text{si } x-y\geq 0 \\
                    y-x    &    \text{si } x-y\leq 0.
               \end{cases}
    \end{equation}
    Nous pouvons maintenant démontrer nos assertions.
    \begin{subproof}
        \item[\ref{ITEMooXJGVooSebiip}]
            En deux parties.
            \begin{subproof}
            \item[\( \Rightarrow\)]
            Nous supposons que \( | x-y |<\epsilon\).

            Si \( x-y\geq 0\) alors l'hypothèse signifie \( x-y<\epsilon\), ce qui donne \( y>x-\epsilon\). Mais l'inégalité \( x-y\geq 0\) donne également \( x\geq y\) et donc \( x+\epsilon\geq y+\epsilon>y\). Notez le jeu de l'inégalité non stricte qui se change en inégalité stricte.

            Si \( x-y\leq 0\) nous pouvons faire le même raisonnement.

            \item[\( \Leftarrow\)]
            Des inégalités \( x-\epsilon<y\) et \( y<x+\epsilon\) nous tirons \( x-y<\epsilon\) et \( y-x<\epsilon\). Donc quel que soit le signe de \( x-y\) nous avons toujours \( | x-y |<\epsilon\).
            \end{subproof}

        \item[\ref{ITEMooRUBBooRayiMs}]

            C'est immédiat parce que
            \begin{equation}
                | x-y |\leq \epsilon<\epsilon'.
            \end{equation}
    \end{subproof}
\end{proof}


\begin{lemma}       \label{LEMooVZNCooRJatKK}
    Tout corps totalement ordonné est de caractéristique nulle.
\end{lemma}

%+++++++++++++++++++++++++++++++++++++++++++++++++++++++++++++++++++++++++++++++++++++++++++++++++++++++++++++++++++++++++++
\section{Les rationnels}
%+++++++++++++++++++++++++++++++++++++++++++++++++++++++++++++++++++++++++++++++++++++++++++++++++++++++++++++++++++++++++++

Une construction très explicite est faite dans \cite{RWWJooJdjxEK}. Ici nous allons prendre plus court :
\begin{definition}
    Le corps des fractions de \( \eZ\) (définition~\ref{DEFooGJYXooOiJQvP}) est noté \( \eQ\) et ses éléments sont les \defe{rationnels}{rationnels}.
\end{definition}

\begin{normaltext}
    Les résultats énoncés ici sont utilisés plus bas et servent de guide à \randomGender{un éventuel contributeur}{une éventuelle contributrice} qui voudrait écrire une partie dédiée à \( \eQ\) et ses propriétés de base\quext{Par exemple, définir une relation d'ordre sur \( \eQ\) et expliciter l'inclusion de \( \eZ\) dans \( \eQ\).}. Nous espérons que des preuves se trouvent dans \cite{RWWJooJdjxEK}. En tout cas, \randomGender{cher lecteur}{chère lectrice}, je t'invite à ne rien prendre pour évident.
\end{normaltext}

\begin{proposition}     \label{PROPooUULNooKbwuEw}
    L'application 
    \begin{equation}
        \begin{aligned}
            f\colon \eZ&\to \eQ \\
            z&\mapsto z/1 
        \end{aligned}
    \end{equation}
    est une injection.
\end{proposition}

\begin{proof}
    Supposons que \( f(a)=f(b)\), c'est-à-dire que \( a/1=b/1\). En vertu de la relation d'équivalence donnée en \ref{DEFooGJYXooOiJQvP}, nous avons \( a1=b1\), c'est-à-dire \( a=b\).
\end{proof}

\begin{normaltext}
    À partir de maintenant, nous allons identifier la partie \( f(\eZ)\) à \( \eZ\). Nous nous autorisons donc à dire que \( 4\in \eQ\) ou que \( -7\in \eQ\), et même que \( 0\in \eQ\).
\end{normaltext}

\begin{proposition}     \label{PROPooDHIAooZysvNs}
    L'ensemble des rationnels est infini dénombrable\footnote{Ouais, je sais, dans les définitions prises ici, un ensemble dénombrable est toujours infini. Mais l'excès de précision ne tue pas, loin s'en faut.}.
\end{proposition}

\begin{proof}
    L'ensemble \( \eZ\) est infini\footnote{Lemme \ref{LEMooJNXIooBmdOVi}.} et la proposition \ref{PROPooUULNooKbwuEw} donne une injection \( f\colon \eZ\to \eQ\). Donc \( f(\eZ)\) est infini.

    L'ensemble \( \eQ\) contient une partie infinie. Il est donc infini par le lemme \ref{LEMooTUIRooEXjfDY}.

    L'application
    \begin{equation}
        \begin{aligned}
            g\colon \eZ^2&\to \eQ \\
            a,b&\mapsto a/b 
        \end{aligned}
    \end{equation}
    est surjective alors que \( \eZ^2\) est dénombrable. Le lemme \ref{LEMooDTAEooIBdHyo} dit alors que \( \eQ\) est fini ou dénombrable. Vu que nous avons déjà prouvé que \( \eQ\) était infini, nous déduisons que \( \eQ\) est infini dénombrable.
\end{proof}


%--------------------------------------------------------------------------------------------------------------------------- 
\subsection{Relation d'ordre}
%---------------------------------------------------------------------------------------------------------------------------

\begin{propositionDef}      \label{DEFooZEXXooUtOhqB}
    Pour \( a,b,c,d\in \eZ\) nous disons que
    \begin{equation}
        \frac{ a }{ b }\geq \frac{ c }{ d }
    \end{equation}
    si et seulement si \( ad\geq bc\) dans \( \eZ\).

    Avec cette définition, \( (\eZ,\geq)\) est un ensemble totalement ordonné.
\end{propositionDef}

\begin{lemma} \label{LEMooEBTIooGMoHsj}
    Tout rationnel est majoré par un naturel.
\end{lemma}


\begin{proposition}     \label{PROPooBTCCooVVvaeL}
    Si \( q<1\) dans \( \eQ\), alors \( qx<x\) pour tout \( x\in \eQ^+\).
\end{proposition}

\begin{proposition}     \label{PROPooMXGPooDUkOuv}
    Le corps \( \eQ\) est archimédien\footnote{Définition~\ref{DEFooLCWLooYrToFv}.}.
\end{proposition}

%--------------------------------------------------------------------------------------------------------------------------- 
\subsection{Caractéristique}
%---------------------------------------------------------------------------------------------------------------------------

\begin{lemma}       \label{LEMooYCPUooNxEPhB}
    Le corps \( \eQ\) est de caractéristique\footnote{Définition \ref{LEMDEFooVEWZooUrPaDw}.} nulle.
\end{lemma}

%+++++++++++++++++++++++++++++++++++++++++++++++++++++++++++++++++++++++++++++++++++++++++++++++++++++++++++++++++++++++++++ 
\section{Suite de Cauchy dans un corps totalement ordonné}
%+++++++++++++++++++++++++++++++++++++++++++++++++++++++++++++++++++++++++++++++++++++++++++++++++++++++++++++++++++++++++++


\begin{lemma}[\cite{ooIBWOooSjOvXd, MonCerveau}]        \label{LEMooLTBIooSZnvsQ}
    Tout corps commutatif de caractéristique nulle contient un sous-corps isomorphe à \( \eQ\).
\end{lemma}

\begin{proof}
    Soit un corps \( \eK\) de caractéristique nulle. Nous savons du lemme \ref{LEMDEFooVEWZooUrPaDw} que
    \begin{equation}
        \begin{aligned}
            \mu\colon \eZ&\to \eK \\
            n&\mapsto n1_{\eK} 
        \end{aligned}
    \end{equation}
    est un morphisme d'anneaux vérifiant \( \ker(\mu)=\{ 0 \}\). Nous posons \( Z=\mu(\eZ)\). L'application \( \mu\colon \eZ\to Z\) est un isomorphisme d'anneaux. Prouvons cela :
    \begin{subproof}
    \item[Morphisme]
        L'application \( \mu\) est un morphisme par le lemme \ref{LEMDEFooVEWZooUrPaDw}.
    \item[Surjectif]
        Par définition les éléments de \( Z\) sont dans l'image de \( \eZ\).
    \item[Injectif] Si \( x,y\in \eZ\) vérifient \( \mu(x)=\mu(y)\), alors \( \mu(x-y)=0\) parce que \( \mu\) est un morphisme. Mais \( \eK\) est de caractéristique nulle, c'est-à-dire \( \ker(\mu)=\{ 0 \}\). Donc \( x-y=0\).
    \end{subproof}
    Le corps \( \eK\) contient donc un sous-anneau isomorphe à \( \eZ\). Puisque \( \eZ\) et \( \eK\) sont commutatifs, la proposition \ref{PROPooIJBEooDjsoHr} s'applique et \( \eK\) contient un sous-corps isomorphe à \( \Frac(\eZ)=\eQ\).
\end{proof}

La proposition suivante donne des précisions à propos du lemme \ref{LEMooLTBIooSZnvsQ}.

\begin{proposition}[\cite{MonCerveau}]      \label{PROPooKNROooFdgIeQ}
    Soit un corps totalement ordonné \( \eK\). Nous considérons l'application
    \begin{equation}
        \begin{aligned}
            \mu\colon \eZ&\to \eK \\
            n&\mapsto n\cdot 1_{\eK} 
        \end{aligned}
    \end{equation}
    et ensuite
    \begin{equation}
        \begin{aligned}
            \sigma\colon \eQ&\to \eK \\
            a/b&\mapsto \mu(a)\mu(b)^{-1}. 
        \end{aligned}
    \end{equation}
    Alors
    \begin{enumerate}
        \item
            L'application \( \sigma\) est bien définie.
        \item
            L'application \( \sigma\) est un morphisme de corps.
        \item
            Si \( q\leq q'\) dans \( \eQ\), alors \( \sigma(q)\leq \sigma(q')\).
    \end{enumerate}
\end{proposition}

\begin{proof}
    En plusieurs morceaux.
    \begin{subproof}
    \item[\( \sigma\) est bien définie]
    Montrons que \( \sigma\) est bien définie. Pour cela nous considérons \( a,b,x,y\in \eZ\) tels que \( a/b=x/y\) dans \( \eQ\). Par définition des classes (définition \ref{DEFooGJYXooOiJQvP} du corps des fractions), nous avons \( ay=bx\) dans \( \eQ\). Vu que \( \mu\) est un morphisme nous avons alors
    \begin{equation}
        \mu(a)\mu(y)=\mu(b)\mu(x)
    \end{equation}
    et donc \( \mu(a)\mu(b)^{-1}=\mu(x)\mu(y)^{-1}\), c'est-à-dire \( \sigma(a/b)=\sigma(x/y)\). L'application \( \sigma\) est donc bien définie.

\item[Morphisme pour la somme]

    L'application \( \mu\) est un morphisme d'anneaux, comme déjà dit depuis le lemme \ref{LEMDEFooVEWZooUrPaDw}. Notons aussi que, parce que \( \eK\) est commutatif,
    \begin{equation}
        \mu(qy)^{-1}=\mu(q)^{-1}\mu(y)^{-1}.
    \end{equation}

    En utilisant la définition \ref{DEFooGJYXooOiJQvP}\ref{ITEMooWBWHooYsXFkO} de la somme nous avons
    \begin{subequations}
        \begin{align}
            \sigma(p/q+x/y)&=\sigma\big( (py+qx)/qy \big)\\
            &=\big[ \mu(py)+\mu(qx) \big]\mu(qy)^{-1}\\
            &=\mu(py)\mu(qy)^{-1}+\mu(qx)\mu(qy)^{-1}\\
            &=\mu(p)\mu(q)^{-1}+\mu(x)\mu(y)^{-1}\\
            &=\sigma(p/q)+\sigma(x/y).
        \end{align}
    \end{subequations}

\item[Morphisme pour le produit]
    Même genre de calculs que pour la somme.
\item[Croissante]

    Nous savons aussi par le lemme \ref{LEMooXJTAooZauchx}\ref{ITEMooRGYAooCUIfss} que \( 1\geq 0\). Puisque \( \mu\) est un morphisme d'anneaux,
    \begin{equation}
        \mu(n+1)=\mu(n)+\mu(1)=\mu(n)+1
    \end{equation}

    La définition \ref{DefKCGBooLRNdJf}\ref{ITEMooZISJooWNxnBj} dit alors que \( \mu(n)\geq 0\) pour tout \( n\in \eN\). Nous avons pour la même raison que si \( m\geq n\) dans \( \eN\), alors \( \mu(m)\geq\mu(n)\) dans \( \eK\).

        Soient maintenant \( p,q\in \eN\), et prouvons que \( \sigma(p/q)\geq 0\). D'abord
        \begin{equation}
            \sigma(p/q)=\mu(p)\mu(q)^{-1}
        \end{equation}
        où \( \mu(p)\geq 0\) et \( \mu(q)\geq 0\). Ensuite le lemme \ref{LEMooXJTAooZauchx}\ref{ITEMooMRNHooLglPKn} nous indique que \( \mu(q)^{-1}\geq 0\). Enfin la condition \ref{DefKCGBooLRNdJf}\ref{CONDooBYYDooElXgPO} nous permet de conclure que \( \sigma(p/q)\geq 0\).

        Finalement, si \( q_1\geq q_2\) dans \( \eQ\), alors \( q_1-q_2\geq 0\), et nous avons
        \begin{equation}
            \sigma(q_1)=\sigma(q_2+q_1-q_2)=\sigma(q_2)+\sigma(q_1-q_2)\geq \sigma(q_2)
        \end{equation}
        par la condition \ref{DefKCGBooLRNdJf}\ref{ITEMooZISJooWNxnBj}.
    \end{subproof}
\end{proof}


\begin{normaltext}      \label{NORMooJRRZooTwTVYG}
    Si \( \eK\) est un corps totalement ordonné, la proposition \ref{PROPooKNROooFdgIeQ} nous donne un morphisme de corps \( \sigma\colon \eQ\to \eK\) qui respecte l'ordre. Pour \( q\in \eQ\) et \( k\in \eK\) nous notons
    \begin{equation}        \label{EQooERFIooMpZVEs}
        qk=\sigma(q)k.
    \end{equation}
    Nous pourrons donc écrire \( \frac{ k }{2}\) pour \( \sigma(1/2)k\).
\end{normaltext}

Le lemme suivant explique que la notation \eqref{EQooERFIooMpZVEs} n'est pas complètement idiote.
\begin{lemma}       \label{LEMooWIONooGTKfcJ}
    Soit un corps commutatif totalement ordonné \( \eK\). Soit \( k\in \eK\). Nous avons
    \begin{equation}
        k+k=2k.
    \end{equation}
\end{lemma}

\begin{proof}
    Vu que \(  \sigma\colon \eQ\to \eK \) est un morphisme, il vérifie \( \sigma(1)=1\), donc
    \begin{subequations}
        \begin{align}
            k+k&=\sigma(1)k+\sigma(1)k\\
            &=\big( \sigma(1)+\sigma(1) \big)k\\
            &=\sigma(2)k\\
            &=2k.
        \end{align}
    \end{subequations}
\end{proof}


\begin{proposition}     \label{PROPooTFVOooFoSHPg}
    Toute suite convergente dans un corps totalement ordonné est de Cauchy.
\end{proposition}

\begin{proof}
    Soit un corps totalement ordonné \( \eK\) et une suite \( x_n\stackrel{\eK}{\longrightarrow}x\). Soit \( \epsilon>0\). Il est important de se rendre compte que \( \epsilon\in \eK\) et que l'inégalité est au sens de l'ordre dans \( \eK\); en particulier ce n'est pas \( \epsilon\in \eR\) ni \( \epsilon\in \eQ\). D'ailleurs nous n'avons pas encore défini \( \eR\).

    Vu que \( (x_n)\) converge vers \( x\), il existe \( N\in \eN\) tel que pour tout \( k>N\),
    \begin{equation}
        | x_k-x |<\epsilon.
    \end{equation}
    Soient \( p,q>N\). Alors en utilisant la majoration du lemme~\ref{LemooANTJooYxQZDw}\ref{ItemooOMKNooRlanvk},
    \begin{equation}        \label{EQooMQYGooLpgEQO}
        | x_p-x_q |=\big| (x_p-x)+(x-x_q) \big|\leq | x_p-x |+| x-x_q |\leq 2\epsilon.
    \end{equation}
    En analyse en général, on s'arrête là et on dit que \( (x_n)\) est de Cauchy parce qu'il n'y a pas vraiment de différence entre réaliser une majoration avec \( \epsilon\) ou avec \( 2\epsilon\). Détaillons toutefois comment ça se passe dans le cas où \( \epsilon\) est un élément d'un corps totalement ordonné.

    Le \( 2\epsilon\) arrivant à la fin de \eqref{EQooMQYGooLpgEQO} est en réalité \( \epsilon+\epsilon=\sigma(2)\epsilon\) en vertu de ce qui est raconté en \ref{NORMooJRRZooTwTVYG} et en vertu du lemme \ref{LEMooWIONooGTKfcJ}.

    Considérons \( \epsilon'=\sigma(1/2)\epsilon\), que nous pouvons noter \( \epsilon'=\epsilon/2\). Vu que \( \epsilon'>0\), il existe un \( N'\) tel que pour tout \( p,q>N'\) nous ayons
    \begin{equation}
        | x_p-x_q |\leq 2\epsilon'=\sigma(2)\sigma(1/2)\epsilon=\sigma(1)\epsilon=\epsilon.
    \end{equation}
    Ce dernier \( \epsilon\) étant bien celui fixé au début de la preuve, nous en déduisons que \( (x_n)\) est de Cauchy.
\end{proof}

%---------------------------------------------------------------------------------------------------------------------------
\subsection{Suites de Cauchy dans les rationnels}
%---------------------------------------------------------------------------------------------------------------------------

\begin{proposition}[\cite{RWWJooJdjxEK}]        \label{PropFFDJooAapQlP}
    Principales propriétés des suites de Cauchy dans \( \eQ\).
    \begin{enumerate}
        \item       \label{ItemRKCIooJguHdji}
            Toute suite convergente est de Cauchy\footnote{Et non la réciproque, qui sera justement la grande innovation des nombres réels.}.
        \item       \label{ItemRKCIooJguHdjii}
            Toute suite de Cauchy est bornée.
        \item       \label{ItemRKCIooJguHdjiii}
            Si \( x_n\to 0\) et si \( (y_n)\) est bornée, alors \( x_ny_n\to 0\)
        \item
            Si \( (x_n)\) et \( (y_n)\) sont de Cauchy alors \( (x_n+y_n)\), \( (x_n-y_n)\) et \( (x_ny_n)\) sont également de Cauchy.
        \item       \label{ITEMooIAFSooAIUpAN}
            Si il existe \( a,b\in \eQ\) tels que \( x_n\to a \) et \( y_n\to b \) alors \( x_n+y_n\to a+b\), \( x_n-y_n\to a-b\) et \(  x_ny_n\to ab  \).
        \item   \label{ItemRKCIooJguHdjvi}
            Soit \( (x_n)\) une suite de Cauchy qui ne converge pas vers zéro. Alors il existe \( n_0\) tel que la suite \( \left( \frac{1}{ x_n } \right)_{n\geq n_0}\) soit de Cauchy.
    \end{enumerate}
\end{proposition}

\begin{proof}
    Point par point.
    \begin{enumerate}
        \item
            C'est la proposition~\ref{PROPooTFVOooFoSHPg}.
        \item
            Soit \( (x_n)\) une suite de Cauchy dans \( \eQ\). Avec \( \epsilon=1\) dans la définition, si \( q>N\), nous avons
            \begin{equation}
                | x_q-x_{N} |\leq 1.
            \end{equation}
            Et donc pour tout \( q\) plus grand que \( N\), \( x_N-1\leq x_q\leq x_N+1\), ou encore, pour tout \( n\) :
            \begin{equation}
                | x_n |\leq\max\{ | x_1 |,| x_2 |,\ldots,| x_N |,| x_N+1 | \}.
            \end{equation}
            La suite est donc bornée.
        \item
            Soit \(\epsilon>0\). Les hypothèses disent qu'il existe un \( N\) tel que \( | x_n |\leq \epsilon\) dès que \( n\geq N\). Et il existe aussi \( M\geq 0\) tel que \( | y_n |\leq M\) pour tout \( n\). Du coup, lorsque \( n\geq N\) nous avons \( | x_ny_n |\leq M\epsilon\).
        \item
            En ce qui concerne la somme,
            \begin{equation}        \label{EqDCNBooAzrrBi}
                | x_p+y_p-x_q-y_q |\leq | x_p-x_q |+| y_p-y_q |.
            \end{equation}
            Soit \( N_1\) tel que si \( p,q\geq N_1\) alors \( | x_p-x_q |\leq \epsilon\) et \( N_2\) de même pour la suite \( (y_n)\). En prenant \( N=\max\{ N_1,N_2 \}\), la somme \eqref{EqDCNBooAzrrBi} est plus petite que \( 2\epsilon\) dès que \( p,q\geq N\).

            Passons à la démonstration du fait que le produit de deux suites de Cauchy est de Cauchy. Les suites \( (x_n)\) et \( (y_n)\) sont bornées et quitte à prendre le maximum, nous disons qu'elles sont toutes les deux bornées par le nombre \( M\) : pour tout \( n\) nous avons \( | x_n |\leq M\) et \( | y_n |\leq M\). Nous avons :
            \begin{equation}
                | x_py_p-x_qy_q |\leq | x_py_p-x_qy_p |+| x_qy_p-x_qy_q |\leq | y_p | |x_p-x_q |+| x_q | |y_p-y_q |.
            \end{equation}
            Puisque \( (x_n)\) et \( (y_n)\) sont de Cauchy, si \( p\) et \( q\) sont assez grands, les deux différences sont majorées par \( \epsilon\) et nous avons
            \begin{equation}
                | x_py_p-x_qy_q |\leq M\epsilon+M\epsilon=2M\epsilon,
            \end{equation}
            ce qui prouve que \( (x_ny_n)\) est de Cauchy.
        \item
            En ce qui concerne la somme, nous pouvons tout de suite calculer
            \begin{equation}
                | x_n+y_n-(a+b) |\leq | x_n-a |+| y_n-b |.
            \end{equation}
            Il existe une valeur de \( n\) à partir de laquelle le premier terme est plus petit que \( \epsilon\) et une à partir de laquelle le second terme est plus petit que \( \epsilon\). En prenant le maximum des deux, la somme est plus petite que \( 2\epsilon\).

            En ce qui concerne le produit,
            \begin{equation}
                | x_ny_n-ab |\leq | x_ny_n-ay_n |+| ay_n-ab |\leq | y_n || x_n-a |+| a || y_n-b |.
            \end{equation}
            Les suites \( | x_n-a |\) et \( | y_n-b |\) convergent vers zéro; la suite \( (y_n)\) est bornée parce que convergente (combinaison des points~\ref{ItemRKCIooJguHdji} et~\ref{ItemRKCIooJguHdjii})  et \( a\) (la suite constante) est également bornée. Donc par le point~\ref{ItemRKCIooJguHdjiii}, nous avons
            \begin{equation}
                y_n| x_n-a |+a| y_n-b |\to 0.
            \end{equation}
            Au passage nous avons également utilisé la propriété de la somme que nous venons de démontrer.
        \item Soit \( (x_n)\) une suite de Cauchy dans \( \eQ\) ne convergeant pas vers zéro : il existe \( \alpha>0\) tel que pour tout \( N\in \eN\), il existe \( n\geq N\) tel que \( | x_n |>\alpha\). Mais notre suite est de Cauchy, donc il existe \( n_0\in \eN\) tel que si \( p,q\geq n_0\) alors
            \begin{equation}
                | x_p-x_q |\leq \frac{ \alpha }{2}.
            \end{equation}
            En fixant \( N = n_0\), on obtient un naturel \( n\geq n_0\) tel que \( | x_n |\geq \alpha\). De plus, comme la suite est de Cauchy, si \( p>n\) nous avons aussi \( | x_n-x_p |\leq \frac{ \alpha }{2}\). Cela implique \( | x_p |\geq \frac{ \alpha }{2}\) et en particulier \( x_p\neq 0\).

            Nous venons de prouver que la suite ne s'annule plus à partir de l'indice \( n\), et même que \( | x_k |\geq\alpha/2\) pour tout \( k\geq n\). La suite \( (1/x_k)_{k\geq n}\) est donc bien définie.

            Soit \( \epsilon>0\). Soit \( n_0\) tel que \( | x_p-x_q |<\epsilon\) pour tout \( p,q>n_0\). Soit \( K\) plus grand que \( n_0\) et que \( n\). En prenant \( p,q\geq K\), nous avons \( |  x_p|>\frac{ \alpha }{2}\) et \( | x_q |>\frac{ \alpha }{2}\). Nous en déduisons que
            \begin{equation}
                \left| \frac{1}{ x_p }-\frac{1}{ x_q } \right| \leq \frac{ | x_q-x_p | }{ | x_px_q | }\leq \frac{ 4 }{ \alpha^2 }| x_q-x_p |\leq \frac{ 4 }{ \alpha^2 }\epsilon.
            \end{equation}
            Donc \( \left( \frac{1}{ x_n } \right)\) est de Cauchy.
    \end{enumerate}
\end{proof}


%+++++++++++++++++++++++++++++++++++++++++++++++++++++++++++++++++++++++++++++++++++++++++++++++++++++++++++++++++++++++++++ 
\section{Insuffisance des rationnels}
%+++++++++++++++++++++++++++++++++++++++++++++++++++++++++++++++++++++++++++++++++++++++++++++++++++++++++++++++++++++++++++

Nous allons voir qu'il n'existe pas de nombres rationnels \( x\) tels que \( x^2=2\), mais que pourtant il existe une infinité de suites de rationnels \( (x_n)\) tels que \(  x_n^2\to 2  \).

\begin{lemma}       \label{LemJPIUooWFHaFM}
    Un entier \( x\) est pair si et seulement si l'entier \( x^2\) est pair.
\end{lemma}

\begin{proof}
    Si \( x\) est un nombre pair, alors il existe un entier \( a\) tel que \( x=2a\) alors \( x^2=4a^2\) est pair.

    Inversement, si \( x\) est impair alors il existe un entier \( a\) tel que \( x=2a+1\) et alors \( x^2=4a^2+4a+1=2(2a^2+2a)+1\) est impair.
\end{proof}

Le théorème~\ref{THOooYXJIooWcbnbm} nous dira que tous les \( \sqrt{n}\) sont irrationnels dès que \( n\) n'est pas un carré parfait. Voici déjà le résultat pour \( n=2\). Le fait que \( \sqrt{ 2 }\) existe dans \( \eR\) sera la proposition \ref{PROPooUHKFooVKmpte}.
\begin{proposition}[Irrationalité de \( \sqrt{2}\)]     \label{PropooRJMSooPrdeJb}
    Il n'existe pas de fractions d'entiers dont le carré soit égal à \( 2\).
\end{proposition}
\index{irrationalité!\( \sqrt{2}\)}

\begin{proof}
    Nous supposons que la fraction d'entiers \( a/b\) est telle que \( a^2/b^2=2\), et nous allons construire une suite d'entiers strictement décroissante et strictement positive, ce qui est impossible.

    Grâce au lemme~\ref{LemJPIUooWFHaFM} nous avons successivement les affirmations suivantes :
    \begin{itemize}
        \item
        \(\frac{ a^2 }{ b^2 }=2 \)  avec \( a\neq 0\) et \( b\neq 0\).
    \item
        \( a^2=2b^2\), donc \( a^2\) est pair.
    \item
        \( a\) est alors pair et \( a^2\) est divisible par \( 4\). Soit \( a^2=4k\).
    \item
        \( 4k/b^2=2\), donc \( 4k=2b^2\), donc \( b^2=2k\) et \( b^2\) est pair.
    \item
        Nous déduisons que \( b\) est pair.
    \end{itemize}
    La fraction \( \frac{ a/2 }{ b/2 }\) est alors une nouvelle fraction d'entiers dont le carré vaut $2$. En procédant de la même façon, en remplaçant \( a\) par \( a/2\) et \( b\) par \( b/2\), on obtient que la fraction d'entiers \( \frac{ a/4 }{ b/4 }\) a la même propriété.

    En particulier, tous les nombres de la forme \( a/2^n\) sont des entiers.  Ils forment une suite strictement décroissante d'entiers strictement positifs. Impossible, me diriez-vous ? Et vous auriez bien raison : toute partie non vide de \( \eN\) admet un plus petit élément\footnote{Voir \cite{RWWJooJdjxEK}, et attention : ce n'est pas tout à fait évident.}. Il n'y a donc pas de fractions d'entiers dont le carré vaut \( 2\).
\end{proof}

\begin{lemma}[Série géométrique]   \label{LEMooOTVUooImvusn}
    Si \( q\neq 1\) dans \( \eQ\) et \( p\in \eN\) nous avons
    \begin{equation}
        \sum_{k=0}^pq^k=\frac{ 1-q^{p+1} }{ 1-q }.
    \end{equation}
\end{lemma}

\begin{proof}
    En posant \( S_p=1+q+q^2+\cdots +q^{p}\), nous avons $S_p-qS_p=1-q^{p+1}$ et donc
    \begin{equation}
        S_p=\sum_{k=0}^pq^k=\frac{ 1-q^{p+1} }{ 1-q }.
    \end{equation}
\end{proof}

\begin{proposition}
    La suite donnée par
    \begin{equation}
        x_n=1+\frac{ 1 }{ 1! }+\cdots +\frac{1}{ n! }
    \end{equation}
    est de Cauchy et ne converge pas dans \( \eQ\).
\end{proposition}

\begin{proof}
    Si \( p>q>0\) nous avons
    \begin{subequations}
        \begin{align}
            x_p-x_q&=\sum_{k=q+1}^p\frac{1}{ k! }\\
            &\leq \sum_{k=q+1}^p\frac{1}{ (q+1)! }\frac{1}{ (q+1)^{k-q-1} }  \label{SUBEQooAXILooEAcpVB}\\
            &\leq \frac{1}{ (q+1)! }\lim_{p\to \infty} \sum_{k=0}^{p}\frac{1}{ (q+1)^k }  \label{SUBEQooNDPTooDSEYEJ}\\
            &\leq \frac{1}{ (q+1)! }\frac{1}{ 1-\frac{1}{ q+1 } } \label{SUBEQooEMHJooSnCUiK}  \\
            &\leq \frac{1}{ (q+1)! }\frac{q+1}{q}\\
            &\leq \frac{1}{ q!q }.
        \end{align}
    \end{subequations}
    Justifications :
    \begin{itemize}
        \item Pour \eqref{SUBEQooAXILooEAcpVB}, il s'agit de remplacer dans \( k!\) tous les facteurs plus grands que \( (q+1)\) par \( q+1\). Cela rend le dénominateur plus petit.
        \item Pour \eqref{SUBEQooNDPTooDSEYEJ}, il y a une inégalité parce que la suite \( p\mapsto \sum_{k=0}^p1/(q+1)^k\) est une suite strictement croissante.

        \item Pour \eqref{SUBEQooEMHJooSnCUiK}, le lemme~\ref{LEMooOTVUooImvusn} donne la valeur de la somme finie. En ce qui concerne la limite, nous avons demandé \( p>q>0\) et donc \( q+1>1\). Dans ce cas la limite fonctionne.
    \end{itemize}

    Cette inégalité une fois établie nous permet de prouver les assertions. La suite \( (x_n) \) est de Cauchy car, pour tout \( \epsilon\in\eQ\) s'écrivant \( \epsilon=\frac{ a }{ b }\) avec \( a,b\in \eN\), en prenant \( p,q>b\), nous avons
    \begin{equation}
        x_p-x_q\leq \frac{1}{ b!b }<\frac{1}{ b }<\frac{ a }{ b }=\epsilon.
    \end{equation}

    Montrons par l'absurde que cette suite ne converge pas dans \( \eQ\). Pour cela, nous supposons que \( \lim_{n\to \infty} x_n=\frac{ a }{b }\in \eQ\). Pour tout \( p>q\) nous avons établi
    \begin{equation}
        0<x_p-x_q<\frac{1}{ qq! }.
    \end{equation}
    Prenons la limite \( p\to \infty\); par stricte croissance de la suite, les inégalités restent strictes :
    \begin{equation}        \label{EqQLCTooOgQOdh}
        0<\frac{ a }{ b }-x_q<\frac{1}{ qq! }.
    \end{equation}
    Si \( n>b\) alors nous pouvons écrire
    \begin{equation}
        \frac{ a }{ b }-x_n=\frac{ \alpha }{ n! }
    \end{equation}
    avec \( \alpha\in \eZ\) parce que le dénominateur commun entre \( \frac{ a }{ b }\) et \( x_n\) est dans \( n!\). En prenant donc \( q>n\) dans \eqref{EqQLCTooOgQOdh} nous pouvons écrire
    \begin{equation}
        0<\frac{ \alpha }{ q! }<\frac{1}{ qq! },
    \end{equation}
    c'est-à-dire \( 0<\alpha<\frac{1}{ q }\), ce qui est impossible pour \( \alpha\in \eZ\).
\end{proof}

\begin{lemma}   \label{LEMooDTXYooKwmlZh}
    Soit \( A>0\) dans \( \eQ\). Il existe un rationnel \( q>0\) tel que \( q^2<A\).
\end{lemma}

\begin{proof}
    Vu que \( \eQ\) est archimédien (proposition \ref{PROPooMXGPooDUkOuv}), il existe \( n\in \eN\) tel que \( 1<nA\). Pour ce \( n\), nous avons
    \begin{equation}
        \left( \frac{1}{ n } \right)^2<\frac{1}{ n }<A.
    \end{equation}
\end{proof}

La proposition suivante donne une suite de rationnels qui convergerait dans \( \eR\) vers \( \sqrt{ A }\) (non encore défini à ce stade). Il est expliqué dans \cite{BIBooMPXEooQLKhku} que la suite peut être vue comme une forme de méthode de Newton \ref{THOooDOVSooWsAFkx}; voir l'exemple \ref{EXooDLSVooMHPpcl}. Si vous aimez les dessins et les approches géométriques, il y a une explication sur Wikipédia\cite{BIBooVCWCooQcolIq}.
\begin{proposition}[\cite{BIBooMPXEooQLKhku}]       \label{PROPooSTQXooHlIGVf}
    Soient \( A>0\) dans \( \eQ\) et \( x_0\in \eQ\). La suite \( (x_k)\) définie par
    \begin{equation}        \label{EQooOUIVooUqWhXe}
        x_{k+1}=\frac{ 1 }{2}\left( x_k+\frac{ A }{ x_k } \right)
    \end{equation}
    a les propriétés suivantes :
    \begin{enumerate}
        \item
            La suite \( y_k=x_k^2 \) converge dans \( \eQ\) vers \( A\).
        \item
            La suite \( (x_k)\) est de Cauchy dans \( \eQ\).
        \item
            La suite \( (x_k)\) ne converge pas dans \( \eQ\) dans le cas de \( A=2\).
    \end{enumerate}
\end{proposition}

\begin{proof}
    En plusieurs points.
    \begin{subproof}
        \item[La suite \( s_k\)]
            En posant \( y_k=x_k^2\) nous calculons que
            \begin{equation}
                y_{k+1}-A=\frac{ (y_k-A)^2 }{ 4y_k }.
            \end{equation}
            Autrement dit, la suite \( s_k=y_k-A\) vérifie
            \begin{equation}
                s_{k+1}=\frac{ s_k^2 }{ 4(A+s_k) }.
            \end{equation}
            Quelle que soit la valeur de \( s_0=x_0^2-A\), nous avons
            \begin{equation}
                s_1=\frac{ s_0^2 }{ 4(A+s_0) }=\frac{ (x_0^2-A)^2 }{ 4(A+x_0^2-A) }=\frac{ (x_0^2-A)^2 }{ 4x_0^2 }>0.
            \end{equation}
            Donc à partir de \( s_1\), tous les éléments sont positifs. Vu que \( A>0\) nous avons aussi
            \begin{equation}
                s_{k+1}<\frac{ s_k^2 }{ 4s_k }=\frac{ s_k }{ 4 }
            \end{equation}
            et donc \( s_k<s_0/4^k\). Donc \( s_k\to 0\).
        \item[La suite \( (y_k)\)]
            Nous venons de prouver que si \( y_k=A+s_k\), alors \( s_k\to 0\). Autrement dit, la suite \( y_k\) converge vers \( A\) dans \( \eQ\). 

            La suite \( (y_k)\) est donc de Cauchy par la proposition \ref{PropFFDJooAapQlP}\ref{ItemRKCIooJguHdji}.
        \item[La suite \( (x_k)\) est de Cauchy]
            Soit \( \epsilon>0\) dans \( \eQ\). Puisque \( (y_k)\) est de Cauchy, il existe \( n_0\in \eN\) tel que 
            \begin{equation}
                | x^2_r-x_s^2 |<\epsilon
            \end{equation}
            pour tout \( r,s\geq n_0\).

            Soit par ailleurs \( q\neq 0\) dans \( \eQ\) tel que \( q^2<A\), assuré par le lemme \ref{LEMooDTXYooKwmlZh}. Quitte à augmenter la valeur de \( n_0\), nous supposons que \( x_r,x_s>q\), et en particulier que \( x_r+x_s\neq 0\). Cela permet d'écrire d'abord
            \begin{equation}
                x_r^2-x_s^2=(x_r+x_s)(x_r-x_s)
            \end{equation}
            et ensuite de prendre la valeur absolue et de diviser par \( | x_r+x_s |\) :
            \begin{equation}
                | x_r-x_s |=\frac{ | x_r^2-x_s^2 | }{ | x_r+x_s | }<\frac{ \epsilon }{ 2q }.
            \end{equation}
            Donc \( (x_k)\) est une suite de Cauchy.
        \item[Pas de convergence pour \( A=2\)]
            Supposons que \( x_k\to a\in \eQ\). Dans ce cas nous aurions \( x_k^2\to a^2=A=2\) (proposition~\ref{PropFFDJooAapQlP}\ref{ITEMooIAFSooAIUpAN}). Mais nous savons par la proposition~\ref{PropooRJMSooPrdeJb} que \( a^2=2\) est impossible dans \( \eQ\).
    \end{subproof}
\end{proof}

Notons que cette proposition ne présume en rien de l'existence ou de la non-existence dans \( \eQ\) d'un élément qui pourrait décemment être nommé \( \sqrt{ A }\). Il se fait que le théorème \ref{THOooYXJIooWcbnbm} dira que \( \sqrt{ n }\) est soit entier, soit irrationnel.

\begin{normaltext}
    Un petit programme en python pour explorer la suite de la proposition \ref{PROPooSTQXooHlIGVf}.
    \lstinputlisting{tex/frido/codeSnip_4.py}
\end{normaltext}
