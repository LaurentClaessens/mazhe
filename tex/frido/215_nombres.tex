% This is part of Mes notes de mathématique
% Copyright (c) 2011-2025
%   Laurent Claessens
% See the file fdl-1.3.txt for copying conditions.

%+++++++++++++++++++++++++++++++++++++++++++++++++++++++++++++++++++++++++++++++++++++++++++++++++++++++++++++++++++++++++++ 
\section{Quelques résultats de cardinalité}
\label{SECooResultatsCardinalite}
%+++++++++++++++++++++++++++++++++++++++++++++++++++++++++++++++++++++++++++++++++++++++++++++++++++++++++++++++++++++++++++

%-------------------------------------------------------
\subsection{Ensembles dénombrables}
\label{SUBooEnsemblesDenombrables}
%----------------------------------------------------

\begin{normaltext}	\label{NORMooEnsembleFiniEquivalenceDef}
	La proposition suivante est à peu près prise comme définition d'un ensemble fini dans \cite{ooVAYLooJxVYex} qui donne également une preuve de l'équivalence avec notre définition.
\end{normaltext}

\begin{proposition}     \label{PROPooBYKCooGDkfWy}
	L'ensemble \( \eN\) est infini\footnote{Définition \ref{DefEOZLooUMCzZR}.}.
\end{proposition}

\begin{proof}
	Nous considérons la partie propre\footnote{Le fait que ce soit une partie propre est dû au fait que \( 0\) n'est pas dedans d'une part parce que le lemme \ref{LEMooWMYPooLTMyWb} dit que \( 0\leq 1\), et d'autre part parce que le lemme \ref{LEMooQBHFooCuCusQ} donne \( 0\neq 1\).}
	\begin{equation}
		A=\{ n\in \eN\tq n\geq 1 \}.
	\end{equation}
	Ensuite nous posons
	\begin{equation}
		\begin{aligned}
			\sigma\colon \eN & \to A         \\
			x                & \mapsto s(x).
		\end{aligned}
	\end{equation}
	Le fait que \( \sigma\) prenne ses valeurs dans \( A\) est parce que \( s\) prend ses valeurs dans \( \eN\setminus\{ 0 \}=A\).
	\begin{subproof}
		\spitem[\( \sigma\) est injective]
		Parce que \( s\) l'est.
		\spitem[\( \sigma\) est surjective]
		C'est dans la définition \ref{DEFooBJBOooWlblAx}\ref{ITEMooQAKJooGKdJsM}
	\end{subproof}
	Nous avons donc une bijection entre \( \eN\) et un sous-ensemble strict.
\end{proof}

\begin{lemma}       \label{LEMooYIWDooIQFSad}
	Si \( a\leq b\) dans \( \eN\), alors la partie\footnote{Définition \ref{DEFooKBUFooLvMHrf}.} \( \{ a,\ldots, b \}\) est finie.
\end{lemma}

\begin{proof}
	En deux parties.
	\begin{subproof}
		\spitem[Pour \( a=0\)]
		%-----------------------------------------------------------
		Nous commençons par prouver que \( \{ 0,\ldots,b \}\) est fini. Par récurrence sur \( b\). Avec \( b=0\), la partie \( \{ 0 \}\) est finie parce que son seul sous-ensemble propre est \( \emptyset\) qui n'est pas en bijection avec \( \{ 0 \}\).

		Supposons que \( \{ 0,\ldots, b \}\) est fini. Alors \( \{ 0,\ldots, b \}\cup \{ b+1 \}\) est fini par le lemme \ref{PROPooVOKDooOStPzU}.

		\spitem[Cas général]
		%-----------------------------------------------------------
		Nous avons \( \{ a,\ldots,b \}\subset\{ 0,\ldots,b \}\). Donc \( \{ a,\ldots,b \}\) est une partie d'un ensemble fini. Elle est finie par \ref{LEMooTUIRooEXjfDY}.
	\end{subproof}
\end{proof}

\begin{lemma}       \label{LEMooJDGOooHdyJnu}
	Si \( p\neq q\), alors il n'existe pas de bijection entre \( \{ 0,\ldots, p \}\) et \( \{ 0,\ldots, q \}\).
\end{lemma}

\begin{proof}
	Supposons pour fixer les idées que \( p\leq q\). Dans ce cas \( \{ 0,\ldots, p \}\) est une partie stricte de \( \{ 0,\ldots, q \}\). Vu que \( \{ 0,\ldots, q \}\) est fini (lemme \ref{LEMooYIWDooIQFSad}), il n'y a pas de bijection avec ses parties strictes.
\end{proof}

\begin{propositionDef}[Cardinal d'un ensemble fini\cite{MonCerveau}]     \label{PROPooJLGKooDCcnWi}
	Soit un ensemble non vide \( I\).
	\begin{enumerate}
		\item   \label{ITEMooMNMTooEOIjdo}
		      L'ensemble \( I\) est fini\footnote{Ensemble fini, définition \ref{DefEOZLooUMCzZR}.} si et seulement si il existe une bijection entre \( I\) et \( \{ 0,\ldots, N \}\) pour un certain \( N\in \eN\).
		\item   \label{ITEMooZJFUooSNUSIk}
		      Si \( I\) est fini, il existe un unique \( N\in \eN\) tel que \( I\) soit en bijection avec \( \{ 0,\ldots, N \}\).
	\end{enumerate}
	Dans ce cas, le nombre \( N+1\) est le \defe{cardinal}{cardinal} de \( I\), et est noté \( \Card(I)\). Pour l'ensemble vide, nous définissons \( \Card(\emptyset)=0\).
\end{propositionDef}

\begin{proof}
	En plusieurs parties.
	\begin{subproof}
		\spitem[\ref{ITEMooMNMTooEOIjdo}, \( \Leftarrow\)]
		Soit un ensemble \( A\) en bijection avec \( \{ 0,\ldots, N \}\). Nous avons vu que \( \{ 0,\ldots, N \}\) est fini dans le lemme \ref{LEMooYIWDooIQFSad}. Le lemme \ref{LEMooPGPVooZzlFvf}\ref{ITEMooKQMFooSzmXrd} conclut que \( A\) est fini.

		\spitem[\ref{ITEMooMNMTooEOIjdo}, \( \Rightarrow\)]
		Le vrai sport est de faire l'implication inverse. Nous supposons que \( A\) est un ensemble fini, et nous allons prouver qu'il est en bijection avec \( \{ 0,\ldots, N \}\) pour un \( N\) bien choisi. Nous allons utiliser le lemme de Zorn. Soit
		\begin{equation}
			\mA=\{ (N,\varphi)\tq  \varphi\colon \{ 0,\ldots, N \}\to A\text{ est injective} \}.
		\end{equation}
		Nous mettons sur \( \mA\) la relation d'ordre donnée par \( (N_1,\varphi_1)\leq (N_2,\varphi_2)\) lorsque
		\begin{enumerate}
			\item
			      \( N_1\leq N_2\)
			\item
			      \( \varphi_2\) étend \( \varphi_1\), c'est-à-dire que \( \varphi_2=\varphi_1\) sur \( \{ 0,\ldots, N_1 \}\).
		\end{enumerate}
		\begin{subproof}
			\spitem[\( \mA\) est inductif]
			Soit une partie \( \mF\) totalement ordonnée de \( \mA\). Nous considérons la partie suivante de \( \eN\) :
			\begin{equation}
				S=\{ n\in \eN\tq \exists  \varphi \tq (n,\varphi)\in \mF \}.
			\end{equation}
			Si \( S\) est majoré, alors il a un maximum (proposition \ref{PROPooMZOWooHmsXzI}\ref{ITEMooKIHZooDRTCdx}). Si \( M\) est le maximum de \( S\), le \( (M,\varphi)\) de \( \mF\) qui correspond à ce maximum est un majorant de \( \mF\).

			Supposons --pour l'absurde-- que \( \mF\) n'est pas majoré dans \( \mA\); en particulier \( S\) n'est pas majoré dans \( \eN\). Pour chaque \( n\in S\), il existe une application \( \varphi_n\) telle que \( (n,\varphi_n)\in\mF\). Cela nous permet de définir
			\begin{equation}
				\begin{aligned}
					\phi\colon S & \to A                 \\
					n            & \mapsto \varphi_n(n).
				\end{aligned}
			\end{equation}
			Montrons que \( \phi\) est injective. Si \( \phi(m)=\phi(n)\), alors \( \varphi_m(m)=\varphi_n(n)\). Supposons pour fixer les idées que \( n\leq m\). Vu que \( (n,\varphi_n)\) et \( (m,\varphi_m)\) sont dans \( \mF\) qui est totalement ordonné, \( \varphi_m\) prolonge \( \varphi_n\); en particulier \( \varphi_n(n)=\varphi_m(n)\). Mais comme \( \varphi_m\) est injective, \( m=n\).

			Nous avons donc une injection \( \phi\colon S\to A\). Mais \( S\) est non borné et donc infini\footnote{Le lemme \ref{LEMooFKLPooPrmeUU} dit qu'il existe une bijection entre \( S\) et \( \eN\). De là nous concluons que \( S\) est infini parce qu'un ensemble en bijection avec un ensemble infini est infini par la proposition \ref{PROPooWKSIooHcfYPN}.}. La proposition \ref{PROPooWKSIooHcfYPN} conclut que \( A\) est infini, ce qui est contraire aux hypothèses.

			Donc \( \mF\) a un majorant et \( \mA\) est inductif.
			\spitem[Lemme de Zorn]
			Le lemme de Zorn dit que \( \mA\) a un élément maximal.
			\spitem[Conclusion]
			Soit \( (N,\varphi)\) un élément maximal de \( \mA\). Nous allons prouver que \( \varphi\colon \{ 0,\ldots, N \}\to A\) est une bijection. Que \( \varphi\) soit injective est une conséquence du fait que \( (N,\varphi)\) est dans \( \mA\).

			Pour prouver que \( \varphi\) est surjective, nous supposons qu'elle ne l'est pas. Soit \( a\in A\) qui n'est pas dans l'image de \( \varphi\). En posant
			\begin{equation}
				\begin{aligned}
					\phi\colon \{ 0,\ldots, N+1 \} & \to A                                     \\
					x                              & \mapsto \begin{cases}
						                                         \varphi(x) & \text{si } x\neq N+1 \\
						                                         a          & \text{si } x=N+1,
					                                         \end{cases}
				\end{aligned}
			\end{equation}
			le couple \( (N+1,\phi)\) majore strictement \( (N,\varphi)\). Ce qui est une contradiction.
		\end{subproof}
		\spitem[\ref{ITEMooZJFUooSNUSIk} existence]
		L'existence est ce que nous venons de montrer ci-dessus.
		\spitem[\ref{ITEMooZJFUooSNUSIk} unicité]
		Supposons que \( I\) soit en bijection avec \( \{ 0,\ldots, M \}\) et avec \( \{ 0,\ldots,N \}\). Il existe donc une bijection entre \(  \{ 0,\ldots, M \}\) et \( \{ 0,\ldots, N \}\). Par le lemme \ref{LEMooJDGOooHdyJnu}, cela implique que \( M=N\).
	\end{subproof}
\end{proof}
Nous ne définissons pas ce qu'est le cardinal d'un ensemble infini; c'est très compliqué et ça ne nous servira pas.


\begin{lemma}       \label{LEMooRWUDooTgoRXH}
	Si \( A\) est une partie infinie de \( \eN\), alors pour tout \( n\), la partie \( A\setminus\{ 0,\ldots, n \}\) est non vide.
\end{lemma}

\begin{proof}
	Si \( A\setminus\{ 0,\ldots, n \}\) était vide, cela signifierait que \( A\) est une partie de \( \{ 0,\ldots, n \}\). Or nous savons que \( \{ 0,\ldots, n \}\) est fini (lemme \ref{LEMooYIWDooIQFSad}), et que toute partie d'un ensemble fini est finie (lemme \ref{LEMooTUIRooEXjfDY}). Donc nous aurions que \( A\) est fini, ce qui est contraire à l'hypothèse.
\end{proof}

\begin{proposition}[\cite{MonCerveau}]	\label{PROPooOTHMooPzPzdW}
	À propos d'ensembles finis.
	\begin{enumerate}
		\item		\label{ITEMooFIEUooRmDcfo}
		      Si \( I\) est un ensemble fini et si \( \{ A_i \}_{i\in I}\) sont des ensembles finis, alors \( \bigcup_{i\in I}A_i\) est fini.
		\item		\label{ITEMooLVSVooHEwNWA}
		      Si \( A\) et \( B\) sont finis, alors \( A\times B\) est fini.
	\end{enumerate}
	%TODOooVLQNooBumUlE. Prouver ça. On a une récurrence à faire, donc on ne peut en parler plus tôt.
\end{proposition}

\begin{lemma}[\cite{MonCerveau}]	\label{LEMooRURDooRIGJnY}
	Si \( \Card(A)=n\) et si \( a\in A\) alors \( \Card(A\setminus\{ a \})=n-1\).
	%TODOooKFQHooGJEnzl. Prouver ça.
\end{lemma}

\begin{normaltext}
	En termes de notations, si \( I\) et \( A\) sont des ensembles et si \(\sigma \colon I\to A  \) est une injection, nous notons souvent
	\begin{equation}
		\{ a_i \}_{i\in I}
	\end{equation}
	l'ensemble \( \sigma(I)\) en écrivant \( a_i\) pour \( \sigma(i)\).

	La proposition \ref{PROPooTJVUooZDxMpU} dit entre autres que \( \Card\big( \{ a_1,\ldots,a_n \} \big)=n\) ou que
	\begin{equation}
		\Card\big( a_i\tq i\in I \big)=\Card(I)
	\end{equation}
	pourvu que les \( a_i\) sont tous différents.

	La proposition \ref{PROPooFRKAooAnqKEh} permet de dire, si \( \Card(A)=k\), que \( A=\{ a_1,\ldots,a_k \}\).
\end{normaltext}


\begin{proposition}[\cite{MonCerveau}]	\label{PROPooTJVUooZDxMpU}
	Si il existe une injection \(\sigma \colon I\to E  \) alors \( \Card\big( \sigma(I) \big)=\Card(I)\).
\end{proposition}

\begin{proof}
	L'application \( \sigma\) est une bijection entre \( I\) et son image \( \sigma(I)\).
\end{proof}

\begin{proposition}[\cite{MonCerveau}]	\label{PROPooFRKAooAnqKEh}
	Soient un ensemble fini \( E\) tel que \( \Card(E)=n\) et une partie finie \( A\subset E\) telle que \( \Card(A)=k\). Nous considérons une bijection \(\sigma \colon \{ 1,\ldots,n \}\to E  \). Alors il existe \( S\subset\{ 1,\ldots,n \}\) tel que
	\begin{enumerate}
		\item
		      \( \Card(S)=k\)
		\item
		      \( A=\sigma(S)\).
	\end{enumerate}
\end{proposition}

\begin{proof}
	Il suffit de poser \( S=\sigma^{-1}(A)\).
\end{proof}

\begin{lemma}[\cite{MonCerveau}]       \label{LEMooVFPNooVmdUXY}
	Union d'ensembles finis.
	\begin{enumerate}
		\item       \label{ITEMooBUCZooYLCuIe}
		      Si \( A\) et \( B\) sont des ensembles finis disjoints, alors \( A\cup B\) est fini et
		      \begin{equation}
			      \Card(A\cup B)=\Card(A) + \Card(B).
		      \end{equation}
		\item       \label{ITEMooCCWOooYwgGBp}
		      Si \( A\) et \( B\) sont des ensembles finis, alors \( A\cup B\) est fini.
		\item       \label{ITEMooYJSZooXQXkOX}
		      Si \( A\) est fini et si \( B\subset A\) alors
		      \begin{equation}
			      \Card(A\setminus B)=\Card(A)-\Card(B).
		      \end{equation}
		\item       \label{ITEMooSWJCooEpBVkG}
		      Si \( A\) et \( B\) sont des ensembles quelconques, alors
		      \begin{equation}
			      \Card(A\cup B)=\Card(A)+\Card(B)-\Card(A\cap B).
		      \end{equation}
		\item       \label{ITEMooJDUUooVMvAOn}
		      Si les \( \{ A_i \}_{i=1,\ldots, n}\) sont des ensembles deux à deux disjoints, alors
		      \begin{equation}
			      \Card\big( \bigcup_{i=1}^nA_i \big)=\sum_{i=1}^n\Card(A_i).
		      \end{equation}
		\item \label{ITEMooNMFSooBvsNyq}
		      Si \( I\) ou \( J\) est infini, alors \( I \cup J\) est infini.
	\end{enumerate}
\end{lemma}

\begin{proof}
	Point par point.
	\begin{subproof}
		\spitem[Pour \ref{ITEMooBUCZooYLCuIe}]
		Puisque \( A\) et \( B\) sont finis, la proposition \ref{PROPooJLGKooDCcnWi} nous dit qu'il existe des naturels \( N\) et \( M\) ainsi que des bijections \( \varphi_A\colon \{ 0,\ldots, N \}\to A\) et \( \varphi_B\colon \{ 0,\ldots, M \}\to B\). Maintenant l'application
		\begin{equation}
			\begin{aligned}
				\varphi\colon \{ 0,\ldots, M+N+1 \} & \to A\cup B                                    \\
				n                                   & \mapsto \begin{cases}
					                                              \varphi_A(n)     & \text{si }  n\leq N \\
					                                              \varphi_B(n-N-1) & \text{si } n>N
				                                              \end{cases}
			\end{aligned}
		\end{equation}
		est une bijection. Le fait que \( A\) et \( B\) soient disjoints est important pour l'injectivité.  La proposition \ref{PROPooJLGKooDCcnWi} nous dit qu'alors \( A\cup B\) est fini. De plus, par définition le cardinal de \( A\cup B\) est \( N+M\).

		\spitem[Pour \ref{ITEMooCCWOooYwgGBp}]
		Nous ne supposons plus que \( A\) et \( B\) sont disjoints. Nous posons \( I=A\) et \( J=B\setminus A\). Avec ça, \( I\) et \( J\) sont disjoints et finis (comme parties des ensembles finis, lemme \ref{LEMooTUIRooEXjfDY}), et vérifient \( I\cup J=A\cup B\). Le point \ref{ITEMooBUCZooYLCuIe} indique que \( I\cup J\) est fini.

		\spitem[Pour \ref{ITEMooYJSZooXQXkOX}]

		L'ensemble \( A\) peut être écrit sous la forme d'une union disjointe : \( A=B\cup(A\setminus B)\). Les ensembles \( B\) et \( A\setminus B\) étant disjoints, nous avons
		\begin{equation}
			\Card(A)=\Card(B)+\Card(A\setminus B).
		\end{equation}

		\spitem[Pour \ref{ITEMooSWJCooEpBVkG}]
		Nous utilisons quelques égalités d'ensembles pour ramener \( A\cup B\) à des cas déjà traités :
		\begin{equation}        \label{EQooSZDYooPBOdYv}
			A\cup B=A\cup(B\setminus A)=A\cup\big( B\setminus(A\cap B) \big).
		\end{equation}
		Nous avons en particulier utilisé \( B\setminus A=B\setminus(A\cap B)\). La chose intéressante dans \eqref{EQooSZDYooPBOdYv} est que l'union est disjointe et que \( A\cap B\subset B\). Nous pouvons donc écrire
		\begin{equation}
			\Card(A\cup B)=\Card(A)+\Card\big( B\setminus(A\cap B) \big)=\Card(A)+\Card(B)-\Card(A\cap B).
		\end{equation}
		\spitem[Pour \ref{ITEMooJDUUooVMvAOn}]
		Récurrence en utilisant le point \ref{ITEMooSWJCooEpBVkG}.
		\spitem[Pour \ref{ITEMooNMFSooBvsNyq}]
		Toute partie d'un ensemble fini est finie (lemme \ref{LEMooTUIRooEXjfDY}). Donc si \( I\cup J\) était fini, \( I\) et \( J\) devraient l'être.
	\end{subproof}
\end{proof}


%-------------------------------------------------------
\subsection{Minumum, maximum}
%----------------------------------------------------


Une démonstration sans utilisation des naturels est dans la proposition \ref{PROPooVZRVooAncMjb}.
\begin{lemma}[\cite{BIBooQNGLooNkeBVr}]       \label{LEMooPCRFooXRGrUr}
	Si \( E\) est un ensemble ordonné et si \( A\) est une partie finie non vide totalement ordonnée de \( E\), alors \( A\) possède un unique minimum et un unique maximum\footnote{Définition \ref{DEFooBZNRooYRPGme}.}.
\end{lemma}

\begin{proof}
	En plusieurs parties.
	\begin{subproof}
		\spitem[Existence maximum]
		%-----------------------------------------------------------
		Soit un ensemble \( E\) totalement ordonné. Nous faisons par récurrence\footnote{Proposition \ref{PROPooXTRCooKwrWkq}.} sur le cardinal de la partie finie (proposition \ref{PROPooJLGKooDCcnWi} pour la définition du cardinal).

		Si \( \Card(A)=1\), alors nous avons \( A=\{ a \}\) pour un certain \( a\in E\) et ce \( a\) est un maximum de \( A\).

		Sinon nous passons à la récurrence. Nous supposons que pour toute partie \( B\) totalement ordonné de cardinal \( n\), il existe un \( m_B\in B\) tel que \( m_B\geq x\) pour tout \( x\in B\). Soit une partie \( A\) totalement ordonnée de cardinal \( n+1\). Nous prouvons que \( A\) possède un maximum.

		Soit \( a\in A\). La partie \( B=A\setminus\{ a \}\) est de cardinal \( n-1\) (lemme \ref{LEMooVFPNooVmdUXY}\ref{ITEMooYJSZooXQXkOX}). Par hypothèse de récurrence, l'ensemble \( B\) admet un maximum que nous notons \( b\). Il y a deux possibilités : \( a>b\) ou \( a<b\).

		\begin{subproof}
			\spitem[Si \( a>b\)]
			%-----------------------------------------------------------
			Alors nous avons \( a\geq x\) pour tout \( x\in A=B\cup\{ a \}\). En effet si \( x\in B\) nous avons \( x<b<a\). Et si \( x=a\) nous avons \( a=x\).

			\spitem[Si \( b>a\)]
			%-----------------------------------------------------------
			Alors \( b\geq x\) pour tout \( x\in A=B\cup\{ a \}\). En effet si \( x\in B\) nous avons \( b\geq x\) par définition de \( b\) et si \( x=a\) nous avons \( b>x\) par hypothèse.
		\end{subproof}

		\spitem[Unicité maximum]
		%-----------------------------------------------------------
		En ce qui concerne l'unicité, si \( a\) et \( b\) sont des maximums d'une partie \( A\), alors \( a\geq b\) parce que \( b\in A\) et \( b\geq a\) parce que \( a\in A\). La définition \ref{DefooFLYOooRaGYRk}\ref{ITEMooQTXOooTZXEnu} de la relation d'ordre conclu que \( a=b\).
	\end{subproof}
	Pour l'existence et l'unicité du minimum, nous adaptons juste ce que nous venons de faire\quext{Je n'ai pas vérifié. Faites-le et écrivez-moi si ça pose un problème.}.
\end{proof}


\begin{lemmaDef}[\cite{MonCerveau}]        \label{LEMooOEJOooOgaxzi}
	Si \( A\) est une partie non vide de \( \eN\), il existe un unique élément \( m\in \eN\) tel que
	\begin{subequations}
		\begin{numcases}{}
			m\in A\\
			m\leq x\,\forall x\in A.
		\end{numcases}
	\end{subequations}
	Cet élément est noté \( \min(A)\) et nommé \defe{minimum de \( A\)}{minimum}.

	Si \( A\) est majoré, il existe un unique élément \( M\in \eN\) tel que
	\begin{subequations}
		\begin{numcases}{}
			M\in A\\
			M\geq x\,\forall x\in A.
		\end{numcases}
	\end{subequations}
	Cet élément est noté \( \max(A)\) et nommé \defe{maximum de \( A\)}{maximum}.
\end{lemmaDef}

\begin{proof}
	En plusieurs parties.
	\begin{subproof}
		\spitem[Existence minimum]
		%-----------------------------------------------------------
		Soit \( a\in A\). Nous considérons \( E=A\cap\{ 0,\ldots,a \}\). C'est un en semble non vide (il contient \( a\)) et fini en tant que partie de \( \{ 0,\ldots,a \}\) (utilisation des lemmes \ref{LEMooTUIRooEXjfDY} et de \ref{LEMooYIWDooIQFSad}). Donc \ref{LEMooPCRFooXRGrUr} dit que \( E\) possède un minimum. Soit \( m\) ce minimum.

		Si \( x\in A\) alors soit \( x\in E\) (et alors \( l\leq x\)), soit \( x\) n'est pas dans \( E\) et alors \( x\geq a\geq m\).

		\spitem[Existence maximum]
		%-----------------------------------------------------------
		Soit un majorant \( M\) de \( A\). Si \( M\in A\) alors on est bon. Sinon nous posons
		\begin{equation}
			E=\bigcap_{a\in A}\{ a,\ldots,M \}.
		\end{equation}
		C'est un ensemble non vide parce qu'il contient \( M\). Et il est fini parce qu'il est contenu dans \( \{ 0,\ldots,M \}\) qui est fini.

		Nous considérons le minimum \( \mu\) de \( E\). Pour tout \( a\in A\) nous avons \( \mu\in\{ a,\ldots,M \}\) et donc \( \mu\geq a\).

		Nous devons encore prouver que \( \mu\in A\). Supposons que ce ne soit pas le cas. Alors pour tout \( a\in A\) nous avons \( \mu\in \{ a,\ldots,M \}\). Mains \( \mu\neq a\), donc \( \mu\in\{ a+1,\ldots,M \}\). Autrement dit, \( \mu\in\{ a+1,\ldots,M \}\). Donc
		\begin{equation}
			\mu-1\in \bigcap_{a\in A}\{ a,\ldots,M \}=E.
		\end{equation}
		Contradiction à propos de la minimalité de \( \mu\).

		\spitem[Unicité minimum]
		%-----------------------------------------------------------
		Soient des minima \( m_1\) et \( m_2\). Nous avons \( m_1\leq m_2\) parce que \( m_2\in A\) et \( m_1\) est une minimum de \( A\). De même \( m_2\leq m_1\). Donc \( m_1=m_2\).
		\spitem[Unicité maximum]
		%-----------------------------------------------------------
		De même que l'unicité du minimum.
	\end{subproof}
\end{proof}

\begin{lemma}       \label{LEMooKUWUooPLWelf}
	Toute partie finie non vide de \( \eN\) est majorée et minorée.
\end{lemma}

\begin{proof}
	Cas particulier du lemme \ref{LEMooPCRFooXRGrUr}.
\end{proof}


\begin{lemma}       \label{LEMooGQUWooYJQfJB}
	Une partie de \( \eN\) admet un maximum si et seulement si elle est finie.
	%TODOooRRGMooVNRcLR. Prouver ça.
\end{lemma}


\begin{definition}\label{DefEnsembleDenombrable}
	Un ensemble est \defe{dénombrable}{dénombrable} si il peut être mis en bijection avec \( \eN\). Il est \defe{non dénombrable}{non dénombrable} si il est infini et ne peut pas être mis en bijection avec \( \eN\).
\end{definition}
Une chose vraiment amusante avec cette définition que l'on met en rapport avec la définition~\ref{DefEOZLooUMCzZR}, c'est qu'un ensemble fini n'est ni dénombrable ni non dénombrable\footnote{Beaucoup de sources disent qu'un ensemble est dénombrable lorsqu'il est en bijection avec une partie de \( \eN\). Cela laisse la porte ouverte aux ensembles finis. Par exemple Wikipédia\cite{ooLMVKooUiQUtb}.}.


%--------------------------------------------------------------------------------------------------------------------------- 
\subsection{Dénombrabilité et ensemble des naturels}
%---------------------------------------------------------------------------------------------------------------------------

\begin{proposition}[\cite{MonCerveau, BIBooZFPUooIiywbk}]      \label{PROPooOBKMooWEGCvM}
	Toute partie infinie de \( \eN\) est dénombrable.
\end{proposition}

\begin{proof}
	Soit \( A\), une partie infinie de \( \eN\).
	\begin{subproof}
		\spitem[Définition de \( \sigma\)]
		Nous voulons construire une application \( \sigma\colon \eN\to A\) telle que
		\begin{subequations}
			\begin{numcases}{}
				\sigma(0)=\min(A)   \label{SUBEQooEIEMooZcTOWT}\\
				\sigma(k+1)=\min\Big( A\setminus \sigma\big( \{ 0,\ldots, k \} \big)\Big)      \label{SUBEQooWWOAooAEfrPx}
			\end{numcases}
		\end{subequations}
		Les lâches, par prudence, diront juste que c'est défini par récurrence et n'insisteront pas. Nous, nous insistons.

		Nous allons définir \( \sigma(n)\) à l'aide du théorème \ref{THOooEJPYooZFVnez}. Pour cela nous posons \( E=\mP(A)\), \( b=\emptyset\) et
		\begin{equation}
			\begin{aligned}
				g\colon E & \to E                                                  \\
				Z         & \mapsto \begin{cases}
					                    A                             & \text{si } Z=A \\
					                    Z\cup\{ \min(A\setminus Z) \} & \text{sinon. }
				                    \end{cases}
			\end{aligned}
		\end{equation}
		Notons que la proposition \ref{PROPooMZOWooHmsXzI}\ref{ITEMooYAJIooEFmOpB} nous indique que toute partie non vide de \( \eN\) possède un minimum; la définition de \( g\) a donc un sens. Le théorème \ref{THOooEJPYooZFVnez} donne alors une application \( f\colon \eN\to E\) telle que
		\begin{enumerate}
			\item
			      \( f(0)=b=\emptyset\)
			\item
			      \( f(n+1)=g\big( f(n) \big)\) pour tout \( n\geq 0\).
		\end{enumerate}
		Prouvons par récurrence que \( f(n)\) est un ensemble fini pour tout \( n\). D'abord \( f(0)=\emptyset\). Ensuite, si \( n\geq 0\) est tel que \( f(n)\) est fini, alors en particulier \( f(n)\neq A\) et nous avons
		\begin{equation}
			f(n+1)=g\big( f(n) \big)=f(n)\cup\{ \min\big( A\setminus f(n) \big)\}.
		\end{equation}
		Dans ce cas, \( f(n+1)\) est également fini comme union de deux ensembles finis.

		Nous posons
		\begin{equation}        \label{EQooGHQHooRnXDdo}
			\sigma(n)=\min\big( A\setminus f(n) \big).
		\end{equation}
		Avec \( n=0\), nous avons \( \sigma(0)=\min\big( A\setminus \emptyset \big)=\min(A)\). La condition \eqref{SUBEQooEIEMooZcTOWT} est donc déjà satisfaite.

		Nous devons encore prouver \eqref{SUBEQooWWOAooAEfrPx}.  Pour tout \( n\), la relation entre \( f(n)\) et \( \sigma(n)\) est donnée par
		\begin{subequations}
			\begin{numcases}{}
				f(0)=\emptyset\\
				f(n+1)=f(n)\cup \sigma(n).
			\end{numcases}
		\end{subequations}
		Par récurrence nous avons alors
		\begin{equation}        \label{EQooPXFEooYzhtBe}
			f(n)=\bigcup_{k=0}^{n-1}\{ \sigma(k) \}=\sigma\big( \{ 0,\ldots, n-1 \} \big)
		\end{equation}
		pour tout \( n\geq 1\). Nous avons alors la condition \ref{SUBEQooWWOAooAEfrPx} en substituant \eqref{EQooPXFEooYzhtBe} dans la définition \eqref{EQooGHQHooRnXDdo} écrite avec \( n+1\) :
		\begin{equation}
			\sigma(n+1)=\min\big( A\setminus f(n+1) \big)=\min\Big( A\setminus \sigma\big( \{ 0,\ldots, n \} \big) \Big).
		\end{equation}

		\spitem[\( \sigma\) est strictement croissante]
		Vu que \( A\setminus\sigma\{ 0,\ldots, k \}\subset A\setminus\sigma\{ 0,\ldots, k-1 \}\), le minimum est plus grand ou égal : \( \sigma(k+1)\geq \sigma(k)\). Mais \( \sigma(k+1)\) est sélectionné dans l'ensemble \( A\setminus\sigma\{ 0,\ldots, k \}\), qui ne contient justement pas \( \sigma(k)\). Donc \( \sigma(k+1)\neq \sigma(k)\).
		\spitem[\( \sigma\) est définie sur \( \eN\)]
		Il faut montrer que pour tout \( k\), l'ensemble \( A\setminus\sigma\{ 0,\ldots, k \}\) est non vide. Si il l'était, cela signifierait que \( A\subset \sigma\{ 0,\ldots, k \}\). Par le lemme \ref{LEMooPGPVooZzlFvf}\ref{ITEMooNCCUooBGrtdn}, la partie \( \sigma\{ 0,\ldots, k \}\) est finie dans \( \eN\). Le lemme \ref{LEMooTUIRooEXjfDY} dit alors qu'en tant que partie de \( \sigma\{ 0,\ldots, k \}\), l'ensemble \( A\) est fini. Mais comme les hypothèses disent que \( A\) est infini, nous avons une contradiction et nous concluons que \( \sigma\) est bien définie sur tout \( \eN\).
		\spitem[\( \sigma\) est injective]
		Une application \( \eN\to \eN\) strictement croissante est injective par la proposition \ref{PROPooFYMJooWihvhk}.
		\spitem[\( \sigma\) est surjective]
		Soit \( a\in A\). Vu que \( \sigma\) est strictement croissante et que \( \sigma(0)\geq 0\), nous avons \( \sigma(a)\geq a\). Si \( \sigma(a)=a\) nous avons terminé. Supposons \( \sigma(a)>a\). Alors
		\begin{equation}        \label{EQooNHTBooQexzwV}
			\min\big( A\setminus\sigma\{ 0,\ldots, a \} \big)>a.
		\end{equation}
		Si \( \sigma\big( \{ 0,\ldots, a \} \big)\) ne contenait pas \( a\), alors \( A\setminus \sigma(\{ 0,\ldots, a \})\) le contiendrait et nous n'aurions pas l'inégalité \eqref{EQooNHTBooQexzwV}. Donc \( a\in \sigma\big( \{ 0,\ldots, a \} \big)\) et \( a\) est bien dans l'image de \( \sigma\).
	\end{subproof}
\end{proof}

\begin{normaltext}	\label{NORMooPartieDenombrableParCantorBernstein}
	La proposition \ref{PROPooOBKMooWEGCvM} pourrait être prouvée plus facilement en acceptant le théorème de Cantor-Schröder-Bernstein \ref{THOooRYZJooQcjlcl}. Il existe une injection \( A\to \eN\) parce que \( A\) est une partie de \( \eN\). Mais puisque \( A\) est infini, il possède une partie dénombrable. Cela donne une surjection \( A\to \eN\) et donc une injection \( \eN\to A\). Le théorème de Cantor-Schröder-Bernstein conclut.

	Cela dit, une telle preuve demanderait des outils plus complexes.
\end{normaltext}


\begin{normaltext}	\label{NORMooBijectionNNMadore}
	La proposition suivante donne une bijection explicite entre \( \eN\) et \( \eN\times \eN\). Elle n'a rien de transcendante, mais je ne résiste pas à la donner ici parce qu'elle est utilisée dans l'article \emph{Un peu de programmation transfinie} de David Madore\footnote{Et comme j'aime beaucoup cet article, il me fallait une excuse pour le placer ici.\\ \url{http://www.madore.org/~david/weblog/d.2017-08-18.2460.html}.}. Son utilité est de pouvoir créer un langage de programmation pouvant traiter des paires d'entiers rien qu'en traitant des entiers.
\end{normaltext}

\begin{proposition}[Une bijection \( \eN\times \eN\to \eN\)]        \label{PROPooLPKUooAlsYJg}
	La fonction
	\begin{equation}
		\begin{aligned}
			f\colon \eN\times \eN & \to \eN                               \\
			(x,y)                 & \mapsto \begin{cases}
				                                y^2+x   & \text{si } x<y      \\
				                                x^2+x+y & \text{si } y\leq x.
			                                \end{cases}
		\end{aligned}
	\end{equation}
	est une bijection.
\end{proposition}

\begin{proof}
	Il s'agit de prouver qu'elle est injective et surjective. Dans la suite, tous les nombres sont des entiers positifs.
	\begin{subproof}
		\spitem[\( f\) est injective]

		Pour \( k\in \eN\) donné, nous allons prouver que
		\begin{enumerate}
			\item
			      l'équation \( f(x,y)=k\) possède au maximum une solution avec \( x<y\),
			\item
			      l'équation \( f(x,y)=k\) possède au maximum une solution avec \( y\leq x\),
			\item
			      si \(   k=y^2+x \) avec \( x<y\) alors il est impossible que \( k=x'^2+x'+y'\) avec \( y'\leq x'\).
		\end{enumerate}
		On y va.
		\begin{enumerate}
			\item
			      Nous supposons \( y^2+x=t^2+z\) avec \( x<y\) et \( z<t\). Pour fixer les idées, nous supposons \( t>y\) et nous posons \( t=y+s\) (\( s\geq 1\)). En substituant, et en isolant \( z\),
			      \begin{subequations}
				      \begin{align}
					      z & =x-2sy-s^2 \\
					        & <x-2sy     \\
					        & <x-2sx     \\
					        & =x(1-2s)   \\
					        & <0.
				      \end{align}
			      \end{subequations}
			      Impossible parce que \( z\geq 0\).
			\item
			      De même nous supposons \( x^2+x+y=z^2+z+t\) avec \( y\leq x\) et \( t\leq z\). Nous posons \( z=x+s\), et nous déballons le même genre de calculs en isolant \( t\).
			\item
			      Enfin nous supposons \( y^2+x=z^2+z+t\) avec \( x<y\) et \( t\leq z\). Les plus courageux diviseront en trois cas : \( y<z\), \( y=z\) et \( y>z\) et feront les calculs. Par exemple, pour le cas \( y>z\) nous posons \( y=z+s\) et nous substituons :
			      \begin{equation}
				      (y+s)^2+x=z^2+z+t
			      \end{equation}
			      qui donne
			      \begin{equation}
				      x=z+t-2zs-s^2<2z-2zs-s^2=2z(1-s)-s^2\leq -s<0
			      \end{equation}
			      parce que \( s\geq 1\), donc \( 1-s\leq 0\).
		\end{enumerate}

		\spitem[\( f\) est surjective]

		Nous devons prouver que tous les éléments de \( \eN\) sont dans l'image de \( \eN\times \eN\) par \( f\). En premier lieu, \( 0=f(0,0)\). C'est un bon début. Soit \( a\in \eN\) non nul; nous montrons que tous les nombres de \( a^2\) à \( (a+1)^2\) sont des images de \( f\). D'abord \( a^2=f(0,a)\), ensuite les nombres
		\begin{equation}
			f(1,a),f(2,a),\ldots, f(a-1,a)
		\end{equation}
		prennent les valeurs \( a^2+1\), \ldots, \( a^2+a-1\). Enfin nous avons \( f(a,0)=a^2+a\) et les nombres \( f(a,1),\ldots, f(a,a)\) prennent les valeurs de \( a^2+a+1\) à \( a^2+2a=(a+1)^2-1\).
	\end{subproof}
\end{proof}

\begin{remark}	\label{REMooBijectionNNOrdinaux}
	Sachez que cette fonction s'étend aux ordinaux (mais là ce n'est plus pour rigoler).
\end{remark}

\begin{corollary}       \label{CORooNRPIooZPSmqa}
	Il existe des parties \( \{ \eN_i \}_{i\in \eN}\) telles que \( \bigcup_{i\in \eN}\eN_i=\eN\) et que chaque \( \eN_i\) soit en bijection avec \( \eN\)
\end{corollary}

\begin{proof}
	Nous considérons la bijection \( f\colon \eN\to \eN\times \eN\) donnée par (l'inverse de celle donnée) par la proposition \ref{PROPooLPKUooAlsYJg}, et nous posons
	\begin{equation}
		\eN_i=f^{-1}(i,\eN).
	\end{equation}
	L'application
	\begin{equation}
		f\colon \eN_i\to \{ (i,k) \}_{k\in \eN}
	\end{equation}
	est une bijection. Or l'ensemble \( \{ (i,k) \}_{k\in \eN}\) est évidemment en bijection avec \( \eN\). Par composition nous avons le résultat.
\end{proof}

\begin{lemma}[\cite{MonCerveau}]        \label{LEMooDLWFooNAJbbq}
	Si il existe une surjection \( \eN\to A\), alors \( A\) est fini ou dénombrable.
\end{lemma}

\begin{proof}
	Pour chaque \( a\in A\), l'ensemble \( f^{-1}(a)\) est une partie de \( \eN\).
	\begin{subproof}
		\spitem[Une application]
		La proposition \ref{PROPooMZOWooHmsXzI}\ref{ITEMooYAJIooEFmOpB} nous permet de poser
		\begin{equation}
			\begin{aligned}
				\sigma\colon A & \to \eN                            \\
				a              & \mapsto \min\big( f^{-1}(a) \big).
			\end{aligned}
		\end{equation}
		\spitem[\( \sigma\) est injective]
		Supposons que \( \sigma(a)=\sigma(b)\). Nous appelons \( x\) ce nombre :
		\begin{equation}
			x=\min\big( f^{-1}(a) \big)=\min\big( f^{-1}(b) \big).
		\end{equation}
		Nous avons \( x\in f^{-1}(a)\cap f^{-1}(b)\), ce qui implique que \( f(x)=a\) et que \( f(x)=b\); donc \( a=b\).

		Donc \( \sigma\) est une injection.
		\spitem[\( A\) est infini]
		Si \( A\) est fini, le lemme est prouvé. Donc à partir de maintenant nous supposons que \( A\) est infini. Le but est de prouver qu'il est dénombrable, c'est-à-dire de construire une bijection \( A\to \eN\).
		\spitem[\( \sigma(A)\) est dénombrable]
		Puisque \( \sigma\colon A\to  \eN\) est injective et que \( A\) est infini, la proposition \ref{PROPooWKSIooHcfYPN} dit que \( \sigma(A)\) est infini dans \( \eN\). La proposition \ref{PROPooOBKMooWEGCvM} nous dit alors que \( \sigma(A)\) est dénombrable.

		Soit une bijection \( \varphi\colon \sigma(A)\to \eN\).
		\spitem[La candidate bijection]
		Nous posons
		\begin{equation}
			f=\varphi\circ \sigma\colon A\to \eN
		\end{equation}
		et nous allons prouver que c'est une bijection.
		\spitem[Injective]
		Puisque \( \varphi\) et \( \sigma\) sont injectives, l'égalité \( (\varphi\sigma)(a)=(\varphi\sigma)(b)\) implique immédiatement \( a=b\).
		\spitem[Surjective]
		Soit \( k\in \eN\). Puisque \( \varphi\) et \( \sigma\) sont des injections, nous pouvons poser \( a=(\sigma^{-1}\varphi^{-1})(k)\). Il est alors immédiat que \( f(a)=k\).
	\end{subproof}
\end{proof}

\begin{proposition}[\cite{MonCerveau,ooLMVKooUiQUtb}]     \label{PROPooENTPooSPpmhY}
	Une union dénombrable d'ensembles finis ou dénombrables est finie ou dénombrable.
\end{proposition}

\begin{proof}
	Soient \( A_i\) des ensembles finis ou dénombrables. Nous posons \( A=\bigcup_{i\in \eN}A_i\), et nous considérons les parties \( \eN_i\) du corolaire \ref{CORooNRPIooZPSmqa}. Puisque \( A_i\) est dénombrable ou fini et que \( \eN_i\) est dénombrable, il existe une surjection \( \varphi_i\colon \eN_i\to A_i\).

	Nous définissons \( s\colon \eN\to \eN\) par \( n\in \eN_{s(n)}\), et nous posons enfin
	\begin{equation}
		\begin{aligned}
			\varphi\colon \eN & \to A                      \\
			n                 & \mapsto \varphi_{s(n)}(n).
		\end{aligned}
	\end{equation}
	Nous prouvons que \( \varphi\) est surjective.

	Soit \( a\in A_i\). Il existe \( n\in \eN_i\) tel que \( a=\varphi_i(n)\). Mais comme \( n\in \eN_i\), nous avons \( s(n)=i\). Donc
	\begin{equation}
		a=\varphi_i(n)=\varphi_{s(n)}(n)=\varphi(n).
	\end{equation}
	Donc \( \varphi\colon \eN\to A\) est surjective.

	Le lemme \ref{LEMooDLWFooNAJbbq} conclut que \( A\) est fini ou dénombrable.
\end{proof}



\begin{lemma}       \label{LEMooDTAEooIBdHyo}
	Si \( A\) est dénombrable et si il existe une surjection \( f\colon A\to B\), alors \( B\) est fini ou dénombrable.
\end{lemma}

\begin{proof}
	Étant donné que \( A\) est dénombrable, il existe une bijection \(\varphi \colon \eN\to A  \). L'application \(f\circ \varphi \colon \eN\to B  \) est donc une surjection de \( \eN\) vers \( B\). Donc \( B\) est fini ou dénombrable par le lemme \ref{LEMooDLWFooNAJbbq}.
\end{proof}<++>


\begin{lemma}       \label{LEMooSRZWooASgEfy}
	Si \( A\) est un ensemble fini ou dénombrable, alors il existe une surjection \( \eN\to A\).
	%TODOooYCRWooTJZgyO. Prouver ça. Et j'en mets deux.
\end{lemma}


\begin{lemma}[\cite{MonCerveau}]		\label{LEMooKFBAooHxgOsg}
	Si \( A\) est dénombrable, alors l'ensemble des parties finies de \( A\) est dénombrable.
	%TODOooCJEUooSXuphm. Prouver ça.
\end{lemma}

\begin{lemma}[\cite{MonCerveau}]        \label{LEMooRXSRooBUWOyb}
	Si \( N\) est un ensemble dénombrable, alors il existe une bijection \( g\colon \{ 1,2 \}\times N\to N\).
\end{lemma}

\begin{proof}
	D'abord nous définissons une bijection \( \varphi\colon \{ 0,1 \}\times \eN\to \eN\) par
	\begin{equation}
		\begin{aligned}
			\varphi\colon \{ 0,1 \}\times \eN & \to \eN       \\
			(n,k)                             & \mapsto 2k+n.
		\end{aligned}
	\end{equation}
	Ensuite si \( f\colon \eN\to N\) est une bijection, il suffit de poser \( g(n,k)=f\big( \varphi(n,k) \big)\).
\end{proof}

\begin{proposition}[Produit d'ensembles dénombrables\cite{ooLMVKooUiQUtb}]     \label{PROPooDMZHooXouDrQ}
	Deux résultats.
	\begin{enumerate}
		\item		\label{ITEMooWNBGooZjoIjZ}
		      Si \( A\) et \( B\) sont dénombrables, alors \( A\times B\) est dénombrable.
		\item
		      Si \( N\) est un ensemble dénombrable, alors pour tout \( n\in \eN\), l'ensemble \( N^n\) est dénombrable.
	\end{enumerate}
	%TODOooDUMVooPcBWkb. Prouver ça.
	% note : c'est déjà dans les pauses de mon cahier.
\end{proposition}

\begin{proof}
	La proposition \ref{PROPooLPKUooAlsYJg} donne une bijection \(f \colon \eN\times \eN\to \eN  \). Par hypothèse sur \( A\) et \( B\) nous avons des bijections \(\varphi \colon A\to \eN  \) et \(\sigma \colon B\to \eN  \). Nous posons
	\begin{equation}
		\begin{aligned}
			f\colon A\times B & \to \eN                                    \\
			(a,b)             & \mapsto f\big( \varphi(a), \sigma(b) \big)
		\end{aligned}
	\end{equation}
	et nous prouvons que c'est une bijection.
	\begin{subproof}
		\spitem[Injectif]
		%-----------------------------------------------------------
		Supposons que \( g(a,b)=g(x,y)\). Vu que \( f\) est une bijection, nous avons \( \varphi(a)=\varphi(x)\) et \( \sigma(b)=\sigma(y)\). Étant donné que  \( \varphi\) et \( \sigma\) sont des bijections, ça donne \( a=x\) et \( b=y\). Donc \( (a,b)=(x,y)\)).

		\spitem[Surjectif]
		%-----------------------------------------------------------
		Soit \( n\in \eN\). Nous posons \( (a,b)=f^{-1}(n)\) puis \( x=\varphi^{-1}(a)\) et \( y=\sigma^{-1}(b)\). Avec ça nous avons
		\begin{equation}
			n=f(a,b)=f\big( \varphi(x),\sigma(y) \big)=g(a,b).
		\end{equation}
		Donc \( n\) est bien dans l'image de \( g\).
	\end{subproof}

	En ce qui concerne le fait que \( A^n\) soit dénombrable, nous le faisons par récurrence. Pour \( n=1\), nous savons que \( A^n=A\) est dénombrable. Si nous supposons que \( A^{n-1}\) est dénombrable, alors nous avons \( A^n=A^{n-1}\times A\) qui est dénombrable en tant que produit cartésien de deux ensembles dénombrables.
\end{proof}

\begin{normaltext}	\label{NORMooDenombrablesPlusPetitsInfinis}
	Les ensembles dénombrables sont les plus petits ensembles infinis possibles, comme en témoigne la proposition suivante.
\end{normaltext}

\begin{proposition}      \label{PROPooUIPAooCUEFme}
	Tout ensemble infini contient une partie en bijection avec \( \eN\).
\end{proposition}

\begin{proof}
	Soient un ensemble infini \( E_0\) et une partie propre \( E_1\) en bijection avec \( E_0\). Nous notons \( \varphi\colon E_0\to E_1\) une bijection.

	Soit \( x_0\in E_0\setminus E_1\) (axiome du choix et tout ça). Nous définissons
	\begin{equation}
		\begin{aligned}
			\psi\colon \eN & \to \{\varphi^k(x_0)\} \\
			n              & \mapsto \varphi^n(x_0)
		\end{aligned}
	\end{equation}
	et nous allons prouver que c'est une bijection. Que ce soit surjectif est immédiat. Pour l'injectivité, soit \( \varphi^k(x_0)=\varphi^l(x_0)\) avec \( k\neq l\). Supposons pour fixer les notations que \( k>l\). Alors, vu que \( \varphi\) est inversible nous pouvons écrire
	\begin{equation}
		x_0=\varphi^{k-l}(x_0)=\varphi\big( \varphi^{k-l-1}(x_0) \big)
	\end{equation}
	où il est entendu que \( \varphi^0(x_0)=x_0\). Cela signifie que \( x_0\) est dans l'image de \( \varphi\), c'est-à-dire dans \( E_1\), ce que nous avons exclu par choix de \( x_0\) dans \( E_0\setminus E_1\). Donc en réalité \( \varphi^k(x_0)\neq \varphi^l(x_0)\) dès que \( k\neq l\).
\end{proof}

\begin{proposition} \label{PropQEPoozLqOQ}
	Toute partie d'un ensemble fini est finie, et toute partie d'un ensemble dénombrable est finie ou dénombrable.
\end{proposition}

\begin{proof}
	Soient un ensemble \( E\) ainsi qu'une partie infinie \( A\subset E\). Nous notons \( \varphi\colon A\to A'\) une bijection entre \( A\) et une partie propre \( A'\) de \( A\). Dans ce cas, l'application
	\begin{equation}
		\begin{aligned}
			\phi\colon E & \to (E\setminus A)\cup A'                         \\
			x            & \mapsto \begin{cases}
				                       x          & \text{si } x\in E\setminus A \\
				                       \varphi(x) & \text{si }x\in A
			                       \end{cases}
		\end{aligned}
	\end{equation}
	est une bijection entre \( E\) et une partie propre de \( E\). Donc \( E\) est infini.

	Par contraposée nous déduisons que toute partie d'un ensemble fini est finie.

	En ce qui concerne les parties d'ensembles dénombrables, soit une partie \( A\) d'un ensemble dénombrable \( E\). Nous avons une bijection \( t\varphi\colon E\to \eN\). La restriction \( \varphi\colon A\to \varphi(A)\) est une bijection entre \( A\) est une partie de \( \eN\).

	\begin{itemize}
		\item Si \( \varphi(E)\) est infinie, elle est dénombrable (proposition \ref{PROPooOBKMooWEGCvM}). Dans ce cas \( A\) est en bijection avec un ensemble dénombrable. Il est donc dénombrable.
		\item
		      Si \( \varphi(E)\) est fini, alors le lemme \ref{LEMooPGPVooZzlFvf}\ref{ITEMooKQMFooSzmXrd} nous dit que \( A\) est fini.
	\end{itemize}
\end{proof}

\begin{lemma}   \label{LEMooGTOTooFbpvzU}
	Soit un ensemble \( E\) non dénombrable ainsi qu'une application \( f\colon E\to F\) où \( F\) est un ensemble quelconque. Si \( f(E)\) est dénombrable (ou fini), alors il existe \( y\in f(E)\) tel que \( f^{-1}(y)\) est indénombrable.
\end{lemma}

\begin{proof}
	Nous avons
	\begin{equation}
		E=\bigcup_{y\in F}f^{-1}(y).
	\end{equation}
	Si tous les \( f^{-1}(y)\) sont dénombrables, alors \( E\) est une union dénombrable (\( F\) est dénombrable) d'ensembles dénombrables. Il serait donc dénombrable (proposition \ref{PROPooENTPooSPpmhY}), ce qui est contraire à l'hypothèse.
\end{proof}


%--------------------------------------------------------------------------------------------------------------------------- 
\subsection{Ajouter ou soustraire des cardinalités}
\label{SUBooOperationsCardinaux}
%---------------------------------------------------------------------------------------------------------------------------

\begin{normaltext}	\label{NORMooOperationsCardinaux}
	Nous allons prouver une série de résultats que nous pourrions résumer en  «ajouter ou retrancher des parties de cardinalité plus petite ne change pas la cardinalité».
\end{normaltext}

\begin{lemma}[\cite{MonCerveau}]        \label{LEMooUFCAooSyZtZj}
	Si \( A\) est infini et \( B\) est fini, alors \( A\cup B\approx A\).
\end{lemma}

\begin{proof}
	Nous supposons que \( A\) et \( B\) sont disjoints\footnote{Adaptez la démonstration au cas où l'intersection n'est pas vide.}. La proposition \ref{PROPooJLGKooDCcnWi} nous permet de considérer une bijection \( \psi\colon \{ 1,\ldots, n \}\to B\).

	Puisque \( A\) est infini, la proposition \ref{PROPooUIPAooCUEFme} nous permet de considérer \( N\subset A\) et une bijection \( \varphi\colon \eN\to N\).

	Maintenant, il s'agit seulement d'insérer \( B\) dans \( A\) en le mettant «au début» de \( N\) et en décalant les autres éléments. La bijection est
	\begin{equation}
		\begin{aligned}
			f\colon A\cup B & \to A                                                                                  \\
			x               & \mapsto \begin{cases}
				                          x                                    & \text{si } x\in A\setminus N \\
				                          \varphi\big( \varphi^{-1}(x)+n \big) & \text{si } x\in N            \\
				                          \varphi\big( \psi^{-1}(x) \big)      & \text{si }x\in B.
			                          \end{cases}
		\end{aligned}
	\end{equation}
\end{proof}

\begin{lemma}       \label{LEMooXMVDooIWLWis}
	Si \( A\) est infini et si \( A\) est surpotent à \( B\), alors \( A\approx A\cup B\).
\end{lemma}

\begin{proof}
	Il existe évidemment une injection \( A\to A\cup B\). Donc le théorème de Cantor-Schröder-Bernstein \ref{THOooRYZJooQcjlcl} nous indique que trouver une injection \( A\cup B\to A\) suffira pour la peine.

	Nous allons utiliser le lemme de Zorn \ref{LemUEGjJBc} avec l'ensemble
	\begin{equation}
		\mA=\Big\{  (X,\varphi_X)  \tq
		\begin{cases}
			X\subset B \\
			\varphi_X\colon A\cup X\to A\text{ est injective.}
		\end{cases}
		\Big\}
	\end{equation}
	muni de l'ordre de l'inclusion : \( (X,\varphi_X)<(Y,\varphi_Y)\) si \( X\subset Y\) et \( \varphi_Y(x)=\varphi_X(x)\) pour tout \( x\in A\cup X\).

	\begin{subproof}
		\spitem[\( \mA\) est inductif]
		Soit une famille \( \mF=\{ (X_i,\varphi_i) \}_{i\in I}\) complètement ordonnée indexée par l'ensemble \( I\). En posant \( X=\bigcup_{i\in I}X_i\) et \( \varphi(x)=\varphi_i(x)\) dès que \( x\in A\cup X_i\), l'élément \( (X,\varphi)\) majore \( \mF\).
		\spitem[Un maximum]
		Le lemme de Zorn nous assure que \( \mA\) possède (au moins) un élément maximal. Soit un tel élément maximum \( (X,\varphi)\).
		\spitem[\( X\approx B\)]
		Ah oui, vous auriez aimé avoir \( X=B\). Mais non; il n'y a pas de garantie. Nous allons montrer que \( X\approx B\), et ça suffira.

		Vu que \( X\subset B\), si \( X\) n'est pas équipotent à \( B\), il est strictement inclus dans \( B\). Nous pouvons donc considérer
		\begin{subequations}
			\begin{align}
				b & \in B\setminus X           \\
				a & \in A\setminus \varphi(X).
			\end{align}
		\end{subequations}
		Nous considérons alors l'élément \( (Y,\psi)\in \mA\) défini par
		\begin{subequations}
			\begin{align}
				Y       & =X\cup\{ b \}                \\
				\psi(x) & =\begin{cases}
					           a          & \text{si } x=b \\
					           \varphi(x) & \text{sinon }.
				           \end{cases}
			\end{align}
		\end{subequations}
		Cet élément majore \( (X,\varphi)\).

		Donc \( X\approx B\).
		\spitem[Résumé de la situation]
		Nous avons \( A\approx A\cup X\) ainsi qu'une injection \( \varphi\colon A\cup X\to A\) et une bijection \( \psi\colon B\to X\).
		\spitem[Conclusion si \( A\) est disjoint de \( B\)]
		Si \( A\) et \( B\) sont disjoints, nous avons une bijection
		\begin{equation}
			\begin{aligned}
				l\colon A\cup B & \to A                                                                   \\
				x               & \mapsto \begin{cases}
					                          \varphi(x)                 & \text{si }  x\in A \\
					                          \varphi\big( \psi(x) \big) & \text{si } x\in B.
				                          \end{cases}
			\end{aligned}
		\end{equation}
		\spitem[Conclusion si \( A\) n'est pas disjoint de \( B\)]
		Il suffit de poser \( C=B\setminus A\) et nous avons
		\begin{equation}
			A\cup B=[A\cup (A\cap B)]\cup C.
		\end{equation}
		Cette union est disjointe, \( A\cup(A\cap B)\) est surpotent à \( A\) et \( C\) est subpotent à \( B\). La conclusion est donc encore valable.
	\end{subproof}
\end{proof}

\begin{normaltext}	\label{NORMooBijectionApplicThMesure}
	La proposition suivante sera utilisée en théorie de la mesure, dans l'exemple~\ref{ExOIXoosScTC}.
\end{normaltext}

\begin{proposition}[\cite{ooFAOQooACUugI,BIBooZFPUooIiywbk}] \label{PropVCSooMzmIX}
	Si \( S\) est un ensemble infini alors il existe une bijection \( \varphi\colon \{ 0,1 \}\times S\to S\).
\end{proposition}
% La preuve de cette proposition a été grandement simplifiée le 6 juin 2021. Ce commentaire sert à retrouver l'ancienne preuve plus
% facilement dans l'historique git. Le premier commit a avoir changé la preuve est celui qui a introduit aussi ce commentaire. ooWADNooUkhTWI

\begin{proof}
	Nous posons \( A=\{ 0 \}\times S\) et \( B=\{ 1 \}\times S\). L'ensemble \( A\) est infini et surpotent à \( B\) (pas strictement, mais quand même).

	Donc \( A\) est idempotent à \( A\cup B\) par le lemme \ref{LEMooXMVDooIWLWis}. Mais \( A\) est idempotent à \( S\), donc
	\begin{equation}
		S\approx A \approx A\cup B=\{ 0,1 \}\times S.
	\end{equation}
\end{proof}

\begin{corollary}       \label{CORooJCSIooOeOICJ}
	Si \( A\) est un ensemble infini, alors \( A\) possède deux sous-ensembles disjoints \( A_1\) et \( A_2\) qui sont tous deux en bijection avec \( A\).
\end{corollary}

\begin{proof}
	La proposition \ref{PropVCSooMzmIX} donne une bijection \( \varphi\colon \{ 0,1 \}\times A\to A\). Il suffit de poser \( A_1=\varphi(0,A)\) et \( A_2=\varphi(1,A)\).
\end{proof}

\begin{normaltext}	\label{NORMooInfiniteCopies}
	Maintenant que nous pouvons mettre dans \( A\) deux copies disjointes de \( A\), il n'est pas très étonnant que nous puissions en mettre une infinité dénombrable. Nous verrons ça dans \ref{PROPooFKBEooKXqujV}.
\end{normaltext}

\begin{proposition} \label{PROPooFKBEooKXqujV}
	Si \( A\) est infini, alors \( A\times \eN\approx A\).
\end{proposition}

\begin{proof}
	La démonstration se base sur le fait qu'à l'intérieur de \( A\), nous pouvons construire autant de copies de \( A\) deux à deux disjointes que nous le voulons. La \( k\)\ieme\ «copie» sera naturellement l'image de \( k\times A\).

	Voyons tout cela en détail.
	\begin{subproof}
		\spitem[Ce que nous allons faire]
		Nous allons construire, pour tout \( i\geq 1 \) des parties \( A_i,B_i\subset A\) telles que
		\begin{itemize}
			\item \( A_i\cap B_i=\emptyset\),
			\item \( A_i,B_i\subset B_{i-1}\),
			\item \( A_i\approx B_i\approx A\),
			\item \( A_i\cap A_j=\emptyset\) si \( i\neq j\)
		\end{itemize}
		\spitem[La construction]
		Nous commençons à zéro en utilisant le corolaire \ref{CORooJCSIooOeOICJ} pour construire des parties disjointes \( A_0\) et \( B_0\) de \( A\) telles que \( A_0\approx B_0\approx A\).

		Ensuite, puisque \( B_0\approx A\), il existe \( A_1\) et \( B_1\) dans \( B_0\) tels que \(  A_1\cap B_1=\emptyset\) et \( A_1\approx B_1\approx B_0\approx A\). Cela est notre construction pour \( i=1\).

		Pour la récurrence, puisque \( A_i\approx B_i\approx A\), nous considérons \( A_{i+1}\) et \( B_{i+1}\) dans \( B_i\) tels que \( A_{i+1}\cap B_{i+1}=\emptyset\) et \( A_{i+1}\approx B_{i+1}\approx B_i\approx A\). C'est encore le corolaire \ref{CORooJCSIooOeOICJ} qui fait le travail.

		\spitem[Les propriétés]
		Nous avons \( A_i\cap A_{i+1}=\emptyset\) parce que \( A_i\cap A_{i+1}\subset A_i\cap B_i=\emptyset\).

		Nous devons encore montrer que \( A_i\cap A_j=\emptyset\) dès que \( i\neq j\). Supposons que \( j>i\). Nous avons les inclusions
		\begin{equation}
			A_j\subset B_{j-1}\subset B_{j-2}\subset \ldots \subset B_i.
		\end{equation}
		Donc \( A_j\cap A_i\subset B_i\cap A_i=\emptyset\).
		\spitem[Une injection]
		Nous pouvons à présent écrire une injection qui termine presque la preuve. Pour cela nous considérons pour tout \( i\), une bijection \( \psi_i\colon A\to A_i\). Ensuite nous posons
		\begin{equation}
			\begin{aligned}
				\varphi\colon A\times \eN & \to A              \\
				(a,k)                     & \mapsto \psi_k(a).
			\end{aligned}
		\end{equation}
		Si \( \varphi(a,k)=\varphi(b,l)\), alors \( \psi_k(a)=\psi_l(b)\). L'élément \( \psi_k(a)\) est donc dans \( A_k\cap A_l\); ce n'est possible que si \( k=l\). Donc \( \psi_l(a)=\psi_l(b)\). Cette dernière égalité n'est possible que si \( a=b\) parce que \( \psi_l\) est une bijection.

		Donc \( \varphi\) est une injection, et nous avons prouvé que \( A\times \eN\preceq A\).
		\spitem[La bijection]
		Nous venons de prouver que \( A\times \eN\preceq A\). La surpotence \( A\times \eN\succeq A\) étant évidente, le théorème de Cantor-Schröder-Bernstein \ref{THOooRYZJooQcjlcl} conclut que \( A\times \eN\approx A\).
	\end{subproof}
\end{proof}

\begin{lemma}[\cite{MonCerveau}]        \label{LEMooDHWSooFqhano}
	Soit un ensemble \( A\) muni de deux sous-ensembles \( B\) et \( B'\) équipotents et disjoints. Alors \( A\setminus B\) est équipotent à \( A\setminus B'\).
\end{lemma}

\begin{proof}
	Soit une bijection \( \psi\colon B'\to B\). L'application
	\begin{equation}
		\begin{aligned}
			\varphi\colon A\setminus B & \to A\setminus B'                       \\
			x                          & \mapsto \begin{cases}
				                                     x       & \text{si } x\notin B' \\
				                                     \psi(x) & \text{si } x\in B'.
			                                     \end{cases}
		\end{aligned}
	\end{equation}
	est la bijection cherchée.
\end{proof}

\begin{lemma}[\cite{BIBooYBGLooUZuTrc}]       \label{LEMooIVCBooHWQiZB}
	Si \( A\) est un ensemble infini et si \( B\prec A\), alors \( A\approx A\setminus B\).
\end{lemma}

\begin{proof}
	Nous pouvons écrire
	\begin{equation}
		A=(A\setminus B)\cup B.
	\end{equation}
	Le théorème de comparabilité cardinale \ref{THOooCBSKooCmzfUf} nous indique que soit \( A\setminus B\preceq B\), soit \( A\setminus B\succeq B\). Nous allons étudier les deux cas.
	\begin{subproof}
		\spitem[Si \( A\setminus B\succeq B\)]
		Dans ce cas, \( (A\setminus B)\cup B\approx A\setminus B\) par le lemme \ref{LEMooXMVDooIWLWis}. Alors, notre résultat est prouvé parce que \( A=(A\setminus B)\cup B\approx A\setminus B\).
		\spitem[Si \( A\setminus B\preceq B\)]
		Dans ce cas, le lemme \ref{LEMooXMVDooIWLWis} nous indique que \( A=(A\setminus B)\cup B\approx B\). Mais \( A\approx B\) est exclu par l'hypothèse. Ce cas est donc impossible.
	\end{subproof}
\end{proof}

\begin{lemma}[\cite{MonCerveau}]        \label{LEMooMRVQooUZSSyL}
	Si \( A\) est infini et si \( B\) est une partie strictement subpotente de \( A\), alors il existe \( U\subset A\) disjoint de \( B\) et équipotent à \( B\).
\end{lemma}

\begin{proof}
	Le lemme \ref{LEMooIVCBooHWQiZB} nous donne une bijection \( \varphi\colon A\to A\setminus B\). Il suffit alors de poser \( U=\varphi(B)\). Cette partie est disjointe de \( B\) parce que \( \varphi\) prend ses valeurs dans \( A\setminus B\).
\end{proof}

\begin{lemma}[\cite{BIBooDLDFooFwXSGV}]
	Soit un ensemble infini \( A\) ainsi qu'un sous-ensemble \( B\subset A\). Nous supposons l'existence d'une fonction surjective \( f\colon B\to B\times B\).

	Alors \( B\preceq B\times B\preceq B\preceq A\).
\end{lemma}

\begin{proof}
	La première est l'hypothèse sur \( f\). La seconde est l'existence (évidente) d'une surjection \( B\times B\to B\). La troisième est le fait que \( B\) soit inclus dans \( A\).
\end{proof}

\begin{lemma}[\cite{BIBooDLDFooFwXSGV}]     \label{LEMooPOEFooXaifhT}
	Soit un ensemble infini \( A\) ainsi qu'un sous-ensemble strictement subpotent \( B\subset A\). Nous supposons l'existence d'une fonction surjective \( f\colon B\to B\times B\).

	Alors \( f\) peut être étendue en une injection \( f\colon D\to D\times D\) où \( D\subset A\) contient strictement \( B\).
\end{lemma}

\begin{proof}
	Par le lemme \ref{LEMooMRVQooUZSSyL}, nous considérons une partie \( U\subset A\) disjointe de \( B\) et équipotente à \( B\). Nous pouvons écrire le développement
	\begin{equation}
		(B\cup U)\times (B\cup U)=(B\times B)\cup(B\times U)\cup (U\times B)\cup (U\times U).
	\end{equation}
	Nous savons que \( B\) est surpotent à \( U\) (il est même équipotent); donc le lemme \ref{LEMooXMVDooIWLWis} nous dit que \( B\cup U\approx B\). De plus il existe une bijection \( B\to U\), donc
	\begin{equation}
		U\approx B\approx B\times B\approx B\times U\approx U\times B\approx U\times U.
	\end{equation}
	Chacun des ensembles  \( U\times B\),  \( B\times U\) et \( U\times U\) est équipotent à \( U\). Leur union est donc équipotente\footnote{Lemme \ref{LEMooXMVDooIWLWis}.} à \( U\) et nous avons une bijection
	\begin{equation}
		\varphi\colon U\to U\times B\cup (B\times U)\cup (U\times U).
	\end{equation}
	Et enfin nous définissons
	\begin{equation}
		\begin{aligned}
			g\colon B\cup U & \to (B\times U)\times (B\cup U)         \\
			x               & \mapsto \begin{cases}
				                          f(x)       & \text{si }  x\in B \\
				                          \varphi(x) & \text{si } x\in U.
			                          \end{cases}
		\end{aligned}
	\end{equation}
	Cette définition est bonne parce que \( U\) et \( B\) sont disjoints, et \( g\) est injective.
\end{proof}

\begin{normaltext}	\label{NORMooBijectionAA}
	Le théorème suivant est une généralisation de la proposition \ref{PropVCSooMzmIX}. Elle implique, entre autres choses, qu'il existe une bijection entre \( \eR\) et \( \eR\times \eR\). Pour le cas de \( \eN\times \eN\approx \eN\), il y a la proposition \ref{PROPooLPKUooAlsYJg} qui donne une bijection explicite et donc sans axiome du choix et sans lemme de Zorn.
\end{normaltext}

\begin{theorem}     \label{THOooDGOVooRdURVi}
	Si \( A\) est infini, alors \( A\approx A\times A\).
\end{theorem}

\begin{proof}
	Une fois de plus, ce sera le lemme de Zorn qui va s'y coller. Soit l'ensemble
	\begin{equation}
		\mA=\Big\{  (X,\varphi)  \tq
		\begin{cases}
			X\subset A \\
			\varphi\colon X\to X\times X\text{ est surjective.}
		\end{cases}
		\Big\}
	\end{equation}
	Cet ensemble est non vide parce que \( A\) est infini; il contient donc une partie dénombrable \( N\), et nous connaissons la surjection \( \eN\to \eN\times \eN\) du lemme \ref{PROPooLPKUooAlsYJg}.

	Nous ordonnons \( \mA\) par l'inclusion : \( (X,\varphi)\leq (Y,\phi)\) lorsque \( X\subset Y\) et \( \phi|_X=\varphi\). La tambouille usuelle montre que \( \mA\) est un ensemble inductif et le lemme de Zorn \ref{LemUEGjJBc} donne l'existence d'un élément maximal que nous notons \( (B,\varphi)\).

	Puisque \( B\) est subpotent à \( A\) (parce qu'il est inclus), soit il est strictement subpotent, soit il est équipotent. Nous commençons par montrer que \( B\) ne peut pas être strictement subpotent à \( A\).

	En effet, si nous avions une surjection \( B\to B\times B\), alors que \( B\) est strictement subpotent à \( A\). Le lemme \ref{LEMooPOEFooXaifhT} nous dit alors que \( \varphi\) peut être étendue, ce qui contredirait la maximalité de \( (B,\varphi)\).

	Donc la partie \( B\) est équipotente à \( A\) : il existe une bijection \( g\colon A\to B\). Mais nous avons une surjection \( B\to B\times B\) et donc aussi une injection \( B\times B\to B\). Vu que nous avons par ailleurs une injection \( B\to B\times B\), le théorème de Cantor-Schröder-Bernstein \ref{THOooRYZJooQcjlcl} nous donne une bijection \( \phi\colon B\times B\to B\). Avec ça, l'application
	\begin{equation}
		\begin{aligned}
			f\colon A\times A & \to A                             \\
			(a,b)             & \mapsto \phi\big( g(a),g(b) \big)
		\end{aligned}
	\end{equation}
	est une bijection. Donc les ensembles \( A\) et \( A\times A\) sont équipotents.
\end{proof}

\begin{lemma}       \label{LEMooNKKDooUvSYPO}
	Si \( A\) est infini, et si pour tout \( i\in \eN\) nous avons \( A_i\approx A\), alors
	\begin{equation}
		\bigcup_{i\in \eN}A_i\approx A.
	\end{equation}
\end{lemma}

\begin{proof}
	Pour chaque \( i\in \eN\) nous avons une bijection \( \varphi_i\colon A_i\to A\). Nous posons
	\begin{equation}        \label{EQooCHJAooRpHypV}
		\begin{aligned}
			\varphi\colon A\times \eN & \to \bigcup_{i=0}^{\infty}A_i \\
			(a,i)                     & \mapsto \varphi_i(a).
		\end{aligned}
	\end{equation}
	Cette application est surjective mais peut-être pas injective parce que les \( A_i\) peuvent avoir des intersections non vides. Nous avons alors le calcul
	\begin{subequations}
		\begin{align}
			A & \approx A\times \eN        \label{SUBEQooICFEooTLuFHZ}             \\
			  & \succeq \bigcup_{i=0}^{\infty}A_i      \label{SUBEQooRVPRooJJevkv} \\
			  & \succeq A      \label{SUBEQooFJRGooJnervy}
		\end{align}
	\end{subequations}
	Justifications :
	\begin{itemize}
		\item Pour \eqref{SUBEQooICFEooTLuFHZ}, c'est la proposition \ref{PROPooFKBEooKXqujV}.
		\item Pour \eqref{SUBEQooRVPRooJJevkv}, c'est la surjection \eqref{EQooCHJAooRpHypV}.
		\item Pour \eqref{SUBEQooFJRGooJnervy}, c'est le fait que seulement \( A_0\) possède déjà une surjection vers \( A\).
	\end{itemize}
	Donc \( \bigcup_iA_i\) est à la fois surpotent et subpotent à \( A\). Il est donc équipotent par le théorème \ref{THOooRYZJooQcjlcl}.
\end{proof}
