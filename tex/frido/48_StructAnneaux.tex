% This is part of Mes notes de mathématique
% Copyright (c) 2011-2022
%   Laurent Claessens
% See the file fdl-1.3.txt for copying conditions.

Attention aux conventions. Dans le Frido, un corps peut être réduit à \( \{ 0 \}\) et un idéal premier ne peut pas être \( \{ 0 \}\). Ces conventions ont une série de conséquences un peu partout, par exemple dans la proposition \ref{PROPooRUQKooIfbnQX} où nous parlons d'idéal maximum propre. Comparez par exemple avec \cite{ooWEUDooQybvIx}. Soyez attentifs.

En cas de doutes, nous suivons les conventions de Wikipédia.

%+++++++++++++++++++++++++++++++++++++++++++++++++++++++++++++++++++++++++++++++++++++++++++++++++++++++++++++++++++++++++++
\section{Inversibles et nilpotents}
%+++++++++++++++++++++++++++++++++++++++++++++++++++++++++++++++++++++++++++++++++++++++++++++++++++++++++++++++++++++++++++

Le concept d'anneau est la définition \ref{DefHXJUooKoovob}.

\begin{lemma}       \label{LEMooOYZEooLivKWI}
	Si \( a\) et \( b\) commutent, nous avons, pour tout \( r \in \eN \) et \( r > 0\), la formule
	\begin{equation}        \label{Eqarpurmkbk}
		a^{r+1}-b^{r+1}=(a-b)\left(\sum_{k=0}^ra^{r-k}b^k\right).
	\end{equation}
\end{lemma}

\begin{proof}
	Démontrons cela par récurrence. Le cas \( r = 0 \) est évident. Pour
	un \( r \) donné, si \eqref{Eqarpurmkbk} est vraie, alors
	\begin{align*}
		a^{r+2}-b^{r+2} & = a^{r+1}a - a^{r+1}b +a^{r+1}b - b^{r+1}b                            \\
		                & = a^{r+1}(a - b) + (a^{r+1} - b^{r+1})b                               \\
		                & = a^{r+1}(a - b) + (a-b)\left(\sum_{k=0}^ra^{r-k}b^k\right)b          \\
		                & = (a - b) \left(a^{r+1} + \left(\sum_{k=0}^ra^{r-k}b^k\right)b\right) \\
		                & = (a - b) \left(a^{r+1} + \sum_{k=0}^ra^{r-k}b^{k + 1}\right)         \\
		                & = (a - b) \left(a^{r+1} + \sum_{k'=1}^{r+1}a^{(r+1)-k'}b^{k'}\right)  \\
		                & = (a - b) \left(\sum_{k'=0}^{r+1}a^{(r+1)-k'}b^{k'}\right).
	\end{align*}
\end{proof}

\begin{proposition}
	Si \( a\) est un élément nilpotent de l'anneau \( A\), alors \( 1-a\) est inversible. Si \( a\) est nilpotent non nul, alors il est diviseur de zéro.
\end{proposition}

\begin{proof}
	Soit \( n\) le minimum tel que \( a^n=0\). En vertu de la formule \eqref{Eqarpurmkbk} nous avons
	\begin{equation}
		1=1-a^n=(1-a)(1+a+\cdots+a^{n-1})=(1+a+\cdots+a^{n-1})(1-a).
	\end{equation}
	La somme \( 1+a+\cdots+a^{n-1}\) est donc un inverse de \( (1-a)\).
\end{proof}

%+++++++++++++++++++++++++++++++++++++++++++++++++++++++++++++++++++++++++++++++++++++++++++++++++++++++++++++++++++++++++++
\section{PGCD, PPCM et éléments inversibles}
%+++++++++++++++++++++++++++++++++++++++++++++++++++++++++++++++++++++++++++++++++++++++++++++++++++++++++++++++++++++++++++

La définition de pgcd et ppcm dans un anneau commutatif est la définition \ref{DefrYwbct}. Dans la plus grande tradition, elle a été introduite sans motivation, et utilisée par-ci par-là. Nous revenons maintenant dessus.

Commençons par donner une autre vision de la divisibilité dans les anneaux intègres.
\begin{proposition}\label{PropDiviseurIdeaux}
	Dans un anneau intègre\footnote{Définition \ref{DEFooTAOPooWDPYmd}.} \( A\), on a l'équivalence suivante concernant deux éléments \( a, b \in A \):
	\begin{equation}
		a\divides b\Leftrightarrow (b)\subset (a).
	\end{equation}
\end{proposition}

Donc la divisibilité devient en réalité une relation d'ordre dont nous pouvons chercher un maximum et un minimum. Si \( S\) est une partie de \( A\), nous notons \( a\divides S\) pour exprimer que \( a\divides x\) pour tout \( x\in S\); de la même façon, \( S\divides b\) signifie que \( x\divides b\) pour tout \( x\in S\).

Nous rappelons également la définition~\ref{DEFooSPHPooCwjzuz} de morphisme d'anneaux. Remarquons que si \( f\) est un morphisme, nous avons \( f(0)=0\) et \( f(x)^{-1}=f(x^{-1})\).

\begin{lemma}[\cite{ooLIOMooBuCPUS}]
	Les éléments inversibles d'un anneau sont diviseurs de tous les éléments.
\end{lemma}

\begin{proof}
	Soit \( k\) inversible d'inverse \( k'\) : \( kk'=1\); soit aussi \( a\in A\). Alors \( a=k(k'a)\), ce qui montre que \( k\) divise \( a\).
\end{proof}

\begin{lemma}[\cite{ooLIOMooBuCPUS}]
	Dans un anneau, \( 1\) est un pgcd de \( a\) et \( b\) si et seulement si les seuls diviseurs communs sont les inversibles.
\end{lemma}

\begin{proof}
	Supposons pour commencer que \( 1\) est un pgcd de \( a\) et \( b\). Un diviseur commun de \( a\) et \( b\) doit donc diviser \( 1\). Or un diviseur de \( 1\) est forcément inversible.

	Dans l'autre sens, les diviseurs communs de \( a\) et \( b\) sont tous inversibles et donc diviseurs de \( 1\). Donc \( 1\) est un pgcd de \( a\) et \( b\).
\end{proof}

%---------------------------------------------------------------------------------------------------------------------------
\subsection{Calcul effectif du PGCD et théorème de Bézout}
%---------------------------------------------------------------------------------------------------------------------------
\label{subSecIpmnhO}

Soient \( a\) et \( b\), deux entiers que nous allons prendre positifs. Nous allons voir maintenant l'algorithme de \defe{Euclide étendu}{Euclide!algorithme étendu} qui est capable, pour \( a\) et \( b\) donnés, de calculer le PGCD de \( a\) et \( b\), et un couple de Bézout \( (u,v)\) tel que \( ua+vb=\pgcd(a,b)\). Ce calcul est indispensable si on veut implémenter RSA (\ref{SecEVaFYi}).

Cela se base sur le lemme suivant.

\begin{lemma}       \label{LemiVqita}
	Soient \( a,b\in \eN\) et des nombres \( q\) et \( r\) tels que \( a=qb+r\). Alors \( \pgcd(a,b)=\pgcd(r,b)\).
\end{lemma}

\begin{proof}
	Il suffit de voir que les diviseurs communs de \( a\) et \( b\) sont diviseurs de \( r\) et que les diviseurs communs de \( r\) et \( b\) divisent \( a\).

	Si \( s\) divise \( a\) et \( b\), alors dans l'équation
	\begin{equation*}
		\frac{ a }{ s }=\frac{ qb }{ s }+\frac{ r }{ s }
	\end{equation*}
	les termes \( a/s\) et \( qb/s\) sont entiers, donc \( r/s\) est aussi entier, et \( s\) divise \( r\).

	Inversement, si \( s\) divise \( r\) et \( b\), alors il divise \( qb+r\) et donc \( a\).
\end{proof}
\begin{remark}
	Ce lemme est surtout intéressant lorsque \( a=qb+r\) est la division euclidienne de \( a\) par \( b\): en effet, dans ce cas \( r < b \), et le calcul du PGCD de deux nombres (\(a \) et \( b\)) est ramené à un calcul de PGCD de deux nombres plus petits (\( b\) et \( r\)).

	L'algorithme pour calculer \( \pgcd(a,b)\) consiste à écrire des divisions euclidiennes successives de la manière suivante:
	\begin{subequations}
		\begin{align}
			a & = q_2 b   + r_2 &  & r_2<b   \\
			b & = q_3 r_2 + r_3 &  & r_3<r_2 \\
			  & \vdots
		\end{align}
	\end{subequations}
	en remarquant que \( \pgcd(a,b)=\pgcd(b,r_2)=\pgcd(r_2,r_3) \). Étant donné que les inégalités \( r_2<b\) et \( r_3<r_2\) sont strictes, en continuant ainsi nous finissons sur zéro, c'est-à-dire qu'il existera un \( n\) pour lequel \( r_{n+1} = 0 \); et donc
	\begin{equation*}
		r_{n-1}=q_{n+1}r_n,
	\end{equation*}
	et à ce moment nous avons \( \pgcd(a,b)=\pgcd(r_{n-1},r_n)=r_n\).
\end{remark}

%///////////////////////////////////////////////////////////////////////////////////////////////////////////////////////////
\subsubsection{Algorithme d'Euclide pour le PGCD}
%///////////////////////////////////////////////////////////////////////////////////////////////////////////////////////////
\label{SUBSECooAEBLooFGJRkg}
\index{pgcd!calcul effectif}

Écrivons l'algorithme\cite{BezoutCos} en détail (parce que Bézout, ça va être la même chose en cinq fois plus compliqué). On pose
\begin{subequations}
	\begin{align}
		r_0=a \\
		r_1=b
	\end{align}
\end{subequations}
(ce qui explique que nous n'ayons pas utilisé \( r_0\) et \( r_1\) précédemment). Ensuite on écrit la division euclidienne \( a=q_2b+r_2\), c'est-à-dire \( r_0=q_2r_1+r_2\). Cela donne \( r_2\) et \( q_2\) en termes de \( r_0\) et \( r_1\) :
\begin{equation}
	r_2=r_0-q_2r_1.
\end{equation}
Ensuite, sachant \( r_2\) nous pouvons continuer :
\begin{equation}
	r_1=q_3r_2+r_3
\end{equation}
donne \( q_3\) et \( r_3=r_1-q_3r_2\). On continue avec \( r_2=q_4r_3+r_4\). Tout cela pour poser la suite
\begin{equation}        \label{SUBEQooDYUEooYNQnII}
	\begin{aligned}[]
		r_0 & =a                      \\
		r_1 & =b                      \\
		r_k & =q_{k+2}r_{k+1}+r_{k+2}
	\end{aligned}
\end{equation}
où la troisième équation définit \( r_{k+2}\) et \( q_{k+2}\) en fonction de \( r_k\) et \( r_{k+1}\), à l'aide du théorème de la division euclidienne. La suite \( (r_k)\) ainsi construite est strictement décroissante et à chaque étape le lemme~\ref{LemiVqita} et le principe de l'algorithme d'Euclide nous donnent
\begin{subequations}
	\begin{numcases}{}
		\pgcd(r_k,r_{k+1})=\pgcd(r_{k+1},r_{k+2})=\pgcd(a,b)\\
		0\leq r_{k+1}<r_k.
	\end{numcases}
\end{subequations}
La suite étant strictement décroissante, nous prenons \( r_n\), le dernier non nul : \( r_{n+1}=0\). Dans ce cas, en prenant \( k=n-1\) dans la la dernière équation \eqref{SUBEQooDYUEooYNQnII} devient :
\begin{equation}
	r_{n-1}=q_{n+1}r_n
\end{equation}
avec \( \pgcd(a, b)=\pgcd(r_n,r_{n-1})=r_n\).

\begin{example}
	Calculons le PGCD de \( 18\) et \( 231\). Pour cela nous écrivons les divisions euclidiennes en chaine :
	\begin{subequations}
		\begin{align}
			231 & = 18 \cdot 12 + 15 \\
			18  & =  1 \cdot 15 + 3  \\
			15  & =  5 \cdot 5  + 0.
		\end{align}
	\end{subequations}
	Donc le PGCD est \( 3\) (le dernier reste non nul).
\end{example}

%///////////////////////////////////////////////////////////////////////////////////////////////////////////////////////////
\subsubsection{Algorithme étendu: calcul effectif des coefficients de Bézout}
%///////////////////////////////////////////////////////////////////////////////////////////////////////////////////////////
\label{SUBSECooRHSQooEuBWbd}
\index{Bézout!calcul effectif}

La difficulté est de construire la suite qui donne des coefficients de Bézout. Elle va être construite à l'envers. Nous supposons déjà connaitre la liste complète des \( r_k\) jusqu'à \( r_n=\pgcd(a,b)\), ainsi que la liste complète des divisions euclidiennes
\begin{equation}
	r_k=q_{k+2}r_{k+1}+r_{k+2}.
\end{equation}

Nous voulons trouver les couples \( (u_k,v_k)\) de telle façon à avoir à chaque étape
\begin{equation}
	r_n=u_kr_k+v_kr_{k-1}.
\end{equation}
Notons que c'est à chaque fois \( r_n\) que nous construisons. La première équation de type Bézout à résoudre est
\begin{equation}
	r_n=u_nr_n+v_nr_{n-1},
\end{equation}
sachant que \( r_{n-1}=q_{n+1}r_n\). On pose \( v_n=0\) et \( u_n=1\) et c'est bon. Pour la récurrence, supposons les coefficients \( u_k\) et \( v_k\) connus, et déterminons les coefficients \( u_{k-1} \) et \( v_{k-1} \). Pour ce faire, nous égalons les deux expressions pour \( r_n\) :
\begin{equation}
	r_n=u_kr_k+v_kr_{k-1}=u_{k-1}r_{k-1}+v_{k-1}r_{k-2};
\end{equation}
dans laquelle nous substituons \( r_{k-2}=q_k r_{k-1}+r_k\):
\begin{align}
	u_kr_k+v_kr_{k-1} & = u_{k-1}r_{k-1}+v_{k-1}(q_k r_{k-1}+r_k)      \\
	                  & = (u_{k-1} + q_k v_{k-1}) r_{k-1} +v_{k-1} r_k
\end{align}
et nous égalons les coefficients de \( r_k\) et \( r_{k-1}\) pour obtenir
\begin{subequations}
	\begin{numcases}{}
		v_{k-1}=u_k\\
		u_{k-1}=v_k-v_{k-1}q_k.
	\end{numcases}
\end{subequations}
Dès que \( u_k\) et \( v_k\) ainsi que \( q_k\) sont connus, on peut calculer \( u_{k-1}\) et \( v_{k-1}\).

La dernière équation, celle avec \( k=1\), est
\begin{equation}
	r_n=u_1r_1+v_1r_0,
\end{equation}
c'est-à-dire
\begin{equation}        \label{EqNDMLooDvaiAc}
	\pgcd(a,b)=u_1b+v_1a.
\end{equation}
Nous avons ainsi trouvé des coefficients de Bézout pour \( a\) et \( b\).

%---------------------------------------------------------------------------------------------------------------------------
\subsection{Générateurs}
%---------------------------------------------------------------------------------------------------------------------------

\begin{normaltext}
	Les éléments de \( \eZ/n\eZ\) ne sont pas des éléments de \( \eZ\); ce sont des parties de \( \eZ\). Pour rappel :
	\begin{equation}
		[a]_n=\{ a+kn \tq k\in \eZ \}.
	\end{equation}

	Pour écrire les éléments de \( \eZ/n\eZ\), nous pouvons écrire
	\begin{equation}
		\{ [i]_n \}_{i=1,\ldots, n}
	\end{equation}
	ou
	\begin{equation}
		\{ [i]_n\tq 1\leq i\leq n \}.
	\end{equation}

	Mais attention : l'ensemble
	\begin{equation}
		\bigcup_{i=1}^n[i]_n
	\end{equation}
	est très différent. Ce dernier ensemble est \( \eZ\).
\end{normaltext}

\begin{proposition}[\cite{BIBooFOGTooQgFAbQ}]   \label{PROPooEWREooUOSMsE}
	Soit \( n\geq 2\).
	\begin{enumerate}
		\item
		      L'anneau \( \eZ/n\eZ\) contient \( n\) éléments
		\item
		      Nous avons \( \eZ/n\eZ=\{ [k]_n \}_{k=0,\ldots, n-1}\).
	\end{enumerate}
\end{proposition}

\begin{proposition}[\cite{BIBooFOGTooQgFAbQ}]       \label{PROPooSKSYooZoDhIP}
	Soient \( n\geq 2\), \( a\) premier avec \( n\), \( b\in \eN\), et \( m_1,\ldots, m_n\in \eN\). Alors
	\begin{enumerate}
		\item
		      Nous avons \( \eZ/n\eZ=\{ [am_i+b]_n \}_{i=1,\ldots, n}\),
		\item
		      Nous avons \( \eZ/n\eZ=\{ [b+k]_n \}_{k=0,\ldots, n-1}\)
		\item
		      Nous avons \( \eZ/n\eZ=\{ [ka]_{k=0,\ldots, n-1} \}\).
	\end{enumerate}
\end{proposition}

\begin{proposition}[\cite{BIBooFOGTooQgFAbQ}]       \label{PROPooMTWGooEMbvDi}
	Soit \( n\geq 2\) et \( m\in \eZ\). Nous avons équivalence entre
	\begin{enumerate}
		\item
		      \( \pgcd(n,m)=1\),
		\item
		      \( [m]_n\) engendre le groupe \( (\eZ/n\eZ,+)\)
		\item
		      \( [m]_n\) est inversible dans \( \big( (\eZ/n\eZ)^*,\cdot \big)\).
	\end{enumerate}
\end{proposition}

%TODOooFBFJooTWsuLV.  Vérifier que les preuves de PROPooEWREooUOSMsE, PROPooSKSYooZoDhIP et PROPooMTWGooEMbvDi sont faites.

%---------------------------------------------------------------------------------------------------------------------------
\subsection{Générateurs pour le groupe multiplicatif}
%---------------------------------------------------------------------------------------------------------------------------

\begin{proposition}[\cite{BIBooYOLUooOAUbYH}]       \label{PROPooKSCRooPyInSv}
	Le groupe \( \big( (\eZ/n\eZ)^*, \cdot\big)\) est cyclique.
\end{proposition}

La proposition suivante est un pas important dans l'algorithme de Shor permettant aux ordinateurs quantiques de factoriser rapidement des grands nombres.

\begin{proposition}[\cite{BIBooBBRQooJxksHX}]       \label{PROPooZCKXooOtocKE}
	Soient \( A,B\in \eN\) tels que \( \pgcd(A,B)=1\). Alors il existe \( p,m\in \eN\) tels que \( A^p=mB+1\).
\end{proposition}

%---------------------------------------------------------------------------------------------------------------------------
\subsection{Décomposition en facteurs premiers}
%---------------------------------------------------------------------------------------------------------------------------

\begin{lemma}[Lemme de Gauss]    \label{LemPRuUrsD}
	Soient \( a,b,c\in \eZ\) tels que \( a\) divise \( bc\). Si \( a\) est premier avec \( c\), alors \( a\) divise \( b\).
\end{lemma}
\index{lemme!de Gauss!pour des entiers}

\begin{proof}
	Puisque \( a\) est premier avec \( c\), nous avons \( \pgcd(a,c)=1\) et le théorème de Bézout~\ref{ThoBuNjam} nous donne donc \( s,t\in \eZ\) tels que \( sa+tc=1\). En multipliant par \( b\), nous avons \( sab+tbc=b\). Mais les deux termes du membre de gauche sont multiples de \( a\) parce que \( a\) divise \( bc\). Par conséquent \( b\) est somme de deux multiples de \( a\) et donc est multiple de \( a\).
\end{proof}

\begin{lemma}[Lemme d'Euclide\cite{BTDWooZCyXfb}]       \label{LemAXINooOeuMJZ}
	Soient \( a,b\in \eZ\). Si le nombre premier \( p\) divise le produit \( ab\), alors \( p\) divise \( a\) ou \( b\).
\end{lemma}
\index{Euclide!lemme}

\begin{proof}
	Comme \( p\) est premier, si il ne divise pas \( a\) c'est que \( \pgcd(a,p)=1\). Dans ce cas le lemme de Gauss~\ref{LemPRuUrsD} implique que \( p\) divise \( b\).
\end{proof}
\index{lemme!d'Euclide}

Le théorème fondamental de l'arithmétique permet de décomposer des nombres en facteurs premiers.

\begin{theorem}[Décomposition en facteurs premiers\cite{RATEooJuqgom}]        \label{ThoAJFJooAveRvY}
	Tout entier strictement positif peut être écrit comme un produit de nombres premiers d'une unique façon, à l'ordre près des facteurs.

	En d'autres termes, pour tout entier \( n>1\), il existe une unique suite finie unique \( (p_1, k_1)\),\ldots \( (p_r, k_r)\) telle que :
	\begin{enumerate}
		\item
		      les \( p_i\) sont des nombres premiers tels que, si \( i < j\), alors \( p_i < p_j\) ;
		\item
		      les \( k_i\) sont des entiers naturels non nuls ;
		\item
		      \( n=\prod_{i=1}^rp_i^{k_i}\).
	\end{enumerate}
\end{theorem}

\begin{proof}
    Soit \( n\) un entier positif. Nous prouvons l'existence d'une décomposition en facteurs premiers par récurrence. Le nombre \( n=1\) est le produit d'une famille finie de nombres premiers : la famille vide\footnote{Voir \ref{NORMooDBOFooQCwbOY}.}.

	Supposons que tout entier strictement inférieur à un certain entier \( n>1\) est produit de nombres premiers. Deux possibilités apparaissent pour \( n\) : il est premier ou non. Si \( n\) est premier, et donc produit d'un unique entier premier, à savoir lui-même, le résultat est vrai. Si \( n\) n'est pas premier, il se décompose sous la forme \( kl\) avec \( k\) et \( l\) strictement inférieurs à \( n\). Dans ce cas, l'hypothèse de récurrence implique que les entiers \( k\) et \( l\) peuvent s'écrire comme produits de nombres premiers. Leur produit aussi, ce qui fournit une décomposition de \( n\) en produit de nombres premiers. Par application du principe de récurrence, tous les entiers naturels peuvent s'écrire comme produit de nombres premiers.

	Nous prouvons maintenant l'unicité. Prenons deux produits de nombres premiers qui sont égaux. Prenons n'importe quel nombre premier \( p\) du premier produit. Il divise le premier produit, et, de là, aussi le second. Par le lemme d'Euclide~\ref{LemAXINooOeuMJZ}, \( p\) doit alors diviser au moins un facteur dans le second produit. Mais les facteurs sont tous des nombres premiers eux-mêmes, donc \( p\) doit être égal à un des facteurs du second produit. Nous pouvons donc simplifier par \( p\) les deux produits. En continuant de cette manière, nous voyons que les facteurs premiers des deux produits coïncident précisément.
\end{proof}

\begin{lemma}[\cite{MonCerveau}]        \label{LEMooDTQQooYoJABt}
	Nous notons \( \mP\) l'ensemble des nombres premiers dans \( \eN\). Soient des suites finies \( (a_p)_{p\in \mP}\) et \( (b_p)_{p\in \mP}\). Nous posons
	\begin{equation}
		\begin{aligned}[]
			a & =\prod_{ p\in\mP}p^{a_p} & \text{ et } &  & b=\prod_{ p\in \mP}p^{b_p}.
		\end{aligned}
	\end{equation}
	Alors \( a\divides b\) si et seulement si \( a_p\leq b_p\) pour tout \( p\).
\end{lemma}

\begin{proof}
	Dire que \( a\divides b\) signifie qu'il existe \( s\in \eN\) tel que \( as=b\); le théorème \ref{ThoAJFJooAveRvY} nous permet de décomposer \( s\) en \( s=\prod_{p\in\mP}p^{s_p}\). Puisque le produit dans \( \eN\) est commutatif et associatif,
	\begin{equation}
		b=as=\prod_{p\in\mP}p^{s_p+a_p}.
	\end{equation}
	Par unicité de la décomposition de \( b\) (toujours le théorème \ref{ThoAJFJooAveRvY}), nous en déduisons que \( b_p=s_p+a_p\geq a_p\).

	Dans l'autre sens, l'hypothèse \( a_p\leq b_p\) implique l'existence de \( s_p\geq 0\) tels que \( b_p=a_p+s_p\). En posant \( s=\prod_{p\in\mP}p^{s_p}\), nous avons
	\begin{equation}
		as=\prod_{p\in\mP}p^{s_p+a_p}=\prod_{p\in \mP}p^{b_p}=b.
	\end{equation}
	Donc \( a\divides b\).
\end{proof}

\begin{proposition}     \label{PROPooBKQNooFglPGI}
	Soient \( x\in \eZ\) ainsi que \( p,q\in \eN\). Si \( q\) divise \( x^p\), alors \( q\) divise \( x\).
\end{proposition}

\begin{lemma}       \label{LEMooGLZHooUcRNgu}
	Soient un nombre premier \( q\) ainsi que \( a\in \eZ\). Soit un entier \( n\geq 1\). Le nombre \( q\) divise \( a\) si et seulement si il divise \( a^n\).
\end{lemma}

\begin{proof}
	Nous numérotons les nombres premiers \( p_i\) pour que \( p_1\) soit \( q\). La décomposition en nombre premiers du théorème \ref{ThoAJFJooAveRvY} nous dit que
	\begin{equation}
		a=q^{a_1}\prod_{i\neq 1}p_i^{a_i}
	\end{equation}
	et
	\begin{equation}
		a^n=q^{na_1}\prod_{i\neq 1}p_i^{na_i}
	\end{equation}
	Nous avons équivalence entre les énoncés suivants :
	\begin{itemize}
		\item \( q\) divise \( a\)
		\item \( a_1\neq 0\)
		\item \( na_1\neq 0\) (parce que \( n\neq 0\))
		\item \( q\) divise \( a^n\).
	\end{itemize}
\end{proof}

\begin{lemma}[\cite{MonCerveau}]       \label{LEMooBJVJooFyuFeN}
	Dans \( \eN\), le pgcd\footnote{Le pgcd et ppcm sont définis dans \ref{DefrYwbct}.} et le ppcm sont uniques.
\end{lemma}

\begin{proof}
	Supposons que \( \delta_1\) et \( \delta_2\) soient des pgcd de la partie \( S\). Puisque \( \delta_1\divides S\), nous avons \( \delta_1\divides \delta_2\) parce que \( \delta_2\) est un pgcd. Le même raisonnement, inversant \( \delta_1\) et \( \delta_2\) montre que \( \delta_2\divides \delta_1\). Si \( (a_p)\) sont les éléments de la décomposition de \( \delta_1\) et \( (b_p)\) ceux de \( \delta_2\), alors le lemme \ref{LEMooDTQQooYoJABt} nous indique que \( a_p\leq b_p\) et \( b_p\leq a_p\), ce qui implique que \( a_p=b_p\).

	La démonstration pour le ppcm s'effectue selon le même principe.
\end{proof}

\begin{lemma}       \label{LEMooJIGRooARiIPC}
	Soit une partie \( S\) de \( \eN\).
	\begin{enumerate}
		\item
		      Le pgcd de \( S\) est le plus grand élément de \( \eN\) divisant tous les éléments de \( S\).
		\item
		      Le ppcm de \( S\) est le plus petit élément de \( \eN\) que tous les éléments de \( S\) divisent.
	\end{enumerate}
\end{lemma}

\begin{lemma}[\cite{MonCerveau}]        \label{LEMooEVIZooPAkQZW}
	Soient \( a,b\in \eZ\) et \( k\in \eN\) tels que \( ab=q^k\) et \( \pgcd(a,b)=1\). Alors il existe \( \alpha,\beta\in\eZ\) tels que \( a=\alpha^k\) et \( b=\beta^k\).
\end{lemma}

\begin{proof}
	Nous décomposons \( a\), \( b\) et \( q\) en facteurs premiers suivant le théorème \ref{ThoAJFJooAveRvY} :
	\begin{subequations}
		\begin{align}
			a & = \prod_ip_i^{a_i}      \label{SUBEQooBJEQooDiWbYg} \\
			b & = \prod_ip_i^{b_i}                                  \\
			q & = \prod_{i}p_i^{q_i}.
		\end{align}
	\end{subequations}
	D'un part, en utilisant la commutativité et l'associativité du produit,
	\begin{equation}
		ab=\prod_ip_i^{a_i+b_i}.
	\end{equation}
	D'autre part, puisque \( ab=q^k\), nous avons
	\begin{equation}
		ab=(\prod_ip_i^{q_i})^k=\prod_ip_i^{kq_i}.
	\end{equation}
	En vertu de l'unicité de la décomposition en facteurs premiers, pour chaque \( i\) nous avons
	\begin{equation}
		a_i+b_i=kq_i.
	\end{equation}
	Comme \( a\) et \( b\) sont premiers entre eux, si \( a_i\neq0\) alors \( b_i=0\) et inversement. Prenons un \( i\) tel que \( a_i\neq 0\). Alors \( b_i=0\) et nous avons \( a_i=kq_i\). Idem pour les \( b_i\).

	Donc tous les \( a_i\) et les \( b_i\) qui sont non nuls sont des multiples de \( k\). Nous posons \( a_i=ks_i\) et nous reportons dans \eqref{SUBEQooBJEQooDiWbYg} :
	\begin{equation}
		a=\prod_ip_i^{ks_i}=(\prod_ip_i^{s_i})^k,
	\end{equation}
	de telle sorte que \( a\) soit une puissance \( k\)\ieme. La même chose tient pour \( b\).
\end{proof}

\begin{proposition}     \label{PROPooNQBOooHWqTvs}
	Soient \( a,b\in \eZ\setminus\{ 0 \}\) décomposés en \( a=\prod_{p\in\mP}p^{a_p}\) et \( b=\prod_{p\in\mP}p^{b_p}\). En posant
	\begin{subequations}
		\begin{align}
			m_p & = \min\{ a_p,b_p \}  \\
			M_p & = \max\{ a_p,b_p \},
		\end{align}
	\end{subequations}
	nous avons
	\begin{subequations}
		\begin{align}
			\pgcd(a,b) & = \prod_{p\in\mP}p^{m_p}  \\
			\ppcm(a,b) & = \prod_{p\in\mP}p^{M_p}.
		\end{align}
	\end{subequations}
	Pour rappel, la définition de pgcd et ppcm sont dans \ref{DefrYwbct}.
\end{proposition}

\begin{proof}
	Nous commençons par le pgcd. Nous notons \( \delta=\prod_{p\in\mP}p^{m_p}\) et nous prouvons que \( \delta\) est un pgcd de \( \{ a,b \}\). Il y a deux propriétés à vérifier.

	\begin{subproof}
		\spitem[\( \delta\) divise \( a\) et \( b\)]
		Puisque \( m_p=\min\{ a_p,b_p \}\), nous avons \( m_p\leq a_p\) et \( m_p\leq b_p\). Le lemme \ref{LEMooDTQQooYoJABt} nous dit alors que \( \delta\divides a\) et \( \delta\divides b\).
		\spitem[Si \( s\) divise \( a\) et \( b\)]
		De même, si \(s\divides a\) et \( s\divides b\), nous avons \( s_p\leq a_p\) et \( s_p\leq b_p\), ce qui montre que \( s_p\leq m_p\) et donc que \( s\divides \delta\).
	\end{subproof}

	Pour le ppcm, nous posons \( \mu=\prod_{p\in\mP}p^{M_p}\), et nous prouvons que \( \mu\) est un ppcm de \( \{ a,b \}\).
	\begin{subproof}
		\spitem[\( a\) et \( b\) divisent \( \mu\)]
		Pour tout \( p\), nous avons \( M_p\geq a_p\). Le lemme \ref{LEMooDTQQooYoJABt} implique que \( a\divides \mu\). Idem pour \( b\), donc tous les éléments de \( \{ a,b \}\) divisent \( \mu\).
		\spitem[Si \( a\) et \( b\) divisent \( r\)]
		Supposons que \( a\) et \( b\) divisent un certain nombre \( r\). Alors \( a_p\leq r_p\) et \( b_p\leq r_p\). Donc \( r_p\geq\max\{ a_p,b_p \}=M_p\). Nous en déduisons que \( \mu\divides r\).
	\end{subproof}

	Puisque les pgcd et ppcm sont uniques (lemme \ref{LEMooBJVJooFyuFeN}), nous avons prouvé que \( \delta\) et \( \mu\) sont les nombres recherchés.
\end{proof}

\begin{corollary}       \label{CORooQIMHooUzLUJY}
    Soit un nombre premier \( p\). Un élément \( m\in \eZ^*\) vérifie \( \pgcd(m,p^n)\neq 1\) si et seulement si \( m=qp\) pour un certain \( q\).
\end{corollary}

\begin{proof}
    Nous considérons les décompositions en facteurs premiers \( m=\prod_{s\in\mP}s^{a_s}\) et \( p^n=\prod_{s\in\mP}s^{b_s}\). Par la partie unicité du théorème \ref{ThoAJFJooAveRvY}, nous savons que \( b_s=0\) pour \( s\neq p\) et \( b_p=n\).

    Nous prenons ensuite l'expression du \( \pgcd\) donné par la proposition \ref{PROPooNQBOooHWqTvs} :
    \begin{equation}
        \pgcd(m,p^n)=\prod_{s\in\mP}s^{\min\{ a_s,b_s \}}.
    \end{equation}
    Le minimum est toujours zéro lorsque \( s\neq p\), donc \( \pgcd(m,p^n)=p^{\min\{ a_s,n \}}\).

    Nous avons donc \( \pgcd(m,p^n)\neq 1\) si et seulement si \( a_p\neq 0\). Mais \( a_p\neq 0\) si et seulement si \( m\) est un multiple de \( p\).
\end{proof}

\begin{lemma}   \label{LemheKdsa}
	Un entier \( n\geq 1\) se décompose de façon unique en produit de la forme \( n=qm^2\) où \( q\) est un entier sans facteurs carrés et \( m\), un entier.
\end{lemma}

\begin{proof}
	Pour \( n=1\), c'est évident. Nous supposons \( n\geq 2\).

	En ce qui concerne l'existence, nous décomposons \( n\) en facteurs premiers\footnote{Théorème~\ref{ThoAJFJooAveRvY}.} et nous séparons les puissances paires des puissances impaires :
	\begin{subequations}
		\begin{align}
			n & = \prod_{i=1}^r p_i^{2\alpha_i}\prod_{j=1}^sq_{j}^{2\beta_j+1}                                                              \\
			  & = \underbrace{\left( \prod_{i=1}^rp_i^{2\alpha_i}\prod_{j=1}^sq^{2\beta_j} \right)}_{m^2}\underbrace{\prod_{j=1}^sq_j}_{q}.
		\end{align}
	\end{subequations}

	Nous passons à l'unicité. Supposons que \( n=q_1m_1^2=q_2m_2^2\) avec \( q_1\) et \( q_2\) sans facteurs carrés (dans leur décomposition en facteurs premiers). Soit \( d=\pgcd(m_1,m_2)\) et \( k_1\), \( k_2\) définis par \( m_1=dk_1\), \( m_2=dk_2\). Par construction, \( \pgcd(k_1,k_2)=1\). Étant donné que
	\begin{equation}        \label{EqWPOtto}
		n=q_1d^2k_1^2=q_2d^2k_2^2,
	\end{equation}
	nous avons \( q_1k_1^2=q_2k_2^2\) et donc \( k_1^2\) divise \( q_2k_2^2\). Mais \( k_1\) et \( k_2\) n'ont pas de facteurs premiers en commun, donc \( k_1^2\) divise \( q_2\), ce qui n'est possible que si \( k_1=1\) (parce que \( k_1^2\) n'a que des facteurs premiers alors que \( q_2\) n'en a pas). Dans ce cas, \( d=m_1\) et \( m_1\) divise \( m_2\). Si \( m_2=lm_1\) alors l'équation \eqref{EqWPOtto} se réduit à  \( n=q_1m_1^2=q_2l^2m_1^2\) et donc
	\begin{equation}
		q_1=q_2l^2,
	\end{equation}
	ce qui signifie \( l=1\) et donc \( m_1=m_2\).
\end{proof}

Les nombres premiers ne sont pas si rares que ça dans \( \eN\). Nous allons voir dans \ref{ThonfVruT} que la somme des inverses des nombres premiers diverge. Pour comparaison, la somme des inverses des carrés converge par la proposition \ref{PROPooFPVZooGnsqrs}. Il y a donc, dans un certains sens, plus de nombres premiers que de carrés; dans un autre sens, il y en a autant : une infinité dénombrable.

%---------------------------------------------------------------------------------------------------------------------------
\subsection{Ordre d'un élément dans un groupe fini}
%---------------------------------------------------------------------------------------------------------------------------

Voir plus d'informations dans la partie \ref{SECooXIHPooWVSjhT} sur les groupes monogènes.

\begin{theorem}[Théorème de Cauchy\cite{ooTZHGooEPFstf}]    \label{THOooSUWKooICbzqM}
	Soit un groupe fini d'ordre\footnote{Définition \ref{DEFooKWBCooMlmpCP}.} \( n\). Pour tout diviseur premier \( p\) de \( n\), le groupe \( G\) possède au moins un élément d'ordre \( p\).
\end{theorem}
\index{théorème!Cauchy!groupe}

Le lemme suivant indique que sous hypothèse de commutativité, l'ordre d'un élément est une notion multiplicative.
\begin{lemma}[\cite{rqrNyg}]    \label{LemyETtdy}
	Soit \( G\) un groupe et \( a,b\in G\) tels que \( ab=ba\) d'ordres respectivement \( r\) et \( s\), deux nombres premiers entre eux. Alors l'élément \( ab\) est d'ordre \( rs\).
\end{lemma}

\begin{proof}
	Étant donné que \( (ab)^{rs}=a^{rs}b^{rs}=1\), l'ordre de \( ab\) divise \( rs\). Et comme \( r\) et \( s\) sont premiers entre eux, l'ordre de \( ab\) s'écrit sous la forme \( r_1s_1\) avec \( r_1\divides r\) et \( s_1\divides s\). Nous avons
	\begin{equation}
		a^{r_1s_1}b^{r_1s_1}=(ab)^{r_1s_1}=1,
	\end{equation}
	que nous élevons à la puissance \( r_2\) où \( r_2\) est défini en posant \(r=r_1r_2\) :
	\begin{equation}
		a^{rs_1}b^{rs_1}=1.
	\end{equation}
	Et comme \( a^{rs_1}=1\), nous concluons que \( b^{rs_1}=1\). Donc \( s\divides rs_1\). Par le lemme de Gauss \ref{LemPRuUrsD}, nous avons en fait \( s\divides s_1\). Puisqu'on a aussi \( s_1\divides s\), nous avons \( s=s_1\).

	Le même type d'argument donne \( r=r_1\), et finalement l'ordre de \( ab\) est \( r_1s_1=rs\).
\end{proof}

\begin{lemma}[\cite{Combes}]    \label{LemSkIOOG}
	Un sous-groupe d'indice \( 2\) est un sous-groupe normal.
\end{lemma}

\begin{proof}
	Si \( H\) est un tel sous-groupe d'un groupe \( G\), alors \( G\) possède exactement deux classes à gauche par rapport à \( H\) (théorème de Lagrange~\ref{ThoLagrange}) et se partitionne donc en deux parties : \( G=H\cup xH\) avec \( x \notin H \). De même pour les classes à droite : \( G=H\cup Hx\). Puisque la classe à droite \( Hx \) n'est pas \( H\), on a \( xH = Hx \), et \( H\) est normal dans \( G\) par la proposition~\ref{propGroupeNormal}.
\end{proof}

\begin{lemma}[\cite{NielsBMorph}]\label{PropubeiGX}
	Soit \( H\), un sous-groupe normal d'indice \( m\) de \( G\). Alors pour tout \( a\in G\) nous avons \( a^m\in H\).
\end{lemma}

\begin{proof}
	Par définition de l'indice, le groupe \( G/H\) est d'ordre \( m\). Donc si \( [a]\in G/H\), nous avons \( [a]^m=[e]\), ce qui signifie \( [a^m]=[e]\), ou encore \( a^m\in H\).
\end{proof}

\begin{proposition}[\cite{NielsBMorph}]     \label{PROPooVWVIooQzuAlA}
	Soit un groupe fini \( G\) et \( H\), un sous-groupe normal d'ordre \( n\) et d'indice \( m\) avec \( m\) et \( n\) premiers entre eux. Alors \( H\) est l'unique sous-groupe de \( G\) à être d'ordre \( n\).
\end{proposition}

\begin{proof}
	Soit \( H'\) un sous-groupe d'ordre \( n\). Si \( h\in H'\) alors \( h^n=1\) et \( h^m\in H\) par le lemme \ref{PropubeiGX}. Étant donné que \( m\) et \( n\) sont premiers entre eux, par le théorème de Bézout~\ref{ThoBuNjam}, il existe \( a,b\in \eZ\) tels que
	\begin{equation}
		am+bn=1.
	\end{equation}
	Et donc, \( h=h^1=(h^m)^a(h^n)^b\). En tenant compte du fait que \( h^n=1\) et \( h^m\in H\), nous avons \( h\in H\). Ce que nous venons de prouver est que \( H'\subset H\) et donc que \( H=H'\) parce que \( | H' |=| H |=| G |/m\).
\end{proof}

\begin{normaltext}
	Notons que cette proposition ne dit pas qu'il existe un sous-groupe d'ordre \( n\) et d'indice \( m\). Il dit juste que si il y en a un et si il est normal, alors il n'y en a pas d'autre.
\end{normaltext}

\begin{lemma}       \label{LemqAUBYn}
	L'ensemble des ordres d'un groupe commutatif est stable par PPCM\footnote{Définition \ref{DefrYwbct}.}.

	Autrement dit, si \( x\in G\) est d'ordre \( r\) et si \( y\in G\) est d'ordre \( s\), alors il existe un élément d'ordre \( \ppcm(r,s)\).
\end{lemma}

\begin{proof}
	Soit \( m=\ppcm(r,s)\). Afin d'écrire \( m\) sous une forme pratique, nous considérons les décompositions en facteurs premiers de \( r\) et \( s\) :
	\begin{subequations}
		\begin{align}
			r & =\prod_{i=1}^kp_i^{\alpha_i} \\
			s & =\prod_{i=1}^kp_i^{\beta_i}
		\end{align}
	\end{subequations}
	où \( \{ p_i \}_{i=1,\ldots, k}\) est l'ensemble des nombres premiers arrivant dans les décompositions de \( r\) et de \( s\). Si nous posons
	\begin{subequations}
		\begin{align}
			r' & =\prod_{\substack{i=1 \\\alpha_i>\beta_i}}^kp_i^{\alpha_i}      \\
			s' & =\prod_{\substack{i=1 \\\alpha_i\leq \beta_i}}^kp_i^{\beta_i},
		\end{align}
	\end{subequations}
	alors \( \ppcm(r,s)=r's'\) et \( r'\) et \( s'\) sont premiers entre eux. L'élément \( x^{r/r'}\) est d'ordre \( r'\) et l'élément \( y^{s/s'}\) est d'ordre \( s'\). Maintenant nous utilisons le fait que \( G\) soit commutatif et le lemme~\ref{LemyETtdy} pour conclure que l'ordre de \( x^{r/r'}y^{s/s'}\) est \( r's'=m\).
\end{proof}

%---------------------------------------------------------------------------------------------------------------------------
\subsection{Écriture des fractions}
%---------------------------------------------------------------------------------------------------------------------------

\begin{theorem}     \label{THOooWYQVooRBaAAM}
	Tout élément de \( \eQ^+\) s'écrit de façon unique comme quotient de deux entiers premiers entre eux.
\end{theorem}

\begin{proof}
	En deux parties\footnote{Définitions des pgcd et ppcm en \ref{DefrYwbct}.}
	\begin{subproof}
		\spitem[Unicité]
		Supposons avoir \( \frac{ a }{ b }=\frac{ c }{ d }\) avec \( \pgcd(a,b)=\pgcd(c,d)=1\). Nous avons
		\begin{equation}
			ad=bc
		\end{equation}
		donc
		\begin{enumerate}
			\item
			      \( a\) divise \( bc\) mais est premier avec \( b\) donc \( a\) divise \( c\) par le lemme de Gauss~\ref{LemPRuUrsD}.
			\item
			      \( c\) divise \( ad\) mais est premier avec \( d\) donc \( c\) divise \( a\) par le lemme de Gauss~\ref{LemPRuUrsD}.
		\end{enumerate}
		En conclusion \( a\) divise \( c\) et \( c\) divise \( a\), ergo \( a=c\). L'égalité \( b=d\) est alors immédiate.
		\spitem[Existence]
		Soit le quotient \( \frac{ a }{ b }\). Nous avons
		\begin{equation}
			\frac{ a }{ b }=\frac{ a/\pgcd(a,b) }{ b/\pgcd(a,b) },
		\end{equation}
		qui est encore un quotient d'entiers parce que \( \pgcd(a,b)\) divise aussi bien \( a\) que \( b\). Il faut montrer que les nombres \( a/\pgcd(a,b)\) et \( b/\pgcd(a,b)\) sont premiers entre eux. Pour cela nous supposons que \( k\) est un diviseur commun. En particulier, les nombres \( a/k\pgcd(a,b)\) et \( b/k\pgcd(a,b)\) sont des entiers, ce qui fait que \( k\pgcd(a,b)\) est un diviseur commun de \( a\) et \( b\). Étant donné que \( \pgcd(a,b)\) est le plus grand tel diviseur, nous devons avoir \( k\pgcd(a,b)=\pgcd(a,b)\) c'est-à-dire que \( k=1\). Donc les nombres \( a/\pgcd(a,b)\) et \( b/\pgcd(a,b)\) sont premiers entre eux.
	\end{subproof}
\end{proof}

\begin{proposition}     \label{PROPooRZDDooLJabov}
    Les entiers \( p\) et \( q\) sont premiers entre eux\footnote{Premiers entre eux, définition \ref{DefZHRXooNeWIcB}.} si et seulement si \( p^2\) et \( q^2\) sont premiers entre eux.

\end{proposition}

\begin{proof}
    En deux parties.
    \begin{subproof}
        \spitem[\( \Rightarrow\)]
        % -------------------------------------------------------------------------------------------- 
        Nous supposons que \( p^2\) et \( q^2\) ne sont pas premiers entre eux. Donc il existe \( \delta\) divisant \( p^2\) et \( q^2\). La proposition \ref{PROPooBKQNooFglPGI} implique que \( \delta\) divise \( p\) et \( q\). Donc \( p\) et \( q\) ne sont pas premiers entre eux.
        \spitem[\( \Leftarrow\)]
        % -------------------------------------------------------------------------------------------- 
	Si \( p^2\) et \( q^2\) sont premiers entre eux, par le théorème de Bézout~\ref{ThoBuNjam} il existe \( a,b\in \eZ\) tels que
	\begin{equation}
		ap^2+bq^2=1
	\end{equation}
	Dans ce cas, \( (ap)p+(bq)q=1\), ce qui montre (par encore Bézout, mais l'autre sens) que \( p\) et \( q\) sont premiers entre eux.
    \end{subproof}
\end{proof}

Une des conséquences de ces résultats sera le fait que \( \sqrt{n}\) est irrationnelle dès que \( n\) n'est pas un carré parfait, théorème~\ref{THOooYXJIooWcbnbm}.

Nous avons déjà vu dans la proposition~\ref{PropooRJMSooPrdeJb} que \( \sqrt{2}\) était irrationnel. En fait le théorème suivant va nous montrer que le nombre \( \sqrt{ n }\) est soit entier, soit irrationnel.
\begin{theorem}     \label{THOooYXJIooWcbnbm}
	Soit \( n\in \eN\). Le nombre \( \sqrt{n}\) est rationnel si et seulement si \( n\) est un carré parfait.
\end{theorem}

\begin{proof}
	Supposons que \( \sqrt{n}\) soit rationnel. Le théorème~\ref{THOooWYQVooRBaAAM} nous donne \( p,q\in \eN\) premiers entre eux tels que \( \sqrt{n}=p/q\). La proposition~\ref{PROPooRZDDooLJabov} nous enseigne de plus que \( p^2\) et \( q^2\) sont premiers entre eux. Nous avons
	\begin{equation}
		p^2=nq^2.
	\end{equation}
	Le nombre \( q\) est alors un diviseur commun de \( q^2\) et de \( p\). Donc \( q=1\) et \( n=p^2\) est un carré parfait.
\end{proof}

%---------------------------------------------------------------------------------------------------------------------------
\subsection{Équation diophantienne linéaire à deux inconnues}
%---------------------------------------------------------------------------------------------------------------------------
\label{subsecZVKNooXNjPSf}

\index{équation!diophantienne}

Soient \( a\), \( b\) et \( c\) des entiers naturels donnés. Nous considérons l'équation
\begin{equation}        \label{EqTOVSooJbxlIq}
	ax+by=c
\end{equation}
à résoudre\cite{PAYUooYVuNAB} pour \( (x,y)\in \eN^2\).

Si \( a\) ou \( b\) est nul, c'est facile; nous supposons donc que \( a\) et \( b\) sont tous deux non nuls. Nous commençons par simplifier l'équation en cherchant les diviseurs communs. Soit \( d=\pgcd(a,b)\) et notons \( a=da'\), \( b=db'\). Nous avons déjà l'équation
\begin{equation}
	d(a'x+b'y)=c,
\end{equation}
et donc si \( c\) n'est pas un multiple de \( d\), il n'y a pas de solution\footnote{Exemple : \( 8x+2y=9\). Le membre de gauche est certainement un nombre pair et il n'y a donc pas de solution.}. Si par contre \( c\) est un multiple de \( d\) alors nous écrivons \( c=c'd\) et l'équation devient
\begin{equation}
	a'x+b'y=c'
\end{equation}
C'est maintenant que nous utilisons le théorème de Bézout~\ref{ThoBuNjam} : puisque \( a'\) et \( b'\) sont premiers entre eux, nous avons la relation  \( a'u+b'v=1\) pour certains\footnote{Nous avons décrit un algorithme pour les trouver en démontrant l'équation~\ref{EqNDMLooDvaiAc}.} nombres entiers \( u\) et \( v\). Nous récrivons notre équation sous la forme \( a'x+b'y=c'(a'u+b'v)\) et rassemblons les termes :
\begin{equation}
	a'(x-c'u)=b'(c'v-y).
\end{equation}
C'est-à-dire que si \( (x,y)\) est une solution, alors \( a'\) divise \( b'(c'v-y)\). Mais comme \( a'\) et \( b'\) sont premiers entre eux, le nombre \( a'\) doit forcément\footnote{C'est Gauss~\ref{LemPRuUrsD} qui le dit, et vous savez que lorsque Gauss dit, c'est \emph{forcément} vrai.} diviser \( c'v-y\). Disons \( c'v-y=ka'\). Alors \( a'(x-c'u)=b'ka'\) et donc
\begin{equation}
	x=b'k+c'u.
\end{equation}
Nous trouvons alors une expression pour \( y\) en injectant ce résultat dans  \( a'x+b'y=c'\) et en utilisant le théorème de Bézout : \( a'c'u=(1-b'v)c'\). Au final nous avons prouvé que toutes les solutions sont de la forme
\begin{subequations}            \label{EqYCQVooZzHuRq}
	\begin{numcases}{}
		x=b'k+c'u\\
		y=vc'-a'k
	\end{numcases}
\end{subequations}
avec \( k\in\eZ\). Si nous ne voulons réellement que les solutions dans \( \eN\) et non dans \( \eZ\), il faut seulement un peu restreindre les valeurs de \( k\). Il en reste un nombre fini parce que l'équation pour \( x\) borne \( k\) vers le bas tandis que celle pour \( y\) borne \( k\) vers le haut.

Par ailleurs, il est très vite vérifié que tous les couples \( (x,y)\) de la forme \eqref{EqYCQVooZzHuRq} sont solutions.

\begin{example}
	Résoudre l'équation \( 2x+6y=52\).

	Nous pouvons factoriser \( 2\) dans le membre de gauche et simplifier alors toute l'équation par \( 2\) :
	\begin{equation}
		x+3y=26.
	\end{equation}
	Nous cherchons une relation de Bézout pour \( u+3v=1\). Ce n'est heureusement pas très compliqué : \( u=-5\), \( v=2\). Nous pouvons alors écrire
	\begin{equation}
		x+3y=26\times (-5+3\times 2),
	\end{equation}
	et donc \( x+5\times 26=-3(y-26\times 2)\), et en posant \( k=y-26\times 2\) nous avons
	\begin{equation}
		x=-3k-130.
	\end{equation}
	En injectant nous trouvons l'équation \( 3y-3k-130=26\) et donc
	\begin{equation}
		y=52+k.
	\end{equation}
\end{example}

%---------------------------------------------------------------------------------------------------------------------------
\subsection{Quotients}
%---------------------------------------------------------------------------------------------------------------------------

Nous écrivons \( a=b\mod p\) essentiellement si il existe \( k\in \eZ\) tel que \( b+kp=a\). Plus généralement nous notons \( [a]_p=\{ a+kp|k\in \eZ \}\)\nomenclature[R]{\( [a]_p\)}{ensemble des \( a+kp\)} et l'écriture «\( a=n\mod p\)» peut tout autant signifier \( a=[b]_p\) que \( a\in [b]_p\). La différence entre les deux est subtile mais peut de temps en temps valoir son pesant d'or.

\begin{proposition}
	Soit \( n\in\eN\). Le groupe \( (\eZ/n\eZ, +)\) est monogène. Si \( n\neq 0\), il est cyclique d'ordre \( n\).
\end{proposition}

\begin{proof}
	Nous considérons la surjection canonique \( \mu\colon \eZ\to \eZ/n\eZ\). Si \( a\in\eZ\), alors \( \mu(a)=a\mu(1)\). Par conséquent \( \eZ/n\eZ=\gr\bigl( \mu(1) \bigr)\) parce que tout groupe contenant \( \mu(1)\) contient tous les multiples de \( \mu(1)\), et par conséquent contient \( \mu(\eZ)=\eZ/n\eZ\).

	Soit \( x\in\eZ/n\eZ\) et considérons \( x_0\), le plus petit naturel représentant \( x\). Nous notons \( x=[x_0]\). Le théorème de la division euclidienne~\ref{ThoDivisEuclide} donne l'existence de \( q\) et \( r\) avec \( 0\leq r<n\) et \( q\geq 0\) tels que
	\begin{equation}
		x_0=nq+r.
	\end{equation}
	Nous avons \( [x_0]=[r]=\mu(r)\) parce que \( x_0-r\) est un multiple de \( n\). Nous avons donc \( [x_0]\in\mu(\eN_{n-1})\). Par conséquent
	\begin{equation}
		\eZ/n\eZ=\mu(\eZ)=\mu(\eN_{n-1}).
	\end{equation}
	La restriction \( \mu\colon \eN_{n-1}\to \eZ/n\eZ\) est donc surjective. Montrons qu'elle est également injective. Si \( \mu(x_0)=\mu(x_1)\), alors \( x_1=x_0+kn\). Si nous supposons que \( x_1>x_0\), alors \( k>0\) et si \( x_0\in\eN_{n-1}\), alors \( x_1>n-1\).

	L'ordre de \( \eZ/n\eZ\) est donc le même que le cardinal de \( \eN_{n-1}\), c'est-à-dire \( n\). Le groupe \( \eZ/n\eZ\) est donc fini, d'ordre \( n\) et monogène parce que \( \eZ/n\eZ=\gr(\mu(1))\). Il est donc cyclique.
\end{proof}

\begin{lemma}[\cite{KXjFWKA}]
	Soit \( q\in \eN\) avec \( q\geq 2\). Soient \( n,d\in \eN\) tels que \( q^d-1\divides q^n-1\). Alors \( d\divides n\).
\end{lemma}

\begin{proof}
	Par le théorème de division euclidienne~\ref{ThoDivisEuclide}, il existe \( a,b\in \eZ\) tels que \( n=ad+b\) avec \( 0\leq b<d\). En remarquant que \( q^d\in[1]_{q^d-1}\) nous avons
	\begin{equation}
		q^n=(q^d)^aq^b\in[1]_{q^d-1}q^b=[q^b]_{q^d-1}.
	\end{equation}
	Pour cela nous avons utilisé d'abord le fait que si \( a\in [z]_k\), alors \( a^n\in[z^n]_k\) et ensuite le fait que \( [1]_kx=[x]_k\). D'autre part l'hypothèse \( q^d-1\divides q^n-1\) implique
	\begin{equation}
		q^n\in[1]_{q^d-1}.
	\end{equation}
	Par conséquent le nombre \( q^n\) est à la fois dans \( [q^b]_{q^d-1}\) et dans \( [1]_{q^d-1}\). Cela implique que
	\begin{equation}
		[1]_{q^d-1}=[q^b]_{q^d-1},
	\end{equation}
	ou encore que \( q^b\in[1]_{q^d-1}\) ou encore que \( q^d-1\divides q^b-1\).

	Étant donné que \( b<d\) et que \( q\geq 2\), nous avons que \( q^b-1<q^d-1\); donc pour que \( q^d-1\) divise \( q^b-1\), il faut que \( q^b-1\) soit zéro, c'est-à-dire \( b=0\).

	Mais dire \( b=0\) revient à dire que \( d\divides n\), ce qu'il fallait démontrer.
\end{proof}

%+++++++++++++++++++++++++++++++++++++++++++++++++++++++++++++++++++++++++++++++++++++++++++++++++++++++++++++++++++++++++++
\section{Binôme de Newton et morphisme de Frobenius}
%+++++++++++++++++++++++++++++++++++++++++++++++++++++++++++++++++++++++++++++++++++++++++++++++++++++++++++++++++++++++++++

\begin{proposition}[\cite{ooPTQCooIWykWP}]     \label{PropBinomFExOiL}
	Pour tout \( x\), \( y\in\eR\) et \( n\in\eN\), nous avons
	\begin{equation}        \label{EqNewtonB}
		(x+y)^n=\sum_{k=0}^n{n\choose k}x^{n-k}y^k
	\end{equation}
	où
	\begin{equation}
		{n\choose k}=\frac{ n! }{ k!(n-k)! }
	\end{equation}
	sont les \defe{coefficients binomiaux}{coefficients binomiaux}.
\end{proposition}

\begin{proof}
	La preuve se fait par récurrence. La vérification pour \( n=0\) et \( n=1\) se fait aisément pour peu que l'on se rappelle que \( x^0=1\) et que \( 0!=1\), ce qui donne entre autres \( {0\choose 0}=1\).

	Supposons que la formule \eqref{EqNewtonB} soit vraie pour \( n\geq1\), et prouvons la pour \( n+1\). Nous avons
	\begin{equation}        \label{EqBinTrav}
		\begin{aligned}[]
			(x+y)^{n+1} & = (x+y)\cdot \sum_{k=0}^n{n\choose k}x^{n-k}y^k                                                     \\
			            & = \sum_{k=0}^n{n\choose k}x^{n-k+1}y^k+\sum_{k=0}^n{n\choose k}x^{n-k}y^{k+1}                       \\
			            & = x^{n+1}+ \sum_{k=1}^n{n\choose k}x^{n-k+1}y^k+\sum_{k=0}^{n-1}{n\choose k}x^{n-k}y^{k+1}+y^{n+1}.
		\end{aligned}
	\end{equation}
	La seconde grande somme peut être transformée en posant \( i=k+1\) :
	\begin{equation}
		\sum_{k=0}^{n-1}{n\choose k}x^{n-k}y^{k+1} = \sum_{i=1}^n{n\choose i-1}x^{n-(i-1)}y^{i-1+1},
	\end{equation}
	dans lequel nous pouvons immédiatement renommer \( i\) par \( k\). En remplaçant dans la dernière expression de \eqref{EqBinTrav}, nous trouvons
	\begin{equation}
		(x+y)^{n+1}=x^{n+1}+y^{n+1}+\sum_{k=1}^n\left[ {n\choose k}+{n\choose k-1} \right]x^{n-k+1}y^k.
	\end{equation}
	La thèse découle maintenant de la formule
	\begin{equation}
		{n\choose k}+{n\choose k-1}={n+1\choose k}
	\end{equation}
	qui est vraie parce que
	\begin{equation}
		\frac{ n! }{ k!(n-k)! }+\frac{ n! }{ (k-1)!(n-k+1)! }=\frac{ n!(n-k+1)+n!k }{ k!(n-k+1)! }=\frac{ n!(n+1) }{  k!(n+1-k)!  },
	\end{equation}
	par simple mise au même dénominateur.
\end{proof}

\begin{lemma}[\cite{MonCerveau}]        \label{LEMooLPCXooYIzJsD}
	Si \( n\geq k\) nous avons
	\begin{equation}
		\frac{ n! }{ k!(n-k)! }\leq \frac{ n^{k-1} }{ k! }.
	\end{equation}
\end{lemma}

\begin{proof}
	Nous décomposons le produit définissant \( n!\) en les facteurs entre \( 1\) et \( n-k\) et ceux entre \( n-k+1\) et \( n\) :
	\begin{equation}
		n!=(n-k)!\prod_{i=n-k+1}^ni\leq n^{k-1}(n-k)!.
	\end{equation}
	Donc
	\begin{equation}
		\frac{ n! }{ k!(n-k)! }\leq \frac{ n^{k-1} }{ k! }.
	\end{equation}
\end{proof}

Tant que nous sommes à démontrer des égalités, en voici une.

\begin{lemma}[\cite{BIBooLDPCooFJcgAl}]     \label{LEMooLPOQooICJYdV}
	Pour \( a,b\in \eR\) et \( n\in \eN\) nous avons
	\begin{equation}
		a^n+(-1)^{n-1}b^n=(a+b)\sum_{k=0}^{n-1}(-1)^ka^{n-1-k}b^k.
	\end{equation}
\end{lemma}

\begin{proof}
	C'est un simple calcul:
	\begin{subequations}
		\begin{align}
			(a+b)\sum_{k=0}^{n-1}(-1)^ka^{n-1-k}b^k & =  \sum_{k=0}^{n-1}(-1)^ka^{n-k}b^k+\sum_{k=0}^{n-1}(-1)^ka^{n-k-1}b^{k+1}                                                 \\
			                                        & = a^n+\sum_{k=1}^{n-1}(-1)^ka^{n-k}b^k+\sum_{k=0}^{n-2}(-1)^ka^{n-k-1}b^{k+1} +(-1)^{n-1}b^n   \label{SUBEQooUCIBooKsuEbh} \\
			                                        & = a^n+(-1)^{n-1}b^n                                            \label{SUBEQooLTIHooZPMwVF}
		\end{align}
	\end{subequations}
	Justifications.
	\begin{itemize}
		\item Pour \eqref{SUBEQooUCIBooKsuEbh}. Dans la première somme, nous avons séparé le terme \( k=0\) et dans la seconde nous avons séparé le terme \( k=n-1\)
		\item Pour \eqref{SUBEQooLTIHooZPMwVF}. Dans la seconde somme, décaler les termes pour sommer de \( 1\) à \( n-1\) et remarquer que ce qu'on obtient annule la première somme.
	\end{itemize}
\end{proof}

\begin{proposition}     \label{Propqrrdem}
	Soit \( A\) un anneau commutatif de caractéristique première \( p\). Alors \( \sigma(x)=x^p\) est un automorphisme de l'anneau \( A\). Nous avons la formule
	\begin{equation}
		(a+b)^p=a^p+b^p
	\end{equation}
	pour tout \( a,b\in A\).
\end{proposition}

\begin{proof}
	Nous utilisons la formule du binôme de la proposition~\ref{PropBinomFExOiL} et le fait que les coefficients binomiaux non extrêmes sont divisibles par \( p\) et donc nuls.
\end{proof}

%+++++++++++++++++++++++++++++++++++++++++++++++++++++++++++++++++++++++++++++++++++++++++++++++++++++++++++++++++++++++++++
\section{Idéal dans un anneau}
%+++++++++++++++++++++++++++++++++++++++++++++++++++++++++++++++++++++++++++++++++++++++++++++++++++++++++++++++++++++++++++

La définition d'un idéal dans un anneau est la définition~\ref{DefooQULAooREUIU}.

\begin{definition}[Idéal engendré par un élément]  \label{DefSKTooOTauAR}
	Si \( p\) est un élément d'un anneau \( A\) alors nous notons \( (p)\)\nomenclature[A]{\( (p)\)}{idéal engendré par \( p\)}\index{engendré!idéal dans un anneau} l'idéal dans \( A\) \defe{engendré}{engendré} par \( p\), c'est-à-dire \( pA\).
\end{definition}

\begin{definition}  \label{DefAJVTPxb}
	Un sous-ensemble \( B\subset A\) d'un anneau est un \defe{sous anneau}{sous-anneau} si
	\begin{enumerate}
		\item
		      \( 1\in B\)
		\item
		      \( B\) est un sous-groupe pour l'addition
		\item
		      \( B\) est stable pour la multiplication.
	\end{enumerate}
\end{definition}

\begin{remark}
	Un idéal n'est pas toujours un anneau parce que l'identité pourrait manquer. Un idéal qui contient l'identité est l'anneau complet.
\end{remark}

\begin{example}
	L'ensemble \( 2\eZ\) est un idéal de \( \eZ\). On peut aussi le noter \( (2) \).
\end{example}

\begin{proposition}[Premier théorème d'isomorphisme pour les anneaux]   \label{PROPooJALPooHFIObB}
	Soient \( A\) et \( B\) des anneaux et un homomorphisme \( f\colon A\to B\). Nous considérons l'injection canonique \( j\colon f(A)\to B\) et la surjection canonique \( \phi\colon A\to A/\ker f\). Alors il existe un unique isomorphisme
	\begin{equation}
		\tilde f \colon A/\ker f\to f(A)
	\end{equation}
	tel que \( f=j\circ\tilde f\circ\phi\).

	\begin{equation}
		\xymatrix{%
			A \ar[r]^{f}\ar[d]_{\phi}       &   B                         \\
			A/\ker f    \ar[r]_{\tilde f}   &   f(A)\subset B \ar[u]^{j}
		}
	\end{equation}
\end{proposition}
\index{théorème!isomorphisme!premier!pour les anneaux}

\begin{proposition}     \label{PropIJJIdsousphi}
	Soient \( I\) un idéal de \( A\) et la projection canonique
	\begin{equation}
		\phi\colon A\to A/I.
	\end{equation}
	C'est une bijection entre les idéaux de \( A\) contenant \( I\) et les idéaux de \( A/I\).

	Dit de façon imagée :
	\begin{equation}        \label{EqKbrizu}
		\{ \text{idéaux de } A\text{ contenant } I\}\simeq\{ \text{idéaux de } A/I \}.
	\end{equation}
\end{proposition}

\begin{proof}
	Si \( I\subset J\) et si \( J \) est un idéal de \( A\), alors \( \phi(J)\) est un idéal dans \( A/I\). En effet un élément de \( \phi(J)\) est de la forme \( \phi(j)\) et un élément de \( A/I\) est de la forme \( \phi(i)\). Leur produit vaut
	\begin{equation}
		\phi(i)\phi(j)=\phi(ij)\in\phi(J).
	\end{equation}

	Soit maintenant \( K\) un idéal dans \( A/I\) et soit \( J=\phi^{-1}(K)\). Étant donné qu'un idéal doit contenir \( 0\) (parce qu'un idéal est un groupe pour l'addition), \( [0]\in K\) et par conséquent \( I\subset\phi^{-1}(K)\).
\end{proof}

\begin{proposition}[\cite{MonCerveau}]     \label{AnnCorpsIdeal}\label{PROPooUOCVooZGAVVk}
	Si \( A\) est un anneau, nous avons les équivalences
	\begin{enumerate}
		\item       \label{ITEMooLAAVooXhTcMe}
		      \( A\) est un corps\footnote{Définition \ref{DefTMNooKXHUd}.}.
		\item       \label{ITEMooDGZIooRopYGx}
		      \( A\) est non nul et ses seuls\footnote{Je vous laisse vous poser de grandes questions sur le fait que le vide est un idéal ou non.} idéaux à gauche sont \( \{ 0 \}\) et \( A\).
		\item       \label{ITEMooLPWHooDJpTbR}
		      \( A\) est non nul et ses seuls idéaux à droite sont \( \{ 0 \}\) et \( A\).
	\end{enumerate}
\end{proposition}

\begin{proof}
	Nous allons montrer que le point \ref{ITEMooLAAVooXhTcMe} est équivalent aux deux autres.
	\begin{subproof}
		\spitem[\ref{ITEMooLAAVooXhTcMe} implique \ref{ITEMooDGZIooRopYGx}]
		Si \( I\) est un idéal à gauche différent de \( \{ 0 \}\), alors il contient un certain \( a\neq 0\). Puisque \( A\) est un corps, il contient un inverse \( a^{-1}\), et comme \( I\) est un idéal, \( a^{-1} I\subset I\). En particulier \( a^{-1}a\in I\). Donc \( 1\in I\) et \( I=A\).
		\spitem[\ref{ITEMooDGZIooRopYGx} implique \ref{ITEMooLAAVooXhTcMe}]
        Supposons que les seuls idéaux de \( A\) soient \( \{ 0 \}\) et \( A\). Soit \( a\in A\). Si \( a\) est non nul, alors \( aA\) est un idéal de \( a\). Vu qu'il contient \( a\neq 0\), nous avons \( aA=A\) (par hypothèse, un idéal qui n'est pas \( \{ 1 \}\) est \( A\)). En particulier, \( 1\in aA\), c'est-à-dire qu'il existe \( b\in A\) tel que \( ab=1\). L'élément \( a\) est donc inversible.
        \spitem[\ref{ITEMooLAAVooXhTcMe} implique \ref{ITEMooLPWHooDJpTbR}]
        % -------------------------------------------------------------------------------------------- 
        Comme pour \ref{ITEMooLAAVooXhTcMe} implique \ref{ITEMooDGZIooRopYGx}. 
        \spitem[\ref{ITEMooLPWHooDJpTbR} implique \ref{ITEMooLAAVooXhTcMe}]
        % -------------------------------------------------------------------------------------------- 
        Comme pour \ref{ITEMooDGZIooRopYGx} implique \ref{ITEMooLAAVooXhTcMe}.
	\end{subproof}
    Notez que je n'ai pas vérifié les deux derniers points. Donc vous devriez le vérifier et m'écrire si il y a un problème.
\end{proof}

\begin{definition}[\cite{ooWEUDooQybvIx}]      \label{DEFIdealMax}
    Soit un anneau \( A\). Un idéal \( I\neq A\) est dit \defe{idéal maximal}{idéal maximal} si il n'existe pas d'idéal \( J\neq A\) contenant strictement \( I\).
\end{definition}

\begin{proposition}[Thème~\ref{THEMEooZYKFooQXhiPD}]     \label{PROPooSHHWooCyZPPw}
	Un idéal \( I\) dans un anneau \( A \) est maximum si et seulement si \( A/I\) est un corps.
\end{proposition}

\begin{proof}
	Soit un idéal maximum \( I\subset A\). Alors les idéaux contenant \( I\) sont \( A\) et \( I\). L'application \( \phi\) de la proposition~\ref{PropIJJIdsousphi} est une bijection, donc l'ensemble des idéaux de \( A/I\) ne contient que deux éléments. Les seuls idéaux de \( A/I\) sont donc \( \{ 0 \}\) et \( A/I\); donc \( A/I\) est un corps par la proposition~\ref{PROPooUOCVooZGAVVk}.

	Dans l'autre sens, c'est la même chose : si \( A/I\) est un corps, il possède exactement deux idéaux, donc \( A\) ne contient que deux idéaux contenant \( I\). Donc \( I\) est un idéal maximum.
\end{proof}

%+++++++++++++++++++++++++++++++++++++++++++++++++++++++++++++++++++++++++++++++++++++++++++++++++++++++++++++++++++++++++++
\section{Polynômes de plusieurs variables}
%+++++++++++++++++++++++++++++++++++++++++++++++++++++++++++++++++++++++++++++++++++++++++++++++++++++++++++++++++++++++++++

\begin{definition}      \label{DEFooZNHOooCruuwI}
	L'ensemble des \defe{polynôme de \( n\) variables}{polynôme de plusieurs variables} sur l'anneau \( A\) est \( A^{(\eN^n)}\), c'est-à-dire l'ensemble des suites indexées par \( \eN^n\) et dont seulement une quantité finie de coefficients sont non nuls.

	Le produit sur \( A[X_1,\ldots, X_n]\) est défini par
	\begin{equation}
		(PQ)(k_1,\ldots, k_n)=\sum_{\substack{ (l_1,\ldots, l_n),(m_1,\ldots, m_n)\in \eN^n\times \eN^n   \\l_i+m_i=k_i}}P_{l_1,\ldots, l_n}Q_{m_1,\ldots, m_n}.
	\end{equation}
\end{definition}

\begin{normaltext}
	Dans \( A[X_1,\ldots, X_n]\), la multiplication n'est pas la multiplication de fonctions \( \eN^n\to \eK\), parce que le but est d'obtenir la multiplication usuelle au niveau des évaluations.
\end{normaltext}

\begin{definition}
	Si \( P\) est un polynôme de \( n\) variables sur \( A\), et si \( (x_1,\ldots, x_n)\in A^n\), \defe{l'évaluation}{évaluation!polynôme plusieurs variables} de \( P\) sur \( (x_1,\ldots, x_n)\) est
	\begin{equation}
		P(x_1,\ldots, x_n)=\sum_{(k_1,\ldots, k_n)\in \eN^n}P_{k_1,\ldots, k_n}x_1^{k_1}\ldots x_n^{k_n}.
	\end{equation}
	Notez que la somme, bien que sur \( \eN^n\), est une somme finie.
\end{definition}

\begin{normaltext}
	Comme dans le cas des polynômes d'une seule variable, les \( X_i\) dans la notation \( A[X_1,\ldots, X_n]\) sont à prendre à la légère. L'anneau des polynômes de \( n\) variables sur \( A\) aurait mieux fait d'être noté par exemple par \( \mP_n(A)\).

	Le fait est que nous avons les polynômes élémentaires définis par
	\begin{equation}
		X_1(k_1,\ldots, k_n)=\begin{cases}
			1 & \text{si } (k_1,\ldots, k_n)=(1,0\ldots, 0) \\
			0 & \text{sinon. }
		\end{cases}
	\end{equation}
	et que l'anneau des polynômes peut être vu comme \( A\) (les polynômes constants) étendus par les \( X_i\).

	Quoi qu'il en soit, les \( X_i\) dans la notation \( A[X_1,\ldots, X_n]\) sont des indices muets. L'anneau \( A[X_1,\ldots, X_n]\) est exactement le même que \( A[T_1,\ldots, T_n]\).
\end{normaltext}

%---------------------------------------------------------------------------------------------------------------------------
\subsection{Divisibilité et classes d'association}
%---------------------------------------------------------------------------------------------------------------------------
\label{DivisibiliteAnneauxIntegres}

\begin{definition}\label{DefrXUixs}
	On dit de deux éléments \( a,b\in A\) qu'ils sont \defe{associés}{associés!éléments d'un anneau} si il existe un inversible \( u\in A\) tel que \( a=ub\).

	%TODO: qu'est ce que 'U(A)' ici ?
	La \defe{classe d'association}{classe d'association}\index{classe!d'association} d'un élément \( a \in A \) est l'ensemble des éléments qui lui sont associés.
\end{definition}

\begin{lemma}\label{LemRmVTRq}
	Si \( A\) est un anneau intègre et si \( a,b\in A\) sont deux éléments vérifiant \( a\divides b\) et \( b\divides a\), alors ils sont associés, c'est à dire qu'il existe un inversible \( u\in A\) tel que \( a=ub\).
\end{lemma}

\begin{proof}
	Les hypothèses à propos de la divisibilité nous indiquent que \( a=xb\) et \( b=ya\) pour certains \( x,y\in A\). Alors,
	\begin{equation}
		b(1-yx)=0.
	\end{equation}
	Étant donné que \( A\) est intègre, cela montre que \( b=0\) ou \( 1-yx=0\). Si \( b=0\) nous avons immédiatement \( a=0\) et le lemme est prouvé. Si au contraire \( yx=1\), c'est que \( y\) et \( x\) sont inversibles et inverses l'un de l'autre.
\end{proof}

\begin{lemma}       \label{LEMooJBOXooYkMRrz}
	Si \( A\) est un anneau et si \( a\in A\), la classe d'association de \( a\) est \( aU(A)\) où \( U(A)\) est l'ensemble des éléments inversibles de \( A\).
\end{lemma}

\begin{example}
	Dans \( \eZ[i]\), les inversibles sont \( \pm 1\) et \( \pm i\). Donc les éléments associés à \( z\) sont \( z\), \( -z\), \( iz\) et \( -iz\).

	Notons au passage que la notion de divisibilité dans \( \eZ[i]\) n'est pas immédiatement intuitive. En effet bien que \( 7\) ne soit pas divisible par \( 2\) (ni dans \( \eZ\) ni dans \( \eZ[i]\)), le nombre \( 7+6i\) est divisible par \( 2+i\) dans \( \eZ[i]\). En effet :
	\begin{equation}
		(2+i)(4+i)=7+6i.
	\end{equation}
\end{example}

\begin{example}
	Si \( \eK\) est un corps, l'élément \( XY\) de \( \eK[X,Y]\) n'est pas premier parce que \( XY\divides X^2Y^2\) alors que \( XY\) ne divise ni \( X^2\) ni \( Y^2\).
\end{example}

\begin{probleme}
	Est-ce que quelqu'un connaît un anneau contenant \( \eZ\) dans lequel \( 7\) est divisible en \( 2\) ?

	Peut-être \( \eZ\) étendu par tous les \( 1/2^n\) ?
\end{probleme}

%---------------------------------------------------------------------------------------------------------------------------
\subsection{PGCD et PPCM}
%---------------------------------------------------------------------------------------------------------------------------

Pour le théorème de Bézout et autres opérations avec des modulo, voir le thème~\ref{THEMEooNRZHooYuuHyt}. Le pgcd et le ppcm sont définis en \ref{DefrYwbct}.

\begin{lemma}
	Soient \( A\) un anneau intègre et \( S\subset A\). Si \( \delta\) est un PGCD de \( S\), alors l'ensemble des PGCD de \( S\) est la classe d'association de \( \delta\).

	De la même façon si \( \mu\) est un PPCM de \( S\), alors l'ensemble des PPCM de \( S\) est la classe d'association de \( \mu\).
\end{lemma}

\begin{proof}
	Soient \( \delta\) un PGCD de \( S\) et \( u\) un inversible dans \( A\). Si \( x\in S\) nous avons \( \delta\divides x\) et donc \( x=a\delta\). Par conséquent \( x=au^{-1}u\delta\) et donc \( u\delta\) divise \( x\). De la même manière, si \( d\) divise \( x\) pour tout \( x\in S\), alors \( d\) divise \( \delta\) et donc \( \delta=ad\) et \( u\delta=aud\), ce qui signifie que \( d\) divise \( u\delta\).

	Dans l'autre sens nous devons prouver que si \( \delta'\) est un autre PGCD de \( S\), alors il existe un inversible \( u\in A\) tel que \( \delta'=u\delta\). Comme \( \delta'\) divise \( x\) pour tout \( x\in S\), nous avons \( \delta'\divides \delta\), et symétriquement nous trouvons \( \delta\divides\delta'\). Par conséquent (lemme~\ref{LemRmVTRq}), il existe un inversible \( u\) tel que \( \delta=u\delta'\).

	Le même type de raisonnement tient pour le PPCM.
\end{proof}

Si \( \delta\) est un PGCD de \( S\), nous dirons \emph{par abus de langage} que \( \delta\) est \emph{le} PGCD de \( S\), gardant en tête qu'en réalité toute sa classe d'association est PGCD. Nous noterons aussi, toujours par abus que \( \delta=\pgcd(S)\).

\begin{remark}
	La classe d'association d'un élément n'est pas toujours très grande. Les inversibles dans \( \eZ\) étant seulement \( \pm 1\), nous pouvons obtenir l'unicité du PGCD et du PPCM en imposant qu'ils soient positifs.

	Pour les polynômes, nous obtenons l'unicité en demandant que le PGCD soit unitaire.

	Dans les cas pratiques, il y a donc en réalité peu d'ambiguïté à parler du PGCD ou du PPCM d'un ensemble.
\end{remark}

%---------------------------------------------------------------------------------------------------------------------------
\subsection{Anneaux intègres et corps}
%---------------------------------------------------------------------------------------------------------------------------

Le fait d'être intègre pour un anneau n'assure pas le fait d'être un corps. Nous avons cependant ce résultat pour les anneaux finis.

\begin{proposition}     \label{PropanfinintimpCorp}
	Un anneau fini intègre est un corps.
\end{proposition}

\begin{proof}
	Soit \( A\) un tel anneau. Soit \( a\neq 0\). Les applications
	\begin{subequations}
		\begin{align}
			l_a\colon x\mapsto ax \\
			r_a\colon x\mapsto xa
		\end{align}
	\end{subequations}
	sont injectives. En tant que applications injectives entre ensembles finis, elles sont surjectives. Il existe donc \( b\) et \( c\) tels que \( 1=ba=ac\). Nous trouvons que \( b\) et \( c\) sont égaux parce que
	\footnote{Il faut être un peu souple avec les notations communément employées dans les ouvrages mathématiques, et que nous reprenons telles quelles. Dans l'équation qui suit, \( b(ac)\) est le produit de \( b\) par l'élément \( ac\), et non quelque chose comme le produit de \( b\) avec l'idéal \( (ac)\) par exemple.}
	\begin{equation}
		b=b(ac)=(ba)c=c.
	\end{equation}
	Par conséquent \( b\) est un inverse de \( a\).
\end{proof}

\begin{proposition}     \label{PropzhFgNJ}
	Soit \( n\in\eN^*\). Les conditions suivantes sont équivalentes :
	\begin{enumerate}
		\item
		      \( n\) est premier.
		\item
		      \( \eZ/n\eZ\) est un anneau intègre.
		\item
		      \( \eZ/n\eZ\) est un corps.
	\end{enumerate}
\end{proposition}

\begin{proof}
	L'équivalence entre les deux premiers points est le contenu du corolaire~\ref{CorZnInternprem}. Le fait que \( \eZ/n\eZ\) soit un corps lorsque \( \eZ/n\eZ\) est intègre est la proposition~\ref{PropanfinintimpCorp}. Le fait que \( \eZ/n\eZ\) soit intègre lorsque \( \eZ/n\eZ\) est un corps est une propriété générale des corps : ce sont en particulier des anneaux intègres (lemme~\ref{LemAnnCorpsnonInterdivzer}).
\end{proof}

%---------------------------------------------------------------------------------------------------------------------------
\subsection{Élément irréductible}
%---------------------------------------------------------------------------------------------------------------------------

\begin{definition}[Élément irréductible\cite{ooWUNIooXKxRya}]  \label{DeirredBDhQfA}
    Un élément d'un anneau commutatif est \defe{irréductible}{irréductible!dans un anneau} si il n'est ni inversible, ni le produit de deux éléments non inversibles. \index{polynôme irréductible}
\end{definition}

\begin{normaltext}
	Nous allons voir dans la section \ref{SECooSWGKooEeOZTO} que le concept d'élément irréductible n'est vraiment utile que dans le cas des anneaux intègres.
\end{normaltext}

\begin{example}
	Un corps n'a pas d'élément irréductible parce qu'à part zéro, tous les éléments sont inversibles. Mais \( 0\) n'est pas irréductible parce qu'il peut être écrit comme produit d'éléments non inversibles : \( 0=0\cdot 0\).
\end{example}

\begin{proposition}     \label{PROPooKDWQooTtScrN}
    Les éléments irréductibles de l'anneau \( \eZ\) sont les nombres premiers\footnote{Nombre premier, définition \ref{DEFooZCRQooWXRalw}.}. 
\end{proposition}

\begin{proof}
    Les seuls inversibles de \( \eZ\) sont \( \pm 1\).

    Si \( p\) est premier et \( p=ab\) avec \( a,b\in \eZ\), alors nous avons soit \( a=\pm 1\) soit \( b=\pm 1\). Donc \( p\) n'est pas le produit de deux éléments non inversibles.

    Dans le sens inverse, supposons que \( p\) soit irréductible dans \( \eZ\). D'abord \( p\) ne peut pas être \( \pm 1\) parce que \( \pm 1\) sont inversibles. Ensuite supposons que \( p=ab\). Vu que \( p\) est irréductible, nous avons \( a=\pm1\) ou \( b=\pm1\). Autrement dit, dans \( p=ab\), soit \( a\) soit \( b\) est un inversible.
\end{proof}

%+++++++++++++++++++++++++++++++++++++++++++++++++++++++++++++++++++++++++++++++++++++++++++++++++++++++++++++++++++++++++++
\section{Anneau factoriel}
%+++++++++++++++++++++++++++++++++++++++++++++++++++++++++++++++++++++++++++++++++++++++++++++++++++++++++++++++++++++++++++

\begin{definition}[Anneau factoriel]        \label{DEFooVCATooPJGWKq}
	Un anneau commutatif \( A\) est \defe{factoriel}{factoriel!anneau}\index{anneau!factoriel} si il vérifie les propriétés suivantes.
	\begin{enumerate}
		\item
		      L'anneau \( A\) est intègre (pas de diviseurs de zéro).
		\item
		      Si \( a\in A\) est non nul et non inversible, alors il admet une décomposition en facteurs irréductibles: \( a=p_1\ldots p_k\) où les \( p_i\) sont irréductibles.
		\item
		      Si \( a=q_1\ldots q_m\) est une autre décomposition de \( a\) en irréductibles, alors \( m=k\) et il existe une permutation\footnote{Définition~\ref{DEFooJNPIooMuzIXd}.} \( \sigma\in S_k\) telle que \( p_i\) et \( q_{\sigma(i)}\) soient associés\footnote{Définition~\ref{DefrXUixs}.}.
	\end{enumerate}
\end{definition}

Un anneau factoriel permet de caractériser le \( \pgcd\) et le \( \ppcm\) de nombres.

\begin{proposition}
	Soit une famille \( \{ a_n \}\) d'éléments de \( A\) qui se décomposent en irréductibles comme
	\begin{equation}
		a_i=\prod_k p_k^{\alpha_{k,i}}.
	\end{equation}
	Alors
	\begin{equation}
		\pgcd\{ a_n \}=\prod_k p_k^{\min_i\{ \alpha_{k,i} \}}.
	\end{equation}

	De plus le PGCD est :
	\begin{enumerate}
		\item
		      Un multiple de tous les diviseurs communs des \( a_i\).
		\item
		      Unique pour cette propriété à multiple près par un inversible\quext{Soyez prudent avec cette affirmation : je n'en n'ai pas de démonstrations sous la main et ne suis pas certain que ce soit vrai.}.
	\end{enumerate}
\end{proposition}

De la même manière,
\begin{equation}
	\ppcm\{ a_n \}=\prod_kp_k^{\max_i\{ \alpha_{k,i} \}}.
\end{equation}
Un anneau factoriel a une relation de préordre partiel\index{ordre!sur un anneau factoriel} donnée par \( a<b\) si \( a\) divise \( b\). En termes d'idéaux, cela donne l'ordre inverse de celui de l'inclusion\footnote{Voir proposition~\ref{PropDiviseurIdeaux}.} : \( a<b\) si et seulement si \( (b)\subset (a)\).

\begin{example} \label{EXooCWJUooCDJqkr}
	L'anneau \( \eZ[i\sqrt{3}]\) n'est pas factoriel parce que \( 4\) a au moins deux décompositions distinctes en irréductibles :
	\begin{equation}
		4=2\cdot 2,
	\end{equation}
	et
	\begin{equation}
		4=(1+i\sqrt{3})(1-i\sqrt{3}).
	\end{equation}
\end{example}

Nous allons voir dans l'exemple~\ref{ExeDufyZI} que \( \eZ[i\sqrt{2}]\) est factoriel parce qu'il sera euclidien.

%+++++++++++++++++++++++++++++++++++++++++++++++++++++++++++++++++++++++++++++++++++++++++++++++++++++++++++++++++++++++++++
\section{Anneau principal et idéal premier}
%+++++++++++++++++++++++++++++++++++++++++++++++++++++++++++++++++++++++++++++++++++++++++++++++++++++++++++++++++++++++++++

\begin{definition}      \label{DEFooMZRKooBPLAWH}
	Un idéal \( I\) dans \( A\) est \defe{principal à gauche}{idéal!principal!à gauche} si il existe \( a\in I\) tel que \( I= A a\). Il est \defe{principal à droite}{idéal!principal!à droite} si il existe \( a\in I\) tel que \( I=a A\). Nous disons qu'il est \defe{principal}{principal!idéal} si il est principal à gauche et à droite.
\end{definition}

\begin{definition}          \label{DEFooGWOZooXzUlhK}
	Un anneau est \defe{principal}{principal!anneau} si
	\begin{enumerate}
		\item
		      il est commutatif et intègre
		\item
		      tous ses idéaux sont principaux.
	\end{enumerate}
\end{definition}

Souvent pour prouver qu'un anneau est principal, nous prouvons qu'il est euclidien (définition~\ref{DefAXitWRL}) et nous utilisons la proposition~\ref{Propkllxnv} qui dit qu'un anneau euclidien est principal.

Une manière de prouver qu'un anneau n'est pas principal est de prouver qu'il n'est pas factoriel, théorème~\ref{THOooANCAooBChmwp}.

\begin{definition}      \label{DEFooAQSZooVhvQWv}
	Nous disons qu'un idéal \( I\) dans \( A\) est \defe{premier}{premier!idéal} si \( I\) est strictement inclus dans \( A\) et si pour tout \( a,b\in A\) tels que \( ab\in I\) nous avons \( a\in I\) ou \( b\in I\).
\end{definition}

\begin{lemma}       \label{LEMooYRPBooYxXXsi}
	L'idéal nul (réduit à \( \{ 0 \}\)) est premier si et seulement si \( A\) est intègre.
\end{lemma}

\begin{proof}
	En deux sens.
	\begin{subproof}
		\spitem[Si \( \{ 0 \}\) est premier]
		Alors \( A\neq \{ 0 \}\) parce que \( I=\{ 0 \}\) est propre (définition d'idéal premier).

		De plus, si \( ab=0\), alors \( ab\in I\) qui est un idéal premier. Donc soit \( a\) soit \( b\) est dans \( I\), c'est-à-dire que soit \( a\) soit \( b\) est nul. Donc \( A\) est intègre.

		\spitem[Si \( A\) est intègre]
		Alors \( A\neq \{ 0 \}\) et l'idéal \( I=\{ 0 \}\) est strictement inclus dans \( A\). Si \( ab\in I\), alors \( ab=0\) et comme \( A\) est intègre, soit \( a\) soit \( b\) est nul, c'est-à-dire appartient à \( I\).
	\end{subproof}
\end{proof}

\begin{proposition}[\cite{ooWEUDooQybvIx}]      \label{PROPooRUQKooIfbnQX}
	Soit un anneau commutatif\footnote{Tous les anneaux du Frido sont commutatifs} et un idéal \( I\) dans \( A\).
	\begin{enumerate}
		\item       \label{ITEMooUGBTooOGrnWl}
		      \( I\) est un idéal premier si et seulement si \( A/I\) est un anneau intègre.
		\item   \label{ITEMooGLXSooUjINqR}
		      \( I\) est un idéal maximal si et seulement si \( A/I\) est un corps.
		\item       \label{ITEMooTFFQooOUajFw}
		      Tout idéal maximal propre est premier.
	\end{enumerate}
\end{proposition}

\begin{proof}
	En plein d'étapes.
	\begin{subproof}
        \spitem[\ref{ITEMooUGBTooOGrnWl}, \( \Rightarrow\)]
		Évacuons le cas trivial pour être sûr. Si \( I=\{ 0 \}\) alors \( A\) est intègre par le lemme \ref{LEMooYRPBooYxXXsi}. Donc \( A/I=A/\{ 0 \}=A\) est intègre également.

		Soient \( a,b\in A\) tels que \( [a][b]=[0]\). Donc \( [ab]=[0]\), c'est-à-dire \( ab\in I\). Puisque \( I\) est un idéal premier nous avons \( a\in I\) ou \( b\in I\), c'est-à-dire \( [a]=0\) ou \( [b]=0\); nous en déduisons que \( A/I\) est un anneau intègre.
        \spitem[\ref{ITEMooUGBTooOGrnWl}, \( \Leftarrow\)]

		Soit \( ab\in I\). Alors \( [ab]=0\), ce qui signifie que \( [a][b]=0\) donc que \( [a]=0\) ou que \( [b]=0\) parce que \( A/I\) est intègre. Mais la condition \( [a]=0\) signifie \( a\in I\), et \( [b]=0\) signifie \( b\in I\). Nous avons donc prouvé que soit \( a\) soit \( b\) est dans \( I\), c'est-à-dire que \( I\) est premier.
        \spitem[\ref{ITEMooGLXSooUjINqR}, \( \Rightarrow\)]

		Nous devons montrer que tout élément non nul de \( A/I\) est inversible. Un élément non nul de \( A/I\) est \( [x]\) avec \( x\in A\setminus I\).

		Nous considérons \( J=Ax+I\), qui est un idéal parce que pour tout \( a\in A\), \( aAx+aI\in Ax+I\). Mais comme \( I\) est maximal, \( J=I\) ou \( J=A\).

		Si \( J=I\), nous aurions que pour tout \( a\in A\) et pour tout \( i\in I\), \( ax+i\in I\). En particulier pour \( a=1\) et \( i=0\) nous aurions \( x\in I\), ce qui est contraire à l'hypothèse faite sur \( x\).

		Donc \( J=A\). En particulier, \( 1\in J\), c'est-à-dire qu'il existe \( a\in A\) et \( i\in I\) tels que \( ax+i=1\). En passant aux classes, \( [ax]=1\), c'est-à-dire \( [a][x]=1\) qui signifie que \( [a]\) est un inverse de \( [x]\) dans \( A/I\).

        Nous avons prouvé que \( A/I\) est un corps.

        \spitem[\ref{ITEMooGLXSooUjINqR}, \( \Leftarrow\)]
		Si \( x\in A\setminus I\), il faut prouver que tout idéal contenant \( I\) et \( x\) est \( A\).

		Un idéal contenant \( I\) et \( x\) doit contenir l'idéal \( J=Ax+I\). Comme \( x\notin I\), nous avons \( [x]\neq 0\) dans \( A/I\). Donc \( [x] \) est inversible et il existe \( a\in A\) tel que \( [ax]=[1]\). C'est-à-dire que \( ax-1\in I\). Nous avons alors
		\begin{equation}
			1=ax+\underbrace{(1-ax)}_{\in I}.
		\end{equation}
		C'est-à-dire que \( 1\in Ax+I\) et donc \( Ax+I=A\).
	\end{subproof}
	Enfin nous prouvons que tout idéal maximal propre est premier.

	Si \( I\) est maximal, \( A/I\) est un corps par le point \ref{ITEMooGLXSooUjINqR}, et vu que \( I\) est propre, le corps \( A/I\) n'est pas réduit à \( \{ 0 \}\). Donc le lemme \ref{LemAnnCorpsnonInterdivzer} dit que \( A/I\) est un anneau intègre. Le point \ref{ITEMooUGBTooOGrnWl} dit alors que \( I\) est un idéal premier.
\end{proof}

\begin{remark}
	Puisqu'un corps peut être réduit à \( \{0\}\), dans \ref{ITEMooGLXSooUjINqR}, l'idéal peut être \( A\). Mais pas dans \ref{ITEMooTFFQooOUajFw}, parce qu'un idéal premier est propre, ça fait partie de la définition \ref{DEFooAQSZooVhvQWv}.
\end{remark}

\begin{proposition}[\cite{ooOYKZooOJBDHS}]     \label{PROPooHABIooBZZQMj}
	Si \( A\) est un anneau commutatif intègre, alors un idéal \( I\) dans \( A\) est premier si et seulement si \( A/I\) est intègre.
\end{proposition}

\begin{proof}
	Supposons que \( I\) soit un idéal premier. Si \( [a],[b] \in A/I\)  sont tels que \( [a][b]=0\), alors \( [ab]=0\), ce qui signifie que \( ab\in I\). Mais alors, puisque \( I\) est premier, soit \( a\) soit \( b\) est dans \( I\). Cela signifie que soit \( [a]\) soit \( [b]\) est nul dans \( A/I\). Cela prouve que \( A/I\) est un anneau intègre.

	Dans l'autre sens, nous supposons que \( A/I\) est intègre. Cela implique immédiatement que \( I\neq A\) parce que \( A/A\) n'est pas un anneau intègre (tout le monde est évidemment diviseur de zéro).

	Soient donc \( a,b\in A\) tels que \( ab\in I\). Alors \( [a][b]=[ab]=0\) dans \( A/I\), mais comme \( A/I\) est intègre, cela implique que soit \( [a]\) soit \( [b]\) est nul. Autrement dit, soit \( a\) soit \( b\) est dans \( I\).
\end{proof}

\begin{proposition}[Thème~\ref{THEMEooZYKFooQXhiPD}, \cite{MonCerveau}] \label{PropomqcGe}
	Soit \( A\) un anneau principal\footnote{Définition \ref{DEFooGWOZooXzUlhK}.} qui n'est pas un corps. Pour un idéal propre \( I\) de \( A\), les conditions suivantes sont équivalentes :
	\begin{enumerate}
		\item       \label{ITEMooNOVFooEHtcwE}
		      \( I\) est un idéal maximal\footnote{Définition \ref{DEFIdealMax}.};
		\item       \label{ITEMooMQWVooNocVEU}
		      \( I\) est un idéal premier non nul\footnote{Définition \ref{DEFooAQSZooVhvQWv}.};
		\item       \label{ITEMooJBXGooEISNuW}
		      il existe \( p\) irréductible\footnote{Définition \ref{DeirredBDhQfA}.} dans \( A\) tel que \( I=(p)\).
	\end{enumerate}
\end{proposition}

\begin{proof}
	En plusieurs implications.
	\begin{subproof}
		\spitem[\ref{ITEMooNOVFooEHtcwE} implique~\ref{ITEMooMQWVooNocVEU}]
		Par hypothèse, \( I\) est un idéal propre, de plus il n'est pas égal à \( \{ 0 \}\), parce que lorsque \( A\) et \( \{ 0 \} \) sont les seuls idéaux, nous avons un corps (proposition~\ref{PROPooUOCVooZGAVVk}). Étant donné que \( I\) est un idéal maximal, le quotient \( A/I\) est un corps par la proposition~\ref{PROPooSHHWooCyZPPw}.

		Soient maintenant, pour entrer dans le vif du sujet, des éléments \( a,b\in A\) tels que \( ab\in I\). Dans le corps \( A/I\) nous avons \( [ab]=0\), et par définition du produit dans le quotient, \( [a][b]=0\). Par intégrité de l'anneau \( A/I\) (un corps est un anneau intègre, lemme \ref{LEMooZSMEooUmSXWZ}) nous avons soit \( [a]=0\), soit \( [b]=0\), soit les deux en même temps. Dans tous les cas, soit \( a\) soit \( b\) est dans \( I\).

		\spitem[\ref{ITEMooMQWVooNocVEU} implique~\ref{ITEMooJBXGooEISNuW}]
		Maintenant \( I\) est un idéal premier non réduit à \( \{ 0 \}\). Puisque \( A\) est un anneau principal, il existe \( x\in A\) tel que \( I=(x)\). Nous devons prouver que \( x\) peut être choisi irréductible; et nous allons faire plus : nous allons prouver que \( x\) ne peut être que irréductible\quext{ça me semble un peu trop facile. Lisez attentivement, et écrivez-moi pour dire si vous êtes d'accord ou pas.}.

		Supposons que \( x\) ne soit pas irréductible. Alors il existe \( a,b\in A\) non inversibles tels que \( x=ab\). Si \( a\in (x)\) alors il existe \( k\in A\) tel que \( a=xk\), et en particulier, \( a=abk\), c'est-à-dire \( 1=bk\) (parce que \( A\) est principal et donc intègre). Cela signifie que \( b\) est inversible alors que nous avions dit qu'il ne l'était pas. Nous en déduisons que \( a\) n'est pas dans \( (x)\). On montre de manière similaire que \( b\) n'est pas dans \( (x)\) non plus.

		Nous nous retrouvons donc avec \( a,b\in A\) tel que \( ab\in I\) sans que ni \( a\) ni \( b\) ne soient dans \( I\). Cela contredit le fait que \( I\) soit un idéal premier. En conclusion, \( x\) est irréductible.

		\spitem[\ref{ITEMooJBXGooEISNuW} implique~\ref{ITEMooNOVFooEHtcwE}]
		Nous avons \( I=(p)\) avec \( p\) irréductible dans \( A\). Supposons que \( J\) est un idéal différent de \( A\) contenant \( I\). Comme \( A\) est principal, il existe \( y\in A\) tel que \( J=(y)\). En particulier \( p\in J\), donc \( p=ay\) pour un certain \( a\in A\). Mais \( p\) est irréductible, donc soit \( a\) est inversible, soit \( y\) est inversible. Si \( y\) est inversible, alors \( J=A\), ce qui est exclu. Si \( a\) est inversible, alors \( (y)=(p)\), et \( I=J\).
	\end{subproof}
\end{proof}

\begin{normaltext}
	Dans la proposition \ref{PropomqcGe}, l'hypothèse d'idéal propre est importante. En effet dans le cas \( I=A\), nous avons évidemment que \( I\) est un idéal maximum. Mais \( A\) n'est d'abord pas un idéal premier parce qu'un idéal premier doit être strictement inclus dans l'anneau. Et ensuite, \( A\) est en général loin d'être garanti d'être égal à \( (p)\) pour un de ses éléments \( p\).
\end{normaltext}

\begin{proposition}     \label{PropoTMMXCx}
	Soit \( A \) un anneau principal, et soit \( p \in A \) un élément irréductible. Alors
	\begin{enumerate}
		\item
		      \( (p)\) est un idéal maximum.
		\item       \label{ITEMooKPJQooWuPZXS}
		      \( A/(p)\) est un corps.
	\end{enumerate}
\end{proposition}

\begin{proof}
	Nous notons \( I=(p)\). Soit un idéal \( J\) contenant \( I\). Comme \( A\) est principal, \( J\) est monogène : \( J=(q)\). Mais comme \( p\) est dans \( I\) qui est dans \( J\), il existe \( a\in A\) tel que \( p=qa\).

	Puisque \( p\) est irréductible, soit \( q\), soit \( a\) est inversible. Si \( q\) est inversible, alors \( J=A\). Si \( a\) est inversible, alors nous avons \( p=qa\), donc \( q=pa^{-1}\), ce qui signifie que \( q\in(p)\) et donc que \( J=I\).

	Cela prouve que \( (p)\) est un idéal maximum.

	Le fait que \( A/(p)\) soit un corps est maintenant la proposition~\ref{PROPooSHHWooCyZPPw}.
\end{proof}

\begin{example}
	L'anneau \( \eZ\) est principal parce qu'il est intègre et que ses seuls idéaux sont les \( n\eZ\) qui sont principaux : \( n\eZ\) est engendré par \( n\).
\end{example}

\begin{example}[Les idéaux de \( \eZ/n\eZ\)]       \label{EXooCJRPooYkWdyr}

	Les idéaux de \( \eZ/n\eZ\) sont principaux, mais l'anneau \( \eZ/n\eZ\) n'est pas principal lorsque \( n\) n'est pas premier. Nous allons voir ça.

	\begin{subproof}
		\spitem[Les idéaux de \( \eZ/n\eZ\) sont principaux]
		Soit un idéal \( S\) dans \( \eZ/n\eZ\). Nous considérons la projection canonique \( \phi\colon \eZ\to \eZ/n\eZ\). La proposition~\ref{PropIJJIdsousphi} dit que  \( S=\phi(J)\) où \( J\) est un idéal de \( \eZ\) contenant \( n\eZ\). Mais le corolaire~\ref{CORooLINXooBlUKPG} nous dit qu'alors \( J=m\eZ\) pour un certain \( m\). Pour que \( m\eZ\) contienne \( n\eZ\), il faut que \( m\) divise \( n\).

		Bref, \( S=\phi(m\eZ)\) avec \( m\divides n\). Nous montrons maintenant que \( S\) est engendré par \( [m]_n\). D'abord, l'élément \( [m]_n\) est bien dans \( \phi(m\eZ)\). Ensuite un élément de \( \phi(m\eZ)\) est de la forme
		\begin{equation}
			[km]_n=k[m]_n\in ([m]_n).
		\end{equation}
		Donc \( S\subset ([m]_n)\). Et l'inclusion dans l'autre sens est tout aussi immédiate : un élément de \( ([m]_n)\) est de la forme
		\begin{equation}
			k[m]_n=[km]_n=\phi(km)\in \phi(m\eZ).
		\end{equation}

		\spitem[Si \( n\) n'est pas premier, \( \eZ/n\eZ\) n'est pas principal]
		Le fait est que lorsque \( n\) n'est pas premier, \( \eZ/n\eZ\) n'est pas intègre (corolaire~\ref{CorZnInternprem}).

		\spitem[Moralité]
		Un anneau comme \( \eZ/6\eZ\) est un anneau dont tous les idéaux sont principaux, mais qui n'est pas principal.
	\end{subproof}
\end{example}

\begin{example}
	Nous verrons dans la proposition~\ref{PROPooVWRPooGQMenV} que l'anneau des fonctions holomorphes sur un compact de \( \eC\) est principal.
\end{example}

\begin{definition}      \label{DEFooXSPFooPumQSy}
	Nous disons que deux éléments d'un anneau principal sont \defe{premiers entre eux}{premier!deux éléments d'un anneau principal} si leur PGCD est \( 1\).
\end{definition}

\begin{theorem}\index{théorème!chinois!anneau principal}        \label{ThofPXwiM}
	Si \( A\) est un anneau principal et si \( p\) et \( q\) sont premiers entre eux dans \( A\), alors on a l'isomorphisme d'anneaux
	\begin{equation}
		A/pqA\simeq A/pA\times A/qA.
	\end{equation}
\end{theorem}
% TODO : trouver une preuve. Je parie que recopier la même que celle dans Z fonctionne très bien.

%---------------------------------------------------------------------------------------------------------------------------
\subsection{Autour du théorème de Bézout}
%---------------------------------------------------------------------------------------------------------------------------

Rappel de notations : si \( A\) est un anneau et si \( p\in A\), nous notons \( (p)\) l'idéal engendré par \( p\), c'est à dire \( pA\). C'est la définition \ref{DefSKTooOTauAR}.

\begin{lemma}[\cite{BIBooGIDYooKaacXf}]         \label{LEMooARNUooXqrJGa}
	Soient un anneau principal\footnote{Définition \ref{DEFooGWOZooXzUlhK}.} \( A\) ainsi que \( a,b\in A\).
	\begin{enumerate}
		\item
		      \( d\) est PGCD de \( \{ a,b \}\) si et seulement si \( (d)=(a)+(b)\).
		\item
		      \( m\) est PPCM de \( \{ a,b \}\) si et seulement si \( (m)=(a)\cap (b)\).
	\end{enumerate}
\end{lemma}

\begin{theorem}[\cite{XPXxPl,BIBooGIDYooKaacXf}]
	Soient un anneau principal et \( S\subset A\) non vide.
	\begin{enumerate}
		\item
		      \( d\) est un PGCD de \( S\) si et seulement si il génère l'idéal engendré par \( S\).
		\item
		      \( m\) est PPCM de \( S\) si et seulement si \( (m)=\bigcap_{s\in S}(s)\).
	\end{enumerate}
	Toute partie d'un anneau principal admet un PGCD et un PPCM\footnote{ Ce n'est pas aussi trivial que ça parce qu'il faut encore qu'il existe des éléments vérifiant les formules données.}.
\end{theorem}

\begin{probleme}
	La preuve est à revoir.     %TODOooOGGCooFRMrQt
\end{probleme}

\begin{proof}
	Puisque l'anneau \( A\) est principal, tous ses idéaux sont principaux et donc engendrés par un seul élément. En particulier il existe \( \delta,\mu\in A\) tels que
	\begin{subequations}
		\begin{align}
			(\delta) & =  \sum_{s\in S}(s)    \\
			(\mu)    & =  \bigcap_{s\in S}(s)
		\end{align}
	\end{subequations}
	\begin{subproof}
		\spitem[PGCD]
		Montrons ce que \( \delta\) est un PGCD de \( S\). Pour tout \( x\in S\), nous avons \( (x)\subset (\delta)\), et donc \( \delta\divides x\). Par ailleurs si \( d\divides x\) pour tout \( x\in S\), nous avons \( (x)\subset (d)\) et donc
		\begin{equation}
			\sum_{x\in S}(x)\subset (d),
		\end{equation}
		puis \( (\delta)\subset (d)\) et finalement \( d\divides \delta\).
		\spitem[PPCM]
		Si \( x\in S\) nous avons \( (\mu)\subset (x)\) et donc \( x\divides \mu\). D'autre part si \( x\divides m\) pour tout \( x\in S\), alors \( (m)\subset (x)\) et donc \( (m)\subset(\mu)\), finalement \( \mu\divides m\).
	\end{subproof}
\end{proof}

\begin{corollary}[Théorème de Bézout\cite{XPXxPl}]\index{Bézout!anneau principal}\label{CorimHyXy}
	Soit un anneau principal \( A\). Deux éléments \( a,b\in A\) sont premiers entre eux si et seulement si il existe un couple \( (u, v)\in A^2 \) tel que
	\begin{equation}
		ua+vb=1.
	\end{equation}
	À la place de \( 1\) on aurait pu écrire n'importe quel inversible.
\end{corollary}
\index{anneau!principal}

\begin{proof}
	Pour cette preuve, nous allons écrire \( \pgcd(a,b)\) l'ensemble de PGCD de \( a\) et \( b\), c'est-à-dire la classe d'association d'un PGCD.

	Si \( a\) et \( b\) sont premiers entre eux, alors
	\begin{equation}
		1\in\pgcd(a,b)=\sum_{x=a,b}(x)=(a)+(b).
	\end{equation}

	À l'inverse, si nous avons \( ua+vb=1\), alors \( 1\in (a)+(b)\), mais puisque \( (a)+(b)\) est un idéal principal, \( (1)=(a)+(b)\) et donc \( 1\in \pgcd(a,b)\).
\end{proof}

Le lemme de Gauss est une application immédiate du théorème de Bézout. Il y aura aussi un lemme de Gauss à propos de polynômes (lemme~\ref{LemEfdkZw}), et une généralisation directe au théorème de Gauss, théorème~\ref{ThoLLgIsig}.
\begin{lemma}[Lemme de Gauss\cite{BIBooEPIDooKerHPs}]    \label{LemSdnZNX}
	Soit \( A\) un anneau principal et \( a,b,c\in A\) tels que \( a\) divise \( bc\). Si \( a\) est premier avec \( c\), alors \( a\) divise \( b\).
\end{lemma}
\index{lemme!Gauss!dans un anneau principal}

\begin{proof}
	Comme \( a\) est premier avec \( c\), nous avons \( \pgcd(a,c)=1\) et Bézout (\ref{ThoBuNjam}) nous donne donc \( s,t\in A\) tels que \( sa+tc=1\). En multipliant par \( b\),
	\begin{equation}
		sab+tbc=b.
	\end{equation}
	Mais les deux termes du membre de gauche sont multiples de \( a\) parce que \( a\) divise \( bc\). Par conséquent \( b\) est somme de deux multiples de \( a\) et donc est multiple de \( a\).
\end{proof}
Un cas usuel d'utilisation est le cas de \( A=\eN^*\).

%---------------------------------------------------------------------------------------------------------------------------
\subsection{Idéal premier}
%---------------------------------------------------------------------------------------------------------------------------

\begin{proposition}[\cite{ooJHFCooSbHtEC,MonCerveau}, thème \ref{THEMEooVIQIooOcFAQS}]     \label{PROPooZICGooNmblhl}
	Soit un anneau principal \( A\) et un élément \( p\neq 0\) dans \( A\). Nous avons équivalence de :
	\begin{enumerate}
		\item   \label{ITEMooBTEAooWlFUTX}
		      \( (p)\) est un idéal premier,
		\item   \label{ITEMooKQRMooBNPDMX}
		      \( p\) est un élément premier,
		\item   \label{ITEMooZYYJooCWiBhL}
		      \( p\) est un élément irréductible,
		\item   \label{ITEMooHPAIooYoQzqD}
		      \( (p)\) est un idéal maximum propre\quext{Ce «propre» n'est pas dans l'énoncé sur Wikipédia. Je ne comprends pas pourquoi, et j'ai posé la question sur la page de discussion.\\\url{https://fr.wikipedia.org/wiki/Discussion:Idéal_premier}}.
	\end{enumerate}
\end{proposition}

\begin{proof}
	En plusieurs implications.
	\begin{subproof}
		\spitem[\ref{ITEMooBTEAooWlFUTX} implique \ref{ITEMooKQRMooBNPDMX}]
		En plusieurs points.
		\begin{itemize}
			\item La condition \( p\neq 0\) est dans les hypothèses de la proposition.
			\item Si \( p\) était inversible, nous aurions \( (p)=A\) et donc pas que \( (p)\) est un idéal premier.
			\item Soient \( a,b\in A\) tels que \( p\divides ab\). En particulier, \( (ab)\in (p)\). Mais comme \( (p)\) est un idéal premier, cela implique soit \( a\in (p)\) soit \( b\in (p)\). Donc soit \( p\) divise \( a\) soit \( p\) divise \( b\).
		\end{itemize}
		\spitem[\ref{ITEMooKQRMooBNPDMX} implique \ref{ITEMooZYYJooCWiBhL}]
		Un anneau principal est intègre; c'est dans la définition \ref{DEFooGWOZooXzUlhK}. Dans un anneau intègre, tout élément premier est irréductible, c'est la proposition \ref{PROPooWMNPooZdvOBt}.
		\spitem[\ref{ITEMooZYYJooCWiBhL} implique \ref{ITEMooHPAIooYoQzqD}]
		Soit un idéal \( I\) contenant \( (p)\). Puisque \( A\) est principal, \( I\) est engendré par un seul élément; soit \( I=(a)\). Vu que \( p\in I\), l'élément \( a\) divise \( p\). Mais comme \( p\) est un élément premier, \( a\divides p\) implique \( a=p\) ou \( a=1\). Dans le premier cas, \( I=(a)=(p)\), et dans le second cas, \( I=(a)=(1)=A\). Donc \( (p)\) est bien un idéal maximum.

		De plus l'idéal \( (p)\) est propre. En effet avoir \( (p)=A\) dirait en particulier que \( 1\in (p)\), c'est-à-dire qu'il existe \( x\in A\) tel que \( xp=1\). Or \( p\) étant irréductible, il est non inversible.
		\spitem[\ref{ITEMooHPAIooYoQzqD} implique \ref{ITEMooBTEAooWlFUTX}]
		C'est la proposition \ref{PROPooRUQKooIfbnQX}\ref{ITEMooTFFQooOUajFw}.
	\end{subproof}
\end{proof}

Un exemple d'élément premier non irréductible est \( [4]_6\) dans l'anneau non principal \( \eZ/6\eZ\). Voir \ref{NORMooAXOKooAQMXoB} et le lemme \ref{LEMooZSELooGOFEIz}.

%---------------------------------------------------------------------------------------------------------------------------
\subsection{Anneau noethérien}
%---------------------------------------------------------------------------------------------------------------------------

\begin{definition}      \label{DEFooPWMHooCnrQuJ}
	Un anneau est dit \defe{noethérien}{anneau!noethérien} si toute suite croissante d'idéaux est stationnaire (à partir d'un certain rang).
\end{definition}

Montrer que tout anneau principal est noethérien est le premier pas pour montrer que tout anneau principal est factoriel.

\begin{lemma}       \label{LEMooHQPVooTfkhRV}
	Tout anneau principal\footnote{Définition \ref{DEFooGWOZooXzUlhK}.} est noethérien.
\end{lemma}

\begin{proof}
	Soit \( (J_n)\) une suite croissante d'idéaux et \( J\) la réunion. L'ensemble \( J\) est encore un idéal parce que les \( J_i\) sont emboités. Étant donné que l'idéal est principal nous pouvons prendre \( a\in J\) tel que \( J=(a)\). Il existe \( N\) tel que \( a\in J_N\). Alors pour tout \( n\geq N\) nous avons
	\begin{equation}
		J\subset J_N\subset J_n\subset J.
	\end{equation}
	La première inclusion est le fait que \( J=(a)\) et \( a\in J_N\). La seconde est la croissance des idéaux et la troisième est le fait que \( J\) est une union. Par conséquent pour tout \( n\geq N\) nous avons \( J_N=J_n=J\). La suite est par conséquent stationnaire.
\end{proof}

\begin{example}
	Il y a moyen d'avoir un anneau noetherien non principal. C'est le cas de \( \eZ/6\eZ\) dont nous parlerons dans \ref{LEMooZSELooGOFEIz}.
\end{example}

\begin{theorem}[\cite{FSwlnf}]      \label{THOooANCAooBChmwp}
	Tout anneau principal est factoriel.
\end{theorem}

\begin{example}[\( \eZ\lbrack i\sqrt{ 5 }\rbrack\) n'est ni factoriel ni principal]     \label{EXooYCTDooGXAjGC}
	Puisque \( (i\sqrt{ 5 })^2=-5\), les éléments de \( \eZ[i\sqrt{ 5 }]\) sont les éléments de \( \eC\) de la forme \( a+bi\sqrt{ 5 }\) avec \( a,b\in \eZ\). Nous définissons la \defe{norme}{norme!sur \( \eZ[i\sqrt{ 5 }]\)} sur \( \eZ[i\sqrt{ 5 }]\) par\footnote{C'est le carré de la norme usuelle, mais c'est l'usage dans le milieu.}
	\begin{equation}
		\begin{aligned}
			N\colon \eZ[i\sqrt{ 5 }] & \to \eN          \\
			z                        & \mapsto z\bar z.
		\end{aligned}
	\end{equation}
	Le fait que ce soit à valeurs dans \( \eN\) est un simple calcul :
	\begin{equation}
		N(x+iy\sqrt{ 5 })=(x+iy\sqrt{ 5 })(x-iy\sqrt{ 5 })=x^2+5y^2.
	\end{equation}
	De plus \( N\) est multiplicative : \( N(z_1z_2)=N(z_1)N(z_2)\).

	Nous pouvons maintenant déterminer les inversibles de \( \eZ[i\sqrt{ 5 }]\). Si \( \alpha\) est inversible, alors il existe \( \beta\) tel que \( \alpha\beta=1\). Au niveau de la norme,
	\begin{equation}
		N(\alpha)N(\beta)=1,
	\end{equation}
	ce qui implique que \( N(\alpha)=1\). Or l'équation \( x^2+5y^2=1\) dans \( \eN\) donne \( y=0\), \( x=\pm 1\).

	Au final, les inversibles de \( \eZ[i\sqrt{ 5 }]\) sont \( \pm 1\).

	L'anneau \( \eZ[i\sqrt{ 5 }]\) n'est alors pas factoriel (définition~\ref{DEFooVCATooPJGWKq}) parce que
	\begin{equation}
		2\times 3=(1+i\sqrt{ 5 })(1-i\sqrt{ 5 }).
	\end{equation}
	Cela donne deux décompositions du nombre \( 6\) en produit d'éléments non associés\footnote{Définition~\ref{DefrXUixs}.} (\( 2\) n'est associé qu'à \( 2\) et \( -2\)) parce que les inversibles sont \( 1\) et \( -1\).

	Le fait que \( \eZ[i\sqrt{ 5 }]\) ne soit pas factoriel implique qu'il ne soit pas principal, théorème~\ref{THOooANCAooBChmwp}.
\end{example}

%+++++++++++++++++++++++++++++++++++++++++++++++++++++++++++++++++++++++++++++++++++++++++++++++++++++++++++++++++++++++++++
\section{Anneau \texorpdfstring{\(  \eZ/6\eZ\)}{Z/6Z}}
%+++++++++++++++++++++++++++++++++++++++++++++++++++++++++++++++++++++++++++++++++++++++++++++++++++++++++++++++++++++++++++
\label{SECooSWGKooEeOZTO}

Nous allons donner quelques propriétés de cet anneau, et en particulier voir que dans cet anneau non intègre, la notion d'élément irréductible n'est pas très intéressante.

Voici pour commencer un calcul de la table de multiplication de \( A=\eZ/6\eZ\). Pour les multiples de (par exemple) \( [4]_6\) nous écrivons
\begin{equation}
	1\times [4]_6=[4_6]
\end{equation}
et ensuite
\begin{equation}
	2\times [4]_6=[8]_6=[2]_6,
\end{equation}
puis
\begin{equation}
	3\times [4]_6=[2+4]_6=[6]_6=[0]_6,
\end{equation}
et caetera. Le résultat est :
\begin{equation}
	\begin{array}{c|c|c|c|c|c|c}
		\times & [0]_6 & [1]_6 & [2]_6 & [3]_6 & [4]_6 & [5]_6 \\
		\hline\hline
		[0]_6  & 0     & 0     & 0     & 0     & 0     & 0     \\
		\hline
		[1]_6  & 0     & 1     & 2     & 3     & 4     & 5     \\
		\hline
		[2]_6  & 0     & 2     & 4     & 0     & 2     & 4     \\
		\hline
		[3]_6  & 0     & 3     & 0     & 3     & 0     & 3     \\
		\hline
		[4]_6  & 0     & 4     & 2     & 0     & 4     & 2     \\
		\hline
		[5]_6  & 0     & 5     & 4     & 3     & 2     & 1     \\
		\hline
	\end{array}
\end{equation}
Pour ne pas alourdir, nous n'avons pas écrit \( [x]_6\) partout au lieu de \( x\).

\begin{normaltext}[Inversibles]
	Les éléments inversibles de \( \eZ/6\eZ\) sont ceux qui ont un \( [1]_6\) dans leur table de multiplication. Ce sont donc
	\begin{equation}
		U(\eZ/6\eZ)=\big\{ [1]_6,[5]_6 \big\}.
	\end{equation}
\end{normaltext}

\begin{normaltext}[Diviseurs de zéro]
	Les diviseurs de zéro sont ceux qui ont un \( [0]_6\) dans leur table de multiplication, c'est-à-dire
	\begin{equation}
		\big\{ [2]_6,[3]_6,[4]_6 \big\}.
	\end{equation}
\end{normaltext}

\begin{normaltext}[Irréductibles]
	Les irréductibles sont ceux qui ne sont ni inversibles ni produit de deux éléments non inversibles. Les non inversibles sont :
	\begin{equation}
		\big\{ [0]_6,[2]_6,[3]_6,[4]_6 \}.
	\end{equation}
	Ils sont candidats à être irréductibles. Les produits de ces éléments (on oublie les crochets) sont :
	\begin{subequations}
		\begin{align}
			2\times 2 & =4  \\
			2\times 3 & =0  \\
			2\times 4 & =2  \\
			3\times 3 & =3  \\
			3\times 4 & =0  \\
			4\times 4 & =4.
		\end{align}
	\end{subequations}
	Donc \( [0]_6\), \( [2]_6\), \( [3]_6\) et \( [4]_6\) ne sont plus candidats à être irréductible. Bref, il ne reste aucun candidats.

	L'anneau \( \eZ/6\eZ\) n'a aucun élément irréductible.
\end{normaltext}

\begin{normaltext}[Éléments premiers]       \label{NORMooAXOKooAQMXoB}
	Les éléments non nuls et non inversibles sont \( 2\), \( 3\) et \( 4\).
	\begin{subproof}
		\spitem[Pour \( 2\)]
		L'élément \( [2]_6\) divise \( 2\), \( 4\) et \( 0\).
		\begin{itemize}
			\item Les \( (a,b)\) tels que \( ab=2\) sont : \( (1,2)\), \( (2,4)\) et \( (5,4)\). L'élément \( 2\) divise donc toujours \( a\) ou \( b\).
			\item Les \( (a,b)\) tels que \( ab=4\) sont : \( (1,4)\), \( (2,5)\) et \( (4,4)\). L'élément \( 2\) divise donc toujours \( a\) ou \( b\).
			\item Les \( (a,b)\) tels que \( ab=0\) sont : \( (0,x)\), \( (3,2)\) et \( (4,3)\). L'élément \( 2\) divise donc toujours \( a\) ou \( b\). En particulier, \( [2]_6\) divise \( [0]_6\); c'est important.
		\end{itemize}
		Donc \( [2]_6\) est un élément premier.
		\spitem[Pour \( 3\)]
		L'élément \( [3]_6\) divise \( 3\) et \( 0\).
		\begin{itemize}
			\item Les \( (a,b)\) tels que \( ab=3\) sont : \( (1,3)\) et \( (3,5)\).             L'élément \( 3\) divise donc toujours \( a\) ou \( b\).
			\item Les \( (a,b)\) tels que \( ab=0\) sont : \( (0,x)\), \( (3,2)\) et \( (4,3)\). L'élément \( 3\) divise donc toujours \( a\) ou \( b\).
		\end{itemize}
		Donc \( [3]_6\) est un élément premier.
		L'élément \( [4]_6\) divise \( 4\), \( 2\) et \( 0\).
		\begin{itemize}
			\item Les \( (a,b)\) tels que \( ab=4\) sont : \( (1,4)\), \( (2,5)\) et \( (4,4)\). L'élément \( 4\) divise donc toujours \( a\) ou \( b\).
			\item Les \( (a,b)\) tels que \( ab=2\) sont : \( (1,2)\), \( (2,4)\) et \( (5,4)\). L'élément \( 4\) divise donc toujours \( a\) ou \( b\).
			\item Les \( (a,b)\) tels que \( ab=0\) sont : \( (0,x)\), \( (3,2)\) et \( (4,3)\). L'élément \( 4\) divise donc toujours \( a\) ou \( b\).
		\end{itemize}
		Donc \( [4]_6\) est un élément premier.
	\end{subproof}
	Au final, les éléments premiers dans \( \eZ/6\eZ\) sont
	\begin{equation}
		\big\{ [2]_6, [3]_6, [4]_6  \big\}.
	\end{equation}
\end{normaltext}

Vous noterez que \( \eZ/6\eZ\) a des éléments premiers non irréductibles. Cela est un contre-exemple à la proposition \ref{PROPooZICGooNmblhl} dans le cas d'un anneau non-intègre.

\begin{lemma}[\cite{MonCerveau}]    \label{LEMooZSELooGOFEIz}
	L'anneau \( \eZ/6\eZ\) est noetherien, mais ni intègre ni principal\footnote{Toutes les définitions dans le thème \ref{THEMEooVIQIooOcFAQS}.}.
\end{lemma}

\begin{proof}
	Comme c'est un anneau fini, toute suite croissante de quoi que ce soit devient stationnaire; donc \( \eZ/6\eZ\) est noetherien.

	Puisque \( \eZ/6\eZ\) a des diviseurs de zéro, il n'est pas intègre. Et s'il n'est pas intègre, il n'est pas factoriel non plus.
\end{proof}
