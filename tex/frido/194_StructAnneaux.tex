% This is part of Mes notes de mathématique
% Copyright (c) 2011-2020, 2022-2023
%   Laurent Claessens
% See the file fdl-1.3.txt for copying conditions.


\begin{example} \label{ExeDufyZI}
	Prouvons que \( \eZ[i\sqrt{2}]\) est un anneau euclidien. Pour cela nous démontrons que
	\begin{equation}    \label{EqOZUIooZGmHWl}
		\begin{aligned}
			N\colon \eZ[i\sqrt{2}] & \to \eN          \\
			a+bi\sqrt{2}           & \mapsto a^2+2b^2
		\end{aligned}
	\end{equation}
	est un stathme euclidien.

	Soient \( z=a+bi\sqrt{2}\), \( t=a'+b'i\sqrt{2}\). Nous cherchons \( q\) et \( r\) tels que la division euclidienne s'écrive \( z=qt+r\). Soient \( \alpha,\beta\in \eQ\) tels que
	\begin{equation}
		\frac{ z }{ t }=\alpha+\beta i\sqrt{2}.
	\end{equation}
	Nous désignons par \( \alpha+\epsilon_1\) et \( \beta+\epsilon_2\) les entiers les plus proches de \( \alpha\) et \( \beta\). Nous avons \( | \epsilon_1 |,| \epsilon_2 |\leq \frac{ 1 }{2}\). Nous posons alors naturellement
	\begin{equation}
		q=(\alpha+\epsilon_1)+(\beta+\epsilon_2)i\sqrt{2}
	\end{equation}
	et nous calculons \( r=z-qt\) :
	\begin{equation}
		2b'\epsilon_2-a'\epsilon_1+i\sqrt{2}\big( \epsilon_1b'-a'\epsilon_2 \big).
	\end{equation}
	Nous trouvons
	\begin{equation}
		N(r)=a'^2\epsilon_1^2+4b'^2\epsilon_2^2+2a'^2\epsilon_1^2+2b'^2\epsilon_2^2\leq \frac{ 3 }{ 4 }a'^2+\frac{ 3 }{2}b'^2.
	\end{equation}
	D'autre part \( N(t)=a'^2+2b'^2\), et nous avons donc bien \( N(r)<N(t)\).

	En ce qui concerne la seconde propriété du stathme, un petit calcul montre que
	\begin{equation}
		N(zt)=(a^2+2b^2)(a'^2+2b'^2),
	\end{equation}
	et tant que \( t\neq 0\) nous avons bien \( N(zt)>N(z)\).
\end{example}

Notons en particulier que \( \eZ[i\sqrt{2}]\) est factoriel et principal.

\begin{example}[Décomposition en facteurs irréductibles dans \( {\eZ[i\sqrt{2}]}\)] \label{ExluqIkE}
	Les éléments inversibles de \( \eZ[i\sqrt{2}]\) sont \( \pm 1\), donc deux éléments \( a\) et \( b\) sont associés (définition~\ref{DefrXUixs}) si et seulement si \( a=\pm b\).

	De plus si \( p\) est irréductible\footnote{Définition \ref{DeirredBDhQfA}}, alors \( -p\) est irréductible. Les éléments irréductibles de \( \eZ[i\sqrt{2}]\) arrivent donc par pairs d'éléments associés. Soit \( \{ p_i \}\) une sélection de un élément irréductible parmi chaque paire. Tout élément \( x\) de \( \eZ[i\sqrt{2}]\) peut alors être écrit \( x=\pm p_1^{\alpha_1}\ldots p_n^{\alpha_n}\). Cette écriture va être pratique pour comparer des décompositions en facteurs irréductibles d'éléments.
\end{example}

Le lemme suivant fait en pratique partie de l'exemple~\ref{ExmuQisZU}, mais nous l'isolons pour plus de clarté\footnote{Merci à \href{http://fr.wikipedia.org/wiki/Utilisateur:Marvoir}{Marvoir} pour m'avoir souligné le manque.}.
\begin{lemma}       \label{LemTScCIv}
	Si \( a\) et \( b\) sont deux éléments premiers entre eux de \( \eZ[i\sqrt{2}]\), et si il existe \( y \in  \eZ[i\sqrt{2}]\) tel que \( ab=y^3\), alors \( a\) et \( b\) sont des cubes (dans \( \eZ[i\sqrt{2}]\)).
\end{lemma}

\begin{proof}
	D'après l'exemple~\ref{ExluqIkE} nous pouvons écrire
	\begin{subequations}
		\begin{align}
			y & =\pm p_1^{\sigma_1}\ldots p_n^{\sigma_n} \\
			a & =\pm p_1^{\alpha_1}\ldots p_n^{\alpha_n} \\
			b & =\pm p_1^{\beta_1}\ldots p_n^{\beta_n}
		\end{align}
	\end{subequations}
	où les \( p_i\) sont les irréductibles de \( \eZ[i\sqrt{2}]\) «modulo \( \pm 1\)» au sens où la liste des irréductibles est \( \{ p_i \}\cup\{ -p_i \}\) (union disjointe). Étant donné que \( a\) et \( b\) sont premiers entre eux, \( \alpha_i\) et \( \beta_i\) ne peuvent pas être non nuls en même temps alors que leur somme doit faire \( 3\sigma_i\). Nous avons donc pour chaque \( i\) soit \( \alpha_i=3\sigma_i\) soit \( \beta_i=3\sigma_i\) (et bien entendu si \( \sigma_i=0\) alors \( \alpha_i=\beta_i=0\)).

	Étant donné que \( \pm 1\) sont également deux cubes, \( a\) et \( b\) sont bien des cubes.

	Notons que nous avons utilisé de façon capitale le fait que \( \eZ[i\sqrt{2}]\) était factoriel.
\end{proof}

%---------------------------------------------------------------------------------------------------------------------------
\subsection{Équations diophantiennes}
%---------------------------------------------------------------------------------------------------------------------------
%TODO : il y a une équation diophantienne qui semple pas mal ici : http://fr.wikipedia.org/wiki/Entier_quadratique#x2_.2B_5.y2_.3D_p

\begin{example} \label{ExZPVFooPpdKJc}
	L'équation diophantienne
	\begin{equation}
		x^2=3y^2+8
	\end{equation}
	n'a pas de solution. En effet si nous prenons l'équation modulo \( 3\) nous obtenons
	\begin{equation}
		[x^2]_3=[3y^2+8]_3=[8]_3=[2]_3.
	\end{equation}
	Or dans \( \eZ/3\eZ\), aucun carré n'est égal à deux : \( 0^2=0\neq 2\), \( 1^2=1\neq 2\) et \( 2^2=4=1\neq 2\).
\end{example}

\begin{example}     \label{ExmuQisZU}
	Résolvons l'équation diophantienne\index{équation!diophantienne}
	\begin{equation}
		x^2+2=y^3.
	\end{equation}
	Une première remarque est que \( x\) doit être impair. En effet si \( x=2k\), nous devons avoir \( y^3\) pair. Mais un cube pair est divisible par \( 8\). Donc \( y^3=8l\) pour un certain \( l\). L'équation devient \( 4k^2+2=8l\), c'est-à-dire \( 2k^2+1=4l\). Le membre de gauche est impair tandis que celui de droite est pair. Impossible.

	Nous pouvons écrire l'équation sous la forme \( x^2+2=(x+i\sqrt{2})(x-i\sqrt{2})\). Et nous considérons \( \eZ[i\sqrt{2}]\) muni de son stathme \( N\) donné par \eqref{EqOZUIooZGmHWl}.

	L'élément \( i\sqrt{2}\) est irréductible parce que \( N(i\sqrt{2})=2\), et si nous avions \( i\sqrt{2}=pq\), alors nous aurions \( N(p)N(q)=2\), ce qui n'est possible que si \( N(p)\) ou \( N(q)\) vaut \( 1\).

	Nous prouvons maintenant que les éléments \( x+i\sqrt{2}\) et \( x-i\sqrt{2}\) sont premiers entre eux. Supposons que \( d\) soit un diviseur commun; alors il divise aussi la somme et la différence. Donc \( d\) divise à la fois \( 2x\) et \( 2i\sqrt{2}\).

	Étant donné que \( i\sqrt{2}\) est irréductible et que \( 2i\sqrt{2}=(-i\sqrt{2})^3\), les diviseurs de \( 2i\sqrt{2}\) sont les puissances de \( (-i\sqrt{2})\). Alors nous devrions avoir \( d=(i\sqrt{2})^{\beta}\) et donc
	\begin{equation}
		x=(i\sqrt{2})^{\beta}q
	\end{equation}
	pour un certain \( q\in\eZ[i\sqrt{2}]\). Dans ce cas nous avons \( N(x)=2^{\beta}N(q)\), mais nous avons déjà précisé que \( x\) ne pouvait pas être pair, donc \( \beta=0\) et nous avons \( d=1\).

	Comme les nombres \( x\pm i\sqrt{2}\) sont premiers entre eux et que leur produit doit être un cube, ils doivent être séparément des cubes (lemme~\ref{LemTScCIv}). Nous devons donc résoudre séparément \( x\pm i\sqrt{2}=y^3\).

	Cherchons les \( x\) et \( y\) entiers tels que \( x+i\sqrt{2}=y^3\). Si nous posons \( z=a+bi\sqrt{2}\), il suffit de calculer \( z^3\) :
	\begin{verbatim}
----------------------------------------------------------------------
| Sage Version 4.8, Release Date: 2012-01-20                         |
| Type notebook() for the GUI, and license() for information.        |
----------------------------------------------------------------------
sage: var('a,b')
(a, b)
sage: z=a+I*sqrt(2)*b
sage: (z**3).expand()
3*I*sqrt(2)*a^2*b - 2*I*sqrt(2)*b^3 + a^3 - 6*a*b^2
    \end{verbatim}
	En identifiant cela à \( x+i\sqrt{2}\) nous trouvons le système
	\begin{subequations}
		\begin{numcases}{}
			x=a^3-6ab^2\\
			1=3a^2b-2b^3
		\end{numcases}
	\end{subequations}
	où, nous le rappelons, \( x\), \( a\) et \( b\) sont des entiers. La seconde équation montre que \( b\) doit être inversible : \( b(3a^2-2b^2)=1\). Il y a donc les possibilités \( b=\pm 1\). Pour \( b=1\) l'équation devient \( 3a^2-2=1\), c'est-à-dire \( a=\pm 1\). Pour \( b=-1\) l'équation devient \( 3a^2-2=-1\), impossible. En conclusion les possibilités sont
	\begin{subequations}
		\begin{align}
			(x,z)=(-5,1+i\sqrt{2}) \\
			(x,z)=(5,-1+i\sqrt{2}) \\
		\end{align}
	\end{subequations}
	Le travail avec \( x-i\sqrt{2}\) donne les mêmes résultats.

	Les deux solutions de l'équation \( x^2+2=y^3\) sont alors \( (5,3)\) et \( (-5,3)\).
\end{example}

%---------------------------------------------------------------------------------------------------------------------------
\subsection{Triplets pythagoriciens et équation de Fermat pour \texorpdfstring{\(  n=4\)}{n=4}}
%---------------------------------------------------------------------------------------------------------------------------

\begin{definition}
	Les solutions entières (positives) de l'équation \( x^2+y^2=z^2\) sont appelés \defe{triplets pythagoriciens}{triplet!pythagoricien}.
\end{definition}

Ils donnent toutes les possibilités de triangles rectangles dont les côtés ont des longueurs entières.

\begin{definition}
	On dit qu'un triplet pythagoricien est \defe{primitif}{primitif!triplet pythagoricien} si les trois nombres sont premiers dans leur ensemble\footnote{Définition \ref{DefZHRXooNeWIcB}.}.
\end{definition}

Remarquons que c'est équivalent à montrer que les trois nombres sont premiers deux à deux: en effet, si deux parmi \( x\), \( y\) et \( z\) sont divisibles par un nombre, alors tous les trois sont divisibles par ce nombre\footnote{Parce que \( k\) et \( k^2\) ont les mêmes facteurs premiers.}, donc les nombres \( x\), \( y\) et \( z\) sont premiers deux à deux.

\begin{lemma}    \label{LemTripletsPythagoriciensPrimitifs}
	Dans un triplet pythagoricien primitif \( (x, y, z) \), on a toujours \( z\) impair et:
	\begin{itemize}
		\item
		      soit \( x\) impair et \( y\) pair;
		\item
		      soit \( x\) pair et \( y\) impair.
	\end{itemize}
\end{lemma}

\begin{proof}
	Remarquons que le fait d'imposer que le triplet soit primitif, interdit aux nombres \( x\) et \( y\) d'être pairs en même temps. En effet, si c'était le cas, alors \( x^2 \) et \( y^2 \) seraient aussi pairs, donc leur somme \( z^2 \) aussi, d'ou \( z\) serait pair et les trois nombres ne seraient pas premiers entre eux.

	Nous montrons à présent que les nombres \( x\) et \( y\) ne sont pas tous les deux impairs. Par l'absurde, si \( x=2a+1\), nous avons \( x^2=4a^2+4a+1\in [1]_4\); de la même manière,  \( y^2 \in [1]_4\). On en déduit alors que \( z^2=x^2+y^2\in [2]_4\). Le nombre \(  z^2\) est donc pair, donc \( z\) est pair : disons \( z=2c\). Alors, \( z^2=4c^2\in [0]_4\). Comme les classes modulo 4 sont disjointes, nous aboutissons à une contradiction.
\end{proof}

\begin{proposition}[Triplets pythagoriciens\cite{fJhCTE,HARRooBvzbXo}]  \label{PropXHMLooRnJKRi}
	Un triplet \( (x,y,z)\in(\eN^*)^3\) est solution de \( x^2+y^2=z^2\) si et seulement si il existe \( d\in \eN\) et \( u,v\in \eN^*\) premiers entre eux tels que
	\begin{subequations}        \label{subeqLVHFooVgWsFx}
		\begin{numcases}{}
			x=d(u^2-v^2)\\
			y=2duv\\
			z=d(u^2+v^2)
		\end{numcases}
	\end{subequations}
	ou
	\begin{subequations}    \label{SUBEQSooRQKBooCBsqYA}
		\begin{numcases}{}
			x=2duv\\
			y=d(u^2-v^2)\\
			z=d(u^2+v^2)
		\end{numcases}
	\end{subequations}
	La différence entre les deux est seulement d'inverser les rôles de \( x\) et \( y\).
\end{proposition}

\begin{proof}
	Montrons d'abord que les formules proposées sont bien des solutions; nous vérifions \eqref{subeqLVHFooVgWsFx} :
	\begin{equation}
		x^2+y^2=d^2(u^2-v^2)^2+4d^2u^2v^2=d^2(u^2+v^2)^2,
	\end{equation}
	qui correspond bien au \( z^2\) proposé.

	Une vérification du même style fonctionne pour \eqref{SUBEQSooRQKBooCBsqYA}.

	Nous allons maintenant prouver la réciproque : toute solution est d'une des deux formes proposées. Déterminer les triplets primitifs suffira parce que si \( (x,y,z)\) n'est pas une solution primitive, alors en posant \( k=\pgcd(x,y,z)\), le triplet \( \big( \frac{ x }{ k },\frac{ y }{ k },\frac{ z }{ k } \big)\) est primitif. Connaissant les triplets primitifs, nous obtenons tous les autres par simple multiplication.

	Soit donc \( (x,y,z)\) un triplet pythagoricien primitif. Sans perte de généralité\footnote{En échangeant les rôles de \( x\) et \( y\) ici, nous obtenons à la fin la seconde forme des solutions.}, grâce au lemme \ref{LemTripletsPythagoriciensPrimitifs}, nous pouvons supposer \( x\) pair tandis que \( y\) et \( z\) seront impairs. Comme \( x^2=(z+y)(z-y)\), nous avons
	\begin{equation}
		\left( \frac{ x }{2} \right)^2=\frac{1}{ 4 }(z+y)(z-y).
	\end{equation}
	Puisque \( z\) et \( y\) sont premiers entre eux, les nombres \( z+y\) et \( z-y\) sont également premiers entre eux\footnote{Si \( z-y=kn\) et \( z+y=km\), faisant la somme et la différence on voit que \( y\) et \( z\) sont divisibles par \( k\).}. Vu que $(z+y)(z-y)$ est divisible par \( 4\), soit \( z+y\) soit \( (z-y)\) est divisible par \( 4\). Pour fixer les idées nous supposons que c'est \( z+y\), et nous écrivons
	\begin{equation}
		\left( \frac{ x }{ 2 } \right)^2=\left( \frac{ z+y }{ 4 } \right)(z-y).
	\end{equation}
	Les facteurs premiers (qui arrivent au moins au carré) de \( (x/2)\) sont chacun soit dans \( (z+y)/4\) soit dans \( (z-y)\). Tout deux sont donc des carrés parfaits.  Nous posons
	\begin{equation}
		\begin{aligned}[]
			\frac{ z+y }{4}=u^2 & , & z-y=v^2.
		\end{aligned}
	\end{equation}
	Bien entendu \( u\) et \( v\) sont non nuls parce que nous avons exclu la possibilité de triplets dont un élément serait nul. Avec tout cela nous avons \( (x/2)^2=u^2v^2\) et donc \( x=2uv\) puis par somme et différence :
	\begin{subequations}
		\begin{numcases}{}
			x=2uv\\
			y=v^2-u^2\\
			z=u^2+v^2,
		\end{numcases}
	\end{subequations}
	ce qu'il fallait.
\end{proof}

\begin{remark}
	Les solutions dans lesquelles \( x\), \( y\) ou \( z\) sont nuls sont faciles à classer. La solution \( (1,0,1)\) n'est pas dans les formes proposées. En effet elle ne peut pas être de la première forme : avoir \( y=0\) demanderait qu'un nombre parmi \( d\), \( u\) et \( v\) soit nul, ce qui est interdit. La solution \( (1,0,1) \) ne peut pas non plus être de la seconde forme parce que \( x\) y est pair.
\end{remark}

\begin{proposition}[\cite{fJhCTE}]      \label{propFKKKooFYQcxE}
	Les équations \( x^4+y^4=z^2\) et \( x^2+y^4=z^4\) n'ont pas de solution dans \( (\eN^*)^3\).
\end{proposition}
\index{équation!diophantienne}

\begin{proof}
	Si la première équation n'a pas de solution, alors la seconde n'en
	a pas non plus parce que \( z^4\) est un carré. Nous nous
	concentrons donc sur l'équation \( x^4+y^4=z^2\) et nous supposons
	qu'il existe au moins une solution dans \( (\eN^*)^3\). Nous en choisissons une \( (x,y,z)\) avec \( z\) minimum (les \( z\) dans différentes solutions étant dans \( \eN\), il en existe forcément un minimum\footnote{Voir quelque chose comme le lemme~\ref{PropQEPoozLqOQ}.}). Du coup, les trois nombres \( x\), \( y\) et \( z\) sont premiers dans leur ensemble parce qu'une
	division par leur \( \pgcd\) donnerait une nouvelle solution qui
	contredirait la minimalité de \( z\).

	Nous posons \( x^4=\bar x^2\) et \( y^4=\bar y^2\). Ils vérifient
	l'équation \( \bar x^2+\bar y^2=z^2\) et par la proposition
	~\ref{PropXHMLooRnJKRi}, il existe \( u,v\in \eN^*\) premiers entre
	eux tels que, sans perte de généralité\footnote{En inversant les
		rôles de \( x\) et \( y\) au besoin.}, on ait
	\begin{subequations}
		\begin{numcases}{}
			\bar x=2uv\\
			\bar y=u^2-v^2\label{eqnFKKKooFYQcxE1}\\
			z=u^2+v^2.\label{eqnFKKKooFYQcxE2}
		\end{numcases}
	\end{subequations}
	Si \( u\) est pair, alors \( v\) est impair (et inversement) parce
	que \( \pgcd(u,v)=1\) Si \( u\) est pair, alors \( u=2a\) et \(
	v=2b+1\), ce qui donne \( \bar y=4a^2-4b^2-4b-1\in[-1]_4\). Or
	nous avons déjà vu qu'un carré est dans \( [0]_4\) ou dans \(
	[1]_4\). Il faut donc que \( u\) soit impair. Le lemme
	~\ref{LemTripletsPythagoriciensPrimitifs} implique alors que \( v\)
	soit pair.

	De l'équation~\ref{eqnFKKKooFYQcxE1}, nous en déduisons que \(
	v^2+\bar y=u^2\); de plus \( u^2\), \( v^2\) et \( \bar y\) sont
	premiers dans leur ensemble: en effet, \( u\) et \( v\) sont premiers
	entre eux, et si l'un parmi \( u^2\) et \( v^2\) a un facteur
	commun avec \( \bar y\), alors l'autre doit l'avoir aussi (dans
	une égalité \( a+b=c\), si deux des nombres ont un diviseur
	commun, le troisième l'a aussi). Comme \( \bar y=y^2\), le triplet
	\( (v,y,u)\) est un triplet pythagoricien primitif. Nous
	appliquons de nouveau la proposition~\ref{PropXHMLooRnJKRi}, en se
	souvenant que \( v\) est pair: il existe donc deux nombres \( r\) et
	\( s\) premiers entre eux tels que
	\begin{subequations} \label{eqnFKKKooFYQcxE3}
		\begin{numcases}{}
			v=2rs\\
			y=r^2-s^2\\
			u=r^2+s^2.
		\end{numcases}
	\end{subequations}
	Avec cela, \( \bar x=2uv=4rs(r^2+s^2)\). Remarquons que les trois
	nombres \( r\), \( s\) et \( r^2+s^2\) sont premiers entre
	eux dans leur ensemble; or, comme \( \bar x\) est un
	carré ces nombres doivent séparément être des carrés :
	\begin{subequations}
		\begin{numcases}{}
			r=\alpha^2\\
			s=\beta^2\\
			r^2+s^2=\gamma^2.
		\end{numcases}
	\end{subequations}
	En mettant les deux premiers dans la troisième équation, nous avons \( \alpha^4+\beta^4=\gamma^2\). Donc \( (\alpha^2,\beta^2,\gamma)\) est une solution. Nous allons montrer que \( \gamma<z\), ce qui terminera la preuve, puisque \( z\) était supposé minimal. Nous avons :
	\begin{align*}
		z & =u^2+v^2          &  & \text{par~\ref{eqnFKKKooFYQcxE2}} \\
		  & =r^2+s^2+4r^2s^2  &  & \text{par~\ref{eqnFKKKooFYQcxE3}} \\
		  & =\gamma^2+4r^2s^2                                        \\
		  & > \gamma^2,
	\end{align*}
	et a fortiori \( \gamma<z\).
\end{proof}

%+++++++++++++++++++++++++++++++++++++++++++++++++++++++++++++++++++++++++++++++++++++++++++++++++++++++++++++++++++++++++++
\section{Polynômes à coefficients dans un anneau commutatif}
%+++++++++++++++++++++++++++++++++++++++++++++++++++++++++++++++++++++++++++++++++++++++++++++++++++++++++++++++++++++++++++
\label{SECooVMABooVdhbPo}

\begin{lemma}       \label{LEMooXFMAooMBgIrN}
	Nous considérons un polynôme \( P\in A[X]\), et le quotient \( A[X]/(P)\). Pour tout polynôme \( Q\in A[X]\) nous avons les égalités
	\begin{equation}
		Q(\bar X)=\overline{ Q(X) }=\bar Q.
	\end{equation}
\end{lemma}

\begin{proof}
	Si \( Q=\sum_ka_kX^k\), alors par la linéarité de la prise de classes,
	\begin{equation}        \label{EQooXQRMooIPGFVM}
		\bar Q=\sum_ka_k\overline{ X^k }.
	\end{equation}
	Nous insistons sur le fait que cette égalité n'est rien d'autre que l'itération de la définition de la somme dans l'espace quotient : \( \bar x+\bar y=\overline{ x+y }\) ainsi que du produit \( k\bar x=\overline{ kx }\) (définition~\ref{PROPooGXMRooTcUGbi}). Toujours par définition du produit appliqué à l'élément \( \bar X\) nous avons \( (\bar X)^2=\overline{ X^2 }\); par récurrence \( \overline{ X^k }=\bar X^k\), et
	\begin{equation}
		\bar Q=\sum_ka_k\bar X^k=Q(\bar X).
	\end{equation}

	Le fait que \( \bar Q=\overline{ Q(X) }\) n'est rien d'autre que le fait que dans \( A[X]\) nous avons \( Q=Q(X)\), comme expliqué dans le lemme~\ref{LEMooGKWQooVOyeDX}.
\end{proof}

%---------------------------------------------------------------------------------------------------------------------------
\subsection{Monômes}
%---------------------------------------------------------------------------------------------------------------------------

\begin{normaltext}
	Les éléments de la forme \( \lambda X^k\) avec \( \lambda\in A\) et \( k\in\eN\) sont des \defe{monômes}{monôme}.

	Nous allons aussi considérer\nomenclature[A]{\( A_n[X]\)}{les polynômes à coefficients dans \( A\) et de degré inférieur ou égal à \( n\)}
	\begin{equation}
		A_n[X]=\{ P\in A[X]\tq \deg(P)\leq n \}.
	\end{equation}
	Cet ensemble est un sous-module libre.
\end{normaltext}

%---------------------------------------------------------------------------------------------------------------------------
\subsection{Évaluation}
%---------------------------------------------------------------------------------------------------------------------------

Soit \( P\in A[X]\). À priori, \( P\) n'est qu'une suite dans \( A\) indexée par \( \eN\).


Nous avons déjà défini son évaluation sur un élément \( \alpha\in A\) dans la définition \ref{DEFooNXKUooLrGeuh} :
\begin{equation}
	P(\alpha)=\sum_ka_k\alpha^k.
\end{equation}
Cette somme est toujours finie.

\begin{normaltext}      \label{NORMooQFTJooLBcPxl}
	L'ensemble \( A[X]\) est une algèbre et donc un espace vectoriel. Il possède un unique élément nul qui est celui dont tous les coefficients sont nuls; cela est immédiat par la construction en tant que suites presque nulles.
\end{normaltext}

Il n'y a à priori pas équivalence entre le fait d'être un polynôme nul et le fait de s'évaluer à zéro sur tous les éléments de \( A\). Cela sera discuté dans le théorème~\ref{ThoLXTooNaUAKR} et l'exemple~\ref{exVQBooBMPLkD}.

\begin{definition}      \label{DEFooRFBFooKCXQsv}
	Soient un anneau \( A\) et un anneau \( B\) qui contient \( A\) (comme sous-anneau). Pour \( \alpha\in B\) nous définissons \( A[\alpha]_B\) comme étant l'intersection de tous les sous-anneaux de \( B\) contenant \( A\).
\end{definition}
Comme dit plus haut, nous nous permettons d'écrire \( A[\alpha]\) sans préciser \( B\) lorsque ce dernier sera clair dans le contexte.

\begin{proposition}     \label{PROPooPMNSooOkHOxJ}
	Soient un anneau \( A\) et un anneau \( B\) qui contient \( A\) (comme sous-anneau). Pour tout \( \alpha\in B\) nous avons
	\begin{equation}
		A[\alpha]=\{ P(\alpha)\tq P\in A[X] \}
	\end{equation}
	où encore une fois, \( P(\alpha)\) est calculé dans \( B\); le contexte est clair là-dessus.
\end{proposition}

\begin{proof}
	Si \( A'\) est un sous-anneau de \( B\) contenant \( A\) et \( \alpha\), alors \( A'\) contient tous les \( P(\alpha)\) avec \( P\in A[X]\). Nous avons donc
	\begin{equation}
		\{ P(\alpha)\tq P\in A[X] \}\subset A[\alpha].
	\end{equation}
	Par ailleurs, \( \{ P(\alpha)\tq P\in A[X] \}\) est un sous-anneau de \( B\) contenant \( A\) et \( \alpha\). Donc \( A[\alpha]\) y est inclus.
\end{proof}

%---------------------------------------------------------------------------------------------------------------------------
\subsection{Polynômes sur un anneau intègre}
%---------------------------------------------------------------------------------------------------------------------------

\begin{theorem}     \label{ThoBUEDrJ}
	L'anneau \( A\) est intègre si et seulement si \( A[X]\) est intègre.
\end{theorem}

\begin{proof}
	Soient \( P\) et \( Q\) des éléments non nuls de \( A[X]\). Puisque l'anneau \( A\) est intègre, nous avons
	\begin{equation}
		\deg(PQ)=\deg(P)+\deg(Q)
	\end{equation}
	et le produit ne peut pas être nul. L'anneau \( A[X]\) est donc intègre.

	Si \( A[X]\) est intègre, \( A\) est intègre parce qu'il peut être considéré comme sous-anneau de \( A[X]\).
\end{proof}

\begin{normaltext}
	Si \( A\) n'est pas intègre, soient \( \alpha,\beta\in A\) non nuls tels que \( \alpha\beta=0\). Le produit des polynômes \( X\mapsto \alpha X\) et \( X\mapsto \beta\) est \( (\alpha X)\cdot(\beta)=0\); le degré du produit n'est pas la somme des degrés.

	Les personnes qui ont tout compris jusqu'ici remarqueront que la notation «\( X\mapsto P(X)\)» n'est pas correcte parce que du point de vue que nous adoptons ici, un polynôme n'est pas une application.
\end{normaltext}

\begin{corollary}
	Si \( A\) est intègre, les inversibles de \( A[X]\) sont les éléments inversibles de \( A\).
\end{corollary}

\begin{proof}
	Pour que \( Q\) soit inversible, il faut un \( P\) tel que \( PQ=1\). Mais l'anneau \( A\) étant intègre, les degrés s'additionnent. Par conséquent ils doivent être de degré zéro et il faut que \( P,Q\in A\). Donc \( Q\) est un inversible de \( A\).
\end{proof}

%---------------------------------------------------------------------------------------------------------------------------
\subsection{Division euclidienne}
%---------------------------------------------------------------------------------------------------------------------------

Le théorème suivant établit la \defe{division euclidienne}{division!euclidienne} dans \( A[X]\) du polynôme \( P\) par un polynôme \( D\).
\begin{theorem}     \label{ThodivEuclPsFexf}
	Soit \( D\neq 0\) dans \( A[X]\) de coefficient dominant inversible dans \( A\). Pour tout \( P\in A[X]\), il existe \( Q,R\in A[X]\) tels que
	\begin{equation}
		P=QD+R
	\end{equation}
	avec \( \deg(R)<\deg(D)\).

	Les polynômes \( Q\) et \( R\) sont déterminés de façon univoque par cette condition.
\end{theorem}

\begin{definition}\label{DefMPZooMmMymG}
	Le polynôme \( Q\) est le \defe{quotient}{quotient} et \( R\) est le \defe{reste}{reste} de la division euclidienne de \( P\) par \( D\). Si le reste de la division de \( P\) par \( D\) est nul on dit que \( D\) \defe{divise}{diviseur!polynôme} \( P\) et on note \( D\divides P\)\nomenclature[A]{\( D\divides P\)}{\( D\) divise \( P\)}. Autrement dit \( D\) divise \( P\) si il existe \( Q\) tel que \( P=QD\).\footnote{Ceci se rapproche tout naturellement des notions générales de divisibilité dans un anneau intègre, vues en sous-section~\ref{DivisibiliteAnneauxIntegres}.}
\end{definition}

\begin{normaltext}
	Le théorème~\ref{ThodivEuclPsFexf} nous incite à utiliser le degré comme stathme euclidien sur \( A[X]\) dès que \( A\) est un anneau intègre. Or cela ne fonctionne pas en général, parce que très peu de polynômes ont à priori un coefficient dominant inversible.
\end{normaltext}

\begin{lemma}[Thème~\ref{THEMEooZYKFooQXhiPD}]       \label{LEMooIDSKooQfkeKp}
	Si \( \eK\) est un corps\footnote{Définition~\ref{DefTMNooKXHUd}.}, alors l'anneau \( \eK[X]\) est euclidien et principal.
\end{lemma}

\begin{proof}
	Puisque \( \eK\) est un corps, tous les éléments sont inversibles et le degré donne un stathme par le théorème~\ref{ThodivEuclPsFexf}. L'anneau \( \eK[X]\) est donc euclidien et par conséquent principal (proposition~\ref{Propkllxnv}).
\end{proof}

Dans le théorème~\ref{ThoCCHkoU} nous donnerons une preuve directe du fait que \( \eK[X]\) est principal en montrant que tous ses idéaux sont principaux. Nous y démontrerons donc un peu moins pour un peu plus cher, mais avec le plaisir de ne pas devoir passer par un stathme.

\begin{definition}[\cite{ooSXFEooEehobn}]  \label{DefDSFooZVbNAX}
	Soit un anneau \( A\). Deux polynômes \( P\) et \( Q\) dans \( A[X]\) sont dits \defe{étrangers}{etranger@étrangers!polynômes} entre eux si \( 1\) est un pgcd\footnote{Définition~\ref{DefrYwbct}.} de \( P\) et \( Q\). Un ensemble de polynômes \( (P_i)_{i\in I}\) est étranger \defe{dans leur ensemble}{étranger!dans leur ensemble} si \( 1\) est un \( \pgcd\) des \( P_i\).

	Les polynômes \( P\) et \( Q\) sont \defe{premiers entre eux}{premier!deux polynômes entre eux} si les seuls diviseurs communs de \( P\) et \( Q\) sont les inversibles.
\end{definition}

Les notions de polynômes étrangers entre eux ou de polynômes premiers entre eux ne sont pas identiques, comme le montre l'exemple suivant.

\begin{example}[\cite{MonCerveau}]
	Soient dans \( \eZ[X]\) les polynômes \( P(X)=2X+2\) et \( Q(X)=2X^2+2\). Le nombre \( 2\) est diviseur commun et n'est pas un diviseur de \( 1\). Donc \( 1\) n'est pas un pgcd de \( P\) et \( Q\). Ils ne sont pas étrangers.

	Mais ils sont premiers entre eux parce qu'ils n'ont pas d'autres diviseurs communs que les inversibles (\( 1\) et \( -1\)).
\end{example}

%---------------------------------------------------------------------------------------------------------------------------
\subsection{Polynôme primitif}
%---------------------------------------------------------------------------------------------------------------------------

\begin{definition}\label{DefContenuPolynome}
	Un \defe{contenu}{contenu}\index{polynôme!contenu} du polynôme \( P=\sum_ia_iX^i\in\eK[X]\) est un pgcd de ses coefficients : \( c(P)=\pgcd\big(\{ a_i \}\big)\).
\end{definition}

\begin{normaltext}		\label{NORooQITDooKsTMKm}
	Quand il y a unicité du pgcd d'un ensemble, on peut parler du contenu au singulier.

	Le contenu d'un polynôme sera surtout utilisé juste pour savoir si il est inversible ou non. Dans un anneau intègre, ça demande un peu d'abus de langage, mais comme le lemme \ref{LEMooSFHMooQoKsPV} dit que si un pcgd est inversible, ils le sont tous, au moins discuter de l'inversibilité du contenu dans un anneau intègre a un sens et est bien défini.
\end{normaltext}

\begin{definition}[Polynôme primitif\cite{BIBooVOBZooEbyWWm}]           \label{DEFooDVOOooKaPZQC}
	Il existe deux notions de polynômes primitifs. Vu qu'il existe des anneaux unitaires qui sont des corps finis\quext{Je pense à \( \eZ/n\eZ\), mais faites attention, j'ai pas vérifié. Écrivez-moi pour me dire si c'est un bon exemple.}, il faut faire attention au contexte.
	\begin{enumerate}
		\item		\label{ITEMooCNHAooVfIYEW}
		      Si \( A\) est un anneau unitaire, \( P\) est un \defe{polynôme primitif}{polynôme primitif} \( c(P) \) est inversible\footnote{Pas mal d'auteurs disent que \( c(P)\) soit être égal à \( 1\). Mais en général ces auteurs se rétractent bien vite en disant que \( c(P)\) n'est définit qu'à inversible près. Voir aussi \ref{NORooQITDooKsTMKm}.}.
		\item
		      Si \( \eK\) est un corps fini, un polynôme dans \( \eK[X]\) est primitif si il est le polynôme minimal d'un générateur du groupe commutatif du corps.
	\end{enumerate}
\end{definition}


\begin{normaltext}
	Pour rappel, il y a plusieurs façons de périphraser le fait que les coefficients soient premiers entre eux. Nous pouvons dire \ldots
	\begin{enumerate}
		\item
		      Le pgcd de ses coefficients est \( 1\) parce que c'est la définition~\ref{DEFooXSPFooPumQSy} pour avoir des nombres premiers entre eux.
		\item
		      Le contenu de ses coefficients est \( 1\). Parce que le contenu est précisément le pgcd, définition~\ref{DefContenuPolynome}.
	\end{enumerate}
\end{normaltext}

La notion de polynôme primitif au sens du pgcd est particulière aux polynômes à coefficients dans un anneau comme le montre le lemme suivant.



\begin{lemma}
	Si \( \eK\) est un corps, tout polynôme unitaire dans \( \eK[X]\) non nul est primitif au sens du pgcd.
\end{lemma}

\begin{proof}
	Un polynôme unitaire a un \( 1\) parmi ses coefficients, donc le pgcd est forcément \( 1\).
\end{proof}



\begin{lemma}[\cite{BIBooXBJQooOCerti}]
	Soit un anneau factoriel\footnote{Définition \ref{DEFooVCATooPJGWKq}. Je rappelle qu'un anneau factoriel est toujours commutatif.} \( A\). Soient des polynômes primitifs \( P_1\) et \( P_2\) dans \( A[X]\). Soient \( a_1,a_2\in A^*\) tels que
	\begin{equation}
		a_1P_1=a_2P_2.
	\end{equation}
	Alors
	\begin{enumerate}
		\item
		      Les éléments \( a_1\) et \( a_2\) sont associés dans \( A\).
		\item
		      Les polynômes \( P_1\) et \( P_2\) sont associés dans \( A[X]\).
	\end{enumerate}
\end{lemma}

\begin{proof}
	Vu que les \( P_i\) sont primitifs, nous avons \( c(a_iP_i)=u_ia_i\) où \( u_i \) est un inversible (\( u_i=c(P_i)\)). Les éléments \( u_1a_1\) et \( u_2a_2\) sont donc deux pgcd du même ensemble : les coefficients de \( a_1P_1\) (qui sont les mêmes que ceux de \( a_2P_2\)). Le lemme \ref{LEMooGWKMooLEepxz} entre en jeu pour dire que \( u_1a_1\) et \( u_2a_2\) sont associés : il existe un inversible \( v\) tel que \( u_1a_1=vu_2a_2\). Au final, \( a_1=u_1^{-1} vu_2 a_2\) et nous voyons que \( a_1\) et \( a_2\) sont associés.

	Nous écrivons l'égalité \( a_1P1=a_2P_2\) avec la valeur trouvée de \( a_1\): \( u_1^{-1}vu_2a_2P_1=a_2P_2\). On voudrait simplifier par \( a_2\) (surtout que \( A\) est commutatif), mais comme \( a_2\) n'est pas spécialement inversible, il faut le justifier. L'anneau \( A\) est intègre; donc l'anneau \( A[X]\) l'est aussi (théorème \ref{ThoBUEDrJ}). Et comme \( A[X]\) est intègre, on peut faire la simplification.

	Au final nous avons \( u_1^{-1}vu_2P_1=P_2\), et comme \( u_1^{-1}vuy\) est inversible, \( P_1\) et \( P_2\) sont associés.
\end{proof}


\begin{lemma}[de Gauss\cite{BIBooXBJQooOCerti}]		\label{LEMooHJECooTeELgN}
	Soit un anneau factoriel \( A\). Soient \( P,Q\in A[X]\).
	\begin{enumerate}
		\item		\label{ITEMooISPDooXdRywE}
		      Les polynômes \( P\) et \( Q\) sont primitifs si et seulement si le polynôme \( PQ\) est primitif.
		\item		\label{ITEMooLWEEooLcDlfp}
		      Il existe un inversible \( u\in A\) tel que \( c(PQ)=uc(P)c(Q)\).
	\end{enumerate}
\end{lemma}

\begin{proof}
	En plusieurs parties.
	\begin{subproof}
		\spitem[\ref{ITEMooISPDooXdRywE}\( \Rightarrow\)]
		%-----------------------------------------------------------
		Nous supposons que \( P\) et \( Q\) sont primitifs. Et, par l'absurde, nous supposons que \( PQ\) n'est pas primitif. Donc \( c(PQ)\) n'est pas inversible dans \( A\). Vu que \( A\) est factoriel, tout élément non inversible est un produit d'irréductibles; c'est le cas de \( c(PQ)\). Soit \( p\) un irréductible entrant dans la décomposition de \( c(pQ)\).

		Le lemme \ref{LEMooGKOSooRKtfDJ} dit que l'anneau \( B=A/pA\) est intègre\quext{Ici, ma source \cite{BIBooXBJQooOCerti} précise que \( p\) est premier. C'est vrai par la proposition \ref{PROPooOQSXooYidPQv}, mais ça me semble inutile. Écrivez-moi pour me dire si le fait que \( p\) soit premier est important.}.

		Considérons la surjection canonique \(\pi \colon A\to B  \) que nous prolongeons en un morphisme d'anneaux
		\begin{equation}
			\begin{aligned}
				\phi\colon A[X] & \to B[X]                   \\
				\sum_ia_iX^i    & \mapsto \sum_i\pi(a_i)X^i.
			\end{aligned}
		\end{equation}

		Notons que \( p\) ne divise pas tous les coefficients de \( P\). En effet, sinon \( c(P)\) serait un multiple de \( p\) (proposition \ref{PROPooOXQMooVEzlyG}) et ne pourrait pas être inversible. Donc \( \phi(P)\neq 0\) dans \( B[X]\). De même nous savons que \( \phi(Q)\neq 0\).

		L'anneau \( B\) est intègre parce que \( A\) est intègre (proposition \ref{ThoBUEDrJ}). Donc \( \phi(P)\phi(A)\neq 0\), et comme \( \phi\) est un morphisme, \( \phi(PQ)\neq 0\), ce qui signifie que \( p\) ne divise pas tous les coefficients de \( PQ\), ce qui est contraire à l'hypothèse de l'absurde.

		Nous en déduisons que \( c(PQ)\) n'est pas inversible, et donc que \( PQ\) est primitif.

		\spitem[\ref{ITEMooISPDooXdRywE}\( \Leftarrow\)]
		%-----------------------------------------------------------

		Nous supposons que \( PQ\) est primitif, et nous montrons que \( P\) et \( Q\) le sont. En factorisant un pgcd des coefficients de \( P\) nous pouvons écrire \( P=c(P)P_1\) avec \( P_1\) primitif (lemme \ref{LEMooZSUNooUmYmgt}). Par la partie précédente nous savons que \( P_1Q_1\) est primitif. Nous avons donc l'égalité
		\begin{equation}
			PQ=c(P)c(Q)P_1Q_1
		\end{equation}
		où à la fois \( PQ\) et \( P_1P_2\) sont primitifs. Vu que \( PQ\) est primitif, \( c(P)c(Q)P_1Q_1\) est primitif, de telle sorte que\footnote{Utilisation du lemme \ref{LEMooMHZQooIcSNSf}.}
		\begin{equation}
			c\Big( c(P)c(Q)P_1Q_1 \Big)=c(P)c(Q)c(P_1Q_1)
		\end{equation}
		est inversible. Vu que \( c(P_1Q_1)\) est également inversible, nous en déduisons que \( c(P)c(Q)\) est inversible. Si \( u\in A\) vérifie \( c(P)c(Q)u=1\), alors \( c(Q)u\) est un inverse de \( c(P)\). De même \( uc(P)\) est un inverse de \( c(Q)\). Bref, ils sont inversibles et les polynômes \( P\) et \( Q\) sont primitifs.

		\spitem[Pour \ref{ITEMooLWEEooLcDlfp}]
		%-----------------------------------------------------------
	\end{subproof}

	Soient \( P,Q\in A[X]\). Nous considérons les polynômes primitifs \( P_1\), \( Q_1\) et \( R_1\) définis par \( P=c(P)P_1\), \( Q=c(Q)Q_1\) et \( PQ=c(PQ)R_1\). Nous avons
	\begin{equation}
		c(P)c(Q)P_1Q_1=PQ=c(PQ)R_1.
	\end{equation}
	En prenant le contenu des deux côtés nous avons
	\begin{equation}
		c\Big( c(P)c(Q)P_1Q_1 \Big)=c\Big( c(PQ)R_1 \Big),
	\end{equation}
	et en tenant compte du lemme \ref{LEMooMHZQooIcSNSf}, cela devient
	\begin{equation}
		c(P)c(Q)c(P_1Q_1)=c(PQ)c(R_1)
	\end{equation}
	où les éléments \( c(P_1Q_1)\) et \( c(R_1)\) sont inversibles. En posant \( u=c(P_1Q_1)^{-1}c(R_1)\), nous avons bien
	\begin{equation}
		c(P)c(Q)=uc(PQ).
	\end{equation}
\end{proof}

\begin{normaltext}
	Dans toute la démonstration, il y a une ambiguïté entre \( c(P)\) qui serait le pgcd des coefficients de \( P\), c'est-à-dire un ensemble et \( c(P)\) qui serait un représentant de cet ensemble.

	Grâce au lemme \ref{LEMooZKASooKstTuK}, nous savons que si \( \delta\) est un pgcd et si \( u\) est inversible, alors \( u\delta\) est un pgcd. Donc à partir de l'équation
	\begin{equation}
		c(P)c(Q)=uc(PQ),
	\end{equation}
	en fait nous savons qu'il existe des représentants de \( c(P)\), \( c(Q) \) et \( c(PQ)\) tels que
	\begin{equation}
		c(P)c(Q)=c(PQ).
	\end{equation}
\end{normaltext}

%---------------------------------------------------------------------------------------------------------------------------
\subsection{Racines des polynômes}
%---------------------------------------------------------------------------------------------------------------------------


\begin{definition}[Ordre d'un polynôme]
	Soit \( P\) un polynôme irréductible\footnote{Définition \ref{DeirredBDhQfA}.} de degré \( n\) sur \( \eF_p[X]\). L'\defe{ordre}{ordre!d'un polynôme} de \( P\) est
	\begin{equation}
		\min\{ k\tq P\divides X^k-1 \}.
	\end{equation}
\end{definition}


\begin{definition}
	Soient \( A \) un anneau et \( P \in A[X] \). On appelle
	\defe{racine}{racine!d'un polynôme} un élément \( \alpha \in A \)
	tel que \( P(\alpha) = 0 \); c'est-à-dire que, en remplaçant toutes
	les occurrences de \( X\) par \( \alpha\) dans l'expression de \( P\), on
	obtient \( 0\).
\end{definition}

\begin{proposition} \label{PropHSQooASRbeA}
	Soient \( A\) un anneau et \( P\) un polynôme non nul dans \( A[X]\). Si \( \alpha\in A\) est une racine de \( P\) alors \( X-\alpha\) divise \( P\), et réciproquement.
\end{proposition}

\begin{proof}
	Nous notons le polynôme \( \mu=X-\alpha\) par analogie avec le polynôme minimal dont il sera question dans la très semblable proposition~\ref{PropXULooPCusvE}. Le sens réciproque est clair: si \( \mu\) divise \( P\), alors \( \alpha\) est racine de \( P\).

	Pour le sens direct, remarquons que si \( \alpha\) est racine de \( P\), alors \( P\) est de degré au moins égal à \( 1\), et nous pouvons donc effectuer la division euclidienne\footnote{Théorème~\ref{ThodivEuclPsFexf}.} de \( P\) par \( \mu\) : il existe des polynômes \( Q\) et \( R\) tels que
	\begin{equation} \label{PropHSQooASRbeA1}
		P=Q\mu+R
	\end{equation}
	avec \( \deg(R)<\deg(\mu)\). Donc \( R\) est une constante,
	élément de \( A\): appelons-le \( a\). En évaluant
	\eqref{PropHSQooASRbeA1} en \( \alpha\), il vient
	\begin{equation}
		0 = P(\alpha)=Q(\alpha)\mu(\alpha)+a,
	\end{equation}
	et nous en déduisons que \( a=0\), ce qui montre que \( P=Q\mu\) et que \( \mu\) divise \( P\).
\end{proof}

\begin{definition}[Racine simple et multiple d'un polynôme]     \label{DEFooTGZYooCYiKQa}
	Soit \( A\) un anneau ainsi qu'un polynôme \( P\in A[X]\) et \( \alpha\in A\) racine de \( P\). La \defe{multiplicité}{multiplicité!racine d'un polynôme} de \( \alpha\) par rapport à \( P\) est l'entier \( h\) tel que \( P\) est divisible par \( (X-\alpha)^h\) mais pas divisible par \( (X-\alpha)^{h+1}\).  Nous noterons \( \theta_{\alpha}(P)\)\nomenclature[A]{\( \theta_{\alpha}(P)\)}{la multiplicité de \( \alpha\) par rapport à \( P\)} la multiplicité de \( \alpha\) par rapport à \( P\).

	Si la multiplicité d'une racine est égale à \( 1\), nous disons que c'est une \defe{racine simple}{racine simple}. Sinon, c'est une \defe{racine multiple}{racine multiple}.
\end{definition}

\begin{normaltext}
	Pour une définition générale d'une racine simple de l'équation \( f(x)=0\), voir la définition~\ref{DEFooXSOQooAnWqKM}. La proposition~\ref{PropHSQooASRbeA} nous indique que toute racine est de multiplicité au moins égale à \( 1\).
\end{normaltext}


\begin{lemma}		\label{LEMooZUUVooQgsdXs}
	Soient un anneau \( A\) ainsi que \( x,r\in A\). Nous avons
	\begin{equation}
		x^n-r^n=(x-r)\sum_{k=0}^{n-1}x^kr^{n-k-1}.
	\end{equation}
\end{lemma}

\begin{proof}
	Nous effectuons la distribution, et nous calculons un peu :
	\begin{subequations}
		\begin{align}
			(x-r)\sum_{k=0}^{n-1}x^kr^{n-k} & =\sum_{k=0}^{n-1}x^{k+1}r^{n-k-1}-\sum_{k=0}^{n-1}x^kr^{n-k} \\
			                                & =\sum_{k=1}^{n}x^kr^{n-k}-\sum_{k=0}^{n-1}x^kr^{n-k}         \\
			                                & = - r^n +\sum_{k=1}^{n-1}(x^kr^{n-k}-x^kr^{n-k})+x^n         \\
			                                & = x^n-r^n.
		\end{align}
	\end{subequations}
\end{proof}

\begin{proposition}[\cite{BIBooNVMLooSsZrwi}]     \label{PROPooQCZSooVokxXQ}
	Soient un anneau \( A\), un polynôme \( P\in A[X]\) de degré \( n\), ainsi qu'une racine \( \alpha\in A\) de \( P\). Alors il existe un polynôme \( Q\in A[X]\) de degré \( n-1\) tel que \( P=(X-\alpha)Q\).
\end{proposition}

\begin{proof}
	D'après le lemme \ref{LEMooZUUVooQgsdXs}, pour chaque \( k\), nous avons \( x^k-r^k=(x-r)S_k(x)\) où \( S_k\) est un polynôme de degré \( k-1\) (ou moins).

	Soit un polynôme \( P\) tel que \( P(\alpha)=0\). Nous pouvons écrire
	\begin{subequations}
		\begin{align}
			P(x) & =P(x)-P(\alpha)                             \\
			     & =\sum_{k=0}^na_kx^k-\sum_{k=0}^na_k\alpha^k \\
			     & =\sum_{k=1}^na_k(x^k-\alpha^k)              \\
			     & =\sum_{k=1}^na_k(x-\alpha)S_k(x)            \\
			     & =(x-\alpha)\sum_{k=1}^na_lS_k(x)            \\
			     & =(x-\alpha)Q(x)
		\end{align}
		où \( Q\) est le polynômde de degré \( n-1\) donné par \( Q=\sum_{k=1}^na_kS_k\).
	\end{subequations}
\end{proof}

\begin{proposition} \label{PropahQQpA}
	L'élément \( \alpha\in A\) est une racine de multiplicité\footnote{Multiplicité d'une racine, définition \ref{DEFooTGZYooCYiKQa}.} \( h\) du polynôme \( P\) si et seulement si il existe \( Q\in A[X]\) tel que \( P=(X-\alpha)^hQ\) avec \( Q(\alpha)\neq 0\).
\end{proposition}

\begin{proof}
	Par définition de la multiplicité de \( \alpha\), le polynôme \( P\) est divisible par \( (X-\alpha)^h\) mais pas par \( (X-\alpha)^{h+1}\). Il existe donc un polynôme \( Q\) tel que
	\begin{equation}
		P=(X-\alpha)^hQ.
	\end{equation}
	Si \( Q(\alpha)\) était nul, la proposition \ref{PROPooQCZSooVokxXQ} nous dirait que \( Q =(X-\alpha)S \) pour un cetain polynôme \( S\). Cela ferait \( P=(X-\alpha)^{h+1}S\), ce qui est contraire à l'hypothèse sur la multiplicité.
\end{proof}

\begin{lemma}       \label{LemIeLhpc}
	Soient \( P\) et \( Q\) des polynômes non nuls de \( A[X]\) et \( \alpha\in A\). Alors
	\begin{enumerate}
		\item
		      \( \theta_{\alpha}(P+Q)\leq\min\{
		      \theta_{\alpha}(P),\theta_{\alpha}(Q) \}\), et l'égalité a
		      lieu si \( \theta_{\alpha}(P)\neq \theta_{\alpha}(Q)\);
		\item     \label{ItemIeLhpciv}
		      \( \theta_{\alpha}(PQ)\geq
		      \theta_{\alpha(P)}+\theta_{\alpha}(Q)\), et l'égalité a
		      lieu si \( A \) est intègre.
	\end{enumerate}
\end{lemma}

Dans le théorème suivant, la partie importante en pratique est le point \ref{ITEMooWGOBooGApPOo} parce qu'il dit que, lorsque nous cherchons les racines d'un polynôme, nous pouvons nous arrêter lorsque nous en avons trouvé autant que le degré, multiplicité comprise.
\begin{theorem} \label{ThoSVZooMpNANi}
	Soit \( A\) un anneau intègre
	et \( P\in A[X]\setminus\{ 0 \}\), un polynôme de degré \( n\).

	\begin{enumerate}
		\item
		      Si \( \alpha_1,\ldots, \alpha_p\in A\) sont des racines deux à deux
		      distinctes de multiplicités \( k_1,\ldots, k_p\), alors il existe \(
		      Q\in A[X]\), de degré \( n-\sum_{i=1}^pk_i\), tel que
		      \begin{equation}
			      P=Q\prod_{i=1}^p(X-\alpha_i)^{k_i}
		      \end{equation}
		      et \( Q(\alpha_i)\neq 0\) pour tout \( i\).
		\item     \label{ITEMooWGOBooGApPOo}
		      La somme des multiplicités des racines de \( P\) est au plus \( \deg(P)\).
	\end{enumerate}
\end{theorem}
\index{factorisation!de polynôme}

\begin{proof}
	Si \( p=1\), soit \( \alpha\) une racine de multiplicité \( k\) de \( P\). La définition de la multiplicité d'une racine nous dit que \( P\) est divisible par \( (X-\alpha)^k\) mais pas par \( (X-\alpha)^{k+1}\). Donc il existe \( Q\in A[X]\) tel que \( P=Q(X-\alpha)^k\). Il reste à voir que \( Q(\alpha)\neq 0\). Cela est une conséquence de la proposition~\ref{PropHSQooASRbeA} : si \( Q(\alpha)\) était nul, on pourrait lui factoriser \( (X-\alpha)\) et donc avoir \( (X-\alpha)^{k+1}\) qui se factorise dans \( P\), ce qui n'est pas possible.

	Nous supposons que \( p\geq 2\) et nous effectuons une récurrence sur \( p\). Nous considérons donc les \( p-1\) premières racines \( \alpha_1,\ldots, \alpha_{p-1}\) et un polynôme \( R\in A[X]\) tel que \( R(\alpha_i)\neq 0\) pour \( i=1,\ldots, p-1\) et
	\begin{equation}
		P=\underbrace{(X-\alpha_1)^{k_1}\ldots (X-\alpha_{p-1})^{k_{p-1}}}_SR.
	\end{equation}
	Par hypothèse \( P(\alpha_p)=S(\alpha_p)R(\alpha_p)=0\). L'anneau \( A\) étant intègre, \( S(\alpha_p)\neq 0\) parce que \( \alpha_i\neq \alpha_p\) pour \( i\neq p\). Par conséquent, \( R(\alpha_p)=0\).

	Nous devons encore vérifier que la multiplicité \( \alpha_p\) est \( k_p\) par rapport à \( R\). Pour cela nous utilisons le point~\ref{ItemIeLhpciv} du lemme~\ref{LemIeLhpc} afin de dire que le degré de \( \alpha_p\) pour \( P=SR\) est \( k_p\). Par conséquent
	\begin{equation}
		R=(X-\alpha_p)^{k_p}T
	\end{equation}
	avec \( T(\alpha_p)\neq 0\) et enfin
	\begin{equation}
		P=\prod_{i=1}^p(X-\alpha_i)T.
	\end{equation}
	De plus \( T(\alpha_i)\neq 0\), sinon \( R(\alpha_i)\) serait nul.
\end{proof}

\begin{corollary}       \label{CORooUGJGooBofWLr}
	Un polynôme de degré \( n\) sur un anneau intègre possède au maximum \( n\) racines distinctes.
\end{corollary}

\begin{proof}
	Le théorème \ref{ThoSVZooMpNANi}\ref{ITEMooWGOBooGApPOo} dit que la somme des multiplicités des racines de \( P\) est au maximum \( n\). Mais la proposition \ref{PropHSQooASRbeA} dit que toutes les racines ont une multiplicité au moins égale à un. Donc il ne peut pas y en avoir plus de \( n\).
\end{proof}

\begin{proposition}[\cite{frwiki178477716,BIBooUMMWooZsakDB}]
	Soit un anneau intègre et \( n>0\). Les racines \( n\)\ieme\ de l'unité forment un groupe cyclique dont l'ordre divise \( n\).
\end{proposition}

\begin{lemma}       \label{LEMooHDJIooPTpUCJ}
	Soit le polynôme \( P=X^2-1\) sur l'anneau intègre\footnote{Définition \ref{DEFooTAOPooWDPYmd}.} \( A\). Si \( 1\neq -1\)\quext{À mon avis cette hypothèse n'est pas nécessaire. Et d'ailleurs j'ai un peu du mal à voir des exemples exotiques d'anneaux intègres dans lesquels \( 1=-1\). Il y aurait \( \mathopen[ 0 , 1 \mathclose]\) modulo \( 2\), certaines de ses parties, comme \( \eZ/2\eZ\). Et quoi d'autre ?}, alors les racines de \( P\) sont \( \pm 1\).
\end{lemma}

\begin{proof}
	Le fait que \( \pm 1\) sont racines est le lemme \ref{LEMooLTERooVKgqjn}\ref{ITEMooUGHCooOPgoeR}\ref{ITEMooYMRKooHVYYKU}. Puisque par hypothèse \( 1\neq -1\), le corolaire \ref{CORooUGJGooBofWLr} termine la preuve.
\end{proof}

\begin{corollary}[Conséquence du lemme de Gauss\cite{ooCDLEooEQGSvn}]       \label{CORooZCSOooHQVAOV}
	Soient \( A\) un anneau factoriel et \( \Frac(A)\) son corps des fractions. Un polynôme non constant \( P\in A[X]\) est irréductible (sur \( A\)) si et seulement si il est irréductible et primitif au sens du pgcd\footnote{Définition~\ref{DEFooDVOOooKaPZQC}.} sur \( \Frac(A)[X]\).
\end{corollary}

\begin{example}
	Il ne faudrait pas croire qu'être irréductible dans un anneau \( A\) implique d'être irréductible dans le corps des fractions. En effet soit \( A=\eZ[\sqrt{ 5 }]\) et \( P=X^2-X-1\). Nous savons que sa factorisation est
	\begin{equation}
		P=\left( X-\frac{ 1+\sqrt{ 5 } }{ 2 } \right)\left( X-\frac{ 1-\sqrt{ 5 } }{ 2 } \right).
	\end{equation}
	Si vous ne le saviez pas, faites juste le calcul pour vous en assurer.

	Ce polynôme est irréductible sur \( \eZ[\sqrt{ 5 }]\) mais pas irréductible sur \( \Frac\big( \eZ[\sqrt{ 5 }] \big)\).
\end{example}

%---------------------------------------------------------------------------------------------------------------------------
\subsection{Quelques identités}
%---------------------------------------------------------------------------------------------------------------------------

\begin{lemma}[\cite{MonCerveau,BIBooFLIAooJadaMP}]   \label{LemISPooHIKJBU}
	Quelques identités de polynômes.
	\begin{enumerate}
		\item   \label{ItemLTBooAcyMtN}
		      Si \( n\) est impair, alors \( 1+X\) divise \( 1+X^n\).
		      \item\label{ItemLTBooAcyMtNii}
		      Pour tout \( n\) nous avons \( X^n-1=(X-1)(1+X+\cdots +X^{n-1})\).
		\item       \label{ITEMooILIVooHtmWLm}
		      \( X^n-a^n=(X-a)\sum_{i=0}^{n-1}a^iX^{n-1-i}\).
	\end{enumerate}
\end{lemma}

\begin{proof}
	Plusieurs points.
	\begin{subproof}
		\spitem[Pour \ref{ItemLTBooAcyMtN}]
		Nous allons utiliser le lemme \ref{LEMooOYZEooLivKWI} avec \( a=1\), \( b=-X\) et \( r=n-1\). Notez que \( n\) étant impair, \( r=n-1\) est encore positif. En ce qui concerne le membre de gauche de \eqref{Eqarpurmkbk}, nous remarquons que \( (-X)^n=(-1)^nX^n=-X^n\). Nous avons donc :
		\begin{equation}
			1+X^n=(a-b)\sum_{k=0}^{n-1}a^{n-1-kb^k}=(1+X)(\cdots).
		\end{equation}
		Cela prouve que \( 1+X\) divise \( 1+X^n\).
		\spitem[Pour \ref{ItemLTBooAcyMtNii}]
		Posons \( P=X+\ldots +X^{n-1}\). Nous avons \( XP=P-X+X^n\) et donc
		\begin{equation}
			(X-1)(1+P)=X+XP-1-P
			=X+(P-X+X^n)-1-P
			=X^n-1.
		\end{equation}
		\spitem[Pour \ref{ITEMooILIVooHtmWLm}]
		Déjà fait dans \eqref{Eqarpurmkbk}.
	\end{subproof}
\end{proof}

%---------------------------------------------------------------------------------------------------------------------------
\subsection{Générateurs pour le groupe multiplicatif}
%---------------------------------------------------------------------------------------------------------------------------

\begin{proposition}[\cite{BIBooYOLUooOAUbYH}]       \label{PROPooKSCRooPyInSv}
	Si \( p\) est un nombre premier, tout sous-groupe de \( \big( (\eZ/p\eZ)^*, \cdot\big)\) est cyclique\footnote{Définition \ref{DefHFJWooFxkzCF}.}.
\end{proposition}

\begin{proof}
	Soit un sous-groupe \( H\) d'ordre \( | H |=n\) de \(   (\eZ/p\eZ)^*\). Nous prouvons par récurrence sur \( n\) que \( H\) est cyclique.

	\begin{subproof}
		\spitem[\( n=1\)]
		%-----------------------------------------------------------
		Si \( H\) est un sous-groupe d'ordre \( 1\), alors \( H=\{ [1]_p \}\) et c'est cyclique, pas de problèmes.
		\spitem[Récurrence]
		%-----------------------------------------------------------
		Nous supposons que tous les sous-groupes de \( \big( (\eZ/p\eZ)^*\) d'ordre plus petit que \( n\) sont cycliques, et nous prouvons que \( H\) est également cyclique.

		Nous allons faire deux cas suivant que \( n\) est la puissance d'un nombre premier ou non.

		\begin{subproof}
			\spitem[Si \( n=q^k\)]
			%-----------------------------------------------------------
			Nous supposons que \( n=q^k\) pour un certain nombre premier \( q\). Nous supposons aussi, par l'absurde que \( H\) n'est pas cyclique. Les éléments de \( H\) ont un ordre qui divise \( q^k\) (corolaire \ref{CorpZItFX}), mais pas d'ordre \( q^k\). Donc pour tout \( x\) dans \( H\) nous avons \( x^{q^{k-1}}=1\). En particulier le polynôme \( P=X^{X^{q-1}}-1\) a au moins \( q^k\) racines dans \( \eZ/p\eZ\) : les éléments de \( H\).

			Hélas le degré de \( P\) étant \( q^{k-1}\), il ne peut pas avoir plus de \( q^{k-1}\) racines distinctes par \ref{ThoSVZooMpNANi}. Contradiction. Donc \( H\) est cylcique.

			\spitem[Cas général]
			%-----------------------------------------------------------

			Nous supposons que \( | H |\) n'est pas la puissance d'un nombre premier.

			\begin{subproof}
				\spitem[Un morphisme]
				%-----------------------------------------------------------
				Le corolaire \ref{CORooWBSQooQOEmaC} nous indique que \( n=ab\) pour deux nombres \( a,b\) tels que \( \pgcd(a,b)=1\). Nous considérons l'application
				\begin{equation}
					\begin{aligned}
						f\colon H & \to H        \\
						x         & \mapsto x^a.
					\end{aligned}
				\end{equation}
				Étant donné que \( H\) est abélien, \( f\) est un morphisme de groupes. Tout élément \( x\in H\) vérifie \( f(x)^b=x^{ab}=1\). Donc
				\begin{equation}
					\Image(f)\subset\{ y\in H\tq y^b=1 \}.
				\end{equation}

				\spitem[Ordre du noyau et de l'image]
				%-----------------------------------------------------------
				Par le coup des racines distinctes de polynôme \( X^b-1\), nous avons \( | \Image(f) |\leq b\). De même nous avons
				\begin{equation}
					\ker(f)=\{ x\in H\tq x^a=1 \},
				\end{equation}
				et donc \( | \ker(f) |\leq a\).

				Le premier théorème d'isomorphisme \ref{ThoPremierthoisomo} nous indique que
				\begin{equation}
					\frac{ H }{ \ker(f) }\simeq \Image(f),
				\end{equation}
				et comme tout est abélien, tous les sous groupes sont normaux\footnote{«distingué» et «normal» sont synonymes.}, et nous avons, par le théorème de Lagrange \ref{ThoLagrange} que
				\begin{equation}
					ab = | H |=\underbrace{| \ker(f) |}_{\leq a}\underbrace{| \Image(f) |}_{\leq b}\leq ab.
				\end{equation}
				Le premier membre est égal au dernier. Donc toutes les inégalités sont des égalités. Nous avons donc \( | \ker(f) |=a\) et \( | \Image(f) |=b\).

				\spitem[Hypothèse de récurrence]
				%-----------------------------------------------------------
				C'est le moment d'utiliser l'hypothèse de récurrence : les groupes \( \ker(f)\) et \( \Image(f)\) sont cycliques. Soient des générateurs \( h\) et \( h'\). Quel est l'ordre de \( hh'\) ? Le lemme \ref{LemyETtdy} nous indique que \( hh'\) est d'ordre \( ab=n\). Donc \( hh'\) est générateur de \( H\) qui est donc cyclique.
			\end{subproof}
		\end{subproof}
	\end{subproof}
\end{proof}
