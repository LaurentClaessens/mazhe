% This is part of Mes notes de mathématique
% Copyright (c) 2011-2018, 2020, 2022-2023
%   Laurent Claessens
% See the file fdl-1.3.txt for copying conditions.

%+++++++++++++++++++++++++++++++++++++++++++++++++++++++++++++++++++++++++++++++++++++++++++++++++++++++++++++++++++++++++++
\section{Espaces de Sobolev}
%+++++++++++++++++++++++++++++++++++++++++++++++++++++++++++++++++++++++++++++++++++++++++++++++++++++++++++++++++++++++++++

Rappel : la définition de la dérivée faible est \ref{DEFooIRJQooMVNopl}.

%---------------------------------------------------------------------------------------------------------------------------
\subsection{Sur un intervalle de \( \eR\)}
%---------------------------------------------------------------------------------------------------------------------------

Sauf mention du contraire dans cette section \( I\) est un intervalle borné ouvert \( I=\mathopen] a , b \mathclose[\) de \( \eR\).

\begin{definition}
	Soit \( I=\mathopen] a , b \mathclose[\) un ouvert borné de \( \eR\). L'\defe{espace de Sobolev}{espace!de Sobolev} \( H^1(I)\)\nomenclature[Y]{\( H^1(I)\)}{espace de Sobolev} est l'ensemble
	\begin{equation}
		H^1(I)=\Big\{   u\in L^2(I)\tq\exists g\in L^2(I)\tq\forall \varphi\in  C^{\infty}_c(I),\int_Iu\varphi'=-\int_Ig\varphi   \Big\}.
	\end{equation}
\end{definition}

L'unique élément \( g\) de \( L^2(I)\) vérifiant \( \int_Iu\varphi'=-\int_Ig\varphi\) est noté \( u'\) et est nommé \defe{dérivée}{dérivée!dans Sobolev \(  H^1(I)\)}; nous verrons dans les prochaines pages pourquoi.

L'espace \( H^1\) accepte le produit scalaire suivant :
\begin{equation}
	\langle u, v\rangle =\int_Iuv+\int_Iu'v',
\end{equation}
et nous notons \( \| . \|_{H^1}\) la norme correspondante qui n'est autre que
\begin{equation}
	\| u \|^2_{H^1}=\langle u, u\rangle =\| u \|^2_{L^2}+\| u' \|^2_{L^2}.
\end{equation}

Nous introduisons l'espace \( L^1_{loc}(I)\)\nomenclature[Y]{\( L^1_{loc}(I)\)}{fonctions intégrables sur les compacts de \( I\)} des fonctions étant \( L^1\) sur tout compact de \( I\).

\begin{corollary}   \label{CorEVJYihj}
	Si \( u\in H^1(I)\) et si \( u'=0\) alors il existe une constante \( C\) telle que \( u=C\) presque partout.
\end{corollary}

\begin{proof}
	L'hypothèse \( u'=0\) signifie que pour toute fonction \( \varphi\in C^{\infty}_c(I)\),
	\begin{equation}
		\int_Iu\varphi'=\int_Iu'\varphi=0.
	\end{equation}
	La proposition~\ref{PropLGoLtcS} nous dit alors qu'il existe une constante \( C\) telle que \( u=C\) presque partout.
\end{proof}

\begin{lemma}   \label{LemMPkbZxX}
	Tout élément de \( H^1(I)\) admet un unique représentant continu.
\end{lemma}
Nous verrons dans le corolaire~\ref{CorCEPJGAu} que ce représentant pourra être prolongé par continuité sur \( \bar I\).

\begin{proof}
	Soient \( y_0\in I\) et \( u\in H^1(I)\). Nous considérons la fonction
	\begin{equation}
		\bar u(x)=\int_{y_0}^xu'(t)dt.
	\end{equation}
	Notons que par définition, \( u'\in L^2\) donc l'intégrale ne pose pas de problèmes. Montrons que \( \bar u\) est continue sur \( \bar I\). Pour cela nous considérons \( x\in\bar I\) et \( h\) tel que \( x+h\in \bar I\). Alors
	\begin{equation}
		\big| \bar u(x+h)-\bar u(x) \big|=\big| \int_x^{x+h}u' \big|\leq \int_x^{x+h}| u' |.
	\end{equation}
	Mais la fonction \( | u' |\) est dans \( L^1_{loc}(I)\) par le lemme~\ref{LemTLHwYzD}; elle est en particulier intégrable sur un ouvert contenant \( x\) et par conséquent la dernière intégrale tend vers zéro lorsque \( h\) tend vers \( 0\).

	Nous prouvons à présent que \( \bar u\) est dans \( H^1(I)\) et que sa dérivée est égale à \( u'\); pour cela nous allons montrer que pour tout \( \varphi\in  C^{\infty}_c(I)\),
	\begin{equation}
		\int_I\bar u\varphi'=-\int_Iu'\varphi.
	\end{equation}
	Nous avons
	\begin{equation}
		\int_I\bar u\varphi'=\int_I\left( \int_{y_0}^xu'(t)dt\right)\varphi'(x)dx
		=\int_{a}^{y_0}\left( \int_{y_0}^xu'(t)dt\right)\varphi'(x)dx+\int_{y_0}^b\left( \int_{y_0}^xu'(t)dt\right)\varphi'(x)dx.
	\end{equation}
	Pour faire plus court, nous notons \( f(t,x)=u'(t)\varphi'(x)\). La première intégrale vaut
	\begin{subequations}
		\begin{align}
			\int_a^{y_0}\left( \int_{y_0}^x u'(t)\varphi'(x) \right) & =\int_a^{y_0}\left(\int_{y_0}^af(t,x)\mtu_{t<x}(t,x)dt\right)dx     \\
			                                                         & =\int_{y_0}^a\int_a^{y_0}f(t,x)\mtu_{t<x}dxdt  \label{subeqBVyBPLp} \\
			                                                         & =\int_{y_0}^a\int_a^tf(t,x)dxdt                                     \\
			                                                         & =-\int_a^{y_0}\int_a^tu'(t)\varphi'(x)dx\,dt
		\end{align}
	\end{subequations}
	La permutation d'intégrales pour obtenir \eqref{subeqBVyBPLp} est due au théorème de Fubini~\ref{ThoFubinioYLtPI}\ref{ItemQMWiolgiii}. Par le même petit jeu, la seconde intégrale vaut
	\begin{equation}
		\int_{y_0}^b\int_t^b u'(t)\varphi'(x)dx\,dt.
	\end{equation}
	En refaisant la somme,
	\begin{subequations}
		\begin{align}
			\int_I\bar u\varphi'
			 & =-\int_a^{y_0}u'(t)\left( \int_a^t\varphi'(x)dx \right)dt+\int_{y_0}^bu'(t)\left( \int_t^b\varphi'(x)dx \right)dt \\
			 & =-\int_a^{y_0}u'(t)\big( \varphi(t)-\varphi(a) \big)dt+\int_{y_0}^bu'(t)\big( \varphi(b)-\varphi(t) \big)         \\
			 & =-\int_a^bu'\varphi                                                                                               \\
			 & =-\int_Iu'\varphi.
		\end{align}
	\end{subequations}
	Notons que \( \varphi(a)=\varphi(b)=0\) parce que \( \varphi\) est à support compact dans \( \mathopen] a , b \mathclose[\). Nous avons donc prouvé que \( \bar u\) est dans \( H^1(I)\) et que \( \bar u'=u'\). Par le corolaire~\ref{CorEVJYihj}, nous avons une constante \( C\) telle que \( \bar u=u+C\) presque partout, c'est-à-dire \( u=\bar u +C\) dans \( H^1(I)\).

	En résumé, \(\tilde u=\bar u+C\) est un représentant continu de \( u\) dans \( L^2(I)\).

	L'unicité du représentant continu est simplement le fait que deux fonctions continues égales presque partout sont égales (proposition ~\ref{PropNCMToWI}).

\end{proof}

\begin{proposition}     \label{PropGWOIoDg}
	Si \( u\in H^1(I)\), alors
	\begin{equation}
		u(x)-u(y)=\int_y^xu'
	\end{equation}
	pour tout \( x,y\in I\).
\end{proposition}

\begin{proof}
	Pour fixer les idées, nous supposons \( x<y\). Nous considérons une suite \( \varphi_n\in C^{\infty}_c(I)\) convergeant uniformément sur \( I\) vers \( \mtu_{\mathopen[ x , y \mathclose]}\). Nous exigeons de plus que
	\begin{itemize}
		\item
		      \( \varphi_n'\) est positive sur \( \mathopen[ a , x+\frac{1}{ n } \mathclose]\)
		\item
		      \( \varphi_n'\) est négative sur \( \mathopen[ y-\frac{1}{ n } , b \mathclose]\)
		\item
		      \( \varphi_n=1\) sur \( \mathopen[ x+\frac{1}{ n } , y-\frac{1}{ n } \mathclose]\).
		\item
		      \( \varphi_n=0\) sur \( \mathopen[ a , x-1/n \mathclose]\) et sur \( \mathopen[ y+1/n , b \mathclose]\).
	\end{itemize}
	Pour chaque \( n\), nous découpons l'intégrale comme
	\begin{equation}        \label{EqRPwqpve}
		-\int_Iu'\varphi_n=\int_Iu\varphi'_n=\int_a^{x-1/n}u\varphi'_n+\int_{x-1/n}^{x+1/n}u\varphi'_n+\int_{x+1/n}^{y-1/n}u\varphi'_n+\int_{y-1/n}^{y+1/n}u\varphi'_n+\int_{y+1/n}^{b}u\varphi'_n.
	\end{equation}
	Par construction de \( \varphi_n\), de ces \( 5\) morceaux, il n'en reste que deux de non nulles :
	\begin{equation}
		\int_Iu\varphi_n'=\underbrace{\int_{x-1/n}^{x+1/n}u(t)\varphi'_n(t)dt}_A+\underbrace{\int_{y-1/n}^{y+1/n}u(t)\varphi'_n(t)dt}_B
	\end{equation}

	Soit \( \epsilon>0\) et \( n\) suffisamment grand pour avoir \( u(t)\in B\big( u(x),\epsilon \big)\) pour tout \( t\in B(x,\frac{1}{ n })\) et (en même temps) \( u(t)\in B\big( u(y),\epsilon \big)\) pour tout \( t\in B(y,\frac{1}{ n })\). C'est la continuité de \( u\) qui permet de trouver un tel \( n\). Pour cette valeur de \( n\), en tenant compte des hypothèses sur la positivité de \( \varphi_n'\) nous avons
	\begin{equation}
		\int_{x-1/n}^{x+1/n}\big( u(x)-\epsilon \big)\varphi'_n(t)dt\leq\int_{x-1/n}^{x+1/n}u(t)\varphi'_n(t)dt\leq\int_{x-1/n}^{x+1/n}\big( u(x)+\epsilon \big)\varphi'_n(t)dt,
	\end{equation}
	mais par hypothèse sur \( \varphi_n\) nous trouvons
	\begin{equation}
		\int_{x-1/n}^{x+1/n}\varphi'_n(t)dt=\varphi_n(x+\frac{1}{ n })-\varphi(x+\frac{1}{ n })=1.
	\end{equation}
	donc
	\begin{equation}    \label{EqLYrpEdb}
		u(x)-\epsilon\leq\int_{x-1/n}^{x+1/n}u(t)\varphi'_n(t)dt\leq u(x)+\epsilon.
	\end{equation}
	Pour encadrer la seconde, il faut être plus prudent avec les signes parce que \( \varphi'_n\) y est négative. En posant \( \psi_n=-\varphi_n\) nous avons
	\begin{equation}
		-B=\int_{y-1/n}^{y+1/n}u(t)\psi_n(t)dt,
	\end{equation}
	et donc
	\begin{equation}
		u(y)-\epsilon\leq -B\leq u(y)+\epsilon
	\end{equation}
	ou encore
	\begin{equation}
		-\epsilon-u(y)\leq B\leq \epsilon-u(y).
	\end{equation}
	En additionnant avec \eqref{EqLYrpEdb} nous voyons que pour tout \( \epsilon>0\) il existe un \( N(\epsilon)\) tel que nous ayons
	\begin{equation}    \label{EqEBwWUxm}
		u(x)-u(y)-2\epsilon\leq\int_Iu'\varphi_{n}\leq u(x)-u(y)+2\epsilon
	\end{equation}
	pour tout \( n\geq N\). Nous voulons évidemment prendre la limite \( \epsilon\to 0\), c'est-à-dire \( n\to \infty\). Étant donné que \( \varphi_n(t)<1\) pour tout \( t\) et pour tout \( n\), la fonction \( t\mapsto u'(t)\varphi_n(t)\) est dominée par \( u'\), qui est dans \( L^1(I)\) par le lemme~\ref{LemTLHwYzD}. Le théorème de la convergence dominée nous permet donc d'affirmer que
	\begin{equation}
		\lim_{n\to \infty} \int_Iu'\varphi_n=\int_Iu'\mtu_{[x,y]}=\int_x^yu',
	\end{equation}
	et donc les inégalités \eqref{EqEBwWUxm} donnent le résultat, grâce au signe dans \eqref{EqRPwqpve}.
\end{proof}

\begin{corollary}   \label{CorCEPJGAu}
	Si \( [u]\in H^1(I)\), le représentant continu \( u\in C^0(I)\) peut être prolongé par continuité en \( u\in C^0(\bar I)\).
\end{corollary}

\begin{proof}
	Soit \( (x_n)\) une suite strictement croissante dans \( \mathopen] a , b \mathclose[\) convergeant vers \( b\). Nous voulons montrer que la suite \( \big( u(x_n) \big)\) est de Cauchy dans \( \eR\), ce qui nous permettra de définir
	\begin{equation}
		u(b)=\lim_{n\to \infty} u(x_n).
	\end{equation}
	qui sera évidemment continue. Cette construction ne dépendra pas du choix de la suite \( (x_n)\) parce que deux fonctions continues sur \( \bar I\) et égales sur \( I\) sont égales sur \( \bar I\).

	En notant \( u'\) la dérivée de \( u\) dans \( H^1\), nous avons par construction du représentant continu : \( u(x)=\int_{y_0}^xu'(t)dt\). Et donc
	\begin{equation}
		\big| u(x_n)-u(x_{n+p}) \big|=\left| \int_{y_0}^{x_n}u'-\int_{y_0}^{x_{n+p}}u' \right| =\left| \int_{x_n}^{x_{n+p}}u' \right| .
	\end{equation}
	Vu que la suite \( (x_n)\) est de Cauchy et que \( u'\) est intégrable (même sur \( \bar I\)), la limite \( n\to\infty\) de cela est zéro, quelle que soit la valeur de \( p\). Donc \( \big( u(x_n) \big)\) est de Cauchy dans \( \eR\) et est donc convergente.
\end{proof}
\index{prolongement!par continuité!dans \( H^1(I)\)}

\begin{proposition}[\cite{KXjFWKA,MonCerveau}]     \label{ThoESIyxfU}
	Quelques propriétés de l'espace de Sobolev \( H^1(I)\) où \( I=\mathopen] a , b \mathclose[\) est un ouvert borné de \( \eR\).
	\begin{enumerate}
		\item       \label{ITEMooITQUooKWwMwu}
		      \( H^1(I)\) est un espace de Hilbert.
		\item
		      \( H^1(I)\) s'injecte de façon compacte dans \( C^0(\bar I)\).
		\item
		      \( H^1(I)\) s'injecte de façon continue dans \( L^2(I)\).
	\end{enumerate}
\end{proposition}
\index{espace!de fonctions!Sobolev \( H^1\)}
\index{espace!de Hilbert!espace de Sobolev \( H^1\)}
\index{espace!\( L^2\)!Sobolev}
\index{dérivation!au sens des distribution!Sobolev}


\begin{proof}
	Nous prouvons point par point.
	\begin{enumerate}
		\item
		      Le seul critère à vérifier est la complétude. Pour cela nous considérons une suite de Cauchy \( (u_n)\) dans \( H^1(I)\). Si \( \epsilon>0\), alors il existe \( N>0\) tel que pour tout \( p\geq 0\) nous ayons \( \| u_{n+p}-u_n \|_{H^1}^2\leq \epsilon\), c'est-à-dire
		      \begin{equation}
			      \| u_{n+p}-u_n \|^2_{L^2}+\| u'_{n+p}-u'_n \|^2_{L^2}\leq \epsilon.
		      \end{equation}
		      En particulier les suites \( (u_n)\) et \( (u'_n)\) sont de Cauchy dans \( L^2\) qui est complet par le théorème de Fischer-Riesz~\ref{ThoGVmqOro}. Nous notons donc
		      \begin{subequations}
			      \begin{align}
				      u_n\stackrel{L^2}{\to}u \\
				      u'_n\stackrel{L^2}{\to}v.
			      \end{align}
		      \end{subequations}

		      Nous allons maintenant montrer quelques limites.
		      \begin{subproof}
			      \spitem[\( u_n\varphi\stackrel{L^2}{\longrightarrow}u\varphi\)]
			      %-----------------------------------------------------------
			      Si \( M\) est une constante qui majore \( \varphi\) alors \( \| u_n\varphi-u\varphi \|_2\leq M\| u_n-u \|_2\to 0\).

			      \spitem[\( u'_n\varphi\stackrel{L^2}{\longrightarrow}v\varphi\)]
			      %-----------------------------------------------------------
			      C'est la même chose avec \( \| u'_n\varphi-v\varphi \|_2\leq M\| u'_n-v \|_2\to 0\).

			      \spitem [\( u\in H^1(I)\) avec \( u'=v\)]
			      %-----------------------------------------------------------
			      Attendu le corolaire~\ref{CORooIIEAooNmbkTo} qui permet de permuter intégrale et limite dans \( L^2(I)\) et les limites que nous venons de prouver,
			      \begin{equation}
				      \int_Iu\varphi'=\lim_{n\to \infty} \int_Iu_n\varphi'=-\lim_{n\to \infty} \int_Iu'_n\varphi=-\int_Iv\varphi.
			      \end{equation}
			      Cela signifie que \( v\) est la dérivée faible de \( u\) : \( u'=v\).

			      \spitem[\( u_n\stackrel{H^1}{\to}u\)]
			      %-----------------------------------------------------------


			      Nous pouvons alors prouver que \( u_n\to u\) dans \( H^1(I)\) :
			      \begin{equation}
				      \| u_n-u \|^2_{H^1(I)}=\| u_n-u \|^2_{L^2}+\| u'_n-u' \|_{L^2}^2.
			      \end{equation}
			      Mais nous savons déjà que \( u_n\to u\) dans \( L^2\) (d'ailleurs c'est la définition de \( u\)) et que \( u'=v\) alors que par définition de \( v\), nous avons \( u'_n\to v\) dans \( L^2\).

		      \end{subproof}

		      Tout cela donne que \( u_n\to u\) dans \( H^1(I)\) et donc que \( H^1(I)\) est un espace complet.

		\item

		      L'application que nous allons prouver être compacte entre \( H^1(I)\) et \( C^0(\bar I)\) est
		      \begin{equation}
			      \begin{aligned}
				      \psi\colon H^1(I) & \to C^0(\bar I)  \\
				      [u]               & \mapsto \tilde u
			      \end{aligned}
		      \end{equation}
		      où \( [u]\) désigne une classe de fonction dans \( H^1(I)\) et \( \tilde u\) est son représentant continu prolongé par continuité à \( \bar I\)\footnote{Encore que par souci d'économie d'encre nous n'allons pas écrire toujours les tildes et noter \( u\) le représentant continu prolongé à \( \bar I\) par le corolaire~\ref{CorCEPJGAu}.}, qui existe par le lemme~\ref{LemMPkbZxX} et le corolaire~\ref{CorCEPJGAu}. Cette application est une injection par l'unicité du représentant continu. Nous allons prouver que c'est une application compacte en utilisant le critère~\ref{ItemJIkpUbLii} de la proposition~\ref{PropDGsPtpU}. Pour cela nous allons commencer par utiliser le théorème d'Ascoli sur l'ensemble \( \tilde \mB\) des représentants continus des éléments de \( \mB\), prolongés par continuité sur \( \bar I\); c'est-à-dire \( \tilde B\subset C^0(\bar I)\).

		      Soit \( u\in \tilde \mB\); par la proposition~\ref{PropGWOIoDg}, nous avons
		      \begin{subequations}
			      \begin{align}
				      \big| u(x)-u(y) \big| & =\big| \int_y^xu'(t)dt \big|                                      \\
				                            & =\left| \int_I\mtu_{[x,y]}(t)u'(t)dt \right|                      \\
				                            & \leq\| \mtu_{\mathopen[ x , y \mathclose]} \|_{L^2}\| u' \|_{L^2} \\
				                            & \leq\sqrt{| x-y |}\| u' \|_{H^1}                                  \\
				                            & \leq\sqrt{| x-y |}.
			      \end{align}
		      \end{subequations}
		      où nous insistons sur le fait que la continuité n'impliquant pas la dérivabilité, le \( u'\) ici est la dérivé au sens de \( H^1\), et non la dérivée usuelle. Quoi qu'il en soit, l'ensemble \(\tilde  \mB\) est équicontinu\footnote{Définition \ref{DEFooDHQDooFfIvsX}.}. Nous montrons à présent qu'il est également borné pour la norme uniforme. Soit \( u\in\tilde \mB\); vu la construction du représentant continu au lemme~\ref{LemMPkbZxX}, nous avons
		      \begin{subequations}
			      \begin{align}
				      \big| u(x) \big| & =\left| \frac{1}{ b-a }\int_a^bu(x)dy \right|                                                           \\
				                       & =\left| \frac{1}{ b-a }\int_a^b\left( \int_y^xu'(t)dt-u(y) \right)dy \right|                            \\
				                       & =\left| \frac{1}{ b-a }\int_a^b\int_y^xu'(t)dtdy-\frac{1}{ b-a }\int_a^b u(y)dy \right|                 \\
				                       & \leq\frac{1}{ b-a }\int_a^b\int_a^b| u'(t) |dt\,dy+\frac{1}{ b-a }\int_a^b| u(y) |dy \label{EqCFwSOxh}.
			      \end{align}
		      \end{subequations}
		      À ce niveau, il faut remarquer que dans la première intégrale, le passage de la valeur absolue à l'intérieur de l'intégrale en même temps que l'élargissement des bornes n'a rien d'innocent. Si \( x<y\), les bornes ne sont pas «dans le bon ordre» et nous ne pouvons pas faire la majoration usuelle en entrant simplement la valeur absolue. Ici nous tenons compte de cela en élargissant les bornes, et en les mettant dans le bon ordre. Le passage exact est le suivant : si \( x,y\in\mathopen] a , b \mathclose[\), nous avons
		      \begin{equation}
			      \left| \int_y^xf(t)dt \right| \leq\left| \int_y^x| f(t) |dt \right| \leq\left| \int_a^b| f(t) |dt \right| =\int_a^b| f(t) |dt.
		      \end{equation}
		      Notons en particulier que dans le cas du passage vers l'équation \eqref{EqCFwSOxh}, le nombre \( x\) est fixé alors que \( y\) est une variable d'intégration. Donc l'ordre des deux est certainement de temps en temps le «mauvais».

		      Quoi qu'il en soit, la première intégrale se réduit à une multiplication par \( b-a\) et le calcul continue :
		      \begin{subequations}
			      \begin{align}
				      \big| u(x) \big| & \leq \int_I| u'(t) |dt+\frac{1}{ b-a }\int_I| u |                                            \\
				                       & \leq \sqrt{b-a}\| u' \|_{L^2}+\frac{1}{ \sqrt{b-a} }\| u \|_{L^2}                            \\
				                       & \leq\left( \sqrt{b-a}+\frac{1}{ \sqrt{b-a} } \right)\big( \| u' \|_{L^2}+\| u \|_{L^2} \big) \\
				                       & \leq\left( \sqrt{b-a}+\frac{1}{ \sqrt{b-a} } \right) \| u \|_{H^1}                           \\
				                       & = \sqrt{b-a}+\frac{1}{ \sqrt{b-a} }.
			      \end{align}
		      \end{subequations}
		      Donc \( \tilde \mB\) est borné pour la norme \( L^{\infty}\). Et c'est même borné par un nombre facilement calculable connaissant \( I\). En particulier l'ensemble
		      \begin{equation}
			      \{ u(x)\tq u\in H^1 \}
		      \end{equation}
		      est, pour tout \( x\), contenu dans la boule de rayon \( \sqrt{b-a}+\frac{1}{ \sqrt{b-a} }\) et donc est relativement compact dans \( \eR\). Par conséquent le théorème d'Ascoli~\ref{ThoKRbtpah} nous dit que l'ensemble \( \tilde \mB\) est relativement compact dans \( C^0(I)\).

		      Par conséquent nous avons montré que l'image par \( \psi\) de la boule unité fermée \( \mB\) de \( H^1(I)\) est relativement compacte dans \( C^0(\bar I)\), ce qui signifie que \( \psi\) est une application compacte.


		\item

		      Les éléments de \( H^1(I)\) sont des éléments de \( L^2(I)\); donc l'identité est une injection. Nous devons seulement étudier la continuité. Si \( (u_n)\) est une suite dans \( H^1\) convergeant dans \( H^1\) vers \( u\), alors
		      \begin{equation}
			      \| u_n-u \|_{L^2}\leq\| u_n-u \|_{L^2}+\| u'_n-u' \|_{L^2}=\| u_n-u \|_{H^1}\to 0.
		      \end{equation}
		      Donc la suite des images (par l'identité) converge dans \( L^2\). L'identité est donc continue.

	\end{enumerate}

\end{proof}

%---------------------------------------------------------------------------------------------------------------------------
\subsection{Sur un ouvert de \( \eR^n\)}
%---------------------------------------------------------------------------------------------------------------------------

Soit \( \Omega\), un ouvert de \( \eR^n\) et \( v\in L^2(\Omega)\) (voir \ref{NORMooUEIEooYtlFse}). Les fonctions considérées sont à valeurs réelles.


%///////////////////////////////////////////////////////////////////////////////////////////////////////////////////////////
\subsubsection{Définition}
%///////////////////////////////////////////////////////////////////////////////////////////////////////////////////////////

\begin{definition}[Espace de Sobolev \( H^1(\Omega)\)]
	Soit \( \Omega\) une partie de \( \eR^n\). L'espace de \defe{Sobolev}{espace de Sobolev} \( H^1(\Omega)\)\nomenclature[Y]{\( H^1(\Omega)\)}{espace de Sobolev sur \( \Omega\)} est:
	\begin{equation}
		H^1(\Omega)=\{ v\in L^2(\Omega)\tq \forall i=1,\ldots, n, \partial_iv\in L^2(\Omega) \}.
	\end{equation}
	Nous munissons cet espace d'un produit scalaire
	\begin{equation}        \label{EQooQRMKooLaMpcp}
		(u,v)_{H^1}=\langle u, v\rangle_{L^2}+\langle \nabla u, \nabla v\rangle_{L^2},
	\end{equation}
	où \( \nabla u=\sum_i\partial_iu\in L^2\).
\end{definition}
L'existence des intégrales dans le produit scalaire est assurée par le fait que \( u\), \( v\), \( \nabla u\) et \( \nabla v\) sont dans \( L^2(\Omega)\). La définition du produit scalaire dans \( L^2\) est la définition~\ref{DefProdScalLubrgTj} (mais sans la conjugaison complexe).

Pour la même raison, \( (u,u)_{H^1}=0\) demande que chacun des deux termes est séparément nul, et nous avons \( u=0\) dans \( L^2\), et donc aussi dans \( H^1\).

\begin{theorem}[\cite{ooYWZMooXnzOQp}]
	L'espace \( H^1(\Omega)\) est un espace de Hilbert\footnote{Définition~\ref{DefORuBdBN}.}.
\end{theorem}

\begin{proof}
	Nous devons nous assurer que l'espace \( H^1\) est complet. Pour cela nous considérons une suite de Cauchy \( (u_n)\) dans \( H^1\). Soit \( \epsilon>0\); il existe \( N>0\) tel que si \( n,m>N\) alors \( \| u_n-u_m \|_{H^1}<\epsilon\). Or en déballant la définition de la norme et du produit scalaire,
	\begin{subequations}
		\begin{align}
			\| u_m-u_n \|_{H^1}^2 & =(u_m-u_n,u_m-u_n)_{H^1}                                                            \\
			                      & =\langle u_m-u_n, u_m-u_n\rangle +\langle \nabla (u_m-u_n), \nabla (u_m-u_n)\rangle \\
			                      & =\| u_m-u_n \|_{L^2}^2+\| \nabla (u_m-u_n) \|_{L^2}^2.
		\end{align}
	\end{subequations}
	En particulier les suites \( (u_n)\) et \( (\nabla u_n)\) sont de Cauchy dans \( L^2\). Vu que \( L^2\), lui, est complet (théorème~\ref{ThoUYBDWQX}), il existe \( u\in L^2\) et \( v_i\in L^2\) tels que
	\begin{subequations}
		\begin{align}
			u_n\stackrel{L^2}{\longrightarrow}u \\
			\partial_iu_n\stackrel{L^2}{\longrightarrow}v_i.
		\end{align}
	\end{subequations}
	Nous savons que l'injection \( i\colon L^2\to \swD'\) est continue par la proposition~\ref{PROPooYAJSooMSwVOm}. Nous avons donc aussi les limites
	\begin{subequations}
		\begin{align}
			T_{u_n}\stackrel{\swD'}{\longrightarrow}T_u \\
			T_{\partial_i u_n}\stackrel{\swD'}{\longrightarrow}T_{v_i}.     \label{SUBEQooMWLIooWakTkx}
		\end{align}
	\end{subequations}
	La dérivée étant une opération continue sur \( \swD'\) nous avons de plus
	\begin{equation}
		\partial_i(T_{u_n})\stackrel{\swD'}{\longrightarrow}\partial_i(T_u)
	\end{equation}
	En utilisant le lemme~\ref{LEMooQRUOooWVjCAV} nous avons alors
	\begin{equation}
		T_{\partial_iu_n}=\partial_i(T_{u_n})\stackrel{\swD}{\longrightarrow}\partial_i(T_u)=T_{\partial_iu}.
	\end{equation}
	En comparant avec \eqref{SUBEQooMWLIooWakTkx} et par l'unicité de la limite, nous avons \( T_{v_i}=T_{\partial_iu}\). Cela implique \( v_i=\partial_iu\).

	Vu que \( v_i\in L^2\) nous avons aussi \( \partial_iu\in L^2\). Par conséquent \( u\in H^1(\Omega)\) parce que ses dérivées sont dans \( L^2\).

	Nous devons maintenant prouver que \( u_n\stackrel{H^1}{\longrightarrow}u\). Nous avons
	\begin{equation}
		\| u_n-u \|^2_{H^1}=\| u_n -u\|^2_{L^2}+\| \nabla u_n-\nabla u \|^2_{L^2}
	\end{equation}
	Le premier terme tend vers zéro parce que \( u_n\stackrel{L^2}{\longrightarrow}u\) et le second parce que \( \partial_iu_n\stackrel{L^2}{\longrightarrow}\partial_iu\).
\end{proof}

%---------------------------------------------------------------------------------------------------------------------------
\subsection{Espace de Sobolev fractionnaire}
%---------------------------------------------------------------------------------------------------------------------------

\begin{definition}
	Pour \( m\in \eN\) et un ouvert \( \Omega\) de \( \eR^d\) nous définissons l'\defe{espace de Sobolev}{espace!de Sobolev}\nomenclature[Y]{\( H^m(M)\)}{espace de Sobolev}
	\begin{equation}
		H^m(\Omega)=\{ u\in L^2(\Omega)\tq \partial^{\alpha}u\in L^2(\Omega)\,\forall | \alpha |\leq m \}.
	\end{equation}
	Nous définissons également un produit scalaire sur \( H^m\) par
	\begin{equation}
		(u,v)_{H^m}=\sum_{| \alpha |\leq m}\langle \partial^{\alpha}u,  \partial^{\alpha}v  \rangle_{L^2}.
	\end{equation}
\end{definition}
En particulier la topologie est celle de la norme dérivée du produit scalaire :
\begin{equation}        \label{EQooMCWMooKKTqzM}
	\| u \|^2_{H^m(\Omega)}=\sum_{| \alpha |\leq m}\| \partial^{\alpha}u \|^2_{L^2(\Omega)}.
\end{equation}

Le lemme suivant montre que la proposition~\ref{LemQPVQjCx} fonctionne encore avec \( L^2\) au lieu de \( \swS\).
\begin{lemma}[Lemme de transfert\cite{ooUYIYooGJyIPi}, thème~\ref{THEMEooJREIooKEdMOl}] \label{LEMooAGBZooWCbPDd}
	Soit \( f\in H^m(\eR^d)\). Alors pour tout multiindice \( \alpha\) avec \( | \alpha |\leq m\) nous avons
	\begin{equation}
		\TF(\partial^{\alpha}f)=\big[ \xi\mapsto i^{| \alpha |}\xi^{\alpha}\hat f(\xi) \big].
	\end{equation}
\end{lemma}

\begin{lemma}
	Il existe des constantes \( c_1\) et \( c_2\) telles que pour tout \( x\in \eR^d\),
	\begin{equation}
		c_1(1+\| x \|^2)^m\leq \sum_{| \alpha |\leq m}(x^{\alpha})^2\leq c_2(1+\| x \|^2)^m.
	\end{equation}
\end{lemma}

\begin{lemma}
	Soit \( u\in L^2(\eR^d)\). Nous avons \( u\in H^m(\eR^d)\) si et seulement si l'application
	\begin{equation}
		\xi\mapsto \big( 1+| \xi |^2 \big)^{k/2}\hat u
	\end{equation}
	est dans \( L^2(\eR^d)\) pour tout \( k\leq m\). Ici \( | \xi |\) est la norme euclidienne de \( \xi\) dans \( \eR^d\).
\end{lemma}

\begin{proof}
	Vu le lemme~\ref{LEMooAGBZooWCbPDd}, il suffit de montrer que
	\begin{equation}        \label{EQooIJXTooWsGNxw}
		\big( 1+| \xi |^2 \big)^{k/2}\hat u
	\end{equation}
	est dans \( L^2\) pour tout \( k\leq m\) si et seulement si
	\begin{equation}        \label{EQooILPQooNGUvjD}
		\xi^{\alpha}\hat u
	\end{equation}
	l'est pour tout \( \alpha\) avec \( | \alpha |\leq m\).

	L'expression \eqref{EQooIJXTooWsGNxw} est une somme d'expressions du type \eqref{EQooILPQooNGUvjD}. Donc l'implication dans un sens est montrée. Pour l'autre sens, nous savons que
	\begin{equation}
		\xi^{\alpha}=\xi_1^{\alpha_1}\ldots \xi_n^{\alpha_n},
	\end{equation}
	et donc
	\begin{equation}
		| \xi^{\alpha} |\leq | \xi_1 |^{\alpha_1}\ldots | \xi_n |^{\alpha_n}.
	\end{equation}
	Or \( | \xi |^{| \alpha |}=| \xi |^{\sum_i\alpha_i}=| \xi |^{\alpha_1}\ldots | \xi |^{\alpha_n}\) et \( | \xi |\geq | \xi_i |\) pour tout \( i\), donc
	\begin{equation}
		| \xi^{\alpha} |\leq | \xi |^{| \alpha |}.
	\end{equation}

	D'autre part pour tout \( x\in \eR^+\) et tout \( k\) positif nous avons
	\begin{equation}
		(1+x^2)^{k/2}\geq x^k
	\end{equation}
	qui est facile à vérifier en prenant le carré des deux membres.

	En remettant tout ensemble,
	\begin{equation}
		| \xi^{\alpha}\hat u |\leq | \xi^{\alpha} | |\hat u |\leq | \xi |^{| \alpha |}| \hat u |\leq \big( 1+| \xi |^2 \big)^{| \alpha |/2}| \hat u |.
	\end{equation}
	Donc si le membre de droite est de carré intégrable, celui de gauche l'est également.
\end{proof}

\begin{definition}[Espace de Sobolev \( H^s\)\cite{ooCVEMooUxVWwU}]     \label{DEFooWEAQooAIWBwx}
	Pour \( s>0\) nous définissons l'espace de Sobolev \( H^s(\eR^d)\) par
	\begin{equation}
		H^s(\eR^d)=\{ u\in L^2(\eR^d)\tq  \big( 1+\| \xi \|^2 \big)^{s/2}\hat u\in L^2(\eR^d)  \}.
	\end{equation}
	Nous y mettons le produit scalaire
	\begin{equation}
		(u,v)_{H^s}=\int_{\eR^d}\hat u(\xi)\overline{ \hat v(\xi) }(1+\| \xi \|^2)^sd\xi.
	\end{equation}
\end{definition}

\begin{normaltext}
	Vu que \( \swD(\eR^d)\) est dense dans \( L^2(\eR^d)\) (théorème~\ref{ThoILGYXhX}), on pourrait croire à la densité a fortiori dans \( H^s(\eR^d)\). Mais attention : \( \swD(\eR^d)\) est dense dans \( L^2\) pour la norme \( L^2\). Nous n'avons encore rien dit pour la norme \( H^s(\eR^d)\).
\end{normaltext}

\begin{proposition}[\cite{ooBYHTooRZtLLi}]      \label{PROPooMKAFooKDNTbO}
	La partie \( \swS(\eR^d)\) est dense dans \( H^s(\eR^d)\).
\end{proposition}

\begin{proof}
	Soit \( u\in H^s(\eR^d)\). Par définition l'application
	\begin{equation}
		\xi\mapsto (1+\| \xi \|^2)^{s/2}\hat u
	\end{equation}
	est dans \( L^2(\eR^d)\). Elle peut donc être approximée au sens \( L^2\) par des fonctions dans \( \swD(\eR^d)\) (théorème~\ref{ThoILGYXhX}\ref{ItemYVFVrOIv}), c'est-à-dire qu'il existe des fonctions \( \phi_n\in\swD(\eR^d)\) telles que
	\begin{equation}
		\phi_n\stackrel{L^2(\eR^d)}{\longrightarrow}(1+\| \xi \|^2)^{s/2}\hat u.
	\end{equation}
	Nous posons
	\begin{equation}
		\psi_n=\frac{ \phi_n }{ (1+\xi^2)^{s/2} }
	\end{equation}
	Cela est encore une fonction de \( \swD(\eR^d)\), et donc de \( \swS(\eR^d)\). Vu que la transformée de Fourier est une bijection de \( \swD(\eR^d)\) (proposition~\ref{PROPooLWTJooReGlaN}), nous pouvons considérer une suite \( \varphi_n\in\swD(\eR^d)\) telle que \( \hat \varphi_n=\psi_n \), et nous allons montrer que \( \varphi_n\stackrel{H^s(\eR^d)}{\longrightarrow}u\).

	Nous avons :
	\begin{subequations}
		\begin{align}
			\| \varphi_n-u \|^2_{H^s} & =\int_{\eR^d}| \hat \varphi_n-\hat u |^2(1+\xi^2)^sd\xi                                       \\
			                          & =\int_{\eR^d}\big| \frac{ \phi_n(\xi) }{ (1+\xi^2)^{s/2} }-\hat u(\xi) \big|^2(1+\xi^2)^sd\xi \\
			                          & =\int_{\eR^d}| \phi_n(\xi)-\hat u(\xi)(1+\xi^2)^{s/2} |^2d\xi                                 \\
			                          & =\| \phi_n-(1+\xi^2)^{s/2}\hat u \|^2_{L^2}.
		\end{align}
	\end{subequations}
	Par définition de la suite \( \phi_n\) nous avons donc bien
	\begin{equation}
		\| \varphi_n-u \|^2_{H^s} =\| \phi_n-(1+\xi^2)^{s/2}\hat u \|^2_{L^2}\to 0.
	\end{equation}

	Notons que même si \( \phi_n\) est dans \( \swD(\eR^d)\), nous n'avons pas prouvé la convergence \( \phi_n\stackrel{H^s}{\longrightarrow}u\), mais bien \( \varphi_n\stackrel{H^s}{\longrightarrow}u\). Or les fonctions \( \varphi_n\) sont dans \( \swS(\eR^d)\), et rien n'assure qu'elles soient à support compact. Nous avons donc bien prouvé la densité de \( \swS\) et non celle de \( \swD\).
\end{proof}

\begin{remark}
	Pour qui a tout compris, cela peut sembler une évidence, mais nous précisions que nous parlons de densité de \( \swS(\eR^d)\) dans \( H^s(\eR^d)\), à aucun moment la topologie de \( \swS(\eR^d)\) n'entre en compte.

	Un peu moins évident : ce que nous avons réellement montré est la densité de \( \iota\big( \swD(\eR^d) \big)\) dans \( H^s(\eR^d)\) où \( \iota\) est l'application «prise de classe». Nous n'avons pas insisté là-dessus, mais il faut dire que dans la preuve de la proposition~\ref{PROPooMKAFooKDNTbO}, \( u\) est un représentant d'un élément choisi dans \( H^s(\eR^d)\).

	Nous avons ensuite prouvé la convergence \( \| \varphi_n-u \|_{H^s(\eR^d)}\to 0\) qui est une convergence d'une suite dans \( \eR\), et dans laquelle l'opération \( \| . \|_{H^s}\) est définie sur un espace de fonctions et n'est pas une norme (c'est pour que cela devienne une norme que l'on prend les classes).

	Nous en avons déduit la convergence \( \varphi_n\stackrel{H^s(\eR^d)}{\longrightarrow}u\) où maintenant \( \varphi_n\) et \( u\) sont des classes dans \( H^s(\eR^d)\).
\end{remark}

\begin{proposition}     \label{PROPooLIQJooKpWtnV}
	La partie \( \swD(\eR^d)\) est dense dans \( \big( H^s(\eR^d),\| . \|_{H^s(\eR^d)} \big)\).
\end{proposition}

\begin{proof}
	Nous savons déjà que \( \swS(\eR^d)\) est dense dans \( H^s(\eR^d)\) par la proposition~\ref{PROPooMKAFooKDNTbO}. Nous devons seulement prouver que \( \swD(\eR^d)\) est dense dans \( \big( \swS(\eR^d),\| . \|_{H^s(\eR^d)} \big)\). Pour cela nous utilisons la densité de \( \swD(\eR^d)\) dans \( \swS(\eR^d)\) de la proposition~\ref{PROPooJNQZooIRbJei}. Soit donc \( f\in\swS(\eR^d)\) et une suite \( f_k\) dans \( \swD(\eR^d)\) telle que
	\begin{equation}
		f_k\stackrel{\swS(\eR^d)}{\longrightarrow}f.
	\end{equation}
	Vu que la transformée de Fourier est continue sur \( \swS(\eR^d)\) (proposition~\ref{PropKPsjyzT}) nous avons aussi
	\begin{equation}
		\hat f_k\stackrel{\swS(\eR^d)}{\longrightarrow}\hat f,
	\end{equation}
	et en particulier pour tout polynôme \( P\) nous avons la convergence uniforme
	\begin{equation}        \label{EQooDTMUooQWphpR}
		P\hat f_k\stackrel{unif}{\longrightarrow}P\hat f.
	\end{equation}

	D'autre part la fonction \( \xi\mapsto | \hat f_k(\xi)-\hat f(\xi) |^2(1+\xi^2)^s\) est Schwartz et en tout point décroissante en \( k\). Soientt \( \epsilon>0\) et \( r>0\) choisis de telle sorte à avoir
	\begin{equation}
		\int_{\| \xi \|>r}| \hat f_{k}(\xi)-\hat f(\xi) |(1+\xi^2)^sd\xi<\epsilon.
	\end{equation}
	pour tout \( k\). La convergence uniforme \eqref{EQooDTMUooQWphpR} permet de considérer \( k_0\) tel que  pour tout \( k>k_0\),
	\begin{equation}
		| \hat f_k-\hat f |(1+\xi^2)^s< \frac{ \epsilon }{ \Vol\big( B(0,r) \big) }
	\end{equation}
	dans \( B(0,r)\). Avec tout cela, dès que \( k>k_0\) nous avons
	\begin{equation}
		\| f_k-f \|_{H^s(\eR^d)}=\int_{\eR}| \hat f_k-\hat f |(1+\xi^2)^sd\xi=\int_{B(0,r)}\ldots +\int_{\| \xi \|>r}\ldots\leq 2\epsilon.
	\end{equation}
	Donc nous avons bien \( \| f_k-f \|_{H^s(\eR^d)}\to 0\) et convergence de \( f_k\) vers \( f\) dans \( H^s(\eR^d)\).
\end{proof}

%+++++++++++++++++++++++++++++++++++++++++++++++++++++++++++++++++++++++++++++++++++++++++++++++++++++++++++++++++++++++++++
\section{Trace}
%+++++++++++++++++++++++++++++++++++++++++++++++++++++++++++++++++++++++++++++++++++++++++++++++++++++++++++++++++++++++++++

\begin{definition}[\cite{ooCVEMooUxVWwU}]
	Nous définissons la \defe{trace}{trace} d'une fonction par
	\begin{equation}
		\begin{aligned}
			\gamma_0\colon   \swD(\eR^d)     & \to \swD(\eR^{d-1})        \\
			(\gamma_0v)(x_1,\ldots, x_{d-1}) & =v(x_1,\ldots, x_{d-1},0).
		\end{aligned}
	\end{equation}
\end{definition}

\begin{theorem}[\cite{ooFGHJooNYYBIp,ooCVEMooUxVWwU}]       \label{THOooXEJZooBKtXBW}
	Si \( s>\frac{ 1 }{2}\), alors \( \gamma_0\) accepte une unique extension en opérateur linéaire borné
	\begin{equation}
		\gamma_0\colon H^s(\eR^d)\to H^{s-\frac{ 1 }{2}}(\eR^{d-1}).
	\end{equation}
\end{theorem}

\begin{proof}
	Nous subdivisons la preuve en plusieurs pas.
	\begin{subproof}
		\spitem[Une inégalité pour \( \varphi\in\swD(\eR^d)\)]
		Nous commençons par considérer \( \varphi\in \swD(\eR^d)\) (fonction \(  C^{\infty}\) à support compact). Nous allons alors prouver que
		\begin{equation}
			\| \gamma_0\varphi \|_{H^{s-\frac{ 1 }{2}}(\eR^{d-1})}\leq K\| \varphi \|_{H^s(\eR^d)}
		\end{equation}
		pour une certaine constante \( K\) (qui ne dépend en particulier pas de \( \varphi\)).

		Nous avons
		\begin{subequations}
			\begin{align}
				\| \gamma_0\varphi \|^2_{H^{s-\frac{ 1 }{2}}(\eR^{d-1})} & =(  \gamma_0\varphi,\gamma_0\varphi  )_{H^{s-\frac{ 1 }{2}}}                                                                                       \\
				                                                         & =\int_{\eR^{d-1}}| \widehat{\gamma_0\varphi(\xi)} |^2(1+\| \xi \|^2)^{s-\frac{ 1 }{2}}d\xi                                                         \\
				                                                         & =\int_{\eR^{d-1}}\Big| \int_{\eR^{d-1}}(\gamma_0\varphi)(x) e^{-i\xi x}dx \Big|^2(1+\| \xi \|^2)^{s-\frac{ 1 }{2}}d\xi \label{SUBEQooKLLYooNRjPwn} \\
			\end{align}
		\end{subequations}
		Nous appliquons la trace en appliquant la formule du corolaire~\ref{CORooAZLZooSviTej},
		\begin{equation}
			(\gamma_0\varphi)(x)= \varphi(x,0) =\frac{1}{ 2\pi }\int_{\eR}\int_{\eR} e^{-iky}\varphi(x,y)dy\,dk
		\end{equation}
		En remplaçant dans \eqref{SUBEQooKLLYooNRjPwn} nous avons
		\begin{equation}
			\| \gamma_0\varphi \|^2_{H^{s-\frac{ 1 }{2}}(\eR^{d-1})}=\frac{1}{ 2\pi }\int_{\eR^{d-1}}\big| \int_{\eR^{d-1}}\int_{\eR}\int_{\eR} e^{-iky} e^{-i\xi x}\varphi(x,y)dy\,dk\,dx \big|^2(1+\| \xi \|^2)^{s-\frac{ 1 }{2}}d\xi.
		\end{equation}
		Nous voudrions permuter les intégrales en \( k\) et en \( x\). Pour cela nous étudions la fonction \( u\colon \eR\times \eR^{d-1}\to \eC\) donnée par
		\begin{equation}
			u(k,x)= e^{-ikx}\int_{\eR} e^{-i\xi x}\varphi(x,y)dy
		\end{equation}
		Effectuer l'intégrale par rapport à \( y\) revient à calculer la transformée de Fourier partielle dont nous parlons dans la proposition~\ref{PROPooMVQMooGYAzSX}\quext{Dont une relecture de la preuve ne serait vraiment pas de trop, ainsi que la preuve de~\ref{CORooZFPSooHCFUSH}.}. Elle est donc une fonction Schwartz de \( k\) et de \( x\) (conjointement et non seulement séparément) et est donc dans \( L^1(\eR\times \eR^{d-1})\). Les intégrales sur \( k \) et sur \( x\) peuvent donc être réunies et permutées par le théorème de Fubini~\ref{ThoFubinioYLtPI} (n'oubliez tout de même pas de vous convaincre que la condition~\ref{ITEMooCYMKooUdizni} est remplie).

		Nous avons donc
		\begin{equation}
			\| \gamma_0\varphi \|^2_{H^{s-\frac{ 1 }{2}}(\eR^{d-1})}=\frac{1}{ 2\pi }\int_{\eR^{d-1}}|\int_{\eR}\int_{\eR^{d-1}}\int_{\eR} e^{-iky} e^{-i\xi x}\varphi(x,y)dy\,dx\,dk|^2(1+\| \xi \|^2)^{s-\frac{ 1 }{2}}d\xi.
		\end{equation}
		Étant donné que \( \varphi\) est à support compact, les intégrales sur \( x\) et sur \( y\) peuvent se réunir en utilisant encore le théorème de Fubini; ces intégrales donnent :
		\begin{equation}
			\int_{\eR^{d-1}\times \eR} e^{-iky} e^{-i\xi x}\varphi(x,y)dx\otimes dy=\int_{\eR^{d-1}\times \eR} e^{  -i(\xi,k)\cdot(x,y)  }\varphi(x,y)dx\otimes dy=\hat \varphi(\xi,k).
		\end{equation}
		Nous restons avec
		\begin{equation}
			\| \gamma_0\varphi \|^2_{H^{s-\frac{ 1 }{2}}(\eR^{d-1})}=\frac{1}{ 2\pi }\int_{\eR^{d-1}}|   \int_{\eR}\hat \varphi(\xi,k)dk    |^2(1+\| \xi \|^2)^{s-\frac{ 1 }{2}}d\xi.
		\end{equation}
		Nous allons maintenant traiter la partie du milieu :
		\begin{equation}
			\clubsuit=|\int_{\eR}\hat \varphi(\xi,k)dk|=| \int_{\eR}\hat \varphi(\xi,k)(1+\xi^2+k^2)^{s/2}\frac{1}{ (1+\xi^2+k^2)^{s/2} }dk |=| \langle f_1, f_2\rangle_{L^2(\eR^d)} |
		\end{equation}
		Ici \( \xi\) est vu comme une constante et les fonctions \( f_1\) et \( f_2\) sont
		\begin{subequations}
			\begin{align}
				f_1 & \colon k\mapsto \hat \varphi(\xi,k)(1+\xi^2+k^2)^{s/2} \\
				f_2 & \colon k\mapsto \frac{1}{ (1+\xi^2+k^2)^{s/2} }
			\end{align}
		\end{subequations}
		Nous pouvons utiliser l'inégalité de Cauchy-Schwarz~\ref{ThoAYfEHG} :
		\begin{equation}
			\clubsuit\leq \left( \int_{\eR}| \hat \varphi(\xi,k) |^2(1+\xi^2+k^2)^sdk \right)^{1/2}\left( \int_{\eR}\frac{1}{ (1+\xi^2+k^2)^s }dk \right)^{1/2}
		\end{equation}
		Nous notons \( g(\xi)\) ce qui se trouve dans la seconde parenthèse (après intégration sur \( k\)). Avec cela nous continuons :
		\begin{equation}
			\| \gamma_0\varphi \|^2_{H^{s-\frac{ 1 }{2}}(\eR^{d-1})}\leq\frac{1}{ 2\pi }\int_{\eR^{d-1}}\int_{\eR}| g(\xi) | |\hat \varphi(\xi,k) |^2(1+\xi^2+k^2)^s(1+\| \xi \|^2)^{s-\frac{ 1 }{2}}dk\,d\xi.
		\end{equation}
		Vu que \( \hat \varphi\) est Schwartz, la fonction qui est à l'intérieur des deux intégrales est dans \( L^1(\eR^{d-1}\times \eR)\) et nous pouvons réunir les deux intégrales :
		\begin{equation}
			\| \gamma_0\varphi \|^2_{H^{s-\frac{ 1 }{2}}(\eR^{d-1})}\leq\frac{1}{ 2\pi }\int_{\eR\times \eR^{d-1}}| g(\xi) | |\hat \varphi(\xi,k) |^2(1+\xi^2+k^2)^s(1+\| \xi \|^2)^{s-\frac{ 1 }{2}}dk\otimes d\xi.
		\end{equation}
		À ce point nous démontrons qu'en réalité la combinaison \( g(\xi)(1+\xi^2)^{s-\frac{ 1 }{2}}\) ne dépend pas de \( \xi\). En effet
		\begin{subequations}
			\begin{align}
				g(\xi)(1+\xi^2)^{s-\frac{ 1 }{2}} & =(1+\xi^2)^{s-\frac{ 1 }{2}}\int_{\eR}\frac{1}{ (1+\xi^2+k^2) }dk                         \\
				                                  & =\frac{1}{ (1+\xi^2)^{1/2} }\int_{\eR}\left( \frac{ 1+\xi^2 }{ 1+\xi^2+k^2 } \right)^sdk  \\
				                                  & =\frac{1}{ (1+\xi^2)^{1/2} }\int\left( \frac{1}{ 1+\frac{ k^2 }{ 1+\xi^2 } } \right)^sdk.
			\end{align}
		\end{subequations}
		Nous effectuons le changement de variables \( t=\frac{ k }{ \sqrt{ 1+\xi^2 } }\), \( dk=(1+\xi^2)^{1/2}dt\), et le tout vaut
		\begin{equation}
			\int_{\eR}\left( \frac{1}{ 1+t^2 } \right)^sdt,
		\end{equation}
		qui est effectivement indépendant de \( \xi\). Nous nommons cela \( K\) (auquel nous ajoutons le \( \frac{1}{ 2\pi }\)) :
		\begin{equation}
			\| \gamma_0\varphi \|^2_{H^{s-\frac{ 1 }{2}}(\eR^{d-1})}\leq K\int_{\eR\times \eR^{d-1}} |\hat \varphi(\xi,k) |^2(1+\xi^2+k^2)^sdk\otimes d\xi=K\| \varphi \|^2_{H^s(\eR^d)}.
		\end{equation}
		Nous avons donc prouvé pour tout \( \varphi\in\swD(\eR^d)\) (avec redéfinition du \( K\)) :
		\begin{equation}        \label{EQooNWCIooALMivH}
			\| \gamma_0\varphi \|_{H^{s-\frac{ 1 }{2}}(\eR^{d-1})}\leq K\| \varphi \|_{H^s(\eR^d)}.
		\end{equation}
		\spitem[À propos de classes]
		Il serait tentant de conclure en disant que \( \swD(\eR^d)\) est dense dans \( H^s(\eR^d)\). Hélas, \href{https://explosm.net/comics/3613/}{techniquement}, l'ensemble \( \swD(\eR^d)\) n'est même pas un sous-ensemble de \( H^s(\eR^d)\) parce que ce dernier est un ensemble de \emph{classes} de fonctions. Ce petit détail a ici son importance parce que \( \gamma_0\) n'est pas une application qui descend aux classes. En effet, \( \eR^{d-1}\) étant de mesure nulle dans \( \eR^d\), deux fonctions de la même classe peuvent différer en \emph{tous} les points de \( \eR^{d-1}\) en même temps.

		Notons \( \iota\) l'application qui consiste à prendre la classe, c'est à dire \( \iota(f)=[f]\). Ce qui est dense dans \( H^s(\eR^d)\), c'est \( \iota(\swD(\eR^d))\). Or chaque classe contient au maximum une seule fonction continue (qui sera même de classe \(  C^{\infty}\) à support compact pour les éléments de \( \iota(\swD)\)).

		L'application \( \gamma_0\) considérée est l'application composée entre le \( \gamma_0\) classique et le choix du représentant continu dans la classe. La formule \eqref{EQooNWCIooALMivH} que nous venons de prouver est valide pour l'application \( \gamma_0\) vue comme
		\begin{equation}
			\gamma_0\colon \iota\big( \swD(\eR^d) \big)\to \swD(\eR^{d-1}).
		\end{equation}
		\spitem[Densité et conclusion]
		Ce que la majoration \eqref{EQooNWCIooALMivH} prouve est la continuité de l'application
		\begin{equation}
			\gamma_0\colon \big( \swD(\eR^d),\| . \|_{H^s(\eR^d)} \big)\to \big( H^{s-\frac{ 1 }{2}}(\eR^{d-1}),\| . \|_{H^{s-\frac{ 1 }{2}}(\eR^{d-1})} \big).
		\end{equation}
		Mais la proposition~\ref{PROPooLIQJooKpWtnV} nous donne la densité de la partie \( \swD(\eR^d)\) dans \( H^s(\eR^d)\). La proposition~\ref{PropTTiRgAq} nous donne alors une extension
		\begin{equation}
			\gamma_0\colon \big(   H^s(\eR^d), \| . \|_{H^s(\eR^d)}   \big)\to \big( H^{s-\frac{ 1 }{2}}(\eR^{d-1}),\| . \|_{H^{s-\frac{ 1 }{2}}(\eR^{d-1})} \big).
		\end{equation}
	\end{subproof}
\end{proof}

\begin{remark}
	L'extension n'est pas évidente parce que les éléments de \( H^s(\eR^d)\) sont en général des classes de fonctions dont les valeurs sur le bord ne sont pas du tout fixées du fait que le bord soit de mesure nulle.
\end{remark}

%+++++++++++++++++++++++++++++++++++++++++++++++++++++++++++++++++++++++++++++++++++++++++++++++++++++++++++++++++++++++++++
\section{Théorème de plongement}
%+++++++++++++++++++++++++++++++++++++++++++++++++++++++++++++++++++++++++++++++++++++++++++++++++++++++++++++++++++++++++++

L'objet des théorèmes de plongement de Sobolev est de montrer que si \( s>\frac{ d }{ 2 }+k\) alors les éléments de \( H^s(\eR^d)\) possèdent des représentants de classe \( C^k\). Avant de démontrer le théorème, pour alléger, nous allons donner deux lemmes.

\begin{lemma}       \label{LEMooZIBIooANHyPy}
	Soit \( (u_j)\) une suite dans \( \swS(\eR^d)\) telle que
	\begin{equation}
		u_j\stackrel{H^s(\eR^d)}{\longrightarrow}u
	\end{equation}
	avec \( s>0\). Alors nous avons aussi la convergence
	\begin{equation}
		u_j\stackrel{L^2(\eR^d)}{\longrightarrow}u.
	\end{equation}
\end{lemma}

\begin{proof}
	Vu que \( s>0\) nous avons \( (1+k^2)^s>1\) (ici nous écrivons \( k^2\) pour \( \| k \|^2\)). Par conséquent
	\begin{equation}
		(u,v)_{H^s(\eR^d)}=\int_{\eR^d}\hat u\overline{ \hat v }(1+k^2)^sdk\geq \int_{\eR^d}\hat u\overline{ \hat v }dk=\langle \hat u, \hat v\rangle_{L^2(\eR^d)}.
	\end{equation}

	Nous avons alors
	\begin{subequations}
		\begin{align}
			\| u_j-u \|_{L^2} & =\frac{1}{ (2\pi)^d }\| \hat u_j-\hat u \|_{L^2}                    \\
			                  & =\frac{1}{ (2\pi)^d }\int_{\eR^d}| \hat u_j-\hat u |^2              \\
			                  & \leq \frac{1}{(2\pi)^d}\int_{\eR^d}| \hat u_j-\hat u |^2(1+k^2)^sdk \\
			                  & =\frac{1}{ (2\pi)^d }\| u_j-u \|_{H^s(\eR^d)}.
		\end{align}
	\end{subequations}
\end{proof}

\begin{lemma}       \label{LEMooGDTXooJRudME}
	Soient des fonctions \(u_j\in\swS(\eR^d) \) telles que
	\begin{equation}
		u_j\stackrel{\big( C_0^0(\eR^d),\| . \|_{\infty} \big)}{\longrightarrow}v.
	\end{equation}
	Alors nous avons la convergence
	\begin{equation}
		\int_{\eR^d}u_j\varphi\to\int_{\eR^d}v\varphi
	\end{equation}
	pour tout \( \varphi\in\swS(\eR^d)\).
\end{lemma}

\begin{proof}
	La suite \( (u_j)\) est équibornée. En effet il existe une queue de suite pour laquelle \( \| u_j-v \|_{\infty}<\epsilon\); cette queue de suite est alors équibornée par \( \| v \|_{\infty}+\epsilon\). Le début de la suite est un nombre fini de fonctions, toutes bornées. Le maximum des bornes donne alors une borne.

	Soit donc \( M>0\) tel que \( | u_j(x) |<M\) pour tout \( x\in \eR^d\) et pour tout \( j\in \eN\). Nous avons alors \( | u_j\varphi |<M| \varphi |\) pour tout \( j\) et les fonctions \(  | u_j\varphi | \) sont majorées par la fonction \( M| \varphi |\) qui est intégrable. Nous pouvons donc utiliser le théorème de la convergence dominée de Lebesgue~\ref{ThoConvDomLebVdhsTf} nous donne
	\begin{equation}
		\lim_{j\to \infty} \int_{\eR^d} u_j\varphi=\int_{\eR^d}v\varphi.
	\end{equation}
\end{proof}

Nous pouvons écrire la conclusion du lemme~\ref{LEMooGDTXooJRudME} sous la forme
\begin{equation}
	\langle u_j, \varphi\rangle_{L^2(\eR^d)}\to \langle v, \varphi\rangle_{L^2(\eR^d)}
\end{equation}
pour tout \( \varphi\in \swS(\eR^d)\) (et non pour tout \( \varphi\in L^2(\eR^d\)).

\begin{theorem}[Théorème de Sobolev avec \( k=0\)\cite{ooFZERooPVhoge}]     \label{THOooOHIPooXSEkVI}
	Soit \( s>\frac{ d }{ 2 }\) et \( u\in H^s(\eR^d)\). Alors \( u\) possède un représentant dans \( C^0_0(\eR^d)\) (les fonctions continues et qui s'annulent à l'infini). Nous écrivons cela \( H^s(\eR^d)\subset C_0^0(\eR^d)\).
\end{theorem}

\begin{proof}
	Nous commençons par supposer que \( u\in H^s(\eR^d)\cap \swS(\eR^d)\), et dans ce cas nous notons \( u\) le représentant dans \( \swS(\eR^d)\). Nous allons prouver l'inégalité
	\begin{equation}
		\| u \|_{\infty}\leq c\| u \|_{H^s(\eR^d)}.
	\end{equation}
	La formule d'inversion de Fourier~\ref{PROPooLWTJooReGlaN} appliquée à \( u_j\) donne
	\begin{equation}        \label{EQooDKYDooAxThqf}
		u(x)=\frac{1}{ (2\pi)^d }\int_{\eR^d} e^{ikx}\hat u(k)dk,
	\end{equation}
	nous avons alors
	\begin{subequations}
		\begin{align}
			(2\pi)^d| u(x) | & \leq\int_{\eR^d}| \hat u(k) |dk                                                                                           \\
			                 & =\int_{\eR^d}(1+k^2)^{s/2}| \hat u(k) |(1+k^2)^{-s/2}dk                                                                   \\
			                 & =\int_{\eR^d}\underbrace{\Big( | \hat u(k) |^2(1+k^2)^s \Big)^{1/2}}_{f}\underbrace{\Big( (1+k^2)^{-s} \Big)^{1/2}}_{g}dk \\
			                 & =\langle f, g\rangle_{L^2(\eR^d)}.
		\end{align}
	\end{subequations}
	Ici il convient nous arrêter un instant pour nous convaincre que \( f\) et \( g\) sont réellement des éléments de \( L^2\). En ce qui concerne \( f\), c'est facile : \( \hat u\) est une fonction Schwartz. En ce qui concerne \( g\) il faut l'intégrabilité de \( | g |^2\), c'est-à-dire de \( k\mapsto (1+k^2)^{-s}\). Cela a lieu si et seulement si \(2s>d\) et donc a lieu dans les hypothèses du théorème. Nous utilisons le théorème de Cauchy-Schwarz\footnote{Formule~\ref{EQooZDSHooWPcryG}.} pour continuer :
	\begin{subequations}
		\begin{align}
			(2\pi)^d| u(x) | & \leq \| f \|_{L^2}\| g \|_{L^2}                               \\
			                 & =c\left( \int_{\eR^d}| \hat u(k) |^2(1+k^2)^sdk \right)^{1/2} \\
			                 & =c\| u \|_{H^s(\eR^d)}.
		\end{align}
	\end{subequations}
	Donc en introduisant le facteur \( (2\pi)^d\) dans la constante \( c\) nous avons
	\begin{equation}
		\| u \|_{\infty}\leq c\| u \|_{H^s(\eR^d)}.
	\end{equation}
	Cela est tout ce que nous voulions faire avec \( u\in \swS(\eR^d)\).

	Nous considérons maintenant \( u\in H^s(\eR^d)\). Vue la densité des fonctions Schwartz dans \( H^s\) (proposition~\ref{PROPooMKAFooKDNTbO}) nous considérons une suite \( (u_j)\) dans \( \swS(\eR^d)\) telle que
	\begin{equation}
		u_j\stackrel{H^s(\eR^d)}{\longrightarrow}u
	\end{equation}
	Ici \( u\) est une classe, mais nous identifions \( u_j\) avec sa classe (parce qu'il ne faut pas exagérer non plus). La suite \( (u_j)\) est de Cauchy dans \( H^s\), donc si \( \epsilon>0\) est donné, il existe \( N\) tel que si \( n,m>N\), \( \| u_m-u_n \|\leq \epsilon\). Nous avons alors aussi
	\begin{equation}
		\| u_m-u_n \|_{\infty}\leq c\epsilon,
	\end{equation}
	ce qui signifie que \( (u_j)\) est également une suite de Cauchy dans \( \big( C^0(\eR^d),\| . \|_{\infty} \big)\) qui est un espace complet par la proposition~\ref{PropSYMEZGU}.

	Il existe donc une fonction \( v\in C_0^0(\eR^d)\) telle que
	\begin{equation}        \label{EQooBNGCooBUAQYN}
		u_j\stackrel{\big( C^0_0(\eR^d), \| . \|_{\infty} \big)}{\longrightarrow}v.
	\end{equation}
	La question est de savoir si nous pouvons déduire que \( v\) est un représentant de \( u\).

	Par le lemme~\ref{LEMooZIBIooANHyPy} nous avons également la convergence
	\begin{equation}
		u_j\stackrel{L^2(\eR^d)}{\longrightarrow} u.
	\end{equation}
	Pour tout \( \varphi\in \swS(\eR^d)\) nous avons alors
	\begin{equation}
		\langle u_j, \varphi\rangle_{L^2}\to\langle u, \varphi\rangle_{L^2}.
	\end{equation}

	Mais en même temps, la convergence \eqref{EQooBNGCooBUAQYN} couplée au lemme~\ref{LEMooGDTXooJRudME} donne également
	\begin{equation}
		\langle u_j, \varphi\rangle_{L^2}\to \langle v, \varphi\rangle_{L^2}.
	\end{equation}
	Par unicité de la limite (dans \( \eR\)) nous avons
	\begin{equation}
		\langle v, \varphi\rangle_{L^2}=\langle u, \varphi\rangle_{L^2}
	\end{equation}
	pour tout \( \varphi\in\swS(\eR^d)\). La proposition~\ref{PropAAjSURG} appliquée à \( u-v\) montre alors que \( u-v=0\) presque partout, c'est-à-dire que \( v\) est bien un représentant de \( u\).

	Le représentant \( v\) de \( u\) est non seulement continu (comme limite uniforme de fonctions continues), mais également bornée, comme limite uniforme de fonctions Schwartz.
\end{proof}

\begin{proposition}[\cite{ooFZERooPVhoge}]
	Si \( u\in H^s(\eR^d)\) (\( s\in \eR\)) alors
	\begin{enumerate}
		\item
		      \( \partial^{\alpha}u\in H^{s-| \alpha |}(\eR^d)\),
		\item
		      l'application
		      \begin{equation}
			      \partial^{\alpha}\colon H^s(\eR^d)\to H^{s-| \alpha |}(\eR^d)
		      \end{equation}
		      est continue.
	\end{enumerate}
\end{proposition}
Note : ici \( \partial\) est l'opération de dérivée faible.

\begin{proof}
	Nous allons seulement prouver que \( \partial_j\colon H^s(\eR^d)\to  H^{s-1}(\eR^d) \) est bien définie\footnote{Au sens où l'espace d'arrivée est bien celui-là.} et continue. Par composition, la thèse suivra.

	Soit \( u\in H^s(\eR^d)\) par le lemme~\ref{LEMooAGBZooWCbPDd} nous avons
	\begin{equation}
		\widehat{\partial_ju}=i\xi_j\hat u.
	\end{equation}

	D'autre part, la fonction
	\begin{equation}
		\begin{aligned}
			f\colon \eR^n & \to \eR                             \\
			x             & \mapsto \frac{ x_i }{ 1+\| x \|^2 }
		\end{aligned}
	\end{equation}
	est bornée (et même indépendamment de \( i\)) par une constante \( K\). Donc nous avons pour tout\quext{Question : dans \cite{ooFZERooPVhoge}, il faut dépendre cette constante de \( s\). Je ne comprends pas pourquoi.} \( s\) :
	\begin{equation}
		k_i(1+\| k \|^2)^{-s}<K(1+\| k \|^2)^{-s+1}.
	\end{equation}

	Avec cela nous pouvons calculer un peu : si \( u\in H^s(\eR^d)\), nous avons
	\begin{subequations}
		\begin{align}
			\| \partial_ju \|_{H^{s-1}(\eR^d)} & =\int_{\eR^d}| \widehat{\partial_ju} |(1+k^2)^{s-1}dk \\
			                                   & =\int_{\eR^d}k_j| \hat u |(1+k^2)^{s-1}dk             \\
			                                   & \leq \int_{\eR^d}K| \hat u |(1+k^2)^sdk               \\
			                                   & =K\| u \|_{H^s(\eR^d)}.
		\end{align}
	\end{subequations}
	Nous avons donc que \( \| \partial_ju \|_{H^{s-1}(\eR^d)}\) est fini lorsque \( u\in H^s(\eR^d)\).

	La majoration \( \| \partial_ju \|\leq K\| u \|\) donne la majoration suivante pour la norme de l'opérateur \( \partial_j\) :
	\begin{equation}
		\| \partial_j \|=\sup_{\| u \|_{H^s}=1}\| \partial_ju \|_{H^{s-1}}\leq K.
	\end{equation}
	Le fait d'être borné implique d'être continu par la proposition~\ref{PROPooQZYVooYJVlBd}.
\end{proof}

\begin{theorem}[Théorème de plongement de Sobolev \cite{ooFZERooPVhoge}]
	Soient \( k\in \eN\) et \( s>\frac{ d }{ 2 }+k\). Alors
	\begin{equation}        \label{EQooIJZOooZiYSnJ}
		H^s(\eR^d)\subset C_0^k(\eR^d).
	\end{equation}
\end{theorem}

Remarques :
\begin{itemize}
	\item
	      L'espace \( C_0^k(\eR^d)\) est l'ensemble des fonctions de classe \( C^k\) qui s'annulent à l'infini.
	\item
	      L'inclusion \eqref{EQooIJZOooZiYSnJ} signifie que tout élément dans \( H^s\) possède un représentant dans \( C_0^k(\eR^d)\).
\end{itemize}

\begin{proof}
	Pour \( k=0\), c'est le théorème~\ref{THOooOHIPooXSEkVI}. Si \( | \alpha |<k\) nous savons que \( \partial^{\alpha}u\in H^{s-k}\subset C_0^0(\eR^d) \). Cela signifie que les dérivées faibles sont continues, mais pas qu'il existe un représentant qui est réellement \( k\) fois continument dérivable.

	Soit \( u\in H^s(\eR^d)\) et une suite \( (u_j)  \) dans \( \swS(\eR^d)\) telle que
	\begin{equation}
		u_j\stackrel{H^s(\eR^d)}{\longrightarrow}u.
	\end{equation}
	Vu que l'espace topologique \( \big( C_0^k(\eR^d),\| . \|_{\infty} \big)\) est complet il existe \( v\in C_0^k\) tel que
	\begin{equation}
		u_j\stackrel{C_0^k}{\longrightarrow}v.
	\end{equation}
	Il reste à montrer que \( v\) est un représentant de \( u\). Cela se fait comme plus haut en montrant que \( u_j\stackrel{L^2}{\longrightarrow}u\).
\end{proof}
