%+++++++++++++++++++++++++++++++++++++++++++++++++++++++++++++++++++++++++++++++++++++++++++++++++++++++++++++++++++++++++++
\section{Langages}
%+++++++++++++++++++++++++++++++++++++++++++++++++++++++++++++++++++++++++++++++++++++++++++++++++++++++++++++++++++++++++++

%---------------------------------------------------------------------------------------------------------------------------
\subsection{Alphabets et mots}
%---------------------------------------------------------------------------------------------------------------------------

\begin{definition}
    Un \defe{alphabet}{alphabet} est un ensemble fini de symboles appelés \defe{lettres}{lettres}.
\end{definition}

On utilise aussi parfois le terme \defe{vocabulaire}{alphabet} pour désigner un alphabet.

\begin{definition}
    Un \defe{mot}{mot} sur l'alphabet \( \Sigma \) est une suite finie et ordonnée, éventuellement vide, de lettres de \( \Sigma \). Le \defe{mot vide}{mot!mot vide} est toujours noté $\varepsilon$.
\end{definition}

% TODO: mettre la notation du mot vide à part ?
% TODO: utiliser une commande pour le mot vide ?

\begin{definition}
    La \defe{longueur d'un mot}{mot!longueur d'un mot} \( w \), noté \( |w| \), est le nombre de lettres constituant le mot \( w \). Le mot vide a une longueur de 0.
\end{definition}

Soit \( w \) un mot de longueur \( k \), on peut désormais noter \( w = w_1 \cdots w_k \), où chacun des \( w_i, 1 \leq i \leq k \) représente une lettre de \( w \). Par convention, si \( k = 0 \), alors le mot \( w \) est le mot vide.

\begin{definition}
    Soient \( w \) un mot sur l'alphabet \( \Sigma \) et \( a \in \Sigma \) une lettre, le \defe{nombre d'occurrences}{mot!nombre d'occurrences} de la lettre \( a \) dans le mot \( w \), noté \( |w|_a \), est le nombre de fois où apparaît la lettre \( a \) dans le mot \( w \), c'est-à-dire le cardinal de l'ensemble \( \{ i \mid w_i = a, 1 \leq i \leq |w| \} \).
\end{definition}

% TODO: je crois que la définition de «cardinal d'un ensemble» (éventuellement fini) n'apparaît nulle part, à moins que ça ne fasse partie de la théorie des ensembles qui est considérée comme acquise

\begin{definition}
    Soit \( \Sigma \) un alphabet, l'\defe{ensemble des mots non-vides}\ sur l'alphabet \( \Sigma \), noté \( \Sigma^+ \), est l'ensemble:
    \begin{equation}
        \Sigma^+ = \{ w = w_1 \ldots w_n, n > 0 \}
    \end{equation}
\end{definition}

\begin{definition}
    Soit \( \Sigma \) un alphabet, l'\defe{ensemble des mots}\ sur l'alphabet \( \Sigma \), noté \( \Sigma^* \), est l'ensemble:
    \begin{equation}
        \Sigma^* = \{ w = w_1 \ldots w_n, n \geq 0 \}
    \end{equation}
\end{definition}

Des deux définitions précédentes, on tire l'égalité suivante:

\begin{equation}
  \Sigma^* = \Sigma^+ \cup \{ \varepsilon \}
\end{equation}

\begin{definition}
    Soient \( \Sigma \) un alphabet et \( x, y \in \Sigma^* \) deux mots sur l'alphabet \( \Sigma \) de longueur respective \( n \) et \( m \), le \defe{produit}{mot!produit} \( w \) de \( x \) et \( y \), noté \( x \cdot y \) est défini par \( w = x_1 \ldots x_n y_1 \ldots y_n \).
\end{definition}

Le produit est également appelé \defe{concaténation}{mot!concaténation}.

\begin{proposition}[Monoïde \( (\Sigma^*, \cdot, \varepsilon) \)]
    L'ensemble \( \Sigma^* \) munie de l'opération produit d'élément neutre \( \varepsilon \) est un monoïde.
\end{proposition}

\begin{proof}
    Soient \( x, y, z \in \Sigma^* \), avec les définitions précédentes, on peut vérifier facilement que:
    \begin{itemize}
    \item
      le produit est une loi interne: $x \cdot y \in \Sigma^*$;
    \item
      le produit est associatif: $x \cdot (y \cdot z) = (x \cdot y) \cdot z$;
    \item
      $\varepsilon$ est l'élément neutre du produit: $x \cdot \varepsilon = \varepsilon \cdot x = x$.
    \end{itemize}
\end{proof}

Le produit n'est pas commutatif.

% TODO: à partir de là, on peut aussi parler de monoïde libre
% Voir https://fr.wikipedia.org/wiki/Mono%C3%AFde#Bases_et_mono%C3%AFdes_libres et aussi (surtout) https://en.wikipedia.org/wiki/Free_monoid
% La notion de monoïde libre est très liée aux définitions précédentes, donc si jamais on souhaite développer la notion de monoïde libre, il semble naturel de le faire ici-même plutôt que dans 42_nombres (où doit être défini monoïde). La page wikipedia anglophone est un bon point d'appui pour développer cette partie.



%---------------------------------------------------------------------------------------------------------------------------
\subsection{Langage}
%---------------------------------------------------------------------------------------------------------------------------



