% This is part of Mes notes de mathématique
% Copyright (c) 2011-2013,2015-2019, 2023-2025
%   Laurent Claessens
% See the file fdl-1.3.txt for copying conditions.


\begin{definition}      \label{DEFooRIXGooECoIbx}
	Soit une fonction \( f\) sur \( \eR^d\), dont nous ne précisons pas la régularité. Sa \defe{transformée de fourier}{transformée!de Fourier} est la fonction \( \hat f\) définie par
	\begin{equation}        \label{EQooCHMUooFhSmaz}
		\hat f(\xi)=\int_{\eR^d} f(x) e^{-i\xi\cdot x}dx
	\end{equation}
	si elle existe.
\end{definition}

\begin{normaltext}
	Ce qui est bien avec cette définition est que si la formule \eqref{EQooCHMUooFhSmaz} ne définit pas \( \hat f\) (parce que l'intégrale n'existe pas, par exemple), nous nous réservons le droit de définir tout de même \( \hat f\) par d'autres biais. Ce sera d'ailleurs l'objet du théorème~\ref{THOooJLCDooAjTvJf} qui définira \( \hat f\) pour tout \( f\in L^2(\eR^d)\) alors que la formule \eqref{EQooCHMUooFhSmaz} ne fonctionne pas sur toutes ces fonctions.

	Une bonne partie de ce qui va suivre aura pour objet de déterminer des espaces de fonctions sur lesquels la transformée est bien définie, et sur lesquels elle a de bonnes propriétés.

	Nous allons par ailleurs utiliser indifféremment les notations \( \TF(f)\) ou \( \hat f\) pour la transformée de Fourier de \( f\). La notation \( \TF\) est pratique pour les transformées de loooooongues expressions ainsi que pour parler de l'application «transformée de Fourier» d'un espace de fonction vers un autre.
\end{normaltext}

\begin{normaltext}
	Nous verrons dans le théorème~\ref{THOooJLCDooAjTvJf} que la transformée de Fourier n'est pas une isométrie de \( L^2\). Pour avoir une isométrie, il aurait fallu choisir des coefficients moins simples.
\end{normaltext}

\begin{proposition}
	La transformée de Fourier est \( \eC\)-linéaire au sens suivant. Soient des fonctions \( f,g\colon \eR^d\to \eC\) telles que \( \TF(f)\) et \( \TF(g)\) existent. Alors
	\begin{enumerate}
		\item
		      La transformée de Fourier de \( f+g\) existe et vaut \( \TF(f+g)=\TF(f)+\TF(g)\)
		\item
		      Pour tout \( \lambda\in \eC\), la transformée de Fourier de \( \lambda f\) existe et vaut \( \TF(\lambda f)=\lambda\TF(f)\).
	\end{enumerate}
\end{proposition}

\begin{proof}
	C'est la proposition \ref{PROPooFIYEooCpdmwZ} qui nous permet de séparer l'intégrale dans le calcul suivant :
	\begin{subequations}
		\begin{align}
			\TF(f+g)(\xi) & =\int_{\eR^d}(f+g)(x) e^{-i\xi\cdot x}dx                                 \\
			              & =\int_{\eR^d}f(x) e^{-i\xi\cdot x}dx+\int_{\eR^d}g(x) e^{-i\xi\cdot x}dx \\
			              & =\TF(f)(\xi)+\TF(g)(\xi).
		\end{align}
	\end{subequations}
	En ce qui concerne \( \TF(\lambda f)\), c'est la même chose, en utilisant la linéarité de l'intégrale.
\end{proof}

%+++++++++++++++++++++++++++++++++++++++++++++++++++++++++++++++++++++++++++++++++++++++++++++++++++++++++++++++++++++++++++
\section{Transformée de Fourier sur \( L^1(\eR^d)\)}
%+++++++++++++++++++++++++++++++++++++++++++++++++++++++++++++++++++++++++++++++++++++++++++++++++++++++++++++++++++++++++++

Nous rappelons que les espaces \( L^p\) sont des ensembles de classes de fonctions, définition~\ref{DEFooKMJQooXeaUtp}. La transformée de Fourier, comme presque tout ce qui a trait aux intégrales, passe aux classes.

\begin{lemma}
	Soit une fonction \( f\colon \eR^d\to \eC\) telle que \( \TF(f)\) existe. Alors pour toute fonction \( g\in[f]\) la transformée \( \TF(g)\) existe et \( \hat g=\hat f\).
\end{lemma}

\begin{proof}
	Par définition des classes, il existe une fonction \( s\colon \eR^d\to \eC\) presque partout nulle telle que \( g=f+s\). Soit \( \xi\in \eR^d\) fixé. La fonction \( x\mapsto s(x) e^{-i\xi x}\) est presque partout nulle et donc intégrable d'intégrale nulle. La proposition~\ref{PROPooFIYEooCpdmwZ} nous permet alors d'affirmer que \( f+s\) est intégrable et que
	\begin{equation}
		\TF(f+s)(\xi)=\int_{\eR^d}(f+s)(x) e^{-i\xi x}dx=\int_{\eR^d}f(x) e^{-i\xi x}dx+\int_{\eR^d}s(x) e^{-i\xi x}dx=\TF(f)(\xi).
	\end{equation}
\end{proof}

À partir de maintenant, lorsque nous parlons de transformée de Fourier d'une fonction dans \( L^p\), nous parlons indifféremment d'une vraie fonction ou d'une classe.

\begin{lemma}
	Si \( f\in L^1(\eR^d)\), alors \( \hat f\) existe.
\end{lemma}

\begin{proof}
	Par définition, si \( f\in L^1(\eR^d)\), alors l'intégrale \( \int_{\eR^d}| f |\) existe et est finie. Alors la fonction qui arrive dans la transformée de Fourier en \( \xi\), la fonction \( s\colon x\to f(x) e^{-i\xi x}\) est également dans \( L^1(\eR^d)\) parce que \( | s |=| f |\).
\end{proof}

\begin{normaltext}
	Si \( f\in L^1(\eR)\), la transformée de Fourier \( \hat f\) existe, mais il n'est pas garanti qu'elle soit elle-même dans \( L^1\), même si il est vrai que \( \hat f\) est continue (proposition \ref{PropJvNfj}) et bornée (proposition \ref{CORooHSNYooZlZoyV}). Nous allons immédiatement montrer un exemple de fonction \( L^1\) dont la transformée de Fourier n'est pas dans \( L^1\).
\end{normaltext}

\begin{lemma}       \label{LEMooROPHooOSguhN}
	La transformée de Fourier de
	\begin{equation}
		\begin{aligned}
			f\colon \eR & \to \eR                                                                                             \\
			x           & \mapsto \begin{cases}
				                      1 & \text{si } x\in \mathopen[ -\frac{ 1 }{2} , \frac{ 1 }{2} \mathclose] \\
				                      0 & \text{sinon }
			                      \end{cases}
		\end{aligned}
	\end{equation}
	est
	\begin{equation}
		\hat f(\xi)=-\frac{ \sin(\xi/2) }{ \xi/2 }.
	\end{equation}

	La fonction \( \hat f\) n'est pas dans \( L^1(\eR)\).
\end{lemma}

\begin{proof}
	Pour \( \xi\in \eR\) fixé, la fonction \( x\mapsto f(x) e^{i\xi x}\) est continue à support compact, donc nous n'avons aucun problèmes avec l'intégrale
	\begin{subequations}
		\begin{align}
			\hat f(\xi) & =\int_{\eR}f(x) e^{-i\xi x}dx                               \\
			            & =\int_{\mathopen[ -1/2 , 1/2 \mathclose]} e^{-i\xi x}       \\
			            & =\big[ -\frac{1}{ i\xi } e^{-i\xi x} \big]_{x=-1/2}^{x=1/2} \\
			            & =-\frac{1}{ i\xi }\big(  e^{-i\xi /2}- e^{i\xi /2} \big)    \\
			            & =-\frac{ 2i\sin(\xi/2) }{ i\xi }                            \\
			            & =\frac{ \sin(\xi/2) }{ \xi/2 }.
		\end{align}
	\end{subequations}

	Le fait que cette fonction ne soit pas dans \( L^1(\eR)\) est le lemme \ref{LEMooEEWSooZwLSAP}.
\end{proof}

\begin{lemma}       \label{LEMooKGDKooVXSMCn}
	Si \( f\in L^1(\eR^d)\) et si \( g(x)=f(\lambda x)\) alors
	\begin{equation}
		\hat g(\xi)=\lambda^{-d}\hat f(\xi/\lambda).
	\end{equation}
\end{lemma}

\begin{proof}
	Il s'agit de faire le changement de variable \( y=\lambda x\) dans l'intégrale
	\begin{equation}
		\hat g(\xi)=\int_{\eR^d}f(\lambda x) e^{-i\xi x}dx.
	\end{equation}
	Dans le changement de variables, vient le coefficient \( dx=\lambda^{-d}dy\).
\end{proof}

\begin{proposition}     \label{PropfqvLOl}
	La transformée de Fourier est un morphisme vis-à-vis de la convolution\index{produit!de convolution!et Fourier} sur \( L^1(\eR^n)\) :
	\begin{equation}
		\widehat{f*g}=\hat f\hat g.
	\end{equation}
\end{proposition}

\begin{proof}
	Nous devons étudier l'intégrale
	\begin{equation}
		\widehat{f*g}(\xi)=\int_{\eR}\left[ \int_{\eR} f(y)g(t-y)dy\right] e^{-it\xi} dt.
	\end{equation}
	Ici nous avons choisi des représentants \( f\) et \( g\) dans les classes de \( L^1\). Montrons que \( f\) est borélienne. D'abord \( f(x)=f_+(x)-f_-(x)\) où \( f_+\) et \( f_-\) sont des fonctions positives. Afin d'alléger les notations nous supposons un instant que \( f\) est positive et nous posons
	\begin{equation}
		f_n(x)=\sum_{k=1}^{2^n} \frac{ k }{ n }\mtu_{f(x)\in\mathopen[ \frac{ k }{ n } , \frac{ k+1 }{ n } [}.
	\end{equation}
	Le fait que \( f\) soit dans \( L^1\) implique que chacune des fonctions \( f_n\) est borélienne\quext{Ceci demanderait plus de justification. Dites moi si vous savez comment justifier que les \( f_n\) soient boréliennes.} et donc que \( f\) l'est aussi en tant que limite ponctuelle de fonctions boréliennes\footnote{Le fait que \( f\) soit borélienne est une conséquence du théorème~\ref{ThoRWEoqY}.}.

	Nous allons appliquer le théorème de Fubini~\ref{CorTKZKwP} à la fonction
	\begin{equation}
		\phi(x,y)=f(x)g(y) e^{-i\xi(x+y)}
	\end{equation}
	qui est borélienne en tant que produit et composée de fonctions boréliennes. Nous avons
	\begin{subequations}
		\begin{align}
			\int_{\eR}\left( \int_{\eR}| f(x) e^{-i\xi x} | |g(y) e^{-i\xi y} |dy \right)dx & =\int_{\eR}\left( | f(x) |\int_{\eR}| g(y) |dy \right)dx \\
			                                                                                & =\int_{\eR}| f(x) |\| g \|_1                             \\
			                                                                                & =\| f \|_1\| g \|_1<\infty.
		\end{align}
	\end{subequations}
	Le théorème est donc applicable. D'abord nous avons :
	\begin{subequations}
		\begin{align}
			\hat f(\xi)\hat g(\xi) & =\left(\int_{\eR}f(x) e^{-i\xi x}dx\right)\left(\int_{\eR}g(y) e^{-i\xi y}dy\right) \\
			                       & =\int_{\eR}\left( \int_{\eR}f(x)g(y) e^{-i\xi(x+y)}dy \right)dx                     \\
			                       & =\int_{\eR}\left( \int_{\eR}f(x)g(t-x) e^{-i\xi t}dt \right)dx.
		\end{align}
	\end{subequations}
	Jusqu'ici nous n'avons pas utilisé Fubini. Nous avons seulement introduit le nombre \( \int_{\eR}g(y) e^{-i\xi y}dy\) dans l'intégrale par rapport à \( x\) et effectué le changement de variables \( y\mapsto t=x+y\). Maintenant nous appliquons le théorème de Fubini pour inverser l'ordre des intégrales :
	\begin{subequations}
		\begin{align}
			\hat f(\xi)\hat g(\xi) & =\int_{\eR}\left( \int_{\eR}f(x)g(t-x) e^{-it\xi}dt \right)dx \\
			                       & =\int_{\eR} e^{-it\xi}\left( \int_{\eR}f(x)g(t-x)dx \right)dt \\
			                       & =\int_{\eR} e^{-it\xi}(f*g)(t)dt                              \\
			                       & =\widehat{f*g}(\xi).
		\end{align}
	\end{subequations}
\end{proof}

\begin{proposition}       \label{PropJvNfj}
	Soit une fonction \( f\in L^1(\eR^d)\). Alors sa transformée de Fourier est continue\index{transformée!de Fourier!continuité}.
\end{proposition}

\begin{proof}
	Nous considérons une fonction \( f\) définie sur \( \eR^d\) et à valeurs dans \( \eR\) ou \( \eC\). Sa transformée de Fourier est donnée par
	\begin{equation}
		\hat f(\xi)=\int_{\eR^d} e^{-i\xi x}f(x)dx.
	\end{equation}
	Pour montrer que cette fonction \( \hat f\) est continue en \( \xi_0\) nous considérons une suite \( (\xi_n)\to \xi_0\) et nous voulons montrer que \( \hat f(\xi_n)\to\hat f(\xi_0)\). Pour cela nous considérons les fonctions
	\begin{equation}
		g_n(x)= e^{-i\xi_nx}f(x)
	\end{equation}
	qui convergent simplement vers \( g(x)= e^{-i\xi_0 x}f(x)\). Étant donné que
	\begin{equation}
		| g_n(x) |<| f(x) |,
	\end{equation}
	le théorème de la convergence dominée donne alors
	\begin{equation}
		\lim_{n\to \infty} \int g_n(x)=\int\lim_{n\to \infty } g_n(x),
	\end{equation}
	c'est-à-dire \( \lim_{n\to \infty} \hat f(\xi_n)=\hat f(\xi_0)\). La fonction \( \hat f\) est donc continue.
\end{proof}

\begin{lemma}       \label{LEMooCBPTooYlcbrR}
	Pour tout \( f\in L^1(\eR^n)\) nous avons \( \| \hat f \|_{\infty}\leq \| f \|_1\).
\end{lemma}

\begin{proof}
	Cela est une simple vérification :
	\begin{equation}
		\hat f(\xi)=\int_{\eR^n}f(x) e^{-ix\xi}dx,
	\end{equation}
	nous avons, pour tout \( \xi\),
	\begin{equation}
		| \hat f(\xi) |\leq\int_{\eR}| f(x) |dx,
	\end{equation}
	ce qui signifie exactement \( \| \hat f \|_{\infty}\leq \| f \|_1\).
\end{proof}

\begin{lemma}[Lemme de Riemann-Lebesgue\cite{MaureyHilbertFourier}]     \label{LesmRLaxXkQV}
	Si \( f\) est une fonction \( L^1(\eR)\) alors \( \lim_{\xi\to\pm\infty} \hat f(\xi)=0\).
\end{lemma}

\begin{proof}
	Nous commençons par prouver le résultat dans le cas d'une fonction \( g\) en escalier, et plus précisément par une fonction caractéristique d'un compact \( K=\mathopen[ a , b \mathclose]\). Au niveau de la transformée de Fourier nous avons
	\begin{equation}
		\hat\mtu_{K}(\xi)=\int_a^b e^{-i\xi x}dx=-\frac{1}{ i\xi }( e^{-ib\xi}- e^{-ia\xi}).
	\end{equation}
	Par conséquent
	\begin{equation}
		| \hat\mtu_K(\xi) |\leq \frac{ 2 }{ | \xi | }.
	\end{equation}
	Plus généralement si \( g=\sum_{i=1}^Nc_i\mtu_{K_i}\), alors
	\begin{equation}
		| \hat g(\xi) |\leq \frac{ 2 }{ | \xi | }\sum_{i=1}^N| c_i |,
	\end{equation}
	et donc nous avons effectivement \( \lim_{\xi\to\pm\infty}| \hat g(\xi) |=0\).

	Nous passons maintenant au cas général \( f\in L^1(\eR)\). Étant donné que les fonctions \( L^1\) en escalier sont denses dans \( L^1\), nous considérons une fonction \( g\in L^1(\eR)\) en escalier telle que \( \| f-g \|_1<\epsilon\). Nous avons donc
	\begin{equation}
		\| \hat f-\hat g \|_{\infty}\leq \| f-g \|_1<\epsilon.
	\end{equation}
	Donc dans
	\begin{equation}
		\| \hat f(\xi) \|\leq \| \hat f(\xi)-\hat g(\xi) \|+| \hat g(\xi) |,
	\end{equation}
	le premier terme est plus petit que \( \epsilon\). Il nous reste à voir que
	\begin{equation}
		\lim_{\xi\to \infty} | \hat g(\xi) |=0,
	\end{equation}
	mais cela est le résultat de la première partie de la preuve.
\end{proof}

\begin{corollary}       \label{CORooHSNYooZlZoyV}
	La transformée de Fourier d'une fonction \( L^1(\eR)\) est bornée.
\end{corollary}

\begin{proof}
	Par le corolaire~\ref{PropJvNfj}, la transformée de Fourier d'une fonction \( L^1\) est continue. Le lemme de Riemann-Lebesgue~\ref{LesmRLaxXkQV} impliquant qu'elle tend vers zéro en \( \pm\infty\), elle doit être bornée.
\end{proof}

%---------------------------------------------------------------------------------------------------------------------------
\subsection{Formule sommatoire de Poisson}
%---------------------------------------------------------------------------------------------------------------------------

\begin{proposition}[Formule sommatoire de Poisson]   \label{ProprPbkoQ}
	Soit \( f\colon \eR\to \eC\) une fonction continue et \( L^1(\eR)\). Nous supposons que
	\begin{enumerate}
		\item
		      il existe \( M>0\) et \( \alpha>1\) tels que
		      \begin{equation}
			      | f(x) |\leq\frac{ M }{ (1+| x |)^{\alpha} },
		      \end{equation}
		\item
		      \( \sum_{n=-\infty}^{\infty}| \hat f(2\pi n) |<\infty\).

	\end{enumerate}
	Alors nous avons
	\begin{equation}
		\sum_{n=-\infty}^{\infty}f(n)=\sum_{n=-\infty}^{\infty}\hat f(2\pi n).
	\end{equation}
\end{proposition}
\index{convergence!rapidité}
\index{série!fonctions}
\index{transformation!Fourier}
\index{Fourier}
\index{série!entière}
\index{série!de Fourier}
\index{Poisson!formule sommatoire}
\index{formule!sommatoire de Poisson}

%TODO : Exprimer ce théorème comme truc sur les distributions et sur les machins tempérées, espace de Schwartz.

\begin{proof}
	\begin{subproof}
		\spitem[Convergence normale]

		Nous commençons par montrer qu'il y a convergence normale sur tout compact séparément des séries sur les \( n\geq 0\) et sur les \( n<0\).

		Soit \( K\) un compact de \( \eR\) contenu dans \( \mathopen[ -A , A \mathclose]\) et \( n\in \eZ\) tel que \( | n |\geq 2A\). Pour \( x\in K\) nous avons
		\begin{equation}
			| x+n |\geq | n |-| x |\geq | n |-A\geq \frac{ | n | }{ 2 }.
		\end{equation}
		Du coup nous avons un \( \alpha>1\) tel que
		\begin{equation}
			| f(x+n) |\leq \frac{ M }{ \big( 1+| x+n | \big)^{\alpha} }\leq \frac{ M }{ \left( 1+\frac{ | n | }{2} \right)^{\alpha} }.
		\end{equation}
		Lorsque \( n\) est grand, cela a le comportement de \( M/| n |^{\alpha}\) et donc la série
		\begin{equation}
			\sum_{n=0}^{\infty}f(x+n)
		\end{equation}
		est une série convergeant normalement. Les deux séries (usuelles)
		\begin{subequations}
			\begin{align}
				a_+=\sum_{n\geq 0}f(x+n) \\
				a_-=\sum_{n< 0}f(x+n)
			\end{align}
		\end{subequations}
		convergent normalement.

		\spitem[Convergence commutative]
		Au sens de la définition~\ref{DefIkoheE} nous avons
		\begin{equation}
			\sum_{n\in \eZ}f(x+n)=a_++a_-.
		\end{equation}
		En effet si nous prenons \( J'_0\subset\eN\) fini tel que \( |\sum_{\eN\setminus J_0'}f(x+n)-a_+|\leq \epsilon\) et \( J'_1\in -\eN\) tel que \( |\sum_{n\in -\eN\setminus J'_1}f(x+n)-a_-|<\epsilon\), et si nous posons \( J_0=J'_0\cup J'_1\) alors si \( K\) est un ensemble fini de \( \eZ\) contenant \( J_0\) nous avons
		\begin{equation}
			| \sum_{n\in K}f(n+x)-(a_++a_-) |\leq | \sum_{n\in K^+}f(n+x)-a_+ |+| \sum_{n\in K^-}f(n+x)-a_- |\leq 2\epsilon
		\end{equation}
		où \( K^+\) sont les éléments positifs de \(K\) et \( K^-\) sont les \emph{strictement} négatifs. Maintenant que la famille \( \{ f(n+x) \}_{n\in \eZ}\) est une famille sommable, nous savons qu'elle est commutativement sommable et que la proposition~\ref{PropoWHdjw} nous permet de sommer dans l'ordre que l'on veut. Nous pouvons donc écrire sans ambigüité l'expression \( \sum_{n\in \eZ}f(x+n)\) ou \( \sum_{n=-\infty}^{\infty}f(x+n)\).

		\spitem[re-convergence normale]

		Nous posons donc sans complexes la série
		\begin{equation}
			F(x)=\sum_{n\in \eZ}f(x+n)
		\end{equation}
		qui converge tant commutativement que normalement. Notons que nous pouvons maintenant dire que la série sur \( \eZ\) converge normalement; pas seulement les deux séries séparément.

		\spitem[Continuité, périodicité]
		Étant donné que chacune des fonctions \( f(x+n)\) est continue, la convergence normale nous assure que \( F\) est continue.

		De plus \( F\) est périodique de période \( 1\) parce que
		\begin{equation}
			F(x+1)=\sum_{n=-\infty}^{\infty}f(x+1+n)=\sum_{p=-\infty}^{\infty}f(x+p)=F(x)
		\end{equation}
		où nous avons posé \( p=1+n\).

		Notons que nous n'avons pas spécialement prouvé que \( F\) n'était pas périodique avec des périodes plus petites que \( 1\). Mais cela n'a pas d'importance ici.

		\spitem[Coefficients de Fourier]

		En vertu de la définition \eqref{EqhIPoPH} et de la périodicité de \( F\),
		% position ooXKRMooLMZPbj. On peut enlever dès que c'est dans l'erratum.
		\begin{subequations}
			\begin{align}
				c_n(F) & =\int_{-1/2}^{1/2}F(t) e^{-2\pi int}dt                \\
				       & =\int_0^1F(t) e^{-2\pi int}dt                         \\
				       & =\int_0^1\sum_{n\in \eZ}f(t+n) e^{-2 i\pi nt}dt       \\
				       & =\sum_{n\in \eZ}\int_n^{n+1}f(u) e^{-2\pi i (u-n)n}du \\
				       & =\int_{-\infty}^{\infty}f(u) e^{-2\pi inu}du          \\
				       & =\hat f(2\pi n).
			\end{align}
		\end{subequations}
		Justifications :
		\begin{itemize}
			\item
			      Changement de variables \( u=t+n\).
			\item
			      permuté l'intégrale et la somme en vertu du fait que la somme converge normalement.
			\item
			      Égalité \( e^{-2\pi i(u-n)n}=e^{-2\pi iun}e^{-2\pi in^2}=e^{-2\pi iun}\).
		\end{itemize}

		\spitem[Conclusion]

		Étant donné l'hypothèse \( \sum_{n\in \eZ}| \hat f(n) |<\infty\) la proposition~\ref{PropSgvPab} nous dit que
		\begin{equation}
			F(x)=\sum_{n\in \eZ}c_n(F) e^{2\pi inx},
		\end{equation}
		c'est-à-dire que
		\begin{equation}
			\sum_{n=-\infty}^{\infty}f(x+n)=\sum_{n=-\infty}^{\infty}\hat f(2\pi n) e^{2\pi i nx}.
		\end{equation}
		En écrivant cette égalité en \( x=0\) nous trouvons le résultat :
		\begin{equation}
			\sum_{n\in \eZ}f(n)=\sum_{n\in \eZ}\hat f(2\pi n).
		\end{equation}
	\end{subproof}
\end{proof}

\begin{example}\label{ExDLjesf}
	\index{convergence!rapidité}
	La formule sommatoire de Poisson peut être utilisée pour calculer des sommes dans l'espace de Fourier plutôt que dans l'espace direct. Nous allons montrer dans cet exemple l'égalité
	\begin{equation}
		\sum_{n=-\infty}^{\infty} e^{-\alpha n^2}=\sum_{n=-\infty}^{\infty}\sqrt{\frac{ \pi }{ \alpha }} e^{-\pi^2 n^2/\alpha}.
	\end{equation}
	Si \( \alpha\) est grand, alors la somme de gauche est plus rapide, tandis que si \( \alpha\) est petit, c'est le contraire.

	Nous appliquons la formule sommatoire de Poisson à la fonction
	\begin{equation}
		f(x)= e^{-\alpha x^2}.
	\end{equation}
	Nous avons
	\begin{subequations}        \label{EqCDeLht}
		\begin{align}
			\hat f(x) & =\int_{\eR} e^{-\alpha t^2-ixt}dt                                                               \\
			          & = e^{-x^2/4\alpha}\int_{\eR}e^{ -(\sqrt{\alpha}t+\frac{ ix }{ 2\sqrt{\alpha} })^2 }             \\
			          & = e^{-x^2/4\alpha}\frac{1}{ \sqrt{\alpha} }\int_{\eR+\frac{ ix }{ 2\sqrt{\alpha} }} e^{-u^2}du.
		\end{align}
	\end{subequations}
	Pour traiter cette intégrale nous utilisons la proposition~\ref{PrpopwQSbJg} en considérant le chemin rectangulaire fermé qui joint les points \( -R\), \( R\), \( R+ai\), \( -R+ai\) et \( f(z)= e^{-z^2}\). Calculons l'intégrale sur les deux côtés verticaux. Nous posons
	\begin{equation}
		\gamma_R(t)=R+tia
	\end{equation}
	avec \( t\colon 0\to 1\). Nous avons
	\begin{subequations}
		\begin{align}
			\int_{\gamma_R}f & =\int_0^1f\big( \gamma_R(t) \big)\| \gamma_R'(t) \|dt \\
			                 & =a e^{-R^2}\int_0^1 e^{-2tRia+at^2}dt,
		\end{align}
	\end{subequations}
	donc en module nous avons
	\begin{equation}
		| \int_{\gamma_R}f |\leq a e^{-R^2}\int_0^1 e^{at^2}dt\leq M e^{-R^2},
	\end{equation}
	où \( M\) est une constante ne dépendant pas de \( R\). Lorsque \( R\to \infty\), la contribution des chemins verticaux s'annule et nous trouvons que
	\begin{equation}    \label{EqjrNxLr}
		\int_{\eR+ai} e^{-u^2}du=\int_{\eR} e^{-u^2}du,
	\end{equation}
	que nous pouvons utiliser pour continuer le calcul \eqref{EqCDeLht}. Nous avons
	\begin{equation}
		\hat f(x)= \frac{ e^{-x^2/4\alpha}}{\sqrt{\alpha}}\int_{\eR} e^{-u^2}du\\
		=\sqrt{\frac{ \pi }{ \alpha }} e^{-x^2/4\alpha}
	\end{equation}
	où nous avons utilisé la formule \eqref{EqFDvHTg}. Par conséquent ce qui rentre dans la formule sommatoire de Poisson est
	\begin{equation}
		\hat f(2\pi n)=\sqrt{\frac{ \pi }{ \alpha }} e^{-\pi^2 n^2/\alpha}.
	\end{equation}
\end{example}

%+++++++++++++++++++++++++++++++++++++++++++++++++++++++++++++++++++++++++++++++++++++++++++++++++++++++++++++++++++++++++++
\section{Suite régularisante}
%+++++++++++++++++++++++++++++++++++++++++++++++++++++++++++++++++++++++++++++++++++++++++++++++++++++++++++++++++++++++++++

\begin{definition}      \label{DEFooRIFYooUUUoha}
	Une \defe{suite régularisante}{suite!régularisante} est une suite \( (\rho_n)\) dans \( L^1(\eR^d)\) telle que
	\begin{enumerate}
		\item       \label{ITEMooEYXYooAkKeXX}
		      pour tout \( n\), \( \rho_n\geq 0\) et \( \int_{\eR^d}\rho_n=1\);
		\item
		      pour tout \( \alpha>0\),
		      \begin{equation}
			      \lim_{n\to \infty} \int_{| t |>\alpha}\rho_n=0.
		      \end{equation}
	\end{enumerate}
\end{definition}
Une telle suite est régularisante parce que souvent \( \rho_n\in\swD(\eR^d)\), ce qui donne \( f*\rho_n\in C^{\infty}\) par le corolaire~\ref{CORooBSPNooFwYQrc}.

\begin{proposition}[\cite{ooYTNMooJsvznx,ooINHXooZWALhj}]       \label{PROPooYUVUooMiOktf}
	Soit une suite régularisante \( \rho_n\in L^1(\eR^d)\). Alors :
	\begin{enumerate}
		\item       \label{ITEMooLWMIooFFamdf}
		      Si \( f\) est continue à support compact, nous avons la convergence uniforme sur \( \eR^d\) :
		      \begin{equation}
			      f*\rho_n\stackrel{unif}{\longrightarrow} f.
		      \end{equation}
		\item       \label{ITEMooEJKKooChcgyM}
		      Si \( g\in L^p\) (\( 1\leq p<\infty\)) alors
		      \begin{equation}
			      g*\rho_n\stackrel{L^p}{\longrightarrow}g.
		      \end{equation}
	\end{enumerate}
\end{proposition}

\begin{proof}
	Si \( f\) est continue à support compact, elle est uniformément continue\footnote{Théorème de Heine~\ref{ThoHeineContinueCompact}.}, et elle est bornée. Soit \( \epsilon>0\) et \( \alpha>0\) tel que pour tout \( x,y\) tels que \( \| x-y \|<\alpha\) nous ayons \( | f(x)-f(y) |<\epsilon\). Nous prenons de plus \(n\) suffisamment grand pour avoir \( \int_{B(0,\alpha)^c}\rho_n<\epsilon\). Nous avons alors
	\begin{subequations}
		\begin{align}
			| f(x)-(f*\rho_n)(x) | & =| \int_{\eR^d}\big( f(x)-f(y) \big)\rho_n(x-y)dy |                                                                                                                 \\
			                       & \leq \int_{B(x,\alpha)}\underbrace{| f(x)-f(y) |}_{\leq \epsilon}\rho_n(x-y)dy+\int_{B(x,\alpha)^c}\underbrace{| f(x)-f(y) |}_{\leq 2\| f \|_{\infty}}\rho_n(x-y)dy \\
			                       & \leq \epsilon(1+2\| f \|_{\infty}).
		\end{align}
	\end{subequations}
	Nous avons prouvé que pour tout \( \epsilon>0\), il existe \( N\) tel que \( n>N\) implique \( \big| f(x)-(f*\rho_n)(x) \big|\leq \epsilon\). Cela prouve l'uniforme convergence sur \( \eR^d\) de \( f*\rho_n\) vers \( f\).

	Pour le point~\ref{ITEMooEJKKooChcgyM} nous considérons \( g\in L^p(\eR^d)\) et \( \phi\in \swD(\eR^d)\). Nous avons la majoration
	\begin{equation}
		\| g*\rho_n-g \|_p\leq \| g*\rho_n-\phi*\rho_n \|_p+\| \phi*\rho_n-\phi \|_p+\| \phi-g \|_p
	\end{equation}
	En ce qui concerne le premier terme;
	\begin{equation}
		\| (g-\phi)*\rho_n \|_p\leq \| g-\phi \|_p
	\end{equation}
	par la proposition~\ref{PROPooDMMCooPTuQuS}. Donc
	\begin{equation}		\label{EQooXWZYooVtsROK}
		\| g*\rho_n-g \|_p\leq 2\| g-\phi \|_p+\| \phi*\rho_n-\phi \|_p.
	\end{equation}
	Par la densité de \( \swD\) dans \( L^p\) (théorème~\ref{ThoILGYXhX}\ref{ItemYVFVrOIv}) nous pouvons considérer une suite \( \phi_i\stackrel{L^p}{\longrightarrow}g\) dans \( \swD(\eR^d)\). L'inégalité \eqref{EQooXWZYooVtsROK} est valable pour chaque \( i\) :
	\begin{equation}
		\| g*\rho_n-g \|_p\leq 2\| g-\phi_i \|_p+\| \phi_i*\rho_n-\phi_i \|_p.
	\end{equation}
	Nous effectuons la limite sur \( n\to \infty\) :
	\begin{equation}
		\lim_{n\to \infty} \| g*\rho_n-g \|_p\leq 2\| g-\phi_i \|+\underbrace{\lim_{n\to \infty} \| \phi_i*\rho_n-\phi_i \|_p}_{=0}
	\end{equation}
	parce que le point~\ref{ITEMooLWMIooFFamdf} s'applique à \( \phi_i\). Nous effectuons ensuite la limite sur \( i\to \infty\) dans
	\begin{equation}
		\lim_{n\to \infty} \| g*\rho_n-g \|\leq 2\| g-\phi_i \|\to 0.
	\end{equation}
\end{proof}


%+++++++++++++++++++++++++++++++++++++++++++++++++++++++
\section{Dérivation dans les bornes}
%+++++++++++++++++++++++++++++++++++++++++++++++++++++++


\begin{proposition}[\cite{MonCerveau}]	\label{PROPooSBBOooHdtWDK}
	Tout élément de \( C_c\big( \mathopen[ a,b\mathclose] \big)\)\footnote{Les applications continues à support compact dans \( \mathopen[ a,b\mathclose]\).} est limite uniforme de fonctions en escalier\footnote{Définition \ref{DefBPCxdel}.}.
\end{proposition}

\begin{proof}
	Soit une application en continue escalier \(\phi \colon \mathopen[ a,b\mathclose]\to \eR  \). Soit \( \epsilon>0\). Soit \( s\in \mathopen[ a,b\mathclose]\). Vu que \( \phi\) est continue en \( s\), il existe \( \delta_s>0\) tel que \( | \phi(x)-\phi(x) |<\epsilon\) pour tout \( x\in B(s,\delta_s)\). Notez que cette boule est pour la topologie induite de \( \eR\) vers \( \mathopen[ a,b\mathclose]\); si \( s=a\), c'est une boule coupée à gauche.

	Étant donné que \( \mathopen[ a,b\mathclose]\) est compact, il existe \( s_1,\ldots,s_m\in \mathopen[ a,b\mathclose]\) tels que \( \{ B(s_i,\delta_{s_i}) \}_{i=1,\ldots,m}\) recouvre \( \mathopen[ a,b\mathclose]\). Nous les supposons triés dans l'ordre croissant.


	Chacun des \( B(s_1,\delta_i)\) est un intervalle dans l'intervalle \( \mathopen[ a,b\mathclose]\). Donc la proposition \ref{PROPooJEXCooHwHAKq} nous assure qu'il existe des intervalles disjoints \( \{ J_k \}_{k=1,\ldots,n}\) tels que chaque \( J_k\) est inclus dans un des \( B(s_i,\delta_i)\).

	Nous considérons une application \(\sigma \colon \{ 1,\ldots,n \}\to \{ 1,\ldots,m \}  \) telle que \( I_k\subset  B(s_{\sigma(k)}, \delta_{\sigma(k)})\), et nous posons \( \alpha_k=f(s_{\sigma(k)})\). Enfin nous considérons la fonction en escalier
	\begin{equation}
		h=\sum_{k=1}^n\alpha_k\mtu_{J_k}.
	\end{equation}

	Nous vérifions que \( \| \phi-h \|_{\infty}<\epsilon\). Soit \( x\in \mathopen[ a,b\mathclose]\). Il existe un unique \( k\) tel que \( x\in J_k\). Nous avons alors \( h(x)=\alpha_k=f(s_{\sigma(k)})\) et \( J_k\subset B(s_{\sigma(k)},\delta_{\sigma(k)})\). Donc
	\begin{equation}
		| \phi(x)-h(x) |=| \phi(x)-\alpha_k |=| \phi(x) -f(s_{\sigma(k)}) |<\epsilon.
	\end{equation}
\end{proof}

Voir le reste du thème \ref{THEMEooDASVooWZLOjw}.


\begin{proposition}[\cite{BIBooFNPXooFAavvy}]	\label{PROPooWFSPooIBogJV}
	Soit \(f \colon \mathopen[ a,b\mathclose]\to \eR  \) croissante et presque partout dérivable. Alors \( f'\) est mesurable et
	\begin{equation}
		\int_a^bf'(t)dt\leq f(b)-f(a)
	\end{equation}
	%TODOooBTVVooPymeIe. Prouver ça. C'est le théorème 1.2 dans la source.
\end{proposition}



%+++++++++++++++++++++++++++++++++++++++++++++++++++++++++++++++++++++++++++++++++++++++++++++++++++++++++++++++++++++++++++
\section{Transformée de Fourier dans l'espace de Schwartz}
%+++++++++++++++++++++++++++++++++++++++++++++++++++++++++++++++++++++++++++++++++++++++++++++++++++++++++++++++++++++++++++

La définition de la transformée de Fourier de \( \varphi\in\swS(\eR^d)\) est
\begin{equation}
	\hat  \varphi(\xi)=\int_{\eR^n}\varphi(x) e^{-ix\cdot \xi}dx.
\end{equation}

Si \( \alpha\) est un multiindice de taille \( m\), nous notons
\begin{equation}
	(M_{\alpha}f)(x)=x_{\alpha_1}\ldots x_{\alpha_m}f(x).
\end{equation}

\begin{lemma}[Lemme de transfert]   \label{LemQPVQjCx}
	Si \( \varphi\in\swS(\eR^d)\) et si \( \alpha\) est un multiindice, alors
	\begin{equation}
		\partial^{\alpha}\hat\varphi=(-i)^{| \alpha |}\widehat{M_{\alpha}\varphi}.
	\end{equation}
	et
	\begin{equation}
		\widehat{\partial^{\alpha}\varphi}(\xi)=(-i)^{| \alpha |}\xi^{\alpha}\hat\varphi(\xi).
	\end{equation}
\end{lemma}

\begin{proof}
	Nous considérons la fonction \( h(x,\xi)=\varphi(x) e^{-ix\cdot \xi}\) dont la dérivée par rapport à \( \xi_i\) est donnée par \( -i(M_{i}\varphi)(x) e^{x\cdot \xi}\). Cette fonction est majorée en norme par
	\begin{equation}
		G(x)=M_i\varphi(x),
	\end{equation}
	qui est encore une fonction à décroissance rapide et donc parfaitement intégrable sur \( \eR^d\). Le théorème~\ref{ThoMWpRKYp} nous dit donc que la dérivée de \( \hat \varphi\) par rapport à \( \xi_j\) existe et vaut
	\begin{equation}
		\frac{ \partial \hat\varphi }{ \partial \xi_j }(\xi)=-i\int_{\eR^n}x_j\varphi(x) e^{-i\xi\cdot x}=-i\widehat{M_j\varphi}(\xi).
	\end{equation}
	En appliquant ce résultat en chaine, nous trouvons la première formule annoncée.

	Nous passons à la seconde formule annoncée. Étant donné que \( \varphi\in\swS\), ses dérivées le sont aussi et par conséquent, il n'y a pas de problèmes pour écrire
	\begin{equation}    \label{EqTYizlnia}
		\widehat{\partial_{x_k}\varphi}(\xi)=\int_{\eR^d}\frac{ \partial \varphi }{ \partial x_k }(x) e^{-ix\cdot \xi}dx.
	\end{equation}
	Étant donné que
	\begin{equation}    \label{EqZAeYaCB}
		\frac{ \partial  }{ \partial x_k }\left( \varphi(x) e^{-ix\cdot\xi} \right)=\frac{ \partial \varphi }{ \partial x_k }(x) e^{-ix\cdot\xi}-i\xi_k\varphi(x) e^{-ix\cdot \xi},
	\end{equation}
	notre tâche sera de prouver que
	\begin{equation}    \label{EqVGvYBNK}
		\int_{\eR^d}\frac{ \partial  }{ \partial x_k }\left( \varphi(x) e^{-ix\cdot \xi} \right)dx=0.
	\end{equation}
	Autrement dit, nous voulons montrer que le terme au bord d'une intégration par partie s'annule. D'abord le fait que \( \varphi\) soit à décroissance rapide nous assure que l'intégrale \eqref{EqVGvYBNK} converge. Pour chaque \( \xi\), la fonction
	\begin{equation}
		f(x,\xi)=\frac{ \partial}{\partial x_k }\left( \varphi(x) e^{-ix\cdot \xi} \right)
	\end{equation}
	est intégrable par rapport à \( x\). De plus, \( f\) est dans \( \swS(\eR)\) pour chacune de ses variables (les autres étant fixées). Le théorème de Fubini~\ref{ThoFubinioYLtPI} nous permet alors de décomposer l'intégrale en
	\begin{equation}
		\int_{\eR^d}f(x,\xi)dx=\int_{\eR}\ldots\int_{\eR} f(x_1,\ldots, x_d)dx_1\ldots dx_d.
	\end{equation}
	De plus nous pouvons intégrer dans l'ordre de notre choix et nous choisissons évidemment d'intégrer d'abord par rapport à \( x_k\).  Étudions donc l'intégrale
	\begin{equation}
		\int_{\eR}\frac{ \partial  }{ \partial x }\left( \varphi(x) e^{-ix\xi} \right)dx=\lim_{A\to\infty}\int_{-A}^A\frac{ \partial  }{ \partial x }\left( \varphi(x) e^{-ix\xi} \right)dx
	\end{equation}
	dans laquelle nous avons un peu allégé les notations. Une primitive de ce qui est intégré est toute trouvée : c'est \( \varphi(x) e^{-ix\xi}\), et nous pouvons utiliser le théorème fondamental du calcul intégral pour écrire que
	\begin{equation}
		\int_{-A}^A\left( \varphi(x) e^{-ix\xi} \right)'dx=\left[ \varphi(x) e^{-ix\xi} \right]_{x=-A}^{x=A}.
	\end{equation}
	Vu que \( \varphi\) est dans \( \swS\), la limite \( A\to\infty\) donne zéro.

	En substituant maintenant \eqref{EqZAeYaCB} dans \eqref{EqTYizlnia} et en tenant compte du terme que nous venons de montrer s'annuler, nous avons
	\begin{equation}
		\widehat{\partial_k\varphi}(\xi)=-i\xi_k\int_{\eR^d}\varphi(x) e^{-ix\cdot \xi}=-i\xi_k\hat\varphi(\xi).
	\end{equation}
	En recommençant la procédure \( | \alpha |\) fois nous trouvons la seconde formule annoncée.
\end{proof}

\begin{proposition}[\cite{MesIntProbb}] \label{PropKPsjyzT}
	À propos de transformée de Fourier dans l'espace de Schwartz.
	\begin{enumerate}
		\item
		      L'espace de Schwartz\footnote{Définition \ref{DefHHyQooK}, avec rappel que \( p_{\alpha\beta}(\varphi)=\| x\mapsto x^{\beta}(\partial^{\alpha}\varphi)(x) \|_{\infty}\).} est stable par transformée de Fourier, c'est-à-dire que \( \TF\big( \swS(\eR^d) \big)\subset \swS(\eR^d)\)\footnote{Nous verrons, avec la formule d'inversion dans la proposition \ref{PROPooLWTJooReGlaN} que c'est même une bijection.}.
		\item
		      L'application \( \TF\colon \swS(\eR^d)\to \swS(\eR^d)\) est continue.
	\end{enumerate}
\end{proposition}

\begin{proof}
	La linéarité découle de celle de l'intégrale. La difficulté est de prouver que pour \( \varphi\in\swS(\eR^d)\) nous avons bien que \( \hat\varphi\in\swS(\eR^d)\) et que cette association est continue\footnote{Pour rappel, en dimension infinie, il n'est pas garanti qu'une application linéaire soit continue.}.
	\begin{subproof}
		\spitem[Première inclusion : \(  \TF\big( \swS(\eR^d) \big)\subset\swS(\eR^d)\)]
		Nous devons prouver que pour tout multiindices \( \alpha\) et \( \beta\), nous avons \( p_{\alpha,\beta}(\hat\varphi)<\infty\). Nous avons
		\begin{equation}
			\xi^{\beta}\partial^{\alpha}\hat\varphi(\xi)=\xi^{\beta}(-i)^{| \alpha |}\widehat{M_{\alpha}\varphi}(\xi)=(-i)^{| \alpha |+| \beta |}\widehat{\partial^{\beta}M_{\alpha}\varphi}(\xi).
		\end{equation}
		Ensuite nous nous souvenons que \( \| \hat f \|_{\infty}\leq \| f \|_1\) parce que
		\begin{equation}
			| \hat f(\xi) |\leq\int_{\eR^d}\big| f(x) e^{-ix\cdot \xi} \big|=\int_{\eR^d}| f(x) |dx=\| f \|_1.
		\end{equation}
		Donc
		\begin{equation}
			p_{\alpha,\beta}(\hat\varphi)=\| \widehat{\partial^{\beta}M_{\alpha}\varphi} \|_{\infty}\leq \| \partial^{\beta}M_{\alpha}\varphi \|_1.
		\end{equation}
		Du fait que \( \varphi\) soit dans \( \swS\), la dernière expression est finie. Cela prouve déjà que
		\begin{equation}
			\TF\big( \swS(\eR^d) \big)\subset\swS(\eR^d).
		\end{equation}

		\spitem[Continuité]

		Nous supposons avoir une suite \( \varphi_n\stackrel{\swS}{\to}\varphi\), et nous devons prouver que \( \hat\varphi_n\stackrel{\swS}{\to}\hat\varphi\). Pour alléger les notations, nous posons \( f_n=\varphi_n-\varphi\). Nous avons
		\begin{subequations}    \label{subEqsSGsGGih}
			\begin{align}
				\| \hat f \|_{\alpha,\beta} & =\| \xi^{\beta}\partial^{\alpha}\hat f \|_{\infty}                                        \\
				                            & =\| \widehat{  \partial^{\beta}M_{\alpha}f  } \|_{\infty}\,\text{lemme~\ref{LemQPVQjCx}.} \\
				                            & \leq \| \partial^{\beta}M_{\alpha}f \|_1
			\end{align}
		\end{subequations}
		La convergence \(f_n\stackrel{\swS}{\to}0\) nous dit entre autres que \( \partial^{\beta}M_{\alpha}f_n\stackrel{\swS}{\to}0\); en particulier la proposition~\ref{PropGNXBeME} nous dit que \( \partial^{\beta}M_{\alpha}f_n\stackrel{L^1}{\to}0\), ce qui signifie, par les majorations \eqref{subEqsSGsGGih} que
		\begin{equation}
			\| \hat f_n \|_{\alpha,\beta}\leq \| \partial^{\beta}M_{\alpha}f_n \|_1\to0,
		\end{equation}
		ce qui prouve la continuité de transformée de Fourier dans \( \swS(\eR^d)\).
		\spitem[Bijection]
		Une preuve peut être trouvée dans \cite{BMoNzTY}.
	\end{subproof}
	% TODOooATERooEiDqLX : Faire le dernier morceau de cette preuve.
\end{proof}

\begin{proposition}[\cite{MonCerveau}]     \label{PROPooMVQMooGYAzSX}
	Soit \( \varphi\in\swS(\eR^n\times \eR^m)\) et la transformée de Fourier partielle
	\begin{equation}
		\tilde \varphi(x,k)=\int_{\eR^m}  e^{-iky}\varphi(x,y)dy.
	\end{equation}
	Alors \( \tilde \varphi\in\swS(\eR^n\times \eR^m  )\).
\end{proposition}

\begin{proof}
	Il s'agit de reprendre les étapes de la partie correspondante de la preuve de la proposition~\ref{PropKPsjyzT}. Soient des multiindices \( \alpha\), \( \alpha'\), \( \beta\) et \( \beta'\) où \( \alpha\) et \( \beta\) se réfèrent à la variable \( x\) tandis que \( \alpha'\) et \( \beta'\) se réfèrent à la variable \( k\).

	Vu que la multiplication par \( k^{\beta'}\) commute avec \( \partial^{\alpha}\) nous avons
	\begin{equation}
		x^{\beta}k^{\beta'}\partial^{\alpha}\partial^{\alpha'}\tilde \varphi(x,k)=x^{\beta}k^{\beta'}\partial^{\alpha}(-i)^{| \alpha' |}\widetilde{M_{\alpha'}\varphi}(x,k)=(-i)^{| \alpha' |+| \beta' |}x^{\beta}\partial^{\alpha}\widetilde{    \partial^{\beta'}M_{\alpha'}\varphi  }(x,k).
	\end{equation}
	D'autre part nous avons \( \partial^{\alpha}\tilde \varphi=\widetilde{\partial^{\alpha}\varphi}\) parce que la fonction \( \partial_x\varphi\) étant Schwartz, la fonction
	\begin{equation}
		G(y)=\sup_{x\in \eR^n}|(\partial_x\varphi)(x,y)|
	\end{equation}
	est dans \( L^1(\eR^m)\) par le corolaire~\ref{CORooZFPSooHCFUSH}. Par conséquent le théorème~\ref{ThoMWpRKYp} permet de permuter la dérivée et l'intégrale dans
	\begin{equation}
		\frac{ \partial  }{ \partial x }\tilde \varphi(x,k)=\frac{ \partial  }{ \partial x }\int_{\eR^m} e^{-iky}\varphi(x,y)dy.
	\end{equation}
	Dans le même ordre d'esprit des difficultés de permutation de limites nous avons \( M_{\beta}\tilde \varphi=\widetilde{M_{\beta}\varphi}\).

	D'autre part nous avons encore \( \| \tilde \varphi \|_{\alpha}<\infty\) parce que
	\begin{equation}
		| \tilde \varphi(x,k) |\leq \int_{\eR^m}| \varphi(x,y) |dy\leq \sup_x\int_{\eR^m}| \varphi(x,y) |dy\leq \int_{\eR^m}| \sup_x\varphi(x,y) |dy<\infty
	\end{equation}
	parce que \( \varphi\) est Schwartz et le corolaire~\ref{CORooZFPSooHCFUSH} donne l'intégrabilité.

	Donc nous avons
	\begin{equation}
		p_{(\alpha\alpha'),(\beta\beta')}(\tilde \varphi)=\|  \widetilde{   \partial^{\beta'}M_{\alpha'}M_{\beta}\partial^{\alpha}\varphi      }    \|_{\infty}<\infty.
	\end{equation}
	Cela prouve que \( \tilde \varphi\) est Schwartz.
\end{proof}

%---------------------------------------------------------------------------------------------------------------------------
\subsection{Quelques transformées de Fourier}
%---------------------------------------------------------------------------------------------------------------------------

\begin{lemma}[Transformée de Fourier de la Gaussienne \cite{ooKDRBooDFsyfV}]       \label{LEMooPAAJooCsoyAJ}
	La transformée de Fourier de
	\begin{equation}
		\begin{aligned}
			g_{\epsilon}\colon \eR^d & \to \eR                         \\
			x                        & \mapsto  e^{-\epsilon\| x \|^2}
		\end{aligned}
	\end{equation}
	est donnée par
	\begin{equation}
		\hat g_{\epsilon}(\xi)=\left( \frac{ \pi }{ \epsilon } \right)^{d/2} e^{-\| \xi \|^2/4\epsilon}
	\end{equation}
\end{lemma}

\begin{proof}
	Nous commençons par la fonction \(  g(x) = e^{-\| x \|^2/2}\) et nous prouvons que sa transformée de Fourier est \( \hat g(\xi)=(2\pi)^{d/2}g(\xi)\).
	\begin{subproof}
		\spitem[Réduction à la dimension \( 1\)]
		La fonction \( g\) est dans l'espace de Schwartz. Par le théorème de Fubini,
		\begin{equation}
			\hat g(\xi)=\int_{\eR^d}\prod_{k=1}^d e^{-x_k^2} e^{-i\xi_kx_k}dx
			=\prod_{k=1}^d\int_{\eR} e^{-t^2/2} e^{-\xi_kx}dt
			=\prod_{k=1}^d\hat f(\xi_k) \label{EQooXRLIooRCfIOd}
		\end{equation}
		où \( f\) est la fonction d'une variable
		\begin{equation}        \label{EQooFKSPooRBdgnk}
			f(x)= e^{-x^2/2}.
		\end{equation}
		Notons que \( f\in\swD(\eR)\).

		\spitem[Une équation différentielle]

		Voyons l'équation différentielle satisfaite par la transformée de Fourier \( \hat f\) de la fonction \eqref{EQooFKSPooRBdgnk}. Grâce au lemme~\ref{LemQPVQjCx} nous trouvons l'équation différentielle\footnote{Une façon alternative, plus directe de déduire cette équation différentielle sera donnée dans l'exemple~\ref{EXooLMXKooFcAZGR}.}
		\begin{equation}
			\xi \hat f(\xi)+(\hat f)'(\xi)=0.
		\end{equation}
		C'est le moment d'utiliser le théorème de Cauchy-Lipschitz \eqref{ThokUUlgU}, appliqué à la fonction \( f(t,y)=-ty\) qui est Lipschitz et continue au problème
		\begin{subequations}        \label{SUBEQSooWZZKooNEKnME}
			\begin{numcases}{}
				y'+ty=0\\
				y(0)=y_0
			\end{numcases}
		\end{subequations}
		possède une unique solution maximale, en l'occurrence \( y(x)= y_0  e^{-x^2/2}  \). En ce qui concerne la condition initiale nous avons
		\begin{equation}
			\hat f(0)=\int_{\eR} e^{-x^2/2}dx=\sqrt{ 2\pi }.
		\end{equation}
		par l'exemple~\ref{EXooLUFAooGcxFUW}. Donc
		\begin{equation}
			\hat f(\xi)=\sqrt{ 2\pi } e^{-\xi^2/2}.
		\end{equation}
		En reformant le produit \eqref{EQooXRLIooRCfIOd} nous concluons.
	\end{subproof}

	Nous passons maintenant à la fonction \( g_{\epsilon}\). Nous pouvons écrire \( g_{\epsilon}\) sous la forme
	\begin{equation}
		g_{\epsilon}(x)=g(\sqrt{ 2\epsilon }x).
	\end{equation}
	Utilisant successivement la transformée de Fourier de \( g\) que nous venons de calculer et~\ref{LEMooKGDKooVXSMCn} (facteur d'échelle) nous trouvons
	\begin{subequations}
		\begin{align}
			\hat g(\xi)            & =(2\pi)^{d/2}g(\xi)                                                                                   \\
			\hat g_{\epsilon}(\xi) & =(2\epsilon)^{-d/2}\hat g\big( \xi/\sqrt{ 2\epsilon } \big)                                           \\
			                       & =\left( \frac{ \pi }{ \epsilon } \right)^{d/2} e^{-| \xi |^2/4\epsilon} \label{SUBEQooFWIKooGMpFbo}..
		\end{align}
	\end{subequations}
	Nous voyons que \( \hat g_{\epsilon}\in\swS(\eR^d)\)  (c'était gagné d'avance par la proposition~\ref{PropKPsjyzT}).
\end{proof}

\begin{lemma}       \label{LEMooTDWSooSBJXdv}
	Si \( g_{\epsilon}(x)= e^{-\epsilon\| x \|^2}\) alors la suite
	\begin{equation}        \label{EQooWQWZooZIYGpq}
		\rho_n=\frac{1}{ (2\pi)^d }\hat g_{1/n}
	\end{equation}
	est une suite régularisante (définition~\ref{DEFooRIFYooUUUoha}).
\end{lemma}

\begin{proof}
	Nous savons déjà la transformée de Fourier de \( g_{\epsilon}\) par le lemme~\ref{LEMooPAAJooCsoyAJ}. Nous montrons que la suite \( \rho_n\) est régularisante. Nous avons \( \hat g_{\epsilon}\in L^1(\eR^d)\) et \( \hat g_{\epsilon}\geq 0\) ainsi que \( \lim_{\epsilon\to 0}\int_{B(0,\alpha)}\hat  g_{\epsilon}=0\) pour tout \( \alpha\). Il y a seulement un couac avec la norme. Nous calculons \( \int_{\eR^d}\hat g_{\epsilon}(\xi)d\xi\) avec la forme \eqref{SUBEQooFWIKooGMpFbo}. En utilisant sauvagement Fubini\footnote{Le pauvre !} pour séparer les intégrales et en effectuant le changement de variable \( u=t/(2\sqrt{ \epsilon })\) nous calculons :
	\begin{subequations}
		\begin{align}
			\int_{\eR^d} e^{-| \xi |^2/4\epsilon}d\xi & =\prod_{k=1}^d\int_{\eR} e^{-t^2/4\epsilon}dt         \\
			                                          & =\prod_{k=1}^d2\sqrt{ \epsilon }\int_{\eR} e^{-u^2}du \\
			                                          & =\prod_{k=1}^d2\sqrt{ \epsilon }\sqrt{ \pi }          \\
			                                          & =2^d(\pi\epsilon)^{d/2}.
		\end{align}
	\end{subequations}
	Nous avons utilisé l'exemple~\ref{EXooLUFAooGcxFUW} pour le calcul de l'intégrale gaussienne. Avec tout cela nous avons
	\begin{equation}
		\int_{\eR^d}\hat g_{\epsilon}=(2\pi)^d.
	\end{equation}
	Donc \( \frac{1}{ (2\pi)^d }\hat g_{1/n}\) est une suite régularisante.
\end{proof}

Le corolaire suivant regroupe les résultats à propos des suites régularisantes, leur utilité et leur existence.
\begin{corollary}       \label{CORooQLELooUjzIoM}
	Si la suite régularisante \( \rho_n\) est dans \( L^1(\eR^d)\cap  C^{\infty}(\eR^d)\) alors pour \( f\in L^p(\eR^d)\) en posant \( f_n=\rho_n*f\) nous avons
	\begin{enumerate}
		\item
		      \( f_n\in C^{\infty}(\eR^d)\cap L^p(\eR^d)\)
		\item
		      \( f_n\stackrel{L^p}{\longrightarrow}f\)
	\end{enumerate}
	De plus, de telles suites existent.
\end{corollary}

\begin{proof}
	Le fait que \( f_n\) soit de classe \(  C^{\infty}\) est le corolaire~\ref{CORooBSPNooFwYQrc}, et la convergence est la proposition~\ref{PROPooYUVUooMiOktf}\ref{ITEMooEJKKooChcgyM}.

	De telles suites existent, par exemple celle donnée par le lemme~\ref{LEMooTDWSooSBJXdv}.
\end{proof}

\begin{example}[\cite{KXjFWKA}] \label{EXooLMXKooFcAZGR}
	Soit la fonction \( g_{\epsilon}(x)= e^{-\epsilon x^2}\). Sa transformée de Fourier a été vue dans le lemme~\ref{LEMooPAAJooCsoyAJ} en utilisant le lemme de transfert~\ref{LemQPVQjCx}. Nous nous proposons ici de déduire de façon directe l'équation différentielle vérifiée par la transformée de Fourier de \( g_{\epsilon}\).

	Nous posons
	\begin{equation}
		I(k)=\int_{\eR} e^{-ikx} e^{-\epsilon x^2}dx.
	\end{equation}
	et nous considérons la fonction
	\begin{equation}
		f(k,x)= e^{-ikx} e^{-\epsilon x^2}.
	\end{equation}
	Elle est de classe \( C^1\) par rapport à \( k\), et intégrable en \( x\) pour chaque \( k\). De plus sa dérivée
	\begin{equation}
		(\partial_k f)(k,x)=-ix e^{-ikx} e^{-\epsilon x^2}
	\end{equation}
	vérifie \( | \partial_kf |\leq x e^{-\epsilon x^2}\). La dérivée est donc majorée (uniformément en \( k\)) par une fonction intégrable. Le théorème~\ref{ThoMWpRKYp} permet de permuter la dérivée et l'intégrale :
	\begin{subequations}
		\begin{align}
			I'(k) & =\int_{\eR}-ix e^{-ikx} e^{-\epsilon x^2}dx                                                                         \\
			      & =i\int_{\eR} e^{-ikx}\frac{1}{ 2\epsilon } \frac{ d  }{ dx }\left(  e^{-\epsilon x^2} \right)dx                     \\
			      & =\frac{ -i }{ 2\epsilon }\int_{\eR}\frac{ d }{ dx }\left(  e^{-ikx} \right) e^{-\epsilon x^2}dx & \text{par partie} \\
			      & =\frac{ -k }{ 2\epsilon }\int_{\eR} e^{-ikx} e^{-\epsilon x^2}dx                                                    \\
			      & =\frac{ -k }{ 2\epsilon }I(k).
		\end{align}
	\end{subequations}
	D'où l'équation différentielle \( I'(k)=-\frac{ k }{ 2\epsilon }I(k)\).
\end{example}



%+++++++++++++++++++++++++++++++++++++++++++++++++++++++
\section{Intégrales et fonctions nulles}
%+++++++++++++++++++++++++++++++++++++++++++++++++++++++


\begin{proposition}[\cite{BIBooWINRooQSnfWf}] \label{PropAAjSURG}
	Soient un ouvert \( \Omega\) de \( \eR\) et une fonction intégrable \( f\colon \big( \Omega,\Borelien(\Omega),\lambda \big)\to \eC\) telle que
	\begin{equation}
		\int_{\Omega}f\varphi=0
	\end{equation}
	pour toute fonction \( \varphi\in\swD(\Omega)\). Alors \( f=0\) presque partout sur \( \Omega\).
\end{proposition}

\begin{proof}
	Nous commençons par prouver que \( f\) est nulle sur tout compact de \( \Omega\). Soit un compact \( K\) de \( \Omega\). Le lemme d'Urysohn \ref{LEMooECTNooKagaRU} nous donne une fonction \( \theta\) à support compact qui vaut \( 1\) sur \( K\).

	Nous considérons une suite régularisante \( (\phi_k)\) de fonctions toujours strictement positives (par exemple celle du lemme \ref{LEMooTDWSooSBJXdv}). Vu que \( f\theta\) est à support compact, elle est dans \( L^p(\Omega)\) et le corolaire \ref{CORooQLELooUjzIoM} s'applique :
	\begin{equation}
		\phi_k*(\theta f)\to \theta f.
	\end{equation}
	Mais, \( x\) et \( k\) étant fixés, nous avons
	\begin{equation}        \label{EQooIUIMooMTAHCY}
		\big( \phi_k* (\theta f) \big)(x)=\int_{\eR}\phi_k(x-t)\theta(t)f(t)dt.
	\end{equation}
	La fonction
	\begin{equation}
		t\mapsto \phi_k(x-t)\theta(t)
	\end{equation}
	étant à support compact, l'hypothèse à propos de \( f\) fait que l'intégrale \eqref{EQooIUIMooMTAHCY} est nulle :
	\begin{equation}
		\phi_k*(\theta f)=0
	\end{equation}
	pour tout \( k\). En prenant la limite \( k\to \infty\),
	\begin{equation}
		\theta f=0.
	\end{equation}
	Vu que \( \theta(x)=1\) pour tout \( x\in K\), nous avons \( f(x)=0\) pour tout \( x\in K\).

	Nous avons démontré que \( f\) était nulle sur tout compact de \( \Omega\).

	Nous considérons maintenant une suite exhaustive \( (K_n)\) de compacts (lemme \ref{LemGDeZlOo}). La fonction \( f\) est nulle sur chaque \( K_n\), et comme \( \Omega=\bigcup_{n=0}^{\infty}K_n\), la fonction \( f\) est nulle sur \( \Omega\).
\end{proof}


\begin{proposition}[\cite{MonCerveau}]	\label{PROPooZOJHooKwZOFW}
	Soit \( f\) intégrable sur \( \mathopen[ a,b\mathclose]\) vérifiant
	\begin{equation}
		\int_a^xf=0
	\end{equation}
	pour tout \( x\in\mathopen[ a,b\mathclose]\). Alors \( f=0\) presque partout.
\end{proposition}

\begin{proof}
	Tout d'abord, notons que si \( I\) est un intervalle dans \( \mathopen[ a,b\mathclose]\), alors \( \int_If=0\) parce que si \( I\) est un intervalle entre \( c\) et \( d\) (ouvert ou fermé), nous avons
	\begin{equation}
		\int_{c}^df=\int_a^df-\int_a^cf=0-0=0.
	\end{equation}
	Soient \( \phi\in C^{\infty}_c\big( \mathopen[ a,b\mathclose] \big)\) et \( \epsilon>0\). Par le lemme \ref{PROPooSBBOooHdtWDK}, il existe des intervalles \( J_i\) tels qu'en posant \( h=\sum_i\alpha_i\mtu_{J_i}\) avec des \( \alpha_i\) bien choisis, nous ayons
	\begin{equation}
		\| \phi-h \|_{\infty}<\epsilon.
	\end{equation}
	Pour une telle fonction \( h\), nous avons
	\begin{equation}
		\int_a^bfh=\sum_i\int_{J_i}\alpha_if(t)dt=\sum_i\alpha_i\int_{J_i}f=0.
	\end{equation}
	Et donc
	\begin{subequations}
		\begin{align}
			| \int_a^bf\phi | & \leq \int_a^b| f(t)(\phi-h)(t)dt |+\underbrace{\int_a^b| f(t)h(t)dt |}_{=0} \\
			                  & \leq\int_a^b|f|\| \phi-h \|                                                 \\
			                  & \leq \epsilon\int_a^b|f|.
		\end{align}
	\end{subequations}
	Bref, pour tout \( \epsilon>0\) nous avons \( \int_a^bf\phi\leq \epsilon\int_a^b| f |\). La proposition \ref{PropAAjSURG} nous permet de déduire que \( f=0\) presque partout sur \( \mathopen[ a,b\mathclose]\).
\end{proof}



\begin{proposition}[\cite{BIBooSQMWooEgERdj}]	\label{PROPooUMIPooQGXaPd}
	Soit une fonction réelle \(  f\in L^1\big( \mathopen[ a,b\mathclose]\big)\). Nous définissons
	\begin{equation}
		\begin{aligned}
			F\colon \mathopen[ a,b\mathclose] & \to \eR                 \\
			F(x)                              & \mapsto \int_a^xf(t)dt.
		\end{aligned}
	\end{equation}
	Alors \( F\) est dérivable presque partout sur \( \mathopen[ a,b\mathclose]\) et là où \( F\) est dérivable, nous avons \( F'=f\).
\end{proposition}

\begin{proof}
	Nous divisons en plusieurs cas de généralité croissante.
	\begin{subproof}
		\spitem[En supposant que \( f\) est bornée]
		%-----------------------------------------------------------
		Nous supposons que \( | f(x) |< K\) pour tout \( x\in \mathopen[ a,b\mathclose]\). Pour chaque \( n\geq 1\), nous posons
		\begin{equation}
			\begin{aligned}
				g_n\colon \mathopen[ a,b\mathclose] & \to \eR                                   \\
				x                                   & \mapsto    \frac{ F(x+1/n)-F(x) }{ 1/n }.
			\end{aligned}
		\end{equation}
		En développant un peu la valeur de \( F\) là-dedans, nous trouvons
		\begin{equation}
			g_n(x)=n\int_x^{x+1/n}f.
		\end{equation}
		Vu que \( | f(x) |<K\), nous avons
		\begin{equation}
			| g_n(x) |\leq n\int_x^{x+1/n}K=Kn\frac{1}{ n}=K.
		\end{equation}
		Par la proposition \ref{PROPooHCCWooHaYitq}, \( F\) est absolument continue; la proposition \ref{PROPooMONZooWVkYWb} dit alors qu'elle est à variation bornée, et enfin \ref{PROPooVHGNooSDlrrq} implique que \( F\) est dérivable presque partout. Autrement dit, la limite \( \lim_{n\to\infty}g_n(x)\) existe pour presque tout \( x\). Et là où elle existe, elle vaut forcément \( F'\).

		Soit \( c\in\mathopen[ a,b\mathclose]\). Utilisons à présent le théorème de la convergence dominée de Lebesgue \ref{ThoConvDomLebVdhsTf} sur \( \mathopen[ a,c\mathclose]\). Vérifions les hypothèses :
		\begin{itemize}
			\item
			      Pour tout \( n\), nous avons \( g_n\in L^1\big( \mathopen[ a,b\mathclose] \big)\).
			\item
			      Nous avons \( g_n\to F'\) ponctuellement presque partout.
			\item
			      La fonction constante \( K\) est intégrable sur \(\mathopen[ a,b\mathclose]\) et majore \( | g_n(x) |\) pour tout \( n\) et pour tout \( x\).
		\end{itemize}
		Nous avons
		\begin{equation}		\label{EQooUCNQooDywXUO}
			\lim_{n\to \infty}\int_a^cg_n=\int_a^cF'.
		\end{equation}
		L'intégrale dans le membre de gauche est :
		\begin{equation}		\label{EQooORZDooUYPcAN}
			\int_a^cg_n=\int_a^c\frac{ F(x+1/x)-F(x) }{ 1/n }dx=n\int_a^c\left[  \int_a^{x+1/n}f-\int_a^xf  \right]dx.
		\end{equation}
		Nous nous intéressons d'abord à la première intégrale avec le changement de variables\footnote{Théorème \ref{THOooUMIWooZUtUSg}.} \( u=x+1/x\) :
		\begin{equation}
			\int_a^c\left[ \int_a^{x+1/n}f(t)dt \right]  =\int_a^cF(x+1/n)                                      =\int_{a+1/n}^{c+1/n}F(u)du.
		\end{equation}
		En remettant dans \eqref{EQooORZDooUYPcAN},
		\begin{equation}
			\int_a^cg_n=n\int_{a+1/n}^{c+1/n}F(x)dx-n\int_a^cF(x)dx=n\int_c^{c+1/n}F-n\int_a^{a+1/n}F
		\end{equation}
		Et nous pouvons enfin passer tranquillement à la limite \( n\to \infty\) en utilisant la proposition \ref{PROPooRUSEooFYGLLU} :
		\begin{equation}
			\lim_{n\to \infty}\int_a^cg_n=F(c)-F(a).
		\end{equation}
		Et en reprenant depuis \eqref{EQooUCNQooDywXUO},
		\begin{equation}
			\int_a^c(F'-f)=0
		\end{equation}
		pour tout \( c\in\mathopen[ a,b\mathclose]\). Cela prouve\footnote{Par la proposition \ref{PROPooZOJHooKwZOFW}.} que \( F'(x)=f(x)\) presque partout\footnote{Ici, le «presque partout» est l'intersection du «presque partout» où \( F\) est dérivable et le «presque partout» de la proposition \ref{PROPooZOJHooKwZOFW}.}.

		\spitem[Si \( f\in L^1\) et \( f\geq 0\)]
		%-----------------------------------------------------------
		Nous supposons maintenant que \( f\in L^1\big( \mathopen[ a,b\mathclose] \big)\) et que \( f\geq 0\). Nous posons
		\begin{equation}
			f_n(x)=\begin{cases}
				f(x) & \text{si } 0\leq f(x)\leq n \\
				0    & \text{sinon, }
			\end{cases}
		\end{equation}
		ainsi que
		\begin{equation}
			F_n(x)=\int_a^xf_n(t)dt
		\end{equation}
		et
		\begin{equation}
			G_n(x)=\int_a^x\big( f(t)-f_n(t) \big).
		\end{equation}
		Nous avons \( F=F_n+G_n\), et, comme \( f(t)-f_n(t)\geq 0\), la fonction \( G\) est croissante.

		La fonction \( f_n\) étant majorée par \( n\), la première partie d'applique et nous avons
		\begin{equation}
			F_n'(x)=f_n(x)
		\end{equation}
		pour presque tout \( x\). Le même raisonnement que précédemment, mettant en scène les propositions \ref{PROPooHCCWooHaYitq}, \ref{PROPooMONZooWVkYWb} et \ref{PROPooVHGNooSDlrrq} montre que \( F\) est dérivable presque partout. Donc dans l'égalité \( F=F_n+G_n\), les fonctions \( F\) et \( F_n\) sont dérivables. Nous en déduisons que \( G_n\) est également dérivable et que
		\begin{equation}
			F'(x)=F'_n(x)+G_n'(x)
		\end{equation}
		pour presque tout \( x\). Comme \( G_n\) est croissante, nous avons \( G_n'(x)\geq 0\) et la majoration
		\begin{equation}
			F'(x)=F_n'(x)+G_n'(x)\geq F_n'(x)=f_n(x).
		\end{equation}
		En passant à la limite et en utilisant la proposition \ref{PROPooDOTQooWaMBrO}, nous avons
		\begin{equation}
			F'(x)\geq f(x)
		\end{equation}
		pour presque tout \( x\in \mathopen[ a,b\mathclose]\). En intégrant cette inégalité de \( a\) à \( b\)\footnote{En posant par exemple \( F'=f=0\) sur la partie où cette inégalité n'est pas vraie. De toutes façon cette partie est de mesure nulle.} nous avons
		\begin{equation}
			\int_a^bF'(x)dx\geq \int_a^bf(x)dx=F(b)-F(a).
		\end{equation}
		Ceci n'a aucun rapport avec cela, mais je te rappelle que la proposition \ref{PROPooWFSPooIBogJV}\footnote{Qui s'applique parce que \( F\) est presque partout croissante du fait que \( f\) est supposée positive.} dit que
		\begin{equation}
			\int_a^bF'(x)dx\leq F(b)-F(a).
		\end{equation}
		Nous avons donc
		\begin{equation}
			\int_a^bF'=F(b)-F(a)=\int_a^bf,
		\end{equation}
		et donc
		\begin{equation}
			\int_a^b\big[  F'-f  \big]=0
		\end{equation}
		Vu que \( F'-f\geq 0\), le fait que l'intégrale soit nulle implique que \( F'-f=0\) presque partout\footnote{Encore moins «presque partout» qu'avant parce que c'est l'intersection entre le «presque partout» d'avant et de celui-ci. Mais bon\ldots Tant qu'on a une quantité dénombrable d'intersections, ça reste de mesure nulle. Alors une de plus ou une de moins\ldots}

		\spitem[Le cas générique]
		%-----------------------------------------------------------
		Si \( f\in L^1\big( \mathopen[ a,b\mathclose] \big)\). Nous décomposons \( f\) en parties positives et négatives : \( f=f^++f^-\). En posant \( F_+=\int_a^b f^+\) et \( F_-=\int_a^bf^-\), nous avons \( F=F_++F_-\).

		En appliquant le cas positif à \( f^+\) nous avons immédiatement \( F_+'=f^+\) presque partout. Ensuite nous considérons \( g(x)=-f^-(x)\) et \( G(x)=\int_a^bG\). Nous avons alors \( G'=g\). Mais comme \( G=-F_-\) et \( g=-f^-\), ça donne \( F'_-=f^-\). En mettant tout ensemble,
		\begin{equation}
			F'=F_+'+F_-'=f^++f^-=f.
		\end{equation}
	\end{subproof}
\end{proof}

\begin{proposition} \label{PropJLnPpaw}
	Si \( f\in L^1(\eR)\), alors la fonction
	\begin{equation}
		F(x)=\int_{-\infty}^xf(t)dt
	\end{equation}
	est presque partout dérivable et pour les points où elle l'est, nous avons \( F'(x)=f(x)\).
	%TODOooJRFHooCVxmUp. Prouver ça.
\end{proposition}
\index{fonction!définie par une intégrale}



%+++++++++++++++++++++++++++++++++++++++++++++++++++++++++++++++++++++++++++++++++++++++++++++++++++++++++++++++++++++++++++ 
\section{L'espace de Lebesgues \( L^2\big( \mathopen[ a , b \mathclose] \big)\)}
%+++++++++++++++++++++++++++++++++++++++++++++++++++++++++++++++++++++++++++++++++++++++++++++++++++++++++++++++++++++++++++
\label{SECooUMEPooPMemJz}

L'espace \( L^2\big( \mathopen[ a , b \mathclose] \big)\) est l'espace générique sur lequel nous allons construire les espaces \( L^2\) sur \( \mathopen[ -T , T \mathclose]\) et \( \mathopen[ 0 , 2\pi \mathclose]\). Pour fixer les idées, nous considérons \( b>a\).

Si \( f\) et \( g\) sont dans \( L^2\big( \mathopen[ a , b \mathclose] \big)\), il n'est pas possible de définir \( f*g\) par la formule intégrale usuelle parce que \( f(x_0+t)\) n'existe pas pour tout \( x_0\) et \( t\) dans \( \mathopen[ a , b \mathclose]\). Donc soit nous utilisons un truc pas très net comme étendre les fonctions sur \( \mathopen[ a , b \mathclose]\) en fonctions périodiques sur \( \eR\), soit nous intégrons vraiment seulement sur \( \mathopen[ a , b \mathclose]\).

Nous n'allons suivre aucune de ces deux voies ou plutôt les deux en même temps. Nous allons seulement tout ramener de \( S^1\) que nous venons de travailler.

\begin{propositionDef}
	Sur \( \mathopen[ a , b \mathclose]\) nous considérons la mesure de Lebesgue \( dx\) usuelle. Si \( f,g\in L^2\big( \mathopen[ a , b \mathclose] \big)\), alors
	\begin{enumerate}
		\item
		      \( f\bar g\in L^1\big( \mathopen[ a , b \mathclose] \big)\),
		\item
		      La formule
		      \begin{equation}    \label{EQooCRSXooPEopzm}
			      \langle f, g\rangle =\int_a^bf(x)\overline{ g(x) }dx.
		      \end{equation}
		      définit un produit hermitien\footnote{Définition \ref{DefMZQxmQ}.}.
	\end{enumerate}
\end{propositionDef}

\begin{proof}
	Pour le premier point, d'abord \( \bar g\in L^2\), et ensuite l'inégalité de Hölder \ref{ProptYqspT}\ref{ITEMooNDKPooRKdmgS} dit que \( f\bar g\) est dans \( L^1\).

	Le fait que la formule donne une forme sesquilinéaire découle des propriétés de l'intégrale. Le fait que ce soit hermitien découle du fait que \( \overline{ \int f }=\int\bar f\).

	Et enfin,
	\begin{equation}
		\langle f,f \rangle =\int_a^b| f(x) |^2dx\geq 0.
	\end{equation}

	Si il existe une partie de mesure non nulle \( A\) sur laquelle \( f\neq 0\), alors
	\begin{equation}
		\int_a^b| f |^2=\int_A| f |^2+\int_{\mathopen[ a , b \mathclose]\setminus A}| f |^2.
	\end{equation}
	Le premier terme est strictement positif, alors que le second est positif ou nul. Donc le tout est strictement positif.
\end{proof}

\begin{normaltext}
	Il y a (au moins) deux conventions possibles pour le produit scalaire :
	\begin{equation}    \label{EQooAJLHooTKraYR}
		\langle f, g\rangle =\int_a^bf(x)\bar g(x)\,dx
	\end{equation}
	et
	\begin{equation}    \label{EQooSJJEooOLGzDG}
		\langle f, g\rangle =\frac{1}{ b-a }\int_a^bf(x)\bar g(x)\,dx
	\end{equation}
	L'argument en faveur de \eqref{EQooAJLHooTKraYR}. Il est plus facile d'être cohérent avec les espaces \( L^p(\Omega, \tribA, \mu)\). En effet, pour de telles espaces, on a vite \( \mu(\Omega)=\infty\) et donc du mal à mettre un coefficient \( \frac{1}{ \mu(\Omega) }\) dans la définition de la norme. Voir la définition \ref{DEFooTHIDooWYzBtn}.

	L'argument en faveur de \eqref{EQooSJJEooOLGzDG}. Le facteur \( dx\) a les mêmes unités que \( b-a\). En mettant donc le facteur \( b-a\), le tout a les unités de \( fg\), comme il se doit pour le produit scalaire.
\end{normaltext}

\begin{proposition}[\cite{BIBooARJKooLuqoxW}]	\label{PROPooSPDIooKntTyz}
	Nous avons \( L^2\big( \mathopen[ a,b\mathclose] \big)\subset L^1\big( \mathopen[ a,b\mathclose] \big)\).
\end{proposition}

\begin{proof}
	Nous avons \( | f |\leq | f |^2+1\). Donc
	\begin{equation}
		\| f \|_1=\int_a^b| f |\leq \int_{\mathopen[ a,b\mathclose]}\big( | f |^2+1 \big)=\int_{\mathopen[ a,b\mathclose]}| f |^2+\int_{\mathopen[ a,b\mathclose]}1.
	\end{equation}
	Le premier terme est fini parce que \( f\in L^2\) et le second vaut \( | b-a |\). Donc le tout est fini et \( \| f \|_1<\infty\).
\end{proof}

\begin{proposition}		\label{PROPooLNALooSVNMfe}
	Nous considérons
	\begin{equation}
		\begin{aligned}
			s\colon \mathopen[ a , b \mathclose] & \to \mathopen[ 0 , 2\pi \mathclose] \\
			x                                    & \mapsto 2\pi\frac{ x-a }{ b-a }
		\end{aligned}
	\end{equation}
	ainsi que l'application usuelle
	\begin{equation}
		\begin{aligned}
			\varphi\colon \mathopen[ 0 , 2\pi \mathclose[ & \to S^1          \\
			t                                             & \mapsto  e^{it}.
		\end{aligned}
	\end{equation}
	L'application
	\begin{equation}
		\begin{aligned}
			\phi\colon L^2\big( \mathopen[ a , b \mathclose] \big) & \to L^2(S^1)                                \\
			\phi(f)(z)                                             & =f\big( (s^{-1}\circ \varphi^{-1})(z) \big)
		\end{aligned}
	\end{equation}
	est une bijection isométrique.
\end{proposition}

\begin{proof}
	La preuve du fait que \( \phi\) est isométrique suffira pour prouver qu'elle prend bien ses valeurs dans \( L^2(S^1)\).
	\begin{subproof}
		\spitem[Isométrique]
		C'est un calcul :
		\begin{subequations}
			\begin{align}
				\| \phi(f) \|^2 & =\langle \phi(f), \phi(f)\rangle                                                                         \\
				                & =\int_{S^1}| \phi(f) |^2                                                                                 \\
				                & =\frac{1}{ 2\pi }\int_{\mathopen\lbrack 0 , 2\pi \mathclose\lbrack}| \phi(f)\big( \varphi(u) \big) |^2du \\
				                & =\frac{1}{ 2\pi }\int_0^{2\pi}| f\big( (s^{-1}\circ\varphi^{-1}\circ\varphi)(u) \big) |^2du              \\
				                & =\frac{1}{ 2\pi }\int_0^{2\pi}| f\big( s^{-1}(u) \big) |^2du.
			\end{align}
		\end{subequations}
		Il est temps de faire le changement de variables\footnote{Nous le faisons de façon un peu informelle; soyez capable de bien justifier.} \( y=s^{-1}(u)\), c'est-à-dire
		\begin{equation}
			y=\frac{ b-a }{ 2\pi }u+a.
		\end{equation}
		En ce qui concerne la différentielle,
		\begin{equation}
			dy=\frac{ b-a }{ 2\pi }du
		\end{equation}
		et pour les bornes, si \( u=0\) alors \( y=a\) et si \( u=2\pi\), \( y=b\). Donc
		\begin{subequations}
			\begin{align}
				\| \phi(f) \|^2 & =\frac{1}{ 2\pi }\int_a^b| f(y) |^2\frac{ 2\pi }{ b-a }dy \\
				                & =\frac{1}{ b-a }\int_a^b| f |^2                           \\
				                & =\| f \|^2.
			\end{align}
		\end{subequations}
		\spitem[Injectif]
		Soit \( f\) telle que \( \phi(f)=0\). Alors pour tout \( z\in S^1\) nous avons
		\begin{equation}
			f\big( (s^{-1}\circ\varphi^{-1})(z) \big)=0.
		\end{equation}
		Vu que \( s^{-1}\circ\varphi^{-1}\colon S^1 \to \mathopen\lbrack a , b \mathclose[\) est une bijection, pour tout \( u\in\mathopen\lbrack a , b \mathclose[\) nous avons \( f(u)=0\). Donc \( f=0\) dans \( L^2\big( \mathopen\lbrack a , b \mathclose] \big)\) parce que du point de vue de \( L^2\), que l'on prenne ou non les bornes, ce n'est pas important.
		\spitem[Surjectif]
		Si \( g\in L^2(S^1)\), alors en posant
		\begin{equation}
			f(u)=g\big( (\varphi\circ s)(u) \big)
		\end{equation}
		nous avons \( g=\phi(f)\).
	\end{subproof}
\end{proof}

\begin{definition}
	En ce qui concerne le produit de convolution, si \( f\) et \( g\) sont des fonctions sur \( \mathopen\lbrack a , b \mathclose]\) nous définissons
	\begin{equation}
		f*g=\phi^{-1}\big( \phi(f)*\phi(g) \big)
	\end{equation}
	tant que les formules ont un sens.
\end{definition}

\begin{definition}
	Le \defe{système trigonométrique}{système trigonométrique} sur \( \mathopen[ a , b \mathclose]\) est l'ensemble de fonctions
	\begin{equation}
		\begin{aligned}
			e_k\colon \mathopen[ a , b \mathclose] & \to \eC                                               \\
			t                                      & \mapsto  \frac{1}{ \sqrt{ b-a } } e^{2\pi i kt/(b-a)}
		\end{aligned}
	\end{equation}
	pour \( k\in \eZ\).
\end{definition}

\begin{normaltext}
	Pour prouver que ce système est une base hilbertienne, il faut prouver que c'est orthonormal et total. Pour prouver que le système est total, il y a (au moins) trois moyens.
	\begin{enumerate}
		\item
		      Prouver que le système est orthonormal maximal et invoquer la proposition \ref{PROPooLDXFooRaxBsI}\ref{ITEMooVUFXooDrVwum}. Cela est fait dans \cite{BIBooZYKMooGGbwyI}.
		\item
		      Prouver que le système trigonométrique sépare les points pour la densité dans les fonctions continues. Ensuite travailler comme dans \cite{BIBooQLKHooOlskCs}.
		\item
		      Adapter le théorème \ref{ThoQGPSSJq} pour prouver directement la densité des polynômes trigonométriques dans \( L^2\big( \mathopen[ a , b \mathclose] \big)\).
	\end{enumerate}
\end{normaltext}

\begin{proposition}[\cite{BIBooZYKMooGGbwyI}]	\label{PROPooKJQKooYeNxIq}
	Il n'existe pas de fonctions \( f\in L^2\big( \mathopen[ -\pi,\pi\mathclose] \big)\) telles que
	\begin{enumerate}
		\item
		      \( f\) est à valeurs réelles,
		\item
		      \( \langle f, e_k \rangle=0\) pour tout \( k\in \eZ\).
		\item
		      Il existe \( h>0\) et \( \alpha>0\) tels que pour tout \( x\in\mathopen[ -h,h\mathclose]\), \( 0<\alpha<f(x)\).
	\end{enumerate}
\end{proposition}


\begin{proof}
	En plusieurs parties.
	\begin{subproof}
		\spitem[Trigonométrie]
		%-----------------------------------------------------------


		Vu que \( a=-\pi\) et \( b=\pi\) nous avons
		\begin{equation}
			e_k(t)=\frac{1}{ \sqrt{2\pi}}\big( \cos(kt)+i\sin(kt) \big).
		\end{equation}
		Étant donné que \( f\) est à valeurs réelles, \( \langle f, e_k \rangle\) se décompose facilement en parties réelles et imaginaires :
		\begin{equation}
			\langle f, e_k \rangle=\frac{1}{ \sqrt{2\pi}}\int_{-\pi}^{\pi}f(t)\cos(kt)dt+\frac{ i }{ \sqrt{2\pi} }\int_{-\pi}^{\pi}f(t)\sin(kt)dt.
		\end{equation}
		L'hypothèse \( \langle f, e_k \rangle=0\) demande en particulier l'annulation de la partie réelle et donc que \( \langle f, c_k \rangle=0\) pour tout \( k\) où
		\begin{equation}
			c_k(t)=\cos(kt).
		\end{equation}

		\spitem[Les fonctions \( P_n\)]
		%-----------------------------------------------------------

		Nous posons
		\begin{equation}
			P_n(x)=\big( 1+\cos(x)-\cos(h) \big)^n.
		\end{equation}
		Grâce aux formules de linéarisation (proposition \ref{PROPooXGABooBAsbsq}), \( P_n\) est une combinaison linéaires des \( c_k\). Donc l'hypothèse \( \langle f, e_k \rangle=0\) pour tout \( k\) implique \( \langle f, P_n \rangle=0\) pour tout \( n\).


		\spitem[Une étude de fonction]
		%-----------------------------------------------------------
		Soit \( s(x)=1+\cos(x)-\cos(h)\). Nous avons :
		\begin{enumerate}
			\item
			      \( s(x)=1\) si et seulement si \( x=\pm h\).
			\item
			      \( s(0)=2-\cos(h)>1\)
			\item
			      \( | s(\pi) |=| -\cos(h) |=| \cos(h) |<1\).
			\item
			      \( | s(-\pi) |<1\).
		\end{enumerate}
		Donc par le théorème des valeurs intermédiaires \ref{ThoValInter} nous avons
		\begin{enumerate}
			\item
			      \( s(x)>1\) pour \( x\in\mathopen] -h,h\mathclose[\) et donc, sur cet intervalle,
			      \begin{equation}
				      P_n(x)\stackrel{ n\to\infty}{\longrightarrow} \infty.
			      \end{equation}
			\item
			      \( | s(x) |<1\) sur \( x\in\mathopen[ -\pi,-h\mathclose[\cup \mathopen] h,\pi\mathclose]\), et donc pour \( x\) dans cette partie,
			      \begin{equation}
				      P_n(x)\stackrel{ n\to \infty}{\longrightarrow} 0.
			      \end{equation}
			      Si vous aimez les études de fonctions, je vous laisse vous demander pour quelles valeurs de \( x\) et de \( h\), cette limite a des signes alternés.
		\end{enumerate}

		\spitem[Une intégrale à découper]
		%-----------------------------------------------------------
		Nous savons que \( \langle f, P_n \rangle=0\) pour tout \( n\), et donc \( \lim_{\to \infty}\langle f, P_n \rangle=0\). Nous allons expliciter ce produit scalaire, faire la limite et trouver une contradiction. Nous avons \( \langle f, P_n \rangle=A_n+B_n+C_n\) avec
		\begin{subequations}
			\begin{align}
				A_n & = \int_{-\pi}^{-h}f(x)P_n(x)dx \\
				B_n & = \int_{-h}^hf(x)P_n(x)dx      \\
				C_n & = \int_h^{\pi}f(x)P_n(x)dx.
			\end{align}
		\end{subequations}

		\spitem[Calcul de \( C_n\)]
		%-----------------------------------------------------------
		Nous utilisons la convergence dominée de Lebesgue \ref{ThoConvDomLebVdhsTf}. La fonction \( fP_n\) est continue sur le compact \( \mathopen[ h,\pi\mathclose]\) et donc majorée et donc dans \( L^1\big( \mathopen[ h,\pi\mathclose] \big)\). De plus \( fP_n\to 0\) simplement et comme \( | fP_n |\) est majorée par une constante parce que \( | P_n |\to 0\) et \( | f |<\alpha\). Tout cela fait que la convergence dominée fonctionne et
		\begin{equation}
			\lim_{n\to \infty}\int_h^{\pi}f(x)P_(x)dx=\int_h^{\pi}f(x)\lim_{n\to \infty}P_n(x)dx=0.
		\end{equation}

		\spitem[Calcul de \( A_n\)]
		%-----------------------------------------------------------
		Même chose que \( C_n\).

		\spitem[Calcul de \( B_n\)]
		%-----------------------------------------------------------
		Ici nous utilisons le lemme de Fatou \ref{LemFatouUOQqyk}. Nous savons que \( \lim_{n\to\infty}\langle f, P_n \rangle\) existe et vaut zéro; de plus \( \langle f, P_n \rangle=A_n+B_n+C_n\) et nous savons déjà que \( \lim_{n\to \infty}A_n\) et \( \lim_{n\to\infty}C_n\) existent et valent zéro. Donc\footnote{Proposition \ref{PROPooZRCBooKiJhDg}\ref{ITEMooSHPAooQyEkgT} utilisée à l'envers.} \( \lim_{n\to \infty}B_n\) existe (et vaut zéro). Cela pour dire qu'à droite du lemme de Fatou nous pouvons mettre une limite usuelle au lieu d'une limite inférieure.

		D'autre part \( f\) est bornée et \( P_n\to \infty\), donc \( \liminf_{n\to \infty}f(x)P_n(x)=\lim_{n\to \infty}f(x)P_n(x)=\infty\). Donc à gauche aussi du lemme Fatou nous pouvons mettre une limite usuelle. Bref :
		\begin{subequations}
			\begin{align}
				\lim_{n\to \infty}\int_{-h}^hf(x)P_n(x)dx & \geq \int_{-h}^h\liminf_{n\to \infty}\big( f(x)P_n(x) \big)dx \\
				                                          & \geq \alpha\int_{-h}^h\lim_{n\to \infty}P_n(x)dx              \\
				                                          & =\infty.
			\end{align}
		\end{subequations}

		\spitem[Conclusion]
		%-----------------------------------------------------------
		Nous avons prouvé que \( \lim_{n\to \infty}B_n=\infty\) alors que ça devait être \( 0\). Contradiction. Il n'existe donc pas de fonctions \( f\) vérifiant toutes les propriétés demandées.
	\end{subproof}
\end{proof}


\begin{lemma}[\cite{MonCerveau}]	\label{LEMooRWKAooLRdUdr}
	Soient \( a,b,c,d\in \eR\) tels que \( a<b\) et \( c<d\). Soient des intervalles fermés \( I\subset\mathopen[ a,b\mathclose] \) et \( J\subset\mathopen[ c,d\mathclose]\). Il existe une bijection continue \(\alpha \colon \mathopen[ a,b\mathclose]\to  \mathopen[ c,d\mathclose] \) telle que \( \alpha(I)=J\).
\end{lemma}

\begin{proof}
	Les intervalles \( I\) et \( J\) sont fermés; la proposition \ref{PROPooBWXYooDcwXrp} dit qu'il existe \( k_1,k_2,l_1,l_2\in \eR\) tels que
	\begin{equation}
		\begin{aligned}[]
			I & =\mathopen[ k_1,k_2\mathclose]\subset \mathopen[ a,b\mathclose]  \\
			J & =\mathopen[ l_1,l_2\mathclose]\subset \mathopen[ c,d\mathclose].
		\end{aligned}
	\end{equation}
	Nous considérons les application linéaire suivantes :
	\begin{enumerate}
		\item
		      \(\alpha_1 \colon \mathopen[ a,k_1\mathclose]\to \mathopen[ c,l_1,\mathclose]  \) telle que \( \alpha_1(a)=c\) et \( \alpha_1(k_1)=l_1\).
		\item
		      \(\alpha_2 \colon \mathopen[ k_1,k_2\mathclose]\to \mathopen[ l_1,l_2\mathclose]  \) telle que \( \alpha_2(k_1)=l_1\) et \( \alpha_2(k_2)=l_2\).
		\item
		      \(\alpha_3 \colon \mathopen[ k_2,b\mathclose]\to \mathopen[ l_2,d\mathclose]  \) telle que \( \alpha_3(k_2)=l_2\) et \( \alpha_3(b)=d\).
	\end{enumerate}
	Et finalement nous posons
	\begin{equation}
		\begin{aligned}
			\alpha\colon \mathopen[ a,b\mathclose] & \to\mathopen[ c,d\mathclose]                                      \\
			t                                      & \mapsto \begin{cases}
				                                                 \alpha_1(t) & \text{si } t\in\mathopen[ a,k_1\mathclose]  \\
				                                                 \alpha_2(t) & \text{si }t\in\mathopen] k_1,k_2\mathclose[ \\
				                                                 \alpha_3(t) & \text{si }t\in\mathopen[ k_2,b\mathclose].
			                                                 \end{cases}
		\end{aligned}
	\end{equation}
	Celle fonction est continue parce que  \( \alpha_1(k_1)=\alpha_2(k_1)=l_1\) et \( \alpha_2(k_2)=\alpha_3(k_2)=l_2\).
\end{proof}


\begin{proposition}[\cite{MonCerveau}]	\label{PROPooMHMHooMwiVbz}
	Soit une application continue \( f\in L^2\big( \mathopen[ a,b\mathclose] \big)\). Nous supposons que \( \langle f, e_k \rangle=0\) pour tout \( k\in \eZ\). Alors \( f=0\).
\end{proposition}

\begin{proof}
	Nous allons procéder par généralisations successives.
	\begin{subproof}
		\spitem[Premier pas]
		%-----------------------------------------------------------
		Nous supposons que
		\begin{itemize}
			\item
			      \( f\) est à valeurs réelles
			\item
			      \( \mathopen[ a,b\mathclose]=\mathopen[ -\pi, \pi,\mathclose]\)
		\end{itemize}
		Si \( f\neq 0\) alors il existe un intervalle \( I\subset\mathopen[ -\pi,\pi\mathclose]\) sur lequel \( f\) est non nulle. Pour fixer les idées, nous disons que \( f\) y est strictement positive : il existe \( \alpha>0\) tel que \( f(x)>\alpha>0\) pour tout \( x\in I\).

		Soit \( \pi/2>h>0\). Le lemme \ref{LEMooRWKAooLRdUdr} donne une bijection continue \(\alpha \colon \mathopen[ -\pi,\pi\mathclose]\to \mathopen[ -\pi,\pi\mathclose]  \) telle que \( \alpha\big( \mathopen[ -h,h\mathclose] \big)=I\). La fonction \( f\circ\alpha\) vérifie les hypothèses de la proposition \ref{PROPooKJQKooYeNxIq} et est donc une contradiction.

		Nous en déduisons que sous les hypothèses de ce point, \( f=0\) comme il se doit.

		\spitem[Deuxième pas]
		%-----------------------------------------------------------
		Nous supposons que
		\begin{itemize}
			\item
			      \( f\) est à valeurs réelles.
		\end{itemize}
		Nous prenons une bijection croissante continue \(\alpha \colon \mathopen[ -\pi,\pi\mathclose]\to \mathopen[ a,b\mathclose]  \). La fonction \( f\circ \alpha\) est dans les hypothèses du premier pas. Donc \( f\circ \alpha=0\) et donc \( f=0\).

		\spitem[Troisième pas]
		%-----------------------------------------------------------
		Plus d'hypothèses. Vu que \( f\) est à valeurs dans \( \eC\) nous pouvons écrire \( f=f_1+if_2\) où \( f_1\) et \( f_2\) sont à valeurs dans \( \eR\). Les fonctions \( f_1\) et \( f_2\) séparément vérifient les hypothèses du deuxième pas. Donc \( f_1=f_2=0\) et donc \( f=0\).
	\end{subproof}
\end{proof}


\begin{theorem}[\cite{BIBooZYKMooGGbwyI}]       \label{THOooAVWIooDhnjpN}
	Le système trigonométrique \( \{ e_k \}_{k\in \eZ}\) est une base hilbertienne de \( L^2\big( \mathopen[ a , b \mathclose] \big)\).
\end{theorem}

\begin{proof}
	En vertu de la proposition \ref{PROPooLDXFooRaxBsI}\ref{ITEMooVUFXooDrVwum}, ils nous suffit de prouver que \( \{ e_k \}_{k\in \eZ}\) est une famille orthonormale maximale\footnote{Définition \ref{DEFooRFATooDRKWoJ}.}.

	\begin{subproof}
		\spitem[Orthonormale]
		Nous calculons le produit :
		\begin{equation}
			\langle e_k, e_l\rangle =\frac{1}{ b-a }\int_a^b e^{2\pi i kt/(b-a)} e^{-2\pi i lt/(b-a)}dt
			=\frac{1}{ b-a }\int_a^b e^{2\pi i t(k-l)/(b-a)}dt.
		\end{equation}
		Justifications.
		\begin{itemize}
			\item
			      Le complexe conjugué de \(  e^{it}\) est \(  e^{-it}\) par le corolaire \ref{CORooWZFIooDTCoRo}.
			\item
			      Les exponentielles sont «fusionnées» avec la proposition \ref{PropdDjisy}\ref{ITEMooRLHCooJTuYKV}.
		\end{itemize}
		Si \( k=l\) nous avons
		\begin{equation}
			\langle e_k, e_k\rangle =\frac{1}{ b-a }\int_a^b1\,dt=1.
		\end{equation}
		Si \( k\neq l\) nous pouvons continuer avec une primitive. Une primitive de \(  e^{at}\) est \( \frac{1}{ a } e^{at}\). Dans notre cas, en regroupant toutes les constantes sous le nom \( C\) nous avons :
		\begin{equation}
			\langle e_k, e_l\rangle =C\left[   e^{2\pi it(k-l)/(b-a)} \right]_a^b=C\left(  e^{2\pi i a(k-l)/(b-a)}- e^{2\pi i b(k-l)/(b-a)} \right).
		\end{equation}
		Cela vaut zéro. Vous n'y croyez pas ? Faites un effort, relisez le corolaire \ref{CORooTFMAooHDRrqi}, et remarquez que
		\begin{equation}
			\frac{ 2\pi a(k-l) }{ b-a }-\frac{ 2\pi b(k-l) }{ b-a }=2\pi (k-l)\in 2\pi \eZ.
		\end{equation}
		\spitem[Maximale]
		Nous prouvons que \( \{ e_k \}_{k\in \eZ}\) est maximale, c'est à dire que nous supposons que \( \langle e_k, f\rangle =0\) pour tout \( k\), et nous montrons que \( f=0\). Nous allons largement confondre \( f\in L^2\) et une fonction \( f\) qui représente la classe.

		Nous considérons la fonction
		\begin{equation}
			\begin{aligned}
				\phi\colon \mathopen[ a,b\mathclose] & \to \eC                 \\
				x                                    & \mapsto \int_a^xf(t)dt.
			\end{aligned}
		\end{equation}
		Vu que \( f\in L^2\big( \mathopen[ a,b\mathclose] \big)\subset L^1\big( \mathopen[ a,b\mathclose] \big)\) (proposition \ref{PROPooSPDIooKntTyz}), la proposition \ref{PROPooANISooKzQrnH} montre que \( \phi\) est continue.

		Étant donné que \( \phi\) est continue sur le compact \( \mathopen[ a,b\mathclose]\), elle est bornée et donc dans \( L^2\big( \mathopen[ a,b\mathclose] \big)\). Nous avons
		\begin{equation}
			\langle \phi, e_n \rangle  =\int_a^be_n(x)\int_a^x\overline{f(t)}dt
			=\int_{a}^b\int_a^xe_n(x)\overline{f(t)}dt
			=\int_a^b\int_a^bs(x,t)dt\,dx
		\end{equation}
		où
		\begin{equation}
			s(x,t)=e_n(x)\overline{f(t)}\mtu_{\mathopen[ a,x\mathclose]}(t).
		\end{equation}
		Nous permutons les intégrales en utilisant \ref{NORMooKIRJooPvyPWQ}. Vu que l'intégration en chaine fonctionne, nous avons \( s\in L^1\big( \mathopen[ a,b\mathclose]\times \mathopen[ a,b\mathclose] \big)\), et donc Fubini permute les intégrales :
		\begin{equation}		\label{EQooVSYBooSknVOj}
			\langle \phi, e_n \rangle=\int_a^b\int_a^be_n(x)\overline{f(t)}\mtu_{\mathopen[ a,x\mathclose]}(t)dx\,dt.
		\end{equation}
		Fixons un \( t\) et concentrons nous sur l'intégrale sur \( x\) :
		\begin{subequations}
			\begin{align}
				\int_a^be_n(x)\mtu_{\mathopen[ a,x\mathclose]}(t)dx & =\int_t^be_n(x)dx                                                               \\
				                                                    & =\frac{ b-a }{ 2\pi i n }\Big( \exp(2i\pi nb/(b-a))-\exp(2i\pi nt/(b-a)) \Big).
			\end{align}
		\end{subequations}
		Nous remettons ça dans l'intégrale \eqref{EQooVSYBooSknVOj}, et nous obtenons
		\begin{equation}
			\langle \phi, e_n \rangle=\frac{ b-a }{ 2i\pi n }(I_1+I_2)
		\end{equation}
		avec
		\begin{equation}
			I_1=\int_a^be^{2i\pi nb/(b-a)}\overline{f(t)}dt
		\end{equation}
		et
		\begin{equation}
			I_2=\int_a^be^{2i\pi nt/(b-a)}\overline{f(t)}dt.
		\end{equation}
		À coefficients près, \( I_1\) est juste \( \int_a^b\overline{f(t)}dt\) qui n'est autre que \( \langle f, e_0 \rangle\) (à autres coefficients près).  Bref, \( I_1=0\). En ce qui concerne \( I_2\), ce qui est dans l'intégrale est, à coefficients près, \( e_n(t)\overline{f(t)}\), et donc \( I_2=C\langle e_n, f \rangle=0\).

		Tout ça pour dire que \( \phi\) est un élément de \( L^2\big( \mathopen[ a,b\mathclose] \big)\) tel que \( \langle \phi, e_n \rangle=0\) pour tout \( n\in \eZ\). La proposition \ref{PROPooMHMHooMwiVbz} nous dit qu'alors \( \phi(x)=0\) pour tout \( x\in \mathopen[ a,b\mathclose]\). La proposition \ref{PROPooZOJHooKwZOFW} indique que \( f=0\) presque partout, ce qu'il fallait démontrer.
	\end{subproof}
\end{proof}



%+++++++++++++++++++++++++++++++++++++++++++++++++++++++++++++++++++++++++++++++++++++++++++++++++++++++++++++++++++++++++++ 
\section{Sur \( \mathopen\lbrack -T , T \mathclose\lbrack\)}
%+++++++++++++++++++++++++++++++++++++++++++++++++++++++++++++++++++++++++++++++++++++++++++++++++++++++++++++++++++++++++++

Pour rappel, les éléments de \( L^2\) sont des classes de fonctions à valeurs dans \( \eC\).

\begin{proposition}     \label{PROPooHNJZooGfRCfU}
	Les fonctions
	\begin{equation}
		\begin{aligned}
			e_n\colon \mathopen[ -T , T \mathclose] & \to \eC                                        \\
			t                                       & \mapsto \frac{1}{ \sqrt{ 2T } } e^{\pi int/T}.
		\end{aligned}
	\end{equation}
	forment une base hilbertienne\footnote{Définition \ref{DEFooADQXooFoIhTG}.} de \( L^2\big( \mathopen[ -T , T \mathclose[ \big)\).
\end{proposition}

\begin{proof}
	C'est un cas particulier du théorème \ref{THOooAVWIooDhnjpN}.
\end{proof}




%--------------------------------------------------------------------------------------------------------------------------- 
\subsection{Le cas dans \( \mathopen[ 0 , 2\pi \mathclose]\)}
%---------------------------------------------------------------------------------------------------------------------------

En pratique, nous n'allons pas souvent travailler avec des fonctions sur intervalle symétrique \( \mathopen[ -T , T \mathclose]\), mais le plus souvent nous serons sur \( \mathopen[ 0 , 2\pi \mathclose]\).

Nous notons ici une conséquence du théorème~\ref{ThoGVmqOro} dans le cas de l'espace \( L^2\). La proposition suivante est une petite partie du corolaire~\ref{CorQETwUdF}, qui sera d'ailleurs démontré de façon indépendante.

\begin{proposition}
	Si nous avons une suite de réels \( (a_k)\) telle que \( \sum_{k=0}^{\infty}| a_k |^2<\infty\) alors la suite
	\begin{equation}
		f_n(x)=\sum_{k=0}^na_k e^{ikx}
	\end{equation}
	converge dans \( L^2\big( \mathopen] 0 , 2\pi \mathclose[ \big)\).
\end{proposition}

\begin{proof}
	Quitte à séparer les parties réelles et imaginaires, nous pouvons faire abstraction du fait que nous parlons d'une série de fonctions à valeurs dans \( \eC\) au lieu de \( \eR\).

	Un simple calcul est :
	\begin{equation}    \label{EqHVdJxZT}
		\| f_n-f_m \|^2\leq\int_0^{2\pi}\sum_{k=n}^m| a_k |^2dx\leq 2\pi\sum_{k=n}^m| a_k |^2.
	\end{equation}
	Par hypothèse le membre de droite est \( | s_m-s_n |\) où \( s_k\) dénote la suite des sommes partielles de la série des \( | a_k |^2\). Cette dernière est de Cauchy (parce que convergente dans \( \eR\)) et donc la limite \( n\to\infty\) (en gardant \( m>n\)) est zéro. Donc la suite des \( f_n\) est de Cauchy dans \( L^2\) et donc converge dans \( L^2\).
\end{proof}

\begin{normaltext}
	Adaptons tout cela pour l'espace \( L^2\big( \mathopen[ 0 , 2\pi \mathclose] \big)\). Nous posons
	\begin{equation}        \label{EQooBFKDooMkCZOt}
		\langle f, g\rangle =\int_0^{2\pi}f(t)\overline{ g(t) }dt
	\end{equation}
	et
	\begin{equation}        \label{EQooKMYOooLZCNap}
		e_n(t)=\frac{1}{ \sqrt{ 2\pi } } e^{int}.
	\end{equation}
\end{normaltext}

\begin{normaltext}
	Attention que \( e_n(x)\) n'est pas exactement \(  e^{inx}\) : il y a un coefficient. Lorsque ça a un sens, la théorie de Fourier permet d'écrire
	\begin{equation}
		f(x)=\sum_{n\in \eZ}c_n(f) e^{inx}.
	\end{equation}
	Ici les \( c_n(f)\) sont les coefficients de Fourier de \( f\). Ce développement n'est pas le même que
	\begin{equation}
		f(x)=\sum_{n\in \eZ}a_n(f)e_n(x).
	\end{equation}
	Dans cette dernière égalité, les \( a_n(f)\) ne sont pas les coefficients de Fourier, mais \( a_n=\langle f, e_n\rangle \). Le lien entre les deux est fondamentalement l'objet du corolaire \ref{CordgtXlC}.
\end{normaltext}


L'importance du système trigonométrique défini en \ref{DEFooGCZAooFecAHB} est d'être une base de \( L^2\big( \mathopen[ 0 , 2\pi \mathclose] \big)\), comme précisé dans le lemme suivant.
\begin{lemma}       \label{LEMooBJDQooLVPczR}
	Le système trigonométrique \( \{ e_n \}_{n\in \eZ}\) est une base hilbertienne\footnote{Définition \ref{DEFooADQXooFoIhTG}.} de \( L^2\big( \mathopen[ 0 , 2\pi \mathclose] \big)\).
\end{lemma}

\begin{proof}
	Cas particulier du théorème \ref{THOooAVWIooDhnjpN}.
\end{proof}

Note : le théorème~\ref{ThoDPTwimI} donne aussi la densité, mais sera démontré plus tard, indépendamment. Voir aussi les thèmes~\ref{THEooPUIIooLDPUuq} et~\ref{THEMooNMYKooVVeGTU}.

Pour un élément donné \( f\in L^2\big( \mathopen[ 0 , 2\pi \mathclose] \big)\), nous définissons\nomenclature[Y]{\( S_nf\)}{somme partielle de série de Fourier}
\begin{equation}
	S_nf=\sum_{k=-n}^n\langle f, e_k\rangle e_k
\end{equation}
et nous avons le théorème suivant, qui récompense les efforts consentis à propos de la densité des polynômes trigonométriques dans \( L^2\).

\begin{theorem} \label{ThoYDKZLyv}
	Soit \( f\in L^2\big( \mathopen[ 0 , 2\pi \mathclose] \big)\). Nous avons égalité\footnote{Notons que la somme sur \( \eZ\) dans \eqref{EqXMMRpSN} est commutative; il n'est donc pas besoin d'être plus précis.}
	\begin{equation}    \label{EqXMMRpSN}
		f=\sum_{n\in \eZ}c_n(f)e_n
	\end{equation}
	dans \( L^2\).

	Nous avons aussi la convergence
	\begin{equation}    \label{EqRBWKsYP}
		S_nf\stackrel{L^2}{\to} f.
	\end{equation}
\end{theorem}

\begin{proof}
	Le système trigonométrique \( \{ e_n \}_{n\in \eZ}\) est total pour l'espace de Hilbert \( L^2\big( \mathopen[ 0 , 2\pi \mathclose] \big)\) (sans périodicité particulière). Donc le point~\ref{ItemQGwoIxi} du théorème~\ref{ThoyAjoqP} nous donne l'égalité demandée.

	La convergence \eqref{EqRBWKsYP} est une reformulation de l'égalité \eqref{EqXMMRpSN}.
\end{proof}

\begin{normaltext}
	Obtenir la convergence \( L^2\) ne demande pas d'hypothèses de périodicité : la convergence \eqref{EqRBWKsYP} est automatique du fait que le système trigonométrique soit total. Ce n'est cependant pas plus qu'une convergence \( L^2\) et elle ne demande pas \( f(0)=f(2\pi)\), même si pour chacun des \( e_k\) nous avons \( e_k(0)=e_k(2\pi)\).

	Si \( f(2\pi)\neq f(0)\), alors il existe tout de même une suite \( (f_n)\) convergente vers \( f\) au sens \( L^2\) telle que \( f_n(0)=f_n(2\pi)\). Cela ne contredit en rien le fait que \( e_k(0)=e_k(2\pi)\) parce que dans \( L^2\), la valeur d'un point seul n'a pas d'importance.

	Si nous voulons une vraie convergence ponctuelle ou uniforme \( (S_nf)(x)\to f(x)\), alors il faut ajouter des hypothèses sur la continuité de \( f\), sa périodicité ou le comportement des coefficients \( c_n\). Voir aussi le thème~\ref{THMooHWEBooTMInve}.
\end{normaltext}

\begin{example}     \label{EXooQDWUooLtuIOm}
	Si \( f\in L^2\big( \mathopen[ 0 , 2\pi \mathclose] \big)\) est (la classe de) une fonction à valeurs réelles, alors on peut la développer avec nettement moins de termes. D'abord nous savons que \( e_{-n}=\overline{ e_n }\), et donc
	\begin{equation}
		\langle f, e_n\rangle =\overline{ \langle f, e_{-n}\rangle  },
	\end{equation}
	ce qui donne
	\begin{equation}
		f=\sum_{n\in\eZ}\langle f, e_n\rangle e_n
		=\sum_{n>0}\langle f, e_n\rangle e_n +\sum_{n<0}\overline{ \langle f, e_n\rangle e_n }+\langle f, e_0 \rangle e_0
		=\sum_{n\in \eN}\Re\big( \langle f, e_n\rangle e_n \big).
	\end{equation}
	Notez que \( f\) étant supposée réelle et \( e_0\) étant la fonction constante (réelle) \( 1/\sqrt{ 2\pi }\), le terme \( n=0\) est bien réel.

	Or
	\begin{equation}        \label{EQooMWJNooSjPCpR}
		\Re\big( \langle f, e_n\rangle e_n \big)=\frac{1}{ (2\pi)^{3/2} }\cos(nx)\int_0^{2\pi}f(t)\cos(nt)dt-\frac{1}{ (2\pi)^{3/2} }\sin(nx)\int_0^{2\pi}f(t)\sin(nt)dt.
	\end{equation}

	Considérons la fonction impaire \( \tilde f\in L^2\big( [-2\pi,2\pi] \big)\) créée à partir de \( f\). Elle se développe de même et nous avons la même formule \eqref{EQooMWJNooSjPCpR} à part quelques coefficients et le fait que les intégrales sont entre \( -2\pi\) et \( 2\pi\). Vu que \( \tilde f\) est impaire, l'intégrale avec \( \cos(nt)\) s'annule et
	\begin{equation}
		\tilde f(x)=\sum_{n\in \eN}c_n\sin(nx)
	\end{equation}
	pour certains coefficients réels \( c_n\). Cette égalité est à considérer dans \( L^2\), c'est-à-dire presque partout et en particulier presque partout sur \( \mathopen[ 0 , 2\pi \mathclose]\).

	Donc les fonctions réelles sur \( \mathopen[ 0 , 2\pi \mathclose]\) peuvent être écrites sous la forme d'une série de seulement des sinus.

	Note : en choisissant \( \tilde f\) paire, nous aurions eu une série de cosinus.
\end{example}




\begin{corollary}[Unicité des coefficients de Fourier\cite{MonCerveau}]   \label{CordgtXlC}
	Soient \( f,g\) deux fonctions continues et \( 2\pi\)-périodiques.
	\begin{enumerate}
		\item       \label{ITEMooPLTIooSDykYF}
		      Si \( c_n(f)=c_n(g)\) alors \( f=g\).
		\item       \label{ITEMooQMMSooEpIFbt}
		      Si \( f(x)=\sum_{n\in \eZ}a_n e^{inx}\), alors \( a_n=c_n(f)\).
	\end{enumerate}
\end{corollary}

\begin{proof}
	En deux points.
	\begin{subproof}
		\spitem[Pour \ref{ITEMooPLTIooSDykYF}]
		Dans le cas de fonctions continues, le théorème de Fejér \ref{ThoJFqczow} nous enseigne que si nous posons
		\begin{equation}
			S_n(f)(x)=\sum_{k=-n}^{n}c_k(f) e^{ikx}
		\end{equation}
		alors nous avons la convergence
		\begin{equation}
			\frac{1}{ N+1 }\sum_{n=0}^NS_n(f)(x)\to f(x).
		\end{equation}
		Donc en supposant que \( c_k(f)=c_k(g)\), nous avons \( S_n(f)(x)=S_n(g)(x)\) et
		\begin{equation}
			f(x)=\lim_{N\to \infty} \frac{1}{ N+1 }\sum_{n=0}^NS_n(f)(x)=\lim_{N\to \infty} \frac{1}{ N+1 }\sum_{n=0}^NS_n(g)(x)=g(x).
		\end{equation}
		\spitem[Pour \ref{ITEMooQMMSooEpIFbt}]
		Nous considérons la restriction
		\begin{equation}
			\begin{aligned}
				\tilde f\colon \mathopen[ 0 , 2\pi \mathclose] & \to \eC       \\
				x                                              & \mapsto f(x).
			\end{aligned}
		\end{equation}
		C'est une fonction bornée parce qu'elle est la restriction de \( f\) qui est continue sur, disons, le compact \( \mathopen[ -\delta , 2\pi+\delta \mathclose]\). Elle est donc dans l'espace de Hilbert \( L^2\big( \mathopen[ 0 , 2\pi \mathclose[ \big)\).

		En utilisant la base trigonométrique \eqref{EQooKMYOooLZCNap} (qui est une base par le lemme \ref{LEMooBJDQooLVPczR}), nous écrivons l'hypothèse sous la forme
		\begin{equation}
			\tilde f(x)=\sum_{n\in \eZ}\sqrt{ 2\pi }a_ne_n(x).
		\end{equation}
		Autrement dit, \( \tilde f=\sum_{n\in \eZ}\sqrt{ 2\pi }a_ne_n\). La proposition \ref{PROPooWTOZooYZdlml} permet d'identifier les coefficients :
		\begin{equation}
			\sqrt{ 2\pi }a_n=\langle \tilde f, e_n\rangle .
		\end{equation}
		Nous avons donc
		\begin{subequations}
			\begin{align}
				a_n & =\frac{1}{ \sqrt{ 2\pi } }\int_0^{2\pi}\tilde f(t)\overline{ e_n(t) }dt                \\
				    & =\frac{1}{ \sqrt{ 2\pi } }\int_0^{2\pi}\tilde f(t)\frac{1}{ \sqrt{ 2\pi } } e^{-int}dt \\
				    & =\frac{1}{ 2\pi }\int_0^{2\pi}\tilde f(t) e^{-int}dt                                   \\
				    & =\frac{1}{ 2\pi }\int_0^{2\pi}f(t) e^{-int}dt                                          \\
				    & =c_n(f),
			\end{align}
		\end{subequations}
		ce qu'il fallait démontrer.
	\end{subproof}
\end{proof}

\begin{normaltext}
	La proposition \ref{PropREkHdol} dit que les hypothèses de continuité et de périodicité ne sont pas suffisantes pour assurer la convergence de la série de Fourier. En particulier, pour \ref{CordgtXlC}\ref{ITEMooQMMSooEpIFbt}, l'hypothèse de la convergence de la série est une vraie hypothèse.
\end{normaltext}

\begin{example}
	Considérons la fonction
	\begin{equation}
		f(x)=1-\frac{ x^2 }{ \pi^2 }
	\end{equation}
	sur \( \mathopen[ -\pi , \pi \mathclose]\). Nous la développons en série trigonométrique, et étant paire il n'y a pas de sinus. Un calcul montre que
	\begin{equation}
		a_0=\frac{ 4 }{ 3 }
	\end{equation}
	et
	\begin{equation}
		a_n=(-1)^{n+1}\frac{ 4 }{ n^2\pi^2 },
	\end{equation}
	de telle sorte que
	\begin{equation}
		f(x)=\frac{ 2 }{ 3 }-\frac{ 4 }{ \pi^2 }\sum_{n=1}^{\infty}(-1)^n\frac{ \cos(nx) }{ n^2 }.
	\end{equation}
	Nous avons \( f(\pi)=0\), mais avec le développement,
	\begin{equation}
		f(\pi)=\frac{ 2 }{ 3 }-\frac{ 4 }{ \pi^2 }\sum_{n=1}^{\infty}\frac{1}{ n^2 },
	\end{equation}
	donc
	\begin{equation}
		\sum_{n=1}^{\infty}\frac{1}{ n^2 }=\frac{ \pi^2 }{ 6 }.
	\end{equation}
\end{example}


%---------------------------------------------------------------------------------------------------------------------------
\subsection{À propos des coefficients}
%---------------------------------------------------------------------------------------------------------------------------

Pour la suite, nous avons besoin d'une notation pour désigner l'ensemble des suites dans \( \eC\) à index dans \( \eZ\), c'est-à-dire l'ensemble \( \Fun(\eZ,\eC)\). Pour alléger les notations, nous allons l'écrire \( \eC^{\eZ}\), conformément à des notations déjà introduites par exemple en \ref{DEFooLCJEooBvVmkV}.

Nous considérons l'application
\begin{equation}
	\begin{aligned}
		c\colon \big( L^1_{2\pi},\| . \|_1 \big) & \to \big( \eC^{\eZ},\| .\|_{\infty} \big) \\
		f                                        & \mapsto (c_n(f))_{n\in \eZ}
	\end{aligned}
\end{equation}
qui à une fonction \( 2\pi\)-périodique fait correspondre la suite (bornée) de ses coefficients de Fourier. Nous rappelons la définition
\begin{equation}
	c_n(f)=\frac{1}{ 2\pi }\int_0^{2\pi}f(t) e^{-int} dt.
\end{equation}
Nous allons montrer que cette application est linéaire, continue, injective et non surjective. Pour la continuité, par la linéarité il suffit de la montrer en \( 0\). Nous devons donc montrer que si nous avons une suite de fonctions \( f_k\) qui tend vers \( 0\) au sens \( L^1\), alors \( c(f_k)\to 0\) au sens de la norme \( \| . \|_{\infty}\) sur l'ensemble des suites.

Si nous posons \( r_k=\int_0^{2\pi}| f_k(t) |dt\), alors \( r_k=\| f_k \|_1\) et nous avons \( r_k\to 0\). Mais par définition
\begin{equation}
	| c_n(f_k) |\leq r_k,
\end{equation}
et donc \( \| c(f_k) \|_{\infty}\leq r_k\). L'application \( c\) est donc continue. L'injectivité est donnée par le corolaire~\ref{CordgtXlC}.

Si nous supposons que l'application \( c\) est continue, alors le théorème d'isomorphisme de Banach (\ref{ThofQShsw}) nous dit que cela devrait être un homéomorphisme, c'est-à-dire que \( c^{-1}\) serait également continue. Nous allons montrer qu'il n'en est rien.

Nous considérons la suite de suite
\begin{equation}    \label{EqdMtbOB}
	(c_n)_k=\begin{cases}
		0 & \text{si } n<0 \\
		1 & \text{si } k<n \\
		0 & \text{sinon}.
	\end{cases}
\end{equation}
Ici \( (c_n)_k\) est le terme numéro \( k\) de la suite \( (c_n)\). Par exemple \( c_0=(0,0,\ldots )\) et \( c_2=(1,1,0,\ldots)\).

Par injectivité de l'application qui à une fonction fait correspondre la suite de ses coefficients de Fourier, l'unique fonction qui possède ces coefficients est
\begin{equation}
	f_n(t)=\sum_{k\in \eN}c_{n,k} e^{ikt}.
\end{equation}
En ce qui concerne la norme de \( f_n\), nous avons
\begin{equation}
	\| f_n \|_1=\frac{1}{ 2\pi }\int_0^{2\pi}\sum_{k\in \eN}(c_n)_k|  e^{ikt} |dt=\sum_{k\in \eN}(c_n)_k=n.
\end{equation}
Étant donné que \( \| f_n \|_1=n\), la suite \( (\| f_n \|_1)\) n'est pas bornée alors que la suite de suites \eqref{EqdMtbOB} est bornée dans l'ensemble des suites parce que \( \| c_n \|_{\infty}=1\).

\begin{lemma}       \label{LEMooPUJDooKRBTaU}
	Soit une fonction \( f\colon \eR\to \eC\) qui est \( T\)-périodique et de classe \( C^1\). Alors
	\begin{equation}
		c_n(f')=\frac{ 2\pi n }{ T }ic_n(f).
	\end{equation}
\end{lemma}

\begin{proof}
	Nous rappelons la définition \eqref{EQooBOFSooFCJXzu} des coefficients de Fourier :
	\begin{equation}
		c_n(f)=\frac{1}{ T }\int_0^Tf(t) e^{-2 i \pi n t/T}dt.
	\end{equation}
	Le coefficient pour \( f'\) ne pose pas de problème d'existence parce que \( f'\) est continue sur le compact \( \mathopen[ 0 , T \mathclose]\). Il vaut
	\begin{subequations}
		\begin{align}
			c_n(f') & =\frac{1}{ T }\int_0^Tf'(t) e^{-2 i \pi n t/T}dt                                                                                                                   \\
			        & =\frac{1}{ T }\left[ f(t) e^{-2i\pi nt/T} \right]_0^T-\frac{1}{ T }\int_0^Tf(t)\left( \frac{ -2i\pi n }{ T } \right) e^{-2i\pi nt/T}dt \label{SUBEQooXYOVooGmoXbZ} \\
			        & =\frac{ 2i\pi n }{ T }\frac{1}{ T }\int_0^Tf(t) e^{-2i\pi nt/T}dt \label{SUBEQooXSCEooIJXFxT}                                                                      \\
			        & =\frac{ 2i\pi n }{ T }c_n(f).
		\end{align}
	\end{subequations}
	Justifications.
	\begin{itemize}
		\item Pour \eqref{SUBEQooXYOVooGmoXbZ}. C'est une intégration par partie avec \( u'=f'\) et \( v= e^{-2i\pi nt/T}\).
		\item Pour \eqref{SUBEQooXSCEooIJXFxT}. Comme \( f(T)=f(0)\), et que \( t\mapsto e^{-2i\pi nt/T}\) est périodique de période \( T\), le terme au bord est nul : \( f(T) e^{-2i\pi n}-f(0) e^{i0}=0\).
	\end{itemize}
\end{proof}

\begin{lemma}[\cite{BIBooUBUAooHyhrlg}]     \label{LEMooYJQWooDVvSyj}
	Soit une fonction \( f\colon \eR\to \eC\) de classe \( C^2\) et \( T\)-périodique. Alors
	\begin{equation}
		| c_n(f) |\leq \left( \frac{ T }{ 2\pi } \right)^2 \frac{ \| f'' \|_{\infty} }{ n^2 }.
	\end{equation}
\end{lemma}

\begin{proof}
	En utilisant la définition \eqref{EQooBOFSooFCJXzu} des coefficients de Fourier,
	\begin{equation}
		| c_n(f) |\leq \frac{1}{ T }\int_0^T| f(t) |dt\leq \frac{ 1 }{ T }\| f \|_{\infty}\int_0^T1dt=\| f \|_{\infty}.
	\end{equation}
	En appliquant le lemme \ref{LEMooPUJDooKRBTaU} à \( f'\) nous avons
	\begin{equation}
		c_n(f'')=\left( \frac{ 2i\pi n }{ T } \right)^2c_n(f).
	\end{equation}
	Donc
	\begin{equation}
		| c_n(f) |=\left( \frac{ T }{ 2\pi n } \right)^2| c_n(f'') |\leq \left( \frac{ T }{ 2\pi n } \right)^2\| f'' \|_{\infty}.
	\end{equation}
\end{proof}


%---------------------------------------------------------------------------------------------------------------------------
\subsection{Le contre-exemple que nous attendions tous}
%---------------------------------------------------------------------------------------------------------------------------

Nous montrons maintenant que la continuité et la périodicité ne sont pas suffisantes pour avoir convergence de la série de Fourier.

\begin{proposition}[\cite{KXjFWKA}] \label{PropREkHdol}
	Soit \( C^0_{2\pi}(\eR)\) l'ensemble des fonctions périodiques continues muni de la norme uniforme. Nous définissons
	\begin{equation}
		S_n(f)(x)=\sum_{k=-n}^nc_k(f) e^{ikx}.
	\end{equation}
	Alors il existe \( f\in C^0_{2\pi}\) tel que la suite \(n\mapsto S_n(f)(0)\) soit divergente. En particulier \( f\) n'est pas la somme de sa série de Fourier.
\end{proposition}

\begin{proof}
	Nous considérons la forme linéaire
	\begin{equation}
		\begin{aligned}
			l_n\colon C^0_{2\pi} & \to \eC                                \\
			f                    & \mapsto S_n(f)(0)=\sum_{k=-n}^nc_k(f).
		\end{aligned}
	\end{equation}
	\begin{subproof}
		\spitem[La forme est continue]
		Nous montrons d'abord que \(  l_n \) est continue en montrant que \( \| l_n \|<\infty\) et en utilisant la proposition~\ref{PROPooQZYVooYJVlBd}. Pour cela nous calculons un peu :
		\begin{equation}    \label{EqBELHGya}
			l_n(f)=\sum_{k=-n}^n\frac{1}{ 2\pi }\int_{-\pi}^{\pi}f(t) e^{-ikt}dt=\frac{1}{ 2\pi }\int_{-\pi}^{\pi}f(t)\sum_{k=-n}^n e^{-ikt}dt=\frac{1}{ 2\pi }\int_{-\pi}^{\pi}f(t)D_n(t)dt
		\end{equation}
		où \( D_n(t)\) est le noyau de Dirichlet dont nous connaissons une formule par le lemme~\ref{LemHPoIkwu}. Nous avons donc
		\begin{equation}
			| l_n(f) |\leq \frac{1}{ 2\pi }\int_{-\pi}^{\pi}| D_n(t) |\| f \|_{\infty}dt.
		\end{equation}
		En prenant \( \| f \|_{\infty}=1\) nous avons la borne suivante pour la norme de \( l_n\) :
		\begin{equation}        \label{EqBXoIUiD}
			\| l_n \|\leq \frac{1}{ 2\pi }\int_{-\pi}^{\pi}| D_n(t) |dt<\infty.
		\end{equation}
		Notons que la convergence de l'intégrale vient de la continuité de la fonction
		\begin{equation}
			t\mapsto \frac{ \sin\left( \frac{ 2n+1 }{2}t \right) }{ \sin\left( \frac{ t }{ 2 } \right) }
		\end{equation}
		qui, elle même, se prouve avec une règle de l'Hospital :
		\begin{equation}
			\lim_{t\to 0} \frac{ \sin(at) }{ \sin(t) }=\lim_{t\to 0} \frac{ a\cos(at) }{ \cos(t) }=a.
		\end{equation}
		Donc \( D_n(t)\) a une limite bien définie pour \( t\to 0\) et est alors une fonction continue sur le compact \( \mathopen[ -\pi , \pi \mathclose]\).

		\spitem[La norme de \( l_n\) (début)]

		Nous avons prouvé que \( \| l_n \|\leq \frac{1}{ 2\pi }\int_{-\pi}^{\pi}| D_n(t) |dt\). Nous allons à présent prouver que ceci est effectivement la norme de \( l_n\). Pour \( \epsilon>0\) nous considérons la fonction
		\begin{equation}
			\begin{aligned}
				f_{\epsilon}\colon \eR & \to \eC                                         \\
				x                      & \mapsto \frac{ D_n(x) }{ | D_n(x) |+\epsilon }.
			\end{aligned}
		\end{equation}
		C'est une fonction continue et \( 2\pi\)-périodique satisfaisant \( \| f_{\epsilon} \|\leq 1\) parce que le dénominateur est toujours plus grand que le numérateur. Nous nous proposons de calculer
		\begin{equation}
			l_n(f_{\epsilon})=\sum_{k=-n}^n\frac{1}{ 2\pi }\int_{-\pi}^{\pi}f_{\epsilon}(t) e^{-ikt}dt.
		\end{equation}
		Puisque \( f_{\epsilon}(t) e^{-ikt}\) vaut en norme \( | f_{\epsilon}(t) |\), qui est une fonction intégrable (ne dépendant pas de \( k\)) sur \( \mathopen[ -\pi , \pi \mathclose]\), le théorème de la convergence dominée~\ref{ThoConvDomLebVdhsTf} nous permet de permuter la somme et l'intégrale :
		\begin{equation}
			l_n(f_{\epsilon})=\frac{1}{ 2\pi }\int_{-\pi}^{\pi}\frac{ D_n(t) }{ | D_n(t) |+\epsilon }\underbrace{\sum_{k=-n}^n e^{-ikt}}_{=D_n(t)}dt=\frac{1}{ 2\pi }\int_{-\pi}^{\pi}\frac{ \big| D_n(t) \big|^2 }{ | D_n(t) |+\epsilon }dt.
		\end{equation}
		Nous avons donc
		\begin{equation}
			\lim_{\epsilon\to 0}l_n(f_{\epsilon})=\frac{1}{ 2\pi }\int_{-\pi}^{\pi}| D_n(t) |dt.
		\end{equation}
		Mais vue l'inégalité \eqref{EqBXoIUiD} nous avons
		\begin{equation}
			\| l_n \|=\frac{1}{ 2\pi }\int_{-\pi}^{\pi}| D_n(t) |dt.
		\end{equation}
		Notre tâche est maintenant de donner une valeur à cette intégrale.

		\spitem[Norme de \( l_n\) tend vers \( \infty\)]
		D'abord nous écrivons
		\begin{equation}
			\| l_n \|=\frac{1}{ 2\pi }\int_{-\pi}^{\pi}\frac{ \left| \sin\left( \frac{ 2n+1 }{2}t \right) \right|  }{ \big| \sin(t/2) \big| }dt,
		\end{equation}
		ensuite nous nous souvenons que \( | \sin(x) |\leq | x |\) pour tout \( x\), ce qui nous permet de changer le dénominateur :
		\begin{equation}
			\| l_n \|\geq \frac{ 2 }{ \pi }\int_0^{\pi}\frac{ \left| \sin\left( \frac{ 2n+1 }{2}t \right) \right|  }{ | t | }dt
		\end{equation}
		Nous y effectuons le changement de variable \( u=\frac{ 2n+1 }{2}t\) qui donne
		\begin{equation}
			\| l_n \|\geq \frac{ 2 }{ \pi }\int_{0}^{(n+\frac{ 1 }{2})\pi}\frac{ \big| \sin(u) \big| }{ | u | }du.
		\end{equation}
		Nous y reconnaissons l'intégrale \eqref{EqKNOmLEd} du sinus cardinal que nous savons diverger. Cela donne
		\begin{equation}
			\lim_{n\to \infty} \| l_n \|=\infty.
		\end{equation}
		\spitem[La conclusion]

		L'espace \( \big( C^0_{2\pi},\| . \|_{\infty} \big)\) est complet\footnote{Parce qu'une limite uniforme de fonctions continues est continue, théorème~\ref{ThoUnigCvCont}.}, donc le théorème de Banach-Steinhaus~\ref{ThoPFBMHBN} s'applique. Par rapport aux notations de l'énoncé de Banch-Steinhaus, nous posons
		\begin{subequations}
			\begin{align}
				E & =\big( C^0_{2\pi},\| . \|_{\infty} \big) \\
				F & =\eR                                     \\
				H & =\{ l_n \}_{n\in \eN}.
			\end{align}
		\end{subequations}
		Comme la suite \( (\| l_n \|)\) n'est pas bornée, il existe \( f\in C^0_{2\pi}\) tel que
		\begin{equation}
			\sup_n\| l_n(f) \|=\infty.
		\end{equation}
		Pour cette fonction nous avons
		\begin{equation}
			\sup_{n\geq 0}S_n(f)(0)=\infty,
		\end{equation}
		et donc la série de Fourier de \( f\) ne converge pas en zéro.

	\end{subproof}
\end{proof}

\begin{normaltext}      \label{NORMooGKKWooFmOBeE}
	La proposition \ref{PropREkHdol} ne contredit pas Fejér \ref{ThoJFqczow}\ref{ItemUNQSPmyiv}. Alors là c'est subtil, donc soyez bien \randomGender{attentif}{attentive}.

	Le système trigonométrique est total dans l'espace des fonctions continues périodiques sur \( \eR\) : \( \big(  C^0_{2\pi}(\eR),\| . \|_{\infty}\big)\). Cela signifie que si \( f\in C^0_{2\pi}(\eR)\), il  existe une suite de polynômes trigonométriques \( P_k\) tels que \( P_k\stackrel{unif}{\longrightarrow}f\).

	Cela ne signifie pas que cette suite soit la suite des sommes partielles de la série de Fourier. Et en effet, le théorème de Fejér ne donne pas la convergence de la suite des sommes partielles de Fourier, mais la convergence au sens de Cesàro de la somme des \( c_k(f)e_k\). Ce n'est pas la même chose.

	Notez que le coefficient de \( e_1\) dans \( F_2\) est \( c_1(f)/2\) alors que dans \( F_3\), il est \( 2c_1(f)/3\).

	Il y a donc bien une suite de polynômes trigonométriques qui converge vers \( f\), mais ce n'est pas la suite des sommes partielles de la série de Fourier.

	De plus, \( C^0_{2\pi}(\eR)\) n'est pas contenu dans \( L^2(\eR)\), donc nous ne pouvons pas invoquer la théorie de Hilbert pour dire que le système trigonométrique serait quelque chose comme une base. Si \( f\in C^0_{2\pi}(\eR)\), il n'est pas garanti qu'il existe des nombres \( (a_i)_{i\in \eZ}\) tels que \( f(x)=\sum_{n\in \eZ}a_n e^{inx}\).

	Enfin, me diriez-vous, les fonctions continues et périodiques sur \( \eR\) sont les mêmes que les fonctions définies sur \( \mathopen[ 0 , 2\pi \mathclose[\) avec une petite condition de \( \lim_{x\to 2\pi} f(x)=f(0)\). Les fonctions qui vérifient cela sont une partie de \( L^2\big( \mathopen[ 0 , 2\pi \mathclose[ \big)\), qui est un espace de Hilbert, lui. Or le système trigonométrique est une base hilbertienne (lemme \ref{LEMooBJDQooLVPczR}). Alors oui, la série de Fourier de \( f\) converge vers \( f\) lorsque \( f\) est continue sur \( \mathopen[ 0 , 2\pi \mathclose[\). Cela n'est cependant pas un contre-argument pour deux raisons :
	\begin{itemize}
		\item
		      La convergence \( S_n(f)\to f\) qu'on a dans \( L^2\big( \mathopen[ 0 , 2\pi \mathclose[ \big)\) est seulement une convergence pour la norme \( L^2\), et non une convergence uniforme.
		\item
		      Une convergence \( L^2\) sur \( \mathopen[ 0 , 2\pi \mathclose[\) ne se prolonge pas spécialement en une convergence \( L^2\) sur \( \eR\), et encore moins en une convergence uniforme sur \( \eR\).
	\end{itemize}
\end{normaltext}



%+++++++++++++++++++++++++++++++++++++++++++++++++++++++++++++++++++++++++++++++++++++++++++++++++++++++++++++++++++++++++++
\section{Série de Laurent}
%+++++++++++++++++++++++++++++++++++++++++++++++++++++++++++++++++++++++++++++++++++++++++++++++++++++++++++++++++++++++++++

\begin{theorem}[Série de Laurent\cite{BIBooUBUAooHyhrlg}]       \label{THOooMKJOooVghZyG}
	Soient la couronne
	\begin{equation}
		C(r_1,r_2)=\{ z\in \eC\tq r_1<| z |<r_2 \}
	\end{equation}
	et une fonction holomorphe \( f\colon C(r_1,r_2)\to \eC\). Alors :
	\begin{enumerate}
		\item
		      Il existe une suite \( (a_n) \) dans \( \eC\) telle que
		      \begin{equation}
			      f(z)=\sum_{n\in \eZ}a_nz^n.
		      \end{equation}
		\item       \label{ITEMooUOPHooSJRGKs}
		      Cette suite est unique : si
		      \begin{equation}
			      f(z)=\sum_{n\in \eZ}a_nz^n=\sum_{n\in \eZ}b_nz^n,
		      \end{equation}
		      alors \( a_n=b_n\), pour tout \( n\).
		\item       \label{ITEMooDGGZooJkDSxC}
		      Si on pose, pour \( r\in \mathopen] r_1 , r_2 \mathclose[\),
		      \begin{equation}
			      \begin{aligned}
				      f_r\colon \eR & \to \eC                   \\
				      \theta        & \mapsto f(r e^{i\theta}),
			      \end{aligned}
		      \end{equation}
		      les valeurs \( a_n\) sont liés aux coefficients de Fourier de \( f_r\) par
		      \begin{equation}
			      a_n=\frac{ c_n(f_r) }{ r^n }.
		      \end{equation}
		\item       \label{ITEMooOYCPooZZAyKs}
		      Cette série converge uniformément sur tout compact contenu dans \( C(r_1,r_2)\).
		\item
		      Pour tout \( r_1<s<r_2\), les coefficients sont donnés par\footnote{Pour le dire clairement, ces \( a_n\) ne dépendent pas de \( s\), même si \( s\) entre dans le membre de droite.}
		      \begin{equation}
			      a_n=\frac{1}{ 2\pi i }\int_{C_s}\frac{ f(z) }{ z^{n+1} }dz
		      \end{equation}
		      où \( C_s\) est un cercle centré en \( 0\), et de rayon \( s\).
	\end{enumerate}
	La série ainsi définie est la \defe{série de Laurent}{série de Laurent} de la fonction \( f\).
\end{theorem}

\begin{proof}
	Pour \( r\in\mathopen] r_1 , r_2 \mathclose[\) nous posons
	\begin{equation}
		\begin{aligned}
			f_r\colon \eR & \to \eC                   \\
			\theta        & \mapsto f(r e^{i\theta}).
		\end{aligned}
	\end{equation}
	\begin{subproof}
		\spitem[Coefficients de Fourier]
		La fonction \( f_r\) est de classe \( C^1\) et périodique. Le théorème \ref{ThozHXraQ} sur les séries de Fourier nous indique que
		\begin{equation}      \label{EQooIHQRooZWqJKL}
			f_r(t)=\sum_{n\in \eZ}c_n(r) e^{int}
		\end{equation}
		avec\footnote{Oui, on devrait écrire \( c_n(f_r)\) pour suivre scrupuleusement les notation introduites plus haut. Mais comme toute la suite de la démonstration sera de voir le tout comme fonction de \( r\), je vous laisse juger.}
		\begin{equation}
			c_n(r)=\frac{1}{ 2\pi }\int_0^{2\pi}f_r(t) e^{-int}dt.
		\end{equation}
		La fonction \( c_n\colon \mathopen] r_1 , r_2 \mathclose[\to \eC\) est une fonction définie par une intégrale que nous voudrions dériver en \( r_0\).
			\spitem[Digression]
			Deux voies s'offrent à nous.
			\begin{itemize}
				\item Le plus immédiatement disponible est le théorème \ref{ThoMWpRKYp}, mais il demande de travailler avec la dérivée (réelle) de \( r\mapsto f(r e^{i\theta})\) et de se poser des questions quant à son lien avec la dérivée (complexe) de \( f\).
				\item Une façon plus indirecte est de considérer une extension
				      \begin{equation}
					      \begin{aligned}
						      h\colon B(r_0,\delta)_{\eC}\times \mathopen[ 0 , 2\pi \mathclose[ & \to \eC                                \\
						      z,\theta                                                          & \mapsto f(z e^{i\theta}) e^{-in\theta}
					      \end{aligned}
				      \end{equation}
				      où \( \delta\) est assez petit pour que le tout reste dans le domaine de \( f\). Alors nous pouvons utiliser le théorème \ref{ThopCLOVN} qui a l'avantage de ne pas devoir majorer la dérivée. Mais cette voie demande de réellement faire le lien entre la dérivée complexe de \( h\) et la dérivée réelle de \( r\mapsto f(r e^{i\theta})\).
			\end{itemize}
			Nous choisissons la première voie parce qu'en réalité, elle évite complètement de parler de dérivée complexe.

			\spitem[Permuter dérivée et intégrale]
			Nous allons essayer de la dériver en \( r_0\in \mathopen] r_1 , r_2 \mathclose[\) en utilisant le théorème \ref{ThoMWpRKYp}. Pour y voir plus clair, ce qui joue le rôle de \( f\) dans l'énoncé de \ref{ThoMWpRKYp} est l'application
			\begin{equation}
				\begin{aligned}
					h\colon \mathopen] r_1 , r_2 \mathclose[\times \mathopen[ 0 , 2\pi \mathclose[ & \to \eC                             \\
					(r,\theta)                                                                     & \mapsto  f_r(\theta) e^{-in\theta}.
				\end{aligned}
			\end{equation}
			Nous considérons un intervalle \( I=B(r_0,\delta)\) assez petit pour être dans \( \mathopen] r_1 , r_2 \mathclose[\). Passons en revue les conditions.
			\begin{subproof}
				\spitem[Pour \ref{ITEMooAFVMooAeCEco}]

				Nous introduisons la fonction
				\begin{equation}
					\begin{aligned}
						\tilde  f\colon \eR\times \mathopen[ 0,2\pi\mathclose] & \to \eC                 \\
						(r,\theta)                                             & \mapsto f(re^{i\theta})
					\end{aligned}
				\end{equation}

				Pour tout \( r\in I\), la fonction \( \theta\mapsto f(r e^{i\theta}) e^{-in\theta}\) est dans \( L^1\big( \mathopen[ 0 , 2\pi \mathclose[ \big)\) parce que
				\begin{equation}
					| f(r e^{i\theta}) e^{-in\theta} |=| f(r e^{i\theta}) |=| \tilde f(r,\theta) |.
				\end{equation}
				La fonction \( \tilde f\) étant continue, elle est bornée sur le compact \( \overline{ I }\times \mathopen[ 0 , 2\pi \mathclose]\).
				\spitem[Pour \ref{ITEMooXIZXooGPYFyT}]
				Pour chaque \( \theta\), la fonction \( r\mapsto f(r e^{i\theta}) e^{-in\theta}\) est dérivable par la proposition \ref{PROPooAGGMooIVQFQB}\ref{ITEMooRTYYooSTgTAQ}.
				\spitem[Pour \ref{ITEMooDTTIooWkldfB}]
				En utilisant la proposition \ref{PROPooAGGMooIVQFQB}\ref{ITEMooUUXTooZoDMHI}, nous savons que la fonction
				\begin{equation}
					\frac{ \partial h }{ \partial r  }(r,\theta)=\frac{ \partial \tilde f }{ \partial r }(r,\theta) e^{-in\theta}
				\end{equation}
				est continue et donc bornée sur le compact \( \overline{ I }\times \mathopen[ 0 , 2\pi \mathclose]\). Une fonction constante majorant de \( \partial_rh\) est intégrable sur le compact \( \mathopen[ 0 , 2\pi \mathclose]\).
			\end{subproof}
			En permutant nous avons donc
			\begin{equation}
				c'_n(r_0)=\frac{1}{ 2\pi }\int_0^{2\pi}(\partial r\tilde f)(r_0,\theta) e^{-in\theta}d\theta.
			\end{equation}
			\spitem[Cauchy-Riemann]
			C'est le moment d'utiliser Cauchy-Riemann en coordonnées polaires sous la forme de la proposition \ref{PROPooAGGMooIVQFQB}\ref{ITEMooDHXTooBjxwjY}. Et tant que nous y sommes, nous notons \( g(\theta)=\tilde f(r_0,\theta)\) pour avoir moins de choses à écrire :
			\begin{subequations}
				\begin{align}
					c'_n(r_0) & =\int_0^{2\pi}(\partial_r\tilde f)(r_0,\theta) e^{-in\theta}d\theta                                        \\
					          & =\frac{1}{ 2\pi }\int_0^{2\pi}\frac{1}{ ir_0 }(\partial_{\theta}\tilde f)(r_0,\theta) e^{-in\theta}d\theta \\
					          & =\frac{1}{ 2\pi ir_0 }\int_{0}^{2\pi}g'(\theta) e^{-in\theta}d\theta.
				\end{align}
			\end{subequations}
			La dernière expression a manifestement envie de se soumettre à une intégration par partie.
			\spitem[Une intégration par partie]
			Nous posons \( u(\theta)= e^{-in\theta}\) et \( v=g\), de telle sorte que
			\begin{subequations}        \label{EQSooSRDJooXLHhgh}
				\begin{align}
					c'_n(r_0) & =\frac{1}{ 2\pi i r_0 }\left( \big[ e^{-in\theta}g(\theta)\big]_0^{2\pi}-\int_0^{2\pi}(-in) e^{-in\theta}g(\theta)d\theta \right) \\
					          & =\frac{n}{ 2\pi r_0 }\int_0^{2\pi}g(\theta) e^{-in\theta}d\theta.           \label{SUBEQooKVZMooUxcRXn}                           \\
					          & =\frac{n}{ r_0 }c_n(r_0).
				\end{align}
			\end{subequations}
			Justification pour \eqref{SUBEQooKVZMooUxcRXn}. Pour rappel, \( g(\theta)=\tilde f(r_0,\theta)=f(r_0 e^{i\theta})\); donc \( g(0)=g(2\pi)\) et le «terme au bord» est nul.
			\spitem[Équation différentielle]
			L'équation \eqref{EQSooSRDJooXLHhgh} dit que \( c_n\) satisfait à l'équation différentielle
			\begin{equation}
				c'_n(r)=\frac{ n }{ r }c_n(r)
			\end{equation}
			pour tout \( r\in \mathopen] r_1 , r_2 \mathclose[\). La fonction \( c_n\) est une fonction à valeurs complexes dont les parties réelles et imaginaires vérifient toutes deux l'équation du lemme \ref{LEMooCSAFooTYasYM}. Il existe donc \( a_n\in \eC\) tel que
			\begin{equation}
				c_n(r)=a_nr^n.
			\end{equation}
			Cela prouve au passage le point \ref{ITEMooDGGZooJkDSxC} parce que \( r\) n'est jamais nul.
			\spitem[La valeur de \( a_n\)]
			Nous avons
			\begin{equation}        \label{EQooFNUHooZbbNAT}
				\begin{aligned}
					a_n & =\frac{ c_n(r) }{ r^n }                                                               \\
					    & =\frac{1}{ 2\pi }\int_0^{2\pi}f(r e^{i\theta}) e^{-in\theta}r^{-n}d\theta             \\
					    & =\frac{1}{ 2\pi }  \int_0^{2\pi}\frac{ f(r e^{i\theta}) }{ ( e^{i\theta}r)^n }d\theta \\
					    & =\frac{1}{ 2\pi i }\int_{C_s}\frac{ f(z) }{ z^{n+1} }dz.
				\end{aligned}
			\end{equation}
			Une justification pour l'intégrale curviligne s'impose. La définition est \ref{DEFooBPLJooZwsmxi}. Dans le cas du cercle, nous considérons
			\begin{equation}
				\begin{aligned}
					C_r\colon \mathopen[ 0 , 2\pi \mathclose[ & \to \eC                \\
					\theta                                    & \mapsto r e^{i\theta},
				\end{aligned}
			\end{equation}
			et donc
			\begin{equation}
				\int_{C_r}\frac{ f(z) }{ z^{n+1} }dz=\int_0^{2\pi}\frac{ f(r e^{i\theta}) }{ (r e^{i\theta})^{n+1} }ri e^{i\theta}d\theta=i\int_0^{2\pi}\frac{ f(r e^{i\theta}) }{ (r e^{i\theta})^n }d\theta.
			\end{equation}
			Le fait que le tout soit égal à \( a_n\) prouve que l'intégrale est en réalité indépendante de \( r\)\quext{Voir ma question \ref{PROPBooYWDNooMXVPLJ}.}.
			\spitem[Conclusion]
			Reprenons la formule \eqref{EQooIHQRooZWqJKL} :
			\begin{equation}
				f(r e^{i\theta})=f_r(\theta)=\sum_nc_n(r) e^{in\theta}=\sum_na_nr^n e^{in\theta}=\sum_na_n(r e^{i\theta})^n.
			\end{equation}
			Autrement dit,
			\begin{equation}
				f(z)=\sum_na_nz^n,
			\end{equation}
			avec les \( a_n\) donnés par la formule \eqref{EQooFNUHooZbbNAT}, comme nous devions le prouver.
			\spitem[Point \ref{ITEMooUOPHooSJRGKs} (unicité)]
			Supposons que \( f(z)=\sum_{n\in \eZ}a_nz^n\), fixons \( r\in \mathopen] r_1 , r_2 \mathclose[\), et posons
			\begin{equation}
				\begin{aligned}
					f_r\colon \eR & \to \eC                   \\
					\theta        & \mapsto f(r e^{i\theta}).
				\end{aligned}
			\end{equation}
			C'est une fonction continue et périodique. Elle peut s'écrire sous la forme
			\begin{equation}
				f_r(\theta)=f(r e^{i\theta})=\sum_{n\in \eZ}a_n(r^n e^{in\theta})=\sum_{n\in \eZ}(a_nr^n) e^{in\theta}.
			\end{equation}
			Le corolaire \ref{CordgtXlC}\ref{ITEMooQMMSooEpIFbt} à propos de l'unicité des coefficients de Fourier implique que \( a_nr^n=c_n(f)\) et donc que
			\begin{equation}        \label{EQooBNSMooGLIBqU}
				a_n=\frac{ c_n(f) }{ r^n }.
			\end{equation}
			Donc les coefficients \( a_n\) sont déterminés par \( f\). Si nous avions fait le calcul en partant de \( f(z)=\sum_{n\in \eZ}b_nz^n\), nous aurions eu \( b_n=\frac{ c_n(f) }{ r^n }\), et donc bien \( a_n=b_n\).
			\spitem[Point \ref{ITEMooOYCPooZZAyKs}]
			La fonction
			\begin{equation}
				\begin{aligned}
					f_r\colon \eR & \to \eC                  \\
					\theta        & \mapsto f(r e^{i\theta})
				\end{aligned}
			\end{equation}
			est continue et périodique de période \( 2\pi\). Le lemme \ref{LEMooYJQWooDVvSyj} nous indique que, pour tout \( r\in \mathopen] r_1 , r_2 \mathclose[\), nous avons
		\begin{equation}        \label{EQooMDNNooPYFQrq}
			| c_n(f_r) |\leq \frac{ \| f_r'' \|_{\infty} }{ n^2 }.
		\end{equation}
		\begin{subproof}
			\spitem[Sur une couronne]
			Soient deux rayons intermédiaires \( r_1<s_1<s_2<r_2\). La couronne fermée \( \overline{ C(s_1,s_2) }\) est compacte. Nous considérons la fonction
			\begin{equation}
				\begin{aligned}
					g\colon \mathopen[ s_1 , s_2 \mathclose]\times \mathopen[ 0 , 2\pi \mathclose] & \to \eC               \\
					(u,\theta)                                                                     & \mapsto f''_u(\theta)
				\end{aligned}
			\end{equation}
			Nous notons \( g_u\) l'application \( \theta\mapsto g(u,\theta)\); c'est une application définie sur \( \mathopen[ s_1 , s_2 \mathclose]\). Par le lemme \ref{LEMooIVAKooUiEENr}, l'application \( u\mapsto \| g_u \|_{\infty}\) est continue et donc majorée : nous pouvons considérer \( M\in \eR\) tel que \( \| g_u \|_{\infty}<M\) pour tout \( u\in \mathopen[ s_1 , s_2 \mathclose]\).

			Puisque, pour \( r\in\mathopen[ s_1 , s_2 \mathclose]\) nous avons \( g_u=f_u''\), en combinant avec \eqref{EQooMDNNooPYFQrq}, nous voyons qu'il existe \( M\) tel que
			\begin{equation}        \label{EQooAIPEooKdfoXr}
				| c_n(f_r) |\leq \frac{ M }{ n^2 }.
			\end{equation}
			En prenant les notations de la définition \ref{DefVBrJUxo} de la convergence normale, nous posons
			\begin{equation}
				\begin{aligned}
					u_n\colon \overline{ C(s_1,s_2) } & \to \eC        \\
					z                                 & \mapsto a_nz^n
				\end{aligned}
			\end{equation}
			En ce qui concerne sa norme\footnote{Faites bien attention que dans cette partie, \( \| . \|_{\infty}\) est la norme uniforme sur \( \overline{ C(s_1,s_2) }\) et non sur \( C(r_1,r2)\) ou sur \( \eC\).}, nous avons
			\begin{subequations}
				\begin{align}
					\| u_n \|_{\infty} & =| a_n |\sup_{z\in\overline{ C(s_1,s_2) }}| z^n |                           \\
					                   & =| a_n |s_2^n                                                               \\
					                   & =\left| \frac{ c_n(f_s) }{ s^n } \right| s_2^n  \label{SUBEQooRKUXooFPxnLG} \\
					                   & \leq \frac{ | c_n(f_s) | }{ s_2^n } s_2^n                                   \\
					                   & \leq \frac{ M }{ n^2 }      \label{SUBEQooZJFJooAgRKHc}
				\end{align}
			\end{subequations}
			Justifications :
			\begin{itemize}
				\item Pour \eqref{SUBEQooRKUXooFPxnLG}. Prendre n'importe quel \( s\in \mathopen[ s_1 , s_2 \mathclose]\) et c'est bon par \eqref{EQooBNSMooGLIBqU}.
				\item Pour \eqref{SUBEQooZJFJooAgRKHc}. Équation \eqref{EQooAIPEooKdfoXr}.
			\end{itemize}
			Comme la somme \( \sum_{n\in \eZ}M/n^2\) converge, nous avons la convergence normale de \( \sum_{n\in \eZ}a_nz^n\) sur \( \overline{C(s_1,s_2)}\).

			\spitem[Sur un compact quelconque]
			Soit \( K\), un compact dans \( C(r_1,r_2)\). La fonction \( z\mapsto | z |\) est continue sur \( K\); donc elle a un minimum et un maximum. Nous posons \( s_1=\min_{z\in K}| z |\) et \( s_2=\max_{z\in K}| z |\).

			Ah ah ! non. En fait nous ne définissons pas \( s_1\) et \( s_2\) de cette manière parce qu'il y a un risque que \( s_1=s_2\) et qu'alors \( \overline{ C(s_1,s_2) }\) soit vide et ne contienne donc pas \( K\) -- pour rappel, \( C(s_1,s_2)\) est la couronne ouverte.

			Nous choisissons donc \( s_1\) et \( s_2\) de telle sorte que
			\begin{equation}
				r_1 <s_1 <\min_{z\in K}| z |\leq \max_{z\in K}| z | <s_2 <r_2.
			\end{equation}
			Tout ça pour dire que \( K\subset \overline{ C(s_1,s_2) }\). La convergence normale sur \( \overline{ C(s_1,s_2) }\) déjà prouvée implique la convergence normale sur \( K\).
		\end{subproof}
	\end{subproof}
\end{proof}

\begin{probleme}        \label{PROPBooYWDNooMXVPLJ}
	Sur Wikipédia\cite{BIBooUBUAooHyhrlg}, le fait que \( a_n\) ne dépende pas de \( r\) est prouvé en disant que \( a_n(s)=c_n(s)/s^n\) et \( a_n(t)=c_n(t)/t^n\) et que
	\begin{equation}        \label{EQooCPDTooBDxIKm}
		\frac{ c_n(s) }{ c_n(t) }=\frac{ s^n }{ t^n }.
	\end{equation}
	En mettant tout cela bout à bout,
	\begin{equation}
		\frac{ a_n(s) }{ a_n(t) }=1.
	\end{equation}
	Je ne comprends pas très bien pourquoi cette justification est nécessaire. À mon avis, \randomGender{le rédacteur}{la rédactrice} de la démonstration sur Wikipédia parvient à déduire la relation \eqref{EQooCPDTooBDxIKm} directement depuis l'équation différentielle, sans réellement avoir besoin de la résoudre.

	Si vous comprenez n'hésitez pas à m'écrire, parce que j'ai l'impression d'avoir manqué quelque chose.
\end{probleme}

\begin{probleme}
	L'énoncé de la proposition \ref{PROPooBMZGooLoaGLK} n'est peut-être pas précis. Si vous avez un énoncé correct sous le coude, écrivez-moi.
\end{probleme}

\begin{proposition}     \label{PROPooBMZGooLoaGLK}
	Si \( f\) est holomorphe sur \( B(a,r)\), alors sa série de Laurent est de la forme
	\begin{equation}
		f(z)=\sum_{n=0}^{\infty}a_n(z-a)^n.
	\end{equation}
	avec
	\begin{equation}
		a_n=\frac{1}{ 2\pi i }\int_{\gamma}\frac{ f(z) }{ (z-a)^{n+1} }dz.
	\end{equation}
	où
	\begin{equation}
		\begin{aligned}
			\gamma\colon \mathopen[ 0 , 2\pi \mathclose[ & \to \eC            \\
			t                                            & \mapsto a+ re^{it}
		\end{aligned}
	\end{equation}
	est le cercle de centre \( a\) et de rayon \( r\).
\end{proposition}

%---------------------------------------------------------------------------------------------------------------------------
\subsection{Inégalité isopérimétrique}
%---------------------------------------------------------------------------------------------------------------------------

Le théorème suivant dit que parmi les courbes \( C^1\), le cercle a la plus grande surface possible à périmètre donné.
\begin{theorem}[Inégalité isopérimétrique\cite{KXjFWKA}]    \label{ThoIXyctPo}
	Soit \( f\colon S^1\to \eC \) une courbe de Jordan\footnote{Définition \ref{DEFooQZMSooYYkGDv}} de classe \( C^1\). Nous notons \( L\) sa longueur et \( S\) l'aire contenue de la surface délimitée\footnote{C'est la partie connexe bornée de \( \eC\setminus\gamma\) dont l'existence est donnée par le théorème de Jordan~\ref{ThoHSPWBuh}.} par \( f\). Alors
	\begin{enumerate}
		\item
		      Nous avons l'\defe{inégalité isopérimétrique}{inégalité!isopérimétrique} : \( L^2\geq 4\pi S\).
		\item
		      Nous avons l'égalité \( L^2=4\pi S\) si et seulement si la courbe donnée par \( f\) est un cercle.
	\end{enumerate}
\end{theorem}
\index{base!hilbertienne!utilisation}
\index{inégalité!isopérimétrique}
\index{géométrique!avec des nombres complexes}
\index{courbe!étude métrique}
\index{série!de Fourier!utilisation}
\index{Fourier!série!utilisation}

\begin{proof}
	Nous commençons par considérer un chemin dont la longueur est \( 2\pi\) et nous en considérons son paramétrage normal. Nous allons exprimer l'aire \( S\) en utilisant le théorème de Green, et plus particulièrement la formule de surface \eqref{EqAJGrtOk}.

	Si \( f(s)=x(s)+iy(s)\), nous devons intégrer \( y'x-x'y\), qui n'est rien d'autre que la partie imaginaire de \( f'(s)\overline{ f(s) }\). Donc
	\begin{equation}    \label{EqCSWKbPX}
		S=\frac{ 1 }{2}\imag\int_0^{2\pi}f'(s)\overline{ f(s) }ds
	\end{equation}
	Nous considérons les coefficients de Fourier de \( f\) donnés par la formule \eqref{EqNDBaXRL} :
	\begin{equation}
		c_n(f)=\frac{1}{ 2\pi }\int_0^{2\pi}f(s) e^{-ins}ds.
	\end{equation}
	Ceux de \( f'\) (qui est aussi continue sur le compact \( S^1\) et donc tout autant \( L^2\)) sont donnés par
	\begin{equation}
		c_n(f')=inc_n(f).
	\end{equation}

	D'autre part en vertu du théorème~\ref{ThoLongueurIntegrale}, la longueur de \( \gamma\) s'exprime en termes de l'intégrale de la norme de sa dérivée :
	\begin{equation}
		2\pi=L=\int_0^{2\pi}| f'(s) |ds=\int_0^{2\pi}| f'(s) |^2ds
	\end{equation}
	parce que nous avons choisi un paramétrage normal qui vérifie automatiquement \( | f'(s) |=1\) pour tout \( s\). L'identité de Parseval sous sa forme \eqref{EqMIuCSfz} appliquée à \( f'\) nous enseigne que
	\begin{equation}        \label{EqXSpHuZI}
		L=2\pi=\int_0^{2\pi}| f'(s) |^2ds=2\pi\sum_{n=-\infty}^{\infty}| c_n(f') |^2=2\pi\sum_{n\in \eZ}n^2| c_n(f) |^2,
	\end{equation}
	et donc que
	\begin{equation}        \label{EQooAXAWooIgSDmu}
		\sum_{n\in \eZ}n^2| c_n(f) |^2=1.
	\end{equation}
	Par ailleurs le système trigonométrique étant une base hilbertienne, et les fonctions \( f\) et \( f'\) étant dans \( L^2\big( \mathopen[ 0 , 2\pi \mathclose] \big)\) (parce que continues sur un compact), elles sont égales à leurs séries de Fourier (au sens \( L^2\)), c'est-à-dire que nous avons l'égalité \eqref{EqXMMRpSN}. Nous avons alors
	\begin{subequations}
		\begin{align}
			\langle f', f\rangle_{L^2} & =\langle \sum_{n\in \eZ}c_n(f')e_n, \sum_{m\in \eZ}c_m(f)e_m\rangle                         \\
			                           & =\sum_m\sum_nc_n(f')\overline{ c_m(f) }\underbrace{\langle e_n, e_m\rangle }_{\delta_{n,m}} \\
			                           & =\sum_{n\in \eZ}c_n(f')\overline{ c_n(f) }                                                  \\
			                           & =\sum_nin| c_n(f) |^2
		\end{align}
	\end{subequations}
	où nous avons utilisé la continuité du produit scalaire pour sortir les sommes. Avec cela nous pouvons exprimer l'aire \eqref{EqCSWKbPX} en termes de coefficients de Fourier :
	\begin{equation}    \label{EqOZBMiat}
		S=\frac{ 1 }{2}\imag2\pi\langle f', f\rangle =\pi\sum_{n\in \eZ}n| c_n(f) |^2.
	\end{equation}
	En utilisant les expressions \eqref{EqXSpHuZI} et \eqref{EqOZBMiat} pour \( L\) et \( S\), et en écrivant \( L=2\pi \), nous avons
	\begin{subequations}
		\begin{align}
			L^2-4\pi S & =4\pi^2\left( \sum_{n\in \eZ}n^2| c_n(f) |^2 \right)^2-4\pi^2\sum_{n\in \eZ}n| c_n(f) |^2       \label{SUBEQooJTEWooSQpQFC} \\
			           & \geq 4\pi^2\sum_{n\in \eZ}n^2| c_n(f) |^2-4\pi^2\sum_{n\in \eZ}n| c_n(f) |^2      \label{SUBEQooBYENooTtoxGt}               \\
			           & =4\pi^2\sum_{n\in \eZ}| c_n(f) |^2(n^2-n)                                                                                   \\
			           & \geq 0.
		\end{align}
	\end{subequations}
	Justifications.
	\begin{itemize}
		\item Pour \eqref{SUBEQooJTEWooSQpQFC}. Expression \eqref{EqXSpHuZI} pour \( L\) et \eqref{EqOZBMiat} pour \( S\).
		\item Pour \eqref{SUBEQooBYENooTtoxGt}. La somme dans le premier terme valant \( 1\) par \eqref{EQooAXAWooIgSDmu}, nous pouvons supprimer le carré.
	\end{itemize}
	Cela prouve l'inégalité demandée dans le cas où \( L=2\pi\).

	Si \( \gamma\) n'est pas de longueur \( 2\pi\) mais \( L\), alors nous considérons le chemin \( \sigma(t)=\frac{ 2\pi\gamma(t) }{ L }\). Sa longueur est \( 2\pi\) et son aire, au vu de la formule de Green \eqref{EqCSWKbPX}, est de \( 4\pi^2\frac{ S }{ L^2 }\). L'inégalité isopérimétrique appliquée au chemin \( \sigma\) donne alors \( L^2\geq 4\pi S\).

	Le cas d'égalité s'obtient uniquement si \( c_n=0\) pour tout \( n\) différent de \( 0\) ou \( 1\). Dans ce cas nous avons
	\begin{equation}
		f(s)=c_0(f)+c_1(f) e^{is},
	\end{equation}
	qui est un cercle de centre \( c_0(f)\) et de rayon \( | c_1(f) |\).
\end{proof}


%---------------------------------------------------------------------------------------------------------------------------
\subsection{Formule d'inversion}
%---------------------------------------------------------------------------------------------------------------------------

\begin{proposition}[Formule d'inversion\cite{MesIntProbb,KXjFWKA}]  \label{PROPooLWTJooReGlaN}
	À propos d'inversion de la transformée de Fourier.
	\begin{enumerate}
		\item       \label{ITEMooLVOTooUDJSWt}
		      Si \( f\in\swS(\eR)\), alors nous avons la formule d'inversion
		      \begin{equation}        \label{EQooHIDAooHARdNZ}
			      f(x)=\frac{1}{ 2\pi }\int_{\eR} e^{ikx}\hat f(k)dk.
		      \end{equation}
		\item       \label{ITEMooWINLooJWcDIX}
		      Nous avons, pour \( f\in \swS(\eR)\),
		      \begin{equation}
			      f(x)=\frac{1}{ 2\pi }\TF(\hat f)(-x).
		      \end{equation}
		\item       \label{ITEMooCZYMooRvKTfS}
		      L'application
		      \begin{equation}
			      \TF\colon \swS(\eR)\to \swS(\eR)
		      \end{equation}
		      est une bijection continue.
		\item
		      Cette formule peut d'écrire de plusieurs autres façons : pour tout \( f\in \swS(\eR^d)\) nous avons
		      \begin{subequations}
			      \begin{align}
				      f(x)=\frac{1}{ 2\pi }\TF(\hat f)(-x).     \label{EQooWBZTooPeBNeh} \\
				      \TF^{-1}(f)(\xi) & =-\frac{1}{ 2\pi }\TF(f)(-\xi).
			      \end{align}
		      \end{subequations}
	\end{enumerate}
\end{proposition}

\begin{proof}
	Le fait que la transformée de Fourier sur \( \swS(\eR^d)\) prenne ses valeurs dans \( \swS(\eR^d)\) est déjà prouvé dans \ref{PropKPsjyzT}. Nous commençons maintenant la preuve de \ref{ITEMooLVOTooUDJSWt}.

	Soit \( f\in \swS(\eR^d)\). Pour \( \epsilon>0\) nous posons
	\begin{equation}
		f_{\epsilon}(k)= e^{-\epsilon k^2} e^{ikx}\hat f(k).
	\end{equation}
	Nous allons calculer
	\begin{equation}
		\lim_{\epsilon\to 0}\int_{\eR} e^{-\epsilon k^2} e^{ikx}\hat f(k)dk
	\end{equation}
	de deux façons; d'abord avec la convergence dominée, et ensuite avec Fubini.

	\begin{subproof}
		\spitem[Premier calcul : convergence dominée]
		D'abord en utilisant directement le théorème de la convergence dominée~\ref{ThoConvDomLebVdhsTf}. La fonction \( \hat f\) est dans \( \swS(\eR)\) (théorème~\ref{PropKPsjyzT}) et par conséquent \( f_{\epsilon}\in L^1(\eR)\) parce que le facteur \(  e^{-\epsilon k^2}\) ne va certainement pas empêcher de converger. De plus \( | f_{\epsilon} |\leq | \hat f |\) et \( \hat f\in L^1\). Le théorème est de la convergence dominée est applicable et
		\begin{equation}        \label{EQooYIYGooXYubbW}
			\lim_{\epsilon\to 0}\int_{\eR} e^{-\epsilon k^2} e^{ikx}\hat f(k)dk=\int_{\eR} e^{ikx}\hat f(k)dk.
		\end{equation}

		\spitem[Deuxième calcul : Fubini]

		Pour le deuxième calcul nous allons faire appel à Fubini\footnote{Parce qu'il est toujours plus simple de refiler le boulot aux autres que de le faire soi-même\ldots pauvre Fubini !} pour la fonction
		\begin{equation}
			\begin{aligned}
				u\colon \eR\times \eR & \to \eR                                     \\
				(k,y)                 & \mapsto  e^{ik(x-y)} e^{-\epsilon k^2}f(y).
			\end{aligned}
		\end{equation}
		D'abord nous nous assurons que \( u\in L^1(\eR\times \eR)\) par le corolaire~\ref{CorTKZKwP}, et ensuite nous utilisons le théorème de Fubini~\ref{ThoFubinioYLtPI} pour manipuler les intégrales (et en particulier les permuter).

		\begin{subproof}
			\spitem[\( u\in L^1(\eR\times \eR)\)]
			Dans un premier temps nous avons :
			\begin{equation}
				\int_{\eR}\int_{\eR}|  e^{ik(x-y)} e^{-\epsilon k^2} f(y) |dy\,dk\leq \int_{\eR} e^{-\epsilon k^2} \big[  \int_{\eR}| f(y) |   dy \big] dk<\infty
			\end{equation}
			parce que \( f\) étant dans \( \swS(\eR)\), l'intégrale intérieure se réduit à un nombre. Nous savons maintenant que \( u\in L^1(\eR\times \eR)\) grâce au corolaire \ref{CorTKZKwP}.

			\spitem[Calcul]

			Nous pouvons alors calculer un peu \ldots
			\begin{subequations}
				\begin{align}
					\int_{\eR} e^{ikx} e^{-\epsilon k^2}\hat f(k)dk & =\int_{\eR}\int_{\eR} e^{ikx} e^{-\epsilon k^2} e^{-iky}f(y)dy\,dk                                                             \\
					                                                & =\int_{\eR}\big[ \int_{\eR} e^{ik(x-y)} e^{-\epsilon k^2}f(y)dk \big]dy    \label{SUBEQooYJATooBdisqE}                         \\
					                                                & =\int_{\eR}f(y)\big[   \int_{\eR} e^{ik(x-y)} e^{-\epsilon k^2}dk  \big]dy                                                     \\
					                                                & =\int_{\eR} f(y)\hat g_{\epsilon}(y-x)dy   \label{SUBEQooXLLMooQazBnM}                                                         \\
					                                                & =\sqrt{ \frac{ \pi }{ \epsilon } }\int_{\eR}f(y) e^{-(y-x)^2/4\epsilon}dy\label{SUBEQooOPQNooPofmvh}                           \\
					                                                & =2\sqrt{ \epsilon }\sqrt{ \frac{ \pi }{ \epsilon } }\int_{\eR}f(x+2\sqrt{ \epsilon }t) e^{-t^2}dt  \label{SUBEQooONHGooNoiBru} \\
					                                                & =2\sqrt{ \pi }\int_{\eR}f(x+2\sqrt{ \epsilon }t) e^{-t^2}dt                                                                    \\
				\end{align}
			\end{subequations}
			Justifications  :
			\begin{itemize}
				\item
				      Pour \eqref{SUBEQooYJATooBdisqE}, c'est Fubini \ref{ThoFubinioYLtPI}.
				\item
				      Pour \eqref{SUBEQooXLLMooQazBnM}, nous avons reconnu dans le crochet la transformée de Fourier de la fonction \( g_{\epsilon}\colon x\mapsto  e^{-\epsilon x^2}\).
				\item
				      Pour \eqref{SUBEQooOPQNooPofmvh}, nous utilisons la transformée de Fourier de \( g_{\epsilon}\) donnée dans le lemme \ref{LEMooPAAJooCsoyAJ}.
				\item
				      Pour \eqref{SUBEQooONHGooNoiBru}, nous avons effectué le changement de variables \( t=(y-x)/(2\sqrt{ \epsilon })\) qui donne \( dt=dy/2\sqrt{ \epsilon }\).
			\end{itemize}

		\end{subproof}

		\spitem[Second passage à la limite]

		Nous avons obtenu l'égalité
		\begin{equation}        \label{EQooUCSTooRvOuhi}
			\int_{\eR} e^{ikx} e^{-\epsilon k^2}\hat f(k)dk = 2\sqrt{ \pi }\int_{\eR}f(x+2\sqrt{ \epsilon }t) e^{-t^2}dt,
		\end{equation}
		et nous voudrions passer à la limite \( \epsilon\to 0\). Le membre de gauche est déjà fait en \eqref{EQooYIYGooXYubbW}.

		Pour le membre de droite, la fonction \( f\) étant Schwartz (en particulier bornée), nous pouvons effectuer la majoration
		\begin{equation}
			f(x+2\sqrt{ \epsilon }t) e^{-t^2}\leq \| f \|_{\infty} e^{-t^2},
		\end{equation}
		qui est une fonction intégrable de \( t\). Nous avons donc le droit de permuter la limite \( \epsilon\to 0\) et l'intégrale dans le calcul suivant :
		\begin{equation}
			\lim_{\epsilon\to 0}\int_{\eR}f(x+2\sqrt{ \epsilon }t) e^{-t^2}=\int_{\eR}f(x) e^{-t^2}dt.
		\end{equation}

		\spitem[Fin]

		Nous avons maintenant les limites des deux membres de \eqref{EQooUCSTooRvOuhi}. Récrivons :
		\begin{equation}
			\lim_{\epsilon\to 0}\int_{\eR} e^{ikx} e^{-\epsilon k^2}\hat f(k)dk =\lim_{\epsilon\to 0} 2\sqrt{ \pi }\int_{\eR}f(x+2\sqrt{ \epsilon }t) e^{-t^2}dt,
		\end{equation}
		À gauche nous avons déjà la limite depuis \eqref{EQooYIYGooXYubbW}, et à droite nous obtenons
		\begin{equation}
			\lim_{\epsilon\to 0} 2\sqrt{ \pi }\int_{\eR}f(x+2\sqrt{ \epsilon }t) e^{-t^2}dt=2\sqrt{ \pi }\int_{\eR}f(x) e^{-t^2}dt=2\sqrt{ \pi }f(x)\sqrt{ \pi }=2\pi f(x)
		\end{equation}
		où nous avons utilisé l'intégrale gaussienne faite dans l'exemple~\ref{EXooLUFAooGcxFUW}.

		En remettant tout ensemble,
		\begin{equation}
			2\pi f(x)=\lim_{\epsilon\to 0}\int_{\eR} e^{-\epsilon k^2} e^{ikx}\hat f(k)dk=\int_{\eR} e^{ikx}\hat f(k)dk,
		\end{equation}
		ce qu'il fallait prouver.
	\end{subproof}

	Le plus gros est fait; nous avons prouvé
	\begin{equation}        \label{EQooYVUZooZXQoEH}
		f(x)=\frac{1}{ 2\pi }\int_{\eR} e^{ikx}\hat f(k)dk.
	\end{equation}
	Pour \ref{ITEMooWINLooJWcDIX}, c'est simplement une reformulation de cela.

	Nous prouvons maintenant \ref{ITEMooCZYMooRvKTfS}.

	\begin{subproof}
		\spitem[La transformée de Fourier est injective]
		Vu qu'elle est linéaire, il suffit de démontrer que si \( \TF(f)=0\), alors \( f=0\). Si \( \hat f=0\), alors la formule \eqref{EQooYVUZooZXQoEH} donne immédiatement \( f=0\).
		\spitem[La transformée de Fourier est surjective]
		Soit \( f\in \swS(\eR)\). Nous devons trouver \( g\in\swS(\eR)\) tel que \( \hat g=f\). La formule \( \TF(\hat f)(-x)=2\pi f(x)\) nous incite à essayer
		\begin{equation}
			g(\xi)=-\frac{1}{ 2\pi }\hat f(-\xi).
		\end{equation}
		Calculons \( \TF(g)\) :
		\begin{subequations}
			\begin{align}
				\TF(g)(x) & =-\int_{\eR} e^{-i\xi x}g(\xi)d\xi                       \\
				          & =-\int_{\eR} e^{-i\xi x}\frac{1}{ 2\pi }\hat f(-\xi)d\xi \\
				          & =\frac{1}{ 2\pi }\int_{\eR} e^{itx}\hat f(t)dt           \\
				          & =\frac{1}{ 2\pi }\TF(\hat f)(-x)                         \\
				          & =f(x).
			\end{align}
		\end{subequations}
		Pour la dernière ligne, nous avons utilisé \( \TF(\hat f)(-x)=2\pi f(x)\).
	\end{subproof}


\end{proof}

\begin{corollary}       \label{CORooAZLZooSviTej}
	Nous avons la formule
	\begin{equation}        \label{EQooRJXRooElEMAa}
		\int_{\eR}\int_{\eR} e^{-ikx}f(x)dx\,dk=2\pi f(0).
	\end{equation}
\end{corollary}

\begin{proof}
	Poser \( x=0\) dans l'équation \eqref{EQooHIDAooHARdNZ}.
\end{proof}

\begin{normaltext}
	Les physiciens qui n'ont que rarement peur écrivent souvent la formule \eqref{EQooRJXRooElEMAa} sous la forme
	\begin{equation}
		\int_{\eR} e^{-ikx}dk=\delta(x)
	\end{equation}
	où \( \delta\) serait la fonction de Dirac qui vaut zéro partout sauf en \( x=0\) où elle vaudrait l'infini, mais pas n'importe quel infini; juste celui qu'il faut pour que sont intégrale vaille \( 1\).
\end{normaltext}

\begin{lemma}   \label{LemYYjFZSa}
	Si \( \phi\in\swS(\eR\times \eR^n)\), alors
	\begin{equation}
		\partial_t\hat\phi=\widehat{\partial_t\phi}
	\end{equation}
	où le chapeau dénote la transformée de Fourier par rapport à la variable dans \( \eR^n\) et non par rapport à celle dans \( \eR\). Le \( t\) par contre est la variable dans \( \eR\).
\end{lemma}

\begin{proof}
	Par définition de la transformée de Fourier nous avons
	\begin{equation}
		(\partial_t\hat\phi)(t,\xi)=\frac{ \partial  }{ \partial t }\int_{\eR^n}\phi(t,x) e^{-i x\xi}dx.
	\end{equation}
	Notre but est de permuter l'intégrale et la dérivée en utilisant le théorème~\ref{ThoMWpRKYp}. Il nous faut une fonction \( G\colon \eR^n\to \eR\) qui soit intégrable sur \( \eR^n\) et telle que
	\begin{equation}
		\left| \frac{ \partial \phi }{ \partial t }\phi(t,x) \right| \leq G(x)
	\end{equation}
	pour tout \( t\in B(t_0,\delta)\). Étant donné que la fonction \( \partial_t\phi\) est tout autant Schwartz que \( \phi\) elle-même nous pouvons alléger les notations et chercher une fonction \( G\) qui convient pour toute fonction \( \varphi\in\swS(\eR\times \eR^n)\). Soit la fonction
	\begin{equation}
		G(x)=\sup_{t\in B(t_0,\delta)}| \varphi(t,x) |.
	\end{equation}
	Pour tout multiindice \( \alpha\) nous avons alors
	\begin{equation}
		\sup_{x\in \eR^n}\big| x^{\alpha}G(x) \big|\leq \sup_{(t,x)\in \eR\times \eR^n}\big| x^{\alpha}\varphi(t,x) \big|\leq M_{\alpha}\in \eR.
	\end{equation}
	Grâce à la proposition~\ref{PropCSmzwGv}, cela signifie que \( \varphi\) décroît plus vite que n'importe quel polynôme; \( G\) est donc intégrable sur \( \eR^n\).
\end{proof}

%+++++++++++++++++++++++++++++++++++++++++++++++++++++++++++++++++++++++++++++++++++++++++++++++++++++++++++++++++++++++++++
\section{Transformée de Fourier sur \( L^2(\eR^d)\)}
%+++++++++++++++++++++++++++++++++++++++++++++++++++++++++++++++++++++++++++++++++++++++++++++++++++++++++++++++++++++++++++

La théorie des transformées de Fourier est intéressante sur \( L^2(\eR^d)\) parce qu'elle y donne une isométrie. Nous allons la donner avec des fonctions à valeurs dans \( \eC\).

\begin{remark}
	Une remarque qui vaut ce qu'elle vaut, mais si \( u\) est une classe de fonction pour la relation \( u\sim v\) si et seulement si \( u (x)=v(x)\) pour presque tout \( v\) alors l'intégrale
	\begin{equation}
		\hat u(\xi)=\int_{\eR^d}u(x) e^{ix\xi}dx
	\end{equation}
	ne dépend pas du choix du représentant. Nous pouvons donc parfaitement parler de transformée de Fourier d'une classe de fonctions.
\end{remark}

%---------------------------------------------------------------------------------------------------------------------------
\subsection{Le problème}
%---------------------------------------------------------------------------------------------------------------------------

Nous avons défini en général la transformée de Fourier d'une fonction \( f\colon \eR\to \eC\) par la formule
\begin{equation}        \label{EQooOPLJooEkqRuC}
	\hat f(\xi)=\int_{\eR} e^{i\xi x } f(x)dx
\end{equation}
tant que cette intégrale existe.

Il se fait que cette intégrale n'existe pas toujours pour des fonctions dans \( L^2(\eR)\). Donc nous devons faire mieux pour définir la transformée de Fourier sur \( L^2\).

\begin{example}[\cite{MonCerveau}]     \label{EXooSWCHooTdHTsl}
	Prenons la fonction
	\begin{equation}
		f(x)=\begin{cases}
			0             & \text{si } x<1      \\
			\frac{1}{ x } & \text{si } x\geq 1.
		\end{cases}
	\end{equation}
	Vu que l'intégrale \( \int_1^{\infty}\frac{1}{ x^2 }dx\) existe et est finie (proposition~\ref{PropBKNooPDIPUc}\ref{ITEMooJFSXooHmgmEj}), la fonction \( f\) est dans \( L^2(\eR)\).

	Cependant l'intégrale \eqref{EQooOPLJooEkqRuC} n'existe pas. Pour nous convaincre de cela, nous pouvons simplement nous souvenir de la définition d'une intégrale à valeurs dans un espace vectoriel (définition~\ref{PROPooOFSMooLhqOsc}). Nous fixons \( \xi\in \eR\) et nous posons \( g(x)=f(x) e^{i\xi x}\).

	Bien évidemment, \( | g(x) |=\frac{1}{ x }\) sur \( \mathopen] 1 , \infty \mathclose[\). Donc \( \int_{\eR}| g |=\infty\), et la fonction \( g\) n'est pas intégrable. Fin de l'histoire.

	Nous pouvons toujours essayer de comprendre mieux. Vu que \( \int_{\eR}| g |=\infty\), la proposition~\ref{PROPooNSCPooCMkrZl} nous dit qu'au moins une des intégrales parmi
	\begin{equation}
		\int g^+_{re}, \int g^+_{im},\int g^-_{re},\int g^-_{im}
	\end{equation}
	est égale à \( +\infty\).

	Note qu'en travaillant un peu, on se convainc qu'en réalité, elles divergent toutes les quatre.
\end{example}

%---------------------------------------------------------------------------------------------------------------------------
\subsection{Extension de \( L^1\cap L^2\) vers \( L^2\)}
%---------------------------------------------------------------------------------------------------------------------------

\begin{theorem}[Extention de la transformée de Fourier vers \( L^2(\eR^d)\)\cite{ooKDRBooDFsyfV}]       \label{THOooJLCDooAjTvJf}
	Soit \( f\in L^1(\eR^d)\cap L^2(\eR^d)\). Alors
	\begin{enumerate}
		\item
		      Nous avons \( \TF(f)\in L^2\) et \( \| \hat f\|_{L^2}= (2\pi)^d  \| f \|_{L^2}\).
		\item
		      L'application \( \TF\colon L^1\cap L^2\to L^2\) peut être étendue en une application \( \TF\colon L^2(\eR^d)\to L^2(\eR^d)\) vérifiant
		      \begin{equation}
			      \| \hat f \|_{L^2}=(2\pi)^d\| f \|_{L^2}
		      \end{equation}
		      pour tout \( f\in L^2(\eR^d)\).
	\end{enumerate}
\end{theorem}

\begin{proof}
	Le fait que \( f\in L^1\) implique \( \| \TF(f) \|_{\infty}\leq \| f \|_1\) (c'est le lemme~\ref{LEMooCBPTooYlcbrR}). En particulier, \( | \TF(f)(\xi) |^2\) est majoré et l'intégrale
	\begin{equation}
		\clubsuit=\int_{\eR^d}| \hat f |^2 e^{-\epsilon\xi^2}d\xi
	\end{equation}
	existe et est finie.
	\begin{subproof}
		\spitem[Découper l'intégrale]
		Dans un premier temps nous développons les intégrales. Dans les égalités suivantes, \( x\xi\) est le produit scalaire \( x\cdot \xi\) dans \( \eR^d\).
		\begin{subequations}
			\begin{align}
				\clubsuit & =\int_{\eR^d}\left( \int_{\eR^d} \overline{ f(x) } e^{ix\xi}dx \right)\left( \int_{\eR^d} f(y)  e^{-y\xi} \right) e^{- \epsilon | \xi |^2}d\xi \\
				          & =\int_{\eR^d}\left[ \int_{\eR^d\times \eR^d}  \overline{ f(x) }  f(y) e^{i\xi(x-y)}dxdy \right] e^{-\epsilon| \xi |^2}d\xi.
			\end{align}
		\end{subequations}
		Nous avons utilisé le théorème de Fubini pour regrouper les intégrales\footnote{Dans la suite nous allons encore utiliser Fubini quelques fois pour regrouper et dégrouper des intégrales.}. Vu que \( f\in L^1(\eR^d)\), la fonction \( (x,y,\xi)\mapsto f(x)f(y) e^{-\epsilon| \xi |^2}\) est dans \( L^1(\eR^d\times \eR^d\times \eR^d)\) et le théorème de Fubini~\ref{ThoFubinioYLtPI} avec \( \Omega_1=\eR^d\times \eR^d\) et \( \Omega_2=\eR^d\)  nous permet de permuter les intégrales pour avoir
		\begin{equation}        \label{EQooSUYWooCmtFeF}
			\clubsuit=\int_{\eR^d\times \eR^d} \overline{ f(x) }f(y)  \left[ \int_{\eR^d} e^{i\xi(x-y)} e^{-\epsilon| \xi |^2}d\xi \right]dxdy.
		\end{equation}
		\spitem[Discuter de cette gaussienne]
		En posant
		\begin{subequations}
			\begin{align}
				g(x)= e^{-| x |^2/2} \\
				g_{\epsilon}(x)=g(\sqrt{ 2\epsilon }x)= e^{-\epsilon| x |^2}
			\end{align}
		\end{subequations}
		nous avons \( g_{\epsilon}\in \swS(\eR^d)\) et le lemme~\ref{LEMooTDWSooSBJXdv} nous autorise à écrire
		\begin{subequations}
			\begin{align}
				\hat g(\xi)            & =(2\pi)^{d/2}g(\xi)                                                     \\
				\hat g_{\epsilon}(\xi) & =\left( \frac{ \pi }{ \epsilon } \right)^{d/2} e^{-| \xi |^2/4\epsilon}
			\end{align}
		\end{subequations}
		Nous voyons que \( \hat g_{\epsilon}\in\swS(\eR^d)\) (c'était gagné d'avance par la proposition~\ref{PropKPsjyzT}) et que \( \hat g_{\epsilon}\) est une fonction paire (encore une fois, c'était gagné d'avance parce que la transformée de Fourier d'une fonction paire est paire).

		Tout cela pour dire que l'intégrale entre crochet dans \eqref{EQooSUYWooCmtFeF} est \( \hat g_{\epsilon}(y-x)=\hat g_{\epsilon}(x-y)\), et donc
		\begin{equation}
			\clubsuit=\int_{\eR^d\times \eR^d} \overline{ f(x) }f(y)  \hat g_{\epsilon}(x-y)  dxdy.
		\end{equation}
		Encore une fois le théorème de Fubini permet de séparer les intégrales et de calculer l'intégrale sur \( y\) en premier. Vu que \( f\in L^1\) et que \( \hat g_{\epsilon}\in \swS(\eR^d)\), le produit de convolution \( f*\hat g_{\epsilon}\) est un élément de \( \swS(\eR^d)\) par la proposition~\ref{PROPooUNFYooYdbSbJ}.
		Nous avons donc
		\begin{equation}
			\clubsuit=\int_{\eR^d}\overline{ f(x) }(f* \hat g_{\epsilon})(x)dx.
		\end{equation}
		Là, nous reconnaissons un produit scalaire dans \( L^2(\eR^d)\), et donc
		\begin{equation}        \label{EQooWIHNooHutHlS}
			\int_{\eR^d}| \hat f |^2 e^{-\epsilon\xi^2}d\xi=\langle f, f*\hat g_{\epsilon}\rangle_{L^2(\eR^d)}.
		\end{equation}
		Notons que tout a un sens : \( f\in L^2(\eR^d)\) et \( f*\hat g_{\epsilon}\in\swS(\eR^d)\subset L^2(\eR^d)\).

		\spitem[Suite régularisante]

		Nous prenons la suite régularisante du lemme~\ref{LEMooTDWSooSBJXdv} donnée par
		\begin{equation}
			\rho_n=\frac{1}{ (2\pi)^d }\hat g_{1/n}.
		\end{equation}

		\spitem[Première conclusion]
		Nous reprenons \eqref{EQooWIHNooHutHlS}
		\begin{equation}
			\int_{\eR^d}| \hat f |^2 e^{-| \xi^2 |/n}d\xi=\langle f, f*\hat g_{1/n}\rangle_{L^2(\eR^d)}=(2\pi)^d\langle f, f* \rho_n \rangle .
		\end{equation}
		En prenant la limite \( n\to \infty\) nous trouvons
		\begin{equation}
			\lim_{n\to \infty}\int_{\eR^d}| \hat f |^2 e^{-\epsilon\xi^2}d\xi=(2\pi)^d\| f \|^2.
		\end{equation}
		Pour effectuer la limite du membre de gauche nous devons remarquer qu'en posant
		\begin{equation}
			g_n(\xi)=| \hat f(\xi) | e^{-| \xi |^2/n},
		\end{equation}
		nous avons une suite décroissante de fonctions (c'est-à-dire que à \( \xi\) fixé, \( g_n(\xi)\) est décroissant en \(n\)). Par ailleurs ces fonctions sont toujours à valeurs dans \( \mathopen[ 0 , \infty \mathclose]\) et nous pouvons utiliser le théorème de la convergence monotone~\ref{ThoRRDooFUvEAN} pour permuter la limite et l'intégrale. Au final :
		\begin{equation}
			\| \hat f \|_{L^2}=(2\pi)^d\| f \|_{L^2}.
		\end{equation}
	\end{subproof}

	En ce qui concerne l'extension, soit \( f\in L^2(\eR^d)\) et une suite \( (f_n)\) dans \( L^1\cap L^2\) telle que \( f_n\stackrel{L^2}{\longrightarrow}f\).
	\begin{subproof}
		\spitem[Existence d'une telle suite]
		Si \( f\in L^2(\eR^d)\), alors nous pouvons poser
		\begin{equation}    \label{EQooHGJYooJsmxoX}
			f_n(x)=f(x) e^{-|x|^2/n^2}.
		\end{equation}
		Par l'inégalité de Hölder \eqref{EqLPKooPBCQYN} nous avons \( f_n\in L^1(\eR^d)\); de plus \( f_n\in L^2(\eR^d)\) parce que pour tout \( x\) nous avons \( | f_n(x) |\leq | f(x) |\). Montrons que \( f_n\stackrel{L^2}{\longrightarrow}f\). Nous avons
		\begin{equation}
			\| f_n-f \|_{L^2}^2=\int_{\eR^d}| f(x)(1- e^{-| x^2 |/n^2}) |^2dx.
		\end{equation}
		Nous voulons prendre la limite \( n\to \infty\). Pour ce faire à droite nous remarquons que \(  e^{-| x |^2/n^2}\) est majoré par \( 1\); ce qui se trouve dans l'intégrale est donc majoré (uniformément en \( n\)) par \( | f(x) |^2\), qui est une fonction \( L^1\) parce que \( f\) est \( L^2\). Le théorème de la convergence dominée~\ref{ThoConvDomLebVdhsTf} nous permet alors de permuter la limite et l'intégrale, ce qui donne
		\begin{equation}
			\lim_{n\to \infty} \| f_n-f \|_2^2=\int_{\eR^d}\lim_{n\to \infty} | f(x)(1- e^{-| x |^2/n^2}) |^2dx=0.
		\end{equation}

		\spitem[Définition de \( \TF\colon L^2\to L^2\)]

		La suite \( (f_n)\) est une suite convergence dans \( L^2\), et elle est donc de Cauchy. De plus pour chaque \( n,m\) nous avons
		\begin{equation}
			\| \hat f_n-\hat f_m \|=(2\pi)^d\| f_n-f_m \|.
		\end{equation}
		Nous voyons donc que la suite \( (\hat f_n)\) est également de Cauchy, dans l'espace \( L^2(\eR^d)\) qui est complet (lemme~\ref{LemIVWooZyWodb}). Nous posons
		\begin{equation}
			\hat f=\lim_{n\to \infty} \hat f_n.
		\end{equation}

		\spitem[Indépendance aux choix]
		Nous devons montrer que la définition de \( \hat f\) ne dépend pas de la suite approximant \( f\) dans \( L^1\cap L^2\). Soient dans deux suites \( f_n\stackrel{L^2}{\longrightarrow}f\) et \( g_n\stackrel{L^2}{\longrightarrow}f\) telles que \( \hat f_n\stackrel{L^2}{\longrightarrow}F\) et \( \hat g_n\stackrel{L^2}{\longrightarrow}G\). Alors
		\begin{equation}
			\| \hat f_n-\hat g_n \|=(2\pi)^d\| f_n-g_n \|\leq (2\pi)^d\| f_n-f \|+(2\pi)^d\| g_n-f \|\to 0.
		\end{equation}
		Par conséquent \( (\hat f_n-\hat g_n)_n\) est une suite qui converge vers zéro. Par unicité de la limite, \( F=G\).
	\end{subproof}
\end{proof}

\begin{remark}
	Une autre suite possible, à la place de \eqref{EQooHGJYooJsmxoX}, est
	\begin{equation}
		f_n(x)=f(x)\mtu_{| x |<n}.
	\end{equation}
	C'est-à-dire la fonction \( f\) limitée à une boule de rayon \( n\) autour de \( 0\).
\end{remark}
