% This is part of Le Frido
% Copyright (c) 2008-2018
%   Laurent Claessens
% See the file fdl-1.3.txt for copying conditions.

%+++++++++++++++++++++++++++++++++++++++++++++++++++++++++++++++++++++++++++++++++++++++++++++++++++++++++++++++++++++++++++
\section{Exponentielle de matrice}
%+++++++++++++++++++++++++++++++++++++++++++++++++++++++++++++++++++++++++++++++++++++++++++++++++++++++++++++++++++++++++++
\label{secAOnIwQM}

\begin{proposition}     \label{PropPEDSooAvSXmY}
    Soit \( V\) un espace vectoriel de dimension finie et \( A\in\End(V)\). La série
    \begin{equation}
        \exp(A)=\mtu+A+\frac{ A^2 }{ 2 }+\frac{ A^3 }{ 3 }+\ldots =\sum_{k=1}^{\infty}\frac{ A^k }{ k! }.
    \end{equation}
    converge normalement dans \( \big( \End(V),\| . \|_{op} \big)\).  L'\defe{exponentielle}{exponentielle!de matrice} de la matrice \( A\) est cette matrice.
\end{proposition}

\begin{proof}
    Vu que la norme opérateur est une norme d'algèbre par le lemme~\ref{LEMooFITMooBBBWGI}, nous avons pour tout \( k\) la majoration \( \| A^k \|\leq \| A \|^k\). Nous avons donc
    \begin{equation}
        \sum_{k=0}^{\infty}\frac{ \| A^k \| }{ k! }\leq \sum_k\frac{ \| A \|^k }{ k! }.
    \end{equation}
    La dernière somme converge en vertu de la convergence de la série exponentielle donnée en exemple~\ref{ExIJMHooOEUKfj}.
\end{proof}

Étant donné que c'est une limite, il y a une question de convergence et donc de topologie. C'est pour cela que nous ne pouvions pas introduire l'exponentielle de matrice avant d'avoir introduit la norme des matrices. La convergence de la série pour toute matrice sera prouvée au passage dans la proposition~\ref{PropFMqsIE}.


La fonction exponentielle \(  x\mapsto e^{x}\) n'est pas un polynôme en \( x\), mais nous avons le résultat marrant suivant.
\begin{proposition} \label{PropFMqsIE}
    Si \( u\) est un endomorphisme, alors \( \exp(u)\) est un polynôme en \( u\)\footnote{Nan, mais j'te jure : \( \exp\) n'est pas un polynôme, mais $\exp(u)$ est un polynôme de \( u\).}.
\end{proposition}

\begin{proof}
    Nous considérons l'application
    \begin{equation}
        \begin{aligned}
            \varphi_u\colon \eK[X]&\to \End(E) \\
            P&\mapsto P(u)
        \end{aligned}
    \end{equation}
    Étant donné que l'image de \( \varphi_u\) est un fermé dans \( \End(E)\), il suffit de montrer que la série
    \begin{equation}
        \sum_{k=0}^{\infty}\frac{ \varphi_u(X)^k }{ k! }
    \end{equation}
    converge dans \( \End(E)\) pour qu'elle converge dans \( \Image(\varphi_u)\). Pour ce faire nous nous rappelons de la norme opérateur\footnote{Définition~\ref{DefNFYUooBZCPTr}.} et de la propriété fondamentale \( \| A^k \|\leq \| A \|^k\). En notant \( A=\varphi_u(X)\),
    \begin{equation}
        \left\| \sum_{k=n}^m\frac{ A^k }{ k! } \right\|\leq \sum_{k=n}^m\frac{ \| A^k \| }{ k! }\leq \sum_{k=n}^m\frac{ \| A \|^k }{ k! },
    \end{equation}
    ce qui est une morceau du développement de \(  e^{\| A \|}\). La limite \( n\to\infty\) est donc zéro par la convergence de l'exponentielle réelle. La suite des sommes partielles de  $e^{A}$ est donc de Cauchy. La série converge donc parce que nous sommes dans un espace vectoriel réel de dimension finie (\( \End(E)\)).
\end{proof}
% TODO : et tant qu'on y est, justifier la convergence de la série de l'exponentielle réelle.

\begin{remark}
    Pourquoi \( \exp(u)\) est-il un polynôme d'endomorphisme alors que \( \exp\) n'est pas un polynôme ? Lorsque nous disons que la fonction \( x\mapsto \exp(x)\) n'est pas un polynôme, nous sommes en train de localiser la fonction \( \exp\) à l'intérieur de l'espace de toutes les fonctions \( \eR\to \eR\), c'est à dire à l'intérieur d'un espace de dimension infinie. Au contraire lorsqu'on parle de \( \exp(u)\) et qu'on le compare aux endomorphismes \( P(u)\), nous sommes en train de repérer \( \exp(u)\) à l'intérieur de l'espace des matrices qui est de dimension finie. Il n'est donc pas étonnant que l'on parvienne moins à faire la distinction.

    Si par contre nous considérons \( \exp\) en tant qu'application \( \exp\colon \End(E)\to \End(E)\), ce n'est pas un polynôme.

    Si \( u\) et \( v\) sont des endomorphismes, nous aurons des polynômes \( P\) et \( Q\) tels que \( e^u=P(u)\) et \( e^v=Q(v)\); mais nous n'aurons en général évidemment pas \( P=Q\). En cela, \( \exp\) n'est pas un polynôme.
\end{remark}

%+++++++++++++++++++++++++++++++++++++++++++++++++++++++++++++++++++++++++++++++++++++++++++++++++++++++++++++++++++++++++++
\section{Espace dual}
%+++++++++++++++++++++++++++++++++++++++++++++++++++++++++++++++++++++++++++++++++++++++++++++++++++++++++++++++++++++++++++
\label{SECooKOJNooQVawFY}

\begin{definition}
    Soit un espace vectoriel normé \( (V,\| . \|)\) sur le corps \( \eC\) ou \( \eR\) (que nous nommons \( \eK\)). Son \defe{dual topologique}{dual topologique}, noté \( V'\) est l'ensemble des applications linéaires continues \( V\to \eK\).
\end{definition}

%---------------------------------------------------------------------------------------------------------------------------
\subsection{Topologies}
%---------------------------------------------------------------------------------------------------------------------------

Il est possible de mettre sur \( V'\) (au moins) deux topologies distinctes. La première est la topologie de la norme opérateur; rien de nouveau pour elle. La seconde est la topologie \( *\)-faible dont nous avons déjà un peu parlé dans la définition~\ref{DefHUelCDD}.

En termes de notations, nous allons noter les semi-normes de la topologie faible par
\begin{equation}
    p_x(\varphi)=| \varphi(x) |
\end{equation}
pour \( x\in V\) et \( \varphi\in V'\). À droite, les barres dénotent soit la valeur absolue (si \( \eK=\eR\)), soit le module (si \( \eK=\eC\)).

\begin{lemma}       \label{LEMooFMAUooQBIeTh}
    Soit \( \varphi\in V'\) et \( x\in V\). Alors
    \begin{equation}
        p_x(\varphi)\leq\frac{ \| \varphi \| }{ \| x \| }.
    \end{equation}
    Si \( \varphi_0\in V'\), si \( r>0\) et si \( x\in V\) nous avons aussi :
    \begin{equation}
        B(\varphi_0,r)\subset B_x(\varphi_0,\frac{ r }{ \| x \| }).
    \end{equation}
\end{lemma}

\begin{proof}
    En posant \( x'=x/\| x \|\) nous avons
    \begin{equation}
        p_x(\varphi)=| \varphi(x) |=\frac{1}{ \| x \| }| \varphi(x') |\leq \frac{1}{ \| x \| }\| \varphi \|.
    \end{equation}

    En ce qui concerne la seconde affirmation, si \( \varphi\in B(\varphi_0,r)\) alors en notant \( x'=x/\| x \|\) nous avons :
    \begin{equation}
        p_x(\varphi_0-\varphi)=| \varphi_0(x)-\varphi(x) |=\frac{1}{ \| x \| }| \varphi_0(x')-\varphi(x') |\leq\frac{1}{ \| x \| }\|\varphi_0-\varphi  \|\leq \frac{ r }{ \| x \| }.
    \end{equation}
    Donc \( \varphi\in B_x\big( \varphi_0,\frac{ r }{ \| x \| } \big)\).
\end{proof}

\begin{proposition}
    En ce qui concerne la convergence d'une suite \( (\varphi_k)\) dans \( V'\) mais si elle vérifie
    \begin{equation}
        \varphi_k\stackrel{\| . \|}{\longrightarrow}\varphi
    \end{equation}
    alors
    \begin{equation}
        \varphi_k\stackrel{*}{\longrightarrow}\varphi.
    \end{equation}
\end{proposition}

\begin{proof}
    Soit une suite \( (\varphi_k)\) dans \( V'\), convergente vers \( \varphi\) pour la topologie de la norme.  Soit \( x\in V\), et \( x'=x/\| x \|\). Nous avons
    \begin{equation}
        p_x(\varphi_k-\varphi)=\frac{1}{ \| x \| }| \varphi_k(x')-\varphi(x) |\leq\frac{1}{ \| x \| }\| \varphi_k-\varphi \|\to 0.
    \end{equation}
\end{proof}

\begin{lemma}       \label{LEMooEAVEooAFveHn}
    La translation dans \( V'\) est une opération continue pour la topologie de la norme opérateur et pour celle de la topologie \( *\).
\end{lemma}

\begin{proof}
    Soit une suite \( \varphi_k\) tendant vers \( 0\); nous devons prouver que \( \tau_{\sigma}(\varphi_k)\to \tau_{\sigma}(0)=\sigma\). Et ce, pour chacune des deux topologies.

    \begin{subproof}
        \item[Norme opérateur]

            L'hypothèse \( \varphi_k\stackrel{\| . \|}{\longrightarrow} 0\) signifie que \( \| \varphi_k \|\to 0\), c'est à dire que
            \begin{equation}
                \sup_{\| v \|=1}| \varphi_k(v) |\to 0.
            \end{equation}
            Nous avons alors
            \begin{equation}
                \| \tau_{\sigma}(\varphi_k)-\sigma \|=\sup_{\| v \|=1}| \tau_{\sigma}(\varphi_k)v-\sigma(v) |=\sup_{\| v \|=1}| \varphi_k(v) |\to 0.
            \end{equation}
            Donc d'accord pour \( \tau_{\sigma}(\varphi)\to \sigma\).

        \item[Topologie $*$]

            Nous supposons maintenant que \( \varphi_k\stackrel{*}{\longrightarrow}0\). Pour tout \( v\in V\) nous avons
            \begin{equation}
                p_v\big( \tau_{\sigma}(\varphi_k)-\sigma \big)=\big| \tau_{\sigma}(\varphi_k)v-\sigma(v) \big|=| \varphi_k(v) |=p_v(\varphi_k).
            \end{equation}
            Mais par hypothèse, \( p_v(\varphi_k)\to 0\).
    \end{subproof}
\end{proof}

Pour la suite, nous allons préfixer par \( N\) les concepts liés à la topologie de \( V'\) associée à la norme opérateur et par \( *\), les concepts de la topologie \( *\).

\begin{proposition}     \label{PROPooFGXAooFRWweD}
    Soit un espace vectoriel normé \( V\). Un \( *\)-ouvert et toujours un \( N\)-ouvert.
\end{proposition}

\begin{proof}
    Soit un \( *\)-ouvert \( \mO\) de \( V'\). Il existe donc \( x\in V\) et \( r>0\) tels que \( B_x(\varphi,r)\subset \mO\). Nous avons alors, en utilisant le lemme~\ref{LEMooFMAUooQBIeTh},
    \begin{equation}
        B(\varphi,r\| x \|)\subset B_x(\varphi,r)\subset \mO.
    \end{equation}
    Donc \( \mO\) est un \( N\)-ouvert.
\end{proof}

\begin{corollary}
    Soit un espace topologique \( X\). Si \( f\colon (V',*)\to X\) est continue, alors \( f\colon (V',\| . \|)\to X\) est continue.
\end{corollary}

\begin{proof}
    Soit un ouvert \( \mO\) de \( X\). Vu que \( f\) est \( *\)-continue, la partie \( f^{-1}(\mO)\) est un \( *\)-ouvert de \( V'\). Il est onc un \( N\)-ouvert de \( V'\) par la proposition~\ref{PROPooFGXAooFRWweD}.
\end{proof}

%---------------------------------------------------------------------------------------------------------------------------
\subsection{Réflexivité}
%---------------------------------------------------------------------------------------------------------------------------

Pour la suite nous notons \( V''\) le dual de \( (V',\| . \|)\). Certes en tant qu'ensembles, \( (V',*)\) et \( (V',\| . \|) \) sont identiques, mais comme ils n'ont pas la même topologie, les duaux ne sont pas les mêmes.

Bref, \( V''\) est l'ensemble des applications linéaires continues \( (V',\| . \|)\to \eC\). Et lorsque nous disons \( \eC\) ici, ça peut aussi bien être \( \eR\) selon le contexte.

De plus nous considérons que \( V''\) la norme opérateur qui dérive de la norme de \( V'\), laquelle dérive de la norme vectorielle sur \( V\).

\begin{propositionDef}      \label{PROPooMAQSooCGFBBM}
    Soit un espace vectoriel normé $V$ sur $\eR$ ou $\eC$. Nous considérons l'application
    \begin{equation}
        \begin{aligned}
            J\colon V&\to V'' \\
            J(x)\varphi&= \varphi(x).
        \end{aligned}
    \end{equation}
    \begin{enumerate}
        \item       \label{ITEMooNVVSooNFXgnE}
            L'application \( J\) est bien définie : \( J(x)\) est continue.
        \item       \label{ITEMooKURHooZZWpbu}
            L'application \( J\) est continue.
        \item       \label{ITEMooTFYVooKhMOjp}
             Elle est injective.
    \end{enumerate}

    Lorsque \( J\) est bijective, l'espace \( V\) est dit \defe{réflexif}{réflexif}.
\end{propositionDef}

\begin{proof}
    Point par point.
    \begin{subproof}
        \item[\ref{ITEMooNVVSooNFXgnE}]
            Nous commençons par montrer que \( J(x)\colon (V',\| . \|)\to \eC\) est continue pour chaque \( x\in V\). Soit une suite \( \varphi_k\stackrel{\| . \|}{\longrightarrow}0\). Nous avons :
            \begin{equation}
                J(x)\varphi_k=\varphi_k(x)\leq \| \varphi_k \|\| x \|\to 0
            \end{equation}
            où vous aurez noté l'utilisation du lemme~\ref{LEMooIBLEooLJczmu}.  Cela prouve que \( J(x)\) est continue et donc que \( J\) est bien à valeurs dans \( V''\).
        \item[\ref{ITEMooKURHooZZWpbu}]

            Soit une suite \( x_k\stackrel{V}{\longrightarrow}0\), et étudions \( \| J(x_k) \|\) pour la norme dans \( V''\). Nous posons \( x'_k=x_k/\| x_k \|\) et nous calculons (encore une fois, nous écrivons «\( \eC\)», mais ça pourrait être \( \eR\))
            \begin{equation}
                \| J(x_k) \|=\sup_{\| \varphi \|=1}| J(x_k)\varphi |_{\eC}=\sup_{\| \varphi \|=1}| \varphi(x_k) |=\| x_k \|\sup_{\| \varphi \|=1}| \varphi(x'_k) |\leq \| x_k \|\to 0.
            \end{equation}
            La dernière inégalité pourrait être sans doute une égalité\quext{Écrivez moi si vous en êtes certain.}, mais nous n'en avons pas besoin ici.
    \end{subproof}
\end{proof}

%+++++++++++++++++++++++++++++++++++++++++++++++++++++++++++++++++++++++++++++++++++++++++++++++++++++++++++++++++++++++++++
\section{Mini introduction aux nombres \texorpdfstring{\( p\)}{p}-adiques}
%+++++++++++++++++++++++++++++++++++++++++++++++++++++++++++++++++++++++++++++++++++++++++++++++++++++++++++++++++++++++++++

\subsection{La flèche d'Achille}\label{s:un}

C'est un grand classique que je donne ici juste comme introduction pour montrer que des séries infinies peuvent donner des nombres finis de manière tout à fait intuitive.

Achille tire une flèche vers un arbre situé à $\unit{10}{\meter}$ de lui. Disons que la flèche avance à une vitesse constante de $\unit{1}{\meter\per\second}$. Il est clair que la flèche mettra $\unit{10}{\second}$ pour toucher l'arbre. En $\unit{5}{\second}$, elle aura parcouru la moitié de son chemin. On le note :
\[
\text{temps}=5s+\ldots
\]
Reste \( \unit{5}{\meter}\) à faire. En $\unit{2.5}{\second}$, elle aura fait la moitié de ce chemin chemin, soit $2.5m=\frac{10}{4}m$. On le note :
\[
\text{temps}=\frac{10}{2}s+\frac{10}{4}s+
\]
Reste $2.5m$ à faire. La moitié de ce trajet, soit $\frac{10}{8}m$, est parcouru en $\frac{10}{8}s$; on le note encore, mais c'est la dernière fois !

\[
\text{temps}=\frac{10}{2}s+\frac{10}{4}s+\frac{10}{8}s+
\]
En continuant ainsi à regarder la flèche qui parcours des demi-trajets puis des demi de demi-trajets et encore des demi de demi de demi-trajets, et en sachant que le temps total est $10s$, on trouve :
\[
10\left( \frac{1}{2}+\frac{1}{4}+\frac{1}{8}+\frac{1}{16}+\ldots  \right)=10.
\]
On doit donc croire que la somme jusqu'à l'infini des inverses des puissances de deux vaut $1$ :
\[
   \sum_{n=1}^{\infty}\frac{1}{2^n}=1.
\]
Cela peut être démontré à la loyale.

\subsection{La tortue et Achille}

Maintenant qu'on est convaincu que des sommes infinies peuvent représenter des nombres tout à fait normaux, passons à un truc plus marrant.

Achille, qui marche peinard à $\unit{10}{\meter\per\hour}$, part avec $1m$ d'avance sur une tortue qui avance à $\unit{1}{\meter\per\hour}$. Le temps que la tortue arrive au point de départ d'Achille, Achille aura parcouru $10m$, et le temps que la tortue mettra pour arriver à ce point, eh bien, Achille ne sera déjà plus là : il sera à $100m$. Si la tortue tient bon pendant un temps infini, et si l'on est confiant en le genre de raisonnements faits à la section~\ref{s:un}, elle rattrapera Achille dans
\[
1m+10m+100m+1000m+\ldots
\]
Autant dire que ça ne risque pas d'arriver. Et pourtant, mettons en équations :
\begin{subequations}
    \begin{numcases}{}
        x_{\text{Achile}}(t)=1+10t\\
        x_{\text{tortue}}(t)=t.
    \end{numcases}
\end{subequations}
La tortue rejoint Achille au temps \( t\) tel que \( x_{\text{Achille}(t)}=x_{\text{tortue}}(t)\). Un mini calcul donne $t=-1/9$. Physiquement, c'est une situation logique. Peut-on en déduire une égalité mathématique du style de
\[
1+10+100+1000+\ldots=-\frac{1}{9}\; ???
\]
Là où les choses deviennent jolies, c'est quand on cherche à voir ce que peut bien être la valeur d'un hypothétique $x=1+10+100+1000+\ldots$. En effet, logiquement on devrait avoir
\begin{equation*}
\begin{split}
\frac{x}{10}&=\frac{1}{10}+1+10+100+\ldots\\
            &=\frac{1}{10}+x.
\end{split}
\end{equation*}
Reste à résoudre l'équation du premier degré : $\frac{x}{10}=x+\frac{1}{10}$. Ai-je besoin de donner la solution ?

%---------------------------------------------------------------------------------------------------------------------------
\subsection{Dans les nombres \texorpdfstring{\( p\)}{p}-adiques, c'est vrai}
%---------------------------------------------------------------------------------------------------------------------------

Nous nous proposons d'apprendre sur les nombres \( p\)-adiques juste ce qu'il faut pour montrer que l'égalité
\begin{equation}
    \sum_{k=0}^{\infty}10^k=-\frac{1}{ 9 }
\end{equation}
est vraie dans les nombres \( 5\)-adiques. Tout ce qu'il faut est sur \wikipedia{fr}{Nombre_p-adique}{wikipedia}.

Soit \( a\in \eN\) et \( p\), un nombre premier. La \defe{valuation}{valuation!$p$-adique} \( p\)-adique de \( a\) est l'exposant de \( p\) dans la décomposition de \( a\) en nombres premiers. On la note \( v_p(a)\). Pour un rationnel on définit
\begin{equation}
    v_p\left( \frac{ a }{ b } \right)=v_p(a)-v_p(b)
\end{equation}
La \defe{valeur absolue}{valeur absolue!$p$-adique} \( p\)-adique de \( r\in \eQ\) est
\begin{equation}
    | r |_p=p^{-v_p(r)}.
\end{equation}
Nous posons \( | 0 |_p=0\). De là nous considérons la distance
\begin{equation}
    d_p(x,y)=| x-y |_p.
\end{equation}

\begin{lemma}
    L'espace \( (\eQ,d_p)\) est un espace métrique\footnote{Définition~\ref{DefMVNVFsX}}.
\end{lemma}
\index{topologie!\( p\)-adique}

Nous considérons maintenant \( p=5\). Étant donné que \( a=5\cdot 2\) nous avons \( v_5(10)=1\) et
\begin{equation}
    v_5\left( \frac{1}{ 9 } \right)=v_5(1)-v_5(9)=0.
\end{equation}
Nous avons
\begin{equation}
    \sum_{k=0}^N10^k+\frac{1}{ 9 }=\frac{ 10^{N+1} }{ 9 }
\end{equation}
mais
\begin{equation}
    v_p\left( \frac{ 10^{N+1} }{ 9 } \right)=v_5(10^{N+1})-v_5(9)=N+1.
\end{equation}
Par conséquent
\begin{equation}
    d_5\big( \sum_{k=0}^N10^k,-\frac{1}{ 9 } \big)=| \frac{ 10^{N+1} }{ 9 } |_p=p^{-(N+1)}.
\end{equation}
En passant à la limite,
\begin{equation}
    \lim_{N\to \infty} d_5\big( \sum_{k=0}^N10^k,-\frac{1}{ 9 } \big)=0,
\end{equation}
ce qui signifie que\footnote{Voir la définition~\ref{DefGFHAaOL} de la convergence d'une série dans un espace métrique.}
\begin{equation}
    \sum_{k=0}^{\infty}10^k=-\frac{1}{ 9 }.
\end{equation}
