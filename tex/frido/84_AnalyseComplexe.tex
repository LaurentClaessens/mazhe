%This is part of Mes notes de mathématique
% Copyright (c) 2012-2019, 2022-2025
%   Laurent Claessens
% See the file fdl-1.3.txt for copying conditions.

%+++++++++++++++++++++++++++++++++++++++++++++++++++++++++++++++++++++++++++++++++++++++++++++++++++++++++++++++++++++++++++
\section{Fonctions holomorphes}
%+++++++++++++++++++++++++++++++++++++++++++++++++++++++++++++++++++++++++++++++++++++++++++++++++++++++++++++++++++++++++++

La dérivée complexe est discutée à la section~\ref{SECooJWNOooOgMiWR}, et la définition d'une fonction holomorphe est \ref{DefMMpjJZ}.

%---------------------------------------------------------------------------------------------------------------------------
\subsection{Équations de Cauchy-Riemann}
%---------------------------------------------------------------------------------------------------------------------------

Notons que la formule \eqref{EqYFmoiM} donne un \defe{développement limité}{développement!limité!fonction holomorphe} pour les fonctions holomorphes. Si \( f\) est holomorphe en \( z_0\) alors si \( z\) est dans un voisinage de \( z_0\), il existe une fonction \( s\colon \eR\to \eC\) telle que \( \lim_{t\to 0} s(t)/t=0\) et
\begin{equation}    \label{EqptwBFG}
	f(z)=f(z_0)+f'(z_0)(z-z_0)+s(| z-z_0 |).
\end{equation}

\begin{normaltext}[Quelque (abus de) notations]

	Nous utilisons l'application
	\begin{equation}
		\begin{aligned}
			\varphi\colon \eR^2 & \to \eC       \\
			(x,y)               & \mapsto x+iy.
		\end{aligned}
	\end{equation}
	Nous notons
	\begin{subequations}
		\begin{align}
			\partial_xf & =\partial_1(f\circ\varphi)  \\
			\partial_yf & =\partial_2(f\circ\varphi).
		\end{align}
	\end{subequations}

	Avec ces définitions, nous introduisons les opérateurs\nomenclature[Y]{\( \partial_z\),\( \partial_{\bar z}\)}{dérivées partielles d'une fonction complexe}
	\begin{subequations}
		\begin{align}
			\frac{ \partial  }{ \partial z }=\partial=\frac{ 1 }{2}\left( \frac{ \partial  }{ \partial x }-i\frac{ \partial  }{ \partial y } \right) \\
			\frac{ \partial  }{ \partial \bar z }=\bar\partial=\frac{ 1 }{2}\left( \frac{ \partial  }{ \partial x }+i\frac{ \partial  }{ \partial y } \right)		\label{SUBEQooNYQKooYdLOzB}
		\end{align}
	\end{subequations}
	La raison de ces définition est d'avoir l'égalité
	\begin{equation}
		f'(z_0)=(\partial_zf)(z_0)
	\end{equation}
	que nous verrons dans la proposition \ref{PROPooKXEQooBPjWAH}.
\end{normaltext}

Si \( f\) est une fonction \( \eC\)-dérivable représentée par la fonction \( F=P+iQ\), les équations de Cauchy-Riemann signifient que \( \Delta P=\Delta Q=0\), c'est-à-dire que les composantes de la fonction \( f\) sont harmoniques\footnote{Une fonction \( u\) est harmonique si \( \Delta u=0\).}.

\begin{proposition}[\cite{MonCerveau}]\label{PropkwIQwg}
	Soit un ouvert \( \Omega\subset \eC\). Nous considérons une application \(f \colon \eC\to \eC  \) s'écrivant sous la forme \( f(x+iy)=u(x,y)+iv(x,y)\) où \(u,v \colon \eR^2\to \eR  \) sont des applications réelles.

	Nous avons équivalence entre :
	\begin{enumerate}
		\item		\label{ITEMooEFBVooGFENJw}
		      \( f\) est \( \eC\)-dérivable.
		\item		\label{ITEMooXHZPooTqtLXc}
		      \( f\) est différentiable et sa différentielle est une similitude\footnote{En réalité ce qui est une similitude, c'est la matrice de \(  \varphi\circ df_{z_0}\circ\varphi^{-1} \colon \eR^2\to \eR^2\).} (définition \ref{DEFooSRKYooLkGVUe}).
		\item	\label{ITEMooTFRIooWrkkvG}
		      \( f \) est différentiable et vérifie
		      \begin{subequations}        \label{EqmblExI}
			      \begin{numcases}{}
				      \frac{ \partial u }{ \partial x }=\frac{ \partial v }{ \partial y }\\
				      \frac{ \partial u }{ \partial y }=-\frac{ \partial v }{ \partial x }
			      \end{numcases}
		      \end{subequations}
		\item		\label{ITEMooHKAFooDFdgni}
		      \( f\) est différentiable et \( \partial_{\bar z}f=0\).
	\end{enumerate}
	Si ces conditions sont remplies, nous avons
	\begin{equation}		\label{EQooYVAJooCjdHvX}
		df_{z_0}(z)=\big( \partial_xu+i\partial_xv \big)(x_0,y_0)z
	\end{equation}
	Les équations \eqref{EqmblExI} sont les équations de \defe{Cauchy-Riemann}{Cauchy-Riemann}.
\end{proposition}

\begin{proof}
	En plusieurs étapes.
	\begin{subproof}
		\spitem[\ref{ITEMooEFBVooGFENJw} \( \Leftrightarrow\) \ref{ITEMooXHZPooTqtLXc}]
		%-----------------------------------------------------------
		C'est la proposition \ref{PropKJUDooJfqgYS}.
		\spitem[\ref{ITEMooXHZPooTqtLXc} \( \Rightarrow\) \ref{ITEMooTFRIooWrkkvG}]
		%-----------------------------------------------------------
		Par hypothèse \( f\) est différentiable. Nous allons calculer \( df_{z_0}(z)\). Nous notons \( z_0=x_0+iy_0\) et nous utilisons une des formules du lemme \ref{LemdfaSurLesPartielles} :
		\begin{subequations}		\label{SUBEQSooCFAXooSQOMUu}
			\begin{align}
				df_{z_0} (x+iy) & =\frac{d}{dt} \left[ u(x_0+tx,y_0+ty)+iv(x_0+tx,y_0+ty)  \right]_{t=0}                         \\
				                & = x(\partial_xu)(x_0,y_0)+y(\partial_yu)(x_0,y_0)\nonumber                                     \\
				                & \quad+ix(\partial_xv)(x_0,y_0)+iy(\partial_yv)(x_0,y_0)                                        \\
				                & = x\big( (\partial_xu)(x_0,y_0)+i(\partial_xv)(x_0,y_0) \big)\nonumber                         \\
				                & \quad +iy\big( (\partial_yv)(x_0,y_0)-i(\partial_yu)(x_0,y_0) \big)		\label{SUBEQooQSLGooFbyYcq}
			\end{align}
		\end{subequations}
		Étant donné que la différentielle est linéaire, les coefficients de \( x\) et de \( iy\) doivent être égaux :
		\begin{equation}
			(\partial_xu)(x_0,y_0)+i(\partial_xv)(x_0,y_0) =  (\partial_yv)(x_0,y_0)-i(\partial_yu)(x_0,y_0).
		\end{equation}
		En égalisant les parties réelles et imaginaires, nous trouvons les relations \eqref{EqmblExI}.

		Maintenant que nous savons que les deux coefficients dans \eqref{SUBEQooQSLGooFbyYcq} sont identiques nous avons
		\begin{equation}
			df_{z_0}(z)=(\partial_xu+u\partial_xv)(x_0,y_0)z,
		\end{equation}
		c'est à dire que nous avons prouvé \eqref{EQooYVAJooCjdHvX} au passage.
		\spitem[\ref{ITEMooTFRIooWrkkvG} \( \Rightarrow\) \ref{ITEMooHKAFooDFdgni}]
		%-----------------------------------------------------------
		Il suffit de vérifier en utilisant la définition \ref{SUBEQooNYQKooYdLOzB}, et en remplaçant \( \partial_yu\) et \( \partial_yv\) par ce qu'on peut d'après \eqref{EqmblExI} :
		\begin{subequations}
			\begin{align}
				\partial_{\bar z}(u+iv) & =\frac{ 1 }{2}(\partial_x+i\partial_y)(u+iv)                         \\
				                        & =\frac{ 1 }{2}(\partial_xu+i\partial_xv+i\partial_yu-\partial_yv)    \\
				                        & = \frac{ 1 }{ 2 }(\partial_xu+i\partial_xv-i\partial_xv-\partial_xu) \\
				                        & = 0.
			\end{align}
		\end{subequations}
		\spitem[\ref{ITEMooHKAFooDFdgni} \( \Rightarrow\) \ref{ITEMooEFBVooGFENJw}]
		%-----------------------------------------------------------
		Nous démontrons \ref{ITEMooHKAFooDFdgni} \( \Rightarrow\) \ref{ITEMooXHZPooTqtLXc}. Nous avons déjà montré que \ref{ITEMooXHZPooTqtLXc} implique \ref{ITEMooEFBVooGFENJw}. Nous posons \( f(x+iy)=u(x,y)+iv(x,y)\) et nous allons calculer \( df_{z_0}(x+iy)\) en reprenant le calcul déjà fait dans \eqref{SUBEQSooCFAXooSQOMUu}. Pour alléger les notations, nous notons
		\begin{equation}
			\begin{aligned}[]
				\alpha & =(\partial_xu)(x_0,y_0) & \beta  & =(\partial_xv)(x_0,y_0)  \\
				\gamma & =(\partial_yu)(x_0,y_0) & \delta & =(\partial_yv)(x_0,y_0).
			\end{aligned}
		\end{equation}
		Nous avons :
		\begin{equation}
			df_{z_0}(x+iy)=x(\alpha+i\beta)+iy(\delta-i\gamma)=\alpha x+\gamma y+i(\beta x+\delta y).
		\end{equation}
		Donc
		\begin{equation}
			(\varphi\circ df_{z_0}\circ\varphi^{-1})(x,y)=\begin{pmatrix}
				\alpha x+\gamma y \\
				\beta x+\delta y
			\end{pmatrix}=
			\begin{pmatrix}
				\alpha & \gamma \\
				\beta  & \delta
			\end{pmatrix}\begin{pmatrix}
				x \\
				y
			\end{pmatrix}.
		\end{equation}
		Par ailleurs, l'hypothèse \( \partial_{\bar z}=0\) donne les équations \eqref{EqmblExI}, c'est à dire \( \delta=\alpha\) et \( \gamma=-\beta\). La matrice de \( (\varphi\circ df_{z_0}\circ\varphi^{-1})\) est donc \( \begin{pmatrix}
			\alpha & -\beta \\
			\beta  & \alpha
		\end{pmatrix}\) qui est une similitude
		%-----------------------------------------------------------
	\end{subproof}
\end{proof}

\begin{proposition}[\cite{MonCerveau}]	\label{PROPooKXEQooBPjWAH}
	Si \( f\) est \( \eC\)-dérivable en \( z_0\) nous avons
	\begin{equation}
		f'(z_0)=(\partial_zf)(z_0).
	\end{equation}
\end{proposition}

\begin{proof}
	La proposition \ref{PropKJUDooJfqgYS} dit déjà que \( f'(z_0)z=(df_{z_0})(z)\). Nous devons montrer que \( (\partial_zf)(z_0)z=(df_{z_0})(z)\). Pour cela nous calculons un peu en utilisant la définition de \( \partial_z\) et les relations \eqref{EqmblExI} :
	\begin{subequations}
		\begin{align}
			(\partial_zf)(z_0) & =\frac{ 1 }{2}(\partial_x-i\partial_y)(u+iv)(x_0,y_0)                      \\
			                   & =\frac{ 1 }{2}(\partial_xu+i\partial_xv-i\partial_yu+\partial_yv)(x_0,y_0) \\
			                   & =\frac{ 1 }{2}(\partial_xu+i\partial_xv+i\partial_xv+\partial_xu)(x_0,y_0) \\
			                   & =(\partial_xu+i\partial_xv)(x_0,y_0).
		\end{align}
	\end{subequations}
	L'équation \eqref{EQooYVAJooCjdHvX} nous permet de terminer :
	\begin{equation}
		(\partial_zf)(z_0)=(\partial_xu+i\partial_xv)(x_0,y_0)=df_{z_0}
	\end{equation}
	au sens où \( df_{z_0}\) est l'opérateur sur \( \eC\) qui consiste à multiplier par ce qui est à gauche.
\end{proof}


\begin{proposition}[Cauchy-Riemann en coordonnées polaires\cite{MonCerveau,BIBooUBUAooHyhrlg}]      \label{PROPooAGGMooIVQFQB}
	Soit une fonction \( f\colon \eC\to \eC\) que nous supposons être \( \eC\)-dérivable dans un voisinage de \( r_0 e^{i\theta_0}\) (\( r_0\) et \( \theta_0\) sont des réels). Nous posons
	\begin{equation}
		\begin{aligned}
			\tilde f\colon \eR\times \eR & \to \eC                   \\
			r,\theta                     & \mapsto f(r e^{i\theta}).
		\end{aligned}
	\end{equation}
	Alors
	\begin{enumerate}
		\item       \label{ITEMooRTYYooSTgTAQ}
		      La fonction \( \tilde f\) admet des dérivées partielles dans les deux directions en \( (r_0,\theta_0)\).
		\item       \label{ITEMooDHXTooBjxwjY}
		      Les dérivées partielles de \( \tilde f\) sont liées par la relation
		      \begin{equation}
			      \frac{ \partial \tilde f }{ \partial \theta }(r_0,\theta_0)=ir_0\frac{ \partial \tilde f }{ \partial r }(r_0,\theta_0).
		      \end{equation}
		\item       \label{ITEMooUUXTooZoDMHI}
		      La fonction \( \tilde f\) est de classe \(  C^{\infty}\).
	\end{enumerate}
\end{proposition}

\begin{proof}
	Nous considérons les fonctions réelles \( u\) et \( v\) donnant les parties réelles et imaginaires de \( f\) :
	\begin{equation}
		f(x+iy)=u(x,y)+iv(x,y).
	\end{equation}
	Nous avons:
	\begin{equation}
		\begin{aligned}[]
			\frac{ \partial \tilde f }{ \partial r }(r_0,\theta_0)
			 & =\frac{ \partial u }{ \partial x }\big( r_0\cos(\theta_0),r_0\sin(\theta_0) \big)\cos(\theta_0)        \\
			 & \quad+\frac{ \partial u }{ \partial y }\big( r_0\cos(\theta_0),r_0\sin(\theta_0) \big)\sin(\theta_0)   \\
			 & \quad+ i\frac{ \partial v }{ \partial x }\big( r_0\cos(\theta_0),r_0\sin(\theta_0) \big)\cos(\theta_0) \\
			 & \quad+ i\frac{ \partial v }{ \partial y }\big( r_0\cos(\theta_0),r_0\sin(\theta_0) \big)\sin(\theta_0)
		\end{aligned}
	\end{equation}
	et
	\begin{equation}
		\begin{aligned}[]
			\frac{ \partial \tilde f }{ \partial \theta }(r_0,\theta_0) & =
			-\frac{ \partial u }{ \partial x }\big( r_0\cos(\theta_0),r_0\sin(\theta_0) \big)r_0\sin(\theta_0)                                                                      \\
			                                                            & \quad+\frac{ \partial u }{ \partial y }\big( r_0\cos(\theta_0),r_0\sin(\theta_0) \big)r_0\cos(\theta_0)   \\
			                                                            & \quad-\frac{ \partial v }{ \partial x }\big( r_0\cos(\theta_0),r_0\sin(\theta_0) \big)r_0\sin(\theta_0)   \\
			                                                            & \quad+i\frac{ \partial v }{ \partial y }\big( r_0\cos(\theta_0),r_0\sin(\theta_0) \big)r_0\cos(\theta_0).
		\end{aligned}
	\end{equation}
	Nous exprimons tout en termes de \( \partial_yv\) et \( \partial_xv\) en utilisant les équations de Cauchy-Riemann de la proposition \ref{PropkwIQwg} ainsi que la formule \( \cos(\theta)+i\sin(\theta)= e^{i\theta}\) du lemme \ref{LEMooHOYZooKQTsXW}. Pour simplifier les notations, nous notons \( a_0=\big( r_0\cos(\theta_0), r_0\sin(\theta_0) \big)\in \eR^2 \). Après quelques calculs :
	\begin{subequations}
		\begin{align}
			(\partial_r\tilde f)(r_0,\theta_0) & = e^{i\theta_0}\frac{ \partial v }{ \partial y}\big( r_0\cos(\theta_0),r_0\sin(\theta_0) \big)
			+i e^{i\theta_0}\frac{ \partial v }{ \partial x}\big( r_0\cos(\theta_0),r_0\sin(\theta_0) \big)                                                    \\
			                                   & = e^{i\theta_0}\left( \frac{ \partial v }{ \partial y }(a_0)+i\frac{ \partial v }{ \partial x }(a_0) \right),
		\end{align}
	\end{subequations}
	et
	\begin{equation}
		(\partial_{\theta}\tilde f)(r_0,\theta_0)=r_0 e^{i\theta_0}\left( i\frac{ \partial v }{ \partial y }(a_0)-\frac{ \partial v  }{ \partial x }(a_0) \right)
	\end{equation}
	En comparant les deux, nous trouvons
	\begin{equation}
		(\partial_{\theta}\tilde f)(r_0,\theta_0)=r_0i(\partial_r\tilde f)(r_0,\theta_0).
	\end{equation}
\end{proof}


\begin{proposition}     \label{PROPooCHUEooYsGcQK}
	Si \( f\colon \eC\to \eC\) est holomorphe, alors nous avons
	\begin{equation}
		df_{z_0}=(\partial_zf)(z_0)
	\end{equation}
	au sens où l'opérateur linéaire \( df_{z_0}\colon \eC\to \eC\) est l'opération de multiplication par le nombre complexe \( (\partial_zf)(z_0)\).
\end{proposition}

\begin{proof}
	Soit \( f(x+iy)=f_1(x,y)+if_2(x,y)\) une fonction holomorphe\footnote{Définition \ref{DefMMpjJZ}.}. Les fonctions réelles \( f_1\) et \( f_2\) sont assujetties aux équations de Cauchy-Riemann de la proposition~\ref{PropkwIQwg} :
	\begin{subequations}
		\begin{numcases}{}
			\partial_xf_1=\partial_yf_2\\
			\partial_xf_2=-\partial_yf_1.
		\end{numcases}
	\end{subequations}
	Nous avons, en recourant à un petit abus de notation entre \( f_i\colon \eR^2\to \eR\) et \( f_i\colon \eC\to \eR\) :
	\begin{subequations}
		\begin{align}
			df_{z_0}(u) & =\Dsdd{ f(z_0+tu) }{t}{0}                                                         \\
			            & =\Dsdd{ f_1(z_0+tu)+if_2(z_0+tu) }{t}{0}                                          \\
			            & =\partial_xf_1u_1+\partial_yf_1u_2+i\big( \partial_xf_2u_1+\partial_yf_2u_2 \big) \\
			            & =\begin{pmatrix}
				               \partial_xf_1 & \partial_yf_1 \\
				               \partial_xf_2 & \partial_yf_2
			               \end{pmatrix}\begin{pmatrix}
				                            u_1 \\
				                            u_2
			                            \end{pmatrix}                                                    \\
			            & =\begin{pmatrix}
				               \partial_xf_1  & \partial_yf_1 \\
				               -\partial_yf_1 & \partial_xf_1
			               \end{pmatrix}
			\begin{pmatrix}
				u_1 \\
				u_2
			\end{pmatrix}.
		\end{align}
	\end{subequations}
	En utilisant le lemme~\ref{LEMooJNFEooZCbJMo} nous reconnaissons la matrice de multiplication par le nombre \( \partial_xf_1-i\partial_yf_1\). Or justement,
	\begin{equation}
		\partial_zf=\frac{ 1 }{2}\left( \frac{ \partial  }{ \partial x }-i\frac{ \partial  }{ \partial y } \right)f=\frac{ 1 }{2}\big( \partial_xf_1+i\partial_xf_2-i\partial_yf_1+\partial_yf_2 \big),
	\end{equation}
	qui se réduit à \( \partial_xf_1-i\partial_yf_1\) lorsque nous y appliquons les équations de Cauchy-Riemann.
\end{proof}

%--------------------------------------------------------------------------------------------------------------------------- 
\subsection{Intégrale sur un chemin dans \( \eC\)}
%---------------------------------------------------------------------------------------------------------------------------

\begin{definition}      \label{DEFooBPLJooZwsmxi}
	Si nous avons une application \( \gamma\colon \mathopen[ a , b \mathclose]\to \eC\) et une fonction \( f\colon \eC\to \eC\), nous définissons
	\begin{equation}
		\int_{\gamma}f(z)dz=\int_a^b(f\circ \gamma)(t)\gamma'(t)dt
	\end{equation}
	pour tous les couples \( (f,\gamma)\) pour lesquels le membre de droite a un sens.
\end{definition}

\begin{normaltext}
	Vous noterez que cette définition n'est pas exactement la même que celle \ref{DEFooFAYUooCaUdyo} d'une intégrale curviligne en analyse réelle. Cette dernière demande de prendre la norme de \( \gamma'\), alors qu'ici nous la gardons telle quelle. D'ailleurs, gardez en tête que \( \gamma\) est une fonction à valeurs dans \( \eC\). Donc \( \gamma'(t)\) peut très bien être un nombre complexe.
\end{normaltext}

\begin{propositionDef}[Intégrale sur un chemin \( C^1\) par morceaux]       \label{PROPooCUBTooZDcdHX}
	Soit un chemin \( C^1\) par morceaux \( \gamma\colon \mathopen[ a , b \mathclose]\to \eC\). Nous considérons deux subdivisions \( a=t_1<t_2<\ldots t_n=b\) et \( a=u_1<u_2<\ldots <u_m=b\) telles que \( \gamma\) soit de classe \( C^1\) sur les intervalles \( \mathopen] t_i , t_{i+1} \mathclose[\) et \( \mathopen] u_j , u_{j+1} \mathclose[\). Alors
	\begin{equation}
		\sum_{i=1}^{n-1}\int_{\gamma\colon \mathopen] t_i , t_{i+1} \mathclose[\to \eC}f= \sum_{j=1}^{m-1}\int_{\gamma\colon \mathopen] u_j , u_{j+1} \mathclose[\to \eC}f.
	\end{equation}
	Cette valeur est notée
	\begin{equation}
		\int_{\gamma}f.
	\end{equation}
\end{propositionDef}

%---------------------------------------------------------------------------------------------------------------------------
\subsection{Intégrales sur des chemins fermés}
%---------------------------------------------------------------------------------------------------------------------------

\begin{lemma}       \label{LemtpEOmi}
	Si \( g\) est une fonction continue dans un ouvert \( \Omega\subset \eC\) et si \( g\) admet une primitive complexe sur \( \Omega\) alors
	\begin{equation}
		\int_{\gamma}g(z)dz=0
	\end{equation}
	pour tout chemin fermé \( \gamma\) de classe \( C^1\) contenu dans \( \Omega\).
\end{lemma}

\begin{proof}
	Nommons \( G\) une primitive de \( g\). Par définition,
	\begin{subequations}
		\begin{align}
			\int_{\gamma}g & =\int_{\gamma}G'                             \\
			               & =\int_0^1G'\big( \gamma(t) \big)\gamma'(t)dt \\
			               & =\int_0^1 (G\circ \gamma)'(t)dt              \\
			               & =G(\gamma(1))-G\big( \gamma(0) \big)         \\
			               & =0
		\end{align}
	\end{subequations}
	parce que le chemin est fermé : \( \gamma(0)=\gamma(1)\).
\end{proof}

\begin{lemma}[Goursat\cite{Holomorphieus}]  \label{LemwbwbUR}
	Soit \( \Omega\) un ouvert dans \( \eC\) et \( f\) une fonction continue sur \( \Omega\), holomorphe sur \( \Omega\) moins éventuellement un point (nommé \( z_1\in\Omega\)). Soit \( T\), un triangle\footnote{Nous considérons ici le triangle «plein».} fermé inclus dans \( \Omega\). Alors nous avons
	\begin{equation}
		\int_{\partial T}f(z)dz=0.
	\end{equation}
\end{lemma}

\begin{proof}
	Nous notons \( \gamma=\partial T\). Dans la suite nous allons définir une suite de triangles \( T^{(n)}\) et nous noterons \( \gamma_n=\partial T^{(n)}\) avec une orientation que nous allons expliquer. Pour commencer nous posons \( T^{(0)}=T\) et \( \gamma_0=\partial T^{(0)}\).

	Nous considérons le cas \( z_1\notin T\), et nous posons
	\begin{equation}
		c=l(\gamma)^{-2}| \int_{\gamma}f |.
	\end{equation}
	Notre objectif est de montrer que \( c=0\). Soit \( A,B,C\) les trois sommets du triangle; nous divisons le triangle de la façon suivante. D'abord nous considérons les points \( A',B',C'\) respectivement milieux de \( BC\), \( AC\) et \( AB\). En traçant le triangle \( A'B'C'\), nous construisons quatre triangles que nous nommons \( T^{(0)}_i\). Le théorème de Thalès\footnote{Théorème \ref{THOooFMMLooLmAnAd}.} assure que le périmètre de chacun des quatre triangles est la moitié du périmètre du grand triangle \( T\).

	Sur \( T\) nous choisissons l'orientation \( ABC\). De façon à être «compatible», nous choisissons les orientations \( AC'B'\), \( BA'C'\) et \( A'CB'\). La somme de ces trois triangles donne \( T\) plus le triangle \( A'C'B'\). Par conséquent nous choisissons sur le triangle central l'orientation (inverse) \( A'B'C'\) de façon à avoir
	\begin{equation}
		\int_{\gamma}f=\sum_{i=1}^4\int_{\partial T^{(0)}_i}f.
	\end{equation}
	Cela implique que pour au moins un des quatre triangles (disons \( T^{(0)}_k\) pour fixer les idées) nous ayons
	\begin{equation}
		\int_{\partial T^{(0)}_k}f\geq \frac{1}{ 4 }\int_{\partial T^{(0)}}f
	\end{equation}
	Nous notons \( T^{(1)}\) ce triangle. Comme noté précédemment nous avons
	\begin{equation}
		l(\partial T^{(1)})=\frac{ 1 }{2}l(\partial T^{(0)}),
	\end{equation}
	et donc
	\begin{equation}
		l(\gamma_1)^{-2}| \int_{\gamma_1} f|=4l(\gamma_0)^{-2}| \int_{\gamma_1}f |\geq 4l(\gamma_0)^{-2}\frac{1}{ 4 }| \int_{\gamma_0}f |=c.
	\end{equation}
	En répétant le procédé nous construisons une suite de triangles \( T^{(n)}\) qui satisfont toujours
	\begin{equation}
		l(\partial T^{(n)})=\frac{1}{ 2^n }l(\partial T^{(0)}).
	\end{equation}
	Ces triangles forment une suite de fermés emboités dont le diamètre tend vers zéro. Leur intersection contient donc exactement un point (lemme~\ref{LemdCOMQM}) que nous nommons \( z_0\) (et qui appartient évidemment à \( \Omega\)). Étant donné que \( f\) est holomorphe nous utilisons le développement limité \eqref{EqptwBFG} autour de \( z_0\) :
	\begin{equation}
		f(z)=f(z_0)+f'(z_0)(z-z_0)+s(| z-z_0 |)(z-z_0)
	\end{equation}
	avec \( \lim_{t\to 0} s(t)=0\). Nous posons \( g(z)=f(z_0)+f'(z_0)(z-z_0)\) et nous considérons \( \epsilon>0\). Soit \( \alpha>0\) tel que
	\begin{equation}
		| f(z)-g(z) |<\epsilon| z-z_0 |
	\end{equation}
	pour tout \( | z-z_0 |<\alpha\). Le \( \alpha\) à choisir pour obtenir cet effet est celui qui donne \( s(| z-z_0 |)<\epsilon\). Soit \( N\in \eN\) tel que \( l(\gamma_n)<\alpha\) pour tout \( n>N\). D'autre part, deux points dans un triangle sont toujours à distance moindre que la longueur d'un côté, donc pour tout \( z\in T^{(n)}\) nous avons \( | z-z_0 |<\alpha\) et par conséquent pour tout \( z\) dans \( T^{(n)}\) nous avons
	\begin{equation}
		| f(z)-g(z) |<\epsilon| z-z_0 |.
	\end{equation}
	Notons que la fonction \( g\) est une dérivée : c'est la dérivée de la fonction
	\begin{equation}
		G(z)=zf(z_0)+\frac{ 1 }{2}f'(z_0)(z-z_0)^2.
	\end{equation}
	Par conséquent nous avons
	\begin{equation}
		\int_{\gamma_n}g=0
	\end{equation}
	par le lemme~\ref{LemtpEOmi}. Nous avons donc
	\begin{subequations}
		\begin{align}
			| \int_{\gamma_n}f | & =|\int_{\gamma_n}(f-g)|                                 \\
			                     & \leq l(\gamma_n)\max\{ | f(z)-g(z) |\tq z\in T^{(n)} \} \\
			                     & \leq \epsilon l(\gamma_n)^2,
		\end{align}
	\end{subequations}
	et par conséquent
	\begin{equation}
		c\leq l(\gamma_n)^{-2}| \int_{\gamma_n}f |\leq \epsilon,
	\end{equation}
	ce qui signifie que \( c=0\) parce que \( \epsilon\) est arbitraire. Nous avons donc prouvé le lemme de Goursat dans le cas où le point de non holomorphie \( z_1\) est en dehors de \( T\).

	Si \( z_1\) est sur un côté, disons sur le côté \( AB\), alors nous considérons un vecteur \( v\in \eC\) tel que \( T_{\epsilon}=T+\epsilon v\) ne contienne \( z_1\) pour aucun \( \epsilon\). Le vecteur \( v=z_1-C\) fait par exemple l'affaire. En vertu du point précédent nous avons
	\begin{equation}
		\int_{\partial T_{\epsilon}}f=0
	\end{equation}
	pour tout \( \epsilon>0\). Étant donné que la fonction \( f\) est continue (y compris en \( z_1\)), l'intégrale sur \( \partial T\) est également nulle.

	Si maintenant le point \( z_1\) est à l'intérieur de \( T\) nous décomposons \( T\) en trois triangles ayant \( z_1\) comme sommet commun. Si nous considérons les orientations \( Az_1C\), \( ABz_1\) et \( BCz_1\), alors nous avons
	\begin{equation}
		\int_Tf=\int_{Az_1C}f+\int_{ABz_1}f+\int_{BCz_1}f,
	\end{equation}
	alors que par le point précédent les trois intégrales du membre de droite sont nulles.
\end{proof}

\begin{proposition}[\cite{Holomorphieus}]   \label{PrpopwQSbJg}
	Soient \( \Omega\) un ouvert étoilé et \( f\) une fonction holomorphe sur \( \Omega\) sauf éventuellement en un point \( z_1\) où \( f\) est seulement continue. Alors si \( \gamma\) est un chemin fermé dans \( \Omega\), nous avons
	\begin{equation}
		\int_{\gamma}f=0.
	\end{equation}
\end{proposition}

\begin{proposition}     \label{PropRZCKeO}
	Si \( f(z)=\sum_na_nz^n\) a pour rayon de convergence \( R\), alors \( f\) est \( \eC\)-dérivable et nous pouvons dériver terme à terme dans la boule ouverte \( B(0,R)\).
\end{proposition}

\begin{proof}
	Cela est exactement la proposition\quext{Je ne suis pas convaincu que ça s'applique à \( \eC\).}\ref{ProptzOIuG}.
	%TODOooXSZPooFlVxHl. Je dois utiliser un théorème pour les fonctions sur C. Le ProptzOIuG n'est pas adapté.
\end{proof}

%--------------------------------------------------------------------------------------------------------------------------- 
\subsection{Homotopie entre applications}
%---------------------------------------------------------------------------------------------------------------------------

\begin{definition}[homotopie entre applications]       \label{DEFooPJKLooCvgxsu}
	Soient des espaces topologiques \( X\) et \( Y\). Deux applications continues \( f_1,f_2\colon X\to Y\) sont \defe{homotopes}{applications homotopes} si il existe une application continue \( H\colon \mathopen[ 0 , 1 \mathclose]\times X\to Y \) telle que pour tout \( x\in X \) nous avons
	\begin{subequations}
		\begin{align}
			H(0,x) & =f_1(x)  \\
			H(1,x) & =f_2(x).
		\end{align}
	\end{subequations}
\end{definition}

\begin{lemma}       \label{LEMooMGFZooGOaGYl}
	La relation «être homotope à\footnote{Définition \ref{DEFooPJKLooCvgxsu}.}»  est une relation d'équivalence sur \( C(X,Y)\).
\end{lemma}

\begin{proof}
	Pour obtenir \( f\sim f\), il suffit de prendre \( H(s,x)=f(x)\).

	Si \( f\sim g\), nous considérons l'homotopie \( H\colon \mathopen[ 0 , 1 \mathclose]\times X\to Y\). Alors l'application
	\begin{equation}
		\begin{aligned}
			M\colon \mathopen[ 0 , 1 \mathclose]\times X & \to Y            \\
			(t,x)                                        & \mapsto H(1-t,x)
		\end{aligned}
	\end{equation}
	est une homotopie pour \( g\sim f\).

	Si \( f\sim g\) et \( g\sim f\), nous avons les applications continues \( H\) et \( M\) telles que
	\begin{subequations}
		\begin{align}
			H(0,x) & =f(x) & H(1,x) & =g(x)  \\
			M(0,x) & =g(x) & M(1,x) & =h(x).
		\end{align}
	\end{subequations}
	L'application
	\begin{equation}
		\begin{aligned}
			S\colon \mathopen[ 0 , 1 \mathclose]\times X & \to Y                                                                                 \\
			(t,x)                                        & \mapsto \begin{cases}
				                                                       H(2t,x)   & \text{si } t\in\mathopen[ 0 , \frac{ 1 }{2} \mathclose[ \\
				                                                       M(2t-1,x) & \text{si }t\in\mathopen[ \frac{ 1 }{2} , 1 \mathclose].
			                                                       \end{cases}
		\end{aligned}
	\end{equation}
	Nous vérifions que \( S\) est continue en vérifiant la valeur en \( t=1/2\). De plus
	\begin{equation}
		S(0,x)=H(0,x)=f(x)
	\end{equation}
	et
	\begin{equation}
		S(1,x)=M(2-1,x)=M(1,x)=h(x).
	\end{equation}
\end{proof}


\begin{lemma}[\cite{MonCerveau}]   \label{LEMooMJKEooCaVhjD}
	Soient des espaces topologiques \( X\) et \( Y\). Si \( Y\) est connexe par arcs, alors toutes les applications constantes \( X\to Y\) sont homotopes.
\end{lemma}

\begin{proof}
	Soient les applications constantes \( u(x)=u_0\) et \( v(x)=v_0\). Étant donné que \( Y\) est connexe par arcs, il existe une application continue \( \gamma\colon \mathopen[ 0 , 1 \mathclose]\to Y\) telle que \( \gamma(0)=u_0\) et \( \gamma(1)=v_0\). Alors l'application
	\begin{equation}
		\begin{aligned}
			H\colon \mathopen[ 0 , 1 \mathclose]\times X & \to Y             \\
			(t,x)                                        & \mapsto \gamma(t)
		\end{aligned}
	\end{equation}
	est une homotopie entre les applications \( u\) et \( v\).
\end{proof}

\begin{proposition}[\cite{BIBooQKARooMHqitK}]       \label{PROPooNABDooFtKukO}
	Soit un compact \( K\) de \( \eR^n\). Deux applications \( f,g\colon K\to \eC^*\) sont homotopes dans \( \eC^*\) si et seulement si \( f/g\) est homotope à la fonction
	\begin{equation}
		\begin{aligned}
			u\colon K & \to \eC    \\
			x         & \mapsto 1.
		\end{aligned}
	\end{equation}
\end{proposition}

\begin{proof}
	Supposons que \( H\colon \mathopen[ 0 , 1 \mathclose]\times K\to \eC^*\) est une homotopie entre \( f\) et \( g\). Dans ce cas, l'application
	\begin{equation}
		\begin{aligned}
			M\colon \mathopen[ 0 , 1 \mathclose]\times K & \to \eC^*                       \\
			(t,x)                                        & \mapsto \frac{ H(t,x) }{ g(x) }
		\end{aligned}
	\end{equation}
	est une homotopie entre \( f/g\) et \( u\). En effet
	\begin{equation}
		M(0,x)=\frac{ H(0,x) }{ g(x) }=\frac{ f(x) }{ g(x) }
	\end{equation}
	et
	\begin{equation}
		M(1,x)=\frac{ H(1,x) }{ g(x) }=1=u(x).
	\end{equation}

	Dans l'autre sens, si \( H\) est une homotopie entre \( f/g\) et \( u\), alors l'application \( M(t,x)=H(t,x)g(x)\) est une homotopie entre \( f\) et \( g\).
\end{proof}


%--------------------------------------------------------------------------------------------------------------------------- 
\subsection{Intégrale et homotopie}
%---------------------------------------------------------------------------------------------------------------------------

\begin{lemma}[\cite{BIBooQKARooMHqitK}]       \label{LEMooOFCEooIsuchR}
	Soit un rectangle fermé \( R\subset \eR^2\). Si l'application \( H\colon R\to \Omega\) est continue, alors pour toute fonction continue \( f\colon \Omega\to \eC\) nous avons
	\begin{equation}
		\int_{\partial H}f=0
	\end{equation}
	où \( \partial H\) désigne la frontière de \( H(R)\) dans \( \eC\), et l'intégrale est celle de la définition \ref{PROPooCUBTooZDcdHX}.
\end{lemma}

\begin{proof}
	La frontière \( \partial H\) peut être décomposée en \( 4\) parties de classe \( C^1\) :
	\begin{enumerate}
		\item
		      \begin{equation}
			      \begin{aligned}
				      \alpha_1\colon \mathopen[ 0 , 1 \mathclose] & \to \Omega      \\
				      t                                           & \mapsto H(t,a),
			      \end{aligned}
		      \end{equation}
		\item
		      \begin{equation}
			      \begin{aligned}
				      \alpha_2\colon \mathopen[ a , b \mathclose] & \to \Omega     \\
				      t                                           & \mapsto H(1,t)
			      \end{aligned}
		      \end{equation}
		\item
		      \begin{equation}
			      \begin{aligned}
				      \alpha_3\colon \mathopen[ 0 , 1 \mathclose] & \to \Omega       \\
				      t                                           & \mapsto H(1-t,b)
			      \end{aligned}
		      \end{equation}
		\item
		      \begin{equation}
			      \begin{aligned}
				      \alpha_4\colon \mathopen[ a , b \mathclose] & \to \Omega               \\
				      t                                           & \mapsto H\big(0, b+a-t).
			      \end{aligned}
		      \end{equation}
	\end{enumerate}
	Étudions les diverses parties \( \alpha_i\).
	\begin{subproof}
		\spitem[Pour \( \alpha_2\)]
		% -------------------------------------------------------------------------------------------- 
		Première observation : \( \alpha_2=\gamma_1\).
		\spitem[Pour \( \alpha_4\)]
		% -------------------------------------------------------------------------------------------- 
		Nous avons \( \alpha_4=\gamma_0(b+a-t)\) et donc
		\begin{equation}        \label{EQooFWIQooJALDpO}
			\int_{\alpha_4}f=\int_a^b(f\circ \alpha_4)(t)\alpha_4'(t)= -\int_a^b(f\circ \gamma_0)(b+a-t)\gamma_0'(b+a-t).
		\end{equation}
		Nous utilisons le changement de variables\footnote{Théorème \ref{THOooUMIWooZUtUSg}\ref{ITEMooAJGDooGHKnvj}.}
		\begin{equation}
			\begin{aligned}
				\phi\colon \mathopen[ a , b \mathclose] & \to \mathopen[ a , b \mathclose] \\
				t                                       & \mapsto b+a-t .
			\end{aligned}
		\end{equation}
		Le jacobien \( | J_{\phi} |\) est égal à \( 1\) et donc nous continuons \eqref{EQooFWIQooJALDpO} de la façon suivante :
		\begin{subequations}
			\begin{align}
				\int_{\alpha_4}f & =-\int_a^b(f\circ\gamma_0)\big( \phi(t) \big)\gamma'\big( \phi(t) \big)dt \\
				                 & =-\int_a^b(f\circ\gamma_0)(t)\gamma_0'(t)dt                               \\
				                 & =-\int_{\gamma_0}f.
			\end{align}
		\end{subequations}
		\spitem[Lien entre \( \alpha_1\) et \( \alpha_3\)]
		% -------------------------------------------------------------------------------------------- 
		Il y a deux possibilités : soit \( \gamma_1\) et \( \gamma_2\) sont des lacets, soit ce sont des chemins normaux.
		\begin{subproof}
			\spitem[Si ils sont des lacets]
			% -------------------------------------------------------------------------------------------- 
			Alors pour tout \( s\), l'application
			\begin{equation}
				\begin{aligned}
					\Gamma_s\colon \mathopen[ a , b \mathclose] & \to C          \\
					t                                           & \mapsto H(s,t)
				\end{aligned}
			\end{equation}
			est un lacet. En particulier \( H(s,a)=H(s,b)\). Donc \( \alpha_3(t)=H(1-t,b)=\alpha_1(1-t)\). Dans ce cas nous avons
			\begin{subequations}
				\begin{align}
					\int_{\alpha_3}f & =\int_a^bf\big( \alpha_3t \big)\alpha_3'(t)                                              \\
					                 & =\int_a^bf\big( \alpha_1(1-t) \big)(-)\alpha_1'(1-t)                                     \\
					                 & =-\int_a^b(f\circ\alpha_1)(t)dt                      & \text{chm. var. \( \phi(t)=1-t\)} \\
					                 & =-\int_{\alpha_1}f.
				\end{align}
			\end{subequations}
			Nous avons alors le calcul
			\begin{equation}
				0=\int_{\partial H}f=\int_{\alpha_1}f+\int_{\gamma_1}f+\underbrace{\int_{\alpha_3}f}_{-\int_{\alpha_1}f}-\int_{\gamma_0}f=\int_{\gamma_1}f-\int_{\gamma_0}f.
			\end{equation}
			Le lemme est prouvé.
			\spitem[Si \( \gamma_0\) et \( \gamma_1\) sont des chemins]
			Dans ce cas nous avons une homotopie à extrémités fixées. Donc \( \alpha_1\) et \( \alpha_3\) sont des chemins constants, et les intégrales dessus sont nulles et
			\begin{equation}
				0=\int_{\partial H}f=\underbrace{\int_{\alpha_1}f}_{=0}+\int_{\gamma_1}f+\underbrace{\int_{\alpha_3}f}_{=0}-\int_{\gamma_0}f=\int_{\gamma_1}f-\int_{\gamma_0}f.
			\end{equation}
		\end{subproof}
		Et le lemme est prouvé.
	\end{subproof}
\end{proof}


\begin{theorem}[Cauchy, version homotopique\cite{BIBooQKARooMHqitK,ADEyNiz}]     \label{THOooVTFXooBgvVyD}
	Soit un ouvert \( \Omega\) dans \( \eC\). Soit une fonction holomorphe \( f\colon \Omega\to \eC\).

	Supposons que l'une des deux conditions suivantes soit respectée :
	\begin{enumerate}
		\item
		      Les chemins \( \gamma_0,\gamma_1\colon \mathopen[ a , b \mathclose]\to \Omega \) sont homotopes à extrémités fixées.
		\item       \label{ITEMooESDVooFgVarr}
		      Les lacets \( \gamma_0,\gamma_1\colon \mathopen[ a , b \mathclose]\to \Omega \) sont homotopes.
	\end{enumerate}
	Alors
	\begin{equation}
		\int_{\gamma_0}f=\int_{\gamma_1}f.
	\end{equation}
\end{theorem}

\begin{proof}
	Considérons l'homotopie \( H\colon \mathopen[ 0 , 1 \mathclose]\times \mathopen[ a , b \mathclose]\to \Omega\). Étant donné que \( \mathopen[ 0 ,1  \mathclose]\times \mathopen[ a , b \mathclose]\) est un rectangle dans \( \eR^2\), le lemme \ref{LEMooOFCEooIsuchR} nous indique que
	\begin{equation}
		\int_{\partial H}f=0.
	\end{equation}
	%TODOooBQGFooHRquzz je n'ai pas l'impression que cette preuve soit terminée.
\end{proof}

\begin{corollary}[\cite{ADEyNiz}]   \label{CorGZXzuZR}
	Soient \( a\in \eC\) ainsi que deux chemins \( \gamma_1\) et \( \gamma_2\) homotopes dans \( \eC\setminus\{ a \}\). Alors \( \Ind(\gamma_1,a)=\Ind(\gamma_2,a)\).
\end{corollary}



%--------------------------------------------------------------------------------------------------------------------------- 
\subsection{Théorème de Tietze (espace normal)}
%---------------------------------------------------------------------------------------------------------------------------

\begin{lemma}[\cite{BIBooQKARooMHqitK}]     \label{LEMooCLVAooTaNGJk}
	Soit un espace topologique normal\footnote{Définition \ref{DEFooNNKVooLtzImT}.} \( X\). Soient \( M\in \eR^+\) et \( A\) fermé dans \( X\), et une application continue \( f\colon A\to \mathopen[ -M , M \mathclose]\).

	Il existe une application continue \( g\colon X\to \mathopen[ -M/3 , M/3 \mathclose]\) telle que pour tout \( x\in A\),
	\begin{equation}
		| f(x)-g(x) |<\frac{ 2M }{ 3 },
	\end{equation}
\end{lemma}

\begin{proof}
	Nous divisons \( A\) en trois parties:
	\begin{subequations}
		\begin{align}
			A_- & =f^{-1}\big( \mathopen[ -M , -M/3 \mathclose] \big),  \\
			A_+ & =f^{-1}\big( \mathopen[ M/3 , M/3 \mathclose] \big),  \\
			A_0 & =f^{-1}\big( \mathopen[ -M/3 , M/3 \mathclose] \big).
		\end{align}
	\end{subequations}
	Les parties \( A_+\) et \( A_-\) sont fermées parce que \( f\) est continue. D'autre part, \( X\) est normal, de telle sorte que le théorème d'Urysohn \ref{THOooKYYEooLFcNpg} s'applique.

	Nous considérons donc une application continue \( g_1\colon X\to \mathopen[ 0 , 1 \mathclose]\) telle que \( g_1^{-1}(A_-)=\{ 0 \}\) et \( g_1^{-1}(A_+)=\{ 1 \}\). Nous posons alors
	\begin{equation}
		g(x)=\left( \frac{ 2M }{ 3 } \right)g_1(x)-\frac{ M }{ 3 }.
	\end{equation}
	Vérification des propriétés de \( g\).
	\begin{subproof}
		\spitem[\( g\) est continue]
		Parce que \( g_1\) est continue.
		\spitem[\( g\) prend ses valeurs dans \( \mathopen\lbrack -M/3 , M/3 \mathclose\rbrack\)]
		% -------------------------------------------------------------------------------------------- 
		Parce que \( g_1\) prend ses valeurs dans \( \mathopen[ 0 , 1 \mathclose]\) et que \( t\mapsto (2M/3)t-M/3\) est croissante.
		\spitem[\( | f(x)-g(x) |\) sur \( A\)]
		% -------------------------------------------------------------------------------------------- 
		Si \( x\in A_-\), alors \( g_1(x)=0\) et donc \( g(x)=-M/3\). Donc
		\begin{equation}        \label{EQooEUOKooFLRJjz}
			| f(x)-g(x) |=| f(x)-M/3 |\leq 2M/3
		\end{equation}
		parce que \( f(x)\in \mathopen[ -M/3 , M/3 \mathclose]\).

		Si \( x\in A_+\), alors \( g_1(x)=1\) et donc \( g(x)=M/3\). Même fin de raisonnement qu'en \eqref{EQooEUOKooFLRJjz}.

		Si \( x\in A_0\), alors \( f(x)\in \mathopen[ -M/3 , M/3 \mathclose]\) et \( g(x)\in\mathopen[ -M/3 , M/3 \mathclose]\) et donc encore \( | f(x)-g(x) |\leq 2M/3\).
	\end{subproof}
\end{proof}

\begin{lemma}[\cite{BIBooQKARooMHqitK}]     \label{LEMooSKSNooEdgFcR}
	Soient un espace topologique normal \( X\) ainsi qu'un fermé \( A\) dans \( X\). Nous considérons une fonction continue \( f\colon A\to \mathopen[ -M , M \mathclose]\).

	Il existe une application \( g\colon X\to \mathopen[ -M , M \mathclose]\) continue prolongeant \( f\).
\end{lemma}

\begin{proof}
	Nous allons commencer par construire une suite d'applications \( g_i\colon X\to \eR\) telles que
	\begin{enumerate}
		\item       \label{ITEMooGAIVooQYBCZj}
		      Pour tout \( x\in A\) et pour tout \(n\in \eN\),
		      \begin{equation}
			      \big| f(x)-\sum_{i=1}^rng_i(x) \big|\leq \left( \frac{ 2 }{ 3 } \right)^nM
		      \end{equation}
		\item       \label{ITEMooOLNAooEJPdbV}
		      Pour tout \( x\in X\) et pour tout \( i\),
		      \begin{equation}
			      | g_i(x) |\leq \left( \frac{ 2 }{ 3 } \right)^i\frac{ M }{2}.
		      \end{equation}
	\end{enumerate}
	Nous commençons par construire \( g_1\) à partir du lemme \ref{LEMooCLVAooTaNGJk} appliqué à la fonction \( f\). Nous avons donc une application \( g_1\colon X\to \mathopen[ -M/3 , M/3 \mathclose]\) telle que
	\begin{equation}        \label{EQooGXUIooWnATPw}
		| f(x)-g_1(x) |\leq \frac{ 2M }{ 3 }
	\end{equation}
	pour tout \( x\in A\). Vu que \( g_1\) prend ses valeurs dans \( \mathopen[ -M/3 , M/3 \mathclose]\), elle vérifie la condition \ref{ITEMooOLNAooEJPdbV}. De plus \eqref{EQooGXUIooWnATPw} montre que la condition \ref{ITEMooGAIVooQYBCZj} est vérifiée pour \( n=1\).

	Et c'est parti pour la récurrence. Nous supposons avoir des applications \( g_i\) pour \( i=1,\ldots, k\) qui vérifient la condition \ref{ITEMooOLNAooEJPdbV}, et telles que la condition \ref{ITEMooGAIVooQYBCZj} est satisfaite pour \(  n=1,\ldots, k\). Nous allons maintenant construire \( g_{k+1}\).

	Nous posons
	\begin{equation}
		\begin{aligned}
			h_k\colon A & \to \eR                          \\
			x           & \mapsto f(x)-\sum_{i=1}^kg_i(x).
		\end{aligned}
	\end{equation}
	Par hypothèse de récurrence, la fonction \( h_k\) ne prend pas n'importe quelles valeurs dans \( \eR\), mais
	\begin{equation}
		h_k\colon A\to \mathopen\big[  -\left( \frac{ 2 }{ 3 } \right)^kM  , \left( \frac{ 2 }{ 3 } \right)^kM \mathclose\big].
	\end{equation}
	Nous construisons \( g_{k+1}\) à partir de ce \( h_k\) et du lemme \ref{LEMooCLVAooTaNGJk}. Nous avons donc
	\begin{equation}
		g_{k+1}\colon X\to \mathopen\Big[  -\frac{1}{ 3 }\left( \frac{ 2 }{ 3 } \right)^kM  , \frac{1}{ 3 }\left( \frac{ 2 }{ 3 } \right)^kM \mathclose\Big]
	\end{equation}
	vérifiant
	\begin{equation}
		| h_k(x)-g_{k+1}(x) |\leq \left( \frac{ 2 }{ 3 } \right)^{k+1}M.
	\end{equation}
	La condition \ref{ITEMooGAIVooQYBCZj} est donc maintenant vérifiée jusqu'à \( n=k+1\). Nous vérifions la condition \ref{ITEMooOLNAooEJPdbV} pour \( i=k+1\). Simple calcul :
	\begin{equation}
		| g_{k+1}(x) |\leq \frac{1}{ 3 }\left( \frac{ 2 }{ 3 } \right)^kM=\left( \frac{ 2 }{ 3 } \right)^{k+1}\frac{ M }{ 2 }.
	\end{equation}
	Et voilà pour la définition des applications \( g_i\).

	Vu que \( 2/3<1\), la série \( \sum_{i=1}^{\infty}\| g_i \|_{\infty}\) converge normalement (définition \ref{DefVBrJUxo}). Notons \( g\) la somme. Le lemme \ref{LEMooJZTBooIopLok} donne alors la convergence uniforme \( g_i\stackrel{unif}{\longrightarrow}g\), et le théorème \ref{ThoSerUnifCont} nous assure que \( g\) est continue sur \( X\).

	En ce qui concerne la norme de \( g\), nous avons, en utilisant la formule \eqref{EqRGkBhrX} avec \( q=2/3\),
	\begin{equation}
		\| g \|_{\infty}\leq \sum_{i=1}^{\infty}\left( \frac{ 2 }{ 3 } \right)^i\frac{ M }{2}\leq M.
	\end{equation}
	Donc \( | g(x) |\leq M\) pour tout \( x\in X\).

	Enfin nous vérifions que \( g\) prolonge \( f\). Soit \( x\in A\). Prenez la limite \( n\to \infty\) dans l'inégalité
	\begin{equation}
		| f(x)-\sum_{i=1}^ng_i(x) |\leq \left( \frac{ 2 }{ 3 } \right)^nM.
	\end{equation}
	Nous trouvons que \( | f(x)-g(x) |=0\).
\end{proof}

\begin{theorem}[Théorème de Tietze]     \label{THOooXKGWooFUYlux}
	Soit une partie fermée \( A\) de l'espace normal \( X\). Si la fonction \( f\colon A\to \eR\) est continue, alors elle se prolonge en une fonction continue \( g\colon X\to \eR\).
\end{theorem}

\begin{proof}
	Nous supposons dans un premier temps que \( f\) prenne ses valeurs dans \( \mathopen] -M , M \mathclose[\). À fortiori, elle prend ses valeurs dans \( \mathopen[ -M , M \mathclose]\) et le lemme \ref{LEMooSKSNooEdgFcR} dit qu'il existe un prolongement continu \( g\colon X\to \mathopen[ -M , M \mathclose]\). Nous allons construire à partir de là un prolongement continu \( h\colon X\to \mathopen] -M , M \mathclose[\) de \( f\).

		Nous posons
		\begin{equation}
			B=g^{-1}\big( \{ -M,M \} \big).
		\end{equation}
		Étant donné que \( f(A)\subset\mathopen] -M , M \mathclose[\) , nous avons \( B\cap A=\emptyset\). De plus \( A\) est fermé par hypothèse et \( B\) est fermé en tant qu'image réciproque du fermé \( \{ -M,M \}\) par l'application continue \( g\). Nous pouvons donc appliquer le théorème d'Urysohn \ref{THOooKYYEooLFcNpg}.

		Nous considérons donc une application \( g_1\colon X\to \mathopen[ 0 , 1 \mathclose]\) telle que \( g_1=0\) sur \( B\) et \( g_1=1\) sur \( A\). Enfin nous posons \( h=gg_1\). Cette application prend ses valeurs dans \( \mathopen] -M , M \mathclose[\), est continue et si \( x\in A\) nous avons
		\begin{equation}
			h(x)=g(x)g_1(x)=f(x)\times 1=f(x).
		\end{equation}
		Ceci règle la question si \( f\) prend ses valeurs dans \( \mathopen] -M , M \mathclose[\).

		Nous considérons à présent le cas général \( f\colon A\to \eR\). Soit un homéomorphisme \( \phi\colon \eR\to \mathopen] -M , M \mathclose[\) (par exemple via la proposition \ref{PROPooKPACooElwCbh}). Nous considérons
		\begin{equation}
			\begin{aligned}
				\tilde f\colon A & \to \mathopen] -M , M \mathclose[ \\
				x                & \mapsto (\phi\circ f)(x).
			\end{aligned}
		\end{equation}
		Nous lui appliquons le premier cas pour avoir une fonction \( \tilde g\colon X\to \mathopen] -M , M \mathclose[\) qui prolonge \( \tilde f\). Il suffit maintenant de poser
	\begin{equation}
		\begin{aligned}
			g\colon X & \to \eR                              \\
			x         & \mapsto (\phi^{-1}\circ\tilde g)(x).
		\end{aligned}
	\end{equation}
	Cela est une application continue et si \( x\in A\), nous avons
	\begin{equation}
		g(x)=(\phi^{-1}\circ \tilde g)(x)=(\phi^{-1}\circ \tilde f)(x)=(\phi^{-1}\circ\phi\circ f)(x)=f(x).
	\end{equation}
\end{proof}

\begin{corollary}
	Soit un fermé \( A\) dans un espace normal \( X\). Si \( f\colon A\to \eC\) est continue, alors elle se prolonge en une fonction continue \( g\colon X\to \eC\).
\end{corollary}

\begin{proof}
	Les parties réelles et imaginaires de \( f\) sont continues. Il suffit de leur appliquer le théorème de Tietze \ref{THOooXKGWooFUYlux}.
\end{proof}



%+++++++++++++++++++++++++++++++++++++++++++++++++++++++++++++++++++++++++++++++++++++++++++++++++++++++++++++++++++++++++++ 
\section{Logarithme complexe}
%+++++++++++++++++++++++++++++++++++++++++++++++++++++++++++++++++++++++++++++++++++++++++++++++++++++++++++++++++++++++++++

%--------------------------------------------------------------------------------------------------------------------------- 
\subsection{La fonction argument}
%---------------------------------------------------------------------------------------------------------------------------

Nous savons la définition~\ref{DefJilXoM} de l'exponentielle complexe.

\begin{definition}
	Un \defe{logarithme}{logarithme!dans \( \eC\)} de \( \alpha\in \eC\) est une solution de l'équation \(  e^{z}=\alpha\).
\end{definition}
Notons bien que cela définit \emph{un} logarithme, et non \emph{le} logarithme.

\begin{lemma}       \label{LEMooUMESooJVzeDb}
	Si \( z_1\) et \( z_2\) sont des logarithmes de \( \alpha\) alors il existe \( k\in \eZ\) tel que \( z_1=z_2+2ik\pi\).
\end{lemma}

\begin{proof}
	Nous commençons par déterminer les logarithmes de \( \alpha=1\). Nous avons besoin de \(  e^{a+bi}=1\) (\( a,b\in \eR\)). Nous avons
	\begin{equation}
		e^{a} e^{bi}=1,
	\end{equation}
	et en prenant la norme nous trouvons \( | e^a |=1\), ce qui donne \( a=0\). Ensuite \(  e^{bi}=1\), qui signifie \( b=2k\pi\). Les logarithmes de \( 1\) sont donc les nombres de la forme \( 2ik\pi\).

	Soient maintenant \( z_1\) et \( z_2\) des logarithmes de \( \alpha\). Alors \(  e^{z_1}= e^{z_2}\), donc\footnote{C'est facile de dire «donc». Il faut surtout citer la proposition~\ref{PropdDjisy}\ref{ITEMooRLHCooJTuYKV}.} \(  e^{z_1-z_2}=1\), ce qui signifie que \( z_1-z_2\) est un logarithme de \( 1\). Donc il existe un \( k\in \eZ\) tel que \( z_1-z_2=2ik\pi\).
\end{proof}

\begin{remark}
	Jusqu'ici nous n'avons pas donné de conditions donnant l'existence d'un logarithme. Nous avons seulement supposé des existences et donné des propriétés sur ces hypothétiques objets.
\end{remark}

\begin{definition}[\cite{ooDQKTooXNjklV}]
	Si \( z\in \eC^*\) nous définissons la \defe{valeur principale}{valeur principale} de son argument le nombre \( \theta\in \mathopen] -\pi , \pi \mathclose]\) tel que
	\begin{equation}
		z=| z | e^{i\theta}
	\end{equation}
	Nous le notons \( \arg(z)\)\nomenclature[Y]{\( \arg(z)\)}{La valeur principale de l'argument de \( z\in \eC\)}.
\end{definition}

\begin{normaltext}      \label{NORMooOGHNooYriCBH}
	Il ne faut pas se ruer sur \( \arg(x+iy)=\arctan(y/x)\). Pour rappel, la fonction \( \arctan\) a été définie dans le théorème~\ref{THOooUSVGooOAnCvC}, et elle prend ses valeurs dans \( \mathopen] -\pi/2 , \pi/2 \mathclose[\). La formule \( v(x,y)=\arctan(y/x)\) n'est donc valable que pour \( x>0\). Les valeurs sont :
	\begin{equation}        \label{EQooPJVFooSEKTny}
		\arg(x+iy)=\begin{cases}
			\arctan(y/x)       & \text{si } x>0                   \\
			\pi+\arctan(y/x)   & \text{si } x<0\text{ et }y\geq 0 \\
			-\pi+\arctan(y/x)  & \text{si } x<0 \text{ et }y<0    \\
			\frac{ \pi }{ 2 }  & \text{si } x=0 \text{ et }y>0    \\
			\frac{- \pi }{ 2 } & \text{si } x=0 \text{ et }y<0.
		\end{cases}
	\end{equation}

	Pour \( x>0\) nous avons \( \arg(x+iy)=\arctan(y/x)  \) parce que justement la fonction \( \arctan\) prend ses valeurs en particulier entre \( -\pi\) et \( \pi\). Pour \( x<0\) et \( y>0 \) nous avons \( \arg(x+iy)=\pi+\arctan(y/x)\) (dans ce cas, \( \arctan(y/x)<0\)) et si \( x<0\), \( y<0\) nous avons \( \arg(x+iy)=-\pi+\arctan(y/x)\).
\end{normaltext}

\begin{normaltext}[Les dérivées partielles de la fonction argument]     \label{NORMooMRBEooVtTcIA}
	Vu que nous en aurons besoin plusieurs fois, nous calculons maintenant les dérivées partielles de la fonction
	\begin{equation}
		\begin{aligned}
			\varphi\colon \eR^2 & \to \eR             \\
			(x,y)               & \mapsto \arg(x+iy).
		\end{aligned}
	\end{equation}
	Nous commençons par la dérivée \( \partial_x\varphi(x,y)\). Et il y a de nombreux cas à séparer.
	\begin{subproof}

		\spitem[\( x>0\)]

		Nous avons
		\begin{equation}
			\frac{ \partial \varphi }{ \partial x }(x,y)=\lim_{\epsilon\to 0}\frac{ \arctan(y/(x+\epsilon))-\arctan(y/x) }{ \epsilon },
		\end{equation}
		qui n'est autre que la dérivée de la fonction \( x\mapsto\arctan(y/x)\). Nous pouvons la calculer facilement avec le théorème~\ref{THOooUSVGooOAnCvC}\ref{ITEMooMNHLooOVhIIb} :
		\begin{equation}
			\frac{ \partial \varphi }{ \partial x }(x,y)=-\frac{ y }{ x^2+y^2 }.
		\end{equation}

		\spitem[\( x<0\)]

		Nous avons
		\begin{equation}
			\frac{ \partial \varphi }{ \partial x }(x,y)=\lim_{\epsilon\to 0}\frac{ \pm\pi+\arctan(y/(x+\epsilon))-\big( \pm\pi+\arctan(y/x) \big) }{ \epsilon }
		\end{equation}
		où les signes \( \pm\) dépendent du signe de \( y\). De toutes façons, les termes en \( \pi\) se simplifient et le calcul est le même que celui du cas \( x>0\). Encore une fois nous avons
		\begin{equation}
			\frac{ \partial \varphi }{ \partial x }(x,y)=-\frac{ y }{ x^2+y^2 }.
		\end{equation}

		\spitem[\( x=0\)]

		Nous devons calculer
		\begin{equation}
			\frac{ \partial \varphi }{ \partial x }(0,y)=\lim_{\epsilon\to 0}\frac{ \arg(\epsilon+ iy)-\arg(iy) }{ \epsilon }.
		\end{equation}
		Il y a quatre cas d'après les signes de \( \epsilon\) (séparer limite à gauche et à droite) et \( y\).

		Si \( \epsilon>0\) et \( y>0\) alors nous avons à faire le calcul
		\begin{equation}
			\lim_{\epsilon\to 0^+}\frac{ \arctan(y/\epsilon)-\pi/2 }{ \epsilon }
		\end{equation}
		qui se traite par la règle de l'Hospital. Cela donne \( -1/y\).

		Les trois autres cas ne se distinguent que par des constantes au numérateur, lesquelles disparaissent en appliquant la règle de l'Hospital\footnote{Nonobstant le fait que ces constantes se mettent bien pour avoir un vrai cas d'indétermination \( 0/0\), sinon la règle de l'Hospital ne s'applique pas.}. Au final,
		\begin{equation}
			\frac{ \partial \varphi }{ \partial x }(0,y)=-\frac{1}{ y }.
		\end{equation}
	\end{subproof}

	Nous avons calculé jusqu'ici :
	\begin{equation}        \label{EQooAOJPooOrvUBR}
		\frac{ \partial \varphi }{ \partial x }(x,y)=\frac{ -y }{ x^2+y^2 }
	\end{equation}
	pour tout \( (x,y)\in \eR^2\setminus\{ (0,0) \}\). En particulier vous avez noté que cette dérivée partielle est continue sur \( \eR^2\setminus\{ (0,0) \}\).

	Nous calculons à présent la dérivée partielle par rapport à \( y\) :
	\begin{equation}
		\frac{ \partial \varphi }{ \partial y }(x,y)=\lim_{\epsilon\to 0}\frac{ \arg(x+iy+i\epsilon)-\arg(x+iy) }{ \epsilon }.
	\end{equation}

	\begin{subproof}

		\spitem[\( x>0\)]

		Nous avons à calculer
		\begin{equation}
			\lim_{\epsilon\to 0}\frac{ \arctan\frac{ y+\epsilon }{ x }-\arctan\frac{ y }{ x } }{ \epsilon },
		\end{equation}
		qui n'est autre que la dérivée de la fonction \( t\mapsto\arctan\frac{ t }{ x }\) en \( t=y\). Résultat :
		\begin{equation}
			\frac{ \partial \varphi }{ \partial y }(x,y)=\frac{ x }{ x^2+y^2 }.
		\end{equation}

		\spitem[\( x<0  \) et \( y\neq 0\)]

		Le calcul à faire est :
		\begin{equation}
			\lim_{\epsilon\to 0}\frac{ \pm\pi+\arctan\frac{ y+\epsilon }{ x }-\left( \pm\pi+\arctan\frac{ y }{ x } \right) }{ \epsilon }
		\end{equation}
		Une chose importante à remarquer est que dans le calcul de la limite nous pouvons supposer que \( y\) et \( y+\epsilon\) aient le même signe, quelle que soit la valeur et le signe de \( \epsilon\) (assez petit). C'est pour cela que les deux termes \( \pm\pi\) arrivent avec le même signe des deux côtés de la différence, et se simplifient. Nous tombons sur une limite déjà faite et
		\begin{equation}
			\frac{ \partial \varphi }{ \partial y }(x,y)=\frac{ x }{ x^2+y^2 }
		\end{equation}

		\spitem[\( x<0\) et \( y=0\)]

		Vu que \( x<0\) nous avons \( \arg(x)=\pi\) et nous devons calculer
		\begin{equation}
			\lim_{\epsilon\to 0}\frac{ \arg(x+i\epsilon)-\pi }{ \epsilon }.
		\end{equation}
		La limite \( \epsilon\to 0^+\) est classique et donne \( 1/x\).

		Mais la limite \( \epsilon\to 0^-\) n'existe pas :
		\begin{equation}
			\lim_{\epsilon\to 0^-}\frac{ -\pi+\arctan(\epsilon/x)-\pi }{ \epsilon }
		\end{equation}
		n'existe pas.

		Donc
		\begin{equation}
			\frac{ \partial \varphi }{ \partial y }(x,0)
		\end{equation}
		n'existe pas pour \( x<0\).

		\spitem[\( x=0\) et \( y\neq 0\)]

		Le calcul est immédiat
		\begin{equation}
			\lim_{\epsilon\to 0}\frac{ \arg(iy+i\epsilon)-\arg(iy) }{ \epsilon }=0,
		\end{equation}
		donc
		\begin{equation}
			\frac{ \partial \varphi }{ \partial y }(0,y)=0.
		\end{equation}


	\end{subproof}
	En ce qui concerne la continuité, nous avons que \( \partial_y\varphi\) est continue partout sauf sur la demi-droite \(  \{ (x,0)\tq x\leq 0 \}   \) où elle n'existe pas.
\end{normaltext}

%--------------------------------------------------------------------------------------------------------------------------- 
\subsection{Une définition possible du logarithme}
%---------------------------------------------------------------------------------------------------------------------------


\begin{definition}      \label{DEFooWDYNooYIXVMC}
	Nous définissons la fonction \defe{logarithme}{logarithme!complexe} par
	\begin{equation}
		\begin{aligned}
			\ln\colon \eC^* & \to \eC                               \\
			z               & \mapsto \ln\big( | z | \big)+i\arg(z)
		\end{aligned}
	\end{equation}
	où le \( \ln\) à droite est le logarithme usuel sur \( \eR^+\).
\end{definition}

\begin{remark}
	Cette fonction généralise le logarithme déjà vu sur \( \mathopen] 0 , \infty \mathclose[\subset \eR\). En effet pour des valeurs de \( z\) dans cette partie nous avons \( \arg(z)=0\) et \( | z |=z\).
\end{remark}

\begin{lemma}
	Le nombre \( \ln(z)\) est un logarithme de \( z\).
\end{lemma}

\begin{proof}
	Nous avons
	\begin{equation}
		e^{\ln(z)}= e^{\ln| z |} e^{i\arg(z)}=| z | e^{i\arg(z)}=z.
	\end{equation}
	Nous avons utilisé le fait que \(  e^{\ln(x)}=x\) pour \( x\in\eR^+\) et \( | z | e^{i\arg(z)}=z\) par définition de la fonction \( \arg\).
\end{proof}

Notons que si on avait pris d'autres conventions pour définir \( \arg\), nous aurions eu d'autres définitions possibles de \( \ln\).

\begin{example}
	Nous avons
	\begin{equation}
		\ln(-1)=\ln(1)+i\arg(-1).
	\end{equation}
	Mais \( \ln(1)=0\) et \( \arg(-1)=\pi\) (et non \( -\pi\)), donc
	\begin{equation}
		\ln(-1)=i\pi.
	\end{equation}

	C'est cette définition du logarithme qui est prise par Sage, et c'est cela qui lui permet de donner la primitive de \( 1/x\) comme \( \ln(x)\) et non \( \ln(| x |)\), parce que Sage connaît les logarithmes de nombres réels négatifs :
	\lstinputlisting{tex/sage/sageSnip010.sage}
\end{example}

Nous avons jusqu'ici défini une fonction sur \( \eC^*\) qui fait correspondre à chaque nombre complexe un de ses logarithmes. Il reste quelques questions à régler :
\begin{itemize}
	\item Est-ce que cette fonction est continue ? Holomorphe ? (réponses : non et non)
	\item Si non, est-ce qu'il y avait moyen de trouver une définition plus efficace ? (réponse : non)
\end{itemize}

\begin{lemma}       \label{LEMooMUOIooCnoWwq}
	La fonction \( \ln\) n'est pas continue sur \( \mathopen] -\infty , 0 \mathclose]\).
\end{lemma}

\begin{proof}
	Attention à bien comprendre l'énoncé. La fonction
	\begin{equation}
		\begin{aligned}
			f\colon \mathopen] -\infty , 0 \mathclose[ & \to \eC        \\
			x                                          & \mapsto \ln(x)
		\end{aligned}
	\end{equation}
	est continue. D'ailleurs c'est \( \ln(x)=\ln(| x |)+i\pi\). Ce dont il est question dans l'énoncé, c'est de la fonction \( \ln\) vue comme fonction sur \( \eC^*\).

	Soit \( x>0\) dans \( \eR\); nous avons
	\begin{equation}
		\ln(-x)=\ln(x)+i\pi.
	\end{equation}
	Cependant \( \lim_{\substack{\lambda\to 0^-\\\lambda\in \eR}}\ln(-x+\lambda i) \) va valoir \( \ln(| x |)-i\pi\). En effet lorsque \( \lambda<0\) est petit, l'argument de \( -x+\lambda i\) se rapproche de \( -\pi\) (et non de \( \pi\)).

	\begin{center}
		\input{auto/pictures_tex/Fig_CWKJooppMsZXjw.pstricks}
	\end{center}

	Donc
	\begin{equation}
		\lim_{\substack{\lambda\to 0^-\\\lambda\in \eR}}\ln(-x+\lambda i)=\lim \ln(| x+\lambda i |)+i\arg(-x+\lambda i)=\ln(| x |)-i\pi.
	\end{equation}
	Nous n'avons donc pas continuité de la fonction logarithme comme fonction sur \( \eC^*\).
\end{proof}

\begin{theorem}     \label{THOooWUXOooYKvLbJ}
	La restriction
	\begin{equation}
		\ln\colon \eC\setminus\mathopen] -\infty , 0 \mathclose]\to \eC
	\end{equation}
	est holomorphe.
\end{theorem}

\begin{proof}
	Nous allons utiliser la proposition~\ref{PropKJUDooJfqgYS} et considérer la fonction
	\begin{equation}
		\begin{aligned}
			F\colon S & \to \eR^2                                    \\
			(x,y)     & \mapsto \big( \ln(| x+iy |),\arg(x+iy) \big)
		\end{aligned}
	\end{equation}
	où \( S=\eR^2\setminus\{ (x,0)\tq x\leq 0 \}\). Nous devons vérifier que \( F\) est différentiable et que sa différentielle en un point de \( S\) est une similitude.

	Nous posons
	\begin{equation}
		u(x,y)=\ln\big( \sqrt{ x^2+y^2 } \big)
	\end{equation}
	et
	\begin{equation}
		v(x,y)=\arg(x+iy).
	\end{equation}
	Les dérivées partielles de \( u\) ne sont pas très compliquées :
	\lstinputlisting{tex/sage/sageSnip011.sage}
	c'est-à-dire
	\begin{subequations}
		\begin{align}
			\frac{ \partial u }{ \partial x }=\frac{ x }{ x^2+y^2 } \\
			\frac{ \partial u }{ \partial y }=\frac{ y }{ x^2+y^2 }.
		\end{align}
	\end{subequations}

	Pour celles de \( v \) par contre, il faut se poser des questions, par exemples résister à la tentation d'écrire \( v(x,y)=\arctan(y/x)\) et lire~\ref{NORMooOGHNooYriCBH}.

	Nous avons déjà calculé les dérivées partielles de \( v\) dans~\ref{NORMooMRBEooVtTcIA}, et nous avons vu qu'elles étaient continues sur \( \eR^2\) privé de la demi-droite.

	Vu que les dérivées partielles sont continues, le théorème \ref{THOooBEAOooBdvOdr} nous dit que \( F\) est différentiable. La matrice de la différentielle est alors la matrice des dérivées partielles
	\begin{equation}
		\begin{pmatrix}
			\frac{ x }{ x^2+y^2 }  & \frac{ y }{ x^2+y^2 } \\
			\frac{ -y }{ x^2+y^2 } & \frac{ x }{ x^2+y^2 }
		\end{pmatrix},
	\end{equation}
	qui a la forme requise \eqref{EQooWZGKooLDEHGr} pour que la proposition~\ref{PropKJUDooJfqgYS} nous assure que \( \ln\) soit \( \eC\)-dérivable, c'est-à-dire holomorphe.
\end{proof}

%--------------------------------------------------------------------------------------------------------------------------- 
\subsection{Pas plus de continuité}
%---------------------------------------------------------------------------------------------------------------------------

Bon. La fonction logarithme que nous avons définie est holomorphe sur \( \eC^*\) privé d'une demi-droite
\begin{equation}		\label{EQooDOELooBQpPyG}
	U=\{ z\in \eC\tq \imag(z)=0,\real(z)\leq 0 \}.
\end{equation}
Et elle n'est pas continue sur \( U\); elle y est cependant continue «par le haut». Pouvons-nous faire mieux ? Nous allons maintenant prouver quelques résultats d'impossibilité de faire mieux que holomorphe partout sauf une partie pas si petite que ça.

\begin{proposition}
	Il n'existe pas de fonctions continues \( f\colon \eC^*\to \eC\) telle que \(  e^{f(z)}=z\) pour tout \( z\in \eC^*\).
\end{proposition}

\begin{proof}
	Pour tout \( z\), le nombre \( f(z)\) est un logarithme de \( z\). Or \( \ln(z)\) en est également un. Donc par le lemme~\ref{LEMooUMESooJVzeDb}
	\begin{equation}
		f(z)=\ln(z)+2i k(z)\pi
	\end{equation}
	pour une certaine fonction \( k\colon \eC^*\to \eZ\). Sur le domaine d'holomorphie de \( \ln\), les fonctions \( \ln\) et \( f\) étant continues, la fonction \( k\) l'est aussi. Mais une fonction continue à valeurs dans \( \eZ\) est constante (son domaine est connexe).

	Il existe donc \( k\in \eZ\) tel que
	\begin{equation}
		f(z)=\ln(z)+2ik\pi
	\end{equation}
	au moins pour tout \( z\in \eC^*\setminus U\). Une telle fonction ne peut pas être continue sur \( U\) parce que \( \ln\) ne l'est pas.
\end{proof}

Ok. Pas continue sur tout \( \eC\). Mais continue sur un peu plus que \( \eC\) privé de toute une demi-droite ? La proposition suivante répond que bof.

\begin{proposition}
	Soit \( \Omega\) un ouvert de \( \eC\) contenant \( S(0,r)\) (le cercle centré en \( 0\) et de rayon \( r>0\)). Il n'existe pas de fonction continue \( f\colon \Omega\to \eC\) telle que \(  e^{f(z)}=z\) pour tout \( z\in \Omega\).
\end{proposition}

\begin{proof}
	Encore une fois, pour tout \( z\in \Omega\) nous avons
	\begin{equation}
		f(z)=\ln(z)+2i\pi k(z)
	\end{equation}
	pour une certaine fonction \( k\colon \Omega\to \eZ\). Nous considérons la demi-droite \( U\) de \eqref{EQooDOELooBQpPyG}. Sur \( \Omega\setminus U\), la fonction \( \ln\) est continue et \( k\) doit également l'être. Donc \( k\) est constante sur les composantes connexes de \( \Omega\setminus U\).


	Vu que \( S(0,r)\) est compact, on peut le recouvrir par un nombre fini de boules centrées en des points de \( S(0,r)\). En prenant le minimum des rayons de ces boules, nous voyons que \( \Omega\) contient une couronne
	\begin{equation}
		\{ z\in \eC\tq r-\delta\leq | z |\leq r+\delta \}.
	\end{equation}
	Soit le point \( x_0=-r\). C'est un point de \( \Omega\) contenu dans \( U\). Nous allons prouver que \( B(x_0,\delta)\setminus U\) est dans une seule composante connexe de \( \Omega\).

	Soit un point \( z_1\in B(x_0,\delta)\) situé au-dessus de \( U\), et \( z_2\) un point de \( B(x_0,\delta)\) situé en dessous de \( U\). Le cercle \( S(0,r)\) coupe \( B(x_0,\delta)\) en deux points : un au-dessus et un en-dessous de \( U\). On peut lier \( z_1\) au point de «sortie» supérieur de \( S(0,r)\) en restant dans \( B(x_0,\delta)\); ce point est ensuite relié en suivant le cercle au point d'entrée inférieur du cercle dans \( B(x_0,\delta)\). Ce dernier point est lié à \( z_2\) par un chemin restant dans la boule.

	Tout cela pour dire que \( z_1\) et \( z_2\) sont dans la même composante connexe de \( \Omega\) et que \( k(z_1)=k(z_2)\). Il existe donc \( k\in \eZ\) tel que
	\begin{equation}
		f(z)=\ln(z)+2ik\pi
	\end{equation}
	sur \( B(x_0,\delta)\setminus U\). Une telle fonction \( f\) ne peut pas être continue.
\end{proof}

%--------------------------------------------------------------------------------------------------------------------------- 
\subsection{Pas d'unicité : autres déterminations de l'argument}
%---------------------------------------------------------------------------------------------------------------------------

\begin{normaltext}      \label{NORMooFCDOooFDzAjp}
	Nous avons pris la fonction d'argument \( \arg\colon \eC\to \mathopen] -\pi , \pi \mathclose]\). Il y en a évidemment beaucoup d'autres de possibles. Par exemple pour \( \alpha\in \eR\) nous pouvons considérer
	\begin{equation}        \label{EQooNKKDooOuJxXe}
		\arg_{\alpha^+}\colon \eC\to \mathopen] \alpha , \alpha+2\pi \mathclose]
	\end{equation}
	ou
	\begin{equation}
		\arg_{\alpha^-}\colon \eC\to \mathopen[ \alpha , \alpha+2\pi \mathclose[.
	\end{equation}
	En posant
	\begin{equation}
		\ln_{\alpha^{\pm}}(z)=\ln(| z |)+i\arg_{\alpha^{\pm}}(z)
	\end{equation}
	nous avons une fonction réciproque de l'exponentielle définie sur \( \eC^*\) et holomorphe sur \( \eC^*\) privé d'une demi-droite \( D_{\alpha}\) (dépendante de la valeur de \( \alpha\)).
\end{normaltext}

La différence entre \( \ln_{\alpha^+}\) et \( \ln_{\alpha^-}\) est seulement la valeur sur la demi-droite de non-holomorphie. L'une sera semicontinue d'un côté et l'autre, de l'autre côté.

\begin{remark}
	La fonction \( \arg_{0^-}\) a déjà été utilisée en \ref{SUBSECooWFNMooOuZBRN} pour écrire un inverse de la fonction
	\begin{equation}
		\begin{aligned}
			\varphi\colon \mathopen[ 0 , 2\pi \mathclose[ & \to S^1          \\
			t                                             & \mapsto  e^{it}.
		\end{aligned}
	\end{equation}
\end{remark}

\begin{definition}[\cite{BIBooIKTTooKPwHdS}]
	Soit un ouvert \( \Omega\subset \eC^*\). Nous disons que la fonction \( f\colon \Omega\to \eC\) est une \defe{détermination}{détermination!logarithme} sur \( \Omega\) si elle est continue et vérifie
	\begin{equation}
		e^{f(z)}=z
	\end{equation}
	pour tout \( z\in \Omega\).
\end{definition}

Les différents résultats vus jusqu'ici montrent qu'il n'existe pas de détermination du logarithme sur \( \eC^*\).

\begin{definition}
	La \defe{détermination principale}{détermination!logarithme!principale} du logarithme est la restriction de notre logarithme~\ref{DEFooWDYNooYIXVMC}
	\begin{equation}
		\begin{aligned}
			\ln\colon \eC^* & \to \eC                     \\
			z               & \mapsto \ln(| z |)+i\arg(z)
		\end{aligned}
	\end{equation}
	à l'ouvert \( \eC^*\setminus U\) où \( U\) est la demi-droite \( \real(z)\leq 0\), \( \imag(z)=0\) de \( \eC\).
\end{definition}

\begin{remark}      \label{REMooFBLLooDnkmjR}
	Beaucoup de sources\cite{ooGUROooApafph} ne définissent pas \( \ln_{\alpha^{\pm}}\) sur la droite \( D_{\alpha}\). C'est-à-dire qu'ils notent \( \ln_{\alpha}\) notre fonction \( \ln_{\alpha^+}\) restreinte à \( \eC^*\setminus D_{\alpha}\). Dans ce cas, les fonctions \( \ln_{\alpha^+}\) et \( \ln_{\alpha^-}\) sont identiques\footnote{Cela n'est pas tout à fait évident; vous devriez y penser.}.

	Cette remarque est importante parce que certains vont vous dire «le logarithme n'est pas défini sur la demi-droite»; de leur point de vue, la fonction que nous avons définie est une prolongation (non continue) à \( U\) du logarithme, qui est continu.

	\begin{enumerate}
		\item
		      Certaines personnes pourraient vous dire que notre logarithme «n'est pas bien défini parce que si on fait le tour dans un sens ou dans l'autre nous n'obtenons pas la même valeur pour \( \ln(z)\) lorsque \( z\) est sur \( U\)». Et cela avec des arguments aussi forts que «\( 2\pi\) et \( 0\), c'est le même point».

		      Nous préférons être bien clairs\quext{Est-ce qu'il faut vraiment un pluriel ici ?} sur ce point : notre fonction \( \ln\) est parfaitement définie sur \( \eC^*\) et \( 2\pi\) n'est pas la même chose que zéro. En particulier \( \arg( e^{2i\pi})=0\) et \(  \arg(e^{-i\pi})=\pi\) et non \( -\pi\).
		\item
		      Il n'en reste pas moins que Sage donne \( \ln(-1)=i\pi\) et que nous avons choisi de faire de même, parce que le Frido n'est pas un cours d'agrégation, mais un texte qui donne quelques éléments de mathématique dans le but d'utiliser Sage efficacement.
		\item
		      Tout ceci pour dire que si vous utilisez ce livre pour l'agrégation, vous devriez sérieusement considérer l'option de ne pas donner du logarithme la définition donnée ici, mais bien sa restriction.
	\end{enumerate}

	En fait notre logarithme est maximum pour la propriété «être une réciproque de l'exponentielle» alors que beaucoup de monde préfère avoir une fonction maximale pour la propriété «être réciproque de l'exponentielle tout en étant continue».

\end{remark}

De toutes les fonctions ayant le droit de vouloir être appelée «logarithme», celle que nous avons choisie (un peu arbitrairement) pour s'appeler «logarithme» et accaparer de la notation «\( \ln\)» est \( \ln_{\pi^+}\). Elle est d'une certaine manière celle qui arrive le plus naturellement.

En effet si nous pensons au logarithme népérien \( \ln\colon \eR^+\to \eR\) que nous voulons prolonger sur \( \eR\), nous devons poser
\begin{equation}
	\ln(-x)=\ln(-1)+\ln(x)
\end{equation}
pour \( x>0\). Que peut valoir \( \ln(-1)\) ? Il doit vérifier \(  e^{\ln(-1)}=-1\). La première valeur qui nous tombe sous la main est \( \ln(-1)=\pi\). Bien entendu, d'autres possibilités existent, comme \( \ln(-1)=2017\pi\) par exemple.

%--------------------------------------------------------------------------------------------------------------------------- 
\subsection{Pas d'unicité : développement en série}
%---------------------------------------------------------------------------------------------------------------------------

Pour \( z_0\in \eC^*\) nous pouvons écrire un développement en série de la réciproque de l'exponentielle autour de \( z_0\). La fonction ainsi définie est holomorphe sur la boule \( B(z_0,| z_0 |)\) et diverge en dehors de cette boule.

Voilà encore une fonction «logarithme» pour chaque point de \( \eC^*\). Nous nommons \( \ln_{z_0}\) la fonction
\begin{equation}
	\ln_{z_0}\colon B(z_0,| z_0 |)\to \eC
\end{equation}
donnée par la série.

En général nous n'avons pas \( \ln_{z_1}=\ln_{z_2}\) sur l'intersection des disques de convergence. Si c'était le cas, de proche en proche nous pourrions construire une fonction continue réciproque du logarithme sur \( \eC^*\), ce qui est impossible.

%--------------------------------------------------------------------------------------------------------------------------- 
\subsection{Pas d'unicité : laquelle choisir ?}
%---------------------------------------------------------------------------------------------------------------------------

Bon. Pour chaque demi-droite \( D\) nous avons une détermination du logarithme sur \( \eC^*\setminus D\). Et pour tout \( z_0\in \eC^*\) nous en avons une sur \( B(z_0,| z_0 |)\).

En pratique, quel logarithme choisir ? Cela dépend du problème.

Si vous avez besoin ou envie de travailler avec des série entières, le mieux est de choisir une détermination donnée par un développement autour d'un point bien choisi par rapport à votre problème.

Si vous avez surtout besoin d'holomorphie, et que vous en avez besoin sur un grand domaine, vous devriez choisir une détermination sur un des ensembles \( \eC^*\setminus D_{\alpha}\) en choisissant \( \alpha\) de telle sorte que la demi-droite maudite ne passe pas par la zone sur laquelle vous travaillez.

Dans tous les cas, vous devez préciser très explicitement la détermination choisie. Dans ce texte, sauf mention du contraire, nous utiliserons la détermination principale, et même son extension (non continue) à \( \eC^*\). Lorsque nous aurions besoin d'holomorphie, nous préciserons que nous considérons la restriction.

%--------------------------------------------------------------------------------------------------------------------------- 
\subsection{Logarithme comme primitive}
%---------------------------------------------------------------------------------------------------------------------------

Tout le monde sait\footnote{Proposition \ref{PROPooPDJLooXphpEM}.} que le logarithme \( \ln\colon \eR^+\to \eR\) est une primitive de la fonction \( x\mapsto 1/x\). Qu'en est-il dans le cas complexe ? Tout d'abord précisons que nous ne comptons pas encore parler d'intégrale sur \( \eC\), mais seulement d'intégrales sur \( \eR\) d'une fonction à valeur complexes.

\begin{proposition}     \label{PROPooNIJVooKueuYJ}
	Si \( z\in \eC\) alors
	\begin{equation}        \label{EQooAHYXooTPGXDS}
		\int\frac{1}{ x+z }dx=\ln(x+z)
	\end{equation}
\end{proposition}

\begin{proof}
	Il est important de comprendre que la formule \eqref{EQooAHYXooTPGXDS} est un abus de notation pour dire que si nous considérons la fonction
	\begin{equation}
		\begin{aligned}
			\varphi\colon \eR & \to \eC          \\
			x                 & \mapsto \ln(x+z)
		\end{aligned}
	\end{equation}
	alors nous avons \( \varphi'(x)=\frac{1}{ x+z }\). Ici la dérivation est une dérivation sur \( \eR\) et l'intégrale est une intégrale sur \( \eR\), c'est-à-dire «composante par composantes». La fonction \(  \varphi\) se décompose en partie réelle et imaginaire qui sont à dériver séparément :
	\begin{equation}
		\varphi(x)=\ln(| x+z |)+i\arg(x+z).
	\end{equation}

	\begin{subproof}

		\spitem[Si \( z\) est imaginaire pur]

		Nous posons \( z=\lambda i\) avec \( \lambda\in \eR^*\). D'abord nous avons
		\begin{equation}
			\frac{1}{ x+\lambda i }=\frac{ x }{ x^2+\lambda^2 }-i\frac{ \lambda }{ x^2+\lambda^2 }.
		\end{equation}
		La partie réelle de \( \varphi(x)\) est
		\begin{equation}
			\varphi_1(x)=\ln\big( \sqrt{ x^2+\lambda^2 } \big),
		\end{equation}
		dont la dérivée est
		\begin{equation}
			\varphi_1'(x)=\frac{ x }{ x^2+\lambda^2 },
		\end{equation}
		qui correspond bien à la partie réelle de \( \frac{1}{ x+\lambda i }\).

		En ce qui concerne la partie imaginaire, \( \varphi_2(x)=\arg(x+\lambda i)\), et sa dérivée n'est rien d'autre que la dérivée partielle par rapport à \( x\) de la fonction argument, déjà calculée en \eqref{EQooAOJPooOrvUBR} :
		\begin{equation}
			\varphi_2'(x)=\frac{ -\lambda }{ x^2+\lambda }.
		\end{equation}
		Cela est bien la partie imaginaire de \( \frac{1}{ x+\lambda i }\).

		Notons que nous n'avons pas de problèmes sur la demi-droite des réels négatifs parce que nous ne considérons au final que la dérivée partielle par rapport à \( x\) de la fonction argument, laquelle existe et est continue, même sur cette partie.

		\spitem[Pour \( z\) quelconque]

		Soit \( z=s+\lambda i\) avec \( s,\lambda\in \eR\). En posant \( \varphi_0(x)=\ln(x+\lambda i)\) nous avons \( \varphi(x)=\varphi_0(x+s)\) et donc
		\begin{equation}
			\varphi'(x)=\varphi_0'(x+s)=\frac{ 1 }{ x+s+\lambda i }=\frac{1}{ x+z }.
		\end{equation}
		Tout va bien.

	\end{subproof}
\end{proof}

\begin{example}     \label{EXooAKEDooZgjocX}
	Un petit calcul d'intégrale, que nous avions déjà faite dans l'exemple~\ref{EXooIPEQooGKDjea} (avec la méthode de Rothstein-Trager). En passant par une décomposition en fractions simples :
	\begin{subequations}
		\begin{align}
			\int\frac{1}{ x^3+x } & =\int\left( \frac{1}{ x }-\frac{ 1/2 }{ x-i }-\frac{ 1/2 }{ x+i } \right) \\
			                      & =\ln(x)-\frac{ 1 }{2}\ln(x-i)-\frac{ 1 }{2}\ln(x+i)                       \\
			                      & =\ln(x)-\frac{ 1 }{2}\ln(x^2+1).       \label{SUBEQooRNQLooScfSlG}
		\end{align}
	\end{subequations}
	Attention aux justifications. Il n'est pas vrai en général dans le cas de nombres complexes \( a\) et \( b\) que \( \ln(ab)=\ln(a)+\ln(b)\). En effet, pour la partie réelle, ça passe parce que \( | ab |=| a | |b |\). Mais en ce qui concerne la partie imaginaire,
	\begin{equation}
		\arg(ab)\neq \arg(a)+\arg(b)
	\end{equation}
	lorsque la somme dépasse les bornes de \( \mathopen] -\pi , \pi \mathclose]\). Le passage à \eqref{SUBEQooRNQLooScfSlG} fonctionne parce que dans le cas particulier des nombres \( x+i\) et \( x-i\), les arguments se somment à zéro : \( \arg(x+i)+\arg(x-i)=0\).
\end{example}

%--------------------------------------------------------------------------------------------------------------------------- 
\subsection{Logarithme sur un chemin}
%---------------------------------------------------------------------------------------------------------------------------

\begin{definition}[logarithme continu]      \label{DEFooBBGFooCEdsFR}
	Soient un espace topologique \( X\), et une application \( f\colon X\to \eC^*\). Nous disons que \( g\colon X\to \eC^*\) est un \defe{logarithme}{logarithme d'une application} de \( f\) si pour tout \( x\in X\) nous avons
	\begin{equation}
		f(x)=\exp\big( g(x) \big).
	\end{equation}
\end{definition}

\begin{definition}[Détermination du logarithme]     \label{DEFooOCDGooGyvvWi}
	Soit un chemin\footnote{Définition \ref{DEFooQZMSooYYkGDv}.} \( \gamma\colon \mathopen[ a , b \mathclose]\to \eC^*\). Nous disons qu'une application \( g\colon \gamma\big( \mathopen[ a , b \mathclose] \big)\to \eC^*\) est une \defe{détermination du logarithme}{détermination du logarithme} sur \( \gamma\) si
	\begin{equation}
		\exp\big( (g\circ\gamma)(t) \big)=\gamma(t)
	\end{equation}
	pour tout \( t\in \mathopen[ a , b \mathclose]\).
\end{definition}

\begin{theorem}     \label{THOooUPANooMiECqe}
	Tout chemin dans \( \eC^*\) admet une détermination continue du logarithme\footnote{Définition \ref{DEFooOCDGooGyvvWi}.}, et si \( l\) est une détermination sur le chemin \( \gamma\), nous avons
	\begin{equation}
		\int_{\gamma}\frac{ dz }{ z }=l\big( \gamma(b) \big)-l\big( \gamma(a) \big).
	\end{equation}
\end{theorem}

Le théorème de Borsuk parle de lien entre logarithme continu et homotopie de fonctions à la fonction constante. Il s'applique donc bien de concert avec la proposition \ref{PROPooNABDooFtKukO} qui dit que si \( f\) et \( g\) sont homotopes, alors \( f/g\) est homotope à une constante.

\begin{theorem}[Théorème de Borsuk\cite{BIBooQKARooMHqitK}]     \label{THOooTCUMooEByCKg}
	Soient un compact \( K\) de \( \eR^n\) ainsi qu'une application continue \( f\colon K\to \eC^*\). Les propriétés suivantes sont équivalentes :
	\begin{enumerate}
		\item   \label{ITEMooKZYDooKoEEbl}
		      \( f\) admet un logarithme continu\footnote{Définition \ref{DEFooBBGFooCEdsFR}.}.
		\item   \label{ITEMooXVNXooVAHklr}
		      \( f\) est homotope à l'application constante \( u\colon K\to \eC^*\), \( u(z)=1\).
		\item   \label{ITEMooQDHXooObjxLA}
		      \( f\) admet une extension continue \( \tilde f\colon \eR^n\to \eC^*\).
	\end{enumerate}
\end{theorem}

\begin{proof}
	En plusieurs points.
	\begin{subproof}
		\spitem[\ref{ITEMooKZYDooKoEEbl} \( \Rightarrow\) \ref{ITEMooQDHXooObjxLA}]
		% -------------------------------------------------------------------------------------------- 
		Soit un logarithme continu \( g\colon K\to \eC\) de \( f\). Vu que \( K\) est fermé, le théorème de Tietze \ref{THOooXKGWooFUYlux} dit que \( g\) possède une extension continue \( \tilde g\colon \eR^n\to \eC\). Nous posons
		\begin{equation}
			\begin{aligned}
				\tilde f\colon \eR^n & \to \eC^*                            \\
				x                    & \mapsto \exp\big( \tilde g(x) \big).
			\end{aligned}
		\end{equation}
		L'application \( \tilde f\) est continue parce que \( \exp\) et \( \tilde g\) le sont. Elle est une extension de \( f\) parce que si \( x\in K\), nous avons
		\begin{equation}
			\tilde f(x)=\exp\big( \tilde g(x) \big)=\exp\big( g(x) \big)=f(x)
		\end{equation}
		parce que \( \tilde g(x)=g(x)\) et \( g\) est un inverse de \( \exp\).
		\spitem[\ref{ITEMooQDHXooObjxLA} \( \Rightarrow\) \ref{ITEMooXVNXooVAHklr}]
		% -------------------------------------------------------------------------------------------- 
		Soit une extension continue \( \tilde f\colon \eR^n\to \eC^*\) de \( f\). Soit \( x_0\in \eR^n\). Nous posons
		\begin{equation}
			\begin{aligned}
				H\colon \mathopen[ 0 , 1 \mathclose]\times K & \to \eC^*                                \\
				(t,x)                                        & \mapsto \tilde f\big( (1-t)x+tx_0 \big).
			\end{aligned}
		\end{equation}
		L'application \( H\) est continue, et pour \( x\in K\) elle vérifie \( H(0,x)=\tilde f(x)=f(x)\) ainsi que \( H(1,x)=\tilde f(x_0)\).

		Donc \( f\) est homotope à l'application constante \( \tilde f(x_0)\). Vu que \( \eC^*\) est connexe par arcs, toutes les applications constantes sont homotopes\footnote{Lemme \ref{LEMooMJKEooCaVhjD}.}. Donc \( f\) est homotope à \( \tilde f(x_0)\) qui est homotope à \( u\). L'homotopie étant une relation d'équivalence\footnote{Lemme \ref{LEMooMGFZooGOaGYl}.}, \( f\) est homotope à \( u\).
		\spitem[\ref{ITEMooXVNXooVAHklr} \( \Rightarrow\) \ref{ITEMooKZYDooKoEEbl}]
		% -------------------------------------------------------------------------------------------- 
		Nous considérons l'homotopie \( H\colon \mathopen[ 0 , 1 \mathclose]\times K\to \eC^*\) entre \( f\) et \( u\). En particulier pour tout \( x\in K\) nous avons \( H(0,x)=f(x)\) et \( H(1,x)=1\).

		La partie \( \mathopen[ 0 , 1 \mathclose]\times K\) est compacte\footnote{Théorème \ref{THOIYmxXuu}.} et \( | H |\) y est continue. Donc elle atteint ses bornes. Mais elle prend ses valeurs dans \( \eC^*\); donc
		\begin{equation}
			\inf_{x,t}| H(x,t) |>0.
		\end{equation}
		Et d'ailleurs cet infimum est un minimum. Nous considérons la norme suivante sur \( \eR\times \eR^n\) :
		\begin{equation}
			\| (t,x) \|_{\infty}=\max\big( | t |,| x | \big).
		\end{equation}
		Toutes les normes étant équivalentes\footnote{Équivalence de normes, théorème \ref{ThoNormesEquiv}.} sur \( \eR\times \eR^n\), nous pouvons caractériser la compacité, la continuité et tout ça en termes de cette norme. L'application \( H\) est uniformément continue\footnote{Théorème de Heine \ref{PROPooBWUFooYhMlDp}.}. Soit \( \epsilon>0\), il existe \( \eta>0\) tel que si \( \| (t,x)-(s,y) \|_{\infty}<\eta\) alors \( \| H(t,x)-H(s,y) \|<\epsilon\).

		Soient un entier \( n>\frac{1}{ \eta }+1\), \( x\in K\)  et \( k\in\{ 0,\ldots, n \}\). Nous avons
		\begin{equation}
			\big\| \big(\frac{ k }{ n },x\big)-\big(  \frac{ k+1 }{ n },x \big) \big\|_{\infty}=\big\|  (-\frac{1}{ n },0)  \big\|=\frac{1}{ n }<\eta.
		\end{equation}
		Nous avons donc également
		\begin{equation}
			\big| H(\frac{ k }{ n },x)-H(\frac{ k+1 }{ n },x) \big|<\epsilon.
		\end{equation}

		Nous posons
		\begin{equation}
			\begin{aligned}
				F_k\colon K & \to \eC^*                               \\
				x           & \mapsto H\big( \frac{ k }{ n },x \big).
			\end{aligned}
		\end{equation}
		Cette application est continue sur \( K\) et pour tout \( x\in K\) nous avons
		\begin{subequations}        \label{SUBEQSooLJIJooYkxyMO}
			\begin{align}
				\big| \frac{ F_k(x) }{ F_{k+1}(x) }-1 \big| & =\big| \frac{ F_k(x)-F_{k+1}(x) }{ F_{k+1}(x) } \big|                                       \\
				                                            & =\big| \frac{ H(\frac{ k }{ n },x)-H(\frac{ k+1 }{ n },x) }{ H(\frac{ k+1 }{ n },x) } \big| \\
				                                            & <\frac{ \epsilon }{ | H(\frac{ k+1 }{ n },x) | }.
			\end{align}
		\end{subequations}
		Pour rappel, pour tout \( \epsilon>0\), il existe un \( \eta>0\) tel qu'en posant \( n>\frac{1}{ \eta }+1\) nous avons \eqref{SUBEQSooLJIJooYkxyMO}. Nous choisissons \( \epsilon\) tel que \( 0<\epsilon<\inf_{(t,x)}| H(t,x) |\), de telle sorte à avoir
		\begin{equation}
			\big| \frac{ F_k(x) }{ F_{k+1}(x) }-1 \big|<\frac{ \epsilon }{ | H(\frac{ k+1 }{ n },x) | }<1.
		\end{equation}

		La fonction \( \ln\colon \eC^*\to \eC^*\) est continue sur \( B(1,1)\)\footnote{Le logarithme est celui défini en \ref{DEFooWDYNooYIXVMC} et sa continuité est le théorème \ref{THOooWUXOooYKvLbJ}.}. Donc la fonction
		\begin{equation}
			\begin{aligned}
				h_k\colon K & \to \eC^*                                               \\
				x           & \mapsto \ln\left( \frac{ F_k(x) }{ F_{k+1}(x) } \right)
			\end{aligned}
		\end{equation}
		est continue. Nous posons
		\begin{equation}
			g=\sum_{k=0}^{n-1}\ln\left( \frac{ F_k }{ F_{k+1} } \right),
		\end{equation}
		qui est également continue sur \( K\). Cette fonction \( g\) est un logarithme continu de \( f\) sur \( K\) parce que\footnote{Dans le calcul suivant, nous utilisons entre autres la formule \( \exp(a+b)=\exp(a)\exp(b)\) de la proposition \ref{PropdDjisy}\ref{ITEMooRLHCooJTuYKV}.}
		\begin{subequations}
			\begin{align}
				\exp\big( g(x) \big) & =\exp\left( \sum_{k=0}^{n-1} \ln\Big( \frac{ f_k(x) }{ F_{k+1}(x) } \Big) \right) \\
				                     & =\prod_{k=0}^{n-1}\exp\left( \ln\big( \frac{ F_k(x) }{ F_{k+1}(x) } \big) \right) \\
				                     & =\prod_{k=0}^{n-1}\frac{ F_k(x) }{ F_{k+1}(x) }                                   \\
				                     & =\frac{ F_0(x) }{ F_n(x) }                                                        \\
				                     & =\frac{ H(0,x) }{ H(1,x) }                                                        \\
				                     & =\frac{ f(x) }{ 1 }                                                               \\
				                     & =f(x).
			\end{align}
		\end{subequations}
	\end{subproof}
\end{proof}

\begin{corollary}[\cite{BIBooQKARooMHqitK}]     \label{CORooXOOZooUJMKxu}
	Soit un compact convexe \( K\) de \( \eR^n\). Toute application continue \( K\to \eC^*\) admet un logarithme continu.
\end{corollary}

\begin{proof}
	Soit une application continue \( f\colon K\to \eC^*\). Soit \( x_0\in K\). Vu que \( K\) est convexe, \( f\) est homotope à la fonction constante \( f(x_0)\). Par arc-connexité de \( \eC^*\), l'application \( f\) est alors homotope à la fonction constante \( 1\).

	Le théorème de Borsuk \ref{THOooTCUMooEByCKg} nous dit alors que \( f\) admet un logarithme continu.
\end{proof}

\begin{lemma}       \label{LEMooJNPTooScfSvA}
	Si \( K\) est compact dans \( \eC\), alors la partie \( \eC\setminus K\) possède exactement une composante connexe non bornée.
\end{lemma}

\begin{proof}
	La partie \( K\) étant compacte, elle est bornée. Il existe donc \( r>0\) tel que \( K\subset B(0,r)\). Soit \( A=\eC\setminus B(0,r)\). Cela est une partie connexe de \( \eC\) et est donc contenue dans une composante connexe de \( \eC\setminus K\). Il y a donc existence d'une composante connexe non bornée de \( \eC\setminus K\).

	Pour l'unicité, les autres composantes connexes de \( \eC\setminus K\) sont contenues dans \( B(0,r)\) et sont donc bornées.
\end{proof}

%---------------------------------------------------------------------------------------------------------------------------
\subsection{Lacets, indice et homotopie}
%---------------------------------------------------------------------------------------------------------------------------

\begin{propositionDef}[\cite{BIBooCLWDooMmMzZe}]      \label{DEFooLFBNooGlvJmp}
	Soit \( \gamma\) un chemin fermé\footnote{Par abus de langage, nous désignerons par \( \gamma\) à la fois le chemin et son image.} dans \( \eC\) que nous supposons continu et \( C^1\) par morceaux\footnote{Définition \ref{DEFooQTAWooOCmSJo}}. L'\defe{indice}{indice!d'une courbe dans \( \eC\)} de la courbe \( \gamma\) est la fonction
	\begin{equation}
		\begin{aligned}
			\Ind_{\gamma}\colon \eC\setminus \gamma & \to \eZ                                                              \\
			z                                       & \mapsto \frac{1}{ 2\pi i }\int_{\gamma}\frac{ d\omega }{ \omega-z }.
		\end{aligned}
	\end{equation}
	Alors:
	\begin{enumerate}
		\item		\label{ITEMooHYHMooWcaJxP}
		      La fonction \( \Ind_{\gamma}\) prend ses valeurs dans \( \eZ\).
		\item
		      Elle est continue sur chaque composante connexe de \( \eC\setminus \gamma\).
		\item
		      La fonction indice est constante sur chaque composante connexe\footnote{Définition \ref{DEFooFHXNooJGUPPn}.} de \( \eC\setminus \gamma\).
		\item
		      Elle est nulle sur les composantes non bornées de \( \eC\setminus\gamma\)\footnote{Elle a une seule telle composante par le lemme \ref{LEMooJNPTooScfSvA}, mais ce n'est pas important ici.}.
	\end{enumerate}
\end{propositionDef}
\index{connexité!indice d'une courbe}

\begin{proof}
	Point par point.
	\begin{subproof}
		\spitem[Prend ses valeurs dans \( \eZ\)]
		%-----------------------------------------------------------

		Explicitons l'intégrale sur \(\gamma \colon \mathopen[ a,b\mathclose]\to \eC  \) :
		\begin{equation}
			\Ind_{\gamma}(z_0)=\frac{1}{ 2\pi i}\int_{\gamma}\frac{ dz }{ z-z_0 }=\frac{1}{ 2\pi i}\int_a^b\frac{ \gamma'(t) }{ \gamma(t)-z_0 }dt.
		\end{equation}
		Note. Comme \( \gamma\) n'est peut-être pas dérivable en certains points, en réalité nous n'avons pas une intégrale entre \( a\) et \( b\), mais une somme d'intégrales entre \( a_i\) et \( b_i\) avec \( a_0=a\) et \( b_n=b\). Nous nommons \( S\) l'ensemble de ces points.

		Nous allons calculer l'exponentielle de ce qui est dans l'intégrale en posant
		\begin{equation}
			\begin{aligned}
				\varphi\colon \mathopen[ a,b\mathclose] & \to \eC                                                                    \\
				u                                       & \mapsto \exp\left( \int_a^u\frac{ \gamma'(t) }{ \gamma(t)-z_0 }dt \right).
			\end{aligned}
		\end{equation}
		Nous savons par le lien entre primitive et intégrale (proposition \ref{PropEZFRsMj}) que la dérivée de \( x\mapsto \int_0^xf\) est \( f\) elle-même. Nous pouvons donc facilement dériver \( \varphi\) :
		\begin{equation}		\label{EQooHEAWooCWnNLZ}
			\varphi'(u)=\frac{ \gamma'(u) }{ \gamma(u)-z_0 }\exp\left(\int_a^u\frac{ \gamma'(t) }{ \gamma(t)-z_0 }dt\right)=\frac{ \gamma'(u)\varphi(u) }{ \gamma(u)-z_0 }.
		\end{equation}
		Cette égalité est vraie pour tout \( u\in\mathopen[ a,b\mathclose]\) sauf en un nombre fini de points (les points de \( S\)). Nous considérons maintenant
		\begin{equation}
			\begin{aligned}
				f\colon \mathopen[ a,b\mathclose] & \to \eC                                       \\
				t                                 & \mapsto \frac{ \varphi(t) }{ \gamma(t)-z_0 },
			\end{aligned}
		\end{equation}
		et nous dérivons :
		\begin{subequations}
			\begin{align}
				f'(t) & =\frac{ \varphi'(t)\big( \gamma(t)-z_0 \big) -\varphi(t)\gamma'(t) }{ \big( \gamma(t)-z_0 \big)^2 } \\
				      & =\frac{ \varphi'(t) }{ \gamma(t)-z_0 }-\frac{\varphi(t)\gamma'(t) }{ \big( \gamma(t)-z_0 \big)^2 }.
			\end{align}
		\end{subequations}
		Nous y injectons l'expression \eqref{EQooHEAWooCWnNLZ} de \( \varphi'\) :
		\begin{equation}
			f'(t)=\frac{ \gamma'(t)\varphi(t) }{ \big( \gamma(t)-z_0 \big)^2 }-\frac{ \varphi(t)\gamma'(t) }{ \big( \gamma(t)-z_0 \big)^2 }=0.
		\end{equation}
		Comme \( f\) est continue\footnote{Fait important : nous utilisons la continuité de \( f\), pas de l'indice. Nous n'avons pas encore démontré que l'indice est une fonction continue.} sur \( \mathopen[ a,b\mathclose]\) et de dérivée nulle partout sauf sur \( S\), elle est constante sur \( \mathopen[ a,b\mathclose]\). Nous avons donc \( f(t)=f(a)\) pour tout \( t\), c'est à dire
		\begin{equation}
			\frac{ \varphi(t) }{ \gamma(t)-z_0 }=\frac{ 1 }{ \gamma(a)-z_0 }.
		\end{equation}
		Nous pouvons isoler de là une valeur pour \( \varphi(t)\) :
		\begin{equation}
			\varphi(t)=\frac{ \gamma(t)-z_0 }{ \gamma(a)-z_0 }.
		\end{equation}

		C'est le moment de calculer \( \varphi(b)\). Dans le calcul suivant, nous utilisons \( \gamma(a)=\gamma(b)\) :
		\begin{equation}
			\varphi(b)=\frac{ \gamma(b)-z_0 }{ \gamma(a)-z_0 }=\frac{ \gamma(b)-z_0 }{ \gamma(b)-z_0 }=1.
		\end{equation}
		Nous avons prouvé que
		\begin{equation}
			\exp\left( \int_a^b\frac{ \gamma'(t) }{ \gamma(t)-z_0 }dt \right)=1.
		\end{equation}
		Pour qu'une exponentielle soit égale à \( 1\), il faut que ce qui se trouve dedans soit égal à \( 2ik\pi\) pour un certain \( k\). Il existe \( k\in \eZ\) tel que
		\begin{equation}
			\int_a^b\frac{ \gamma'(t) }{ \gamma(t)-z_0 }dt=2i\pi k.
		\end{equation}

		Pour cette valeur de \( k\) nous avons \( \Ind_{\gamma}(z_0)=k\).

		\spitem[Une borne pour \( | \gamma' |\)]
		%-----------------------------------------------------------
		L'application \( \gamma\) est \( C^1\) par morceaux\footnote{Nous utilisons ici crucialement le fait que les morceaux sont fermés.}. La restriction \(\gamma' \colon \mathopen[ a_i,b_i\mathclose]\to \eC  \) est continue sur un compact et est donc bornée. Pour chaque \( i\) nous avons donc une borne \( M_i\) pour \( | \gamma'j |\). En nommant \( M\) le maximum des ces bornes \( M_i\) nous avons \( \gamma'(t)<M\) pour tout \( t\in \mathopen[ a,b\mathclose]\).

		\spitem[L'indice est ponctuellement continu]
		%-----------------------------------------------------------
		Nous utilisons le théorème \ref{ThoKnuSNd} avec l'application
		\begin{equation}
			\begin{aligned}
				f\colon \mathopen[ a,b\mathclose]\times \eC\setminus\gamma & \to \eC                                     \\
				(t,z)                                                      & \mapsto \frac{ \gamma'(t) }{ \gamma(t)-z }.
			\end{aligned}
		\end{equation}
		Sauf que non. Il ne va pas être possible de majorer cette fonction uniformément en \( z\), comme il le faudrait pour remplir la condition \ref{ItemNAuYNG}. La raison est que quand \( z\) s'approche de \( \gamma\), le dénominateur n'est pas contrôlé.

		Nous allons donc restreindre le domaine de \( f\). Soit \( z_0\in \eC\setminus\gamma\). Vu que \( \gamma\) est continue et que \( \mathopen[ a,b\mathclose]\) est compact, la fonction \( t\mapsto1/(\gamma(t)-z_0)\) atteint une borne inférieure. Disons que cette borne est atteinte pour \( t=t_0\). Nous considérons \( r<d\big( z_0,\gamma(t_0) \big)\) et nous prenons une borne \( M\) pour \( | \gamma' |\). Nous avons alors, pour tout \( t\) que
		\begin{equation}
			\frac{ | \gamma'(t) | }{ | \gamma(t)-z_0 | }<\frac{ M }{ r }.
		\end{equation}
		Le \( z_0\) étant fixé, la fonction constante \( M/r\) majore \( f\) uniformément en \( t\). La théorème de continuité sous l'intégrale \ref{ThoKnuSNd} dit alors que l'indice est continue en \( z_0\).

		\spitem[Continuité globale]
		%-----------------------------------------------------------
		La fonction indice étant continue en chaque point de \( \eC\setminus \gamma\), elle est continue sur \( \eC\setminus \gamma\) (théorème \ref{ThoESCaraB}).

		\spitem[Sur les composantes connexes]
		%-----------------------------------------------------------
		Soit une composante connexe \( C\) de \( \eC\setminus\gamma\). L'application \( \Ind_{\gamma}\) est continue et à valeurs entières sur \( C\). La proposition \ref{PropConnexiteViaFonction} dit alors que l'indice est constant sur \( C\).

		\spitem[Nulle sur la non bornée]
		%-----------------------------------------------------------
		En majorant \( | \gamma' |\) par \( M\), nous avons
		\begin{equation}
			| \Ind_{\gamma}(z_0) |\leq\frac{1}{ 2\pi}\int_a^b\frac{ | \gamma'(t) | }{ | \gamma(t)-z_0 | }\leq \frac{ M }{ 2\pi }\int_a^b\frac{ dt }{ | \gamma(t)-z_0 | }.
		\end{equation}
		Soit \( \epsilon>0\) assez petit pour que \(\epsilon M(b-a)/2\pi<1\). Vu que \( \gamma\) est bornée, il existe \( R>0\) tel que \( \gamma(t)\in B(0,R)\) pour tout \( t\). Nous considérons \( r\) assez grand pour que \( r-R>1/\epsilon\).

		Comme nous étudions \( \Ind_{\gamma}\) sur une composante non bornée, nous considérons \( | z_0 |>r\). Avec ça nous avons \( | \gamma(t)-z_0 |>r-R\) et donc
		\begin{equation}
			| \Ind_{\gamma}(z_0) |\leq \frac{ M }{ 2\pi }\int_a^b\epsilon dt=\frac{ M(b-a) }{ 2\pi }\epsilon<1.
		\end{equation}
		La dernière inégalité est un choix de \( \epsilon\) fait au départ.

		Étant donné que \( \Ind_{\gamma}(z_0)\in \eZ\), nous avons forcément \( \Ind_{\gamma}(z_0)=0\). Et comme l'indice est constant sur les composantes connexe, il est nul partout.
	\end{subproof}
\end{proof}


\begin{proposition}     \label{PROPooXWULooFZgHfL}
	Soit \( p\in \eC\). Soit un chemin \( \gamma\colon \mathopen[ 0 , 1 \mathclose]\to \eC\setminus\{ p \}\). Quelques hypothèses :
	\begin{enumerate}
		\item
		      Il existe une demi-droite \( D\) d'origine \( p\) et tel que \( \gamma^{-1}(D)\) est fini.
		\item
		      Le chemin \( \gamma\) traverse \( D\) en tout point de \( \gamma^{-1}(D)\).
		\item
		      Soit \( n^+\) le nombre de points de \( \gamma^{-1}(D)\) où\( \gamma\) traverse positivement \( D\) et \( n^-\) idem négativement.
	\end{enumerate}
	Alors
	\begin{equation}
		\Ind(\gamma,p)=n^+-n^-.
	\end{equation}
\end{proposition}

\begin{proposition}[\cite{BIBooQKARooMHqitK}]       \label{PROPooEKFHooOWcIMk}
	Soient \( p\in \eC\) ainsi que deux chemins \( \gamma_1,\gamma_2\colon \mathopen[ a , b \mathclose]\to \eC\setminus \{ p \} \). Si \( \gamma_1\) et \( \gamma_2\) sont homotopes dans \( \eC\setminus \{ p \}\). Alors
	\begin{equation}
		\Ind(\gamma_1,p)=\Ind(\gamma_2,p).
	\end{equation}
\end{proposition}

\begin{proof}
	Posons
	\begin{equation}
		\begin{aligned}
			f\colon \eC\setminus\{ p \} & \to \eC                   \\
			z                           & \mapsto \frac{1}{ z-p } .
		\end{aligned}
	\end{equation}
	La partie \( \eC\setminus\{ p \}\) est ouverte. Le théorème \ref{THOooVTFXooBgvVyD} d'invariance de l'intégrale par homotopie nous indique que
	\begin{equation}
		\Ind(\gamma_1,p)=\frac{1}{ 2\pi i }\int_{\gamma_1}f=\frac{1}{ 2\pi i }\int_{\gamma_2}f=\Ind(\gamma_2,p).
	\end{equation}
\end{proof}

\begin{definition}  \label{DefECnFJQp}
	Si \( \gamma_1\) et \( \gamma_2\) sont deux lacets en \( x_0\in X\) (un espace topologique), une \defe{équivalence d'homotopie}{equivalence@équivalence!homotopie} est une application \( f\colon \mathopen[ 0 , 1 \mathclose]\times \mathopen[ 0 , 1 \mathclose]\to X\) telle que
	\begin{enumerate}
		\item
		      \( f(0,t)=\gamma_1(t)\) pour tout \( t\);
		\item
		      \( f(1,t)=\gamma_2(t)\) pour tout \( t\);
		\item
		      pour chaque \( t\in \mathopen[ 0 , 1 \mathclose]\), l'application \( s\mapsto f(s,t)\) est continue;
		\item
		      pour chaque \( s\in \mathopen[ 0 , 1 \mathclose]\), l'application \( t\mapsto f(s,t)\) est un lacet basé en \( x_0\).
	\end{enumerate}
\end{definition}


\begin{lemma} 		\label{LEMooMEICooOOGzdf}
	Si \( \gamma\) est un cercle de centre \( z_0\in \eC\) et de rayon \( r\), alors
	\begin{equation}
		\Ind_{\gamma}(z)=\begin{cases}
			1 i & \text{si } z\in B(z_0,r) \\
			0   & \text{sinon}.
		\end{cases}
	\end{equation}
\end{lemma}

\begin{proof}
	La seconde ligne provient directement de la proposition~\ref{DEFooLFBNooGlvJmp}. Pour la première, le cercle \( \gamma\) se paramètre par
	\begin{equation}
		\gamma(\theta)=z_0+r e^{i\theta}.
	\end{equation}

	Nous commençons par calculer \( \Ind_{\gamma}(z_0)\). Il vaut
	\begin{equation}
		\frac{1}{ 2\pi i}\int_{\gamma}\frac{ d\omega }{ \omega-z_0 }=\frac{1}{ 2\pi i}\int_0^{2\pi}\frac{1}{ r e^{i\theta} }ir e^{i\theta}d\theta=1.
	\end{equation}

	Nous considérons maintenant un point \( p\in B(z_0,r)\) et nous calculons \( \Ind_{\gamma}(p)\). D'abord si \( \sigma\) est le cercle \( \sigma(t)=\gamma(t)+p-z_0=re^{it}+p\) de rayon \( r\) centré en \( p\), nous avons
	\begin{equation}
		\Ind_{\sigma}(p)=1
	\end{equation}
	par ce que nous venons de voir.

	Ensuite les chemins \( \sigma\) et \( \gamma\) sont homotopes dans \( \eC\setminus\{ p \}\). En effet nous posons
	\begin{equation}
		H(u,t)=(1-u)\gamma(t)+u\sigma(t).
	\end{equation}
	Nous avons
	\begin{equation}
		H(u,t)-p=(1-u)z_0+re^{it}+(u-1)p=(1-u)(z_0-p)+re^{it}.
	\end{equation}
	En termes de modules, nous avons d'une part
	\begin{equation}
		| (1-u)(z_0-p) |\leq | z_0-p |<r
	\end{equation}
	et \( | re^{it} |=r\). Donc la somme des deux ne peut pas faire zéro. Nous avons donc bien \( H(u,t)\neq p\) pour tout \( u\in \mathopen[ 0,1\mathclose]\) et pour tout \( t\in\mathopen[ 0,2\pi\mathclose]\).

	Les chemins \( \gamma\) et \( \sigma\) étant homotopes dans \( \eC\setminus\{ p \}\), la proposition \ref{PROPooEKFHooOWcIMk} dit que
	\begin{equation}
		\Ind_{\gamma}(p)=\Ind_{\sigma}(p)=1.
	\end{equation}
\end{proof}

\begin{proposition}[\cite{BIBooQKARooMHqitK}]       \label{PROPooGAOIooFTOuli}
	Soit un intervalle \( J\subset \eR\). Nous considérons l'application
	\begin{equation}
		\begin{aligned}
			\psi\colon \mL(J, \eC^*) & \to \eZ                \\
			\gamma                   & \mapsto \Ind(\gamma,0)
		\end{aligned}
	\end{equation}
	où \( \mL(J, \eC^*)\) est l'ensemble des lacets \( J\to \eC^*\) sur lequel nous considérons la loi de groupe multiplicatif tandis que nous considérons la loi additive sur \( \eZ\). Nous avons :
	\begin{enumerate}
		\item       \label{ITEMooKCRIooSEyhlp}
		      \( \psi\) est effectivement à valeurs dans \( \eZ\),
		\item
		      \( \psi\) est un morphisme de groupes surjectif.
		\item       \label{ITEMooZFXTooDCXTVU}
		      Le noyau de \( \psi\) est
		      \begin{equation}
			      \ker(\psi)=\{ \text{lacets homotopes dans \( \eC^*\) à un lacet constant} \}.
		      \end{equation}
	\end{enumerate}
\end{proposition}

\begin{proof}
	Pour \ref{ITEMooKCRIooSEyhlp}, c'est la proposition \ref{DEFooLFBNooGlvJmp}\ref{ITEMooHYHMooWcaJxP}.

	Soient deux lacets \( \gamma_1\) et \( \gamma_2\) dans \( \eC^*\). Le théorème \ref{THOooUPANooMiECqe} nous permet de considérer des logarithmes continus \( l_i\colon \mathopen[ a , b \mathclose]\to \eC\) le long de \( \gamma_i\). L'application \( l_1+l_2\) est une détermination du logarithme le long de \( \gamma_1\gamma_2\) parce que\footnote{Nous ne nous lasserons jamais de citer la proposition \ref{PropdDjisy}\ref{ITEMooRLHCooJTuYKV}.}
	\begin{equation}
		\exp\big( l_1(t)+l_2(t) \big)=\exp\big( l_1(t) \big)\exp\big( l_2(t) \big)=\gamma_1(t)\gamma_2(t).
	\end{equation}
	\begin{subproof}
		\spitem[Morphisme]
		% -------------------------------------------------------------------------------------------- 
		Nous avons
		\begin{subequations}
			\begin{align}
				\Ind(\gamma_1\gamma_2,0) & =\frac{1}{ 2\pi i }\int_a^bf\big( (\gamma_1\gamma_2)(t)\big)(\gamma_1\gamma_2)'(t)dt \\
				                         & =\frac{1}{ 2\pi i }\int_{\gamma_1\gamma_2}\frac{ dz }{ z }                           \\
				                         & =(l_1+l_2)(b)-(l_1+l_2)(a)                                                           \\
				                         & =\big( l_1(b)-l_1(a) \big)+\big( l_2(b)-l_1(a) \big)                                 \\
				                         & =\Ind(\gamma_1,0)+\Ind(\gamma_2,0).
			\end{align}
		\end{subequations}
		Nous avons prouvé que \( \psi(\gamma_1\gamma_2)=\psi(\gamma_1)+\psi(\gamma_2)\), et donc que \( \psi\) est un morphisme.
		\spitem[Surjection]
		% -------------------------------------------------------------------------------------------- 
		Nous considérons le lacet
		\begin{equation}
			\begin{aligned}
				\gamma\colon \mathopen[ a , b \mathclose] & \to \eC^*                              \\
				t                                         & \mapsto \exp\big( 2i\pi kt(b-a) \big),
			\end{aligned}
		\end{equation}
		et nous montrons que \( \psi(\gamma)=k\). D'abord nous remarquons que
		\begin{equation}
			\gamma'(t)=\gamma(t)\frac{ 2i\pi k }{ (b-a) },
		\end{equation}
		et ensuite nous calculons :
		\begin{subequations}
			\begin{align}
				\psi(\gamma) & =\frac{1}{ 2\pi i }\int_a^b\frac{1}{ \gamma(t) }\gamma'(t)dt                         \\
				             & =\frac{1}{ 2\pi i }\frac{1}\int_a^b{ \gamma(t) }\frac{ 2\pi ik }{ (b-a) }\gamma(t)dt \\
				             & =\frac{1}{ 2\pi i }\int_a^b\frac{ 2i\pi k }{ (b-a) }dt                               \\
				             & =k.
			\end{align}
		\end{subequations}
		\spitem[Noyau]
		% -------------------------------------------------------------------------------------------- 
		Notons pour la simplicité
		\begin{equation}
			\mH=\{ \text{lacets homotopes dans \( \eC^*\) à un lacet constant} \}.
		\end{equation}
		Nous notons \( u\) le chemin constant \( u(t)=1\). Nous avons \( \psi(u)=0\) parce que, en notant \( f(z)=1/z\),
		\begin{equation}
			\Ind(u,0)=\frac{1}{ 2\pi i }\int_uf(z)dz=\frac{1}{ 2\pi i }\int_0^1f\big( u(t) \big)u'(t)dt=0
		\end{equation}
		parce que \( u'(t)=0\) pour tout \( t\). Nous avons donc \( u\in\ker(\psi)\). Si \( \gamma\in\mH\), alors, utilisant le théorème de Cauchy homotopique \ref{THOooVTFXooBgvVyD}\ref{ITEMooESDVooFgVarr}, nous avons \( \Ind(\gamma,0)=\Ind(u,0)=0\). Donc \( \mH\subset\ker(\psi)\).

		Soient \( \gamma\in\ker(\psi)\) et une détermination continue \( l\) du logarithme le long de \( \gamma\). Nous avons
		\begin{equation}
			0=\psi(\gamma)=l(b)-l(a).
		\end{equation}
		Nous posons
		\begin{equation}
			\begin{aligned}
				H\colon \mathopen[ 0 , 1 \mathclose]\times \mathopen[ a , b \mathclose] & \to \eC^*                                 \\
				(s,t)                                                                   & \mapsto  \exp\big( sl(a)+(1-s)l(t) \big).
			\end{aligned}
		\end{equation}
		Cette application continue vérifie \( H(0,t)= e^{l(t)}=\gamma(t)\) et \( H(1,t)= e^{l(a)}=\gamma(a)\). Donc elle est une homotopie entre \( \gamma\) et le chemin constant \( \gamma(a)\). Encore par Cauchy homotopique, \( \psi(\gamma)=0\) parce que \( \psi\) est nul sur les chemins constants.
	\end{subproof}
\end{proof}

\begin{lemma}[\cite{MonCerveau}]    \label{LEMooCTIBooNRAyZH}
	Si \( \gamma\) est un lacet dans \( \eC\setminus\{ p \}\), alors
	\begin{equation}
		\Ind(\gamma,p)=\Ind(\gamma-p,0)
	\end{equation}
\end{lemma}

\begin{proposition}[\cite{BIBooQKARooMHqitK}]       \label{PROPooPRIIooKWCHBZ}
	Soient \( p\in \eC\) ainsi que deux lacets \( \gamma_1\) et \( \gamma_2\) dans \( \eC\setminus\{ p \}\). Si
	\begin{equation}
		\Ind(\gamma_1,p)=\Ind(\gamma_2,p),
	\end{equation}
	alors \( \gamma_1\) et \( \gamma_2\) sont homotopes dans \( \eC\setminus \{ p \}\).
\end{proposition}

\begin{proof}
	Vu que \( \gamma_1\) ne passe pas par \( p\), nous pouvons considérer le chemin
	\begin{equation}
		\begin{aligned}
			\gamma\colon \mathopen[ a , b \mathclose] & \to \eC^*                                        \\
			t                                         & \mapsto \frac{ \gamma_0(t)-p }{ \gamma_1(t)-p }.
		\end{aligned}
	\end{equation}
	En posant \( \sigma_i(t)=\gamma_i(t)-p\), nous avons
	\begin{subequations}
		\begin{align}
			\Ind(\gamma,0) & =\Ind(\sigma_0\sigma_1^{-1},0)                                                  \\
			               & =\Ind(\sigma_0,0)+\Ind(\sigma_1^{-1},0) & \text{prop. \ref{PROPooGAOIooFTOuli}} \\
			               & =\Ind(\sigma_0,0)-\Ind(\sigma_1,0)                                              \\
			               & =\Ind(\gamma_0,p)-\Ind(\gamma_1,p)      & \text{lem. \ref{LEMooCTIBooNRAyZH}}   \\
			               & =0.
		\end{align}
	\end{subequations}
	Vu que l'indice de \( \gamma\) est nul, ce chemin est homotope dans \( \eC^*\) au chemin constant \( u\) par la proposition \ref{PROPooGAOIooFTOuli}\ref{ITEMooZFXTooDCXTVU}. Soit une homotopie \( H\colon \mathopen[ 0 , 1 \mathclose]\times \mathopen[ a , b \mathclose]\to \eC^*\) entre \( \gamma\) et \( u\). Et nous posons
	\begin{equation}
		\begin{aligned}
			S\colon \mathopen[ 0 , 1 \mathclose]\times \mathopen[ a , b \mathclose] & \to \eC^*                                 \\
			(s,t)                                                                   & \mapsto \big( \gamma_1(t)-p \big)H(s,t) .
		\end{aligned}
	\end{equation}
	Cette application est une homotopie entre
	\begin{equation}
		t\mapsto S(0,t)=\big( \gamma_2(t)-p \big)H(0,t)=\big( \gamma_2(t)-p \big)\gamma(t)=\gamma_1(t)-p
	\end{equation}
	et
	\begin{equation}
		t\mapsto S(1,t)=\big( \gamma_2(t)-p \big)H(1,t)=\big( \gamma_2(t)-p \big)u(t)=\gamma_2(t)-p.
	\end{equation}
	Donc \( S+p\) est une homotopie entre \( \gamma_1\) et \( \gamma_2\) dans \( \eC\setminus\{ p \}\).
\end{proof}

\begin{proposition}[\cite{BIBooQKARooMHqitK}]       \label{PROPooCFMFooXjlhfV}
	Soit un espace topologique \( X\) localement connexe par arcs. Soit une application continue \( f\colon X\to \eC^*\). Il y a équivalence entre
	\begin{enumerate}
		\item
		      \( f\) admet un logarithme continu.
		\item
		      Pour tout lacet \( \gamma\) dans \( X\), nous avons \( \Ind(f\circ\gamma,0)=0\).
	\end{enumerate}
\end{proposition}

%---------------------------------------------------------------------------------------------------------------------------
\subsection{Théorème de Cauchy et analycité}
%---------------------------------------------------------------------------------------------------------------------------

Cette sous-section veut prouver le théorème de Cauchy.


\begin{theorem}[Formule de Cauchy]    \label{ThoUHztQe}
	Soient \( \Omega\) ouvert dans \( \eC\), \( z_0\in \Omega\) et \( f\) une fonction holomorphe sur \( \Omega\). Soit \( r>0\) tel que \( B(z_0,r)\subset \Omega\). Alors pour tout \( z\in B(z_0,r)\) nous avons
	\begin{equation}    \label{EqPzUABM}
		f(z)=\frac{1}{ 2\pi i }\int_{\partial B(z_0,r)}\frac{ f(\omega) }{ \omega-z }d\omega.
	\end{equation}
\end{theorem}
\index{formule!de Cauchy}
\index{Cauchy!formule}

\begin{proof}
	Soit \( z\in B(z_0,r)\). Considérons la fonction
	\begin{equation}
		g(\omega)=\begin{cases}
			\frac{ f(\omega)-f(z) }{ \omega-z } & \text{si } \omega\neq z \\
			f'(z)                               & \text{si } \omega=z.
		\end{cases}
	\end{equation}
	Cette fonction est holomorphe sur \( B(z_0,r)\setminus\{ z \}\) et continue en \( z\). Elle vérifie donc la proposition~\ref{PrpopwQSbJg} et nous avons
	\begin{equation}
		\int_{\gamma}g=0
	\end{equation}
	où \( \gamma\) est le cercle de centre \( z_0\) et de rayon \( r\). Nous avons donc
	\begin{equation}
		0=\int_{\gamma}\frac{ f(\omega) }{ \omega-z }-\int_{\gamma}\frac{ f(z) }{ \omega-z },
	\end{equation}
	et ayant déjà calculé la seconde intégrale dans l'exemple~\ref{LEMooMEICooOOGzdf} nous en déduisons
	\begin{equation}
		\int_{\gamma}\frac{ f(\omega) }{ \omega-z }d\omega=2\pi if(z),
	\end{equation}
	ce qu'il fallait.
\end{proof}

\begin{theorem}     \label{ThomcPOdd}
	Soient \( \Omega\) ouvert dans \( \eC\) et \( f\) holomorphe sur \( \Omega\). Soient encore \( z_0\in \Omega\) et \( r_0\) tels que \( B(z_0,r_0)\subset \Omega\). Alors :
	\begin{enumerate}
		\item       \label{ITEMooYWSOooHJtxGr}
		      Sur \( B(z_0,r_0)\), la fonction \( f\) s'écrit
		      \begin{equation}
			      f(z)=\sum_{n=0}^{\infty}a_n(z-z_0)^n.
		      \end{equation}
		\item
		      Nous avons
		      \begin{equation}
			      a_n=\frac{ f^{(n)}(z_0) }{ n! }=\frac{1}{ 2\pi i }\int_{\gamma}\frac{ f(\omega) }{ (\omega-z_0)^{n+1} }d\omega
		      \end{equation}
		      où \( \gamma=\partial B(z_0,r)\) avec \( | z-z_0 |<r<r_0\).
		\item   \label{ItemMRRTooMChmuZ}
		      En particulier \( f\) est infiniment dérivable.
	\end{enumerate}
\end{theorem}
\index{série entière!fonctions holomorphes}

\begin{proof}
	Soit \( r>0\) tel que \( | z-z_0 |<r<r_0\). La formule de Cauchy (théorème~\ref{ThoUHztQe}) nous dit que
	\begin{equation}
		f(z)=\frac{1}{ 2\pi i }\int_{\gamma}\frac{ f(\omega)}{ \omega-z }d\omega
	\end{equation}
	où \( \gamma=\partial B(z_0,r)\). Nous pouvons paramétrer ce chemin par \( \omega=z_0+r e^{i\theta}\) et \( \theta\in \mathopen[ 0 , 2\pi \mathclose]\). Nous avons
	\begin{subequations}
		\begin{align}
			f(z) & =\frac{1}{ 2\pi i }\int_0^{2\pi}\frac{ f(z_0+r e^{i\theta}) }{ z_0+r e^{i\theta}-z }ri e^{i\theta}d\theta \\
			     & =\frac{1}{ 2\pi }\int_0^{2\pi}\frac{ f(z_0+r e^{i\theta}) }{ 1- e^{-i\theta}(z-z_0)/r }d\theta.
		\end{align}
	\end{subequations}
	Nous pouvons développer l'intégrante en puissance de \( (z-z_0)\) en utilisant la formule~\ref{EqVmuaqT}. Ici le rôle de \( x\) est tenu par
	\begin{equation}
		e^{-i\theta}(z-z_0)/r
	\end{equation}
	dont le module est bien plus petit que \( 1\), par hypothèse sur \( r\). Nous avons donc
	\begin{equation}
		f(z)=\frac{1}{ 2\pi }\int_0^{2\pi}\sum_{n=0}^{\infty}f(z_0+r e^{i\theta}) e^{-in\theta}r^{-n}(z-z_0)^nd\theta.
	\end{equation}
	L'art est maintenant de permuter la somme et l'intégrale. Pour cela nous remarquons que ce qui se trouve dans la somme est majoré en module par
	\begin{equation}        \label{EqbykTLD}
		M\left| \frac{ z-z_0 }{ r } \right|^n
	\end{equation}
	où \( M\) est le maximum de \( | f |\) sur \( \gamma\). La borne \eqref{EqbykTLD} ne dépend pas de \( \theta\); par conséquent la convergence de la somme est uniforme en \( \theta\) par le critère de Weierstrass (théorème~\ref{ThoCritWeierstrass}). Le théorème~\ref{ThoCciOlZ} s'applique\footnote{Étant donné que nous savions déjà que la somme était une fonction intégrable, nous sommes loin d'avoir utilisé toute la puissance du théorème.} et nous pouvons permuter la somme avec l'intégrale.

	Ce que nous trouvons est que
	\begin{equation}
		f(z)=\sum_{n=0}^{\infty}a_n(z-z_0)^n
	\end{equation}
	où
	\begin{equation}
		a_n=\frac{1}{ 2\pi }\int_0^{2\pi}f(z_0+r e^{i\theta}) e^{-in\theta}r^{-n}d\theta=\frac{1}{ 2\pi i }\int_{\gamma}\frac{ f(\omega) }{ (\omega-z_0)^{n+1} }.
	\end{equation}
	Cette formule est valable pour \( | z-z_0 |<r\). Sur cette boule, la fonction est donc une série entière. Le théorème \ref{THOooCESLooZxAzOg} nous permet donc d'affirmer que \( f\) est partout infiniment continument dérivable (parce que en chaque point on a un voisinage sur lequel c'est vrai), et d'identifier les coefficients (qui, eux, ne sont valables que localement) sous la forme
	\begin{equation}
		a_n=\frac{ f^{(n)}(z_0) }{ n! }.
	\end{equation}
\end{proof}


\begin{corollary}       \label{CORooJISDooFgwOPh}
	Soit une application \(f \colon B_{\eC}(0,R)\to \eC  \) ainsi que des \( (a_n)\) dans \( \eC\) tels que
	\begin{equation}		\label{EQooWLEQooOsLGfP}
		f(z)=\sum_{n=0}^{\infty}a_nz^n
	\end{equation}
	pour tout \( z\in B(0,r)\).

	Alors \( a_0=f(0)\) et \( a_1=f'(0)\).
\end{corollary}

\begin{proof}
	En posant \( z=0\) à droite dans \eqref{EQooWLEQooOsLGfP} nous trouvons tout de suite \( f(0)=a_0\). En ce qui concerne \( a_1\), nous passons par le théorème \ref{THOooCESLooZxAzOg}\footnote{Qui infiniment plus général que ce dont nous avons besoin, mais qui fait quand même bien le boulot.} nous dit que sur \( B(0,r)\), l'application \( f\) est de classe \( C^{\infty}\) et que
	\begin{equation}
		df_z(u)=\sum_{n=1}^{\infty}na_nz^{n-1}u.
	\end{equation}
	Notez que dans le théorème \ref{THOooCESLooZxAzOg}, il est dit \( \alpha_n(x-x_0)^{n-1}\) qui est l'application linéaire qui à \( u\in \eC\) fait correspondre \( \alpha_n(x-x_0,\ldots,x-x_0,u)\). Dans notre cas, ce qui joue le rôle de \( \alpha_n\) est l'application \( (z_1,\ldots,z_n)\mapsto a_nz_1,\ldots,z_n\).

	Ce qui nous dit la relation \eqref{EqPAEFooYNhYpz}, c'est que, si \( f\) est différentiable\footnote{C'est le cas de notre \( f\), comme nous venons de le dire.}, le lien entre dérivée et différentielle est que la différentielle est l'application qui consiste à multiplier par la dérivée. Bref, nous avons
	\begin{equation}
		f'(z)=\sum_{n=1}^{\infty}na_nz^{n-1}.
	\end{equation}
	En posant \( z=0\) nous trouvons bien \( f'(0)=a_1\).
\end{proof}

\begin{corollary}       \label{CorwfHtJu}
	Soit \( f\) une fonction continue sur un ouvert \( \Omega\) telle que pour toute boule \( B(a,r)\) contenue dans \( \Omega\), nous ayons
	\begin{equation}
		f(a)=\frac{1}{ 2\pi i }\int_{\partial B(a,r)}\frac{ f(\xi) }{ \xi-a }d\xi.
	\end{equation}
	Alors \( f\) est holomorphe.
\end{corollary}

\begin{proof}
	Il suffit de recopier la démonstration du théorème~\ref{ThomcPOdd} pour savoir que \( f\) se développe en série de puissances et est donc en particulier dérivable.
\end{proof}

Le fait qu'une fonction holomorphe soit \(  C^{\infty}\) comme dit dans la proposition~\ref{ThomcPOdd} permet de démonter un résultat de dérivation sous l'intégrale, qui dépend de pouvoir majorer la différentielle.

\begin{proposition}     \label{PROPooZCLYooUaSMWA}
	Soit une fonction continue \( g\colon \eR\times \eC\to \eC\). Nous supposons que pour tout \( t\), la fonction \( z\mapsto g(t,z)\) est \( \eC\)-dérivable (définition~\ref{DEFooVJVXooKlnFkh}) et différentiable. Soit \( B\) compact dans \( \eR\) et la fonction
	\begin{equation}
		G(z)=\int_B g(t,z)dt.
	\end{equation}
	que nous supposons exister pour tout \( z\).

	Alors
	\begin{equation}
		G'(z)=\int_Bg'(t,z)dt
	\end{equation}
	où le prime réfère à la \( \eC\)-dérivée par rapport à la variable \( z\) à \( t\) fixé.
\end{proposition}

\begin{proof}
	Nous fixons \( z\in \eC\) et nous considérons la suite de fonctions
	\begin{equation}
		g_i(t)=\frac{ g(t,z+\epsilon_i)-g(t,z) }{ \epsilon_i }
	\end{equation}
	où \( \epsilon_i\) est une suite de nombres complexes tendant vers zéro (\( \epsilon_i\stackrel{\eC}{\longrightarrow}0\)). Si la limite existe et ne dépend pas de la suite choisie, alors \( \lim_{i\to \infty} g_i(t)=g'(t,z)\). Et vu que \( g\) est supposée dérivable, c'est le cas.

	Nous avons aussi, par linéarité de l'intégrale :
	\begin{equation}
		G'(z)=\lim_{i\to \infty} \int_B g_i(t)dt.
	\end{equation}
	La difficulté est de permuter la limite et l'intégrale. Pour cela nous allons utiliser la convergence dominée de Lebesgue (théorème~\ref{ThoConvDomLebVdhsTf}). Afin de majorer \( | g_i(t) |\) par une fonction intégrable en \( t\) (uniformément en \( i\)), nous exploitons le théorème des accroissements finis, théorème~\ref{ThoNAKKght}. En notant \( dg\) la différentielle de \( g\) par rapport à \( z\) à \( t\) fixé, pour chaque \( t\) et chaque \( i\) nous avons
	\begin{equation}
		| g(t,z+\epsilon_i)-g(t,z) |\leq \sup_{\xi\in\mathopen[ z , z+\epsilon_i \mathclose]}\| dg_{\xi} \|\| \epsilon_i \|.
	\end{equation}
	Vu que \( z\) est fixé et que \( \xi\) est dans le compact \( \mathopen[ z , z+\epsilon_i \mathclose]\) et que \( dg\) est continue (parce que la \( \eC\)-dérivabilité implique la continuité de la différentielle parce que nous avons l'analycité par le théorème~\ref{ThomcPOdd}), nous pouvons majorer \( \| dg_{\xi} \|\) par une constante \( M_i(z)\) qui dépend à priori de \( i\) et de \( z\).

	Heureusement, nous pouvons prendre a fortiori le supremum sur \( \overline{ B(z,|\epsilon_i|) }\) (qui est tout autant compact) et supposer que \( | \epsilon_i |\) est strictement décroissante; de toutes façons, il y a un maximum parce que \( | \epsilon_i |\to 0\). Dans ce cas, il suffit de prendre le supremum de \( \| dg_{\xi} \|\) pour \( \xi\in \overline{ B(z,| \epsilon_1 |) }\) et ça contente tout le monde.

	Quoi qu'il en soit nous avons une constante \( M(z)\) telle que
	\begin{equation}
		| g(t,z+\epsilon_i)-g(t,z) |\leq M(z)\| \epsilon_i \|
	\end{equation}
	et donc \( | g_i(t) |\leq M(z)\). La constante (par rapport à \( t\)) \( M(z)\) est évidemment intégrable sur le compact \( B\) et nous pouvons permuter la limite avec l'intégrale :
	\begin{equation}
		G'(z)=\lim_{i\to \infty} \int_Bg_i(t)dt=\int_B\lim_{i\to \infty} g_i(t)dt=\int_Bg'(t,z)dt.
	\end{equation}
\end{proof}

\begin{proposition}\label{PropZOkfmO}
	Une fonction continue \( f\) est holomorphe si et seulement si la \( 1\)-forme différentielle \( f(z)dz\) est localement exacte.
\end{proposition}

\begin{proof}
	Si \( f\) est holomorphe, alors nous avons vu que \( f\) était différentiable et que \( df_{z}=f(z)dz\) par la formule~\ref{EqPropZOkfmO}.

	Dans le sens inverse, supposons que \( f(z)dz\) est localement exacte, et soit \( F\) telle que \( dF=f(z)dz\). Ce que nous allons faire est montrer que la dérivée de \( F\) existe et vaut \( f\). En effet, la définition de la différentielle nous dit que
	\begin{equation}
		\lim_{h\to 0} \left| \frac{ F(z+h)-F(z)-dF_z(h) }{ h } \right| =0.
	\end{equation}
	La limite vaut évidemment encore zéro si nous enlevons les modules :
	\begin{subequations}
		\begin{align}
			0 & =\lim_{h\to 0} \frac{ F(z+h)-F(z)-f(z)h }{ h } \\
			  & =\lim_{h\to 0} \frac{ F(z+h)-F(z) }{ h }-f(z).
		\end{align}
	\end{subequations}
	Donc \( F'=f\). Cela montre que \( F\) est \( \eC\)-dérivable et donc holomorphe. En conséquence du théorème~\ref{ThomcPOdd}, la fonction \( F\) est infiniment dérivable et \( f\) l'est alors aussi. La fonction \( f\) est donc holomorphe\footnote{Dire que la dérivée d'une fonction holomorphe est holomorphe est un raisonnement classique.}.
\end{proof}

%---------------------------------------------------------------------------------------------------------------------------
\subsection{Théorème de Brouwer en dimension \texorpdfstring{\(  2\)}{2}}
%---------------------------------------------------------------------------------------------------------------------------
Pour d'autres versions du théorème de Brouwer, voir la sous-section~\ref{subSecZCCmMnQ}.

\begin{theorem}[Brouwer en dimension \( 2\)\cite{KXjFWKA}]     \label{ThoLVViheK}
	Soit \( \mB=\overline{B(0,1)}\) la boule unité fermée de \( \eR^2\). Alors toute application continue de \( \mB\) dans elle-même admet un point fixe.
\end{theorem}
\index{théorème!Brouwer!dimension \( 2\)}
\index{connexité!utilisation!Brouwer}
\index{théorème!point fixe!Brouwer}

\begin{proof}
	Supposons que la fonction \( f\in C^0(\mB,\mB)\) n'admette pas de points fixes sur \( \mB=\overline{ B(0,1) }\). Pour \( x\in \mB\) nous notons \( g(x)\) l'intersection entre \( \partial \mB\) et la demi-droite allant de \( f(x)\) vers \( x\). C'est bien parce que \( f\) n'a pas de points fixes que \( g\) est bien définie.

	%TODOooVGSFooFlrqhG il faut le faire explicitement ici.
	En reprenant le même début de la preuve de la proposition~\ref{PropDRpYwv} nous savons que la fonction
	\begin{equation}
		\begin{aligned}
			g\colon \overline{ B(0,1) } & \to \partial B(0,1)                       \\
			x                           & \mapsto \lambda(x)\big( x-f(x) \big)+f(x)
		\end{aligned}
	\end{equation}
	est continue. De plus \( g(x)=x\) sur \( \partial B(0,1)\). Nous allons montrer qu'une telle fonction\footnote{Qui est nommée \emph{rétraction} de la sphère sur elle-même.} ne peut pas exister.

	Pour \( s\in\mathopen[ 0 , 1 \mathclose]\) nous paramétrons le cercle \( \partial B(0,s)\) par
	\begin{equation}
		\begin{aligned}
			x_s\colon \mathopen[ 0 , 1 \mathclose] & \to \partial B(0,s)                              \\
			t                                      & \mapsto \big( s\cos(2\pi t),s\sin(2\pi t) \big).
		\end{aligned}
	\end{equation}
	Ensuite nous considérons les chemins
	\begin{equation}
		\begin{aligned}
			\gamma_s\colon \mathopen[ 0 , 1 \mathclose] & \to \partial B(0,1)    \\
			t                                           & \mapsto g\circ x_s(t).
		\end{aligned}
	\end{equation}
	L'application \( \gamma_s\) est continue et \( \gamma_s(0)=\gamma_s(1)\). Les chemins \( \gamma_s\) sont des lacets; nous nous intéressons maintenant à l'indice au point \( 0\) de \( \gamma_0\) et \( \gamma_1\). D'une part \( \gamma_0(t)=g(0)\) (lacet constant) et \( \gamma_1(t)= e^{2i\pi t}\) (parce que \( g(x)=x\) sur le bord). Nous avons donc, en utilisant l'indice de la définition~\ref{DEFooLFBNooGlvJmp},
	\begin{equation}
		\Ind_{\gamma_0}(0)=\frac{1}{ 2\pi i }\int_{\gamma_0}\frac{ d\omega }{ \omega }=\frac{1}{ 2\pi i }\int_0^1\frac{ \gamma_0'(t) }{ \gamma_0(t) }dt=0,
	\end{equation}
	alors que
	\begin{equation}
		\Ind_{\gamma_1}(0)=\frac{1}{ 2\pi i }\int_0^1\frac{ 2i\pi e^{2i\pi t} }{  e^{2i\pi t} }dt=1.
	\end{equation}

	Nous considérons l'homotopie
	\begin{equation}
		\begin{aligned}
			\gamma\colon \mathopen[ 0 , 1 \mathclose]\times \mathopen[ 0 , 1 \mathclose] & \to \overline{ B(0,1) }              \\
			(s,t)                                                                        & \mapsto \gamma_s(t)=(g\circ x_s)(t).
		\end{aligned}
	\end{equation}
	Nous avons \( g(0)\neq 0\) parce que \( g\) prend ses valeurs sur le bord. Vu que c'est une équivalence d'homotopie\footnote{Définition~\ref{DefECnFJQp}} entre \( \gamma_1\) et \( \gamma_2\), les indices devraient être égaux par le corolaire~\ref{CorGZXzuZR}.
\end{proof}

%---------------------------------------------------------------------------------------------------------------------------
\subsection{Principe des zéros isolés}
%---------------------------------------------------------------------------------------------------------------------------

\begin{lemma}   \label{LEMooYYZQooClmOgG}
	Si \( f\) est une fonction holomorphe\footnote{Définition \ref{DEFooQSMCooOoWVZk}.} sur le compact \( K\), alors il existe un polynôme \( P_f\) et une fonction holomorphe \( h_f \) ne s'annulant pas sur \( K\) telles que \( f=h_fP_f \).
\end{lemma}

\begin{proof}
	Soit une fonction \( f\) vérifiant les conditions. Si \( f\) est identiquement nulle, alors il suffit de prendre \( P_f=0\) et c'est fait. Nous supposons donc que \( f\) n'est pas identiquement nulle.

	\begin{subproof}
		\spitem[Quantité finie de racines]

		D'abord \( f\) ne peut s'annuler qu'un nombre fini de fois sur \( K\). Sinon, on pourrait considérer une suite des racines\quext{Notez l'utilisation de la proposition~\ref{PROPooBYKCooGDkfWy} que je vous invite à ne pas considérer comme une trivialité absolue.} de \( f\) dans \( K\). Vu qu'une suite dans un compact contient une sous-suite convergente (théorème~\ref{ThoBWFTXAZNH}), la fonction \( f\) aurait un point d'accumulation de racines. Alors le principe des zéros isolés (théorème~\ref{ThoukDPBX}) nous donne un ouvert sur lequel \( f\) est nulle et donc le corolaire~\ref{CORooFBXXooZyfUQi} nous dit que \( f\) est identiquement nulle.

		\spitem[Autour d'une racine]

		Bref, la fonction \( f\) possède un nombre fini de racines sur \( K\). Soit \( z_0\) l'une d'elles.

		Par le théorème~\ref{ThomcPOdd}\ref{ITEMooYWSOooHJtxGr}, nous avons, sur un voisinage de \( z_0\) :
		\begin{equation}
			f(z)=\sum_{n=0}^{\infty}a_n(z-z_0)^n.
		\end{equation}
		En particulier, \( 0=f(z_0)=a_0\). Donc \( a_0=0\). Soit \( k\), le plus petit naturel pour lequel \( a_k\neq 0\). Nous avons
		\begin{equation}
			f(z)=(z-z_0)^kg(z)
		\end{equation}
		avec \( g(z)= \sum_{n=0}^{\infty}a_{k+n}(z-z_0)^n.\). Vu que \( a_{k}\neq 0\) nous avons \( g(z_0)\neq 0\). Montrons à présent que \( g\) est holomorphe sur un voisinage de \( z_0\). Vu que la série définissant \( g\) est une sous-série d'une série convergente sur un voisinage, elle converge sur un voisinage et la proposition~\ref{PropRZCKeO} nous dit que \( g\) est \( \eC\)-dérivable. C'est-à-dire holomorphe par définition.

		\spitem[Autour de toutes les racines]

		Soient \( (z_i)\) les racines (en nombre fini). Pour chaque \( i\) nous avons une boule \( B(z_i,r_i)\) sur laquelle \( f=P_ig_i\) où \( P_i\) est un polynôme de la forme \( (z-z_i)^k\) et \( g_i\) est holomorphe sur \( B(z_i,r_i)\). Nous définissons la fonction suivante :
		\begin{equation}
			h(z)=\begin{cases}
				\dfrac{ f(z) }{ \prod_kP_k(z) }        & \text{si } z\neq z_i \\
				\dfrac{ g_i }{ \prod_{k\neq i}P_k(z) } & \text{si } z=z_i.
			\end{cases}
		\end{equation}
		Cette fonction ne s'annule jamais. Mais est-elle holomorphe ?

		Si \( z\neq z_i\) (sous-entendu : pour tout \( i\)), alors sur un voisinage, \( h=f/\prod P_k\) qui est un quotient de fonctions holomorphes dont le dénominateur ne s'annule pas. Elle est donc holomorphe sur ce voisinage par le lemme~\ref{LEMooVDXOooUyFHXZ}.

		Pour les autres notons que pour tout \( z\in B(z_i,r_i)\),
		\begin{equation}
			h=\frac{ g_i }{\prod_{k\neq i}P_k}.
		\end{equation}
		Cela est encore un quotient dont le dénominateur ne s'annule pas\footnote{Nous avons choisi les \( r_i\) de telle sorte que les boules ne s'intersectent pas.}.

		\spitem[La réponse]

		Nous avons, pour tout \( z\in K\) :
		\begin{equation}
			f(z)=h(z)\prod_{k}P_k(z).
		\end{equation}

	\end{subproof}
\end{proof}

Afin de détendre l'atmosphère, nous allons laisser tomber l'analyse quelques instants et prouver un résultat d'algèbre.
\begin{proposition}[\cite{MonCerveau,ooACALooBfyhba}]       \label{PROPooVWRPooGQMenV}
	L'anneau des fonctions holomorphes sur un compact\footnote{Être holomorphe sur un compact signifie qu'il existe une extension holomorphe à un ouvert contenant le compact.} donné de \( \eC\) est principal\footnote{Définition~\ref{DEFooGWOZooXzUlhK}}.
\end{proposition}

\begin{proof}
	Nous nommons \( A\) l'ensemble des fonctions holomorphes sur le compact \( K\), et \( J\) un idéal de \( A\).

	\begin{subproof}

		\spitem[\( A\) est un anneau]

		Le point délicat de la définition \ref{DefHXJUooKoovob} est le fait que la somme et le produit d'éléments de \( A\) sont des éléments de \( A\) parce que les résultats type «la somme de deux fonctions holomorphes est holomorphes sont valides sur des ouverts alors que nous sommes ici sur un compact. Soient \( f\) et \( g\) dans \( A\); nous nommons \( \Omega_f\) et \( \Omega_g\) des ouverts contenant \( K\) tels que \( f\) est holomorphe sur \( \Omega_f\) et \( g\) sur \( \Omega_g\).

		L'ensemble \( \Omega_f\cap\Omega_g\) est un ouvert (intersection d'ouverts) contenant \( K\) et sur lequel \( f\) et \( g\) sont holomorphes. Donc \( f+g\) et \( fg\) y sont holomorphes.

		\spitem[Engendré par des polynômes]

		Pour chaque \( f\in J\) nous écrivons \( f=P_fh_f\) en vertu de la décomposition donnée par le lemme~\ref{LEMooYYZQooClmOgG}. Vu que \( h_f\) ne s'annule pas, \( 1/h_f\) est encore holomorphe sur \( K\) et nous déduisons que \( P_f=f/h_f\) est dans \( J\).  La partie
		\begin{equation}
			S=\{ P_f\tq f\in J \}
		\end{equation}
		est génératrice de \( J\) parce que, par construction, tous les éléments de \( J\) sont des produits d'éléments de \( S\) par des fonctions holomorphes sur \( K\) (donc, des éléments de \( A\)). Mais tous les éléments de \( S\) sont dans \( J\), donc \( (S)=J\).

		\spitem[Un polynôme pour tous les engendrer]

		Soit \( M\), l'idéal de \( \eC[X]\) engendré par \( S\). Attention : \( J\) est l'idéal de \( A\) engendré par \( S\). Mais l'idéal de \( \eC[X]\) engendré par \( S\) est peut-être autre chose.  Vu que \( \eC\) est un corps, le lemme~\ref{LEMooIDSKooQfkeKp} dit que \( \eC[X]\) est principal. Donc \( M\) est un idéal principal de \( \eC[X]\) et nous avons un polynôme \( p\in \eC[X]\) tel que
		\begin{equation}
			M=\eC[X]p.
		\end{equation}
		Si vous avez compris le chausse trappe, vous saurez pourquoi il faut écrire \( M=\eC[X]p\) et non utiliser l'écriture plus simple «\( M=(p)\)».

		\spitem[\( A\eC\lbrack X\rbrack=A\)]

		L'inclusion \( A\subset A\eC[X]\) est dûe au fait que \( 1\in \eC[X]\), et l'autre inclusion est le fait que \( \eC[X]\subset A\) alors que \( A\) est un anneau.

		\spitem[Suite des opérations]

		Nous avons :
		\begin{equation}
			J=AS\subset A\eC[X]p.
		\end{equation}
		Voilà une inclusion de montrée. Reste à faire l'autre.

		Vu que \( p\in J\) nous avons aussi \( Ap\subset J\). Et donc
		\begin{equation}
			A\eC[X]p = Ap\subset J.
		\end{equation}

		Avec ces deux inclusions, \( J=A\eC[X]p=Ap\). Donc \( J\) est engendré par un seul élément et est principal.
	\end{subproof}
\end{proof}

%---------------------------------------------------------------------------------------------------------------------------
\subsection{Prolongement de fonctions holomorphes}
%---------------------------------------------------------------------------------------------------------------------------

\begin{proposition} \label{PropDRnYkKP}
	Soient \( \Omega\) un ouvert de \( \eC\) et \( f\colon \Omega\to \eC\) une fonction holomorphe sur \( \Omega\setminus\{ a \}\) (\( a\in \Omega\)). Nous supposons qu'il existe \( r>0\) tel que \( f\) est bornée sur \( B(a,r)\cap\Omega\). Alors \( f\) se prolonge en une fonction holomorphe sur \( \Omega\).
\end{proposition}
Le théorème de prolongement de Riemann~\ref{ThoTLQOEwW} donnera plus d'informations.

\begin{proof}
	Nous définissons la fonction \( g\colon \Omega\to \eC\) par
	\begin{equation}
		g(z)=\begin{cases}
			(z-a)f(z) & \text{si } z\neq a \\
			0         & \text{si } z=a.
		\end{cases}
	\end{equation}
	Sur \( \Omega\setminus\{ a \}\), la fonction \( g\) est holomorphe (produit de fonctions holomorphes), et elle est continue en \( a\). Par conséquent elle est holomorphe sur \( \Omega\). Nous la développons en série entière sur une boule \( B(a,r)\) :
	\begin{equation}
		g(z)=\sum_{n=0}^{\infty}c_n(z-a)^n.
	\end{equation}
	Nous avons \( g(a)=c_0=0\). Nous posons
	\begin{equation}
		\varphi(z)=\sum_{n=0}^{\infty}c_{n+1}(z-a)^n.
	\end{equation}
	Si \( z\neq a\), alors \( \varphi(z)=f(a)\) parce que \( \varphi(z)=g(z)/(z-a)\). Mais \( \varphi\) est continue en \( a\), et donc holomorphe en \( a\).

	La fonction \( \varphi\) est par conséquent un prolongement holomorphe de \( f\) en \( a\).
\end{proof}

%---------------------------------------------------------------------------------------------------------------------------
\subsection{Théorème de Runge}
%---------------------------------------------------------------------------------------------------------------------------

Le théorème que nous allons prouver n'est en réalité qu'une partie de ce qui est usuellement appelle le théorème de Runge.
\begin{theorem}[Théorème de Runge]\index{théorème!Runge}     \label{ThoMvMCci}
	Soit \( K\), un compact de \( \eC\) tel que \( \complement K\) soit connexe. Si \( a\in \complement K\) alors la fonction
	\begin{equation}
		\varphi_a(z)=\frac{1}{ z-a }
	\end{equation}
	est limite uniforme de polynômes sur \( K\).
\end{theorem}
\index{connexité!théorème de Runge}
\index{approximation!polynomiale}

\begin{proof}
	Nous considérons \( P(K)\), l'adhérence des polynômes sur \( K\) pour la norme uniforme (sur \( K\)). Nous devons montrer que pour tout \( a\in \complement K\), la fonction \( \varphi_a\) est dans \( P(K)\). Pour cela nous considérons l'ensemble
	\begin{equation}
		A=\{ a\in\complement K\tq \varphi_a\in P(K) \}
	\end{equation}
	et nous allons montrer qu'il est à la fois non vide, ouvert et fermé dans le connexe \( \complement K\).

	Je répète : nous allons prouver l'ouverture et la fermeture \emph{pour la topologie de \( \complement K\)}. Nous n'allons pas prouver que \( A\) est un ouvert de \( \eC\). Ce qui sera par conséquent prouvé est que \( A=\complement K\).

	\begin{subproof}
		\spitem[Non vide] Soit \( R=\sup_{z\in K}| z |\) et \( a\in \complement K\) tel que \( | a |>R\). Nous avons
		\begin{equation}
			\varphi_a(z)=\frac{1}{ a }\frac{1}{ \frac{ z }{ a }-1 }
			=-\frac{1}{ a }\frac{1}{ 1-\frac{ z }{ a } }
			=-\frac{1}{ a }\sum_{k=0}^{\infty}\left( \frac{ z }{ a } \right)^k
			=-\sum_{k=0}^{\infty}\frac{ z^k }{ a^{k+1} }.
		\end{equation}
		Ici la convergence de la série et sa limite sont assurées par le fait que \( | z/a |<1\) par choix de \( R\) et \( a\). La suite de polynômes
		\begin{equation}
			P_n(z)=\sum_{k=0}^n\frac{ z^k }{ a^{k+1} }
		\end{equation}
		converge uniformément sur \( B(0,R)\) et en particulier sur \( K\). Donc \( P_n\to \varphi_a\).

		\spitem[Fermé]

		Nous allons montrer que la fermeture de \( A\) (dans \( \complement K\)) est incluse dans \( A\), et donc qu'elle est égale à \( A\) et donc que \( A\) est fermé. Par le lemme~\ref{LEMooCXHEooKNmLma}, la fermeture de \( A\) dans \( \complement K\) est l'ensemble \( \bar A\cap\complement K\) où \( \bar A\) est la fermeture de \( A\) au sens usuel.

		Bref, soit \( a\in \bar A\cap\complement K\), et montrons que \( \varphi_a\in \overline{ P(K) }\). Vu que \( P(K)\) est déjà une fermeture, nous aurons en fait \( \varphi_a\in P(K)\) et donc \( a\in A\), ce qui signifierait que \( \bar A\cap\complement A=A\) et donc que \( A\) est fermé.

		Au travail.

		Soit \( (a_n)\in A\) une suite convergente vers \( a\). Soit aussi \( d=d(a,K)\); on a \( d>0\) parce que \( K\) est compact et \( a\) est hors de \( K\) alors le complémentaire de \( K\) est ouvert. Nous choisissons en plus la suite \( a_n\) pour avoir \( | a_n-a |<\frac{ d }{2}\); au pire on prend la queue de suite. Soit \( z\in K\); nous avons
		\begin{equation}    \label{EqYHWQhI}
			| \varphi_{a_n}(z)-\varphi_a(z) |=\left| \frac{1}{ z-a_n }-\frac{1}{ z-a } \right| =  \left| \frac{ a_n-a }{ (z-a_n)(z-a) } \right|.
		\end{equation}
		Vu que \( a_n\in B(a,\frac{ d }{2})\) et que \( z\in K\) et \( d=d(a,K)\) nous avons \( | a_n-z |\geq \frac{ d }{2}\); et aussi \( | a-z |\geq \frac{ d }{2}\). Nous pouvons donc majorer \eqref{EqYHWQhI} par
		\begin{equation}
			| \varphi_{a_n}(z)-\varphi_a(z) |\leq 2\frac{ | a_n-a | }{ d^2 }.
		\end{equation}
		Donc nous avons
		\begin{equation}
			\| \varphi_a-\varphi_{a_n} \|_K\leq 2\frac{ | a_n-a | }{ d^2 }\to 0
		\end{equation}
		où la norme \( \| . \|_K\) est la norme supremum sur \( K\). Donc \( a\in \overline{ P(K) }=P(K)\) et \( A\) est fermé.

		\spitem[Ouvert] Vu que \( K\) est compact, il est fermé et donc \( \complement K\) est ouvert. Par conséquent, ainsi que précisé dans l'exemple~\ref{ExloeyoR}, les ouverts de \( \complement K\) sont les ouverts de \( \eC\) contenus dans \( \complement K\). Afin de prouver que \( A\) est ouvert, nous prenons  \( a\in A\) et nous cherchons une boule (au sens de \( \eC\)) autour de \( a\) qui serait incluse dans \( A\).

		Soit donc \( h\in \eC\) «petit» dans un sens que nous allons préciser plus tard. Encore une fois nous posons \( d=d(a,K)\). Nous avons
		\begin{equation}        \label{EqgBSxFB}
			\varphi_{a+h}(z)=\frac{1}{ z-a-h }=\frac{1}{ z-a }\frac{1}{ 1-\frac{ h }{ z-a } }=\sum_{k=0}^{\infty}\frac{ h^k }{ (z-a)^{k+1} }.
		\end{equation}
		Déjà ici nous demandons \( h<\sup_{z\in K}| z-a |\). Puisque \( | z-a |>d\), nous avons alors
		\begin{equation}
			| \varphi_{a+h}(z) |\leq \sum_{k=0}^{\infty}\frac{ h^k }{ d^{k+1} }<\infty.
		\end{equation}
		Cela pour dire que la somme à droite de \eqref{EqgBSxFB} converge bien pourvu que \( h\) soit bien petit. Nous pouvons donc poursuivre :
		\begin{equation}    \label{EqTSSdttylSDX}
			\varphi_{a+h}(z)=\sum_{k=0}^{\infty}\frac{ h^k }{ (z-a)^{k+1} }=\sum_{k=0}^{\infty}h^k\varphi_a(z)^{k+1}.
		\end{equation}
		Nous montrons maintenant que la convergence de la somme \eqref{EqTSSdttylSDX} est en réalité uniforme en \( z\). En effet
		\begin{subequations}
			\begin{align}
				\big| \varphi_{a+h}(z)-\sum_{k=0}^Nh^k\varphi_a(z)^{k+1} \big| & =\big| \sum_{k=N+1}^{\infty}h^k\varphi_a(z)^{k+1} \big| \\
				                                                               & \leq\sum_{k=N+1}^{\infty}| h |^k| \varphi_a(z) |^{k+1}.
			\end{align}
		\end{subequations}
		Étant donné que \( \varphi_a\) est continue sur le compact \( K\), elle y est majorée en module; on peut même être plus précis :
		\begin{equation}
			|\varphi_a(z)|=\frac{1}{ | z-a | }\leq \frac{1}{ d }.
		\end{equation}
		Nous pouvons donc écrire
		\begin{equation}
			\big| \varphi_{a+h}(z)-\sum_{k=0}^Nh^k\varphi_a(z)^{k+1} \big|\leq\frac{1}{ d }\sum_{k=N+1}^{\infty}\left| \frac{ h }{ d } \right|^k.
		\end{equation}
		Étant donné que la somme \( \sum_{k=0}^{\infty}| h/d |^k\) converge, la limite \( N\to \infty\) est nulle et nous avons
		\begin{equation}
			\lim_{N\to \infty} \| \varphi_{a+h}-\sum_{k=0}^Nh^k\varphi_a^{k+1} \|_K=0.
		\end{equation}
		Pour avoir \( \varphi_{a+h}\in P(K)\), il faut encore savoir si les fonctions \( \varphi_a^{k}\) sont dans \( P(K)\) pour tout \( k\). Dans ce cas pour chaque \( N\) la somme sera encore dans \( P(K)\) et \( \varphi_{a+h}\) sera limite uniforme d'éléments de \( P(K)\).

		Par hypothèse, \( \varphi_a\in P(K)\); soit \( P_n\) une suite de polynômes qui converge uniformément vers \( \varphi_a\). Nous allons montrer qu'alors la suite de polynômes \( P_n^k\) converge uniformément vers \( \varphi_a^k\). Soit \( n\) tel que \( \| P_n-\varphi_a \|_{K}\leq \epsilon\) et utilisons le produit remarquable\index{produit remarquable}
		\begin{equation}
			a^k-b^k=(a-b)\sum_{i=0}^{k-1}a^ib^{k-1-i}
		\end{equation}
		pour obtenir
		\begin{equation}
			| P_n(z)^k-\varphi_a(z)^k |\leq | P_n(z)-\varphi_a(z) |\sum_{i=0}^{k-1}| P_n(z)^i\varphi_a(z)^{k-1-i} |.
		\end{equation}
		Vu que \( P_n\) et \( \varphi_a\) sont continues sur le compact \( K\), on peut majorer la somme par une constante \( M\), et il restera
		\begin{equation}
			| P_n(z)^k-\varphi_a(z)^k |\leq M | P_n(z)-\varphi_a(z) |,
		\end{equation}
		ou encore
		\begin{equation}
			\| P_n^k-\varphi_a^k \|\leq M\epsilon.
		\end{equation}
		Cela prouve que \( \varphi_a^{k}\in P(K)\) et donc que \( \varphi_{a+h}\) est limite uniforme (sur \( K\)) d'éléments de \( P(K)\) et donc fait partie de \( P(K)\) lui aussi.

		Ceci achève de prouver que \( A\) est ouvert dans \( \complement K\).
		\spitem[Conclusion]

		L'ensemble \( A\) est non vide, ouvert et fermé dans \( \complement K\), donc il est égal à \( \complement K\). Le théorème est ainsi démontré.
	\end{subproof}
\end{proof}

%+++++++++++++++++++++++++++++++++++++++++++++++++++++++++++++++++++++++++++++++++++++++++++++++++++++++++++++++++++++++++++
\section{Intégrales de fonctions holomorphes}
%+++++++++++++++++++++++++++++++++++++++++++++++++++++++++++++++++++++++++++++++++++++++++++++++++++++++++++++++++++++++++++

Nous commençons par le lemme technique.
\begin{lemma}[\cite{Holomorphieus}]       \label{LemNAnweA}
	Soit \( f\) une fonction holomorphe sur \( B(z_0,r_0)\). Pour tout \( z\in B(z_0,r)\) (avec \( r<r_0\)) nous avons
	\begin{equation}
		| f'(z) |\leq \frac{ r }{ \big( r-| z-z_0 | \big)^2 }\max\big\{ f(z_0+r e^{i\theta}) \big\}_{\theta\in \eR}.
	\end{equation}
\end{lemma}

\begin{proof}
	Par translation nous pouvons supposer que \( z_0=0\). Étant donné que \( f\) est holomorphe, elle admet un développement en séries entières
	\begin{equation}
		f(z)=\sum_{n=0}^{\infty}a_nz^n
	\end{equation}
	et nous notons \( M=\max\{ f(z)\tq z\in \overline{ B(0,r) } \}\). Nous avons \( r^n| a_n |\leq M\). Par conséquent
	\begin{subequations}
		\begin{align}
			| f'(z) | & =\left| \sum_{n=1}^{\infty}na_nz^{n-1} \right|                            \\
			          & \leq\frac{1}{ r }\sum r^n| a_n |n\left( \frac{ | z | }{ r } \right)^{n-1} \\
			          & <\frac{ M }{ r }\sum n\left( \frac{ | z | }{ r } \right)^{n-1}.
		\end{align}
	\end{subequations}
	À ce point nous devons utiliser la série de l'exemple~\ref{ExGxzLlP}. Nous avons alors
	\begin{equation}
		| f'(z) |\leq \frac{ M }{ r }\frac{ 1 }{ \left( 1-\frac{ | z | }{ r } \right)^2 }=\frac{ Mr }{ (r-| z |)^2 }.
	\end{equation}
\end{proof}

%--------------------------------------------------------------------------------------------------------------------------- 
\subsection{Holomorphie sous l'intégrale}
%---------------------------------------------------------------------------------------------------------------------------

\begin{normaltext}
	Notez une grande différence entre les théorèmes \ref{ThopCLOVN} et \ref{ThoMWpRKYp} : la condition \eqref{EQooUHQYooHtwfML} demande de contrôler l'intégrabilité de la dérivée de \( f\), alors que la condition \eqref{EQooYILMooAlBQof} demande de contrôler l'intégrabilité de \( f\) elle-même. Oh oui, on voudrait faire de l'analyse, mais ces fonctions holomorphes \ldots tellement  \href{https://www.dragonball-multiverse.com/fr/page-1835.html}{déloyal}!!
\end{normaltext}

\begin{theorem}[Holomorphie sous l'intégrale\cite{Holomorphieus}] \label{ThopCLOVN}
	Soit un espace mesuré \( (\Omega,\mu)\), un ouvert \( A\) dans \( \eC\) et une fonction \( f\colon A\times \Omega\to \eC\). Nous voulons étudier la fonction
	\begin{equation}
		F(z)=\int_{\Omega}f(z,\omega)d\mu(\omega)
	\end{equation}
	pour tout \( z\in A\). Nous supposons que
	\begin{enumerate}
		\item
		      la fonction \( f(.,\omega)\) est holomorphe sur \( A\) pour chaque \( \omega\).
		\item
		      La fonction \( f(z,.)\) est mesurable sur \( (\Omega,\mu)\).
		\item
		      Pour tout compact \( K\subset A\), il existe une fonction \( g_K\colon \Omega\to \eR\) telle que \( | f(z,\omega) |\leq g_K(\omega)\) et telle que
		      \begin{equation}        \label{EQooYILMooAlBQof}
			      \int_{\Omega}g_K(\omega)d\mu(\omega)
		      \end{equation}
		      existe.
	\end{enumerate}
	Alors la fonction \( F\) est holomorphe et
	\begin{equation}
		F'(z)=\int_{\Omega}\frac{ \partial f }{ \partial z }(z,\omega)d\mu(\omega).
	\end{equation}
\end{theorem}

\begin{proof}
	Soient \( z_0\in A\) et \( r>0\) tels que \( K=\overline{ B(z_0,r) }\subset A\). Pour chaque \( \omega\in \Omega\) nous considérons la fonction
	\begin{equation}
		\begin{aligned}
			f_{\omega}\colon \overline{ B(z_0,r) } & \to \eC              \\
			z                                      & \mapsto f(z,\omega).
		\end{aligned}
	\end{equation}
	Étant donné que \( \overline{ B(z_0,r) }\) est compacte, la fonction \( | f_{\omega} |\) est majorée par un nombre que nous notons \( f_K(\omega)\) qui est indépendant de \( z\) (pour autant que \( z\in K\)). Nous désignons par \( S(z_0,r)\) la frontière de la boule \( B(z_0,r)\). Étant donné que la majoration est valable sur \( \overline{ B(z_0,r) }\), nous avons en particulier
	\begin{equation}
		| f_{\omega}(z) |\leq f_K(\omega)
	\end{equation}
	pour tout \( z\in S\). En utilisant la lemme~\ref{LemNAnweA} nous avons
	\begin{subequations}
		\begin{align}
			| f'_{\omega}(z) | & \leq \frac{ r }{ (r-| z-z_0 |)^2 }\max\{ f(z_0+r e^{i\theta}) \}_{\theta\in \eR} \\
			                   & \leq \frac{ rf_K(\omega) }{ (r-| z-z_0 |)^2 }.
		\end{align}
	\end{subequations}
	Cette majoration est valable pour tout \( z\in B(z_0,r)\). Si nous supposons de plus que \( z\in B(z_0,r/2)\)  nous avons
	\begin{equation}
		| f'(z) |\leq \frac{ rf_K(\omega) }{ \left( r-\frac{ r }{2} \right)^2 }=\frac{ 4 }{ r }f_K(\omega).
	\end{equation}
	Étant donné que la boule \( B(z_0,r/2)\) est convexe, la fonction \( f_{\omega}\) est Lipschitz et pour tout \( h\in \eC\) tel que \( | h |<r/2\) nous avons
	\begin{equation}
		\left| \frac{ f_{\omega}(z_0+h)-f_{\omega}(z_0) }{ h } \right| \leq \frac{ 4f_K(\omega) }{ r }.
	\end{equation}
	Soit maintenant une suite \( (h_n)\) qui converge vers \( 0\) dans \( \eC\). Nous considérons la suite de fonctions correspondantes
	\begin{equation}
		g_n(\omega)=\frac{ f(z_0+h_n,\omega)-f(z_0,\omega) }{ h_n }.
	\end{equation}
	Cette suite de fonctions vérifie la convergence ponctuelle
	\begin{equation}
		g_n(\omega)\to\frac{ \partial f }{ \partial z }(z_0,\omega).
	\end{equation}
	De plus \( g_n\) est une fonction (de \( \omega\)) dominée par \( \frac{ 4f_K }{ r }\) qui est intégrable. Par conséquent le théorème de la convergence dominée\footnote{Lebesgue, théorème \ref{ThoConvDomLebVdhsTf}.} nous indique que
	\begin{equation}
		\int_{\Omega}g_n(\omega)d\mu(\omega)\to \int_{\Omega}\frac{ \partial f }{ \partial z }(z_0,\omega)d\mu(\omega),
	\end{equation}
	tandis que
	\begin{equation}
		F'(z)=\lim_{n\to \infty} \frac{ F(z_0+h_n)-F(z_0) }{ h_n }=\lim_{n\to \infty} \int_{\Omega}g_n(\omega)d\mu(\omega).
	\end{equation}
\end{proof}

\begin{normaltext}
	Dans le corolaire suivant, l'intégrale sur \( \partial B(z,r)\) est une intégrale sur un chemin.
\end{normaltext}

\begin{corollary}       \label{CorNxTjEj}
	Si \( f\) est une fonction holomorphe sur un ouvert \( \Omega\) contenant la fermeture de la boule \( B(z_0,r)\), alors pour tout \( z\) dans \( B(z_0,\rho)\) (\( \rho<r\)) les dérivées de \( f\) s'expriment par la formule suivante :
	\begin{equation}        \label{EQooBPIQooNhOTtB}
		f^{(k)}(z)=\frac{k!}{ 2\pi i }\int_{\partial B(z_0,r)}\frac{ f(\xi) }{ (\xi-z)^{k+1} }d\xi.
	\end{equation}
\end{corollary}
\index{compacité}

\begin{proof}
	Nous faisons par récurrence.
	\begin{subproof}
		\spitem[Pour la dérivée première]
		Soit \( \rho<r\). Nous utilisons le théorème \ref{ThoUHztQe} pour écrire
		\begin{equation}
			f(z)=\frac{1}{ 2\pi i}\int_{\partial B(z_0,r)}\frac{ f(\xi) }{ \xi-z }d\xi.
		\end{equation}
		Ensuite nous dérivons sous l'intégrale en appliquant le théorème \ref{ThopCLOVN} à la fonction
		\begin{equation}
			\begin{aligned}
				g\colon B(z_0,\rho)\times \partial B(z_0,r) & \to \eC                           \\
				(z,\xi)                                     & \mapsto \frac{ f(\xi) }{ \xi-z }.
			\end{aligned}
		\end{equation}
		Étant donné que \( f\) est holomorphe, la fonction \( g\) est continue et donc bornée sur tout compact \( K\subset A\) par une constante \( M\) (qui dépend du compact choisi).  D'autre part, nous avons toujours \( | \xi-z |>r-\rho\) et donc
		\begin{equation}
			| g(z,\xi) |\leq \frac{ M }{ r-\rho }.
		\end{equation}
		La fonction constante \( g_K=\frac{ M }{ r-\rho }\) est évidemment intégrable. Le théorème conclut que \( f\) est holomorphe (cela, nous le savions déjà\footnote{Et cela fournit une preuve alternative à la réciproque du théorème de Cauchy : une fonction continue qui vérifie la formule de Cauchy est holomorphe.}), et
		\begin{equation}
			f'(z)=\frac{1}{ 2i\pi }\int_{\partial B}\frac{ f(\xi) }{ (\xi-z)^2 }d\xi.
		\end{equation}

		\spitem[Les dérivées suivantes]
		Pour la récurrence\cite{ooKZJHooZhNpkf} nous supposons que
		\begin{equation}
			f^{(k)}(z)=\frac{k!}{ 2i\pi }\int_{\partial B(z_0,r)}\frac{ f(\xi) }{ (\xi-z)^{k+1} }d\xi,
		\end{equation}
		et nous tentons de calculer \( f^{(k+1)}(z)\). Pour cela nous paramétrons l'intégrale de façon très usuelle :
		\begin{equation}
			f^{(k)}(z)=\frac{ k! }{ 2\pi i }\int_0^{2\pi}\frac{ f(r e^{it}) }{ (r e^{it}-z) }ir e^{it}dt.
		\end{equation}
		Nous permettons de permuter la \( \eC\)-dérivation (par rapport à \( z\)) et l'intégrale en vertu de la proposition~\ref{PROPooZCLYooUaSMWA} appliquée à la fonction
		\begin{equation}
			g(t,z)=\frac{ f(r e^{it}) }{ (r e^{it}-z)^{k+1} }ir e^{it}.
		\end{equation}
		Cela donne
		\begin{equation}
			f^{(k+1)}(z)=\frac{ k! }{ 2i\pi }\int_0^{2\pi}f(r e^{it})ir e^{it}\frac{ k+1 }{ (r e^{it}-z)^{k+1} }dt=\frac{ (k+1)! }{ 2i\pi }\int_{\partial B}\frac{ f(\xi) }{ (\xi-z)^{k+2} }d\xi.
		\end{equation}
	\end{subproof}
\end{proof}

\begin{theorem}     \label{THOooSULFooHTLRPE}
	Si \( f\) est une fonction holomorphe sur le disque ouvert \( B(z_0,R)\) alors
	\begin{equation}
		f(z)=\sum_{n=0}^{\infty}\frac{ f^{(n)}(z_0) }{ n! }(z-z_0)^n
	\end{equation}
	et cette série converge uniformément sur tout compact.
\end{theorem}

\begin{proof}
	Sans perte de généralité nous supposons que \( z_0=0\). La formule de Cauchy (théorème~\ref{ThoUHztQe}) fournit, pour \( z\in B(0,R)\),
	\begin{equation}
		f(z)=\frac{1}{ 2\pi i }\int_{\partial B}\frac{ f(\xi) }{ \xi-z }d\xi=\frac{1}{ 2\pi i }\int_{\partial B}\frac{ f(\xi) }{ 1-(z/\xi) }\frac{ d\xi }{ \xi }.
	\end{equation}
	En particulier notons que \( z\in B(0,R)\) alors que \( \xi\) est sur le bord de cette boule ouverte. Donc \( | \xi |>| z |\) pour tous les \( \xi\) et \( z\) qui interviennent. Nous utilisons la série géométrique
	\begin{equation}
		\frac{1}{ 1-(z/\xi) }=\sum_{n=0}^{\infty}\left( \frac{ z }{ \xi } \right)^n.
	\end{equation}

	\begin{subproof}
		\spitem[Permuter une intégrale et une somme]
		En utilisant la mesure de comptage\footnote{Définition~\ref{DEFooILJRooByDzhs}.} sur \( \eN\) (qui est \( \sigma\)-finie), nous pouvons écrire
		\begin{equation}        \label{EQooWOLOooFHSrsx}
			\int_{\partial B}\sum_{n=0}^{\infty}\frac{ z^nf(\xi) }{ \xi^{n+1} }d\xi= \int_{\partial B}\left( \int_{\eN}g(\xi,n)dm(n) \right)d\xi
		\end{equation}
		où \begin{equation}
			\begin{aligned}
				g\colon \partial B\times \eN & \to \eC                                  \\
				(\xi,n)                      & \mapsto \frac{ z^nf(\xi) }{ \xi^{n+1} }.
			\end{aligned}
		\end{equation}
		Nous allons permuter les intégrales en utilisant le théorème de Fubini, selon la procédure décrite en~\ref{NORMooKIRJooPvyPWQ}. Nous commençons par l'intégrale sur \( \eN\) :
		\begin{equation}        \label{EQooLQLOooKiSAKH}
			\int_{\eN}| g(n,\xi) |=| \frac{ f(\xi) }{ \xi } |\sum_{n\in \eN}| \frac{ z }{ \xi } |^n=\frac{1}{ R }| f(\xi) |\frac{1}{ 1-| z |/R }.
		\end{equation}
		Ici nous avons utilisé \( | \xi |=R\). Notons que \( z\) est fixé depuis longtemps à l'intérieur de la boule de rayon \( R\) de telle sorte que \( | z/\xi |\) est une constante strictement inférieure à \( 1\).

		L'intégrale sur \( \xi\in \partial B\) n'a pas à être effectuée explicitement : nous nous contentons de prouver qu'elle est finie.
		La fonction \( f\) est continue sur le compact \( \partial B\). Cela parce que \( B\) est une boule fermée dans l'ouvert \( \Omega\) sur lequel \( f\) est continue. Au final l'expression à droite de \eqref{EQooLQLOooKiSAKH} est bornée sur le compact \( \partial B\) et son intégrale donne un nombre fini.

		Tout ceci pour invoquer le corolaire~\ref{CorTKZKwP} qui nous indique que \( g\in L^1(\eN\times \partial B)\).

		Une fois \( g\) intégrable sur l'espace produit \( \eN\times \partial B\), nous pouvons utiliser Fubini~\ref{ThoFubinioYLtPI} pour permuter les intégrales.
	\end{subproof}

	Une fois la somme et l'intégrale permutées, nous avons
	\begin{equation} \label{EqXSgZGw}
		f(z)=\frac{1}{ 2\pi i }\sum_{n=0}^{\infty}\int_{\partial B}\frac{ z^nf(\xi) }{ \xi^{n+1} }
		=\sum_{n=0}^{\infty}\left( \frac{1}{ 2\pi i }\int_{\partial B}\frac{ f(\xi) }{ \xi^{n+1} } \right)z^n.
	\end{equation}
	Nous devons maintenant montrer que ce qui se trouve dans la grande parenthèse vaut \( f^{(n)}(0)/n!\). Cela est immédiat en comparant avec la formule \eqref{EQooBPIQooNhOTtB}.

\end{proof}

\begin{proposition}[Morera \cite{NEBgfg}]   \label{ThoRckxes}
	Soit \( \Omega\) ouvert dans \( \eC\) et \( f\) continue. Si
	\begin{equation}
		\int_{\partial T}f=0
	\end{equation}
	pour tout triangle (plein) \( T\) contenu dans \( \Omega\), alors \( f\) est holomorphe sur \( \Omega\).
\end{proposition}

\begin{proof}
	Il est suffisant de prouver que \( f\) est holomorphe sur toute boule ouverte \( B(a,r)\) incluse dans \( \Omega\). Nous posons, pour tout \( z\in B(a,r)\),
	\begin{equation}
		F(z)=\int_{[a,z]}f,
	\end{equation}
	et nous considérons le chemin triangulaire \( a\to z\to z+h\to a\) où \( h\in \eC\) est choisi assez petit pour que \( z+h\in B(a,r)\). L'intégrale sur le triangle étant nulle, nous avons
	\begin{equation}
		0=\int_{a\to z}f+\int_{z\to z+h}f+\int_{z+h\to a}f,
	\end{equation}
	c'est-à-dire
	\begin{equation}
		F(z+h)-F(z)=\int_{z\to z+h}f.
	\end{equation}
	En paramétrant le chemin par \( z+th\) avec \( t\in\mathopen[ 0 , 1 \mathclose]\), et en tenant compte de la remarque~\ref{RemiqswPd},
	\begin{subequations}
		\begin{align}
			F'(z) & =\lim_{h\to 0} \frac{ F(z+h)-F(z) }{ h }        \\
			      & =\lim_{h\to 0} \frac{1}{ h }\int_0^1f(z+th)hdt,
		\end{align}
	\end{subequations}
	ce qui prouve que \( F\) est dérivable et \( F'=f\). Par définition (\ref{DefMMpjJZ}), \( F\) est holomorphe, et donc \( C^{\infty}\) par le théorème~\ref{ThomcPOdd}. Du coup \( f\) est également \(  C^{\infty}\) et donc en particulier holomorphe.
\end{proof}

%---------------------------------------------------------------------------------------------------------------------------
\subsection{Mesure de Radon}
%---------------------------------------------------------------------------------------------------------------------------

Soit un compact \( K\) de \( \eC\).  Le lemme \ref{LemdLKKnd} nous dit que \( \big( C(K),\| . \|_{\infty} \big) \) est complet. Mais cela n'est pas vraiment utile pour l'instant. Je dis ça juste pour le garder dans un coin de la tête.

\begin{definition}      \label{DEFooTLRKooGbKkAj}
	Une \defe{mesure de Radon}{mesure!de Radon} sur un compact \(  K\) de \( \eC\) est une forme linéaire continue sur \( \big( C(K),\| . \|_{\infty} \big)\). Si \( \mu\) est une mesure de Radon, on définit la \defe{transformée de Cauchy}{transformée!de Cauchy} de \( \mu\) par
	\begin{equation}
		\begin{aligned}
			\hat \mu\colon \eC\setminus K & \to \eC                          \\
			z                             & \mapsto -\frac{1}{ \pi }\mu(f_z)
		\end{aligned}
	\end{equation}
	où, pour tout \( z\in \eC\setminus K\) nous avons défini la fonction \( f_z\) par
	\begin{equation}        \label{EQooNHTLooWkIofe}
		\begin{aligned}
			f_z\colon K & \to \eC                    \\
			\xi         & \mapsto \frac{1}{ \xi-z }.
		\end{aligned}
	\end{equation}
\end{definition}

\begin{normaltext}
	Les fonctions \( f_z\) sont définies pour \( z\in \eC\setminus K\), et elles sont définies sur \( K\). Donc il n'y a pas de danger que \( \xi=z\) dans le dénominateur de \eqref{EQooNHTLooWkIofe}. De plus si \( f_z\) est définie (c'est à dire si \( z\in \eC\setminus K\)), alors \( f_{z+h}\) est également définie tant que \( h\in \eC\) n'est pas trop grand parce que \( \eC\setminus K\) est un ouvert de \( \eC\).
\end{normaltext}

\begin{theorem}     \label{ThoJVNTzn}
	Si \( \mu\) est une mesure de Radon\footnote{Définition \ref{DEFooTLRKooGbKkAj}.} sur \( K\) alors \( \hat \mu\) est infiniment \( \eC\)-dérivable\footnote{Définition \ref{DEFooVJVXooKlnFkh}.} sur \( \Omega=\eC\setminus K\) et nous avons
	\begin{equation}
		\hat\mu^{(n)}(z)=-\frac{ n! }{ \pi }\mu\left( \frac{1}{ (\xi-z)^{n+1} } \right).
	\end{equation}
\end{theorem}

\begin{proof}
	Calculons la dérivée complexe de \( \hat \mu\). Si la limite existe, nous avons
	\begin{equation}
		\hat \mu'(z)=\lim_{_{\substack{h\to 0\\h\in \eC}}}\frac{ \hat \mu(z+h)-\hat\mu(z) }{ h }.
	\end{equation}
	Nous allons nous concentrer sur ce qui est dans la limite pour montrer qu'elle existe, et pour calculer la valeur. Nous avons
	\begin{subequations}
		\begin{align}
			\frac{ \hat\mu(z+h)-\hat\mu(z) }{ h } & =-\frac{1}{ \pi }\frac{ \mu(f_{z+h})-\mu(f_z) }{ h }                                             \\
			                                      & =-\frac{1}{ \pi }\mu\left( \frac{ f_{z+h}-f_z }{ h } \right)      & \text{\( \mu\) est linéaire} \\
			                                      & =-\frac{1}{ \pi }\mu\left( \frac{ 1 }{ (\xi-z-h)(\xi-z) } \right)
		\end{align}
	\end{subequations}
	où la dernière ligne est un abus de notations pour dire que \( \mu\) est appliquée à la fonction \( \xi\mapsto 1/(\xi-z-h)(\xi-z)\).

	Histoire d'éviter de regretables abus de notations, nous introduisons la fonction
	\begin{equation}
		\begin{aligned}
			\phi_{z,h}\colon K & \to \eC                               \\
			\xi                & \mapsto \frac{1}{ (\xi-z-h)(\xi-z) },
		\end{aligned}
	\end{equation}
	ainsi que la famille
	\begin{equation}
		\begin{aligned}
			f_{z,n}\colon K & \to \eC                        \\
			\xi             & \mapsto \frac{1}{ (\xi-z)^n }.
		\end{aligned}
	\end{equation}
	Ce que nous avons montré est que
	\begin{equation}
		\frac{ \hat \mu(z+h)-\hat \mu(z) }{ h }=-\frac{1}{ \pi }\mu(\phi_{z,h}).
	\end{equation}
	De plus nous avons la convergence
	\begin{equation}
		\phi_{z,h}\stackrel{\big( C(K),\| . \|_{\infty} \big)}{\longrightarrow}f_{z,2}
	\end{equation}
	lorsque \( h\to 0\). Vu que \( \mu\) est continue, nous pouvons permuter la limite avec \( \mu\) et avoir :
	\begin{equation}
		\hat\mu'(z)=\lim_{h\to 0} \mu(\phi_{z,h})=-\frac{1}{ \pi }\mu(f_{z,2}).
	\end{equation}
	Cela prouve le résultat pour \( n=1\).

	Pour le reste, vous faites un récurrence\quext{Je n'ai pas vérifié. Faites-le et écrivez-moi pour me dire si c'est bon.}.
\end{proof}

Cette théorie permet de fournir une démonstration plus technologique du corolaire~\ref{CorNxTjEj}.
\begin{lemma}
	Si \( f\) est holomorphe sur \( \Omega\) et si \( B\) est une boule fermée dans \( \Omega\) alors pour tout \( z\in \Int(B)\) nous avons
	\begin{equation}
		f^{(k)}(z)=\frac{ k! }{ 2i\pi }\int_{\partial B}\frac{ f(\xi) }{ (\xi-z)^{k+1} }d\xi.
	\end{equation}
\end{lemma}

\begin{proof}
	Appliquer le théorème~\ref{ThoJVNTzn} à la mesure de Radon
	%TODOooMPPRooMAomag: prouver que c'est une mesure de Radon
	\begin{equation}
		\mu(\phi)=\int_{\partial B}\phi(\xi)d\xi.
	\end{equation}
\end{proof}

Tout ce petit monde à propos de la mesure de Radon permet également de redémontrer que

\begin{equation}
	\left( \frac{1}{ 2\pi i }\int_{\partial B}\frac{ f(\xi) }{ \xi^{n+1} } \right)=f^{(n)}(0)/n!,
\end{equation}
comme nous l'avons déjà fait autour de l'équation \eqref{EqXSgZGw}. Nous utilisons le théorème de Radon~\ref{ThoJVNTzn} à la mesure
\begin{equation}
	\mu(\phi)=\int_{\partial B}\phi(\xi)d\xi.
\end{equation}
La transformée de Cauchy est
\begin{equation}        \label{EqTzkmeL}
	\hat \mu(z)=-\frac{1}{ \pi }\mu\left( \frac{1}{ \xi-z } \right)=-\frac{1}{ \pi }\int_{\partial B}\frac{1}{ \xi-z }d\xi,
\end{equation}
et le théorème assure que
\begin{equation}
	\hat\mu^{(n)}(z)=-\frac{ n! }{ \pi }\mu\left( \frac{1}{ (\xi-z)^{n+1} } \right)=-\frac{ n! }{ \pi }\int_{\partial B}\frac{ 1 }{ (\xi-z)^{n+1} }d\xi.
\end{equation}
En comparant la formule \eqref{EqTzkmeL} avec la formule de Cauchy nous voyons que \( \hat\mu(z)=-2i f(z)\). Par conséquent
\begin{equation}
	f^{(n)}(z)=-\frac{1}{ 2i }\hat\mu^{(n)}(z)=\frac{ n! }{ 2\pi i }\int_{\partial B}\frac{1}{ (\xi-z)^{n+1} }d\xi,
\end{equation}
et
\begin{equation}
	f^{(n)}(0)=\frac{ n! }{ 2\pi i }\int_{\partial B}\frac{1}{ \xi^{n+1} }d\xi.
\end{equation}

%+++++++++++++++++++++++++++++++++++++++++++++++++++++++++++++++++++++++++++++++++++++++++++++++++++++++++++++++++++++++++++
\section{Conditions équivalentes à l'holomorphie}
%+++++++++++++++++++++++++++++++++++++++++++++++++++++++++++++++++++++++++++++++++++++++++++++++++++++++++++++++++++++++++++

Nous nous proposons de lister les conditions que nous avons vues être équivalentes à l'holomorphie.

\begin{theorem}         \label{THOooOGOCooUalFaG}
	Soient \( \Omega\) un ouvert de \( \eC\) et \( f\colon \Omega\to \eC\) une fonction continue. Les conditions suivantes sont équivalentes.
	\begin{enumerate}
		\item   \label{ItemOtPcTb}
		      \( f\) est holomorphe.
		\item   \label{ItemHWRnxx}
		      Pour tout triangle (plein) \( T\) contenu dans \( \Omega\), \( \int_Tf=0\).
		\item   \label{ItempBBPVv}
		      \( f\) est \( \eC\)-dérivable.
		\item   \label{ItemmLhzbB}
		      \( f\) est \(  C^{\infty}\)
		\item   \label{ItemCCrSrLj}
		      \( \partial_{\bar z}f  =0\); ce sont les équations de Cauchy-Riemann.
		\item   \label{ItemEvxRSn}
		      La \( 1\)-forme différentielle \( f(z)dz\) est localement exacte.
		\item   \label{ItemVSCHtY}
		      Pour toute boule \( B(a,r)\) contenue dans \( \Omega\) nous avons
		      \begin{equation}
			      f(a)=\frac{1}{ 2\pi i }\int_{\partial B(a,r)}\frac{ f(z) }{ z-a }dz.
		      \end{equation}
	\end{enumerate}

	La fonction \( f\) est holomorphe en \( z_0\) si et seulement si il existe un voisinage \( B(z_0,r)\) de \( z_0\) et des nombres \( a_k\) tels que sur la boule,
	\begin{equation}
		f(z)=\sum_{n=0}^{\infty}a_n(z-z_0)^n.
	\end{equation}
	Dans ce cas, \( f\) est holomorphe sur toute la boule.
\end{theorem}

\begin{proof}
	~\ref{ItemOtPcTb} implique~\ref{ItemHWRnxx} est le lemme de Goursat~\ref{LemwbwbUR}.~\ref{ItemHWRnxx} implique~\ref{ItemOtPcTb} est le théorème de Morera~\ref{ThoRckxes}.

	~\ref{ItempBBPVv} est la définition de l'holomorphie, définition~\ref{DefMMpjJZ}.

	~\ref{ItemmLhzbB} implique~\ref{ItemOtPcTb} est un a fortiori sur la définition.~\ref{ItemOtPcTb} implique~\ref{ItemmLhzbB} est contenu dans le théorème de développement en série entière~\ref{ThomcPOdd}.

	L'équivalence entre~\ref{ItemCCrSrLj} et l'holomorphie est le théorème~\ref{PropkwIQwg}.

	L'équivalence entre~\ref{ItemEvxRSn} et~\ref{ItemOtPcTb} est la proposition~\ref{PropZOkfmO}.

	L'équivalence entre~\ref{ItemOtPcTb} et~\ref{ItemVSCHtY} est d'une part le théorème~\ref{ThomcPOdd} et d'autre part le corolaire~\ref{CorwfHtJu}.

	En ce qui concerne la dernière affirmation, si \( f\) est holomorphe en \( z_0\), alors le théorème~\ref{ThomcPOdd}\ref{ITEMooYWSOooHJtxGr} donne la série. Si au contraire nous avons la série, la proposition~\ref{PropRZCKeO} nous donne le résultat grâce au point \ref{ItempBBPVv}.
\end{proof}


%+++++++++++++++++++++++++++++++++++++++++++++++++++++++
\section{Limites, séries et dérivée}
%+++++++++++++++++++++++++++++++++++++++++++++++++++++++

\begin{proposition}[\cite{BIBooYQTTooOscORo}]	\label{PROPooKHBHooAIsbCG}
	Soit un ouvert \( \Omega\subset \eC\). Soit une suite \( (f_n)\) de fonctions holomorphes sur \( \Omega\). Nous supposons que \( f_n\) converge uniformément sur tout compact vers une fonction \( f\).

	Alors
	\begin{enumerate}
		\item
		      \( f\) est holomorphe.
		\item
		      Pour tout \( k\geq 1\) nous avons \( f_n^{(k)}\to f^{(k)}\) uniformément sur tout compact.
	\end{enumerate}
	%TODOooYMPPooUoDBUG. Prouver ça.
\end{proposition}


\begin{proposition}[\cite{MonCerveau}]	\label{PROPooMIDPooUgSXRp}
	Soit une suite \( (f_n \colon \Omega\to \eC)_{n\in \eN}\) de fonctions holomorphes. Si la série
	\begin{equation}
		f(z)=\sum_{n=0}^{\infty}f_n(z)
	\end{equation}
	converge uniformément, la fonction \( f\) est holomorphe sur \( \omega\).

	De plus
	\begin{equation}
		f'(z)=\sum_{n=0}^{\infty}f_n'(z).
	\end{equation}
	%TODOooZQNUooCUHjGY. Prouver ça.
\end{proposition}
