% This is part of Mes notes de mathématique
% Copyright (c) 2011-2020, 2023
%   Laurent Claessens, Carlotta Donadello
% See the file fdl-1.3.txt for copying conditions.

%+++++++++++++++++++++++++++++++++++++++++++++++++++++++++++++++++++++++++++++++++++++++++++++++++++++++++++++++++++++++++++
\section{Tribu produit}
%+++++++++++++++++++++++++++++++++++++++++++++++++++++++++++++++++++++++++++++++++++++++++++++++++++++++++++++++++++++++++++

%---------------------------------------------------------------------------------------------------------------------------
\subsection{Produit d'espaces mesurables}
%---------------------------------------------------------------------------------------------------------------------------

\begin{definition}      \label{DefTribProfGfYTuR}
	Si \( \tribA_1\) et \( \tribA_2\) sont deux tribus sur deux ensembles \( \Omega_1\) et \( \Omega_2\), nous définissons la \defe{tribu produit}{tribu!produit} \( \tribA_1\otimes\tribA_2\) comme étant la tribu engendrée par
	\begin{equation}
		\{ X\times Y\tq X\in\tribA_1,Y\in\tribA_2 \}.
	\end{equation}
	Ces ensembles sont appelés \defe{rectangles}{rectangle!produit de tribus} de \( (\Omega_1,\tribA_1)\otimes (\Omega_2,\tribA_2)\).
\end{definition}

\begin{proposition}[\cite{KEQWooJsCGiw}]        \label{PropLJJWooKqWlTr}
	Soient deux espaces mesurables \( (S_1,\tribF_1)\) et \( (S_2,\tribF_2)\). Si \( \tribC_i\) est une classe de parties de \( S_i\) avec \( \tribF_i=\sigma(\tribC_i)\) et \( S_i\in\tribC_i\). Alors
	\begin{equation}
		\tribF_1\otimes \tribF_2=\sigma(\tribC_1\times \tribC_2).
	\end{equation}
\end{proposition}

\begin{proof} 
	Nous notons \( p_1\) et \( p_2\) les projections de \( S_1\times S_2\) vers \( S_1\) et \( S_2\). Nous commençons par prouver que
	\begin{equation}    \label{eqSGPBooLpQHfq}
		\tribF_1\otimes \tribF_2=\sigma\big( p_1^{-1}(\tribF_1)\cup p_2^{-1}(\tribF_2) \big).
	\end{equation}
	En effet cette union est dans \( \tribF_1\otimes \tribF_2\) parce que ce sont tous des produits de la forme \( A_1\times S_2\) et \( S_1\times A_2\) où \( A_i\in \tribF_i\). Inversement, tous les produits de la forme \( A_1\times A_2\) sont dans la tribu engendrée par l'union parce que
	\begin{equation}
		A_1\times A_2=(A_1\times S_2)\cap(S_1\times A_2).
	\end{equation}
	Par conséquent, la partie \( p_1^{-1}(\tribF_1)\cup p_2^{-1}(\tribF_2)\) engendre tous les produits qui \href{https://fr.wikisource.org/wiki/Bible_Crampon_1923/Matthieu}{ engendrent } la tribu \( \tribF_1\otimes\tribF_2\). L'égalité \eqref{eqSGPBooLpQHfq} est donc correcte.

	Si \( C_1\in\tribC_1\) alors
	\begin{equation}
		p_1^{-1}(C_1)=C_1\times S_2\in\tribC_1\times \tribC_2
	\end{equation}
	et donc \( p_1^{-1}(\tribC_1)\subset \tribC_1\times \tribC_2\). En utilisant le lemme de transport \ref{LemOQTBooWGYuDU} nous avons alors
	\begin{equation}        \label{EqDQLYooVOLqMZ}
        p_1^{-1}(\tribF_1)=p_1^{-1}\big( \sigma(\tribC_1) \big)=\sigma\big( p_1^{-1}(\tribC_1) \big)\subset\sigma(\tribC_1\times \tribC_2)
	\end{equation}
	et de la même façon,
	\begin{equation}        \label{EqMTRCooVHNTHJ}
		p_2^{-1}(\tribF_2)\subset\sigma(\tribC_1\times \tribC_2).
	\end{equation}

	Vu les relations \eqref{EqDQLYooVOLqMZ}, \eqref{EqMTRCooVHNTHJ} et \eqref{eqSGPBooLpQHfq} nous avons
	\begin{equation}
		\tribF_1\otimes\tribF_2=\sigma\big( p_1^{-1}(\tribF_1)\cup p_2^{-1}(\tribF_2) \big)\subset\sigma(\tribC_1\times \tribC_2).
	\end{equation}

	Réciproquement, si \( C_1\in \tribC_1\) et \( C_2\in \tribC_2\) alors
	\begin{equation}
		C_1\times C_2=(C_1\times S_2)\cap(S_1\times C_2)=p_1^{-1}(C_1)\cap p_2^{-1}(C_2)\in\tribF_1\otimes\tribF_2.
	\end{equation}
\end{proof}

%---------------------------------------------------------------------------------------------------------------------------
\subsection{Le cas des boréliens}
%---------------------------------------------------------------------------------------------------------------------------

Si \( X_1\) et  \( X_2\) sont des espaces topologiques et si nous notons \( \mO_i\) l'ensemble de leurs ouverts, par définition \( \Borelien(X_i)=\sigma(\mO_i)\). De plus par la proposition~\ref{PropLJJWooKqWlTr} nous savons que
\begin{equation}        \label{EqOHMSooRSLrDk}
	\sigma(\mO_1\times \mO_2)=\Borelien(X_1)\otimes \Borelien(X_2).
\end{equation}

\begin{lemma}       \label{LemDEDQooJyzXgC}
	Si \( (X_i,\mO_i)\) sont des espaces topologiques, alors
	\begin{equation}
		\Borelien(X_1)\otimes \Borelien(X_2)\subset \Borelien(X_1\times X_2)
	\end{equation}
\end{lemma}

\begin{proof}
	Si \( A_i\in \mO_i\) alors \( A_1\times A_2\) est un ouvert de \( X_1\times X_2\) (voir la définition~\ref{DefIINHooAAjTdY}). Par conséquent, \( \mO_1\times \mO_2\) est contenu dans l'ensemble des ouverts de \( X_1\times X_2\) ou encore
	\begin{equation}
		\mO_1\times \mO_2\subset\Borelien(X_1\times X_2),
	\end{equation}
	et donc
	\begin{equation}
		\sigma(\mO_1\times \mO_2)\subset\sigma\big( \Borelien(X_1\times X_2) \big)
	\end{equation}
	finalement, par \eqref{EqOHMSooRSLrDk}
	\begin{equation}
		\Borelien(X_1)\otimes\Borelien(X_2)\subset\Borelien(X_1\times X_2).
	\end{equation}
\end{proof}

Il n'y a en général pas égalité, mais nous allons immédiatement voir que dans (presque) tous les cas raisonnables, les boréliens sur un produit sont le produit des boréliens.

\begin{proposition}[\cite{KEQWooJsCGiw}]        \label{PropNAAJooBPbjkX}
	Soient \( (X_1,d_1)\) et \( (X_2,d_2)\) des espaces métriques séparables. Alors
	\begin{equation}
		\Borelien(X_1\times X_2)=\Borelien(X_1)\otimes \Borelien(X_2).
	\end{equation}
\end{proposition}

\begin{proof}
	Nous savons par le lemme~\ref{LemDUJXooWsnmpL} que tout ouvert de \( X_1\times X_2\) est une réunion dénombrable d'éléments de \( \mO_1\times\mO_2\). Donc tout ouvert de \( X_1\times X_2\) est dans \( \Borelien(X_1)\otimes \Borelien(X_2)\). Par conséquent
	\begin{equation}
		\Borelien(X_1\times X_2)\subset \Borelien(X_1)\otimes \Borelien(X_2).
	\end{equation}
	L'inclusion inverse étant déjà acquise par le lemme~\ref{LemDEDQooJyzXgC}, nous avons l'égalité.
\end{proof}

\begin{proposition}     \label{CorWOOOooHcoEEF}
	Les boréliens sur \( \eR^N\) sont ceux qu'on croit.
	\begin{enumerate}
		\item
		      \( \Borelien(\eR^2)=\Borelien(\eR)\otimes \Borelien(\eR)\)
		\item
		      \( \Borelien(\eR^{N+1})=\Borelien(\eR^N)\otimes \Borelien(\eR)\)
	\end{enumerate}
\end{proposition}

\begin{proof}
	Cela n'est rien d'autre que la proposition~\ref{PropNAAJooBPbjkX}.
\end{proof}

\begin{proposition}
	Soit un espace mesurable \( (S,\tribF)\) et des applications \( f_k\colon S\to \eR\) (\( k=1,\ldots, N\)). Alors l'application
	\begin{equation}
		\begin{aligned}
			f\colon (S,\tribF) & \to (\eR^N,\Borelien(\eR^N))              \\
			x                  & \mapsto \big( f_1(x),\ldots, f_N(x) \big)
		\end{aligned}
	\end{equation}
	est mesurable si et seulement si chacun des \( f_i\) est mesurable.
\end{proposition}

\begin{proof}
	Division en deux.
	\begin{subproof}
		\spitem[Condition nécessaire]
		Nous supposons que les \( f_i\) sont mesurables. Nous avons
		\begin{subequations}
			\begin{align}
				f^{-1}\big( \prod_{k=1}^N\mathopen] a_k , b_k \mathclose[ \big) & =\{ x\in S\tq f_1(x)\in\mathopen] a_1 , b_1 \mathclose[ ,\cdots f_N(x)\in\mathopen] a_N , b_N \mathclose[\} \\
				                                                                & =\bigcap_{k=1}^Nf_k^{-1}\big( \mathopen] a_k , b_k \mathclose[ \big).
			\end{align}
		\end{subequations}
		Cela est une intersection finie d'éléments de \( \tribF\) et est donc un élément de \( \tribF\). Mais les pavés ouverts engendrent \( \Borelien(\eR^N)\) parce qu'ils sont une base dénombrable de la topologie (proposition~\ref{PROPooYEkvbWBz}). Le théorème~\ref{ThoECVAooDUxZrE} nous assure alors que \( f\) est mesurable parce que l'image inverse d'une base de la tribu est mesurable.
		\spitem[Condition suffisante]
		Si \( f\) est mesurable alors en particulier
		\begin{equation}
			f_k^{-1}\big( \mathopen] a , b \mathclose[ \big)=f^{-1}\big( \eR\times\ldots\times \mathopen] a , b \mathclose[\times \eR\times\ldots\times \eR \big)\in\tribF.
		\end{equation}
		Pour cela nous avons utilisé la proposition~\ref{CorWOOOooHcoEEF} qui nous indique que le produit dans la parenthèse est un borélien de \( \eR^N\) en tant que produit de boréliens de \( \eR\).

		Encore une fois \( f_k^{-1}\) tombe dans \( \tribF\) pour une base dénombrable de la topologie de \( \eR\) et est donc mesurable.
	\end{subproof}
\end{proof}

