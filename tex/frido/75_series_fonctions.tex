% This is part of Mes notes de mathématique
% Copyright (c) 2008, 2009, 2011-2021
%   Laurent Claessens
% See the file fdl-1.3.txt for copying conditions.

Les généralités sur les suites et séries de fonctions, c'est dans la section \ref{SECooTDZNooJvjPks}.

%+++++++++++++++++++++++++++++++++++++++++++++++++++++++++++++++++++++++++++++++++++++++++++++++++++++++++++++++++++++++++++ 
\section{Séries de fonctions}
%+++++++++++++++++++++++++++++++++++++++++++++++++++++++++++++++++++++++++++++++++++++++++++++++++++++++++++++++++++++++++++

%---------------------------------------------------------------------------------------------------------------------------
\subsection{Intégration de séries de fonctions}
%---------------------------------------------------------------------------------------------------------------------------

\begin{theorem}      \label{ThoCciOlZ}
    La somme uniforme de fonctions intégrables sur un ensemble de mesure fini est intégrable et on peut permuter la somme et l'intégrale.

    En d'autres termes, supposons que \( \sum_{n=0}^{\infty}f_n\) converge uniformément vers \( F\) sur \( A\) avec \( \mu(A)<\infty\). Si \( F\) et \( f_n\) sont des fonctions intégrables sur \( A\) alors
    \begin{equation}
        \int_AF(x)d\mu(x)=\sum_{n=0}^{\infty}\int_Af_n(x)d\mu(x).
    \end{equation}
\end{theorem}
\index{permuter!somme et intégrale}

\begin{proof}
    Ce théorème est une conséquence du théorème~\ref{ThoUnifCvIntRiem}. En effet nous définissons la suite des sommes partielles
    \begin{equation}
        F_N=\sum_{n=0}^Nf_n.
    \end{equation}
    La limite \( \lim_{N\to \infty} F_N=F\) est uniforme. Par conséquent la fonction \( F\) est intégrable et
    \begin{equation}
        \int_A F=\lim_{N\to \infty} \int_AF_N=\lim_{N\to \infty} \int_A\sum_{n=0}^Nf_n=\lim_{N\to \infty} \sum_{n=0}^N\int_Af_n=\sum_{n=0}^{\infty}\int_Af_n.
    \end{equation}
    La première égalité est le théorème~\ref{ThoUnifCvIntRiem}, les autres sont de simples manipulations rhétoriques.
\end{proof}

Le théorème suivant est une paraphrase du théorème de la convergence dominée de Lebesgue (\ref{ThoConvDomLebVdhsTf}).
\begin{theorem}     \label{ThoockMHn}
    Soient des fonctions \( (f_n)_{n\in \eN}\) telles que \( \sum_{n=0}^Nf_n\) soit intégrable sur \( (\Omega,\tribA,\mu)\) pour chaque \( N\). Nous supposons que la somme converge simplement vers
    \begin{equation}
        f(x)=\sum_{n=0}^{\infty}f_n(x)
    \end{equation}
    et qu'il existe une fonction \( g\) telle que
    \begin{equation}
        \left| \sum_{n=0}^Nf_n \right| <g
    \end{equation}
    pour tout \( N\in \eN\). Alors
    \begin{enumerate}
        \item
            \( \sum_{n=0}^{\infty}f_n\) est intégrable,
        \item
            on peut permuter somme et intégrale :
            \begin{equation}
                \lim_{N\to \infty} \int_{\Omega}\sum_{n=0}^Nf_nd\mu=\int_{\Omega}\sum_{n=0}^{\infty}f_n,
            \end{equation}
        \item
            \begin{equation}
                \lim_{N\to \infty} \int_{\Omega}\left| \sum_{n=0}^Nf_n-\sum_{n=0}^{\infty}f_n \right| =\lim_{N\to \infty} \int_{\Omega}\left| \sum_{n=N}^{\infty}f_n \right| =0.
            \end{equation}
    \end{enumerate}
\end{theorem}


\begin{theorem} \label{ThoCSGaPY}
    Soit \( f_n\) des fonctions \( C^1\mathopen[ a , b \mathclose]\) telles que
    \begin{enumerate}
        \item
            la série \( \sum_n f_n(x_0)\) converge pour un certain \( x_0\in\mathopen[ a , b \mathclose]\),
        \item
            la série des dérivées \( \sum_n f'_n\) converge uniformément sur \( \mathopen[ a , b \mathclose]\).
    \end{enumerate}
    Alors la série \( \sum_n f_n\) converge vers une fonction \( F\) et
    \begin{enumerate}
        \item
            La convergence est uniforme sur \( \mathopen[ a , b \mathclose]\).
        \item
            La fonction \( F\) est dérivable
        \item
            \( F'(x)=\sum_nf'_n(x)\).
    \end{enumerate}
\end{theorem}

%---------------------------------------------------------------------------------------------------------------------------
\subsection{Différentiabilité}
%---------------------------------------------------------------------------------------------------------------------------

\begin{lemma}
    Soient \( E\) et \( F\) deux espaces vectoriels normés. Si la suite \( (T_n)\) converge vers \( T\) dans \( \aL(E,F)\), alors pour tout \( v\in E\) nous avons
    \begin{equation}
        \left( \sum_{n=0}^{\infty}T_n \right)(v)=\sum_{n=0}^{\infty}T_n(v).
    \end{equation}
\end{lemma}

\begin{probleme}
    À mon avis si on a un ouvert connexe par arcs dans un espace vectoriel normé, alors il est connexe par arcs de classe \( C^1\), c'est-à-dire que deux points peuvent être liés par un chemin de classe \( C^1\).

    Je n'en suis pas certain.

    Si vous êtes sûr de vous, vous pouvez affaiblir les hypothèses du théorème \ref{ThoLDpRmXQ} et supprimer la définition \ref{DEFooHOXOooKUqTQU} qui ne sert à rien d'autre.
\end{probleme}

\begin{definition}      \label{DEFooHOXOooKUqTQU}
    Soit un espace vectoriel normé \( E\). Un ouvert \( \Omega\) est dit connexe par arcs de classe \( C^1\) si pour tout choix de \( a,b\in \Omega\), il existe une application \( \gamma\colon \mathopen[ 0 , 1 \mathclose]\to \Omega\) de classe \( C^1\) telle que \( \gamma(0)=a\) et \( \gamma(1)=b\).
\end{definition}

\begin{probleme}
    Le théorème \ref{ThoLDpRmXQ} se démontre ici avec des intégrales. Je suis presque certain qu'on doit pouvoir adapter la démonstration du théorème \ref{ThoSerUnifDerr} pour ne pas avoir à utiliser d'intégrales.
\end{probleme}

\begin{theorem}[\cite{DHdwZRZ}] \label{ThoLDpRmXQ}
    Soient \( E\) et \( F\) deux espaces vectoriels normés, \( \Omega\) un ouvert connexe par arcs de classe \( C^1\) de \( E\). Soit \( (u_n)\) une suite de fonctions \( u_n\colon \Omega\to F\) telle que
    \begin{enumerate}
        \item
            pour tout \( n\), la fonction \( u_n\) est de classe \( C^1\) sur \( \Omega\),
        \item
            la série \( \sum_nu_n\) converge simplement sur \( \Omega\),
        \item
            la série des différentielles \( \sum_n(du_n)\) converge normalement sur tout compact de \( \Omega\).
    \end{enumerate}
    Alors la somme \( u=\sum_nu_n\) est de classe \( C^1\) sur \( \Omega\) et sa différentielle est donnée par
    \begin{equation}
        du=\sum_{n=0}^{\infty}du_n.
    \end{equation}
\end{theorem}

\begin{proof}
    Pour chaque \( n\), la fonction \( du_n\colon \Omega\to \aL(E,F)\) est une fonction continue parce que \( u_n\) est de classe \( C^1\). La série convergeant normalement, la fonction \( \sum_{n=0}^{\infty}du_n\) est également continue par la proposition~\ref{PropOMBbwst}. La difficulté de ce théorème est donc de prouver que cela est bien la différentielle de la fonction \( \sum_nu_n\), c'est-à-dire que
    \begin{equation}
        d\left( \sum_{n=0}^{\infty}u_n \right)=\sum_{n=0}^{\infty}du_n.
    \end{equation}

    Soient \( a,x\in \Omega\). Nous considérerions bien le segment \( \mathopen[ a , x \mathclose]\), mais vu que \( \Omega\) n'est supposé que connexe par arcs de classe \( C^1\) (définition \ref{DEFooHOXOooKUqTQU}), nous ne pouvons pas faire mieux pour joindre \( a\) à \( x\) que choisir un chemin de classe \( C^1\)
    \begin{equation}
        \gamma\colon \mathopen[ 0 , 1 \mathclose]\to \Omega
    \end{equation}
    tel que \( \gamma(0)=a\) et \( \gamma(1)=b\).

    L'astuce est de poser
    \begin{equation}
        \begin{aligned}
            f_n\colon \mathopen[ 0 , 1 \mathclose]&\to \eR \\
            t&\mapsto (du_n)_{\gamma(t)}\big( \gamma'(t) \big),
        \end{aligned}
    \end{equation}
    et d'en étudier l'intégrale\footnote{Cela revient à étudier l'intégrale de la forme différentielle \( du_n\) sur le chemin \( \gamma\). Voir la définition \ref{DEFooRMHGooFtMEPB} et tout ce qui s'en suit.}.

    \begin{subproof}
        \item[Permuter somme et intégrale]
            Nous voudrions permuter la somme et l'intégrale dans l'expression \( \int_0^1\sum_if_i(t)dt\). Pour cela nous commençons par regarder quelques majorations de normes.

            D'abord \( \gamma\) est de classe \( C^1\), ce qui fait que \( \gamma'\) est continue. Vu que la norme est une application continue, la fonction \( t\mapsto \| \gamma'(t) \|\) est également continue sur le compact \( \mathopen[ 0 , 1 \mathclose]\). Elle est donc majorée par une constante que nous nommons \( M\). C'est le théorème de Weierstrass \ref{ThoWeirstrassRn}.

            Ensuite nous avons le calcul
            \begin{equation}
                    \| f_i(t) \|=\| (du_i)_{\gamma(t)}\big( \gamma'(t) \big) \|
                    \leq\| (du_i)_{\gamma(t)} \|\| \gamma'(t) \|  
                    \leq M\| du_i \|_{\infty}<\infty.
            \end{equation}
            Justifications :
            \begin{itemize}
                \item 
                    Pour la première inégalité. C'est le lemme \ref{LEMooIBLEooLJczmu}. 
                \item Pour la seconde inégalité. Il s'agit de l'inégalité évidente
                    \begin{equation}
                        \| du_i \|_{\infty}=\sup_{x\in \gamma\big( \mathopen[ 0 , 1 \mathclose] \big)}\| (du_i)_x \|
                    \end{equation}
                    Notons que la norme \( \| . \|_{\infty}\) ne réfère pas à un supremum sur \( E\), mais seulement sur l'image de \( \gamma\). Nous aurions pu faire preuve d'un peu de créativité dans les notations.
                \item 
                    L'application \( du_i\) est continue sur le compact \( \gamma\big( \mathopen[ 0 , 1 \mathclose] \big)\). Donc le supremum est fini et atteint.
            \end{itemize}

            Maintenant nous posons
            \begin{equation}
                g_n(t)=\sum_{i=0}^nf_i(t).
            \end{equation}
            Nous avons la majoration
            \begin{equation}
                \| g_n(t) \|\leq \sum_{i=0}^n\| f_i(t) \|\leq M\sum_{i=0}^n\| du_i \|_{\infty}<\infty.
            \end{equation}
            Le fait que le tout soit fini est l'hypothèse de convergence normale sur tout compact. Le compact en question est \( \gamma\big( \mathopen[ 0 , 1 \mathclose] \big)\).

            C'est le moment d'utiliser le théorème de la convergence dominée de Lebesgue \ref{ThoConvDomLebVdhsTf}. Attention aux notations un peu décalées. Nous avons \( g_n\to \sum_{i=1}^{\infty}f_i\) (convergence simple) et \( \| g_n(t) \|\leq A\) où \( A\) est une constante que nous voyons comme une fonction constante intégrable sur le compact \( \mathopen[ 0 , 1 \mathclose]\). Nous permutons la limite et l'intégrale :
            \begin{subequations}
                \begin{align}
                \int_0^1\sum_{i=0}^{\infty}f_i(t)dt&=\int_0^1(\lim_{n\to \infty} g_n)(t)dt\\
                &=\lim_{n\to \infty} \int_0^1g_n(t)dt\\
                &=\lim_{n\to \infty} \int_{0}^1\sum_{i=0}^nf_i(t)dt\\
                &=\lim_{n\to \infty} \sum_{i=0}^n\int_0^1f_i(t)dt\\
                &=\sum_{i=0}^{\infty}\int_0^1f_i(t)dt.
                \end{align}
            \end{subequations}
        \item[Accroissements]

            Nous pouvons maintenant faire le petit calcul suivant :
            \begin{equation}
                \sum_n\int_0^1(du_n)_{\gamma(t)}\big( \gamma'(t) \big)dt=\sum_n\Big( u_n\big( \gamma(1) \big)-u_n\big( \gamma(0) \big) \Big)=\sum_n\big( u_n(x)-u_n(a) \big)=u(x)-u(a)
            \end{equation}
            où nous avons utilisé le théorème fondamental du calcul intégral sous la forme de la proposition \ref{PROPooJYIAooXLkbMx}.

            Nous retenons l'égalité
            \begin{equation}        \label{EQooTXYWooIDxVri}
                u(x)=u(a)+\int_0^1\sum_n(du_n)_{\gamma(t)}\big( \gamma'(t) \big)dt.
            \end{equation}
        \item[Remarque]
            La formule \eqref{EQooTXYWooIDxVri} n'est pas une forme de formule des accroissements finis qui parlerait d'évaluer une fonction \( u\) en \( x\) en partant de \( a\) et en intégrant \( du\) le long d'un chemin joignant \( a\) et \( x\).

            Ce serait le cas si nous pouvions permuter la somme et la différentielle qui se trouvent dans l'intégrale. Or permuter somme et différentielle est précisément l'objet du théorème que nous sommes en train de prouver.

        \item[Différentielle]
        
            Forts de la formule \eqref{EQooTXYWooIDxVri}, nous calculons \( du_a(v)\), c'est-à-dire la différentielle de \( u\) au point \( a\) appliquée au vecteur \( v\in F\). Pour cela, nous savons que \( \Omega\) est ouvert, donc \( \Omega\) contient une boule de rayon \( r\) autour de \( a\), ce qui nous permet de dire que pour un \( \) donné, le point \( a+sv\) est dans \( \Omega\) pour tout \( s\in B(0,\epsilon)\) lorsque \( \epsilon\) n'est pas trop grand. Pour chacun de ces \( s\), nous considérons un chemin de classe \( C^1\) joignant \( a\) à \( a+sv\). Ce chemin sera noté
            \begin{equation}
                \gamma_s\colon \mathopen[ 0 , 1 \mathclose]\to \Omega
            \end{equation}
            et \( \gamma(0)=a\), \( \gamma(1)=a+sv\). Nous avons le calcul
            \begin{subequations}
                \begin{align}
                    du_a(v)&=\Dsdd{ u(a+sv) }{s}{0}\\
                    &=\Dsdd{ \int_0^1\sum_{n=0}^{\infty}(du_n)_{\gamma_s(t)}\big( \gamma_s'(t) \big)dt }{s}{0}\\
                    &=\Dsdd{ \sum_{n=0}^{\infty} \int_0^1(du_n)_{\gamma_s(t)}\big( \gamma'_s(t) \big)dt  }{s}{0}    \label{SUBEQooFSPSooCPErXj}\\
                    &=\Dsdd{ \sum_n\big( u_n(a+sv)-u_n(a) \big) }{s}{0}\\
                    &=\Dsdd{ (\sum_nu_n)(a+sv) }{s}{0}\\
                    &=\sum_n\Dsdd{ u_n(a+sv) }{s}{0}    \label{SUBEQooDQODooIPMfDo}\\
                    &=\sum_n(du_n)_a(v)\\
                    &=\Big( \sum_n(du_n)_a \Big)(v)\\
                    &=\Big( \sum_ndu_n \Big)_a(v)\label{SUBEQooQGOQooLvXuaX}.
                \end{align}
            \end{subequations}
            Justifications :
            \begin{itemize}
                \item Pour \ref{SUBEQooFSPSooCPErXj}. Permuter la somme et l'intégrale comme plus haut.
                \item Pour \ref{SUBEQooDQODooIPMfDo}. Permuter une somme et une dérivée classique des fonctions \( \eR\to F\) données par \( s\mapsto u_n(a+sv)\). Il s'agit d'utiliser le théorème \ref{THOooXZQCooSRteSr} sur chaque composantes dans \( F\).
                \item Pour \ref{SUBEQooQGOQooLvXuaX}. Chaque \( du_n\) est une application \( du_n\colon E\to \aL(E,F)\). Au fait près que la notation est plus lourde, il s'agit simplement d'une définition de la somme ponctuelle d'une suite de fonctions : \( \sum_nf_n(a)=(\sum_nf_n)(a)\). Dans ce cas-ci, le tout est encore un élément de \( \aL(E,F)\) que nous appliquons à \( v\).
            \end{itemize}
    \end{subproof}
\end{proof}


\begin{lemma}       \label{LEMooRIQTooLomsqD}
Sur $[\frac{ 1 }{2},1]$, étudier la convergence de la série
\begin{equation}
	f(x)=\sum_{n=1}^{\infty}\frac{1}{ n }\left( \frac{ x-1 }{ x } \right)^n.
\end{equation}
Étudier la convergence de la série dérivée; en déduire que
\begin{equation}
	\sum_{n=1}^{\infty}\frac{ (-1)^{n-1} }{ n }=\ln(2).
\end{equation}
\end{lemma}


\begin{proof}
Nous posons $y(x)=\frac{ x-1 }{ x }$, et nous regardons la série
\begin{equation}		\label{EqSeryexotreize}
	\tilde f(y)=\sum_{n=1}^{\infty}\frac{1}{ n }y^n.
\end{equation}
Cette série de puissance converge absolument pour $| y |<1$, voir l'exemple 4 de la page 123bis du cours de première. Cette série converge également simplement en $y=-1$, par le corolaire de la page 123 du même cours\footnote{Un étudiant avait dit se souvenir qu'Abel s'appliquait seulement aux séries alternées; c'est ce corolaire (critère des séries alternées) qui l'a induit en erreur. En effet, Abel (proposition 5, page 122) est plus général, mais s'applique particulièrement bien aux séries alternées.}. Nous sommes dans le cas d'une série de puissance dont le disque de convergence est centré en $0$, et dont le rayon est $1$, mais qui converge (en plus) simplement sur un des bords du disque. Cela est le cadre du théorème \ref{ThoAbelSeriePuiss} qui nous permet de dire que pour tout $\epsilon>0$, la série \eqref{EqSeryexotreize} converge uniformément sur $[-1,1-\epsilon]$.

La fonction $\tilde f(y)$ est donc continue sur $[-1,1-\epsilon]$, et donc en particulier sur $[-1,0]$. Par ailleurs, la fonction $y(x)$ est continue en $x\neq 0$. En tant que composée de fonctions continues, la fonction $f(x)=\tilde f\big( y(x) \big)$ est continue sur $[\frac{ 1 }{2},1]$.

Nous la mettons la série des dérivées sous la forme d'une série de puissances :
\begin{equation}		\label{EqSerieDerrTreize}
	g(x)=\frac{1}{ x^2 }\sum_{n=1}^{\infty} \left( \frac{ x-1 }{ x } \right)^{n-1}.
\end{equation}
Afin d'éviter tout malentendu, nous insistons sur le fait que $g$ est la série des dérivée de la série $f$. Nous ne savons pas encore si $g$ existe (c'est-à-dire si elle converge), ni si sa somme est la dérivée de $f$. C'est cela que nous allons tenter d'établir maintenant.

Nous posons à nouveau $y(x)=\frac{ x-1 }{ x }$, et nous savons que la série de puissances $\sum_ny^n$ converge uniformément pour $y\in[-1+\epsilon,1-\epsilon]$ pour tout $\epsilon>0$. En repassant aux variables $x$, pour tout $\epsilon>0$, nous avons convergence uniforme de la série 
\begin{equation}
	\sum_{n=0}^{\infty} \left( \frac{ x-1 }{ x } \right)^n, 
\end{equation}
sur le compact $x\in[\frac{ 1 }{ 2-\epsilon },\frac{1}{ \epsilon }]$, ou, pour parler plus simplement, sur $x\in[\frac{ 1 }{2}+\epsilon,a]$ pour tout $\epsilon$ (petit) et $a$ (grand). Nous avons donc également convergence uniforme de la série des dérivées \eqref{EqSerieDerrTreize} sur le même intervalle. Maintenant, le théorème \ref{ThoLDpRmXQ} montre que la série des dérivée est bien la dérivée de la série, c'est-à-dire que 
\begin{equation}
	g(x)=f'(x)
\end{equation}
sur $]\frac{ 1 }{ 2 },1]$. Notez que la convergence uniforme \emph{sur tout compact} de la série des dérivées est suffisante.

Une bonne nouvelle est qu'il est possibles de sommer explicitement la série $\sum_ky^k$. En effet, il est montré à la page 115 du cours de première que $\sum_{k=0}^n z^k=\frac{ 1-z^{n+1} }{ 1-z }$, donc
\begin{equation}		\label{EqFormSomGeometrze}
	\sum_{k=0}^{\infty}y^n=\lim_{n\to\infty}\frac{1-y^{n+1}}{ 1-y }=\frac{1}{ 1-y },
\end{equation}
lorsque $| y |<1$. Du coup, nous avons simplement
\begin{equation}
	f'(x)=g(x)=\frac{1}{ x^2 }\sum_{n=1}^{\infty}\left( \frac{ x-1 }{ x } \right)^{n-1}=\frac{1}{ x^2 }\sum_{n=0}^{\infty}\left( \frac{ x-1 }{ x } \right)^n=\frac{1}{ x^2 }\left( \frac{1}{  1-\left( \frac{ x-1 }{ x } \right)  } \right)=\frac{1}{ x },
\end{equation}
donc la fonction $f$ a la forme simple $f(x)=\ln(x)+C$. Notez bien le petit jeu de variables de sommation. Au départ $g(x)$ est une somme qui part de $1$ avec un exposant $n-1$, et nous la transformons en une somme qui part de $0$ avec un exposant $n$. C'est cela qui nous permet d'appliquer la formule \eqref{EqFormSomGeometrze}.

 Étant donné que $f(1)=0$, nous avons
\begin{equation}
	f(x)=\ln(x)
\end{equation}
pour tout $x\in[\frac{ 1 }{2}+\epsilon,1]$. Mais nous avons vu que la fonction $f$ était continue sur $[\frac{ 1 }{2},1]$. Étant donné que $\ln(x)$ et $f(x)$ sont deux fonctions continues sur $[\frac{ 1 }{2},1]$ qui sont égales sur tout compact $[\frac{ 1 }{2}+\epsilon,1]$, nous déduisons que ces deux fonctions sont en réalité égales sur tout l'entièreté du compact $[\frac{ 1 }{2},1]$.

En particulier, en $x=\frac{ 1 }{2}$, nous avons
\begin{equation}
	f(\frac{ 1 }{2})=\sum_{n=1}^{\infty}\frac{1}{ n }(-1)^n=\ln(1/2)=-\ln(2).
\end{equation}
\end{proof}

%+++++++++++++++++++++++++++++++++++++++++++++++++++++++++++++++++++++++++++++++++++++++++++++++++++++++++++++++++++++++++++
\section{Séries entières}
%+++++++++++++++++++++++++++++++++++++++++++++++++++++++++++++++++++++++++++++++++++++++++++++++++++++++++++++++++++++++++++

Dans cette section nous allons parler de séries complexes autant que de séries réelles. L'étude des propriétés à proprement parler complexes des séries entières (holomorphie) sera effectuée dans le chapitre dédié, voir le théorème~\ref{ThomcPOdd} et ses conséquences.

%---------------------------------------------------------------------------------------------------------------------------
\subsection{Disque de convergence}
%---------------------------------------------------------------------------------------------------------------------------

Une \defe{série de puissance}{série!de puissance} est une série de la forme
\begin{equation}		\label{eqseriepuissance}
	\sum_{k=0}^{\infty}c_k(z-z_0)^k
\end{equation}
où $z_0\in \eC$ est fixé, $(c_k)$ est une suite complexe fixée, et $z$ est un paramètre complexe. Nous disons que cette série est \emph{centrée} en $z_0$.

\begin{definition}
    Une \defe{série entière}{série!entière} est une somme de la forme
    \begin{equation}
        \sum_{n=0}^{\infty}a_nz^n
    \end{equation}
    avec \( a_n,z\in\eC\).
\end{definition}
Une série entière peut définir une fonction
\begin{equation}
    f(z)=\sum_na_nz^n.
\end{equation}
Le but de cette section est d'étudier des conditions sur la suite \( (a_n)\) qui assurent la continuité de \( f\) ou la possibilité de dériver ou intégrer la série terme à terme.

\begin{definition}  \label{DefZWKOZOl}
    Soit \( \sum_{n\in \eN}a_nz^n\) une série entière. Le \defe{rayon de convergence}{rayon!de convergence} de cette série est le nombre
    \begin{equation}
        R=\sup\{ r\in \eR^+\tq \text{la suite }(a_nr^n)\text{ est bornée} \}\in\mathopen[ 0 , \infty \mathclose].
    \end{equation}
    La boule \( B(0,R)\) est le \defe{disque de convergence}{disque de convergence} de la série.
\end{definition}

En réalité, il serait plus correct de parler du rayon de convergence de la suite \( (a_n)\) parce qu'au moment où on l'étudie, nous ne savons pas encore si la somme existera. Il ne devrait donc pas être autorisé d'écrire «étudions le rayon de convergence de \( \sum_na_nz^n\)».

Le rayon de convergence d'une série ne dépend que des réels \( | a_n |\), même si à la base \( a_n\in \eC\).

\begin{normaltext}
    Sur Wikipédia\cite{BIBooRVNCooAKQeld}, le rayon de convergence est défini par le supremum des \( | z |\) tels que la série \( \sum_na_nz^n\) converge. Je vous invite à vous étonner que cela est équivalent à la définition donnée ici.

    Il est dingue que demander que la suite \( (a_nr^n)\) soit bornée soit suffisant pour que la série converge. En réalité ce n'est pas tout à fait le cas; les séries qui convergent sont celles pour \( | z |\) strictement plus petit que le rayon de convergence. Et là ça marche. En effet, si \( x<R\) alors \( x=\epsilon R\) avec \( \epsilon<1\) et nous avons
    \begin{equation}
        a_nx^n=a_n(\epsilon R)^n=(a_nx^n)\epsilon^n\to 0.
    \end{equation}
    
    Le critère d'Abel \ref{LemmbWnFI} va formaliser ça.
\end{normaltext}

\begin{remark}      \label{REMooYOTEooKvxHSf}
    Si pour tout \( n\) nous avons \( | b_n |\geq | a_n |\) alors le rayon de convergence de la série \( \sum_na_nz^n\) est au moins aussi grand que celui de la série \( \sum_nb_nz^n\). Cela y compris lorsque l'un ou l'autre des rayons de convergences est infini.
\end{remark}

\begin{lemma}[\cite{MonCerveau}]    \label{LEMooVCTNooCQHkzs}
    Le rayon de convergence pour la suite \( b_n=a_{n+k}\) est le même que celui pour \( a_n\).
\end{lemma}

\begin{proof}
    Soit \( r>0\). Nous avons \( r^kb_nr^n=a_{n+k}r^{n+k}\). Vu que les \( k\) premiers termes d'une suite ne changement pas le fait que la suite soit bornée, la suite \( (a_nr^n)_{n\geq k}\) est bornée si et seulement si la suite \( (a_nr^n)_{n\in \eN}\) est bornée.
\end{proof}
<++>

\begin{lemma}[Critère d'Abel]\index{critère!Abel}   \label{LemmbWnFI}
    Soit \( R>0\) le rayon de convergence de la somme \( \sum_na_nz^n\) et \( z\in \eC\).
    \begin{enumerate}
        \item
            Si \( | z |<R\) alors la série converge absolument.
        \item
            Si \( | z |>R\) alors la série diverge.
    \end{enumerate}
\end{lemma}

\begin{proof}
    Démonstration en deux parties.
    \begin{enumerate}
        \item

            Si \( | z |<R\) alors la suite \( (a_nz^n)\) est bornée et il existe un nombre \( M\in \eR\) tel que \( | a_n |r^n\leq M\) pour tout \( n\). Nous considérons alors un \( r\) tel que \( | z |<r<R\) et nous pouvons calculer :
            \begin{equation}
                | a_nz^n |=| a_n |r^n\big( \frac{ | z | }{ r } \big)^n\leq M\left( \frac{ | z | }{ r } \right)^n
            \end{equation}
            Vu que \( | z |<r\) nous tombons sur la série géométrique \eqref{EqZQTGooIWEFxL} qui converge. Par le critère de comparaison\footnote{Lemme~\ref{LemgHWyfG}.} la série \( \sum_{n=0}^{\infty}| a_nz^n |\) converge.

        \item
            Par définition du rayon de convergence, la suite \( (a_nz^n)\) n'est donc pas bornée et la série ne peut pas converger à cause de la proposition~\ref{PROPooYDFUooTGnYQg}.
    \end{enumerate}
\end{proof}

\begin{corollary}       \label{CORooCUDSooTfMvAB}
    Soit une série entière \( \sum_na_nz^n\). Soit un nombre \( \rho\) tel que
    \begin{enumerate}
        \item
            La série converge pour \( | z |<\rho\).
        \item
            La série diverge pour \( | z |>\rho\).
    \end{enumerate}
    Alors le rayon de convergence est \( \rho\).
\end{corollary}

\begin{proof}
    Nous notons \( R\) le rayon de convergence de la série.
    \begin{subproof}
    \item[\( R\geq \rho\)]
        Si \( R<\rho\), nous prenons \( r\) strictement entre \( R\) et \( \rho\). Le critère d'Abel \ref{LemmbWnFI} nous dit, pour \( | z |=r\), que la série diverge. Par hypothèse, elle converge; contradiction.
    \item[\( R\leq \rho\)]
        De même si \( R>\rho\), alors nous prenons \( \rho<r<R\). Le critère d'Abel nous dit que la série diverge pour \( | z |=r\). L'hypothèse nous dit le contraire. Nouvelle contradiction.
    \end{subproof}
\end{proof}

Le critère d'Abel parle bien de convergence absolue, et non de convergence normale. Pour chaque \( t\), la série \( \sum_k | a_nt^k |\) converge. Si par contre nous posons \( u_k(t)=a_kt^k\), nous n'avons à priori pas la convergence normale \( \sum_k\| u_k \|_{\infty}\), même pas si la norme est la norme supremum sur \( B(0,R)\)\quext{Il y aurait par contre bien convergence sur tout compact ? \randomGender{Chère lectrice}{Cher lecteur}, dites moi ce que vous en pensez}. Prenons comme exemple simplement \( a_k=1\) pour tout \( k\). Pour tout \( | t |<1\), la série \( \sum_k t^k\) converge absolument (série géométrique), mais nous aurions \( \| u_k \|_{\infty}=1\) et donc divergence évidente de \( \sum_k\| u_k \|_{\infty}\).

La proposition suivante sera surtout utile lorsqu'on parlera de dérivée.
\begin{proposition}[\cite{KOWMooXhcOoy}]        \label{PropHDIUooKTbVSX}
    Quel que soit le nombre \( \alpha\in \eR\), les séries \( \sum_na_nz^n\) et \( \sum_nn^{\alpha}a_nz^n\) ont même rayon de convergence.
\end{proposition}

\begin{proof}
    Nous posons
    \begin{subequations}
        \begin{align}
            E=\{ r\in \eR^+\tq \text{  } (a_nr^n)\text{ est borné } \}
            E'=\{ r\in \eR^+\tq \text{  } (n^{\alpha}a_nr^n)\text{ est borné } \}
        \end{align}
    \end{subequations}
    Et aussi \( R=\sup(E)\), \( R'=\sup(E')\). Le fait que \( E'\geq E\) est facile. Nous supposons \( R>0\) et nous considérons \( r<R\) (c'est-à-dire \( r\in E\)).  Nous allons montrer que \( r\in E'\). Pour cela nous prenons un nombre \( s\) tel que \( r<s<R\). Nous avons
    \begin{equation}
        n^{\alpha}a_nr^n=n^{\alpha}a_n\left( \frac{ r }{ s } \right)^ns^n=n^{\alpha}\left( \frac{ r }{ s } \right)^na_ns^n.
    \end{equation}
    Mais \( r/s<1\), donc le lemme~\ref{LemLJOSooEiNtTs} dit que \( n^{\alpha}(r/s)^n\to 0\). Cela est donc borné par une constante \( M\). Donc
    \begin{equation}
        n^{\alpha}a_nr^n\leq Ma_ns^n.
    \end{equation}
    Mais la suite \( (a_ns^n)\) est bornée. Donc la suite \( n^{\alpha}a_nr^n\) est également bornée, ce qui prouve que \( r\in E'\).
\end{proof}

\begin{remark}
    Au fond, cette proposition n'est rien d'autre que dire que dans \( n^\alpha r^n\), l'effet «convergent» est \( r^n\) qui est une décroissance exponentielle tandis que l'effet «divergent» est \( n^{\alpha}\) qui a une croissance seulement polynomiale.
\end{remark}

\begin{theorem}[Formule de Hadamard] \label{ThoSerPuissRap}
    Le rayon de convergence\footnote{Définition \ref{DefZWKOZOl}.} de la série entière \( \sum_n c_n z^n\) est donné par une des deux formules
    \begin{equation}		\label{EqRayCOnvSer}
        \frac{1}{ R } =\limsup\sqrt[k]{| a_k |}
    \end{equation}
    ou
    \begin{equation}		\label{EqAlphaSerPuissAtern}
        \frac{1}{ R }=\limite k \infty \abs{\frac{a_{k+1}}{a_k}}
    \end{equation}
    lorsque $a_k$ est non nul à partir d'un certain $k$.

    Si une de ces formules donne \( 1/R=0\), alors le rayon de convergence est infini.
\end{theorem}
\index{formule!Hadamard}\index{Hadamard!formule}		

Le disque $| z-z_0 |\leq R$ est le \defe{disque de convergence}{disque de convergence} de la série \( \sum_n a_n(z-z_0)^n\). Notons que le critère d'Abel ne dit rien pour les points tels que $| z-z_0 |=R$. Il faut traiter ces points au cas par cas. Et le pire, c'est qu'une série donnée peut converger pour certain des points sur le bord du disque, et diverger en d'autres. Le théorème d'Abel radial (théorème~\ref{ThoLUXVjs}) nous donnera quelques informations sur le sujet.

Il y a un dessin à la figure~\ref{LabelFigDisqueConv}.
\newcommand{\CaptionFigDisqueConv}{À l'intérieur du disque de convergence, la convergence est absolue. En dehors, la série diverge. Sur le cercle proprement dit, tout peut arriver.}
\input{auto/pictures_tex/Fig_DisqueConv.pstricks}

Si les suites \( a_n\) et \( b_n\) sont équivalentes, alors les séries correspondantes auront le même rayon de convergence. Cela ne signifie pas que sur le bord du disque de convergence, elles aient même comportement. Par exemple nous avons
\begin{equation}
    \frac{1}{ \sqrt{n} }\sim \frac{1}{ \sqrt{n} }+\frac{ (-1)^n }{ n }.
\end{equation}
En même temps, en \( z=-1\) la série
\begin{equation}
    \sum_{n\geq 1}\frac{ z^n }{ \sqrt{n} }
\end{equation}
converge par le critère des séries alternées\footnote{Théorème \ref{THOooOHANooHYfkII}.}. Par contre la série
\begin{equation}
    \sum_{n\geq 1}\left( \frac{1}{ \sqrt{n} }+\frac{ (-1)^n }{ n } \right)z^n
\end{equation}
ne converge pas pour \( z=-1\).

\begin{example}
    Soit \( \alpha\in \eR\) et considérons la série \( \sum_{n\geq 1}a_nz^n\) où \( a_n\) est la \( n\)-ième décimale de \( \alpha\). Si \( \alpha\) est un nombre décimal limité, la suite \( (a_n)\) est finie et le rayon de convergence est infini. Sinon, pour tout \( N\) il existe un \( n>N\) tel que \( a_n\neq 0\) et la suite \( (a_n)\) ne tend pas vers zéro. Par conséquent la série
    \begin{equation}
        \sum_{n}a_nz^n
    \end{equation}
    diverge pour \( z=1\) et le rayon de convergence satisfait \( R\leq 1\). Nous avons aussi \( | a_n |\leq 9\), de telle manière à ce que la série soit bornée et par conséquent majorée en module par \( 9z^n\), ce qui signifie que \( R\geq 1\).

    Nous déduisons alors \( R=1\).
\end{example}

%---------------------------------------------------------------------------------------------------------------------------
\subsection{Somme et produit de séries}
%---------------------------------------------------------------------------------------------------------------------------

\begin{theorem}\label{THOooSDQQooIawBOk}
    Soient \( \sum_na_nz^n\) et \( \sum b_nz^n\) deux séries de rayon de convergences respectivement \( R_a\) et \( R_b\).  Si \( R_s\) est le rayon de convergence de \( \sum_n(a_n+b_n)z^n\), nous avons
    \begin{equation}
        R_s\geq \min\{ R_a,R_b \}.
    \end{equation}
\end{theorem}

\begin{theorem}     \label{THOooINHDooZxErnp}
    Soient \( \sum_na_nz^n\) et \( \sum b_nz^n\) deux séries de rayon de convergences respectivement \( R_a\) et \( R_b\).  
    \begin{enumerate}
        \item
            Si \( \lambda\neq 0\) la série \( \sum_n(\lambda a_n)z^n\) a le même rayon de convergence que la série \( \sum_na_nz^n\) 
        \item
            Si \( | z |<R_a\) nous avons
    \begin{equation}
        \sum_{n=0}^{\infty}(\lambda a_n)z^n=\lambda\sum_{n=0}^{\infty}a_nz^n.
    \end{equation}
    \end{enumerate}
\end{theorem}


\begin{lemma}[\cite{MonCerveau}]       \label{LEMooNYAXooKUuQFe}
    Soient deux suites de nombres complexes \( (a_n)\) et \( (b_n)\). Soit \( n\in \eN\). Nous avons :
    \begin{equation}
        \sum_{k=0}^n\sum_{l=0}^ka_lb_{k-l}=\sum_{l=0}^n\big( \sum_{j=0}^{n-l}b_j \big)a_l.
    \end{equation}
\end{lemma}

\begin{proof}
    Le problème est qu'à gauche la borne de la somme sur \( l\) dépend de \( k\); cela nous empêche de permuter les sommes\footnote{Vu que toutes les sommes sont finies, ce ne sont certainement pas les questions de convergence qui nous retiennent.}. Qu'à cela ne tienne : nous complétons la somme en introduisant
    \begin{equation}
         \sigma_{lk}=\begin{cases}
            1    &   \text{si } l\leq k\\
            0    &    \text{sinon }.
        \end{cases}
    \end{equation}
    Nous avons alors
    \begin{subequations}
        \begin{align}
            \sum_{k=0}^n\sum_{l=0}^ka_lb_{k-l}&=\sum_{k=0}^n\sigma_{lk}a_lb_{k-l}=\sum_{l=0}^n\sum_{k=0}^n\sigma_{lk}a_lb_{k-l} =\sum_{l=0}^n\big( a_l\sum_{k=l}^nb_k-l \big)\\
            &=\sum_{l=0}^n\big( a_l\sum_{j=0}^{n-l}b_j \big)=\sum_{l=0}^n\big( \sum_{j=0}^{n-l}b_j \big)a_l.
        \end{align}
    \end{subequations}
\end{proof}

\begin{lemma}[\cite{MonCerveau}]        \label{LEMooLPBCooRWuvJB}
    Si la série \( \sum_{n=0}^{\infty}a_nz^n\) a un rayon de convergence \( R\), alors nous avons
    \begin{equation}
        \big( \sum_{n=0}^{\infty}a_nz^n \big)z=\sum_{n=0}^{\infty}a_nz^{n+1}
    \end{equation}
    et les rayons de convergences sont égaux à \( R\).
\end{lemma}

\begin{proposition}[\cite{BIBooWNXZooQIdAns}]       \label{PROPooPKGEooZKyxwo}
    Soient \( (a_n)\) et \( (b_n)\) des suites dans \( \eC\). Nous supposons que \( \sum_na_n\) est absolument convergente\footnote{Définition \ref{DefVFUIXwU}.} et que \( \sum_nb_n\) est convergente.

    Alors en posant
    \begin{equation}
        c_n=\sum_{k=0}^na_kb_{n-k},
    \end{equation}
    la série \( \sum_nc_n\) est convergente et
    \begin{equation}
        \sum_{n=0}^{\infty}c_n=\big( \sum_ia_i \big)\big( \sum_j b_j \big).
    \end{equation}
\end{proposition}

\begin{proof}
    Nous commençons par quelque notations sur les sommes partielles et leurs limites. Nous posons \( A_n=\sum_{k=0}^na_n\), \( B_n=\sum_{k=0}^nb_n\) et nous avons, par hypothèse, les convergences \( A_n\stackrel{\eC}{\longrightarrow}A\) et \( B_n\stackrel{\eC}{\longrightarrow}B\).

    En ce qui concerne la somme partielle pour les \( (c_n)\), en appliquant le lemme \ref{LEMooNYAXooKUuQFe},
    \begin{equation}
        C_n=\sum_{k=0}^nc_k=\sum_{k=0}^n\sum_{l=0}^ka_lb_{k-l}=\sum_{l=0}^n\big( \sum_{j=0}^{n-l}b_j \big)a_l=\sum_{l=0}^na_lB_{n-l}.
    \end{equation}
    
    Soit \( \epsilon>0\). 
    \begin{subproof}
    \item[Des indices assez grands]
    Nous définissons \( N_1, N_2\in \eN\) de la façon suivante :
    \begin{itemize}
        \item 
    Vu que \( B_j\to B\), il existe \( N_1\in \eN\) tel que pour tout \( j\geq N_1\), \( | B_j-B |\leq \epsilon\).
\item
    Vu que \( \sum_na_n\) converge, la proposition \ref{PROPooYDFUooTGnYQg} nous dit que \( | a_i |\to 0\) (ce n'est pas ici que nous utilisons la convergence absolue). Il existe donc \( N_2\in \eN\) tel que \( | a_i |\leq \epsilon/N_1\) pour tout \( i\geq N\2\).
\item
    Nous considérons \( N\geq N_1+N_2\).
    \end{itemize}
\item[Un majorant]
    Vu que la série \( \sum_na_n\) converge absolument, la somme \( \sum_{n=0}^{\infty}| a_n |\) est bornée. De même, la suite \( j\to| B_j-B |\) est bornée et nous choisissons \( M\) assez grand pour majorer les deux en même temps :
    \begin{subequations}
        \begin{align}
            \sum_{n=0}^{\infty}| a_n |&<M           \label{SUBEQooBKCVooXnampA}\\
            | B_j-B |&<M\quad \forall j.
        \end{align}
    \end{subequations}
\item[Et on calcule un peu]
        Nous avons assez préparé de notations et de majorations. C'est le moment de prouver que \( C_n-A_nB \to 0\). Nous avons
        \begin{subequations}
            \begin{align}
                | C_n-A_nB |&=| \sum_{l=0}^na_k(B_{n-l-B}) |\\
                &\leq \sum_{l=0}^n| a_l | |B_{n-l}-B |\\
                &=\sum_{l=0}^{n-N_1}| a_l | |B_{n-l} |+\sum_{l=n-N_1+1}^n| a_l | |B_{n-l}-B |\\
                &\leq \sum_{l=0}^{n-N_1}| a_l |\epsilon+M\sum_{l=n-N_1+1}^n| a_l |      \label{SUBEQooIQFZooJMqFWo}\\
                &\leq \epsilon M +M\sum_{l=n-N_1+1}^n\frac{ \epsilon }{ N_1 }      \label{SUBEQooKAJFooQsIxWo}\\
                &=2\epsilon M.   \label{SUBEQooDXQMooOvBCXH}
            \end{align}
        \end{subequations}
        Justifications :
        \begin{itemize}
            \item Pour \eqref{SUBEQooIQFZooJMqFWo}. Dans la première somme, \( n-l\geq n-(n-N_1)=N_1\), donc \( | B_{n-l}-B |\leq \epsilon\). Dans la seconde somme nous avons seulement majoré \( | N_{n-l}-B |\leq M\).
            \item Pour \eqref{SUBEQooKAJFooQsIxWo}. Dans la première somme, il s'agit de la majoration \eqref{SUBEQooBKCVooXnampA}. Dans le seconde somme, \( l\geq n-N_1+1\geq N_1+N_2-N_1+1\geq N_2+1\), ce qui implique \( | a_l |\leq \epsilon/N_1\).
            \item Pour \eqref{SUBEQooDXQMooOvBCXH}. La somme contient \( n-(n-N_1+1)+1=N_1\) termes. Chaque terme valant \( \epsilon/N_1\).
        \end{itemize}
    \item[Conclusion]
        Nous avons prouvé que pour tout \( \epsilon>0\), il existe \( N\) tel que \( | C_n-A_nB |\leq \epsilon\). Nous écrivons maintenant
        \begin{equation}
            C_n=(C_n-A_nB)+A_nB
        \end{equation}
        Vu que \( C_n-A_nB\to 0\) et que \( A_nB\to AB\), la somme des deux suites converge vers\footnote{C'est la proposition \ref{PROPooICZMooGfLdPc}.} \( 0+AB=A\) et donc
        \begin{equation}
            C_n\to AB.
        \end{equation}
    \end{subproof}
\end{proof}

Le théorème suivant donne une formule (dit «produit de Cauchy») pour le produit de deux séries entières. Nous en donnons une adaptation dans le cas de séries de puissances dans une algèbre normée dans la proposition \ref{PROPooFMEXooCNjdhS}.
\begin{theoremDef}[Produit de Cauchy dans \( \eC\)\cite{BIBooVMCQooIBokOv}]     \label{ThokPTXYC}
    Soient \( \sum_na_nz^n\) et \( \sum b_nz^n\) deux séries de rayon de convergences respectivement \( R_a\) et \( R_b\). La série entière
    \begin{equation}        \label{EqFPGGooDQlXGe}
        \sum_{n=0}^{\infty}\left( \sum_{k=0}^{n}a_kb_{n-k} \right)z^n.
    \end{equation}
    est le \defe{produit de Cauchy}{produit de Cauchy} des séries \( \sum_na_nz^n\) et \( \sum_nb_nz^n\).

    Nous notons \( R_p\) le rayon de convergence de la série.
    \begin{enumerate}
        \item   \label{ITEMooFOVPooBaVknN}
            Nous avons l'inégalité $R_p\geq \min\{ R_a,R_b \}$.
        \item   \label{ITEMooHRNZooWviigD}
            Si \( | z |<\min\{ R_a,R_b \}\) alors
            \begin{equation}        \label{EQooSGXHooHwjOEV}
                \sum_{n=0}^{\infty}\left( \sum_{i+j=n}a_ib_j \right)z^n=\left( \sum_{n=0}^{\infty}a_nz^n \right)\left( \sum_{n=0}^{\infty}b_nz^n \right).
            \end{equation}
    \end{enumerate}
\end{theoremDef}

\begin{proof}
    En plusieurs parties.
    \begin{subproof}
        \item[Préambule]
            Nous allons fixer \( z\), et utiliser la proposition \ref{PROPooPKGEooZKyxwo}. Ce que nous appelons \( a_n\) là-bas est \( a_nz^n\) ici. Idem pour les \( b_n\) qui sont \( b_nz^n\) et \( c_n\) qui devient \( c_nz^n\). Vu que \( z\) sera fixé, tout cela n'est pas très profond.

            Ces substitutions sont très courantes lorsque nous prouvons des résultats sur les séries entières comme corollaires de résultats généraux sur les séries.
        \item[Pour \ref{ITEMooFOVPooBaVknN}]
            Si \( | z |<\min\{ R_a,R_b \}\), alors les séries \( \sum_na_nz^n\) et \( \sum_nb_nz^n\) sont absolument convergentes par lemme d'Abel \ref{LemmbWnFI}. La proposition \ref{PROPooPKGEooZKyxwo} pour les suites \( (a_nz^n)\) et \( (b_nz^n)\) fait alors le boulot : en posant 
            \begin{equation}
                c_n=\sum_{k=0}^na_nb_{n-k},
            \end{equation}
            la série \( \sum_nc_nz^n\) converge et vaut le produit des deux.

            Vu que la série \( \sum_nc_nz^n\) converge sur \( \{ z\in \eC\tq | z |<\min\{ R_a,R_b \} \}\). Donc en posant \( r<\min\{ R_a,R_b \}\), la suite \( (c_nr^n)\) est bornée (dans \( \eC\)) et nous avons que \( R_p\geq r\) (utilisation très littérale de la définition du rayon de convergence). Donc \( R_p\geq\min \{ R_a,R_b \}\).
        \item[Pour \ref{ITEMooHRNZooWviigD}]
            Le travail est déjà fait.
    \end{subproof}
\end{proof}

\begin{example}
    Montrons un produit de Cauchy dont le rayon de convergence est strictement plus grand que le minimum. D'abord nous considérons
    \begin{equation}
        A=1-z,
    \end{equation}
    c'est-à-dire \( a_0=1\), \( a_1=-1\), \( a_{n\geq 2}=0\) avec \( R_a=\infty\). Ensuite nous considérons
    \begin{equation}
        B=\sum_nz^n,
    \end{equation}
    c'est-à-dire \( B=(1-z)^{-1}\) et \( R_b=1\). Le produit de Cauchy de ces deux séries valant \( 1\), le rayon de convergence est infini.

    Notons qu'alors l'égalité \eqref{EQooSGXHooHwjOEV} a lieu dans \( B(0,1)\), mais pas au-delà.

    Donc le «produit de Cauchy» de deux séries peut ne pas être égal au produit des deux séries, au sens où il est possible que le produit existe là où une des deux séries n'existe plus.
\end{example}

\begin{example}
    Nous montrons que
    \begin{equation}
        \sum_{n=0}^{\infty}(n+1)x^n=\frac{1}{ (1-x)^2 }
    \end{equation}
    pour \( x\in\mathopen] -1 , 1 \mathclose[\).

    Étant donné que pour tout \( r\) dans \( \mathopen] -1 , 1 \mathclose[\) la suite \( (n+1)r^n\) est bornée, le rayon de convergence est correct. Pour les \( x\) dans ce domaine nous avons
    \begin{equation}        \label{EqIwbuTk}
        \frac{1}{ (1-x)^2 }=\frac{1}{ (1-x) }\frac{1}{ (1-x) }=\left( \sum_{n=0}^{\infty}x^n \right)\left( \sum_{m=0}^{\infty}z^m \right).
    \end{equation}
    Nous devons expliciter ce produit de Cauchy en utilisant le théorème~\ref{ThokPTXYC}. Pour tout \( i\) nous avons \( a_i=b_i=1\). Par conséquent le produit \eqref{EqIwbuTk} devient
    \begin{equation}
        \sum_{n=0}^{\infty}\sum_{i+j=n}x^n=\sum_{n=0}^{\infty}(n+1)x^n.
    \end{equation}
\end{example}

Nous voulons maintenant faire le produit de Cauchy à plus que deux facteurs. Pour cela nous prouvons d'abord un certain nombre de lemmes traitant de la combinatoire du problème.

Nous posons, pour \( n,N\in \eN\) :
\begin{equation}        \label{EQooJCBSooMSbaCd}
    V_n(N)=\{ x\in \eN^N\tq \sum_{i=1}^Nx_i=n \}.
\end{equation}

\begin{lemma}       \label{LEMooRKEVooDdpuHt}
    Nous avons
    \begin{equation}
        V_n(N+1)=\bigcup_{y=0}^n\bigcup_{x\in V_y(N)}(x,n-y).
    \end{equation}
\end{lemma}

\begin{proof}
    En deux parties.
    \begin{subproof}
    \item[Première inclusion]
        Un élément de \( \bigcup_{y=0}^n\bigcup_{x\in V_u(N)}(x,n-y)\) est un élément \( z\in \eN^{N+1}\) de la forme \( z=(x,n-y)\) tel que \( x\in V_y(N)\). Donc
        \begin{equation}
            \sum_{i=1}^{N+1}z_i=\sum_{i=1}^Nx_i+(n-y)=y+n-y=n.
        \end{equation}
        Donc \( z\in V_n(N+1)\).
        \begin{subproof}
        \item[L'autre inclusion]
            Un élément de \( V_n(N+1)\) est de la forme \( z=(x,y)\) avec \( x\in \eN^N\) et \( y\in \eN\). Posons \( t=\sum_{i=1}^Nx_i\), de telle sorte que \( x\in V_t(N)\).

            Vu que \( z\in V_n(N+1)\) nous avons d'autre part \( y=n-\sum_{i=1}^Nx_i=n-t\).
        \end{subproof}
    \end{subproof}
\end{proof}

La proposition suvante généralise le produit de Cauchy du théorème \ref{ThokPTXYC} au cas de plus de deux facteurs. Nous ne pouvons cependant pas considérer \ref{ThokPTXYC} comme un cas particulier de \ref{PROPooJPVVooLqSdSn}, parce que la démonstration va utiliser le cas à deux facteurs.
\begin{proposition}[Produit de Cauchy\cite{MonCerveau}]     \label{PROPooJPVVooLqSdSn}
    Soient des nombres complexes \( a_{ik}\) tels que les séries entières
    \begin{equation}
        s_i(z)=\sum_{k=0}^{\infty}a_{ik}z^k
    \end{equation}
    soient convergentes avec un rayon de convergence \( R_i\). Nous posons
    \begin{equation}
        c_n=\sum_{x\in V_n(N)}\prod_{i=1}^Na_{ix_i},
    \end{equation}
    et nous appelons \( R_p\) le rayon de convergence de la série \( \sum_nc_nz^n\).
    \begin{enumerate}
        \item
            Nous avons l'inégalité \( R_p\geq \min\{R_i\}\) .
        \item       \label{ITEMooUVNXooLxlawx}
            Pour \( | z |\leq \min\{R_i\}\) nous avons l'égalité
            \begin{equation}        \label{EQooHCUGooDRhxzt}
                \prod_{i=1}^Ns_i(z)=\sum_{n=1}^N\left( \sum_{x\in V_n(N)}\prod_{i=1}^Na_{ix_i} \right)z^n.
            \end{equation}
    \end{enumerate}
\end{proposition}
\index{produit de Cauchy}

\begin{proof}
    Nous prouvons cela par récurrence sur \( N\). D'abord pour \( N=1\) nous avons \( V_n(1)=\{ n \}\). Donc \( c_n=\prod_{i=1}^1a_{ix_i}=a_{1n}\) et donc la série à droite dans \eqref{EQooHCUGooDRhxzt} est seulement \( \sum_{n=0}^{\infty}a_{1n}z^n=s_1(z)\).

    Nous supposons le théorème prouvé pour touttes valeurs jusqu'à \( N\) et nous prouvons pour \( N+1\). Si \( | z |<\min\{ R_i \}\) alors toutes les séries convergent et en utilisant l'associativité du produit dans \( \eC\) nous avons :
    \begin{equation}
        \prod_{i=1}^{N+1}s_i(z)=\left( \prod_{i=1}^Ns_i(z) \right)s_{N+1}(z)=\sum_{n=1}^{\infty}\left( \sum_{x\in V_n(N)}\prod_{i=1}^Na_{ix_i} \right)z^ns_{N+1}(z).
    \end{equation}
    Nous allons maintenant utilliser le produit de Cauchy à deux termes du théorème \ref{PROPooPKGEooZKyxwo}. Notez que c'est bien l'utilisation de ce théorème qui nous permet d'obtenir la convegence dans notre pas de récurrence, et non l'hypothèse de réccurence actuelle. Bref, nous posons
    \begin{subequations}
        \begin{align}
            b_{1k}&=\sum_{x\in V_k(N)}\prod_{i=1}^Na_{ix_i}\\
            b_{2k}&=a_{N+1,k}.
        \end{align}
    \end{subequations}
    L'utilisation du produit de Cauchy à deux facteurs donne le coefficient de \( z^n\) sous la forme suivante :
    \begin{subequations}
        \begin{align}
            c_n&=\sum_{y\in V_n(2)}b_{1y_1}b_{2y_2}=\sum_{y=V_n(2)}\sum_{x\in V_{y_1}(N)}\prod_{i=1}^Na_{ix_i}a_{N+1,y_2}\\
            &=\sum_{y=0}^n\sum_{x\in V_n(N)}\prod_{i=1}^Na_{ix_i}a_{N+1,n-y}        \label{SUBEQooFPEMooKpLQBd}\\
            &=\sum_{(x,y)\in A_n(N+1)}\prod_{i=1}^Na_{ix_i}a_{N+1, y}       \label{SUBEQooHXZRooSdxFTf}\\
            &=\sum_{x\in V_n(N+1)}\prod_{i=1}^{N+1}a_{ix_i}.
        \end{align}
    \end{subequations}
    Justifications :
    \begin{itemize}
        \item Pour \eqref{SUBEQooFPEMooKpLQBd}. L'ensemble \( V_n(2)\) n'est pas très compliqué à expliciter :
    \begin{equation}
        V_n(2)=\{ (y,n-y)\tq y=0,\ldots, n \}.
    \end{equation}
\item   Pour \eqref{SUBEQooHXZRooSdxFTf}. La somme porte sur les \( (x,t)\) avec \(x\in \eN^N\) et \( y\in \eN\) tels que \( (x,y) \) est dans \( V_n(N+1)\), et la justification de l'égalité est le lemme \ref{LEMooRKEVooDdpuHt}.
    \end{itemize}
\end{proof}

%--------------------------------------------------------------------------------------------------------------------------- 
\subsection{Convergence normale}
%---------------------------------------------------------------------------------------------------------------------------

\begin{theorem}
    Une série entière converge normalement sur tout disque fermé inclus au disque de convergence.
\end{theorem}

\begin{proof}
    Toute boule fermée inclue à \( B(0,R)\) est inclue à la boule \( \overline{ B(0,r) }\) pour un certain \( r<R\). Nous nous concentrons donc sur une telle boule fermée.

    Pour chaque \( n\) nous posons \( u_n(z)=a_nz^n\) que nous voyons comme une fonction sur \( \overline{ B(0,r) }\). Pour tout \( n\in \eN\) et tout \( z\in\overline{ B(0,r) }\) nous avons
    \begin{equation}
        \| u_n \|_{\infty}\leq| a_nz^n |\leq | a_n |r^n.
    \end{equation}
    Étant donné que \( r<R\) la série \( \sum_n | a_n |r^n\) converge et la série \( \sum_n\| u_n \|\) est convergente. La série \( \sum_na_nz^n\) est alors normalement convergente.
\end{proof}

\begin{example}
    Encore une fois nous n'avons pas d'informations sur le comportement au bord. Par exemple la série \( \sum_nz^n\) a pour rayon de convergence \( R=1\), mais \( \sup_{z\in B(0,1)}| z^n |=1\) et nous n'avons pas de convergence normale sur la boule fermée.
\end{example}

La convergence normale n'est donc pas de mise sur tout l'intérieur du disque de convergence. La continuité, par contre est effective sur la boule. En effet si \( z_0\in B(0,R)\) alors il existe un rayon \( 0<r<R\) tel que \( B(z_0,r)\subset B(0,R)\). Sur \( B(z_0,r)\) nous avons convergence normale et donc continuité en \( z_0\).

La différence est que la continuité est une propriété locale tandis que la convergence normale est une propriété globale.

\begin{proposition}
    Soit \( f(z)=\sum_na_nz^n\) avec un rayon de convergence \( R\). Si \( \sum | a_n |R^n\) converge alors
    \begin{enumerate}
        \item
            la série \( \sum_na_nz^n\) converge normalement sur \( \overline{ B(0,R) }\),
        \item
            \( f\) est continue sur \( \overline{ B(0,R) }\).
    \end{enumerate}
\end{proposition}

\begin{proof}
    La conclusion est claire dans l'intérieur du disque de convergence. En ce qui concerne le bord, chacune des sommes partielles est une fonction continue. De plus nous avons \( \| u_n \|\leq | a_n |R^n\), dont la série converge. Par conséquent nous avons convergence normale sur le disque fermé.
\end{proof}

Le théorème suivant permet de donner, dans le cas de fonctions réelle, des informations sur la convergence en une des deux extrémités de l'intervalle de convergence.
\begin{theorem}[Convergence radiale de Abel]\index{Abel!convergence radiale} \label{ThoLUXVjs}
    Soit \( f(x)=\sum_na_nx^n\) une série réelle de rayon de convergence \( 0<R<\infty\).
    \begin{enumerate}
        \item
            Si \( \sum a_nR^n\) converge, alors \( f\) est continue sur \( \mathopen[ 0 , R \mathclose]\).
        \item
            Si \( \sum_na_n(-R)^n\) converge, alors \( f\) est continue sur \( \mathopen[ -R , 0 \mathclose]\).
    \end{enumerate}
\end{theorem}

La proposition \ref{PROPooKPBIooJdNsqX} donnera un exemple d'utilisation pour la série de \( \ln(1-x)\) (qui n'est pas encore définie à ce moment).


Le résultat suivant permet d'identifier deux séries complexes lorsque leurs valeurs sur \( \eR\) sont identiques.
\begin{proposition}
    Soient les séries \( f(z)=\sum a_nz^n\) et \( g(z)=\sum b_n z^n\) convergentes dans \( B(0,R)\). Si \( f(x)=g(x)\) pour \( x\in \mathopen[ 0 , R [\) alors \( a_n=b_n\).
\end{proposition}

\begin{proof}
    Soit \( n_0\) le plus petit entier tel que \( a_{n_0}\neq b_{n_0}\). Pour tout \( z\in B(0,R)\) nous avons
    \begin{equation}
        f(z)-g(z)=\sum_{n=n_0}^{\infty}(a_n-b_n)z^n=z^{n_0}\varphi(z)
    \end{equation}
    où
    \begin{equation}
        \varphi(z)=\sum_{n\geq 0}(a_{n+n_0}-b_{n+n_0})z^n.
    \end{equation}
    Par le théorème \ref{THOooSDQQooIawBOk} le rayon de convergence de \( \varphi\) est plus grand que \( R\) et la fonction \( \varphi\) est continue en \( 0\). Étant donné que \( \varphi(0)=a_{n_0}-b_{n_0}\neq 0\) et que \( \varphi\) est continue nous avons un \( \rho\) tel que \( \varphi\neq 0\) sur \( B(0,\rho)\). Or cela n'est pas possible parce que au moins sur la partie réelle de cette dernière boule, \( \varphi\) doit être nulle.
\end{proof}

\begin{proposition}[\cite{GYDXooJJusGH,MonCerveau}]     \label{PropSNMEooVgNqBP}
    Si la série entière \( \sum_{n\geq 0}a_nz^n\) a un rayon de convergence \( R\) alors
    \begin{enumerate}
        \item
            La somme est une fonction holomorphe\footnote{Définition \ref{DefMMpjJZ}.} dans le disque de convergence.
        \item       \label{ItemUULDooEGRNiA}
            La somme est différentiable et
            \begin{equation}
                du_{z_0}(z)=\sum_{n=1}^{\infty}na_nz_0^{n-1}z.
            \end{equation}
        \item
    De plus pour tout \( z_0\in B(0,R)\), on pose\footnote{Pour rappel, dans tout ce texte, \( B(a,r)\) est une boule \emph{ouverte}.}
    \begin{subequations}
        \begin{align}
            S(z)&=\sum_{n\geq 0}a_nz^n\\
            T(z)&=\sum_{n\geq 1}na_nz^{n-1}=\sum_{n=0}^{\infty}(n+1)a_{n+1}z^n.
        \end{align}
    \end{subequations}
    Alors  nous avons
    \begin{equation}    \label{EqVQDPooOPICwN}
        \lim_{z\to z_0}\frac{ S(z)-S(z_0) }{ z-z_0 }=T(z_0).
    \end{equation}
    \end{enumerate}
\end{proposition}

\begin{proof}
    Nous allons prouver, en utilisant le théorème~\ref{ThoLDpRmXQ}, que la somme est une fonction différentiable et que la différentielle est \( \eC\)-linéaire. La proposition~\ref{PropKJUDooJfqgYS} nous dira alors que la somme est \( \eC\)-dérivable.

    Nous posons \( u_n(z)=a_nz^n\), qui est une fonction de classe \( C^1\). En ce qui concerne sa différentielle nous considérons \( z_0\in B(0,R)\)  et nous avons    (si \( n=0\) alors la différentielle est nulle)
    \begin{subequations}
        \begin{align}
            (du_n)_{z_0}(z)&=\Dsdd{ u_n(z_0+tz) }{t}{0}\\
            &=\Dsdd{ a_n(z_0+tz)^n }{t}{0}\\
            &=\Dsdd{ na_n(z_0^{n-1}tz) }{t}{0}\\
            &=na_nz_0^{n-1}z.
        \end{align}
    \end{subequations}
    En cours de calcul nous avons développé \( (z_0+tz)^n\) et gardé seulement les termes de degré \( 1\) en \( t\). Il y en a \( n\) et ils sont tous égaux à \( z_0^{n-1}tz\).

    La convergence simple \( \sum_nu_n\) est dans les hypothèses. Il reste à prouver que la somme des différentielles converge uniformément sur tout compact autour de \( z_0\) ne débordant pas du disque ouvert de convergence. Soit \( K\) un compact autour de \( z_0\). Dans le calcul suivant nous utilisons une première fois la norme uniforme de \( du_n\) vu comme fonction de \( K\) vers \( \aL(\eC,\eC)\) et une fois la norme opérateur\footnote{Définition~\ref{DefNFYUooBZCPTr}.} de \( (du_n)_{z_0}\) comme application linéaire \( \eC\to \eC\) :
    \begin{subequations}
        \begin{align}
            \| du_n \|_k&=\sup_{z_0\in K}\| (du_n)_{z_0} \|\\
            &=\sup_{z_0\in K}\sup_{| z |=1}| (du_n)_{z_0}(z) |\\
            &=\sup_{z_0\in K}\sup_{| z |=1}| na_nz_0^{n-1}z |\\
            &=\sup_{z_0\in K}n| a_n | |z_0 |^{n-1}.
        \end{align}
    \end{subequations}
    Vu que \( z\mapsto| z |^{n-1}\) est une application continue sur le compact \( K\), elle atteint son maximum (théorème~\ref{ThoWeirstrassRn}). Nous considérons \( z_K\), un point qui réalise le supremum. Ce nombre est dans le disque de convergence parce que \( K\) est un compact autour de \( z_0\).

    Nous devons prouver que \( \sum_nn| a_n | |z_K |^{n-1}\) converge. Vu que \( | z_K |\) est une constante (par rapport à \( n\)) nous pouvons étudier la convergence en écrivant \( | z_K |^n\) au lieu de \( | z_K |^{n-1}\).

    La suite \( (a_n| z_K |^n)\) est une suite bornée. Soit \( M\) tel que \( | a_n | |z_K |^n<M\) pour tout \( n\). Nous considérons de plus \( r\) de telle sorte que \( K\subset B(0,r)\subset B(0,R)\). En particulier \( | z_K |<r\) et nous avons
    \begin{equation}
        n| a_n | |z_K |^n\leq n| a_n |r^n\left( \frac{ | z_K | }{ r } \right)^n\leq nM\left( \frac{ | z_K | }{ r } \right)^n.
    \end{equation}
    Nous savons que ce qui est dans la parenthèse est plus petit que \( 1\), mais que \( \sum_nnx^n\) converge dès que \( | x |<1\). Par conséquent
    \begin{equation}
        \sum_n\| du_n \|_K
    \end{equation}
    converge et le théorème~\ref{ThoLDpRmXQ} fonctionne : \( du=\sum_{n=1}^{\infty}du_n\) et la somme \( \sum_nu_n\) est de classe \( C^1\).

    La différentielle de \( \sum_nu_n\) s'exprime explicitement par
    \begin{equation}        \label{EqJBFMooMjSABz}
        du_{z_0}(z)=\sum_{n=1}^{\infty}na_nz_0^{n-1}z.
    \end{equation}
    Cette forme montre que \( du_{z_0}\) est une application \( \eC\)-linéaire et donc la somme est \( \eC\)-dérivable par la proposition~\ref{PropKJUDooJfqgYS}. Ergo holomorphe sur le disque de convergence par définition~\ref{DefMMpjJZ}.

    En ce qui concerne la formule \eqref{EqVQDPooOPICwN}, elle provient de la formule \eqref{EqPAEFooYNhYpz} : \( f'(z_0)\) est donné par la facteur multiplicatif de \( du_{z_0}\). En l'occurrence la formule \eqref{EqJBFMooMjSABz} nous donne
    \begin{equation}
        f'(z_0)=\sum_{n\geq 1}na_nz_0^{n-1}.
    \end{equation}
\end{proof}

%---------------------------------------------------------------------------------------------------------------------------
\subsection{Dérivation}
%---------------------------------------------------------------------------------------------------------------------------

\begin{lemma}       \label{LemFVMaSD}
    Soit une série entière \( \sum a_nz^n\) de rayon de convergence \( R\). Les séries
    \begin{equation}
        \sum \frac{ a_n }{ n+1 }z^{n+1}
    \end{equation}
    et
    \begin{equation}
        \sum_{n\geq 1}na_nz^{n-1}
    \end{equation}
    ont même rayon de convergence \( R\).
\end{lemma}

Notons toutefois que nonobstant ce lemme, les séries dont il est question peuvent se comporter différemment sur le bord du disque de convergence. En effet la série
\begin{equation}
    \sum \frac{1}{ n }z^n
\end{equation}
diverge pour \( z=1\) alors que
\begin{equation}
    \sum\frac{1}{ n(n+1) }z^{n+1}
\end{equation}
converge pour \( z=1\).


Les théorèmes de dérivation et d'intégration de séries de fonctions (théorèmes~\ref{ThoCciOlZ} et~\ref{ThoCSGaPY}) fonctionnent bien dans le cas des séries entières. Ils donnent la proposition~\ref{ProptzOIuG} pour la dérivation et~\ref{PropfeFQWr} pour l'intégration.

\begin{proposition}     \label{ProptzOIuG}
    Soit la série entière
    \begin{equation}
        f(x)=\sum_{n=0}^{\infty}a_n x^n
    \end{equation}
    de rayon de convergence \( R\). Alors la fonction \( f\) est \( C^1\) sur \( \mathopen] -R , R \mathclose[\) et se dérive terme à terme :
    \begin{equation}
        f'(x)=\sum_{n=1}^{\infty}na_nx^{n-1}
    \end{equation}
    pour tout \( x\in\mathopen] -R , R \mathclose[\).
\end{proposition}
\index{permuter!série entière et dérivation}

\begin{proof}
    Nous savons que la série \( \sum_{n=1}^{\infty}na_nx^{n-1}\) a le même rayon de convergente que celui de la série \( f\). En particulier cette série des dérivées converge normalement sur tout compact dans \( \mathopen] -R , R \mathclose[\) et la somme est continue. Le théorème~\ref{ThoCSGaPY} conclu.
\end{proof}

\begin{remark}
    À part lorsqu'on parle de fonction \( \eR\to \eR\), la notion de classe \( C^k\) s'entend au sens de la différentielle, et non de la dérivée, voir les définitions~\ref{DefPNjMGqy}. C'est cela qui explique la structure de la démonstration de la proposition~\ref{PropSNMEooVgNqBP}.
\end{remark}

\begin{corollary}[\cite{GYDXooJJusGH,MonCerveau}]       \label{CorCBYHooQhgara}
    La somme d'une série entière est de classe \( C^{\infty}\) sur le disque ouvert de convergence.
\end{corollary}

\begin{proof}
    La proposition~\ref{PropSNMEooVgNqBP} a démontré en réalité nettement plus : sur le disque ouvert de convergence, la somme est une fonction holomorphe. Il n'est cependant pas possible de conclure ainsi parce que le fait qu'une fonction holomorphe est \( C^{\infty}\) ne sera démontré qu'au coût de nombreux efforts dans le théorème~\ref{ThomcPOdd}\ref{ItemMRRTooMChmuZ}.

    \begin{subproof}
    \item[Cas réel]
        Nous considérons la série entière \( \sum_na_nx^n\) pour \( x\in \eR\) de rayon de convergence \( R\). Une simple récurrence sur la proposition~\ref{ProptzOIuG} donne le résultat.
    \item[Cas complexe]
        Attention : le fait d'être de classe \( C^k\) est le fait d'être \( k\) fois \emph{différentiable}. Rien à voir avec la \( \eC\)-dérivabilité.

        En ce qui concerne la différentiabilité nous avons la proposition~\ref{PropSNMEooVgNqBP} qui dit que dans le disque de convergence, la fonction \( u(z)=\sum_na_nz^n\) a pour différentielle l'application \( du\colon \eC\to \aL_{\eC}(\eC,\eC)\),
        \begin{equation}
            \begin{aligned}
                du\colon \eC&\to \aL_{\eC}(\eC,\eC) \\
                du_{z_0}(z)&=\big( \sum_{n=0}^{\infty}(n+1)a_{n+1}z_0^n \big)z.
            \end{aligned}
        \end{equation}
        Nous allons éviter de considérer la différentielle seconde comme une application
        \begin{equation}
            d^2u\colon \eC\to \aL\big( \eC,\aL(\eC,\eC) \big)
        \end{equation}
        parce que ça nous mènerait trop loin pour parler de la différentielle \( k\)\ieme. Au lieu de cela nous allons considérer l'isomorphisme d'espace vectoriel
        \begin{equation}
            \begin{aligned}
                \psi\colon \eC&\to \aL_{\eC}(\eC,\eC) \\
                z_0&\mapsto \psi(z_0) z=z_0z.
            \end{aligned}
        \end{equation}
        Dans cette optique nous écrivons :
        \begin{equation}
            du_{z_0}=\psi\big( \sum_{n=0}^{\infty}(n+1)a_{n+1} z_0^n\big)
        \end{equation}
        ou encore :
        \begin{equation}
            (\psi^{-1}\circ d)u(z_0)=\sum_{n\geq 0}(n+1)a_{n+1}z_{0}^n.
        \end{equation}
        Nous allons prouver par récurrence que l'égalité suivante est vraie (y compris le fait que la somme converge) :
        \begin{equation}
            (\psi^{-1}\circ d)^ku(z_0)=\sum_{n=0}^{\infty}\frac{ (n+k)! }{ n! }a_{n+k}z_0^n.
        \end{equation}
        Prouvons d'abord que cette somme converge pour tout \( k\). Nous avons \( (n+k)!/n!<(n+k)^k\) et donc il suffit de prouver que la série de coefficients \( n^ka_n\) converge. C'est le cas par la proposition~\ref{PropHDIUooKTbVSX}.

        Nous pouvons calculer la différentielle de \( (\psi^{-1}\circ d)^ku\) en dérivant terme à terme en utilisant (encore) la proposition~\ref{PropSNMEooVgNqBP}\ref{ItemUULDooEGRNiA} :
        \begin{subequations}
            \begin{align}
                d\big( (\psi^{-1}\circ d)^k u\big)_{z_0}(z)&=\sum_{n=1}^{\infty}\frac{ (n+k)! }{ n! }a_{n+k}na_{0}^{n-1}z\\
                &=\sum_{n=0}^{\infty}\frac{ (n+k+1)! }{ n! }a_{n+k+1}z_{0}^nz.
            \end{align}
        \end{subequations}
        Nous appliquons \( \psi^{-1}\) à cela :
        \begin{equation}
            (\psi^{-1}\circ d)^{k+1}u(z_0)=\sum_{k=0}^{\infty}\frac{ (n+k+1)! }{ n! }a_{n+k+1}z_0^n.
        \end{equation}

    \item[Dérouler à l'envers]

        Nous allons maintenant utiliser la proposition~\ref{PropEKLTooSvZjdW} pour montrer que \( u\) est de classe \( C^k\) pour tout \( k\). Nous avons démontré que \( (\psi^{-1}\circ d)^ku\) était différentiable. Par conséquent, \( d\big( (\psi^{-1}\circ d)^{k-1}u \big)\) est différentiable et donc \( (\psi^{-1}\circ d)^{k-1}\) est de classe \( C^1\). En continuant ainsi, \( (\psi^{-1}\circ d)^{k-l}u\) est de classe \( C^l\) et \( u\) est de classe \( C^k\).
    \end{subproof}
\end{proof}

Le lemme suivant est encore essentiellement valable dans un espace de Banach (proposition~\ref{PropQAjqUNp}).
\begin{lemma}       \label{LemPQFDooGUPBvF}
    Plusieurs choses sur des séries entières.
    \begin{enumerate}
        \item
    La série entière \( \sum_{n\geq 0}z^{nk}\) a un rayon de convergence \( 1\) et converge vers la fonction
    \begin{equation}
        \sum_{n\geq 0}z^{nk}=\frac{1}{ 1-z^k }.
    \end{equation}
    \item
    Lorsque \( | \omega |=1\), la série
    \begin{equation}        \label{EqSSHZooLwCBAZ}
        \frac{1}{ \omega-z }=\sum_{k\geq 0}\frac{ z^k }{ \omega^{k+1} }.
    \end{equation}
    a un rayon de convergence égal à \( 1\).
\item   \label{ITEMooHFVHooPCgzZV}
    Si \( | \omega |=1\), la série
    \begin{equation}
        \frac{1}{ (\omega-z)^k }=\frac{1}{ (k-1)! }\sum_{s=0}^{\infty}\frac{ (s+k-1)! }{ s! }\frac{ z^s }{ \omega^{s+k+1} }
    \end{equation}
    a un rayon de convergence égal à \( 1\).
    \end{enumerate}
\end{lemma}

\begin{proof}
    Les coefficients de la série sont \( a_n=1\) lorsque \( n\) est multiple de \( k\) et \( a_n=0\) autrement. Donc pour \( r=1\) la suite \( r^na_n\) reste bornée\footnote{Utilisation directe de la définition~\ref{DefZWKOZOl}.}. Cela prouve que le rayon de convergence est au moins \( 1\). Par ailleurs si \( r>1\) alors clairement la suite \( (a_nr^n)\) n'est pas bornée. Cela prouve le rayon de convergence égal à \( 1\).

    Soit donc \( z\in B(0,1)\). Nous avons, par le lemme \ref{LEMooLPBCooRWuvJB},
    \begin{subequations}
        \begin{align}
            \left( \sum_{n=0}^{\infty}z^{nk} \right)(1-z^k)&=\sum_{n=0}^{\infty}z^{nk}-\sum_{n=0}^{\infty}z^{(n+1)k}\\
            &=1+\sum_{n=1}^{\infty}z^{nk}-\sum_{n=0}^{\infty}z^{(n+1)k}\\
            &=1+\sum_{n=0}^{\infty}z^{(n+1)k}-\sum_{n=0}^{\infty}z^{(n+1)k}\\
            &=1
        \end{align}
    \end{subequations}

    En ce qui concerne la série \eqref{EqSSHZooLwCBAZ}, elle s'obtient facilement :
    \begin{equation}
        \frac{1}{ \omega-z }=\frac{1}{  \omega }\frac{1}{ 1-\frac{ z }{ \omega } }=\frac{1}{ \omega }\sum_{s=0}^{\infty}\left( \frac{ z }{ \omega } \right)^s=\sum_s\omega^{-s-1}z^s.
    \end{equation}

    La troisième série s'obtient en dérivant la seconde, ce qui est permis dans le disque de convergence par la proposition~\ref{ProptzOIuG}.
\end{proof}

\begin{remark}
    Sur le bord du disque de convergence, la série \( \sum_nz^{nk}\) ne converge pas. En effet le rayon étant \( 1\), sur le bord nous avons la série \( \sum_n e^{ink\theta}\) dont la norme du terme général ne tend pas vers zéro.
\end{remark}

%---------------------------------------------------------------------------------------------------------------------------
\subsection{Intégration}
%---------------------------------------------------------------------------------------------------------------------------

\begin{proposition} \label{PropfeFQWr}
    Soit la série entière $\sum a_nx^n$ de rayon de convergence \( R\).
    \begin{enumerate}
        \item
            Pour tout segment \( \mathopen[ a , b \mathclose]\subset\mathopen] -R , R \mathclose[\) nous pouvons intégrer terme à terme :
            \begin{equation}
                \int_a^b\left( \sum_{n=0}^{\infty}a_nx^n\right)dx=\sum_{n=0}^{\infty}a_n\int_a^bx^ndx.
            \end{equation}
        \item
            La série entière obtenue en intégrant terme à terme a le même rayon de convergence que celui de la série de départ.
    \end{enumerate}
\end{proposition}
\index{permuter!série entière et intégration}

\begin{proof}
    La première assertion est un cas particulier du théorème général~\ref{ThoCciOlZ}. Pour le rayon de convergence, le lemme~\ref{LemFVMaSD} fait le travail.
\end{proof}

Vu que le rayon de convergence ne varie pas par la dérivation ou par l'intégration et qu'une série entière est de classe \(  C^{\infty}\) sur son disque de convergence, nous pouvons dériver terme à terme autant de fois que nous le voulons sans faire de fautes dans le disque de convergence.

%+++++++++++++++++++++++++++++++++++++++++++++++++++++++++++++++++++++++++++++++++++++++++++++++++++++++++++++++++++++++++++ 
\section{Séries de Taylor}
%+++++++++++++++++++++++++++++++++++++++++++++++++++++++++++++++++++++++++++++++++++++++++++++++++++++++++++++++++++++++++++
\label{SECooDWRMooUKSuPh}

\begin{normaltext}
    Avant de commencer, une petite formule de dérivation toute simple que nous allons utiliser souvent :
    \begin{equation}        \label{EqSOFdwhw}
        (z^k)^{(l)}=\begin{cases}
            0   &   \text{si } l>k\\
            \frac{ k! }{ (k-l)! }z^{k-l}    &    \text{sinon.}
        \end{cases}
    \end{equation}

    Dans les cas où il est permis de dériver terme à terme, nous avons la formule
    \begin{equation}        \label{EQooTNOMooJZClvE}
        f^{(p)}(x)=\sum_ka_k(x^k)^{(p)}=\sum_{k=p}^{\infty}a_k\frac{ k! }{ (k-p)! }x^{k-p}
    \end{equation}
\end{normaltext}

%--------------------------------------------------------------------------------------------------------------------------- 
\subsection{Polynôme de Taylor d'une série entière}
%---------------------------------------------------------------------------------------------------------------------------

Le polynôme de Taylor d'une fonction définie par une série entière s'obtient en tronquant la série. Cela est une assez bonne nouvelle que nous allons démontrer maintenant.

\begin{proposition}[\cite{MonCerveau}]      \label{PROPooQLHNooRsBYbe}
    Soit une série entière
    \begin{equation}
        f(x)=\sum_ka_kx^k
    \end{equation}
    de rayon de convergence \( R>0\).

    Pour tout \( n\in \eN\), il existe une fonction \( \alpha\) telle que
    \begin{equation}    \label{EQooSXUJooFjsVek}
        f(x)=\sum_{k=0}^na_kx^k+\alpha(x)x^n
    \end{equation}
    et 
    \begin{equation}
        \lim_{t\to 0} \alpha(t)=0.
    \end{equation}
    Tout ceci étant convenu que
    \begin{itemize}
        \item 
            l'égalité \eqref{EQooSXUJooFjsVek} est uniquement valable sur le disque de convergence,
        \item La fonction \( \alpha\) dépend de \( n\).
    \end{itemize}
\end{proposition}

\begin{proof}
    Le corolaire \ref{CorCBYHooQhgara} nous indique que \( f\) est de classe \(  C^{\infty}\) sur \( \mathopen] -R , R \mathclose[\) et que nous pouvons dériver terme à terme.

        En utilisant la formule \eqref{EQooTNOMooJZClvE} et en l'évaluant en \( x=x_0\), tous les termes s'annulent sauf \( k=p\):
        \begin{equation}
            f^{(p)}(0)=p!a_p.
        \end{equation}
        Le théorème de Taylor \ref{ThoTaylor} nous indique alors qu'il existe \( \alpha\colon \eR\to \eR\) telle que \( \lim_{t\to 0} \alpha(t)=0\) et
        \begin{equation}
            f(x)=\sum_{k=0}^{n}a_kx^k+\alpha(x)x^n.
        \end{equation}
\end{proof}

%--------------------------------------------------------------------------------------------------------------------------- 
\subsection{Une majoration pour le reste}
%---------------------------------------------------------------------------------------------------------------------------

\begin{lemma}       \label{LEMooOVPIooAPWFOm}
    Soit une fonction \( f\colon \eR\to \eR\) dérivable \( n+1\) fois sur \( B(a,R)\). Alors pour tout \( x\in B(a,r)\),
    \begin{equation}
        f(x)=f(a)+\sum_{k=1}^{n-1}\frac{ (x-a)^{k} }{ k! }f^{(k)}(a)+\int_a^{x}\ldots\int_a^{u_{n-1}}f^{(n)}(u_n)du_n\ldots du_1.
    \end{equation}
\end{lemma}

\begin{proof}
    Nous allons intensivement utiliser le théorème fondamental du calcul intégral \ref{ThoRWXooTqHGbC} sous la forme de la formule \eqref{EqooBBCYooNweVrF}. Nous avons d'abord
    \begin{equation}
        f(x)=f(a)+\int_a^x f'(u_1)du_1=\int_a^x\big[ f'(a)+\int_a^{u_1}f''(u_2)du_2 \big]du_1.
    \end{equation}
    Toute l'astuce de ce théorème est de continuer à substituer \( f^{(k)}(t)\) par \( f^{(k)}(a)\) plus une intégrale de \( a\) à \( t\) de \( f^{(k+1)}(u)\). Nous démontrons ainsi par récurrence que
    \begin{equation}        \label{EQooOWJMooHATpMV}
        f(x)=f(a)+\sum_{k=1}^{n-1}\frac{ (x-a)^k }{ k! }f^{(k)}(a)+\int_a^x\cdots\int_a^{u_{n-1}}f^{(n)}(u_n)du_n\ldots du_1.
    \end{equation}
    La preuve de cela se fait en substituant
    \begin{equation}
        f^{(n)}(u_n)=f^{(n)}(a)+\int_{a}^{u_n}f^{(n+1)}(u_{n+1})du_{n+1}
    \end{equation}
    et en remarquant (encore par récurrence par exemple) que
    \begin{equation}
        \int_a^x\ldots \int_a^{u_{n-1}}du_n\ldots du_1=\frac{ (x-a)^n }{ n! }.   
    \end{equation}
\end{proof}

Le théorème suivant donne majoration du reste du polynôme de Taylor. Il est un premier pas dans la démonstration de formules comme
\begin{equation}
    \lim_{n\to \infty} P_n(x)=f(x)
\end{equation}
lorsque \( P_n\) est un polynôme de Taylor autour d'un point \( a\neq x\). Nous ne saurions trop insister sur le fait que de telles formules ne seraient valables que pour une classe relativement restreintes de fonctions.
\begin{theorem}[Inégalité de Taylor\cite{ooNVJGooGKwDWG}]       \label{THOooEUVEooXZJTRL}
    Soit une fonction \( f\colon \eR\to \eR\) dérivable \( n+1\) fois et telle que \( | f^{(n+1)}(x) |\leq M_N\) sur \( B(a,d)\). Alors
    \begin{equation}
        | R_n(x) |\leq \frac{ M_n }{ (n+1)! }| x-a |^{n+1}
    \end{equation}
    où \( R(x)=f(x)-P_n(x)\) et où \( P_n\) sont les polynômes de Taylor autour de \( a\in \eR\).
\end{theorem}

\begin{proof}
    Nous pouvons écrire la formule du lemme \ref{LEMooOVPIooAPWFOm} pour \( n+1\) au lieu de \( n\); cela donne
    \begin{equation}
        f(x)=P_n(x)+\int\cdots,
    \end{equation}
    et donc
    \begin{equation}
        | R_n(x) |=| P_n(x)-f(x) |=\int_a^x\ldots\int_a^{u_n}f^{(n+1)}(x)du_n\cdots du_{1}
    \end{equation}
    En effectuant toutes les intégrales nous trouvons\quext{Je me demande si je n'ai pas une faute entre \( n\) et \( n+1\) quelque part. Relisez attentivement et écrivez-moi si vous trouvez une faute.}
    \begin{equation}
        | R_n(x) |\leq \frac{ M_n }{ (n+1)! }| x-a |^{n+1}.
    \end{equation}
\end{proof}
Cette formule pour le reste est très bien, mais pour l'exploiter au maximum de ses possibilités, il faudra la notion de convergence de suite de fonctions, et en particulier la notion de série de fonctions, pour pouvoir écrire 
\begin{equation}
    f(x)=\sum_{k=0}^{\infty}\frac{ f^{(k)}(a) }{ k! }x^k
\end{equation}
lorsque cela est possible. Nous renvoyons donc aux séries de Taylor, section \ref{SECooDWRMooUKSuPh}, et en particulier aux fonctions analytiques de la sous-section \ref{SUBSECooXKHWooEzqGRJ}.

%--------------------------------------------------------------------------------------------------------------------------- 
\subsection{Fonctions analytiques}
%---------------------------------------------------------------------------------------------------------------------------
\label{SUBSECooXKHWooEzqGRJ}

Nous avons vu les polynômes de Taylor et déjà noté qu'il n'est pas en général vrai que \( \lim_{n\to \infty} P_n(x)=f(x)\) pour des \( x\) même proches du point autour duquel les polynômes de Taylor \( P_n\) sont calculés.

Nous allons maintenant étudier la classe des fonctions pour lesquelles la série de Taylor est égale à la fonction de départ. D'abord une proposition montrant que les coefficients de Taylor sont les seuls pour lesquels il est possible d'espérer avoir une telle propriété.
\begin{proposition}[\cite{ooSBUJooIuujhF}]      \label{PROPooTRWVooETTtbP}
    Soit une fonction donnée par la série entière
    \begin{equation}
        f(x)=\sum_{k=0}^{\infty}c_n(x-a)^n
    \end{equation}
    sur la boule de convergence \( B(a,R)\) avec \( R>0\) (hypothèse : le rayon de convergence est strictement positif). Alors
    \begin{equation}
        c_n=\frac{ f^{(n)}(a) }{ n! }.
    \end{equation}
\end{proposition}

\begin{proof}
    Par hypothèse, nous avons un rayon de convergence \( R>0\), et le corolaire \ref{CorCBYHooQhgara} nous indique que \( f\) y est de classe \(  C^{\infty}\). Et nous pouvons dériver terme à terme par la proposition \ref{ProptzOIuG}. Cela pour dire qu'il nous est autorisé d'utiliser la formule \eqref{EQooTNOMooJZClvE} pour calculer les dérivées de \( f\) au point \( a\). Nous avons d'abord
    \begin{equation}
        f^{(p)}(x)=\sum_{n=p}^{\infty}c_n\frac{ n! }{ (n-p)! }(x-a)^{n-p},
    \end{equation}
    et donc
    \begin{equation}
        f^{(p)}(a)=c_pp!
    \end{equation}
    qui donne immédiatement le résultat.
\end{proof}

\begin{proposition}
    Soit l'intervalle \( I=B(a,r)\). Si il existe \( M\) tel que 
    \begin{equation}
        | f^{(n)}(x) |\leq \frac{ M }{ r^n }n!
    \end{equation}
    pour tout \( x\in B(a,r)\). Alors nous avons la convergence simple
    \begin{equation}
        P_n\to f
    \end{equation}
    sur \( B(a,r)\). Ici, \( P_n\) est le polynôme de Taylor d'ordre \( n\) pour la fonction \( f\) autour du point \( a\)\footnote{Pour être complet, il faut préciser que \( P_n\) est calculé dans ZFC. C'est pour cela que nous n'écrivons pas des lourdeurs comme \( P_{n,a}(f)(x)\); si il fallait donner tout le contexte dans la notation, on n'en sortirait pas.

Ah, et tant que j'y suis si vous ne savez pas ce qu'est ZFC, je vous déconseille fortement de répéter cela à un jury d'agrég, entre autres parce que vous allez attirer la question «vraiment ? Vous utilisez C ? Où ? Pourquoi ?». Et là, bonne chance.}.
\end{proposition}

\begin{proof}
    Vu que nous avons \( | f^{(n)}(x) |\leq \frac{ M }{ r^n }n!\) pour tout \( x\), nous pouvons poser 
    \begin{equation}
        M_n=\frac{ M }{ r^n }n!
    \end{equation}
    dans le théorème \ref{THOooEUVEooXZJTRL} pour le faire fonctionner. Nous avons alors
    \begin{equation}
        | R_n(x) |\leq \frac{ M }{ r^n }n!\frac{1}{ (n+1)! }| x-a |^{n+1}=\frac{ M }{ n+1 }| x-a |\left| \frac{ x-a }{ r } \right|^n.
    \end{equation}
    Vu que \( x\in B(a,r)\) nous avons \( | x-a |<r\) et donc \( |(x-a)/r |^n<1\). Nous pouvons aussi majorer \( | x-a |\) par \( r\) et écrire
    \begin{equation}
        | R_n(x) |\leq \frac{ rM }{ n+1 }.
    \end{equation}
    Nous avons donc bien \( \lim_{n\to \infty} R_n(x)\to 0\).
\end{proof}
