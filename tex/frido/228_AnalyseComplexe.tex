% This is part of Le Frido
% Copyright (c) 2021-2023, 2025
%   Laurent Claessens
% See the file fdl-1.3.txt for copying conditions.

%+++++++++++++++++++++++++++++++++++++++++++++++++++++++++++++++++++++++++++++++++++++++++++++++++++++++++++++++++++++++++++ 
\section{Analyse complexe en plusieurs variables}
%+++++++++++++++++++++++++++++++++++++++++++++++++++++++++++++++++++++++++++++++++++++++++++++++++++++++++++++++++++++++++++

Pour rappel, la définition d'une application holomorphe est \ref{DefMMpjJZ} et elle est directement donnée en plusieurs variables.

\begin{definition}[\cite{BIBooQYDXooDWJciq}]
	Nous définissons le \defe{polydisque}{polydisque} centré en \( a\in \eC^n\) et de «rayons» \( r\in \eC^n\) par
	\begin{equation}
		D(a,r)=\{ z\in \eC^n\tq | z_j-a_j |<r_j\,\forall j=1,\ldots, n \}.
	\end{equation}
\end{definition}

\begin{definition}
	Pour une fonction \( f\colon \eC^n\to \eC^n\), nous notons
	\begin{equation}
		\frac{ \partial f }{ \partial z_j }=\frac{ 1 }{2}\left( \frac{ \partial  }{ \partial x_j }-i\frac{ \partial  }{ \partial y_j } \right)f.
	\end{equation}
\end{definition}

\begin{proposition}	\label{PROPooKGOIooWRIoAq}
	Une application \(f \colon \Omega\to \eC  \) est holomorphe si et seulement si pour tout \( j\in\{ 1,\ldots,n \}\) et pour tout \( z_1,\ldots,z_{j-1}, j_{j+1},\ldots,j_n\in \Omega\), l'application partielle
	\begin{equation}
		z\mapsto f(z_1,\ldots,z_{j_1}, z,z_{j+1},\ldots,z_n)
	\end{equation}
	est holomorphe.
	%TODOooCRAUooMxNCzY. Prouver ça.
\end{proposition}


\begin{definition}[\cite{BIBooXNQTooNgvlKd}]
	Soient des ouverts \( B_1\) et \( B_2\) dans \( \eC^n\). Nous disons que \( f\colon B_1\to B_2\) est \defe{biholomorphe}{biholomorphe} si
	\begin{enumerate}
		\item
		      \( f\) est holomorphe,
		\item
		      \( f\) est une bijection entre \( B_1\) et \( B_2\),
		\item
		      \( f^{-1}\) est holomorphe.
	\end{enumerate}
\end{definition}

%-------------------------------------------------------
\subsection{Série}
%----------------------------------------------------

\begin{proposition}	\label{PROPooJWDGooLLnldZ}
	Soient des application holomorphe \(f_n \colon B_{_\eC^n}(0,r)\to \eC  \) telles que \( \| f_n \|<M_n\) avec \( \sum_{n=0}^{\infty}M_n<\infty\). Alors l'application
	\begin{equation}
		\sum_{n=0}^{\infty}f_n \colon B_{\eC^n(0,r)}\to \eC
	\end{equation}
	est holomorphe.
	%TODOooHVZFooRsQJNy. Prouver ça.
\end{proposition}


%-------------------------------------------------------
\subsection{Inversion locale}
%----------------------------------------------------


\begin{theorem}[Inversion locale, version holomorphe]       \label{THOooNBGZooHuGtxW}
	Soient des ouverts \( B_1\) et \( B_2\) dans \( \eC^n\). Nous considérons une fonction holomorphe \( f\colon B_1\to B_2\). Soit \( z_0\in B_1\) et \( w_0=f(z_0)\). Il y a équivalence entre
	\begin{enumerate}
		\item
		      \( \det(df_{z_0})\neq 0\)
		\item
		      Il existe un voisinage \( U\) de \( z_0\) dans \( B_1\) et \( V\) de \( w_0\) dans \( B_2\) tels que la restriction \( f\colon U\to V\) est biholomorphe.
	\end{enumerate}
	%TODOooNTYPooQpqvhk. Prouver ça.
\end{theorem}


\begin{proposition}[\cite{BIBChatGPT}]	\label{PROPooSOAWooDZswKV}
	Soient un ouvert \( \Omega\subset \eC\), un ouvert \( \Lambda\) de \( \eC^n\) ainsi que \( \lambda\in\Lambda\) et \( z_0\in \Omega\). Nous considérons une application holomorphe \(f \colon \Omega\times \Lambda\to \eC  \) et
	\begin{equation}		\label{EQooERLJooYwqRxO}
		\begin{aligned}
			g\colon \Omega & \to \eC                            \\
			z              & \mapsto \int_{z_0}^zf(s,\lambda)ds
		\end{aligned}
	\end{equation}
	où l'intégrale est celle sur le chemin droit joignant \( z_0\) à \( z\) dans \( \eC\).

	Alors
	\begin{enumerate}
		\item
		      \( g\) est holomorphe sur \( \Omega\)
		\item		\label{ITEMooJWJOooCNjrDl}
		      \( g'(z)=f(z,\lambda)\).
	\end{enumerate}
\end{proposition}

\begin{proof}
	Il s'agit de vérifier les hypothèses de \ref{ThopCLOVN}, en notant une subtilité de notations : dans \ref{ThopCLOVN}, l'intégration se fait sur la deuxième variable alors que dans \eqref{EQooERLJooYwqRxO} l'intégration se fait sur la première; il faut donc s'adapter. Les hypothèses \ref{ITEMooMWCTooJgLVKa} et \ref{ITEMooAJHBooCFqhfq} sont vérifiées parce que \( f\) est holomorphe et donc ses fonctions partielles aussi (proposition \ref{PROPooKGOIooWRIoAq}).

	Soit un compact \( K\) dans \( \Lambda\). Vu que \( f\) est holomorphe, nous pouvons considérer la fonction constante
	\begin{equation}
		\begin{aligned}
			g_K\colon \mathopen[ z_0,z\mathclose] & \to \eR                                                                                   \\
			s                                     & \mapsto \sup_{(u,\lambda)\in \mathopen[ z_0,z\mathclose]\times K}| f(u,\lambda) |<\infty.
		\end{aligned}
	\end{equation}
	En tant qu'intégrale sur le compact \( \mathopen[ z_0,z\mathclose]\) nous avons
	\begin{equation}
		\int_{z_0}^zg_K(s)ds<\infty.
	\end{equation}
	Cela vérifie l'hypothèse \ref{ITEMooPNFYooXRzWby}. Et conclu donc la preuve.
\end{proof}

%-------------------------------------------------------
\subsection{Extension d'analytique vers holomorphe}
%----------------------------------------------------


\begin{lemma}[\cite{MonCerveau}]	\label{LEMooSOYJooLLilgz}
	Soit un polynôme \( P\) homogène de degré \( k\) sur \( \eC^n\). Nous avons
	\begin{equation}
		\sup_{z\in B(z_0,r)}| P(z-z_0) |\leq \| z-z_0 \|^k\sup_{z\in B(z_0,1)}| P(z-z_0) |.
	\end{equation}
	%TODOooNSLMooFquzrE. Prouver ça.
\end{lemma}


\begin{lemma}[\cite{BIBChatGPTDifficile}]	\label{LEMooAPNOooJSpgUi}
	Si \( P\) est un polynôme réel homogène de degré \( k\) sur \( \eR^n\), nous notons \( \tilde P\) son extension complexe\footnote{C'est à dire : mêmes coefficients (réels), mais en acceptant les variables complexes.}.

	Soient \( k,n\in \eN\). Il existe des réels \( C_{k,n}>0\) tels que pour tout polynôme réel homogène de degré \( k\),
	\begin{enumerate}
		\item
		      \begin{equation}
			      \sup_{\| z \|\leq 1}| \tilde P(z) |\leq C_{k,n}\sup_{\| x \|\leq 1}| P(x) |.
		      \end{equation}
		\item
		      Si \( x_0\in \eR^n\), alors
		      \begin{equation}
			      \sup_{z\in B(x_0,r)}| \tilde P(z-x_0) |\leq C_{k,n}\sup_{x\in B(x_0,r)}| P(x-x_0) |.
		      \end{equation}
		\item
		      Pour chaque \( n\),
		      \begin{equation}
			      \lim_{k\to \infty}C_{k,n}^{1/k}=1.
		      \end{equation}
	\end{enumerate}
	Notez le point important : le même nombre \( C_{k,n}\) fonctionne pour tous les polynôme en même temps.
	%TODOooPLSVooFSYtIB. Prouver ça.
\end{lemma}


Le théorème suivant dit que si \( f\) est analytique sur une \( \eR^{n}\)-boule autour de \( x_0\), alors en prenant la même formule, nous avons une application analytique sur une \( \eC^n\)-boule à peine plus petite\footnote{Plus petite en rayon, mais de dimension deux fois plus grande.}.
\begin{theorem}[\cite{BIBChatGPTDifficile}]	\label{THOooXJNJooFucWqn}
	Soient \( 0<\rho<r\) et une application analytique\footnote{Définition \ref{DEFooAIHMooKbWsBt}.} \(f \colon B_{\eR^n}(x_0,r)\to \eR  \) s'écrivant sous la forme
	\begin{equation}
		f(x)=\sum_{k=0}^{\infty} a_k(x-x_0),
	\end{equation}
	avec
	\begin{equation}
		a_k(x)=\sum_{\alpha\in\Lambda_k}a_{k,\alpha}x^{\alpha}.
	\end{equation}
	Pour chaque \( k\) nous posons
	\begin{equation}
		\begin{aligned}
			\tilde a_k\colon B_{\eC^n}(x_0,\rho) & \to \eC                                                   \\
			z                                    & \mapsto \sum_{\alpha\in \Lambda_k}a_{k,\alpha}z^{\alpha}.
		\end{aligned}
	\end{equation}
	Alors
	\begin{enumerate}
		\item
		      l'application
		      \begin{equation}
			      \begin{aligned}
				      \tilde f\colon B_{\eC^n}(x_0,\rho) & \to \eC                                      \\
				      z                                  & \mapsto \sum_{k=0}^{\infty}\tilde a_k(z-x_0)
			      \end{aligned}
		      \end{equation}
		      est holomorphe.
		\item
		      Une telle extension est unique : si \( \tilde f_1\) et \( \tilde f_2\) sont deux telles extensions alors \( \tilde f_1=\tilde f_2\).
	\end{enumerate}
\end{theorem}

\begin{proof}
	D'abord en tant que sommes finies, chacune des \( \tilde a_k\) est holomorphe; ce sont des polynômes homogènes de degré \( k\). Soit \( \rho>0\); pour tout \( z\in B(z_0,\rho)\) et pour tout \( k\), en utilisant le lemme \ref{LEMooAPNOooJSpgUi} nous avons
	\begin{subequations}
		\begin{align}
			| \tilde a_k(z-x_0) | & \leq \| z-x_0 \|^k\sup_{s\in B_{\eC^n}(x_0,1)}| \tilde a_k(s-x_0) | \\
			                      & \leq \| z-x_0 \|^kC_k\sup_{x\in B_{\eR^n}(x_0,1)}| a_k(x-x_0) |     \\
			                      & \leq \| z-x_0 \|^kC_k\| a_k \|                                      \\
			                      & \leq \rho^k C_k\| a_k \|
		\end{align}
	\end{subequations}
	Vu que \( f\) est analytique nous avons par ailleurs
	\begin{equation}
		\sum_{k=0}^{\infty}r^k\| a_k \|<\infty.
	\end{equation}
	Vu que \( C_k^{1/k}\to 1\) nous savons que \( C_k\) peut très bien croître polynômialement; sans hypothèses supplémentaires, nous ne pouvons rien dire à propos de la convergence de
	\begin{equation}
		\sum_{k=0}^{\infty}| \tilde a_k(z-x_0) |\leq \sum_{k=0}^{\infty}\rho^k C_k\| a_k \|.
	\end{equation}
	Nous choisissons donc \( \rho<r\) pour avoir
	\begin{equation}
		\rho^kC_k\| a_k \|=\left( \frac{ \rho }{ r } \right)^kr^kC_k\| a_k \|.
	\end{equation}
	Étant donné que \( r^kC_k\) a au plus une croissance polynômiale et que \( (\rho/r)^k\) a une décroissance exponentielle, nous avons
	\begin{equation}
		| \tilde a(z-x_0) |\leq \left( \frac{ \rho }{ r } \right)^kr^kC_k\| a_k \|\leq r^k\| a_k \|
	\end{equation}
	dès que \( k\) est grand (lemme \ref{LemLJOSooEiNtTs}). Tout cela pour dire que
	\begin{equation}
		\| \tilde a_k \|_{\infty}<\left( \frac{ \rho }{ r } \right)^kr^kC_k\| a_k \|<r^k\| a_k \|
	\end{equation}
	où à gauche \( \| \tilde a_k \|_{\infty}\) est la norme uniforme sur \( B_{\eC^n}(x_0,\rho)\) et non la norme en tant qu'application \( k\)-multiniléaire. Donc en posant
	\begin{equation}
		M_k=r^k\| a_k \|,
	\end{equation}
	nous avons \( \| \tilde a_k \|<M_k\) avec \( \sum_{k=0}^{\infty}M_k<\infty\). La proposition \ref{PROPooJWDGooLLnldZ} nous dit alors qu'en posant
	\begin{equation}
		\tilde f(z)=\sum_{k=0}^{\infty}\tilde a_k(z-x_0),
	\end{equation}
	nous avons une fonction holomorphe sur \( B(x_0,\rho)\).

	Nous prouvons à présent l'unicité. D'abord \( x_0\) n'est pas un zéro isolé de \( \tilde f_1-\tilde f_2\) parce que \( \tilde f_1=\tilde f_2\) sur un voisinage réel de \( x_0\). Donc, par le principe des zéros isolée (théorème \ref{ThoukDPBX}), \( \tilde f_1-\tilde f_2=0\) sur un voisinage complexe de \( x_0\). Ensuite le corollaire \ref{CORooFBXXooZyfUQi} nous dit que \( \tilde f_1-\tilde f_2=0\) sur tout \( B_{\eC^n}(x_0,\rho)\).

	\begin{probleme}
		Pour être tout à fait correct, il faudrait une version à plusieurs variables du principe des zéros isolés. Vous êtes \randomGender{le bienvenu}{la bienvenue} pour m'envoyer un énoncé avec une démonstration.
		%TODOooBYHKooEvhJOB. Le faire. C'est déjà sur ma feuille des choses à faire.
	\end{probleme}
\end{proof}

%-------------------------------------------------------
\subsection{Espace de fonctions holomorphes}
%----------------------------------------------------

\begin{proposition}	\label{PROPooFFXSooBLoTSq}
	Soit \( z_0\in \eC^n\) et
	\begin{equation}
		X=\{\varphi \colon\overline{B(z_0),r}\to \eC^{m}\tq
		\begin{cases}
			\varphi\text{ est holomorphe sur \( B(z_0,r)\)} \\
			\varphi\text{ est continue sur \( \overline{B(z_0,a)}\)}.
		\end{cases}
		\}
	\end{equation}
	Alors \( (X,\|  \|_{\infty})\) est fermé dans \( \big( C^0(z_0,r),\| . \|_{î} \big)\).
	%TODOooCJOKooNtQTYq. Prouver ça.
\end{proposition}


%-------------------------------------------------------
\subsection{Cauchy-Lipschitz holomorphe}
%----------------------------------------------------

\begin{proposition}[Cauchy-Lipschitz holomorphe\cite{BIBChatGPTDifficile, MonCerveau}]	\label{PROPooYYYDooXeEcVw}
	Soient \( (t_0,y_0,\lambda_0)\in \eC\times \eC^n\times \Lambda\)	 (ici \( \Lambda=C^m\)) ainsi qu'une application \(f \colon \eC\times \eC^n\times \Lambda\to \eC^n  \) holomorphe\footnote{Définition \ref{DefMMpjJZ}.} sur un voisinage de \( (t_0,y_0,\lambda_0)\).
	Alors :
	\begin{enumerate}
		\item
		      Il existe \( a,\rho>0\) tels que le système
		      \begin{subequations}		\label{SUBEQSooPTXEooQMupUi}
			      \begin{numcases}{}
				      (\partial_ty)(t,\lambda)=f\big( t,y(t,\lambda),\lambda \big)\\
				      y(t_0,\lambda)=y_0
			      \end{numcases}
		      \end{subequations}
		      admette une unique solution \(y \colon B(t_0,a)\times B(\lambda_0,\rho)\to \eC^n  \).
		\item
		      Cette solution \( y\) est holomorphe.
		\item
		      Il existe un \( r>0\) tel que pour tout \( (t,\lambda)\in \overline{B(t_0,a)}\times \overline{B(\lambda_0,\rho)}\), nous avons
		      \begin{equation}
			      \| y(t,\lambda)-y_0 \|<r.
		      \end{equation}
	\end{enumerate}
\end{proposition}

\begin{proof}
	En plusieurs parties.
	\begin{subproof}
		\spitem[Sécurité]
		%-----------------------------------------------------------

		Soit \( R>0\) tel que \( f\) soit holomorphe sur
		\begin{equation}
			U=\overline{B(t_0,R)}\times \overline{B(y_0,R)}\times \overline{B(\lambda_0,R)}.
		\end{equation}
		Vu que \( U\) est compact et que \( f\) y est continue, elle est majorée et nous posons
		\begin{equation}
			\sup_{(t,y,\lambda)\in U}| f(t,y,\lambda) |\leq M.
		\end{equation}

		\spitem[Une application \( g\) intermédiaire]
		%-----------------------------------------------------------

		Nous fixons ensuite \( (t_1,\lambda_1)\in B(t_0,R)\times B(\lambda_0,R)\) ainsi que l'application
		\begin{equation}
			\begin{aligned}
				g\colon \overline{B(y_0,R)} & \to \eC^n                   \\
				y                           & \mapsto g(t_1,y,\lambda_1).
			\end{aligned}
		\end{equation}
		Cette application est holomorphe par la proposition \ref{PROPooKGOIooWRIoAq}. Nous appliquons à chaque composante \( g_i\) l'estimateur de Cauchy (proposition \ref{PROPooRKVVooNmfHkz}\ref{ITEMooMWMSooRoXLgL}) avec la majoration \( 1/R\geq 1/(R-r)\) :
		\begin{equation}
			| g_i'(y) |\leq \frac{1}{ R}\sup_{| w-y_0 |=R}| g(w) |\leq \frac{ M }{ R }.
		\end{equation}
		La norme sur \( \eC^n\) est celle de la définition \ref{DEFooGUXNooXwCsrq}, et nous avons donc
		\begin{equation}
			\| g'(z) \|  =\sqrt{ \langle g'(z), g'(z) \rangle }
			=\sqrt{ \sum_{i=1}^n| g_i(z) |^2 }
			\leq\sqrt{\sum_{i=1}^n\frac{ M }{ R }}
			\leq \frac{ \sqrt{n}M }{ R }.
		\end{equation}

		\spitem[Une condition Lipschitz]
		%-----------------------------------------------------------
		Soient \( y,z\in B(y_0,R)\). Nous utilisons les accroissements finis \ref{val_medio_2} :
		\begin{equation}
			\| g(y)-g(z) \|\leq \sup_{x\in \mathopen[ y,z,\mathclose]}\| dg_x \|\| y-z \|.
		\end{equation}
		Mais nous savons que pour une application holomorphe, la différentielle est l'application qui consiste à multiplier par la dérivée complexe (proposition \ref{PropKJUDooJfqgYS}). Donc en posant
		\begin{equation}
			L=\frac{ \sqrt{n}M }{ R }
		\end{equation}
		nous avons
		\begin{equation}
			\| f(t_1,yl_1)-f(t_1,z,\lambda_1) \|\leq L\| y-z \|.
		\end{equation}

		\spitem[L'espace de Banach pour la contraction]
		%-----------------------------------------------------------
		Soient \( a,\rho<R\). Nous posons
		\begin{equation}
			X=\{\varphi \colon\overline{B(t_0),a}\times \overline{B(\lambda_0,\rho)}\to \eC^{n}\tq
			\begin{cases}
				\varphi\text{ est holomorphe sur \( B(t_0,a)\times B(\lambda_0,\rho)\)}                     \\
				\varphi\text{ est continue sur \( \overline{B(t_0,a)}\times \overline{B(\lambda_0,\rho)}\)} \\
				\| \varphi-y_0 \|\leq r
			\end{cases}
			\}
		\end{equation}
		Et maintenant on parle de fermeture et de complétude.
		\begin{enumerate}
			\item
			      Par le lemme \ref{LemdLKKnd}), l'espace \( \big( C^0(\ldots),\| . \|_{\infty} \big)\) est de Banach
			\item
			      Nous considérons \( \mF=\big( X,\| . \|_{\infty} \big)\).
			\item
			      La partie \( \mF\) est fermée dans \( C^0(\ldots)\) par proposition \ref{PROPooFFXSooBLoTSq}.
			\item
			      En tant que fermé dans un complet, la partie \( \mF\) est complète.
		\end{enumerate}
		Nous pourrions donc utiliser le théorème de Picard sur \( \mF\).

		\spitem[La suite]
		%-----------------------------------------------------------
		Nous allons définir la contraction \( \mF\to\mF\) qui à \( \varphi\) fait correspondre
		\begin{equation}
			\phi(t,\lambda)=y_0+\int_{t_0}^tf\big( s,\varphi(s,\lambda),\lambda \big)ds
		\end{equation}
		Nous allons maintenant prouver ce qu'il faut pour que cette application soit bien définie et soit une contraction.

		Pour la suite nous considérons \( \varphi\in \mF\) et nous posons
		\begin{equation}
			\begin{aligned}
				\phi\colon \overline{B(t_0,a)}\times \overline{B(\lambda_0,\rho)} & \to \eC^n                                                            \\
				(t,\lambda)                                                       & \mapsto y_0+\int_{t_0}^tf\big( s,\varphi(s,\lambda),\lambda \big)ds.
			\end{aligned}
		\end{equation}
		Nous allons prouver que si \( a\) et \( \rho\) sont bien choisis, \( \phi\in \mF\).

		\spitem[L'intégrale a un sens]
		%-----------------------------------------------------------
		Si on fixe \( \lambda_1\in \overline{B(\lambda_0,\rho)}\) l'application
		\begin{equation}
			\begin{aligned}
				g\colon \mathopen[ t_0,t\mathclose] & \to \eC^n                                             \\
				s                                   & \mapsto f\big( s,\varphi(s,\lambda_1),\lambda_1 \big)
			\end{aligned}
		\end{equation}
		est continue en tant que composée d'applications continues. Donc l'intégrale sur le compact \( \mathopen[ t_0,t\mathclose]\) ne pose pas de problèmes.

		\spitem[\( \phi\) est continue sur l'ouvert]
		%-----------------------------------------------------------
		Attention que l'intégrale \( \int_{t_0}^t\) est l'intégrale sur le segment dans \( \eC\) joignant \( t_0\) à \( t\). Il n'est donc pas vrai que \( \int_{t_0}^{t+\delta}=\int_{t_0}^t+\int_t^{t+\delta}\). La définition plus explicite est que
		\begin{equation}
			\phi(t,\lambda)=y_0+\int_{t_0}^tf\big( s,\varphi(s,\lambda),\lambda \big)ds=y_0+\int_{0}^1f\Big( \gamma(u),\varphi\big( \gamma(u),\lambda \big),\lambda  \Big)du
		\end{equation}
		avec \( \gamma(u)=tu+(1-u)t_0\).

		De même, en prenant \( \delta\in \eC\) tel que \( t+\delta\) reste dans \( \overline{(t_0,a)}\) nous avons
		\begin{equation}
			\phi(t+\delta,\lambda+\delta)=y_0+\int_0^1f\Big( \sigma(u),\varphi\big( \sigma(u),\lambda \big),\lambda \Big)du
		\end{equation}
		avec \( \sigma(u)=(t+\delta)u+(1-u)t_0\).

		Nous avons donc
		\begin{equation}
			\| \phi(t,\lambda)-\phi(t+\delta,\lambda+\delta) \|\leq \int_{0}^1\| f\Big( \sigma(u),\varphi\big( \sigma(u),\lambda+\delta \big),\lambda+\delta \Big)-f\Big( \gamma(u),\varphi\big( \gamma(u),\lambda \big),\lambda \Big) \|du.
		\end{equation}
		Là dedans, tout est continu en \( \delta\) et \( \gamma(u)-\sigma(u)=\delta u\). Nous avons donc bien
		\begin{equation}
			\lim_{\delta\to 0}\| \phi(t,\lambda)-\phi(t+\delta,\lambda+\delta) \|=0,
		\end{equation}
		et donc la continuité de \( \phi\).

		\spitem[\( \phi\) est holomorphe sur le fermé]
		%-----------------------------------------------------------
		C'est la proposition \ref{PROPooSQXZooBzIrji} qui le dit.

		\spitem[Si \( a<rM\) alors \( \phi\in \mF\)]
		%-----------------------------------------------------------
		Nous avons supposé que \( \| \varphi-y_0 \|<r\). Nous prouvons à présent que si \( a<rM\) alors \( \| \phi-y_0 \|<r\). Soit donc \( (t,\lambda)\in \overline{B(t_0,a)}\times \overline{B(\lambda_0,\rho)}\). Nous avons
		\begin{equation}
			\| \phi(t,\lambda)-y_0 \|\leq\int_{t_0}^t\| f\big( s,\varphi(s,\lambda),\lambda \big) \|ds<aM.
		\end{equation}

		\spitem[L'application \( T\)]
		%-----------------------------------------------------------
		Tout ce que nous venons de faire nous permet de considérer l'application
		\begin{equation}
			\begin{aligned}
				T\colon \mF           & \to \mF                                                        \\
				T(\varphi)(t,\lambda) & = y_0+\int_{t_0}^tf\big( s,\varphi(s,\lambda),\lambda \big)ds.
			\end{aligned}
		\end{equation}

		\spitem[Si \( aL<1\) alors \( T\) est une contraction]
		%-----------------------------------------------------------
		Soient \( \varphi,\phi\in\mF\) ainsi que \( (t,\lambda)\in\overline{B(t_0,a)}\times\overline{B(\lambda_0,\rho)}\). Nous avons
		\begin{subequations}
			\begin{align}
				\| T(\phi)-T(\varphi) \| & \leq \int_{t_0}^t\| f\big( s,\varphi(s,\lambda),\lambda \big)-f\big( s,\phi(s,\lambda),\lambda \big) \|ds \\
				                         & \leq L\int_{t_0}^t\| \varphi(s,t)-\phi(s,t) \|                                                            \\
				                         & \leq aL \| \varphi-\phi \|_{\infty}.
			\end{align}
		\end{subequations}
		Donc si \( aL<1\), l'application \( T\) est une contraction.

		\spitem[Picard]
		%-----------------------------------------------------------
		Nous appliquons le théorème de point fixe de Picard \ref{ThoEPVkCL} et nous considérons l'unique point fixe \( \varphi_0\in\mF\) de \( T\).

		\spitem[Point fixe \( \Rightarrow\) solution]
		%-----------------------------------------------------------
		Nous prouvons que \( \varphi_0\) est une solution. Vu que \( \varphi_0\in \mF\), nous savons déjà qu'elle vérifie
		\begin{equation}
			\varphi_0(t,\lambda)=y_0+\int_{t_0}^tf\big( s,\varphi_0(s,\lambda),\lambda \big)ds.
		\end{equation}
		Nous fixons \( \lambda_1\in B(\lambda_0,\rho)\) et nous considérons la fonction
		\begin{equation}
			\begin{aligned}
				g\colon B(t_0,a) & \to \eC^n                       \\
				t                & \mapsto \varphi_0(t,\lambda_1),
			\end{aligned}
		\end{equation}
		c'est à dire
		\begin{equation}
			g(t)=y_0+\int_{t_0}^tf\big( s,g(s,\lambda_1),\lambda_1 \big)ds.
		\end{equation}
		C'est le moment d'utiliser \ref{PROPooSOAWooDZswKV}\ref{ITEMooJWJOooCNjrDl} :
		\begin{equation}
			g'(z)=f\big( t,g(s,\lambda_1),\lambda_1 \big).
		\end{equation}
		Donc \( \varphi_0\) est bien une solution.

		\spitem[Solution \( \Rightarrow\) point fixe]
		%-----------------------------------------------------------
		Nous prouvons que si \( y\) est une solution de \eqref{SUBEQSooPTXEooQMupUi} alors \( T(y)=y\). Nous fixons \( \lambda_1\in\Lambda\) et nous posons \( g(t)=y(t,\lambda_1)\). Donc
		\begin{equation}
			g'(t)=f\big( t,y(t,\lambda_1),\lambda_1 \big).
		\end{equation}
		Nous avons alors, en utilisant le théorème \ref{THOooMFFMooBvJdFK} :
		\begin{subequations}
			\begin{align}
				T(y)(t,\lambda_1) & =y_0+\int_{t_0}^tf\big( s,y(s,\lambda_1),\lambda_1 \big)ds \\
				                  & =y_0+\int_{t_0}^tg'(s)                                     \\
				                  & =y_0+g(t)-g(t_0)                                           \\
				                  & =y_0+g(t)-\underbrace{y(t_0,\lambda_1)}_{=y_0}             \\
				                  & =g(t)                                                      \\
				                  & =y(t,\lambda_1).
			\end{align}
		\end{subequations}
		Nous avons donc bien \( T(y)=y\).
	\end{subproof}
\end{proof}
