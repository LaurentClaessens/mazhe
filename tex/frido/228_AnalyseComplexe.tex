% This is part of Le Frido
% Copyright (c) 2021-2023
%   Laurent Claessens
% See the file fdl-1.3.txt for copying conditions.

%+++++++++++++++++++++++++++++++++++++++++++++++++++++++++++++++++++++++++++++++++++++++++++++++++++++++++++++++++++++++++++ 
\section{Analyse complexe en plusieurs variables}
%+++++++++++++++++++++++++++++++++++++++++++++++++++++++++++++++++++++++++++++++++++++++++++++++++++++++++++++++++++++++++++

\begin{definition}[\cite{BIBooQYDXooDWJciq}]
	Nous définissons le \defe{polydisque}{polydisque} centré en \( a\in \eC^n\) et de «rayons» \( r\in \eC^n\) par
	\begin{equation}
		D(a,r)=\{ z\in \eC^n\tq | z_j-a_j |<r_j\,\forall j=1,\ldots, n \}.
	\end{equation}
\end{definition}

\begin{definition}
	Pour une fonction \( f\colon \eC^n\to \eC^n\), nous notons
	\begin{equation}
		\frac{ \partial f }{ \partial z_j }=\frac{ 1 }{2}\left( \frac{ \partial  }{ \partial x_j }-i\frac{ \partial  }{ \partial y_j } \right)f.
	\end{equation}
\end{definition}

\begin{definition}[\cite{BIBooQYDXooDWJciq}]
	Soit un ouvert\footnote{La topologie sur \( \eC^n\) est celle de la norme définie en \ref{DEFooGUXNooXwCsrq}.} \( D\) dans \( \eC^n\). Nous disons que \( f\colon D\to \eC\) est \defe{holomorphe}{holomorphe} si
	\begin{enumerate}
		\item
		      \( f\) est continue sur \( D\),
		\item
		      pour tout \( z\in D\) et pour tout \( 1\leq j\leq n\), le nombre \( \frac{ \partial f }{ \partial z_j }(z)\) existe et est fini.
	\end{enumerate}
\end{definition}

\begin{definition}[\cite{BIBooXNQTooNgvlKd}]
	Soient des ouverts \( B_1\) et \( B_2\) dans \( \eC^n\). Nous disons que \( f\colon B_1\to B_2\) est \defe{biholomorphe}{biholomorphe} si
	\begin{enumerate}
		\item
		      \( f\) est holomorphe,
		\item
		      \( f\) est une bijection entre \( B_1\) et \( B_2\),
		\item
		      \( f^{-1}\) est holomorphe.
	\end{enumerate}
\end{definition}

\begin{theorem}[Inversion locale, version holomorphe]       \label{THOooNBGZooHuGtxW}
	Soient des ouverts \( B_1\) et \( B_2\) dans \( \eC^n\). Nous considérons une fonction holomorphe \( f\colon B_1\to B_2\). Soit \( z_0\in B_1\) et \( w_0=f(z_0)\). Il y a équivalence entre
	\begin{enumerate}
		\item
		      \( \det(df_{z_0})\neq 0\)
		\item
		      Il existe un voisinage \( U\) de \( z_0\) dans \( B_1\) et \( V\) de \( w_0\) dans \( B_2\) tels que la restriction \( f\colon U\to V\) est biholomorphe.
	\end{enumerate}
	%TODOooNTYPooQpqvhk. Prouver ça.
\end{theorem}
