% This is part of Mes notes de mathématique
% Copyright (c) 2012-2017, 2019-2021
%   Laurent Claessens
% See the file fdl-1.3.txt for copying conditions.

Nous donnons ici quelques idées de développements associés aux leçons données dans le rapport du jury 2016\cite{ooJECQooJvIKEJ}. Parfois, il est bon d'ajouter quelques lemmes au développement proposé, s'il est trop court. Si l'un ou l'autre ne vous semble pas adapté à l'énoncé de la leçon, faites le moi savoir.

%+++++++++++++++++++++++++++++++++++++++++++++++++++++++++++++++++++++++++++++++++++++++++++++++++++++++++++++++++++++++++++
\section{Algèbre et géométrie}
%+++++++++++++++++++++++++++++++++++++++++++++++++++++++++++++++++++++++++++++++++++++++++++++++++++++++++++++++++++++++++++


%+++++++++++++++++++++++++++++++++++++++++++++++++++++++++++++++++++++++++++++++++++++++++++++++++++++++++++++++++++++++++++
\section{Analyse}
%+++++++++++++++++++++++++++++++++++++++++++++++++++++++++++++++++++++++++++++++++++++++++++++++++++++++++++++++++++++++++++

\paragraph{Extremums : existence, caractérisation, recherche. Exemples et applications.}
\paragraph{Approximation d'une fonction par des fonctions régulières.  Exemples et applications.}
\paragraph{Théorème d'inversion locale, théorème des fonctions implicites. Exemples et applications en analyse et en géométrie.}
\paragraph{Applications différentiables définies sur un ouvert de $\eR^n$ . Exemples et applications.}
\paragraph{Équations différentielles ordinaires. Exemple de résolution et d'études de solutions en dimension \( 1\) et \( 2\).}
\paragraph{Exemples d'équations aux dérivées partielles linéaires}
\paragraph{Suites numériques. Convergence, valeurs d'adhérence. Exemples et applications.}
\paragraph{Suites vectorielles et réelles définies par une relation de récurrence \( u_{n+1}=f(u_n)\). Exemples. Applications à la résolution approchée d'équations.}


\begin{itemize}
    \item Extrema liés, théorème~\ref{ThoRGJosS}.
    \item Théorème d'inversion locale, théorème~\ref{ThoXWpzqCn}.
    \item Lemme de Morse, lemme~\ref{LemNQAmCLo}.
\end{itemize}

\paragraph{Espaces de Hilbert. Bases hilbertiennes. Exemples et applications.}
\begin{itemize}
    \item Espace de Sobolev \( H^1(I)\), théorème~\ref{ThoESIyxfU}.
    \item Inégalité isopérimétrique, théorème~\ref{ThoIXyctPo}.
    \item Dual de \( L^p\big( \mathopen[ 0 , 1 \mathclose] \big)\) pour \( 1<p<2\), proposition~\ref{PropOAVooYZSodR}.
        % Fonctions de Haar
\end{itemize}

\paragraph{Espaces de fonctions : exemples et applications.}
\begin{itemize}
    \item Théorème de Fischer-Riesz~\ref{ThoGVmqOro}.
    \item Espace de Sobolev \( H^1(I)\), théorème~\ref{ThoESIyxfU}.
    \item Théorème de Cauchy-Lipschitz~\ref{ThokUUlgU}.
    \item Dual de \( L^p\big( \mathopen[ 0 , 1 \mathclose] \big)\) pour \( 1<p<2\), proposition~\ref{PropOAVooYZSodR}.
\end{itemize}

\paragraph{Utilisation de la notion de compacité.}
\begin{itemize}
    \item Le théorème de Weierstrass sur la limite uniforme de fonctions holomorphes, théorème~\ref{ThoArYtQO}.
    \item Suite telle que \( \lim_{k\to \infty} d(u_{k+1},u_k)=0\), théorème~\ref{PropLHWACDU}.
    \item Sous-groupes compacts de \( \GL(n,\eR)\), lemme~\ref{LemOCtdiaE} ou proposition~\ref{PropQZkeHeG}.
    \item Théorème de Montel~\ref{ThoXLyCzol}.
    \item Ellipsoïde de John-Loewner, proposition~\ref{PropJYVooRMaPok}.
\end{itemize}
\paragraph{Connexité. Exemples et applications.}
\begin{itemize}
    \item Théorème de Runge~\ref{ThoMvMCci}.
    \item Suite telle que \( \lim_{k\to \infty} d(u_{k+1},u_k)=0\), théorème~\ref{PropLHWACDU}.
    \item Théorème de Brouwer en dimension \( 2\) via l'homotopie~\ref{ThoLVViheK}.
    \item Théorème de Lie-Kolchin~\ref{ThoUWQBooCvutTO}.
\end{itemize}
\paragraph{Espaces complets. Exemples et applications.}
\begin{itemize}
    \item La proposition~\ref{PropWoywYG} qui donne des indications sur la notion de classes dans \( L^p\).
    \item Prolongement de fonction définie sur une partie dense, théorème~\ref{ThoPVFQMi}
    \item Complétion d'un espace métrique, théorème~\ref{ThoKHTQJXZ}.
    \item Théorème de Fischer-Riesz~\ref{ThoGVmqOro}.
    \item Théorème de Cauchy-Lipschitz global~\ref{THOooZIVRooPSWMxg}.
\end{itemize}
\paragraph{Prolongement de fonctions. Exemples et applications.}
\begin{itemize}
    \item Prolongement de fonction définie sur une partie dense, théorème~\ref{ThoPVFQMi}
    \item Lemme de Borel~\ref{LemRENlIEL}.
    \item Prolongement méromorphe de la fonction \( \Gamma\) d'Euler.
    \item Théorème de Tietze~\ref{ThoFFQooGvcLzJ}.
        % théorème de Hadamard
\end{itemize}
\paragraph{Espaces vectoriels normés, applications linéaires continues. Exemples.}
\begin{itemize}
    \item Théorème de Fischer-Riesz~\ref{ThoGVmqOro}.
    \item Théorème de Banach-Steinhaus~\ref{ThoPFBMHBN}.
    \item Dual de \( L^p\big( \mathopen[ 0 , 1 \mathclose] \big)\) pour \( 1<p<2\), proposition~\ref{PropOAVooYZSodR}.
\end{itemize}
