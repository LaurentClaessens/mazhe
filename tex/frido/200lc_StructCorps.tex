%---------------------------------------------------------------------------------------------------------------------------
\subsection{Polynômes à plusieurs variables}
%---------------------------------------------------------------------------------------------------------------------------

Nous avons déjà vu \( A[X,Y]\) lorsque \( A\) est un anneau en la définition~\ref{DEFooZNHOooCruuwI}.

\begin{definition}      \label{DEFooRHRKooPqLNOp}
    Soit un corps \( \eK\). Le corps \( \eK(X_1,\ldots, X_n)\) est le corps des fractions de l'anneau \( \eK[X_1,\ldots, X_n]\).
\end{definition}

\begin{definition}  \label{DEFooOCPHooXneutp}
    Soient un corps \( \eK\) et une extension \( \eL\) de \( \eK\) contenant les éléments \( \alpha_1\),\ldots, \( \alpha_n\) de \( \eK\). Nous définissons \( \eK(\alpha_1,\ldots, \alpha_n)\) comme étant l'intersection de tous les sous-corps de \( \eL\) contenant \( \eK\) et les \( \alpha_i\).
\end{definition}

La proposition suivante est analogue à~\ref{PROPooYSFNooFGbbCi}\ref{ITEMooATPTooVXKdlK}.

\begin{lemma}[\cite{MonCerveau}]        \label{LEMooQEJHooAmSNxU}
    Soient un corps \( \eK\), une extension \( \eL\) et des éléments \( \alpha_1,\ldots, \alpha_n\) dans \( \eL\). Alors
    \begin{equation}
        \eK(\alpha_1,\ldots, \alpha_n)=\{ r(\alpha_1,\ldots, \alpha_n)\tq r\in \eK(X_1,\ldots, X_n) \}.
    \end{equation}
\end{lemma}

\begin{proof}
    Ce que nous avons à droite est un corps : par exemple pour l'inverse, si \( r=P/Q\) alors \( r(\alpha_1,\ldots,\alpha_n)=P(\alpha_1,\ldots, \alpha_n)Q(\alpha_1,\ldots, \alpha_n)^{-1}\). Cet élément a un inverse en la personne de \( (Q/P)(\alpha_1,\ldots, \alpha_n)\).

    Donc à droite nous avons un sous-corps de \( \eL\) contenant \( \eK\) ainsi que les \( \alpha_i\). Donc
    \begin{equation}
        \eK(\alpha_1,\ldots, \alpha_n)\subset \big\{ r(\alpha_1,\ldots, \alpha_n)\tq r\in \eK(X_1,\ldots, X_n) \big\}.
    \end{equation}

    D'autre part, tout corps contenant \( \eK\) et les \( \alpha_i\) doit contenir tous les \( P(\alpha_1,\ldots, \alpha_n)\) (\( P\in \eK[X_1,\ldots, X_n]\)), leurs inverses ainsi que leurs produits; bref doit contenir tous les \( r(\alpha_1,\ldots, \alpha_n)\) avec \( r\in\eK[X_1,\ldots, X_n]\).
\end{proof}

%---------------------------------------------------------------------------------------------------------------------------
\subsection{Racines de polynômes}
%---------------------------------------------------------------------------------------------------------------------------

\begin{corollary}[Factorisation d'une racine]   \label{CorDIYooEtmztc}
    Soit \( P\in \eK[X]\), un polynôme de degré \( n\) et \( \alpha\in \eK\) tel que \( P(\alpha)=0\). Alors il existe un polynôme \( Q\) de degré \( n-1\) tel que \( P(x)=(X-\alpha)Q\).
\end{corollary}
\index{factorisation!de polynôme}

\begin{proof}
    Il s'agit d'un cas particulier de la proposition~\ref{PropXULooPCusvE} : si \( \alpha\in \eK\) alors son polynôme minimal dans \( \eK\) est \( X-\alpha\); donc \( X-\alpha\) divise \( P\). Il existe un polynôme \( Q\) tel que \( P=(X-\alpha)Q\). Le degré est alors immédiat.
\end{proof}

Avant de lire l'énoncé suivant, allez relire la définition \ref{NORMooQFTJooLBcPxl} pour savoir ce qu'est un polynôme nul.
\begin{theorem}[Polynôme qui a tellement de racines qu'il s'annule]\label{ThoLXTooNaUAKR}
    Soit \( \eK\) un corps et \( P\in \eK[X]\) un polynôme de degré \( n\) possédant \( n+1\) racines distinctes \( \alpha_1\),\ldots, \( \alpha_{n+1}\), alors \( P=0\).
\end{theorem}
\index{racine!de polynôme}

\begin{proof}
    Si \( P\) est de degré \( 1\), il s'écrit \( P=aX+b\); s'il a comme racines \( \alpha\) et \( \beta\), nous avons le système
    \begin{subequations}
        \begin{numcases}{}
            a\alpha+b=0\\
            a\beta+b=0.
        \end{numcases}
    \end{subequations}
    La différence entre les deux donne \( a(\alpha-\beta)=0\). Vu que \( \alpha\neq \beta\), la règle du produit nul (lemme~\ref{LemAnnCorpsnonInterdivzer}) nous donne \( a=0\). Maintenant que \( a=0\), l'annulation de \( b\) est alors immédiate.

    Nous faisons maintenant la récurrence en supposant le théorème vrai pour le degré \( n\) et en considérant un polynôme \( P\) de degré \( n+1\) possédant \( n+2\) racines distinctes. Vu que \( P(\alpha_1)=0\), le corollaire~\ref{CorDIYooEtmztc} nous donne un polynôme \( Q\) de degré \( n\) tel que
    \begin{equation}    \label{EqQGSooNdTWfz}
        P=(X-\alpha_1)Q.
    \end{equation}
    Étant donne que pour tout \( i\neq 1\) nous avons \( \alpha_i\neq \alpha_1\),
    \begin{equation}
        0=P(\alpha_i)=\underbrace{(\alpha_i-\alpha_1)}_{\neq 0}Q(\alpha_i),
    \end{equation}
    et la règle du produit nul donne \( Q(\alpha_i)=0\). Par conséquent le polynôme \( Q\) est de degré \( n\) et possède \( n+1\) racines distinctes; tous ses coefficients sont alors nuls par hypothèse de récurrence. Tous les coefficients du produit \eqref{EqQGSooNdTWfz} sont alors également nuls.
\end{proof}

\begin{probleme}
On a déjà utilisé par ailleurs le fait qu'un polynôme ayant davantage de racines que son degré s'annule. Donc ce théorème doit être énoncé et prouvé plus haut.
\end{probleme}

\begin{example}\label{ExGRHooBNpjSP}
    Un polynôme à plusieurs variables peut s'annuler en une infinité de points sans être nul. Par exemple le polynôme \( X^2+Y^2-1\in\eR[X,Y]\) s'annule sur tout un cercle de \( \eR^2\) mais n'est pas nul, loin s'en faut.

    Nous verrons dans la proposition~\ref{PropTETooGuBYQf} une condition pour qu'un polynôme à plusieurs variables s'annule du fait qu'il ait «trop» de racines.
\end{example}

\begin{remark}
    L'intérêt du théorème~\ref{ThoLXTooNaUAKR} est que si l'on prouve qu'un polynôme s'annule sur un corps infini, alors il s'annulera sur n'importe quel autre corps. Nous aurons un exemple d'utilisation de cela dans le théorème de Cayley-Hamilton~\ref{ThoHZTooWDjTYI}.
\end{remark}


