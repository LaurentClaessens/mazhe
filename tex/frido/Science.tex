% This is part of Mes notes de mathématique
% Copyright (c) 2006-2013,2015, 2018
%   Laurent Claessens, Carlotta Donadello
% See the file fdl-1.3.txt for copying conditions.

Dans ce chapitre nous donnons des applications de divers théorèmes dans les autres sciences que la mathématique.

%+++++++++++++++++++++++++++++++++++++++++++++++++++++++++++++++++++++++++++++++++++++++++++++++++++++++++++++++++++++++++++
\section{Démystification du MRUA}
%+++++++++++++++++++++++++++++++++++++++++++++++++++++++++++++++++++++++++++++++++++++++++++++++++++++++++++++++++++++++++++
\label{SecMRUAsecondeGGdQoT}

%---------------------------------------------------------------------------------------------------------------------------
\subsection{Preuve de la formule}
%---------------------------------------------------------------------------------------------------------------------------

Nous sommes maintenant en mesure de donner une démonstration complète de la formule du MRUA :
\begin{equation}    \label{EqMRUAINT}
	x(t) = \frac{ at^2 }{ 2 } + v_0t +x_0.
\end{equation}

Au niveau de la physique, nous considérons un mobile qui se déplace avec une accélération constante \( a\). Nous notons par \( v_0\) sa vitesse initiale et par \( x_0\) sa position initiale.

Nous savons que, pour tout mouvement, si \( x(t)\) est la position en fonction du temps, et si \( v(t)\) et \( a(t)\) représentent la vitesse et l'accélération en fonction du temps, alors
\begin{equation}
	\begin{aligned}[]
		v(t) & =x'(t) & \text{et} &  & a(t)=v'(t)=x''(t).
	\end{aligned}
\end{equation}
Afin de trouver \( x(t)\) en connaissant \( a(t)\), il « suffit » donc de prendre deux fois la primitive. Essayons ça dans le cas facile du MRUA où \( a(t)=a\) est constante.

La vitesse \( v(t)\) doit être une primitive de la constante \( a\). Il est facile de voir que \( v(t)=at\) est une primitive de \( a\). Par le corolaire \ref{CorZeroCst}(bis),
\begin{equation}    \label{EqvtatC}
	v(t)=at+C_1
\end{equation}
pour une certaine constante \( C_1\). Afin de fixer \( C_1\), il faut faire appel à la physique : d'après la formule \eqref{EqvtatC}, la vitesse initiale est \( v(0)=C_1\). Donc il faut identifier \( C_1\) à la vitesse initiale : \( C_1=v_0\). Nous avons donc déjà obtenu que
\begin{equation}
	v(t)=at+v_0.
\end{equation}
Afin de trouver \( x(t)\), il faut trouver une primitive de \( v(t)\). Il n'est pas très difficile de voir que \( at^2/2 + v_0t\) fonctionne, donc il existe une constante \( C_2\) telle que
\begin{equation}
	x(t)=\frac{ at^2 }{ 2 }+v_0t+C_2.
\end{equation}
Encore une fois, regardons la condition initiale : la formule donne comme position initiale \( x(0)=C_2\), et donc nous devons identifier \( C_2\) avec la position initiale \( x_0\). En définitive, nous avons bien
\begin{equation}
	x(t) = \frac{ at^2 }{ 2 } + v_0t +x_0.
\end{equation}

Cette formule est donc maintenant \emph{démontrée} à partir de la seule définition de la vitesse comme dérivée de la position et de l'accélération comme dérivée de la vitesse.

Remarquons cependant que la preuve complète fut \emph{très} longue. En effet, nous avons utilisé les règles de dérivation de la proposition \ref{PROPooOUZOooEcYKxn}, pour la démonstration desquels, les résultats \ref{ThoLimLin} et \ref{ThoLimLinMul} ont étés utiles. Mais nous avons surtout utilisé le corolaire \ref{CorZeroCst}(bis) qui repose sur le théorème de Rolle \ref{ThoRolle}, qui lui-même demande le théorème de Borel-Lebesgue \ref{ThoBOrelLebesgue} dans lequel la notion d'ensemble compact a été cruciale.

%---------------------------------------------------------------------------------------------------------------------------
\subsection{Interprétation graphique}
%---------------------------------------------------------------------------------------------------------------------------

La distance parcourue \( x(t)\) en un temps \( t\) est la primitive de la vitesse. Nous avons, par ailleurs, que l'opération inverse de la dérivée donnait la surface. Pour reprendre les mêmes notations, nous notons \( S_v(t)\) la surface contenue en dessous de la fonction \( v\) entre \( 0\) et \( x\). Nous ne serions donc pas étonné que
\begin{equation}        \label{EqEncoreMRUASvt}
	S_v(t) = \frac{ at^2 }{ 2 }+v_0t+x_0
\end{equation}
soit la surface en dessous de la fonction \( v(t)=at+v_0\). Nous voyons que la surface totale sous la fonction \( v(t)=at+v_0\) est exactement
\begin{equation}
	S_v(t)=\frac{ at^2 }{ 2 }+v_0t.
\end{equation}
Cela est un bon début, mais hélas nous ne retrouvons pas le terme « \( +x_0\) » de la formule \eqref{EqEncoreMRUASvt}. Cela n'est pas tout à fait étonnant parce que nous savons que la surface sous une fonction était \emph{une} primitive de la fonction, mais nous n'avons pas dit \emph{laquelle}. D'après le fameux corolaire \ref{CorZeroCst}(bis), la primitive n'est définie qu'à une constante près. Ici, c'est la constante \( x_0\) qu'on a perdue en chemin.

Nous parlerons plus en détail du lien entre les surfaces et les primitives dans la section dédié à l'intégration.
