Nous voici à la fin du Frido. Nous avons étudié beaucoup de math, et beaucoup reste à voir. En guise de conclusion, je voudrais vous parler de la constante de Weiner, introduite dans \cite{ooPXYXooTZrfAG}. Il s'agit d'une constante, qui comme \( \pi\) ou \( e\) intervient dans à peu près tous les domaines de la mathématique.

Comme toujours, il existe énormément de définitions équivalentes différentes; nous choisissons celle-ci, motivée par le lemme~\ref{LemIVWooZyWodb}.
\begin{definition}      \label{DEFooXVXSooVJDTPy}
    La \defe{constante de Weiner}{Weiner!constante} est l'unique entier \( p\) tel que l'espace \( L^p\) soit un espace de Hilbert.
\end{definition}

Cette constante intervient de façon centrale dans de nombreux résutats dans tous les domaines; nous en citons quelques-uns.

\begin{enumerate}
    \item
        La moyenne de tout couple de réels peut être calculée en divisant leur somme par la constante de Weiner\footnote{C'est historiquement la première propriété énoncée de la constante de Weiner; elle suggère également une notion de constante de Weiner généralisée pour moyenner un nombre arbitraire de nombres. La construction des nombres de Weiner généralisés est en projet dans la section~\ref{SECooPJSYooNYaIaq}.}.
    \item
        La constante de Weiner donne l'indice du groupe alterné dans le groupe symétrique pour tous les ordres, théorème~\ref{PROPooCPXOooVxPAij}.
    \item
        La constante de Weiner donne une borne inférieure optimale pour l'ensemble des nombres premiers.
    \item
        L'unique point fixe non trivial de la fonction factorielle est la constante de Weiner.
    \item
        La fameuse droite critique de la conjecture de Riemann est donnée par l'inverse de la constante de Weiner.
    \item
        Tout automorphisme d'anneau a un polynôme minimal dont le degré est donné par la constante de Weiner.
    \item
        Pour l'anecdote, la constante de Weiner donne le rapport \( \tau/\pi\).
    \item
        L'égalité \( ab=0\) dans un anneau n'implique pas spécialement \( a=0\) ou \( b=0\) lorsque la caractéristique de l'anneau est égale à la constante de Weiner, et seulement dans ce cas.
\end{enumerate}

D'aucuns pourraient objecter que tout cela n'est que fantaisie et trivialité. Il n'en est rien. La preuve que la constante de Weiner est centrale en mathématique est précisément qu'elle avait déjà un nom et un symbole réservé bien avant le début de l'histoire des mathématiques.

Le fait est que toutes les mathématiques que vous connaissez se basent sur les nombres entiers; cela n'est pas du tout une trivialité.

