% Copyright (c) 2008-2021
%   Laurent Claessens
% See the file fdl-1.3.txt for copying conditions.

%+++++++++++++++++++++++++++++++++++++++++++++++++++++++++++++++++++++++++++++++++++++++++++++++++++++++++++++++++++++++++++
\section{Espace des fonctions continues}
%+++++++++++++++++++++++++++++++++++++++++++++++++++++++++++++++++++++++++++++++++++++++++++++++++++++++++++++++++++++++++++

\begin{definition}
    Soit \( I\), un intervalle de \( \eR\). L'\defe{oscillation}{oscillation!d'une fonction} sur \( I\) est le nombre
    \begin{equation}
        \omega_f(I)=\sup_{x\in I}f(x)-\inf_{x\in I}f(x).
    \end{equation}
\end{definition}
    Pour chaque \( x\) fixé, la fonction
    \begin{equation}
        x\mapsto \omega_f\big( B(x,\delta) \big)
    \end{equation}
    est une fonction positive, croissante et a donc une limite (pour \( \delta\to 0\)). Nous notons \( \omega_f(x)\) cette limite qui est l'\defe{oscillation}{oscillation!d'une fonction en un point} de \( f\) en ce point. Une propriété immédiate est que \( f\) est continue en \( x_0\) si et seulement si \( \omega_f(x_0)=0\).

    \begin{lemma}       \label{LemuaPbtQ}
    L'ensemble des points de discontinuité d'une fonction \( f\colon \eR\to \eR\) est une réunion dénombrable de fermés.
\end{lemma}

\begin{proof}
    Soit \( D\) l'ensemble des points de discontinuité de \( f\). Nous avons
    \begin{equation}
        D=\bigcup_{n=1}^{\infty}\{ x\tq \omega_f(x)\geq \frac{1}{ n } \}.
    \end{equation}
    Il nous suffit donc de montrer que pour tout \( \epsilon\), l'ensemble
    \begin{equation}
        \{ x\tq \omega_f(x)<\epsilon \}
    \end{equation}
    est ouvert. Soit en effet \( x_0\) dans cet ensemble. Il existe \( \delta\) tel que \( \omega_f\big( B(x_0,\delta) \big)<\epsilon\). Si \( x\in B(x_0,\delta)\), alors si on choisit \( \delta'\) tel que \( B(x,\delta')\subset B(x_0,\delta)\), nous avons \( \omega_f\big( B(x,\delta') \big)<\epsilon\), ce qui justifie que \( \omega_f(x)<\epsilon\) et donc que \( x\) est également dans l'ensemble considéré.
\end{proof}

\begin{theorem}
    L'ensemble des points de discontinuité d'une limite simple de fonctions continues est de première catégorie.
\end{theorem}

\begin{proof}
    Soit \( (f_n)\) une suite de fonctions qui converge simplement vers \( f\). Nous devons écrire l'ensemble des points de discontinuité de \( f\) comme une union dénombrable d'ensembles tels que sur tout intervalle \( I\), aucun de ces ensembles n'est dense. Nous savons déjà par le lemme~\ref{LemuaPbtQ} que l'ensemble des points de discontinuité  de \( f\) est donné par
    \begin{equation}
        D=\bigcup_{n=1}^{\infty}\{ x\tq \omega_f(x)\geq \frac{1}{  n } \}.
    \end{equation}
    Nous essayons donc de prouver que pour tout \( \epsilon\), l'ensemble
    \begin{equation}
        F=\{ x\tq \omega_f(x)\geq \epsilon \}
    \end{equation}
    est nulle part dense. Soit
    \begin{equation}
        E_n=\bigcap_{i,j>n}\{ x\tq | f_i(x)-f_j(x) |<\epsilon \}.
    \end{equation}
    Nous montrons que cet ensemble est fermé en étudiant le complémentaire. Soit \( x\notin E_n\); alors il existe un couple \( (i,j)\) tel que
    \begin{equation}
        | f_i(x)-f_j(x) |>\epsilon.
    \end{equation}
    Par continuité, cette inégalité reste valide dans un voisinage de \( x\). Donc il existe un voisinage de \( x\) contenu dans \( \complement E_n\) et \( E_n\) est donc fermé.

    De plus nous avons \( E_n\subset E_{n+1}\) et \( \bigcup_nE_n=\eR\). Ce dernier point est dû au fait que pour tout \( x\), il existe \( N\) tel que \( i,j>N\) implique \( | f_i(x)-f_j(x) |\leq \epsilon\). Cela est l'expression du fait que la suite \( \big( f_n(x) \big)_{n\in \eN}\) est de Cauchy.

    Soit \( I\), un intervalle fermé de \( \eR\). Nous voulons trouver un intervalle \( J\subset I\) sur lequel \( f\) est continue. Nous écrivons \( I\) sous la forme
    \begin{equation}
        I=\bigcup_{n=1}^{\infty}(E_n\cap I).
    \end{equation}
    Tous les ensembles \( J_n=E_n\cap I\) ne peuvent être nulle part dense en même temps (à cause du théorème de Baire~\ref{ThoQGalIO}). Il existe donc un \( n\) tel que \( J_n\) contienne un ouvert \( J\). Le but est de montrer que \( f\) est continue sur \( J\). Pour ce faire, nous n'allons pas simplement majorer \( | f(x)-f(x_0) |\) par \( \epsilon\) lorsque \( | x-x_0 |\) est petit. Nous allons majorer l'oscillation de \( f\) sur \( B(x_0,\delta)\) lorsque \( \delta\) est petit. Pour cela nous prenons \( x_0\) et \( x\) dans \( J\) et nous écrivons
    \begin{equation}
        | f(x)-f(x_0) |\leq | f(x)-f_n(x) |+| f_n(x)-f_n(x_0) |.
    \end{equation}
    À ce niveau nous rappelons que \( n\) est fixé par le choix de \( J\), dans lequel \( \epsilon\) est déjà inclus. Nous choisissons évidemment \( | x-x_0 |\leq \delta\) de telle sorte que le second terme soit plus petit que \( \epsilon\) en vertu de la continuité de \( f_n\). Pour le premier terme, pour tout \( i,j\geq n\) nous avons
    \begin{equation}
        | f_i(x)-f_j(x) |<\epsilon.
    \end{equation}
    Si nous posons \( j=n\) et \( i\to\infty\), en tenant compte du fait que \( f_i\to f\) simplement,
    \begin{equation}
        | f(x)-f_n(x) |\leq \epsilon.
    \end{equation}
    Nous avons donc obtenu \( | f(x)-f_n(x_0) |\leq 2\epsilon\). Cela signifie que dans un voisinage de rayon \( \delta\) autour de \( x_0\), les valeurs extrêmes prises par \( f(x) \) sont \( f_n(x_0)\pm 4\epsilon\). Nous avons donc prouvé que pour tout \( \epsilon\), il existe \( \delta\) tel que
    \begin{equation}
        \omega_f\big( \mathopen[ x_0-\delta , x_0+\delta \mathclose] \big)\leq 4\epsilon.
    \end{equation}
    De là nous concluons que
    \begin{equation}
        \lim_{\delta\to 0}\omega_f\big( \mathopen[ x_0-\delta , x_0+\delta \mathclose] \big)=0,
    \end{equation}
    ce qui signifie que \( f\) est continue en \( x_0\).
\end{proof}

\begin{example}
    Une fonction discontinue sur \( \eQ\) et continue ailleurs. La fonction
    \begin{equation}
        f(x)=\begin{cases}
            0    &   \text{si } x\notin \eQ\\
            \frac{1}{ q }    &    \text{si } x=p/q
        \end{cases}
    \end{equation}
    où par «\( x=p/q\)» nous entendons que \( p/q\) est la fraction irréductible.

    Cette fonction est discontinue sur \( \eQ\) parce que si \( q\in \eQ\) alors \( f(q)\neq 0\) alors que dans tous voisinage de \( q\) il existe un irrationnel sur qui la fonction vaudra zéro.

    Montrons que \( f\) est continue sur les irrationnels. Si \( x_0\notin \eQ\) alors \( f(x_0)=0\). Mais si on prend un voisinage suffisamment petit de \( x_0\), nous pouvons nous arranger pour que tous les rationnels aient un dénominateur arbitrairement grand. En effet si nous nous fixons un premier rayon \( r_0>0\) alors il existe un nombre fini de fractions de la forme \( 1\), \( \frac{ k }{2}\), \( \frac{ k }{ 3 }\),\ldots, \( \frac{ k }{ N }\) dans \( B(x_0,r_0)\). Il suffit maintenant de choisir \( 0<r\leq r_0\) tel que ces fractions soient toutes hors de \( B(x_0,r)\). Dans cette boule nous avons \( f<\frac{1}{ N }\). Du coup \( f\) est continue en \( x_0\).
\end{example}

\begin{definition}[Point périodique\cite{TMCHooOaTrJL}]
    Soit \( f\colon I\to I\) une application d'un ensemble \( I\) dans lui-même. Si \( x\in I\) vérifie \( f^n(x)=x\) et \( f^k(x)\neq x\) pour \( k=1,\ldots, n-1\) alors on dit que \( x\) est un point \( n\)-périodique.
\end{definition}

\begin{lemma}       \label{LemAONBooGZBuYt}
    Soit \( I\) un segment\footnote{définition~\ref{DefLISOooDHLQrl}. Un segment est un intervalle fermé borné.} de \( \eR\) et une fonction continue \( f\colon I\to I\). Si \( K\) est un segment fermé avec \( K\subset f(I)\) alors il existe un segment fermé \( L\subset I\) tel que \( K=f(L)\).
\end{lemma}

\begin{proof}
    Mentionnons immédiatement que \( f\) est continue sur \( I\) qui est compact\footnote{Par le lemme~\ref{LemOACGWxV}.}. Par conséquent tous les nombres dont nous allons parler sont finis parce que \( f\) est bornée par le théorème~\ref{ThoMKKooAbHaro}.

    Soit \( K=\mathopen[ \alpha , \beta \mathclose]\). Si \( \alpha=\beta\) alors le segment \( L=\{ a \}\) convient. Nous supposons donc que \( \alpha\neq \beta\) et nous considérons \( a,b\in I\) tels que \( \alpha=f(a)\) et \( \beta=f(b)\). Vu que \( a\neq b\) nous supposons \( a<b\) (le cas \( a>b\) se traite de façon similaire).

    Nous posons
    \begin{equation}
        A=\{ x\in\mathopen[ a , b \mathclose]\tq f(x)=\alpha \}.
    \end{equation}
    C'est un ensemble borné par \( a\) et \( b\). De plus il est fermé; ce dernier point n'est pas tout à fait évident parce que \( f\) n'est pas définit sur \( \eR\) mais sur \( I\) qui est fermé, le corolaire~\ref{CorNNPYooMbaYZg} n'est donc pas immédiatement utilisable. Prouvons donc que \( Z=\{ x\in \eR\tq f(x)=\alpha \}\) est fermé. Si \( x_0\) est hors de \( Z\) alors soit \( x_0\) est dans \( I\) soit il est hors de \( I\). Dans ce second cas, le complémentaire de \( I\) étant ouvert, on a un voisinage de \( x_0\) hors de \( I\) et par conséquent hors de \( Z\). Si au contraire \( x_0\in I\) alors il y a (encore) deux cas : soit \( x_0\in\Int(I)\) soit \( x_0\) est sur le bord de \( I\). Dans le premier cas, le théorème des valeurs intermédiaires\footnote{Théorème~\ref{ThoValInter}.} fonctionne. Pour le second cas, nous supposons \( x_0=\max(I)\) (le cas \( x_0=\min(I)\) est similaire). Le théorème des valeurs intermédiaires dit que sur \( \mathopen[ x_0-\epsilon , x_0 \mathclose]\), \( f\neq \alpha\) et en même temps, sur \( \mathopen] x_0 , x_0+\epsilon \mathclose]\), nous sommes en dehors du domaine. Au final \( \{ f(x)=\alpha \}\) est fermé et \( A\) est alors fermé en tant que intersection de deux fermés.

    L'ensemble \( A\) étant non vide (\( a\in A\)), il possède donc un maximum que nous nommons \( u\) :
    \begin{equation}
        u=\max(A).
    \end{equation}
    Nous posons aussi
    \begin{equation}
        B=\{ x\in \mathopen[ u , b \mathclose]\tq f(x)=\beta \}
    \end{equation}
    qui est encore fermé, borné et non vide. Nous pouvons donc définir
    \begin{equation}
        v=\min(B).
    \end{equation}
    Nous prouvons maintenant que \( f\big( \mathopen[ u , v \mathclose] \big)=\mathopen[ \alpha , \beta \mathclose]\). D'abord \( f\big( \mathopen[ u , v \mathclose] \big)\) est un intervalle compact\footnote{Corolaire \ref{CorImInterInter} et théorème~\ref{ThoImCompCotComp}.} contenant \( f(u)=\alpha\) et \( f(v)=\beta\). Par conséquent \( \mathopen[ \alpha , \beta \mathclose]\subset f\big( \mathopen[ u , v \mathclose] \big)\). Pour l'inclusion inverse supposons \( t\in \mathopen[ u , v \mathclose]\) tel que \( f(t)>\beta\). Vu que \( f(a)=\alpha\) et \( \alpha<\beta\) le théorème des valeurs intermédiaires il existe \( t_0\in \mathopen[ a , t \mathclose]\) tel que \( f(t_0)=\beta\). Cela donne \( t_0<v\) et donc contredit la minimalité de \( v\) dans \( B\). Nous en déduisons que \( f\big( \mathopen[ u , v \mathclose] \big)\) ne contient aucun élément plus grand que \( \beta\). Même jeu pour montrer que ça ne contient aucun élément plus petit que \( \alpha\).

    En définitive, le segment \( L=\mathopen[ u , v \mathclose]\) fonctionne.
\end{proof}

Lorsque \( I_2\subset f(I_1)\) nous notons \( I_1\to I_2\) ou, si une ambiguïté est à craindre, \( I_1\stackrel{f}{\longrightarrow}I_2\). Cette flèche se lit «recouvre».
\begin{lemma}[\cite{PAXrsMn,TMCHooOaTrJL}]      \label{LemSSPXooMkwzjb}
    Soient les segments \( I_0,\ldots, I_{n-1}\) tels que nous ayons le cycle
    \begin{equation}
        I_0\to I_1\to\ldots\to I_{n-1}\to I_0.
    \end{equation}
    Alors \( f^n\) admet un point fixe \( x_0\in I_0\) tel que \( f^k(x_0)\in I_k\) pour tout \( k=0,\ldots, n-1\).
\end{lemma}

\begin{proof}
    Nous prouvons les cas \( n=1\) et \( n=2\) séparément.
    \begin{subproof}
    \item[\( n=1\)]
        Nous avons \( I_0\to I_0\), c'est-à-dire que $I_0\subset f(I_0)$. Si \( I_0=\mathopen[ a , b \mathclose]\) alors nous posons \( a=f(\alpha)\) et \( b=f(\beta)\) pour certains \( \alpha,\beta\in I_0\). Nous posons ensuite \( g(x)=f(x)-x\).

        Dans un premier temps, \( g(\alpha)=a-\alpha\leq 0\) parce que \( a=\in(I_0)\) et \( \alpha\in I_0\). Pour la même raison, \( g(\beta)=b-\beta\geq 0\). Le théorème des valeurs intermédiaires donne alors \( t_0\in \mathopen[ \alpha , \beta \mathclose]\subset I_0\) tel que \( g(t_0)=0\). Nous avons donc \( f(t_0)=t_0\).
    \item[\( n=2\)]
        Nous avons \( I_0\to I_1\to I_0\). Vu que \( I_1\subset f(I_0)\), le lemme~\ref{LemAONBooGZBuYt} donne un segment \( J_1\subset I_0\) tel que \( f(J_1)=I_1\). Mézalors
        \begin{equation}
            J_1\subset I_0\subset f(I_1)=f^2(J_1).
        \end{equation}
        Nous avons donc \( J_1\stackrel{f^2}{\longrightarrow}J_1\) et par le cas \( n=1\) traité plus haut, la fonction \( f^2 \) a un point fixe \( x_0\) dans \( J_1\). De plus
        \begin{equation}
            f(x_0)\in f(J_1)=I_1,
        \end{equation}
        le point \( x_0\) est donc bien celui que nous cherchions.
    \item
        Cas général. Nous avons
        \begin{equation}
            I_0\to I_1\to\ldots\to I_{n-1}\to I_0.
        \end{equation}
        Vu que \( I_1\subset f(I_0)\), il existe \( J_1\subset I_0\) tel que \( f(J_1)=I_1\). Mais
        \begin{equation}
            I_2\subset f(I_1)=f^2(J_1),
        \end{equation}
        donc il existe \( J_2\subset J_1\) tel que \( I_2=f^2(J_2)\). En procédant encore longtemps ainsi nous construisons les ensembles \( J_1,\ldots, J_{n-1}\) tels que
        \begin{equation}
            J_{n-1}\subset J_{n-2}\subset\ldots\subset J_1\subset J_0
        \end{equation}
        tels que \( I_k=f^k(J_k)\) pour tout \( k=1,\ldots, n-1\). La dernière de ces inclusions est \( I_{n-1}=f^{n-1}(J_{n-1})\), mais \( I_{n-1}\to I_0\), c'est-à-dire que
        \begin{equation}
            I_0\subset f(I_{n-1})=f^n(J_{n-1}),
        \end{equation}
        et il existe \( J_n\subset J_{n-1}\) tel que \( I_0\subset f^n(J_n)\). Mais comme \( J_n\subset J_0\) nous avons en particulier \( J_n\subset f^n(J_n)\).

        Cela donne un point fixe \( x_0\in J_n\) pour \( f^n\). Par construction nous avons \( J_n\subset J_{n-1}\subset\ldots\subset J_1\subset J_0\) et donc \( x_0\in J_k\) pour tout \( k\). En  particulier
        \begin{equation}
            f^k(x_0)\in f^k(J_k)=I_k
        \end{equation}
        pour tout \( k\).
    \end{subproof}
\end{proof}

\begin{theorem}[Théorème de Sarkowski\cite{PAXrsMn,TMCHooOaTrJL}]
    Soit \( I\), un segment de \( \eR\) et une application continue \( f\colon I\to I\). Si \( f\) admet un point \( 3\)-périodique, alors \( f\) admet des points \( n\)-périodiques pour tout \( n\geq 1\).
\end{theorem}

\begin{proof}
    Soit \( a\in I\) un point \( 3\)-périodique pour \( f\) et notons \( b=f(a)\), \( c=f(b)\). Les points \( b\) et \( c\) sont également des points \( 3\)-périodiques. Quitte à renommer, nous pouvons supposer que \( a\) est le plus petit des trois. Il reste deux possibilités : \( a<b<c\) et \( a<c<b\). Nous traitons d'abord le premier cas.

    Supposons \( a<b<c\). Nous posons \( I_0=\mathopen[ a , b \mathclose]\) et \( I_1=\mathopen[ b , c \mathclose]\). Nous avons immédiatement \( I_1\subset f(I_0)\) et comme \( f(b)=c\) et \( f(c)=a\), \( f(I_1)\) recouvre \( \mathopen[ a , c \mathclose]\) et donc recouvre en même temps \( I_1\) et \( I_2\). Nous avons donc \( I_0\to I_1\), \( I_1\to I_0\) et \( I_1\to I_1\).
    \begin{subproof}
    \item[Un point \( 1\)-périodique]
        Nous avons \( I_1\to I_1\) qui prouve que \( f\) a un point fixe dans \( I_1\). C'est le cas \( n=1\) du lemme~\ref{LemSSPXooMkwzjb}. Voilà un point \( 1\)-périodique.
    \item[Un point \( 2\)-périodique]
        Nous avons \( I_0\to I_1\to I_0\). Par conséquent, le lemme~\ref{LemSSPXooMkwzjb} dit que \( f^2\) a un point fixe \( x_0\in I_0\) tel que \( f(x_0)\in I_1\). Montrons que \( f(x_0)\neq x_0\). Pour avoir \( x_0=f(x_0)\), il faudrait \( x_0\in I_0\cap I_1=\{ b \}\). Mais \( b\) est un point \( 3\)-périodique, donc ne vérifiant certainement pas \( f^2(b)=b\). Nous en déduisons que \( f(x_0)\neq x_0\) et donc que \( x_0\) est \( 2\)-périodique.
    \item[Un point \( 3\)-périodique]
        On en a par hypothèse.
    \item[Un point \( n\)-périodique pour \( n\geq 4\)]
        Nous avons le cyle
        \begin{equation}
            I_0\to \underbrace{I_1\to I_1\to\ldots\to I_1}_{\text{n-1} fois}\to I_0.
        \end{equation}
        Le lemme donne alors un point fixe \( x\in I_0\) pour \( f^n\) tel que \( f^k(x)\in I_1\) pour \( k=1,\ldots, n-1\). Est-ce possible que \( x=b\) ? Non parce que \( f^2(b)=a\in I_0\) alors que \( f^2(x)\in I_1\). Mais \( I_0\cap I_1=\{ b \}\).

        Par conséquent la relation \( f^k(x)\in I_1\) exclu d'avoir \( f^k(x)=x\), et le point \( x\) est bien \( n\)-périodique.
    \end{subproof}

    Passons au cas \( a<c<b\). Alors nous posons \( I_0=\mathopen[ a , c \mathclose]\) et \( I_1=\mathopen[ c , b \mathclose]\). Encore une fois \( f(I_0)\) contient \( a\) et \( b\), donc \( I_0\to I_0\) et \( I_0\to I_1\). Mais en même temps \( f(I_1)\) contient \( a\) et \( c\), donc \( I_1\to I_0\).

    Nous pouvons donc refaire comme dans le premier cas, en inversant les rôles de \( I_0\) et \( I_1\). En particulier nous pouvons considérer le cycle
    \begin{equation}
        I_1\to I_0\to I_0\to\ldots\to I_0\to I_1.
    \end{equation}
\end{proof}

%+++++++++++++++++++++++++++++++++++++++++++++++++++++++++++++++++++++++++++++++++++++++++++++++++++++++++++++++++++++++++++
\section{Uniforme continuité}		\label{SecUnifContinue}
%+++++++++++++++++++++++++++++++++++++++++++++++++++++++++++++++++++++++++++++++++++++++++++++++++++++++++++++++++++++++++++

\begin{definition}
	Une partie $A\subset\eR^m$ est dite \defe{bornée}{bornée!partie de $\eR^m$} s'il existe un $M>0$ tel que $A\subset B(0,M)$. Le \defe{diamètre}{diamètre} de la partie $A$ est\nomenclature[T]{$\Diam(A)$}{Diamètre de la partie $A$} le nombre
	\begin{equation}
		\Diam(A)=\sup_{x,y\in A}\| x-y \|\in\mathopen[ 0 , \infty \mathclose].
	\end{equation}
\end{definition}
Lorsque $A$ est borné, il existe un $M$ tel que $\| x \|\leq M$ pour tout $x\in A$.

\begin{lemma}
	Si $A$ est une partie non vide de $\eR^m$, alors $\Diam(A)=\Diam(\bar A)$.
\end{lemma}
Nous n'allons pas donner de démonstrations de ce lemme.


Si $(x_n)$ est une suite et $I$ est un sous-ensemble infini de $\eN$, nous désignons par $x_I$ la suite des éléments $x_n$ tels que $n\in I$. Par exemple la suite $x_{\eN}$ est la suite elle-même, la suite $x_{2\eN}$ est la suite obtenue en ne prenant que les éléments d'indice pair.

Les suites $x_I$ ainsi construites sont dites des \defe{sous-suites}{sous-suite} de la suite $(x_n)$.


Pour une fonction $f\colon D\subset\eR^m\to \eR$, la continuité au point $a$ signifie que pour tout $\varepsilon>0$,
\begin{equation}
	\exists\delta>0\tq 0<\| x-a \|<\delta\Rightarrow | f(x)-f(a) |<\varepsilon.
\end{equation}
Le $\delta$ qu'il faut choisir dépend évidemment de $\varepsilon$, mais il dépend en général aussi du point $a$ où l'on veut tester la continuité. C'est-à-dire que, étant donné un $\varepsilon>0$, nous pouvons trouver un $\delta$ qui fonctionne pour certains points, mais qui ne fonctionne pas pour d'autres points.

Il peut cependant également arriver qu'un même $\delta$ fonctionne pour tous les points du domaine. Dans ce cas, nous disons que la fonction est uniformément continue sur le domaine.

\begin{definition}
	Une fonction $f\colon D\subset\eR^m\to \eR$ est dite \defe{uniformément continue}{continue!uniformément} sur $D$ si
	\begin{equation}	\label{EqConditionUnifCont}
		\forall\varepsilon>0,\,\exists\delta>0\tq\,\forall x,y\in D,\,\| x-y \|\leq\delta \Rightarrow| f(x)-f(a) |<\varepsilon.
	\end{equation}
\end{definition}

Il est intéressant de voir ce que signifie le fait de \emph{ne pas} être uniformément continue sur un domaine $D$. Il s'agit essentiellement de retourner tous les quantificateurs de la condition \eqref{EqConditionUnifCont} :
\begin{equation}	\label{EqConditionPasUnifCont}
	\exists\varepsilon>0\tq\forall\delta>0,\,\exists x,y\in D\tq \| x-y \|<\delta\text{ et }\big| f(x)-f(y) \big|>\varepsilon.
\end{equation}
Dans cette condition, les points $x$ et $y$ peuvent être fonction du $\delta$. L'important est que pour tout $\delta$, on puisse trouver deux points $\delta$-proches dont les images par $f$ ne soient pas $\varepsilon$-proches.

\begin{example}
	Prenons la fonction $f(x)=\frac{1}{ x }$, et demandons nous pour quel $\delta$ nous sommes sûr d'avoir
	\begin{equation}
		| f(a+\delta)-f(a) |=\left| \frac{1}{ a+\delta }-\frac{1}{ a } \right| <\varepsilon.
	\end{equation}
	Pour simplifier, nous supposons que $a>0$. Nous calculons
	\begin{equation}
		\begin{aligned}[]
			\frac{ 1 }{ a }-\frac{1}{ a+\delta }&<	\varepsilon\\
			\frac{ \delta }{ a(a+\delta) }&<\varepsilon\\
			\delta&<\varepsilon a^2+\varepsilon a\delta\\
			\delta(1-\varepsilon a)&<\varepsilon a^2\\
			\delta&<\frac{ \varepsilon a^2 }{ 1-\varepsilon a }.
		\end{aligned}
	\end{equation}
	Notons que, à $\varepsilon$ fixé, plus $a$ est petit, plus il faut choisir $\delta$ petit. La fonction $x\mapsto\frac{1}{ x }$ n'est donc pas uniformément continue. Cela correspond au fait que, proche de zéro, la fonction monte très vite. Une fonction uniformément continue sera une fonction qui ne montera jamais très vite.
\end{example}

\begin{proposition}
	Quelques propriétés des fonctions uniformément continues.
	\begin{enumerate}
		\item
			Toute application uniformément continue est continue;
		\item
			la composée de deux fonctions uniformément continues est uniformément continue;
	\end{enumerate}
\end{proposition}
Nous verrons qu'une application lipschitzienne est uniformément continue (proposition~\ref{PROPooVZSAooUneOQK}).

Une fonction peut être uniformément continue sur un domaine et pas sur un autre. Le théorème suivant donne une importante indication à ce sujet.
\begin{theorem}[Heine]\index{théorème!Heine}\index{Heine (théorème)}		\label{ThoHeineContinueCompact}
    Soit \( K\) un compact de \( \eR^n\). Une fonction continue \( f\colon \eR^n\to \eR^m\) est uniformément continue sur \( K\).
\end{theorem}

La démonstration qui suit est valable pour une fonction \( f\colon \eR^n\to \eR^m\) et utilise le fait que le produit cartésien de compacts est compact. Dans le cas de fonctions sur \( \eR\), nous pouvons modifier la démonstration pour ne pas utiliser ce résultat; voir plus bas.
%TODO : trouver où se trouve la preuve du produit de compacts et la référentier ici.
\begin{proof}
	Nous allons prouver ce théorème par l'absurde. Nous commençons par écrire la condition \eqref{EqConditionPasUnifCont} qui exprime que $f$ n'est pas uniformément continue sur le compact \( K\) :
	\begin{equation}
		\exists\varepsilon>0\tq\forall\delta>0,\,\exists x,y\in K\tqs \| x-y \|<\delta\text{ et }\big| f(x)-f(y) \big|>\varepsilon.
	\end{equation}
	En particulier (en prenant $\delta=\frac{1}{ n }$ pour tout $n$), pour chaque $n$ nous pouvons trouver $x_n$ et $y_n$ dans $K$ qui vérifient simultanément les deux conditions suivantes :
	\begin{subequations}
		\begin{numcases}{}
			\| x_n-y_n \|<\frac{1}{ n }\\
			\big| f(x_n)-f(y_n) \big|>\varepsilon.	\label{EqCond3107fxfyepsppt}
		\end{numcases}
	\end{subequations}
    Nous insistons que c'est le même $\varepsilon$ pour chaque $n$. L'ensemble $K$ étant compact, l'ensemble \( K\times K \) est compact (théorème~\ref{THOIYmxXuu}) et nous pouvons trouver une sous-suite convergente \emph{du couple} \( (x_n,y_n)\) dans \( K\times K\). Quitte à passer à ces sous-suites, nous  nous supposons que \( (x_n,y_n)\) converge dans \( K\times K\) et en particulier que les suites $(x_n)$ et $(y_n)$ sont convergentes. Étant donné que pour chaque $n$ elles vérifient $\| x_n-y_n \|<\frac{1}{ n }$, les limites sont égales :
	\begin{equation}
		\lim x_n=\lim y_n=x.
	\end{equation}
	L'ensemble $K$ étant fermé, la limite $x$ est dans $K$. Par continuité de $f$, nous avons finalement
	\begin{equation}
		\lim f(x_n)=\lim f(y_n)=f(x),
	\end{equation}
	mais alors
	\begin{equation}
		\lim_{n\to\infty}\big| f(x_n)-f(y_n) \big|=0,
	\end{equation}
	ce qui est en contradiction avec le choix \eqref{EqCond3107fxfyepsppt}.

	Tout ceci prouve que $f(K)$ est bornée supérieurement et que $f$ atteint son supremum (qui est donc un maximum). Le fait que $f(K)$ soit borné inférieurement se prouve en considérant la fonction $-f$ au lieu de $f$.

\end{proof}

\begin{remark}
    Nous pouvons ne pas utiliser le fait que le produit de compacts est compact. Cela est particulièrement commode lorsqu'on considère des fonctions de \( \eR\) dans \( \eR\) parce que dans ce cadre nous ne pouvons pas supposer connue la notion de produit d'espace topologiques.

    Pour choisir les sous-suites \( (x_n)\) et \( (y_n)\), il suffit de prendre une sous-suite convergente de \( (x_n)\) et d'invoquer le fait que \( \| x_n-y_n \|\leq \frac{1}{ n }\). Les suites \( (x_n)\) et \( (y_n)\) étant adjacentes\footnote{Définition \ref{DEFooDMZLooDtNPmu}.}, la convergence de \( (x_n)\) implique la convergence de \( (y_n)\) vers la même limite.

    Il est donc un peu superflus de parler de la convergence du couple \( (x_n,y_n)\).
\end{remark}

\begin{proposition}[Heine\cite{ooNDDIooKLdIWH}]     \label{PROPooBWUFooYhMlDp}
    Toute application continue d'un espace métrique compact dans un espace métrique quelconque est uniformément continue.
\end{proposition}

%+++++++++++++++++++++++++++++++++++++++++++++++++++++++++++++++++++++++++++++++++++++++++++++++++++++++++++++++++++++++++++
\section{Fonctions sur un compact}
%+++++++++++++++++++++++++++++++++++++++++++++++++++++++++++++++++++++++++++++++++++++++++++++++++++++++++++++++++++++++++++

Par le théorème des valeurs intermédiaires \ref{ThoValInter}, l'image d'un intervalle par une fonction continue est un intervalle, et nous avons l'importante propriété suivante des fonctions continues sur un compact.

Le théorème suivant est un cas particulier du théorème~\ref{ThoMKKooAbHaro}.
\begin{theorem}
    Si $f$ est une fonction continue sur l'intervalle compact $[a,b]$. Alors $f$ est bornée sur $[a,b]$ et elle atteint ses bornes.
\end{theorem}

\begin{proof}
    Étant donné que $[a,b]$ est un intervalle compact, son image est également un intervalle compact, et donc est de la forme $[m,M]$. Ceci découle du théorème~\ref{ThoImCompCotComp} et le corolaire~\ref{CorImInterInter}. Le maximum de $f$ sur $[a,b]$ est la borne $M$ qui est bien dans l'image (parce que $[m,M]$ est fermé). Idem pour le minimum $m$.
\end{proof}

%+++++++++++++++++++++++++++++++++++++++++++++++++++++++++++++++++++++++++++++++++++++++++++++++++++++++++++++++++++++++++++ 
\section{Polynômes, théorème de d'Alembert}
%+++++++++++++++++++++++++++++++++++++++++++++++++++++++++++++++++++++++++++++++++++++++++++++++++++++++++++++++++++++++++++

L'algèbre des polynômes sur un anneau est définie en \ref{DEFooFYZRooMikwEL}. Si \( P\in A[X]\) et si \( \alpha\in A\) nous avons également défini l'évaluation de \( P\) en \( \alpha\); c'est la définition \ref{DEFooNXKUooLrGeuh}. Dans le cadre de l'analyse, lorsque nous considérons des polynômes, nous allons complètement confondre le polynôme avec la fonction qu'il définit.

%--------------------------------------------------------------------------------------------------------------------------- 
\subsection{Polynômes sur les réels}
%---------------------------------------------------------------------------------------------------------------------------

\begin{proposition}     \label{PROPooJKYJooFqbQMr}
    Tout polynôme à coefficients réels de degré impair possède une racine réelle.
\end{proposition}

\begin{proof}
    Nous mettons le plus haut degré en facteur :
    \begin{equation}
        P(x)=\sum_{k=0}^na_kx^k=x^n\sum_{k=0}^n\frac{ a_k }{ x^{n-k} }.
    \end{equation}
    Le terme \( k=0\) vaut \( a_nx^n\) tandis que les autres sont de la forme (à coefficient près) \( \frac{1}{ x^l }\) pour un \( l\geq 1\). Lorsque \( x\to \infty\), chacun de ces termes s'annule (lemme \ref{LEMooFCIXooJuHFqk}). Nous avons donc
    \begin{equation}
        \lim_{x\to \infty} P(x)=+\infty,
    \end{equation}
    et de même, \( n\) étant impair, \( \lim_{x\to -\infty} P(x)=-\infty\). Le théorème des valeurs intermédiaires \ref{ThoValInter} nous donne alors l'existence d'un réel sur lequel \( P\) s'annule.
\end{proof}

%--------------------------------------------------------------------------------------------------------------------------- 
\subsection{Polynômes sur les complexes}
%---------------------------------------------------------------------------------------------------------------------------

Nous allons parler de comportement asymptotique de polynômes définis sur \( \eC\). La topologique que nous considérons est celle de la compactification en un point décrite en \ref{PROPooHNOZooPSzKIN}.

Le lemme suivant donne une caractérisation de la limite en l'infini dans le compactifié \( \hat \eC\). Dans beaucoup de cas, cette caractérisation est prise comme la définition de la limite. Hélas, dans le Frido nous sommes des extrémistes et nous ne parvenons pas à dire le mot «limite» si il n'y a pas une topologie.
\begin{lemma}[\cite{MonCerveau}]        \label{LEMooERABooQjLBzW}
    Nous considérons la compatification en un point d'Alexandrov\footnote{Définition \ref{PROPooHNOZooPSzKIN}.}. Soit une fonction \( f\colon \eC\to \eC\). Nous avons \( \lim_{z\to \infty} f(z)=\infty\) si et seulement si pour tout \( M>0\), il existe \( R>0\) tel que \( | z |>R\) implique \( | f(z) |>M\).
\end{lemma}

\begin{proof}
    Souvenons-nous que, en général\footnote{Définition \ref{DefYNVoWBx}.}, nous avons
    \begin{equation}
        \lim_{x\to a} f(x)=b
    \end{equation}
    si pour tout voisinage \( V\) de \( b\), il existe un voisinage \( W\) de \( a\) tel que \( z\in W\setminus\{ a \}\) implique \( f(z)\in V\).

    Précisons encore un point de notation. Si \( K\) est une partie de \( \eC\), nous notons \( K^c\) son complémentaire dans \( \eC\), pas dans \( \hat  \eC\).

    Ceci étant dit, nous passons à la preuve.
    \begin{subproof}
        \item[Sens direct]
            Nous supposons que \( \lim_{z\to \infty} f(z)=\infty\). Soit \( M>0\); nous considérons le voisinage \( V=\overline{ B(0,M) }^c\cup\{ \infty \}\). Par définition de la limite, il existe un voisinage \( W\) de \( \infty\) tel que \( z\in W\Rightarrow f(z)\in V\setminus\{ \infty \}=\overline{ B(0,M) }^c\). Ce voisinage est de la forme \( K^c\cup\{ \infty \}\). Vu que \( K\) est compact, il est borné et il existe \( R>0\) tel que \( K\subset B(0,R)\).

            Avec tout cela nous avons la chaine suivante d'implications :
            \begin{equation}
                | z |>R\Rightarrow z\in K^c\Rightarrow z\in W\Rightarrow f(z)\in V\setminus\{ \infty \}=\overline{ B(0,M) }^c\Rightarrow | f(z) |>M.
            \end{equation}
            C'est bien la propriété que nous voulions.
        \item[Sens réciproque]
            Soit un voisinage \( V\) de \( \infty\). Nous avons \( V=K^c\cup\{ \infty \}\) où \( K\) est compact dans \( \eC\). Il existe \( M>0\) tel que \( K\subset B(0,M)\).

            Par hypothèse, il existe \( R\) tel que \( | z |>R\Rightarrow | f(z) |>M\). Soit \( W=\overline{ B(0,R) }^c\cup\{ \infty \}\). Nous avons la chaine
            \begin{equation}
                z\in W\Rightarrow| z |>R\Rightarrow| f(z) |>M\Rightarrow f(z)\in K^c\Rightarrow f(z)\in V.
            \end{equation}
    \end{subproof}
\end{proof}

\begin{proposition}[\cite{MonCerveau}]     \label{PROPooPWVWooGuftxZ}
    Soit le polynôme
    \begin{equation}
        \begin{aligned}
            P\colon \eC&\to \eC \\
            z&\mapsto \sum_{i=0}^na_iz^i 
        \end{aligned}
    \end{equation}
    où nous sous-entendons que \( a_n\neq 0\). La fonction \( z\mapsto | P(z) |\) est équivalente\footnote{Définition \ref{DEFooWDSAooKXZsZY}.} en l'infini à la fonction
    \begin{equation}
        \begin{aligned}
            w\colon \eC&\to \eR^+ \\
            z&\mapsto | a_nz^n |. 
        \end{aligned}
    \end{equation}
\end{proposition}

\begin{proof}
    Nous voudrions prouver qu'il existe une fonction \( \alpha\colon \eC\to \eR\) telle que 
    \begin{subequations}     \label{EQooGXWZooDJZNzE}
        \begin{numcases}{}
        | \sum_{i=0}^na_iz^i |=\big( 1+\alpha(z) \big)| a_nz^n |.
         \lim_{z\to \infty} \alpha(z)=0.
        \end{numcases}
    \end{subequations}
    Nous trouvons un candidat pour être une telle fonction en isolant simplement \( \alpha(z)\) de cette égalité. Nous trouvons
    \begin{equation}
        \alpha(z)=\big| \sum_{i=0}^n\frac{ a_i }{ a_n }z^{i-n} \big|-1.
    \end{equation}
    Elle vérifie immédiatement \eqref{EQooGXWZooDJZNzE}. Le point qui fait intervenir la topologie de  est de vérifier que \( \lim_{z\to \infty} \alpha(z)=0\). Le terme \( i=0\) de la somme vaut \( 1\). Il suffit donc de montrer que pour \( i\neq 0\) nous avons
    \begin{equation}
        \lim_{z\to \infty} \frac{1}{ z^{n-i} }=0.
    \end{equation}
    Soit \( \epsilon>0\). Nous devons prouver qu'il existe un voisinage \( V\) de \( \infty\) dans \( \hat \eC\) tel que
    \begin{equation}
        | \frac{1}{ z^{n-i} }-0 |\leq \epsilon
    \end{equation}
    pour tout \( z\in V\).
    
    En utilisant la proposition \ref{PROPooXLARooYSDCsF} nous avons déjà
    \begin{equation}
        | \frac{1}{ z^{n-i} } |=\frac{1}{ | z^{n-i} | }=\frac{1}{ | z |^{n-i} }.
    \end{equation}
    Soit \( R>0\) tel que \( \frac{1}{ R }<\epsilon\). Nous considérons le voisinage \( \{ | z |>R \}\cup \{ \infty \}\) de \( \infty\). Dans ce voisinage, nous avons
    \begin{equation}
        \frac{1}{ | z |^{n-i} }\leq \frac{1}{ | z | }\leq \frac{1}{ R }<\epsilon.
    \end{equation}
    Et voila.
\end{proof}

Le lemme suivant parle de polynôme sur \( \eC\). Vous pouvez l'adapter à \( \hat \eR\) et \( \bar \eR\).
\begin{lemma}       \label{LEMooYZVGooXZvBAc}
    Si \( P\colon \eC\to \eC\) est un polynôme, alors \( | P |\) atteint une borne inférieure globale.
\end{lemma}

\begin{proof}
    Nous savons, par l'équivalence de fonctions prouvée dans la proposition \ref{PROPooPWVWooGuftxZ} que \( \lim_{z\to \infty} P(z)=\infty\). Soit \( a>0\) dans \( \eR\). Par le lemme \ref{LEMooERABooQjLBzW} il existe un \( R>a\) tel que \( | z |>R\Rightarrow | f(z) |>| f(a) |\).

    La fonction \( | P |\) est continue sur le compact \( \overline{ B(0,R) }\). Soit \( z_0\) le point de minimum\footnote{Théorème de Weierstrass \ref{ThoWeirstrassRn}.} de \( | P |\) sur \( \overline{ B(0,R) }\).

    Nous devons prouver que \( z_0\) donne même un minimum global. Vu que \( a\in\overline{ B(0,R) }\) nous avons
    \begin{equation}
        | f(z_0) |\leq | f(a) |.
    \end{equation}
    Si \( z\in \overline{ B(0,R) }^c\), nous avons
    \begin{equation}
        | f(z) |>| f(a) |\geq | f(z_0) |.
    \end{equation}
    Donc ce \( z_0\) est un minimum sur \( B(0,R)\) et sur \( \overline{ B(0,R) }^c\). Bref, un minimum global.
\end{proof}

\begin{lemma}       \label{LEMooTTOYooXaukuH}
    Soit le polynôme
    \begin{equation}
        \begin{aligned}
            P\colon \eC&\to \eC \\
            z&\mapsto \sum_{i=0}^na_iz^i. 
        \end{aligned}
    \end{equation}
    La fonction \( P\) est équivalente à \( a_0+a_1z\) en \( z=0\).
\end{lemma}

\begin{proof}
    En posant \( g(z)=a_0+a_1z\), nous devons trouver une fonction \( \alpha\) telle que
    \begin{equation}        \label{EQooZFJBooVAYVBv}
        P(z)=\big( 1+\alpha(z) \big)g(z).
    \end{equation}
    Si \( a_0\neq 0\), il existe un voisinage de \( z=0\) sur lequel la fonction
    \begin{equation}        \label{EQooVCOVooAKWJxF}
        \alpha(z)=\frac{ z^2\sum_{i=2}^na_iz^{i-2} }{ a_0+a_1z }
    \end{equation}
    existe. Il n'y a aucun problème à ce que \( \alpha(z)\to 0\) pour \( z\to 0\)\footnote{En remarquant toutefois que c'est une limite à deux dimensions. Sachez la définir.}, et un simple calcul\footnote{En fait, la formule \eqref{EQooVCOVooAKWJxF} est obtenue en isolant \( \alpha(z)\) dans \eqref{EQooZFJBooVAYVBv}.} donne \eqref{EQooVCOVooAKWJxF}.

    Si par contre \( a_0=0\), nous faisons le calcul intermédiaire suivant :
    \begin{equation}
        \alpha(z)g(z)=P(z)-g(z)=z^2\sum_{i=2}^na_iz^{i-2},
    \end{equation}
    et donc, en isolant \( \alpha(z)\) et en simplifiant par \( z\), nous voyons que la fonction \( \alpha\) définie par
    \begin{equation}
        \alpha(z)=\frac{z}{ a_1 }\sum_{i=2}^na_iz^{i-2}
    \end{equation}
    fonctionne.
\end{proof}

\begin{proposition}[\cite{ooRIPVooMlBiAH,MonCerveau}]       \label{PROPooLBBLooQwEiHr}
    Soient \( a,b\in \eR\).
    \begin{enumerate}
        \item       \label{ITEMooSPSWooKLtqzZ}
            L'équation \( z^2=a+bi\) a une solution dans \( \eC\).
        \item       \label{ITEMooQOJDooWjfGXv}
            Pour tout \( l\), l'équation \( z^{2^l}=a+bi\) a une solution dans \( \eC\).
    \end{enumerate}
    Nous ne disons pas que ces solutions sont uniques\footnote{Comme vous en conviendrez en pensant à \( z^2=1\) qui a déjà les solutions \( 1\) et \( -1\).}.
\end{proposition}

\begin{proof}
    Pour prouver \ref{ITEMooSPSWooKLtqzZ}, l'équation \( z^2=a+bi\) a pour solution \( \pm\xi\) où
        \begin{equation}
            \xi=\sqrt{ \frac{ 1 }{2}a+\frac{ 1 }{2}\sqrt{ a^2+b^2 } }+i\signe(b)\sqrt{ -\frac{ 1 }{2}a+\frac{ 1 }{2}\sqrt{ a^2+b^2 } }.
        \end{equation}
        Nous n'avons en fait pas besoin de montrer que \( \pm\xi\) sont toutes deux des solutions, ni que ce sont les seules. Un calcul direct montre que \( \xi^2=a+bi\) et nous sommes content.

    Pour \ref{ITEMooQOJDooWjfGXv}, nous faisons une récurrence sur \( l\). Nous savons que
        \begin{equation}
            z^{2^{k+1}}=(z^{2^k})^2.
        \end{equation}
        Soit \( \xi\in \eC\) tel que \( \xi^{2^k}=a+bi\); un tel \( \xi\) existe par hypothèse de récurrence. Alors si \( z\) est tel que \( z^2=\xi\), nous avons 
        \begin{equation}
            z^{2^{k+1}}=a+bi.
        \end{equation}
\end{proof}

Le théorème de d'Alembert possède de nombreuses démonstrations. En voici une qui à ma connaissance est celle demandant le moins d'analyse; une démonstration à base de théorie de Galois peut être trouvée dans \cite{rqrNyg,ooPSLMooAVODjn}. Si vous lisez ces lignes pour savoir qu'un polynôme de degré \( n\) possède au \emph{maximum} \( n\) racines, ce n'est pas ici qu'il faut regarder, mais le corolaire \ref{CORooUGJGooBofWLr}.
\begin{theorem}[d'Alembert\cite{ooRIPVooMlBiAH}]   \label{THOooIRJYooBiHRyW}
    Tout polynôme non constant à coefficients complexes admet au moins une racine complexe.
\end{theorem}

\begin{proof}
    Nous effectuons une preuve tout à la fois par l'absurde et par récurrence en supposant que le polynôme
    \begin{equation}
        \begin{aligned}
            f\colon \eC&\to \eC \\
            z&\mapsto z^n+a_1z^{n-1}+\ldots+a_n 
        \end{aligned}
    \end{equation}
    n'a pas de racines dans \( \eC\), et que \( n\) soit le plus petit entier pour lequel un tel polynôme existe. Nous notons
    \begin{equation}
        n=2^km
    \end{equation}
    où \( m\) est impair.

    Le lemme \ref{LEMooYZVGooXZvBAc} donne un point \( z_0\) qui réalise le minimum global de \( | f |\) sur $\eC$. Nous posons \( g(z)=f(z_0+z)\) et nous définissons ses coefficients \( A_i\) par
    \begin{equation}
        g(z)=\sum_{i=0}^nA_iz^i.
    \end{equation}
    Nous avons \( A_n=1\) et \( | A_0 |=| f(z_0) |\). Soit \( A_r\) le premier à être non nul parmi les \( A_1\), \( A_2\), \ldots.
    \begin{subproof}
        \item[Si \( r<n\)]
            Par hypothèse de récurrence, il existe \( \xi\in \eC\) tel que \( \xi^r=-A_1/A_r\). Nous avons
            \begin{equation}
                g(t\xi)=A_0+\frac{ -A_rt^rA_0 }{ A_r }+t^{r+1}\sum_{i=r+1}^nA_i\xi^it^{i-r-1}.
            \end{equation}
            En notant \( P(t)\) le dernier polynôme, nous pouvons écrire cela sous forme compacte :
            \begin{equation}
                g(t\xi)=A_0-t^rA_0+t^{r+1}P(t).
            \end{equation}
            Vu que
            \begin{equation}
                \lim_{t\to 0} \frac{ t^{r+1}P(t) }{ t^r| A_0 | }=\lim_{t\to 0} tP(t)=0,
            \end{equation}
            il existe \( t_0>0\) tel que
            \begin{equation}
                | t_0^{r+1}P(t_0) |<| A_0t_0r |.
            \end{equation}
            Nous choisissons de plus \( t_0<1\), de telle sorte que \( 1-t^r>0\). Avec cela nous avons
            \begin{equation}
                | g(t\xi) |\leq | A_0 |(1-t^r)+| t^{r+1}P(t) |=| A_0 |\underbrace{-t^r| A_0 |+| t^{r+1}P(t) |}_{<0}<| A_0 |.
            \end{equation}
            Or \( | A_0 |\) était un minimum global de \( | g |\). Contradiction.

        \item[Si \( r=n\)]

            Dans ce cas,
            \begin{equation}
                g(z)=f(z_0+z)=A_0+z^n,
            \end{equation}
            et nous rappelons que \( n=2^km\) où \( m\) est impair. Nous allons trouver une contradiction dans les quatre cas \( \real{A_0}>0\), \( \real(A_0)<0\), \( \imag(A_0)>0\) et \( \imag(A_0)<0\). Bien entendu ces cas se recouvrent largement, mais en toute généralité, nous avons besoin des quatre.
            \begin{subproof}
                \item[Si \( \real(A_0)>0\)]
                    La proposition \ref{PROPooLBBLooQwEiHr} nous permet de considérer \( v\in \eC\) tel que \( v^{2^k}=-1\). Nous avons alors
                    \begin{equation}
                        g(tv)=A_0+(tv)^n=A_0+t^n(v^{2^k})^m=A_0+t^n(-1)^m=A_0-t^n
                    \end{equation}
                    parce que \( m\) est impair. Nous avons \( \imag\big( g(tv) \big)=\imag(A_0)\). Si \( t\) est assez petit pour que \( t^n<| \real(A_0) |\) nous avons aussi \( |\real\big( g(tv) \big)|<| \real(A_0) |\). Donc
                    \begin{equation}
                        | g(tv) |^2=| \real\big( g(tv) \big) |^2+| \real\big( g(tv) \big) |^2<| \real(A_0) |^2+| \imag(A_0) |^2=| A_0 |^2.
                    \end{equation}
                    Donc \( | g(tv) |<| A_0 |\). Contradiction.
                \item[Si \( \real(A_0)<0\)]
                    Nous prenons \( v=1\), et même histoire.
                \item[Si \( \imag(A_0)<0\)]
                    Nous prenons \( w\in \eC\) tel que
                    \begin{equation}
                        w^{2^k}=i(-1)^{\frac{ 1 }{2}(m-1)}.
                    \end{equation}
                    Là, il y a un peu d'arrachage de cheveux pour bien voir les cas. La difficulté est que les puissances de \( i\) alternent entre \( 1\), \( -1\), \( i\) et \( -i\). Vu que \( m\) est impair, nous avons un \( l\) tel que \( m=2l+1\). Nous subdivisons les cas \( l\) pair et \( l\) impair.
                    \begin{subproof}
                        \item[Si \( l\) est pair]
                            Alors d'une part \( \frac{ 1 }{2}(m-1)=l\) est pair et donc 
                            \begin{equation}
                                (-1)^{\frac{ 1 }{2}(m-1)}=1.
                            \end{equation}
                            Et d'autre part, \( i^{2l+1}=(-1)^li=i\). En tout,
                            \begin{equation}
                                i^m(-1)^{\frac{ 1 }{2}(m-1)}=i.
                            \end{equation}
                        \item[Si \( l\) est impair]
                            Alors \( \frac{ 1 }{2}(m-1)=l\) et \( (-1)^{\frac{ 1 }{2}(m-1)}=-1\). Mais en même temps, \( i^{2l+1}=-i\), ce qui donne encore une fois
                            \begin{equation}
                                i^m(-1)^{\frac{ 1 }{2}(m-1)}=i.
                            \end{equation}
                    \end{subproof}
                    Bref, que \( l\) soit pair ou impair, nous avons \( i^m(-1)^{\frac{ 1 }{2}(m-1)}=i\).
            \end{subproof}
            Nous avons donc \( \real\big( g(tw) \big)=\real(A_0)\) et \( \imag\big( g(tw) \big)<\imag(A_0)\). Encore contradiction.
                \item[Si \( \imag(A_0)=0\)]
                    Même chose que ce que nous venons de faire, mais avec
                    \begin{equation}
                        w^{2^k}=-i(-1)^{\frac{ 1 }{2}(m-1)}.
                    \end{equation}
    \end{subproof}
\end{proof}

\begin{corollary}
    Le corps $\eC$ est algébriquement clos.
\end{corollary}

\begin{corollary}[\cite{MonCerveau}]       \label{CORooKKNWooWEQukb}
    Tout polynôme de degré \( 3\) à coefficients réels possède au moins une racine réelle.
\end{corollary}

\begin{proof}
    Soient les racines \( \lambda_1\), \( \lambda_2\) et \( \lambda_3\) du polynôme en question. Toutes trois sont dans \( \eC\). Supposons que \( \lambda_1\) ne soit pas réelle. Alors \( \lambda_2\) ou \( \lambda_3\) doit être égale à \( \bar\lambda_1\). Disons \( \lambda_2\). Nous avons donc les racines \( \lambda_1\), \( \bar\lambda_1\) et \( \lambda_3\). Le polynôme se factorise alors en
    \begin{equation}        \label{EQooELMMooNbpBgg}
        a(X-\lambda_1)(X-\bar\lambda_1)(X-\lambda_3).
    \end{equation}
    Le coefficient \( a\) doit être réel parce qu'il est le coefficient du terme en \( X^3\) (réel par hypothèse). Si \( \lambda_3\) n'est pas réel, alors ce polynôme ne peut pas avoir des coefficients réels. Entre autres parce que terme indépendant est \( a| \lambda_1 |^2\lambda_3\), qui est réel si et seulement si \( \lambda_3\) est réel\footnote{Notez l'utilisation du lemme~\ref{LEMooONLNooXLNbtB}.}.
\end{proof}
Tant que vous y êtes, vous pouvez voir que le polynôme \eqref{EQooELMMooNbpBgg} est à coefficient réels si et seulement si \( a\in \eR\) et \( \lambda_3\in \eR\).

\begin{example}     \label{EXooIPLOooSNfiWg}
    Toute application linéaire \( \eR^3\to \eR^3\) a un vecteur propre. En effet si \( R\colon \eR^3\to \eR^3\) est linéaire, son polynôme caractéristique \( \chi_R\) est de degré \( 3\). Le corolaire \ref{CORooKKNWooWEQukb} indique qu'un tel polynôme possède au moins une racine réelle.
    Une telle racine est une valeur propre de \( R\) par le théorème \ref{ThoWDGooQUGSTL}.
\end{example}

\begin{definition}
    Si \( \lambda\in\eK\) est une racine de \( \chi_u\), l'ordre de l'annulation est la \defe{multiplicité algébrique}{multiplicité!valeur propre!algébrique} de la valeur propre \( \lambda\) de \( u\). À ne pas confondre avec la \defe{multiplicité géométrique}{multiplicité!valeur propre!géométrique} qui sera la dimension de l'espace propre.
\end{definition}

\begin{proposition}
    Un polynôme irréductible à coefficients réels est soit de degré un soit de degré \( 2\) avec un discriminant négatif.
\end{proposition}

\begin{proof}
    Soit un polynôme \( P\) à coefficients réels de degré plus grand que \( 1\). Alors le théorème de d'Alembert-Gauss (théorème~\ref{THOooIRJYooBiHRyW}) implique l'existence d'une racine \( \alpha \in \eC \). Si $\alpha$ est un réel, $P$ est réductible. Si \( \alpha\) n'est pas réel, alors conjugué complexe \( \bar \alpha\) est également une racine. Par conséquent les polynômes \( (X-\alpha)\) et \( (X-\bar \alpha)\) divisent \( P\) dans \( \eC[X]. \).

    Ces deux polynômes sont premiers entre eux parce que
    \begin{equation}
        a(X-\alpha)+b(X-\bar \alpha)=0
    \end{equation}
    implique \( a=b=0\). Par conséquent le produit
    \begin{equation}
        X^2-(\alpha+\bar \alpha)X+\alpha\bar\alpha
    \end{equation}
    divise également \( P\). Ce dernier est un polynôme à coefficients réels de degré \( 2\). Donc tout polynôme de degré \( 3\) ou plus est réductible.
\end{proof}

\begin{proposition}     \label{PROPooLXGSooXmVcVG}
    Si \( E\) est un espace vectoriel sur \( \eC\), tout endomorphisme possède au moins une valeur propre.
\end{proposition}

\begin{proof}
    Soit un endomorphisme \( u\) sur \( E\). Le théorème \ref{ThoWDGooQUGSTL} dit que \( \lambda\in \eC\) est une valeur propre si et seulement si \( \lambda\) est une racine du polynôme caractéristique \( \chi_u\). Or ce polynôme possède au moins une racine dans \( \eC\) par le théorème de d'Alembert \ref{THOooIRJYooBiHRyW}.
\end{proof}

%+++++++++++++++++++++++++++++++++++++++++++++++++++++++++++++++++++++++++++++++++++++++++++++++++++++++++++++++++++++++++++ 
\section{Géométrie dans l'espace}
%+++++++++++++++++++++++++++++++++++++++++++++++++++++++++++++++++++++++++++++++++++++++++++++++++++++++++++++++++++++++++++

\begin{normaltext}
    Les notions de droites, plans et parallélisme sont des notions vectorielles qui auraient pu être traitées beaucoup plus haut. La chose qui rend la géométrie un peu piquante est la notion de perpendicularité. Cette notion demande un produit scalaire et fait intervenir ici et là des polynômes du second degré. Travailler avec le second degré demande la connaissance des racines carrés\footnote{Définition \ref{DEFooGQTYooORuvQb}.} et donc d'un peu de topologie réelle et de continuité.
\end{normaltext}<++>

\begin{definition}      \label{DEFooVTXWooVXfUnc}
    Soient deux espaces vectoriels \( E\) et \( V\). Une application \( f\colon E\to V\) est \defe{affine}{application affine} si il existe une application linéaire \( u\colon E \to V\) et un élément \( v\in V\) tel que
    \begin{equation}
        f(x)=u(x)+v
    \end{equation}
    pour tout \( x\in E\).
\end{definition}

\begin{definition}      \label{DEFooTQIFooKcloeY}
    Soit un espace vectoriel \( E\). 
    \begin{enumerate}
        \item
        Une \defe{droite vectorielle}{droite vectorielle} dans \( E\) est un sous-espace vectoriel de dimension \( 1\). 
    \item
        Une \defe{droite affine}{droite affine} est une partie de \( E\) de la forme \( a+V\) où \( a\in E\) et \( V\) est un sous-espace vectoriel de dimension \( 1\) de \( E\).
    \item
        Un \defe{plan vectoriel}{plan vectoriel} est un sous-espace vectoriel de dimension \( 2\).
    \item
        Une partie \( P\) est un \defe{plan affin}{plan affin} si il existe un \( v\in E\) tel que \( P-v\) est un plan vectoriel.
    \end{enumerate}
    Le plus souvent, nous parlerons de «droite» et «plan» sans préciser «vectoriel» ou «affine». Dans ces cas, le plus souvent, ce sera «affine».
\end{definition}

\begin{definition}[Perpendiculaires et parallèles]
    Deux notions importantes.
    \begin{enumerate}
        \item
            Nous disons que les droites \( a+V\) et \( b+W\) sont \defe{parallèles}{droites parallèles} lorsque \( V=W\).
        \item
            Nous disons que les droites \( a+V\) et \( b+W\) sont \defe{perpendiculaires}{droites perpendiculaires} si pour tout \( v\in V\) et \( w\in W\) nous avons \( v\cdot w=0\).
    \end{enumerate}
    Vous noterez que le parallélisme est une notion vectorielle alors que la perpendicularité dépend du produit scalaire; c'est une notion comme qui dirait «métrique».
\end{definition}

\begin{proposition}     \label{PROPooADJNooMyXUxG}
    Les propriétés usuelles.
    \begin{enumerate}
        \item
            Deux droites parallèles ayant une intersection sont confondues.
        \item
            Le parallélisme est une relation d'équivalence sur l'ensemble des droites de \( E\).
        \item
            Si la droite \( d_1\) est parallèle à la droite \( d_2\), alors une droite est perpendiculaire à \( d_1\) si et seulement si elle est perpendiculaire à \( d_2\).
    \end{enumerate}
\end{proposition}

\begin{lemma}       \label{LEMooRLFQooJADark}
    Deux droites perpendiculaires ont un unique point d'intersection.
\end{lemma}

\begin{proposition}     \label{PROPooPWNWooYuyrOc}
    Soient une droite \( d\) et un point \( p\).
    \begin{enumerate}
        \item
            Il existe une unique droite parallèle à \( d\) contenant \( p\).
        \item
            Il existe une unique droite perpendiculaire à \( d\) contenant \( p\).
    \end{enumerate}
\end{proposition}

\begin{lemma}       \label{LEMooQQFFooEZYeck}
    Si \( D\) est une droite et si \( a,b\in D\), alors \( D-a=D-b\) et \( D-a\) est une droite vectorielle.
\end{lemma}

\begin{proof}
    Vu que \( D\) est une droite, il existe \( v\in V\) tel que \( D-v\) soit une droite vectorielle que nous notons \( L\). Nous allons montrer que \( D-a=D-v\). Vu que \( a\) est arbitraire, cela suffit.

    \begin{subproof}
        \item[\( D-a\subset D-v\)]
            Un élément de \( D-a\) est de la forme \( x-a\) avec \( x\in D\). Nous écrivons \( x-a\) sous la forme \( y-v\) et nous espérons que \( y\in D\). Allons-y : d'abord nous isolons \( y\) dans \( x-a=y-v\) :
            \begin{subequations}
                \begin{align}
                    y=x-a+v=(x-v)-(a-v)+v.
                \end{align}
            \end{subequations}
            Vu que \( x-v\) et \( a-v\) sont des éléments de \( L\), la somme est dans \( L\) et donc \( y=l+v\) pour un certain élément de \( l\in L\). Nous avons donc prouvé que \( y\in D\) et donc que \( x-a=y-v\in D-v\).
        \item[\( D-v\subset D-a\)]
            Nous notons \( x-v\) un élément générique que \( D-v\) (\( x\in D\)). En posant \( y-a=x-v\), nous trouvons
            \begin{equation}
                y=x-v+a=\underbrace{x-v}_{\in L}+\underbrace{(a-v)}_{\in L}+v
            \end{equation}
            Donc \( y\in D\) et \( x-v=y-a\in D-a\).
    \end{subproof}
\end{proof}

\begin{proposition}     \label{PROPooNTHVooWWyafJ}
    L'image d'une droite par une application affine\footnote{Définition \ref{DEFooVTXWooVXfUnc}.} est une droite.
\end{proposition}

\begin{lemma}
    À propos de droites.
    \begin{enumerate}
        \item       \label{ITEMooYQCIooOrhRwj}
            Si \( L\) est une droite vectorielle, alors pour tout \( a\neq 0\) dans \( L\), nous avons \( L=\Image(f)\) où \( f\) est l'application linéaire donnée par
            \begin{equation}
                \begin{aligned}
                    f\colon \eK&\to V \\
                    \lambda&\mapsto \lambda a. 
                \end{aligned}
            \end{equation}
        \item       \label{ITEMooZIGMooGruFMP}
            Si \( D\) est une droite affine, alors pour tout \( a\neq b\) sur \( D\) nous avons \( D=\Image(f)\) où \( f\) est l'application affine donnée par
            \begin{equation}
                \begin{aligned}
                    g\colon \eK&\to V \\
                    \lambda&\mapsto a+\lambda(b-a). 
                \end{aligned}
            \end{equation}
    \end{enumerate}
\end{lemma}

\begin{proof}
    En deux parties.
    \begin{subproof}
        \item[Pour \ref{ITEMooYQCIooOrhRwj}]
            Vu que \( L\) est un sous-espace de dimension \( 1\), il possède une baser, disons \( \{ b \}\). En particulier \( a=\mu b\) pour un certain \( \mu\in \eK\). Si \( x\in L\) nous avons \( x=\lambda_x b\) pour un certain \( \lambda_x\), et donc
            \begin{equation}
                x=\frac{ \lambda_x }{ \mu }a.
            \end{equation}
            Donc \( x=f(\lambda_x/\mu)\). Cela prouve que \( L\subset\Image(f)\).

            L'inclusion inverse est simplement le fait que \( \lambda a\in L\) dès que \( a\in L\) parce que \( L\) est vectoriel.
        \item[Pour \ref{ITEMooZIGMooGruFMP}]
            Le lemme \ref{LEMooQQFFooEZYeck} nous indique qu'il existe une droite vectorielle \( L\) telle que \( D-x=L\) pour tout \( x\in D\).
            \begin{subproof}
                \item[\( D\subset\Image(g)\)]
                    Nous nommons \( f\colon \eK\to V\) l'application linéaire qui donne \( L\). Vu que \( b-a\in L\) nous avons
                    \begin{equation}
                        f(\lambda)=\lambda(b-a),
                    \end{equation}
                    et tout élément de \( L\) est de la forme \( f(\lambda)\). Nous avons aussi \( D=L+a\); donc un élément de \( D\) est de la forme \( f(\lambda)+a\) et donc de la forme \( \lambda(b-a)+a=g(\lambda)\).
                \item[\( \Image(g)\subset D\)]
                    Un élément de \( \Image(g)\) est de la forme \( a+\lambda(b-a)\) avec \( \lambda\in \eK\). Mais \( b-a\in L\), donc \( \lambda(b-a)\in L\) et 
                    \begin{equation}
                        g(\lambda)=a+\lambda(b-a)\in a+L=D.
                    \end{equation}
            \end{subproof}
    \end{subproof}
\end{proof}

\begin{example}
	Les exemples les plus courants d'applications affines sont les droites et les plans ne passant pas par l'origine.
	\begin{description}
		\item[Les droites] Une droite dans $\eR^2$ (ou $\eR^3$) qui ne passe pas par l'origine est l'image d'une fonction de la forme $s(t) =u t +v$, avec $t \in \eR$, et $u$ et $v$ dans $\eR^2$ ou $\eR^3$ selon le cas. 

		En choisissant des coordonnées adéquates, les droites peuvent être aussi vues comme graphes de fonctions affines. Dans le cas de $\eR^2$, on retrouve la fonction de l'exemple~\ref{ex_affine}, pour \( n = m = 1 \).

		\item[Les plans]
			De la même façon nous savons que tout plan qui ne passe pas par l'origine dans $\eR^3$ est le graphe d'une application affine, $P(x,y)= (a,b)^T\cdot(x,y)^T+(c,d)^T$, lorsque les coordonnées sont bien choisies.
	\end{description}
\end{example}

\begin{lemma}[Équation de droite\index{équation de droite}]       \label{LEMooYIHXooEwmlPo}
    Si \( D\) est une droite dans \( \eR^2\), alors \( D\) est d'une des deux formes suivantes :
    \begin{itemize}
        \item Soit il existe \( a\in \eR\) tel que
            \begin{equation}
                D=\{ (x,y)\in \eR^2\tq x=a \},
            \end{equation}
        \item soit il existe \( a,b\in \eR\) tels que
            \begin{equation}
                D=\{ (x,y)\in \eR^2\tq y=ax+b \}.
            \end{equation}
    \end{itemize}
    Le premier cas correspond aux droites verticales.
\end{lemma}

%---------------------------------------------------------------------------------------------------------------------------
\subsection{Projection orthogonale}
%---------------------------------------------------------------------------------------------------------------------------

Le théorème suivant n'est pas indispensablissime parce qu'il est le même que le théorème de la projection sur les espaces de Hilbert\footnote{Théorème~\ref{ThoProjOrthuzcYkz}}. Cependant la partie existence est plus simple en se limitant au cas de dimension finie.
\begin{theoremDef}[Théorème de la projection]  \label{ThoWKwosrH}
    Soit \( E\) un espace vectoriel réel ou complexe de dimension finie, \( x\in E\), et \( C\) un sous-ensemble fermé convexe de \(E\).
    \begin{enumerate}
        \item
            Les deux conditions suivantes sur \( y\in E\) sont équivalentes:
    \begin{enumerate}
        \item   \label{zzETsfYCSItemi}
            \( \| x-y \|=\inf\{ \| x-z \|\tq z\in C \}\),
        \item\label{zzETsfYCSItemii}
            pour tout \( z\in C\), \( \real\langle x-y, z-y\rangle \leq 0\).
    \end{enumerate}
\item
    Il existe un unique \( y\in E\), noté \( y=\pr_C(x)\) vérifiant ces conditions.
    \end{enumerate}
\end{theoremDef}

\begin{proof}
    Nous commençons par prouver l'existence et l'unicité d'un élément dans \( C\) vérifiant la première condition. Ensuite nous verrons l'équivalence.

    \begin{subproof}
        \item[Existence]

            Soit \( z_0\in C\) et \( r=\| x-z_0 \|\). La boule fermée \( \overline{ B(x,r) }\) est compacte\footnote{C'est ceci qui ne marche plus en dimension infinie.} et intersecte \( C\). Vu que \( C\) est fermé, l'ensemble \( C'=C\cap\overline{ B(x,r) }\) est compacte. Tous les points qui minimisent la distance entre \( x\) et \( C\) sont dans \( C'\); la fonction
            \begin{equation}
                \begin{aligned}
                     C'&\to \eR \\
                    z&\mapsto d(x,z)
                \end{aligned}
            \end{equation}
            est continue sur un compact et donc a un minimum qu'elle atteint\footnote{Théorème~\ref{ThoMKKooAbHaro}.}. Un point \( P\) réalisant ce minimum prouve l'existence d'un point vérifiant la première condition.

        \item[Unicité]
            Soient \( y_1\) et \( y_2\), deux éléments de \( C\) minimisant la distance avec \( x\), et soit \( d\) ce minimum. Nous avons par l'identité du parallélogramme \eqref{EqYCLtWfJ} que
            \begin{equation}
                \| y_1-y_2 \|^2=-4\left\| \frac{ y_1+y_2-x }{2} \right\|^2+2\| y_1-x \|^2+2\| y_2-x \|^2\leq -4d+2d+2d=0.
            \end{equation}
            Par conséquent \( y_1=y_2\).

        \item[\ref{zzETsfYCSItemi}\( \Rightarrow\)~\ref{zzETsfYCSItemii}]

            Soit \( z\in C\) et \( t\in \mathopen] 0 , 1 \mathclose[\); nous notons \( P=\pr_Cx\). Par convexité le point \( z=ty+(1-t)P\) est dans \( C\), et par conséquent,
                \begin{equation}
                    \| x-P \|^2\leq\| x-tz-(1-t)P \|^2=\| (x-P)-t(z-P) \|^2.
                \end{equation}
                Nous sommes dans un cas \( \| a \|^2\leq | a-b |^2\), qui implique \( 2\real\langle a, b\rangle \leq \| b \|^2\). Dans notre cas,
                \begin{equation}
                    2\real\langle x-P , t(z-P)\rangle \leq t^2\| z-P \|^2.
                \end{equation}
                En divisant par \( t\) et en faisant \( t\to 0\) nous trouvons l'inégalité demandée\footnote{Ici nous utilisons la proposition \ref{PROPooKPOXooEHIXJs}, et c'est une des choses qui font que cette partie sur la «géométrie élémentaire» demande en réalité d'être placée après déjà une partie de l'analyse réelle.} :
                \begin{equation}
                    2\real\langle x-P, z-P\rangle \leq 0.
                \end{equation}

        \item[\ref{zzETsfYCSItemii}\( \Rightarrow\)~\ref{zzETsfYCSItemi}]

            Soit un point \( P\in C\) vérifiant
            \begin{equation}
                \real\langle x-P, z-P\rangle \leq 0
            \end{equation}
            pour tout \( z\in C\). Alors en notant \( a=x-P\) et \( b=P-z\),
            \begin{equation}
                \begin{aligned}[]
                \| x-z \|^2=\| x-P+P-z \|^2&=\| a+b \|^2\\
                &=\| a \|^2+\| b \|^2+2\real\langle a, b\rangle \\
                &=\| a \|^2+\| b \|^2-2\real\langle x-P, z-P\rangle \\
                &\geq \| b \|^2,
                \end{aligned}
            \end{equation}
            ce qu'il fallait.
    \end{subproof}
\end{proof}

\begin{proposition}
    Soient une droite \( d\) dans \( \eR^3\) ainsi qu'un point \( p\). La projection\footnote{Définition \ref{ThoWKwosrH}.} \( \pr_d(p)\) est le point d'intersection\footnote{Lemme \ref{LEMooRLFQooJADark}.} entre \( d\) et la perpendiculaire à \( d\) passant par \( p\).
\end{proposition}

\begin{proof}
    Nous considérons la droite \( d=\{ a+\lambda v \}_{\lambda\in \eR}\) et un point \( p\in \eR^3\). Nous notons \( x(\lambda)=a+\lambda v\) le point courant dans \( d\). Conformément à la définition \ref{ThoWKwosrH} de la projection orthogonale, nous allons minimiser la distance \( \| p-x(\lambda) \|\) par rapport à \( \lambda\).

    Vu que \( \| p-x(\lambda) \|\) est toujours positif, nous pouvons chercher à minimiser le carré :
    \begin{equation}
        \| p- x(\lambda) \|=\| p \|^2-2p\cdot a-2\lambda p\cdot v+\| a \|^2+| \lambda |^2\| v \|^2+2\lambda a\cdot v.
    \end{equation}
    Quitte à minimiser ça par rapport à \( \lambda\), nous pouvons oublier les termes ne contenant pas \( \lambda\). Nous posons donc
    \begin{equation}
        f(\lambda)=2\lambda (a-p)\cdot v+ \lambda^2\| v \|^2.
    \end{equation}
    Vu que le coefficient de \( \lambda^2\) est positif, cette fonction aura un minimum (et non un maximum). Nous le cherchons avec la proposition \ref{PROPooNVKXooXtKkuz} qui nous demande de dériver :
    \begin{equation}
        f'(\lambda)=2(a-p)\cdot v+2\| v \|^2\lambda.
    \end{equation}
    Cela s'annule pour
    \begin{equation}
        \lambda_0=\frac{ (p-a)\cdot v }{ \| v \| }.
    \end{equation}
    Nous avons trouvé la valeur de \( \lambda\) pour laquelle
    \begin{equation}
        \pr_d(p)=x(\lambda_0).
    \end{equation}
    
    Nous devons voir maintenant que \( \big( p-x(\lambda_0) \big)\cdot v=0\). Il suffit d'un peu déballer :
    \begin{equation}
        \big( p-x(\lambda_0) \big)\cdot v=p\cdot v-a\cdot v-\frac{ (p-a)\cdot v }{ \| v^2 \| }\| v \|^2=0.
    \end{equation}
\end{proof}

%--------------------------------------------------------------------------------------------------------------------------- 
\subsection{Plan médiateur}
%---------------------------------------------------------------------------------------------------------------------------

\begin{proposition}[plan médiateur\cite{MonCerveau}]
    Soient un espace euclidien \( V\) ainsi que deux points distincts \( a,b\in V\). Nous avons
    \begin{equation}
        \{ x\in V\tq x-m\perp b-a \}=\{ x\in V\tq \| x-a \|=\| x-b \| \}.
    \end{equation}
    Dans le cas de \( V=\eR^3\), alors cet ensemble est un plan\footnote{Définition \ref{DEFooTQIFooKcloeY}.}.

    Ce plan est le \defe{plan médiateur}{plan médiateur} du segment \( [a,b]\).
\end{proposition}

\begin{proof}
    Nous notons 
    \begin{subequations}
        \begin{align}
    M&=\{ x\in V\tq x-m\perp b-a \},\\
    N&=\{ x\in V\tq \| x-a \|=\| x-b \| \}.
        \end{align}
    \end{subequations}
    \begin{subproof}
        \item[\( M\subset N\)]
            Soit \( x\in M\). Nous avons \( (x-m)\cdot (b-a)=0\), et nous pouvons utiliser Pythagore \ref{THOooHXHWooCpcDan} dans les triangles \( xbm\) et \( xma\):
            \begin{subequations}        \label{SUBEQSooVEPCooKnyPoq}
                \begin{align}
                    \| x-a \|^2&=\| x-m \|^2+\| a-m \|^2\\
                    \| x-b \|^2&=\| x-m \|^2+\| m-b \|^2.
                \end{align}
            \end{subequations}
            Vu que \( m\) est le milieu, nous avons \( a-m=m-b\) et donc \( \| a-m \|=\| m-b \|\). Nous voyons donc que les membres de droites des deux équations \eqref{SUBEQSooVEPCooKnyPoq} sont égaux. Donc \( \| x-a \|^2=\| x-b \|^2\). Comme une norme est toujours positive, les carrés peuvent être simplifiés : \( \| x-a \|=\| x-b \|\).

            Donc \( x\in N\).
        \item[\( N\subset M\)]
            klml
    \end{subproof}
    <++>
\end{proof}


%--------------------------------------------------------------------------------------------------------------------------- 
\subsection{Tétraèdre}
%---------------------------------------------------------------------------------------------------------------------------

\begin{definition}[\cite{MonCerveau}]   \label{DEFooMUUMooFVxKyb}
    Un \defe{tétraèdre régulier}{tétraèdre régulier} est un ensemble de \( 4\) points \( A\), \( B\), \( C\) et \( D\) de \( \eR^3\) deux à deux équidistants.

    Nous allons nommer \( \{ a_i \}\) les segments entre les points, \( \{ d_i \}\) les droites sur ces segments, et \( \{ s_i \}\) les sommets.
\end{definition}

\begin{lemma}
    Un tétraèdre régulier existe.
\end{lemma}

\begin{proof}
    Prenez un triangle équilatérale \( ABC\) dans le plan \( (.,.,0)\), et prenez ensuite un point \( D\) à la verticale du centre, placé à la bonne hauteur pour que les longueurs \( \| AD \|\), \( \| BD \|\) et \( \| CD \|\) soient égales à \( \| AB \|\).
\end{proof}

\begin{lemma}       \label{LEMooNWELooZeSEMN}
    Si \( T\) est un tétraèdre régulier, nous avons \( d_i\cap T=a_i\).
\end{lemma}

\begin{lemma}       \label{LEMooUSKVooQJiBuz}
    Les droites \( \{ d_i \}_{i=1,\ldots, 6}\) ne sont pas confondues ni parallèles.
\end{lemma}

\begin{proof}
    Si trois points \( A\), \( B\), \( C\) sont alignés, il n'est pas possible d'avoir \( \| AB \|=\| AC \|=\| BC \|\). Donc il n'y a pas deux droites parmi les \( \{ d_i \}\) qui sont confondues.

    Supposons que deux des droites \( AB\) et \( CD\) sont parallèles. En particulier, les points \( A\), \( B\), \( C\) et \( D\) sont dans un même plan : le plan \( A+\Span\{ B-A, C-A \}\). Il n'est pas possible d'avoir \( 4\) points dans un plan, tous équidistants deux à deux. 
    %Pas 4 points équidistants: TODOooVXINooTNdhAG
\end{proof}

Dans la suite, quand nous parlerons du «tétraèdre», nous parlerons de ses six points et six segments les joignant. L'ensemble \( T\subset \eR^3\) ne contient pas les surfaces et les volumes.

\begin{lemma}   \label{LEMooJCMKooOjMqtw}
    Soit un tétraèdre régulier \( T\). Un point de \( \eR^3\) est un sommet si et seulement si il est l'intersection de deux des droites \( \{ d_i \}\) différentes.
\end{lemma}

\begin{proof}
    En deux parties.
    \begin{subproof}
    \item[Sens direct]
        
    Par définition les sommets sont les points \( A \), \( B\), \( C\), \( D\); et les droites \( d_i\) sont les droites \( (AB)\), \( (AC)\), \( (AD)\), \( (BC)\), \( (DB)\) et \( (CD)\). Donc oui, les sommets sont à des intersections de ces droites.

    \item[Sens inverse]
    Soit un point \( X\in \eR^3\) à l'intersection entre deux des \( d_i\). Nous avons déjà vu dans le lemme \ref{LEMooUSKVooQJiBuz} que ces droites ne sont ni parallèles ni confondues. Donc elles ont au plus un point d'intesection. Voyons les couples possibles de droites.
    %Position relative de droites dans R^3: TODOooUKNZooPsYyDy

    On a une série de possibilités comme \( (AB)\cap(AC)\). Dans ce cas, l'intersection entre ces deux droits est \( A\) qui est un des sommets. Ensuite nous avons une série de possibilités comme \( (AB)\cap (CD)\). Ces deux droites n'ont pas d'intersection parce que si elles en avaient, les points \( A\), \( B\), \( C\) et \( D\) seraient dans le même plan, ce qui est impossible.
    % Deux droites de R^3 qui ont une intersection sont dans un même plan : TODOooFHYSooJZyuaf
    Donc deux droites \( d_i\) ont soit pas d'intersection soit une intersection qui est un sommet.
    \end{subproof}
\end{proof}
