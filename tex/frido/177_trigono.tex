% This is part of (everything) I know in mathematics
% Copyright (c) 2011-2020
%   Laurent Claessens
% See the file fdl-1.3.txt for copying conditions.

%+++++++++++++++++++++++++++++++++++++++++++++++++++++++++++++++++++++++++++++++++++++++++++++++++++++++++++++++++++++++++++ 
\section{Très modeste approximation de \texorpdfstring{$ \pi$}{pi}}
%+++++++++++++++++++++++++++++++++++++++++++++++++++++++++++++++++++++++++++++++++++++++++++++++++++++++++++++++++++++++++++

Nous sommes en droit de vouloir une valeur approchée de \( \pi\).
\begin{lemma}       \label{LEMooJWSGooExmtDA}
    Nous avons l'approximation numérique
    \begin{equation}
        2\sqrt{ 2 }<\pi<4.
    \end{equation}
\end{lemma}

\begin{proof}
    Grace au lemme~\ref{LEMooIGNPooPEctJy} nous savons que la fonction \( \sin\) passe de \( 0\) à \( \sqrt{ 2 }/2\) sur un intervalle de taille \( \pi/4\) avec une dérivé majorée par \( 1\). Par conséquent
    \begin{equation}
        \frac{ \pi }{ 4 }>\frac{ \sqrt{ 2 } }{2}
    \end{equation}
    et donc\footnote{Sérieusement, êtes vous capables de trouver une approximation de \( \sqrt{ 2 }\) en ne vous basant que sur des choses vues jusqu'ici ?}
    \begin{equation}
        \pi>2\sqrt{ 2 }\simeq 2.82
    \end{equation}
    De plus la fonction \( \sin\) passe de \( 0\) à \( \sqrt{ 2 }/2\) sur un intervalle de taille \( \pi/4\) avec une dérivée majorée par \( \sqrt{ 2 }/2\), donc
    \begin{equation}
        \frac{ \pi }{ 4 }<\frac{ \sqrt{ 2 }/2 }{ \sqrt{ 2 }/2 },
    \end{equation}
    ce qui donne
    \begin{equation}
        \pi<4.
    \end{equation}
\end{proof}

Pour avoir une meilleur approximation de \( \pi\), nous pouvons remarquer que \( \pi\in\mathopen] 2.82 , 4 \mathclose[\), et que cet intervalle est suffisamment petit pour ne pas recouvrir l'intervalle correspondant pour \( 2\pi\). L'équation \( \cos(x)=-1\) possède donc une unique solution dans cet intervalle (et cette solution est \( \pi\)). Nous pouvons donc faire une dichotomie pour trouver la valeur de \( \pi\), pourvu que nous ayons une façon d'évaluer des valeurs de \( \cos(x)\) de façon pas trop ridicule.

%+++++++++++++++++++++++++++++++++++++++++++++++++++++++++++++++++++++++++++++++++++++++++++++++++++++++++++++++++++++++++++
\section{Cercle trigonométriques}
%+++++++++++++++++++++++++++++++++++++++++++++++++++++++++++++++++++++++++++++++++++++++++++++++++++++++++++++++++++++++++++

\begin{proposition}[\cite{ooIEJXooIYpBbd}]      \label{PROPooWZFGooMVLtFz}
    Soient des fonctions \( f,g\colon I\to \eR\) de classe \(  C^{1}\) sur l'ouvert \( I\) de \( \eR\) telles que \( f^2+g^2=1\). Soit \( t_0\in I\) et \( \theta_0\) tel que \( f(t_0)=\cos(\theta_0)\) et \( g(t_0)=\sin(\theta_0)\).

    Alors il existe une unique fonction continue \( \theta\colon I\to \eR\) telle que
    \begin{subequations}
        \begin{numcases}{}
            \theta(t_0)=\theta_0\\
            f=\cos\circ \theta\\
            g=\sin\circ \theta.
        \end{numcases}
    \end{subequations}
\end{proposition}

\begin{proof}
    Nous commençons par l'existence, en passant par les nombres complexes. Soit \( h\colon I\to \eC\) définie par \( h=f+ig\). Nous avons \( h\bar h=1\) et nous définissons
    \begin{equation}
        \theta(t)=\theta_0-i\int_{t_0}^th'(s)\overline{ h(s) }ds.
    \end{equation}
    Cette intégrale existe pour tout \( t\) parce que les fonctions \( f\) et \( g\) étant de classe \(  C^{\infty}\), elles sont bornées sur le compact \( \mathopen[ t_0 , t  \mathclose]\). De plus \( \theta\) est une fonction continue parce que c'est une primitive (proposition~\ref{PropEZFRsMj})\footnote{En réalité nous appliquons la proposition~\ref{PropEQRooQXazLz} à chacune des parties réelles et imaginaires de la fonction $s\mapsto h'(s)\overline{ h(s) }$.}.

    La dérivée de \( \theta\) est la fonction \( s\mapsto -i h'(s)\overline{ h(s) }\).

    Utilisant la formule du lemme~\ref{LEMooHOYZooKQTsXW} sur la forme trigonométrique des nombres complexes, nous calculons :
    \begin{equation}
        \Dsdd{ h e^{-i\theta} }{t}{0}= e^{-i\theta}(h'-h\theta')= e^{-i\theta}(h'-ih(-i)h'\bar h)=0.
    \end{equation}
    Par conséquent il existe \( c\in \eC\) tel que \( h e^{-i\theta}=c\). Mais \( h(t_0)=f(t_0)+ig(t_0)=\cos(\theta_0)+i\sin(\theta_0)= e^{i\theta_0}\), du coup
    \begin{equation}
        h(t_0) e^{-i\theta(t_0)}=c
    \end{equation}
    donne immédiatement \( c=1\), ou encore \(  e^{i\theta(t)}=h(t)\), c'est-à-dire que
    \begin{equation}
        f+ig=\cos\circ\theta+i\sin\circ\theta,
    \end{equation}
    ce qu'il fallait pour l'existence.

    Pour l'unicité nous supposons avoir une autre fonction, \(\alpha\) qui satisfait aux exigences. Pour tout \( t\in I\) nous avons
    \begin{equation}
        e^{i\theta(t)}= e^{i\alpha(t)}.
    \end{equation}
    Il existe donc une fonction \( n\colon I\to \eN\) telle que \( \theta(t)=\alpha(t)+2n(t)\pi\). Par continuité de \( \theta\) et \( \alpha\), la fonction \( n\) doit être constante, mais vu que \( \theta(t_0)=\alpha(t_0)\) nous avons \( n=1\).
\end{proof}


%---------------------------------------------------------------------------------------------------------------------------
\subsection{Les fonctions tangente et arc tangente}
%---------------------------------------------------------------------------------------------------------------------------

\begin{definition}
    La fonction \defe{tangente}{tangente} est :
    \begin{equation}
        \tan(x)=\frac{ \sin(x) }{ \cos(x) }
    \end{equation}
    où \( \sin\) et \( \cos\) sont de la définition~\ref{PROPooZXPVooBjONka}.
\end{definition}
La fonction tangente n'est pas définie sur les points de la forme \( x=\frac{ \pi }{2}+k\pi\), \( k\in \eZ\). Une interprétation géométrique, qui justifie le nom, est donnée sur la figure~\ref{LabelFigTgCercleTrigono}.
\newcommand{\CaptionFigTgCercleTrigono}{Interprétation géométrique de la fonction tangente. La tangente de l'angle $\theta$ est positive (et un peu plus grande que $1$) tandis que celle de la tangente de l'angle $\varphi$ est négative.}
\input{auto/pictures_tex/Fig_TgCercleTrigono.pstricks}

\begin{proposition}
    La fonction
    \begin{equation}
        \begin{aligned}
        \tan\colon \mathopen] -\frac{ \pi }{ 2 } , \frac{ \pi }{2} \mathclose[&\to \eR \\
            x&\mapsto \tan(x)
        \end{aligned}
    \end{equation}
    est une bijection.
\end{proposition}

\begin{proof}
    Le cosinus ne s'annulant pas sur l'intervalle donné, la fonction est bien définie. Nous avons
    \begin{equation}
        \lim_{x\to \pi/2^-} \tan(x)=+\infty
    \end{equation}
    parce que la limite du sinus est \( 1\) est celle du cosinus est zéro par les valeurs positives. Le même raisonnement donne la limite en \( -\pi/2\) qui vaut \( -\infty\). Le théorème des valeurs intermédiaires\footnote{Théorème~\ref{ThoValInter}.} dit que la fonction tangente est alors surjective sur \( \eR\).

    Par ailleurs en utilisant les règles de calcul comme la dérivation du quotient~\ref{PROPooOUZOooEcYKxn}\ref{ITEMooMUNQooLiKffz} nous trouvons
    \begin{equation}
        \tan'(x)=\tan^2(x)+1,
    \end{equation}
    ce qui nous donne une dérivée partout strictement positive, et donc une fonction strictement croissante et donc injective.
\end{proof}

Le graphe de la fonction tangente est sur la figure~\ref{LabelFigPVJooJDyNAg}. % From file PVJooJDyNAg
\newcommand{\CaptionFigPVJooJDyNAg}{Le graphe de la fonction tangente.}
\input{auto/pictures_tex/Fig_PVJooJDyNAg.pstricks}

En ce qui concerne la bijection réciproque nous avons le théorème suivant.
\begin{theorem}     \label{THOooUSVGooOAnCvC}
    La fonction inverse de la tangente,
    \begin{equation}
        \begin{aligned}
        \arctan\colon \eR&\to \left] -\frac{ \pi }{2} , \frac{ \pi }{2} \right[ \\
            x&\mapsto \arctan(x)
        \end{aligned}
    \end{equation}
    nommée \defe{arc tangente}{arc tangente} est
    \begin{enumerate}
        \item
            impaire et strictement croissante sur \( \eR\).
        \item       \label{ITEMooMNHLooOVhIIb}
            dérivable sur \( \eR\) de dérivée
            \begin{equation}        \label{EQooGCHGooPlwYWt}
                \arctan'(x)=\frac{1}{ 1+x^2 }.
            \end{equation}
    \end{enumerate}
\end{theorem}

\begin{proof}
    Il est immédiatement visible sur son développement de définition \eqref{EQooCMRFooCTtpge} que la fonction sinus est impaire. Une vérification similaire montre que la fonction cosinus est paire. La fonction tangente est alors impaire et sa réciproque l'est tout autant.

    La fonction arc tangente est également dérivable (donc continue) par la proposition~\ref{PropMRBooXnnDLq} parce que la fonction tangente l'est. Notons qu'ici nous nous sommes restreint à \( \mathopen] -\pi/2 , \pi/2 \mathclose[\). Sinon, le résultat est faux.

    La formule proposée pour la dérivée provient également de la proposition~\ref{PropMRBooXnnDLq} et de la dérivée de la tangente :
\end{proof}

\begin{lemma}       \label{LEMooHRDCooGtnyeQ}
    Nous avons les limites
    \begin{enumerate}
        \item
            $\lim_{x\to \infty} \arctan(x)=\frac{ \pi }{2}$,
        \item
            \( \lim_{x\to -\infty} \arctan(x)=-\frac{ \pi }{2}\).
    \end{enumerate}
\end{lemma}

\begin{lemma}       \label{LEMooJKIUooEMMOrs}
    Nous avons la valeur remarquable
    \begin{equation}
        \arctan(1/\sqrt{ 3 })=\frac{ \pi }{ 6 }.
    \end{equation}
\end{lemma}

Le nombre \( \arctan(x_0)\) se calcule en cherchant l'angle \( \theta\in\mathopen[ -\frac{ \pi }{2} , \frac{ \pi }{2} \mathclose]\) dont la tangente vaut \( x_0\). Nous obtenons le tableau de valeurs suivant :

\begin{lemma}       \label{LEMooPQNCooDkEUyw}
    Quelques valeurs remarquables de l'arc tangente :
\begin{equation}
    \begin{array}[]{|c|c|c|c|c|}
        \hline
        x&0&\frac{1}{ \sqrt{3} }&1&\sqrt{3}\\
        \hline
        \arctan(x)&0&\frac{ \pi }{ 6 }&\frac{ \pi }{ 4 }&\frac{ \pi }{ 3 }\\
        \hline
    \end{array}
\end{equation}
\end{lemma}

En ce qui concerne la représentation graphique de la fonction \( x\mapsto\arctan(x)\), elle s'obtient «en retournant» la partie entre \( -\frac{ \pi }{2}\) et \( \frac{ \pi }{ 2 }\) du graphique de la fonction tangente :
\begin{center}
   \input{auto/pictures_tex/Fig_UQZooGFLNEq.pstricks}
\end{center}

%---------------------------------------------------------------------------------------------------------------------------
\subsection{La fonction arc sinus}
%---------------------------------------------------------------------------------------------------------------------------

Nous voulons étudier la fonction
\begin{equation}
    \begin{aligned}
        \sin\colon \eR&\to \mathopen[ -1 , 1 \mathclose] \\
        x&\mapsto \sin(x)
    \end{aligned}
\end{equation}
et sa réciproque éventuelle.

La fonction sinus est continue sur \( \eR\) mais n'est pas bijective : elle prend une infinité de fois chaque valeur de \( J=\mathopen[ -1 , 1 \mathclose]\). Pour définir une bijection réciproque de la fonction sinus en utilisant le théorème~\ref{ThoKBRooQKXThd}, nous devons donc choisir un intervalle à partir duquel la fonction sinus est monotone. Nous choisissons l'intervalle
\begin{equation}
    I=\mathopen[ -\frac{ \pi }{ 2 } , \frac{ \pi }{2} \mathclose].
\end{equation}
La fonction
\begin{equation}
    \begin{aligned}
        \sin\colon \mathopen[ -\frac{ \pi }{2} , \frac{ \pi }{2} \mathclose]&\to \mathopen[ -1 , 1 \mathclose] \\
        x&\mapsto \sin(x)
    \end{aligned}
\end{equation}
est une bijection croissante et continue. Nous avons donc le résultat suivant.
\begin{theorem}[Définition et propriétés de arc sinus]
    Nous nommons \defe{arc sinus}{arc sinus} la bijection inverse de la fonction \( \sin\colon I\to J\). La fonction
    \begin{equation}
        \begin{aligned}
            \arcsin\colon \mathopen[ -1 , 1 \mathclose]&\to \mathopen[ -\frac{ \pi }{2} , \frac{ \pi }{2} \mathclose] \\
            x&\mapsto \arcsin(x)
        \end{aligned}
    \end{equation}
    ainsi définie est
    \begin{enumerate}
        \item
            continue et strictement croissante;
        \item
            impaire : pour tout \( x\in\mathopen[ -1 , 1 \mathclose]\) nous avons \( \arcsin(-x)=-\arcsin(x)\).
    \end{enumerate}
\end{theorem}

\begin{proof}
    Nous prouvons le fait que \( \arcsin\) est impaire. Un élément de l'ensemble de définition de \( \arcsin\) est de la forme \( y=\sin(x)\) avec \( x\in\mathopen[ -\pi/2 , \pi/2 \mathclose]\). La relation \eqref{EqHQRooNmLYbF} s'écrit dans notre cas
    \begin{equation}    \label{EqVUWooUwVxVp}
        x=\arcsin\big( \sin(x) \big).
    \end{equation}
    Nous écrivons d'une part cette équation avec \( -x\) au lieu de \( x\) :
    \begin{equation}    \label{EqRLYooIwOvSz}
        -x=\arcsin\big( \sin(-x) \big)=\arcsin\big( -\sin(x) \big)=\arcsin(-y);
    \end{equation}
    et d'autre part nous multiplions \eqref{EqVUWooUwVxVp} par \( -1\) :
    \begin{equation}    \label{EqTGIooDeRYyT}
        -x=-\arcsin\big( \sin(x) \big)=-\arcsin(y).
    \end{equation}
    En égalisant les valeurs \eqref{EqRLYooIwOvSz} et \eqref{EqTGIooDeRYyT} nous trouvons
    \begin{equation}
        \arcsin(-y)=-\arcsin(y),
    \end{equation}
    ce qui signifie que \( \arcsin\) est une fonction impaire.
\end{proof}
Notons que cette preuve repose sur le fait que tout élément de l'ensemble de définition de la fonction arc sinus peut être écrit sous la forme \( \sin(x)\) pour un certain \( x\).

Si \( x_0\in\mathopen[ -1 , 1 \mathclose]\) est donné, calculer \( \arcsin(x_0)\) revient à trouver un angle \( \theta_0\) dans \( \mathopen[ -\frac{ \pi }{2} , \frac{ \pi }{2} \mathclose]\) pour lequel \( \sin(\theta_0)=x_0\). Un tel angle sera forcément unique.

\begin{remark}
  La définition de arc sinus découle du choix de l'intervalle $I$, qui est une convention. Il aurait été possible de faire un choix différent : pourriez-vous trouver la réciproque de la fonction sinus sur l'intervalle $[\pi/2, 3\pi/2]$ ? Le mieux est de l'écrire comme une translatée de arc sinus, en utilisant le fait que sinus est une fonction périodique.
\end{remark}

\begin{example}
    Pour calculer \( \arcsin(1)\), il faut chercher un angle entre \( -\frac{ \pi }{2}\) et \( \frac{ \pi }{ 2 }\) ayant \( 1\) pour sinus : résoudre \( \sin(\theta)=1\). La solution est \( \theta=\frac{ \pi }{2}\) et nous avons donc \( \arcsin(1)=\frac{ \pi }{2}\).
\end{example}

À l'aide des valeurs remarquables de la fonction sinus nous obtenons le tableau suivant de valeurs remarquables pour l'arc sinus.
\begin{equation*}
    \begin{array}[]{|c|c|c|c|c|c|}
        \hline
        x&0&\frac{ 1 }{2}&\frac{ \sqrt{2} }{2}&\frac{ \sqrt{3} }{2}&1\\
          \hline
          \arcsin(x)&0&\frac{ \pi }{ 6 }&\frac{ \pi }{ 4 }&\frac{ \pi }{ 3 }&\frac{ \pi }{ 2 }\\
          \hline
           \end{array}
\end{equation*}
Les autres valeurs remarquables peuvent être déduites du fait que l'arc sinus est une fonction impaire.

En ce qui concerne la dérivabilité de la fonction arc sinus, en application de la proposition~\ref{PropMRBooXnnDLq} elle est dérivable en tout \( y=\sin(x)\) tel que \( \sin'(x)\neq 0\), c'est-à-dire tel que \( \cos(x)\neq 0\). Or \( \cos(x)=0\) pour \( x=\pm\frac{ \pi }{2}\), ce qui correspond à \( y=\sin(\pm\frac{ \pi }{2})=\pm 1\). La fonction arc sinus est donc dérivable sur \( \mathopen] -1 , 1 \mathclose[\). Nous avons donc la propriété suivante pour la dérivabilité.

\begin{proposition}
    La fonction arc sinus est continue sur \( \mathopen[ -1 , 1 \mathclose]\) et dérivable sur \( \mathopen] -1 , 1 \mathclose[\). Pour tout \( y\in\mathopen] -1 , 1 \mathclose[\), la dérivée est donnée par la formule \eqref{EqWWAooBRFNsv}, qui dans ce cas s'écrit
        \begin{equation}
            \arcsin'(y)=\frac{1}{ \cos\big( \arcsin(y) \big) }=\frac{1}{ \sqrt{1-y^2} }.
        \end{equation}
\end{proposition}
La dernière égalité viens du fait que si $x=\arcsin(y)$ alors $y = \sin(x)$ et $\cos(x)= \sqrt{1-\sin^2(x)} = \sqrt{1-y^2}$.

Pour comprendre la dernière égalité, remarquer que dans le dessin suivant, \( \theta=\arcsin(y)\), donc $y = \sin(\theta)$, et \( x=\cos(\theta)\).
\begin{center}
    \input{auto/pictures_tex/Fig_BIFooDsvVHb.pstricks}
\end{center}

Notons enfin que le graphe de la fonction arc sinus est donné à la figure~\ref{LabelFigFGRooDhFkch}. % From file FGRooDhFkch
\newcommand{\CaptionFigFGRooDhFkch}{Le graphe de la fonction \( x\mapsto \arcsin(x)\)}
\input{auto/pictures_tex/Fig_FGRooDhFkch.pstricks}

%---------------------------------------------------------------------------------------------------------------------------
\subsection{La fonction arc cosinus}
%---------------------------------------------------------------------------------------------------------------------------

Nous voulons étudier la fonction
\begin{equation}
        \cos\colon \eR\to \mathopen[ -1 , 1 \mathclose]
\end{equation}
et son éventuelle réciproque. Encore une fois il n'est pas possible d'en prendre la réciproque globale parce que ce n'est pas une bijection; ne fut-ce que parce qu'elle est périodique (proposition~\ref{PROPooFRVCooKSgYUM}). Nous choisissons de considérer l'intervalle \( \mathopen[ 0 , \pi \mathclose]\) sur lequel la fonction cosinus est continue et strictement monotone décroissante.

Nous avons alors le résultat suivant :

\begin{propositionDef}     \label{PROPooZOZHooSMoYQD}
    Pour définir la fonction arcsinus.

    \begin{enumerate}
        \item
    La fonction
    \begin{equation}
            \cos\colon \mathopen[ 0 , \pi \mathclose]\to \mathopen[ -1 , 1 \mathclose]
    \end{equation}
    est une bijection continue strictement décroissante.
    \item
    Sa bijection réciproque est la fonction
    \begin{equation}
            \arccos\colon \mathopen[ -1 , 1 \mathclose]\to \mathopen[ 0 , \pi \mathclose] \\
    \end{equation}
    nommée \defe{arc cosinus}{arc cosinus}.
    \item
        La fonction arc cosinus est continue, strictement décroissante.
    \item
        Elle est dérivable et pour tout \( y\in\mathopen] -1 , 1 \mathclose[\), sa dérivée est donnée par
    \begin{equation}
        \arccos'(y)=\frac{1}{ -\sin\big( \arccos(y) \big) }=\frac{ -1 }{ \sqrt{1-y^2} }.
    \end{equation}
    \end{enumerate}
\end{propositionDef}

\begin{proof}
    La fonction cosinus est continue et même de classe \(  C^{\infty}\) par la proposition~\ref{PROPooZXPVooBjONka}. Elle est strictement décroissant parce que sa dérivée (\( -\sin\)) y est strictement positive (strictement à dans l'intérieur du domaine).

    Le fait que arc cosinus soit une bijection continue strictement monotone est dans le théorème de la bijection~\ref{ThoKBRooQKXThd}. La dérivabilité et la formule sont de la proposition~\ref{PropMRBooXnnDLq}.
\end{proof}

Pour \( y_0\in\mathopen[ -1 , 1 \mathclose]\), trouver la valeur de \( \arccos(y_0)\) revient à résoudre l'équation \( \cos(x_0)=y_0\). Cela nous permet de construire une tableau de valeurs :
\begin{equation*}
    \begin{array}[]{|c|c|c|c|c|c|c|c|c|c|}
        \hline
        x&-1&-\frac{ \sqrt{3} }{2}&-\frac{ \sqrt{2} }{2}&-\frac{ 1 }{2}&0&\frac{ 1 }{2}&\frac{ \sqrt{2} }{2}&\frac{ \sqrt{3} }{2}&1\\
          \hline
          \arccos(x)&\pi&\frac{ 5\pi }{ 6 }&\frac{ 3 }{ 4 }\pi&\frac{ 2 }{ 3 }\pi&\frac{ 1 }{2}\pi&\frac{ \pi }{ 3 }&\frac{1}{ 4 }\pi&\frac{1}{ 6 }\pi&0\\
          \hline
           \end{array}
\end{equation*}

\begin{remark}
    Certes la fonction cosinus est paire (vue sur \( \eR\)), mais la fonction arc cosinus ne l'est pas car elle est une bijection entre \(\mathopen[ -1 , 1 \mathclose]\) et \(\mathopen[ 0 , \pi \mathclose]\).
\end{remark}

\begin{example}
    Cherchons \( \arccos(\frac{ 1 }{2})\). Il faut trouver un angle \( \theta\in\mathopen[ 0 , \pi \mathclose]\) tel que \( \cos(\theta)=\frac{ 1 }{2}\). La solution est \( \theta=\frac{ \pi }{ 3 }\). Donc \( \arccos(\frac{ 1 }{2})=\frac{ \pi }{ 3 }\).

    Il n'est cependant pas immédiat d'en déduire la valeur de \( \arccos(-\frac{ 1 }{2})\). En effet \( \theta=\arccos(-\frac{ 1 }{2})\) si et seulement si \( \cos(\theta)=-\frac{ 1 }{2}\) avec \( \theta\in\mathopen[ 0 , \pi \mathclose]\). La solution est \( \theta=\frac{ 2\pi }{ 3 }\).
\end{example}

En ce qui concerne la représentation graphique, il suffit de tracer la fonction cosinus entre \( 0\) et \( \pi\) puis de prendre le symétrique par rapport à la droite \( y=x\).

\begin{center}
    \input{auto/pictures_tex/Fig_GMIooJvcCXg.pstricks}
\end{center}

%---------------------------------------------------------------------------------------------------------------------------
\subsection{Une meilleure approximation de \( \pi\)}
%---------------------------------------------------------------------------------------------------------------------------

Nous avions laissé le nombre \( \pi\) avec l'approximation assez minable de \( 2\sqrt{ 2 }<\pi<4\) en le lemme~\ref{LEMooJWSGooExmtDA}. Nous pouvons maintenant faire nettement mieux.

Le lemme~\ref{LEMooJKIUooEMMOrs} donne
\begin{equation}
    \arctan(1/\sqrt{ 3 })=\pi/6
\end{equation}
et l'idée est de donner un développement de \( \arctan\) autour de zéro, de l'évaluer en \( 1/\sqrt{ 3 }\) et d'égaliser le résultat à \( \pi/6\). Tout cela donne lieu à des calcules peut-être fastidieux, mais comme un gars l'a fait dès l'an 1424\cite{ooOMUNooGROVUu} pour trouver \( 16\) décimales correctes, nous faisons comme si c'était facile.

Pour trouver le développement en série de Taylor (théorème~\ref{ThoTaylor}) de arc tangente autour de \( x=0\), il faut partir de la formule \eqref{EQooGCHGooPlwYWt} et sans doute pas mal calculer et faire une récurrence\quext{Je n'ai pas fait le calcul, merci de me faire savoir si il y a une astuce.}. Le résultat est :
\begin{equation}
    \arctan(x)=\sum_{k=0}^{\infty}\frac{ (-1)^{k}x^{2k+1} }{ 2k+1 },
\end{equation}
valable pour \( x\in \mathopen] -1 , 1 \mathclose[\). Avec cela nous avons
\begin{equation}
    \arctan(\frac{1}{ \sqrt{ 3 } })=\sum_{k=0}^{\infty}\frac{ (-1)^k }{ (2k+1)3^k }\times \frac{1}{ \sqrt{ 3 } }=\frac{ \pi }{ 6 },
\end{equation}
et donc
\begin{equation}
    \pi=\frac{ 6 }{ \sqrt{ 3 } }\sum_{k=0}^{\infty}\frac{ (-1)k }{ (2k+1)3^k }.
\end{equation}

Pour donner une idée du fait que ça fonctionne pas mal, voici le calcul pour quelques termes :
\lstinputlisting{tex/sage/sageSnip012.sage}
Calculer \( 5\) termes donne déjà \( 3.15\). Et on est à \( 10^{-6}\) de la bonne réponse avec \( 20\) termes. Et avec $58$ termes, on n'est à \( 10^{-16}\).

\begin{probleme}
    Pour bien faire, il faudrait étudier le reste et donner un encadrement.
\end{probleme}

%---------------------------------------------------------------------------------------------------------------------------
\subsection[Forme trigonométrique des nombres complexes]{Forme polaire ou trigonométrique des nombres complexes}
%---------------------------------------------------------------------------------------------------------------------------

Un nombre complexe étant représenté par deux nombres, on peut le représenter dans un plan appelé « plan de Gauss ». La plupart des opérations sur les nombres complexes ont leur interprétation géométrique dans ce plan.

Dans le plan de Gauss, le module d'un complexe $z$ représente la distance entre $0$ et $z$. On appelle \Defn{argument} de $z$ (noté $\arg z$) l'angle (déterminé à $2\pi$ près) entre le demi-axe des réels positifs et la demi-droite qui part de $0$ et passe par $z$. Le module et l'argument d'un complexe permettent de déterminer univoquement ce complexe puisqu'on a la formule
\[z = a + bi = \module z \left( \cos(\arg(z)) + i \sin(\arg(z)) \right)\]

L'argument de $z$ se détermine via les formules
\[\frac a {\module z} = \cos(\arg(z)) \quad \frac b {\module z} = \sin(\arg(z))\]
ou encore par la formule
\[
\frac b a = \tan(\arg(z)) \quad \text{en vérifiant le quadrant.}
\]
La vérification du quadrant vient de ce que la tangente ne détermine l'angle qu'à $\pi$ près.

%---------------------------------------------------------------------------------------------------------------------------
\subsection{Angle entre deux vecteurs}
%---------------------------------------------------------------------------------------------------------------------------

\begin{propositionDef} \label{DEFooSVDZooPWHwFQ}
    Soient des vecteurs \( X,Y\in \eR^2\). Il existe un unique \( \theta\in \mathopen[ 0 , \pi \mathclose]\) tel que
    \begin{equation}		\label{eqDefAngleVect}
        \cos(\theta)=\frac{ X\cdot Y }{ \| X \|\| Y \| }.
    \end{equation}
    Ce réel est appelé \defe{angle}{angle entre deux vecteurs} entre \( X\) et \( Y\).
\end{propositionDef}

\begin{proof}
    Si $a$ et $b$ sont des réels, l'inégalité $| a |\leq b$ peut se développer en une double inégalité
    \begin{equation}
        -b\leq a\leq b.
    \end{equation}
    L'inégalité de Cauchy-Schwarz \eqref{EQooZDSHooWPcryG} devient alors
    \begin{equation}
        -\| X \|\| Y \|\leq X\cdot Y\leq\| X \|\| Y \|.
    \end{equation}
    Si $X\neq 0$ et $Y\neq 0$, nous en déduisons
    \begin{equation}
        -1\leq\frac{ X\cdot Y }{ \| X \|\| Y \| }\leq 1.
    \end{equation}
    Il existe donc par la proposition~\ref{PROPooZOZHooSMoYQD} un angle $\theta\in\mathopen[ 0 , \pi \mathclose]$ tel que
    \begin{equation}	
        \cos(\theta)=\frac{ X\cdot Y }{ \| X \|\| Y \| }.
    \end{equation}
\end{proof}

\begin{normaltext}
    Certains n'hésitent pas à écrire la formule
    \begin{equation}		\label{eqPropCosThet}
        X\cdot Y=\| X \|\| Y \|\cos(\theta).
    \end{equation}
    comme une définition du produit scalaire. C'est ce qui arrive lorsqu'on défini les fonctions trigonométriques à partir de relations dans les triangles rectangles.
\end{normaltext}

Notez que les angles entre deux vecteurs sont toujours plus petits ou égaux à \unit{180}{\degree}.

La longueur de la projection du point $P$ sur la droite horizontale va naturellement être égale à $\cos(\theta)$. En effet, si nous notons $X$ un vecteur horizontal de norme $1$, cette projection est donné par $P\cdot X$. Mais en reprenant l'équation \eqref{eqPropCosThet}, nous voyons que
\begin{equation}
	P\cdot X=\| P \|\| X \|\cos(\theta),
\end{equation}
tandis qu'ici nous avons $\| P \|=\| X \|=1$.

Nous appelons $\sin(\theta)$ la longueur de la projection sur l'axe vertical.

Quelques dessins nous convainquent que
\begin{equation}
	\begin{aligned}[]
		\sin(\theta+2\pi)&=\sin(\theta)&\cos(\theta+2\pi)&=\sin(\theta),\\
		\sin(\theta+\frac{ \pi }{2})&=\cos(\theta)&\cos(\theta+\frac{ \pi }{2})&=-\sin(\theta),\\
		\sin(\pi-\theta)&=\sin(\theta)&\cos(\pi-\theta)&=-\cos(\theta).
	\end{aligned}
\end{equation}
Le théorème de Pythagore nous montre aussi l'importante relation
\begin{equation}
	\sin^2(\theta)+\cos^2(\theta)=1.
\end{equation}

Quelques valeurs remarquables pour les sinus et cosinus :
\begin{equation}
	\begin{aligned}[]
		\sin 0&=0,&\sin\frac{ \pi }{ 6 }&=\frac{ 1 }{2},&\sin\frac{ \pi }{ 4 }&=\frac{ \sqrt{2} }{2},&\sin\frac{ \pi }{ 3 }&=\frac{ \sqrt{3} }{2},&\sin\frac{ \pi }{2}&=1,&\sin\pi&=0\\
		\cos 0&=1,&\cos\frac{ \pi }{ 6 }&=\frac{ \sqrt{3} }{2},&\cos\frac{ \pi }{ 4 }&=\frac{ \sqrt{2} }{2},&\cos\frac{ \pi }{ 3 }&=\frac{ 1 }{2},&\cos\frac{ \pi }{2}&=0,&\cos\pi&=-1
	\end{aligned}
\end{equation}

Nous pouvons prouver simplement que $\sin(\unit{30}{\degree})=\frac{ 1 }{2}$ et $\cos(\unit{30}{\degree})=\frac{ \sqrt{3} }{2}$ en s'inspirant de la figure~\ref{LabelFigGVDJooYzMxLW}. % From file GVDJooYzMxLW
\newcommand{\CaptionFigGVDJooYzMxLW}{Un triangle équilatéral de côté $1$.}
\input{auto/pictures_tex/Fig_GVDJooYzMxLW.pstricks}

%---------------------------------------------------------------------------------------------------------------------------
\subsection{Aire du parallélogramme}
%---------------------------------------------------------------------------------------------------------------------------

\begin{remark}      \label{RemaAireParalProdVect}
    Le nombre $\| a \|\| b \|\sin(\theta)$ est l'aire du parallélogramme\footnote{Défnition de ce qu'est une aire : \ref{DEFooPZRDooWbbBXy}. Preuve dans le cas d'un parallélogramme : \ref{PROPooAVVNooOOlSzr}.} formé par les vecteurs $a$ et $b$, comme cela se voit sur la figure~\ref{LabelFigBNHLooLDxdPA}. % From file BNHLooLDxdPA
\newcommand{\CaptionFigBNHLooLDxdPA}{Calculer l'aire d'un parallélogramme.}
\input{auto/pictures_tex/Fig_BNHLooLDxdPA.pstricks}
\end{remark}

\begin{proposition}     \label{PropNormeProdVectoabsint}
    Nous avons
    \begin{equation}
        \| a\times b \|=\| a \|\| b \|\sin(\theta)
    \end{equation}
    où $\theta\in\mathopen[ 0.\pi \mathclose]$ est l'angle formé par $a$ et $b$.
\end{proposition}

\begin{proof}
    En utilisant la décomposition du produit vectoriel, nous avons
    \begin{equation}
        \begin{aligned}[]
            \| a\times b \|^2&=\begin{vmatrix}
                a_2    &   a_3    \\
                b_2    &   b_3
            \end{vmatrix}^2+\begin{vmatrix}
                a_1    &   a_3    \\
                b_1    &   b_3
            \end{vmatrix}^2+\begin{vmatrix}
                a_1    &   a_2    \\
                b_1    &   b_2
            \end{vmatrix}^2\\
            &=(a_2b_3-b_2a_3)^2+(a_1b_3-a_3b_1)^2+(a_1b_2-a_2b_1)^2\\
            &=(a_1^2+a_2^2+a_3^2)(b_1^2+b_2^2+b_3^2)-(a_1b_1+a_2b_2+a_3b_3)^2\\
            &=\| a \|^2\| b \|^2-(a\cdot b)^2\\
            &=\| a \|^2\| b \|^2-\| a \|^2\| b \|^2\cos^2(\theta)\\
            &=\| a \|^2\| b \|^2\big( 1-\cos^2(\theta) \big)\\
            &=\| a \|^2\| b \|^2\sin^2(\theta).
        \end{aligned}
    \end{equation}
    D'où le résultat. Nous avons utilisé la formule de la définition \eqref{DEFooSVDZooPWHwFQ} donnant l'angle en fonction du produit scalaire.
\end{proof}

\begin{normaltext}      \label{NORMooWWOKooWzScnZ}
Si les vecteurs $a$, $b$ et $c$ ne sont pas coplanaires, alors la valeur absolue du produit mixte (voir équation \eqref{EqProduitMixteDet}) $a\cdot(b\times c)$ donne le volume du parallélépipède construit sur les vecteurs $a$, $b$ et $c$.

En effet si $\varphi$ est l'angle entre $b\times c$ et $a$, alors la hauteur du parallélépipède vaut $\| a \|\cos(\varphi)$ parce que la direction verticale est donnée par $b\times c$, et la hauteur est alors la «composante verticale» de $a$. Par conséquent, étant donné que $\| b\times c \|$ est l'aire de la base, le volume du parallélépipède vaut\footnote{Le calcul de ce volume mériterait une certaine réflexion, surtout à partir du moment où nous avons décidé de définir les fonctions trigonométriques à partir de son développement (définition~\ref{PROPooZXPVooBjONka}).}
\begin{equation}
    V=\| b\times c\|  \| a \|\cos(\varphi).
\end{equation}
Or cette formule est le produit scalaire de $a$ par $b \times c$; ce dernier étant donné par le déterminant de la matrice formée des composantes de $a$, $b$ et $c$ grâce à la formule \eqref{EqProduitMixteDet}.
\end{normaltext}

La valeur absolue du déterminant
\begin{equation}        \label{EqDeratb}
    \begin{vmatrix}
        a_1    &   a_2    \\
        b_1    &   b_2
    \end{vmatrix}
\end{equation}
est l'aire du parallélogramme déterminé par les vecteurs $\begin{pmatrix}
    a_1    \\
    a_2
\end{pmatrix}$ et $\begin{pmatrix}
    b_1    \\
    b_2
\end{pmatrix}$. En effet, d'après la remarque~\ref{RemaAireParalProdVect}, l'aire de ce parallélogramme est donnée par la norme du produit vectoriel
\begin{equation}
    \begin{pmatrix}
        a_1    \\
        a_2    \\
        0
    \end{pmatrix}\times
    \begin{pmatrix}
          b_1  \\
        b_2    \\
        0
    \end{pmatrix}=\begin{vmatrix}
        e_x    &   e_y    &   e_z    \\
        a_1    &   a_2    &   0    \\
        b_1    &   b_2    &   0
    \end{vmatrix}=
    \begin{vmatrix}
        a_1    &   a_2    \\
        b_1    &   b_2
    \end{vmatrix}e_z,
\end{equation}
donc la norme $\| a\times b \|$ est bien donnée par la valeur absolue du déterminant \eqref{EqDeratb}.

%+++++++++++++++++++++++++++++++++++++++++++++++++++++++++++++++++++++++++++++++++++++++++++++++++++++++++++++++++++++++++++ 
\section{Paramétrisation du cercle}
%+++++++++++++++++++++++++++++++++++++++++++++++++++++++++++++++++++++++++++++++++++++++++++++++++++++++++++++++++++++++++++

Nous allons parler de paramértisation du cercle. L'ensemble \( S^1\) sera vu tantôt comme le cercle dans \( \eR^2\), tantôt comme le cercle dans \( \eC\). Nous n'allons pas pousser le vice jusqu'à écrire explicitement les isomorphismes lorsque nous passons d'une représentation à l'autre. Parmi les identifications que nous allons faire sans ménagement, il y a l'identification entre les applications
\begin{equation}
    \begin{aligned}
        \gamma\colon \mathopen[ 0 , 2\pi \mathclose]&\to \eR^2 \\
        t&\mapsto \big( \cos(t),\sin(t) \big) 
    \end{aligned}
\end{equation}
et
\begin{equation}
    \begin{aligned}
        \varphi\colon \mathopen[ 0 , 2\pi \mathclose[&\to \eC \\
            t&\mapsto  e^{it}. 
    \end{aligned}
\end{equation}
C'est évidemment la formule \(  e^{ti}=\cos(t)+i\sin(t)\) (lemme \ref{LEMooHOYZooKQTsXW}) qui permet de transformer \( \gamma\) en \( \varphi\) et inversement. De plus \( \eR^2\) et \( \eC\) sont isomorphes en tant qu'espaces vectoriels normés (et aussi donc topologiques).

Nous allons voir deux choses à propos de cette application :
\begin{itemize}
\item 
    Elle est continue, mais son inverse n'est pas continue. En considérant seulement la restriction \( \varphi\colon \mathopen] 0 , 2\pi \mathclose[\to S^2\setminus\{ (1,0) \}\) nous avons un difféomorphisme, et donc une possibilité de changement de variables dans l'intégrale (théorème \ref{THOooUMIWooZUtUSg}).

    Le fait qu'il manque un point est sans importante parce que nous n'allons considérer que la mesure de Lebesgue ou des variations simples autour de la mesure de Lebesgue.        

\item
    La fonction \( \varphi\colon \mathopen[ 0 , 2\pi \mathclose[\to S^1\) est une bijection borélienne d'inverse borélien\footnote{Proposition \ref{PROPooQFYHooEajmbW}.}. Donc nous pouvons transposer toute la théorie de la mesure de \( S^1\) à \( \mathopen[ 0 , 2\pi \mathclose[\) sans «triche».
\end{itemize}

Tout cela pour dire que nous allons donner un tas de justifications pour écrire des égalités du type
\begin{equation}
    \int_{S^1}f=\int_{0}^{2\pi}f\circ\varphi.
\end{equation}

%--------------------------------------------------------------------------------------------------------------------------- 
\subsection{Bijection continue}
%---------------------------------------------------------------------------------------------------------------------------

\begin{proposition}     \label{PROPooKSGXooOqGyZj}
    L'application
    \begin{equation}
        \begin{aligned}
            \gamma\colon \mathopen[ 0 , 2\pi \mathclose[&\to S^1\subset \eR^2 \\
            t&\mapsto \big( \cos(t),\sin(t) \big)
        \end{aligned}
    \end{equation}
    est une bijection continue.
\end{proposition}

\begin{proof}
    La continuité découle de la continuité des composantes. Le fait que l'image de \( \gamma\) soit dans \( S^1\) découle immédiatement du fait que \( \sin^2+\cos^2=1\).

    Pour la bijection, il faut injectif et surjectif.
    \begin{subproof}
        \item[Injectif]
            Soient \( x_1<x_2\) tels que \( \sin(x_1)=\sin(x_2)\) et \( \cos(x_1)=\cos(x_2)\). Supposons pour fixer les idées que \( \sin(x_1)>0\) et \( \cos(x_1)>0\) : si ce n'est pas le cas, il faut traiter séparément les \( 4\) possibilités de combinaisons de signes.

            Nous avons obligatoirement \( x_1,x_2\in\mathopen[ 0 , \frac{ \pi }{ 2 } \mathclose[\). Vu que \( \sin(x_1)=\sin(x_2)\), il existe par le théorème de Rolle~\ref{ThoRolle} un élément \( c\in \mathopen] x_1 , x_2 \mathclose[\) tel que \( \sin'(c)=0\), c'est-à-dire \( \cos(c)=0\). Cela contredirait la proposition~\ref{PROPooMWMDooJYIlis}\ref{ITEMooQKPKooEPeHER} à moins que \( x_1=x_2\).

            \item[Surjectif]

                Soient \( x,y\) tels que \( x^2+y^2=1\). Supposons pour varier les plaisirs que \( x<0\) et \( y>0\). Vu que la fonction \( \cos\) va de \( 0\) à \( -1\) lorsque \( x\) va de \( \pi/2\) à \( \pi\), le théorème des valeurs intermédiaires donne \( t\in\mathopen[ \pi/2 , \pi \mathclose]\) tel que \( \cos(t)=x\). Pour cette valeur de \( x\) nous avons
                \begin{equation}
                    \cos^2(x)+\sin^2(x)=1,
                \end{equation}
                et donc \( \sin^2(x)=y^2\), ce qui donne \( \sin(x)=\pm y\). Mais pour \( x\in \mathopen[ \pi/2 , \pi \mathclose]\) nous avons \( \sin(t)>0\). Par conséquent \( \sin(t)=y\).
    \end{subproof}
\end{proof}

\begin{example} \label{EXooJFDPooBZADKs}
    L'application
    \begin{equation}
        \begin{aligned}
            \varphi\colon \mathopen] 0 , 2\pi \mathclose[&\to S^1 \\
                x&\mapsto \begin{pmatrix}
                    \cos(x)    \\
                    \sin(x)
                \end{pmatrix}
        \end{aligned}
    \end{equation}
est un continue par la proposition \ref{PROPooKSGXooOqGyZj}. Vu que \( \mathopen] 0 , 2\pi \mathclose[\) est connexe (proposition~\ref{PropInterssiConn}) la proposition~\ref{PropGWMVzqb} implique que le cercle privé d'un point est connexe.
\end{example}


Allez\ldots Dans l'intro nous avions dit que nous n'allions pas faire explicitement les isomorphismes. Faisons-le quand même une fois, mais c'est bien parce que c'est vous hein.
\begin{proposition}     \label{PROPooZEFEooEKMOPT}
    L'application
    \begin{equation}
        \begin{aligned}
            f\colon \mathopen[ 0 , 2\pi \mathclose[&\to S^1 \\
                x&\mapsto  e^{ix}
        \end{aligned}
    \end{equation}
    est une bijection. Ici, \( S^1\) est l'ensemble des nombres complexes de norme \( 1\).
\end{proposition}

\begin{proof}
    Nous savons que
    \begin{equation}
        \begin{aligned}
            \varphi\colon \eR^2&\to \eC \\
            (x,y)&\mapsto x+iy
        \end{aligned}
    \end{equation}
    est une bijection isométrique. C'est pour cela que nous allons nous permettre de noter \( S^1\) le cercle unité dans \( \eR^2\) aussi bien que l'ensemble des nombres complexes de norme \( 1\).

    Sur \( \eR^2\) nous avons l'application
    \begin{equation}
        \begin{aligned}
            \gamma\colon \mathopen[ 0 , 2\pi \mathclose[&\to S^1\subset \eR^2 \\
                t&\mapsto \begin{pmatrix}
                    \cos(t)    \\
                    \sin(t)
                \end{pmatrix}
        \end{aligned}
    \end{equation}
    qui est une bijection continue (c'est la proposition~\ref{PROPooKSGXooOqGyZj}). Et enfin le lemme~\ref{LEMooHOYZooKQTsXW} nous donne \(  e^{ix}=\cos(x)+i\sin(x)\).

    Avec tout ça, l'application \( \varphi^{-1}\circ f\colon \mathopen[ 0 , 2\pi \mathclose[\to S^1 \) est une bijection continue. Et comme \( \varphi\) l'est également, \( f\) est une bijection continue.
\end{proof}

La proposition suivante donne les coordonnées polaires sur \( \eC\). La régularité est l'objet du théorème \ref{THOooBETSooXSQhdX} (à part le fait que ce dernier parle de \( \eR^2\) et non de \( \eC\)).
\begin{proposition}     \label{PROPooRFMKooURhAQJ}
    Pour tout nombre complexe \( z\), il existe un unique \( \theta\in\mathopen[ 0 , 2\pi \mathclose[\) tel que
        \begin{equation}
            z=| z | e^{i\theta}.
        \end{equation}
\end{proposition}

\begin{proof}
    Soit \( z\in \eC\). Nous considérons \( z'=z/| z |\) qui est de norme \( 1\). Donc il existe un unique \( \theta\in\mathopen[ 0 , 2\pi \mathclose[\) tel que \( z'= e^{i\theta}\) (proposition \ref{PROPooZEFEooEKMOPT}).

    Pour ce \( \theta\) nous avons \( z=| z | e^{i\theta}\).
\end{proof}
Bien entendu, le \( \theta\) est unique dans \( \mathopen[ 0 , 2\pi \mathclose[\), mais il n'est pas du tout unique dans \( \eR\).

%--------------------------------------------------------------------------------------------------------------------------- 
\subsection{Inverse}
%---------------------------------------------------------------------------------------------------------------------------
\label{SUBSECooWFNMooOuZBRN}

Nous pouvons écrire un inverse de la fonction \( \varphi\) grâce à la fonction arc tangente introduite au théorème \ref{THOooUSVGooOAnCvC}. 
La fonction que nous écrivons à présent est la fonction \( \arg_{0^{-}} \) définie par \eqref{EQooNKKDooOuJxXe}. Elle n'est pas exactement la fonction argument définie par \eqref{EQooPJVFooSEKTny}.

Nous avons :
\begin{equation}        \label{EQooSAYFooRFVSPc}
    \begin{aligned}
        \varphi^{-1}\colon S^1&\to \mathopen[ 0 , 2\pi \mathclose[ \\
        x+iy&\mapsto  
    \begin{cases}
        \arctg(y/x)    &   \text{si } x>0,y\geq 0\\
        \frac{ \pi }{2}    &    \text{si }(x,y)=(0,1)\\
        \pi-\arctg(-y/x)    &    \text{si }x<0,y\geq 0\\
        \pi+\arctg(y/x)    &    \text{si }x<0,y<0\\
        2\pi-\arctg(-y/x)    &    \text{si }x>0,y<0
    \end{cases}
    \end{aligned}
\end{equation}
Chacune des branche est continue parce que la fonction arc tangente l'est. Trois des raccords sont également continus grâce aux limites du lemme \ref{LEMooHRDCooGtnyeQ}.

L'application \( \varphi^{-1}\) n'est cependant pas continue au point \( (1,0)\)\footnote{Vu que nous avons considéré \( S^1\subset \eC\), nous aurions dû noter «\( 1\)» ce point. Mais vous vous imaginez le clash de notation avec le \( 1\in \mathopen[ 0 , 2\pi \mathclose[\subset \eR\)?}. C'est l'objet du lemme suivant.

\begin{lemma}       \label{LEMooEQVRooMAffCw}
    L'application \( \varphi^{-1}\colon S^1\to \mathopen[ 0 , 2\pi \mathclose[\) n'est pas continue en \( (1,0)\). Mais elle est continue ailleurs. Autrement dit,
        \begin{equation}
        \varphi^{-1}\colon S^1\setminus\{ (1,0) \}\to \mathopen] 0 , 2\pi \mathclose[
        \end{equation}
        est continue.
\end{lemma}

\begin{proof}
    En effet, \( \varphi^{-1}\) serait continue si l'image de tout ouvert de \( \mathopen[ 0 , 2\pi \mathclose[\) par \( \varphi\) serait ouverte dans \( S^1\) (topologie induite de \( \eC\)). Prenons un petit ouvert \( \mathopen[ 0 , \epsilon \mathclose[\) (si vous êtes étonnés, c'est que vous n'avez pas bien la topologie induites en tête). Son image contient le point \( (1,0)\), mais aucun point \( (x,y)\) avec \( y<0\).
       
    Montrons que tout voisinage de \( (1,0)\) dans \( \eC\) contient des points \( x+iy\) de \( S^1\) avec \( y<0\). Un point de \( S^1\) est de la forme \( \cos(t)+i\sin(t)\). Nous avons :
    \begin{equation}
        | \cos(t)+i\sin(t)-1 |^2=\big( \cos(t)-1 \big)^2+\sin^2(t)=2\big( 1-\cos(t) \big).
    \end{equation}
    Soit \( \delta>0\), et montrons que \( B\big( (1,0),\delta \big)\cap S^1\) contient des points d'ordonnées négatives. D'abord il existe \( \epsilon>0\) tel que pour \( t=2\pi-\epsilon\),
    \begin{equation}
        2\big( 1-\cos(t) \big)<\delta.
    \end{equation}
    Ensuite pour de tels \( t\), nous avons \( \sin(t)<0\). Donc les points de \( S^1\) correspondant à \( 2\pi-\epsilon\) sont dans \( S^1\cap B\big( (1,0),\delta \big)\).

    Bref, l'image de \( \mathopen[ 0 , \epsilon \mathclose[\) n'est pas un ouvert de \( S^1\).
\end{proof}

%---------------------------------------------------------------------------------------------------------------------------
\subsection{Cercle trigonométrique}
%---------------------------------------------------------------------------------------------------------------------------

% Si la figure est inclue après la définition, le détecteur de références vers le futur râle puis crache parce 
% qu'il ne trouve pas la ligne du label.
\newcommand{\CaptionFigCercleTrigono}{Le cercle trigonométrique.}
\input{auto/pictures_tex/Fig_CercleTrigono.pstricks}

\begin{definition}[\cite{BIBooNPBYooDrvLky}]
    Le cercle trigonométrique est le cercle dans \( \eR^2\) de rayon $1$ centré en \( (0,0)\) représenté à la figure~\ref{LabelFigCercleTrigono}. Nous n'hésiterons pas à parler de cercle trigonométrique dans \( \eC\).
\end{definition}

Nous verrons plus tard que la longueur de l'arc de cercle intercepté par un angle $\theta$ est égal à $\theta$. Les radians sont donc l'unité d'angle les plus adaptés au calcul de longueurs sur le cercle.

%TODOooLMZOooWNDjgq : remettre ce lien après le fork
%Voir exercice~\ref{exoGeomAnal-0034}.

%--------------------------------------------------------------------------------------------------------------------------- 
\subsection{Du point de vue de la tribu, mesure et co.}
%---------------------------------------------------------------------------------------------------------------------------

Nous avons considéré sur \( S^1\) la topologie induite de \( \eC\). Nous allons y mettre la tribu induite de celle de Lebesgue de \( \eC\). Mais nous n'allons pas y mettre la \emph{mesure} induite de \( \eC\); sinon tout serait toujours de mesure nulle.

\begin{proposition}[\cite{MonCerveau}]      \label{PROPooQFYHooEajmbW}
    L'application \( \varphi\) est borélienne d'inverse borélien, c'est-à-dire
    \begin{equation}
        \Borelien(S^1)=\varphi\big( \Borelien(\mathopen[ 0 , 2\pi \mathclose[) \big).
    \end{equation}
\end{proposition}

\begin{proof}
    L'inclusion  \(\Borelien(S^1)\subset\varphi\big( \Borelien(\mathopen[ 0 , 2\pi \mathclose[) \big) \) est la plus simple : si \( A\in\Borelien(S^1)\), alors \( \varphi^{-1}(A)\in\Borelien\big( \mathopen[ 0 , 2\pi \mathclose[ \big)\) parce que \( \varphi\colon \mathopen[ 0 , 2\pi \mathclose[\to S^1\) est continue et donc borélienne (théorème \ref{ThoJDOKooKaaiJh}).

    Pour l'autre inclusion, il faudra faire par étapes.
    \begin{subproof}
        \item[Ouvert ne contenant pas zéro]
            Si \( A\) est un ouvert de \( \mathopen[ 0 , 2\pi \mathclose[\) ne contenant pas \( 0\), il est un ouvert de \( \eR\) ou de \( \mathopen] 0 , 2\pi \mathclose[\). Le lemme \ref{LEMooEQVRooMAffCw} nous indique que son image par \( \varphi\) est ouverte dans \( S^1\). En particulier, \( \varphi(A)\in\Borelien(S^1)\).
            \item[Ouvert de la forme \( \mathopen[ 0 , \epsilon \mathclose[\)] 
                Nous supposons que \( \epsilon\) est petit. Disons pour fixer les idées, plus petit que \( \pi/2\). Nous avons :
                \begin{equation}
                    \varphi\big( \mathopen[ 0 , \epsilon \mathclose[ \big)=\varphi\big( \mathopen] 0 , \epsilon \mathclose[ \big)\cup\varphi\big( \{ 0 \} \big). 
                \end{equation}
                Le premier élément de l'union est un ouvert, et le second un unique point. L'union est un borélien.
            \item[Ouvert général]
                Si un ouvert de \( \mathopen[ 0 , 2\pi\mathclose[\) ne contient pas \( 0\), son image est ouverte. Nous nous penchons sur le cas d'un ouvert contenant \( 0\).

                Si un ouvert de \( \mathopen[ 0 , 2\pi \mathclose[\) contient \( 0\), alors il contient un ouvert de la forme \( \mathopen[ 0 , \epsilon \mathclose[\), parce qu'un ouvert contient une boule autour de chacun de ses points (théorème \ref{ThoPartieOUvpartouv} couplé au fait que nous sommes dans la topologie induite de \( \eR\)).

                Si \( A\) est un ouvert contenant zéro, alors
                \begin{equation}
                    A=\mathopen[ 0 , \epsilon \mathclose[\cup\big( A\setminus\mathopen[ 0 , \frac{ \epsilon }{2} \mathclose] \big).
                \end{equation}
                Nous avons déjà vu que l'image du premier élément de l'union est un borélien. Étant donné que \( A\setminus \mathopen[ 0 , \frac{ \epsilon }{2} \mathclose]\) est un ouvert ne contenant pas zéro, son image est un ouvert. Donc le l'image de \( A\) est un borélien.

            \item[Pause]
                Nous avons déjà vu que l'image par \( \varphi\) de tout ouvert de \( \mathopen[ 0 , 2\pi \mathclose[\) était un borélien de \( S^1\). Nous devons en déduire que l'image de tout borélien de \( \mathopen[ 0 , 2\pi \mathclose[\) est un borélien de \( S^1\).

                    C'est ce que nous faisons maintenant

                \item[Boréliens]
                
                    Nous utilisons le lemme de transport \ref{LemOQTBooWGYuDU} avec l'application \( \varphi^{-1}\) et l'ensemble des ouverts :
                    \begin{equation}
                        \varphi\big( \sigma(\tribC) \big)=\sigma\big( \varphi(\tribC) \big)
                    \end{equation}
                    où \( \tribC\) est la tribu des ouverts dans \( \mathopen[ 0 , 2\pi \mathclose[\). L'ensemble \( \sigma(\tribC)\) est par définition l'ensemble \( \Borelien\big( \mathopen[ 0 , 2\pi \mathclose[ \big)\). D'autre part nous avons vu que l'image d'un ouvert est un borélien : \( \varphi(\tribC)\subset\Borelien(S^1)\). Nous avons donc
                        \begin{equation}
                                \varphi\big( \Borelien(\mathopen[ 0 , 2\pi \mathclose[) \big)=\sigma\big( \varphi(\tribC) \big)\subset\sigma\big( \Borelien(S^1) \big)\subset\Borelien(S^1).
                        \end{equation}
    \end{subproof}
    La preuve est terminée. 
\end{proof}

\begin{proposition}[Boréliens sur \( S^1\)\cite{MonCerveau}]      \label{PROPooHMSCooRIjcJq}
    Soit la structure usuelle d'espace mesurable \( (\eC,\Borelien(\eC))\). Nous considérons
    \begin{itemize}
        \item la tribu \( \Borelien(\eC)_{S^1}\) induite de la tribu des boréliens  de \( \eC\) vers \( S^1\),
        \item la tribu \( \Borelien(S^1)\) des boréliens de \( S^1\) construite à partir de la topologie induite de \( \eC\) vers \( S^1\).
        \item la bijection \( \varphi\colon \mathopen[ 0 , 2\pi \mathclose[\to S^1\),
            \item la mesure de Lebesgue sur \( \mathopen[ 0 , 2\pi \mathclose[\) (induite de celle sur \( \eR\)) et sur \( \eC\), que nous noterons toutes deux \( \lambda\).
    \end{itemize}
    Alors 
    \begin{enumerate}
        \item       \label{ITEMooSUNEooRhAdep}
            Nous avons les expressions
            \begin{subequations}
                \begin{align}
                    \Borelien(\eC)_{S^1}&=\{A\in\Borelien(\eC)\tq A\subset S^1\} \\
                    &=\{A\cap S^1\tq A\in\Borelien(\eC)\}       \label{SUBEQooYZGCooDqXmft}
                \end{align}
            \end{subequations}
        \item       \label{ITEMooGYPNooRaZbNW}
            Nous avons
            \begin{equation}
                \Borelien(S^1) = \Borelien(\eC)_{S^1}=\varphi\Big( \Borelien\big( \mathopen[ 0 , 2\pi \mathclose[ \big) \Big).
            \end{equation}
       \item\label{ITEMooFUXKooFQdoaw}
           En définissant \( \mu\colon \Borelien(S^1)\to \eR\) par
           \begin{equation}         \label{EQooKHZRooSrFMdo}
               \mu(A)=\frac{ \lambda\big( \varphi^{-1}(A) \big) }{ 2\pi },
           \end{equation}
           le triple \( \big( S^1,\Borelien(S^1), \mu \big)\) est un espace mesuré.
       \item\label{ITEMooBQLRooOsqesg}
           L'espace mesuré \( \big( S^1,\Borelien(S^1), \mu \big)\) est fini et
            \begin{equation}
                \mu(S^1)=1.
            \end{equation}
    \end{enumerate}
\end{proposition}

\begin{proof}
    Point par point.
    \begin{subproof}
        \item[Pour \ref{ITEMooSUNEooRhAdep}]
            C'est la proposition \ref{PROPooUNNSooMUQKfp}.
        \item[Pour \ref{ITEMooGYPNooRaZbNW}]
            La première égalité est le lemme \ref{LEMooUPYDooPVjscA}. Le fait que \( \Borelien(S^1)=\varphi\Big( \Borelien\big( \mathopen[ 0 , 2\pi \mathclose[ \big) \Big)\) est déjà la proposition \ref{PROPooQFYHooEajmbW}.
            \item[Pour \ref{ITEMooFUXKooFQdoaw}]
                Nous devons d'abord nous assurer que la formule ait un sens. Cela est chose aisée; si \( A\in \Borelien(S^1)\), le point \ref{ITEMooGYPNooRaZbNW} nous indique que \( \varphi^{-1}(A)\in \Borelien\big( \mathopen[ 0 , 2\pi \mathclose[ \big)\). Ensuite, nous devons vérifier les deux conditions de la définition \ref{DefBTsgznn} pour avoir un espace mesuré.

                En premier lieu,
                \begin{equation}
                    \mu(\emptyset)=\frac{1}{ 2\pi }\lambda\big( \varphi^{-1}(\emptyset) \big)=\frac{1}{ 2\pi }(\emptyset)=0.
                \end{equation}
                En en second lieu, si les \( A_i\in \Borelien(S^1)\) sont disjoints, les \( \varphi^{-1}(A_i)\) sont également disjoints parce que \( \varphi^{-1}\) est une bijection. Donc
                \begin{subequations}
                    \begin{align}
                        \mu(\bigcup_iA_i)&=\frac{1}{ 2\pi }\lambda\big( \bigcup_i\varphi^{-1}(A_i) \big)\\
                        &=\frac{1}{ 2\pi }\sum_i\lambda\big( \varphi^{-1}(A_i) \big)\\
                        &=\sum_i\frac{ \lambda\big( \varphi^{-1}(A_i) \big) }{ 2\pi }\\
                        &=\sum_i\mu(A_i).
                    \end{align}
                \end{subequations}
                D'accord.
            \item[Pour \ref{ITEMooBQLRooOsqesg}]
                En ce qui concerne la mesure de \( S^1\) pour \(\mu\) nous avons simplement
                \begin{equation}
                    \mu(S^1)=\frac{ \lambda\big( \mathopen[ 0 , 2\pi \mathclose[ \big) }{ 2\pi }=1.
                \end{equation}
    \end{subproof}
\end{proof}

Maintenant que \( (S^1,\Borelien(S^1), \mu)\) est un espace mesuré, nous pouvons compléter la tribu \( \Borelien(S^1)\) pour la mesure \( \mu\).

\begin{definition}
    La \defe{tribu de Lebesgue}{tribu de Lebesgue sur $ S^1$} sur \( S^1\) est la mesure complétée pour
    \begin{equation}
        \big( S^1,\Borelien(S^1),\mu \big)
    \end{equation}
    où \( \mu\) est la mesure définie par la proposition \ref{PROPooHMSCooRIjcJq}. Nous notons \( \Lebesgue(S^1)\) la tribu et encore \( \mu\) la mesure.
\end{definition}

\begin{proposition}[Lebesgue sur \( S^1\)\cite{MonCerveau}]     \label{PROPooDLBCooUfQZOa}
    Soit la structure d'espace mesuré complet \( \big( S^1,\Lebesgue(S^1), \mu \big)\). Nous considérons
    \begin{itemize}
        \item la tribu \( \Lebesgue(\eC)_{S^1}\) induite de la tribu des boréliens  de \( \eC\) vers \( S^1\),
        \item la bijection \( \varphi\colon \mathopen[ 0 , 2\pi \mathclose[\to S^1\),
    \end{itemize}
    Alors 
    \begin{enumerate}
        \item               \label{ITEMooQMHDooHEThPf}
            La tribu \( \Lebesgue(\eC)_{S^1}\) est la tribu de toutes les parties de \( S^1\).
        \item       \label{ITEMooNIRNooKSeyCa}
            La tribu \( \Lebesgue(S^1)\) est donnée par 
            \begin{equation}
                \Lebesgue(S^1)=\varphi\big( \Lebesgue(\eR)_{\mathopen[ 0 , 2\pi \mathclose[} \big)=\varphi\big( \Lebesgue(\mathopen[ 0 , 2\pi \mathclose[) \big)
            \end{equation}
            où \( \Lebesgue(\mathopen[ 0 , 2\pi \mathclose[)\) est la tribu sur \( \mathopen[ 0 , 2\pi \mathclose[\) obtenue par completion de la tribu des boréliens de la topologie induite.
        \item       \label{ITEMooXDBTooYnauyi}
            Nous avons l'inclusion stricte
            \begin{equation}
                \Lebesgue(S^1)\subsetneq\Lebesgue(\eC)_{S^1}.
            \end{equation}
    \end{enumerate}
\end{proposition}

\begin{proof}
    Point par point.
    \begin{subproof}
        \item[Pour \ref{ITEMooQMHDooHEThPf}]
            Si \( A\subset S^1\), alors \( A\) est une partie de \( S^1\) qui est mesurable et de mesure nulle pour \( \eC\). Donc \( A\) est \( \lambda\)-négligeable et par conséquent mesurable.
        \item[Pour \ref{ITEMooNIRNooKSeyCa}]
            Il s'agit de prouver que
            \begin{equation}
                \widehat{\Borelien(S^1)}=\varphi\big( \widehat{\Borelien\big( \mathopen[ 0 , 2\pi \mathclose[ \big)} \big).
            \end{equation}
            Ce n'est rien d'autre que la proposition \ref{PROPooORDCooJEsjzR}. La seconde partie de l'égalité est la proposition \ref{PROPooAMIEooRomnMG}
        \item[Pour \ref{ITEMooXDBTooYnauyi}]
            Comme indiqué au point \ref{ITEMooQMHDooHEThPf}, la tribu \( \Lebesgue(\eC)_{S^1}\) est la tribu de toutes les parties de \( S^1\); l'incusion est donc évidente. Le point pas tout à fait évident à prouver est l'existence de parties de \( S^1\) à n'être pas dans \( \Lebesgue(S^1)\).

            Soit \( V\) non mesurable dans \( \mathopen[ 0 , 2\pi \mathclose[\) (prenez quelque chose comme l'ensemble de Vitali de l'exemple \ref{EXooCZCFooRPgKjj}). Vu que, par le point \ref{ITEMooNIRNooKSeyCa},
                \begin{equation}
                    \Lebesgue(S^1)=\varphi\big( \Lebesgue(\eR)_{\mathopen[ 0 , 2\pi \mathclose[} \big),
                \end{equation}
                la partie \( \varphi^{-1}(V)\) ne peut pas être dans \( \Lebesgue(S^1)\).
    \end{subproof}
\end{proof}

Si vous en voulez plus à propos de \( S^1\) et la façon dont on passe la structure depuis \( \mathopen[ 0 , 2\pi \mathclose[\), vous pouvez lire la proposition \ref{PROPooDJERooYirMru} qui donne la structure de
\begin{equation}
    L^2\big( S^1,\Lebesgue(S^1), \mu \big)
\end{equation}
qui sera, sans surprises la même que celle de
\begin{equation}
    L^2\big( \mathopen[ 0 , 2\pi \mathclose[,\Lebesgue\big( \mathopen[ 0 , 2\pi \mathclose[ \big), \lambda \big).
\end{equation}

%+++++++++++++++++++++++++++++++++++++++++++++++++++++++++++++++++++++++++++++++++++++++++++++++++++++++++++++++++++++++++++
\section{Exemples trigonométriques}
%+++++++++++++++++++++++++++++++++++++++++++++++++++++++++++++++++++++++++++++++++++++++++++++++++++++++++++++++++++++++++++

Nous mettons ici quelques exemples concernant les fonctions trigonométriques, qui n'ont pas pu être mis dans les chapitres le plus adapté, parce que ces derniers sont plus haut dans la table des matière.

\begin{example}     \label{EXooSPFDooSluUGV}
    Prouvons que la fonction\footnote{La définition de la fonction sinus est \ref{PROPooZXPVooBjONka}.} $f(x)=x\sin(x)$ tend vers zéro lorsque $x$ tend vers $0$. D'abord, nous coinçons la fonction entre deux fonctions connues :
	\begin{equation}
		0\leq| x\sin(x) |=| x | |\sin(x) |\leq | x |.
	\end{equation}
	Donc $| x\sin(x) |$ est coincé entre $g(x)=0$ et $h(x)=| x |$. Ces deux fonctions tendent vers $0$ lorsque $x\to 0$, et donc $f(x)$ tend vers zéro.
\end{example}

%--------------------------------------------------------------------------------------------------------------------------- 
\subsection{Quelques équations trigonométriques}
%---------------------------------------------------------------------------------------------------------------------------

La proposition suivante se voit très facilement sur le cercle trigonométrique, mais il faut le démontrer.
\begin{proposition}[\cite{MonCerveau}]     \label{PROPooTUUUooVrAGQo}
    Si \( \theta_0\in \mathopen[ 0 , 2\pi \mathclose[\) vérifie \( \cos(\theta_0)=x_0\), alors l'ensemble de solutions de l'équation \( \cos(\theta)=x_0\) (d'inconnue \( \theta\)) est
        \begin{equation}
            \{ \theta_0,2\pi-\theta_0 \}.
        \end{equation}
        Cet ensemble est un singleton si et seulement si \( x_0=\pm1\).
\end{proposition}

\begin{proof}
    Commençons par prouver que \( \theta_0\) et \( 2\pi-\theta_0\) sont des solutions. Le nombre \( \theta_0\) est solution par hypothèse. En ce qui concerne \( 2\pi-\theta_0\), il est possible d'utiliser la formule d'addition d'angle \eqref{EQooCVZAooQfocya} :
    \begin{equation}        \label{EQooUCAOooTQsUUq}
        \cos(2\pi-\theta_0)=\cos(2\pi)\cos(\theta_0)+\sin(2\pi)\sin(\theta_0).
    \end{equation}
    La proposition \ref{PROPooMWMDooJYIlis}\ref{ITEMooRJZHooCXcKmM} nous indique que \( \cos(2\pi)=1\) et \( \sin(2\pi)=0\). Donc l'égalité \eqref{EQooUCAOooTQsUUq} se réduit à \( \cos(2\pi -x_0)=\cos(2\pi)\).

    Le lemme \ref{LEMooAEFPooGSgOkF} dit que si \( \cos(\theta)=x_0\), alors
    \begin{equation}
        \sin(\theta)=\pm\sqrt{ 1-x_0^2 }.
    \end{equation}
    Nous avons donc soit
    \begin{equation}
        \begin{pmatrix}
            \cos(\theta)    \\ 
            \sin(\theta)    
        \end{pmatrix}=\begin{pmatrix}
            x_0    \\ 
            \sqrt{ 1-x_0^2 }    
        \end{pmatrix},
    \end{equation}
    soit
    \begin{equation}
        \begin{pmatrix}
            \cos(\theta)    \\ 
            \sin(\theta)    
        \end{pmatrix}=\begin{pmatrix}
            x_0    \\ 
            -\sqrt{ 1-x_0^2 }    
        \end{pmatrix},
    \end{equation}
    Vu que \( \theta\mapsto\big( \cos(\theta),\sin(\theta) \big)\) est une bijection avec \( S^1\) (proposition \ref{PROPooKSGXooOqGyZj}), chacune de ces deux possibilités possède une unique solution. L'ensemble des solutions de \( \cos(\theta)=x_0\) possède donc au maximum deux éléments.

    L'ensemble des solutions possède exactement une solution lorsque les points \( \big( x_0,\sqrt{ 1-x_0^2 } \big)\) et \( \big( x_0,-\sqrt{ 1-x_0^2 } \big)\) sont identiques. Cela est le cas si et seulement si \( \sqrt{ 1-x_0^2 }=0\), c'est-à-dire si et seulement si \( x_0=\pm 1\).
\end{proof}

%--------------------------------------------------------------------------------------------------------------------------- 
\subsection{Développements en série}
%---------------------------------------------------------------------------------------------------------------------------

\begin{proposition}[Taylor pour cosinus]     \label{PROPooNPYXooTuwAHP}
    Le développement du cosinus est donné par
	\begin{equation}
		\cos(x)=1-\frac{ x^2 }{ 2 }+\frac{ x^4 }{ 4! }-\frac{ x^6 }{ 6! }\cdots
	\end{equation}
    C'est-à-dire que pout tout \( n\in  \eN\), il existe une fonction \( \alpha\colon \eR\to \eR\) telle que \( \lim_{t\to 0} \alpha(t)=0\) et
    \begin{equation}        \label{EQooGQOIooIkwbJV}
        \cos(x)=\sum_{k=0}^{n}\frac{ (-1)^{2k} }{ (2k)! }x^{2k}+\alpha(x)x^{2n+1}.
    \end{equation}
    En ce qui concerne le sinus, pour tout \( n\) nous avons une fonction \( \alpha\colon \eR\to \eR\) telle que \( \lim_{t\to 0} \alpha(t)=0\) et
    \begin{equation}        \label{EQooKYJAooRebHgc}
        \sin(x)=\sum_{k=0}^n\frac{ (-1)^kx^{2k+1} }{ (2k+1)! }+x^{2n+2}\alpha(x).
    \end{equation}
\end{proposition}

\begin{proof}
    Il s'agit d'utiliser la proposition \ref{PROPooQLHNooRsBYbe}, en faisant attention à l'ordre. Le fait est que dans \eqref{EQooGQOIooIkwbJV}, nous avons écrit le polynôme de degré \( 2n+1\) (et non seulement \( 2n\)), en sachant que le terme d'ordre \( 2n+1\) est nul.

    C'est pour cela que nous avons pu écrire \( \alpha(x)x^{2n+1}\) au lieu de \( \alpha(x)x^{2n}\) qui aurait été attendu.

    Même raisonnement pour le développement du sinus.
\end{proof}

\begin{remark}
    Quelques remarques concernant l'ordre du polynôme.
    \begin{enumerate}
        \item
            
Notons que nous aurions aussi pu écrire le reste sous la forme \( \alpha(x)x^{2n}\), mais ça aurait été avec une autre fonction \( \alpha\) : celle correspondant au développement à l'ordre \( 2n\) au lieu de \( 2n+1\).
\item

Les développements de sinus et de cosinus ont un terme sur deux qui est nul. C'est pour cela qu'en ayant une polynôme de degré \( 2p\), nous avons le développement d'ordre \( 2p+1\).

\item

   Nous aurions pu utiliser les dérivées données dans la proposition \ref{LEMooBBCAooHLWmno} et les valeurs spéciales \eqref{SUBEQooTTNNooXzApSM}.
    \end{enumerate}
\end{remark}

\begin{corollary}
    Il existe une fonction \( \alpha\colon \eR\to \eR\) telle que \( \lim_{t\to 0} \alpha(t)/t=0\) et 
    \begin{equation}        \label{EQooDLGIooXyfmtC}
        \sin(x)=x+\alpha(x).
    \end{equation}
    Nous avons la limite
    \begin{equation}
        \lim_{x\to 0} \frac{ \sin(x) }{ x }=1.
    \end{equation}
\end{corollary}

\begin{proof}
    Il s'agit de prendre la formule \eqref{EQooKYJAooRebHgc} avec \( n=0\). Cela donne tout de suite \eqref{EQooDLGIooXyfmtC}. Pour la limite, on divise par \( x\), ce qui donne (pour tout \( x\neq 0\)) 
    \begin{equation}
        \frac{ \sin(x) }{ x }=1+\frac{ \alpha(x) }{ x }.
    \end{equation}
    Et justement la fonction \( \alpha\) la la propriété que \( \lim_{x\to 0} \alpha(x)/x=0\).
\end{proof}

\begin{example}
    Cherchons le développement limité à l'ordre \( 5\) de \( \tan(x)=\frac{ \sin(x) }{ \cos(x) }\). Nous utilisons les développements de la proposition \ref{PROPooNPYXooTuwAHP} : 
    \begin{subequations}
        \begin{align}
            \sin(x)&=x-\frac{ x^3 }{ 6 }+\frac{ x^5 }{ 120 }+x^5\alpha_1(x)\\
            \cos(x)&=1-\frac{ x^2 }{ 2 }+\frac{ x^4 }{ 24 }+x^5\alpha_2(x).
        \end{align}
    \end{subequations}
    Nous calculons alors la division des deux polynômes, en classant les puissances dans l'ordre croissant (c'est le sens inverse de ce qui est fait pour la divisions euclidienne !) :
    \begin{equation*}
        \begin{array}[]{ccccccccccc|c}
            &x&-&\frac{1}{ 6 }x^3&+&\frac{1}{ 120 }x^5&&&&&&1-\frac{ 1 }{2}x^2+\frac{1}{ 24 }x^4\\
            \cline{12-12}
            -\Big( &x&-&\frac{ 1 }{2}x^3&+&\frac{1}{ 24 }x^5&\Big)& & && &x+\frac{1}{ 3 }x^3+\frac{ 2 }{ 15 }x^5\\
            \cline{2-6}
            & & &\frac{1}{ 3 }x^3&-&\frac{1}{ 30 }x^5& & & & && \\
            &&-\Big(  &\frac{1}{ 3 }x^3&-&\frac{1}{ 6 }x^5&+&\frac{1}{ 72 }x^7&\Big) & & \\
            \cline{4-8}
            & & & & &\frac{ 2 }{ 15 }x^5&-&\frac{1}{ 72 }x^7& & & \\
            & & &  &-\Big(  &\frac{ 2 }{ 15 }x^5&-&\frac{1}{ 15 }x^7&+&\frac{1}{ 180 }x^9&\Big)& \\
            \cline{6-10}
            & & & & & & &\frac{ 29 }{ 360 }x^7&-&\frac{1}{ 180 }x^9&& \\
        \end{array}
    \end{equation*}
    Nous avons continué la division jusqu'à obtenir un reste de degré plus grand que \( 5\). Le développement à l'ordre $5$ de la fonction tangente autour de zéro est alors (proposition \ref{PROPooMANAooXhuanS})
    \begin{equation}
        \tan(x)=x+\frac{1}{ 3 }x^3+\frac{ 2 }{ 15 }x^5+x^5\alpha(x).
    \end{equation}
    Notons que, vu que le reste ne nous intéresse pas vraiment, nous aurions pu ne pas calculer les coefficients des termes en \( x^7\) et \( x^8\). La dernière soustraction était également inutile.
\end{example}

%--------------------------------------------------------------------------------------------------------------------------- 
\subsection{Intégration}
%---------------------------------------------------------------------------------------------------------------------------

\begin{example}
    Comme nous le voyons sur le dessin suivant,
    \begin{equation}
        \int_{-3\pi/2}^{3\pi/2}\sin(x)\,dx=0
    \end{equation}
    parce que les deux parties bleues s'annulent avec les deux parties rouges (qui sont comptées comme des aires négatives).
    \begin{center}
       \input{auto/pictures_tex/Fig_JSLooFJWXtB.pstricks}
    \end{center}
\end{example}

%--------------------------------------------------------------------------------------------------------------------------- 
\subsection{Changement de variables dans une intégrale}
%---------------------------------------------------------------------------------------------------------------------------

\begin{example}     \label{EXooNIOZooWxciAC}
    Soit $V$ la région trapézoïdale de sommets $(0,-1)$, $(1,0)$, $(2,0)$, $(0,-2)$, comme à la figure~\ref{LabelFigZTTooXtHkcissLabelSubFigZTTooXtHkci0}. Calculons ensemble l'intégrale double
    \[
    \int_{V}e^{\frac{x+y}{x-y}}\,dV,
    \]
    avec le changement de variable $\psi(x,y)=(x+y,x-y)$. C'est-à-dire que nous considérons les nouvelles variables
    \begin{subequations}
        \begin{numcases}{}
            u=x+y\\
            v=x-y.
        \end{numcases}
    \end{subequations}
    Il faut remarquer d'abord que le changement de variable proposé est dans le mauvais sens. On écrit alors $\phi(u,v)=\psi^{-1}(u,v)=\big((u+v)/2, (u-v)/2\big)$, c'est-à-dire
    \begin{subequations}
        \begin{numcases}{}
            x=\frac{ u+v }{ 2 }\\
            y=\frac{ u-v }{2}.
        \end{numcases}
    \end{subequations}
    La région qui correspond à $V$ est $U$, le trapèze de sommets  $(-1,1)$, $(1,1)$, $(2,2)$ et $(-2,2)$, qu'on voit sur la figure~\ref{LabelFigZTTooXtHkcissLabelSubFigZTTooXtHkci1} et qu'on décrit par
    \[
    U=\{ (u,v)\in\eR^2\,\vert\, 1\leq v\leq 2, \, -v\leq u\leq v\}.
    \]

    % Celui-ci a été supprimée le 17 juillet 2014
    %\ref{LabelFigexamplechangementvariables}
    %\newcommand{\CaptionFigexamplechangementvariables}{Avant et après le changement de variables}
    %\input{auto/pictures_tex/Fig_examplechangementvariables.pstricks}

    %The result is on figure~\ref{LabelFigZTTooXtHkci}. % From file ZTTooXtHkci
    %See also the subfigure~\ref{LabelFigZTTooXtHkcissLabelSubFigZTTooXtHkci0}
    %See also the subfigure~\ref{LabelFigZTTooXtHkcissLabelSubFigZTTooXtHkci1}
    \newcommand{\CaptionFigZTTooXtHkci}{Avant et après le changement de variables}
    \input{auto/pictures_tex/Fig_ZTTooXtHkci.pstricks}

    On observe que $U$ est une région du premier type tandis que $V$ n'est pas du premier ou du deuxième type. Le déterminant de la  matrice  jacobienne de $\psi^{-1}$ est  $J_{\psi^{-1}}$,
    \begin{equation}
     J_{\psi^{-1}}(u,v)= \left\vert\begin{array}{cc}
    \frac{1}{2} & \frac{1}{2} \\
    \frac{1}{2}  & -\frac{1}{2}
    \end{array}\right\vert= -\frac{1}{2}.
    \end{equation}
    On a alors, en utilisant le fait que \( F(x)=a e^{x/a}\) est une primitive de \( f(x)= e^{x/a}\) (proposition \ref{ThoKRYAooAcnTut}) ainsi que le théorème fondamental de l'analyse (théorème \ref{ThoRWXooTqHGbC}),
    \[
    \int_{V}e^{\frac{x+y}{x-y}}\,dV=\int_{U}e^{\frac{u}{v}}\,\frac{1}{2}\,dV=\int_1^2\int_{-v}^{v}e^{\frac{u}{v}}\,\frac{1}{2}\, du\,dv= \frac{3}{4}(e-e^{-1}).
    \]
\end{example}


