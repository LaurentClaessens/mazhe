% This is part of Mes notes de mathématique
% Copyright (c) 2011-2013,2016-2018
%   Laurent Claessens
% See the file fdl-1.3.txt for copying conditions.

%+++++++++++++++++++++++++++++++++++++++++++++++++++++++++++++++++++++++++++++++++++++++++++++++++++++++++++++++++++++++++++ 
\section{Isométriques du cube}
%+++++++++++++++++++++++++++++++++++++++++++++++++++++++++++++++++++++++++++++++++++++++++++++++++++++++++++++++++++++++++++
\label{SecPVCmkxM}
\index{isométrie!espace euclidien!isométries du cube}
\index{groupe!et géométrie!isométries du cube}
Les isométries du cube proviennent de \cite{KXjFWKA}.

\begin{wrapfigure}{r}{5.0cm}
   \vspace{-0.5cm}        % à adapter.
   \centering
   \input{auto/pictures_tex/Fig_MCKyvdk.pstricks}
\end{wrapfigure}
Nous considérons le cube centré en l'origine de \( \eR^3\) et \( G\), le groupe des isométries de \( \eR^3\) préservant ce cube. Nous notons aussi \( G^+\) le sous-groupes de \( G\) constitué des éléments de déterminant positif. Nous notons 
\begin{equation}
    \mD=\{ D_1,\ldots, D_4 \}
\end{equation}
l'ensemble des grandes diagonales, c'est à dire les segments \( [AG]\), \( [EC]\), \( [FD]\), et \( [BH]\). Nous savons que \( G\) préserve les longueurs et que ces segments sont les plus longs possibles à l'intérieur du cube. Donc \( G\) agit sur \( \mD\) parce qu'il ne peut transformer une grande diagonale qu'en une autre grande diagonale. Nous avons donc un morphisme de groupes
\begin{equation}
    \rho\colon G\to S_4.
\end{equation}
Nous montrons ce que morphisme est surjectif en montrant qu'il contient les transpositions. Le groupe \( G\) contient la symétrie axiale passant par le milieu \( M\) de \( [AE]\) et le milieu \( N\) de \( CG\). Il est facile de voir que cette symétrie permute \( [AG]\) avec \( [EC]\). De plus elle laisse \( [FD]\) inchangée. En effet, aussi incroyable que cela paraisse en regardant le dessin, nous avons \( FD\perp MN\), parce qu'en termes de vecteurs directeurs,
\begin{equation}
    \begin{aligned}[]
        \vect{ ON }&=\begin{pmatrix}
            1    \\ 
            -1    \\ 
            0    
        \end{pmatrix}&\vect{ OF }&=\begin{pmatrix}
            1    \\ 
            1    \\ 
            -1    
        \end{pmatrix}.
    \end{aligned}
\end{equation}

Étudions à présent le noyau \( \ker(\rho)\). Si \( f\in\ker(\rho)\) n'est pas l'identité, alors \( f(D_i)=D_i\) pour tout \( i\), mais au moins pour une des diagonales les sommets sont inversés. Quitte à renommer les sommets du cube nous supposons que la diagonale \( [AG]\) est retournée : \( f(A)=G\) et \( f(G)=A\). Regardons où peut partir \( B\) sous l'effet de \( f\). Étant donné que \( f\) préserve les diagonales, \( f(B)\in\{ B,C \}\), mais étant donné que \( f\) est une isométrie, \( d\big( f(B),f(G) \big)=d(B,G)\), et nous concluons que \( f(B)=H\). Donc la diagonale \( [BH]\) est retournée sous l'effet de \( f\). En raisonnant de même, nous voyons que \( f\) retourne toutes les diagonales. Donc les éléments non triviaux de \( \ker(\rho)\) retournent toutes les diagonales; il n'y en a donc qu'un seul et c'est la symétrie centrale :
\begin{equation}
    \ker(\rho)=\{ \id,s_0 \}.
\end{equation}
Le premier théorème d'isomorphisme \ref{ThoPremierthoisomo} nous permet d'écrire le quotient de groupes :
\begin{equation}
    \frac{ G }{ \{ \id,s_0 \} }\simeq S_4.
\end{equation}
Une classe d'équivalence modulo \( \ker(\rho)\) dans \( G\) est donc toujours de la forme \( \{ f,f\circ s_0 \}\). Et vu que \( s_0\) est de déterminant \( -1\), une classe contient toujours exactement un élément de déterminant \( 1\) et un de déterminant \( -1\).

D'autre part \( \ker(\rho)\) est normal dans \( G\) parce que en tant que matrice, \( s_0=-\mtu\), donc les problèmes de commutativité ne se posent pas. L'application
\begin{equation}
    \begin{aligned}
        \varphi\colon \frac{ G }{ \{ \id,s_0 \} }&\to G^+ \\
        [g]&\mapsto \begin{cases}
            g    &   \text{si } \det(g)>0\\
            g\circ s_0    &    \text{sinon}
        \end{cases}
    \end{aligned}
\end{equation}
est un isomorphisme de groupes. Et enfin nous pouvons écrire 
\begin{equation}
    G^+\simeq S_4.
\end{equation}

Nous allons maintenant utiliser le corollaire \ref{CoroGohOZ} pour montrer que \( G=G^+\times_{\sigma}\ker(\rho)\). Les conditions sont remplies :
\begin{itemize}
    \item \( \ker(\rho)\) normalise \( G^+\) parce que \( \ker(\rho)\) ne contient que \( \pm\mtu\).
    \item \( \ker(\rho)\cap G^+=\{ \id \}\).
    \item \( \ker(\rho)G^+=G\) parce que les classes d'équivalence de \( G\) modulo \( \ker(\rho)\) sont composées de \( \{ f,f\circ s_0 \}\).
\end{itemize}
Vu que \( G^+\simeq S_4\) et \( \ker(\rho)\simeq \eZ/2\eZ\) nous pouvons écrire de façon plus brillante que
\begin{equation}
    G\simeq S_4\times_{\sigma}\eZ/2\eZ.
\end{equation}
Mais étant donné que la conjugaison par \( s_0\) est triviale, le produit semi-direct est un produit direct :
\begin{equation}
    G\simeq S_4\times\eZ/2\eZ.
\end{equation}
Il est maintenant du meilleur goût de pouvoir identifier géométriquement ces éléments. Les éléments de \( \eZ/2\eZ=\{ \id,s_0 \}\) ne font pas de mystères. Dans \( S_4\) nous avons les classes de conjugaison des éléments \( \id\), \( (12)\), \( (123)\), \( (1234)\) et \( (12)(34)\) déterminées durant l'exemple \ref{ExVYZPzub}.
\begin{enumerate}
    \item
        L'élément \( (12)\) consiste à permuter deux diagonales et laisser les autres en place. Nous avons déjà vu que c'était une symétrie axiale passant par les milieux de deux côtés opposés. Cela fait \( 6\) axes d'ordre \( 2\).
    \item
        L'élément \( (123)\) fixe une des diagonales. C'est donc la symétrie axiale le long de la diagonale fixée. Par exemple la symétrie d'axe \( (AG)\) fait bouger le point \( B\) de la façon suivante :
        \begin{equation}
            B\to D\to E\to B.
        \end{equation}
        C'est une rotation est d'angle \( \frac{ 2\pi }{ 3 }\). Cela sont \( 8\) rotations d'ordre \( 3\).

        Notons à ce propos que la différence entre \( (234)\) et \( (243)\) est que la première fait une rotation d'angle \( 2\pi/3\) tandis que la seconde fait une rotation d'angle \( -2\pi/3\).

    \item
        L'élément \( (1234)\) ne maintient aucune des diagonales et est d'ordre \( 4\). C'est donc la rotation d'angle \( \pi/2\) ou \( -\pi/2\) autour de l'axe passant par les milieux de deux faces opposées. Il y en a \( 6\) comme ça (\( 3\) paires de faces puis pour chaque il y a \( \pi/2\) et \( -\pi/2\)), et ça tombe bien \( 6\) est justement la taille de la classe de conjugaison de \( (1234)\) dans \( S_4\).

    \item
        L'élément \( (12)(34)\) est le carré de la précédente\footnote{En fait c'est \( (13)(24)\), le carré de la précédente, mais c'est la même classe de conjugaison.}, c'est à dire les rotations d'angle \( \pi\) autour des mêmes axes. Cela fait \( 3\) éléments d'ordre \( 2\).
        
\end{enumerate}

