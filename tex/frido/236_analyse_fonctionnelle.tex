% This is part of Mes notes de mathématique
% Copyright (c) 2011-2024
%   Laurent Claessens
% See the file fdl-1.3.txt for copying conditions.

%+++++++++++++++++++++++++++++++++++++++++++++++++++++++++++++++++++++++++++++++++++++++++++++++++++++++++++++++++++++++++++
\section{Espace de Schwartz}
%+++++++++++++++++++++++++++++++++++++++++++++++++++++++++++++++++++++++++++++++++++++++++++++++++++++++++++++++++++++++++++

Pour un multiindice \( \alpha=(\alpha_1,\ldots, \alpha_d)\in \eN^d\), nous notons
\begin{equation}
	\partial^{\alpha}\varphi=\partial_{x_1}^{\alpha_1}\ldots\partial_{x_d}^{\alpha_d}\varphi
\end{equation}
pour peu que la fonction \( \varphi\) soit \( | \alpha |=\alpha_1+\cdots +\alpha_d\) fois dérivable.

\begin{definition}  \label{DefHHyQooK}
	Soit \( \Omega\subset\eR^d\). L'\defe{espace de Schwartz}{espace!de Schwartz} \( \swS(\Omega)\) est le sous-ensemble de \(  C^{\infty}(\Omega)\) des fonctions dont toutes les dérivées décroissent plus vite que tout polynôme :
	\begin{equation}
		\swS(\Omega)=\big\{   \varphi\in C^{\infty}(\Omega)\tq\forall \alpha,\beta\in \eN^d, p_{\alpha,\beta}(\varphi)<\infty   \big\}
	\end{equation}
	où nous avons considéré
	\begin{equation}    \label{EqOWdChCu}
		p_{\alpha,\beta}(\varphi)=\sup_{x\in \Omega}| x^{\beta}(\partial^{\alpha}\varphi)(x) |=\| x^{\beta}\partial^{\alpha}\varphi \|_{\infty}.
	\end{equation}
\end{definition}

Pour simplifier les notations (surtout du côté de Fourier), nous allons parfois écrire \( M_i\varphi\)\nomenclature[Y]{\( M_i\varphi\)}{La fonction \( x\mapsto x_i\varphi(x)\)} pour la fonction \( x\mapsto x_i\varphi(x)\).

\begin{example}
	La fonction \(  e^{-x^2}\) est une fonction à décroissance rapide sur \( \eR\).
\end{example}

\begin{definition}
	Une fonction \( f\colon \eR^d\to \eC\) est dite à \defe{décroissance rapide}{fonction!à décroissance rapide} si elle décroît plus vite que n'importe quel polynôme. Plus précisément, si pour tout polynôme \( Q\), il existe un \( r>0\) tel que \(  | f(x) |<\frac{1}{ | Q(x) | } \) pour tout \( \| x \|\geq r\).
\end{definition}

\begin{proposition} \label{PropCSmzwGv}
	Une fonction Schwartz est à décroissance rapide.
\end{proposition}

\begin{proof}
	Nous commençons par considérer un polynôme \( P\) donné par
	\begin{equation}
		P(x)=\sum_kc_kx^{\beta_k}
	\end{equation}
	où les \( \beta_k\) sont des multiindices, les \( c_k\) sont des constantes et la somme est finie. Nous avons la majoration
	\begin{equation}
		\sup_{x\in \eR^d}| \varphi(x)P(x) |\leq\sum_k\sup_x\big| c_k\varphi(x) x^{\beta_k} \big|\leq\sum_k| c_k |p_{0,\beta_k}(\varphi)<\infty.
	\end{equation}
	Nous allons noter \( M_P\) la constante \( \sum_k| c_k |p_{0,\beta_k}(\varphi)\), de sorte que pour tout \( x\in \eR^d\) nous ayons \( | \varphi(x)P(x) |\leq M_P\) et donc
	\begin{equation}
		| \varphi(x) |\leq \frac{ M_P }{ | P(x) | }=\frac{1}{ | \frac{1}{ M_P }P(x) | }.
	\end{equation}
	Notons que cette inégalité est a fortiori correcte pour les \( x\) sur lesquels \( P\) s'annule.

	Soit maintenant un polynôme \( Q\). Nous considérons le polynôme \( P(x)=\| x \|Q(x)\). Étant de plus haut degré, pour toute constante \( C\) il existe un rayon \( r_C\) tel que \( | P(x) |\geq C| Q(x) |\) pour tout \( | x |\geq r_C\). En particulier pour \( | x |\geq r_{M_P}\) nous avons
	\begin{equation}
		| P(x) |\geq M_P| Q(x) |
	\end{equation}
	et donc, pour ces \( x\),
	\begin{equation}
		| \varphi(x) |\leq \frac{1}{ | \frac{1}{ M_P }P(x) | }\leq \frac{1}{ | Q(x) | }.
	\end{equation}
	La première inégalité est valable pour tout \( x\), et la seconde pour \( \| x \|\geq r_{M_P}\).
\end{proof}

\begin{corollary}[\cite{MonCerveau}]        \label{CORooZFPSooHCFUSH}
	Soit \( \varphi\) une fonction Schwartz sur \( \eR^m\times \eR^n\). Alors la fonction
	\begin{equation}
		y\mapsto \sup_{x\in \eR^n}| \varphi(x,y) |
	\end{equation}
	est intégrable.
\end{corollary}

\begin{proof}
	Soit un polynôme \( Q\) en la variable \( y\). Par la proposition~\ref{PropCSmzwGv}, il existe \( r>0\) tel que
	\begin{equation}
		| \varphi(x,y) |<\frac{1}{ Q(y) }
	\end{equation}
	pour tout \( \| (x,y) \|>r\). A fortiori l'inégalité tient pour tout \( | y |>r\). Donc
	\begin{equation}
		\int_{\eR^m}\sup_{x\in \eR^n}| \varphi(x,y) |dy=\int_{\| y \|\leq r}\sup_{x}| \varphi(x,y) |dy+\int_{ \| y \|>r  }\sup_{x}| \varphi(x,y) |dy.
	\end{equation}
	La première intégrale est bornée par \( \Vol\big( B(0,r) \big)\| \varphi \|_{\infty}\) tandis que la seconde est bornée par l'intégrale de \( \frac{1}{ Q(y) }\). En prenant \( Q\) de degré suffisamment élevé en toutes les composantes de \( y\) nous avons intégrabilité.
\end{proof}

%---------------------------------------------------------------------------------------------------------------------------
\subsection{Topologie}
%---------------------------------------------------------------------------------------------------------------------------

\begin{lemmaDef}        \label{LEMDEFooZEFVooMMmiBr}
	Les \( p_{\alpha,\beta}\) donnés par l'équation \eqref{EqOWdChCu} ci-dessus sont des seminormes\footnote{Définition~\ref{DefPNXlwmi}.}. La topologie considérée sur \( \swS(\eR^d)\) est celle des seminormes \( p_{\alpha,\beta}\).
\end{lemmaDef}
%TODO : une preuve pour égayer la galerie.

\begin{normaltext}      \label{NORMooVQESooRwJShl}
	Nous avons un enchainement de résultats qui nous aident à prouver la continuité d'une application \( T\colon \swS(\eR^d)\to X\).
	\begin{enumerate}
		\item
		      La topologie de \( \swS(\eR^d)\) est donnée par une famille dénombrable de seminormes. Donc la proposition~\ref{PROPooMJEQooHtIyeX} nous dit que \( \swS(\eR^d)\) est métrisable.
		\item
		      La proposition~\ref{PROPooKNVUooMbLZoy} nous dit alors que si \( X\) est métrique, toute application séquentiellement continue \( T\colon \swS(\eR^d)\to X\) est continue.
		\item
		      Donc si \( X\) est métrique, il suffit de prouver que pour \( f_n\stackrel{\swS(\eR^d)}{\longrightarrow}0\) nous avons \( T(f_n)\stackrel{X}{\longrightarrow} 0\) où \( f_n\colon \swS(\eR^d)\to X\). Dans les cas usuels, \( T\) sera une distribution et \( X=\eC\).
		\item
		      En vertu de la proposition~\ref{PropQPzGKVk}, la convergence \( f_n\stackrel{\swS(\eR^d)}{\longrightarrow}0\) signifie que pour tout choix de multiindice \( \alpha\) et \( \beta\),  \( p_{\alpha,\beta}(f_n)\to 0\), c'est-à-dire
		      \begin{equation}        \label{EQooPUJPooNbtNFh}
			      \| x^{\beta}\partial^{\alpha}f_n \|_{\infty}\to 0.
		      \end{equation}
		\item
		      Et enfin, la technique pour montrer que \( T\colon \swS(\eR^d)\to \eC\) est continue est de montrer que sous l'hypothèse d'avoir \eqref{EQooPUJPooNbtNFh} pour tout choix de \( \alpha\) et \( \beta\), nous avons \( T(f_n)\to 0\) dans \( \eC\).
	\end{enumerate}
\end{normaltext}

\begin{lemma}[\cite{OEVAuEz}]   \label{LemRJhCbkO}
	La topologie sur \( \swS(\eR^d)\) est donnée aussi par les seminormes
	\begin{equation}
		q_{n,m}=\max_{| \alpha |\leq n}\sup_{x\in \eR^d}\big( 1+\| x \| \big)^m\big| \partial^{\alpha}\varphi(x) \big|.
	\end{equation}
	Autrement dit, une suite \( \varphi_n\stackrel{\swS(\eR^d)}{\to}0\) si et seulement si \( q_{n,m}(\varphi)\to 0\) pour tout \( n\) et \( m\).
\end{lemma}
Le fait que les \( q_{n,m}(\varphi)\) restent bornés est la proposition~\ref{PropCSmzwGv}. Cependant ce lemme est plus précis parce qu'en disant seulement que \( \varphi\) est majoré par des polynôme, nous ne disons pas que les polynômes correspondants aux \( \varphi_n\) tendent vers zéro si \( \varphi_n\stackrel{\swS}{\to}0\). Et d'ailleurs on ne sait pas très bien ce que signifierait \( P_n\to 0\) pour une suite de polynômes.

\begin{proposition}     \label{PropGNXBeME}
	Pour \( p\in\mathopen[ 1 , \infty \mathclose]\), l'espace \( \swS(\eR^d)\) s'injecte continument dans \( L^p(\eR^d)\).
\end{proposition}

\begin{proof}
	L'injection dont nous parlons est l'identité ou plus précisément l'identité suivie de la prise de classe. Il faut vérifier que cela est correct et continu, c'est-à-dire d'abord qu'une fonction à décroissance rapide est bien dans \( L^p\) et ensuite que si \( f_n\stackrel{\swS}{\to}0\), alors \( f_n\stackrel{L^p}{\to}0\).

	Commençons par \( p=\infty\). Alors \( \| f_n \|_{\infty}=p_{0,0}(f_n)\to 0\) parce que si \( f_n\stackrel{\swS}{\to}0\), alors en particulier \( p_{0,0}(f_n)\to 0\).

	Au tour de \( p<\infty\) maintenant. Nous savons qu'en dimension \( d\), la fonction
	\begin{equation}
		x\mapsto \frac{1}{ (1+\| x \|)^s }
	\end{equation}
	est intégrable dès que \( s>d\).
	%TODOooWSXPooQrnDEU : il faudrait une petite preuve de ça.
	Pour toute valeur de \( m\) nous avons
	\begin{equation}
		\| \varphi \|_p^p=\int_{\eR^d}| \varphi(x) |^pdx=\int_{\eR^d}\frac{ \big|    (1+\| x \|)^m\varphi(x)   \big|^p }{ \big( 1+\| x \| \big)^{mp} }\leq\int_{\eR^d}\frac{q_{0,m}(\varphi)^p}{ \big( 1+\| x \| \big)^{mp} }.
	\end{equation}
	En choisissant \( m\) de telle sorte que \( mp>d\), nous avons convergence de l'intégrale et donc \( \| \varphi \|_p<\infty\). Nous retenons que
	\begin{equation}    \label{EqVWfEFMk}
		\| \varphi \|_p^p\leq Cq_{0,m}(\varphi)^p
	\end{equation}
	pour une certaine constante \( C\) et un bon choix de \( m\).

	Ceci prouve que \( \swS(\eR^d)\subset L^p(\eR^d)\). Nous devons encore vérifier que l'inclusion est continue. Si \( \varphi_n\stackrel{\swS}{\to}0\), alors en particulier nous avons \( q_{0,m}(\varphi_n)\to 0\) par le lemme~\ref{LemRJhCbkO}. Par conséquent la majoration \eqref{EqVWfEFMk} nous dit que \( \| \varphi_n \|_p\to 0\) également.

\end{proof}
En résumé, si \( \varphi_n\stackrel{\swS(\eR^d)}{\to}\varphi\) alors \( \varphi_n\stackrel{L^p}{\to}\varphi\).

\begin{theorem}[\cite{MesIntProbb}]      \label{ThoRWEoqY}
	Soit \( \mu\) une mesure sur les boréliens de \( \eR^n\) finie sur les compacts. Alors \( \swD(\eR^n)\) est dense dans \( L^1(\eR^n,\Borelien(\eR^n),\mu)\).
\end{theorem}
\index{densité!de \( \swD(\eR^n)\) dans \( L^1(\eR^n)\)}

\begin{proposition}[\cite{ooIKXSooRlKVJR}]      \label{PROPooJNQZooIRbJei}
	La partie \( \swD(\eR^d)\) est dense dans \( \swS(\eR^d)\).
\end{proposition}

\begin{proof}
	Soit \( f\in \swS(\eR^d)\), et \( \phi\), une fonction de \( \swD(\eR^d)\) telle que \( \phi(x)=1\) pour \(| x |\leq 1 \) (l'existence de telles fonctions est discutée en~\ref{subsecOSYAooXXCVjv}). Soit aussi \( \phi_k(x)=\phi(x/k)\). Nous posons
	\begin{equation}
		f_k(x)=\phi_k(x)f(x),
	\end{equation}
	et nous allons prouver que pour tout multiindice \( \alpha\) et \( \gamma\),
	\begin{equation}
		p_{\alpha,\gamma}(f_k-f)=\| x^{\gamma}\partial^{\alpha}(f_k-f)  \|_{\infty}\to 0.
	\end{equation}
	Pour cela nous allons noter \(  \beta\leq \alpha  \) lorsque \( \beta\) est un multiindice contenu dans \( \alpha\). En utilisant la dérivée du produit nous avons
	\begin{subequations}
		\begin{align}
			(\partial^{\alpha}f_k)(x) & =\sum_{\beta\leq \alpha}(\partial^{\alpha-\beta}\phi_k)(x)\partial^{\beta}f(x)                                                         \\
			                          & =\sum_{\beta\leq \alpha}k^{-| \alpha-\beta |}(\partial^{\alpha-\beta}\phi)(x/k)(\partial^{\beta}f)(x)                                  \\
			                          & =\sum_{\beta< \alpha}k^{-| \alpha-\beta |}(\partial^{\alpha-\beta}\phi)(x/k)(\partial^{\beta}f)(x) + \phi(x/k)(\partial^{\alpha}f)(x).
		\end{align}
	\end{subequations}
	Nous devons donc étudier et majorer
	\begin{equation}
		\begin{aligned}[]
			\sup_{x\in \eR^d}| x^{\gamma}\partial^{\alpha}(f_k-f) | & \leq \sup\big| x^{\gamma}  \sum_{\beta< \alpha}k^{-| \alpha-\beta |}(\partial^{\alpha-\beta}\phi)(x/k)(\partial^{\beta}f)(x)  \big| \\
			                                                        & \quad+\sup \big| x^{\gamma}\big( \phi(x/k)-1 \big)(\partial^{\alpha}f)(x) \big|                                                     \\
		\end{aligned}
	\end{equation}
	En ce qui concerne le second terme, soit \( \epsilon>0\), vu que \( f\) est Schwartz, il existe \( R\) tel que
	\begin{equation}
		| x^{\gamma}(\partial^{\alpha}f)(x) |<\epsilon
	\end{equation}
	dès que \( \| x \|>R\). En prenant \( k>R\),
	\begin{equation}
		| x^{\gamma}(\partial^{\alpha}f)(x) |\begin{cases}
			=0            & \text{si } \| x \|<R         \\
			\leq \epsilon & \text{si } \| x \|>R\text{.}
		\end{cases}
	\end{equation}
	En ce qui concerne le premier terme,
	\begin{subequations}
		\begin{align}
			\sup_{x\in \eR^d}\big| x^{\gamma} & \sum_{\beta<\alpha}k^{-|\alpha-\beta |}(\partial^{\alpha-\beta}\phi)(x/k)(\partial^{\beta}f)(x) \Big|                       \\
			                                  & \leq \frac{1}{ k }\sup_{x}\big| \sum_{\beta<\alpha}(\partial^{\alpha-\beta}\phi)(x/k)(x^{\gamma}\partial^{\beta}f)(x) \big| \\
			                                  & = \frac{1}{ k }\sup_{x}\big| \sum_{\beta<\alpha}(\partial^{\alpha-\beta}\phi)(x/k)  p_{\beta,\gamma}(f)   \big|
		\end{align}
	\end{subequations}
	La somme ne contient qu'un nombre fini de \( \beta\) différents, donc nous pouvons considérer un nombre \( K\) qui majore tous les \( p_{\beta,\gamma}(f)\) en même temps. La partie avec \( \phi\) peut être majorée par \( \| \partial^{\alpha-\beta}\phi \|_{\infty}\) (qui est fini) dont nous pouvons prendre le maximum sur \(\beta<\alpha\). Toute l'expression dans la somme est donc majorée par un nombre qui ne dépend ni de \( x\) ni de \( \beta\). Vu que la somme est finie, elle est majorée par ce nombre multiplié par le nombre de termes dans la somme et au final
	\begin{equation}
		\sup_{x\in \eR^d}\big| x^{\gamma}\sum_{\beta<\alpha}k^{-|\alpha-\beta |}(\partial^{\alpha-\beta}\phi)(x/k)(\partial^{\beta}f)(x) \Big|\leq \frac{ K' }{ k }.
	\end{equation}
	La limite \( k\to \infty\) ne fait alors plus de doutes.
\end{proof}

\begin{remark}
	Vu la topologie de \( \swS(\eR^d)\) (définition~\ref{LEMDEFooZEFVooMMmiBr}), la convergence \( f_k\stackrel{\swS(\eR^d)}{\longrightarrow}f\) peut être exprimée par le fait que pour tout \( k,l\),
	\begin{equation}
		t^kf_n^{(l)}\stackrel{unif}{\longrightarrow}t^kf^{(l)}.
	\end{equation}
	C'est-à-dire convergence uniforme de toutes les dérivées multipliées par n'importe quel polynôme.
\end{remark}

%---------------------------------------------------------------------------------------------------------------------------
\subsection{Produit de convolution}
%---------------------------------------------------------------------------------------------------------------------------

\begin{proposition}[Stabilité de Schwartz par convolution\footnote{Définition~\ref{DEFooHHCMooHzfStu}.} \cite{CXCQJIt}]     \label{PROPooUNFYooYdbSbJ}
	Si \( \varphi\in L^1(\eR^d)\) et \( \psi\in\swS(\eR^d)\), alors \( \varphi * \psi\in \swS(\eR^d)\).
\end{proposition}

\begin{proof}
	Nous devons prouver que
	\begin{equation}
		p_{\alpha,\beta}(\varphi*\psi)=\sup_{x\in \eR^d}| x^{\beta}(\partial^{\alpha}(\varphi*\psi))(x) |
	\end{equation}
	est borné pour tout multiindices \( \alpha\) et \( \beta\). En appliquant \( | \alpha |\) fois la proposition~\ref{PropHNbdMQe}, nous mettons toutes les dérivées sur \( \psi\) : \( \partial^{\alpha}(\varphi*\psi)=(\varphi*\partial^{\alpha}\psi)\). Cela étant fait, nous majorons
	\begin{subequations}
		\begin{align}
			\big| x^{\beta}(\varphi*\partial^{\alpha}\psi)(x) \big| & \leq| x^{\beta} |\int_{\eR^d} |\varphi(y)|\underbrace{\big| (\partial^{\alpha}\psi)(x-y)\big|}_{\leq\| \partial^{\alpha}\psi \|_{\infty}} dy \big| \\
			                                                        & \leq | x^{\beta} |  \| \partial^{\alpha}\psi \|_{\infty}\int_{\eR^d}| \varphi(y) |dy                                                               \\
			                                                        & \leq p_{\alpha,\beta}(\psi)\| \varphi \|_{_{L^1}}.
		\end{align}
	\end{subequations}
	Par conséquent, \( p_{\alpha,\beta}(\varphi*\psi)\leq \| \varphi \|_{L^1}p_{\alpha,\beta}(\psi)<\infty\).
\end{proof}

%+++++++++++++++++++++++++++++++++++++++++++++++++++++++++++++++++++++++++++++++++++++++++++++++++++++++++++++++++++++++++++
\section{Théorème de Montel}
%+++++++++++++++++++++++++++++++++++++++++++++++++++++++++++++++++++++++++++++++++++++++++++++++++++++++++++++++++++++++++++

\begin{theorem}[Montel\cite{KXjFWKA}]   \label{ThoXLyCzol}
	Soient \( \Omega\) un ouvert de \( \eC\) et \( \mF\) une famille de fonctions holomorphes sur \( \Omega\), uniformément bornée sur tout compact de \( \Omega\). Alors de toute suite dans \( \mF\) nous pouvons extraire une sous-suite convergeant uniformément sur tout compact de \( \Omega\).
\end{theorem}
\index{théorème!Montel}
\index{compacité!utilisation!théorème de Montel}
\index{suite!de fonctions!théorème de Montel}
\index{fonction!holomorphe!théorème de Montel}

\begin{proof}

	\begin{subproof}
		\spitem[Un ensemble équicontinu]

		Nous commençons par prendre une suite de compacts dans \( \Omega\) comme dans le lemme~\ref{LemGDeZlOo}, et une suite \( \delta_n\) de réels strictement positifs tels que
		\begin{equation}
			B(z,2\delta_n)\subset K_{n+1}
		\end{equation}
		pour tout \( z\in K_n\). Soient \( x,y\in K_n\) tels que \( | x-y |<\delta_n\); nous notons \( \partial B(x,2\delta_n)\) le cercle de rayon \( 2\delta_n\) autour de \( x\), parcouru dans le sens positif. La formule de Cauchy~\ref{EqPzUABM} nous donne
		\begin{equation}
			f(x)-f(y)=\frac{1}{ 2\pi i }\int_{\partial B}\left( \frac{ f(\xi) }{ \xi-x }-\frac{ f(\xi) }{ \xi-y } \right)d\xi
			=\frac{ x-y }{ 2\pi i }\int_{\partial B}\frac{ f(\xi) }{ (\xi-x)(\xi-y) }d\xi
		\end{equation}
		Nous majorons ça par
		\begin{equation}
			\big| f(x)-f(y) \big|\leq\frac{ | x-y | }{ 2\pi }\int_{\partial B}\frac{ | f(\xi) | }{ 2\delta_n^2 }d\xi\leq \frac{ | x-y | }{ \delta_n }M_n.
		\end{equation}
		Justifications :
		\begin{itemize}
			\item
			      \( | \xi-x |=2\delta_n\) et \( | \xi-y |\geq \delta_n\) parce que \( \xi\) est au mieux sur le rayon passant par \( x\) et \( y\).
			\item
			      \( | f(\xi) |\leq M_n\) où \( M_n\) est la borne uniforme de \( \mF\) sur le compact \( K_n\).
			\item
			      Nous avons aussi fini par calculer l'intégrale dans laquelle il ne restait plus rien, ça a donné la circonférence du cercle de rayon \( 2\delta_n\).
		\end{itemize}
		Jusqu'à présent nous avons prouvé que l'ensemble
		\begin{equation}
			\mF_n=\{ f|_{K_n}\tq f\in\mF \}
		\end{equation}
		est équicontinu. Il est aussi équiborné par hypothèse.

		\spitem[Application du théorème d'Ascoli]

		L'ensemble \( \mF_n\) vérifie les hypothèses du théorème d'Ascoli~\ref{ThoKRbtpah}. Donc l'ensemble \( \mF_n\) est relativement compact dans \( C(K_n,\eC)\) pour la norme uniforme. Autrement dit l'ensemble \( \bar\mF\) est compact et si nous avons une suite de fonctions dans \( \mF_n\), il existe une sous-suite convergeant dans \( \bar\mF_n\), c'est-à-dire uniformément. Autrement dit il existe une fonction strictement croissante \( \varphi\colon \eN\to \eN\) telle que la suite \( k\mapsto f_{\varphi(k)}\) converge uniformément sur \( K_n\). La limite n'est cependant pas spécialement dans \( \mF_n\).

		\spitem[L'argument diagonal]

		La suite \( k\mapsto f_{\varphi_1\circ\ldots\varphi_k(k)}\) converge uniformément sur tous les \( K_n\). Si \( K\) est un compact de \( \Omega\), alors les petites propriétés sympas du lemme~\ref{LemGDeZlOo} nous disent que \( K\subset \Int(K_m)\) pour un certain \( m\). Ladite suite convergeant uniformément sur \( K_m\), elle converge uniformément sur \( K\) et nous avons montré la convergence uniforme sur tout compact de \( \Omega\).

	\end{subproof}
\end{proof}

\begin{corollary}[\cite{KXjFWKA}]
	Soient \( \Omega\) un ouvert connexe borné de \( \eC\) et \( a\in \Omega\). Soit \( f\) holomorphe sur \( \Omega\) telle que \( f(a)=a\) et \( | f'(a) |<1\).

	Alors de \( (f^n)\) on peut extraire une sous-suite convergeant uniformément sur tout compact de \( \Omega\) vers la fonction constante \( a\).
\end{corollary}
\index{prolongement!analytique!utilisation}

\begin{proof}
	Nous considérons un voisinage de \( a\) inclus dans \( \Omega\); sachant que \( | f(a) |<1\), nous trouvons un voisinage encore plus petit de \( a\) sur lequel \( | f'(z) |<1\).  Soit donc \( r\) tel que \( \overline{ B(a,r) }\subset \Omega\) et tel que \( | f'(z) |<1\) sur \( \overline{ B(a,r) }\). Étant donné que \( f'(z)\) est continue sur le compact \( \overline{ B(a,r) }\), nous en prenons le maximum \( \lambda\) (qui est strictement inférieur à \( 1\)) et nous avons au final
	\begin{equation}
		| f'(z) |\leq \lambda< 1
	\end{equation}
	pour tout \( z\in \overline{ B(a,r) }\). Le théorème des accroissements finis~\ref{val_medio_2} nous dit que
	\begin{equation}
		\big| f(z)-a \big|\leq \lambda| z-a |
	\end{equation}
	pour tout \( z\in\overline{ B(a,r) }\). C'est ici que nous utilisons l'hypothèse de convexité de \( \Omega\). Nous montrons alors par récurrence que
	\begin{equation}    \label{EqIQUzKpg}
		\big| f^n(z)-a \big|\leq \lambda^n| z-a |\leq \lambda^nr\leq r.
	\end{equation}
	L'ensemble \( A=\{ f^n\tq n\geq 1 \}\) est donc uniformément borné sur \( \overline{ B(a,r) }\) par \( a+r\). Autre manière de le dire : pour tout \( z\in\overline{ B(a,r) }\) nous avons
	\begin{equation}
		f^n(z)\in\overline{ B(a,r) }.
	\end{equation}
	La suite \( (f^n)\) est donc uniformément bornée sur tout compact de \( B(a,r)\). Le théorème de Montel~\ref{ThoXLyCzol} nous indique que l'on peut extraire une sous-suite convergente uniformément sur tout compact. Au vu de \eqref{EqIQUzKpg} cette convergence ne peut avoir lieu que vers une fonction \( g\) qui vaut la constante \( a\) sur \( B(a,r)\).

	D'autre part la fonction \( g\) est holomorphe en tant que limite uniforme de fonctions holomorphes, théorème~\ref{ThoArYtQO}. Or une fonction holomorphe constante sur un ouvert est constante sur tout son domaine d'holomorphie (principe d'extension analytique, théorème~\ref{ThoAVBCewB}).
\end{proof}


%+++++++++++++++++++++++++++++++++++++++++++++++++++++++++++++++++++++++++++++++++++++++++++++++++++++++++++++++++++++++++++
\section{Espaces de Bergman}
%+++++++++++++++++++++++++++++++++++++++++++++++++++++++++++++++++++++++++++++++++++++++++++++++++++++++++++++++++++++++++++

Source : \cite{ytMOpe}.

Soit \( \Omega\) un borné dans \( \eC\) et \( D\) le disque unité ouvert de \( \eC\).

\begin{definition}
	L'\defe{espace de Bergman}{espace!de Bergman}\index{Bergman (espace)} sur \( \Omega\), noté \( A^2(\Omega)\)\nomenclature[Y]{\( A^2(\Omega)\)}{espace de Bergman} est l'espace des fonctions holomorphes sur \( \Omega\) qui sont en même temps dans \( L^2(\Omega)\).
\end{definition}
Nous mettons sur \( A^2(\Omega)\) le produit scalaire usuel hérité de \( L^2\) :
\begin{equation}
	\langle f, g\rangle =\int_{\Omega}f(z)\overline{ g(z) }dz.
\end{equation}

\begin{lemma}   \label{LemIZxKfB}
	Soient un compact \( K\subset \Omega\) et une fonction \( f\in A^2(\Omega)\). Alors
	\begin{equation}
		\max_{z\in K}| f(z) |\leq \frac{1}{ \sqrt{\pi} }\frac{1}{ d(K,\partial \Omega) }\| f \|_2.
	\end{equation}
\end{lemma}

\begin{proof}
	Soient \( a\in \Omega\) et \( r>0\) tels que \( B(a,r)\subset\Omega\). Nous considérons aussi \( \rho\leq r\). La formule de Cauchy \eqref{EqPzUABM} nous donne
	\begin{equation}
		f(a)=\frac{1}{ 2\pi i }\int_{B(a,\rho)}\frac{ f(\xi) }{ \xi-a }d\xi=\frac{1}{ 2\pi }\int_0^{2\pi}f(a+\rho e^{i\theta})d\theta
	\end{equation}
	où nous avons utilisé le chemin \( \gamma(\theta)=a+\rho e^{i\theta}\), \( \gamma'(\theta)=i\rho e^{i\theta}\) et \( \rho=| \xi-a |\). Maintenant une astuce est d'écrire
	\begin{equation}
		\frac{ r^2 }{2}f(a)=\int_0^rf(a)\rho d\rho,
	\end{equation}
	et d'y substituer la valeur de \( f(a)\) que nous venons de calculer :
	\begin{subequations}
		\begin{align}
			\frac{ r^2 }{2}f(a) & =\int_0^r\frac{1}{ 2\pi }\int_0^{2\pi}f(a+\rho e^{i\theta})d\theta\rho d\rho                                       \\
			                    & =\frac{1}{ 2\pi }\int_{B(a,r)}f(z)dz                                         & \text{passage aux polaires}         \\
			                    & =\frac{1}{ 2\pi }\langle 1, f\rangle_B                                       & \text{produit scalaire sur } B(a,r) \\
			                    & \leq\frac{1}{ 2\pi }\sqrt{\langle 1, 1\rangle_B\langle f, f\rangle_B }
		\end{align}
	\end{subequations}
	Nous avons donc
	\begin{equation}
		r^2f(a)\leq \frac{1}{ \pi }\sqrt{\langle 1, 1\rangle_B\langle f, f\rangle_B},
	\end{equation}
	et donc
	\begin{equation}
		\pi r^2 f(a)\leq \sqrt{\pi r^2}\| f \|_2,
	\end{equation}
	parce que \( \langle f, f\rangle_B\leq \| f \|_2^2\). En effet le produit scalaire \( \| . \|_2\) est donné par une intégrale sur \( \Omega\) alors que \( B(a,r)\subset \Omega\) et que la fonction qu'on y intègre est positive (c'est \( | f(z) |^2\)). En simplifiant,
	\begin{equation}
		f(a)\leq \frac{1}{ \sqrt{\pi}r }\| f \|_2.
	\end{equation}
	Mais \( r\) a été choisi pour avoir \( B(a,r)\subset\Omega\), donc \( r\leq d(a,\partial \Omega)\) et
	\begin{equation}
		| f(a) |\leq \frac{1}{ d(a,\partial\Omega)\sqrt{\pi} }\| f \|_2.
	\end{equation}

	Maintenant si nous prenons \( a\in K\), nous avons encore la minoration \( d(a,\partial K)\leq d(a,\partial \Omega)\) et donc
	\begin{equation}
		| f(a) |\leq\frac{1}{ d(a,\partial K)\sqrt{\pi} }\| f \|_2.
	\end{equation}

\end{proof}

\begin{theorem}
	Soit \( \Omega\) un ouvert de \( \eC\).
	\begin{enumerate}
		\item
		      L'espace \( A^2(\Omega)\) est un espace de Hilbert.
		\item
		      Si \( D\) est la boule unité dans \( \eC\), une base hilbertienne de \( A^2(D)\) est donnée par les fonctions
		      \begin{equation}
			      e_n(z)=\sqrt{\frac{ n+1 }{ \pi }}z^n
		      \end{equation}
		      pour \( n\geq 0\).
	\end{enumerate}
\end{theorem}

\begin{proof}
	Nous commençons par montrer que \( A^2(\Omega)\) est complet. Pour cela nous considérons une suite de Cauchy \( (f_n)\) dans \( A^2(\Omega)\) et un compact \( K\subset \Omega\). Nous savons par le lemme~\ref{LemIZxKfB} que
	\begin{equation}
		\max_{z\in K}\big| f_n(z)-f_m(z) \big|\leq \frac{1}{ \sqrt{\pi}d(K,\partial\Omega) }\| f_n-f_m \|_2.
	\end{equation}
	Donc \( f_n\) converge uniformément sur \( K\). Par le théorème de Weierstrass~\ref{ThoArYtQO}, la fonction \( f\) est holomorphe. Il existe donc une fonction holomorphe \( f\) qui est limite uniforme sur tout compact de \( \Omega\) de la suite \( (f_n)\).

	Mais \( L^2(\Omega)\) étant complet, la suite \( (f_n)\) a une limite \( g\in L^2(\Omega)\). Ce que nous voudrions faire est prouver que \( f=g\). Notons que tel quel, ce n'est pas vrai parce que \( f\) est une vraie fonction alors que \( g\) est une classe. Ce que nous enseigne la proposition~\ref{PropWoywYG} est qu'il existe une sous-suite (qu'on note \( (g_n)\)) qui converge vers \( g\) presque partout. Dans cette dernière phrase, \( g_n\) et \( g\) sont de vraies fonctions, des représentants des classes dans \( L^2\).

	Nous déduisons que \( f=g\) presque partout (ici \( f\) et \( g\) sont les fonctions) parce que la sous-suite converge uniformément vers \( f\) en même temps que presque partout vers \( g\). Donc \( f=g\) dans \( L^2(\Omega)\) (ici \( f\) et \( g\) sont les classes). Donc \( f\in L^2(\Omega)\) et l'espace \( A^2(\Omega)\) est de Hilbert.

	Il nous faut encore prouver que \( (e_n)_{n\geq 0}\) est une base orthonormale. En ce qui concerne les produits scalaires,
	\begin{subequations}
		\begin{align}
			\langle e_m, e_n\rangle & =\sqrt{\frac{ (m+1)(n+1) }{ \pi }}\int_Dz^n\overline{ z^m }dz                                                                \\
			                        & =\sqrt{\frac{ (m+1)(n+1) }{ \pi^2 }}\int_0^1\rho\,d\rho\int_0^{2\pi}d\theta \rho^{m+n} e^{i\theta(n-m)}                      \\
			                        & =\sqrt{\frac{ (m+1)(n+1) }{ \pi^2 }}\frac{1}{ m+n+2 }\underbrace{\int_{0}^{2\pi} e^{i\theta(n-m)}d\theta}_{2\pi \delta_{mn}} \\
			                        & =\sqrt{\frac{ (n+1)^2 }{ \pi^2 }}\frac{1}{ 2n+2 }2\pi \delta_{nm}                                                            \\
			                        & =\delta_{nm}.
		\end{align}
	\end{subequations}
	Donc les fonctions données sont bien orthonormales. Nous devons montrer qu'elles sont denses dans \( A^2(D)\). Soit \( f\in A^2(D)\) et \( c_n(f)=\langle f, e_n\rangle \); nous allons montrer que
	\begin{equation}
		\| f \|_2^2=\sum_{n=0}^{\infty}| \langle f, e_n\rangle  |^2,
	\end{equation}
	parce que le point~\ref{ItemQGwoIx} du théorème~\ref{ThoyAjoqP} nous indique que ce sera suffisant pour avoir une base hilbertienne.

	Étant donné que \( f\) est holomorphe sur \( D\), le théorème~\ref{ThoUHztQe} nous développe \( f\) en série entière :
	\begin{equation}    \label{EqObkbPK}
		f(z)=\sum_{k=0}^{\infty}a_kz^k.
	\end{equation}
	En permutant la somme avec le produit scalaire,
	\begin{equation}
		c_n(f)=\int_Df(z)\bar e_n(z)=\sqrt{\frac{ n+1 }{ \pi }}\int_Df(z)\bar z^ndz.
	\end{equation}
	Afin de profiter de la convergence uniforme de la série \eqref{EqObkbPK} à l'intérieur de \( D\), nous allons exprimer l'intégrale sur \( D\) comme une intégrale sur \( | z |<r\) en faisant tendre \( r\) vers \( 1\) (par le bas). Pour ce faire nous considérons les fonctions
	\begin{equation}
		g_k(z)=\begin{cases}
			f(z)\bar z^n & \text{si } | z |<1-1/k \\
			0            & \text{sinon.}
		\end{cases}
	\end{equation}
	Ces fonctions sont intégrables sur \( D\) et dominées par \( f(z)\bar z^n\) qui est intégrable sans dépendre de \( k\). Mais nous avons évidemment \( g_k(z)\to f(z)\bar z^n\). Le théorème de la convergence dominée permet alors de permuter l'intégrale et la limite \( k\to \infty\). Cela nous permet d'écrire
	\begin{equation}
		c_n(f)=\sqrt{\frac{ n+1 }{ \pi }}\lim_{r\to 1^-}\int_{| z |<r}\bar z^nf(z)dz=\sqrt{\frac{ n+1 }{ \pi }}\lim_{r\to 1^-}\int_{| z |<r}\sum_{k=0}^{\infty}a_kz^k\bar z^n.
	\end{equation}
	Par la convergence uniforme de la série entière \emph{à l'intérieur} du disque \( D\) nous pouvons permuter l'intégrale et la somme (proposition~\ref{PropfeFQWr}) :
	\begin{equation}
		c_n(f)=\sqrt{\frac{ n+1 }{ \pi }}\lim_{r\to 1^-}\sum_{k=0}^{\infty}a_k\int_{| z |<r}z^k\bar z^ndz.
	\end{equation}
	L'intégrale proprement dite est vite calculée et vaut
	\begin{equation}
		\int_{| z |<1}\bar z^nz^kdz=\frac{ \pi r^{2n+2} }{ n+1 }\delta_{kn}.
	\end{equation}
	Nous pouvons donc continuer le calcul de \( c_n(f)\) en effectuant la somme sur \( k\) qui se réduit à changer \( k\) en \( n\) puis en effectuant la limite :
	\begin{equation}
		c_n(f)=\sqrt{\frac{ n+1 }{ \pi }}\lim_{r\to 1^-}\sum_ka_k\frac{ \pi r^{2n+2} }{ n+1 }\delta_{kn}=\sqrt{\frac{ \pi }{ n+1 }}a_n.
	\end{equation}

	Nous effectuons le même genre de calculs pour évaluer \( \| f \|^2_2\) :
	\begin{subequations}
		\begin{align}
			\| f \|_2^2 & =\int_D| f(z) |^2dz                                                                                                \\
			            & =\lim_{r\to 1^-}\int_{| z |<r}f(z)\sum_{k=0}^{\infty}\bar a_k\bar z_kdz                                            \\
			            & =\lim_{r\to 1^-}\sum_{k=0}^{\infty}\bar a_k\int_{| z |<r}f(z)\bar z^kdz    & \text{permuter } \sum\text{ et } \int \\
			            & =\lim_{r\to 1^-}\sum_{k=0}^{\infty}\bar a_ka_k\frac{ \pi r^{2k+2} }{ k+1 } & \text{intégrale déjà faite}.
		\end{align}
	\end{subequations}
	Mais nous savons déjà que \( c_n(f)=\sqrt{\pi/(n+1)}\), donc ce qui est dans la somme est \( \pi\bar a_ka_k/(n+1)=| c_k(f) |^2\). Nous avons donc
	\begin{equation}
		\| f \|^2_2=\lim_{r\to 1^-}\sum_{k=0}^{\infty}| c_k(f) |^2\, r^{2k+2}.
	\end{equation}
	La fonction (de \( r\)) constante \( | c_k(f) |^2\) domine \( | c_k(f)r^{2k+2} |\) tout en ayant une somme (sur \( k\)) qui converge; en effet la proposition~\ref{PropHKqVHj} nous indique que \( \sum_k| c_k(f) |^2\leq \| f \|_2^2\). Le théorème de la convergence dominée nous permet d'inverser la limite et la somme pour obtenir le résultat attendu :
	\begin{equation}
		\| f \|_2^2=\sum_{k=0}^{\infty}| c_k(f) |^2.
	\end{equation}
\end{proof}
