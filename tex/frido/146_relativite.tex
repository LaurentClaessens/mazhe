% This is part of Mes notes de mathématique
% Copyright (c) 2006-2013,2015, 2018, 2023-2024
%   Laurent Claessens, Carlotta Donadello
% See the file fdl-1.3.txt for copying conditions.

Dans ce chapitre nous donnons des applications de divers théorèmes dans les autres sciences que la mathématique.

%+++++++++++++++++++++++++++++++++++++++++++++++++++++++++++++++++++++++++++++++++++++++++++++++++++++++++++++++++++++++++++
\section{Démystification du MRUA}
%+++++++++++++++++++++++++++++++++++++++++++++++++++++++++++++++++++++++++++++++++++++++++++++++++++++++++++++++++++++++++++
\label{SecMRUAsecondeGGdQoT}

%---------------------------------------------------------------------------------------------------------------------------
\subsection{Preuve de la formule}
%---------------------------------------------------------------------------------------------------------------------------

Nous sommes maintenant en mesure de donner une démonstration complète de la formule du MRUA :
\begin{equation}    \label{EqMRUAINT}
	x(t) = \frac{ at^2 }{ 2 } + v_0t +x_0.
\end{equation}

Au niveau de la physique, nous considérons un mobile qui se déplace avec une accélération constante \( a\). Nous notons par \( v_0\) sa vitesse initiale et par \( x_0\) sa position initiale.

Nous savons que, pour tout mouvement, si \( x(t)\) est la position en fonction du temps, et si \( v(t)\) et \( a(t)\) représentent la vitesse et l'accélération en fonction du temps, alors
\begin{equation}
	\begin{aligned}[]
		v(t) & =x'(t) & \text{et} &  & a(t)=v'(t)=x''(t).
	\end{aligned}
\end{equation}
Afin de trouver \( x(t)\) en connaissant \( a(t)\), il « suffit » donc de prendre deux fois la primitive. Essayons ça dans le cas facile du MRUA où \( a(t)=a\) est constante.

La vitesse \( v(t)\) doit être une primitive de la constante \( a\). Il est facile de voir que \( v(t)=at\) est une primitive de \( a\). Par le corolaire \ref{CorZeroCst}(bis),
\begin{equation}    \label{EqvtatC}
	v(t)=at+C_1
\end{equation}
pour une certaine constante \( C_1\). Afin de fixer \( C_1\), il faut faire appel à la physique : d'après la formule \eqref{EqvtatC}, la vitesse initiale est \( v(0)=C_1\). Donc il faut identifier \( C_1\) à la vitesse initiale : \( C_1=v_0\). Nous avons donc déjà obtenu que
\begin{equation}
	v(t)=at+v_0.
\end{equation}
Afin de trouver \( x(t)\), il faut trouver une primitive de \( v(t)\). Il n'est pas très difficile de voir que \( at^2/2 + v_0t\) fonctionne, donc il existe une constante \( C_2\) telle que
\begin{equation}
	x(t)=\frac{ at^2 }{ 2 }+v_0t+C_2.
\end{equation}
Encore une fois, regardons la condition initiale : la formule donne comme position initiale \( x(0)=C_2\), et donc nous devons identifier \( C_2\) avec la position initiale \( x_0\). En définitive, nous avons bien
\begin{equation}
	x(t) = \frac{ at^2 }{ 2 } + v_0t +x_0.
\end{equation}

Cette formule est donc maintenant \emph{démontrée} à partir de la seule définition de la vitesse comme dérivée de la position et de l'accélération comme dérivée de la vitesse.

Remarquons cependant que la preuve complète fut \emph{très} longue. En effet, nous avons utilisé les règles de dérivation de la proposition \ref{PROPooOUZOooEcYKxn}, pour la démonstration desquels, les résultats \ref{ThoLimLin} et \ref{PropOpsSimplesLimites} ont étés utiles. Mais nous avons surtout utilisé le corolaire \ref{CorZeroCst}(bis) qui repose sur le théorème de Rolle \ref{ThoRolle}, qui lui-même demande le théorème de Borel-Lebesgue \ref{ThoBOrelLebesgue} dans lequel la notion d'ensemble compact a été cruciale.

%---------------------------------------------------------------------------------------------------------------------------
\subsection{Interprétation graphique}
%---------------------------------------------------------------------------------------------------------------------------

La distance parcourue \( x(t)\) en un temps \( t\) est la primitive de la vitesse. Nous savons, par ailleurs, que la primitive est liée à la surface\footnote{L'intégrale d'une fonction entre deux valeurs de \( t\) donne l'aire sous la courbe.} sous la courbe. Pour reprendre les mêmes notations, nous notons \( S_v(t)\) la surface contenue en dessous de la fonction \( v\) entre \( 0\) et \( x\). Nous ne serions donc pas étonné que
\begin{equation}        \label{EqEncoreMRUASvt}
	S_v(t) = \frac{ at^2 }{ 2 }+v_0t+x_0
\end{equation}
soit la surface en dessous de la fonction \( v(t)=at+v_0\). Nous voyons que la surface totale sous la fonction \( v(t)=at+v_0\) est exactement
\begin{equation}
	S_v(t)=\frac{ at^2 }{ 2 }+v_0t.
\end{equation}
Cela est un bon début, mais hélas nous ne retrouvons pas le terme « \( +x_0\) » de la formule \eqref{EqEncoreMRUASvt}. Cela n'est pas tout à fait étonnant parce que nous savons que la surface sous une fonction était \emph{une} primitive de la fonction, mais nous n'avons pas dit \emph{laquelle}. D'après le fameux corolaire \ref{CorZeroCst}(bis), la primitive n'est définie qu'à une constante près. Ici, c'est la constante \( x_0\) qu'on a perdue en chemin.

Nous parlerons plus en détail du lien entre les surfaces et les primitives dans la section dédiée à l'intégration.

\section{Relativité en mécanique newtonienne}
%++++++++++++++++++++++++++++++++++++++++++++

\subsection{Relativité du mouvement}
%-----------------------------------

Prenons quelqu'un qui court le cent mètres en onze secondes. Par rapport à un spectateur dans les gradins, il se sera déplacé de cent mètres. Mais si je cours à côté de lui de telle façon à avoir parcouru \( 80\) mètres le temps qu'il en fasse cent, alors par rapport à moi l'athlète ne se sera déplacé que de \( 20\) mètres. Par contre, par rapport à mon chronomètre, il aura également mis onze secondes : ce n'est pas parce que je cours que mon chronomètre s'affole !

Entre moi et les spectateurs, on a donc une loi de transformation
\begin{align}		\label{EqTransGal}
	x' & =x-vt & t'=t.
\end{align}
C'est-à-dire que la distance \( x'\) qu'aura parcouru l'athlète par rapport à moi vaut la distance \( x\) parcourue par le spectateur moins la distance que j'ai courue moi-même, c'est-à-dire moins \( vt\).

\subsection{Bob et Alice}
%------------------------

Formalisons le concept de changement de repères. Pour cela, prenons deux amoureux, Bob et Alice\footnote{C'est plus poétique que dire « soient \( A\) et \( B\) deux observateurs ».}. Mettons que Bob reste assis sur un banc pendant qu'Alice court en ligne droite à une vitesse \( v\). Tout deux déclenchent leur chronomètre quand Alice passe devant Bob. À tout moment, Bob et Alice ont leur repères de temps et d'espace. Par exemple si après un temps \( t\), Alice voit une peau de banane à \( 1\) mètre devant elle, elle va dire « Il y a une peau de banane à un mètre.», tandis que Bob va dire « Il y a une peau de banane à \( (1+vt)\) mètres ».

Plus généralement, si il se passe quelque chose à la position \( x\) au temps \( t\) pour Bob, ce quelque chose se passera au temps \( t'=t\) à l'endroit \( x'=x-vt\) pour Alice parce qu'en un temps \( t\), elle aura déjà avancé d'une distance \( vt\).

Ça c'est ce dont tout le monde était persuadé depuis Galilée jusqu'au début du vingtième siècle.

%%%%%%%%%%%%%%%%%%%%%%%%%%
%
\section{Invariance de la vitesse de la lumière}
%
%%%%%%%%%%%%%%%%%%%%%%%%

\subsection{Champ de gravitation et électrique}
%-----------------------------------------------

Nous savons que la force de gravitation s'écrit :
\[
	F_{grav}=G\frac{ mm' }{ r^2 },
\]
tandis que la force électrique entre deux charges \( q\) et \( q'\) est donnée par
\begin{equation}	\label{EqRappelFelec}
	F_{elec}=k\frac{ qq' }{ r^2 }.
\end{equation}
Nous avons aussi fait remarquer que dans le cas de la gravitation, la force a l'air d'être instantanée, et que cela posait quelques problèmes conceptuels. La force électrique a apparemment le même problème. Une différence entre les deux est qu'une charge électrique c'est tout petit et qu'on peut expérimenter à souhait, tandis que pour avoir une masse dont on peut mesurer le champ de gravitation correctement, il faut quelque chose grand comme la Terre\footnote{Une autre différence fondamentale est qu'il existe des charges électriques négatives, mais pas de masses négatives; de ce fait on ne peut pas construire d'isolant gravitationnel, contrairement aux isolants électriques qui existent. Cela augmente encore la difficulté de faire des expériences avec la gravitation.}.

\subsubsection{Finitude de la vitesse de propagation de la force électrique}
%///////////////////////////////////////////////////////////////////////////

Si un micro est placé juste à côté de ton oreille, et qu'il commence à faire biiiiip, tu l'entends directement. Quand il s'arrête, tu ne l'entends plus. Si le micro est placé à \unit{600}{\meter} de toi, tu ne commenceras à l'entendre que deux secondes après le commencement du son, et tu continueras à l'entendre deux secondes après qu'il a fini.

Eh bien, pour la force électrique, on a pu mesurer que c'est la même chose (sauf que ça va beaucoup plus vite). Si on place une charge quelque part, on ne ressent la force \eqref{EqRappelFelec} qu'après qu'elle a eut le temps d'arriver. Si on déplace la charge électrique, on continue à ressentir la même force pendant un certain temps : il faut que la modification du champ électrique ait le temps d'arriver. Exactement comme quand on fait des remous quelque part dans un étang : il faut du temps que les remous arrivent plus loin.

On a pu faire des dizaines d'expériences de ce type avec l'électricité, le magnétisme et la lumière; et les résultats sont clairs : il faut du temps pour que ça se déplace. Tout cela provoque des ondes électromagnétiques qui se déplacent à une vitesse finie. On peut produire de telles ondes avec n'importe quel courant électrique alternatif.

\subsubsection{Pourquoi pas la gravitation ?}
%//////////////////////////////////////////

La gravitation telle que donnée par Newton pose le même problème de vitesse de propagation que l'électricité. Est-ce qu'en réalité la gravitation se propage également à une vitesse finie ?

Avec la gravitation c'est beaucoup plus compliqué parce qu'elle est beaucoup plus faible, et donc c'est beaucoup plus difficile à détecter. D'après la théorie d'Einstein de la gravitation, la gravitation devrait également produire des ondes gravitationnelles. Seulement, si un simple courant électrique suffit pour mesurer une onde électromagnétique, afin de mesurer une onde gravitationnelle, il faudrait un déplacement de masse de l'ampleur d'une étoile qui explose. Or ça, on ne sait pas produire dans un laboratoire.

Des ondes gravitationnelles ont été observées pour la première fois en 2016\cite{BIBooAZZPooQvHWLV}.


\subsection{Support du champ : pas d'éther}
%------------------------------------------

Nous avons dit qu'une onde électromagnétique se propage comme une onde sonore (quoique beaucoup plus vite). Une question se pose alors. En effet, une onde sonore est matérialisée par de l'air qui vibre. Qu'est-ce qui vibre pour une onde électromagnétique ?

Étant donné que les ondes électromagnétiques se propagent dans le vide (c'est pour ça que la radio fonctionne dans l'espace), la question est problématique. Les physiciens ont donc supposé que tout l'univers était rempli d'un fluide invisible appelé \defe{l'éther}{Éther}. L'électromagnétisme consiste en une vibration de l'éther, exactement comme l'acoustique consiste en une vibration de l'air.

En fait, vérifier cette hypothèse n'est pas très compliqué. En effet il n'y a aucune raison que l'éther suive la Terre dans son mouvement. Or la Terre se déplace à environ \( \unit{30}{\kilo\meter\per\second}\) autour du Soleil. Donc les ondes électromagnétiques doivent se propager plus vite dans le sens du mouvement de la Terre que dans le sens perpendiculaire. Tout comme le son se propage plus vite dans le sens du vent.

La célèbre expérience d'interférométrie de Michelson-Morley\cite{BIBooJSDSooIUhMQQ} a mesuré cet effet \ldots et ce fut la consternation : il n'y a aucun effet ! Or, la lumière se déplace à \unit{300.000}{\kilo\meter\per\second}\footnote{\unit{299.792.458}{\meter\per\second} très exactement}; une variation de \( \unit{30}{\kilo\meter\per\second}\) devrait être détectable !

Mais rien ! On a recommencé les expériences dans tous les sens, à tous les mois de l'année, à tous les endroits de la Terre. On n'a pas observé un poil de variation de la vitesse de la lumière. Et ça, ça pose un gros problème à la physique.


\subsection{Le problème}
%-----------------------

Si je joue au football dans un train qui avance à \unit{100}{\kilo\meter\per\hour} et que je lance une balle à \unit{20}{\kilo\meter\per\hour}, quelqu'un au sol mesura la vitesse de la balle soit à \unit{120}{\kilo\meter\per\hour} soit à \unit{80}{\kilo\meter\per\hour} d'après que l'on ait shooté vers l'avant ou l'arrière du train. Cela paraît logique. Mais ce qu'on vient de voir c'est que ça ne marche pas avec la lumière.

Si un train avance à \( \unit{100.000}{\kilo\meter\per\second}\) et qu'on y allume une lampe de poche, la lumière avancera à \unit{300.000}{\kilo\meter\per\second} par rapport au train et \unit{400.000}{\kilo\meter\per\second} par rapport au sol. Non ! Justement pas ! La lumière avancera quand même à \unit{300.000}{\kilo\meter\per\second} par rapport au sol.

Là encore, on a fait des dizaines d'expériences partout, sur Terre, dans des avions, dans l'espace avec des atomes, des lampes de poche et des horloges atomiques, et dans tous les sens, le sens de déplacement de la Terre, le sens inverse, le sens perpendiculaire, vers le haut, vers le bas : rien ! Personne n'a jamais observé un rayon de lumière se déplacer à une autre vitesse que \unit{300.000}{\kilo\meter\per\second}.

Le problème est que le principe d'addition des vitesses est faux pour la lumière. Puisque l'expérience nous force, nous devons faire avec.

\begin{loiphyz}		\label{LoiVitLum}
	La réalité est que la vitesse de la lumière est la même dans tous les référentiels. On note \( c\) cette vitesse. C'est une constante fondamentale de la Nature.
\end{loiphyz}
Étant donné que c'est une loi expérimentale, nous n'y pouvons rien. C'est la nature qui est comme ça. En particulier tu ne peux pas en vouloir à ton prof de physique d'avoir inventé une théorie compliquée. Ce n'est pas lui qui l'a inventée et ce n'est pas de sa faute.


\section{Conséquences}
%++++++++++++++++++++++

C'est maintenant que les choses vraiment graves commencent (cela soit dit sans vouloir te faire peur). Afin d'un peu simplifier les choses, nous n'allons  étudier que les mouvements en une dimension, c'est-à-dire sur une droite.

\subsection{Ligne d'univers}
%---------------------------

Un événement a une coordonnée \( (t,x)\). Si je pose un objet juste à mes pieds (disons en \( x=0\)), ses coordonnées seront à tout moment \( (t,0)\). Il est bon de voir cette coordonnée comme l'équation paramétrique d'une droite horizontale dans le plan des coordonnées \( t\) et \( x\). Plus généralement quand un mobile effectue un mouvement \( x(t)\), on appelle la \defe{ligne d'univers}{Ligne d'univers} du mobile la ligne (pas forcément droite) \( (t,x(t))\). Dans le premier exemple, on avait \( x(t)=0\) pour tout \( t\).

Le cas d'un mobile se déplaçant à vitesse constante \( v\) donne comme ligne d'univers la droite\footnote{bon exercice de révision de ton cours de math de vérifier que c'est une droite.} \( (t,x_0+vt)\), et un objet qui se déplace selon un MRUA a comme ligne d'univers
\[
	\big( t, x_0+v_0t+\frac{ at^2 }{ 2 } \big).
\]

\subsection{Transformations de Lorentz}
%--------------------------------------

Reprenons les amours scientifiques de Bob et Alice, mais cette-fois, analysons celles-ci en tenant compte du fait que la vitesse de la lumière soit invariante. Maintenant, si Bob voit se passer quelque chose au temps \( t\) à l'endroit \( x\), on va dire qu'Alice voit cette chose au temps \( t'\) à la position \( x'\), et on va chercher \( (t',x')\) en fonction de \( (t,x)\).

Posé en termes mathématiques, le problème s'énonce ainsi : trouver les fonctions \( f\) et \( g\) telles que les formules
\begin{subequations}	\label{SubEqLorGen}
	\begin{align}
		t' & =f(t,x) \\
		x' & =g(t,x)
	\end{align}
\end{subequations}
donnent les coordonnées vues par Alice pour un événement vu par Bob à l'instant \( t\) au point \( x\). Une première étape importante est franchie par la proposition suivante\footnote{dont je te suggère fortement de ne pas lire la preuve si tu ne veux pas que ton cerveau éclate.}.

\begin{proposition}
	Les fonctions \( f\) et \( g\) contenues dans les transformations \eqref{SubEqLorGen} sont nécessairement linéaires (affines), c'est-à-dire qu'elles doivent s'écrire sous la forme
	\[
		\begin{split}
			t'&=\alpha t+\beta x+p	\\
			x'&=\gamma t+\delta x+q
		\end{split}
	\]
	pour certaines fonctions \( \alpha\), \( \beta\), \( \gamma\), \( \delta\), \( p\) et \( q\) de la vitesse d'Alice relativement à Bob.
\end{proposition}

\begin{proof}
	Pendant qu'Alice court et que Bob la regarde, Ève tente de lancer une pierre sur Alice (Ève est jalouse). Bob et Alice regardent deux événements. Le premier est la pierre qui quitte la main de Ève, et le second est la pierre qui percute le sol. Pour Bob, le jet s'est passée au temps \( t_0\) au point \( x_0\), et la pierre touche le sol un petit peu plus tard, au temps \( t_0+\Delta t\) et un peu plus loin, au point \( x_0+\Delta x\). Bob écrit donc ceci sur sa feuille de papier :
	\[
		\begin{split}
			E_1&=(t_0,x_0)\\
			E_2&=(t_0+\Delta t,x_0+\Delta x),
		\end{split}
	\]
	tandis qu'Alice, en observant les mêmes deux événements, aura noté
	\[
		\begin{split}
			E'_1&=\big( f(t_0,x_0),g(t_0,x_0) \big) \\
			E'_2&=\big( f(t_0+\Delta t,x_0+\Delta x), g(t_0+\Delta t,x_0+\Delta x) \big).
		\end{split}
	\]
	Bob et Alice se demandent combien de temps la pierre est restée en l'air et quelle distance elle a parcourue. Par le principe général d'homogénéité, les deux seules quantités pertinentes (qui ont un sens physique) pour Bob sont \( (t_0+\Delta t)-t_0\) et \( (x_0+\Delta x)-x_0\), c'est-à-dire \( \Delta t\) et \( \Delta x\). En effet, si Bob avait choisi de s'asseoir autre part et si Alice avait commencé à courir un peu plus tard, ça n'aurait rien changé à la longueur du jet de Ève.

	D'une façon ou d'une autre, il doit exister une façon de déduire les mesures de Alice en connaissant celles de Bob; je ne connais pas avec quelles formules, mais ces formules ne peuvent contenir que \( \Delta t\), \( \Delta x\) et \( v\) parce que ce sont les seules quantités qui définissent tous les événements.

	Cela dit, Alice va caractériser le mouvement de la pierre avec la différence des coordonnées entre le jet et la chute sur le sol mesurées par elle-même. En d'autres termes, pour Alice ce qui compte c'est la différence entre \( E'_1\) et \( E'_2\), soit
	\begin{equation}	\label{EqAlfxdelmoins}
		\big( f(t_0+\Delta t,x_0+\Delta x), g(t_0+\Delta t,x_0+\Delta x) \big)-\big( f(t_0,x_0),g(t_0,x_0) \big).
	\end{equation}
	Mais nous venons de signaler que ce qu'Alice mesurait devait pouvoir être exprimé en termes de \( \Delta t\) et \( \Delta x\). Nous concluons que la différence \eqref{EqAlfxdelmoins} ne dépend en fait pas de \( x\) et \( t\) mais seulement de \( \Delta x\) et \( \Delta t\).

	Prenons maintenant une notation plus compacte et notons \( X=(t,x)\), \( \Delta X=(\Delta t,\Delta x)\) puis \( F=(f,g)\). Avec ça, l'expression \eqref{EqAlfxdelmoins} se note \( F(X+\Delta X)-F(X)\). Comme mentionné, cette expression ne dépend que de \( \Delta x\). En particulier, elle ne dépend pas de \( X\).

	Maintenant tu vas comprendre pourquoi on apprend les dérivées dans ton cours de math. Comme \( F(X+\Delta X)-F(X)\) ne dépend pas de \( X\), le rapport \( \big( F(X+\Delta X)-F(X) \big)/\Delta X\) non plus. La limite de ce rapport quand \( \Delta X\) tend vers zéro non plus :
	\begin{equation}
		\lim_{\Delta X\to 0}\frac{ F(X+\Delta X)-F(X) }{ \Delta X }
	\end{equation}
	ne dépend pas de \( X\). Tu reconnais là la dérivée de \( F\) au point \( X\). En d'autres termes, \( F'(X)\) est constante, elle ne dépend pas de \( X\). Disons donc que \( F'(X)=a\). Tu connais beaucoup de fonctions dont la dérivée est constante ? Non ? En effet, il n'y en a pas beaucoup. Les fonctions qui vérifient \( F'(X)=a\) sont toutes de la forme
	\[
		F(X)=aX+b.
	\]
	À ce niveau, il convient de re-déballer les notations compactes : si \( a=\begin{pmatrix}
		\alpha & \beta \\\gamma&\delta
	\end{pmatrix}\) et \( b=(p,q)\) on trouve
	\begin{subequations}		\label{EqLoUn}
		\begin{align}
			f(t,x) & =\alpha t+\beta x+p   \\
			g(t,x) & =\gamma t+\delta x+q,
		\end{align}
	\end{subequations}
	comme annoncé.

\end{proof}


Nous savons que lorsque \( (t,x)=(0,0)\), alors \( (t',x')=(0,0)\). En effet, Bob et Alice ont lancé leurs chronos en même temps au moment où ils étaient au même endroit. En mettant \( (t,x)=(0,0)\) dans les équations \eqref{EqLoUn}, on trouve \( (t',x')=(p,q)\), et donc \( p=q=0\). Ça fait une chose de réglée; on se retrouve avec
\begin{subequations}\label{EqLoDeux}
	\begin{numcases}{}
		t'=\alpha t+\beta x\\
		x'=\gamma t+\delta x.
	\end{numcases}
\end{subequations}
Quelles sont les contraintes à vérifier pour que ces transformations décrivent correctement la physique que l'on cherche à écrire ?
\begin{enumerate}
	\item Il faut que les transformations décrivent correctement que Alice avance à une vitesse \( v\) par rapport à Bob,
	\item dans le même ordre d'idée, il faut que l'on trouve que Bob avance à la vitesse \( -v\) par rapport à Alice,
	\item il faut que si Alice et Bob observent un rayon lumineux, ce rayon aille à la vitesse \( c\) par rapport à Alice et à la \emph{même} vitesse \( c\) par rapport à Bob,
	\item enfin, il faut avoir le principe de relativité, c'est-à-dire que comme les équations \eqref{EqLoDeux} disent ce que Alice voit en fonction de ce que Bob voit, on demande que les équations qui disent ce que Bob voit en fonction de ce que Alice voit soient les mêmes. En d'autres termes, il faut que les transformations et les transformations inverses soient les mêmes au changement près du signe de \( v\).
\end{enumerate}

Étudions une à une ce que chacune de ses contraintes impose. Rappelons que \( (t,x)\) et \( (t',x')\) sont les coordonnées que Bob et Alice mettent sur le même événement. Par exemple sur l'événement qui consiste à ce que Ève, par jalousie envers Bob, jette une peau de banane sous les pieds d'Alice. Cet événement a lieu à un certain moment, à un certain endroit. C'est ce moment et cet endroit qui sont notés \( (t,x)\) et \( (t',x')\).

\begin{enumerate}
	\item Les coordonnées \( (t,x)\) et \( (t',x')\) peuvent décrire n'importe quoi. Regardons les coordonnées de Alice qui cours. Pour Alice, cela correspond à \( (t',x')=(t',0)\) parce que si \( x'\) désigne la position de Alice par rapport à Alice, alors \( x'\) est toujours nul. Pour Bob par contre, Alice ne reste pas en place, mais se déplace à une vitesse \( v\). C'est-à-dire que si \( (t,x)\) sont les coordonnées de Alice pour Bob, alors \( x/t=v\). Écrivons les équations \eqref{EqLoDeux} en tenant compte de tout ça : avec \( x'=0\), la seconde équation donne
	      \begin{equation}
		      0=\gamma t+\delta x,
	      \end{equation}
	      d'où on déduit que \( x/t=-\gamma/\delta\). En imposant que cela soit \( v\), on trouve \( \gamma=-v\delta\), et on ré-écrit les transformations en tenant compte de ça :
	      \begin{subequations}
		      \begin{numcases}{}
			      t'=\alpha t+\beta x\\
			      x'=-v\delta t+\delta x.
		      \end{numcases}
	      \end{subequations}
	      Nous voilà débarrassé d'un paramètre.
	\item Maintenant, on regarde ce qu'il se passe quand \( (t,x)\) et \( (t',x')\) décrivent les positions de Bob. On a \( (t,x)=(t,0)\) parce que selon Bob, Bob est au repos. Les équations deviennent :
	      \begin{align}
		      t' & =\alpha t & x'=-v\delta t.
	      \end{align}
	      La vitesse de Bob par rapport à Alice est \( -v\), donc on exige que \( x'/t'=-v\), c'est-à-dire que
	      \[
		      \frac{ -v\delta t }{ \alpha t }=-v,
	      \]
	      ce qui implique que \( \delta=\alpha\). On avance encore un peu. Écrivons à nouveau les lois de transformation en en tenant compte :
	      \begin{subequations}
		      \begin{numcases}{}
			      t'=\alpha t+\beta x\\
			      x'=-v\alpha t+\alpha x.
		      \end{numcases}
	      \end{subequations}

	\item Si maintenant Bob et Alice regardent un même rayon de lumière (comme c'est romanesque !), alors \( (t,x)\) et \( (t',x')\) expriment les coordonnées d'un rayon lumineux. Le fait que Bob regarde un rayon lumineux fait que \( x=ct\), et donc que les coordonnées du rayon lumineux, observé par Alice sont :
	      \begin{align}
		      t' & =\alpha t+\beta ct & x' & =-\alpha v t+\alpha c t.
	      \end{align}
	      L'invariance de la vitesse de la lumière exige que Alice mesure une vitesse \( c\) pour le rayon de lumière, c'est-à-dire \( x'=ct'\). On exige donc que
	      \[
		      -\alpha v t+\alpha ct=c\alpha t+\beta c^2t,
	      \]
	      ce qui implique que
	      \[
		      \beta=-\frac{ \alpha v }{ c^2 }.
	      \]
	      Une fois de plus, l'avant-dernière,  on ré-écrit les lois de transformations en tenant compte de ce fait; mais cette fois, on fait l'effort d'écrire aussi les transformations inverses :
	      \begin{align}	\label{EqLorAvd}
		      t' & =\alpha t-\frac{ \alpha v }{ c^2 }x & t & =\frac{1}{ \Delta }\big( \alpha t'+\frac{ \alpha v }{ c^2 }x' \big) \\
		      x' & =-\alpha vt+\alpha x                & x & =\frac{1}{ \Delta }(\alpha v t'+\alpha x')
	      \end{align}
	      où \( \Delta=\alpha^2-\frac{ \alpha^2 v^2 }{ c^2 }\) que tu noteras au passage être toujours positif, et nul uniquement quand \( v=c\).

	\item Maintenant il reste à imposer le principe de relativité. Les transformations \eqref{EqLorAvd} montrent comment Alice voit le monde (c'est-à-dire \( (t',x'))\) en fonction de la façon dont Bob voit le monde (c'est-à-dire \( (t,x)\)). On se demande donc quelle seraient, pour Bob, les coordonnées \( (t,x)\) d'un point vu en \( (t',x')\) par Alice. Cela signifie que l'on impose que les deux systèmes \eqref{EqLorAvd} soient en réalité les mêmes, à un changement de signe près.

	      Attention : il est à priori faux de dire qu'en changeant le signe de \( v\) dans \( \alpha v/c^2\), j'obtiens \( -\alpha v/c^2\) parce que \( \alpha\) est une fonction de \( v\). En réalité, il faut noter \( \alpha(v)v/c^2\) et donc le changement de signe de \( v\) donne \( -\alpha(-v)v/c^2\). Ceci étant clair, on peut un petit peu calculer.

	      Commençons par égaliser le coefficient de \( x\) dans \( t'\) à celui de \( x'\) dans \( t\), en changeant le signe de \( v\) :
	      \[
		      \frac{ \alpha(-v)v }{ c^2 }=\frac{ \alpha(v)v }{ c^2 },
	      \]
	      et donc \( \alpha(v)=\alpha(-v)\). Ça c'est une bonne nouvelle. Égalisons maintenant le coefficient de \( t\) dans \( t'\) à celui de \( t'\) dans \( t\) en changeant le signe de \( v\) :
	      \[
		      \alpha(-v)=\frac{ \alpha(v) }{ \Delta(v) }=\frac{ \alpha(v) }{ \alpha(v)^2\big( 1-\frac{ v^2 }{ c^2 } \big) }.
	      \]
	      Comme \( \alpha(-v)=\alpha(v)\), on en déduit que
	      \begin{equation}		\label{EqalphaLo}
		      \alpha(v)=\frac{1}{ \sqrt{1-\frac{ v^2 }{ c^2 }} }.
	      \end{equation}
\end{enumerate}

Maintenant qu'on a tout, on peut écrire les transformations de Lorentz. On met donc l'expression \eqref{EqalphaLo} dans les lois de transformations \eqref{EqLorAvd} :
\begin{equation}
	\begin{aligned}		\label{EqTrLorentz}
		t' & = \frac{1}{ \sqrt{1-\frac{ v^2 }{ c^2 }} }\left( t-\frac{ v }{ c^2 }x \right) & t=	\frac{1}{ \sqrt{1-\frac{ v^2 }{ c^2 }} }\left(t'+\frac{ v }{ c^2 }x'\right) \\
		x' & =\frac{1}{ \sqrt{1-\frac{ v^2 }{ c^2 }} }(x-vt)                               & x=\frac{1}{ \sqrt{1-\frac{ v^2 }{ c^2 }} }(vt'+x').
	\end{aligned}
\end{equation}
Tu remarqueras que \( \Delta=1\); si tu ne sais pas ce qu'est le déterminant d'une application linéaire, ça n'a pas d'importance. Mais si tu sais ce qu'est le déterminant d'une application linéaire, alors ce \( \Delta=1\) est crucial !

Afin d'avoir des équations un peu plus courtes, à partir de maintenant nous allons noter
\[
	\gamma(v)=\sqrt{1-\frac{ v^2 }{ c^2 }}.
\]


\subsection{Conditions d'existence}
%----------------------------------

Comme tu vois une racine carrée et un dénominateur dans ces formules, tu dois te demander quelles sont les conditions d'existence. Étant donné que \( v<c\), on a \( v^2/c^2\leq 1\) et en particulier, \( v^2/c^2=1\) si et seulement si \( v=c\).

Ce qui se trouve dans la racine carrée ne pose donc jamais de problèmes parce que ce n'est jamais négatif.

Le dénominateur est par contre plus problématique : quand \( v=c\) il n'y a plus rien qui fonctionne. Quelle est la physique de ce problème ? Pour le comprendre, il faut se souvenir ce que représente \( v\). Nous avons dit que \( v\) est la vitesse à laquelle Alice court. Ce que la condition d'existence nous enseigne, c'est que personne ne peut courir à la vitesse de la lumière. C'est une vitesse que l'on ne peut pas atteindre.

Dit en termes plus savants, on ne peut pas choisir un repère qui se déplace à la vitesse de la lumière.

La question qui se pose alors est « ah bon, on ne peut pas atteindre la vitesse de la lumière ! Et la lumière, comment elle fait ? ». Bonne question, merci de l'avoir posée. Hélas la réponse sort du cadre de ce cours.

\begin{loiphyz}\label{loivitlumlimite}
	Aucun objet ne peut atteindre la vitesse de la lumière.
\end{loiphyz}

\begin{loiphyz}
	Tu ne dois pas te demander pourquoi la lumière elle-même se déplace à la vitesse de la lumière malgré la loi numéro~\ref{loivitlumlimite}.
\end{loiphyz}

\subsection{La notion d'intervalle}
%----------------------------------


Un \defe{événement}{Événement} est quelque chose qui se passe à un endroit à un certain moment. C'est donc caractérisé par le moment et le lieu. Comme on travaille à une dimension, c'est un couple de réels \( (t,x)\).

Regardons un rayon de lumière. Un événement est le fait d'allumer une lampe de poche, et un autre est le fait que la lumière arrive sur l'objet qu'on éclaire. Appelons-les \( (t_1,x_1)\) et \( (t_2,x_2)\). Comme d'habitude, on note \( \Delta t=t_2-t_1\) et \( \Delta x=x_2-x_1\). Comme le rayon de lumière va à la vitesse \( c\), on a \( c=\Delta x/\Delta t\), ou encore
\[
	c^2\Delta t^2-\Delta x^2=0.
\]
Pour cette raison, on va dire que l'\defe{intervalle}{Intervalle} entre deux événements \( (t_1,x_1)\) et \( (t_2,x_2)\) vaut en général
\begin{equation}
	s^2=c^2(t_2-t_1)^2-(x_2-x_1)^2.
\end{equation}
Par invariance de la vitesse de la lumière, si un intervalle est nul pour un observateur, il sera nul pour tous les observateurs.


\subsubsection{En mécanique newtonienne}
%///////////////////////////////////////

Afin de voir un peu mieux l'enjeu de l'invariance de l'intervalle, regardons un exemple chiffré.  Si par exemple je me déplace de \unit{10}{\meter} en \unit{5}{\second}, mon intervalle mesuré par une personne extérieure est
\[
	c^2\Delta t^2-\Delta x^2=(300.000.000)^2\cdot (5)^2 - (10)^2=\unit{2,25\cdot \power{10}{18}}{\meter}.
\]
Si je fais le calcul pour moi, j'ai que \( \Delta x'=0\) parce que je ne me déplace pas, et \( \Delta t'=5\) parce que je me suis déplacé en \( 5\) secondes. Le truc est que à côté de \( (300.000.000)^2\), l'intervalle spatial \( \Delta x\) ne pèse pas grand chose. Ça ne change presque rien qu'il soit de \( 5\) mètres ou de zéro. Ça ne change pas grand chose, mais ça change quand même ! Entre moi qui calcule ou une personne extérieure, l'intervalle change de \( 100\) sur un nombre de la grandeur de \( 200.000.000.000.000.000.0000\) !

Reprenons plus clairement le raisonnement. D'après la mécanique classique, l'intervalle mesuré par deux personnes est différent, mais très peu différent. Inutile de dire que du temps de Newton, on n'avait pas les moyens techniques de mesurer si cet intervalle est effectivement différent ou bien si il est en réalité égal. C'est un peu comme si on te mettait un spot dans les yeux et qu'on te demandait si c'est un spot de \unit{1000}{\watt} ou de \unit{1001}{\watt}. Bonne chance pour le dire !

\subsubsection{En mécanique relativiste}
%//////////////////////////////////////

Maintenant qu'on a des moyens techniques nettement plus poussés que Newton, on a pu mesurer que l'intervalle est égal. L'intervalle est un invariant. Cela n'est pas un nouveau principe physique parce qu'il découle des transformations de Lorentz.


\subsection{Le cône de lumière d'un point}
%-----------------------------------------


Il est intéressant de dessiner dans le plan \( (t,x)\) l'ensemble des points atteints par le rayon lumineux. Le point \( (t,x)\) est atteint si \( c^2t^2-x^2=0\), ou encore si \( x=\pm ct\). Cela forme deux droites dans le plan tracé par les coordonnées \( t\) et \( x\). Ces deux droites forment ce qu'on appelle le \defe{cône de lumière}{Cône de lumière} du point \( (0,0)\).

\subsection{Contraction des longueurs}
%-------------------------------------

Bob prend un morceau de bois qu'il mesure de longueur \( l\) et le dépose devant lui. À l'instant \( t\) (de Bob), les deux extrémités sont aux coordonnées \( e_1=(t,0)\) et \( e_2=(t,l)\).

Afin de savoir quelle est la longueur de ce même morceau de bois pour Alice, il faut qu'elle mesure les deux extrémités en même temps (pour elle), et qu'elle fasse la différence. Comme Bob et Alice déclenchent leurs chronomètres en même temps, le plus simple est de faire la mesure à cet instant.

Pour Bob, c'est clair : les coordonnées des deux extrémités sont \( e_1=(0,0)\) et \( e_2=(0,l)\). La longueur du bois est \( l\). Pour savoir quelle est la longueur mesurée par Alice, on utilise les transformations de Lorentz qui donnent les coordonnées \( e'_1\) et \( e'_2\) relatives à Alice. On trouve \( e'_1=(0,0)\) et
\begin{equation}		\label{Eqepdeuxfaut}
	e'_2=\left(   \frac{ -vl/c^2 }{ \gamma(v) },\frac{ l }{ \gamma(v) }   \right).
\end{equation}
En d'autres termes, on a \( x_1=0\) et \( x_2=l/\gamma(v)\), ce qui fait que la longueur observée par Alice est \( l'=x_2-x_1=l/\gamma(v)\).

Eh bien ce résultat est faux. Si tu vois pourquoi sans lire la suite, tu es très fort.

Pour mesurer la longueur d'un objet, il faut mesurer la position des deux bouts \emph{en même temps} puis faire la différence entre les deux. Effectivement, \( e_1\) et \( e_2\) sont en même temps pour Bob, et donc Bob peut mesurer la longueur de son bout de bois en faisant la différence \( x_2-x_1\). Mais comme le montre les coordonnées \eqref{Eqepdeuxfaut}, les événements \( e'_1\) et \( e'_2\) ne se passent pas en même temps pour Alice ! Eh oui : \( t'_1=0\) et \( t'_2=-vl/c^2\gamma(v)\); c'est pas la même chose.

Il faut donc trouver un événement qui pour Alice correspond à l'extrémité du bout de bois au temps \( t'=0\). Comme l'événement général qui correspond au bout du bois pour Bob est \( (t,l)\), l'événement général est pour Alice
\begin{align}
	t' & =\frac{ t-\frac{ v }{ c^2 }l }{ \gamma(v) }
	   & x'                                          & =\frac{ l-vt }{ \gamma(v) }.
\end{align}
Afin d'avoir \( t'=0\), il faut \( t=vl/c^2\). En mettant cette valeur de \( t\) dans \( x'\), on trouve
\[
	x'=\frac{ l-v\left( \frac{ vl }{ c^2 } \right)) }{ \sqrt{1-\frac{ v^2 }{ c^2 }} }=\frac{ l\left( 1-\frac{ v^2 }{ c^2 } \right)) }{ \sqrt{1-\frac{ v^2 }{ c^2 }} }=l\gamma(v).
\]
Et là, c'est la bonne formule. Si un objet a une longueur \( l\) dans le référentiel où il est au repos, il aura une longueur
\begin{equation}
	l'=l\sqrt{1-\frac{ v^2 }{ c^2 }}
\end{equation}
dans un référentiel qui se déplace à la vitesse \( v\) par rapport à l'objet.


\subsection{Dilatation des intervalles de temps}
%-----------------------------------------------

Encore un petit effort et nous passons à une application concrète que tu connais des bizarreries de la relativité. Mais en attendant, regarde bien ta montre, tu ne va pas en croire tes yeux !

Reprenons Bob et Alice. On se rappelle que Bob et Alice avaient déclenché leurs chronomètres en même temps quand Alice était passée devant Bob. Un peu plus tard, Alice regarde sa montre qui indique un temps \( t\). Et elle se demande si Bob a aussi à ce moment une montre qui indique un temps \( t\).

Ce serait dingue que non hein !?! En effet, si je synchronise ma montre avec quelqu'un et que je pars faire un tour, ma montre ne sera pas tout d'un coup  désynchronisée. Oui, mais Alice, elle cours presque à la vitesse de la lumière \ldots et à ces vitesses-là, on a déjà vu des choses incroyables. Calculons donc pour en avoir le c\oe ur net.

Le fait qu'Alice regarde sa montre est un événement qui se passe pour Alice aux coordonnées \( (t',0)\) (le zéro c'est parce que par rapport à elle-même, Alice est toujours au repos). À quelles coordonnées \( (t,x)\) pour Bob correspond cet événement ?

L'équation de \( t\) en fonction de \( t'\) et \( x'\) dans les transformations de Lorentz \eqref{EqTrLorentz} prise avec \( x'=0\) donnent
\[
	t=\frac{ t' }{ \gamma(v) }.
\]
Et si, juste pour le plaisir, on faisait l'inverse ? Bob regarde sa montre, il voir un temps \( t\) et sa coordonnée spatiale est \( x=0\). À quel temps d'Alice cela correspond ? Mettons \( x=0\) dans la transformation de Lorentz de \( t'\) en fonction de \( t\) et \( x\). Ce qu'on obtient c'est
\[
	t'=\frac{ t }{ \gamma(v) }.
\]
N'est-ce pas génial ? C'est la même ! Évidemment, ça ne pouvait pas être autre chose : le principe de relativité demande qu'on ne puisse pas faire la différence entre Alice qui cours vers la droite avec Bob assis et Alice assise avec Bob qui cours vers la gauche. C'est exactement pour ça que dans une gare, quand le train d'à côté démarre, il t'arrive de croire que c'est ton train qui démarre : tu ne peux pas faire la différence, c'est un principe physique.


\subsection{Invariance de l'intervalle}
%--------------------------------------

Dans deux secondes, je vais te montrer comment une utilisation intelligente des exponentielles permet de trouver un résultat très fort en relativité. Quoi ? Les exponentielles, les mêmes qu'au cours de math ? Eh oui : la même exponentielle que celle qu'on t'a introduit avec des populations de bactéries qui se multiplient, cette même exponentielle qui la la miraculeuse propriété d'être égale à sa propre dérivée.

Mais n'anticipons pas.

Nous avons déjà signalé que si la quantité \( \Delta s^2=c^2\Delta t^2-\Delta x^2\) était nulle pour un observateur, alors elle était nulle pour tous les observateurs. Supposons deux événements \( A\) et \( B\) observés par Alice et Bob. Bob les note aux coordonnées \( (t_a,x_a)\), et \( (t_b,x_b)\) tandis qu'Alice les note en \( (t'_a,x'_a)\) et \( (t'_b,x'_b)\).

L'intervalle entre les deux événements mesuré par Bob sera
\[
	s^2=c^2(t_b-t_a)^2-(x_b-x_a)^2,
\]
tandis que ce même intervalle mesuré par Alice sera
\[
	s'{}^2=c^2(t'_b-t'_a)^2-(x'_b-x'_a)^2.
\]
On peut bien entendu remplacer dans la première équation les \( t_a\), \( t_b\), \( x_a\) et \( x_b\) par leurs valeurs en termes de \( t'_a\), \( t'_b\), \( x'_a\) et \( x'_b\) données par les transformations de Lorentz. Tu paries que les trois quart des termes dans le calcul se simplifient et qu'il restera exactement \( s'{}^2\) ? Je te dis que oui, et je te conseille de me croire sur parole, sinon tu vas devoir lire le calcul suivant :
\begin{align*}
	s^2=c^2(t_b-t_a)^2-(x_b-x_a)^2 & =c^2\left(   \frac{1}{ \gamma(v) }(t'_b+\frac{ v }{ c^2 }x'_b)-\frac{1}{ \gamma(v) }(t'_a+\frac{ v }{ c^2 }x'_a)   \right)^2 \\
	                               & \quad-\left(  \frac{1}{ \gamma(v) }(vt'_b+x'_b)-\frac{1}{ \gamma(v) }(vt'_a+x'_a)  \right)^2.
\end{align*}
Jusqu'ici, on n'a fait que remplacer les choses par leurs valeurs données par les transformations de Lorentz. Maintenant on regroupe à l'intérieur de chaque parenthèse les termes de façon à faire apparaitre \( \Delta x'\) et \( \Delta t'\) :
\begin{align*}
	s^2 & =\frac{ c^2 }{ \gamma(v)^2 }\left( (t'_b-t'_a)+\frac{ v }{ c^2 }(x'_b-x'_a) \right)^2                                          \\
	    & \quad-\frac{1}{ \gamma(v)^2 }\big( (x'_b-x'_a)+v(t'_b-t'_a) \big)^2                                                            \\
	    & =\frac{ c^2 }{ \gamma(v)^2 }\left( (\Delta t')^2+2\frac{ v }{ c^2 }\Delta t'\Delta x'+\frac{ v^2 }{ c^4 }(\Delta x')^2 \right) \\
	    & \quad-\frac{1}{ \gamma(v)^2 }\Big( (\Delta x')^2+2v\Delta x'\Delta t'+v^2(\Delta t')^2 \Big).
\end{align*}
Là, on a utilisé le produit remarquable \( (a+b)^2=a^2+2ab+b^2\), et on a systématiquement renommé tous les intervalles avec la notation \( \Delta\) pour être plus compact. Maintenant, on va regrouper tous les termes contentant \( (\Delta t')^2\) ensemble, tous ceux qui contiennent \( \Delta t'\Delta x'\) ensemble et ceux qui contiennent \( (\Delta x')^2\) ensemble. Autre manière de le dire, on met les \( \Delta\) en évidence comme on peut. On trouve ceci :
\[
	\begin{split}
		(\Delta t')^2\left( \frac{ c^2 }{ \gamma(v)^2 }-\frac{ v^2 }{ \gamma(v)^2 } \right)+\Delta &t'\Delta x'\left( \frac{ 2vc^2 }{ \gamma(v)^2c^2 }-\frac{ 2v }{ \gamma(v)^2 } \right)\\
		&+(\Delta x')^2\left( \frac{ c^2 v^2 }{ c^4\gamma(v)^2 }-\frac{1}{ \gamma(v)^2 } \right).
	\end{split}
\]
À partir de là, je te laisse vérifier (en utilisant le fait que \( \gamma(v)^2=1-v^2/c^2\)) que les coefficients se simplifient beaucoup et valent finalement respectivement \( c^2\), \( 0\) et \( -1\) comme il se doit. Avec tout ça, nous avons montré le résultat très important suivant :
\begin{quote}
	L'intervalle entre deux événements est invariant sous les changements de repères d'inertie, c'est-à-dire que la valeur mesurée par n'importe qui qui se déplace en MRU sera toujours la même.
\end{quote}
Pourquoi cela est tellement important ? À cause de Pythagore et d'une petite démonstration à coups d'exponentielles\footnote{oui oui tout ton cours de math va finir par y passer.}.

\subsubsection{Rappel de trigonométrie hyperbolique}
%///////////////////////////////////////////////////
\label{SUBSUBSECooZVHLooYwuhAj}

Les fonctions de trigonométrie hyperboliques sont :
\begin{align}
	\cosh(x) & =\frac{  e^{x}+ e^{-x} }{ 2 } & \sinh & =\frac{  e^{x}- e^{-x} }{ 2 }.
\end{align}
Elles ont pas mal de propriétés en commun avec les sinus cosinus et normaux. D'abord, leurs dérivées sont faciles à calculer :
\[
	\begin{split}
		\cosh'(x)&=\sinh(x)\\
		\sinh'(x)&=\cosh(x)
	\end{split}
\]
où tu noteras qu'il n'y a pas de signe moins qui apparaît, contrairement au cas de la trigonométrie normale. Une autre propriété qui ressemble fort à une propriété de la trigonométrie est :

\begin{proposition}
	Pour tout \( x\in\eR\),
	\begin{equation}
		\cosh^2(x)-\sinh^2(x)=1
	\end{equation}
	avec un signe moins comme différence avec la trigonométrie.
\end{proposition}

\begin{proof}
	La preuve revient simplement à calculer en utilisant le produit remarquable de \( (a+b)^2\). D'abord, on a :
	\[
		\cosh^2(x)=\frac{1}{ 4 }\big(  e^{x}+ e^{-x} \big)^2=\frac{1}{ 4 }\big(  e^{2x}+2 e^{x} e^{-x}+ e^{-2x} \big)=\frac{1}{ 4 }( e^{2x}+2+ e^{-2x})
	\]
	où l'on a utilisé le fait que \( (e^{x})^2= e^{2x}\) et que \(  e^{x} e^{-x}=1\). Il te reste à faire la même chose pour \( \sinh^2(x)\), la réponse est :
	\[
		\sinh^2(x)=\frac{1}{ 4 }\big(  e^{2x}-2+ e^{-2x} \big).
	\]
	En faisant la différence entre les deux, il reste \( 1\).
\end{proof}
Une propriété qui est par contre très différente entre la trigonométrie plane et la trigonométrie hyperbolique, c'est la périodicité. Les fonctions usuelles \( \cos\) et \( \sin\) sont périodiques. Pas les fonctions hyperboliques.

\begin{proposition}
	La fonction \( \sinh\colon \eR\to\eR \) est bijective.
\end{proposition}

\begin{proof}
	Il faut démontrer que sinus hyperbolique est injective et surjective. Calculons d'abord les limites. Comme tu sais que \( \lim_{x\to\infty} e^{x}=\infty\) et \( \lim_{x\to-\infty} e^{x}=0\), tu vois facilement que
	\begin{align}
		\lim_{x\to-\infty}\sinh(x) & =-\infty & \lim_{x\to\infty}\sinh(x)=\infty.
	\end{align}
	Par ailleurs, la fonction sinus hyperbolique est continue et respecte donc le théorème de la valeur intermédiaire\footnote{Théorème \ref{ThoValInter}.}. Soit \( y\in\eR\). Voyons si il existe un \( x\in\eR\) tel que \( \sinh(x)=y\). Les deux limites indiquent qu'il existe \( x_1\in\eR\) tel que \( \sinh(x_1)<y\) et \( x_2\in\eR\) tel que \( \sinh(x_2)>y\). Le théorème de la valeur intermédiaire conclut qu'il existe un \( x\) entre \( x_1\) et \( x_2\) tel que \( \sinh(x)=y\). Cela prouve la surjectivité.

	Pour l'injectivité, on va utiliser le théorème de Rolle \ref{ThoRolle} et une petite preuve par l'absurde. Supposons que \( \sinh(x_1)=\sinh(x_2)\) avec \( x_1\neq x_2\). Dans ce cas, le théorème de Rolle nous dit qu'il existe un \( x\) entre \( x_1\) et \( x_2\) tel que \( \sinh'(x)=0\). La dérivée de sinus hyperbolique étant cosinus hyperbolique, il faut se demander il existe un \( x\) tel que \( \cosh(x)=0\). Étant donné que \(  e^{x}>0\) pour tout \( x\), en fait le cosinus hyperbolique ne s'annule jamais.
\end{proof}

\newcommand{\CaptionFigSYNKooZBuEWsWw}{En rouge, la fonction \( x\mapsto \sinh(x)\) et en bleu, la fonction \( x\mapsto\cosh(x)\).}
\input{auto/pictures_tex/Fig_SYNKooZBuEWsWw.pstricks}

Un très bon exercice serait de faire un étude complète des fonctions cosinus et sinus hyperbolique. Leur graphes sont donnés à la figure~\ref{LabelFigSYNKooZBuEWsWw}

Un corolaire de la surjectivité de \( \sinh\) sur \( \eR\) est que si je prends n'importe quel deux nombres dont la différence des carrés vaut \( 1\), alors ces carrés sont représentables avec des fonctions hyperboliques :
\[
	\forall x,y\in\eR \text{ tels que } x^2-y^2=1,\,\exists\xi\in\eR \text{ tel que } x^2=\cosh(\xi)\text{ et }y^2=\sinh(\xi).
\]
La \defe{tangente hyperbolique}{Tangente hyperbolique} est définie par
\begin{equation}
	\tanh(x)=\frac{ \sinh(x) }{ \cosh(x) }.
\end{equation}
Un bon exercice est de prouver les deux relations suivantes :
\begin{align}		\label{EqRelSinhthcosh}
	\sinh(x) & =\frac{ \tanh(x) }{ \sqrt{1-\tanh^2(x)} } & \cosh(x) & =\frac{1}{ \sqrt{1-\tanh^2(x)} }.
\end{align}

\subsubsection{Les transformations de Lorentz (bis)}
%///////////////////////////////////////////////////

Nous avons prouvé qu'en relativité, l'intervalle est un invariant. Pour cela, nous avons utilisé les transformations de Lorentz démontrées à partir de l'hypothèse d'invariance de la vitesse de la lumière. Eh bien, oublions un instant que la vitesse de la lumière soit invariante, et posons à la place comme hypothèse que l'intervalle soit invariant. C'est-à-dire que si Bob mesure un événement aux coordonnées \( (t,x)\) et Alice en \( (t',x')\), alors \( c^2t^2-x^2=c^2(t')^2-(x')^2\).

\begin{theorem}
	Les transformations de Lorentz sont les seules qui laissent l'intervalle invariant.
\end{theorem}

\begin{proof}
	Toute la partie comme quoi les transformations doivent êtres linéaires reste parce que cette partie ne demandait pas l'invariance de la vitesse de la lumière.

	Nous cherchons donc les transformations entre Alice et Bob sous la forme
	\[
		\begin{split}
			t'&=\alpha t+\beta x\\
			x'&=\gamma t+\delta x
		\end{split}
	\]
	telles que \( c^2(t')^2-(x')^2=c^2t^2-x^2\). Lorsque Alice passe devant Bob, ils déclenchent tous deux leurs chronomètre et leurs axes. C'est-à-dire que si à ce moment un événement se trouve à droite pour Alice, il est aussi à droite pour Bob. On doit donc avoir, quand \( t=t'=0\), que \( x>0\) implique \( x'>0\). Cela donne la contrainte que \( \delta>0\). D'autre part, comme leurs chronomètres vont dans le même sens (ils choisissent tout les deux de \emph{compter} le temps et non \emph{décompter}), on a \( \alpha>0\).

	En développant l'expression de \( (s')^2\) en termes de \( t\) et \( x\), on trouve la condition d'invariance de l'intervalle sous la forme :
	\begin{equation}	\label{EqCondInvINter}
		c^2(\alpha^2 t^2+2\alpha\beta tx+\beta^2x^2)-(\gamma^2t^2+2\gamma\delta tx+\delta^2x^2)=c^2t^2-x^2,
	\end{equation}
	qui doit être valable pour tout \( t\) et pour tout \( x\). En \( t=0\) on trouve la condition
	\begin{equation}
		\delta^2-c^2\beta^2=1.
	\end{equation}
	Cela implique qu'il existe un \( \xi\in\eR\) tel que \( \delta^2=\cosh^2(\xi)\) et \( c^2\beta^2=\sinh(\xi)\). La première équation donne \( \delta=\cosh(\xi)\) (il faut rejeter \( \delta=-\cosh(\xi)\) parce qu'on a demandé que \( \delta>0\)). Pour la seconde, on trouve \( c\beta=\sinh(\xi)\) où l'on peut oublier la possibilité \( c\beta=-\sinh(\xi)\) parce que cela revient juste à renommer \( \xi\to-\xi\) (la fonction sinus hyperbolique est impaire). Bref, il existe un \( \xi\) tel que
	\begin{equation}	\label{EqCondxi}
		\begin{split}
			\delta&=\cosh(\xi)\\
			\beta&=\frac{\sinh(\xi)}{c}
		\end{split}
	\end{equation}
	En mettant maintenant \( x=0\) dans la condition \eqref{EqCondInvINter}, on trouve la condition
	\[
		\alpha^2-\frac{ \gamma^2 }{ c^2 }=1.
	\]
	Pour les mêmes raisons qu'avant, il existe un \( \eta\in\eR\) tel que
	\begin{equation}	\label{EqCondeta}
		\begin{split}
			\alpha&=\cosh(\eta)\\
			\gamma&=c\sinh(\eta).
		\end{split}
	\end{equation}
	Rien qu'en regardant deux cas très particuliers, on a déjà bien avancé, non ? Remettons maintenant les valeurs \eqref{EqCondxi} et \eqref{EqCondeta} dans la condition \eqref{EqCondInvINter}. En utilisant l'identité \( \cosh^2(x)-\sinh^2(x)=1\), et en séparant les termes en \( t^2\), \( x^2\) et \( tx\) pour satisfaire la condition, il faut
	\begin{equation}	\label{EqConsetaxi}
		\cosh(\eta)\sinh(\xi)=\sinh(\eta)\cosh(\xi)
	\end{equation}
	parce que les termes en \( t^2\) et \( x^2\) donnent exactement \( c^2t^2-x^2\) et qu'il faut que le terme en \( tx\) s'annule. Mettons la condition \eqref{EqConsetaxi} au carré, et substituons \( \cosh^2(\eta)=1+\sinh^2(\eta)\) et \( \cosh^2(\xi)=1+\sinh^2(\xi)\), il reste
	\[
		\sinh^2\xi=\sinh^2\eta,
	\]
	ce qui signifie \( \sinh\xi=\pm\sinh\eta\), ou encore \( \xi=\pm\eta\). On voit que \( \xi=-\eta\) ne fonctionne pas dans \eqref{EqConsetaxi}, donc on reste avec \( \xi=\eta\) et les transformations prennent la forme
	\begin{equation}	\label{EqLorxi}
		\begin{split}
			t'&=\cosh(\xi) t+\frac{ \sinh(\xi) }{ c }x\\
			x'&=c\sinh(\xi) t+\cosh(\xi)x.
		\end{split}
	\end{equation}
	Ce que nous avons prouvé, c'est qu'il existe un \( \xi\in\eR\) tel que les transformations entre Alice et Bob aient cette forme. Il faut trouver ce que vaut \( \xi\) en fonction de la vitesse \( v\) à laquelle Alice court.

	Pour ce faire, étudions le mouvement d'Alice. Bob la voit aux coordonnées \( (t,vt)\), ce qui correspond à
	\[
		x'=c\sinh(\xi)t+\cosh(\xi)vt
	\]
	pour Alice. Mais ces coordonnées sont celles de Alice elle-même, donc \( x'=0\), ce qui donne\footnote{Conditions d'existence : \( \cosh(\xi)\neq 0\); heureusement, nous avons vu que le cosinus hyperbolique ne s'annule jamais.} \( vt=- c\sinh(\xi)t/ \cosh(\xi)\), ou encore
	\begin{equation}
		\tanh(\xi)=-\frac{ v }{ c }
	\end{equation}
	En utilisant les relations \eqref{EqRelSinhthcosh}, on trouve
	\begin{align}
		\cosh(\xi) & =\frac{1}{ \sqrt{1-\frac{ v^2 }{ c^2 }} } & \sinh(\xi) & =\frac{ -v/c }{ \sqrt{1-\frac{ v^2 }{ c^2 }} }.
	\end{align}
	En remettant ces valeurs dans les transformations \eqref{EqLorxi}, on trouve
	\begin{align}
		t' & =\frac{ t-\frac{ v }{ c^2 }x }{ \gamma(v) } \\
		x' & =\frac{ x-vt }{ \gamma(v) },
	\end{align}
	exactement les transformations de Lorentz !

\end{proof}


Ce résultat est important pour une raison assez simple : maintenant, la théorie de la relativité est indépendante de toute considérations sur la lumière. En effet, ce que nous venons de prouver, c'est que si il existe une vitesse \( c\) telle que \( c^2t^2-x^2=c^2(t')^2-(x')^2\), alors \( (t,x)\) et \( (t',x')\) sont liés par les transformations de Lorentz.

%---------------------------------------------------------------------------------------------------------------------------
\subsection{Vitesse limite}
%---------------------------------------------------------------------------------------------------------------------------

Afin de nous passer de l'hypothèse d'invariance de la vitesse de la lumière, nous avons prouvé que l'hypothèse d'invariance de l'intervalle était suffisante. Mais il faut avouer que cette hypothèse n'est pas très intuitive. Nous allons montrer maintenant que l'existence d'une vitesse limite est une troisième hypothèse qui peut être utilisée comme alternative aux deux premières.

\section{Applications}
%++++++++++++++++++++

Une première application sympa est le logiciel\footnote{jeu de mot sur « application » ! ah ah !} \emph{lightspeed}. Si tu es sous Ubuntu-Linux, installe juste le paquet nommé lightspeed, et régales-toi ! Tu verras c'est marrant. Si tu utilise des fenêtres, je laisse faire l'adage « Windows c'est facile ».

\subsection{Le GPS}
%------------------

%http://fr.wikipedia.org/wiki/Global_Positioning_System
Pour qu'un système \href{http://fr.wikipedia.org/wiki/Global\_Positioning\_System}{GPS} puisse te localiser, en gros, il t'envoie un signal, tu lui réponds et il mesure le temps qu'il a fallu à la lumière pour faire l'aller-retour. Déjà, tu remarques que cela n'est possible que grâce au fait que la vitesse de la lumière soit finie. Sinon, le GPS ne fonctionnerait pas. Mais il y a mieux.

Comme pour te localiser il faut plusieurs satellites en plus de ton appareil, il faut que les horloges internes de tout ce petit monde soient bien synchronisées, sinon pour mesurer des intervalles de temps et calculer des distances, c'est mal parti. Eh mais un satellite, ça bouge assez vite (surtout que les mesures doivent être très précises), et en plus ça ne fait même pas un MRU, vu que ça tourne en rond. Comme tu vois tout le travail qu'il a fallu faire pour trouver les transformations de Lorentz d'un MRU, tu t'imagines le travail pour un mouvement circulaire ! Eh bien ce travail a été fait, et le résultat est que si on en tient pas compte, les contractions temporelles liées à la relativité sont suffisamment grandes pour complètement dérégler le GPS.


\subsection{Les ondes électromagnétiques}
%----------------------------------------

Tu te souviens qu'au début du chapitre, nous avons dit que le problème qui a amené la relativité était la propagation du champ électrique. Maintenant que nous avons déjà vu une partie des conséquences du problème, il est temps de se rendre compte que les champs électriques et magnétiques sont les objets les plus soumis aux bizarreries relativistes du monde : elles se propagent à la vitesse de la lumière. Regarde un coup autour de toi; tout ce qui est champ électromagnétique a besoin de la relativité pour être bien compris : GSM, lumière, four à micro-onde, radio, wifi, fibre optique, \ldots

Si un jour un ingénieur te dit qu'il n'y a pas besoin de connaitre la relativité pour inventer la radio (c'est vrai : la radio a été inventée avant la relativité), ni pour construire une fibre optique, dis lui en pensant à moi qu'il utilise tout le temps les équations de Maxwell\footnote{C'est sous ce nom là qu'on nomme l'ensemble des équations de l'électromagnétisme comme la loi de l'induction.}, et que ces équations sont relativistes.

Bref, soit convaincu que tu vis dans un monde relativiste et que les transformations de Lorentz te suivent à chacun de tes pas.

\section{Mécanique relativiste}
%++++++++++++++++++++++++++++++

Cela est bien beau, mais la dilatation du temps, et les contractions de longueurs doivent bien avoir des répercussions sur la cinématique et la dynamique des objets. Est-ce que le théorème de l'énergie cinétique est encore valable ? est-ce que les lois de Newton tiennent encore la route ?


\subsection{Des problèmes, toujours des problèmes}
%-------------------------------------------------

%http://fr.wikipedia.org/wiki/Les_Chevaliers_du_Zodiaque

Attardons-nous un peu pour faire quelques commentaires sur cette citation du chevalier pégase dans \href{http://fr.wikipedia.org/wiki/Les_Chevaliers_du_Zodiaque}{les chevaliers du zodiaque} :
\begin{quote}
	Ses coups vont à la vitesse de la lumière et pourtant je les vois distinctement arriver.
\end{quote}
Est-ce possible ? Nous avons vu qu'il y avait des dénominateurs qui s'annulent quand des objets se déplacent plus vite que la lumière; or pour voir venir un rayon de lumière qui vient vers soi, il faudrait que le rayon émette de la lumière devant elle. Ça semble un peu mal parti pour respecter les lois de la relativité, non ?

Cela pose en tout cas une question qu'il faudra résoudre. On \emph{entend} venir une ambulance parce qu'elle émet du son qui avance plus vite qu'elle. Pas de problèmes avec ça. Mais quid de la \emph{voir} venir ?

On peut voir venir un tram parce qu'il émet de la lumière; cette lumière allant plus vite que le tram, elle arrive à nos yeux avant le tram lui-même. Cela est très bien. Mettons que le tram avance à \unit{50}{\kilo\meter\per\hour}; pour le conducteur, la lumière de son phare avant avance devant lui à la vitesse \( c\). Par conséquent pour un observateur au sol, cette même lumière devrait avancer à la vitesse \( c+50\). Encore une fois, on a un problème d'invariance de la vitesse de la lumière; mais comme c'est de la lumière, on est habitué à ce que des trucs bizarres arrivent. On ne sera pas étonné que \( c+50=c\) d'une manière ou d'une autre\footnote{et je ne te cache pas que c'est ce qui va arriver.}. Pire. Si un vaisseau spatial avance à la vitesse \unit{200000}{\kilo\meter\per\second} et qu'il envoie en reconnaissance un vaisseau devant lui à la vitesse de \unit{150000}{\kilo\meter\per\second}, le vaisseau de reconnaissance ira à la vitesse \unit{150000}{\kilo\meter\per\second} par rapport au vaisseau principal. Et par rapport au sol, il ira à la vitesse \( \unit{150000+200000=350000}{\kilo\meter\per\second}\), ce qui est impossible. Il faudra trouver quelque chose pour que ça se passe bien.

Un autre problème maintenant.

Prenons une masse \( m\) que l'on soumet à une force constante \( F\). Par la loi de Newton, \( a=F/m\) est constante et la vitesse après un temps \( t\) vaut \( v=Ft/m\). Pas de bol, ça devient plus grand que la vitesse de la lumière à partir du temps \( t=cm/F\). Ça est un problème hein ? Il faut trouver un truc pour qu'avec une force constante, l'accélération diminue.

\subsection{Loi d'addition des vitesses}
%---------------------------------------

Si Bob observe un objet se déplacer à la vitesse \( V\), alors Alice devrait l'observer bouger à la vitesse \( V-v\). Tout comme si une vache voit passer un train à \unit{90}{\kilo\meter\per\hour}, alors le vélo qui avance à \unit{25}{\kilo\meter\per\hour} le voit passer à \unit{65}{\kilo\meter\per\hour}.

Maintenant, tu es habitué à ce que rien ne se passe comme d'habitude, donc tu te doutes bien qu'en réalité la bonne formule ne va pas être \( V-v\).

Bob observe l'objet aux coordonnées \( (t,Vt)\), ce qui fait pour Alice :
\[
	\left( \frac{ t-\frac{ v }{ c^2 }Vt }{ \gamma(v) },\frac{ Vt-vt }{ \gamma(v) } \right).
\]
En divisant le \( x'\) d'Alice par le \( t'\) d'Alice, on trouve la vitesse mesurée par Alice :
\[
	V'=\frac{ (V-v)t }{ \gamma(v) }\frac{ \gamma(v) }{ t\left( 1-\frac{ vV }{ c^2 } \right) }=\frac{ V-v }{ 1-\frac{ vV }{ c^2 } }.
\]
La loi de transformation des vitesses relativiste est donc
\begin{equation}	\label{EqAddRelVit}
	V'=\frac{ V-v }{ 1-\frac{ vV }{ c^2 } }.
\end{equation}
Qu'en est-il de notre \( c+50=c\) ? Disons que Bob lance un bisou à Alice pendant qu'elle arrive vers lui. Le bisou arrive à la vitesse de la lumière (càd \( V=c\)) tandis que Alice s'approche de Bob à la vitesse \unit{50}{\meter\per\second} (càd \( v=50\)). Donc la vitesse à laquelle Alice devrait voir arriver le bisou est bien \( c+50\). En utilisant la formule d'addition relativiste des vitesses \eqref{EqAddRelVit}, nous trouvons
\[
	V'=\frac{ c-50 }{ 1-\frac{ 50c }{ c^2 } }=\frac{ c-50 }{ 1-\frac{ 50 }{ c } }=\frac{ c(c-50) }{ c-50 }=c.
\]
Donc effectivement en relativité quand on additionne des vitesses il faut penser à la règle du « \( c+50=c\) ».

\subsection{L'action d'une force}
%--------------------------------

L'équation fondamentale de la mécanique classique est
\[
	F=ma.
\]
Or tu n'es pas sans savoir que l'accélération est la dérivée seconde de la position par rapport au temps. Nous noterions donc \( F=mx''(t)\). Le problème est évidemment que si \( F\) est constante, on trouve \( v=Ft/m\) qui dépasse toujours la vitesse \( c\) quand \( t\) est assez grand. Il faudra donc modifier la loi \( F=ma\). Pour cela, posons-nous des questions sur la dérivée \( x'(t)\). On dérive par rapport au temps; oui mais nous avons vu que le temps n'est pas le même pour tout le monde. Introduisons donc la notation
\begin{equation}	\label{Eqvdxdt}
	v=\frac{ dx }{ dt }
\end{equation}
qui ne signifie rien d'autre que nous dérivons \( x\) par rapport à \( t\) et non par rapport au temps \( t'\) de quelqu'un d'autre. Dans le cadre de la relativité, ce que signifie l'équation \eqref{Eqvdxdt} est que \( v\) est la dérivée de \( x\) par rapport à \( t\). Dans le cas où \( x\) et \( t\) sont les coordonnées de la position d'Alice mesurées par Bob, cela signifie qu'on dérive la position \emph{mesurée par Bob} par rapport au temps \emph{mesuré par Bob}.

Ce que dit la relativité est que cette quantité \( v\) ne peut pas varier proportionnellement à la force sous peine de dépasser la vitesse de la lumière. La subtilité est de modifier la loi de Newton en disant que la quantité qui varie sous l'action d'une force n'est plus \( dx/dt=v\), mais
\[
	\frac{ dx }{ dt' }=   \frac{ v }{ \sqrt{1-\frac{ v^2 }{ c^2 }} },
\]
c'est-à-dire la dérivée de la position \emph{mesurée par Bob} par rapport au temps \emph{mesuré par Alice} !  La loi de Newton \( v=Ft/m\) devient donc
\begin{equation}	\label{EsNEwModif}
	\frac{ v }{ \sqrt{1-\frac{ v^2 }{ c^2 }} }=\frac{ Ft }{ m }.
\end{equation}
Est-ce que cela résoud le problème ? Pour le savoir, regardons la vitesse acquise par le mobile de masse \( m\) soumit à la force \( F\) pendant un temps \( t\) Il faut résoudre l'équation \eqref{EsNEwModif} par rapport à \( v\) et voir si cela reste bien toujours inférieur à \( c\). On commence par mettre la racine à droite et à élever toute l'équation au carré :
\[
	\begin{split}
		v^2&=\frac{ F^2t^2 }{ m^2 }\left( 1-\frac{ v^2 }{ c^2 } \right)\\
		v^2\left( 1+\frac{ F^2t^2 }{ c^2m^2 } \right)&=\frac{ F^2t^2 }{ m^2 }\\
		v&=\frac{ \sqrt{F^2t^2/m^2} }{ \sqrt{1+\frac{ F^2t^2 }{ c^2m^2 }} },
	\end{split}
\]
et donc finalement
\begin{equation}	\label{EqVfntRel}
	v(t)=\frac{ Ft }{ m\sqrt{1+\frac{ F^2t^2 }{ c^2m^2 }} }.
\end{equation}
Tu dois remarquer que si \( F\) et \( t\) ne sont pas trop grands, l'expression \( F^2t^2/c^2m^2\) est minuscule parce que \( c\) est énorme. Si on fait l'approximation \( F^2t^2/c^2m^2=0\) dans cette expression, on retrouve \( v=Ft/m\). Cela montre qu'à moins de faire des expérience avec de très grandes forces pendant énormément de temps, on ne peut pas voir la différence entre la mécanique de Newton et la mécanique relativiste.

\begin{center}
	\input{auto/pictures_tex/Fig_KKJAooubQzgBgP.pstricks}
\end{center}

Sur le graphe suivant, la vitesse en fonction du temps lorsqu'une particule de masse \( m=1\) est soumise à une force constante. Pour les besoins du graphique, nous avons mis à \( 1\) la vitesse \( c\). Tu vois que quand la vitesse n'est pas très grande, le graphique est presque celui d'une droite; et à partir d'un certain moment, la courbe s'infléchit pour tendre vers \( 1\) sans l'atteindre.

Remarque que si on maintient une accélération constante égale à celle de la gravité terrestre pendant deux heures, on arrive déjà sur la Lune, à une vitesse de \unit{75}{\kilo\meter\per\second}, c'est-à-dire encore rien par rapport à la vitesse de la lumière ! Cela pour te dire que la formule \eqref{EqVfntRel} a l'air d'être très différente de la formule classique \( v=Ft/m\), mais en réalité tant qu'on n'atteint pas des forces énormes, elle ressemble très fort.

Vérifions maintenant que la formule \eqref{EqVfntRel} n'est pas en contradiction avec l'impossibilité de dépasser la vitesse de la lumière. Pour cela, regardons ce qu'il se passe si on applique une force constante \( F\) sur un objet de masse \( m\) pendant un temps très long. C'est-à-dire : calculons la limite
\[
	\lim_{t\to\infty}v(t).
\]
Tu vois tout de suite qu'on est sur un cas \( \frac{ \infty }{ \infty }\), ce qui t'oblige à utiliser la règle de l'Hospital. On peut cependant un peu réfléchir et deviner la réponse sans passer par des math trop compliquées.

En effet, quand \( t\) est vraiment énorme, l'expression \( \frac{ F^2t^2 }{ m^2c^2 }\) devient très grande, et le \( 1\) qui se trouve à côté ne vaut plus grand chose, on peut le négliger.
\begin{equation}
	\begin{split}
		\lim_{t\to\infty}\frac{ Ft }{ m\sqrt{1+\frac{ F^2t^2 }{ c^2m^2 }} }&=\lim_{t\to\infty}\frac{ Ft }{ m\sqrt{\frac{ F^2t^2 }{ m^2c^2 }} }\\
		&=\lim_{v\to\infty}\frac{ Ft }{ m\frac{ Ft }{ cm } }\\
		&=c.
	\end{split}
\end{equation}
Tout est bien : on arrive au maximum à la vitesse de la lumière, mais il faut un temps infini pour y parvenir. Conclusion : il n'est pas possible d'accélérer un objet jusqu'à atteindre la vitesse de la lumière.

\subsection{Équivalence entre la masse et l'énergie}
%---------------------------------------------------

Le moment est venu de montrer ce que signifie la fameuse formule \( E=mc^2\).

\section{Principe de correspondance}
%++++++++++++++++++++++++++++++++++++

Nous ne sommes pas parvenu à démontrer la formule \eqref{EsNEwModif} de la mécanique relativiste qui montre comme un objet accélère sous l'effet d'une force constante. Nous avons juste montré qu'il fallait modifier la loi \( v=Ft/m\) et nous avons prit la première modification qui nous soit tombée sous la main, à savoir qu'il faut dériver la position par rapport au temps de l'objet qu'on observe plutôt que par rapport au temps de l'observateur.

En fait, il est possible de prouver rigoureusement\footnote{Mais il n'existe pas de démonstrations simples à ma connaissance.} la formule
\[
	\frac{ Ft }{ m }=\frac{ \alpha v }{ \sqrt{1-\frac{ v^2 }{ c^2 }} }.
\]
Mais il n'y a pas moyen de trouver la valeur de la constante \( \alpha\). Tout ce qu'il y a moyen de trouver avec l'hypothèse de l'invariance de la vitesse de la lumière est l'existence d'une constante telle que cette formule soit vraie.

Afin de fixer la constante \( \alpha\), il faut faire intervenir un principe physique supplémentaire, le \defe{principe de correspondance}{Principe de correspondance}
\begin{loiphyz}
	Lorsque la vitesse d'une particule est faible, les équations doivent être en première approximation les mêmes que celles de la mécanique classique.
\end{loiphyz}
Que signifie \emph{en première approximation} ? Tu sais qu'une fonction \( x\mapsto f(x)\) peut être approximée (pour des petits \( x\)) par la formule
\[
	f(x)\simeq f(0)+xf'(0).
\]
Nous voudrions donc que \( Ft/m\) soit en première approximation égal à \( v\). Nous devons étudier la fonction
\[
	f(v)=\frac{ \alpha v }{ \sqrt{1-\frac{ v^2 }{ c^2 }} }.
\]
Voir ce que vaut cette fonction en première approximation lorsque \( v\) est petit est un exercice de dérivation. En utilisant la règle de dérivation des fractions, on trouve que
\[
	f'(v)=\frac{ \alpha }{ \sqrt{1-\frac{ v^2 }{ c^2 }} }+\frac{ \alpha v^2 }{ c^2\left( 1-\frac{ v^2 }{ c^2 } \right)^{3/2} },
\]
et donc que \( f'(0)=\alpha\). Bien entendu, \( f(0)=0\). En première approximation, nous trouvons donc
\begin{equation}
	f(v)\simeq \alpha v
\end{equation}
qui doit être égal à la quantité non relativiste \( v\). Nous en déduisons qu'il faut fixer \( \alpha=1\), et on tombe sur la formule relativiste proposée plus haut
\[
	\frac{ Ft }{ m }=\frac{ v }{ \sqrt{1-\frac{ v^2 }{ c^2 }} }
\]

L'utilisation cruciale du principe de correspondance a une répercussion énorme sur notre vision de la physique. En effet, la relativité d'Einstein ne parvient pas à \emph{remplacer} la mécanique de Newton. On a besoin d'invoquer la mécanique de Newton pour fixer la théorie. On peut écrire l'axiome suivant :
\begin{equation}
	\lim_{v\to 0}\text{Einstein}=\text{Newton}.
\end{equation}
Cela n'est pas une propriété de la théorie d'Einstein, mais un de ses axiomes !

La relativité ne fait donc pas table rase des principes physiques de la mécanique newtonienne : elle les complète et les contient.

% Le fichier
% http://isites.harvard.edu/fs/docs/icb.topic192578.files/chap11.pdf
% donne des précisions sur comment faire du Lorentz sans faire appel à l'invariance de la lumière.
% Il est dans ta littérature sur les cours d'humanité
