% This is part of Mes notes de mathématique
% Copyright (c) 2024
%   Laurent Claessens
% See the file fdl-1.3.txt for copying conditions.

%---------------------------------------------------------------------------------------------------------------------------
\subsection{L'espace \texorpdfstring{\(  L^{\infty}\)}{Linfinity}}
%---------------------------------------------------------------------------------------------------------------------------
\label{SUBSECooYFJTooBqrLXv}


Ne pas confondre la norme suprémum (définition \ref{DEFooSFNFooBygeXX}) avec la norme \( L^{\infty}\), bien que les deux peuvent être écrites \( \| f \|_{\infty}\).

\begin{normaltext}
	Il n'est pas possible de définir le supremum d'une fonction définie à ensemble de mesure nulle près parce que toute classe contient des fonctions qui peuvent être arbitrairement grandes en n'importe que point. Nous cherchons alors à définir une notion de supremum qui ne tient pas compte des ensembles de mesure nulle.
\end{normaltext}

\begin{definition}      \label{DEFooIQOOooLpJBqi}
	Soit \( f\colon \Omega\to \eC\). Un nombre \( M\) est un \defe{majorant essentiel}{majorant!essentiel} de \( f\) si
	\begin{equation}
		\mu\big( | f(x) |\geq M \big)=0.
	\end{equation}
	Nous posons alors
	\begin{equation}
		N_{\infty}(f)=\inf\{ M\tq | f(x) |\leq M\text{ presque partout} \}.
	\end{equation}

	Le nombre (peut-être infini) \( N_{\infty}(f)\) sera souvent écrite \( \| f \|_{L^{\infty}}\) ou \( \| f \|_{\infty}\). Il est nommé le \defe{suprémum esstiel}{supremum essentiel} de \( f\). Il pourra également être noté
	\begin{equation}		\label{EQooZHKBooPgdzCp}
		\esssup_{x\in \Omega}| f(x) |.
	\end{equation}
\end{definition}
Cela revient à prendre le supremum à ensemble de mesure nulle près.

\begin{definition}      \label{DEFooXUKHooXYrlYq}
	Nous définissons alors les espaces de Lebesgue correspondants :
	\begin{equation}
		\mL^{\infty}(\Omega)=\{ f\colon \Omega\to \eC\tq N_{\infty}(f)<\infty \},
	\end{equation}
	et \( L^{\infty}\) en est le quotient usuel.
\end{definition}

\begin{normaltext}
	Le point sur les notations telles qu'elles devraient être respectées :
	\begin{enumerate}
		\item
		      Si \( f\) est une fonction, \( N_{\infty}(f)\) est son supremum essentiel.
		\item
		      Si \( f\) est une fonction, \( \| f \|_{\infty}\) est sa norme supremum. Ce n'est pas la même chose que \( N_{\infty}(f)\).
		\item
		      Si \( [f]\) est une classe de fonctions (pour l'égalité presque partout), alors \( \| [f] \|_{L^{\infty}}\) est la norme de cette classe dans \( L^{\infty}\), c'est-à-dire \( \| [f] \|_{L^{\infty}}=N_{\infty}(f)\) où \( f\) est un représentant.
	\end{enumerate}
	Le point sur les abus tolérables :
	\begin{enumerate}
		\item
		      Si \( f\) est une fonction, on peut écrire \( \| f \|_{L^{\infty}}\) pour \( N_{\infty}(f)\). Attention toutefois que \( N_{\infty}(f)\) peut valoir \( \infty\), alors que les éléments de \( L^{\infty}\) sont sélectionnés pour être les classes des fonctions telles que \( N_{\infty}(f)<\infty\). Donc il est parfois possible, lorsque \( f\) est une fonction, de parler de \( \| f \|_{L^{\infty}}\) alors que \( [f]\) n'est pas un élément de \( L^{\infty}\).
		\item
		      Si \( [f]\) est une classe, on peut écrire \( N_{\infty}([f])\) pour \( \| [f] \|_{L^{\infty}}\).
		\item
		      Noter \( f\) la classe de la fonction \( f\). Attention qu'alors, écrire \( \| f \|_{\infty}\) n'a pas de sens.
	\end{enumerate}
	Le point sur les abus intolérables :
	\begin{enumerate}
		\item
		      Si \( f\) est une fonction, noter \( \| f \|_{\infty}\) pour \( N_{\infty}(f)\).
		\item
		      Si \( [f]\) est une classe de fonctions, noter \( \| [f] \|_{\infty}\) pour \( \| [f] \|_{L^{\infty}}\).
	\end{enumerate}
\end{normaltext}

\begin{normaltext}
	Tout ceci pour dire que si \( f_k\) sont des fonctions et si \( f\) est une fonction, nous avons la convergence
	\begin{equation}
		f_k\stackrel{L^{\infty}}{\longrightarrow}f
	\end{equation}
	si et seulement si \( N_{\infty}(f_k-f)\to 0\). Cette convergence (qui se sert des abus tolérables de notations) est équivalente à la convergence
	\begin{equation}
		[f_k]\stackrel{L^{\infty}}{\longrightarrow}[f].
	\end{equation}
	Mais ce n'est pas du tout équivalent à \( \| f_k -f \|_{\infty}\to 0\). Tout au plus, il est vrai que si \( \alpha\in L^{\infty}\) (donc \( \alpha\) est une classe), il existe un représentant \( f\in \alpha\) tel que \( \| f \|_{\infty}=\|\alpha\|_{L^{\infty}}\).
\end{normaltext}

\begin{lemma}[\cite{MonCerveau}]
	Si \( f_k\) est une suite de fonctions telle que \( \| f_k-f \|_{\infty}\to 0\), alors
	\begin{equation}
		f_k\stackrel{L^{\infty}}{\longrightarrow}f
	\end{equation}
	où la convergence signifie \( N_{\infty}(f_k-f)\to 0\).
\end{lemma}

\begin{proof}
	Si \( g\) est une fonction nous avons toujours \( N_{\infty}(g)\leq \| g \|_{\infty}\). Donc
	\begin{equation}
		N_{\infty}(f_k-f)\leq \| f_k-f \|_{\infty}\to 0.
	\end{equation}
\end{proof}

\begin{normaltext}
	Attention toutefois que ce lemme ne signifie pas que si \( \| f_k-f \|_{\infty}\to 0\), alors \(  [f_k]\stackrel{L^{\infty}}{\longrightarrow}[f]  \) parce que nous pourrions avoir \( N_{\infty}(f_k)=\infty\) et alors \( [f_k]\) n'est pas un élément de \( L^{\infty}\).

	Cela arrive par exemple pour \( f_k(x)=x\) et \( f(x)=x\). Nous avons \( \| f_k-f \|_{\infty}=0\) pour tout \( k\), alors que ni \( f_k\) ni \( f\) ne donnent lieu à un élément de \( L^{\infty}\).
\end{normaltext}

\begin{example}
	Même si \( g\) est bornée, nous n'avons pas spécialement \( \| g \|_{\infty}=N_{\infty}(g)\). Par exemple
	\begin{equation}
		\begin{aligned}
			g\colon \eR & \to \eR                    \\
			x           & \mapsto \begin{cases}
				                      1 & \text{si } x=0 \\
				                      0 & \text{sinon }.
			                      \end{cases}
		\end{aligned}
	\end{equation}
	Cette fonction vérifie \( \| g \|_{\infty}=1\) mais \( N_{\infty}(g)=0\).
\end{example}

%---------------------------------------------------------------------------------------------------------------------------
\subsection{Quelques identifications}
%---------------------------------------------------------------------------------------------------------------------------

Il est intuitivement clair que ce qui peut arriver à une fonction en un seul point ne va pas influencer la fonction lorsqu'elle est vue dans \( L^p\). En tout cas lorsqu'on considère des mesures pour lesquelles les singletons sont de mesure nulle, et c'est bien le cas de la mesure de Lebesgue. Il est peut-être intuitivement moins clair que l'on peut non seulement modifier le comportement d'une fonction en un point, mais également modifier l'ensemble de base. En voici un exemple.

\begin{proposition}
	Nous avons les égalités suivantes d'espaces
	\begin{equation}
		L^p\big( \mathopen] 0 , 2\pi \mathclose[ \big)=L^p\big( \mathopen[ 0 , 2\pi \mathclose] \big)=L^p(S^1)
	\end{equation}
	au sens où il existe des bijections isométriques de l'un à l'autre. Ici nous sous-entendons la mesure de Lebesgue partout\footnote{Vu que la mesure de Lebesgue est définie pour \( \eR^d\) munie de sa tribu des boréliens (complétée), vous êtes en droit de vous demander quelle est la tribu et la mesure que nous considérons sur le cercle \( S^1\).}.
\end{proposition}


\begin{proof}
	Voici une application où le crochet dénote la prise de classe :
	\begin{equation}
		\begin{aligned}
			\psi\colon L^p\big(\mathopen] 0 , 2\pi \mathclose[\big) & \to L^p\big(\mathopen[ 0 , 2\pi \mathclose]\big)                                          \\
			[f]                                                     & \mapsto \text{la classe de } f_e(x)=\begin{cases}
				                                                                                              f(x) & \text{si } x\in\mathopen] 0 , 2\pi \mathclose[ \\
				                                                                                              0    & \text{si } x=0\text{ ou } x=2\pi.
			                                                                                              \end{cases}
		\end{aligned}
	\end{equation}
	\begin{subproof}
		\spitem[Bien définie]
		%-------------------------------------------------------------
		Si \( [f]=[g]\) dans \( L^p\big(\mathopen] 0 , 2\pi \mathclose[\big)\) alors \( f_e(x)=g_e(x)\) pour tout \( x\in \mathopen] 0 , 2\pi \mathclose[ \) sauf une partie de mesure nulle. L'union de cette partie avec \( \{ 0,2\pi \}\) est encore de mesure nulle dans \( \mathopen[ 0 , 2\pi \mathclose]\). Les images par \( \psi\) sont donc égales dans \( L^p\big( \mathopen[ 0 , 2\pi \mathclose] \big)\).

		\spitem[Injective]
		%-----------------------------------------------------------
		Supposons que \( [f], [g]\in L^p\big( \mathopen] 0,2\pi\mathclose[ \big)\) soient tels que \( \psi\big( [f] \big)=\psi\big( [g] \big)\). En particulier \( [f_e]=[g_e]\) et donc \( f_e=g_e\) presque partout sur \( \mathopen[ 0,2\pi\mathclose]\). Vu que \( \{ 0 \}\cup\{ 2\pi \}\) est de mesure nulle, nous avons \( f_e=g_e\) presque partout sur \( \mathopen] 0,2\pi\mathclose[\). Mais pour \( x\in\mathopen] 0,2\pi\mathclose[\), nous avons \( f_e(x)=f(x)\) et \( g_e(x)=g(x)\). Bref nous avons \( f(x)=g(x)\) pour presque tout \( x\in\mathopen] 0,2\pi\mathclose[\).

		\spitem[Surjective]
		%-------------------------------------------------------------
		Un élément de \( L^p\big( \mathopen[ 0 , 2\pi \mathclose] \big)\) est l'image de sa restriction \ldots\ ou plutôt l'image de la classe de la restriction d'un quelconque de ses représentants.
		\spitem[Isométrie]
		%-------------------------------------------------------------
		L'intégrale qui donne la norme sur \( L^p\) ne change pas selon que nous ajoutions ou non les bornes au domaine d'intégration.
	\end{subproof}

	De la même manière nous avons
	\begin{equation}
		L^p\big( \mathopen[ 0 , 2\pi \mathclose[ \big)=L^p\big( \mathopen[ 0 , 2\pi \mathclose] \big).
	\end{equation}

	En ce qui concerne l'identification avec \( L^p(S^1)\), il faut passer par l'isométrie \( \varphi\colon \mathopen[ 0 , 2\pi \mathclose[\to S^1\) donnée par \( \varphi(t)= e^{it}\), et être heureux que ce soit bien une isométrie parce qu'il faudra l'utiliser pour un changement de variables pour montrer que
	\begin{equation}
		\int_0^{2\pi}f(t)dt=\int_{S^1}(f\circ\varphi^{-1})(z)dz.
	\end{equation}
\end{proof}

%---------------------------------------------------------------------------------------------------------------------------
\subsection{Inégalité de Young, Jensen, Hölder et de Minkowski}
%---------------------------------------------------------------------------------------------------------------------------

\begin{proposition}[Inégalité de Young\cite{BIBooBDWZooTNOtoB,BIBooAGENooJauerr}]     \label{PROPooCQUBooCvtMSi}
	Soient \( a,b\geq 0\) ainsi que \( p,q>0\) tels que
	\begin{equation}
		\frac{1}{ p }+\frac{1}{ q }=1.
	\end{equation}
	Alors
	\begin{equation}
		ab\leq \frac{ a^p }{ p }+\frac{ b^q }{ q }.
	\end{equation}
	Il y a égalité si et seulement si \( a^p=b^q\).
\end{proposition}

\begin{proof}
	Nous fixons \( b\geq 0\), et nous considérons la fonction
	\begin{equation}
		\begin{aligned}
			f\colon \eR^+ & \to \eR                                         \\
			x             & \mapsto \frac{ x^p }{ p }+\frac{ b^q }{ q }-xb.
		\end{aligned}
	\end{equation}
	On remarque vite fait que \( f'(x)=x^{p-1}-\). Nous avons \( f'(x)>0\) si et seulement si \( x^{p-1}\geq b\). Nous posons \( x_0=b^{1/(p-1)}\), et nous savons que \( f\) est strictement décroissante sur \( \mathopen[ 0,x_0\mathclose]\) et strictement croissante sur \( \mathopen[ x_0,\infty\mathclose[\). Cette fonction a donc un minimum global en \( x=x_0\). Rapide calcul :
	\begin{equation}
		f(x_0)=\frac{ b^{p/(p-1)} }{ p }+\frac{ b^q }{ q }-b^{1+1/(p-1)}.
	\end{equation}
	En utilisant \( q=p(p-1)\) nous trouvons vite que
	\begin{equation}
		f(x_0)=b^q\left( \frac{1}{ p}+\frac{1}{ q}-1 \right)=0.
	\end{equation}
	Donc pour tout \( x\neq x_0\) nous avons \( f(x)>f(x_0)=0\), c'est à dire
	\begin{equation}
		\frac{ a^p }{ p }+\frac{ b^q }{ q }\geq ab
	\end{equation}
	avec égalité si et seulement si \( a=x_0\), c'est à dire quand \( a=b^{1/(p-1)}\) ou encore, en passant à la puissance \( p\) :
	\begin{equation}
		a^p=b^{p/(p-1)}=b^q.
	\end{equation}
\end{proof}

\begin{proposition}[Inégalité de Jensen\cite{MesIntProbb}] \label{PropXISooBxdaLk}
	Soit un espace mesuré de probabilité\footnote{C'est-à-dire que \( \int_{\Omega}d\mu=1\).} \( (\Omega,\tribA,\mu)\) ainsi qu'une fonction convexe \( f\colon \eR\to \eR\) et une application \( \alpha\colon \Omega\to \eR\) tels que \( \alpha\) et \( f\circ \alpha\) soient intégrables sur \( \Omega\). Alors
	\begin{equation}
		f\Big( \int_{\Omega}\alpha\,d\mu \Big)\leq \int_{\Omega}(f\circ\alpha) d\mu.
	\end{equation}
\end{proposition}
\index{inégalité!Jensen!version intégrale}

\begin{proof}
	Soient \( a\in \eR\) et \( c_a\) le nombre donné par la proposition~\ref{PropNIBooSbXIKO} : pour tout \( \omega\in \Omega\) nous avons
	\begin{equation}    \label{EqOUMooIwknIP}
		f\big( \alpha(\omega) \big)-f(a)\geq c_a\big( \alpha(\omega)-a \big).
	\end{equation}
	Cela est en particulier vrai pour \( a=\int_{\Omega}\alpha\,d\mu\). Nous intégrons l'inégalité \eqref{EqOUMooIwknIP} sur \( \Omega\) en nous souvenant que \( \int d\mu=1\) :
	\begin{subequations}
		\begin{align}
			\int_{\Omega}(f\circ \alpha)d\mu-\int_{\Omega}f(a)d\mu & \geq c_a\big( \int_{\Omega}\alpha-\int_{\Omega}a \big) \\
			\int_{\Omega}(f\circ \alpha)d\mu-f(a)                  & \geq c_a(a-a)                                          \\
			f(a)                                                   & \leq \int_{\Omega}(f\circ\alpha)d\mu.
		\end{align}
	\end{subequations}
	Cette dernière inégalité est celle que nous devions prouver.
\end{proof}

\begin{corollary}		\label{CORooRYUAooOHRsjL}
	Soit un espace mesuré de probabilité \( (\Omega,\tribA,\mu)\) et une application \( \alpha\in L^1(\Omega,\mu)\) et \( \alpha\in L^p(\Omega)\) avec \( 1\leq p<+\infty\). Alors
	\begin{equation}
		| \int_{\Omega}\alpha(s)d\mu(s) |\leq \| \alpha \|_p.
	\end{equation}
\end{corollary}

\begin{proof}
	Il suffit d'utiliser l'inégalité de Jensen sur la fonction convexe \( f(x)=| x |^p\). Nous avons alors
	\begin{equation}
		| \int_{\Omega}\alpha(s)d\mu(s) |^p\leq \int_{\Omega}| \alpha(s) |^pd\mu(s),
	\end{equation}
	c'est-à-dire
	\begin{equation}
		| \int_{\Omega}\alpha(s)d\mu(s) |\leq  \left[  \int_{\Omega}| \alpha(s) |^pd\mu(s)\right]^{1/p}=\| \alpha \|_p
	\end{equation}
	où ma norme \( \| . \|_p\) est prise au sens de la mesure \( \mu\).
\end{proof}

\begin{proposition}[Inégalité de Hölder\cite{BIBooAUHHooGDOxIA}]       \label{ProptYqspT}
	Soit \(  (\Omega,\tribA,\mu) \) un espace mesuré et \( 1\leq p\), \( q\leq\infty\) satisfaisant \( \frac{1}{ p }+\frac{1}{ q }=1\). Si \( f\in L^p(\Omega)\), \( g\in L^q(\Omega)\), alors nous avons les choses suivantes :

	\begin{enumerate}
		\item       \label{ITEMooNDKPooRKdmgS}
		      Le produit \( fg\) est dans \( L^1(\Omega)\) et nous avons
		      \begin{equation}    \label{EqLPKooPBCQYN}
			      \| fg \|_1\leq \| f \|_p\| g \|_q.
		      \end{equation}
		\item           \label{ITEMooQHLPooRWWMOP}
		      Si \( \frac{1}{ p }+\frac{1}{ q }=\frac{1}{ r }\) alors
		      \begin{equation}    \label{EqAVZooFNyzmT}
			      \| fg \|_r\leq \| f \|_p\| g \|_q
		      \end{equation}
		\item       \label{ITEMooBOYJooRkiAqJ}
		      Nous avons \( \| fg \|_1=\| f \|_p\| g \|_q\) si et seulement si nous sommes dans un des trois cas suivants :
		      \begin{itemize}
			      \item \( f=0\) presque partout,
			      \item \( g=0\) presque partout,
			      \item Il existe \( \lambda> 0\) tel que \( | f |^p=\lambda| g |^q\) presque partout.
		      \end{itemize}
	\end{enumerate}
\end{proposition}
\index{inégalité!Hölder}
Ce qui est appelé « inégalité de Hölder » est généralement l'inéquation \eqref{EqLPKooPBCQYN}.

\begin{proof}
	En plusieurs points.
	\begin{subproof}
		\spitem[Pour \ref{ITEMooNDKPooRKdmgS}]
		Nous allons le voir comme cas particulier de \ref{ITEMooQHLPooRWWMOP}.
		\spitem[Pour \ref{ITEMooQHLPooRWWMOP}]
		D'abord nous supposons \( \| g \|_q=1\) et nous posons
		\begin{equation}
			A=\{ x\in\Omega\tq | g(x) |>0 \}.
		\end{equation}
		Hors de \( A\), les intégrales que nous allons écrire sont nulles. Nous avons
		\begin{equation}
			\| fg \|_r^p=\Big|  \int_A| f |^r| g |^{r-q}| g |^q  \Big|^{p/r},
		\end{equation}
		et le coup tordu est de considérer cette intégrale comme étant une intégrale par rapport à la mesure \( \nu=| g |^qd\mu\) qui a la propriété d'être une mesure de probabilité par hypothèse sur \( g\). Nous pouvons alors utiliser l'inégalité de Jensen\footnote{Proposition~\ref{PropXISooBxdaLk}.} parce que \( p/r>1\), ce qui fait de \( x\mapsto | x |^{p/r}\) une fonction convexe. Nous avons alors
		\begin{subequations}
			\begin{align}
				\| fg \|_r^p & \leq\int_A\big( | f |^r| g |^{r-q} \big)^{p/r}| g |^qd\mu \\
				             & =\int_A| f |^{p}| g |^{p(r-q)/r}| g |^qd\mu
			\end{align}
		\end{subequations}
		La puissance de \( | g |\) dans cette expression est : \( q+\frac{ p(r-q) }{ r }=0\) parce que \( p(q-r)=rq\). Nous avons alors montré que
		\begin{equation}
			\| fg \|_r^p\leq \int_A| f |^pd\mu\leq \| f \|_p^p.
		\end{equation}
		La dernière inégalité est le fait que le domaine \( A\) n'est pas tout le domaine \( \Omega\).

		Si maintenant \( \| g \|_q\neq 1\) alors nous calculons
		\begin{equation}
			\| fg \|_r=\| g \|_q\| f\frac{ g }{ \| g \|_q } \|_r\leq \| g \|_q\| f \|_p
		\end{equation}
		en appliquant la première partie à la fonction \( \frac{ g }{ \| g \|_q }\) qui est de norme \( 1\).

		\spitem[Pour \ref{ITEMooBOYJooRkiAqJ}]
		Si \( f\) ou \( g\) est nulle presque partout, il y a immédiatement égalité. Nous supposons donc que \( f\) et \( g\) ne sont pas nulles presque partout et donc que \( \| f \|_p\) et \( \| g \|_q\) sont non nuls.
		\begin{subproof}
			\spitem[Deux fonctions intermédiaires]
			Nous posons
			\begin{equation}
				\begin{aligned}[]
					\hat f=\frac{ | f | }{ \| f \|_p }, &  & \hat g=\frac{ | g | }{ \| g \|_q }.
				\end{aligned}
			\end{equation}
			\spitem[Égalité préliminaire]
			Nous avons
			\begin{equation}
				\frac{1}{ p }\int\hat f^p=\frac{1}{ p }\int\frac{ | f |^p }{ \| f \|_p^p }=\frac{1}{ p\| f \|_p^p }\int| f |^p=\frac{1}{ p },
			\end{equation}
			et de même avec \( g\) et \( q\) au lieu de \( f\) et \( p\). Nous avons donc
			\begin{equation}
				\frac{1}{ p }\int\hat f^p+\frac{1}{ q }\int\hat g^q=1.
			\end{equation}
			\spitem[Les équivalences]
			Les choses suivantes sont équivalentes.
			\begin{enumerate}
				\item
				      \( \int | fg |=\| f \|_p\| g \|_q\)
				\item
				      \( \int\hat f\hat g=1\)
				\item
				      \begin{equation}
					      \int\hat f\hat g=\frac{1}{ p }\int \hat f^p+\frac{1}{ q }\int\hat g^q
				      \end{equation}
				\item
				      \begin{equation}
					      \int\left( \hat g\hat f-\frac{1}{ p }\hat f^p-\frac{1}{ q }\hat g^q \right)=0.
				      \end{equation}
				\item
				      \begin{equation}
					      \hat g\hat f-\frac{1}{ p }\hat f^p-\frac{1}{ q }\hat g^q=0
				      \end{equation}
				      presque partout.

				      En effet l'inégalité de Young\footnote{Proposition \ref{PROPooCQUBooCvtMSi}.} dit que l'intégrante est positive partout. Pour que l'intégrale soit nulle, il faut que l'intégrante soit nulle; c'est le lemme \ref{Lemfobnwt}.
				\item \( \hat f(x)^p=\hat g(x)^q\) pour presque tout \( x\). C'est le cas d'égalité dans l'inégalité de Young.
				\item
				      \begin{equation}
					      \frac{ | f |^p }{ \| f \|_p^p }=\frac{ | g |^q }{ \| g \|_q^q }.
				      \end{equation}
				\item
				      \begin{equation}
					      | f |^p=\lambda | g |^q
				      \end{equation}
				      avec \( \lambda=\| f \|_p^p/\| g \|_q^q\).
			\end{enumerate}
			\spitem[Conclusion]
			En lisant les implications de haut en bas, nous avons la condition nécessaire au cas d'égalité. Pour traiter la condition suffisante, nous supposons qu'il existe \( \lambda>0\) tel que \( | f |^p=\lambda| g |^q\). Alors nous avons
			\begin{equation}
				\| f \|_p^p=\int| f |^p=\lambda\int| g |^q=\lambda\| g \|_q^q,
			\end{equation}
			ce qui donne immédiatement \( \lambda=\| f \|_p^p/\| g \|_q^q\). Nous pouvons donc remonter les équivalences.
		\end{subproof}
	\end{subproof}
\end{proof}


\begin{remark}      \label{RemNormuptNird}
	Dans le cas d'un espace de probabilité, la fonction constante \( g=1\) appartient à \( L^p(\Omega)\). En prenant \( p=q=2\) nous obtenons
	\begin{equation}
		\| f \|_1\leq\| f \|_2.
	\end{equation}
\end{remark}

\begin{proposition}		\label{PROPooDVCCooKqbAwo}
	Soit un espace mesuré \( (\Omega,\tribA,\mu)\) vérifiant \( \mu(\Omega)=1\). Soient des nombres \( 0<r\leq s\) ainsi qu'une application \(f \colon \Omega\to \eR  \).

	Si \( \int_{\Omega}| f |^2<\infty\), alors \( \int_{\Omega}| f |^r<\infty\).
\end{proposition}

\begin{proof}
	Si \( s=r\), la conclusions est immédiate. Nous supposons donc \( r\neq s\). Nous considérons \( q\) tel que \( \frac{1}{ s}+\frac{1}{ q}=\frac{1}{ r}\), c'est-à-dire \( q=rs/(s-r)\). Vu que \( s>r\), nous avons \( q>0\), pas de problèmer.

	Par hypothèse, nous avons \( f\in L^s(\Omega)\). Et comme \( \Omega\) est de mesure finie, la fonction constante \( 1_{\Omega}\) est dans \( L^q(\Omega)\). Nous povons donc écrire l'inégalité de Hölder \ref{EqAVZooFNyzmT} :
	\begin{equation}
		\| f1_{\Omega} \|_r\leq \| f \|_s\| 1_{\Omega} \|_q.
	\end{equation}
	Cela donne \( \| f \|_r\leq \| f \|_s\), et en déballant,
	\begin{equation}
		\left( \int_{\Omega}| f |^r \right)^{1/r}\leq \left( \int_{\Omega}| f |^s \right)^{1/s}.
	\end{equation}
	Vu que \( r>0\), l'application \( t\mapsto t^r\) est croissante, et nous pouvons écrire
	\begin{equation}
		\int_{\Omega}| f |^r\leq\left( \int_{\Omega}| f |^s \right)^{r/s}<\infty.
	\end{equation}
	Notez l'utilisation de la proposition \ref{PROPooDWZKooNwXsdV}.
\end{proof}

\begin{lemma}   \label{LemTLHwYzD}
	À propos de \( L^1_{loc}(I)\).
	\begin{enumerate}
		\item
		      Lorsque \( I\) est borné nous avons \( L^2(I)\subset L^1(I)\).
		\item
		      Si \( I\) n'est pas borné alors \( L^2(I)\subset L^1_{loc}(I)\).
	\end{enumerate}
\end{lemma}

\begin{proof}
	En effet si \( I\) est borné, alors la fonction constante \( 1\) est dans \( L^2(I)\) et l'inégalité de Hölder~\ref{ProptYqspT} nous dit que le produit \( 1u\) est dans \( L^1(I)\).

	Si \( I\) n'est pas borné, nous refaisons le même raisonnement sur un compact \( K\) de \( I\).
\end{proof}

\begin{corollary}[\cite{ooAFCJooXLYoqf}]        \label{CORooIIEAooNmbkTo}
	Soit l'espace \( L^2(I)\) avec \( I=\mathopen] 0 , 1 \mathclose[\) avec la mesure de Lebesgue. Si \( u_n\in L^2\) converge vers \( u\) dans \( L^2\) alors nous pouvons permuter l'intégrale et la limite :
	\begin{equation}
		\lim_{n\to \infty} \int_Iu_n=\int_Iu.
	\end{equation}
\end{corollary}

\begin{proof}
	Nous considérons la forme linéaire
	\begin{equation}
		\begin{aligned}
			T\colon L^2(I) & \to \eR           \\
			u              & \mapsto \int_Iu .
		\end{aligned}
	\end{equation}
	Elle est bien définie par l'inégalité de Hölder \( \| fg \|_1\leq \| f \|_2\| g \|_2\) appliqué à \( g(x)=1\) qui vérifie \( \| g \|_2=1\). Nous avons aussi
	\begin{equation}
		T(u)\leq \int_I| u |\leq \| u \|_1\leq\| u \|_2
	\end{equation}
	où la dernière inégalité est celle de Hölder~\ref{ProptYqspT}. Bref, \( T\) est continue. Cela signifie que si \( u_n\stackrel{L^2(I)}{\longrightarrow}u\) alors \( T(u_n)=T(u)\). Cela est l'égalité demandée.
\end{proof}

%-------------------------------------------------------
\subsection{Inégalité de Minkowski}
%----------------------------------------------------

%-----------------------------------
\subsubsection{Forme usuelle}


\begin{probleme}
	Dans la proposition suivante, la partie «égalité» est très personnelle. Je n'en n'ai pas trouvé de preuve complète. Donc soyez doublement \randomGender{vigilant}{vigilante} et écrivez-moi si vous avez quelque chose à dire.
\end{probleme}

\begin{proposition}[Inégalité de Minkowski\cite{TUEWwUN,ooKFDRooNYNKqI,BIBooQYCIooITcxkY,BIBooTMMQooTqQouA,BIBooTHSZooBXgJmO,MonCerveau}]     \label{PropInegMinkKUpRHg}
	Si \( 1\leq p<\infty\) et si \( f,g\in L^p(\Omega,\tribA,\mu)\) alors
	\begin{enumerate}
		\item   \label{ItemDHukLJi}
		      \( \| f+g \|_p\leq \| f \|_p+\| g \|_p\)
		\item       \label{ITEMooGRXBooMLRMww}
		      Si \( p=1\), il y a égalité si et seulement si \( f(x)\overline{ g(x) }\geq 0\) pour presque tout \( x\).
		\item       \label{ITEMooQCSHooNUDwtM}
		      Si \( p>1\), il y a égalité si et seulement si il existe des réels positifs \( \alpha,\beta\) pas tous deux nuls tels que
		      \begin{equation}
			      \alpha f(x)=\beta g(x)
		      \end{equation}
		      pour presque tout \( x\).
	\end{enumerate}
\end{proposition}
\index{inégalité!Minkowski}

\begin{proof}
	En plusieurs points.
	\begin{subproof}
		\spitem[Pour \ref{ItemDHukLJi}]
		En utilisant l'inégalité \( | z_1+z_2 |\leq | z_1 |+| z_2 |\) (proposition \ref{PROPooEEFGooACcCll}\ref{ITEMooDVMDooFDmOur}) pour chaque \( x\) dans l'intégrale, de façon à pouvoir majorer
		\begin{subequations}
			\begin{align}
				| f(x)+g(x) |^p & =| f(x)+g(x) | |f(x)+g(x) |^{p-1}                     \\
				                & \leq\big( | f(x) |+| g(x) | \big)| f(x)+g(x) |^{p-1}.
			\end{align}
		\end{subequations}
		Nous mettons ça dans une intégrale et nous calculons un peu :
		\begin{subequations}        \label{SUBEQSooGWMTooDBXSgL}
			\begin{align}
				\| f+g \|^p_p & =\int| f+g |^pd\mu                                                                                  \\
				              & \leq \int\big( | f |+| g | \big)| f+g |^{p-1}d\mu                       \label{SUBEQooYCTWooQpHNqa} \\
				              & =\int| f | |f+g |^{p-1}+\int| g | |f+g |^{p-1} \label{SUBEQooNQKBooQFtTIJ}
			\end{align}
		\end{subequations}

		Lorsque \( p=1\), nous nous arrêtons ici parce que \eqref{SUBEQooNQKBooQFtTIJ} s'écrit
		\begin{equation}
			\| f+g \|_1\leq\int| f |+\int| g |=\| f \|_1+\| g \|_1.
		\end{equation}

		Lorsque \( p>1\), nous devons continuer et utiliser Hölder. Attardons nous sur le premier terme. Nous posons \( q=p/(p-1)\), \( a=f\) et \( b=| f+g |^{p-1}\), et nous utilisons l'inégalité de Hölder\footnote{Proposition \ref{ProptYqspT}.} :
		\begin{subequations}        \label{SUBEQSooFINUooQfIdMS}
			\begin{align}
				\int| f | |f+g |^{p-1} & =\| ab \|_1                                                      \\
				                       & \leq \| a \|_p\| b \|_q                                          \\
				                       & =\| f \|_p \left[ \int \big( | f+g |^{p-1} \big)^q \right]^{1/q} \\
				                       & =\| f \|_p\left( \int| f+g |^p \right)^{1-\frac{1}{ p }}         \\
				                       & =\| f \|_p\left( \int| f+g |^p \right)^{(p-1)/p}                 \\
				                       & = \| f \|_p\| f+g \|_p^{p-1}
			\end{align}
		\end{subequations}
		Nous avons utilisé la règle de produit d'exposants et de somme d'exposants\footnote{Propositions \ref{PROPooDWZKooNwXsdV} et \ref{PROPooVADRooLCLOzP}\ref{ITEMooSCJBooNVJZah}.}

		Nous utilisons cette inégalité dans les deux termes de \eqref{SUBEQSooGWMTooDBXSgL} :
		\begin{subequations}		\label{SUBEQSooKHKOooOCfndf}
			\begin{align}
				\| f+g \|_p^p & \leq \int | f || f+g |^{p-1}+\int | g || f+g |^{p-1}     \\
				              & \leq \| f \|_p\| f+g \|_p^{p-1}+\| g \|_p\| f+g \|^{p-1}
			\end{align}
		\end{subequations}
		Nous obtenons le résultat en divisant le tout par \( \| f+g \|_p^{p-1}\).
		\spitem[Pour \ref{ITEMooGRXBooMLRMww}]
		Toutes les inégalités de \eqref{SUBEQSooGWMTooDBXSgL} sont des égalités. En particulier nous avons celle-ci :
		\begin{equation}
			\int | f+g |d\mu=\int(| f |+| g |)d\mu
		\end{equation}
		Cela donne
		\begin{equation}
			\int\big( | f+g |-| f |-|g| \big)d\mu=0.
		\end{equation}
		Vue l'inégalité de la proposition \ref{PROPooEEFGooACcCll}\ref{ITEMooDVMDooFDmOur}, la fonction intégrée est toujours négative. Pour que l'intégrale soit nulle, il faut que la fonction intégrée soit presque partout nulle :
		\begin{equation}
			| f(x)+g(x) |=| f(x) |+| g(x) |
		\end{equation}
		pour presque tout \( x\). La partie «égalité» de la proposition \ref{PROPooEEFGooACcCll}\ref{ITEMooDVMDooFDmOur} donne alors le résultat.
	\end{subproof}

	\begin{center}
		\noindent\rule{2cm}{0.4pt}
		Cas d'égalité, préliminaire
		\noindent\rule{2cm}{0.4pt}
	\end{center}

	Nous commençons par prouver que si \( f+g=0\) alors \( f=g=0\). Pour cela nous faisons le calcul
	\begin{subequations}		\label{SUBEQSooGHMTooWSTtOn}
		\begin{align}
			\| f \|_p+\| g \|_p & =\| f+g \|		\label{SUEQooPUOGooAIaFFa}                                        \\
			                    & \leq \big\| | f |+| g |  \big\|		\label{SUBEQooFXHBooYXMbDf}                  \\
			                    & \leq \big\| | f |  \big\|_p+\big\|  | g | \big\|	\label{SUBEQooQZFXooPVgQrI} \\
			                    & = \| f \|_p+\| g \|_p.
		\end{align}
	\end{subequations}
	Justifications:
	\begin{itemize}
		\item
		      Pour \eqref{SUEQooPUOGooAIaFFa}. L'hypothèse d'égalité dans l'inégalité de Minkowski.
		\item
		      Pour \eqref{SUBEQooFXHBooYXMbDf}. Intégrale de l'inégalité de la proposition \ref{PROPooEEFGooACcCll}\ref{ITEMooDVMDooFDmOur}.
		\item
		      Pour \eqref{SUBEQooQZFXooPVgQrI}. Inégalité de Minkowski.
	\end{itemize}
	Étant donné que le premier et le dernier membre de \eqref{SUBEQSooGHMTooWSTtOn} sont égaux, toutes les inégalités sont des égalités. En particulier,
	\begin{equation}
		\| f+g \|_p=\big\| | f |+| g |  \big\|.
	\end{equation}
	En termes d'intégrales, cette égalité donne
	\begin{equation}		\label{EQooRYQRooGdgtLN}
		\int| f+g |^p-\big( | f |+| g | \big)^p=0.
	\end{equation}
	L'inégalité déjà mentionnée \( | z_1+z_2 |^p\leq \big( | z_1 |+| z_2 | \big)^p\) dit que la fonction à intégrer dans \eqref{EQooRYQRooGdgtLN} est partout négative. Pour que l'intégrale soit nulle, il faut donc que la fonction soit presque partout nulle. Autrement dit, pour presque tout \( x\) nous avons
	\begin{equation}		\label{EQooULHAooWXViEH}
		| f(x)+g(x) |=| f(x) |+| g(x) |.
	\end{equation}
	Cette égalité nous sera utile dans la suite. En particulier, retenez que si \( f+g=0\) presque partout, alors \( f=g=0\) presque partout. Dans ce cas, l'égalité \( \alpha f+\beta g=0\) fonctionne pour tout \( \alpha\) et \( \beta\).

	\begin{center}
		\noindent\rule{2cm}{0.4pt}
		Cas d'égalité \ref{ITEMooGRXBooMLRMww}, $p=1$.
		\noindent\rule{2cm}{0.4pt}
	\end{center}

	\begin{subproof}
		\spitem[\( \Rightarrow\)]
		%-----------------------------------------------------------

		L'inégalité \eqref{SUBEQooYCTWooQpHNqa} est une égalité :
		\begin{equation}
			\int | f+g |^{p-1}\big( | f+g |-| f |-| g | \big)=0.
		\end{equation}
		Par la proposition \ref{PROPooEEFGooACcCll}\ref{ITEMooDVMDooFDmOur}, la fonction intégrée est presque partout nulle ou positive. Pour que l'intégrale soit nulle, il faut que la fonction soit presque partout nulle (lemme \ref{Lemfobnwt}). Presque chaque \( x\) vérifie sont une des deux conditions suivantes :
		\begin{enumerate}
			\item
			      \( | f(x)+g(x) |=0\)
			\item
			      \( | f(x)+g(x) |-| f(x) |-| g(x) |=0\).
		\end{enumerate}
		Pour les \( x\) vérifiant la première condition, nous avons déjà vu que \( f=g=0\), juste en dessous de \eqref{EQooULHAooWXViEH}. En particulier, nous avons \( f(x)\overline{g(x)}\geq 0\) pour tout \( x\) dans ce cas.

		En ce qui concerne les \( x\) vérifiant la seconde condition, la proposition \ref{PROPooEEFGooACcCll}\ref{ITEMooDVMDooFDmOur} montre qu'ils vérifient \( f(x)\overline{g(x)}\geq 0\).

		Pour les \( x\) dans le second cas, la proposition \ref{PROPooEEFGooACcCll}\ref{ITEMooDVMDooFDmOur} donne $f(x)\overline{g(x)}\geq 0$.

		Bref, dans tous les cas nous avons \( f(x)\overline{g(x)}\geq 0\) pour presque tout \( x\).
		\spitem[\( \Leftarrow\)]
		%-----------------------------------------------------------

		Nous supposons que \( f(x)\overline{g(x)}\geq 0\) pour presque tout \( x\). La proposition \ref{PROPooEEFGooACcCll}\ref{ITEMooDVMDooFDmOur} (utilisée dans l'autre sens) dit que \( | f(x)+g(x) |-| f(x) |-| g(x) |=0\) pour presque tout \( x\). L'inégalité \eqref{SUBEQooYCTWooQpHNqa} est alors une égalité.

	\end{subproof}


	\begin{center}
		\noindent\rule{2cm}{0.4pt}
		Cas d'égalité \ref{ITEMooQCSHooNUDwtM}, $p>1$.
		\noindent\rule{2cm}{0.4pt}
	\end{center}

	Il y a deux sens.
	\begin{subproof}


		\spitem[\(  \Rightarrow\), les inégalités]
		%-----------------------------------------------------------------------------------------
		Nous supposons que \( f,g\in L^p(\Omega,\tribA,\mu)\) vérifient $\| f+g \|_p=\| f \|_p+\| g \|_p$, et nous allons voir ce que cela implique. Regroupons les différentes inégalités qui mènent à Minkowski.
		\begin{subequations}
			\begin{align}
				\| f+g \|^p_p & =\int| f+g |^pd\mu                                                                                                            \\
				              & \leq \int\big( | f |+| g | \big)| f+g |^{p-1}d\mu        & \text{eq. \eqref{SUBEQSooGWMTooDBXSgL}}\label{SUBEQooXPMYooJBKwZH} \\
				              & \leq \| f \|_p\| f+g \|_p^{p-1}+\int | g || f+g |^{p-1}  & \text{eq. \eqref{SUBEQSooKHKOooOCfndf}}\label{SUBEQooUPPAooIleVbN} \\
				              & \leq \| f \|_p\| f+g \|_p^{p-1}+\| g \|_p\| f+g \|^{p-1} & \text{eq. \eqref{SUBEQSooKHKOooOCfndf}}                            \\
				              & = \big( \| f \|_p+\| g \|_p \big)\| f+g \|_p^{p-1}                                                                            \\
				              & = \| f+g \|_p^p                                          & \text{hyp. égalité.}
			\end{align}
		\end{subequations}
		La dernière ligne est l'hypothèse d'égalité \( \| f \|_p+\| g \|_p=\| f+g \|_p\). Comme le premier et le dernier membre de ces inégalités sont égaux, toutes les inégalités sont des égalités.


		\spitem[\(  \Rightarrow\), première inégalité]
		%-----------------------------------------------------------------------------------------

		L'inégalité \eqref{SUBEQooXPMYooJBKwZH} est une égalité :
		\begin{equation}
			\int | f+g |^{p-1}\big( | f+g |-| f |-| g | \big)=0.
		\end{equation}
		Par la proposition \ref{PROPooEEFGooACcCll}\ref{ITEMooDVMDooFDmOur}, la fonction intégrée est presque partout nulle ou positive. Pour que l'intégrale soit nulle, il faut que la fonction soit presque partout nulle (lemme \ref{Lemfobnwt}). Presque chaque \( x\) vérifie sont une des deux conditions suivantes :
		\begin{enumerate}
			\item
			      \( | f(x)+g(x) |=0\)
			\item
			      \( | f(x)+g(x) |-| f(x) |-| g(x) |=0\).
		\end{enumerate}
		Pour les \( x\) vérifiant la première condition, nous avons déjà vu que \( \alpha f+\beta g=0\) pour n'importe quels \( \alpha\) et \( \beta\), juste en dessous de \eqref{EQooULHAooWXViEH}.

		En ce qui concerne les \( x\) vérifiant la seconde condition, la proposition \ref{PROPooEEFGooACcCll}\ref{ITEMooDVMDooFDmOur} montre qu'ils vérifient \( f(x)\overline{g(x)}\geq 0\).

		Nous avons donc déjà montré que si \( f\) et \( g\) vérifient une égalité, alors pour presque tout \( x\), nous avons
		\begin{equation}
			f(x)\overline{g(x)}\geq 0.
		\end{equation}

		\spitem[\(  \Rightarrow\), seconde inégalité (Hölder)]
		%-----------------------------------------------------------------------------------------

		La seconde inégalité à être une égalité est \eqref{SUBEQooUPPAooIleVbN}:
		\begin{equation}
			\int | f || f+g |^{p-1}=\| f \|_p\| f+g \|_p^{p-1}.
		\end{equation}
		Il s'agit du cas d'égalité de Hölder avec \( a=f\), \( b=| f+g |^{p-1}\) traité dans la proposition \ref{ProptYqspT}\ref{ITEMooBOYJooRkiAqJ}. Nous sommes donc dans un des trois cas suivants :
		\begin{itemize}
			\item \( f=0\) presque partout,
			\item \( f+g=0\) presque partout,
			\item il existe \( \lambda>0\) tel que \( | a |^p=\lambda | b |^q\).
		\end{itemize}
		Dans le premier cas, nous avons \( \alpha f=\beta g\) avec \( \alpha=1\) et \( \beta=0\). Le second cas donne \( f=g=0\) et donc \( \alpha f+\beta g=0\) pour n'importe quels \( \alpha\) et \( \beta\) (voir autour de \eqref{EQooULHAooWXViEH}).

		Nous supposons donc être dans le troisième cas. Étant donné que \( q=p/(p-1)\), nous avons
		\begin{equation}
			| b |^q=\big( | f+g |^{p-1} \big)^q=| f+g |^p,
		\end{equation}
		et donc la contrainte est l'existence de \( \lambda>0\) tel que \( | f |^p=\lambda | f+g |^p\).

		\spitem[\(  \Rightarrow\), troisième inégalité (Hölder)]
		%-----------------------------------------------------------------------------------------

		La seconde application de Hölder est avec \( a=g\) et \( b=| f+g |^{p-1}\). Nous sommes dans un cas d'égalité si nous sommes dans un des trois cas suivants :
		\begin{itemize}
			\item \( g=0\)
			\item \( f+g=0\)
			\item il existe \( \sigma>0\) tel que \( | a |^p=\sigma | b |^q\).
		\end{itemize}
		Encore une fois nous supposons être dans le troisième cas. Il existe donc \( \sigma>0\) tel que \( | g |^p=\sigma | f+g |^p\).

		\spitem[Petit résumé]
		%-----------------------------------------------------------

		Nous avons prouvé l'existence de \( \lambda,\sigma>0\) tels que
		\begin{subequations}
			\begin{numcases}{}
				f(x)\overline{ g(x) }\geq 0 \quad\text{pour presque tout } x\\
				| f |^p=\lambda| f+g |^p\\
				| g |^p=\sigma| f+g |^p
			\end{numcases}
		\end{subequations}

		\spitem[Des calculs et des cas]
		%-----------------------------------------------------------
		En passant à la racine \( p\)\ieme :
		\begin{subequations}
			\begin{numcases}{}
				f(x)\overline{ g(x) }\geq 0 \quad\text{pour presque tout } x\\
				| f |=s| f+g |\\
				| g |=t| f+g |.
			\end{numcases}
		\end{subequations}
		Les deux dernières égalités du système permettent d'écrire
		\begin{equation}
			t| f |=st| f+g |=s| g |.
		\end{equation}
		Nous considérons donc le système, valide pour presque tout \( x\) :
		\begin{subequations}        \label{SUBEQSooLKGDooQouvhW}
			\begin{numcases}{}
				f(x)\overline{ g(x) }\geq 0\\
				t| f(x) |=s| g(x) |
			\end{numcases}
		\end{subequations}
		Avec la contrainte \( (s,t)\neq (0,0)\).

		La proposition \ref{PROPooYLVPooCVnpFw} appliqué à chaque couple \( f(x)\), \( g(x)\) donne l'existence d'une fonction réelle \( a\) telle que pour chaque \( x\) nous ayons un des deux cas suivants :
		\begin{itemize}
			\item \( g(x)=0\)
			\item \( f(x)=a(x)g(x)\).
		\end{itemize}
		De plus \( a(x)\in \eR^+\) parce que \( f(x)\overline{ g(x) }\) est non seulement réel, mais également positif.

		\begin{subproof}
			\spitem[\(  s=0 \) ]
			Alors nous avons \( t| f(x) |=0\) et donc \( f(x)=0\) pour presque tout \( x\) parce que \( t\neq 0\). Nous avons alors \( \alpha f(x)+\beta g(x)=0\) avec \( \alpha=1\) et \( \beta=0\).
			\spitem[\(  s\neq 0 \) ]
			Les points de \( \Omega\) se séparent en deux parties : \( g(x)=0\) et les autres.
			\begin{subproof}
				\spitem[\(  g(x)\neq0 \) ]
				Dans ce cas nous avons le système
				\begin{subequations}
					\begin{numcases}{}
						f(x)=a(x)g(x)\\
						t| f(x) |=s| g(x) |.
					\end{numcases}
				\end{subequations}
				Vu que \( s\neq 0\) et \( g(x)\neq 0\), la dernière équation donne \( t| f(x) |\neq 0\) et donc \( t\neq 0\) et \( f(x)\neq 0\).

				La première équation donne \( a(x)=f(x)/g(x)\) que nous pouvons mette dans la secondes pour obternir :
				\begin{equation}
					| a(x) |=\frac{ s }{ t }.
				\end{equation}
				Vu que \( a\) est à valeurs dans \( \eR^+\), cela donne
				\begin{equation}
					a(x)=\frac{ s }{ t }
				\end{equation}
				pour tout \( x\) tel que \( g(x)\neq 0\).
				\spitem[\(  g(x)=0 \) ]
				Dans ce cas, le système \eqref{SUBEQSooLKGDooQouvhW} devient
				\begin{equation}
					t| f(x) |=0.
				\end{equation}
				Vu que \( s\neq 0\), nous avons \( f(x)=0\). Donc n'importe quel choix de \( \alpha\) et \( \beta\) fait l'affaire pour ces points. Il n'y a pas de contraintes.
			\end{subproof}

			\spitem[\( \Leftarrow\)]
			Nous supposons \( \alpha f(x)=\beta g(x)\) pour presque tout \( x\). Pour fixer les idées, nous supposons que \( \beta\neq 0\) (sinon, refaire le raisonnement en inversant les rôles de \( f\) et \( g\)). En posant \( \lambda=\alpha/\beta\) nous avons
			\begin{equation}
				g(x)=\lambda f(x)
			\end{equation}
			pour presque tout \( x\). En passant par l'intégrale,
			\begin{equation}
				\| f+g \|_p=\| (1+\lambda)f \|_p=(1+\lambda)\| f \|_p=\| f \|_p+\lambda\| f \|_p=\| f \|_p+\| \lambda f \|_p=\| f \|_p+\| g \|_p.
			\end{equation}
			Le \( \lambda\) et \( (1+\lambda)\) rentrent et sortent des normes parce qu'ils sont positifs.
		\end{subproof}
	\end{subproof}
\end{proof}


%-----------------------------------
\subsubsection{Inégalité intégrale de Minkowski}


Dans le cas où nous n'avons pas une somme de deux fonctions mais d'une infinité paramétrée par \( y\in Y\), nous pouvons convertir la somme de l'inégalité de Minkowski en une intégrale :
\begin{equation}
	\left[ \int_X\Big( \int_Y| f(x,y) |d\nu(y) \Big)^pd\mu(x) \right]^{1/p}\leq \int_Y\Big( \int_X| f(x,y) |^pd\mu(x) \Big)^{1/p}d\nu(y).
\end{equation}
Ce sera la proposition \ref{PROPooINXBooTrTxwg}.

\begin{lemma}[\cite{MonCerveau}]	\label{LEMooMZWYooIsJJms}
	Soit un espace mesuré \( X\). Soient \( p,q\) tels que \( \frac{1}{ p}+\frac{1}{ q}=1\). Soient des applications \(f,g \colon X\to \eC  \) telles que \( f\in L^p(X)\) et \( g^{p-1}\in L^q(X)\). Alors
	\begin{equation}
		\int_Xf(x)g^{p-1}(x)dx\leq\| f \|_p\| g \|_p^{p-1}.
	\end{equation}
\end{lemma}

\begin{proof}
	Nous utilisons Hölder \ref{ProptYqspT}\ref{ITEMooNDKPooRKdmgS} :
	\begin{equation}
		\int_Xf(x)g^{p-1}(x)dx=\| fg^{p-1} \|_1\leq \| f \|_p\| g^{p-1} \|_q.
	\end{equation}
	Le seul truc à savoir est que \( q(p-1)=p\) et \( p/q=p-1\); à part ça, nous calculons
	\begin{subequations}
		\begin{align}
			\| g^{p-1} \|_q & =\left(  \int_X| g^{p-1}(x) |^q  \right)^{1/q}                                                                  \\
			                & =\left(  \int_X| g(x)^{q(p-1)}  \right)^{1/q}                                                                   \\
			                & =\left(\int_Xg(x)^p\right)^{1/q}                                                                                \\
			                & =\Big( \| g \|_p^p \Big)^{1/q}                 =\| g \|_p^{p/q}                               =\| g \|_p^{p-1}.
		\end{align}
	\end{subequations}
\end{proof}


\begin{proposition}[\cite{BIBooRVWYooRoVzuN,MonCerveau}]	\label{PROPooYOWUooXoNLbL}
	Soit \( 1\leq p\leq \infty\) ainsi que \( q\in \eR\) tel que \( \frac{1}{ p}+\frac{1}{ q}=1\). Soit un espace mesuré \( \sigma\)-fini \( (\Omega,\tribA,\mu)\). Soit \(f \colon \Omega\to \eC  \).
	\begin{enumerate}
		\item		\label{ITEMooONTQooMHHOxk}
		      Si \( p<\infty\) et si \( \| f \|_p<\infty\), alors
		      \begin{equation}
			      \| f \|_p=\max_{\| \varphi \|_q\leq 1}\big|  \int_{\Omega}f\varphi\,d\mu \big|.
		      \end{equation}
		\item	\label{ITEMooBZXAooAfgWvF}
		      Si \( p\leq\infty\), alors
		      \begin{equation}
			      \| f \|_p=\sup_{\| \varphi \|_q\leq 1}\big|  \int_{\Omega}f\varphi\,d\mu \big|.
		      \end{equation}
	\end{enumerate}
\end{proposition}

\begin{proof}
	Notez l'inégalité de Hölder \ref{ProptYqspT}\ref{ITEMooNDKPooRKdmgS} :
	\begin{equation}
		| \int_{\Omega}f\varphi |\leq \int_{\Omega}| f\varphi |=\| f\varphi \|_1\leq \| f \|_1\| \varphi \|_q.
	\end{equation}
	Donc l'inégalité dans un sens est toujours vérifiée. Nous ne nous concentrons que sur l'inégalité dans l'autre sens.

	En plein de parties.
	\begin{subproof}
		\spitem[Pour \ref{ITEMooONTQooMHHOxk}]
		%-----------------------------------------------------------
		Notons tout de suite qu'on suppose \( \| f \|_p\neq 0\). Sinon \( f\) est presque partout nulle et n'importe que \( \varphi\) fait l'affaire. Nous subdivisons encore en deux parties.
		\begin{subproof}
			\spitem[Si \( \| f \|_p=1\)]
			%-----------------------------------------------------------
			Nous posons
			\begin{equation}
				\begin{aligned}
					\varphi\colon \Omega & \to \eC                                                                    \\
					\omega               & \mapsto \begin{cases}
						                               \frac{ | f(\omega) |^p }{ f(\omega) } & \text{si } f(\omega)\neq 0 \\
						                               0                                     & \text{sinon. }
					                               \end{cases}
				\end{aligned}
			\end{equation}
			Nous avons bien \( \| \varphi \|_q=1\) parce que, en remarquant que \( q(p-1)=1\),
			\begin{equation}
				\| \varphi \|_q^q=\int_{\Omega}| \varphi(\omega) |^qd\omega=\int_{\Omega_1}\big( | f(\omega) |^{p-1} \big)^q=\int_{\Omega}| f(\omega) |pd\omega=1.
			\end{equation}
			\spitem[Si \( \| f \|_p<\infty\)]
			%-----------------------------------------------------------
			Nous posons \( g=f/\| f \|_p\). Vu que \( \| g \|_p=1\), nous reprenons le \( \varphi\) du cas précédent qui va pour \( g\). Et nous vérifions que ce \( \varphi\) fonctionne aussi pour \( f\) :
			\begin{equation}
				\int_{\Omega}f\varphi=\| f \|_p\int_{\Omega}g\varphi=\| f \|_p.
			\end{equation}
		\end{subproof}
		\spitem[Pour \ref{ITEMooBZXAooAfgWvF}, avec \( p<\infty\) et \( \| f \|_p=\infty\)]
		%-----------------------------------------------------------
		Nous supposons que \( p<\infty\) et que \( \| f \|_p=\infty\). Il y a deux cas. Si \( f=\infty\) sur une partie de mesure non nulle, il suffit de prendre \( \varphi\) non nulle sur cette partie, et on aura facilement \( \int_{\Omega}f\varphi=\infty\).

		Nous supposons que \( | f |=\infty\) seulement sur une partie de mesure nulle, et nous considérons des parties \( \Omega_k\) comme dans la proposition \ref{PROPooRSDBooHEjQDq}. Ensuite nous posons
		\begin{equation}
			\begin{aligned}
				\alpha_k\colon \Omega & \to \eC                                                                          \\
				\omega                & \mapsto \begin{cases}
					                                0                                     & \text{si } \omega\not\in\Omega_k \\
					                                0                                     & \text{si } f(\omega)=0           \\
					                                \frac{ | f(\omega) |^p }{ f(\omega) } & \text{sinon }
				                                \end{cases}
			\end{aligned}
		\end{equation}
		et \( \varphi_k=\alpha_k/\| \alpha_k \|_q\). Nous avons manifestement \( \| \varphi_k \|_q=1\).

		Commençons par calculer ce que vaut réellement \( \| \alpha_k \|_q\). En notant \( \Omega'_k=\{ \omega\in \Omega_k\tq f(\omega)\neq 0 \}\) nous avons un petit calcul :
		\begin{subequations}
			\begin{align}
				\| \alpha_k \|_q & =\left(  \int_{\Omega}| \alpha_k(\omega) |^q \right)^{1/q}                                     \\
				                 & =\left(  \int_{\Omega'_k}\frac{ | f(\omega) |^{pq} }{ | f(\omega) |^q }d\omega   \right)^{1/q} \\
				                 & =\left( \int_{\Omega'_k}| f(\omega) |^p \right)^{1/q}                                          \\
				                 & = \| f \|_{k,p}^{p/q}
			\end{align}
		\end{subequations}
		où nous avons noté \( \| f \|_{k,p}\) la norme \( L^p\) de \( f\) sur \( \Omega_k\).

		Nous pouvons maintenant prouver notre énoncé :
		\begin{subequations}
			\begin{align}
				\int_{\Omega}f\varphi_k & =\frac{1}{ \| \alpha_k \|_q}\int_{\Omega'_k}f(\omega)\frac{ | f(\omega) |^k }{ f(\omega) }d\omega \\
				                        & = \frac{1}{ \| \alpha_k \|_q}\int_{\Omega_k}| f(\omega) |^pd\omega                                \\
				                        & =\frac{1}{ \| \alpha_k \|_q}\| f \|^p_{k,p}                                                       \\
				                        & =\| f \|_{k,p}^{p-p/q}                                                                            \\
				                        & =\| f \|_{k,p}.
			\end{align}
		\end{subequations}
		La proposition \ref{PROPooQNNVooTKRdyC} conclu que
		\begin{equation}
			\int_{\Omega}f\varphi_k=\| f \|_{k,p}\to \| f \|_p=\infty.
		\end{equation}

		\spitem[Pour \ref{ITEMooBZXAooAfgWvF}, avec \( p=\infty\)]
		%-----------------------------------------------------------
		Nous passons au cas \( p=\infty\)\footnote{Et c'est le moment de ne pas confondre \( \| f \|_{\infty}\) qui signifie en réalité \( \| f \|_{L^{\infty}}\) avec la norme suprémum qui n'a rien à voir avec que nous faisons ici.}. Soit \( \epsilon\in\mathopen] 0,1\mathclose[\), et posons
		\begin{equation}
			A_{\epsilon}=\{ \omega\in\Omega\tq | f(\omega)>1-\epsilon | \}.
		\end{equation}
		Notez que \( A_{\epsilon}\in\tribA\).

		\begin{subproof}
			\spitem[Si \( \| f \|_{\infty}=1\)]
			%-----------------------------------------------------------
			Nous avons \( \mu(A_{\epsilon})>0\), sinon nous aurions \( | f(x) |<1-\epsilon\) presque partout et donc \( \| f \|_{\infty}<1\). Vu que \( (\Omega,\mu)  \) est \( \sigma\)-fini, il existe une partie mesurable \( B_{\epsilon}\subset A_{\epsilon}\) vérifiant \( 0<\mu(B_{\epsilon})<\infty\). Nous posons alors
			\begin{equation}
				\begin{aligned}
					\varphi_{\epsilon}\colon \Omega & \to \eC                                                                                 \\
					x                               & \mapsto \begin{cases}
						                                          \frac{ | f(x) | }{ \mu(B_{\epsilon})f(x) } & \text{si } x\in B_{\epsilon} \\
						                                          0                                          & \text{sinon. }
					                                          \end{cases}
				\end{aligned}
			\end{equation}
			Notons que \( B_{\epsilon}\subset A_{\epsilon}\) et donc \( f(x)\neq 0\) sur \( B_{\epsilon}\). La fonction est donc bien définie( pas de problèmes avec le dénominateur).

			Sur \( B_{\epsilon}\), nous avons \( | \varphi_{\epsilon}(x)=1/\mu(B_{\epsilon}) |<\infty\). Donc \( \varphi_{\epsilon}\in L^1(\Omega)\) parce que \( B_{\epsilon}\) est de mesure finie. De plus
			\begin{equation}
				\int_{\Omega}| \varphi_{\epsilon}(x) |dx=\int_{B_{\epsilon}}\frac{1}{ \mu(B_{\epsilon})}dx=1.
			\end{equation}
			Nous avons prouvé que pour tout \( \epsilon\in \mathopen] 0,1\mathclose[\), nous avons
			\begin{equation}
				\sup_{\| \varphi \|=1}\int_{\Omega}f\varphi\geq 1-\epsilon,
			\end{equation}
			et donc que
			\begin{equation}
				\sup_{\| \varphi \|=1}\int_{\Omega}f\varphi=1=\| f \|_{\infty}.
			\end{equation}
		\end{subproof}

		\spitem[Si \( \| f \|_{\infty}\neq 1\)]
		%-----------------------------------------------------------
		Nous posons \( g=f/\| f \|_{\infty}\), de telle sorte que \( \| g \|_{\infty}=1\). Nous prenons les \( \varphi_{\epsilon}\) correspondant à \( g\) et nous avons
		\begin{equation}
			\int_{\Omega}f\varphi_{\epsilon}=\| f \|_{\infty}\int_{\Omega}g\varphi_{\epsilon}\geq \| f \|_{\infty}(1-\epsilon),
		\end{equation}
		et donc ça marche encore.
	\end{subproof}
\end{proof}

\begin{normaltext}
	Pour la suite, nous considérons deux espaces mesurés \( \sigma\)-finis \( X\) et \( Y\) ainsi qu'une application \(h \colon X\times Y\to \eC  \). Nous notons alors
	\begin{equation}
		\begin{aligned}
			h_{1,y}\colon X & \to \eC         \\
			x               & \mapsto h(x,y),
		\end{aligned}
	\end{equation}
	et
	\begin{equation}
		\begin{aligned}
			h_{2,x}\colon Y & \to \eC         \\
			y               & \mapsto h(x,y),
		\end{aligned}
	\end{equation}
	et
	\begin{equation}
		\begin{aligned}
			N_1\colon Y & \to \eR                  \\
			y           & \mapsto \| h_{1,y} \|_p,
		\end{aligned}
	\end{equation}
	et
	\begin{equation}
		\begin{aligned}
			N_2\colon X & \to \eR                 \\
			x           & \mapsto \| h_{2,x} \|_1
		\end{aligned}
	\end{equation}
	pour tous les \( p\in \eR\) pour lesquels \( N_2\) a un sens.
\end{normaltext}

\begin{proposition}[\cite{BIBooLOITooKzeYpM}]	\label{PROPooUMSFooJWFJzt}
	Soient deux espaces mesurés \( \sigma\)-finis \( X\) et \( Y\). Soit \( 1\leq p< \infty\). Soit une application mesurable \(h \colon X\times Y\to \eC  \). Alors l'application
	\begin{equation}
		\begin{aligned}
			N_1\colon Y & \to \eR                 \\
			y           & \mapsto \| h_{1,y} \|_p
		\end{aligned}
	\end{equation}
	est mesurable.
\end{proposition}

\begin{proof}
	Explicitons :
	\begin{equation}
		N(y)=\int_X| h(x,y) |^pdy.
	\end{equation}
	Le théorème de Fubini-Tonelli \ref{ThoWTMSthY}\ref{ITEMooUTMNooVIBdpP} appliqué à la fonction
	\begin{equation}
		\begin{aligned}
			f\colon X\times Y & \to \eR              \\
			(x,y)             & \mapsto | h(x,y) |^p
		\end{aligned}
	\end{equation}
	dit que \( N\) est mesurable.
\end{proof}

\begin{proposition}[Inégalité intégrale de Minkowski\cite{MonCerveau,BIBooZTBVooEzpwGG,BIBooNIKJooFJpQdt,BIBChatGPT}]	\label{PROPooINXBooTrTxwg}
	Soit \( 1\leq p\leq \infty\). Soient deux espaces mesurables \( \sigma\)-finis \( X\) et \( Y\). Soit \(h \colon X\times Y\to \eC  \). Nous supposons :
	\begin{enumerate}
		\item
		      \( h\) est mesurable.
		\item
		      Pour presque tout \( y\in Y\), nous avons \( h_{1,y}\in L^p(X)\).
		\item
		      \( N_1\in L^1(Y)\).
	\end{enumerate}
	Alors nous avons :
	\begin{enumerate}
		\item		\label{ITEMooCCQKooAyHfke}
		      \( h_{2,x}\in L^1(Y)\) pour presque tout \( x\in X\).
		\item		\label{ITEMooVWLZooUYAYuJ}
		      \( N_2\in L^p(X)\).
		\item	\label{ITEMooTBUEooQPnLEA}
		      La formule :
		      \begin{equation}		\label{EQooFMJCooOttKYt}
			      \left(   \int_X\Big|    \int_Y h(x,y) dy \Big|^pdx  \right)^{1/p}\leq  \int_Y\left( \int_X| h(x,y) |^pdx \right)^{1/p}dy
		      \end{equation}
		      ou, en posant
		      \begin{equation}
			      \begin{aligned}
				      F\colon X & \to \eC                 \\
				      x         & \mapsto \int_Yh(x,y)dy,
			      \end{aligned}
		      \end{equation}
		      l'inégalité
		      \begin{equation}		\label{EQooRAIDooBhDMPe}
			      \| F \|_p \leq \int_Y\| h_{1,y} \|_pdy.
		      \end{equation}
	\end{enumerate}
	Note : l'équation \eqref{EQooRAIDooBhDMPe} est plus adaptée que \eqref{EQooFMJCooOttKYt} pour l'énoncé dans le cas \( p=\infty\).
\end{proposition}

\begin{proof}
	Nous subdivisons selon les valeurs de \( p\). Et d'ailleurs, les points \ref{ITEMooCCQKooAyHfke}, \ref{ITEMooVWLZooUYAYuJ} et \ref{ITEMooTBUEooQPnLEA} ne seront pas démontrés dans l'ordre.
	\begin{subproof}
		\spitem[Pour \( p=1\)]
		%-----------------------------------------------------------
		Point par point.
		\begin{subproof}
			\spitem[Pour \ref{ITEMooTBUEooQPnLEA}]
			%-----------------------------------------------------------
			En utilisant Fubini-Tonelli \ref{ThoWTMSthY}\ref{ITEMooFKQUooCoCOLV},
			\begin{equation}
				\int_X\Big|\int_Yh(x,y)dy\Big|dx\leq \int_X\Big( \int_Y| h(x,y) |dy \Big)dx=\int_Y\big( \int_X| h(x,y) |dx \big)dy.
			\end{equation}
			\spitem[Pour \ref{ITEMooCCQKooAyHfke}]
			%-----------------------------------------------------------
			En posant
			\begin{equation}
				\begin{aligned}
					F\colon X & \to \mathopen[ 0,\infty\mathclose] \\
					x         & \mapsto \| h_{2,x} \|_1,
				\end{aligned}
			\end{equation}
			nous avons
			\begin{equation}
				F(x)=\| h_{2,x} \|_1=\int_Y| h(x,y) |dy.
			\end{equation}
			En permutant les intégrales par Fubini-Tonelli,
			\begin{subequations}
				\begin{align}
					\int_XF(x) & =\int_X\Big( \int_Y| h(x,y) |dy \Big)dx \\
					           & =\int_Y\Big( \int_X| h(x,y) |dx \Big)dy \\
					           & =\int_YN_1(y)                           \\
					           & <\infty.
				\end{align}
			\end{subequations}
			La dernière ligne est l'hypothèse que \( N_1\in L^1(Y)\). Le fait que \( \int_XF(x)<\infty\) implique que \( F(x)<\infty\) pour presque tout \( x\).
			\spitem[Pour \ref{ITEMooVWLZooUYAYuJ}]
			%-----------------------------------------------------------
			Nous avons un calcul comme d'habitude en utilisant Fubini-Tonelli :
			\begin{subequations}
				\begin{align}
					\| N_2 \|_1 & =\int_X\Big( \int_Y| h(x,y) |dy \Big)dx \\
					            & =\int_Y\| h_{1,y} \|_1dy                \\
					            & =\int_YN_1(y)dy                         \\
					            & <\infty.
				\end{align}
			\end{subequations}
		\end{subproof}
		\spitem[Pour \( p=\infty\)]
		%-----------------------------------------------------------
		Nous définissons
		\begin{equation}
			\begin{aligned}
				F\colon X & \to \eC                  \\
				x         & \mapsto \int_Y h(x,y)dy.
			\end{aligned}
		\end{equation}
		Nous avons \( \big| F(x) \big|\leq \int_Y| h(x,y) |dy\), et \( | h_{1,y}(x) |\leq \| h_{1,y} \|_{\infty}\) pour presque tout \( x\in X\). Armé de cela, nous pouvons faire un petit calcul avec le supremum essentiel \eqref{EQooZHKBooPgdzCp} :
		\begin{subequations}
			\begin{align}
				\| F \|_{L^{\infty}} & = \esssup_{x\in X}\Big| \int_Yh(x,y)   \Big|            \\
				                     & \leq \esssup_{x\in X}\int_Y\| h_{1,y} \|_{L^{\infty}}dy \\
				                     & =\int_Y\| h_{1,y} \|_{L^{\infty}}.
			\end{align}
		\end{subequations}
		Cela est bien la relation \eqref{EQooRAIDooBhDMPe} avec \( p=\infty\).

		\spitem[Si \( 1<p<\infty\)]
		%-----------------------------------------------------------
		Plusieurs parties.
		\begin{subproof}
			\spitem[Un premier calcul]
			%-----------------------------------------------------------

			Nous posons
			\begin{equation}
				\begin{aligned}
					F\colon X & \to \eC                 \\
					x         & \mapsto \int_Yh(x,y)dy,
				\end{aligned}
			\end{equation}
			nous considérons un certain \( \varphi\in L^q(X)\), et nous calculons \( s = | \int_XF\varphi |\) :
			\begin{subequations}
				\begin{align}
					s & \leq \int_X| F || \varphi |                                 \\
					  & =\int_X\left[  \int_Y| h(x,y) || \varphi(x) |dy  \right]dx.
				\end{align}
			\end{subequations}
			Le théorème de Fubini-Tonelli \ref{ThoWTMSthY}\ref{ITEMooFKQUooCoCOLV} appliqué à la fonction \( f(x,y)=| h(x,y) || g(x) |\) permet de permuter les intégrales parce que \( f\) est mesurable et positive\footnote{Nous n'avons pas (encore) de garanties que \( | f(x,y) |<\infty\), mais ce n'est pas important pour le moment.} :
			\begin{equation}
				s \leq \int_Y\left( \int_X| h(x,y) || \varphi(x) |dx \right)dy.
			\end{equation}
			Nous pouvons utiliser Hölder \ref{ProptYqspT}\ref{ITEMooNDKPooRKdmgS} pour l'intégrale sur \( X\) parce que \( \varphi\in L^q\) et \( h_{1,y}\in L^p\) :
			\begin{equation}
				\int_X| h(x,y) || \varphi(x) |=\| h_{1,y}\varphi \|_1\leq \| h_{1,y} \|_p\| \varphi \|_q = N_1(y)\| \varphi \|_q.
			\end{equation}
			Parmi les hypothèses, nous avons \( N_1\in L^1(Y)\). Donc l'intégrale sur \( Y\) se passe en réalité assez bien et nous avons
			\begin{equation}
				s\leq \int_Y\| h_{1,y} \|_p\| \varphi \|_qdy=\| \varphi \|_q\int_YN_1(y)dy<\infty.
			\end{equation}

			Nous avons prouvé que pour tout \( \varphi\in L^q(Y)\),
			\begin{equation}
				\big|\int_XF\varphi\big| \leq \| \varphi \|_q\| N_1 \|_1<\infty.
			\end{equation}
			Nous prenons maintenant le supremum sur tous les \( \varphi\in L^q(Y)\) satisfaisant \( \| \varphi \|_q=1\). La proposition \ref{PROPooYOWUooXoNLbL}\ref{ITEMooBZXAooAfgWvF} nous dit que le membre de gauche devient \( \| F \|_p\). Quant au membre de droite, il devient simplement \( \| N_1 \|_1\) :
			\begin{equation}		\label{EQooKIFGooRHBGXl}
				\| F \|_p\leq\| N_1 \|_1<\infty.
			\end{equation}
			\spitem[Un second calcul]
			%-----------------------------------------------------------
			Nous faisons exactement le même calcul que le premier, mais en partant de
			\begin{equation}
				\begin{aligned}
					G\colon X & \to \mathopen[ 0,\infty\mathclose] \\
					x         & \mapsto \int_Y| h(x,y) |dy.
				\end{aligned}
			\end{equation}
			Nous considérons \( \varphi\in L^q(X)\), et nous calculons en suivant les mêmes étapes :
			\begin{subequations}
				\begin{align}
					\| G\varphi \|_1 & =\int_X| G(x)\varphi(x) |dx      \\
					                 & =\ldots                          \\
					                 & \leq \| \varphi \|_q\| N_1 \|_1.
				\end{align}
			\end{subequations}
			Cela montre que pour tout \( \varphi\in L^q(X)\) nous avons
			\begin{equation}
				\big| \int_XG\varphi \big|\leq \int_X| G\varphi |= \| G\varphi \|_1\leq \| \varphi \|_q\| N_1 \|_1<\infty,
			\end{equation}
			et en passant au supremum, nous avons
			\begin{equation}
				\| G \|_p\leq \| N_1 \|_1<\infty
			\end{equation}

			\spitem[Pour \ref{ITEMooTBUEooQPnLEA}]
			%-----------------------------------------------------------
			Nous avons, en utilisant \ref{EQooKIFGooRHBGXl} :
			\begin{equation}
				\left(  \int_X\big| \int_Yh(x,y)dy \big|^pdx   \right)^{1/p}=\left( \int_X| F(x) |^pdx \right)^{1/p}=\| F \|_p\leq \| N_1 \|_1
			\end{equation}
			\spitem[Pour \ref{ITEMooCCQKooAyHfke}]
			%-----------------------------------------------------------
			Nous avons vu que \( \| G \|_p<\infty\). Donc \( G<\infty\) presque partout. Cela prouve que \( h_{2,x}\in L^1(Y)\) pour presque tout \( x\).
			\spitem[Pour \ref{ITEMooVWLZooUYAYuJ}]
			%-----------------------------------------------------------
			Nous devons voir que \( N_2\in L^p(X)\). Calcul avec permutation d'intégrales de Fubini-Tonelli :
			\begin{subequations}
				\begin{align}
					\| N_2 \|_p^p & =\int_X\Big|  \| h_{2,x} \|_1 \Big|^pdx      \\
					              & =\int_X\left( \int_Y| f(x,y) |dy \right)^pdx \\
					              & =\int_X| G(x) |^p                            \\
					              & =\| G \|_p^p                                 \\
					              & <\infty.
				\end{align}
			\end{subequations}
		\end{subproof}
	\end{subproof}
\end{proof}


\begin{lemma}[\cite{TUEWwUN}]       \label{LEMooPSBWooGLggTe}
	Si \( (X,\mu)\) est un espace mesuré \( \sigma\)-fini\footnote{Définition \ref{DefBTsgznn}.} non nul\footnote{C'est-à-dire que \( \mu\) n'est pas la mesure identiquement nulle.}, il existe une fonction mesurable strictement positive en escalier \( k\colon X\to \mathopen] 0 , \infty \mathclose[\) telle que
	\begin{equation}
		\int_Xkd\mu=1.
	\end{equation}
\end{lemma}

\begin{proof}
	Vue que \( X\) est \( \sigma\)-fini, il existe des parties \( (A_n)\) de mesures finies telles que \( X=\bigcup_{i=1}^{\infty}A_n\). Afin d'avoir des parties disjointes, nous posons
	\begin{equation}
		B_n=A_n\setminus\big( \bigcup_{i=1}^{n-1}A_n \big).
	\end{equation}

	Nous posons
	\begin{equation}
		k=\sum_{n=0}^{\infty}\frac{ \alpha_n }{ \mu(B_n) }\mtu_{B_n}
	\end{equation}
	où \( (\alpha_n)\) est une suite de réels strictement positifs dont la somme vaut \( 1\). En vertu de la somme géométrique \eqref{EqPZOWooMdSRvY}, prendre \( \alpha_n=2^{-n}\) fonctionne.

	Notons que les \( B_n\) étant disjoints, pour chaque \( x\), la somme définissant \( k(x)\) est non seulement finie, mais se réduit à un seul terme. L'application \( k\) est en escalier; son intégrale vaut
	\begin{equation}
		\int_Xkd\mu=\sum_{n=0}^{\infty}\frac{ \alpha_n }{ \mu(B_n) }\mu(B_n)=\sum_{n=1}^{\infty}\alpha_n=1.
	\end{equation}
\end{proof}

À mon avis l'énoncé de \ref{PROPooGZJZooXfZdqn} n'est pas tellement correct. Il manque des hypothèse. Ne l'utilisez que si vous êtes \randomGender{sûr}{sûre}; et préférez \ref{PROPooINXBooTrTxwg}.
\begin{proposition}[Forme intégrale de Minkowski\cite{TUEWwUN}]     \label{PROPooGZJZooXfZdqn}		% Ne pas trop référentiel ceci. Préférer PROPooINXBooTrTxwg
	Soient deux espaces \( \sigma\)-finis \( (X,\mu)\) et \( (Y,\nu)\). Si \( f\colon X\times Y\to \eC\) est mesurable\footnote{Pour l'espace mesuré produit \(  X\times Y, \mu\otimes \nu\).} et si \( p\geq 1\), alors
	\begin{equation}        \label{EQooAEXWooYJtGGR}
		\left[ \int_X\Big( \int_Y| f(x,y) |d\nu(y) \Big)^pd\mu(x) \right]^{1/p}\leq \int_Y\Big( \int_X| f(x,y) |^pd\mu(x) \Big)^{1/p}d\nu(y).
	\end{equation}
	De façon plus compacte :
	\begin{equation}
		\left\|   x\mapsto\int_Y f(x,y)d\nu(y)   \right\|_p\leq \int_Y  \| f_y \|_pd\nu(y)
	\end{equation}
	où \( f_y(x)=f(x,y)\).
\end{proposition}

\begin{proof}
	En plusieurs parties.
	\begin{subproof}
		\spitem[La fonction \( M\)]
		Pour toute application mesurable \( g\colon X\times Y\to \eC\) nous posons
		\begin{equation}
			M(g)=\int_Y \left[ \int_X| g(x,y) |^pd\mu(x) \right]^{1/p}d\nu(y).
		\end{equation}
		Cette fonction a une propriété d'homogénéité : \( M(\lambda g)=\lambda M(g)\) pour tout \( \lambda\in \mathopen[ 0 , \infty \mathclose[\).
			\spitem[Si \(  M(g)<1 \)]
			Nous considérons une fonction mesurable \( g\colon X\times Y\to \eC\) telle que \( M(g)<1\). Soit \( k\colon Y\to \mathopen] 0 , \infty \mathclose[\) une fonction donnée par le lemme \ref{LEMooPSBWooGLggTe}, c'est-à-dire telle que \( \int_Yk=1\). Nous posons
		\begin{equation}
			\begin{aligned}
				s\colon Y & \to \mathopen[ 0 , \infty \mathclose]                   \\
				y         & \mapsto \left( \int_X| g(x,y) |^pd\mu(x) \right)^{1/p},
			\end{aligned}
		\end{equation}
		ainsi que
		\begin{equation}
			h(y)=s(y)+\big(1-M(g)\big)k(y).
		\end{equation}
		Vu que \( M(g)<1\) et que \( k>0\), nous avons l'inégalité stricte \( h(y)>s(y)\geq 0\) pour tout \( y\in Y\). Nous avons de plus
		\begin{equation}
			\int_Y h(y)d\nu(y)=\underbrace{\int_Ys}_{=M(g)}+\big( 1-M(g) \big)\underbrace{\int_Yk}_{=1}=1.
		\end{equation}
		Nous considérons maintenant la mesure \( \rho=h\nu\) (produite d'une mesure par une fonction, définition \ref{PropooVXPMooGSkyBo}). Nous venons de montrer que c'est une mesure de probablilité. En posant \( \alpha(x)=| x |^p\), nous pouvons faire quelques calculs avec les justifications en-dessous :
		\begin{subequations}        \label{SUBEQSooPPZIooSEDcpS}
			\begin{align}
				\alpha\big( \int_Y|g(x,y)|d\nu(y)\big) & =\alpha\big( \int_Yh(y)^{-1}|g(x,y)|h(y)d\nu(y)\big)                                     \\
				                                       & =\alpha\left( \int_Y [|g(x,y)|h(y)^{-1}]d\rho(y) \right)                                 \\
				                                       & \leq \int_Y\alpha\big( |g(x,y)|h(y)^{-1} \big)d\rho(y)       \label{SUBEQooDAWRooOOfQBE} \\
				                                       & =\int_Y| g(x,y) |^p| h(y) |^{1-p}d\nu(y)           \label{SUBEQooLDEHooJelfxU}           \\
			\end{align}
		\end{subequations}
		Justifications.
		\begin{itemize}
			\item Pour \eqref{SUBEQooDAWRooOOfQBE}. Inégalité de Jensen, proposition \ref{PropXISooBxdaLk}
			\item Pour \eqref{SUBEQooLDEHooJelfxU}. La fonction \( h\) est à valeurs positives; donc \( h=| h |\) et nous pouvons coller le \( h\) qui ressort de \( \rho\) avec le \( | h^{-1} |^p\).
		\end{itemize}
		Nous continuons en intégrant l'inégalité \eqref{SUBEQSooPPZIooSEDcpS} sur \( X\) :
		\begin{subequations}
			\begin{align}
				\int_X\left( \int_Y| g(x,y) |d\nu(y) \right)^pd\mu(x) & \leq\int_X\left[ \int_Y | g(x,y) |^ph(y)^{1-p}d\nu(y) \right]d\mu(x)                           \\
				                                                      & =\int_Y\int_X\big[ | g(x,y) |^ph(y)^{1-p}d\mu(x) \big]d\nu(y)      \label{SUBEQooSDNWooIZcFNB} \\
				                                                      & =\int_Y\left( \int_X\big[ | g(x,y) |^pd\mu(x) \big]\right)h(y)^{1-p}d\nu(y)                    \\
				                                                      & =\int_Y s(y)^ph(y)^{1-p}d\nu(y)                                                                \\
				                                                      & \leq \int_Yh(y)d\nu(y)     \label{SUBEQooEFTLooPwpovo}                                         \\
				                                                      & =1.
			\end{align}
		\end{subequations}
		Justifications.
		\begin{itemize}
			\item Pour \eqref{SUBEQooSDNWooIZcFNB}. Les intégrales sont permutées grâce au théorème de Fibini-Tonelli \ref{ThoWTMSthY}.
			\item Pour \eqref{SUBEQooEFTLooPwpovo}. Nous savons que \( h>s\geq 0\), donc \( s(y)^p/h(y)^p<1 \).
		\end{itemize}
		Au final nous avons prouvé que si \( M(g)<1\), alors nous avons
		\begin{equation}        \label{EQooTGDMooZGYbGx}
			\int_X\left( \int_Y| g(x,y) |d\nu(y) \right)^pd\mu(x)\leq 1.
		\end{equation}
		\spitem[La preuve proprement dite]
		Nous considérons une fonction mesurable \( f\colon X\times Y\to \eC\) comme dans les hypothèses. Astuce : considérons \( 0<t<1\) et appliquons l'inégalité \eqref{EQooTGDMooZGYbGx} à \( g=\frac{ t }{ M(f) }f\). Cela est possible parce que
		\begin{equation}
			M(g)=\frac{ t }{ M(f) }M(f)=t<1.
		\end{equation}
		Nous avons :
		\begin{equation}
			\int_X\left( \int_Y| \frac{ t }{ M(f) } | |f(x,y) |^pd\nu(y) \right)^pd\mu(x)\leq 1,
		\end{equation}
		et donc, vu que \( M(f)\) et \( t\) sont positifs,
		\begin{equation}
			\int_X\left( \int_Y| f(x,y) |^pd\nu(y) \right)^pd\mu(x)\leq \frac{ M(f) }{ t }.
		\end{equation}
		Cette inégalité est valable pour tout \( t<1\) et, ô incroyable!, ce \( t\) n'est qu'à droite. Nous pouvons prendre la limite \( t\to 1\) dans cette inégalité\footnote{Nous ne pouvons pas simplement poursuivre les inégalités en majorant \(  t\) par \(  1\) parce que le \(  t\) est au dénominateur : l'inégalité irait dans le mauvais sens.}  :
		\begin{equation}
			\int_X\left( \int_Y| f(x,y) |^pd\nu(y) \right)^pd\mu(x)\leq M(f).
		\end{equation}
		Et ça, c'est ce que nous devions démontrer.
	\end{subproof}

\end{proof}


%---------------------------------------------------------------------------------------------------------------------------
\subsection{Ni inclusions ni inégalités}
%---------------------------------------------------------------------------------------------------------------------------

Aucun espace \( L^p(\eR)\) n'est inclus dans aucun autre ni aucune norme n'est plus grande qu'une autre (sur les intersections). Nous verrons cependant en la proposition~\ref{PropIRDooFSWORl} que de telles inclusions et inégalités sont possibles pour \( L^p\big( \mathopen[ 0 , 1 \mathclose] \big)\).

Nous allons donner des exemples de tout ça en supposant \( p<q\) et en nous appuyant lourdement sur les intégrales de \( \frac{1}{ x^{\alpha} }\) étudiées par la proposition~\ref{PropBKNooPDIPUc}.

\begin{subproof}
	\spitem[\( L^p\nsubseteq L^q\)]

	La fonction
	\begin{equation}    \label{EqXIEooZpxObV}
		f(x)=\begin{cases}
			\frac{1}{ x^{1/q} } & \text{si } 0<x<1 \\
			0                   & \text{sinon}
		\end{cases}
	\end{equation}
	est dans \( L^p\) mais pas dans \( L^q\). En effet
	\begin{equation}
		\| f \|_p^p=\int_0^1\frac{1}{ x^{p/q} }dx<\infty
	\end{equation}
	parce que \( p<q\) et \( p/q<1\). Par contre
	\begin{equation}
		\| f \|_q^q=\int_0^1\frac{1}{ x }dx=\infty.
	\end{equation}

	\spitem[\( L^q\nsubseteq L^p\)]

	La fonction
	\begin{equation}
		f(x)=\begin{cases}
			\frac{1}{ x^{1/p} } & \text{si } x>1 \\
			0                   & \text{sinon}
		\end{cases}
	\end{equation}
	est dans \( L^q\) mais pas dans \( L^p\). En effet
	\begin{equation}
		\| f \|_p^p=\int_1^{\infty}\frac{1}{ x }=\infty
	\end{equation}
	alors que
	\begin{equation}
		\| f \|_q^q=\int_1^{\infty}\frac{1}{ x^{q/p} }dx<\infty.
	\end{equation}

	\spitem[Exemple de \( \| f \|_p>\| f \|_q\)]

	La fonction
	\begin{equation}
		f(x)=\begin{cases}
			1 & \text{si } x\in\mathopen[ 0 , 2 \mathclose] \\
			0 & \text{sinon.}
		\end{cases}
	\end{equation}
	Nous avons
	\begin{subequations}
		\begin{align}
			\| f \|_p & =2^{1/p}  \\
			\| f \|_q & =2^{1/q}.
		\end{align}
	\end{subequations}
	Mais comme \( p<q\) donc \( \| f \|_p>\| f \|_q\).


	\spitem[Exemple de \( \| f \|_p<\| f \|_q\)]

	La fonction
	\begin{equation}
		f(x)=\begin{cases}
			1 & \text{si } x\in\mathopen[ 0 , \frac{ 1 }{2} \mathclose] \\
			0 & \text{sinon.}
		\end{cases}
	\end{equation}
	Alors
	\begin{subequations}
		\begin{align}
			\| f \|_p & =\frac{1}{ 2^{1/p} } \\
			\| f \|_q & =\frac{1}{ 2^{1/q} }
		\end{align}
	\end{subequations}
	et donc \( \| f \|_p<\| f \|_q\).
\end{subproof}

Ces exemples donnent un exemple de fonction \( f\) telle que \( \| f \|_p<\| f \|_q\) pour tout espace \( L^p(I)\) et \( L^q(I)\) avec \( I\subset \eR\). Par contre l'exemple \( \| f \|_p>\| f \|_q\) ne fonctionne que si la taille de \( I\) est plus grande que \( 1\). Et pour cause : il y a des inclusions si \( I\) est borné.

\begin{proposition}[\cite{MathAgreg}] \label{PropIRDooFSWORl}
	Inclusions et inégalités dans le cas d'un ensemble de mesure finie.
	\begin{enumerate}
		\item
		      Soit \( (\Omega,\tribA,\mu)\) un espace mesuré fini et \( 1\leq p\leq +\infty\). Alors \( L^q(\Omega)\subset L^p(\Omega)\) dès que \( p\leq q\).
		\item   \label{ItemWSTooLcpOvXii}
		      Si \( 1<p<2\) et si \( f\in L^2\big( \mathopen[ 0 , 1 \mathclose] \big)\) alors \( \| f \|_p\leq \| f \|_2\).
	\end{enumerate}
\end{proposition}
\begin{proof}
	Pour la simplicité des notations nous allons noter \( L^p\) pour \( L^p(\Omega)\), et pareillement pour \( L^q\). Soit \( f\in L^q\). Nous posons
	\begin{equation}
		A=\{ x\in\Omega\tq | f(x) |\geq 1 \}.
	\end{equation}
	Étant donné que \( p\leq q\) nous avons \( | f |^p\leq | f |^q\) sur \( A\); par conséquent \( \int_A| f |^p\) converge parce que \( \int_A| f |^q\) converge.

	L'ensemble \( A^c\) est évidemment borné (complémentaire dans \(  \Omega \)) et sur \( A^c\) nous avons \( | f(x) |\leq 1\) et donc \( | f |^p\leq 1\). L'intégrale \( \int_{A^c}| f |^p\) converge donc également.

	Au final \( \int_{\Omega}| f |^p\) converge et \( f\in L^p\).


	Soit à présent \( f\in L^2\); par le premier point nous avons immédiatement \( f\in L^2\cap L^p\). Soit aussi \( r\in \eR\) tel que \( \frac{1}{ 2/p }+\frac{1}{ r }=1\). Nous avons \( | f |^p\in L^{2/p}\), et vu que nous sommes sur un domaine borné, \( 1\in L^r\). Nous écrivons l'inégalité de Hölder \eqref{EqLPKooPBCQYN} avec ces fonctions. D'une part
	\begin{equation}
		\| f \|_1=\| | f |^p \|_1=\| f \|_p^p.
	\end{equation}
	D'autre part
	\begin{equation}
		\| | f |^p \|_{2/p}=\left( \int| f |^2 \right)^{p/2}=\| f \|_2^p.
	\end{equation}
	Donc \( \| f \|_p^p\leq \| f \|_2^p\), ce qui prouve l'assertion~\ref{ItemWSTooLcpOvXii} parce que \( p>1\).
\end{proof}

\begin{remark}
	Nous n'avons cependant pas \( L^2\big( \mathopen[ 0 , 1 \mathclose] \big)=L^p\big( \mathopen[ 0 , 1 \mathclose] \big)\) parce que l'exemple \eqref{EqXIEooZpxObV} fonctionne encore :
	\begin{equation}
		f(x)=\frac{1}{ \sqrt{x} }
	\end{equation}
	pour \( x\in\mathopen[ 0 , 1 \mathclose]\) donne bien
	\begin{equation}
		\| f \|_2=\int_0^1\frac{1}{ x }=\infty
	\end{equation}
	et \( \| f \|_p=\int_0^1\frac{1}{ x^{p/2} }<\infty\) parce que \( 1<p<2\).
\end{remark}

%---------------------------------------------------------------------------------------------------------------------------
\subsection{Complétude}
%---------------------------------------------------------------------------------------------------------------------------

\begin{theorem}[\cite{SuquetFourierProba,UQSGIUo}]  \label{ThoUYBDWQX}
	Pour \( 1\leq p<\infty\), l'espace \( L^p(\Omega,\tribA,\mu)\) est complet.
\end{theorem}
\index{complétude}

\begin{proof}
	Soit \( (f_n)_{n\in\eN}\) une suite de Cauchy dans \( L^p\). Pour tout \( i\), il existe \( N_i\in\eN\) tel que \( \| f_k-f_l \|_p\leq 2^{-i}\) pour tout \( k,l\geq N_i\). Nous considérons la sous-suite \( g_i=f_{N_i}\), de telle sorte qu'en particulier
	\begin{equation}    \label{EqJLoDID}
		\|g_i-g_{i-1}\|_p\leq 2^{-i+1}.
	\end{equation}
	Pour chaque \( j\) nous considérons la somme télescopique
	\begin{equation}
		g_j=g_0+\sum_{i=1}^j(g_i-g_{i-1})
	\end{equation}
	et l'inégalité
	\begin{equation}
		| g_j |\leq | g_0 |+\sum_{i=1}^j| g_i-g_{i-1} |.
	\end{equation}
	Nous allons noter
	\begin{equation}        \label{EqSomPaFPQOWC}
		h_j=| g_0 |+\sum_{i=1}^j| g_i-g_{i-1} |.
	\end{equation}
	La suite de fonctions \( (h_j)\) ainsi définie est une suite croissante de fonctions positives qui converge donc (ponctuellement) vers une fonction \( h\) qui peut éventuellement valoir l'infini en certains points. Par continuité de la fonction \( x\mapsto x^p\) nous avons
	\begin{equation}
		\lim_{j\to \infty} h_j^p=h^p,
	\end{equation}
	puis par le théorème de la convergence monotone (théorème~\ref{ThoRRDooFUvEAN}) nous avons
	\begin{equation}
		\lim_{j\to \infty} \int_{\Omega}h_j^pd\mu=\int_{\Omega}h^pd\mu.
	\end{equation}
	Utilisant à présent la continuité de la fonction \( x\mapsto x^{1/p}\) nous trouvons
	\begin{equation}
		\lim_{j\to \infty} \left( \int h_j^p \right)^{1/p}=\left( \int | h |^p \right)^{1/p}.
	\end{equation}
	Nous avons donc déjà montré que
	\begin{equation}
		\lim_{j\to \infty} \| h_j \|_p=\left( \int | h |^p \right)^{1/p}
	\end{equation}
	où, encore une fois, rien ne garantit à ce stade que l'intégrale à droite soit un nombre fini. En utilisant l'inégalité de Minkowski (proposition~\ref{PropInegMinkKUpRHg}) et l'inégalité \eqref{EqJLoDID} nous trouvons
	\begin{equation}
		\|h_j\|_p\leq \|g_0\|_p+\sum_{i=1}^j\|g_i-g_{i-1}\|_p\leq \|g_0\|_p+1.
	\end{equation}
	En passant à la limite,
	\begin{equation}
		\left( \int| h |^p \right)^{1/p}=\lim_{j\to \infty}\|h_j\|_p \leq \|g_0\|_p+1<\infty.
	\end{equation}
	Par conséquent \( \int| h |^p\) est finie et
	\begin{equation}    \label{EqgLpdUPOBP}
		h\in L^p(\Omega,\tribA,\mu).
	\end{equation}
	En particulier, l'intégrale \( \int h\) est finie (parce que \( p\geq 1\)) et donc que \( h(x)<\infty\) pour presque tout \( x\in\Omega\).

	Nous savons que \( h(x)\) est la limite des sommes partielles \eqref{EqSomPaFPQOWC}, en particulier la série
	\begin{equation}
		\sum_{j=1}^{\infty}| g_i-g_{i-1} |
	\end{equation}
	converge ponctuellement. En vertu du corolaire~\ref{CorCvAbsNormwEZdRc}, la série de terme général \( g_i-g_{i-1}\) converge ponctuellement. La suite \( g_i\) converge donc vers une fonction que nous notons \( g\). Par ailleurs la suite \( g_i\) est dominée par \( h\in L^p\), le théorème de la convergence dominée (théorème~\ref{ThoConvDomLebVdhsTf}) implique que
	\begin{equation}
		\lim_{j\to \infty} \|g_j-g\|_p=0.
	\end{equation}
	Nous allons maintenant prouver que \( \lim_{n\to \infty}\|f_n-g\|_p =0\). Soit \( \epsilon>0\). Pour tout \( n\) et \( i\) nous avons
	\begin{equation}
		\|f_n-g\|_p=\|f_n-f_{N_i}+f_{N_i}-g\|_p\leq\|f_n-f_{N_i}\|_p+\|f_{N_i}-g\|_p.
	\end{equation}
	Pour rappel, \( f_{N_i}=g_i\). Si \(i\) et \( n\) sont suffisamment grands nous pouvons obtenir que chacun des deux termes est plus petit que \( \epsilon/2\).

	Il nous reste à prouver que \( g\in L^p(\Omega,\tribA,\mu)\). Nous avons déjà vu (équation \eqref{EqgLpdUPOBP}) que \( h\in L^p\), mais \( | g_i |\leq h^p\), par conséquent  \( g\in L^p\).

	Nous avons donc montré que la suite de Cauchy \( (f_n)\) converge vers une fonction de \( L^p\), ce qui signifie que \( L^p\) est complet.
\end{proof}

\begin{theorem}[Riesz-Fischer\cite{KXjFWKA,BIBooOKEOooYLVCHb}] \label{ThoGVmqOro}
	Soient un espace mesuré \( (\Omega,\tribA,\mu)\) et \( p\in\mathopen[ 1 , \infty \mathclose]\). Alors
	\begin{enumerate}
		\item\label{ItemPDnjOJzi}
		Toute suite convergente dans \( L^p(\Omega,\tribA,\mu)\) admet une sous-suite convergente presque partout sur \( \Omega\).
		\item\label{ItemPDnjOJzii}
		La sous-suite donnée en~\ref{ItemPDnjOJzi} est dominée par un élément de \( L^p(\Omega,\tribA,\mu)\).
		\item\label{ItemPDnjOJziii}
		L'espace \( L^p(\Omega,\tribA,\mu)\) est de Banach.
	\end{enumerate}
\end{theorem}
\index{espace!de fonctions!\( L^p\)}
\index{complétude!espaces \(  L^p\)}

\begin{proof}
	Le cas \( p=\infty\) est à séparer des autres valeurs de \( p\) parce qu'on y parle de norme uniforme, et aucune sous-suite n'est à considérer.
	\begin{subproof}
		\spitem[Cas \( p=\infty\).]
		Nous commençons par prouver dans le cas \( p=\infty\). Soit \( (f_n)\) une suite de Cauchy dans \( L^{\infty}(\Omega)\), ou plus précisément une suite de représentants d'éléments de \( L^p\). Pour tout \( k\geq 1\), il existe \( N_k\geq 0\) tel que si \( m,n\geq N_k\), on a
		\begin{equation}
			\| f_m-f_n \|_{\infty}\leq \frac{1}{ k }.
		\end{equation}
		En particulier, l'ensemble \( E_k\) sur lequel
		\begin{equation}
			| f_m(x)-f_n(x) |\geq\frac{1}{ k }
		\end{equation}
		est de mesure nulle. D'après le lemme \ref{LemIDITgAy}, la partie \( E=\bigcup_{k\in \eN}E_k\), est encore de mesure nulle. En  résumé, nous avons un \( N_k\) tel que si \( m,n\geq N_k\), alors
		\begin{equation}    \label{EqKAWSmtG}
			| f_n(x)-f_m(x) |\leq \frac{1}{ k }
		\end{equation}
		pour tout \( x\) hors de \( E\). Donc pour chaque \( x\in\Omega\setminus E\), la suite \( n\mapsto f_n(x)\) est de Cauchy dans \( \eR\) et converge donc. Cela définit donc une fonction
		\begin{equation}
			\begin{aligned}
				f\colon \Omega\setminus E & \to \eR                            \\
				x                         & \mapsto \lim_{n\to \infty} f_n(x).
			\end{aligned}
		\end{equation}
		Cela prouve le point~\ref{ItemPDnjOJzi} : la convergence ponctuelle.

		En passant à la limite \( n\to \infty\) dans l'équation~\eqref{EqKAWSmtG} et tenant compte que cette majoration tient pour presque tout \( x\) dans \( \Omega\), nous trouvons
		\begin{equation}
			\| f-f_n \|_{\infty}\leq \frac{1}{ k }.
		\end{equation}
		Donc non seulement \( f\) est dans \( L^{\infty}\), mais en plus la suite \( (f_n)\) converge vers \( f\) au sens \( L^{\infty}\), c'est-à-dire uniformément. Cela prouve le point~\ref{ItemPDnjOJziii}. En ce qui concerne le point~\ref{ItemPDnjOJzii}, la suite \( f_n\) est entièrement (à partir d'un certain point) dominée par la fonction \( 1+| f |\) qui est dans \( L^{\infty}\).

		\spitem[Cas \( p<\infty\).]

		Toute suite convergente étant de Cauchy, nous considérons une suite de Cauchy \( (f_n)\) dans \( L^p(\Omega)\) et ce sera suffisant pour travailler sur le premier point. Pour montrer qu'une suite de Cauchy converge, il est suffisant de montrer qu'une sous-suite converge. Soit \( \varphi\colon \eN\to \eN\) une fonction strictement croissante telle que pour tout \( n\geq 1\) nous ayons
		\begin{equation}
			\| f_{\varphi(n+1)}-f_{\varphi(n)} \|_p\leq \frac{1}{ 2^{n} }.
		\end{equation}
		Pour créer la fonction \( \varphi\), il est suffisant de prendre le \( N_k\) donné par la condition de Cauchy pour \( \epsilon=1/2^k\) et de considérer la fonction définie par récurrence\footnote{Utilisation du théorème \ref{THOooEJPYooZFVnez}. Vous n'êtes pas \randomGender{obligé}{obligée} de le citer à chaque fois, mais c'est bien de garder en tête que la définition de fonctions par récurrence n'est pas quelque chose de complètement trivial.} par \( \varphi(1)=N_1\) et \( \varphi(n+1)>\max\{ N_n,\varphi(n) \}\). Ensuite nous considérons la fonction
		\begin{equation}
			g_n(x)=\sum_{k=1}^n| f_{\varphi(k+1)}(x)-f_{\varphi(k)}(x) |.
		\end{equation}
		Notons que pour écrire cela nous avons considéré des représentants \( f_k\) qui sont alors des fonctions à l'ancienne. Étant donné que \( g_n\) est une somme de fonctions dans \( L^p\), c'est une fonction \( L^p\), comme nous pouvons le constater en calculant sa norme :
		\begin{equation}
			\| g_n \|_p\leq \sum_{k=1}^n\| f_{\varphi(k+1)}-f_{\varphi(k)} \|_p\leq\sum_{k=1}^n\frac{1}{ 2^k }\leq\sum_{k=1}^{\infty}\frac{1}{ 2^k }=1.
		\end{equation}
		Étant donné que tous les termes de la somme définissant \( g_n\) sont positifs, la suite \( (g_n)\) est croissante. Mais elle est bornée en norme \( L^p\) et donc sujette à obéir au théorème de Beppo-Levi~\ref{ThoRRDooFUvEAN} sur la convergence monotone. Il existe donc une fonction \( g\in L^p(\Omega)\) telle que \( g_n\to g\) presque partout.

		Soit un \( x\in \Omega\) pour lequel \( g_n(x)\to g(x)\); alors pour tout \( n\geq 2\) et \( \forall q\geq 0\),
		\begin{subequations}    \label{EqWTHojCq}
			\begin{align}
				| f_{\varphi(n+q)}(x)-f_{\varphi(n)}(x) | & =\Big| \underbrace{_{\varphi(n+q)}(x)+\sum_{k=1}^{q-1}f_{\varphi(n+k)}(x)}_{=\sum_{k=1}^q} \underbrace{\sum_{k=1}^{q-1}f_{\varphi(n+k)}(x)-f_{\varphi(n)}(x)}_{=\sum_{k=1}^q} \Big| \\
				                                          & =\left| \sum_{k=1}^qf_{\varphi(n+k)}(x)-\sum_{k=1}^qf_{\varphi(n+k-1)}(x) \right|                                                                                                   \\
				                                          & \leq \sum_{k=1}^q\Big| f_{\varphi(n+k)}(x)-f_{\varphi(n+k-1)}(x) \Big|                                                                                                              \\
				                                          & =g_{n+q+1}(x)-g_{n+1}(x)                                                                                                                                                            \\
				                                          & \leq g(x)-g_{n+1}(x).
			\end{align}
		\end{subequations}
		Nous prenons la limite \( n\to \infty\); la dernière expression tend vers zéro et donc
		\begin{equation}
			| f_{\varphi(n+q)}(x)-f_{\varphi(n)}(x) |\to 0
		\end{equation}
		pour tout \( q\). Donc pour presque tout \( x\in \Omega\), la suite \( n\mapsto f_{\varphi(n)}(x)\) est de Cauchy dans \( \eR\) et donc y converge vers un nombre que nous nommons \( f(x)\). Cela définit une fonction
		\begin{equation}
			\begin{aligned}
				f\colon \Omega\setminus E & \to \eR                                      \\
				x                         & \mapsto \lim_{n\to \infty} f_{\varphi(n)}(x)
			\end{aligned}
		\end{equation}
		où \( E\) est de mesure nulle. Montrons que \( f\) est bien dans \( L^p(\Omega)\); pour cela nous complétons la série d'inégalités \eqref{EqWTHojCq} en
		\begin{equation}
			\big| f_{\varphi(n+q)}(x)-f_{\varphi(n)}(x) \big|\leq g(x)-g_{n-1}(x)\leq g(x).
		\end{equation}
		En prenant la limite \( q\to \infty\) nous avons l'inégalité
		\begin{equation}    \label{EqMQbDRac}
			| f(x)-f_{\varphi(n)}(x) |\leq g(x)
		\end{equation}
		pour presque tout \( x\in\Omega\), c'est-à-dire pour tout \( x\in\Omega\setminus E\). Cette inégalité implique deux choses valables pour presque tout \( x\) dans \( \Omega\) :
		\begin{subequations}
			\begin{align}
				f(x)              & \in B\big( g(x),f_{\varphi(n)}(x) \big) \\
				f_{\varphi(n)}(x) & \leq | f(x) |+| g(x) |.
			\end{align}
		\end{subequations}

		La première inégalité assure que \( | f |^p\) est intégrable sur \( \Omega\setminus E\) parce que \( | f |\) est majorée par \( | g |+| f_{\varphi(n)} |\). Elle prouve par conséquent le point~\ref{ItemPDnjOJzi} parce que \(n\mapsto f_{\varphi(n)}\) est une sous-suite convergente presque partout. La seconde montre le point~\ref{ItemPDnjOJzii}.

		Attention : à ce point nous avons prouvé que \( n\mapsto f_{\varphi(n)}\) est une suite de fonctions qui converge \emph{ponctuellement presque partout} vers une fonction \( f\) qui s'avère être dans \( L^p\). Nous n'avons pas montré que cette suite convergeait au sens de \( L^p\) vers \( f\). Ce que nous devons montrer est que
		\begin{equation}    \label{EqJLfnEvj}
			\| f-f_{\varphi(n)} \|_p\to 0.
		\end{equation}
		L'inégalité \eqref{EqMQbDRac} nous donne aussi, toujours pour presque tout \( x\in \Omega\) :
		\begin{equation}
			\big| f(x)-f_{\varphi(n)}(x) \big|^p\leq g(x)^p
		\end{equation}
		ce qui signifie que la suite\quext{À ce point, \cite{KXjFWKA} se contente de majorer \( | f_{\varphi(n)}(x) |\) par \( | f(x) |+|g(x)|\), mais je ne comprends pas comment cette majoration nous permet d'utiliser la convergence dominée de Lebesgue pour montrer \eqref{EqJLfnEvj}.} \(    | f-f_{\varphi(n)} |^p    \) est dominée par la fonction \( | g |^p\) qui est intégrable sur \( \Omega\setminus E\) et tout autant sur \( \Omega\) parce que \( E\) est négligeable; cela prouve au passage le point~\ref{ItemPDnjOJzii}, et le théorème de la convergence dominée de Lebesgue (\ref{ThoConvDomLebVdhsTf}) nous dit que
		\begin{equation}
			\lim_{n\to \infty} \int_{\Omega} \big| f(x)-f_{\varphi(n)}(x) \big|^pdx=\int_{\Omega}\lim_{n\to \infty} \big| f(x)-f_{\varphi(n)}(x) \big|dx=0.
		\end{equation}
		Cette dernière suite d'égalités se lit de la façon suivante :
		\begin{equation}
			\lim_{n\to \infty} \| f-f_{\varphi(n)} \|_p=\big\| \lim_{n\to \infty} | f-f_{\varphi(n)} | \big\|_p=0.
		\end{equation}
		Nous en déduisons que la suite \( n\mapsto f_{\varphi(n)}\) est convergente vers \( f\) au sens de la norme \( L^p(\Omega)\). Or la suite de départ \( (f_n)\) était de Cauchy (pour la norme \( L^p\)); donc l'existence d'une sous-suite convergente implique la convergence de la suite entière vers \( f\), ce qu'il fallait démontrer.
	\end{subproof}
\end{proof}

Le théorème suivant est souvent cité en disant que \( L^p\) est un espace de Hilbert si et seulement si \( p=2\). Comme vous le voyez, il faut un peu plus d'hypothèses.

Je précise que je suis le seul à nommer ce théorème par le nom de Weinersmith. Je ne sais pas si il a déjà un nom; alors pourquoi pas celui-là plutôt qu'un autre ? La raison de ce choix est dans la constante de Weiner, définition \ref{DEFooXVXSooVJDTPy}.
\begin{theorem}[Théorème de Weinersmith\cite{BIBooVBCVooWNVKZA,BIBooPRUMooMKIALQ}]      \label{THOooCCMBooGulxkQ}
	Nous considérons un espace mesuré \( (\Omega,\tribA, \mu)\) ainsi qu'un nombre \( p\in \mathopen[ 1 , \infty \mathclose]\). Nous supposons
	\begin{enumerate}
		\item
		      \( L^p(\Omega,\tribA,\mu)\) est un espace de Hilbert,
		\item
		      Il existe des parties \( A,B\subset L^p(\Omega,\tribA,\mu)\) telles que \( A\cap B=\emptyset\) et \( 0<\mu(A)<\infty\) et \( 0<\mu(B)<\infty\).
	\end{enumerate}
	Alors \( p=2\).
\end{theorem}

\begin{proof}
	Vu que \( L^p\) est un espace de Hilbert (hypothèse), il vérifie l'identité du parallélogramme de la proposition \ref{PROPooSSYJooHAXAnC}, c'est-à-dire
	\begin{equation}        \label{EQooAKKYooURIbvi}
		\| f+g \|^2_p+\| f-g \|^2_p=2\| f \|_p^2+2\| g \|^2_p.
	\end{equation}

	\begin{subproof}
		\spitem[Pour \( 1\leq p<\infty\)]


		Soient donc \( A,B\) comme dans l'hypothèse. Nous considérons les fonctions
		\begin{subequations}
			\begin{align}
				f & =\frac{1}{ \mu(A)^{1/p} }\mtu_A  \\
				g & =\frac{1}{ \mu(B)^{1/p} }\mtu_B.
			\end{align}
		\end{subequations}
		En ce qui concerne les normes \( L^p\) de \( f\) et \( g\), c'est un calcul simple :
		\begin{equation}
			\| f \|_p^2=\left( \int_{\Omega}| f(\omega) |^pd\mu \right)^{2/p}=\left( \int_A\frac{1}{ \mu(A) }d\mu \right)^{2/p}=1.
		\end{equation}
		De même pour \( g\) : \( \| f \|_p^2=\| g \|_p^2=1\). Donc
		\begin{equation}
			2\| f \|_p^2+2\| g \|_p^2=4
		\end{equation}

		En ce qui concerne la somme,
		\begin{equation}
			\| f+g \|_p^2=\left( \int_A\frac{ d\mu }{ \mu(A) }+\int_B\frac{d\mu}{ \mu(B) } \right)^{2/p}=2^{2/p}.
		\end{equation}
		Pour la différence, la seule subtilité à voir est que
		\begin{equation}
			\int_{\Omega}| \mtu_A-\mtu_B |^p=
			\int_A| \mtu_A-\mtu_B |^p+\int_B| \mtu_A-\mtu_B |^p
			=\int_A| \mtu_A |+\int_B| -\mtu_B |=\int_A\mtu_A+\int_B\mtu_B.
		\end{equation}
		Ce n'est pas de la magie que le moins se change en plus. Bref, pour la différence nous avons
		\begin{subequations}
			\begin{align}
				\| f-g \|^2 & =\left( \int_{\Omega}| f(\omega)-g(\omega) |^pd\mu \right)^{2/p}                                                                     \\
				            & =\left( \int_{\Omega}| \frac{1}{ \mu(A)^{1/p} }\mtu_A(\omega)-\frac{1}{ \mu(B)^{1/p} }\mtu_B(\omega) |^pd\mu \right)^{2/p}           \\
				            & =\left( \int_A| \frac{ \mtu_A(\omega) }{ \mu(A)^{1/p} } |^p   +\int_B| -\frac{ \mtu_B(\omega) }{ \mu(B)^{1/p} } |^p    \right)^{2/p} \\
				            & =2^{2/p}.
			\end{align}
		\end{subequations}
		Donc \( \| f+g \|^{2}_p+\| f-g \|_p^2=2\times 2^{2/p}\).

		Vu que \( L^p\) est un espace de Hilbert, nous avons finalement
		\begin{equation}
			4=2\times 2^{2/p}.
		\end{equation}
		Cela est uniquement valable pour \( p=2\).
		\spitem[Pour \( p=\infty\)]
		Il suffit de prendre \( f=\mtu_A\) et \( g=\mtu_B\). Nous avons alors \(  \| f \|_{L^{\infty}}^2= \| g \|_{L^{\infty}}^2  =  \| f+g \|_{L^{\infty}}^2=\| f-g \|_{L^{\infty}}^2=1\).

		L'égalité \eqref{EQooAKKYooURIbvi} devient \( 2=4\), ce qui est faux.
	\end{subproof}
\end{proof}

Si \( (\Omega,\tribA,\mu)\) est un espace mesuré, est-ce que \( L^p(\Omega,\tribA,\mu)\) est assuré de n'être pas de Hilbert ? Non.

\begin{example}[\cite{BIBooPRUMooMKIALQ}]
	Soit la mesure de Dirac sur \( \eR\), c'est-à-dire
	\begin{equation}
		\delta(A)=\begin{cases}
			1 & \text{si } 0\in A \\
			0 & \text{sinon. }
		\end{cases}
	\end{equation}
	Nous allons prouver que pour tout \( p\in \mathopen[ 0 , \infty \mathclose]\), l'espace \( L^p(\eR,\Borelien(\eR),\delta)\) est un espace de Hilbert. Pour cela nous introduisons le produit hermitien
	\begin{equation}
		\langle f, g\rangle =f(0)\overline{ g(0) }.
	\end{equation}

	Nous avons par ailleurs la norme
	\begin{equation}
		\| f \|^p=\int_{\eR}| f(x) |^pd\delta=f(0)^p.
	\end{equation}
	Donc oui, \( \| f \|_p=\sqrt{ \langle f, f\rangle  }\).
\end{example}

%--------------------------------------------------------------------------------------------------------------------------- 
\subsection{Théorèmes d'approximation}
%---------------------------------------------------------------------------------------------------------------------------

\begin{proposition}[\cite{ooRCYWooNAeaTA}, thème \ref{THEMEooKLVRooEqecQk}]      \label{PROPooUQUBooAWgNhm}
	Soient \( 1\leq p\leq\infty\) et un espace mesuré \( (\Omega,\tribA,\mu)\). Alors les fonctions étagées\footnote{Définition \ref{DefBPCxdel}. Pour rappel, une fonction est simple lorsqu'elle prend un nombre fini de valeurs, et elle est étagée lorsqu'elle est en outre mesurable.} dans \( L^p(\Omega)\) sont denses dans \( L^p(\Omega)\).
\end{proposition}

\begin{proof}
	Soit \( f\in L^1(\Omega,\tribA,\mu)\). Nous supposons dans un premier temps que \( f\colon \Omega\to \mathopen[ 0 , \infty \mathclose[\). Pour les fonctions à valeurs dans \( \eC\), nous verrons plus bas.

	Notons que la partie \( \{ x\in \Omega\tq f(x)=\infty \}\) est de mesure nulle, donc nous pouvons vraiment choisir un représentant à valeurs dans \( \mathopen[ 0 , \infty \mathclose[\) et non à valeurs dans \( \mathopen[ 0 , \infty \mathclose]\) comme le serait un représentant un peu quelconque.

	Par le théorème \ref{THOooXHIVooKUddLi}, il existe une suite croissante de fonctions étagées \( \phi_n\colon \Omega\to \mathopen[ 0 , \infty \mathclose[\) telles que \( \phi_n\to f\) ponctuellement. Notons que ce théorème fonctionne parce que les fonctions \( L^p\) (en tout cas leurs représentants) sont mesurables parce que c'est dans la définition \ref{DEFooTHIDooWYzBtn}.  Notre devoir est maintenant de prouver que sous l'hypothèse que \( f\) est dans \( L^p\), alors la convergence \( \phi_n\to f\) est une convergence dans \( L^p\).

	\begin{subproof}
		\spitem[\( 1\leq p<\infty\)]
		Vu que \( f\) et \( \phi_n\) sont à valeurs positives nous avons \( | f-\phi_n |^p\leq | f |^p\). Mais par hypothèse \( | f |^p\in L^1(\Omega)\). Donc la suite \( g_n=| f-\phi_n |^p\) est majorée (uniformément en \( n\)) par \( | f |^p\) qui est dans \( L^1\). Le théorème de la convergence dominée permet de permuter
		\begin{equation}
			\lim_{n\to \infty} \int g_n=\int \lim_{n\to \infty} g_n=0.
		\end{equation}
		Cela revient à dire que
		\begin{equation}
			\lim_{n\to \infty} \| f-\phi_n \|^p_p=\lim_{n\to \infty} \int| f-\phi_n |^p=\int\lim_{n\to \infty} | f-\phi_n |^p=0,
		\end{equation}
		ce qui signifie que \( \phi_n\stackrel{L^p(\Omega)}{\longrightarrow}f\).
		\spitem[Pour \( p=\infty\)]
		Si \( f\in L^{\infty}(\Omega)\), alors nous pouvons prendre un représentant borné. Avec lui, nous avons \( \phi_n\stackrel{\| . \|_{\infty}}{\longrightarrow}f\). Avec cela nous avons, pour \( \epsilon\) donné, un \( n\) assez grand pour avoir
		\begin{equation}
			N_{\infty}(\phi_n-f)\leq \| \phi_n-f \|_{\infty}<\epsilon.
		\end{equation}
	\end{subproof}
	Si \( f\) est à valeurs dans \( \eC\) au lieu de \( \mathopen[ 0 , \infty \mathclose[\), il suffit de faire valoir le travail que nous venons de faire quatre fois, pour les valeurs réelles, imaginaires, positives et négatives.
\end{proof}

\begin{proposition}[Continuité de l'intégrale par rapport à la mesure\cite{BIBooPSYBooAMIHCm}]	\label{PROPooPTZRooCBkZvr}
	Soit \( p\in \mathopen[ 1,\infty\mathclose[\). Nous considérons \( f\in L^p(\Omega,\tribA,\mu)\). Alors pour tout \( \epsilon>0\), il existe un \( \delta>0\) tel que
	\begin{equation}
		\int_A| f |^pd\mu<\epsilon
	\end{equation}
	dès que \( \mu(A)<\delta\).
\end{proposition}

\begin{proof}
	Soit \( \epsilon>0\). La proposition \ref{PROPooUQUBooAWgNhm} dit que les applications étagées sont denses dans \( L^p\). Il existe donc une fonction étagée \( \phi\in L^p(\Omega,\tribA,\mu)\) telle que \( \| f-\phi \|_p<\epsilon\). Étant donné que \( \phi\) est étagée, elle prend un nombre fini de valeurs et aucune de ces valeurs est \( \infty\). Donc il existe \( M>0\) tel que \( | \phi(x) |<M\) pour tout \( x\in \Omega\).

	Nous posons \( \delta=(\epsilon/2M)^p\). Soit \( A\in\tribA\) tel que \( \mu(A)<\delta\). Nous utilisons l'inégalité de Minkowski de la proposition \ref{PropInegMinkKUpRHg}\ref{ItemDHukLJi} :
	\begin{equation}
		\| f \|_{A,p}\leq\| f-\phi \|_{A,p}+\| \phi \|_{A,p}\leq \frac{ \epsilon }{2}+M\mu(A^{1/p})<\epsilon.
	\end{equation}
	Donc nous avons bien \( \int_A| f |^p<\epsilon\).
\end{proof}

\begin{corollary}[\cite{MonCerveau}]	\label{CORooSGMXooCNfGNk}
	Si \( f\in L^1\big( \mathopen[ a,b\mathclose] \big)\), alors
	\begin{equation}
		\lim_{h\to 0}\int_a^{a+h}f=0.
	\end{equation}
\end{corollary}

\begin{proof}
	Nous notons \( \lambda\) la mesure de Lebesgue. Soit \( \epsilon>0\). La proposition \ref{PROPooPTZRooCBkZvr} dit qu'il existe \( \delta>0\) tel que \( \lambda(A)<\delta\) implique \( \int_af=0\). Si \( | h |<\delta\) nous avons \( \lambda\big( \mathopen[ a,a+h\mathclose] \big)<\delta \) et donc
	\begin{equation}
		\int_a^{a+h}fd\lambda=\int_{\mathopen[ a,a+h\mathclose]}fd\lambda<\epsilon.
	\end{equation}
\end{proof}

\begin{proposition}[\cite{MonCerveau}]	\label{PROPooANISooKzQrnH}
	Soit un intervalle \( I\subset \eR\). Soient \( f\in L^1(I)\) et \( a\in I\). Alors l'application
	\begin{equation}
		\begin{aligned}
			\phi\colon I & \to \eR                \\
			x            & \mapsto \int_a^xf(t)dt
		\end{aligned}
	\end{equation}
	est continue.
\end{proposition}

\begin{proof}
	Soit \( h>0\) tel que \( B(a,h)\subset I\), étant compris que la boule est au sens de la topologie sur \( I\); donc si \( a\) est une extrémité de \( I\), cela ne comprend que des \( h\geq 0\) ou \( h\leq 0\). Nous avons
	\begin{equation}
		\phi(x+h)-\phi(x)=\int_a^{x+h}-\int_a^xf=\int_x^{x+h}f.
	\end{equation}
	La limite \( h\to0\) donne zéro par le corolaire \ref{CORooSGMXooCNfGNk}.
\end{proof}

%---------------------------------------------------------------------------------------------------------------------------
\subsection{Densité des fonctions infiniment dérivables à support compact}
%---------------------------------------------------------------------------------------------------------------------------

\begin{definition}
	Une fonction est \defe{étagée par rapport à \( L^p\)}{fonction!étagée} si elle est de la forme
	\begin{equation}
		f=\sum_{k=1}^Nc_k\mtu_{B_k}
	\end{equation}
	où les \( B_k\) sont des mesurables disjoints et \( \mtu_{B_k}\in L^p\) pour tout \( k\).
\end{definition}

\begin{lemma}   \label{LemWHIRdaX}
	Si \( f\) est une fonction étagée en même temps qu'être dans \( L^p\), alors elle est étagée par rapport à \( L^p\).
\end{lemma}

\begin{proof}
	Nous pouvons écrire
	\begin{equation}
		f=\sum_{k=1}^Nc_k\mtu_{B_k}
	\end{equation}
	où les \( B_k\) sont disjoints. Par hypothèse \( \| f \|_p\) existe. Donc chacune des intégrales \( \int_{\Omega}| \mtu_{B_k} |^p\) doit exister parce que les \( B_k\) étant disjoints, nous pouvons inverser la norme et la somme ainsi que la somme et l'intégrale :
	\begin{equation}
		\int_{\Omega}|f|^p=\int_{\Omega}\sum_{k=1}^N| c_k\mtu_{B_k}(x) |^pdx=\sum_{k=1}^N\int| c_k\mtu_{B_k}(x) |^pdx=\sum_{k=1}^N| c_k |^p\int_{\Omega}| \mtu_{B_k}(x) |^pdx.
	\end{equation}
\end{proof}

L'ensemble \(  C^{\infty}_c(\eR^d)\) des fonctions de classe \(  C^{\infty}\) et à support compact sur \( \eR^d\) est souvent également noté \( \swD(\eR^d)\).
\begin{theorem}[\cite{TUEWwUN}] \label{ThoILGYXhX}
	Nous avons des densités emboitées. Ici \( D\) est un borélien borné de \( \eR^d\) contenu dans \( B(0,r)\) et \( K\) est un compact contenant \( B(0,r+2)\).
	\begin{enumerate}
		\item
		      Les fonctions étagées par rapport à \( L^p\) sur \( \eR^d\) sont denses dans \( L^p(\eR^d)\). A fortiori les fonctions étagées sont denses dans \( L^p\), mais nous n'en aurons pas besoin ici.
		      \item\label{ItemYVFVrOIii}
		      Il existe une suite \( f_n\) dans \(  C(K,\eC)\) telle que
		      \begin{equation}
			      f_n\stackrel{L^p}{\to}\mtu_{D}.
		      \end{equation}
		      \item\label{ItemYVFVrOIiii}
		      Si \( A\) est un borélien tel que \( \mtu_A\in L^p(\eR^d)\)\quext{Je pense que cette hypothèse manque dans \cite{TUEWwUN}. En tout cas je vois mal comment je pourrais justifier les différentes étapes de la preuve en prenant par exemple \( A=\eR^d\).}, alors il existe une suite de boréliens bornée \( (D_n)_{n\in \eN}\) tels que
		      \begin{equation}
			      \mtu_{D_n}\stackrel{L^p}{\to}\mtu_A.
		      \end{equation}
		      \item\label{ItemYVFVrOIiv}
		      Il existe une suite \( \varphi_n\) dans \( \swD(\eR^d)=  C^{\infty}_c(\eR^d)\) telle que
		      \begin{equation}
			      \varphi_n\stackrel{L^p}{\to}\mtu_{D}.
		      \end{equation}

		      \item\label{ItemYVFVrOIv}
		      L'ensemble \(\swD(\eR^d)= C^{\infty}_c(\eR^d)\) est dense dans \( L^p(\eR^d)\) pour tout \( 1\leq p<\infty\).
	\end{enumerate}
\end{theorem}
\index{densité!de \( C^{\infty}_c(\eR^d)\) dans \( L^p(\eR^d)\)}
\index{densité!des fonctions étagées dans \( L^p\)}

\begin{proof}
	Nous allons montrer les choses point par point.
	\begin{enumerate}
		\item
		      Si \( f\in L^1(\eR^d)\), nous savons par le théorème \ref{THOooXHIVooKUddLi} qu'il existe une suite \( f_n\) de fonctions étagées convergeant ponctuellement vers \( f\) telle que \( | f_n |\leq | f |\). La proposition~\ref{PropBVHXycL} nous dit qu'alors \( f_n\stackrel{L^p}{\to}f\).

		      La fonction \( f_n\) étant étagée et dans \( L^p\) en même temps, elle est automatiquement étagée par rapport à \( L^p\) par le lemme~\ref{LemWHIRdaX}.

		      \item\label{ItemYVFVrOIi}

		      C'est le théorème d'approximation~\ref{ThoAFXXcVa} appliqué au borélien \( D\) contenu dans l'espace mesuré \( K\).

		\item

		      En vertu du point~\ref{ItemYVFVrOIii}, il existe \( f\in C^0(K,\eR)\) telle que
		      \begin{equation}
			      \| f-\mtu_D \|_p\leq \epsilon.
		      \end{equation}
		      Ensuite, par le théorème de Weierstrass, il existe \( \varphi\in C^{\infty}(K,\eR)\) telle que \( \| f-\varphi \|_{\infty}\leq \epsilon\). Nous avons aussi
		      \begin{equation}
			      \| \varphi-f \|^p_p=\int_K| \varphi(x)-f(x) |^pdx\leq\mu(X)\| \varphi-f \|_{\infty}^p\leq \epsilon^p\mu(K).
		      \end{equation}
		      Quitte à prendre un \( \varphi\) correspondant à un \( \epsilon\) plus petit, nous avons
		      \begin{equation}
			      \| \varphi-f \|\leq \epsilon.
		      \end{equation}
		      En combinant et en passant à \( \epsilon/2\) nous avons trouvé une fonction \( \varphi\in  C^{\infty}(K,\eR)\) telle que
		      \begin{equation}
			      \| \varphi-\mtu_D \|\leq \epsilon.
		      \end{equation}

		\item

		      Nous considérons les boréliens fermés \( D_n=A\cap B(0,n)\). Alors \( \mtu_{D_n}\in L^p\) et nous avons pour \( n\) assez grand :
		      \begin{equation}
			      \int_{\eR^d}| \mtu_{D_n}(x)-\mtu_{A}(x) |^pdx=\int_{\eR^d\setminus B(0,n)}| \mtu_A(x) |^p<\epsilon,
		      \end{equation}
		      c'est-à-dire que \( \mtu_{D_n}\stackrel{L^p}{\to}\mtu_A\).

		\item

		      Il suffit de remettre tout ensemble. Si \( f\in L^p(\eR^d)\), par le point~\ref{ItemYVFVrOIi} nous commençons par prendre \( \sigma\) étagée par rapport à \( L^p\) telle que
		      \begin{equation}
			      \| \sigma-f \|_p\leq\epsilon.
		      \end{equation}
		      Ensuite nous écrivons \( \sigma\) sous la forme
		      \begin{equation}
			      \sigma=\sum_{k=1}^Nc_k\mtu_{B_k}
		      \end{equation}
		      et nous appliquons le point~\ref{ItemYVFVrOIiii} à chacune des \( \mtu_{B_k}\) pour trouver des boréliens bornés \( D_k\) tels que
		      \begin{equation}
			      \| \mtu_{D_k}-\mtu_{B_k} \|_p\leq \epsilon.
		      \end{equation}
		      Enfin nous appliquons le point~\ref{ItemYVFVrOIiv} pour trouver des fonctions \( \varphi_k\in C^{\infty}_c(\eR^d)\) telles que
		      \begin{equation}
			      \| \varphi_k-\mtu_{D_k} \|_p\leq \epsilon.
		      \end{equation}

		      Il n'est pas compliqué de calculer que
		      \begin{equation}
			      \big\| \sum_{k=1}^Nc_k\varphi_k-f \big\|_p\leq 2\epsilon\sum_kc_k+\epsilon.
		      \end{equation}

	\end{enumerate}
\end{proof}

\begin{corollary}   \label{CorFZWooYNbtPz}
	Si \( 1<p<\infty\) alors la partie\footnote{Nous parlons bien ici de l'\emph{ensemble} \( L^2\) parce que nous le considérons sans norme ou topologie particulière. La densité dont nous parlons ici est celle pour la topologique de \( L^p\).} \( L^2\big( \mathopen[ 0 , 1 \mathclose] \big)\cap L^p\big( \mathopen[ 0 , 1 \mathclose] \big)\) est dense dans \( L^p\big( \mathopen[ 0 , 1 \mathclose] \big)\).
\end{corollary}
\index{densité!de \( L^2\big( \mathopen[ 0 , 1 \mathclose] \big)\) dans \( L^p\big( \mathopen[ 0 , 1 \mathclose] \big)\)}

\begin{proof}
	Nous savons du théorème~\ref{ThoILGYXhX}\ref{ItemYVFVrOIv} que \(  C^{\infty}_c\big( \mathopen[ 0 , 1 \mathclose] \big)\) est dense dans \( L^p\). Mais nous avons évidemment \(  C^{\infty}_c\subset L^2\cap L^p\), donc \( L^2\cap L^p\) est dense dans \( L^p\).
\end{proof}

\begin{lemma}[\cite{TUEWwUN,ooBBNWooHJPWci}]   \label{LemCUlJzkA}
	Soit \( 1\leq p<\infty\) et \( f\in L^p(\Omega)\). Nous notons \( \tau_v\) l'opérateur de translation par \( v\) :
	\begin{equation}
		\begin{aligned}
			\tau_v\colon L^p(\Omega) & \to L^p(\Omega)                      \\
			f                        & \mapsto \Big[ x\mapsto f(x-v) \Big].
		\end{aligned}
	\end{equation}
	Pour chaque \( f\in L^p(\Omega)\), l'application
	\begin{equation}
		\begin{aligned}
			\tau(f)\colon \eR^d & \to L^p(\Omega)   \\
			v                   & \mapsto \tau_v(f)
		\end{aligned}
	\end{equation}
	est continue en \( v=0\), c'est-à-dire
	\begin{equation}
		\lim_{v\to 0}\| \tau_v(f)-f \|_p=0.
	\end{equation}
\end{lemma}

\begin{proof}
	Nous commençons par supposer que \( f\) est dans \( \swD(\Omega)\), et nous verrons ensuite comment généraliser.

	\begin{subproof}
		\spitem[Si \( f\in\swD(\Omega)\)]

		Soit une suite \( v_i\stackrel{\eR^d}{\longrightarrow}0\), et posons \( f_i=\tau_{v_i}(f)\); le but est de montrer que \( f_i\stackrel{L^p}{\longrightarrow}f\). Pour cela, la fonction \( f-f_i\) est également à support compact, et qui plus est, si \( \supp(f)\subset B(0,r)\), alors \( \supp(f-f_i)\subset B(0,r+| v_i |)\), et l'ensemble
		\begin{equation}
			S=\overline{B \big( 0,r+\max_i| v_i | \big)}
		\end{equation}
		est un compact contenant les supports de tous les \( f-f_i\). Le maximum existe parce que \( v_i\to 0\). Voilà qui «majore» le domaine de \( f-f_i\) uniformément en \( i\).

		Majorons maintenant \( | f-f_i |^p\) de façon uniforme en \( i\). Soit le nombre
		\begin{equation}
			M=2\max_{x\in \eR^d}\{ f(x) \}.
		\end{equation}
		La fonction qui vaut \( M^p\) sur \( S\) et zéro ailleurs est une fonction intégrable qui majore \( | f-f_i |^p\). Nous pouvons donc utiliser la convergence dominée de Lebesgue (théorème~\ref{ThoConvDomLebVdhsTf}) pour écrire
		\begin{equation}
			\lim_{i\to \infty} \| f-f_i \|^p_p=\lim_{i\to \infty} \int_{\Omega} | f(x)-f(x-v_i) |^pdx=\int_{\Omega}\lim_{i\to \infty} | f(x)-f(x-v_i) |dx=0.
		\end{equation}

		\spitem[Pour \( f\in L^p(\Omega)\)]

		Soit \( \epsilon>0\), \( f\in L^p(\Omega)\) et \( \varphi\in\swD(\Omega)\) tel que \(  \| f-\varphi \|_p\leq \epsilon\). Cela est possible par la densité de \( \swD(\Omega)\) dans \( L^p(\Omega)\) vue en~\ref{ThoILGYXhX}\ref{ItemYVFVrOIv}. Nous choisissons de plus \( | v |\) assez petit pour avoir \( \| \tau_v(\varphi)-\varphi \|_p<\epsilon\), qui est possible en vertu de ce que nous venons de démontrer à propos des fonctions à support compact. De plus \( \tau_v\) étant une isométrie de \( L^p\) nous avons \( \| \tau_v(\varphi)-\tau_v(f) \|=\| \varphi-f \|<\epsilon\). Nous avons tout pour majorer :
		\begin{equation}
			\| f-\tau_v(f) \|\leq \| f-\varphi \|+\| \varphi-\tau_v(\varphi) \|+\| \tau_v(\varphi)-\tau_v(f) \|\leq 3\epsilon.
		\end{equation}
		Nous avons donc bien \( \lim_{v\to 0} \| f-\tau_v(f) \|=0\).
	\end{subproof}
\end{proof}
