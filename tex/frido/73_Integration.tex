% This is part of Mes notes de mathématique
% Copyright (c) 2009-2018
%   Laurent Claessens
% See the file fdl-1.3.txt for copying conditions.

%+++++++++++++++++++++++++++++++++++++++++++++++++++++++++++++++++++++++++++++++++++++++++++++++++++++++++++++++++++++++++++
\section{Formes différentielles exactes et fermées}
%+++++++++++++++++++++++++++++++++++++++++++++++++++++++++++++++++++++++++++++++++++++++++++++++++++++++++++++++++++++++++++

Nous avons déjà parlé de formes différentielles et de leurs intégrales sur un chemin dans la section~\ref{SecFormDiffRappel}.

\begin{definition}  \label{DefEFKQmPs}
La forme différentielle $\omega$ est \defe{exacte}{forme!différentielle!exacte} si il existe une fonction $f$ telle que $\omega=df$; elle est dite \defe{fermée}{forme!différentielle!fermée} si $d\omega=0$.
\end{definition}

Dire que la forme différentielle $\omega=fdx+gdy$ est fermée, c'est dire que
\begin{equation}
    \frac{ \partial g }{ \partial x }=\frac{ \partial f }{ \partial y }.
\end{equation}

Il est naturel de se demander si toutes les formes différentielles sont des différentielles de fonctions. Une réponse complète est délicate à établir, mais a d'innombrables conséquences en physique, notamment en ce qui concerne l'existence d'un potentiel vecteur pour le champ magnétique dans les équations de Maxwell.

Le fait qu'une forme exacte soit fermée est relativement facile à établir; c'est la proposition suivante. La question plus délicate est la réciproque : sous quelles conditions une forme fermée est-elle exacte ?
\begin{proposition}
	Si $\omega$ est une $1$-forme exacte de classe $C^1$, alors $\omega$ est fermée.
\end{proposition}

\begin{proof}
	Le fait que $\omega$ soit exacte implique l'existence d'une fonction $f$ telle que $\omega=df$, c'est-à-dire
	\begin{equation}
		\omega_x=\sum_i a_i(x)dx_i=\sum_i\frac{ \partial f }{ \partial x_i }(x)dx_i,
	\end{equation}
	c'est-à-dire que $a_i(x)=\frac{ \partial f }{ \partial x_i }(x)$. L'hypothèse que $\omega$ est $C^1$ implique que $f$ est $C^2$, et donc que nous pouvons inverser l'ordre de dérivation pour les dérivées secondes $\partial^2_{ij}f=\partial^2_{ji}f$. Nous pouvons donc faire le calcul suivant :
	\begin{equation}
		\frac{ \partial a_i }{ \partial x_j }=\frac{ \partial  }{ \partial x_j }\frac{ \partial f }{ \partial x_i }=\frac{ \partial  }{ \partial x_i }\frac{ \partial f }{ \partial x_j }=\frac{ \partial a_j }{ \partial x_i },
	\end{equation}
	ce qu'il fallait démontrer.
\end{proof}

La réciproque est vraie sur un ouvert simplement connexe.
\begin{theorem}        \label{ThoFermeExactFormRappel}
Supposons que $D\subset\eR^n$ soit un ouvert simplement connexe. Alors toute forme différentielle de degré $1$ et de classe $C^1$ sur $D$ qui est fermée est exacte.
\end{theorem}

Nous allons prouver ce théorème dans un cas un peu moins général : celui d'un domaine étoilé de \( \eR^2\) plutôt que simplement connexe de \( \eR^n\).

\begin{theorem} \label{ThoMSofFxL}
Soit $D\subset\eR^2$, une ouvert étoilé, et $\omega$, une $1$-forme fermée de classe $C^1$. Alors $\omega$ est exacte.
\end{theorem}
\begin{proof}

Soit $D\subset\eR^2$, un ouvert étoilé par rapport à l'origine. Soient $f\colon D\to \eR$, $g\colon D\to \eR$, des fonctions de classe $C^1$ telles que
\begin{equation}
	\frac{ \partial f }{ \partial y }=\frac{ \partial g }{ \partial x }
\end{equation}
sur $D$, et
\begin{equation}		\label{EqIMDefFformI33}
	F(x,y)=\int_0^1\big[  f(tx,ty)x+g(tx,ty)y  \big]dt
\end{equation}
pour tout $(x,y)\in D$.

Étant donné que nous ne définissons $F(x,y)$ que pour des $(x,y)\in D$, la fonction $t\mapsto f(tx,ty)$ est $C^1$ sur tout le compact $[0,1]$ et aucune divergence de l'intégrale n'est à craindre. Nous sommes donc dans le cadre de la proposition~\ref{PropDerrSSIntegraleDSD}, et nous pouvons dériver sous le signe intégral.

Nous calculons, en utilisant la règle de dérivation de fonctions composées
\begin{equation}		\label{EqIMI33dsdsFlolo}
	\begin{aligned}[]
		\frac{ \partial F }{ \partial x }(x,t)	&=\int_0^1\left[   f\frac{ \partial f }{ \partial x }(tx,ty)x+f(tx,ty)+t\frac{ \partial g }{ \partial x }(tx,ty)y  \right]dt\\
		&=\int_0^1\left[ t\Big( x\frac{ \partial f }{ \partial x }(tx,ty)+y\frac{ \partial f }{ \partial y }(tx,ty) \Big)+f(tx,ty) \right]dt
	\end{aligned}
\end{equation}
où nous avons utilisé l'hypothèse $\partial_yf=\partial_xg$. Ce qui se trouve dans la parenthèse n'est autre que $\partial_t\big( f(tx,ty) \big)$, plus précisément, si nous posons $\mF(x,y,t)=f(tx,ty)$, nous avons
\begin{equation}
	\frac{ \partial \mF }{ \partial t }(x,y,t)= x\frac{ \partial f }{ \partial x }(tx,ty)+y\frac{ \partial f }{ \partial y }(tx,ty).
\end{equation}
En recopiant le résultat \eqref{EqIMI33dsdsFlolo} en termes de $\mF$, nous avons
\begin{equation}
	\begin{aligned}[]
		\frac{ \partial F }{ \partial x }(x,t)	&=\int_0^1\left( t\frac{ \partial \mF }{ \partial t }(x,y,t)+\mF(x,y,t) \right)dt\\
		&=\int_0^1\partial_t\big( t\mF(x,y,t) \big)dt\\
		&=\big[ f\mF(x,y,t) \big]_0^1\\
		&=\mF(x,y,1)\\
		&=f(x,y).
	\end{aligned}
\end{equation}
Le résultat correspondant pour $\frac{ \partial F }{ \partial y }(x,y)=g(x,y)$ s'obtient de la même manière. Nous avons donc obtenu que
\begin{equation}		\label{EqIMFormI33Fffdd}
	\begin{aligned}[]
		\frac{ \partial F }{ \partial x }&=f,  &\text{et}&& \frac{ \partial F }{ \partial y }=g.
	\end{aligned}
\end{equation}
En ayant prouvé cela, nous avons prouvé que si $\omega=fdx+gdy$ avec $\partial_yf=\partial_xg$, alors $\omega=dF$ où $F$ est définie par \eqref{EqIMDefFformI33}.
\end{proof}

\begin{proof}[Démonstration alternative du théorème~\ref{ThoMSofFxL}]
Nous posons $u=tx$ et $v=ty$, ainsi que $\mF(x,y,t)=f(u,v)$ et $\mG(x,y,t)=g(u,v)$. Avec cette notation, nous avons 
\begin{equation}
    F(x,y)=\int_0^1\big( x\mF(x,y,t)+y\mG(x,y,t) \big)dt,
\end{equation}
et
\begin{equation}
	\begin{aligned}[]
		\frac{ \partial \mF }{ \partial x }&=\frac{ \partial f }{ \partial u }\frac{ \partial u }{ \partial x }+\frac{ \partial f }{ \partial v }\frac{ \partial v }{ \partial x }=t\frac{ \partial f }{ \partial u },\\
		\frac{ \partial \mG }{ \partial x }&=t\frac{ \partial g }{ \partial u }.
	\end{aligned}
\end{equation}
Ainsi,
\begin{equation}
	\begin{aligned}[]
		\frac{ \partial F }{ \partial x }	&=\int_0^1\left( x\frac{ \partial \mF }{ \partial x }+\mF+y\frac{ \partial G }{ \partial x } \right)dt\\
							&=\int_0^1\left( xt\frac{ \partial f }{ \partial u } +\mF+yt\frac{ \partial g }{ \partial u } \right)dt\\
							&=\int_0^1\left[  t\left( x\frac{ \partial f }{ \partial u }+y\frac{ \partial f }{ \partial v } \right)+\mF  \right]dt.
	\end{aligned}
\end{equation}
où nous avons utilisé le fait que, par hypothèse, $\frac{ \partial g }{ \partial u }=\frac{ \partial f }{ \partial v }$. Nous calculons par ailleurs que
\begin{equation}
	\frac{ \partial F }{ \partial t }=\frac{ \partial f }{ \partial u }\frac{ \partial u }{ \partial t }+\frac{ \partial f }{ \partial v }\frac{ \partial v }{ \partial t }=x\frac{ \partial f }{ \partial u }+y\frac{ \partial f }{ \partial v }.
\end{equation}
Donc, nous avons
\begin{equation}
	\frac{ \partial F }{ \partial x }=\int_0^1\left( t\frac{ \partial \mF }{ \partial t }+\mF \right)dt=\int_0^1\frac{ \partial  }{ \partial t }(t\mF)dt.
\end{equation}
Par conséquent,
\begin{equation}
	\frac{ \partial F }{ \partial x }=[t\mF]_0^1=\mF(x,y,1)=f(x,y).
\end{equation}
Le même genre de calculs fournit $\frac{ \partial F }{ \partial y }=g(x,y)$.
\end{proof}

%+++++++++++++++++++++++++++++++++++++++++++++++++++++++++++++++++++++++++++++++++++++++++++++++++++++++++++++++++++++++++++
\section{Théorème d'Abel angulaire}
%+++++++++++++++++++++++++++++++++++++++++++++++++++++++++++++++++++++++++++++++++++++++++++++++++++++++++++++++++++++++++++

\begin{theorem}[Abel angulaire\cite{KXjFWKA}]   \label{ThoTGjmeen}
    Soit \( \sum_{n}a_nz^n\) une série entière de rayon de convergence plus grand ou égal à \( 1\) et de somme \( f\). Soit \( \theta_0\in\mathopen[ 0 , \frac{ \pi }{2} \mathclose[\). Nous posons
    \begin{equation}
        \Delta_{\theta_0}=\{ z=1-\rho e^{i\varphi}\tq \rho>0,\varphi\in\mathopen[ \theta_0 , \theta_0 \mathclose],| z |<1 \}.
    \end{equation}
    Nous supposons de plus que \( \sum_na_n\) converge. Alors
    \begin{equation}
        \lim_{\substack{z\to1\\z\in\Delta_0}}f(z)=\sum_{k=0}^{\infty}a_k.
    \end{equation}
\end{theorem}
\index{Abel!angulaire}
\index{convergence!suite numérique!Abel angulaire}
\index{somme partielles!Abel angulaire}
\index{série!entière!Abel angulaire}

\begin{proof}

    Le résultat de ce théorème est que l'on peut calculer la limite \( z\to 1\) avec des chemins contenus dans un domaine de la forme de celui dessiné à la figure~\ref{LabelFigJGuKEjH}. % From file JGuKEjH
    \newcommand{\CaptionFigJGuKEjH}{La zone dans laquelle peut être le chemin qui va vers \( z=1\).}
    \input{auto/pictures_tex/Fig_JGuKEjH.pstricks}

    De façon très classique nous posons
    \begin{equation}
        \begin{aligned}[]
            S&=\sum_{k=0}^{\infty}a_k&S_n&=\sum_{k=0}^na_k,
        \end{aligned}
    \end{equation}
    et \( R_n=S-S_n\). En particulier \( a_n=R_{n-1}-R_n\).

    Le but du théorème est de montrer que \( \sum a_nz^n\) converge vers \( S\) lorsque \( z\) converge vers \( 1\) à l'intérieur de \( \Delta_{\theta_0}\). Pour cela nous calculons pour un \( N\) donné la différence \( \sum_{n=0}^{N}a_nz^n-S_N\) en triant les termes par ordre de \( R_n\), en isolant le terme \( R_0\) et le terme \( R_N\) :
    \begin{subequations}
        \begin{align}
            \sum_{n=0}^Na_nz^n-S_N&=\sum_{n=1}^Na_n(z^n-1)\\
            &=\sum_{n=1}^N(R_{n-1}-R_n)(z^n-1)\\
            &=R_0(z-1)+\sum_{n=1}^{N-1}R_n(z^{n+1}-1-z^n+1)+R_N(z^N-1)\\
            &=R_0(z-1)+\sum_{n=1}^{N-1}R_nz^n(z-1)+R_N(z^N-1)\\
            &=(z-1)\sum_{n=0}^{N-1}R_nz^n+R_N(z^N-1).
        \end{align}
    \end{subequations}
    Cela est valable pour tout \( N\) et \( | z |<1\). Nous avons donc
    \begin{equation}
        \sum_{n=0}^Na_nz^n-S_N=(z-1)\sum_{n=0}^{N-1}R_nz^n+R_N(z^N-1).
    \end{equation}
    Par hypothèse nous avons \( \lim_{N\to \infty} R_N=0\). Et de plus le membre de gauche converge parce que chacun des deux termes converge séparément. En passant à la limite nous avons pour tout \( | z |<1\) :
    \begin{equation}
        f(z)-S=(z-1)\sum_{n=0}^{\infty}R_nz^n.
    \end{equation}
    Nous voudrions étudier le comportement de la différence \( f(z)-S\) lorsque \( z\) tend vers \( 1\). Pour cela nous nous fixons \( \epsilon>0\) et \( N\geq 1\) tel que \( | R_n |<\epsilon\) dès que \( n\geq N\). Alors pour tout \( | z |<1\) nous avons
    \begin{subequations}
        \begin{align}
            | f(z)-S |&\leq | z-1 |\left( \sum_{n=0}^N| R_n | \underbrace{|z^n |}_{\leq 1} +\sum_{n=N+1}^{\infty}\underbrace{| R_n |}_{\leq \epsilon} |z^n | \right)\\
            &\leq | z-1 |\sum_{n=0}^N| R_n |+\epsilon\frac{ | z-1 | }{ 1-| z | }
        \end{align}
    \end{subequations}
    où nous avons utilisé la somme de la série géométrique \eqref{EqASYTiCK} et l'égalité \( | z^n |=| z |^n\). Avant de nous particulariser à \( z\in\Delta_{\theta_0}\) nous devons anticiper un problème au dénominateur en multipliant par le binôme conjugué :
    \begin{equation}
        \frac{ | z-1 | }{ 1-| z | }=\frac{ | z-1 |(1+| z |) }{ 1-| z |^2 }.
    \end{equation}
    C'est maintenant que nous nous particularisons à \( z\in\Delta_{\theta_0}\) en posant \( z=\rho e^{i\varphi}\) et en remarquant que \( | z |^2=1-2\rho\cos(\varphi)+\rho^2\). Nous avons le calcul suivant :
    \begin{subequations}
        \begin{align}
            \frac{ | z-1 | }{ 1-| z | }&=\frac{ \rho(1+| z |) }{ 2\rho\cos(\varphi)-\rho^2 }\\
            &=\frac{ 1+| z | }{ 2\cos(\varphi)-\rho}\\
            &\leq\frac{ 2 }{ 2\cos(\varphi)-\rho }\\
            &\leq\frac{ 2 }{ 2\cos(\varphi)-\cos(\theta_0) }\\
            &\leq\frac{ 2 }{ 2\cos(\theta_0)-\cos(\theta_0) }\\
            &=\frac{ 2 }{ \cos(\theta_0) }.
        \end{align}
    \end{subequations}
    Quelques justifications.
    \begin{itemize}
        \item Vu que nous avons dans l'idée de faire \( \rho\to 0\) nous supposons que \( \rho<\cos(\theta_0)\).
        \item Nous avons \( \cos(\varphi)>\cos(\theta_0)\) parce que \( z\) est dans \( \Delta_{\theta_0}\).
    \end{itemize}
    Nous avons donc, pour tout \( z\in\Delta_{\theta_0}\) que
    \begin{equation}
        | f(z)-S |\leq | z-1 |\sum_{n=0}^N| R_n |+\epsilon\frac{ 2 }{ \cos(\theta_0) }.
    \end{equation}
    Il suffit de prendre \( \rho\) assez petit pour que
    \begin{equation}
        | z-1 |\sum_{n=0}^N| R_n |<\epsilon
    \end{equation}
    et nous avons
    \begin{equation}
        | f(z)-S |\leq \epsilon\left( 1+\frac{ 2 }{ \cos(\theta_0) } \right).
    \end{equation}
    Nous avons donc bien \( \lim_{\substack{z\to 1\\z\in\Delta_0}}f(z)=S\), comme nous le voulions.
\end{proof}

La réciproque du théorème d'Abel angulaire est que si \( f(z)=\sum_na_nz^n\) sur \( B(0,1)\) se prolonge par continuité en \( z=1\) alors cette prolongation se fait par \( f(1)=\sum_na_n\). Cela est faux comme le montre l'exemple suivant.

\begin{example}
    Nous considérons la série entière \( \sum_{n=0}^{\infty}(-1)^nz^n\) qui converge\footnote{C'est la série géométrique de raison \( -z\).} vers
    \begin{equation}
        f(z)=\frac{1}{ 1+z }
    \end{equation}
    sur \( B(0,1)\). De plus nous avons
    \begin{equation}
        \lim_{\substack{z\to 1\\    | z |<1}}\frac{1}{ 1+z }=\frac{ 1 }{2}.
    \end{equation}
    Donc la fonction converge bien vers quelque chose lorsque \( z\) tend vers \( 1\). La fonction \( f\) se prolonge par continuité en \( 1\). Pourtant la série es coefficients \( \sum_n(-1)^n\) ne converge pas.
\end{example}

Le théorème suivant donne une espèce d'inverse au théorème d'Abel angulaire. En effet il dit que si la série converge  en allant vers \( 1\) le long de l'axe réel, alors ça converge vers la somme des coefficients. Il faut cependant une hypothèse en plus sur les \( a_n\).
\begin{theorem}[Théorème taubérien faible\cite{KXjFWKA}]
    Soit \( \sum_na_nz^n\) une série entière de rayon de convergence \( 1\) et de somme \( f\). Nous supposons
    \begin{enumerate}
        \item
            Il existe \( S\in \eC\) tel que \( \lim_{\substack{x\to 1\\x\in\mathopen] -1 , 1 \mathclose[}}f(x)=S\).
            \item
                \( \lim_{n\to \infty} na_n=0\).
    \end{enumerate}
    Alors la série \( \sum_{n=0}^{\infty}a_n\) converge et vaut \( S\).
\end{theorem}
\index{théorème!taubérien faible}

\begin{proof}
    Nous notons \( S_n=\sum_{k=0}a_k\) et \( M=\sup_{k\geq 1}k| a_k |\), qui est fini par hypothèse. Pour \( x\in \mathopen] 0 , 1 \mathclose[\) et \( n\geq 0\) nous avons
    \begin{equation}
        S_n-f(x)=\sum_{k=1}^na_k-\sum_{k=1}^na_kx^k-\sum_{k=n+1}^{\infty}a_kx^k=\sum_{k=1}^na_k(1-x^k)-\sum_{k=n+1}^{\infty}a_kx^k.
    \end{equation}
    Nous utilisons la série géométrique sous la forme \( 1-x^k=(1-x)\sum_{i=0}^nx^i\) pour écrire
    \begin{subequations}
        \begin{align}
            S_n-f(x)&=\sum_{k=1}^na_k(1-x)\underbrace{\sum_{i=0}^{k-1}x^i}_{\leq k}-\sum_{k=n+1}^{\infty}a_kx^k\\
            &\leq\sum_{k=1}^nka_k(1-x)-\sum_{k=n+1}^{\infty}a_kx^k,
        \end{align}
    \end{subequations}
    donc en passant à la norme
    \begin{subequations}
        \begin{align}
            \big| S_n-f(x) \big|&\leq (1-x)Mn+\sum_{k=n+1}| a_k |x^k\\
            &\leq (1-x)Mn+\sum_{k=n+1}^{\infty}\underbrace{\frac{ k }{ n }| a_k |}_{\leq M/n}x^k\\
            &\leq (1-x)Mn+\frac{ M }{ n }\sum_{k=n+1}^{\infty}x^k\\
            &\leq (1-x)Mn+\frac{ M }{ n }\frac{1}{ 1-x }.
        \end{align}
    \end{subequations}
    Ce que nous cherchons à étudier est le comportement \( x\to 1\) et montrer que \( S_n\to S\), ce qui nous incite à calculer \( | S_n-f(1-\frac{ \epsilon }{n  }) |\) avec \( 0<\epsilon<1\) :
    \begin{equation}
        \big| S_n-f\big( 1-\frac{ \epsilon }{ n } \big) \big|\leq \epsilon M+\epsilon.
    \end{equation}
    Nous choisissons \( N_1\) tel que \( \frac{ M }{ n }\leq \epsilon^2\) dès que \( n\geq N_1\). En sus nous savons que
    \begin{equation}
        \lim_{\epsilon\to 0}f(1-\epsilon)=S.
    \end{equation}
    Nous choisissons \( N_2\) de telle sorte à avoir
    \begin{equation}
        \left| f\left( 1-\frac{ \epsilon }{ n } \right)-S \right| <\epsilon,
    \end{equation}
    et en prenant \( n\geq\max(N_1,N_2)\) nous avons
    \begin{equation}
        | S_n-S |\leq \left| S_n-f\left( 1-\frac{ \epsilon }{ n } \right) \right| +\left| f\left( 1-\frac{ \epsilon }{ n } \right)-S \right|  \leq \epsilon M+2\epsilon.
    \end{equation}
    Il suffit de choisir \( \epsilon\) suffisamment petit (en particulier pour que \( \epsilon M\) soit petit) pour montrer que \( | S_n-S |\) est borné par un nombre arbitrairement petit.
\end{proof}



%---------------------------------------------------------------------------------------------------------------------------
					\subsection{Passage à la limite sous le signe intégral}
%---------------------------------------------------------------------------------------------------------------------------

Un autre résultat très important pour l'étude de l'intégrabilité est le théorème de la \defe{convergence dominée de Lebesgue}{}:
\begin{theorem}
	Soit $E\subset \eR^n$ un ensemble mesurable et $\{ f_k \}$, une suite de fonctions intégrables sur $E$ qui converge simplement vers une fonction $f\colon E\to \overline{ \eR }$. Supposons qu'il existe une fonction $g$ intégrable sur $E$ telle que pour tout $k$,
\begin{equation}
	| f(x) |\leq g(x)
\end{equation}
pour tout $x\in E$. Alors $f$ est intégrable sur $E$ et
\begin{equation}
	\int_Ef=\lim_{k\to\infty}\int_Ef_k.
\end{equation}
\end{theorem}

%---------------------------------------------------------------------------------------------------------------------------
					\subsection{Intégrale en dimension un}
%---------------------------------------------------------------------------------------------------------------------------

\begin{proposition}[Critère de comparaison]
Soit $f$ mesurable sur $]a,\infty[$ et bornée sur tout $]a,b]$, et supposons qu'il existe un $X_0\geq a$, tel que sur $]X_0,\infty[$,
\begin{equation}
	| f(x) |\leq g(x)
\end{equation}
où $g(x)$ est intégrable. Alors $f(x)$ est intégrable sur $]a,\infty[$.
\end{proposition}

\begin{corollary}[Critère d'équivalence]
Soient $f$ et $g$ des fonctions mesurables et positives ou nulles sur $]a,\infty[$, bornées sur tout $]a,b]$, telles que
\begin{equation}
	\lim_{x\to\infty}\frac{ f(x) }{ g(x) }=L
\end{equation}
existe dans $\bar\eR$.
\begin{enumerate}
\item Si $L\neq\infty$ et $\int_{a}^{\infty}g(x)$ existe, alors $\int_a^{\infty}f(x)dx$ existe,
\item Si $L\neq 0$ et si $\int_a^{\infty}f(x)dx$ existe, alors $\int_a^{\infty}g(x)dx$ existe,
\end{enumerate}
\end{corollary}

\begin{corollary}[Critère des fonctions test]			\label{CorCritFonsTest}
Soit $f(x)$ une fonction mesurable et positive ou nulle sur $]a,\infty[$ et bornée pour tout $]a,b]$. Nous posons
\begin{equation}
	L(\alpha)=\lim_{x\to\infty}x^{\alpha}f(x),
\end{equation}
et nous supposons qu'elle existe.
\begin{enumerate}
\item Si il existe $\alpha>1$ tel que $L(\alpha)\neq\infty$, alors $\int_a^{\infty}f(x)dx$ existe,
\item Si il existe $\alpha\leq1$ et $L(\alpha)\neq 0$, alors $\int_a^{\infty}f(x)dx$ n'existe pas.
\end{enumerate}
\end{corollary}

\begin{corollary}		\label{CorAlphaLCasInteabf}
	Soit $f\colon ]a,b]\to \eR$ une fonction mesurable, positive ou nulle, et bornée sur $[a+\epsilon,b]$ $\forall\epsilon>0$. Si $\lim_{x\to a}(x-a)^{\alpha}f(x)=L$ existe, alors
	\begin{enumerate}
		\item Si $\alpha<1$ et $L\neq\infty$, alors $\int_a^bf(x)dx$ existe,
		\item Si $\alpha\geq 1$ et $L\neq 0$, alors $\int_a^bf(x)dx$ n'existe pas.
	\end{enumerate}
\end{corollary}

%---------------------------------------------------------------------------------------------------------------------------
					\subsection{Intégrales convergentes}
%---------------------------------------------------------------------------------------------------------------------------

\begin{definition}
    Soit $f$, une fonction mesurable sur $[a,\infty[$, bornée sur tout intervalle $[a,b]$. On dit que l'intégrale
    \begin{equation}
        \int_a^{\infty}f(x)dx
    \end{equation}
    \defe{converge}{intégrale!convergente} si la limite
    \begin{equation}		\label{EqDEfConvergeZeroInftX}
        \lim_{X\to\infty}\int_a^{X}f
    \end{equation}
    existe et est finie.
\end{definition}
