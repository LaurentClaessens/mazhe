% This is part of Mes notes de mathématique
% Copyright (c) 2008-2019, 2021
%   Laurent Claessens, Carlotta Donadello
% See the file fdl-1.3.txt for copying conditions.

%+++++++++++++++++++++++++++++++++++++++++++++++++++++++++++++++++++++++++++++++++++++++++++++++++++++++++++++++++++++++++++
\section{Fonctions}		\label{Sect_fonctions}
%+++++++++++++++++++++++++++++++++++++++++++++++++++++++++++++++++++++++++++++++++++++++++++++++++++++++++++++++++++++++++++

Soient $(E,\| . \|_E)$ et $(F,\| . \|_F)$ deux espaces vectoriels normés, et une fonction $f$ de $E$ dans $F$. Il est maintenant facile de définir les notions de limites et de continuité pour de telles fonctions en copiant les définitions données pour les fonctions de $\eR$ dans $\eR$ en changeant simplement les valeurs absolues par les normes sur $E$ et $F$.

La proposition suivante explicite la définition \ref{DefYNVoWBx} dans le cas où la topologie est donnée par des boules.
\begin{proposition}[Caractérisation de la limite]\label{PropHOCWooSzrMjl}
    Soient des espaces vectoriels normés.  Soit $f\colon E\to F$ une fonction de domaine \( \Domaine(f)\subset E\) et soit $a$ un point d'accumulation de $\Domaine(f)$.
    \begin{enumerate}
        \item
            Si \( F\) est séparé\footnote{C'est le cas en dimension finie et en particulier pour \( \eR^n\). En dimension infinie, il faut être très prudent.} et si \( f\) admet une limite en \( a\), alors cette limite est unique.
        \item       \label{ITEMooSHKNooStKGKH}
    La fonction \( f\) admet une limite en $a\in E$ si et seulement si il existe un élément $\ell\in F$ tel que pour tout $\varepsilon>0$, il existe un $\delta>0$ tel que pour tout $x\in D=\Domaine(f)$,
    \begin{equation}        \label{EqDefLimzxmasubV}
		0<\| x-a \|_E<\delta\,\Rightarrow\,\| f(x)-\ell \|_F<\varepsilon.
	\end{equation}
    \end{enumerate}
	Si la limite existe et est unique, nous écrivons $\lim_{x\to a} f(x)=\ell$ et nous disons que $\ell$ est la \defe{limite}{limite} de $f$ lorsque $x$ tend vers $a$.
\end{proposition}

\begin{proof}
    L'unicité est la proposition \ref{PropFObayrf}.

    \begin{subproof}
    \item[\( \Rightarrow\)]
        La définition \ref{DefYNVoWBx} nous assure de l'existence d'un élément \( \ell\) tel que pour tout voisinage \( S\) de \( \ell\), il existe un ouvert \( U\) autour de \( a\) tel que \( f\big( U\cap D\setminus\{ a \} \big)\subset S\).

        Soit \( \epsilon>0\). Nous posons \( S=B(\ell,\epsilon)\). Il existe un voisinage \( U\) de \( a\) tel que \( f\big( U\cap D\setminus\{ a \} \big)\subset B(\ell,\epsilon)\). Vu que \( S\) est un voisinage de \( a\), il contient une boule centrée en \( a\) (c'est dans la définition \ref{ThoORdLYUu} de la topologie métrique). Soit donc \( \delta>0\) tel que \( B(a,\delta)\subset U\).

        Un élément de \( D\) qui est dans \( B(0,\delta)\setminus \{ a \}\) est un élément de \( D\) qui vérifie \( 0<\| x-a \|<\delta\). Nous avons donc, pour \( x\in D\) que
        \begin{equation}
            0<\| x-a \|<\delta\Rightarrow \| f(x)-\ell \|<\epsilon.
        \end{equation}
    \item[\( \Leftarrow\)] C'est le même raisonnement.
    \end{subproof}
\end{proof}


\begin{remark}
    Le fait que nous limitions la formule \eqref{EqDefLimzxmasubV} aux \( x\) dans le domaine de \( f\) n'est pas anodin. Considérons la fonction \( f(x)=\sqrt{x^2-4}\), de domaine \( | x |\geq 2\). Nous avons
    \begin{equation}
        \lim_{x\to 2} \sqrt{x^2-4}=0.
    \end{equation}
    Nous ne pouvons pas dire que cette limite n'existe pas en justifiant que la limite à gauche n'existe pas. Les points \( x<2\) sont hors du domaine de \( f\) et ne comptent dons pas dans l'appréciation de l'existence de la limite.

    Vous verrez plus tard que ceci provient de la \wikipedia{fr}{Topologie_induite}{topologie induite} de \( \eR\) sur l'ensemble \( \mathopen[ 2 , \infty [\).
\end{remark}

%+++++++++++++++++++++++++++++++++++++++++++++++++++++++++++++++++++++++++++++++++++++++++++++++++++++++++++++++++++++++++++
\section{Sous espaces caractéristiques}
%+++++++++++++++++++++++++++++++++++++++++++++++++++++++++++++++++++++++++++++++++++++++++++++++++++++++++++++++++++++++++++

% TODO : lire le blog de Pierre Bernard; en particulier celle-ci : http://allken-bernard.org/pierre/weblog/?p=2299

Lorsqu'un opérateur n'est pas diagonalisable, les valeurs propres jouent quand même un rôle important.

\begin{definition}  \label{DefFBNIooCGbIix}
    Soit \( E\) un \( \eK\)-espace vectoriel  \( f\in\End(E)\). Pour \( \lambda\in \eK\) nous définissons
    \begin{equation}
        F_{\lambda}(f)=\{ v\in E\tq (f-\lambda\mtu)^nv=0, n\in\eN \}
    \end{equation}
    et nous appelons cet ensemble un \defe{sous-espace caractéristique}{sous-espace!caractéristique} de \( f\).
\end{definition}
L'espace \( F_{\lambda}(f)\) est l'ensemble de nilpotence de l'opérateur \( f-\lambda\mtu\) et

\begin{lemma}   \label{LemBLPooHMAoyJ}
    L'ensemble \( F_{\lambda}(f)\) est non vide si et seulement si \( \lambda\) est une valeur propre de \( f\). L'espace \( F_{\lambda}(f)\) est invariant sous \( f\).
\end{lemma}

\begin{proof}
    Si \( F_{\lambda}(f)\) est non vide, nous considérons \( v\in F_{\lambda}(f)\) et \( n\) le plus petit entier non nul tel que \( (f-\lambda)^nv=0\). Alors \( (f-\lambda)^{n-1}v\) est un vecteur propre de \( f\) pour la valeur propre \( \lambda\). Inversement si \( v\) est un vecteur propre de \( f\) pour la valeur propre \( \lambda\), alors \( v\in F_{\lambda}(f)\).

    En ce qui concerne l'invariance, remarquons que \( f\) commute avec \( f-\lambda\mtu\). Si \( x\in F_{\lambda}(f)\) il existe \( n\) tel que \( (f-\lambda\mtu)^nx=0\). Nous avons aussi
    \begin{equation}
        (f-\lambda\mtu)^nf(x)=f\big( (f-\lambda\mtu)^nx \big)=0,
    \end{equation}
    par conséquent \( f(x)\in F_{\lambda}(f)\).
\end{proof}

\begin{remark}  \label{RemBOGooCLMwyb}
    Toute matrice sur \( \eC\) n'est pas diagonalisable : nous en avons déjà donné un exemple simple en~\ref{ExBRXUooIlUnSx}. Nous en voyons maintenant un moins simple. Considérons en effet l'endomorphisme \( f\) donné par la matrice
    \begin{equation}
        \begin{pmatrix}
            a&    \alpha    &   \beta    \\
            0    &   a    &   \gamma    \\
            0    &   0    &   b
        \end{pmatrix}
    \end{equation}
    où \( a\neq b\), \( \alpha\neq 0\), \( \beta\) et \( \gamma\) sont des nombres complexes quelconques.
    Son polynôme caractéristique est
    \begin{equation}
        \chi_f(\lambda)=(a-\lambda)^2(b-\lambda),
    \end{equation}
    et les valeurs propres sont donc \( a\) et \( b\). Nous trouvons les vecteurs propres pour la valeur \( a\) en résolvant
    \begin{equation}
        \begin{pmatrix}
            a    &   \alpha    &   \beta    \\
            0    &   a    &   \gamma    \\
            0    &   0    &   b
        \end{pmatrix}\begin{pmatrix}
            x    \\
            y    \\
            z
        \end{pmatrix}=\begin{pmatrix}
            ax    \\
            ay    \\
            az
        \end{pmatrix}.
    \end{equation}
    L'espace propre \( E_a(f)\) est réduit à une seule dimension générée par \( (1,0,0)\). De la même façon l'espace propre correspondant à la valeur propre \( b\) est donné par
    \begin{equation}
        \begin{pmatrix}
            \frac{1}{ b-a }\left( \beta+\frac{ \alpha\gamma }{ b-a } \right)    \\
            \frac{ \gamma }{ b-a }    \\
            1
        \end{pmatrix}.
    \end{equation}
    Il n'y a donc pas trois vecteurs propres linéairement indépendants, et l'opérateur \( f\) n'est pas diagonalisable.

    Par contre nous pouvons voir que
    \begin{equation}
        (f-\alpha\mtu)^2\begin{pmatrix}
             0   \\
            1    \\
            0
        \end{pmatrix}=
        \begin{pmatrix}
            a    &   \alpha    &   \beta    \\
            0    &   a    &   \gamma    \\
            0    &   0    &   b
        \end{pmatrix}
        \begin{pmatrix}
            \alpha    \\
            0    \\
            0
        \end{pmatrix}-\begin{pmatrix}
            a\alpha    \\
            0    \\
            0
        \end{pmatrix}=\begin{pmatrix}
            0    \\
            0    \\
            0
        \end{pmatrix},
    \end{equation}
    de telle sorte que le vecteur \( (0,1,0)\) soit également dans l'espace caractéristique \( F_a(f)\).

    Dans cet exemple, la multiplicité algébrique de la racine \( a\) du polynôme caractéristique vaut \( 2\) tandis que sa multiplicité géométrique vaut seulement \( 1\).
\end{remark}

%---------------------------------------------------------------------------------------------------------------------------
\subsection{Théorèmes de décomposition}
%---------------------------------------------------------------------------------------------------------------------------

%TODO : Je crois qu'on peut remplacer l'hypothèse de corps algébriquement clos par le polynôme caractéristique scindé.
\begin{theorem}[Théorème spectral, décomposition primaire]\index{théorème!spectral}     \label{ThoSpectraluRMLok}
    Soit \( E\) un espace vectoriel de dimension finie sur le corps algébriquement clos \( \eK\) et \( f\in\End(E)\). Alors
    \begin{equation}    \label{EqCTFHooBSGhYK}
        E=F_{\lambda_1}(f)\oplus\ldots\oplus F_{\lambda_k}(f)
    \end{equation}
    où la somme est sur les valeurs propres distinctes de \( f\).

    Les projecteurs sur les espaces caractéristique forment un système complet et orthogonal.
\end{theorem}
\index{décomposition!primaire}
\index{décomposition!spectrale}
\index{décomposition!sous-espaces caractéristiques}

\begin{proof}
    Soit \( P\) le polynôme caractéristique de \( f\) et une décomposition
    \begin{equation}
        P=(f-\lambda_1)^{\alpha_1}\ldots(f-\lambda_r)^{\alpha_r}
    \end{equation}
    en facteurs irréductibles. Par le théorème des noyaux (\ref{ThoDecompNoyayzzMWod}) nous avons
    \begin{equation}        \label{EqDeFVSaYv}
        E=\ker(f-\lambda_1)^{\alpha_1}\oplus\ldots\oplus\ker(f-\lambda_r)^{\alpha_r}.
    \end{equation}
    Les projecteurs sont des polynômes en \( f\) et forment un système orthogonal. Il nous reste à prouver que \( \ker(f-\lambda_i)^{\alpha_i}=F_{\lambda_i}(f)\). L'inclusion
    \begin{equation}    \label{EqzmNxPi}
        \ker(f-\lambda_i)^{\alpha_i}\subset F_{\lambda_i}(f)
    \end{equation}
    est évidente. Nous devons montrer l'inclusion inverse. Prouvons que la somme des \( F_{\lambda_i}(f)\) est directe. Si \( v\in F_{\lambda_i}(f)\cap F_{\lambda_j}(f)\), alors il existe \( v_1=(f-\lambda_i)^nv\neq 0\) avec \( (f-\lambda_i)v_1=0\). Étant donné que \( (f-\lambda_i)\) commute avec \( (f-\lambda_j)\), ce \( v_1\) est encore dans \( F_{\lambda_j}(f)\) et par conséquent il existe \( w=(f-\lambda_j)^mv_1\) non nul tel que
    \begin{subequations}
        \begin{numcases}{}
            (f-\lambda_i)w=0\\
            (f-\lambda_j)w=0.
        \end{numcases}
    \end{subequations}
    Ce \( w\) serait donc un vecteur propre simultané pour les valeurs propres \( \lambda_i\) et \( \lambda_j\), ce qui est impossible parce que les espaces propres sont linéairement indépendants. Les espaces \( F_{\lambda_i}\) sont donc en somme directe et
    \begin{equation}
        \sum_i\dim F_{\lambda_i}(f)\leq \dim E.
    \end{equation}
    En tenant compte de l'inclusion \eqref{EqzmNxPi} nous avons même
    \begin{equation}
        \dim E=\sum_i\dim\ker(f-\lambda_i)^{\alpha_i}\leq\sum_i \dim F_{\lambda_i}(f)\leq \dim E.
    \end{equation}
    Par conséquent nous avons \( \dim\ker(f-\lambda_i)^{\alpha_i}=\dim F_{\lambda_i}(f)\) et l'égalité des deux espaces.
\end{proof}


\begin{probleme}
    Dans le cas où le corps n'est pas algébriquement clos, il paraît qu'il faut remplacer «diagonalisable» par «semi-simple».

    Si vous connaissez un énoncé précis et une démonstration, écrivez-moi.
\end{probleme}
%TODO : peut-être qu'il y a la réponse dans http://www.math.jussieu.fr/~romagny/agreg/dvt/endom_semi_simples.pdf

Si l'espace vectoriel est sur un corps algébriquement clos, alors les endomorphismes semi-simples\footnote{Définition~\ref{DEFooBOHVooSOopJN}.} sont les endomorphismes diagonaux.


%TODO : Je crois qu'on peut remplacer l'hypothèse de corps algébriquement clos par le polynôme caractéristique scindé.
\begin{theorem}[Décomposition de Dunford] \label{ThoRURcpW}
    Soit \( E\) un espace vectoriel sur le corps algébriquement clos \( \eK\) et \( u\in\End(E)\) un endomorphisme de \( E\).

    \begin{enumerate}
        \item

            L'endomorphisme \( u\) se décompose de façon unique sous la forme
            \begin{equation}
                u=s+n
            \end{equation}
            où \( s\) est diagonalisable, \( n\) est nilpotent et \( [s,n]=0\).
        \item
            Les endomorphismes \( s\) et \( n\) sont des polynômes en \( u\) et commutent avec \( u\).
        \item   \label{ItemThoRURcpWiii}
            Les parties \( s\) et \( n\) sont données par
            \begin{subequations}
                \begin{align}
                    s&=\sum_i\lambda_ip_i\\
                    n&=\sum_i(s-\lambda_i\mtu)p_i
                \end{align}
            \end{subequations}
            où les sommes sont sur les valeurs propres distinctes\footnote{C'est-à-dire sur les sous-espaces caractéristiques.} de \( f\) et où \( p_i\colon E\to F_{\lambda_i}(u)\) est la projection de \( E\) sur \( F_{\lambda_i}(u)\).
    \end{enumerate}
\end{theorem}
\index{décomposition!Dunford}
\index{Dunford!décomposition}
\index{réduction!d'endomorphisme}
\index{endomorphisme!sous-espace stable}
\index{polynôme!d'endomorphisme!décomposition de Dunford}
\index{endomorphisme!diagonalisable!Dunford}
\index{endomorphisme!nilpotent!Dunford}
%TODO : comprendre comment on calcule des exponentielles de matrices avec Dunford.

\begin{proof}
    Le théorème spectral~\ref{ThoSpectraluRMLok} nous indique que
    \begin{equation}
        E=\bigoplus_iF_{\lambda_i}(f).
    \end{equation}
    Nous considérons l'endomorphisme \( s\) de \( E\) qui consiste à dilater d'un facteur \( \lambda\) l'espace caractéristique \( F_{\lambda}(f)\) :
    \begin{equation}
        s=\sum_i\lambda_ip_i
    \end{equation}
    où \( p_i\colon E\to F_{\lambda_i}(u)\) est la projection de \( E\) sur \( F_{\lambda_i}(u)\).

    Nous allons prouver que \( [s,f]=0\) et \( n=f-s\) est nilpotent. Cela impliquera que \( [s,n]=0\).

    Si \( x\in F_{\lambda}(f)\), alors nous avons \( sf(x)=\lambda f(x)\) parce que \( f(x)\in F_{\lambda}(f)\) tandis que \( fs(x)=f(\lambda x)=\lambda f(x)\). Par conséquent \( f\) commute avec \( s\).

    Pour montrer que \( f-s\) est nilpotent, nous en considérons la restriction
    \begin{equation}
        f-s\colon F_{\lambda}(f)\to F_{\lambda}(f).
    \end{equation}
    Cet opérateur est égal à \( f-\lambda\mtu\) et est par conséquent nilpotent.

    Prouvons à présent l'unicité. Soit \( u=s'+n'\) une autre décomposition qui satisfait aux conditions : \( s'\) est diagonalisable, \( n'\) est nilpotent et \( [n',s']=0\). Commençons par prouver que \( s'\) et \( n'\) commutent avec \( u\). En multipliant \( u=s'+n'\) par \( s'\) nous avons
    \begin{equation}
        s'u=s'^2+s'n'=s'^2+n's'=(s'+n')s'=us',
    \end{equation}
    par conséquent \( [u,s']=0\). Nous faisons la même chose avec \( n'\) pour trouver \( [u,n']=0\). Notons que pour obtenir ce résultat nous avons utilisé le fait que \( n'\) et \( s'\) commutent, mais pas leur propriétés de nilpotence et de diagonalisibilité.


    Si \( s'+n'=s+n\) est une autre décomposition, \( s'\) et \( n'\) commutent avec \( u\), et par conséquent avec tous les polynômes en \( u\). Ils commutent en particulier avec \( n\) et \( s\). Les endomorphismes \( s\) et \( s'\) sont alors deux endomorphismes diagonalisables qui commutent. Par la proposition~\ref{PropGqhAMei}, ils sont simultanément diagonalisables. Dans la base de diagonalisation simultanée, la matrice de l'opérateur \( s'-s=n-n'\) est donc diagonale. Mais \( n-n'\) est également nilpotent, en effet si \( A\) et \( B\) sont deux opérateurs nilpotents,
    \begin{equation}
        (A+B)^n=\sum_{k=0}^n\binom{k}{n}A^kB^{n-k}.
    \end{equation}
    Si \( n\) est assez grand, au moins un parmi \( A^k\) ou \( B^{n-k}\) est nul.

    Maintenant que \( n-n'\) est diagonal et nilpotent, il est nul et \( n=n'\). Nous avons alors immédiatement aussi \( s=s'\).

\end{proof}

%---------------------------------------------------------------------------------------------------------------------------
\subsection{Diverses conséquences}
%---------------------------------------------------------------------------------------------------------------------------

\begin{theorem}
    Soit une matrice \( A\in \eM(n,\eC)\). La suite \( (A^kx)\) tend vers zéro pour tout \( x\) si et seulement si \( \rho(A)<1\) où \( \rho(A)\)\index{rayon!spectral} est le rayon spectral de $A$
\end{theorem}
\index{décomposition!Dunford!exponentielle de matrice}

\begin{proof}
    Dans le sens direct, il suffit de prendre comme \( x\), un vecteur propre de \( A\). Dans ce cas nous avons \( A^kx=\lambda^kx\). Mais \( \lambda^kx\) ne tend vers zéro que si \( \lambda<1\). Donc toutes les valeurs propres de \( A\) doivent être plus petites que \( 1\) et \( \rho(A)<1\).

    Pour l'autre sens nous utilisons la décomposition de Dunford (théorème~\ref{ThoRURcpW}) : il existe une matrice inversible \( P\) telle que
    \begin{equation}
        A=P^{-1}(D+N)P
    \end{equation}
    où \( D\) est diagonale, \( N\) est nilpotente et \( [D,N]=0\). Étant donné que \( D+N\) est triangulaire, son polynôme caractéristique est
    \begin{equation}
        \chi_{D+N}(\lambda)=\prod_i D_{ii}-\lambda.
    \end{equation}
    Par similitude, c'est le même polynôme caractéristique que celui de \( A\) et nous savons alors que la diagonale de \( D\) contient les valeurs propres de \( A\).

    Par ailleurs nous avons
    \begin{subequations}
        \begin{align}
            A^k&=P^{-1}(D+N)^kP\\
            &=P^{-1}\sum_{j=0}^k{j\choose k}D^{j-k}N^jP\\
            &=P^{-1}\sum_{j=0}^{n-1}{j\choose k}D^{j-k}N^jP
        \end{align}
    \end{subequations}
    où nous avons utilisé le fait que \( D\) et \( N\) commutent ainsi que \( N^{n-1}=0\) parce que \( N\) est nilpotente. Nous utilisons la norme matricielle usuelle, pour laquelle \( \| D \|=\rho(D)=\rho(A)\). Nous avons alors
    \begin{equation}
        \| (D+N)^k \|\leq \sum_{j=0}^k{j\choose k}\rho(D)^{k-j}\| N \|^j.
    \end{equation}
    Du coup si \( \rho(D)<1\) alors \( \| (D+N)^k \|\to 0\) (et c'est même un si et seulement si).
\end{proof}

Une application de la décomposition de Jordan est l'existence d'un logarithme pour les matrices. La proposition suivante va d'une certaine manière donner un logarithme pour les matrices inversibles complexes. Dans le cas des matrices réelles \( m\) telles que \( \| m-\mtu \|<1\), nous donnerons au lemme~\ref{LemQZIQxaB} une formule pour le logarithme sous forme d'une série; ce logarithme sera réel.

%---------------------------------------------------------------------------------------------------------------------------
\subsection{Valeurs singulières}
%---------------------------------------------------------------------------------------------------------------------------

\begin{definition}
    Soit \( M\) une matrice \( m\times n\) sur \( \eK\) (\( \eK\) est \( \eR\) ou \( \eC\)). Un nombre réel \( \sigma\) est une \defe{valeur singulière}{valeur!singulière} de \( M\) si il existent des vecteurs unitaires \( u\in \eK^m\), \( v\in \eK^n\) tels que
    \begin{subequations}
        \begin{align}
            Mv&=\sigma u\\
            M^*u&=\sigma v.
        \end{align}
    \end{subequations}
\end{definition}

\begin{theorem}[Décomposition en valeurs singulières]
    Soit \( M\in \eM(m\times n,\eK)\) où \( \eK=\eR,\eC\). Alors \( M\) se décompose en
    \begin{equation}
        M=ADB
    \end{equation}
    où
    il existe deux matrices unitaires \( A\in \gU(m\times m)\), \( B\in \gU(n\times n)\) et une matrice (pseudo)diagonale \( D\in \eM(m\times n)\) tels que
    \begin{enumerate}
        \item
            \( A\in\gU(m\times m)\), \( B\in\gU(n\times n)\) sont deux matrices unitaires,
        \item
            \( D\) est (pseudo)diagonale,
        \item
            les éléments diagonaux de \( D\) sont les valeurs singulières de \( M\),
        \item
            le nombre d'éléments non nuls sur la diagonale de \( D\) est le rang\footnote{Définition~\ref{DefALUAooSPcmyK}.} de \( M\).
    \end{enumerate}
\end{theorem}

\begin{corollary}
    Soit \( M\in \eM(n,\eC)\). Il existe un isomorphisme \( f\colon \eC^n\to \eC^n\) tel que \( fM\) soit autoadjoint.
\end{corollary}

\begin{proof}
    Si \( M=ADB\) est la décomposition de \( M\) en valeurs singulières, alors nous pouvons prendre \( f=\overline{ B }^tA^{-1}\) qui est une matrice inversible. Pour la vérification que ce \( f\) répond bien à la question, ne pas oublier que \( D\) est réelle, même si \( M\) ne l'est pas.
\end{proof}
