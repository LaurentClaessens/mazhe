% This is part of Mes notes de mathématique
% Copyright (c) 2011-2022
%   Laurent Claessens
% See the file fdl-1.3.txt for copying conditions.

Pour rappel, la notion de groupe est définie en \ref{DEFooBMUZooLAfbeM}.

%+++++++++++++++++++++++++++++++++++++++++++++++++++++++++++++++++++++++++++++++++++++++++++++++++++++++++++++++++++++++++++
\section{Groupes}
%+++++++++++++++++++++++++++++++++++++++++++++++++++++++++++++++++++++++++++++++++++++++++++++++++++++++++++++++++++++++++++

\begin{definition}[centralisateur\cite{Kropholler}]         \label{defGroupeCentre}
	Soient un groupe \( G\), un sous-groupe \( H\) et un élément \( h\in H\). Le \defe{centralisateur}{centralisateur} de \( h\) dans \( G\) est l'ensemble des éléments de \( G\) qui commutent avec \( h\) :
	\begin{equation}
		Z_G(h)=\{z\in G\tq hz=zh\}.
	\end{equation}
	Le centralisateur de \( H\) dans \( G\) est l'ensemble des éléments de \( G\) qui commutent avec tous les éléments de \( H\) :
	\begin{equation}
		Z_G(H)=\bigcap_{h\in H}Z_G(h).
	\end{equation}
	Le \defe{centre}{centre!d'un groupe} d'un groupe \( G\) est l'ensemble des éléments de \( G\) qui commutent avec tous les autres:
	\begin{equation}
		Z_G=Z_G(G)=\{ z\in G\tq gz=zg , \forall g\in G \}.
	\end{equation}
\end{definition}

\begin{definition}[normalisateur\cite{Kropholler}]          \label{DEFooZTSMooBislIy}
	Soient un groupe \( G\) et un sous-groupe \( H\). Le \defe{normalisateur}{normalisateur} de \( H\) dans \( G\) est
	\begin{equation}
		\mN_G(H)=\{ g\in G\tq gH=Hg \}.
	\end{equation}
\end{definition}

\begin{definition}[Sous-groupe normal]                      \label{DEFooNIIMooFkZgvX}
	Un sous-groupe \( N\) de \( G\) est \defe{normal}{normal!sous-groupe} ou \defe{distingué}{distingué!sous-groupe} si pour tout \( g\in G\) et pour tout \( n\in N\), \( gng^{-1}\in N\). Autrement dit lorsque \( gNg^{-1}\subset N\).

	Lorsque \( N\) est normal dans \( G\) il est parfois noté \( N\normal G\)\nomenclature[]{\(N \normal G\)}{Le sous-groupe \( N\) est normal dans \( G\)}.
\end{definition}

\begin{definition}      \label{DEFooUXXTooCCLmQe}
	Un sous-groupe \( H\) de \( G\) est un sous-groupe \defe{caractéristique}{sous-groupe!caractéristique}\index{caractéristique!sous-groupe} si \( \alpha(H)\subset H\) pour tout automorphisme\footnote{Automorphisme de groupe, définition \ref{DEFooBEHTooMeCOTX}.} \( \alpha\) de \( G\).
\end{definition}

\begin{lemma}[\cite{BIBooOHKHooAcewiw}]
	Si \( H\) est un sous-groupe caractéristique de \( G\), alors \( \alpha(H)=H\) pour tout automorphisme \( \alpha\) de \( G\).
\end{lemma}

\begin{proof}
	Si \( \alpha\) est un automorphisme de \( G\), alors \( \alpha^{-1}\) est encore un automorphisme de \( G\). En particulier \( \alpha^{-1}(H)\subset H\).

	Soit \( h\in H\). Nous devons prouver que \( h\in \alpha(H)\). Pour cela :
	\begin{equation}
		h=\alpha\big( \alpha^{-1}(h) \big)\in \alpha\big( \alpha^{-1}(H) \big)\subset\alpha(H).
	\end{equation}
\end{proof}

\begin{definition}[Groupe simple]                 \label{DefGroupeSimple}
	Un groupe est dit \defe{simple}{simple!groupe} si il est non trivial et si les seuls sous-groupes normaux qu'il admet sont lui-même et le sous-groupe réduit à l'élément neutre.
\end{definition}

\begin{definition}[Sous-groupe engendré]          \label{DefooRDRXooEhVxxu}
	Soit \( A\) une partie du groupe \( G\). Le sous-groupe \defe{engendré}{sous-groupe!engendré}\index{engendré!sous-groupe} par \( A\) est l'intersection de tous les sous-groupes de \( G\) contenant \( A\). Nous notons ce groupe \( \gr_G(A)\)\nomenclature[R]{\( \gr_G\)}{groupe engendré}.

	Lorsque \( A \) est fini (disons \( A = \{a_1, \dots, a_n\} \)), on note aussi le sous-groupe engendré \( \langle a_1, \dots, a_n \rangle \).
\end{definition}

\begin{normaltext}
	Un sous-groupe engendré n'est jamais vide parce qu'il contient toujours au moins le neutre (parce que c'est un sous-groupe). Si \( G\) est un groupe, le sous-groupe \( \gr_G(\emptyset)\) lui-même contient \( e\)\footnote{Demandez-vous si il est possible que \( \gr(\emptyset)\) contienne d'autres éléments que \( e\).}.
\end{normaltext}

\begin{normaltext}
	Dans de nombreux cas, le groupe «ambiant» \( G\) est entendu par le contexte et nous noterons \( \gr(A)\) au lieu de \( \gr_G(A)\).

	Si par exemple \( A\) est la matrice \( \begin{pmatrix}
		4 & 5 \\
		6 & 7
	\end{pmatrix}\), le groupe \( \gr(A)\) est à comprendre dans \( \GL(2,\eR)\). Il faudrait être fou pour avoir en tête un autre groupe que \( \GL(2,\eR)\) sans le préciser.

	D'ailleurs, connaissez-vous un groupe contenant la matrice \( A\) et n'étant pas un sous-groupe de \( \GL(2,\eC)\) ?
\end{normaltext}

\begin{lemma}
	Si \( G\) est un groupe et \( A\) une partie de \( G\), alors \( \gr(A)\) est un sous-groupe de \( G\).
\end{lemma}

Le sous-groupe engendré par \( A \) est le plus petit (pour l'inclusion) groupe de \( G\) contenant \( A\). Plus formellement, nous avons le résultat suivant :
\begin{lemma}
	Tout sous-groupe de \( G\) contenant \( A\) contient \( \gr(A)\).
\end{lemma}

\begin{proof}
	Si \( H\) est un sous-groupe de \( G\) contenant \( A\), alors \( \gr(A)\) est l'intersection de \( H\) avec tous les autres sous-groupes de \( G\) contenant \( A\). Il contient donc \( \gr(A)\).
\end{proof}

\begin{lemma}[\cite{BIBooERNQooTXQPvD}]   \label{LemFUIZooBZTCiy}
	Si \( A\) est une partie du groupe \( G\), alors le sous-groupe \( \gr(A)\) engendré\footnote{Définition~\ref{DefooRDRXooEhVxxu}.} par \( A\) est l'ensemble de tous les produits finis d'éléments de \( A\) et de \( A^{-1}\) (l'identité est le produit à zéro éléments).

	C'est-à-dire que tout élément de \( \gr(A)\) peut être écrit sous la forme\footnote{Les \( a_i\) négatifs correspondent aux inverses. Notons que si \( g\in A\), il n'y a pas de garanties que \( g^{-1}\) soit également dans \( A\).}
	\begin{equation}
		\prod_{i=1}^ng_i^{a_i}
	\end{equation}
	où \( a_i\in \eZ\) et \( g\colon \eN\to A\) n'est pas spécialement injective : il peut arriver que \( g_i=g_j\).
\end{lemma}

\begin{proof}
	Puisqu'un produit vide est égal à l'identité\footnote{Voir \ref{NORMooDBOFooQCwbOY}.}, le lemme est vrai (un peu trivialement) dans le cas où \( A=\emptyset\). À partir de maintenant, nous supposons que \( A\) est non vide.

	Nous nommons \( \gr(A)\) le groupe engendré par \( A\) et \( H\), l'ensemble
	\begin{equation}
		H=\{ g_1\ldots g_n\tq g_i\in A\cup A^{-1} \}.
	\end{equation}
	Nous commençons par prouver que \( H\) est un groupe.
	\begin{itemize}
		\item Puisque \( A\) est non vide, nous considérons \( a\in A\). Dans ce cas, \( e=aa^{-1}\in H\). Donc \( e\in H\).
		\item L'inverse de \( g_1\ldots g_n\) est \( g_n^{-1}\ldots g_1^{-1}\) qui est également dans \( H\).
		\item Le produit de \( g_1\ldots g_n\) par \( h_1\ldots h_n\), tous éléments de \( H\), est également dans \( H\)\footnote{Et c'est ici qu'on se rend compte que la décomposition n'est probablement que rarement unique.}.
	\end{itemize}
	Comme \( H\) est un groupe contenant \( A\), nous avons \( \gr(A)\subset H\) parce que \( \gr(A)\) est une intersection dont un des éléments est \( H\).

	Par ailleurs tout groupe contenant \( A\) doit contenir les inverses et les produits finis, donc \( H\subset \gr(A)\).

	Au final, \( H=\gr(A)\), ce qu'il fallait.
\end{proof}

\begin{lemma}       \label{LEMooCFTVooKvmyKN}
	Soit un groupe \( G\) et un sous-groupe \( H=\gr(h_1,\ldots, h_n)\). Si \( \alpha\in G\), alors
	\begin{equation}
		\alpha H\alpha^{-1}=\gr(\alpha h_1\alpha^{-1},\ldots, \alpha h_n\alpha^{-1}).
	\end{equation}
\end{lemma}

\begin{proof}
	Il s'agit d'une conséquence du lemme \ref{LemFUIZooBZTCiy}. Un élément de \( \gr(\alpha h_1\alpha^{-1},\ldots, \alpha h_n\alpha^{-1})\) est un produit d'éléments de \( G\) de la forme \( \alpha h_i\alpha^{-1}\) ou \( (\alpha h_j\alpha^{-1})^{-1}=\alpha h_j^{-1}\alpha^{-1}\). Or nous avons
	\begin{equation}
		\alpha h_i\alpha^{-1}\alpha h_j\alpha^{-1}=\alpha h_ih_j\alpha^{-1}\in \alpha H\alpha^{-1}.
	\end{equation}
	Donc
	\begin{equation}
		\gr(\alpha h_1\alpha^{-1},\ldots, \alpha h_n\alpha^{-1})\subset \alpha H\alpha^{-1}.
	\end{equation}
	L'inclusion dans l'autre sens est du même tonneau.
\end{proof}

\begin{definition}[Partie génératrice, groupe monogène]  \label{DEFooWMFVooLDqVxR}
	Soient un groupe \( G\), et une partie \( A\subset G\). Si \( \gr(A)=G\), alors nous disons que \( A\) est une \defe{partie génératrice}{partie génératrice} du groupe \( G\).

	Un groupe est \defe{monogène}{monogène} si il a une partie génératrice réduite à un seul élément.
\end{definition}

\begin{definition}[Groupe cyclique]     \label{DefHFJWooFxkzCF}
	Un élément \( a\in G\) est un \defe{générateur}{générateur} de \( G\) si tous les éléments de \( G\) s'écrivent sous la forme \( a^n\) pour un certain \( n\in\eZ\). Un groupe fini et monogène est dit \defe{cyclique}{groupe cyclique}.
\end{definition}

\begin{example}
	Soit le groupe \( \big( \eZ/10\eZ,+ \big)\). L'élément \( [2]_{10}\) n'est pas générateur parce que ses puissances\footnote{Attention aux notations; en général on écrit la loi de groupe de façon multiplicative et on parle des puissances d'un élément, mais ici on écrit la loi de groupe additivement, donc les «puissances» sont en réalité les multiples.} sont
	\begin{equation}
		\gr([2]_{10})=\{ [2]_{10},[4]_{10},[6]_{10},[8]_{10},[0]_{10} \}.
	\end{equation}
	Par contre l'élément \( [3]_{10}\) est générateur : ses puissances sont dans l'ordre
	\begin{equation}
		[3]_{10}, [6]_{10}, [9]_{10}, [2]_{10}, [5]_{10}, [8]_{10},[1]_{10},[4]_{10},[7]_{10},[0]_{10}.
	\end{equation}
\end{example}

Un exemple presque identique, mais un peu masqué sera l'exemple \ref{EXooOXAAooZMdDfP}.

%+++++++++++++++++++++++++++++++++++++++++++++++++++++++++++++++++++++++++++++++++++++++++++++++++++++++++++++++++++++++++++
\section{Sous-groupe normal}
%+++++++++++++++++++++++++++++++++++++++++++++++++++++++++++++++++++++++++++++++++++++++++++++++++++++++++++++++++++++++++++

\begin{proposition}\label{propGroupeNormal}
	Soit \( N\) un sous-groupe de \( G\). Les propriétés suivantes sont équivalentes :
	\begin{enumerate}
		\item       \label{ITEMooDYEUooOuKEqQ}
		      \( N\) est normal\footnote{Définition \ref{DEFooNIIMooFkZgvX}.} dans \( G\).
		\item       \label{ITEMooPYTEooZhvrUa}
		      \( N\) est une union de classes de conjugaison\footnote{Définition \ref{DEFooOLXPooWelsZV}.} de \( G\),
		\item       \label{ITEMooJWTLooBRmriQ}
		      \( gNg^{-1}\subseteq N\) pour tout \( g\in G\),
		\item       \label{ITEMooVRZIooAorhRY}
		      \( gNg^{-1}= N\) pour tout \( g\in G\),
		\item       \label{ITEMooJGUOooYshOZa}
		      \( gN=Ng\) pour tout \( g\in G\),
		\item       \label{ITEMooMRYRooZifCCe}
		      Le normalisateur\footnote{Définition \ref{DEFooZTSMooBislIy}.} de \( N\) est \( G\) : \( \mN_G(N)=G\).
	\end{enumerate}
\end{proposition}

\begin{proof}
	En plusieurs parties.
	\begin{subproof}
		\item[\ref{ITEMooDYEUooOuKEqQ} implique \ref{ITEMooJWTLooBRmriQ}]
		C'est la définition de sous-groupe normal.
		\item[\ref{ITEMooJWTLooBRmriQ} implique \ref{ITEMooVRZIooAorhRY}]
		Soit \( g\in G\). Nous avons \( gNg^{-1}\subset N\), mais aussi (en appliquant l'hypothèse à \( g^{-1}\)) \( g^{-1}Ng\subset N\). En combinant nous avons
		\begin{equation}
			N\subset g(g^{-1} Ng)g^{-1}\subset g Ng^{-1}.
		\end{equation}
		Nous avons l'inclusion dans les deux sens. Donc l'égalité.
		\item[\ref{ITEMooVRZIooAorhRY} implique \ref{ITEMooJGUOooYshOZa}]
		Soit \( g\in G\). Un élément général de \( gN\) est de la forme \( gn\) avec \( n\in N\). Nous devons trouver un \( n'\in N\) tel que \( gn=n'g\). En posant \( n'=gng^{-1}\) nous avons
		\begin{equation}
			n'=gng^{-1}\in gNg^{-1}\subset N.
		\end{equation}
		Il est immédiat de prouver que \( gn=n'g\). Cela prouve que \( gN\subset Ng\).

		Le même raisonnement donne \( Ng\subset gN\).
		\item[\ref{ITEMooJGUOooYshOZa} implique \ref{ITEMooJWTLooBRmriQ}]
		Un élément de \( g Ng^{-1}\) est \( a=gng^{-1}\) avec \( n\in N\). Nous devons prouver que \( a\in N\). Puisque \( gn\in gN\), par hypothèse il existe \( n'\) tel que \( gn=n'g\). En remplaçant dans la définition de \( a\),
		\begin{equation}
			a=gng^{-1}=n'gg^{-1}=n'\in N.
		\end{equation}
		\item[\ref{ITEMooJGUOooYshOZa} implique \ref{ITEMooPYTEooZhvrUa}]
		Pour chaque \( a\in G\) nous notons \( C_a\) la classe de conjugaison de \( a\) dans \( G\) :
		\begin{equation}        \label{EQooJIEVooCAshfe}
			C_a=\{ gag^{-1}\tq g\in G \}.
		\end{equation}
		Comme \( a\in C_a\) (prendre \( g=e\) dans \eqref{EQooJIEVooCAshfe}.) nous avons forcément
		\begin{equation}
			N\subset\bigcup_{n\in N}C_n.
		\end{equation}
		Prouvons maintenant l'inclusion inverse. Nous avons déjà prouvé que \ref{ITEMooJGUOooYshOZa} implique \ref{ITEMooJWTLooBRmriQ}. Donc si \( n\in N\), alors \( gng^{-1}\in N\). Nous avons alors
		\begin{equation}
			C_n=\{ gng^{-1}\tq g\in G \}\subset N.
		\end{equation}
		Donc il est vrai que \( N=\bigcup_{n\in N}C_n\).
		\item[\ref{ITEMooPYTEooZhvrUa} implique \ref{ITEMooDYEUooOuKEqQ}]
		Nous supposons que \( N\subset G\) est un sous-groupe de la forme
		\begin{equation}
			N=\bigcup_{a\in I}C_a
		\end{equation}
		où \( I\) est une partie de \( G\). Nous devons montrer que pour tout \( g\in G\) et pour tout \( n\in N\) nous avons \( gng^{-1}\in N\). Puisque \( n\in N\), il existe \( a\in I\) tel que \( n\in C_a\) et donc il existe \( k\in G\) tel que \( n=kak^{-1}\). Nous avons donc
		\begin{equation}
			gng^{-1}=g(kak^{-1})g^{-1}=(gk)a(gk)^{-1}\in C_a\subset N.
		\end{equation}

		\randomGender{Le lecteur attentif}{La lectrice attentive} aura remarqué l'utilisation de l'axiome du choix. La prudence l'incitera à ne pas le faire remarquer au jury.
		\item[\ref{ITEMooMRYRooZifCCe} si et seulement si \ref{ITEMooJGUOooYshOZa}]
		C'est la définition du normalisateur.
	\end{subproof}
\end{proof}

\begin{definition}
	Soit \( g\in G\) et \( n\in \eZ\). Nous définissons \( g^n\) par
	\begin{enumerate}
		\item
		      \( g^0=e\) et \( g^n=gg^{n-1}\) si \( n\) est positif.
		\item
		      si \( n<0\), nous posons \( g^n=(g^{-1})^{-n}\).
	\end{enumerate}
\end{definition}

L'ordre d'un groupe et l'ordre d'un élément d'un groupe sont deux choses différentes.

\begin{definition}[Ordre d'un groupe]    \label{DEFooKWBCooMlmpCP}
    Soit un groupe \( G\).
	\begin{enumerate}
		\item
            Si \( G\) est un ensemble fini, l'\defe{ordre}{ordre d'un groupe} de \( G\) est son cardinal\footnote{Définition \ref{PROPooJLGKooDCcnWi}.}, et nous le notons \( | G |\).
        \item
            Si l'ensemble \( G\) est infini, nous disons que \( | G |=\infty\) et qu'il est d'ordre infini.
	\end{enumerate}
    Oui : nous pourrions simplement toujours dire «cardinalité» et écrire \( \Card(G)\). Au lieu de ça, dans le cas particulier des groupes, il y a une tradition de dire «ordre» et d'écrire \( | G |\).
\end{definition}

\begin{definition}[Ordre d'un élément]      \label{DEFooKSTVooOObpgC}
      L'\defe{ordre}{ordre!d'un élément} d'un élément \( g\) de \( G\) est le naturel
      \begin{equation}
          \min\{ n\in\eN\setminus\{ 0 \}\tq g^n=e \},
      \end{equation}
      si il existe; dans le cas contraire, nous disons que l'ordre de \( g\) est infini.
\end{definition}

\begin{normaltext}
    Nous verrons que le corolaire~\ref{CorpZItFX} au théorème de Lagrange dira que l'ordre d'un élément divise l'ordre du groupe.
\end{normaltext}

\begin{lemma}[\cite{PDFpersoWanadoo,BIBooZFPUooIiywbk}]\label{LemHUkMxp}
	Soient un groupe \( G\) et deux sous-groupes normaux\footnote{Sous-groupe normal, définition \ref{DEFooNIIMooFkZgvX}.} \( H\) et \( K\) tels que \( H\cap K=\{ e \}\). Alors :
	\begin{enumerate}
		\item       \label{ITEMooDFVBooSnnlgR}
		      Tout élément de \( H\) commute avec tout élément de \( K\).
		\item       \label{ITEMooVVBGooZSJqjp}
		      \( HK\) est un sous-groupe de \( G\).
		\item       \label{IMTEooPCBZooQoZFOD}
		      L'application
		      \begin{equation}
			      \begin{aligned}
				      \varphi\colon H\times K & \to HK     \\
				      (h,k)                   & \mapsto hk
			      \end{aligned}
		      \end{equation}
		      est un isomorphisme de groupes.
	\end{enumerate}
\end{lemma}

\begin{proof}
	Point par point.
	\begin{subproof}
		\item[\ref{ITEMooDFVBooSnnlgR}]
		Soient \( h\in H\) et \( k\in K\). Nous voulons montrer que \( hk=kh\). Pour cela nous considérons l'élément \( a=hkh^{-1}k^{-1}\). Comme \( H \) est normal dans \( G\), nous avons
		\begin{equation}
			kh^{-1}k^{-1}\in H
		\end{equation}
		et donc \( a\in H\). De même \( K\) étant normal dans \( G\), nous avons \( hkh^{-1}\in K\) et donc \( a\in K\). Au final \( a\in H\cap K=\{ e \}\). Nous avons prouvé que
		\begin{equation}
			hkh^{-1}k^{-1}=e,
		\end{equation}
		et donc que \( hk=kh\).
		\item[\ref{ITEMooVVBGooZSJqjp}]
		Puisque \( H\) et \( K\) sont des sous-groupes, \( \{ e \}\) est dans les deux, de telle sorte que \( e\in HK\). De plus si \( h_i\in H\) et \( k_i\in K\), la commutativité du point \ref{ITEMooDFVBooSnnlgR} donne
		\begin{equation}
			(h_1k_1)(h_2k_2)=h_1h_2k_1k_2\in HK.
		\end{equation}
		Donc le produit de deux éléments de \( HK\) est dans $HK$.
		\item[\ref{IMTEooPCBZooQoZFOD}]
		En trois sous-parties.
		\begin{subproof}
			\item[Morphisme]
			Soient \( h_i\in H\) et \( k_i\in K\). En utilisant la commutativité du point \ref{ITEMooDFVBooSnnlgR} nous avons
			\begin{subequations}
				\begin{align}
					\varphi\big( (h_1,k_1)(h_2,k_2) \big) & =\varphi(h_1h_2,k_1k_2)            \\
					                                      & =(h_1h_2)(k_1k_2)                  \\
					                                      & =(h_1k_1)(h_2k_2)                  \\
					                                      & =\varphi(h_1,k_1)\varphi(h_2,k_2).
				\end{align}
			\end{subequations}
			\item[Injectif]
			Si \( \varphi(h_1,k_1)=\varphi(h_2,k_2)\) nous avons successivement
			\begin{subequations}
				\begin{align}
					h_1k_1                 & =h_2k_2       \\
					h_1 k_1 h_2^{-1}       & =k_2          \\
					h_1k_1h_2^{-1}k_1^{-1} & =k_2k_1^{-1}  \\
					h_1h_2^{-1}            & =k_2k_1^{-1}.
				\end{align}
			\end{subequations}
			Le membre de gauche est un élément de \( H\) et le membre de droite un élément de \( K\). Comme \( H\cap K=\{ e \}\) nous avons \( h_1h_2^{-1}=e\) et \( k_2k_1^{-1}=e\), c'est-à-dire \( h_1=h_2\) et \( k_1=k_2\).
			\item[Surjectif]
			Un élément général de \( HK\) est \( hk\) avec \( h\in H\) et \( k\in K\), c'est à dire \( \varphi(h,k)\).
		\end{subproof}
	\end{subproof}
\end{proof}

\begin{definition}  \label{DefvtSAyb}
	L'\defe{exposant}{exposant!d'un groupe} du groupe \( G\) est le plus petit entier non nul \( n\) tel que \( g^n=e\) pour tout \( g\in G\). S'il n'existe pas un tel \( n\), nous disons que l'exposant du groupe est infini.
\end{definition}

\begin{proposition} \label{PROPooSWHHooOzqWkw}
    À propos d'exposant de groupe et de ppcm.
	\begin{enumerate}
		\item
		      Si l'ensemble des ordres de tous les éléments d'un groupe est majoré, alors l'exposant du groupe est le plus petit commun multiple des ordres des éléments du groupe.
		\item
		      Pour un groupe fini, l'exposant est le \( \ppcm\) des ordres des éléments du groupe.
	\end{enumerate}
\end{proposition}

Le théorème de Burnside~\ref{ThooJLTit} nous donnera un bon paquet d'exemples de groupes d'exposant fini dans \( \GL(n,\eC)\).

\begin{proposition} \label{PropSRMJooIDPBoW}
	Soit un groupe \( G\). Nous considérons un sous-groupe normal \( H\) de \( G\) ainsi qu'un morphisme \( \psi\colon G\to H\). Alors
	\begin{enumerate}
		\item
		      \( \psi(H)\) est normal dans \( \psi(G)\)
		\item
		      Si \( G/H\) est abélien alors \( \psi(G)/\psi(H)\) est abélien.
	\end{enumerate}
\end{proposition}

\begin{proof}
	Soient \( h\in H\) et \( g\in G\). Alors \( \psi(g)\psi(h)\psi(g)^{-1}=\psi(ghg^{-1})\in\psi(H)\). Donc \( \psi(H)\) est normal dans \( \psi(G)\).

	Pour la seconde partie nous notons \( [\ldots]\) les classes par rapport à \( \psi(H)\) et \( \overline{ \vphantom{g}\ldots }\) celles par rapport à \( H\). Nous avons
	\begin{subequations}
		\begin{align}
			[\psi(g_1)][\psi(g_2)] & =\big[ \psi(g_1)\psi(g_2) \big]            \\
			                       & =\big[ \psi(g_1g_2) \big]                  \\
			                       & =\{ \psi(g_1g_2)\psi(h)\tq h\in H \}       \\
			                       & =\{ \psi(g_1g_2h)\tq h\in H \}             \\
			                       & =\psi\Big(  \{ g_1g_2h\tq h\in H \}  \Big) \\
			                       & =\psi\big( \overline{ g_1g_2 } \big)       \\
			                       & =\psi(\overline{ g_2g_1 })                 \\
			                       & =\text{refaire à l'envers}                 \\
			                       & =[\psi(g_2)][\psi(g_1)].
		\end{align}
	\end{subequations}
	Par conséquent \( \psi(G)/\psi(H)\) est abélien.
\end{proof}

%+++++++++++++++++++++++++++++++++++++++++++++++++++++++++++++++++++++++++++++++++++++++++++++++++++++++++++++++++++++++++++
\section{Groupe dérivé}
%+++++++++++++++++++++++++++++++++++++++++++++++++++++++++++++++++++++++++++++++++++++++++++++++++++++++++++++++++++++++++++

\begin{definition}      \label{DEFooVHZAooUgmesE}
	Si \( G\) est un groupe et si \( g,h\in G\), nous notons \( [g,h]=ghg^{-1}h^{-1}\)\nomenclature[R]{\( [g,h]\)}{commutateur dans un groupe} le \defe{commutateur}{commutateur!dans un groupe} de \( g\) et \( h\).
\end{definition}

L'élément neutre est toujours un commutateur : pour \( g=h \), \( [g,g]=ggg^{-1}g^{-1}=e \).

\begin{definition}      \label{DEFooBNLPooShKYXa}
	Le \defe{groupe dérivé}{dérivé!groupe}\index{groupe!dérivé} de \( G\) est le sous-groupe noté \( D(G)\)\nomenclature[R]{\( D(G)\)}{groupe dérivé} ou \( [G,G]\)\nomenclature[R]{\( [G,G]\)}{groupe dérivé} engendré\footnote{Définition \ref{DefooRDRXooEhVxxu}.} par les commutateurs.
\end{definition}
Autrement dit, \( D(G)\) est l'intersection de tous les sous-groupes de \( G\) contenant tous les commutateurs. Le groupe \( D(G)\) contient toujours au moins le neutre parce que c'est un groupe.

En vertu du lemme~\ref{LemFUIZooBZTCiy}, le groupe dérivé de \( G\) est l'ensemble des produits finis de commutateurs. C'est-à-dire que si \( S_m\) est l'ensemble des produits de \( m\) commutateurs, alors
\begin{equation}
	D(G)=\bigcup_{m=1}^{\infty}S_m.
\end{equation}

\begin{lemma}   \label{LemMMOCooDJJJhy}
	Le groupe dérivé est un sous-groupe caractéristique\footnote{Définition \ref{DEFooUXXTooCCLmQe}.}, et un sous-groupe normal\footnote{Définition \ref{DEFooNIIMooFkZgvX}.}.
\end{lemma}

\begin{proof}
	Il est évident que si \( \alpha\in\Aut(G)\) alors
	\begin{equation}
		\alpha\big( [g,h] \big)=\big[ \alpha(g),\alpha(h) \big],
	\end{equation}
	c'est-à-dire que \( D(G)\) est un sous-groupe caractéristique. En particulier si \( c\) est un commutateur, alors \( xcx^{-1}\) en est encore un, ce qui montre que \( D(G)\) est normal dans \( G\). Plus spécifiquement,
	\begin{subequations}
		\begin{align}
			x(ghg^{-1}h^{-1})x^{-1} & =(xgx^{-1})(xhx^{-1})(xg^{-1}x^{-1})(xh^{-1}x^{-1})  \\
			                        & =(xgx^{-1})(xhx^{-1})(xgx^{-1})^{-1}(xhx^{-1})^{-1}.
		\end{align}
	\end{subequations}
\end{proof}

\begin{proposition}\label{PropAPRGooHBkELf}
	Le groupe quotient \( G/D(G)\) est abélien.
\end{proposition}

\begin{proof}
	En ce qui concerne le fait que \( G/D(G)\) soit abélien, nous savons que pour tout \( g,h\in G\) nous avons \( h^{-1}g^{-1}hg\in D(G)\) et donc
	\begin{equation}
		[g][h]=[gh]=[ghh^{-1}g^{-1}hg]=[hg]=[h][g].
	\end{equation}
\end{proof}

Le groupe quotient \( G/D(G)\) est appelé l'\defe{abélianisé}{abélianisé} de \( G\) et est parfois noté \( G^{ab}\)\nomenclature[R]{\( G^{ab}\)}{groupe abélianisé de \( G\)}.

Si \( f\colon G\to A\) est un morphisme entre le groupe \( G\) et un groupe abélien \( A\), alors \( f\big( D(G) \big)=\{ 0 \}\). Du coup \( f\) passe au quotient de \( G\) par \( D(G)\), et il existe une unique application \( \bar f\colon G/D(G)\to A\) telle que \( f=\bar f\circ \pi\) où \( \pi\colon G\to G/D(G)\) est la projection canonique.

%+++++++++++++++++++++++++++++++++++++++++++++++++++++++++++++++++++++++++++++++++++++++++++++++++++++++++++++++++++++++++++
\section{Théorèmes d'isomorphismes}
%+++++++++++++++++++++++++++++++++++++++++++++++++++++++++++++++++++++++++++++++++++++++++++++++++++++++++++++++++++++++++++

\begin{definition}      \label{DEFooWBIYooGNRYOp}
	Soient un groupe \( G\), un ensemble $X$ et une application \( f\colon X\to G\). Le \defe{noyau}{noyau!application vers un groupe} de \( f\) est la partie
	\begin{equation}
		\Kernel f = \ker(f)=\{ x\in X\tq f(x)=e \}
	\end{equation}
	où \( e\) est l'élément neutre de \( G\).
\end{definition}

Si \( G\) est un groupe et si \( N\) est un sous-groupe normal, alors l'ensemble \( G/N\) a une structure de groupe et la projection canonique \( \pi\colon G\to G/N\) est un morphisme surjectif de noyau~\( N\).

\begin{theorem}[Premier théorème d'isomorphisme]        \label{ThoPremierthoisomo}
	Soit \( \theta\colon G\to H\) un morphisme de groupe. Alors
	\begin{enumerate}
		\item
		      \( \Kernel\theta\) est normal dans \( G\),
		\item
		      \( \Image \theta\) est un sous-groupe de \( H\)
		\item   \label{ItemWLCLdk}
		      nous avons un isomorphisme naturel
		      \begin{equation}
			      G/\Kernel\theta\simeq \Image\theta
		      \end{equation}
	\end{enumerate}
\end{theorem}
\index{théorème!isomorphisme!premier!pour les groupes}

\begin{proof}
	Point par point.
	\begin{enumerate}
		\item
		      Le fait que  \( \Kernel\theta\) soit un sous-groupe de \( G\) est clair; montrons qu'il est normal. Si \( g \in G \) et \( u \in \Kernel\theta\), alors \(\theta (g^{-1} u g) = \theta(g^{-1})\theta(u)\theta(g) = \bigl(\theta(g)\bigr)^{-1}\theta(g) = 1_H \), et donc \( g^{-1} u g \in \Kernel\theta\).
		\item
		      Il suffit de remarquer que si \( h = \theta(g) \) et \( h' = \theta(g') \), alors \( h^{-1} h' = \theta(g^{-1} g') \).
		\item
		      Si \( [g]\) représente la classe de \( g\) dans \( G/\Kernel\theta\), l'isomorphisme est donné par \( \varphi([g])=\theta(g)\).
	\end{enumerate}
\end{proof}

\ifbool{isGiulietta}{
	Ce premier théorème d'isomorphismes permet entre autres de prouver que $SO(3)=\SU(2)/\eZ_2$, voir la proposition \ref{PROPooDKPTooBnLflt}.
}{}
\begin{theorem}[Deuxième théorème d'isomorphisme]   \label{THOooURXUooQJvkjx}
    Soient \( H\) et \( N\) deux sous-groupes de \( G\) et supposons que \( N\) soit normal\footnote{Si \( N\) n'est pas normal, il y aura la proposition \ref{PROPooVBGMooPTlyLF}.}. Alors
	\begin{enumerate}
		\item
		      \( NH=HN\) est un sous-groupe.
		\item
		      Le groupe \( N\) est normal dans \( NH\).
		\item
		      Le groupe \( N\cap H\) est normal dans \( H\).
		\item   \label{ItemjRPajc}
		      Nous avons l'isomorphisme
              \begin{equation}
			      HN / N \simeq H / H\cap N .
		      \end{equation}
	\end{enumerate}
\end{theorem}
\index{théorème!isomorphisme!second}

\begin{proof}
	Point par point.
	\begin{enumerate}
		\item
		      Il est clair que \( 1_G \in NH \). Soient $nh$ et $n'h'$ deux éléments de \( NH \); alors en tenant compte du fait que \( N\) est normal,
		      \begin{equation}
			      nhn'h'=n\underbrace{hn'h^{-1}}_{\in N}hh'\in NH.
		      \end{equation}
		      Cela prouve que \( NH\) est un groupe.

		      De la même façon, nous prouvons que \( HN\) est un groupe par
		      \begin{equation}
			      hnh'n'=hh'\underbrace{h'^{-1}nh'}_{\in N}n'\in HN
		      \end{equation}

		      Nous devons encore prouver que \( HN=NH\). Pour cela, \( nh \in HN \), car \( nh = hh^{-1}nh \), les trois derniers facteurs formant un  élément de \( N \) par normalité; de même \( hn \in NH \), montrant que \( NH = HN \). Enfin, comme \( (nh)^{-1} = h^{-1} n^{-1} \), les inverses de \( NH \) sont dans \( HN = NH \).
		\item
		      \( N\) est normal dans \( G \), a fortiori dans l'un de ses sous-groupes.
		\item
		      Il suffit de voir que, si \( h \in H \) et \( n \in N \cap H \), alors \( hnh^{-1} \in N \cap H \). Or, \( hnh^{-1} \in H \) puisque \( H\) est un sous-groupe; et \( hnh^{-1} \in N \) car \( N \) est un sous-groupe normal de \( G \).
		\item
		      Il faut d'abord remarquer que \( H\) et \( N\) étant des groupes et le produit \( NH\) étant un groupe, nous avons \( NH=HN\). Soit le morphisme injectif
		      \begin{equation}
			      \begin{aligned}
				      j\colon H & \to HN    \\
				      h         & \mapsto h
			      \end{aligned}
		      \end{equation}
		      et la surjection canonique
		      \begin{equation}
			      \sigma\colon HN\to HN/N
		      \end{equation}
		      Nous considérons ensuite l'application composée
		      \begin{equation}
			      \begin{aligned}
				      f\colon H & \to HN/N    \\
				      h         & \mapsto hN.
			      \end{aligned}
		      \end{equation}

		      \begin{subproof}
			      \item[\( f\) est surjective]
			      L'application \( f\) est surjective parce que l'élément \( hnN\in HN/N\) est l'image de \( h\), étant donné que \( hnN=hN\).

			      \item[\( \Kernel f = H\cap N\)]
			      Si \( a\in H\cap N\), nous avons \( f(a) =aN = N\), et donc \( H\cap N\subset \Kernel f\). D'autre part, si \( h\in H\) vérifie \( h\in\Kernel f\), alors \( f(h)=hN=N\), ce qui est uniquement possible lorsque \( h\in N\).

		      \end{subproof}
		      Le premier théorème d'isomorphisme implique alors que \( H/\Kernel f\simeq \Image f\), c'est-à-dire
		      \begin{equation}
			      H/N\cap H\simeq HN/N.
		      \end{equation}
	\end{enumerate}
\end{proof}

\begin{proposition}[Deuxième théorème d'isomorphisme (suite)]     \label{PROPooVBGMooPTlyLF}
    Soient \( N\) et \( H\) des sous-groupes de \( G\). Si \( H\) normalise \( N\), c'est à dire si \( hNh^{-1}\in N\) pour tout \( h\in H\), alors nous avons l'isomorphisme
  \begin{equation}
      HN / N \simeq  H / H\cap N .
  \end{equation}
\end{proposition}

\begin{theorem}[Troisième théorème d'isomorphisme]  \label{ThoezgBep}
	Soient \( N\) et \( M\) deux sous-groupes normaux de \( G\) avec \( M\subset N\). Alors \( N/M\) est normal dans \( G/M\) et
	\begin{equation}
		(G/M)/(N/M)\simeq G/N.
	\end{equation}
\end{theorem}
\index{théorème!isomorphisme!troisième}

\begin{proof}
	Afin de montrer que \( N/M\) est normal dans \( G/M\), nous considérons \( g\in G\), \( nM\in N/M\) et nous calculons
	\begin{equation}
		gnMg^{-1}=gng^{-1}\underbrace{gMg^{-1}}_{=M}=\underbrace{gng^{-1}}_{\in N}M\in N/M.
	\end{equation}

	Pour prouver l'isomorphisme nous considérons le morphisme
	\begin{equation}
		\begin{aligned}
			\varphi\colon G/M & \to G/N     \\
			gM                & \mapsto gN.
		\end{aligned}
	\end{equation}
	Ce morphisme est surjectif et son noyau est \( N/M\), parce que \( \varphi(gM)=N\) uniquement si \( g\in N\). Nous pouvons appliquer le premier théorème d'isomorphisme à \( \varphi\) en écrivant
	\begin{equation}
		(G/M)/\Kernel \varphi\simeq\Image \varphi,
	\end{equation}
	c'est-à-dire
	\begin{equation}
		(G/M)/(N/M)\simeq G/N.
	\end{equation}
\end{proof}

%+++++++++++++++++++++++++++++++++++++++++++++++++++++++++++++++++++++++++++++++++++++++++++++++++++++++++++++++++++++++++++
\section{Indice d'un sous-groupe et ordre des éléments}
%+++++++++++++++++++++++++++++++++++++++++++++++++++++++++++++++++++++++++++++++++++++++++++++++++++++++++++++++++++++++++++

\begin{lemma}       \label{LEMooFNVRooRCkjLc}
	Lorsque \( H\) est normal dans \( G\), alors la définition
	\begin{equation}        \label{EQooEUESooSeUWHK}
		[a]\cdot[b]=[ab]
	\end{equation}
	définit une loi de groupe sur l'ensemble \( G/H\).
\end{lemma}

\begin{proof}
	Le neutre est \( [e]\) et l'associativité ne pose pas plus de problème que l'existence d'un inverse. Le point à vérifier est que la formule \eqref{EQooEUESooSeUWHK} est une bonne définition : \( [ah]\cdot [bh']=[ab]\) pour tout \( h,h'\in H\). Nous avons :
	\begin{equation}
		[ah]\cdot [ah']=[ahah']=[ahb].
	\end{equation}
	Pour montrer que c'est \( [ab]\), l'astuce est d'introduire \( bb^{-1}\) à côté du \( a\) :
	\begin{equation}
		[ahb]=[abb^{-1}hb]=[ab]
	\end{equation}
	parce que \( b^{-1} hb\in H\) du fait que \( H\) soit normal dans \( G\).
\end{proof}

\begin{example}[\cite{ooTVPEooGOLEom}]      \label{EXooFNIKooHxePSs}
	Il ne faudrait pas croire que le groupe quotient \( G/H\) est forcément un sous-groupe de \( G\). Par exemple le quotient \( \eZ/2\eZ\) est l'ensemble \( \{ 0,1 \}\) muni de l'addition. En particulier \( 1+1=0\), ce qui est évidemment faux dans \( \eZ\). Le groupe \( (\eZ,+)\) ne possède aucun élément d'ordre \( 2\).

	Il n'en est pas moins vrai que l'application
	\begin{equation}
		\begin{aligned}
			f\colon G & \to G/H     \\
			g         & \mapsto [g]
		\end{aligned}
	\end{equation}
	est un morphisme de groupes.
\end{example}

\begin{definition}      \label{DEFooMPIAooIeZNaR}
	Si \( H\) est un sous-groupe d'un groupe fini, l'\defe{indice}{indice} de \( H\) dans \( G\) est le nombre \( | G |/| H |\), souvent noté \( | G:H |\).
\end{definition}

Le théorème de Lagrange dira en particulier que l'indice est toujours un nombre entier. C'est à ne pas confondre avec le degré d'une extension de corps (définition~\ref{DefUYiyieu}).

\begin{theorem}[Théorème de Lagrange]   \label{ThoLagrange}
	Soit \( H\) un sous-groupe du groupe fini \( G\).  Alors
	\begin{enumerate}
		\item   \label{ITEMooDPKSooNpOusd}
		      L'ordre de \( H\) divise l'ordre de \( G\).
		\item
		      Les trois nombres suivants sont égaux :
		      \begin{itemize}
			      \item
			            le nombre de classes de \( H\) à gauche,
			      \item
			            le nombre de classes de \( H\) à droite,
			      \item
			            l'indice de \( H\) dans \( G\).
		      \end{itemize}
	\end{enumerate}
	En particulier si \( H\) est distingué dans \( G\), nous avons
	\begin{equation}
		| G/H |=\frac{ | G | }{ | H | }.
	\end{equation}
\end{theorem}
\index{théorème!Lagrange}
\index{Lagrange!théorème}

\begin{proof}
	Nous commençons par montrer que les classes de \( H\) ont toutes le même nombre d'éléments que \( H\). En effet pour chaque \( g\in G\) nous avons la bijection
	\begin{equation}
		\begin{aligned}
			\varphi\colon H & \to gH      \\
			h               & \mapsto gh.
		\end{aligned}
	\end{equation}
	L'injectivité de \( \varphi\) est le fait que \( gh=gh'\) implique \( h=h'\). La surjectivité est par définition de la classe.

	Les classes à gauche formant une partition de \( G\), le cardinal de \( G\) est le produit de la taille des classes par le nombre de classes :
	\begin{equation}
		| G |=| H |\cdot\text{nombre de classes}.
	\end{equation}
	En particulier nous voyons que \( | H |\) divise \( | G |\).

	La dernière formule exprime simplement que \( G/H\) est par définition le nombre de classes de \( H\) à gauche (ou à droite) dans \( G\).
\end{proof}

\begin{corollary}       \label{CorpZItFX}
	L'ordre d'un élément d'un groupe fini divise l'ordre du groupe. En particulier dans un groupe d'ordre \( n\) tous les éléments vérifient \( g^n=e\).
\end{corollary}

\begin{proof}
	Soit \( G\) un groupe fini et considérons, à \( g \in G \) fixé, le sous-groupe
	\begin{equation}
		H=\{ g^k\tq k\in\eN \}.
	\end{equation}
	Par le théorème de Lagrange~\ref{ThoLagrange}, l'ordre de \( H\) divise \( | G |\), mais l'ordre de \( H\) est le plus petit \( k\) tel que \( g^k=e\), c'est-à-dire l'ordre de \( g\).
\end{proof}

D'autres résultats à propos d'ordres et d'indices de groupes finis dans la proposition \ref{PROPooVWVIooQzuAlA} et le lemme \ref{LemqAUBYn}. En particulier le théorème de Cauchy \ref{THOooSUWKooICbzqM} qui dit : si \( p\) divise l'ordre du groupe \( G\), alors \( G\) contient au moins un élément d'ordre \( p\).

%+++++++++++++++++++++++++++++++++++++++++++++++++++++++++++++++++++++++++++++++++++++++++++++++++++++++++++++++++++++++++++
\section{Suite de composition}
%+++++++++++++++++++++++++++++++++++++++++++++++++++++++++++++++++++++++++++++++++++++++++++++++++++++++++++++++++++++++++++
\index{sous-groupe!normal}\index{groupe!quotient}\index{quotient!de groupe}

%TODO : citer la page de la wikiversité sur Jordan-Hölder.
%TODO : donner la définition d'un raffinement de suite de composition.

\begin{definition}[Suite de composition]  \label{DefJWZSooNcntfK}
    Soit un groupe \( G\).
	\begin{enumerate}
		\item
		      Une \defe{suite de composition}{composition!suite de}\index{suite de composition} dans \( G\) est une suite finie de sous-groupes \( (G_i)_{i=0,\ldots, n}\) telle que
		      \begin{equation}
				  \{ e \}=G_n\subseteq G_{n-1}\subseteq\ldots\subseteq G_1\subseteq G_0=G
		      \end{equation}
		      et telle que \( G_{i+1}\) est normal\footnote{Nous rappelons au cas où, que «normal» signifie «distingué».} dans \( G_i\).
          \item
              Les groupes \( G_i/G_{i+1}\) sont les \defe{quotients}{quotient!dans une suite de composition} de la suite de composition.
		\item
		      Une suite de \defe{Jordan-Hölder}{suite!de Jordan-Hölder}\index{Jordan-Hölder} est une suite de composition dont tous les quotients sont simples.
	\end{enumerate}
\end{definition}
L'objet de nos prochaines pérégrinations mathématiques est de montrer que tout groupe fini admet une suite de Jordan-Hölder (théorème~\ref{ThoLgxWIC}).

\begin{lemma}[du papillon ou de Zassenhaus\cite{NjCCfW}]\label{LemsKpXCG}
	Soient \( G\) un groupe et des sous-groupes \( A\) et \( B\). Soient \( A'\) normal dans \( A\) et \( B'\) normal dans \( B\). Alors
	\begin{enumerate}
		\item
		      \( A'(A\cap B')\) est normal dans \( A'(A\cap B)\)
		\item
		      \( (A'\cap B)B'\) est normal dans \( (A\cap B)B'\)
		\item
		      Nous avons les isomorphismes de groupes
		      \begin{equation}
			      \frac{ A'(A\cap B) }{ A'(A\cap B') }\simeq\frac{ (A\cap B)B' }{ (A'\cap B)B' }\simeq\frac{ B'(A\cap B) }{ B'(A'\cap B) }.
		      \end{equation}
	\end{enumerate}
\end{lemma}

\begin{proof}
	Nous n'allons pas démontrer chacun des points; pour plus de détails, nous dirons simplement que «la preuve est très similaire dans les autres cas».

	Commençons par montrer que \( A'(A\cap B')\) est un groupe. Si \( a,b\in A'\) et \( x,y\in A\cap B'\),
	\begin{equation}
		axby=xx^{-1}axbx^{-1}xy
	\end{equation}
	En utilisant la normalité, \( x^{-1}ax\in A'\), donc \( xx^{-1}axbx^{-1}\in A'\) et donc le tout est dans \( A'(A\cap B')\). L'ensemble \( A'(A\cap B')\) est également stable pour l'inverse parce que
	\begin{equation}
		x^{-1}a^{-1}=\underbrace{x^{-1}a^{-1}x}_{\in A'}x^{-1}.
	\end{equation}

	Nous montrons maintenant que \( A'(A\cap B')\) est normal dans \( A'(A\cap B)\). Soient \( a,b\in A'\), \( x\in A\cap B'\) et \( f\in A\cap B\). Alors
	\begin{subequations}
		\begin{align}
			(bf)^{-1}(ax)(bf) & =(bf)^{-1}(a\underbrace{xbx^{-1}}_{=c\in A'}xf)        \\
			                  & =f^{-1}b^{-1}acxf                                      \\
			                  & =f^{-1}b^{-1}acf\underbrace{f^{-1}xf}_{=y\in A\cap B'} \\
			                  & =\underbrace{f^{-1}b^{-1}acf}_{\in A'}y                \\
			                  & \in A'(A\cap B').
		\end{align}
	\end{subequations}

	Pour prouver l'isomorphisme
	\begin{equation}
		\frac{ A'(A\cap B) }{ A'(A\cap B') } \simeq \frac{ (A\cap B)B' }{ (A'\cap B)B' },
	\end{equation}
	nous allons utiliser le deuxième théorème d'isomorphisme (\ref{PROPooVBGMooPTlyLF}) que nous appliquons à \( H=A\cap B\) et \( N=A'(A\cap B')\). La vérification que \( H\) normalise \( N\) est usuelle. Nous commençons par écrire
	\begin{equation}    \label{EqkphNsE}
		\frac{ A'(A\cap B')(A\cap B) }{ A'(A\cap B') }\simeq\frac{ A\cap B }{ A\cap B\cap A'(A\cap B') }.
	\end{equation}
	Pour simplifier un peu cette expression nous prouvons d'abord que
	\begin{equation}    \label{EqkhsyNh}
		(A\cap B)\cap A'(A\cap B')=(A'\cap B)(A\cap B').
	\end{equation}
	L'inclusion \( \supset\) est facile. Pour l'autre sens, étant donné que \( A'(A\cap B')\subset A\) nous avons
	\begin{equation}        \label{EQooEEVIooHCbasF}
		A\cap B\cap A'(A\cap B)=B\cap A'(A\cap B).
	\end{equation}
	Un élément de \( B\cap A'(A\cap B)\) est un élément de \( B\) qui s'écrit sous la forme \( s=ax\) avec \( a\in A'\) et \( x\in A\cap B\). Nous avons alors \( a=sx^{-1}\) avec \( s\in B\) et \( x^{-1} \in A\cap B\). Par conséquent \( a\in B\) et donc \( a\in A'\cap B\). Donc un élément de \( B\cap A'(A\cap B)\) s'écrit sous la forme \( ax\) avec \( a\in A'\cap B\) et \( x\in A\cap B\). Autrement dit
	\begin{equation}
		B\cap A'(A\cap B)\subset (A'\cap B)(A\cap B)
	\end{equation}
	et nous avons
	\begin{equation}
		(A\cap B)\cap A'(A\cap B)=B\cap A'(A\cap B)\subset (A'\cap B)(A\cap B'),
	\end{equation}
	et donc l'égalité \eqref{EqkhsyNh}. Toujours dans l'idée de simplifier \eqref{EqkphNsE} nous remarquons que \( A\cap B'\) est un sous-ensemble de \( A\cap B\), donc \( A'(A\cap B')(A\cap B)=A'(A\cap B)\). Il reste donc
	\begin{equation}
		\frac{ A'(A\cap B) }{ A'(A\cap B') }=\frac{ A\cap B }{ (A'\cap B)(A\cap B') }.
	\end{equation}
	Étant donné que les hypothèses sur \( A\) et \( B\) sont symétriques, le membre de droite peut aussi s'écrire en inversant \( A\) et \( B\). Nous en sommes à
	\begin{equation}
		\frac{ B'(A\cap B) }{ B'(A'\cap B) }=\frac{ A'(A\cap B) }{ A'(A\cap B') }.
	\end{equation}
	Nous devons encore justifier \( B'(A\cap B)=(A\cap B)B'\) et \( B'(A'\cap B)=(A'\cap B)B'\). Vérifions la première égalité, et laissons la seconde \href{https://abstrusegoose.com/395}{\randomGender{au lecteur}{à la lectrice}}.
	Si \( b\in B'\) et \( x\in A\cap B\), alors
	\begin{equation}
		bx=x\underbrace{x^{-1}bx}_{\in B'}\in (A\cap B)B'.
	\end{equation}
\end{proof}

\begin{proposition}     \label{PROPooUIXNooJdKuAs}
	Si \( G\) est un groupe fini et si \( (G_i)\) est une suite de composition pour \( G\), alors l'ordre de \( G\) est le produit des ordres de ses quotients.
\end{proposition}

\begin{proof}
	Étant donné que \( G_{i+1}\) est toujours normal dans \( G_i\), le théorème de Lagrange \ref{ThoLagrange} s'applique et, à chaque pas de la suite de composition, nous avons :
	\begin{equation}
		\left| \frac{ G_i }{ G_{i+1} } \right| = \frac{ | G_i | }{ | G_{i+1} | }.
	\end{equation}
	Il suffit maintenant d'écrire \( | G |\) de façon télescopique :
	\begin{equation}
		| G |=\prod_{0\leq i\leq n-1}\frac{ | G_i | }{ | G_{i+1} | }
	\end{equation}
\end{proof}

\begin{definition}
	Nous disons que les deux suites de composition \( (G_i)_{0\leq i\leq r}\) et \( (G_j)_{0\leq j\leq s}\) sont \defe{équivalentes}{equivalence@équivalence!suite de composition} si \( r=s\) et si il existe une permutation \( \sigma\in S_{r-1}\) telle que
	\begin{equation}
		\frac{ G_i }{ G_{i+1} }\simeq\frac{ G_{\sigma(i)} }{ G_{\sigma(i)+1} }.
	\end{equation}
\end{definition}

\begin{proposition}[Schreider]\index{lemme!de Schreider}
	Deux suites de composition d'un même groupe admettent des raffinements équivalents.
\end{proposition}

\begin{proof}
	Soient les suites de composition
	\begin{subequations}
		\begin{align}
			\{ e \}=G_m\subseteq\ldots\subseteq G_1\subseteq G_0=G \\
			\{ e \}=H_m\subseteq\ldots\subseteq H_1\subseteq H_0=G
		\end{align}
	\end{subequations}
	Nous raffinons la suite \( (G_i)\) en remplaçant \( G_{i+1}\subseteq G_i\) par
	\begin{equation}
		G_{i+1}=G_{i+1}(G_i\cap H_n)\subset G_{i+1}(G_i\cap H_{n-1})\subseteq\ldots\subseteq G_{i+1}(G_i\cap H_0)=G_i,
	\end{equation}
	et de même pour \( (H_j)\). Le groupe \( G_{i+1}(G_i\cap H_k)\) est normal dans \( G_{i+1}(G_i\cap H_{k-1})\) parce que \( G_{i+1}\) étant normal dans \( G_i\), et \( H_k\) dans \( H_{k-1}\), le lemme~\ref{LemsKpXCG} s'applique. Nous avons donc bien défini un raffinement.

	Nous devons maintenant prouver que les deux raffinements ainsi construits sont des suites de composition équivalentes. D'abord elles ont la même longueur \( mn\) parce que chacun des \( m\) éléments de la suite \( (G_i)\) a été remplacé par \( n\) éléments et inversement, chacun des \( n\) éléments de la suite \( (H_j)\) a été remplacé par \( m\) éléments.

	Par ailleurs, les quotients du raffinement de \( (G_i)\) sont de la forme
	\begin{equation}    \label{EqPAYTCB}
		\frac{ G_{i+1}(G_i \cap H_k) }{ G_{i+1}(G_i\cap H_{k+1}) }\simeq \frac{ H_{k+1}(H_k\cap G_i) }{ H_{k+1}(H_k\cap G_{i+1}) }
	\end{equation}
	en vertu du lemme du papillon (\ref{LemsKpXCG}). Le membre de droite de \eqref{EqPAYTCB} est un des quotients du raffinement de \( (H_j)\).
\end{proof}

\begin{lemma}[Schreider strictement décroissant]    \label{LemBSicRJ}
	Soient \( \Sigma_1\) et \( \Sigma_2\), deux suites de composition strictement décroissantes du groupe \( G\). Alors elles admettent des raffinements équivalents strictement décroissants.
\end{lemma}

\begin{proof}
	Par hypothèse, \( \Sigma_1\) et \( \Sigma_2\) n'ont pas de répétitions. Soient \( \Sigma''_1\) et \( \Sigma''_2\), des raffinements équivalents donnés par le lemme de Schreider. Étant donné que ce sont des suites de composition équivalentes, elles ont le même nombre de quotients réduits à \( \{ e \}\), c'est-à-dire le même nombre de répétitions.

	Les suites \( \Sigma'_1\) et \( \Sigma'_2\) obtenues en retirant les répétitions de \( \Sigma''_1\) et \( \Sigma''_2\) sont des raffinements équivalents de \( \Sigma_1\) et \( \Sigma_2\) et strictement décroissants.
\end{proof}

\begin{theorem}[Jordan-Hölder]\label{ThoLgxWIC}
	À propos de suites de Jordan-Hölder dans un groupe fini.
	\begin{enumerate}
		\item       \label{ITEMooRSDDooNHkFYO}
			Tout groupe fini admet une suite de Jordan-Hölder.
		\item       \label{ITEMooGBOCooBAgnyt}
			Toutes les suites de Jordan-Hölder dans un groupe fini sont équivalentes.
	\end{enumerate}
\end{theorem}

\begin{proof}
	En deux parties.
	\begin{subproof}
	\item[Pour \ref{ITEMooRSDDooNHkFYO}]
%TODO
	\item[Pour \ref{ITEMooGBOCooBAgnyt}]

	Par définition, une suite de Jordan-Hölder n'a pas de raffinement strictement décroissant (à part elle-même) parce que \( G_{i+1}\) est normal maximum dans \( G_i\). Si \( \Sigma_1\) et \( \Sigma_2\) sont des suites de Jordan-Hölder nous pouvons considérer les raffinements équivalents strictement décroissants \( \Sigma'_1\) et \( \Sigma'_2\) du lemme de Schreider~\ref{LemBSicRJ}. Nous avons \( \Sigma'_1\sim\Sigma'_2\), mais par ce que nous venons de dire à propos de la maximalité, \( \Sigma'_1=\Sigma_1\) et \( \Sigma'_2=\Sigma_2\). D'où le résultat.
    \end{subproof}
\end{proof}

%+++++++++++++++++++++++++++++++++++++++++++++++++++++++++++++++++++++++++++++++++++++++++++++++++++++++++++++++++++++++++++
\section{Groupes résolubles}
%+++++++++++++++++++++++++++++++++++++++++++++++++++++++++++++++++++++++++++++++++++++++++++++++++++++++++++++++++++++++++++

\begin{definition}  \label{DefOSYNooTROIKs}
	Le groupe \( G\) est \defe{résoluble}{groupe!résoluble} si il existe une suite finie de sous-groupes \( G_i\)
	\begin{equation}
		\{ e \}=G_n\subset G_{n-1}\subset\ldots\subset G_1\subset G_0=G
	\end{equation}
	avec \( G_i\) normal dans \( G_{i+1}\) et \( G_i/G_{i+1}\) abélien.
\end{definition}
Il s'agit d'un groupe qui admet une suite de composition\footnote{Voir définition~\ref{DefJWZSooNcntfK}.} dont les quotients sont abéliens.

\begin{lemma}[\cite{HQRooKGAfpu}]   \label{LemOARMooYhYmbH}
	Soit \( G\) un groupe et \( H\) un sous-groupe normal. Le groupe \( G/H\) est abélien si et seulement si\footnote{Ici \( D(G)\) est le groupe dérivé de \( G\), voir \ref{DEFooBNLPooShKYXa}.} \( D(G)\subset H\).
\end{lemma}

\begin{proof}
	Les propositions suivantes sont équivalentes :
	\begin{itemize}
		\item Le groupe \( G/H\) est abélien
		\item pour tout \( x,y\in G\), \( [x][y]=[y][x]\)
		\item \( [x][y][x^{-1}][y^{-1}]=[e]\)
		\item \( [xyx^{-1}y^{-1}] =[e]\)
		\item \( [x,y]\in H\), voir la définition \ref{DEFooVHZAooUgmesE}.
		\item \( D(G)\subset H\).
	\end{itemize}
\end{proof}

\begin{proposition}[\cite{HQRooKGAfpu}] \label{PropRWYZooTarnmm}
	Un groupe est résoluble si et seulement si sa suite dérivée termine sur \( \{ e \}\).
\end{proposition}

\begin{proof}
	Grâce au lemme~\ref{LemMMOCooDJJJhy} et à la proposition~\ref{PropAPRGooHBkELf}, si la suite dérivée termine sur \( \{ e \}\) alors la suite dérivée est une suite qui répond aux conditions de la définition~\ref{DefOSYNooTROIKs} de groupe résoluble.

	Il faut donc encore montrer le sens direct. Nous supposons que \( G\) est un groupe résoluble et nous étudions sa suite dérivée. Nous avons une suite
	\begin{equation}
		\{ e \}=G_n\subset G_{n-1}\subset\ldots\subset G_1\subset G_0=G
	\end{equation}
	avec \( G_i/G_{i+1}\) abélien et \( G_{i+1}\) normal dans \( G_i\). Nous allons prouver par récurrence que \( D^i(G)\subset G_i\).

	Pour \( i=0\) nous avons bien \( G\subset G_0\). Notre hypothèse de récurrence est :
	\begin{equation}    \label{EqEAQEooEaeIEo}
		D^i(G)\subset G_i
	\end{equation}
	Par le lemme~\ref{LemOARMooYhYmbH} nous avons aussi
	\begin{equation}    \label{EqEDJXooLOLQcr}
		D(G_i)\subset G_{i+1}.
	\end{equation}
	En dérivant \eqref{EqEAQEooEaeIEo} et en tenant compte de \eqref{EqEDJXooLOLQcr}, \( D^{i+1}(G)\subset D(G_i)\subset G_{i+1}\). Donc par récurrence nous avons bien \( D^k(G)\subset G_k\) pour tout \( k\). Mais \( G_n=\{ e \}\) pour un certain \( n\), donc pour ce \( n\) nous avons \( D^n(G)=\{ e \}\), ce qu'il fallait.
\end{proof}

\begin{proposition} \label{PropBNEZooJMDFIB}
	Soient des groupes \( G\) et \( H\). Nous supposons que \( G\) est résoluble et nous considérons un morphisme \( \psi\colon G\to H\). Alors \( \psi(G)\) est résoluble.
\end{proposition}

\begin{proof}
	Puisque \( G\) est résoluble, il existe une suite de sous-groupes \( G_i\) tels que
	\begin{equation}
		\{ e \}=G_n\subset G_{n-1}\subset\ldots\subset G_1\subset G_0=G
	\end{equation}
	avec \( G_i\) normal dans \( G_{i+1}\) et \( G_i/G_{i+1}\) abélien. Nous posons \( \psi(G)_i=\psi(G_i)\) et nous avons \( \psi(G)_n=\psi\big( \{ e \} \big)=\{ e \}\) ainsi que \( \psi(G)_0=\psi(G)\); donc
	\begin{equation}
		\{ e \}=\psi(G)_n\subset \psi(G)_{n-1}\subset\ldots\subset \psi(G)_1\subset \psi(G)_0=\psi(G).
	\end{equation}

	La proposition \ref{PropSRMJooIDPBoW} nous indique que \( \psi(G)_i\) est normal dans \( \psi(G)_{i+1}\), et que \( \psi(G)_i/\psi(G)_{i+1}\) est abélien.
\end{proof}

%+++++++++++++++++++++++++++++++++++++++++++++++++++++++++++++++++++++++++++++++++++++++++++++++++++++++++++++++++++++++++++
\section{Action de groupes}
%+++++++++++++++++++++++++++++++++++++++++++++++++++++++++++++++++++++++++++++++++++++++++++++++++++++++++++++++++++++++++++
Le concept d'action d'un groupe est donné par la définition \ref{DefActionGroupe}

\begin{lemma}
	Pour tout \( g\in G\),
	\begin{enumerate}
		\item
		      L'application \( \phi_g\colon E\to E\) est injective,
		\item
		      Pour l'inverse : \( (\phi_g)^{-1}=\phi_{g^{-1}}\).
	\end{enumerate}
\end{lemma}

\begin{proof}
	Si \( x,y\in E\) sont tels que \( \phi_g(x)=\phi_g(y)\) alors en appliquant \( \phi_{g^{-1}}\) aux deux membres nous trouvons
	\begin{equation}
		(\phi_{g^{-1}}\phi_g)(x)=(\phi_{g^{-1}}\phi_g)(y),
	\end{equation}
	ce qui donne \( x=y\) parce que \( \phi_{g^{-1}}\phi_g=\phi_{g^{-1}g}=\phi_e=\id\).

	Les trois dernières égalités écrites disent que \( \phi_{g^{-1}}\) est l'inverse\footnote{Si vous décidez de dire ça à un jury dans un concours, soyez prêts à préciser les domaines.} de \( \phi_g\).
\end{proof}

Pour alléger les notations, on convient d'écrire $g \cdot x$, voire plus simplement $gx$ au lieu de \( \phi_g(x) \). Le deuxième axiome d'action de groupe dit que la notation $ghx$ ne souffre d'aucune ambiguïté.

\begin{definition}[Orbite]
	Si \( G\) agit sur un ensemble \( E\), nous notons \( G\cdot x\) l'\defe{orbite}{orbite!d'un point sous une action} de \( x\in E\) sous l'action de $G$:
	\begin{equation*}
		G\cdot x = \{ gx \tq g \in G\}.
	\end{equation*}
\end{definition}

\begin{definition}[Stabilisateur]       \label{DEFooMDYGooLrOERP}
	Si \( G\) est un groupe agissant sur l'ensemble \( E\), et si \( x\in E\), nous notons \( G_x\) ou \( \Stab(x)\) le \defe{stabilisateur}{stabilisateur} de \( x\) :
	\begin{equation}
		\Stab(x)=G_x=\{ g\in G\tq g\cdot x=x \}.
	\end{equation}
\end{definition}

\begin{definition}[Fixateur]
	Si \( G\) est un groupe agissant sur l'ensemble \( E\), et si \( g\in G\), nous notons enfin \( \Fix(g)\) le \defe{fixateur}{fixateur} de \( g\) :
	\begin{equation}
		\Fix(g)=\{ x\in E\tq g\cdot x=x \}.
	\end{equation}
\end{definition}

\begin{definition}  \label{DefuyYJRh}
	L'action de \( G\) sur \( E\) est \defe{fidèle}{fidèle (action)}\index{action!fidèle} si l'identité est le seul élément de \( G\) à fixer tous les points de \( E\), c'est-à-dire si \( gx=x, \forall x\in E\Rightarrow g=e\).
\end{definition}

Un exemple d'action fidèle tout à fait non trivial sera donné avec l'action du groupe modulaire sur le plan de Poincaré dans le théorème~\ref{ThoItqXCm}.

Le groupe \( G\) agit toujours sur lui même à gauche et à droite. L'action à gauche est \( g\cdot h=gh\); celle à droite est \( g\cdot h=hg^{-1}\).

\begin{definition}      \label{DEFooCORTooEeOLPT}
	L'action \defe{adjointe}{action!adjointe} définie par \( g\cdot h=ghg^{-1}\) est une manière pour un groupe d'agir sur lui-même par automorphismes. Cela est souvent noté \( \AD(g)h=ghg^{-1}\).
\end{definition}
En effet pour tout \( g\in G\), l'application \( \AD(g)\colon G\to G\) est un automorphisme de \( G\).

Si \( H\) est un sous-groupe de  \( G\), nous notons \( G/H\) le quotient de $G$ par la relation \( g\sim gh\) pour tout \( h\in H\). Lorsque la distinction est importante, nous noterons \( (G/H)_g\)\nomenclature[R]{$(G/H)_g$}{classes à gauche} pour les classes à gauche et \( (G/H)_d\) pour les classes à droite.

Nous avons une relation d'équivalence à gauche et une à droite. D'abord
\begin{equation}
	x\sim_g y\Leftrightarrow xh=y
\end{equation}
pour un certain \( h\in H\). Ensuite
\begin{equation}
	x\sim_d y\Leftrightarrow hx=y
\end{equation}
pour un certain \( h\in H\).

Le lemme suivant est une généralisation du théorème de Lagrange~\ref{ThoLagrange}.

\begin{lemma}
	L'ensemble \( (G/H)_g\) est fini si et seulement si l'ensemble \( (G/H)_d\) est fini. Si il en est ainsi, alors \( (G/H)_g\) et \( (G/H)_d\) ont même cardinal, qui vaut l'indice de \( H\) dans \( G\).
\end{lemma}

\begin{proof}
	L'application
	\begin{equation}
		\begin{aligned}
			f\colon (G/H)_g & \to (G/H)_d        \\
			[x]_g           & \mapsto [x^{-1}]_d
		\end{aligned}
	\end{equation}
	est une bijection bien définie. En effet si \( x\sim_g y\), nous avons \( h\in H\) tel que \( y^{-1}h=x^{-1}\), c'est-à-dire que \( x^{-1}\sim_d y^{-1}\) et \( f\) est bien définie. Le fait que \( f\) soit surjective est évident. Pour l'injectivité, soient \( x, y \in G \) tels que
	\begin{equation}
		f([x]_g)=f([y]_g).
	\end{equation}
	Alors \( x^{-1}\sim_d y^{-1}\), ce qui implique l'existence de \( h\in H\) tel que \( hx^{-1}=y^{-1}\), ou encore que \( xh^{-1}=y\), ce qui signifie que \( x\sim_gy\).

	Pour l'énoncé à propos de l'indice, nous procédons en plusieurs étapes simples.
	\begin{enumerate}
		\item
		      Les classes (les éléments de \( (G/H)_g\)) forment une partition de $G$.
		\item
		      Toutes les classes ont le même nombre d'éléments par la bijection
		      \begin{equation}
			      \begin{aligned}
				      f\colon [x]_g & \to [y]_g   \\
				      xh            & \mapsto yh.
			      \end{aligned}
		      \end{equation}
		\item
		      Le nombre d'éléments dans une classe est égal à \( | H |\) par la bijection
		      \begin{equation}
			      \begin{aligned}
				      g\colon [x]_g & \to H      \\
				      xh            & \mapsto h.
			      \end{aligned}
		      \end{equation}
	\end{enumerate}
	Par conséquent
	\begin{equation}
		| G |=| H |\cdot \text{nombre de classes}=| H | \cdot \Card\((G/H)_g\),
	\end{equation}
	et nous avons bien
	\begin{equation}
		\Card\((G/H)_g\)=\frac{ | G | }{ | H | }=| G:H |.
	\end{equation}
\end{proof}

\begin{proposition}[Orbite-stabilisateur\cite{Combes}]      \label{Propszymlr}
	Soient \( G\) un groupe agissant sur un ensemble \( E\) et \( x\in E\).
	\begin{enumerate}
		\item
		      Les ensembles \( G \cdot x\) et \( G/G_x\) sont équipotents.
		\item       \label{ITEMooCWUGooCOFHYk}
		      L'orbite de \(\Fix(x)\) est finie si et seulement si \( \Fix(x)\) est d'indice fini dans \( G\). Dans ce cas nous avons
		      \begin{equation}        \label{EqnLCHCE}
			      \Card(G\cdot x)=| G:\Fix(x) |.
		      \end{equation}
		      Une autre façon d'écrire la même formule :
		      \begin{equation}        \label{EqCewSXT}
			      | G |=| \Fix(x) | |\mO_x |.
		      \end{equation}
	\end{enumerate}
\end{proposition}
\index{équation!orbite-stabilisateur}
C'est la formule \eqref{EqnLCHCE} qui est nommée \wikipedia{fr}{Action_de_groupe_(mathématiques)\#Formule_des_classes.2C_formule_de_Burnside}{formule des classes} sur wikipédia.

\begin{proof}
	En deux points.
	\begin{enumerate}
		\item
		      Soit l'application
		      \begin{equation}
			      \begin{aligned}
				      \psi\colon G\cdot x & \to G/G_x    \\
				      a\cdot x            & \mapsto [a].
			      \end{aligned}
		      \end{equation}
		      Cette application est bien définie parce que si \( a\cdot x=b\cdot x\), alors il existe \( h\in G_x\) tel que \( b=ah\), et par conséquent \( [a]=[b]\). Cette application est une bijection et par conséquent \( G\cdot x\) est équipotent à \( G/G_x\).
		\item
		      Soit \( y\in \mO_x\) et \( A_y=\{ g\in G\tq g\cdot x=y \}\). L'ensemble \( A_y\) est une classe à gauche de \( \Fix(x)\), par conséquent \( | A_y |=|\Fix(x)|\) pour tout \( y\in\mO_x\). Les \( A_y\) pour différents \( y\) sont disjoints et nous avons de plus
		      \begin{equation}
			      \bigcup_{y\in\mO_x}A_y=G.
		      \end{equation}
		      Les ensembles \( A_y\) divisent donc \( G\) en \( | \mO_x |\) paquets de \( | \Fix(x) |\) éléments. D'où la formule \eqref{EqCewSXT}.
	\end{enumerate}
\end{proof}

\begin{corollary}       \label{CORooRRVHooTyCjZZ}
	Soit \( C_g\) la classe de conjugaison d'un élément  \( g\) du groupe fini \( G\). Alors
	\begin{equation}
		\Card(C_g)=| G:Z_G(g) |
	\end{equation}
	où $Z_G(g)$ est le centralisateur de \( g\) dans \( G\)\footnote{Définition~\ref{defGroupeCentre}.} de \( G\).
\end{corollary}

\begin{proof}
	C'est une application de la proposition~\ref{Propszymlr} (formule \eqref{EqnLCHCE}) dans le cas de l'action adjointe de \( G\) sur lui-même.

	En effet, si nous considérons l'action adjointe, l'orbite est la classe de conjugaison : \( C_g=G\cdot g\). Et le stabilisateur de \( g\) pour l'action adjointe n'est autre que le centralisateur de \( g\) :
	\begin{subequations}
		\begin{align}
			\Fix(g) & =\{ h\in G\tq h\cdot g=g \}  \\
			        & =\{ h\in G\tq hgh^{-1}=g \}  \\
			        & =\{ h\in G\tq gh=hg \}       \\
			        & =Z_G(g).
		\end{align}
	\end{subequations}

	Donc la formule \( \Card(G\cdot g)=| G:G_g |\) devient, dans le cas de l'action adjointe de \( G\) sur lui-même : \( \Card(C_g)=| G:Z_G(g) |\).
\end{proof}

\begin{lemma}
	Soit \( G\) un groupe agissant sur l'ensemble \( E\). On définit \( x\sim x'\) si et seulement si il existe \( g\in G\) tel que \( g\cdot x=x'\). Alors
	\begin{enumerate}
		\item
		      la relation \( \sim\) est une relation d'équivalence.
		\item
		      la classe \( [x]\) est l'orbite \( \mO_x\) de \( x\) sous \( G\).
	\end{enumerate}
\end{lemma}

\begin{corollary}[Équation des orbites]\index{équation!des orbites} \label{CorARFVMP}
	Soient \( G\) un groupe agissant sur l'ensemble \( E\) et \( \mO_1,\ldots, \mO_k  \) la liste des orbites (distinctes). Alors
	\begin{enumerate}
		\item
		      \( E=\bigcup_i\mO_i\), l'union est disjointe,
		\item
		      \( \Card(E)=\sum_i\Card(\mO_i)\).
	\end{enumerate}
\end{corollary}

\begin{definition}  \label{DefcSuYxz}
	Soit \( G\) un groupe agissant sur l'ensemble \( E\). Un \defe{domaine fondamental}{domaine!fondamental d'une action}\index{fondamental!domaine d'une action}\index{action!domaine fondamental} ou une \defe{transversale}{transversale} est une partie de \( E\) contenant un et un seul élément de chaque orbite.
\end{definition}
Autrement dit, les images des éléments d'un domaine fondamental \( F\) forment une partition de l'ensemble :
\begin{equation}
	E=\bigsqcup_{g\in G}g(F)
\end{equation}
où  \( g(F) = \phi_g(F) = \{ \phi_g(x) \tq x \in F\} \). L'union est disjointe, c'est-à-dire que si \( g\neq g'\), alors \( g(F)\cap g'(F)=\emptyset\).


\begin{proposition}[Équation des classes\cite{FabricegPSFinis}]       \label{PropUyLPdp}
	Soit \( G\), un groupe fini opérant sur un ensemble \( E\). Si \( E''\) est un ensemble contenant exactement un élément de chaque orbite dans \( E\setminus\Fix_G(E)\), alors
	\begin{equation}        \label{EqobuzfK}
		| G |=| \Fix_G(E) |+\sum_{x\in E''}\frac{ | G | }{ | \Fix_G(x) | }.
	\end{equation}
	Si de plus \( G\) est un $p$-groupe, alors
	\begin{equation}    \label{EqbzLEVJ}
		| E |=| \Fix_G(E) |\mod p.
	\end{equation}
\end{proposition}

\begin{proof}
	Par le corolaire~\ref{CorARFVMP}, nous avons \( | G |=\sum_{x\in E'}| \mO_x |\) où \( E'\) est une transversale.  En séparant la somme entre les orbites à un élément et les autres,
	\begin{equation}    \label{EqeggkBs}
		| G |=\Card(\Fix_G(E))+\sum_{x\in E''}\frac{ | G | }{ | \Fix_G(x) | }
	\end{equation}  \label{EqDgYbhm}
	où nous avons utilisé le fait que \( | G |=| \Fix_G(x) | |\mO_x |\).

	Si \( G\) est un \( p\)-groupe alors si \( x\in E''\), \( \Fix_G(x)\) est un sous-groupe propre de \( G\) et donc \( | \Fix_G(x) |\) est un diviseur propre de \( | G |\). Du coup la fraction \( | G |/|\Fix_G(x)|\) est une puissance non nulle de \( p\) et l'équation \eqref{EqobuzfK} devient immédiatement \eqref{EqbzLEVJ}.
\end{proof}

\begin{corollary}[Équation des classes]\index{équation!des classes}
	Soient \( G\) un groupe, et \( C_1\),\ldots, \( C_l\) la liste de ses classes de conjugaison contenant plus d'un élément. Alors
	\begin{equation}    \label{EqkgGmoq}
		\Card(G)=\Card\big( Z(G) \big)+\sum_i| G:Z_{g_i} |=\Card\big( Z(G) \big)+\sum_i\frac{ \Card(G) }{ \Card\big( \Fix(g_i) \big) }
	\end{equation}
	si \( g_i\in C_i\).
\end{corollary}

\begin{proof}
	Étant donné que les classes de conjugaison sont disjointes, le cardinal de \( G\) est bien la somme des cardinaux de ses classes. Les classes ne contenant qu'un seul élément sont celles des éléments de \( Z(G)\). En ce qui concerne les autres orbites, \( \Card(C_{g_i})=| G:Z_{g_i} |\) par le théorème orbite-stabilisateur (proposition~\ref{Propszymlr}).
\end{proof}

\begin{theorem}[\wikipedia{fr}{Action_de_groupe_(mathématiques)}{Formule de Burnside}]      \label{THOooEFDMooDfosOw}
	Si \( G\) est un groupe fini agissant sur l'ensemble fini \( E\) et si \( \Omega\) est l'ensemble des orbites, alors
	\begin{equation}    \label{EqTUsblv}
		\Card(\Omega)=\frac{1}{ | G | }\sum_{g\in G}\Card\big( \Fix(g) \big).
	\end{equation}
\end{theorem}
\index{Burnisde!formule}
\index{formule!Burnside}

\begin{proof}
	Nous considérons l'ensemble
	\begin{equation}
		A=\{ (g,x)\in G\times E\tq gx=x \},
	\end{equation}
	et nous en calculons le cardinal de deux façons. D'abord
	\begin{subequations}
		\begin{align}
			\Card(A) & =\sum_{x\in E}\Card\{ g\in G\tq gx=x \}                                      \\
			         & =\sum_{x\in E}\Card(\Fix(x))                                                 \\
			         & =\sum_{\omega\in \Omega}\sum_{x\in \omega}\Card(\Fix(x))                     \\
			         & =\sum_{\omega\in \Omega}\frac{ | G | }{ \Card(\omega) }     \label{EqyVtkyf} \\
			         & =| G |.
		\end{align}
	\end{subequations}
	Pour obtenir \eqref{EqyVtkyf} nous avons utilisé l'équation des classes \eqref{EqCewSXT}. L'autre façon de calculer \( \Card(A)\) est de regrouper ainsi :
	\begin{equation}
		\Card(A)=\sum_{g\in G}\Card\{ x\in E\tq gx=x \}=\sum_{g\in G}\Card(\Fix(g)).
	\end{equation}
	En égalisant les deux expressions de \( \Card(A)\) nous trouvons
\begin{equation}
		| G |\Card(\Omega)=\sum_{g\in G}\Card(\Fix(g)).
	\end{equation}
\end{proof}

\begin{proposition}     \label{PROPooMYKKooLetZWi}
	Soient \( G\) un groupe, et \( H\), un sous-groupe du centre de \( G\).
	\begin{enumerate}
		\item
		      \( H\) est normal dans \( G\).
		\item
		      Si \( G/H\) est monogène, alors \( G\) est abélien.
		\item
%TODO: Le centre Z ? pas H plutot ?    Pas de démonstration ?
		      Si \( G\) est fini de centre \( Z\), alors \( | G:H |\) n'est pas premier.
	\end{enumerate}
\end{proposition}

\begin{theorem}     \label{THOooRGSTooIWyhqt}
	Soit \( G\) un groupe cyclique\footnote{Définition \ref{DefHFJWooFxkzCF}.} d'ordre \( n\).
	\begin{enumerate}
		\item
		      Tout sous-groupe de \( G\) est cyclique.
		\item
		      Pour chaque diviseur \( d\) de \( n\), il existe un unique sous-groupe \( H_d\) de \( G\) d'ordre \( d\).
	\end{enumerate}
	Si \( a\) est un générateur de \( G\), alors \( H_d\) peut être décrit des façons suivantes :
	\begin{equation}
		H_d=\{ x\in G\tq x^d=e \}=\{ x\in G\tq\exists y\in G\tq y^{n/d}=x \}=\langle a^{n/d}\rangle.
	\end{equation}
\end{theorem}


\begin{definition}      \label{DEFooQDHPooCfDEuL}
	Soit \( G\) un groupe agissant sur un ensemble \( E\). Nous disons que l'action est \defe{transitive}{transitive}\index{action!transitive} si elle possède une seule orbite. L'action est \defe{libre}{libre!action}\index{action!libre} si \( g\cdot x=g'\cdot x\) implique \( g=g'\).
\end{definition}
