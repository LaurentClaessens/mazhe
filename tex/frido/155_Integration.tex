% This is part of Mes notes de mathématique
% Copyright (c) 2011-2018, 2021-2022
%   Laurent Claessens
% See the file fdl-1.3.txt for copying conditions.

%+++++++++++++++++++++++++++++++++++++++++++++++++++++++++++++++++++++++++++++++++++++++++++++++++++++++++++++++++++++++++++
\section{Surfaces paramétrées}
%+++++++++++++++++++++++++++++++++++++++++++++++++++++++++++++++++++++++++++++++++++++++++++++++++++++++++++++++++++++++++++

De la même façon qu'un chemin dans \( \eR^3\) est décrit comme une application \( \sigma\colon \eR\to \eR^3\), une surface dans \( \eR^3\) sera vue comme une application \( \varphi\colon \eR^2\to \eR^3\). Une \defe{surface paramétrée}{surface paramétrée} dans \( \eR^3\) est une application
\begin{equation}
	\begin{aligned}
		\varphi\colon D\subset\eR^2 & \to \eR^3                                       \\
		(u,v)                       & \mapsto \varphi(u,v)=\begin{pmatrix}
			x(u,v) \\
			y(u,v) \\
			z(u,z)
		\end{pmatrix}.
	\end{aligned}
\end{equation}
Nous allons parler de la «surface \( \varphi\)» pour désigner l'image de \( \varphi\) dans \( \eR^3\).

Si on fixe le paramètre \( u-u_0\), alors l'application
\begin{equation}
	v\mapsto\varphi(u_0,v)
\end{equation}
est un chemin dans la surface. Un vecteur tangent à ce chemin sera tangent à la courbe :
\begin{equation}
	\frac{ \partial \varphi }{ \partial v }(u_0,v_0)=
	\begin{pmatrix}
		\frac{ \partial x }{ \partial v }(u_0,v_O) \\
		\frac{ \partial y }{ \partial v }(u_0,v_O) \\
		\frac{ \partial z }{ \partial v }(u_0,v_O)
	\end{pmatrix}.
\end{equation}
De même, en fixant \( v_0\), on considère le chemin
\begin{equation}
	u\mapsto\varphi(u,v_0).
\end{equation}
Le vecteur tangent à ce chemin est égalent tangent à la surface :
\begin{equation}
	\frac{ \partial \varphi }{ \partial u }=
	\begin{pmatrix}
		\frac{ \partial x }{ \partial u }(u_0,v_O) \\
		\frac{ \partial y }{ \partial u }(u_0,v_O) \\
		\frac{ \partial z }{ \partial u }(u_0,v_O)
	\end{pmatrix}
\end{equation}

\begin{definition}      \label{DefSurfReguliere}
	Nous disons que la surface est \defe{régulière}{régulière!surface} si les vecteurs \( \partial_u\varphi(u_0,v_0)\) et \( \partial_v\varphi(u_0,v_0)\) sont non nuls et non colinéaires.
\end{definition}
Si la surface est régulière, les vecteurs tangents à le paramétrage forment le plan tangent à la surface au point \( \varphi(u_0,v_0)\).

Un vecteur orthogonal à la surface (et donc au plan tangent) est donc donné par le produit vectoriel :
\begin{equation}
	n(u_0,v_0)=\frac{ \partial \varphi }{ \partial u }(u_0,v_0)  \times \frac{ \partial \varphi }{ \partial v }(u_0,v_0).
\end{equation}
L'équation du plan tangent est alors obtenue par
\begin{equation}        \label{EqPlanTgSurfaceParm}
	\begin{pmatrix}
		x-x_0 \\
		y-y_0 \\
		z-z_0
	\end{pmatrix}\cdot n(u_0,v_0)=0
\end{equation}
où \( x_0=x(u_0,v_0)\), \( y_0=y(u_0,v_0)\), \( z_0=z(u_0,v_0)\).

%---------------------------------------------------------------------------------------------------------------------------
\subsection{Graphe d'une fonction}
%---------------------------------------------------------------------------------------------------------------------------

Soit la fonction \( f\colon D\subset\eR^2\to \eR\). Le graphe de \( f\) est l'ensemble des points de la forme
\begin{equation}
	\big( x,y,f(x,y) \big)
\end{equation}
tels que \( (x,y)\in D\). Cela est une surface paramétrée par
\begin{equation}
	\begin{aligned}
		\varphi\colon D & \to \eR^3                           \\
		(x,y)           & \mapsto \begin{pmatrix}
			x \\
			y \\
			f(x,y)
		\end{pmatrix}.
	\end{aligned}
\end{equation}
Les vecteurs tangents sont
\begin{equation}
	\begin{aligned}[]
		\frac{ \partial \varphi }{ \partial x } & =\begin{pmatrix}
			1 \\
			0 \\
			\frac{ \partial f }{ \partial x }
		\end{pmatrix},
		                                        & \frac{ \partial \varphi }{ \partial y } & =\begin{pmatrix}
			0 \\
			1 \\
			\frac{ \partial \varphi }{ \partial y }
		\end{pmatrix}.
	\end{aligned}
\end{equation}
La surface est donc partout régulière parce que ces deux vecteurs ne sont jamais nuls ou colinéaires. Un vecteur normal à cette surface au point \( (x_0,y_0,f(x_0,y_0))\) est donné par le produit vectoriel
\begin{equation}
	n=\begin{vmatrix}
		e_x & e_y & e_z                  \\
		1   & 0   & \partial_xf(x_0,y_0) \\
		0   & 1   & \partial_yf(x_0,y_0)
	\end{vmatrix}
	=-\frac{ \partial f }{ \partial x }(x_0,y_0)e_x-\frac{ \partial f }{ \partial y }(x_0,y_0)e_y+e_z.
\end{equation}
En suivant l'équation \eqref{EqPlanTgSurfaceParm}, nous avons l'équation suivante pour le plan :
\begin{equation}
	\begin{pmatrix}
		x-x_0 \\
		y-y_0 \\
		z-z_0
	\end{pmatrix}\cdot
	\begin{pmatrix}
		-\frac{ \partial f }{ \partial x }(x_0,y_0) \\
		-\frac{ \partial f }{ \partial y }(x_0,y_0) \\
		1
	\end{pmatrix}=0,
\end{equation}
c'est-à-dire
\begin{equation}
	-(x-x_0)\frac{ \partial f }{ \partial x }(x_0,y_0)-(y-y_0)\frac{ \partial f }{ \partial y }(x_0,y_0)+z-f(x_0,y_0)=0,
\end{equation}
ce qui revient à
\begin{equation}
	z-f(x_0,y_0)=\frac{ \partial f }{ \partial x }(x_0,y_0)(x-x_0)+\frac{ \partial f }{ \partial y }(x_0,y_0)(y-y_0).
\end{equation}
Nous retrouvons donc l'équation du plan tangent à un graphe.

\begin{example}
	La sphère de rayon \( R\) peut être paramétrée par les angles sphériques :
	\begin{equation}
		\phi(\theta,\varphi)=\begin{pmatrix}
			R\sin\theta\cos\varphi \\
			R\sin\theta\sin\varphi \\
			R\cos\theta
		\end{pmatrix}
	\end{equation}
	avec \( (\theta,\varphi)\in\mathopen[ 0 , \pi \mathclose]\times \mathopen[ 0 , 2\pi \mathclose]\).

	Tentons d'en trouver le plan tangent au point \( (x,y,z)=(R,0,0)\). Un petit dessin nous montre que c'est un plan vertical d'équation \( x=R\). Montrons cela en utilisant la théorie que nous venons de découvrir. D'abord le point \( (R,0,0)\) correspond à \( \theta_0=\frac{ \pi }{ 2 }\) et \( \varphi=0\). Les vecteurs tangents sont
	\begin{equation}        \label{EqTthetaSph}
		T_{\theta}=\frac{ \partial \phi }{ \partial \theta }(R,\frac{ \pi }{2},0)=\begin{pmatrix}
			R\cos\theta\cos\varphi \\
			R\cos\theta\sin\varphi \\
			-R\sin\theta
		\end{pmatrix}=\begin{pmatrix}
			0 \\
			0 \\
			-R
		\end{pmatrix},
	\end{equation}
	et
	\begin{equation}    \label{EqTvarphiSph}
		T_{\varphi}=\frac{ \partial \phi }{ \partial \varphi }(R,\frac{ \pi }{2},0)=\begin{pmatrix}
			-R\sin\theta\sin\varphi \\
			R\sin\theta\cos\varphi  \\
			0
		\end{pmatrix}=\begin{pmatrix}
			0 \\
			R \\
			0
		\end{pmatrix}.
	\end{equation}
	Cela sont de toute évidence bien les deux vecteurs tangents à la sphère au point \( (x,y,z)=(R,0,0)\). Le vecteur normal est
	\begin{equation}
		\begin{vmatrix}
			e_x & e_y & e_z \\
			0   & 0   & -R  \\
			0   & R   & 0
		\end{vmatrix}=R^2e_x.
	\end{equation}
	Ici encore, nous avons le vecteur que nous attendions sur un dessin. L'équation du plan tangent est maintenant
	\begin{equation}
		\begin{pmatrix}
			x-R \\
			y   \\
			z
		\end{pmatrix}\cdot
		\begin{pmatrix}
			R^2 \\
			0   \\
			0
		\end{pmatrix}=0,
	\end{equation}
	c'est-à-dire \( R^2(x-R)=0\) et donc \( x=R\).
\end{example}

%---------------------------------------------------------------------------------------------------------------------------
\subsection{Intégrale sur une partie de \( \eR^m\)}
%---------------------------------------------------------------------------------------------------------------------------

Soit \( M\) une variété de dimension \( n\) dans \( \eR^m\). Soit \( F : W \subset \eR^n \to M\) un paramétrage d'un ouvert relatif de \( M\).

Si \( f\) est une fonction définie sur un sous-ensemble \( A \subset F(W)\) tel que \( F^{-1}(A)\) est mesurable, l'\Defn{intégrale de \( f\) sur \( A\)} est définie par
\begin{equation*}
	\int_A f = \int_{F^{-1}(A)} f(F(w)) \sqrt{\det(J_F(w)^t {J_F(w)})} dw
\end{equation*}
où l'intégrale est l'intégration usuelle (de Lebesgue) sur \( F^{-1}(A) \subset \eR^n\). On écrit parfois cette intégrale \( \int_{F^{-1}(A)} f(F(w)) d\sigma\) où
\begin{equation}        \label{EQooARMAooQPhQAL}
	d\sigma = \sqrt{\det(J_F(w)^t {J_F(w)})} dw
\end{equation}
est l'\Defn{élément infinitésimal de volume} de la variété.

Si \( m = 3\) et \( n = 2\), l'élément infinitésimal de volume vaut
\begin{equation*}
	d \sigma = \norme{\pder F {w_1} \times \pder F {w_2}} dw
\end{equation*}
où \( \times\) représente le produit vectoriel dans \( \eR^3\), et \( (w_1,w_2)\) sont les coordonnées sur \( W \subset \eR^2\). Dans la suite, nous ne regarderons plus que ce cas.

%+++++++++++++++++++++++++++++++++++++++++++++++++++++++++++++++++++++++++++++++++++++++++++++++++++++++++++++++++++++++++++
\section{Intégrales de surface}
%+++++++++++++++++++++++++++++++++++++++++++++++++++++++++++++++++++++++++++++++++++++++++++++++++++++++++++++++++++++++++++
\label{secintsurfaciques}

\subsection{Intégrale d'un champ de vecteurs}
Dans l'intégration curviligne, on a noté que si l'intégrale d'une fonction ne dépendait pas de l'orientation du chemin, l'intégrale d'un champ de vecteurs ou d'une forme différentielle en dépendait. Ce problème d'orientation apparait également dans l'intégration sur des surfaces de l'espace.

\begin{definition}      \label{DEFooFTQLooXXbtOQ}
	Une \defe{orientation}{orientation} sur une surface \( S \subset \eR^3\) est le choix d'un champ de vecteurs continu \( \nu : S \to \eR^3\) dont la norme en tout point de \( S\) vaut \( 1\).
\end{definition}

On remarque qu'ayant fait un tel choix
d'orientation \( \nu(x)\) en un point \( x\), le seul autre choix possible
en \( x\) est \( -\nu(x)\).
%% Page
Si \( S\) est le bord d'un ouvert \( D \subset \eR^3\), l'\Defn{orientation
	induite par \( D\) sur \( S\)} est, si elle existe, l'orientation qui
pointe hors de \( D\) en tout point de \( S\). Plus précisément, il faut que
pour tout \( x \in D\) il existe \( \epsilon > 0\) vérifiant, pour tout \( 0 <
t < \epsilon\), la relation \( t \nu(x) \notin D\). Dans ce cas, le champ
de vecteurs \( \nu\) est appelé le \Defn{vecteur normal unitaire
	extérieur} à \( D\) et il est forcément unique.

Soit \( G\) un champ de vecteurs défini sur une surface orientée par un
champ \( \nu\). L'intégrale de \( G\) sur \( S\), aussi appelée le \Defn{flux
	de \( G\) à travers \( S\)}, est
\begin{equation}\label{eqflux-star}
	\iint_S G \cdot d S \pardef \iint_S \scalprod{G}{\nu} d \sigma.
\end{equation}
Si on suppose que la surface est paramétrée par une application
\begin{equation*}
	F : W \subset \eR^2 \to \eR^3 : (u,v) \mapsto (F_1(u,v),F_2(u,v),F_3(u,v))
\end{equation*}
alors un vecteur unitaire \( \nu\) peut s'écrire sous la forme
\begin{equation*}
	\nu = \frac{\pder F u \times \pder F v}{\norme{\pder F u \times \pder F v}}
\end{equation*}
et grâce à ce paramétrage l'intégrale \eqref{eqflux-star}
devient
\begin{equation*}
	\iint_S G \cdot d S = \iint_W \scalprod{G(F(u,v))}{\pder F u \times \pder F v} d u
	d v.
\end{equation*}
où on utilise l'expression de \( d \sigma\) obtenue précédemment dans le cas qui nous intéresse (surface dans l'espace).

%+++++++++++++++++++++++++++++++++++++++++++++++++++++++++++++++++++++++++++++++++++++++++++++++++++++++++++++++++++++++++++
\section{Intégrales de surface}
%+++++++++++++++++++++++++++++++++++++++++++++++++++++++++++++++++++++++++++++++++++++++++++++++++++++++++++++++++++++++++++

%---------------------------------------------------------------------------------------------------------------------------
\subsection{Aire d'une surface paramétrée}
%---------------------------------------------------------------------------------------------------------------------------

Lorsque nous avions vu la longueur d'une courbe paramétrée, nous avions pris comme «élément de longueur» la norme du vecteur tangent. Il est donc naturel de prendre comme «élément de surface» une petite surface que l'on peut construire à partir des deux vecteurs tangents à la surface.

Au point \( \varphi(u_0,v_0)\), nous avons les deux vecteurs tangents
\begin{equation}
	\begin{aligned}[]
		T_u & =\frac{ \partial \varphi }{ \partial u }(u_0,v_0) & T_v & =\frac{ \partial \varphi }{ \partial v }(u_0,v_0).
	\end{aligned}
\end{equation}
L'élément de surface que nous pouvons construire à partir de ces deux vecteurs est la surface du parallélogramme, donnée par la norme du produit vectoriel :
\begin{equation}        \label{EQooNYWSooZuvcPe}
	dS=\| T_u\times T_v \|.
\end{equation}

L'aire de la surface donné par \( \varphi\colon D\subset\eR^2\to \eR^3\) sera donc donnée par
\begin{equation}
	Aire\big( \varphi(D) \big)=\iint_D\| T_u\times T_v \|du\,dv.
\end{equation}

\begin{example}
	Calculons l'aire de la sphère. Les vecteurs tangents ont déjà été calculés aux équations \eqref{EqTthetaSph} et \eqref{EqTvarphiSph} :
	\begin{equation}
		\begin{aligned}[]
			T_{\theta} & =\begin{pmatrix}
				R\cos\theta\cos\varphi \\
				R\cos\theta\sin\varphi \\
				-R\sin\theta
			\end{pmatrix},
			           & T_{\varphi}                  & =\begin{pmatrix}
				-R\sin\theta\sin\varphi \\
				R\sin\theta\cos\varphi  \\
				0
			\end{pmatrix}.
		\end{aligned}
	\end{equation}
	Le produit vectoriel vaut
	\begin{equation}
		\begin{aligned}[]
			T_{\theta}\times T_{\varphi} & =
			\begin{vmatrix}
				e_x                     & e_y                    & e_z          \\
				R\cos\theta\cos\varphi  & R\cos\theta\sin\varphi & -R\sin\theta \\
				-R\sin\theta\sin\varphi & R\sin\theta\cos\varphi & 0
			\end{vmatrix}                                                                                             \\
			                             & =(R^2\sin^2\theta\cos\varphi)e_x+(R^2\sin^2\theta\sin\varphi)e_y                        \\
			                             & \quad +(R^2\cos\theta\sin\theta\cos^2\varphi+R^2\sin\theta\cos\theta \sin^2\varphi)e_z.
		\end{aligned}
	\end{equation}
	La norme demande quelques calculs et mises en évidences. Le résultat est :
	\begin{equation}        \label{EqProdVectTTSPh}
		\| T_{\theta}\times T_{\varphi} \|=R^2\sin\theta.
	\end{equation}
	L'aire de la sphère est donc donnée par
	\begin{equation}
		Aire=\int_0^{2\pi}d\varphi\int_0^{\pi} R^2\sin\theta d\theta=2\pi R^2[-\cos\theta]_0^{\pi}=4\pi R^2.
	\end{equation}

	Il est bon de se souvenir que, en coordonnées sphériques,
	\begin{equation}
		\| T_{\theta}\times T_{\varphi} \|=R^2\sin\theta.
	\end{equation}
	Or nous savons que ce vecteur est dirigé dans le sens de \( e_r\) parce que ce dernier est le vecteur qui est constamment dirigé radialement. En coordonnées sphériques nous avons donc
	\begin{equation}        \label{EqNormalEnSpeh}
		T_{\theta}\times T_{\varphi}=R^2\sin(\theta)e_r.
	\end{equation}

\end{example}

\begin{remark}
	L'équation \eqref{EqProdVectTTSPh} donne l'élément de surface pour la sphère. Notez que cela est justement l'expression du jacobien des coordonnées sphériques. Cela n'est évidemment pas une coïncidence.
\end{remark}

\begin{example}
	Nous pouvons donner l'aire du graphe d'une fonction quelconque. La surface est paramétrée par
	\begin{equation}
		\varphi(x,y)=\begin{pmatrix}
			x \\
			y \\
			f(x,y)
		\end{pmatrix}.
	\end{equation}
	Les vecteurs tangents sont
	\begin{equation}
		\begin{aligned}[]
			T_x & =\begin{pmatrix}
				1 \\
				0 \\
				\partial_xf
			\end{pmatrix}, & T_y & =\begin{pmatrix}
				0 \\
				1 \\
				\partial_yf
			\end{pmatrix}.
		\end{aligned}
	\end{equation}
	Le produit vectoriel est donné par
	\begin{equation}
		T_x\times T_y=\begin{vmatrix}
			e_x & e_y & e_z         \\
			1   & 0   & \partial_xf \\
			0   & 1   & \partial_yf
		\end{vmatrix}=(-\partial_xf)e_x-(\partial_yf)e_y+e_z.
	\end{equation}
	L'élément de surface est par conséquent
	\begin{equation}
		dS=\sqrt{\left( \frac{ \partial f }{ \partial x } \right)^2+\left( \frac{ \partial f }{ \partial y } \right)^2+1},
	\end{equation}
	et la surface du graphe sera
	\begin{equation}
		Aire=\iint_D\sqrt{\left( \frac{ \partial f }{ \partial x }(x,y) \right)^2+\left( \frac{ \partial f }{ \partial y }(x,y) \right)^2+1}\,dx\,dy
	\end{equation}

\end{example}

%---------------------------------------------------------------------------------------------------------------------------
\subsection{Intégrale d'une fonction sur une surface}
%---------------------------------------------------------------------------------------------------------------------------

Si \( S\) est une surface dans \( \eR^3\) paramétrée par
\begin{equation}
	\begin{aligned}
		\varphi\colon D & \to \eR^3                  \\
		(u,v)           & \mapsto \varphi(u,v)\in S,
	\end{aligned}
\end{equation}
et si \( f\) est une fonction \( f\colon \eR^3\to \eR\) définie au moins sur \( S\), l'intégrale de \( f\) sur \( S\) est logiquement définie par
\begin{equation}
	\int_S f\,dS=\iint_D f\big( \varphi(u,v) \big)\| T_u(u,v)\times T_v(u,v) \|dudv
\end{equation}
où \( T_u=\frac{ \partial \varphi }{ \partial u }\) et \( Y_v=\frac{ \partial \varphi }{ \partial v }\). La quantité
\begin{equation}
	\| T_u(u,v)\times T_v(u,v) \|dudv
\end{equation}
est appelé \defe{élément de surface}{element@élément!de surface}.

Encore une fois, si on prend \( f=1\), alors on retrouve la surface de \( S\) :
\begin{equation}
	\int_SdS=Aire(S).
\end{equation}

\begin{remark}
	Le nombre \( \int_SfdS\) ne dépend pas de le paramétrage choisie pour \( S\).
\end{remark}


%---------------------------------------------------------------------------------------------------------------------------
\subsection{Intégrale d'une \( 2\)-forme}
%---------------------------------------------------------------------------------------------------------------------------

Nous considérons \( \omega\), une \( 2\)-forme différentielle sur \( \eR^2\).
\begin{definition}
	Si \( \omega_{(x,y)}=u(x,y)dx\wedge dy\) et si \( D\) est un ouvert de \( \eR^2\) alors nous définissons
	\begin{equation}
		\int_D\omega=\int_D u(x,y)dx\,dy.
	\end{equation}
\end{definition}

Nous voulons maintenant intégrer une \( 2\)-forme sur une surface dans \( \eR^3\). Soit \( S\subset \eR^3\), une surface orientée (c'est-à-dire que nous avons un choix continu d'un vecteur normal unitaire \( n\)). Nous supposons de plus avoir un paramétrage \( \phi\colon D\to S\) de \( S\) avec \( D\) ouvert dans \( \eR^2\) compatible avec l'orientation, c'est-à-dire que pour tout \( (t,s)\in D\),
\begin{equation}
	n\big( \phi(t,s) \big)=\frac{ \partial \phi }{ \partial t }(t,s)\times \frac{ \partial \phi }{ \partial s }(t,s).
\end{equation}

\begin{definition}
	Pour intégrer \( \omega\) sur \( S\) nous faisons
	\begin{equation}
		\int_S\omega=\int_D\phi^*\omega
	\end{equation}
	où \( \phi^*\omega\) est de la forme \( F(t,s)dt\wedge ds\).
\end{definition}
Montrons ce que cela fait. Soient \( u,v\) des vecteurs de \( D\) et calculons
\begin{subequations}
	\begin{align}
		(\phi^*\omega)(u,v) & =\omega\big( d\phi(u),d\phi(v) \big)                                                                                                                                                                    \\
		                    & =\omega\left(    u_1\frac{ \partial \phi }{ \partial x_1 }+u_2\frac{ \partial \infty }{ \partial x_2 },v_1\frac{ \partial \phi }{ \partial x_1 }+v_2\frac{ \partial \infty }{ \partial x_2 }   \right).
	\end{align}
\end{subequations}
Les termes en \( u_1v_1\) et \( u_2v_2\) sont nuls; par exemple :
\begin{equation}
	\omega\left( u_1\frac{ \partial \phi }{ \partial x_1 },v_1\frac{ \partial \phi }{ \partial x_1 } \right)=u_1v_1\omega\left( \frac{ \partial \infty }{ \partial x_1 },\frac{ \partial \phi }{ \partial x_1 } \right)=0
\end{equation}
parce que \( \omega\) est antisymétrique. Il nous reste donc
\begin{subequations}
	\begin{align}
		(\phi^*\omega)(u,v) & =(u_1v_2-u_2v_1)\omega\left( \frac{ \partial \phi }{ \partial x_1 },\frac{ \partial \phi }{ \partial x_2 } \right)     \\
		                    & =\omega\left( \frac{ \partial \phi }{ \partial x_1 },\frac{ \partial \phi }{ \partial x_2 } \right)(dt\wedge ds)(u,v).
	\end{align}
\end{subequations}
Cette dernière ligne est bien de la forme \( \phi^*\omega=F(t,s)dt\wedge ds\).

%+++++++++++++++++++++++++++++++++++++++++++++++++++++++++++++++++++++++++++++++++++++++++++++++++++++++++++++++++++++++++++
\section{Flux d'un champ de vecteurs à travers une surface}
%+++++++++++++++++++++++++++++++++++++++++++++++++++++++++++++++++++++++++++++++++++++++++++++++++++++++++++++++++++++++++++

Nous voulons construire un moulin à eau. Comment placer les pales pour maximiser le travail de la pression de l'eau ? On n'a pas attendu l'invention du calcul intégral pour répondre à cette question. Trois paramètres rentrent en ligne de compte :
\begin{enumerate}
	\item
	      plus il y a d'eau, plus ça pousse;
	\item
	      plus la surface de la palle est grande, plus on va utiliser d'eau;
	\item
	      plus la palle est perpendiculaire au courant, plus on va en profiter.
\end{enumerate}
Nous voyons sur la figure~\ref{LabelFigUUNEooCNVOOs} que lorsque la palle du moulin est inclinée, non seulement elle prend moins d'eau sur elle, mais qu'en plus elle la prend avec un moins bon angle : une partie de la force ne sert pas à la faire tourner.

\newcommand{\CaptionFigUUNEooCNVOOs}{La partie rouge de la force est perdue si l'eau ne pousse pas perpendiculairement. De plus lorsque la palle est inclinée, elle prend moins d'eau sur elle.}
\input{auto/pictures_tex/Fig_UUNEooCNVOOs.pstricks}
%See also the subfigure~\ref{LabelFigUUNEooCNVOOsssLabelSubFigUUNEooCNVOOs0}
%See also the subfigure~\ref{LabelFigUUNEooCNVOOsssLabelSubFigUUNEooCNVOOs1}

L'idée du flux d'un champ de vecteurs à travers une surface est de savoir quelle est la quantité «utile» de vecteurs qui traverse la surface. Ce sera simplement l'intégrale sur la surface de la composante du champ de vecteurs normale à la surface. Il reste deux problèmes à régler : le premier est de savoir quel est le vecteur normal à la surface, et le second est de savoir comment «sélectionner» la composante normale d'un champ de vecteurs \( F\).

Le problème de trouver un vecteur normal est résolu par le produit vectoriel des vecteurs tangents. Si la surface est donnée par \( \varphi\colon D\subset\eR^2\to \eR^3\), les vecteurs tangents sont \( T_u=\partial_u\varphi(u,v)\) et  \( T_v=\partial_v\varphi(u,v)\). Le normal de norme \( 1\) est donné par :
\begin{equation}
	n(u,v)=\frac{ T_u\times T_v }{ \| T_u\times T_v \| }.
\end{equation}

Si \( p\) est un point de la surface \( \varphi(D)\), la composante de \( F(p)\) qui est normale à la surface au point \( p\) est donnée par le produit scalaire
\begin{equation}
	F(p)_{\perp}=F(p)\cdot n(p).
\end{equation}
C'est ce nombre là que nous intégrons sur la surface.

\begin{definition}
	Le \defe{flux du champ de vecteurs}{flux d'un champ de vecteurs} à travers la surface \( S=\varphi(D)\) est
	\begin{equation}
		\int F\cdot dS=\int F \cdot n\,dudv.
	\end{equation}
\end{definition}

Une petite simplification se produit lorsqu'on veut calculer effectivement cette intégrale. En effet \( F\cdot n\) est, en soi, une fonction sur \( S\). Pour l'intégrer, il faut donc la multiplier par \( \| T_u\times T_v \|\) (c'est la définition de l'intégrale d'une fonction sur une surface). Donc, étant donné que \( n=(T_u\times T_v)/\| T_u\times T_v \|\), nous avons
\begin{equation}
	\int F\cdot dS=\iint_D F\big( \varphi(u,v) \big)\cdot (T_u\times T_v)\,dudv
\end{equation}
où \( T_u=\frac{ \partial \phi }{ \partial u }\) et \( T_v=\frac{ \partial \varphi }{ \partial v }\).


\begin{example}
	Soit le champ de vecteurs
	\begin{equation}
		F=\begin{pmatrix}
			2x \\
			2y \\
			2z
		\end{pmatrix}.
	\end{equation}
	Calculons son flux au travers de la sphère de rayon \( R\).

	Nous choisissons de paramétrer la sphère en coordonnées sphériques avec \( \phi(\theta,\varphi)\). Nous pouvons reprendre le résultat \eqref{EqNormalEnSpeh} :
	\begin{equation}
		T_{\theta}\times T_{\varphi}=R^2\sin(\theta).
	\end{equation}
	Nous savons aussi que
	\begin{equation}
		F\big( \phi(\theta,\varphi) \big)=2e_r.
	\end{equation}
	L'intégrale à calculer est donc
	\begin{equation}
		I=\int_0^{\pi}d\theta\int_0^{2\pi}d\varphi\, 2e_r\cdot\big( R^2\sin(\theta)e_r \big).
	\end{equation}
	Vu que le produit scalaire \( e_r\cdot e_r\) vaut \( 1\), nous calculons
	\begin{equation}
		\begin{aligned}[]
			I=4\pi R^2\int_0^{\pi}\sin(\theta)d\theta=8\pi R^2.
		\end{aligned}
	\end{equation}

\end{example}

\begin{example}
	Calculons le flux du champ de force de gravitation d'une masse au travers de la sphère de centre \( R\) centrée autour la masse. À un coefficient constant près, le champ vaut
	\begin{equation}
		G(r,\theta,\varphi)=\frac{1}{ r^2 }e_r.
	\end{equation}
	Sur la sphère de rayon \( R\), nous avons
	\begin{equation}
		G\big( \phi(\theta,\varphi) \big)=\frac{1}{ R^2 }e_r.
	\end{equation}
	L'intégrale est donc
	\begin{equation}
		\int_0^{\pi}d\theta\int_0^{2\pi}\frac{1}{ R^2 }e_r\cdot \big( R^2\sin(\theta)e_r \big)d\varphi=8\pi.
	\end{equation}
	Ce flux ne dépend pas de \( R\).
\end{example}

\begin{example}
	Soit \( S\) le disque de rayon \( 5\) placé horizontalement à la hauteur \( 12\). Calculer le flux du champ de vecteurs
	\begin{equation}
		F(x,y,z)=xe_x+ye_y+ze_z.
	\end{equation}
	Les équations de la surface sont \( z=12\), \( x^2+y^2\leq 25\). Nous prenons le paramétrage en coordonnées cylindriques :
	\begin{equation}
		\varphi(r,\theta)=\begin{pmatrix}
			r\cos(\theta) \\
			r\sin(\theta) \\
			12
		\end{pmatrix}.
	\end{equation}
	Les vecteurs tangents sont
	\begin{equation}
		\begin{aligned}[]
			T_r=\frac{ \partial \varphi }{ \partial r } & =\begin{pmatrix}
				\cos\theta \\
				\sin\theta \\
				0
			\end{pmatrix} & T_{\theta}=\frac{ \partial \varphi }{ \partial \theta } & =\begin{pmatrix}
				-r\sin\theta \\
				r\cos\theta  \\
				0
			\end{pmatrix}.
		\end{aligned}
	\end{equation}
	Le vecteur normal est alors
	\begin{equation}
		T_r\times T_{\theta}=re_z.
	\end{equation}
	Sur la surface, le champ de vecteurs s'écrit
	\begin{equation}
		F\big( \varphi(r,\theta) \big)=r\cos(\theta)e_x+r\sin(\theta)e_y+12e_z.
	\end{equation}
	Par conséquent
	\begin{equation}
		F\cdot(T_r\times T_{\theta})=12r.
	\end{equation}
	L'intégrale à calculer est
	\begin{equation}
		\begin{aligned}[]
			\int_0^5dr\int_0^{2\pi}12r\,d\theta & =12\cdot 2\pi\int_0^5r\,dr \\
			                                    & =\frac{ 25 }{ 2 }24\pi     \\
			                                    & =300\pi.
		\end{aligned}
	\end{equation}

\end{example}

%+++++++++++++++++++++++++++++++++++++++++++++++++++++++++++++++++++++++++++++++++++++++++++++++++++++++++++++++++++++++++++
\section{Divergence, Green, Stokes}
%+++++++++++++++++++++++++++++++++++++++++++++++++++++++++++++++++++++++++++++++++++++++++++++++++++++++++++++++++++++++++++

Le théorème de Stokes (et ses variations) peut se voir comme une généralisation du théorème fondamental du calcul différentiel et intégral qui stipule que
\begin{equation*}
	\int_a^b f^\prime(x) d x = f(b) - f(a)
\end{equation*}
c'est-à-dire qui relie l'intégrale de \( f^\prime\) sur \( I = [a,b]\) aux valeurs de \( f\) sur le bord \( \partial I = \{a,b\}\). Le signe \( -\) qui apparait vient de l'orientation ; celle-ci requiert de la prudence dans l'utilisation des théorèmes.

Voici, pour votre culture générale, un énoncé général :
\begin{theorem} \label{ThoATsPuzF}
	Si \( M\) est une variété orientable de dimension \( n\) avec un bord noté \( \partial  M\), alors pour toute forme différentielle \( \omega\) de degré \( n-1\) on a
	\begin{equation*}
		\int_{ M} d \omega = \int_{\partial  M} \omega.
	\end{equation*}
	où \( d \omega\) désigne la différentielle extérieure de \( \omega\).
\end{theorem}
Nous allons maintenant voir quelques cas particuliers.

Une des nombreuses formes du théorème de Stokes (théorème~\ref{ThoATsPuzF}) est que si la forme différentielle \( \omega\) est exacte alors son intégrale est facile.
\begin{theorem}[\cite{MonCerveau}] \label{ThoUJMhFwU}
	Si \( \gamma\) est une chemin de classe \( C^1\) dans un ouvert \( \Omega\) et si \( \omega\) est la forme différentielle exacte \( \omega=df\), alors
	\begin{equation}
		\int_{\gamma}df=f\big( \gamma(1) \big)-f\big( \gamma(0) \big).
	\end{equation}
\end{theorem}

\begin{proof}
	C'est une application du lemme \ref{LEMooKNBVooQSowos} et du théorème fondamental du calcul intégral \ref{ThoRWXooTqHGbC}.
\end{proof}

\subsection{Théorème de la divergence}

Si nous considérons une surface dans \( \eR^n\) et un champ de vecteurs, il est bon de se demander quelle « quantité de vecteurs » traverse la surface. Soit \( D\), un ouvert borné de \( \eR^n\) telle que \( \partial D\) soit une variété de dimension \( n-1\), et \( G\), un champ de vecteurs défini sur \( \bar D\). Afin de compter combien de \( G\) traverse \( \partial D\), il faudra faire en sorte de ne considérer que la composante de \( G\) normale à \( \partial D\) : pas question d'intégrer par exemple la norme de \( G\) sur \( \partial D\).

Comme nous le savons, la composante du vecteur \( v\) dans la direction \( w\) est le produit scalaire \( v\cdot 1_w\) où \( 1_w\) est le vecteur de norme \( 1\) dans la direction \( w\). Nous allons donc introduire le concept de vecteur normal extérieur.

\begin{definition}
	Soit \( x\in\partial D\) et \( \nu\in\eR^n\), nous disons que \( \nu\) est un \defe{vecteur normal extérieur}{normal extérieur!vecteur} de \( \partial D\) si
	\begin{enumerate}

		\item
		      \( \langle \nu, v\rangle =0\) pour tout vecteur tangent \( v\) à \( \partial D\) au point \( x\). Pour rappel, \( \partial D\) étant une variété de dimension \( n-1\), il y a \( n-1\) tels vecteurs \( v\) linéairement indépendants.

		\item
		      Il existe un \( \delta>0\) tel que \( \forall t\in\mathopen] 0 , \delta \mathclose[\), nous avons \( c+t\nu\notin \bar D\) et \( x-t\nu\in D\).
	\end{enumerate}
\end{definition}

Nous pouvons maintenant définir le concept de flux. Soit \( D\subset \eR^n\) tel que \( \partial D\) soit une variété de dimension \( n-1\) qui admette un vecteur normal extérieur \( \nu(x)\) en chaque point. Soit aussi \( G\colon \bar D\to \eR^n\), un champ de vecteur de classe \( C^1\). Le \defe{flux}{flux!d'un champ de vecteur} de \( G\) au travers de \( \partial D\) est le nombre
\begin{equation}
	\int_{\partial D}\langle G(x), \nu(x)\rangle d\sigma(x).
\end{equation}

Cette intégrale est en général très compliquée à calculer parce qu'il faut trouver le champ de vecteur normal, puis un paramétrage de la surface \( \partial D\) et ensuite appliquer la méthode décrite au point~\ref{secintsurfaciques}.

Heureusement, il y a un théorème qui nous permet de calculer plus facilement : sans devoir trouver de vecteurs normaux.

Il n'est pas plus contraignant d'énoncer ce théorème dans le cadre d'une hypersurface de \( \eR^n\), ce que nous faisons donc~:
\begin{theorem}[Formule de la divergence]
	Soit \( D\) un ouvert borné de \( \eR^n\) dont le bord est « assez régulier par morceaux », c'est-à-dire~:
	\begin{equation}
		\partial D = A_1 \cup \ldots A_p \cup N
	\end{equation}
	où
	\begin{enumerate}
		\item \( A_1, \ldots, A_p, N\) sont deux à deux disjoints,
		\item pour tout \( i \leq p\), \( A_i\) est un ouvert relatif d'une certaine variété \( M_i\) de dimension \( (n-1)\)
		\item \( \bar A_i \subset M_i\)
		\item \( N\) est un compact contenu dans une réunion finie de variétés de dimensions \( (n-2)\).
	\end{enumerate}
	Supposons également qu'en chaque point de \( A_1 \cup \ldots \cup A_p\) il existe un vecteur normal extérieur \( \nu\).

	Si \( G\) est un champ de vecteurs de classe \( C^1\) sur \( \bar D\) alors
	\begin{equation}
		\int_D \nabla\cdot G = \sum_{i=1}^p \int_{A_i} \scalprod{G}{\nu}.
	\end{equation}
	L'intégrale du membre de gauche est l'intégrale sur un ouvert d'une simple fonction.
\end{theorem}

\subsection{Lacets et homotopie}

\begin{definition}[chemin, lacet\cite{BIBooQKARooMHqitK,ooTXKNooIgJrPw}]       \label{DEFooQZMSooYYkGDv}
    Plusieurs notions autour de chemins dans un espace topologique \( X\).
    \begin{enumerate}
        \item
    Un \defe{chemin}{chemin} dans un espace vectoriel \( X\) est une application continue \( \gamma\colon \mathopen[ a , b \mathclose]\to X\) avec \( a<b\).
        \item
    Un \defe{lacet}{lacet} dans \( X\) est un chemin \( \gamma\colon \mathopen[ a , b \mathclose]\to X\) tel que \( \gamma(a)=\gamma(b)\).
\item
    Un chemin \( \gamma\colon \mathopen[ 0 , 1 \mathclose]\to \eR^n\) est \defe{régulier}{chemin régulier} si il est \( C^1\) et si \( \gamma'(t)\neq 0\) pour tout \( t\)
\item
    Le chemin \( \gamma\colon \mathopen[ a , b \mathclose]\to X\) est un \defe{chemin de Jordan}{chemin de Jordan} si il est injective.
\item
    Un lacet \( \gamma\colon \mathopen[ a , b \mathclose]\to X\) est un \defe{lacet de Jordan}{lacet de Jordan} si \( \gamma\colon \mathopen[ a , b \mathclose[\to X\) est injective.
        \item
            Une \defe{courbe de Jordan}{courbe de Jordan} est l'image d'un lacet de Jordan.
        \item 
            Une \defe{courbe simple}{courbe simple} est l'image d'un lacet de Jordan.
    \end{enumerate}
    L'homotopie sera définie en \ref{DEFooHJQTooYUFcee}.
\end{definition}


\begin{normaltext}
    Un chemin de Jordan peut évidemment être vue comme une application \( \gamma\colon \mathopen[ 0 , 2\pi ]\to \eR^2\) telle que \( \gamma(0)=\gamma(2\pi)\). En particulier il n'est jamais mauvais de se rappeler qu'on peut choisir un paramétrage normal par la proposition~\ref{PropExisteChmNorm}.
\end{normaltext}

\begin{definition}[homotopie de lacets]     \label{DEFooHJQTooYUFcee}
    Soit un espace topologique \( X\). Les lacets\footnote{Définition \ref{DEFooQZMSooYYkGDv}.} \( \gamma_0,\gamma_1\colon \mathopen[ a , b \mathclose]\to X\) sont \defe{homotopes}{homotopie} dans \( X\) si il existe une application continue
	\begin{equation}
		H\colon \mathopen[ 0 , 1 \mathclose]\times \mathopen[ a , b \mathclose]\to Y
	\end{equation}
	telle que
	\begin{enumerate}
        \item
            Pour tout \( s\in\mathopen[ 0 , 1 \mathclose]\), l'application
            \begin{equation}
                \begin{aligned}
                    \Gamma_s\colon \mathopen[ a , b \mathclose]&\to X \\
                    t&\mapsto H(s,t) 
                \end{aligned}
            \end{equation}
            est un lacet dans \( X\).
        \item
            Nous avons \( \Gamma_0=\gamma_0\) et \( \Gamma_1=\gamma_1\).
	\end{enumerate}
	L'application \( H\) est l'homotopie entre \( \gamma_0\) et \( \gamma_1\).
\end{definition}

\begin{definition}[Homotopie à extrémités fixées\cite{BIBooQKARooMHqitK}]   \label{DEFooLXDTooDPgxqL}
    Soit un espace topologique \( X\). Soient des chemins \( \gamma_0,\gamma_1\colon \mathopen[ a , b \mathclose]\to X\). Nous supposons que leurs extrémités soient égales, et nous les notons \( p,q\) : \( \gamma_0(a)=\gamma_1(a)=p\) et \( \gamma_0(b)=\gamma_1(b)=q\). Nous disons que \( \gamma_0\) et \( \gamma_1\) sont \defe{homotopes avec les extrémités fixés}{homotopie à extrémité fixées} si il existe une application continue \( H\colon \mathopen[ 0 , 1 \mathclose]\times \mathopen[ a , b \mathclose]\to X\) telle que
    \begin{enumerate}
        \item
            les extrémités sont fixées : \( H(s,0)=p\) et \( H(s,1)=q\) pour tout \( s\in \mathopen[ 0 , 1 \mathclose]\).
        \item
            pour tout \( t\in\mathopen[ a , b \mathclose]\), nous avons \( H(0,t) =\gamma_0(t) \) et \( H(1,t)=\gamma_1(t)\).
    \end{enumerate}
\end{definition}

\subsection{Formule de Green}

La formule de Green est un cas particulier du théorème de la divergence dans le cas \( n = 2\), légèrement reformulé.
\begin{theorem}     \label{THOooQSWMooAZasTl}
	Soit \( D \subset \eR^2\) ouvert borné tel que son bord est est la réunion finie d'un certain nombre de chemins de classe \( C^1\) de Jordan réguliers.  Supposons qu'en chaque point de son bord, \( D\) possède un vecteur normal unitaire extérieur \( \nu\). Soient \( P\) et \( Q\) deux fonctions réelles de classe \( C^1\) sur \( \bar D\). Alors
	\begin{equation}  \label{EqYLblSqV}
		\iint_D (\partial_xQ - \partial_yP)dx\,dy = \oint_{\partial D}
		Pd x + Q d y
	\end{equation}
	où chaque chemin \( \gamma\) formant le bord de \( D\) est orienté de
	sorte que \( T \nu = \frac{\dot\gamma}{\norme{\dot\gamma}}\) où \( T\)
	représente la rotation d'angle \( +\frac\pi2\).
\end{theorem}

Justifions le fait que cela soit un cas particulier de la formule de Stokes du théorème~\ref{ThoATsPuzF}. Nous considérons la forme différentielle
\begin{equation}
	\omega=Pdx+Qdy,
\end{equation}
et sa différentielle
\begin{subequations}
	\begin{align}
		d\omega & =\sum_id\omega_i\wedge dx_i                                                                                                                                                                      \\
		        & =\left( \frac{ \partial P }{ \partial x }dx+\frac{ \partial P }{ \partial y }dy \right)\wedge dx+\left( \frac{ \partial Q }{ \partial x }dx+\frac{ \partial Q }{ \partial y }dy \right)\wedge dy \\
		        & =\left( \frac{ \partial Q }{ \partial x }-\frac{ \partial P }{ \partial y } \right)dx\wedge dy.
	\end{align}
\end{subequations}

Intégrons cette forme \( d\omega\) sur le domaine ouvert \( D\) que nous paramétrons de façon triviale par
\begin{equation}
	\begin{aligned}
		\varphi\colon D & \to \eR^2      \\
		(u,v)           & \mapsto (u,v).
	\end{aligned}
\end{equation}
Ce que nous avons est
\begin{equation}\label{EqKYjFEGF}
	\iint_D d\omega=\iint_D d\omega_{(u,v)}\left( \frac{ \partial \varphi }{ \partial u },\frac{ \partial \varphi }{ \partial v } \right)dudv
\end{equation}
Nous avons aussi \( T_u=\frac{ \partial \varphi }{ \partial u }=\begin{pmatrix}
	1 \\
	0
\end{pmatrix}\) et \(T_v= \frac{ \partial \varphi }{ \partial v }=\begin{pmatrix}
	0 \\
	1
\end{pmatrix}\) et donc
\begin{equation}
	(dx\wedge dy)(T_u,T_v)=dx(T_u)dy(T_v)-dx(T_v)dy(T_u)=1-0=1.
\end{equation}
L'intégrale \eqref{EqKYjFEGF} se développe donc en
\begin{equation}
	\iint_Dd\omega=\iint_D\left( \frac{ \partial Q }{ \partial x }(u,v)-\frac{ \partial P }{ \partial y }(u,v) \right)(dx\wedge dy)(T_u,T_v)dudv=\iint_D\left( \frac{ \partial Q }{ \partial x }-\frac{ \partial P }{ \partial y } \right)dudv.
\end{equation}
Par conséquent la formule de Stokes nous donne la formule \eqref{EqYLblSqV}.

La formule de Green nous permet de calculer l'aire de la surface délimitée par une courbe fermée en termes de l'intégrale d'une forme bien choisie le long du contour. Pour cela nous prenons la forme
\begin{equation}    \label{EqZNXYMQb}
	\omega=-\frac{ y }{2}dx+\frac{ x }{2}dy,
\end{equation}
de telle sorte que \( \partial_xQ-\partial_yQ=1\) et que
\begin{equation}
	\iint_Dd\omega=\iint_Dddudv=S,
\end{equation}
et au final l'aire est donnée par
\begin{equation}
	S=\int_{\partial D}\left( -\frac{ y }{2}dx+\frac{ x }{2}dy \right).
\end{equation}

Lorsque le bord de \( D\) est paramétré par
\begin{equation}
	\begin{aligned}
		\gamma\colon \mathopen[ a , b \mathclose] & \to \eR^2                            \\
		u                                         & \mapsto \begin{pmatrix}
			x(u) \\
			y(u)
		\end{pmatrix},
	\end{aligned}
\end{equation}
nous avons
\begin{equation}
	(Pdx+Qdy)\gamma'(u)=Px'+Qy',
\end{equation}
et alors
\begin{equation}
	\int_{\partial D}Pdx+Qdy=\int_a^b P\big( x(u),y(u) \big)x'(u)+Q\big( x(u),y(u) \big)y'(u)du.
\end{equation}
En ce qui concerne l'aire de la surface, nous prenons les \( P\) et \( Q\) de la forme~\ref{EqZNXYMQb} :
\begin{equation}    \label{EqAJGrtOk}
	S=\frac{ 1 }{2}\int_a^b\Big( -y(u)x'(u)+x(u)y'(u) \Big)du.
\end{equation}

\subsection{Formule de Stokes}
\label{secstokesusuel}

La formule de Stokes est la version classique, qui permet d'exprimer la circulation d'un champ de vecteur le long d'une courbe de \( \eR^3\) comme le flux de son rotationnel à travers n'importe quel surface dont le bord est la courbe. La version présentée ici suppose que la surface peut se paramétrer en un seul morceau~:
\begin{theorem}[Formule de Stokes]      \label{THOooDCPKooMqNOMU}
	Soit \( F : W\subset \eR^2 \to \eR^3\) un paramétrage (carte) d'une surface dans \( \eR^3\), supposée de classe \( C^2\). Soit \( D\) un ouvert de \( \eR^2\) vérifiant les hypothèses de la formule de Green, et tel que \( \bar D \subset W\). Soit \( G\) un champ de vecteurs de classe \( C^1\) défini sur \( F(\bar D)\), et soit \( N\) le champ normal unitaire donné par le paramétrage
	\begin{equation}
		N = \frac{\pder F u \wedge \pder F v}{\norme{\pder F u \wedge \pder F v}}
	\end{equation}
	alors
	\begin{equation}\label{EqStokesTho}
		\iint_{F(D)} \scalprod{\nabla\times G}{N} d\sigma_F = \int_{F(\partial D)} G
	\end{equation}
	où les chemins formant le bord \( \partial D\) sont orientés comme dans le théorème de Green.
\end{theorem}
Notons, juste pour avoir une bonne nouvelle de temps en temps, que
\begin{equation}
	d\sigma_F=\left\| \frac{ \partial F }{ \partial u }\times\frac{ \partial F }{ \partial v }  \right\|dudv,
\end{equation}
mais cette norme apparaît exactement au dénominateur de \( N\). Il ne faut donc pas la calculer parce qu'elle se simplifie.

Sous forme un peu plus physicienne\footnote{et surtout plus explicite.}, la formule \eqref{EqStokesTho} s'écrit
\begin{equation}
	\int_{F(D)}\langle \nabla\times G, N(x)\rangle\, d\sigma_F(x)=\int_{F(\gamma)}\langle G, T\rangle\, ds
\end{equation}
où \( T\) est le vecteur unitaire tangent à \( F(\gamma)\).

%///////////////////////////////////////////////////////////////////////////////////////////////////////////////////////////
\subsubsection{Quelle est la bonne orientation ?}
%///////////////////////////////////////////////////////////////////////////////////////////////////////////////////////////

Le signe du vecteur normal \( N\) dépend du choix de l'ordre des coordonnées dans la carte. Supposons que je veuille paramétrer la surface \( x^2+y^2=1\), \( z=1\). Nous prenons naturellement comme carte le cercle \( C\) de rayon \( 1\) dans \( \eR^2\) et la carte
\begin{equation}
	F(r,\theta)=\begin{pmatrix}
		r\cos\theta \\
		r\sin\theta \\
		1
	\end{pmatrix}.
\end{equation}
Mais nous aurions aussi pu mettre les coordonnées \( r\) et \( \theta\) dans l'autre ordre :
\begin{equation}
	\tilde F(\theta,r)=\begin{pmatrix}
		r\cos\theta \\
		r\sin\theta \\
		1
	\end{pmatrix}.
\end{equation}
Les vecteurs normaux ne sont pas les même : la carte \( F\) donnera \( \partial_rF\times\partial_{\theta}F\), tandis que l'autre donnera \( \partial_{\theta}\tilde F\times\partial_r\tilde F\). Le signe change !

Il faut savoir laquelle choisir. Le cercle \( C\subset \eR^2\) a une orientation donnée par le théorème de Green. Nous choisissons l'ordre des coordonnées pour que \( 1_{\theta}\) et \( 1_{r}\) soient dans la même orientation que les vecteurs \( \nu\) et \( T\) tels que donnés par le théorème de Green, et tels que dessinés sur la figure~\ref{LabelFigCercleTnu}.
\newcommand{\CaptionFigCercleTnu}{L'orientation sur le cercle. Si nous les prenons dans l'ordre, les vecteurs \( (1_r,1_{\theta})\) ont la même orientation que celle donnée par les vecteurs \( (\nu,T)\) donnés par la convention de Green.}
\input{auto/pictures_tex/Fig_CercleTnu.pstricks}

%\ref{LabelFigCercleTnu}.
%\newcommand{\CaptionFigCercleTnu}{L'orientation sur le cercle. Si nous les prenons dans l'ordre, les vecteurs \( (1_r,1_{\theta})\) ont la même orientation que celle donnée par les vecteurs \( (\nu,T)\) donnés par la convention de Green.}
%\input{auto/pictures_tex/Fig_CercleTnu.pstricks}

Plus généralement, nous choisissons l'ordre des coordonnées \( u\) et \( v\) pour que la base \( (1_u,1_v)\) ait la même orientation que \( (\nu,T)\) où \( T\) a le sens convenu dans le théorème de Green.

%+++++++++++++++++++++++++++++++++++++++++++++++++++++++++++++++++++++++++++++++++++++++++++++++++++++++++++++++++++++++++++
\section{Résumé des intégrales vues}
%+++++++++++++++++++++++++++++++++++++++++++++++++++++++++++++++++++++++++++++++++++++++++++++++++++++++++++++++++++++++++++

Nous sommes maintenant capables de revoir tous les types d'intégrales vues jusqu'ici de façon très cohérentes. Nous commencerons par les intégrales de fonctions et nous ferons ensuite les intégrales de champs de vecteurs.

%---------------------------------------------------------------------------------------------------------------------------
\subsection{L'intégrale d'une fonction sur les réels}
%---------------------------------------------------------------------------------------------------------------------------

Si \( f\colon \mathopen[ a , b \mathclose]\subset\eR\to \eR\) est une fonction usuelle, sont intégrale est
\begin{equation}
	\int_a^bf(x)dx=F(b)-F(a)
\end{equation}
où \( F\) est une primitive de \( f\).

%---------------------------------------------------------------------------------------------------------------------------
\subsection{Intégrale d'une fonction sur un chemin}
%---------------------------------------------------------------------------------------------------------------------------

Si \( f\) est une fonction sur \( \eR^3\) et si \( \sigma\colon \mathopen[ a , b \mathclose]\to \eR^3\) est un chemin dans \( \eR^3\), l'intégrale de \( f\) sur \( \sigma\) est, par définition,
\begin{equation}
	\int f\,d\sigma=\int_a^b f\big( \sigma(t) \big)\| \sigma'(t) \|dt.
\end{equation}

%---------------------------------------------------------------------------------------------------------------------------
\subsection{Intégrale d'une fonction sur une surface}
%---------------------------------------------------------------------------------------------------------------------------

Nous devons paramétrer la surface \( S\) par une application \( \varphi\colon D\subset\eR^2\to \eR^3\). À partir d'un tel paramétrage, nous construisons un élément de surface en prenant le produit vectoriel des deux vecteurs tangents :
\begin{equation}
	dS=\frac{ \partial \varphi }{ \partial u }\times\frac{ \partial \varphi }{ \partial v }dudv.
\end{equation}
L'intégrale est
\begin{equation}        \label{EqDefIntSurffS}
	\int f\,dS=\iint_Df\big( \varphi(u,v) \big)\left\| \frac{ \partial \varphi }{ \partial u }\times\frac{ \partial \varphi }{ \partial v } \right\|dudv.
\end{equation}

Il ne faut pas rajouter de jacobien : la norme du produit vectoriel \emph{est} le jacobien.

\begin{remark}
	La formule \eqref{EqDefIntSurffS} est autant valable pour des surfaces dans \( \eR^2\) que dans \( \eR^3\). Si nous considérons une surface dans \( \eR^2\), nous la voyons dans \( \eR^3\) en ajoutant un zéro comme troisième composante.
\end{remark}

\begin{example}
	Les coordonnées polaires sont données par
	\begin{equation}
		\varphi(r,\theta)=\begin{pmatrix}
			r\cos\theta \\
			r\sin\theta \\
			0
		\end{pmatrix}.
	\end{equation}
	Les vecteurs tangents à ce paramétrage sont
	\begin{equation}
		\begin{aligned}[]
			T_r & =\frac{ \partial \varphi }{ \partial r }=\begin{pmatrix}
				\cos\theta \\
				\sin\theta \\
				0
			\end{pmatrix}, & T_{\theta} & =\frac{ \partial \varphi }{ \partial \theta }=\begin{pmatrix}
				-r\sin\theta \\
				r\cos\theta  \\
				0
			\end{pmatrix}.
		\end{aligned}
	\end{equation}
	Le vecteur normal est
	\begin{equation}
		\frac{ \partial \varphi }{ \partial r }\times\frac{ \partial \varphi }{ \partial \theta }=\begin{vmatrix}
			e_x          & e_y         & e_z \\
			\cos\theta   & \sin\theta  & 0   \\
			-r\sin\theta & r\cos\theta & 0
		\end{vmatrix}=re_z.
	\end{equation}
	Nous trouvons donc que l'élément de surface est la norme de \( re_z\), c'est-à-dire \( r\), le jacobien connu.
\end{example}

%---------------------------------------------------------------------------------------------------------------------------
\subsection{Intégrale d'une fonction sur un volume}
%---------------------------------------------------------------------------------------------------------------------------

Si \( V\) est un volume dans \( \eR^3\), nous effectuons la même procédure : nous trouvons un paramétrage, et nous formons un élément de volume avec les vecteurs tangents de le paramétrage. Nous avons donc un volume déterminé par l'application
\begin{equation}
	\varphi\colon D\subset\eR^3\to \eR^3,
\end{equation}
et ses trois vecteurs tangents
\begin{equation}
	\begin{aligned}[]
		T_u & =\frac{ \partial \varphi }{ \partial u }  \\
		T_v & =\frac{ \partial \varphi }{ \partial v }  \\
		T_w & =\frac{ \partial \varphi }{ \partial w }.
	\end{aligned}
\end{equation}
Comment former un volume avec trois vecteurs ? Réponse : le produit mixte. L'intégrale de \( f\) sur \( V\) sera
\begin{equation}
	\int f\,dV=\iiint_D f\big( \varphi(u,v,w) \big)\left\| \frac{ \partial \varphi }{ \partial u }\cdot \left( \frac{ \partial \varphi }{ \partial v }\times\frac{ \partial \varphi }{ \partial w }\right) \right\|dudv.
\end{equation}

Encore une fois, le produit mixte \emph{est} le jacobien. Prenons les coordonnées sphériques :
\begin{equation}
	\begin{aligned}[]
		x(r,\theta,\varphi) & =r\sin(\theta)\cos(\varphi) \\
		y(r,\theta,\varphi) & =r\sin(\theta)\sin(\varphi) \\
		z(r,\theta,\varphi) & =r\cos(\theta)
	\end{aligned}
\end{equation}
Les trois vecteurs tangents seront
\begin{subequations}
	\begin{align}
		T_r         & =\begin{pmatrix}
			\frac{ \partial x }{ \partial r } \\
			\frac{ \partial y }{ \partial r } \\
			\frac{ \partial z }{ \partial r }
		\end{pmatrix}=\begin{pmatrix}
			\sin(\theta)\cos(\varphi) \\
			\sin(\theta)\sin(\varphi) \\
			\cos(\theta)
		\end{pmatrix} \\
		T_{\theta}  & =\begin{pmatrix}
			\frac{ \partial x }{ \partial \theta } \\
			\frac{ \partial y }{ \partial \theta } \\
			\frac{ \partial z }{ \partial \theta }
		\end{pmatrix}=\begin{pmatrix}
			r\cos(\theta)\cos(\varphi) \\
			r\cos(\theta)\sin(\varphi) \\
			-r\sin(\theta)
		\end{pmatrix} \\
		T_{\varphi} & =\begin{pmatrix}
			\frac{ \partial x }{ \partial \varphi } \\
			\frac{ \partial y }{ \partial \varphi } \\
			\frac{ \partial z }{ \partial \varphi }
		\end{pmatrix}=\begin{pmatrix}
			-r\sin(\theta)\sin(\varphi) \\
			r\sin(\theta)\cos(\varphi)  \\
			0
		\end{pmatrix}
	\end{align}
\end{subequations}
Nous avons vu que le produit mixte revient à mettre toutes les composantes dans une matrice. Ici nous avons donc
\begin{equation}
	\frac{ \partial \phi }{ \partial r }\cdot\left( \frac{ \partial \phi }{ \partial \theta }\times\frac{ \partial \phi }{ \partial \varphi } \right)=\begin{vmatrix}
		\frac{ \partial x }{ \partial r }       & \frac{ \partial y }{ \partial r }       & \frac{ \partial z }{ \partial r }       \\
		\frac{ \partial x }{ \partial \theta }  & \frac{ \partial y }{ \partial \theta }  & \frac{ \partial z }{ \partial \theta }  \\
		\frac{ \partial x }{ \partial \varphi } & \frac{ \partial y }{ \partial \varphi } & \frac{ \partial z }{ \partial \varphi }
	\end{vmatrix}
\end{equation}
Cela est précisément le jacobien dont nous parlions plus haut.

%---------------------------------------------------------------------------------------------------------------------------
\subsection{Conclusion pour les fonctions}
%---------------------------------------------------------------------------------------------------------------------------

Lorsque nous intégrons une fonction sur un chemin, une surface ou un volume, la technique est toujours la même :
\begin{enumerate}
	\item
	      Trouver un paramétrage à une, deux ou trois variables.
	\item
	      Dériver le paramétrage par rapport à ses variables.
	\item
	      Construire un élément de longueur, surface ou volume à partir des vecteurs que l'on a. Cela se fait en prenant la norme, le produit vectoriel ou le produit mixte.
\end{enumerate}

%---------------------------------------------------------------------------------------------------------------------------
\subsection{Circulation d'un champ de vecteurs}
%---------------------------------------------------------------------------------------------------------------------------

Pour les champs de vecteurs, nous faisons la même chose, mais au lieu de \emph{multiplier} par l'élément de longueur ou de surface, nous prenons le produit scalaire. Si nous considérons la courbe paramétrée \( \sigma\colon \mathopen[ a , b \mathclose]\to \eR^3\) et le champ de vecteurs \( F\), nous avons donc
\begin{equation}
	\int_{\sigma}F=\int F\cdot d\sigma=\int_a^bF\big( \sigma(t) \big)\cdot\sigma'(t)dt.
\end{equation}

%---------------------------------------------------------------------------------------------------------------------------
\subsection{Flux d'un champ de vecteurs}
%---------------------------------------------------------------------------------------------------------------------------

Si la surface \( S\subset\eR^3\) est paramétrée par
\begin{equation}
	\begin{aligned}
		\phi\colon D\subset\eR^2 & \to \eR^3          \\
		(u,v)                    & \mapsto \phi(u,v),
	\end{aligned}
\end{equation}
et si \( F\) est un champ de vecteurs, alors on a
\begin{equation}        \label{EqResIntFluxPhi}
	\int_SF=\int_S F\cdot dS=\iint_D F\big( \phi(u,v) \big)\cdot\left( \frac{ \partial \phi }{ \partial u }\times\frac{ \partial \phi }{ \partial v } \right)\,dudv.
\end{equation}

%---------------------------------------------------------------------------------------------------------------------------
\subsection{Conclusion pour les champs de vecteurs}
%---------------------------------------------------------------------------------------------------------------------------

La circulation et le flux ne représentent pas tout à fait la même chose. En effet pour la circulation, nous sélectionnons la composante \emph{tangente} à la courbe, c'est-à-dire la partie du vecteurs qui «circule» le long de la courbe. Une force perpendiculaire au mouvement ne travaille pas.

La situation est exactement le contraire pour le flux. Étant donné que le vecteur
\begin{equation}
	\frac{ \partial \phi }{ \partial u }\times\frac{ \partial \phi }{ \partial v }
\end{equation}
est normal à la surface, le fait de prendre le produit scalaire du champ de vecteurs avec lui sélectionne la composante \emph{normale} à la surface, c'est-à-dire la partie du vecteur qui traverse la surface.

%---------------------------------------------------------------------------------------------------------------------------
\subsection{Attention pour les surfaces fermées !}
%---------------------------------------------------------------------------------------------------------------------------

Si nous considérons une surface fermée, il faut faire attention à choisir une \emph{orientation}. Les vecteurs normaux doivent soit tous pointer vers l'intérieur soit tous vers l'extérieur. En effet, en tant que vecteur normal, nous avons choisi de prendre
\begin{equation}
	T_u\times T_v.
\end{equation}
Mais le vecteur \( T_v\times T_u\) est tout aussi normal ! Il n'y a pas à priori de façon standard pour choisir l'un ou l'autre. Il faut juste être cohérent : il faut que si on divise la surface en plusieurs morceaux, tous les vecteurs pointent dans le même sens.

Notez que si vous faites un choix et que votre voisin fait le choix inverse, vous obtiendrez des réponses qui diffèrent d'un signe. Sans plus de précisions\footnote{Il faudrait définir ce qu'est une surface \emph{orientable} et choisir une orientation.}, les deux réponses sont correctes.

Un exemple de ce problème est donné dans l'exemple~\ref{EXooAJRLooSTPChN}.

\begin{example}     \label{EXooAJRLooSTPChN}

	Calculer le flux du champ de vecteurs
	\begin{equation}
		F(x,y,z)=e_x
	\end{equation}
	au travers du cylindre de rayon \( R\) et de hauteur \( h\) autour de l'axe \( z\).

	Même question si le cylindre est autour de l'axe \( x\).

	Remarque : ces cylindres sont considérés \emph{avec} leur «couvercles».


	Un paramétrage du cylindre autour de l'axe \( z\) est
	\begin{equation}
		\phi(\theta,z)=\begin{pmatrix}
			R\cos\theta \\
			R\sin\theta \\
			z
		\end{pmatrix}.
	\end{equation}
	Les vecteurs tangents sont
	\begin{equation}
		\begin{aligned}[]
			T_{\theta} & =\begin{pmatrix}
				-R\sin\theta \\
				R\cos\theta  \\
				0
			\end{pmatrix},
			T_z        & =\begin{pmatrix}
				0 \\
				0 \\
				1
			\end{pmatrix}.
		\end{aligned}
	\end{equation}
	Le vecteur normal est donc
	\begin{equation}
		T_{\theta}\times T_z=R\cos(\theta)e_x+R\sin(\theta)e_y.
	\end{equation}
	C'est un vecteur dirigé vers l'extérieur.

	Le champ de vecteurs considéré est constant : \( F(\theta,z)=e_x\). Nous avons donc
	\begin{equation}
		F(\theta,z)\cdot(T_{\theta}\times T_z)=R\cos(\theta)
	\end{equation}
	et le flux vaut
	\begin{equation}
		\Phi=\int_0^{2\pi}d\theta\int_0^hR\cos(\theta)dz=0.
	\end{equation}

	En ce qui concerne les couvercles haut au bas, ils sont paramétrés par
	\begin{equation}
		\begin{aligned}[]
			\phi_1(r,\theta) & =\begin{pmatrix}
				R\cos(\theta) \\
				R\sin(\theta) \\
				h
			\end{pmatrix},
			\phi_2(r,\theta) & =\begin{pmatrix}
				R\cos(\theta) \\
				R\sin(\theta) \\
				0
			\end{pmatrix}.
		\end{aligned}
	\end{equation}
	Les vecteurs normaux correspondants sont dans la direction de \( e_z\), de façon que le produit scalaire avec \( F(r,\theta)\) soit nul. Le flux total est donc nul.

	Regardons maintenant le cylindre le long de l'axe \( x\). Un paramétrage est
	\begin{equation}
		\phi(\theta,x)=\begin{pmatrix}
			x             \\
			R\cos(\theta) \\
			R\sin(\theta)
		\end{pmatrix},
	\end{equation}
	et le vecteurs tangents sont
	\begin{equation}
		\begin{aligned}[]
			T_{\theta} & =\begin{pmatrix}
				0            \\
				-R\sin\theta \\
				R\cos\theta
			\end{pmatrix},
			T_x        & =\begin{pmatrix}
				1 \\
				0 \\
				0
			\end{pmatrix}.
		\end{aligned}
	\end{equation}
	Le vecteur normal est alors donné par
	\begin{equation}
		T_{\theta}\times T_x=R\cos(\theta)e_y+R\sin(\theta)e_z.
	\end{equation}
	Nous avons par conséquent \( F(\theta,x)\cdot (T_{\theta}\times T_x)=0\). Pas de flux par le côté du cylindre.

	Regardons les «couvercles». Le premier est donné par le paramétrage
	\begin{equation}
		\phi_1(r,\theta)=\begin{pmatrix}
			0             \\
			r\cos(\theta) \\
			r\sin(\theta)
		\end{pmatrix}.
	\end{equation}
	Le vecteur normal serait \( T_r\times T_{\theta}=re_x\), et le flux
	\begin{equation}
		\Phi=\int_0^{2\pi}d\theta\int_0^Rr\,dr=\pi R^2.
	\end{equation}

	Le second couvercle est donné par
	\begin{equation}
		\phi_2(r,\theta)=\begin{pmatrix}
			h             \\
			r\cos(\theta) \\
			r\sin(\theta)
		\end{pmatrix}.
	\end{equation}
	Le vecteur normal est encore \( re_x\), et le flux est à nouveau \( \pi R^2\).

	Le flux total serait donc \( 2\pi R^2\).

	Cela n'est pas possible parce que tous les vecteurs qui «rentrent» d'un côté doivent «sortir» de l'autre côté. L'erreur est le le premier vecteur normal est un vecteur qui pointe vers l'intérieur du cylindre, tandis que le second pointe vers l'extérieur. Si nous choisissons, par convention, de prendre uniquement les vecteurs extérieurs, il faut changer le vecteur normal du premier couvercle en \( -re_x\). Le premier flux vaudra donc
	\begin{equation}
		-\pi R^2,
	\end{equation}
	de telle sorte que le flux total sera nul.
\end{example}
