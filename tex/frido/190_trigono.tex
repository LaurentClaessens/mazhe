% This is part of (everything) I know in mathematics
% Copyright (c) 2011-2017, 2019, 2021-2022
%   Laurent Claessens
% See the file fdl-1.3.txt for copying conditions.

%+++++++++++++++++++++++++++++++++++++++++++++++++++++++++++++++++++++++++++++++++++++++++++++++++++++++++++++++++++++++++++
\section{Isométries de l'espace euclidien}
%+++++++++++++++++++++++++++++++++++++++++++++++++++++++++++++++++++++++++++++++++++++++++++++++++++++++++++++++++++++++++++

Nous considérons l'espace affine euclidien \( A=\affE_n(\eR)\) modelé sur \( \eR^n\) avec sa métrique usuelle. Un premier grand résultat sera le théorème~\ref{ThoDsFErq} qui dira que les isométries de cet espace sont des applications linéaires.

%---------------------------------------------------------------------------------------------------------------------------
\subsection{Structure du groupe  \texorpdfstring{\( \Isom(\eR^n)\)}{Isom(Rn)} }
%---------------------------------------------------------------------------------------------------------------------------

Si vous ne voulez pas savoir ce qu'est un produit semi-direct de groupes, vous pouvez lire seulement le point~\ref{ITEMooLLUIooIGsknv} du théorème suivant, et passer directement à la remarque~\ref{REMooLUEZooIwvTqu}.
\begin{theorem}     \label{THOooQJSRooMrqQct}
	Un peu de structure sur \( \Isom(\eR^n)\).
	\begin{enumerate}
		\item       \label{ITEMooLLUIooIGsknv}
		      L'application
		      \begin{equation}
			      \begin{aligned}
				      \psi\colon T(n)\times \gO(n) & \to \Isom(\eR^n)           \\
				      (v,\Lambda)                  & \mapsto \tau_v\circ\Lambda
			      \end{aligned}
		      \end{equation}
		      est une bijection. Ici,  \( T(n)\) est le groupe des translations de \( \eR^n\).
		\item
		      Un couple \( (v,\Lambda)\in T(n)\times\SO(n)\) agit sur \( x\in \eR^n\) par
		      \begin{equation}
			      (v,\Lambda)x=\Lambda x+v
		      \end{equation}
		      au sens où \( \psi(v,\Lambda)x=\Lambda x+v\).
		\item       \label{ITEMooEWSIooNKzRxB}
		      En tant que groupes,
		      \begin{equation}
			      \Isom(\eR^n)\simeq T(n)\times_{\rho}\gO(n)
		      \end{equation}
		      où \( \rho\) représente l'action adjointe de \( \gO(n)\) sur \( T(n)\) et \( \times_{\rho}\) dénote le produit semi-direct de la définition~\ref{DEFooKWEHooISNQzi}.
		\item     \label{ITEMooSKUPooBDvNWX}
		      Une isométrie de \( \eR^n\) est une application affine\footnote{Définition \ref{DEFooUAWZooXcMKve}.}.
		\item     \label{ITEMooQLNPooSyHaps}
		      La partie linéaire\footnote{Définition \ref{LEMooYJCDooOGAHkF}.} d'une isométrie \( f\) est \( \psi^{-1}(f)_2\).
	\end{enumerate}
\end{theorem}

\begin{proof}
	Point par point.
	\begin{enumerate}
		\item
		      Prouvons que l'application proposée est injective et surjective. Notons aussi que ce point ne parle pas de structure de groupe, mais seulement d'une bijection en tant qu'ensembles.
		      \begin{subproof}
			      \spitem[Injection]
			      Si \( \psi(v,\Lambda)=\psi(w,\Lambda')\) alors en appliquant sur \( x=0\) nous avons tout de suite \( v=w\). Et ensuite \( \Lambda=\Lambda'\) est immédiat.
			      \spitem[Surjection]
			      Une isométrie \( g\in\Isom(\eR^n)\) est une application \( g\colon \eR^n\to \eR^n\) telle que \( d(x,y)=d\big( g(x),g(y) \big)\). Dans le cas de \( \eR^n\) cela se traduit par
			      \begin{equation}
				      \| x-y \|=\big\| g(x)-g(y) \big\|,
			      \end{equation}
			      Comme \( x\mapsto\| x \|\) est une forme quadratique, elle tombe sous le coup du théorème~\ref{ThoDsFErq}, ce qui nous permet de dire que \( g\) est affine. Or par définition une application est affine lorsqu'elle est la composée d'une translation et d'une application linéaire.

			      Donc \( g=\tau_v\circ \Lambda\) pour une certaine application linéaire isométrique \( \Lambda\colon \eR^n\to \eR^n\). L'application \( \Lambda\) est donc dans \( \gO(n)\) par la proposition \ref{PropKBCXooOuEZcS}\ref{ITEMooOWMBooHUatNb}.
		      \end{subproof}
		\item
		      C'est seulement le fait que \( (\tau_v\circ\Lambda)x=\tau_v\big( \Lambda x \big)=\Lambda(x)+v\).
		\item
		      Nous allons étudier l'application
		      \begin{equation}
			      \psi\colon T(n)\times_{\rho}O(n)\to \Isom(\eR^n).
		      \end{equation}
		      \begin{subproof}
			      \spitem[Le produit semi-direct est bien défini]
			      Il faut montrer que
			      \begin{equation}
				      \begin{aligned}
					      \rho\colon O(n) & \to \Aut\big( T(n) \big) \\
					      \Lambda         & \mapsto \AD(\Lambda)
				      \end{aligned}
			      \end{equation}
			      est correcte.

			      D'abord pour \( \Lambda\in O(n)\), nous avons bien \( \rho_{\Lambda}(\tau_v)\in T(n)\) parce qu'en appliquant à \( x\in \eR^n\),
			      \begin{equation}
				      (\Lambda\tau_v\Lambda^{-1})(x)=\Lambda\big( \tau_v(\Lambda^{-1} x) \big)=\Lambda\big( \Lambda^{-1}x+v \big)=x+\Lambda(v)=\tau_{\Lambda(v)}(x).
			      \end{equation}
			      Donc \( \rho_{\Lambda}(\tau_v)=\tau_{\Lambda(v)}\).

			      De plus, \( \rho_{\Lambda}\in\Aut\big( T(n) \big)\) parce que
			      \begin{equation}
				      \rho_{\Lambda}\big( \tau_v\circ \tau_w \big)=\rho_{\Lambda}(\tau_v)\circ\rho_{\Lambda}(\tau_v),
			      \end{equation}
			      comme on peut aisément vérifier que les deux membres sont égaux à \( \tau_{\Lambda(v+w)}\).
			      \spitem[\( \psi\) est une bijection]
			      Cela est déjà vérifié.
			      \spitem[\( \psi\) est un morphisme]
			      Nous avons d'une part
			      \begin{equation}
				      \psi\big( (v,g)(w,h) \big)=\psi\big( v\rho_g(w),gh \big)=\tau_v\circ g\circ\tau_w\circ g^{-1}\circ g\circ h=\tau_v\circ g\circ\tau_w\circ h.
			      \end{equation}
			      Et d'autre part,
			      \begin{equation}
				      \psi(v,g)\circ\psi(w,h)=\tau_v\circ g\circ \tau_w\circ h,
			      \end{equation}
			      ce qui est la même chose.
		      \end{subproof}
		\item
		      Si \(f \) est une isométrie de \( \eR^n\), et si \( \psi(f)=(v,\Lambda)\), nous avons
		      \begin{equation}
			      f(a+x)=\tau_v\big( \Lambda(a)+\Lambda(x) \big)=\Lambda(a)+\Lambda(x)+v=f(a)+\Lambda(x).
		      \end{equation}
		      Donc \( f\) est affine et l'application qui serait notée \( u_M\) dans \eqref{EqMqIoWX} est \( u_M=\Lambda\) pour tout \( M\).
		\item
		      Si \( f(x)=\psi(v,\Lambda)x=\Lambda x+v\), nous avons
		      \begin{equation}
			      f(a+x)=\psi(v,\Lambda)(a+x)=f(a)+\Lambda(x),
		      \end{equation}
		      et donc \( \Lambda\) est bien la partie linéaire de \( f\).
	\end{enumerate}
\end{proof}

\begin{remark}      \label{REMooLUEZooIwvTqu}
	Notons au passage la loi de groupe sur les couples qui est donnée, pour tout \( v,v'\in \eR^n\), \( \Lambda,\Lambda'\in\SO(n)\), par
	\begin{equation}    \label{EqDiHcut}
		(v,\Lambda)\cdot(v',\Lambda')=(\Lambda v'+v,\Lambda\Lambda')
	\end{equation}
	comme le montre le calcul suivant :
	\begin{subequations}
		\begin{align}
			(v,\Lambda)\cdot(v',\Lambda')x & =(v,\Lambda)(\Lambda'x+v')        \\
			                               & =\Lambda\Lambda'x+\Lambda v'+v    \\
			                               & =(\Lambda v'+v,\Lambda\Lambda')x.
		\end{align}
	\end{subequations}
\end{remark}

\begin{proposition}[\cite{ooZYLAooXwWjLa}]      \label{PROPooDHYWooXxEXvl}
	Soient \( n\geq 1\) et \( R\) un élément de \( \gO(n)\) de déterminant \( -1\) tels que \( R^2=\id\). En posant \( C_2=\{ \id,R \}\) nous avons
	\begin{equation}
		\gO(n)=\SO(n)\times_{\rho} C_2
	\end{equation}
\end{proposition}

\begin{proof}
	Notons qu'un élément \( R\) comme décrit dans l'énoncé existe. Par exemple il y a l'application  \( (x_1,\ldots, x_n)\mapsto (-x_1,x_2,\ldots, x_n)\).

	Cela étant dit, nous allons montrer que
	\begin{equation}
		\begin{aligned}
			\psi\colon \SO(n)\times C_2 & \to \gO(n)  \\
			(A,h)                       & \mapsto Ah.
		\end{aligned}
	\end{equation}
	est un isomorphisme.
	\begin{subproof}
		\spitem[Injectif]
		Soient \( A,B\in \SO(n)\) et \( h,k\in C_2\) tels que \( \psi(A,h)=\psi(B,k)\), c'est-à-dire tels que \( Ah=Bk\). Puisque \( \det(A)=\det(B)=1\), nous avons \( \det(h)=\det(k)\). Mais comme \( C_2\) contient un élément de déterminant \( 1\) et un élément de déterminant \( -1\), nous avons \( h=k\). De là \( A=B\).
		\spitem[Surjectif]
		Soit \( X\in\gO(n)\). Si \( \det(X)=1\) alors \( X\in \SO(n)\) et \( X=\psi(X,\mtu)\). Si par contre \( \det(X)=-1\), alors \( XR\in\SO(n)\) parce que \( \det(XR)=1\), et nous avons
		\begin{equation}
			\psi(XR,R)=XR^2=X.
		\end{equation}
		\spitem[Morphisme]
		Nous avons
		\begin{equation}
			\psi\Big( (A,h)(B,k) \Big)=\psi\big( A\rho_h(B),hk \big)=A(hBh^{-1})hk=AhBk,
		\end{equation}
		tandis que
		\begin{equation}
			\psi(A,h)\psi(B,k)=AhBk,
		\end{equation}
		qui est la même chose.
	\end{subproof}
\end{proof}

%+++++++++++++++++++++++++++++++++++++++++++++++++++++++++++++++++++++++++++++++++++++++++++++++++++++++++++++++++++++++++++
\section{Isométries dans \( \eR^n\)}
%+++++++++++++++++++++++++++++++++++++++++++++++++++++++++++++++++++++++++++++++++++++++++++++++++++++++++++++++++++++++++++

\begin{definition}
	Un \defe{hyperplan}{hyperplan} de \( \eR^n\) est un sous-espace affine de dimension \( n-1\).
\end{definition}

\begin{lemmaDef}
	Si un hyperplan \( H\) de \( \eR^n\) est donné, et si \( x\in \eR^n\), il existe un unique point \( y\in \eR^n\) tel que
	\begin{enumerate}
		\item
		      \( x-y\perp H\),
		\item
		      Le segment \( [x,y]\) coupe \( H\) en son milieu.
	\end{enumerate}
	La \defe{réflexion}{réflexion!par rapport à un hyperplan} \( \sigma_H\) est l'application \( \sigma_H\colon \eR^n\to \eR^n \) qui à \( x\) fait correspondre ce \( y\).
\end{lemmaDef}

\begin{proof}
	Il faut vérifier que les conditions données définissent effectivement un unique point de \( \eR^n\). Soit \( H_0\) le sous-espace vectoriel parallèle à \( H\) et une base orthonormée \( \{ e_1,\ldots, e_{n-1} \}\) de \( H_0\). Nous complétons cette partie en une base orthonormée de \( \eR^n\) avec un vecteur \( e_n\). Si \( H=H_0+v\), quitte à décomposer \( v\) en une partie parallèle et une partie perpendiculaire à \( H\), nous avons
	\begin{equation}
		H=H_0+\lambda e_n
	\end{equation}
	pour un certain \( \lambda\).

	Une droite passant par \( x\) et perpendiculaire à \( H\) est de la forme \( t\mapsto x+te_n\). Si \( x=\sum_{i=1}^{n}x_ie_i\) alors l'unique point de cette droite à être dans \( H\) est le point tel que \(   x_ne_n+te_n=\lambda e_n   \), c'est-à-dire \( t=-x_n\). L'unique point \( y\) sur cette droite à être tel que \( [x,y ]\) coupe \( H\) en son milieu est celui qui correspond à \( t=-2x_n\).
\end{proof}

Notons au passage que cette preuve donne une formule pour \( \sigma_H\) :
\begin{equation}        \label{EQooRTWLooLPsUpY}
	\sigma_H(x)=\sum_{i=1}^{n-1}x_ie_i-x_ne_n.
\end{equation}
Il s'agit donc de changer le signe de la composante perpendiculaire à \( H\).

\begin{lemma}       \label{LEMooWYVRooQmWqvM}
	Dans cette même base, si \( H_0\) est l'hyperplan parallèle à \( H\) et passant par l'origine, nous écrivons \( H=H_0+\lambda e_n\) pour un certain \( \lambda\). Alors
	\begin{equation}
		\sigma_H=\sigma_{H_0}+2\lambda e_n.
	\end{equation}
\end{lemma}

\begin{proof}
	Un élément \( x\in \eR^n\) peut être décomposé dans la base adéquate en \( x=x_H+x_ne_n\). Nous savons de la formule \eqref{EQooRTWLooLPsUpY} que
	\begin{equation}
		\sigma_H(x)=x_H-x_ne_n.
	\end{equation}
	Mais puisque \( \sigma_{H_0}(x_H)=x_H-2\lambda e_n\) nous avons
	\begin{equation}
		\sigma_{H_0}(x)+2\lambda e_n=\sigma_{H_0}(x_H+x_ne_n)+2\lambda e_N=x_H-2\lambda e_n-x_ne_n+2\lambda e_n=x_H-x_ne_n.
	\end{equation}
\end{proof}

Le lemme suivant est une généralisation du fait que tous les points de la médiatrice d'un segment sont à égale distance des deux extrémités du segment (très utile lorsqu'on étudie les triangles isocèles).
\begin{lemma}[\cite{ooZYLAooXwWjLa}]        \label{LEMooDPLYooJKZxiM}
	Soient deux points distincts \( x_0,y_0\in \eR^n\) l'ensemble \( H\subset \eR^n\) donné par
	\begin{equation}
		H=\{ x\in \eR^n\tq d(x,x_0)=d(x,y_0) \}.
	\end{equation}
	Alors \( H\) est l'hyperplan orthogonal au vecteur \( v=y_0-x_0\) et \( H\) passe par le milieu du segment \( [x_0,y_0] \).
\end{lemma}

\begin{proof}
	Nous savons que
	\begin{equation}
		d(x,x_0)^2=\langle x-x_0, x-x_0\rangle =\| x \|^2+\| x_0 \|^2-2\langle x, x_0\rangle,
	\end{equation}
	ou encore
	\begin{equation}
		\| x_0 \|^2-\| y_0 \|^2=2\langle x, x_0-y_0\rangle .
	\end{equation}
	En posant \( v=y_0-x_0\) et en considérant la forme linéaire
	\begin{equation}
		\begin{aligned}
			\beta\colon \eR^n & \to \eR                       \\
			x                 & \mapsto \langle x, v\rangle ,
		\end{aligned}
	\end{equation}
	Nous avons \( x\in H\) si et seulement si \( \beta(x)=\frac{ 1 }{2}\big( \| y_0 \|^2-\| x_0 \|^2 \big)=\lambda\). En d'autres termes, \( H=\beta^{-1}(\lambda)\). Par la proposition~\ref{PROPooAKJBooMkmsiV} la partie \( H\) est un sous-espace affine. C'est même un translaté de \( \ker(\beta)\), et comme \( \ker(\beta)\) est l'espace vectoriel des vecteurs perpendiculaires à \( v\), nous avons \( \dim(H)=\dim\big( \ker(\beta) \big)=n-1\).

	Le fait que \( H\) contienne le milieu du segment \( [x_0,y_0]\) est par définition.
\end{proof}

Pour le lemme suivant, et pour que la récurrence se passe bien nous disons que l'ensemble vide est un espace vectoriel de dimension \( -1\).
\begin{lemma}[\cite{ooYVHDooLeexeT}]       \label{LEMooJCDRooGAmlwp}
	Soit un espace euclidien \( E\) de dimension \( n\).
	\begin{enumerate}
		\item       \label{ITEMooFYEDooIJZBjP}
		      Si \( f\) est une isométrie de \( E\) satisfaisant
		      \begin{equation}
			      \dim\big( \Fix(f) \big)=n-k
		      \end{equation}
		      alors \( f\) peut être écrit comme composition de \( k\) réflexions hyperplanes.
		\item       \label{ITEMooJTZVooWvyfDD}
		      Une isométrie de \( E\) peut être écrite sous la forme de \( \rank(f-\id)\) réflexions, mais pas moins.
		\item       \label{ITEMooUCZWooSbyPwt}
		      Toute isométrie de \( \eR^n\) peut être écrite comme composition de \( n+1\) réflexions.
	\end{enumerate}
\end{lemma}

\begin{proof}
	Les deux parties importantes à démontrer sont les points \ref{ITEMooFYEDooIJZBjP} et la partie «pas moins» de \ref{ITEMooJTZVooWvyfDD}. Le reste proviendra de reformulations.
	\begin{subproof}
		\spitem[Pour \ref{ITEMooFYEDooIJZBjP}]
		Nous faisons une récurrence sur \( k\geq 0\).

		Pour l'initialisation, si \( k=0\) alors \( \dim\big( \Fix(f) \big)=n\), c'est-à-dire que \( f\) fixe tout \( \eR^n\), autant dire que \( f\) est l'identité, une composition de zéro réflexions.

		Pour la récurrence, nous supposons que le lemme est démontré jusqu'à \( k\geq 0\). Soit donc \( f\in\Isom(\eR^n)\) tel que
		\begin{equation}
			\dim\big( \Fix(f) \big)=n-(k+1).
		\end{equation}
		Puisque \( k\geq 0\), la dimension de \( \Fix(f)\) est strictement plus petite que \( n\), donc il existe un \( x_0\in \eR^n\) tel que \( f(x_0)\neq x_0\). Nous posons
		\begin{equation}
			H=\{ x\in E\tq d(x,x_0)=d\big( x,f(x_0) \big)  \}.
		\end{equation}
		Par le lemme~\ref{LEMooDPLYooJKZxiM}, ce \( H\) est l'hyperplan orthogonal à \( v=f(x_0)-x_0\) et passant par le milieu du segment \( [x_0,f(x_0)]\).

		Nous posons \( g=\sigma_H\circ f\). Comme \( g(x_0)=\sigma_H(f(x_0))=x_0\), ce \( x_0\) est un point fixe de \( g\). Le fait que \( \sigma_H\big( f(x_0) \big)=x_0\) est vraiment la définition de l'hyperplan \( H\).

		Nous avons donc
		\begin{equation}
			x_0\in\Fix(g)\setminus\Fix(f).
		\end{equation}
		Mais nous prouvons de plus que \( \Fix(f)\subset\Fix(g)\). En effet si \( y\in \Fix(f)\) alors \( y\in H\) parce que
		\begin{equation}
			d(y,x_0)=d\big( f(y),f(x_0) \big)=d\big( y, f(x_0) \big).
		\end{equation}
		Puisque \( y\in H\) nous avons \( y\in \Fix(g)\) parce que
		\begin{equation}
			g(y)=\sigma_H\big( f(y) \big)=\sigma_H(y)=y.
		\end{equation}
		Tout cela pour dire que l'ensemble \( \Fix(g)\) est \emph{strictement} plus grand que \( \Fix(f)\). Et comme ce sont des espaces affines, nous pouvons parler de dimension :
		\begin{equation}
			\dim\big( \Fix(g) \big)>\dim\big( \Fix(f) \big).
		\end{equation}
		Par hypothèse de récurrence, l'application \(  g\) peut être écrite comme composition de \( k\) réflexions. Donc l'application
		\begin{equation}
			f=\sigma_H\circ g
		\end{equation}
		est une composition de \( k+1\) réflexions.
		\spitem[Pour \ref{ITEMooJTZVooWvyfDD}, existence]

		Ce point est une reformulation du point \ref{ITEMooFYEDooIJZBjP}. Le fait est que \( \Fix(f)=\ker(f-\id)\) parce que \( x\in\Fix(f)\) si et seulement si \( f(x)=x\) si et seulement si \( (f-\id)x=0\). Nous utilisons le théorème du rang \ref{ThoGkkffA} à l'endomorphisme \( f-\id\) :
		\begin{equation}
			\dim\big( \Fix(f) \big)=\dim\big( \ker(f-\id) \big)=\dim(E)-\rank(f-\id).
		\end{equation}
		En remplaçant par les valeurs :
		\begin{equation}
			n-k=n-\rank(f-\id).
		\end{equation}
		Or le point \ref{ITEMooFYEDooIJZBjP} donnait \( f\) comme composée de \( n-k\) réflexions. Donc \( f\) est composée de \( \rank(f-\id)\) réflexions.
		\spitem[Pour \ref{ITEMooJTZVooWvyfDD}, «pas moins»]

		Supposons que \( f=\sigma_1\circ\ldots \circ \sigma_r\) où \( \sigma_i\) est la réflexion de l'hyperplan \( H_i\). Nous devons prouver que \( r\geq \rank(f-\id)\). Nous avons
		\begin{equation}
			\bigcap_{i=1}^rH_i\subset \ker(f-\id).
		\end{equation}
		D'autre part, la proposition \ref{PROPooRCLNooJpIMMl} nous donne \( \dim\bigcap_iH_i\geq n-r\). Donc
		\begin{equation}
			n-r\leq \dim\big( \bigcap_{i=1}^r H_i\big)\leq\dim\big( \ker(f-\id) \big)=n-\rank(f-\id).
		\end{equation}
		Donc \( n-r\leq n-\rank(f-\id)\) ou encore
		\begin{equation}
			r\geq \rank(f-\id).
		\end{equation}

		\spitem[Pour \ref{ITEMooUCZWooSbyPwt}]
		La première partie de ce théorème n'est rien d'autre que le lemme~\ref{LEMooJCDRooGAmlwp} parce que le pire cas est celui où le fixateur de \( f\) est réduit à l'ensemble vide, et dans ce cas l'application \( f\) est une composition de \( n+1\) réflexions.
	\end{subproof}
\end{proof}

\begin{proposition}     \label{PROPooUSKEooUbNVfs}
	Un élément de \( \SO(3)\) qui fixe deux vecteurs linéairement indépendants est l'identité.
\end{proposition}

\begin{proof}
	Soit un élément \( A\in \SO(3)\) et deux vecteurs linéairement indépendants \( v_1,v_2\in \eR^3\) tels que \( Av_1=v_1\) et \( Av_2=v_2\). Puisque \( v_1\) et \( v_2\) sont linéairement indépendants, le théorème de la base incomplète \ref{ThonmnWKs} nous permet de considérer \( v_3\in \eR^3\) tel que \( \{ v_1,v_2,v_3 \}\) soit une base. Dans cette base, la matrice de \( A\) est de la forme
	\begin{equation}
		A=\begin{pmatrix}
			1 & 0 & a \\
			0 & 1 & b \\
			0 & 0 & c
		\end{pmatrix}.
	\end{equation}
	Le déterminant de cette matrice est \( c\). Or \( \det(A)=1\) parce qu'elle est dans \( \SO(3)\). Donc \( c=1\). Le fait que \( A\) soit orthogonale implique que la troisième colonne doit être un vecteur de norme \( 1\). Donc \( a=b=0\).

	Donc \( A=\id\).
\end{proof}

\begin{corollary}       \label{CORooJCURooSRzSFb}
	Tout élément de \( \SO(3)\) peut être écrit comme composée de deux réflexions.
\end{corollary}

\begin{proof}
	Un élément de \( \SO(3)\) est une isométrie de \( \eR^3\) parce que si \( A\in\SO(3)\) alors\footnote{Opérateur orthogonal, définition \ref{DEFooYKCSooURQDoS}.}
	\begin{equation}
		\langle Ax, Ay\rangle =\langle A^*Ax, y\rangle =\langle x, y\rangle .
	\end{equation}
	Donc si le rang de \( A\) est \( k\), alors \( A\) est la composée de \( 3-k\) réflexions par le lemme \ref{LEMooJCDRooGAmlwp}.

	Si \( A=\id\), c'est bon parce que l'identité est la composée de deux réflexions égales. Nous supposons que \( A\) n'est pas l'identité.

	Comme discuté dans l'exemple \ref{EXooIPLOooSNfiWg}, l'opérateur \( A\) possède trois valeurs propres dans \( \eC\) dont une réelle, et deux complexes conjuguées. Nous les notons \( \lambda\in \eR\) et \( \alpha,\bar \alpha\in \eC\). Le déterminant de \( A\), qui vaut \( 1\), est le produit de ces trois valeurs propres, c'est-à-dire \( \lambda| \alpha |^2\). En particulier \( \lambda>0\).

	Si \( v\) est un vecteur propre correspondant à la valeur propre \( \lambda\), nous avons \(  \| v \|= \| Av \|=| \lambda |\| v \|\) parce que \( A\) est une isométrie. Donc \( \lambda=\pm 1\).

	Au final, \( \lambda=1\). Cela signifie que \( A\) laisse au moins un vecteur invariant. Vu que \( A\) n'est pas l'identité, la proposition \ref{PROPooUSKEooUbNVfs} nous indique qu'il n'y a pas d'autres vecteurs de \( \eR^3\) à être fixé par \( A\). Donc \( \dim\big( \Fix(A) \big)=1\) et le lemme \ref{LEMooJCDRooGAmlwp}\ref{ITEMooFYEDooIJZBjP} s'écrit avec \( n=3\), \( k=2\) et implique que \( A\) est la composée de deux réflexions.
\end{proof}

\begin{lemma}       \label{LEMooMCVKooKzmlAg}
	Soit un hyperplan \( H\) et un vecteur \( v\) de \( \eR^n\). Nous avons
	\begin{equation}
		\tau_v\circ \sigma_H\circ\tau_v^{-1}=\sigma_{\tau_v(H)}.
	\end{equation}
\end{lemma}

\begin{proof}
	Pour ce faire, nous considérons une base adaptée. Les vecteurs \( \{ e_1,\ldots, e_{n-1} \}\) forment une base orthonormée de \( H_0\) et \( e_n\) complète en une base orthonormée de \( \eR^n\). Soit \( H_0\) l'hyperplan parallèle à \( H\) et passant par l'origine; nous avons, pour un certain \( \lambda\in \eR\),
	\begin{equation}
		H=H_0+\lambda e_n
	\end{equation}
	D'un autre côté, le vecteur \( v\) peut être décomposé en \( v=v_1+v_2\) où \( v_1\perp H\) et \( v_2\parallel H\). Alors
	\begin{equation}
		\tau_v(H)=H+v=H+v_2=H_0+\lambda e_n+v_2.
	\end{equation}
	Nous pouvons maintenant utiliser le lemme~\ref{LEMooWYVRooQmWqvM} pour exprimer la transformation \( \sigma_{\tau_v(H)}\) :
	\begin{equation}        \label{EQooNYKFooXprXav}
		\sigma_{\tau_v(H)}(x)=\sigma_{H_0}(x)+ 2\lambda e_n+2v_2
	\end{equation}

	Mais d'autre part,
	\begin{equation}
		(\tau_v\circ \sigma_H\circ\tau_{v}^{-1})(x)=v+\sigma_H(x-v)=v+\sigma_{H_0}(x-v)+2\lambda e_n.
	\end{equation}
	Vue la décomposition de \( v=v_1+v_2\) nous avons \( \sigma_{H_0}(v)=-v_1+v_2\) et donc
	\begin{equation}        \label{EQooGOHEooALPRFB}
		(\tau_v\circ \sigma_H\circ\tau_{v}^{-1})(x)= v+  \sigma_{H_0}(x)+v_1-v_2+2\lambda e_n=\sigma_{H_0}+2v_1+2\lambda e_n.
	\end{equation}
	Les expressions \eqref{EQooNYKFooXprXav} et \eqref{EQooGOHEooALPRFB} coïncident, d'où l'égalité recherchée.
\end{proof}

\begin{theorem}[\cite{ooZYLAooXwWjLa}]      \label{THOooWBIYooCtWoSq}
	Une isométrie de \( (\eR^n,d)\) préserve l'orientation si et seulement si est elle composition d'un nombre pair de réflexions.
\end{theorem}

\begin{proof}
	Nous définissons
	\begin{equation}
		\begin{aligned}
			\epsilon\colon \Isom(\eR^n) & \to \{ \pm 1 \}      \\
			\tau_v\circ \alpha          & \mapsto \det(\alpha)
		\end{aligned}
	\end{equation}
	où nous nous référons à la décomposition unique d'un élément de \( \Isom(\eR^n)\) sous la forme \( \tau_v\circ \alpha\) avec \( \alpha\in O(n)\) donnée par le théorème~\ref{THOooQJSRooMrqQct}\ref{ITEMooEWSIooNKzRxB}.

	Le noyau de \( \epsilon\) est alors la partie
	\begin{equation}
		\ker(\epsilon)=\eR^n\times_{\AD}\SO(n).
	\end{equation}
	Une isométrie \( f\) préserve l'orientation si et seulement si \( \epsilon(f)=1\). Puisque toutes les isométries sont des compositions de réflexions (première partie), il nous suffit de montrer que \( \epsilon(\epsilon_H)=-1\) pour qu'une isométrie préserve l'orientation si et seulement si elle est composition d'un nombre pair de réflexions.
	%TODOooIRXQooENJxEt: que signifie «préserve l'orientation» ?

	Nous commençons par prouver que pour tout vecteur \( v\), \( \epsilon\big( \sigma_H \big)=\epsilon\big( \sigma_{\tau_v(H)} \big)\). Pour cela nous utilisons le lemme~\ref{LEMooMCVKooKzmlAg} et le fait que \( \epsilon\) est un morphisme :
	\begin{equation}
		\epsilon(\sigma_{\tau_v(H)})=\epsilon(\tau_v)\epsilon(\sigma_H)\epsilon(\tau_v^{-1})=\epsilon(\sigma_H)
	\end{equation}
	parce que la partie linéaire d'une translation est l'identité (et donc \( \epsilon(\tau_v)=1\) pour tout \( v\)).

	Nous avons donc \( \epsilon(\sigma_H)=\epsilon(\sigma_{H_0})\). En ce qui concerne \( \sigma_{H_0}\), dans la base adaptée, la matrice est
	\begin{equation}
		\sigma_{H_0}=\begin{pmatrix}
			1      & 0      & \dots  & 0      \\
			0      & \ddots & \ddots & \vdots \\
			\vdots & \ddots & 1      & 0      \\
			0      & \dots  & 0      & -1
		\end{pmatrix},
	\end{equation}
	dont le déterminant est \( -1\).
\end{proof}

\begin{definition}[Isométrie positive et négative]      \label{DEFooOKGSooUhDIfu}
	Nous utilisons la bijection \( \psi\colon T(n)\times \gO(n)\to \Isom(\eR^n)\) décrite par le théorème \ref{THOooQJSRooMrqQct}, et nous définissons
	\begin{equation}
		\begin{aligned}
			\epsilon\colon \Isom(\eR^n) & \to \{ \pm 1 \}                         \\
			f                           & \mapsto \det\big( \psi^{-1}(f)_2 \big).
		\end{aligned}
	\end{equation}
	Nous disons que \( f\in\Isom(\eR^n)\) est \defe{positive}{isométrie positivé} si \( \epsilon(f)=1\), et nous notons
	\begin{equation}
		\Isom^+(\eR^n)=\epsilon^{-1}(1).
	\end{equation}
\end{definition}

\begin{lemma}       \label{LEMooVRELooESIWQl}
	La partie \( \Isom^+(\eR^n)\) des isométries positives est un sous-groupe de \( \Isom(\eR^n)\).
\end{lemma}

Pour en savoir plus sur le groupe des isométries, il faut lire le théorème de Cartan-Dieudonné dans \cite{JGAdTA}.

%+++++++++++++++++++++++++++++++++++++++++++++++++++++++++++++++++++++++++++++++++++++++++++++++++++++++++++++++++++++++++++
\section{Groupes finis d'isométries}
%+++++++++++++++++++++++++++++++++++++++++++++++++++++++++++++++++++++++++++++++++++++++++++++++++++++++++++++++++++++++++++

\begin{definition}      \label{DEFooCUYLooAlbtzv}
	Si \( X\) est une partie finie de \( \eR^n\), le \defe{barycentre}{barycentre!cas vectoriel} de \( X\) est le point
	\begin{equation}
		B_X=\frac{1}{ | X | }\sum_{x\in X}x
	\end{equation}
	où \( | X |\) est le cardinal de \( X\).
\end{definition}
Cela est à mettre en relation avec la définition dans le cadre affine~\ref{LemtEwnSH}.

\begin{lemma}[\cite{ooZYLAooXwWjLa}]        \label{LEMooSEZYooYceLIb}
	Les applications affines de \( \eR^n\) préservent le barycentre\footnote{Définition \ref{DEFooCUYLooAlbtzv}.} des parties finies.
\end{lemma}

\begin{proof}
	Soit une partie finie \( X\) de \( \eR^n\) et une application affine \( f\in\Aff(\eR^n)\). Nous devons prouver que
	\begin{equation}
		f(B_X)=B_{f(X)}.
	\end{equation}

	Nous savons que toute application affine est une composée de translation et d'une application linéaire : \( f=\tau_v\circ g\) avec \( v\in \eR^n\) et \( g\in \GL(n,\eR)\). Nous vérifions le résultat séparément pour \( \tau_v\) et pour \( g\).

	D'une part,
	\begin{equation}
		B_{\tau_v(X)}=\frac{1}{ | \tau_v(X) | }\sum_{y\in \tau_v(X)}y=\frac{1}{ | X | }\sum_{x\in X}(x+v)=B_x+\frac{1}{ | X | }\sum_{x\in X}v=B_x+v=\tau_v(B_X).
	\end{equation}
	Nous avons utilisé le fait que \( X\) et \( \tau_v(X)\) possèdent le même nombre d'éléments, ainsi que le fait d'avoir une somme de \( | X |\) termes tous égaux à \( v\).

	D'autre part,
	\begin{equation}
		B_{g(X)}=\frac{1}{ | X | }\sum_{x\in X}g(x)=g\big( \frac{1}{ |X | }\sum_{x\in X}x \big)=g(B_X)
	\end{equation}
	où nous avons utilisé la linéarité de \( g\) dans tous ses retranchements.
\end{proof}

\begin{proposition}     \label{PROPooLAEBooWdcBoe}
	Points fixes d'un sous-groupe.
	\begin{enumerate}
		\item
		      Soit \( H\) un sous-groupe fini des isométries de \( (\eR^n,d)\). Alors il existe \( v\in \eR^n\) tel que \( f(v)=v\) pour tout \( f\in H\).
		\item
		      Si \( H\) est un sous-groupe de \( \Isom(\eR^n,d)\) n'acceptant pas de point fixe, alors il est infini.
	\end{enumerate}
\end{proposition}

\begin{proof}
	Le groupe \( H\) agit sur \( \eR^n\), et si \( x\in \eR^n\) nous pouvons considérer son orbite
	\begin{equation}
		Hx=\{f(x)\tq f\in H\},
	\end{equation}
	qui est une partie finie de \( \eR^n\). Considérons son barycentre \( v=B_{Hx}\). Soit \( f\in H\). Alors
	\begin{subequations}
		\begin{align}
			f(v) & =f(B_{Hx})                                 \\
			     & =B_{f(Hx)}     \label{SUBEQooOQBZooYlIbgN} \\
			     & =B_{Hx}        \label{SUBEQooXWEGooSoezYg} \\
			     & =v,
		\end{align}
	\end{subequations}
	Justifications:
	\begin{itemize}
		\item Pour \eqref{SUBEQooOQBZooYlIbgN}, c'est le lemme \ref{LEMooSEZYooYceLIb}.
		\item Pour \eqref{SUBEQooXWEGooSoezYg}, c'est le fait que, \( f\in H\) étant donné, l'application \( g\mapsto fg\) est une bijection de \( H\), donc
		      \begin{equation}
			      f(Hx)=\{(fg)(x)\tq h\in H\}=\{g(x)\tq g\in H\}=Hx.
		      \end{equation}
	\end{itemize}
	Bref, \( v\) est fixé par \( H\).

	La seconde affirmation n'est rien d'autre que la contraposée de la première.
\end{proof}

\begin{proposition}     \label{PROPooEUFIooDUIYzi}
	À propos de groupes finis d'isométries.
	\begin{enumerate}
		\item
		      Tout sous-groupe fini de \( \Isom(\eR^n)\) est isomorphe à un sous-groupe fini de \( \gO(n)\).
		\item
		      Tout sous-groupe fini de \( \Isom^+(\eR^n)\) est isomorphe à un sous-groupe fini de \( \SO(n)\).
	\end{enumerate}
\end{proposition}

\begin{proof}
	Soit \( H\) un sous-groupe fini de \( \Isom(\eR^n)\) et \( v\in \eR^n\) un élément fixé par \( H\) (comme garanti par la proposition~\ref{PROPooLAEBooWdcBoe}). Nous posons
	\begin{equation}
		\begin{aligned}
			\phi\colon H & \to \Isom(\eR^n)                        \\
			f            & \mapsto \tau_v^{-1}\circ f\circ \tau_v.
		\end{aligned}
	\end{equation}

	\begin{subproof}
		\spitem[\( \phi\) est un morphisme]
		Les opération du type \( \phi=\AD(\tau_v)\) sont toujours des morphismes.
		\spitem[\( \phi\) consiste à extraire la partie linéaire]
		Si \( f=\tau_w\circ g\) alors
		\begin{subequations}
			\begin{align}
				\phi(f)(x) & =(\tau_{-v}\circ\tau_w\circ g\circ\tau_v)(x) \\
				           & =\tau_{w-v}(   g(x)+g(v)  )                  \\
				           & =g(x)+g(v)-v+w
			\end{align}
		\end{subequations}
		Mais \( g(v)+w=f(v)\) et nous savons que \( f(v)=v\). Donc il ne reste que \( \phi(f)(x)=g(x)\).
		\spitem[\( \phi\) est injective]
		Si \( f=\tau_w\circ g\) vérifie \( \phi(f)=\id\), il faut en particulier que \( g=\id\). Mais \( H\) est fini et ne peut donc pas contenir de translations non triviales. Donc \( w=0\) et \( f=\id\).
	\end{subproof}
	Donc \( \phi\) est une injection à valeur dans les transformations linéaires de \( \Isom(\eR^n)\). Autrement dit, \( \phi\) est un isomorphisme entre \( H\) et son image, laquelle image est dans \( \gO(n)\).

	En ce qui concerne la seconde partie, si \( f\in\Isom^+(\eR^n)\), alors \( \phi(f)\), qui est la partie linéaire de \( f\) (théorème \ref{THOooQJSRooMrqQct}\ref{ITEMooQLNPooSyHaps}), est dans \( \gO(n)\) avec un déterminant égal à \( 1\). Autrement dit, \( \phi(f)\in\SO(n)\).
\end{proof}

L'extraction de la partie linéaire est injective ? Certe c'est prouvé, mais on peut se demander ce qu'il se passe si \( H\) contient deux éléments qui ont la même partie linéaire. Cela n'est pas possible parce si \( f_1=\tau_{w_1}\circ g\) et \( f_2=\tau_{w_2}\circ g\) sont dans \( H\) alors \( f_1f_2^{-1}=\tau_{w_1+w_2}\) est également dans \( H\), ce qui n'est pas possible si \( H\) est fini.

%---------------------------------------------------------------------------------------------------------------------------
\subsection{Groupe diédral}
%---------------------------------------------------------------------------------------------------------------------------
\label{subsecHibJId}

\begin{proposition}     \label{PROPooUPPTooZBFvPg}
	Les racines de l'unité dans \( \eC\), c'est-à-dire la partie
	\begin{equation}
		\{  e^{2ik\pi/n},k=0,\ldots, n-1 \},
	\end{equation}
	forment un polynôme régulier.
\end{proposition}
% TODOooVDNMooJFwymI : prouver que les racines de l'unité forment un polygone régulier, PROPooUPPTooZBFvPg

%///////////////////////////////////////////////////////////////////////////////////////////////////////////////////////////
\subsubsection{Définition et générateurs : vue géométrique}
%///////////////////////////////////////////////////////////////////////////////////////////////////////////////////////////

\begin{definition}  \label{DEFooIWZGooAinSOh}
	Le \defe{groupe diédral}{groupe!diédral} \( D_n\)\nomenclature[R]{\( D_n\)}{groupe diédral} (\( n\geq 3\)) est le groupe des isométries de \( (\eC,d)\) laissant invariant l'ensemble
	\begin{equation}
		\{  e^{2ik\pi/n},k=0,\ldots, n-1 \}
	\end{equation}
	des racines de l'unité.
\end{definition}
\index{groupe!agissant sur un ensemble!diédral}
\index{groupe!en géométrie}
\index{groupe!fini!diédral}
\index{groupe!permutation!diédral}

\begin{normaltext}
	La proposition \ref{PROPooUPPTooZBFvPg} nous permet de dire que le groupe diédral est le groupe des isométries de \( \eR^2\) laissant invariant un polygone régulier à \( n\) côtés.
	C'est un peu pour cela que nous n'avons défini \( D_n\) que pour \( n\geq 3\); et un peu aussi pour une raison technique qui arrivera dans la preuve du lemme \ref{LEMooCUVPooMZKnzo}.
\end{normaltext}

\begin{lemma}       \label{LEMooCUVPooMZKnzo}
	Nous avons
	\begin{equation}
		D_n\subset O(2,\eR).
	\end{equation}
\end{lemma}

\begin{proof}
	Si \( f\in D_n\), alors \( f( e^{2ik\pi/n}) \) doit être l'un des \(  e^{2ik'\pi/n}\), et puisque \( f\) conserve les longueurs dans \( \eC\), nous devons avoir
	\begin{equation}
		1=d(0, e^{2ik\pi/n})=d\big( f(0), e^{2ik'\pi/n} \big).
	\end{equation}
	Donc \( f(0)\) est à l'intersection de tous les cercles de rayon \( 1\) centrés en les \(  e^{2ik\pi/n}\), ce qui montre que \( f(0)=0\) (dès que \( n\geq 3\)). Par conséquent notre étude du groupe diédral ne doit prendre en compte que les isométries vectorielles de \( \eR^2\). En d'autres termes
	\begin{equation}
		D_n\subset O(2,\eR).
	\end{equation}
\end{proof}

\begin{proposition}[\cite{tzHydF}]      \label{PROPooELOIooVJtuZN}
	Le groupe \( D_n\) contient un sous-groupe cyclique d'ordre \( 2\) et un sous-groupe cyclique d'ordre \( n\).
\end{proposition}

\begin{proof}
	Nous notons \( s\) la conjugaison complexe\footnote{C'est une réflexion; la réflexion d'axe \( \eR\) dans \( \eC\).}. C'est un élément d'ordre \( 2\) qui est dans \( D_n\) parce que
	\begin{equation}    \label{EqSUshknP}
		s\big(  e^{2ki\pi/n} \big)= e^{2(n-k)i\pi/n}.
	\end{equation}

	De la même façon, la rotations d'angle \(2\pi/n\), que l'on note \( r\), agit sur les racines de l'unité et engendre le groupe d'ordre \( n\) des rotations d'angle \(2 k\pi/n\).
\end{proof}

\begin{proposition}[\cite{tzHydF}]
	Si \( s\) est la conjugaison complexe et \( r\) la rotation d'angle \( 2\pi/n\), alors \( (sr)^2=\id\).
\end{proposition}

\begin{proof}
	Si \( z^n=1\), alors
	\begin{equation}
		(srsr)z=(srs)\big(  e^{2 i\pi/n}z \big)=(sr)\big(  e^{-2i\pi /n}\bar z \big)=s(\bar z)=z.
	\end{equation}
\end{proof}

\begin{proposition}[\cite{tzHydF}] \label{PropLDIPoZ}
	Nous notons \( s\) la conjugaison complexe et \( r\) la rotation d'angle \( 2\pi/n\).
	\begin{enumerate}
		\item
		      Le groupe diédral \( D_n\) est engendré par \( s\) et \( r\).
		\item       \label{ITEMooOEBHooULRmZk}
		      Tous les éléments de \( D_n\) s'écrivent sous la forme \( r^m\) ou \( s\circ r^m\).
	\end{enumerate}
\end{proposition}
\index{groupe!diédral!générateurs (preuve)}
\index{racine!de l'unité}
\index{géométrie!avec nombres complexes}
\index{géométrie!avec des groupes}
\index{isométrie!de l'espace euclidien \( \eR^2\)}

\begin{proof}
	Nous considérons les points \( A_0=1\) et \( A_k= e^{2ki\pi/n}\) avec \( k\in\{ 1,\ldots, n-1 \}\). Par convention, \( A_n=A_0\). L'action des éléments \( s\) et \( r\) sur ces points est
	\begin{subequations}
		\begin{align}
			r(A_k) & =A_{k+1}  \\
			s(A_k) & =A_{n-k}.
		\end{align}
	\end{subequations}
	Cette dernière est l'équation \eqref{EqSUshknP}.

	Soit \( f\in D_n\). Étant donné que c'est une isométrie de \( \eR^2\) avec un point fixe (le point \( 0\)), \( f\) est soit une rotation, soit une réflexion.
	%TODOooBYJRooIYcCFq : il faut démontrer ce point et mettre un lien vers ici.

	Supposons pour commencer qu'un des \( A_k\) soit fixé par \( f\). Dans ce cas \( f\) a deux points fixes : \( O\) et \( A_k\) et est donc la réflexion d'axe \( (OA_k)\). Dans ce cas, nous avons \( f=s\circ r^{n-2k}\). En effet
	\begin{equation}
		s\circ r^{n-2k}(A_k)=s(A_{k+n-2k})=s(A_{n-k})=A_k.
	\end{equation}
	Donc \( O\) et \( A_k\) sont deux points fixes de l'isométrie \( f\); donc \( f\) est bien la réflexion sur le bon axe.

	Nous passons à présent au cas où \( f\) ne fixe aucun des \( A_k\).
	\begin{enumerate}
		\item
		      Supposons que \( f\) soit une rotation. Si \( f(A_k)=A_m\), alors l'angle de la rotation est
		      \begin{equation}
			      \frac{ 2(m-k)\pi }{ n },
		      \end{equation}
		      et donc \( f=r^{m-k}\), qui est de la forme demandée.
		\item
		      Supposons à présent que \( f\) soit une réflexion d'axe \( \Delta\). Cette fois, \( \Delta\) ne passe par aucun des points \( A_k\), par contre \( \Delta\) passe par \( 0\). Nous commençons par montrer que \( \Delta\) doit être la médiatrice d'un des côtés \( [A_p,A_{p+1}]\) du polygone. Comme \( \Delta\) passe par \( O\) et n'est aucune des droites \( (OA_k)\), cette droite passe par l'intérieur d'un des triangles \( OA_pA_{p+1}\) et intersecte donc le côté correspondant.

		      Notre tâche est de montrer que \( \Delta\) coupe \( [A_p,A_{p+1}]\) en son milieu. Dans ce cas, \( \Delta\) sera automatiquement perpendiculaire parce que le triangle \( OA_pA_{p+1}\) est isocèle en \( O\). Nommons \( l\) la longueur des côtés du polygone \( P=\Delta\cap[A_p,A_{p+1}]\), \( x=d(A_p,P)\) et \( \delta=d(A_p,\Delta)\). Vu que \( f\) est la symétrie d'axe \( \Delta\), nous avons aussi \( d\big( f(A_p),\Delta \big)=\delta\) et \( d\big( A_p,f(A_p) \big)=2\delta\). D'autre part, par la définition de la distance, \( \delta<x\). Si \( x<\frac{ l }{2}\), alors \( \delta<\frac{ \delta }{2}\) et donc \( d\big( A_p,f(A_p) \big)<l\). Or cela est impossible parce que le polygone ne possède aucun sommet à distance plus courte que \( l\) de \( A_p\).

		      De la même manière si \( x>\frac{ l }{2}\), nous raisonnons avec \( A_{p+1}\) pour obtenir une contradiction. Nous en concluons que la seule possibilité est \( x=\frac{ l }{2}\), et donc \( f(A_p)=A_{p+1}\). Montrons alors que \( f=s\circ r^{n-2p-1}\). Il faut montrer que c'est une réflexion qui envoie \( A_p\) sur \( A_{p+1}\). D'abord c'est une réflexion parce que
		      \begin{equation}
			      \det(sr^{n-2p-1})=\det(s)\det(r^{n-2p-1})=-1
		      \end{equation}
		      parce que \( \det(s)=-1\) alors que \( \det(r^k)=1\) parce que \( r\) est une rotation dans \( \SO(2)\). Ensuite nous avons
		      \begin{equation}
			      s\circ r^{n-2p-1}(A_p)=s(A_{p+n-2p-1})=s(A_{n-p-1})=A_{n-(n-p-1)}=A_{p+1}.
		      \end{equation}

		      Donc \( s\circ r^{n-2p-1}\) est bien une réflexion qui envoie \( A_p\) sur \( A_{p+1}\).

	\end{enumerate}
\end{proof}

\begin{corollary}   \label{CorWYITsWW}
	La liste des éléments de \( D_n\) est
	\begin{equation}
		D_n=\{ 1,r,\ldots, r^{n-1},s,sr,\ldots, sr^{n-1} \}
	\end{equation}
	et \( | D_n |=2n\).
\end{corollary}

\begin{proof}
	Nous savons par la proposition~\ref{PropLDIPoZ} que tous les élément de \( D_n\) s'écrivent sous la forme \( r^k\) ou \( sr^k\). Puisque \( r\) est d'ordre \( n\), il ne faut considérer que \( k\in\{ 1,\ldots, n-1 \}\). Les éléments \( 1\), \( r\),\ldots, \( r^{n-1}\) sont tous différents, et sont (pour des raisons de déterminant) tous différents des \( sr^k\). Les isométries \( sr^k\) sont toutes différentes entre elles pour essentiellement la même raison :
	\begin{equation}
		sr^k(A_p)=s(A_{p+k})=A_{n-p+k}
	\end{equation}
	donc si \( k\neq k'\), \( sr^k(A_p)\neq sr^{k'}(A_p)\). La liste des éléments de \( D_n\) est donc
	\begin{equation}
		D_n=\{ 1,r,\ldots, r^{n-1},s,sr,\ldots, sr^{n-1} \}
	\end{equation}
	et donc \( | D_n |=2n\).
\end{proof}

\begin{example}     \label{EXooHNYYooUDsKnm}
	Nous considérons le carré \( ABCD\) dans \( \eR^2\) et nous cherchons les isométries de \( \eR^2\) qui laissent le carré invariant. Nous nommons les points comme sur la figure~\ref{LabelFigIsomCarre}. La symétrie d'axe vertical est nommée \( s\) et la rotation de \( 90\) degrés est notée \( r\).
	\newcommand{\CaptionFigIsomCarre}{Le carré dont nous étudions le groupe diédral.}
	\input{auto/pictures_tex/Fig_IsomCarre.pstricks}

	Il est facile de vérifier que toutes les symétries axiales peuvent être écrites sous la forme \( r^is\). De plus le groupe engendré par \( s\) agit sur le groupe engendré par \( r\) parce que
	\begin{equation}
		(srs^{-1})(A,B,C,D)=sr(B,A,D,C)=s(A,D,C,B)=(B,C,D,A),
	\end{equation}
	c'est-à-dire \( srs^{-1}=r^{-1}\). Nous sommes alors dans le cadre du corolaire~\ref{CoroGohOZ} et nous pouvons écrire que
	\begin{equation}
		D_4=\gr(r)\times_{\sigma}\gr(s).
	\end{equation}
\end{example}

%///////////////////////////////////////////////////////////////////////////////////////////////////////////////////////////
\subsubsection{Table de multiplication}
%///////////////////////////////////////////////////////////////////////////////////////////////////////////////////////////

La proposition \ref{PropLDIPoZ} nous indique que tous les éléments de \( D_n\) s'écrivent sous la forme \( s^{\epsilon}\circ r^m\) avec \( \epsilon\in\{ 0,1 \}\). Nous allons maintenant écrire la table de multiplication pour de telles transformations de \( \eC\).

\begin{lemma}       \label{LEMooBNJFooAbhsUa}
	Si \( R\) est une rotation autour de \( 0\) (dans \( \eC\)), et si \( s\) est la conjugaison complexe, alors
	\begin{equation}
		rs=sr^{-1}
	\end{equation}
\end{lemma}

\begin{proof}
	Il s'agit seulement d'un calcul en écrivant \( R\) comme la multiplication par \(  e^{i\alpha}\). Nous avons
	\begin{equation}
		(Rs)z= e^{i\alpha}\bar z=s\big(  e^{-i\alpha}z \big)=sR^{-1}z.
	\end{equation}
\end{proof}

\begin{proposition}     \label{PROPooPYDLooLgiUjk}
	Si \( \epsilon_1,\epsilon_2\in\{ 0,1 \}\) et si \( k,l\in \eZ\) nous avons
	\begin{equation}
		(s^{\epsilon_1}r^k)(s^{\epsilon_2}r^l)=s^{\epsilon_1+\epsilon_2}r^{l+(-1)^{\epsilon_1}k}.
	\end{equation}
\end{proposition}

\begin{proof}
	Si \( \epsilon_2=1\) alors nous utilisons le lemme \ref{LEMooBNJFooAbhsUa} pour trouver
	\begin{equation}
		(s^{\epsilon_1}r^k)(s^{\epsilon_2}r^l)=s^{\epsilon_1}(r^ks^{\epsilon_2})r^l=s^{\epsilon_1}s^{\epsilon_2}r^{-k}r^l.
	\end{equation}
	La proposition est déjà prouvée dans ce cas.

	Passons à \( \epsilon_2=0\). Dans ce cas nous avons
	\begin{equation}
		(s^{\epsilon_1}r^k)(s^{\epsilon_2}r^l)=s^{\epsilon_1}r^{k+l},
	\end{equation}
	et c'est bon.
\end{proof}

%///////////////////////////////////////////////////////////////////////////////////////////////////////////////////////////
\subsubsection{Générateurs : vue abstraite}
%///////////////////////////////////////////////////////////////////////////////////////////////////////////////////////////

\begin{normaltext}      \label{NORMooCCUEooRRENed}
	Nous allons montrer que \( D_n\) peut être décrit de façon abstraite en ne parlant que de ses générateurs. Nous considérons un groupe \( G\) engendré par des éléments \( a\) et \( b\) tels que
	\begin{enumerate}
		\item
		      \( a\) est d'ordre \( 2\),
		\item
		      \( b\) est d'ordre \( n\) avec \( n\geq 3\),
		\item
		      \( abab=e\).
	\end{enumerate}
	Nous allons prouver que ce groupe doit avoir la même liste d'éléments que celle du corolaire~\ref{CorWYITsWW}.
\end{normaltext}

\begin{proposition}[\cite{tzHydF}]
	Le groupe \( G\) n'est pas abélien.
\end{proposition}

\begin{proof}
	Nous savons que \( abab=e\), donc \( abab^{-1}=b^{-2}\), mais \( b^{-2}\neq e\) parce que \( b\) est d'ordre \( n>2\). Donc \( abab^{-1}\neq e\). En manipulant un peu :
	\begin{equation}
		e\neq abab^{-1}=(ab)(ba^{-1})^{-1}=(ab)(ba)^{-1}
	\end{equation}
	parce que \( a^{-1}=a\). Donc \( ab\neq ba\).
\end{proof}

\begin{lemma}[\cite{tzHydF}]        \label{LemKKXdqdL}
	Pour tout \( k\) entre \( 1\) et \( n-1\) nous avons
	\begin{equation}
		\AD(a)b^k=ab^ka^{-1}=ab^ka=b^{-k}.
	\end{equation}
\end{lemma}

\begin{proof}
	Nous faisons la démonstration par récurrence. D'abord pour \( k=1\), nous devons avoir \( aba=b^{-1}\), ce qui est correct parce que par construction de \( G\) nous avons \( abab=e\). Ensuite nous supposons que le lemme tient pour \( k\) et nous regardons ce qu'il se passe avec \( k+1\) :
	\begin{equation}
		ab^{k+1}ba=ab^kba=\underbrace{ab^ka}_{b^{-k}}\underbrace{aba}_{b^{-1}}=b^{-k}b^{-1}=b^{-(k+1)}.
	\end{equation}
\end{proof}

\begin{proposition}     \label{PROPooVQARooWuKHMZ}
	L'élément \( a\) n'est pas une puissance de \( b\).
\end{proposition}

\begin{proof}
	Supposons le contraire : \( a=b^k\). Dans ce cas nous aurions
	\begin{equation}
		e=(ab)(ab)=b^{k+1}b^{k+1}=b^{2k+2}=b^{2k}b^2=a^2b^2=b^2,
	\end{equation}
	ce qui signifierait que \( b\) est d'ordre \( 2\), ce qui est exclu par construction.
\end{proof}

\begin{proposition}[\cite{tzHydF}]      \label{PROPooEPVGooQjHRJp}
	La liste des éléments de \( G\) est donnée par
	\begin{equation}
		G=\{ 1,b,\cdots,b^{n-1},a,ab,\ldots, ab^{n-1} \}=\{ a^{\epsilon}b^k\}_{\substack{\epsilon=0,1\\k=0,\ldots, n-1}}
	\end{equation}
	Les éléments de ces listes sont distincts.
\end{proposition}

\begin{proof}
	Étant donné que \( a\) n'est pas une puissance de \( b\), les éléments \( 1\), \( a\), \( b\),\ldots, \( b^{n-1}\) sont distincts. De plus si \( k\) et \( m=k+p\) sont deux éléments distincts de \( \{ 1,\ldots, n-1 \}\), nous avons \( ab^k\neq ab^m\) parce que si \( ab^k=ab^{k+p}\), alors \( a=ab^p\) avec \( p<n\), ce qui est impossible. Pour la même raison, \( ab^k\neq e\), et \( ab^k\neq b^m\).

	Au final les éléments \( 1,a,b,\ldots, b^{n-1},ab,\ldots, ab^{n-1}\) sont tous différents. Nous devons encore voir qu'il n'y en a pas d'autres.

	Par définition le groupe \( G\) est engendré par \( a\) et \( b\), donc tout élément \( x\in G\) s'écrit \( x=a^{m_1}b^{k_1}\ldots a^{m_r}b^{k_r}\) pour un certain \( r\) et avec pour tout \( i\), \( k_i\in\{ 1,\ldots, n-1 \}\) (sauf \( k_r\) qui peut être égal à zéro) et \( m_i=1\), sauf \( m_1\) qui peut être égal à zéro. Donc
	\begin{equation}
		x=a^mb^{k_1}ab^{k_2}a\ldots b^{k_{r-1}}ab^{k_r}
	\end{equation}
	où \( m\) et \( k_r\) peuvent éventuellement être zéro. En utilisant le lemme~\ref{LemKKXdqdL} sous la forme \( b^{k_i}a=ab^{-k_i}\), quitte à changer les valeurs des exposants, nous pouvons passer tous les \( a \) à gauche et tous les \( b\) à droite pour finir sous la forme \( x=a^kb^m\).

	Donc non, il n'existe pas d'autres éléments dans \( G\) que ceux déjà listés.
\end{proof}

\begin{lemma}[\cite{MonCerveau}]        \label{LemooNFRIooPWuikH}
	Tout élément de \( G\) s'écrit de façon unique sous la forme \( a^{\epsilon}b^k\) ou \( b^ka^{\epsilon}\) avec \( \epsilon\in\{ 0,1 \}\) et \( k\in\{ 0,\ldots, n-1 \}\).
\end{lemma}

\begin{proof}
	Nous commençons par la forme \( a^{\epsilon}b^k\). L'existence est la proposition~\ref{PROPooEPVGooQjHRJp}. Pour l'unicité nous supposons \( a^{\epsilon}b^k=a^{\sigma}b^l\) et nous décomposons en \( 4\) cas distincts.
	\begin{subproof}
		\spitem[\( \epsilon=0\), \( \sigma=0\)]
		Alors \( b^k=b^l\). Mais \( b\) étant d'ordre \( n\) et \( k,l\) étant égaux au maximum à \( n-1\), cette égalité implique \( k=l\).
		\spitem[\( \epsilon=0\), \( \sigma=1\)]
		Alors \( b^k=ab^l\), ce qui donne \( a=b^{k-l}\), ce qui est interdit par la proposition~\ref{PROPooVQARooWuKHMZ}.
		\spitem[\( \epsilon=1\), \( \sigma=0\)]
		Même problème que ci-dessus.
		\spitem[\( \epsilon=1\), \( \sigma=1\)]
		Encore une fois \( b^k=b^l\) implique \( k=l\).
	\end{subproof}
	En ce qui concerne la forme \( b^ka^{\epsilon}\), l'existence est à montrer. Soit l'élément \( g=a^{\epsilon}b^k\); cherchons à le mettre sous la forme \( b^la^{\sigma}\). Si \( \epsilon=0\) c'est évident. Sinon \( \epsilon=1\) et nous avons par le lemme~\ref{LemKKXdqdL}
	\begin{equation}
		ab^k=b^{-k}a^{-1}=b^{-k}b^na=b^{-k}a.
	\end{equation}
	En ce qui concerne l'unicité, nous distinguons \( 4\) cas pour \( b^ka^{\epsilon}=b^la^{\sigma}\). Comme précédemment ils se traitement exactement comme précédemment.
\end{proof}

\begin{theorem}     \label{THOooYITHooTNTBuG}
	Les groupes \( G\) et \( D_n\) sont isomorphes.
\end{theorem}

\begin{proof}
	Nous utilisons l'application
	\begin{equation}
		\begin{aligned}
			\psi\colon G & \to D_n         \\
			a^kb^m       & \mapsto s^kr^m.
		\end{aligned}
	\end{equation}
	C'est évidemment bien défini et bijectif, mais c'est également un morphisme parce que si nous calculons \( \psi\) sur un produit, nous devons comparer
	\begin{equation}        \label{EqBULPilp}
		\psi\big( a^{k_1}b^{m_1}a^{k_2}b^{m_2} \big)
	\end{equation}
	avec
	\begin{equation}        \label{EqIVEIphI}
		\psi\big( a^{k_1}b^{m_1}\big)\psi\big(a^{k_2}b^{m_2} \big)= s^{k_1}r^{m_1}s^{k_2}r^{m_2}.
	\end{equation}
	Vu que \( D_n\) et \( G\) ont les mêmes propriétés qui permettent de permuter \( a\) et \( b\) ou \( s\) et \( r\), l'expression à l'intérieur du \( \psi\) dans \eqref{EqBULPilp} se simplifie en \( a^kb^m\) avec les même \( k\) et \( m\) que l'expression à droite dans \eqref{EqIVEIphI} et se simplifie en \( s^kr^m\).
\end{proof}

\begin{corollary}
	Toutes les propriétés démontrées pour \( G\) sont vraies pour \( D_n\). En particulier, avec quelques redites :
	\begin{enumerate}
		\item
		      Le groupe \( D_n\) peut être défini comme étant le groupe engendré par un élément \( s\) d'ordre \( 2\) et un élément \( r\) d'ordre \( n-1\) assujettis à la relation \( srsr=e\).
		\item
		      Le groupe \( D_n\) n'est pas abélien.
		\item
		      Pour tout \( k\in\{ 1,\ldots, n-1 \}\) nous avons \( sr^ks=r^{-k}\).
		\item
		      L'élément \( s\) ne peut pas être obtenu comme une puissance de \( r\).
		\item
		      La liste des éléments de \( D_n\) est
		      \begin{equation}
			      D_n=\{ 1,r,\ldots, r^{n-1},s,sr,\ldots, sr^{n-1} \}
		      \end{equation}
		\item
		      Le groupe diédral \( D_n\) est d'ordre \( 2n\).
	\end{enumerate}
\end{corollary}

\begin{proposition}
	En posant \( C_n=\{ r^k \}_{k=0,\ldots, n-1}\) et \( C_2=\{ a^{\epsilon} \}_{\epsilon=0,1}\), nous pouvons exprimer \( D_n\) comme le produit semi-direct
	\begin{equation}
		D_n=C_n\times_{\rho}C_2
	\end{equation}
	où \( \rho\) désigne l'action adjointe.
\end{proposition}

\begin{proof}
	L'isomorphisme est :
	\begin{equation}
		\begin{aligned}
			\psi\colon C_n\times_{\rho}C_2 & \to D_n                  \\
			(b^k,a^{\epsilon})             & \mapsto b^ka^{\epsilon}.
		\end{aligned}
	\end{equation}
	\begin{subproof}
		\spitem[Action adjointe]
		L'application \( \rho_{a^{\epsilon}}=\AD(a^{\epsilon})\) est toujours un morphisme. Comme \( a^{\epsilon}\) est, soit \( e\), soit \( a\), nous allons nous restreindre à \( a\) et oublier l'exposant \( \epsilon\). Il faut montrer que \( \AD(a)\in\Aut(C_n)\). En utilisant le lemme~\ref{LemKKXdqdL},
		\begin{equation}
			\AD(a)b^k=ab^ka^{-1}=b^{-k}=b^{n-k}.
		\end{equation}
		L'application \( \AD(a)\colon C_n\to C_n\) est donc bijective et homomorphique. Ergo isomorphisme.
		\spitem[Injectif]
		Si \( \psi(b^k,a^{\epsilon})=\psi(b^l,a^{\sigma})\), alors par unicité du lemme~\ref{LemooNFRIooPWuikH} nous avons \( k=l\) et \( \epsilon=\sigma\).
		\spitem[Surjectif]
		Par la partie «existence»  du lemme~\ref{LemooNFRIooPWuikH}.
		\spitem[Morphisme]
		Lorqu'on prend deux sous-groupes d'un même groupe (ici le groupe des isométries de \( \eR^2\)), et que l'on tente de faire un produit semi-direct en utilisant l'action adjointe, nous avons toujours un morphisme. Dans notre cas, le calcul est :
		\begin{equation}
			\psi\big( (b^k,a^{\epsilon})(b^l,a^{\sigma}) \big)=b^k\rho_{a^{\epsilon}}(b^l)a^{\epsilon+\sigma}=b^ka^{\epsilon}b^la^{-\epsilon}a^{\epsilon+\sigma}=b^ka^{\epsilon}b^la^{\sigma}=\psi(b^k,a^{\epsilon})\psi(b^l,a^{\sigma}).
		\end{equation}
	\end{subproof}
\end{proof}

%///////////////////////////////////////////////////////////////////////////////////////////////////////////////////////////
\subsubsection{Classes de conjugaison}
%///////////////////////////////////////////////////////////////////////////////////////////////////////////////////////////
\label{subsubsecZQnBcgo}

Pour les classes de conjugaison du groupe diédral nous suivons \cite{HRIMAJJ}.

D'abord pour des raisons de déterminants\footnote{Vous notez qu'ici nous utilisons un argument qui utilise la définition de \( D_n\) comme isométries de \( \eR^2\). Si nous avions voulu à tout prix nous limiter à la définition «abstraite» en termes de générateurs, il aurait fallu trouver autre chose.}, les classes des éléments de la forme \( r^k\) et de la forme \( sr^k\) ne se mélangent pas. Nous notons \( C(x)\) la classe de conjugaison de \( x\), et \( y\cdot x=yxy^{-1}\).

Les relations que nous allons utiliser sont
\begin{subequations}
	\begin{align}
		sr^ks=r^{-k} \\
		rs=sr^{-1}=sr^{n-1}.
	\end{align}
\end{subequations}

La classe de conjugaison qui ne rate jamais est bien entendu \( C(1)={1}\). Nous commençons les vraies festivités avec \( C(r^{m})\). D'abord \( r^k\cdot r^m=r^m\), ensuite
\begin{equation}
	(sr^k)\cdot r^m=sr^kr^mr^{-k}s^{-1}=sr^ms^{-1}=r^{-m}.
\end{equation}
Donc
\begin{equation}    \label{EqVFfFxgi}
	C(r^m)=\{ r^m,r^{-m} \}.
\end{equation}
À ce niveau il faut faire deux remarques. D'abord si \( m>\frac{ n }{2}\), alors \( C(r^m)\) est la classe de \( C^{n-m}\) avec \( n-m<\frac{ n }{2}\). Donc les classes que nous avons trouvées sont uniquement à lister avec \( m<\frac{ n }{2}\). Ensuite si \( m=\frac{ n }{2}\) alors \( r^m=r^{-m}\) et la classe est un singleton. Cela n'arrive que si \( n\) est pair.

Nous passons ensuite à \( C(s)\). Nous avons
\begin{equation}
	r^k\cdot s=r^ksr^{-k}=ssr^ksr^{-k}=sr^{-k}r^{-k}=sr^{n-2k},
\end{equation}
et
\begin{equation}
	(sr^k)\cdot s=\underbrace{sr^ks}_{r^{-k}}r^{-k}s^{-1}=r^{-2k}s=r^{n-2k}s=sr^{(n-1)(n-2k)}=sr^{n^2-2kn-n+2k}=sr^{2k}.
\end{equation}
donc
\begin{equation}
	C(s)=\{ sr^{n-2k},sr^{2k} \}_{k=0,\ldots, n-1}.
\end{equation}
Ici aussi l'écriture n'est pas optimale : peut-être que pour certains \( k\) il y a des doublons. Nous reportons l'écriture exacte à la discussion plus bas qui distinguera \( n\) pair de \( n\) impair. Notons juste que si \( n\) est pair, l'élément \( sr\) n'est pas dans la classe \( C(s)\).

Nous en faisons donc à présent le calcul en gardant en tête le fait qu'il n'a de sens que si \( n\) est pair. D'abord
\begin{equation}
	s\cdot (sr)=ssrs=rs=sr^{n-1}.
\end{equation}
Ensuite
\begin{equation}
	(sr^k)\cdot (sr)=sr^ksrr^{-k}s=r^{-2k+1}s=sr^{2k-1}.
\end{equation}
Avec \( k=\frac{ n }{2}\), cela rend \( s\cdot (sr)\), donc pas besoin de le recopier. Nous avons
\begin{equation}
	C(sr)=\{ sr^{2k-1} \}_{k=1,\ldots, n-1}.
\end{equation}

%///////////////////////////////////////////////////////////////////////////////////////////////////////////////////////////
\subsubsection{Le compte pour \(  n\) pair}
%///////////////////////////////////////////////////////////////////////////////////////////////////////////////////////////
\label{SubsubsecROVmHuM}

Si \( n\) est pair, nous avons les classes
\begin{subequations}
	\begin{align}
		C(1)       & =\{ 1 \}                                       &               &                   & 1\text{ élément}                               \\
		C(r^m)     & =\{ r^m,r^{m-1} \}                             & \text{ pour } & 0<m<\frac{ n }{2} & \frac{ n }{2}-1\text{ fois } 2\text{ éléments} \\
		C(r^{n/2}) & =\{ r^{n/2} \}                                 &               &                   & 1\text{ élément}                               \\
		C(s)       & =\{ sr^{2k} \}_{k=0,\ldots, \frac{ n }{2}-1}   &               &                   & \frac{ n }{2}\text{ éléments}                  \\
		C(sr)      & =\{ sr^{2k+1} \}_{k=0,\ldots, \frac{ n }{2}-1} &               &                   & \frac{ n }{2}\text{ éléments}.
	\end{align}
\end{subequations}
Au total nous avons bien listé \( 2n\) éléments comme il se doit, dans \(  \frac{ n }{2}+3\) classes différentes.

%///////////////////////////////////////////////////////////////////////////////////////////////////////////////////////////
\subsubsection{Le compte pour \(  n\) impair}
%///////////////////////////////////////////////////////////////////////////////////////////////////////////////////////////
\label{GJIzDEP}

Si \( n\) est impair, nous avons les classes
\begin{subequations}
	\begin{align}
		C(1)   & =\{ 1 \}                      &               &                     & 1\text{ élément}                               \\
		C(r^m) & =\{ r^m,r^{m-1} \}            & \text{ pour } & 0<m<\frac{ n-1 }{2} & \frac{ n-1 }{2}\text{ fois } 2\text{ éléments} \\
		C(s)   & =\{ sr^k \}_{k=0,\ldots, n-1} &               &                     & n\text{ éléments}
	\end{align}
\end{subequations}
Au total nous avons bien listé \( 2n\) éléments comme il se doit, dans \(  \frac{ n+3 }{2}\) classes différentes.

%---------------------------------------------------------------------------------------------------------------------------
\subsection{Applications : du dénombrement}
%---------------------------------------------------------------------------------------------------------------------------

%///////////////////////////////////////////////////////////////////////////////////////////////////////////////////////////
\subsubsection{Le jeu de la roulette}
%///////////////////////////////////////////////////////////////////////////////////////////////////////////////////////////
\label{pTqJLY}
\index{groupe!fini}
\index{groupe!de permutations}
\index{groupe!et géométrie}
\index{combinatoire}
\index{dénombrement}

Soit une roulette à \( n\) secteurs que nous voulons colorier en \( q\) couleurs\cite{HEBOFl}. Nous voulons savoir le nombre de possibilités à rotations près. Soit d'abord \( E\) l'ensemble des coloriages possibles sans contrainte; il y a naturellement \( q^n\) possibilités. Sur l'ensemble \( E\), le groupe cyclique \( G\) des rotations d'angle \( 2\pi/n\) agit. Deux coloriages étant identiques si ils sont reliés par une rotation, la réponse à notre problème est donnée par le nombre d'orbites de l'action de \( G\) sur \( E\) qui sera donnée par la formule du théorème de Burnside~\ref{THOooEFDMooDfosOw}.

Nous devons calculer \( \Card\big( \Fix(g) \big)\) pour tout \( g\in G\). Soit \( g\), un élément d'ordre \( d\) dans \( G\). Si \( g\) agit sur la roulette, chaque secteur a une orbite contenant \( d\) éléments. Autrement dit, \( g\) divise la roulette en \( n/d\) secteurs. Un élément de \( E\) appartenant à \( \Fix(g)\) doit colorier ces \( n/d\) secteurs de façon uniforme; il y a \( q^{n/d}\) possibilités.

Il reste à déterminer le nombre d'éléments d'ordre \( d\) dans \( G\). Un élément de \( G\) est donné par un nombre complexe de la forme \(  e^{2ik\pi/n}\). Les éléments d'ordre \( d\) sont les racines primitives\footnote{Une racine non primitive \( 8\)ième de l'unité est par exemple \( i\). Certes \( i^8=1\), mais \( i^4=1\) aussi. Le nombre \( i\) est d'ordre \( 4\).} \( d\)-ièmes de l'unité. Nous savons que --par définition-- il y a \( \varphi(d)\) telles racines primitives de l'unité. Bref il y a \( \varphi(d)\) éléments d'ordre \( d\) dans \( G\).

La formule de Burnside nous donne maintenant le nombre d'orbites :
\begin{equation}
	\frac{1}{ n }\sum_{d|n}\varphi(d)q^{n/d}.
\end{equation}
Cela est le nombre de coloriage possibles de la roulette à \( n\) secteurs avec \( q\) couleurs.

%///////////////////////////////////////////////////////////////////////////////////////////////////////////////////////////
\subsubsection{L'affaire du collier}
%///////////////////////////////////////////////////////////////////////////////////////////////////////////////////////////
\label{siOQlG}

Nous avons maintenant des perles de \( q\) couleurs différentes et nous voulons en faire un collier à \( n\) perles. Cette fois non seulement les rotations donnent des colliers équivalents, mais en outre les symétries axiales (il est possible de retourner un collier, mais pas une roulette). Le groupe agissant sur \( E\) est maintenant le groupe diédral\footnote{Définition~\ref{DEFooIWZGooAinSOh}.}\index{diédral}\index{groupe!diédral} \( D_n\) conservant un polygone à \( n\) sommets.

Nous devons séparer le cas \( n\) impair, du cas \( n\) pair.

Si \( n\) est impair, alors les axes de symétries passent par un sommet et par le milieu du côté opposé. Le groupe \( D_n\) contient \( n\) symétries axiales. Nous avons donc maintenant
\begin{equation}
	| G |=2n.
\end{equation}
Nous écrivons la formule de Burnside
\begin{equation}
	\Card(\Omega)=\frac{1}{ 2n }\sum_{g\in G}\Card\big( \Fix(g) \big).
\end{equation}
Si \( g\) est une rotation, le travail est déjà fait. Si \( g\) est une symétrie, nous avons le choix de la couleur du sommet par lequel passe l'axe et le choix de la couleur des \( (n-1)/2\) paires de sommets. Cela fait
\begin{equation}
	qq^{(n-1)/2}=q^{\frac{ n+1 }{2}}
\end{equation}
possibilités. Nous avons donc
\begin{equation}
	\Card(\Omega)=\frac{1}{ 2n }\left( \sum_{d|n}q^{n/d}\varphi(d)+nq^{\frac{ n+1 }{2}} \right).
\end{equation}

Si \( n\) est pair, le choses se compliquent un tout petit peu. En plus de symétries axiales passant par un sommet et le milieu du côté opposé, il y a les axes passant par deux sommets opposés. Pour colorier un collier en tenant compte d'une telle symétrie, nous pouvons choisir la couleur des deux perles par lesquelles passe l'axe ainsi que la couleur des \( (n-2)/2\) paires de perles. Cela fait en tout
\begin{equation}
	q^2q^{\frac{ n-2 }{2}}=q^{\frac{ n+2 }{2}}.
\end{equation}
Le groupe \( G\) contient \( n/2\) tels axes.

Notons que cette fois \( G\) ne contient plus que \( n/2\) symétries passant par un sommet et un côté. L'ordre de \( G\) est donc encore \( 2n\). La formule de Burnside donne
\begin{equation}
	\Card(\Omega)=\frac{1}{ 2n }\left( \sum_{d\divides n}\varphi(d)q^{n/d}+\frac{ n }{2}q^{(n+2)/2}+\frac{ n }{2}q^{n/2} \right).
\end{equation}

%---------------------------------------------------------------------------------------------------------------------------
\subsection{Points fixés par une affinité}
%---------------------------------------------------------------------------------------------------------------------------

\begin{lemma}[\cite{JGAdTA}]        \label{LEMooGUEGooTUXRsQ}
	Si \( n\geq 3\), alors toute droite est intersection de deux plans non isotropes.
\end{lemma}

\begin{proposition}[\cite{ooZYLAooXwWjLa}]      \label{PROPooVEEUooJQmmkN}
	Si une isométrie de \( \eR^n\) fixe un ensemble \( F\) de points, alors elle fixe l'espace affine engendrée par \( F\).
\end{proposition}

\begin{proof}
	Soit \( f\in \Isom(\eR^n)\) fixant \( F\). Par le théorème~\ref{ThoDsFErq}, c'est une application affine et l'ensemble \( \Fix(f)\) des points fixés par \( f\) est un sous-espace affine de \( \eR^n\), grâce à la proposition~\ref{PROPooYRCJooIcmUVI}.

	Donc \( \Fix(f)\) est un espace affine contenant \( F\). Puisque l'espace affine engendré par \( F\) est l'intersection de tous les espaces affines contenant \( F\), il est en particulier contenu dans \( \Fix(f)\).
\end{proof}

\begin{corollary}       \label{CORooZHZZooDgTzsW}
	Si \( f\) et \( g\) sont des isométries de \( \eR^n\) qui coïncident sur \( F\), alors elles coïncident sur l'espace affine engendré par \( F\).
\end{corollary}

\begin{proof}
	Nous considérons \( h=g^{-1}\circ f\) qui est une isométrie de \( \eR^n\) fixant \( F\). Elle fixe donc, par la proposition~\ref{PROPooVEEUooJQmmkN}, l'espace affine engendré par \( F\). Or tout point fixé par \( h\) est un point sur lequel \( g\) et \( f\) coïncident.
\end{proof}
