% This is part of Mes notes de mathématique
% Copyright (c) 2008-2017
%   Laurent Claessens, Carlotta Donadello
% See the file fdl-1.3.txt for copying conditions.

%+++++++++++++++++++++++++++++++++++++++++++++++++++++++++++++++++++++++++++++++++++++++++++++++++++++++++++++++++++++++++++
\section{Sommes de familles infinies}
%+++++++++++++++++++++++++++++++++++++++++++++++++++++++++++++++++++++++++++++++++++++++++++++++++++++++++++++++++++++++++++
\label{SECooHHDXooUgLhHR}

%---------------------------------------------------------------------------------------------------------------------------
\subsection{Convergence commutative}
%---------------------------------------------------------------------------------------------------------------------------

\begin{definition}
    Soit \( x_k\) une suite dans un espace vectoriel normé \( E\). Nous disons que la suite \defe{converge commutativement}{convergence!commutative} vers \( x\in E\) si \( \lim_{n\to \infty}\| x_n-x \| =0\) et si pour toute bijection \( \tau\colon \eN\to \eN\) nous avons aussi
    \begin{equation}
        \lim_{n\to \infty} \| x_{\tau(k)}-x \|=0.
    \end{equation}
    La notion de convergence commutative est surtout intéressante pour les séries. La somme
    \begin{equation}
        \sum_{k=0}^{\infty}x_k
    \end{equation}
    converge commutativement vers \( x\) si \( \lim_{N\to \infty} \| x-\sum_{k=0}^Nx_k \|=0\) et si pour toute bijection \( \tau\colon \eN\to \eN\) nous avons
    \begin{equation}
        \lim_{N\to \infty} \| x-\sum_{k=0}^Nx_{\tau(k)} \|=0.
    \end{equation}
\end{definition}

Nous démontrons maintenant qu'une série converge commutativement si et seulement si elle converge absolument.

Pour le sens inverse, nous avons la proposition suivante.
\begin{proposition}
    Soit \( \sum_{k=0}^{\infty}a_k\) une série réelle qui converge mais qui ne converge pas absolument. Alors pour tout \( b\in \eR\), il existe une bijection \( \tau\colon \eN\to \eN\) telle que \( \sum_{i=0}^{\infty}a_{\tau(i)}=b\).
\end{proposition}
Pour une preuve, voir \href{http://gilles.dubois10.free.fr/analyse_reelle/seriescomconv.html}{chez Gilles Dubois}.

\begin{proposition} \label{PopriXWvIY}
    Soit \( (a_i)_{i\in \eN}\) une suite dans \( \eC\) convergent absolument. Alors elle converge commutativement.
\end{proposition}

\begin{proof}
    Soit \( \epsilon>0\). Nous posons \( \sum_{i=0}^\infty a_i=a\) et nous considérons \( N\) tel que
    \begin{equation}
        | \sum_{i=0}^Na_i-a |<\epsilon.
    \end{equation}
    Étant donné que la série des \( | a_i |\) converge, il existe \( N_1\) tel que pour tout \( p,q>N_1\) nous ayons \( \sum_{i=p}^q| a_i |<\epsilon\). Nous considérons maintenant une bijection \( \tau\colon \eN\to \eN \). Prouvons que la série \( \sum_{i=0}^{\infty}| a_{\tau(i)} |\) converge. Nous choisissons \( M\) de telle sorte que pour tout \( n>M\), \( \tau(n)>N_1\); alors si \( p,q>M\) nous avons
    \begin{equation}
        \sum_{i=p}^q| a_{\tau(i)} |<\epsilon.
    \end{equation}
    Par conséquent la somme de la suite \( (a_{\tau(i)})\) converge. Nous devons montrer à présent qu'elle converge vers la même limite que la somme «usuelle» \( \lim_{N\to \infty} \sum_{i=0}^Na_i\).

    Soit \( n>\max\{ M,N \}\). Alors
    \begin{equation}
        \sum_{k=0}^na_{\tau(k)}-\sum_{k=0}^na_k=\sum_{k=0}^Ma_{\tau(k)}-\sum_{k=0}^Na_k+\underbrace{\sum_{M+1}^na_{\tau(k)}}_{<\epsilon}-\underbrace{\sum_{k=N+1}^na_k}_{<\epsilon}.
    \end{equation}
    Par construction les deux derniers termes sont plus petits que \( \epsilon\) parce que \( M\) et \( N\) sont les constantes de Cauchy pour les séries \( \sum a_{\tau(i)}\) et \( \sum a_i\). Afin de traiter les deux premiers termes, quitte à redéfinir \( M\), nous supposons que \( \{ 1,\ldots, N \}\subset \tau\{ 1,\ldots, M \}\); par conséquent tous les \( a_i\) avec \( i<N\) sont atteints par les \( a_{\tau(i)}\) avec \( i<M\). Dans ce cas, les termes qui restent dans la différence
    \begin{equation}
        \sum_{k=0}a_{\tau(k)}-\sum_{k=0}^Na_k
    \end{equation}
    sont des \( a_k\) avec \( k>N\). Cette différence est donc en valeur absolue plus petite que \( \epsilon\), et nous avons en fin de compte que
    \begin{equation}
        \left| \sum_{k=0}^na_{\tau(k)}-\sum_{k=0}^na_k \right| <\epsilon.
    \end{equation}
\end{proof}

\begin{proposition}     \label{PropyFJXpr}
    Soit \( \sum_{i=0}^{\infty}a_i\) une série qui converge mais qui ne converge pas absolument. Pour tout \( b\in \eR\), il existe une bijection \( \tau\colon \eN\to \eN\) telle que \( \sum_{i=}^{\infty}a_{\tau(i)}=b\).
\end{proposition}

Les propositions \ref{PopriXWvIY} et \ref{PropyFJXpr} disent entre autres qu'une série dans \( \eC\) est commutativement sommable si et seulement si elle est absolument sommable.

Soit \( (a_i)_{i\in I}\) une famille de nombres complexes indexée par un ensemble \( I\) quelconque. Nous allons nous intéresser à la somme \( \sum_{i\in I}a_i\).


Soit \( \{ a_i \}_{i\in I}\) des nombres positifs. Nous définissons la somme
\begin{equation}
    \sum_{i\in I}a_i=\sup_{ J\text{ fini}}\sum_{j\in J}a_j.
\end{equation}
Notons que cela est une définition qui ne fonctionne bien que pour les sommes de nombres positifs. Si \( a_i=(-1)^i\), alors selon la définition nous aurions \( \sum_i(-1)^i=\infty\). Nous ne voulons évidemment pas un tel résultat.

Dans le cas de familles de nombres réels positifs, nous avons une première définition de la somme. 
\begin{definition}  \label{DefHYgkkA}
Soit \( (a_i)_{i\in I}\) une famille de nombres réels positifs indexés par un ensemble quelconque \( I\). Nous définissons
\begin{equation}
    \sum_{i\in I}a_i=\sup_{ J\text{ fini dans } I}\sum_{j\in J}a_j.
\end{equation}
\end{definition}

\begin{definition}  \label{DefIkoheE}
    Si \( \{ v_i \}_{i\in I}\) est une famille de vecteurs dans un espace vectoriel normé indexée par un ensemble quelconque \( I\). Nous disons que cette famille est \defe{sommable}{famille!sommable} de somme \( v\) si pour tout \( \epsilon>0\), il existe un \( J_0\) fini dans \( I\) tel que pour tout ensemble fini \( K\) tel que \( J_0\subset K\) nous avons
    \begin{equation}
        \| \sum_{j\in K}v_j-v \|<\epsilon.
    \end{equation}
\end{definition}
Notons que cette définition implique la convergence commutative.

\begin{example}
    La suite \( a_i=(-1)^i\) n'est pas sommable parce que quel que soit \( J_0\) fini dans \( \eN\), nous pouvons trouver \( J\) fini contenant \( J_0\) tel que \( \sum_{j\in J}(-1)^j>10\). Pour cela il suffit d'ajouter à \( J_0\) suffisamment de termes pairs. De la même façon en ajoutant des termes impairs, on peut obtenir \( \sum_{j\in J'}(-1)^i<-10\).
\end{example}

\begin{example}
    De temps en temps, la somme peut sortir d'un espace. Si nous considérons l'espace des polynômes \( \mathopen[ 0 , 1 \mathclose]\to \eR\) muni de la norme uniforme, la somme de l'ensemble
    \begin{equation}
        \{ 1,-1,\pm\frac{ x^n }{ n! } \}_{n\in \eN}
    \end{equation}
    est zéro.

    Par contre la somme de l'ensemble \( \{ 1,\frac{ x^n }{ n! } \}_{n\in \eN}\) est l'exponentielle qui n'est pas un polynôme.
\end{example}

\begin{example}
    Au sens de la définition \ref{DefIkoheE} la famille
    \begin{equation}
        \frac{ (-1)^n }{ n }
    \end{equation}
    n'est pas sommable. En effet la somme des termes pairs est \( \infty\) alors que la somme des termes impairs est \( -\infty\). Quel que soit \( J_0\in \eN\), nous pouvons concocter, en ajoutant des termes pairs, un \( J\) avec \( J_0\subset J\) tel que \( \sum_{j\in J}(-1)^j/j\) soit arbitrairement grand. En ajoutant des termes négatifs, nous pouvons également rendre \( \sum_{j\in J}(-1)^j/j\) arbitrairement petit.
\end{example}

\begin{proposition} \label{PropVQCooYiWTs}
    Si \( (a_{ij})\) est une famille de nombres positifs indexés par \( \eN\times \eN\) alors
    \begin{equation}
        \sum_{(i,j)\in \eN^2}a_{ij}=\sum_{i=1}^{\infty}\Big( \sum_{j=1}^{\infty}a_{ij} \Big)
    \end{equation}
    où la somme de gauche est celle de la définition \ref{DefHYgkkA}.
\end{proposition}
%TODO : cette proposition peut être vue comme une application de Fubini pour la mesure de comptage. Le faire et référentier ici.

\begin{proof}
    Nous considérons \( J_{m,n}=\{ 0,\ldots, m \}\times \{ 0,\ldots, n \}\) et nous avons pour tout \( m\) et \( n\) :
    \begin{equation}
        \sum_{(i,j)\in \eN^2}a_{ij}\geq \sum_{(i,j)\in J_{m,n}}a_{ij}=\sum_{i=1}^m\Big( \sum_{j=1}^na_{ij} \Big).
    \end{equation}
    Si nous fixons \( m\) et que nous prenons la limite \( n\to \infty\) (qui commute avec la somme finie sur \( i\)) nous trouvons
    \begin{equation}
        \sum_{(i,j)\in \eN^2}a_{ij}\geq =\sum_{i=1}^m\Big( \sum_{j=1}^{\infty}a_{ij} \Big).
    \end{equation}
    Cela étant valable pour tout \( m\), c'est encore valable à la limite \( m\to \infty\) et donc
    \begin{equation}
        \sum_{(i,j)\in \eN^2}a_{ij}\geq \sum_{i=1}^{\infty}\Big( \sum_{j=1}^{\infty}a_{ij} \Big).
    \end{equation}
    
    Pour l'inégalité inverse, il faut remarquer que si \( J\) est fini dans \( \eN^2\), il est forcément contenu dans \( J_{m,n}\) pour \( m\) et \( n\) assez grand. Alors
    \begin{equation}
        \sum_{(i,j)\in J}a_{ij}\leq \sum_{(i,j)\in J_{m,n}}a_{ij}=\sum_{i=1}^m\sum_{j=1}^na_{ij}\leq \sum_{i=1}^{\infty}\Big( \sum_{j=1}^{\infty}a_{ij} \Big).
    \end{equation}
    Cette inégalité étant valable pour tout ensemble fini \( J\subset \eN^2\), elle reste valable pour le supremum.    
\end{proof}

La définition générale de la somme \ref{DefIkoheE} est compatible avec la définition usuelle dans les cas où cette dernière s'applique.
\begin{proposition}[commutative sommabilité]\label{PropoWHdjw}
    Soit \( I\) un ensemble dénombrable et une bijection \( \tau\colon \eN\to I\). Soit \( (a_i)_{i\in I}\) une famille dans un espace vectoriel normé. Alors
    \begin{equation}
        \sum_{k=0}^{\infty}a_{\tau(k)}=\sum_{i\in I}a_i
    \end{equation}
    dès que le membre de droite existe. Le membre de gauche est définit par la limite usuelle.
\end{proposition}

\begin{proof}
    Nous posons \( a=\sum_{i\in I}a_i\). Soit \( \epsilon>0\) et \( J_0\) comme dans la définition. Nous choisissons
    \begin{equation}
        N>\max_{j\in J_0}\{ \tau^{-1}(j) \}.
    \end{equation}
    En tant que sommes sur des ensembles finis, nous avons l'égalité
    \begin{equation}
        \sum_{k=0}^Na_{\tau(k)}=\sum_{j\in J_0}a_j
    \end{equation}
    où \( J\) est un sous-ensemble de \( I\) contenant \( J_0\). Soit \( J\) fini dans \( I\) tel que \( J_0\subset J\). Nous avons alors
    \begin{equation}
        \| \sum_{k=0}^Na_{\tau(k)}-a \|=\| \sum_{j\in J}a_j-a \|<\epsilon.
    \end{equation}
    Nous avons prouvé que pour tout \( \epsilon\), il existe \( N\) tel que \( n>N\) implique \( \| \sum_{k=0}^na_{\tau(k)}-a\| <\epsilon\).
\end{proof}

\begin{corollary}
    Nous pouvons permuter une somme dénombrable et une fonction linéaire continue. C'est à dire que si \( f\) est une fonction linéaire continue sur l'espace vectoriel normé \( E\) et \( (a_i)_{i\in I}\) une famille sommable dans \( E\) alors
    \begin{equation}
        f\left( \sum_{i\in I}a_i \right)=\sum_{i\in I}f(a_i).
    \end{equation}
\end{corollary}

\begin{proof}
    En utilisant une bijection \( \tau\) entre \( I\) et \( \eN\) avec la proposition \ref{PropoWHdjw} ainsi que le résultat connu à propos des sommes sur \( \eN\), nous avons
    \begin{subequations}
        \begin{align}
            f\left( \sum_{i\in I}a_i \right)&=f\left( \sum_{k=0}^{\infty}a_{\tau(k)} \right)\\
            &=\sum_{k=0}^{\infty}f(a_{\tau(k)}) \label{SUBEQooCVUTooPmnHER}\\
            &=\sum_{i\in I}f(a_i).
        \end{align}
    \end{subequations}
    Notons que le passage à \eqref{SUBEQooCVUTooPmnHER} n'est pas du tout une trivialité à deux francs cinquante. Il s'agit d'écrire la somme comme la limite des sommes partielles, et de permuter \( f\) avec la limite en invoquant la continuité, puis de permuter \( f\) avec la somme partielle en invoquant sa linéarité.

    Ah, tiens et tant qu'on y est à dire qu'il y a des chose évidentes qui ne le sont pas, oui, il existe des applications linéaires non continues, voir le thème \ref{THEMEooYCBUooEnFdUg}.
\end{proof}

La proposition suivante nous enseigne que les sommes infinies peuvent être manipulée de façon usuelle.
\begin{proposition} \label{PropMpBStL}
    Soit \( I\) un ensemble dénombrable. Soient \( (a_i)_{i\in I}\) et \( (b_i)_{i\in I}\), deux familles de réels positifs telles que \( a_i<b_i\) et telles que \( (b_i)\) est sommable. Alors \( (a_i)\) est sommable.

    Si \( (a_i)_{i\in I}\) est une famille de complexes telle que \( (| a_i |)\) est sommable, alors \( (a_i)\) est sommable.
\end{proposition}

\begin{proposition}[\cite{MonCerveau}]     \label{PROPooWLEDooJogXpQ}
    Soit un espace vectoriel normé \( E\) et une famille sommable\footnote{Définition \ref{DefIkoheE}.} \( \{ v_i \}_{i\in I}\) d'éléments de \( E\). Soit \( f\colon E\to \eC\) une application sur laquelle nous supposons
    \begin{enumerate}
        \item
            \( f\) est linéaire et continue;
        \item
            la partie \( \{ f(v_i)_{i\in I} \} \) est sommable.
    \end{enumerate}
    Alors nous pouvons permuter la somme et \( f\) :
    \begin{equation}        \label{EQooONHXooKqIEbY}
        f\big( \sum_{i\in I}v_i \big)=\sum_{i\in I}f(v_i).
    \end{equation}
\end{proposition}

\begin{proof}
    Soit \( \epsilon>0\); vu que les familles \( \{ v_i \}_{i\in I}\) et \( \{ f(v_i) \}_{i\in I}\) sont sommables, nous pouvons considérer les parties finies \( J_1\) et \( J_2\) de \( I\) telles que
    \begin{equation}
        \big\| \sum_{j\in J_1}v_j-\sum_{i\in I}v_i \big\|\leq \epsilon
    \end{equation}
    et
    \begin{equation}
        \big\| \sum_{j\in J_2}f(v_j)-\sum_{i\in I}f(v_i) \big\|\leq \epsilon
    \end{equation}
    Ensuite nous posons \( J=J_1\cup J_2\). Avec cela nous calculons un peu avec les majorations usuelles :
    \begin{equation}
        \| f(\sum_{i\in I}v_i) -\sum_{i\in I}f(v_i) \|\leq \| f(\sum_{i\in I}v_i)- f(\sum_{j\in J}v_j) \|+  \| f(\sum_{j\in J}v_j)-\sum_i\in If(v_i) \|.
    \end{equation}
    Le second terme est majoré par \( \epsilon\), tandis que le premier, en utilisant la linéarité de \( f\) possède la majoration
    \begin{equation}
        \| f(\sum_{i\in I}v_i)- f(\sum_{j\in J}v_j) \|=\| f(\sum_{i\in I}v_i-\sum_{j\in J}v_j) \|\leq \| f \| \| \sum_{i\in I}v_i- \sum_{j\in J}v_j\|\leq \epsilon\| f \|.
    \end{equation}
    Donc pour tout \( \epsilon>0\) nous avons
    \begin{equation}
        \| f(\sum_{i\in I}v_i) -\sum_{i\in I}f(v_i) \|\leq \epsilon(1+\| f \|).
    \end{equation}
    D'où l'égalité \eqref{EQooONHXooKqIEbY}.
\end{proof}


%+++++++++++++++++++++++++++++++++++++++++++++++++++++++++++++++++++++++++++++++++++++++++++++++++++++++++++++++++++++++++++
\section{Fonctions}		\label{Sect_fonctions}
%+++++++++++++++++++++++++++++++++++++++++++++++++++++++++++++++++++++++++++++++++++++++++++++++++++++++++++++++++++++++++++

Soient $(V,\| . \|_V)$ et $(W,\| . \|_W)$ deux espaces vectoriels normés, et une fonction $f$ de $V$ dans $W$. Il est maintenant facile de définir les notions de limites et de continuité pour de telles fonctions en copiant les définitions données pour les fonctions de $\eR$ dans $\eR$ en changeant simplement les valeurs absolues par les normes sur $V$ et $W$.

La caractérisation suivante est un recopiage de la définition \ref{DefOLNtrxB} lorsque la topologie est donnée par des boules.
\begin{proposition}\label{PropHOCWooSzrMjl}
	Soit $f\colon V\to W$ une fonction de domaine \( \Domaine(f)\subset V\) et soit $a$ un point d'accumulation de $\Domaine(f)$. 
    La fonction \( f\) admet une limite en $a\in V$ si et seulement s'il existe un élément $\ell\in W$ tel que pour tout $\varepsilon>0$, il existe un $\delta>0$ tel que pour tout $x\in \Domaine(f)$,
    \begin{equation}        \label{EqDefLimzxmasubV}
		0<\| x-a \|_V<\delta\,\Rightarrow\,\| f(x)-\ell \|_W<\varepsilon.
	\end{equation}
	Dans ce cas, nous écrivons $\lim_{x\to a} f(x)=\ell$ et nous disons que $\ell$ est la \defe{limite}{limite} de $f$ lorsque $x$ tend vers $a$.
\end{proposition}

\begin{remark}
    Le fait que nous limitions la formule \eqref{EqDefLimzxmasubV} aux \( x\) dans le domaine de \( f\) n'est pas anodin. Considérons la fonction \( f(x)=\sqrt{x^2-4}\), de domaine \( | x |\geq 2\). Nous avons
    \begin{equation}
        \lim_{x\to 2} \sqrt{x^2-4}=0.
    \end{equation}
    Nous ne pouvons pas dire que cette limite n'existe pas en justifiant que la limite à gauche n'existe pas. Les points \( x<2\) sont hors du domaine de \( f\) et ne comptent dons pas dans l'appréciation de l'existence de la limite.

    Vous verrez plus tard que ceci provient de la \wikipedia{fr}{Topologie_induite}{topologie induite} de \( \eR\) sur l'ensemble \( \mathopen[ 2 , \infty [\).
\end{remark}
