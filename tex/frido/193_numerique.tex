% This is part of Mes notes de mathématique
% Copyright (C) 2010-2018, 2022
%   Laurent Claessens
% See the file LICENCE.txt for copying conditions.

\begin{probleme}
    Dans toute la partie sur la méthode des différences finies, il y a un flottement entre \( \Omega\) et \( \bar \Omega\).

    D'une part je ne vois pas bien pourquoi on ne peut pas se contenter de travailler avec une fonction \( u\colon \Omega\to \eR\) avec \( \Omega\) ouvert dans \( \eR\) et \( u\) même pas définie sur le bord.

    Mais d'autre part, de nombreuses sources demandent de la régularité sur un fermé, à commencer par wikipédia :

    \url{https://fr.wikipedia.org/wiki/Méthode_des_différences_finies}

    Si vous avez une idée sur la question, écrivez-moi, ou répondez directement sur la page de discussion de la page Wikipédia, sur laquelle j'ai laissé une question.
\end{probleme}

%+++++++++++++++++++++++++++++++++++++++++++++++++++++++++++++++++++++++++++++++++++++++++++++++++++++++++++++++++++++++++++
\section{Problèmes de dimension un}
%+++++++++++++++++++++++++++++++++++++++++++++++++++++++++++++++++++++++++++++++++++++++++++++++++++++++++++++++++++++++++++

Soit \( u\colon \eR\to \eR\) une fonction et \( h>0\). Nous définissons les opérations suivantes (qui sont supposées approximer la dérivée \( u'(x)\) lorsqu'elle existe).

\begin{definition}
	La \defe{différence progressive}{différence!progressive} est
	\begin{equation}
		(D^+_hu)(x)=\frac{ u(x+h)-u(x) }{ h },
	\end{equation}
	la \defe{différence régressive}{différence!régressive} est
	\begin{equation}
		(D^-_hu)(x)=\frac{ u(x)-u(x-h) }{ h },
	\end{equation}
	la \defe{différence centrée}{différence!centrée} est
	\begin{equation}
		(D^0_hu)(x)=\frac{ u(x+h)-u(x-h) }{ 2h }.
	\end{equation}
\end{definition}
Nous ne noterons pas toujours la dépendance en \( h\), c'est-à-dire que nous noterons \( D^+u\) au lieu de \( D^+_hu\) lorsque cela ne pose pas de problème.

Notons que \( u''\) peut être approximé par \( D_h^+D_h^+u\), \( D_h^0D_h^+\), \( D_h^+D_h^-\), et encore de nombreuses autres possibilités.

Voici un lemme qui dit que tout cela n'est pas si mal, pourvu que \( u\) soit assez régulière.

\begin{lemma}       \label{LEMooZECZooVKxOZZ}
	Soit un ouvert connexe \( \Omega\) de \( \eR\), soit \( x\in \Omega\) et \( h>0\) tel que \( \overline{ B(x,h) }\subset \Omega\).
	\begin{enumerate}
		\item
		      Si \( u\in C^2( \Omega )\) alors
		      \begin{equation}
			      | u'(x)-D_h^+u(x) |\leq \frac{ h }{2}\| u'' \|_{\Omega}
		      \end{equation}
		      et
		      \begin{equation}
			      | u'(x)-D_h^-u(x) |\leq \frac{ h }{2}\| u'' \|_{\Omega}.
		      \end{equation}
		\item       \label{ITEMooSAWJooJUTWAb}
		      Si \( u\in C^3(\Omega)\) alors
		      \begin{equation}
			      | u'(x)-D_h^0(x) |\leq \frac{ h^2 }{2}\| u^{(3)} \|_{\Omega}
		      \end{equation}
		\item       \label{ITEMooRWUHooZJLKuL}
		      Si \( u\in C^4(\Omega)\) alors
		      \begin{equation}
			      | u''(x)-D_h^-D_h^+u(x) |\leq \frac{ h^2 }{ 12 }\| u^{(4)} \|_{\Omega}.
		      \end{equation}
	\end{enumerate}
\end{lemma}

\begin{proof}
	Nous prouvons le point~\ref{ITEMooRWUHooZJLKuL}. D'abord nous regardons de quoi nous avons besoin :
	\begin{equation}        \label{EQooBLIIooWHXbqD}
		D^-D^+u(x)=\frac{ (D^+u)(x)-(D^+u)(x-h) }{ h }=\frac{ u(x+h)-2u(x)+u(x-h) }{ h^2 }
	\end{equation}
    Nous allons y mettre les approximations de \( u(x+h)\) et \( u(x-h)\) par Taylor, proposition~\ref{PropResteTaylorc}\ref{ITEMooVGBVooGXXvIz} :
	\begin{equation}
		u(x+h)=u(x)+hu'(x)+\frac{ h^2 }{2}u''(x)+\frac{ h^3 }{ 6 }u'''(x)+\frac{ h^4 }{ 24 }u^{(4)}(x+\theta_1h)
	\end{equation}
	avec \( \theta_1\in \mathopen[ 0 , 1 \mathclose]\). De même,
	\begin{equation}
		u(x-h)=u(x)-hu'(x)+\frac{ h^2 }{2}u''(x)-\frac{ h^3 }{ 6 }u'''(x)+\frac{ h^4 }{ 24 }u^{(4)}(x-\theta_2h)
	\end{equation}
	avec \( \theta_2\in \mathopen[ 0 , 1 \mathclose]\).

	Donc
	\begin{equation}
		u(x+h)+u(x-h)-2u(x)=h^2u''(x)+\frac{ h^4 }{ 4! }\Big( u^{(4)}(x+\theta_1h)+u^{(4)}(x-\theta_2h) \Big),
	\end{equation}
	ce qui donne
	\begin{equation}
		(D^-D^+u)(x)=u''(x)+\frac{ h^2 }{ 4! }\Big( u^{(4)}(x+\theta_1h)+u^{(4)}(x-\theta_2h) \Big).
	\end{equation}
    Chacun des deux termes dans la parenthèse peut être majoré par \( \| u^{(4)} \|_{\Omega}\) parce que \( x+\theta_1h\) ne prend ses valeurs que dans \( \mathopen[ x , x+h \mathclose]\subset\overline{ B(x,h) }\subset\Omega\). Quoi qu'il en soit nous ne pouvons pas dire mieux que
	\begin{equation}
		| u''(x)-D^-D^+u(x) |\leq \frac{ h^2 }{ 12 }\| u^{(4)} \|_{\overline{ \Omega }}.
	\end{equation}
\end{proof}

\begin{remark}[\cite{MonCerveau}]
    Si nous avons l'égalité
	\begin{equation}        \label{EQooHSPFooTJIoFy}
		| u'(x)-D_h^+u(x) |\leq \delta
	\end{equation}
	pour tout \( x\), il faut faire attention en écrivant
	\begin{equation}
		\| u'-D_h^+u \|_{\infty}\leq \delta
	\end{equation}
    parce que l'inégalité \eqref{EQooHSPFooTJIoFy} n'est valable que pour les \( x\) tels que \( \mathopen[ x-h , x+h \mathclose]\subset \Omega\), de telle sorte que l'inégalité n'est pas spécialement correcte sur \( \overline{ \Omega }\). Il faut donc d'abord se mettre d'accord sur ce que signifie \( \| . \|_{\infty}\). Est-ce une norme supremum sur \( \Omega\) ou sur \( \bar \Omega\) ?
\end{remark}

%---------------------------------------------------------------------------------------------------------------------------
\subsection{Un schéma à cinq points}
%---------------------------------------------------------------------------------------------------------------------------

%///////////////////////////////////////////////////////////////////////////////////////////////////////////////////////////
\subsubsection{Poser le système}
%///////////////////////////////////////////////////////////////////////////////////////////////////////////////////////////

Soit \( \Omega=\mathopen] 0 , 1 \mathclose[\) et l'équation différentielle
	\begin{equation}        \label{EQooXJBWooRhCsLy}
		\begin{cases}
			-u''(x)+c(x)u(x)=f & \text{sur } \Omega \\
			u(0)=\alpha                             \\
			u(1)=\beta
		\end{cases}
	\end{equation}
	où \( c\) est une fonction positive et \( \alpha,\beta\in \eR\). Nous considérons \( h>0\) assez petit pour que le reste ait un sens. Si nous cherchons des solutions dans \( C^4(\Omega)\), le lemme~\ref{LEMooZECZooVKxOZZ} nous dit que
	\begin{equation}
		| u''(x)-D^-D^+u(x) |=\eta(h^2)
	\end{equation}
	où \( \eta\) est une fonction telle que \( \lim_{t\to 0} \eta(t)=0\). Nous pouvons récrire l'équation différentielle sous la forme
	\begin{equation}
		-D^-D^+u(x)+c(x)u(x)=f(x)+\eta(h^2).
	\end{equation}
	Si nous négligeons le terme \( \eta(h^2)\) qui est supposé être petit nous pouvons tenter de résoudre pour la fonction \( u_h\)
	\begin{equation}
		-D^-D^+u_h(x)+c(x)u_h(x)=f(x).
	\end{equation}
	Notons ici l'importance de la notion de problème bien posé parce qu'en remplaçant le paramètre (fonctionnel) \( f\) par \( f+\eta(h^2)\), nous modifions les solutions. Dans la mesure où le problème est bien posé, cette petite modification ne modifiera pas trop la solution et nous pouvons espérer que \( \| u-u_h \|\) soit petit pour une norme ou une autre.

	Utilisant l'expression \eqref{EQooBLIIooWHXbqD} pour \( D^-D^+\) nous avons l'équation suivante pour \( u_h\) :
	\begin{equation}        \label{EQooECLXooFxZEeA}
		\frac{1}{ h^2 }\Big( 2u_h(x)-u_h(x+h)-u_h(x-h) \Big)+c(x)u_h(x)=f(x).
	\end{equation}
	Avons-nous gagné quelque chose ? Pas encore. L'idée de la discrétisation est de ne considérer \( u_h\) qu'en certains points, et de prendre ces points à intervalles réguliers de taille \( h\). Soient donc \( N\) un nombre entier et \( h=1/(N+1)\). Nous posons
	\begin{equation}
		x_k=kh
	\end{equation}
	pour \( k=0,\ldots, N+1\). Avec cela nous avons
	\begin{subequations}
		\begin{align}
			\overline{ \Omega } & =\bigcup_{k=0}^{N}\mathopen[ x_k , x_{k+1} \mathclose] \\
			x_0                 & =0                                                      \\
			x_{N+1}             & =1.
		\end{align}
	\end{subequations}
	Nous posons surtout
	\begin{equation}
		\Omega_h=\{ x_i \}_{i=1,\ldots, N}
	\end{equation}
	et
	\begin{equation}
		\overline{ \Omega_h }=\{ x_i \}_{i=0,\ldots, N+1}.
	\end{equation}
	Enfin, nous ne considérons plus \( u_h\) que comme une fonction \( u_h\colon \overline{ \Omega_h }\to \eR\). C'est-à-dire que \( u_h\) est un vecteur à \( N+2\) composantes.

	L'équation \eqref{EQooECLXooFxZEeA} devient
	\begin{equation}        \label{EQooZMVMooTqlpkF}
		\frac{1}{ h^2 }\big( 2u_h(x_i)-u_h(x_{i+1})-u_h(x_{i-1})+c(x_i)u_h(x_i) \big)=f(x_i)
	\end{equation}
	pour \( i=1,\ldots, N\). Sur les bords, cette équation n'est pas possible parce que \( x_{i-1}\) ou \( x_{i+1}\) n'existerait pas. Au contraire, sur les bords nous avons les conditions aux bords
	\begin{equation}
		u_h(x_0)=\alpha
	\end{equation}
	et
	\begin{equation}
		u_h(x_{N+1})=\beta.
	\end{equation}

    Nous posons \( c_i=c(x_i)\) et \( u_i=u_h(x_i)\). Les nombres \( u_0\) et \( u_{N+1}\) sont donnés par les conditions aux bords, et    les inconnues du problème sont donc les nombres \( u_i\) (\( i=1,\ldots, N\)). Pour les déterminer, nous devons résoudre un système d'équations linéaire.

	L'écriture du système linéaire à résoudre consiste essentiellement à écrire \eqref{EQooZMVMooTqlpkF} en séparant les cas \( i=1\) et \( i=N\) parce que nous connaissons déjà les valeurs de \( u_0\) et \( u_{N+1}\). Le système que nous avons est :
	\begin{subequations}
		\begin{numcases}{}
			\left( \frac{ 2 }{ h^2 }+c_1 \right)u_1-\frac{1}{ h^2 }u_2=f_1+\frac{ \alpha }{ h^2 }  & $i=1$\\
			\left( \frac{ 2 }{ h^2 }+c_N \right)u_N-\frac{1}{ h^2 }u_{N-1}=f_N+\frac{ \beta }{ h^2 }  &\( i=N\) \\
			\left( \frac{ 2 }{ h^2 }+c_i \right)u_i-\frac{1}{ h^2 }u_{i+1}-\frac{1}{ h^2 }u_{i-1}=f_i & autres.
		\end{numcases}
	\end{subequations}
	Cela se met sous la forme matricielle
	\begin{equation}
		L_hU_h=F_h
	\end{equation}
	pour
	\begin{equation}        \label{EQooMNTJooYPYoAj}
		F_h=\big( f_1+\frac{ \alpha }{ h^2 },f_2,\ldots, f_{N-1},f_N+\frac{ \beta }{ h^2 } \big)
	\end{equation}
	et les éléments non nuls de \( L_h\) sont :
	\begin{subequations}
		\begin{align}
			(L_h)_{i,i-1} & =-\frac{1}{ h^2 }      & \text{ pour }i & =2,\ldots, N   \\
			(L_h)_{i,i+1} & =-\frac{1}{ h^2 }      & \text{ pour }i & =1,\ldots, N-1 \\
			(L_h)_{i,i}   & =\frac{ 2 }{ h^2 }+c_i & \text{ pour }i & =1,\ldots, N.
		\end{align}
	\end{subequations}
	Cette matrice est pleine de zéros, à part les trois diagonales centrales, et il existe des méthodes efficaces pour résoudre le système d'équations correspondant.
