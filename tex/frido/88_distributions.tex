% This is part of Mes notes de mathématique
% Copyright (c) 2011-2023
%   Laurent Claessens
% See the file fdl-1.3.txt for copying conditions.

Nous donnons ici une partie de la théorie sur les distributions. L'utilisation des distributions dans le cadre des équations différentielles est mise dans le chapitre sur les équations différentielles, section~\ref{SecTNgeNms}.

\begin{proposition}[\cite{BIBooWINRooQSnfWf}] \label{PropAAjSURG}
	Soient un ouvert \( \Omega\) de \( \eR\) et une fonction intégrable \( f\colon \big( \Omega,\Borelien(\Omega),\lambda \big)\to \eC\) telle que
	\begin{equation}
		\int_{\Omega}f\varphi=0
	\end{equation}
	pour toute fonction \( \varphi\in\swD(\Omega)\). Alors \( f=0\) presque partout sur \( \Omega\).
\end{proposition}

\begin{proof}
	Nous commençons par prouver que \( f\) est nulle sur tout compact de \( \Omega\). Soit un compact \( K\) de \( \Omega\). Le lemme d'Urysohn \ref{LEMooECTNooKagaRU} nous donne une fonction \( \theta\) à support compact qui vaut \( 1\) sur \( K\).

	Nous considérons une suite régularisante \( (\phi_k)\) de fonctions toujours strictement positives (par exemple celle du lemme \ref{LEMooTDWSooSBJXdv}). Vu que \( f\theta\) est à support compact, elle est dans \( L^p(\Omega)\) et le corolaire \ref{CORooQLELooUjzIoM} s'applique :
	\begin{equation}
		\phi_k*(\theta f)\to \theta f.
	\end{equation}
	Mais, \( x\) et \( k\) étant fixés, nous avons
	\begin{equation}        \label{EQooIUIMooMTAHCY}
		\big( \phi_k* (\theta f) \big)(x)=\int_{\eR}\phi_k(x-t)\theta(t)f(t)dt.
	\end{equation}
	La fonction
	\begin{equation}
		t\mapsto \phi_k(x-t)\theta(t)
	\end{equation}
	étant à support compact, l'hypothèse à propos de \( f\) fait que l'intégrale \eqref{EQooIUIMooMTAHCY} est nulle :
	\begin{equation}
		\phi_k*(\theta f)=0
	\end{equation}
	pour tout \( k\). En prenant la limite \( k\to \infty\),
	\begin{equation}
		\theta f=0.
	\end{equation}
	Vu que \( \theta(x)=1\) pour tout \( x\in K\), nous avons \( f(x)=0\) pour tout \( x\in K\).

	Nous avons démontré que \( f\) était nulle sur tout compact de \( \Omega\).

	Nous considérons maintenant une suite exhaustive \( (K_n)\) de compacts (lemme \ref{LemGDeZlOo}). La fonction \( f\) est nulle sur chaque \( K_n\), et comme \( \Omega=\bigcup_{n=0}^{\infty}K_n\), la fonction \( f\) est nulle sur \( \Omega\).
\end{proof}

%+++++++++++++++++++++++++++++++++++++++++++++++++++++++++++++++++++++++++++++++++++++++++++++++++++++++++++++++++++++++++++ 
\section{Dérivée faible}
%+++++++++++++++++++++++++++++++++++++++++++++++++++++++++++++++++++++++++++++++++++++++++++++++++++++++++++++++++++++++++++

%---------------------------------------------------------------------------------------------------------------------------
\subsection{Dérivée partielle au sens faible}
%---------------------------------------------------------------------------------------------------------------------------

\begin{lemmaDef}        \label{DEFooIRJQooMVNopl}
	Soit \( f\in L^p(I)\) où \( I\) est l'intervalle ouvert \( \mathopen] a , b \mathclose[\). Il existe au maximum\footnote{En réalité, c'est une classe au sens de l'égalité presque partout.} une fonction \( g\) telle que
	\begin{equation}
		\int_If\varphi'=-\int_Ig\varphi
	\end{equation}
	pour tout \( \varphi\in \swD(I)   \). Lorsqu'une telle fonction existe, nous la nommons \defe{dérivée faible}{dérivée faible} de \( f\).
\end{lemmaDef}

\begin{proof}
	Soient \( g,h\in L^2\) tels que
	\begin{equation}
		\int_Iu\varphi'=-\int_Ig\varphi=-\int_Ih\varphi
	\end{equation}
	pour tout \( \varphi\in C^{\infty}_c(I)\). Nous avons alors
	\begin{equation}
		\int_I(g-h)\varphi=0.
	\end{equation}
	Cela implique que \( g-h=0\) presque partout par la proposition~\ref{PropUKLZZZh}\footnote{Ou alors par le lemme~\ref{LemDQEKNNf} qui est moins général mais tout aussi bien pour ici.}.
\end{proof}

\begin{example}[Dérivée faible de \(  \mtu_{\eQ}\)]
	Vu que \( \eQ\) est de mesure nulle dans \( \eR\), nous avons
	\begin{equation}
		\int_{\eR}\mtu_{\eQ}\varphi'=0
	\end{equation}
	pour tout \( \varphi\in\swD(\eR)\). Pour \( g=0\) nous avons aussi \( \int_{\eR}g\varphi=0\). Donc \( g=0\) est la dérivée faible de \( \mtu_{\eQ}\).

	Cela n'est pas étonnant du fait qu'en théorie de l'intégration, les parties de mesure nulle ne comptent pas. De ce point de vue, \( \mtu_{\eQ}=0\). D'ailleurs cette égalité est vraie dans \( L^p\) (les classes et tout ça).
\end{example}

\begin{example}[La fonction de Heaveside n'a pas de dérivée faible]     \label{EXooRVGHooTWOCtF}
	Nous montrons que la fonction
	\begin{equation}
		H(x)=\begin{cases}
			0 & \text{si } x<0     \\
			1 & \text{si } x\geq 0
		\end{cases}
	\end{equation}
	n'a pas dérivée faible. Nous nommons \( g\) une hypothétique fonction vérifiant les conditions pour être la dérivée faible de \( H\).

	Soit une fonction \( \varphi\in\swD(\eR)\), dont le support est contenu dans \( \mathopen] 0 , \infty \mathclose[\). Sur le support de \( \varphi\), et donc aussi de \( \varphi'\), nous avons \( H(x)=1\) et donc
		\begin{equation}
			\int_{\eR}\varphi'=-\int_{\eR}g\varphi.
		\end{equation}
		Vu que \( \varphi\) est à support compact, \( \int_{\eR}\varphi'=0\). En effet, si le support de \( \varphi\) est contenu dans \( \mathopen[ -M , M \mathclose]\), alors en utilisant le théorème fondamental de l'analyse \ref{ThoRWXooTqHGbC}, nous trouvons \( \int_{\eR}\varphi=\int_{-M}^M\varphi'=\varphi(M)-\varphi(-M)=0-0=0\).

		Donc \( g\) doit satisfaire
		\begin{equation}
			\int_{\eR}g(x)\varphi(x)=0
		\end{equation}
		pour tout \( \varphi\in\swD(x>0)\). La proposition \ref{PropAAjSURG} nous dit que \( g=0\) presque partout sur \( \mathopen] 0 , \infty \mathclose[\).

	Le même raisonnement dit que \( g=0\) presque partout sur les négatifs. Que \( g\) soit maintenant nulle ou non en \( x=0\) ne change pas le fait que \( g=0\) presque partout sur \( \eR\).

	Par conséquent, \( \int_{\eR}g\varphi=0\) pour toute \( \varphi\in \swD(\eR)\). Hélas, nous avons d'autre part
	\begin{equation}
		\int_{\eR}H(x)\varphi'(x)dx=\int_0^{\infty}\varphi'(x)dx,
	\end{equation}
	qui n'est pas forcément nul. Notons que pour avoir un exemple de \( \varphi\) qui donne \( \int_{\eR}H\varphi\neq 0\), il faut chercher des fonctions dont le support contient des négatifs et des positifs.

	La fonction \( H\) n'a donc pas de dérivée faible. Notons cependant que cela ne présume en rien la possibilité d'accepter une dérivée au sens des distributions.
\end{example}

\begin{normaltext}
	Nous verrons dans la proposition \ref{PROPooVUDVooAlwZzB} que la dérivée de Heaveside au sens des distributions est le delta de Dirac.
\end{normaltext}

\begin{example}[Dérivée faible de la valeur absolue]
	L'exemple de base de fonction continue qui n'est pas dérivable est la valeur absolue \( f(x)=| x |\) prise en \( x=0\). Nous allons montrer ici que la fonction
	\begin{equation}
		H(x)=\begin{cases}
			-1 & \text{si } x<0 \\
			1  & \text{si } x>0
		\end{cases}
	\end{equation}
	est la dérivée faible de \( f\).

	Commençons par noter que \( H\) peut valoir la valeur qu'on veut en zéro; de toutes façons la dérivée faible n'est définie qu'à partie de mesure nulle près.

	Soit \( \varphi\in \swD(\eR)\) et \( M>0\) tel que le support de \( \varphi\) soit contenu dans \( \mathopen[ -M , M \mathclose]\). Nous avons d'une part
	\begin{subequations}
		\begin{align}
			\int_{\eR}| x |\varphi'(x)dx & =-\int_{-M}^0x\varphi'(x)dx+\int_0^Mx\varphi'(x)dx                     \\
			                             & =-[\varphi x]_{-M}^0+\int_{-M}^0\varphi+[x\varphi]_0^M-\int_0^M\varphi
		\end{align}
	\end{subequations}
	où nous avons utilisé l'intégration par partie de la proposition \ref{PROPooRLFIooQHnyJY} en posant
	\begin{subequations}
		\begin{align}
			u  & =x &  & v'=\varphi' \\
			u' & =1 &  & v=\varphi.
		\end{align}
	\end{subequations}
	Tout cela pour dire que
	\begin{equation}
		\int_{\eR}| x |\varphi'(x)dx=\int_{-M}^0\varphi-\int_0^M\varphi.
	\end{equation}
	D'autre part, l'égalité
	\begin{equation}
		\int_{\eR}H(x)\varphi(x)dx= -\int_{-M}^0\varphi+\int_0^M\varphi
	\end{equation}
	est immédiate.

	Nous en déduisons que \( H\) est bien la dérivée faible de \( x\mapsto | x |\).
\end{example}

\begin{remark}      \label{REMooBGJFooPBkFqm}
	La dérivée faible ne doit pas être confondue avec la dérivée au sens des distributions qui sera définie en \ref{PropKJLrfSX}. Nous avons donc trois notions distinctes de dérivation pour une fonction :
	\begin{itemize}
		\item la dérivée usuelle,
		\item la dérivée au sens des distributions,
		\item la dérivée faible.
	\end{itemize}
	La dérivée faible d'une fonction reste une fonction, tandis que la dérivée distributionnelle d'une fonction est une distribution.

	Je vous mets en garde contre l'idée que l'existence de l'une impliquerait trop facilement l'existence d'une autre\quext{Wikipédia cite l'exemple de la fonction de Cantor qui est dérivable presque partout au sens usuel, mais qui n'est pas faiblement dérivable. Écrivez-moi si vous connaissez des théorèmes qui lient les trois notions de dérivée.}.
\end{remark}

%--------------------------------------------------------------------------------------------------------------------------- 
\subsection{Dérivée faible partielle}
%---------------------------------------------------------------------------------------------------------------------------

La notion de dérivée partielle faible est la même que l'autre. Histoire de nous mettre dans le bain, nous écrivons la définition avec les notation du produit scalaire au lieu de l'intégrale.

\begin{definition}      \label{DEFooBRFCooPncSCE}
	Si \( i=1,\ldots, n\), la \defe{dérivée faible}{dérivée!faible} de \( v\) dans la direction \( e_i\) est l'application\footnote{En fait c'est une classe au sens de l'égalité presque partout.} notée \( \partial_iv\) définie par
	\begin{equation}        \label{EQooMRZUooFoqPqv}
		\langle \partial_iv, \phi\rangle =-\langle v, \partial_i\phi\rangle
	\end{equation}
	pour tout \( \phi\in  C^{\infty}_c(\Omega)\).
\end{definition}

\begin{lemma}
	Si \( v\in L^2\) admet une dérivée faible, alors cette dernière est unique.
\end{lemma}

\begin{proof}
	Supposons \( f,g\) telles que \( \langle g, \phi\rangle \) et \( \langle f, \phi\rangle \) soient tous deux égaux à \( -\langle v, \partial_i\phi\rangle \). En particulier pour tout \( \phi\in  \swD(\Omega)\) nous avons \( \langle (f-g), \phi\rangle =0\).

	Cela donne \( f-g=0\) par la proposition~\ref{PropUKLZZZh}.
\end{proof}

Soit \( \Omega\) un ouvert de \( \eR^d\). Le but de notre histoire est de définir une distribution comme étant un élément de l'espace dual (topologique, voir définition~\ref{DEFooKSDFooGIBtrG}) de l'espace \( \swD(\Omega)\) des fonctions \( C^{\infty}\) à support compact dans \( \Omega\). Pour ce faire nous devons voir un peu de topologie sur différents espaces de fonctions. Notons que l'espace \( \swD(\Omega)\) n'est pas réduit à la fonction nulle comme en témoigne l'exemple donné par l'équation \eqref{EqOBYNEMu}.

\begin{propositionDef}     \label{PROPooVZFHooKfSpfO}      % TODOooXAOSooLBWFDs Voir si la DEFooZTKAooWYUyDa est très différence de ceci.
	Pour chaque \( K\) compact dans \( \Omega\) et chaque entier \( m\), l'application
	\begin{equation}  \label{EQooZSQUooAJRIFe}
		\begin{aligned}
			p_{K,m}\colon \ C^{\infty}(\Omega) & \to \eR                                                        \\
			f                                  & \mapsto  \sum_{| \mu |\leq m}\| \partial^{\mu}f \|_{K,\infty}.
		\end{aligned}
	\end{equation}
	est une seminorme.
\end{propositionDef}

En particulier,
\begin{equation}
	p_{K,0}(f)=\sup_{x\in K}| f(x) |=\| f \|_{\infty,K}.
\end{equation}

\begin{lemma}[Formule de Leibnitz\cite{BIBooYDMJooGDtdbo}]      \label{LEMooOLQTooEHJuBc}
	Soient des fonctions \( f,g\) \( p\) fois dérivables sur \( \eR^d\). Alors pour tout multiindices \( \alpha\) de taille plus petite ou égale à \( p\), nous avons
	\begin{equation}
		\partial^{\alpha}(fg)=\sum_{\beta\leq \alpha}\binom{ \alpha }{ \beta }\partial^{\beta}f\partial^{\alpha-\beta}g.
	\end{equation}
	Attention : lisez \cite{BIBooYDMJooGDtdbo} pour savoir ce que signifie la notation \( \binom{ \alpha }{ \beta } \) dans le contexte des multiindices.
\end{lemma}
\index{Formule de Leibnitz}

%+++++++++++++++++++++++++++++++++++++++++++++++++++++++++++++++++++++++++++++++++++++++++++++++++++++++++++++++++++++++++++ 
\section{Topologie et convergence sur des espaces de fonctions}
%+++++++++++++++++++++++++++++++++++++++++++++++++++++++++++++++++++++++++++++++++++++++++++++++++++++++++++++++++++++++++++

%-------------------------------------------------------
\subsection{Limite inductive}
%----------------------------------------------------

\begin{definition}[limite inductive\cite{TQSWRiz}]		\label{DEFooFCLUooSGJIKJ}
	Soit un espace vectoriel \( E\) ainsi que des sous-espaces vectoriels \( (E_i)_{i[in I]} \) tels que \( E=\bigcup_{i\in I}E_i\). Nous supposons avoir les seminormes \( \{ p_j^{(i)} \}_{j\in J_i}\) sur \( E_i\). La topologie \defe{limite inductive}{limite inductive} sur \( E\) est celle définie par l'ensemble des seminomes \( p\) sur \( E\) telles que \( p|_{E_i}\) soit continue pour tout \( i\in I\).
\end{definition}

%-------------------------------------------------------
\subsection{Les espaces classiques}
%----------------------------------------------------


\begin{normaltext}
	Ici, \( \Omega\) est un ouvert de \( \eR^d\), et \( K\) est un compact de \( \eR^d\). Toutes les fonctions sont sur \( \eR^d\) à valeurs dans \( \eC\). Nous rappelons que les seminormes \( p_{K,m}\) sont définies en \ref{PROPooVZFHooKfSpfO} et que la topologie engendrée par une seminorme est définie en \ref{DEFooZTKAooWYUyDa}.

	En tant qu'ensemble, \( \swD(\Omega)\) est l'ensemble des fonctions de classe \(  C^{\infty}\) sur \( \eR^d\) et dont le support est un compact dans \( \Omega\). L'ensemble \( \swD(K)\) est l'ensemble des fonctions de classe \(  C^{\infty}\) sur \( \eR^d\) et dont le support est un compact dans \( K\).
\end{normaltext}

\begin{definition}[Topologie sur \( C^{\infty}(\Omega)\)]  \label{DefFGGCooTYgmYf}
	Sur \(  C^{\infty}(\Omega)\), la topologie des seminormes \( p_{K,m}\) de la définition \ref{PROPooVZFHooKfSpfO} où \( K\) parcourt les compacts de \( \Omega\) et \( m\) parcourt \( \eN\).
\end{definition}
\index{topologie!sur \(  C^{\infty}(\Omega)\) }

\begin{normaltext}
	Pour l'ensemble des fonctions \( C^{\infty}\) à support compact dans \( A\), la notation devrait être \( C^{\infty}_c(A)\), et non \( \swD(A)\). Hélas je ne suis pas consistant dans les notations.
\end{normaltext}


\begin{definition}[Toplogie sur \( C^{\infty}_c(K)\)]		\label{DEFooJJJYooIrekSp}
	Soient un ouvert \( \Omega\) dans \( \eR^n\), ainsi qu'un compact \( K\subset \Omega\). Nous notons
	\begin{equation}
		C^{\infty}(K)=\{ f\in C^{\infty}(\Omega)\tq \supp(f)\subset K \}.
	\end{equation}
	Attention : il n'est dit que les fonctions dans \( C^{\infty}(K)\) ont un support égal à \( K\). La topologie dessus est celle des seminomes \( (p_{K,m})_{m\in \eN}\)
	\begin{equation}		\label{EQooXGINooYJqmtD}
		p_{K,m}(f)=\max_{| \alpha |\leq m}\| \partial^{\alpha}f \|_K
	\end{equation}
\end{definition}
\index{topologie!sur \(  \swD(K)\) }

\begin{definition}[Bonne seminorme, topologie sur \( C^{\infty}_c(\Omega)\)]		\label{DEFooVSCRooLyYBzT}
	Soit un ouvert \( \Omega\) dans \( \eR^n\). Une seminorme\footnote{Seminorme, définition \ref{DefPNXlwmi}.} \( p\) sur \( C^{\infty}_c(\Omega)\) est \defe{bonne}{bonne seminorme} si pour tout compact \( K\subset \Omega\), la restriction \( p|_{C^{\infty}_c(K)}\) est continue.

	Sur \( C^{\infty}_c(\Omega)\), nous considérons la topologie des bonnes seminormes.
\end{definition}
\index{topologie!sur \(  \swD(\Omega)\) }

Cela n'est pas très explicite, mais heureusement nous n'aurons souvent pas besoin de plus que de la notion de convergence dans \( \swD'(\Omega)\). Rappelons que la topologie d'un espace donne la notion de convergence par la définition~\ref{DefXSnbhZX}.


\begin{proposition}[\cite{BIBooFBCEooQXpHce, MonCerveau}]		\label{PROPooRAZSooDttIbK}
	Nous avons
	\begin{equation}
		\phi_k\stackrel{ C^{\infty}_c(\Omega)}{\longrightarrow} \phi
	\end{equation}
	si et seulement si il existe un compact \( K\subset \Omega\) tel que
	\begin{enumerate}
		\item
		      \( \supp(\phi_k)\subset K\) pour tout \( k\).
		\item
		      \( \supp(\phi)\subset K\).
		\item
		      \( \phi_k\stackrel{ C^{\infty}_c(K)}{\longrightarrow} \phi\).
	\end{enumerate}
	Autrement dit nous avons convergence dans \( C^{\infty}_c(\Omega)\) si et seulement si il existe un compact dans lequel tout se passe y compris la convergence.
\end{proposition}

\begin{proof}
	En deux parties.
	\begin{proofpart}
		Sens direct, \( \Rightarrow\).
	\end{proofpart}

	Nous supposons que \( \phi_k\stackrel{ C^{\infty}_c(\Omega)}{\longrightarrow} \phi\). Nous supposons, par l'absurde qu'il n'existe pas de compacts \( K\) dans \( \Omega\) tel que \( \supp(\phi_k)\subset K\) et \( \supp(\phi)\subset K\).

	\begin{subproof}
		\spitem[Une sous-suite exhaustive]
		%-----------------------------------------------------------

		Soit \( (K_i)\) une suite exhaustive de compacts pour \( \Omega\) (lemme \ref{LemGDeZlOo}). La partie \( \supp(\phi)\) est un compact dans \( \Omega\), et le lemme \ref{LemGDeZlOo}\ref{ITEMooBPYPooEMhSmY} dit qu'il existe un \( n\) tel que \( \supp(\phi)\subset K_n\). Nous considérons une nouvelle suite exhaustive \( K'_i=K_{i+n}\). Autrement dit, \( \supp(\phi)\subset K'_i\) pour tout \( i\).

		Nous oublions la suite \( (K_i)\) et nous ne travaillons plus qu'avec \( (K'_i)\) que nous allons écrire \( (K_i)\) en oubliant le prime.

		\spitem[Les \( k_i\)]
		%-----------------------------------------------------------

		Par hypothèse d'absurdité, nous avons alors qu'il n'existe aucun \( i\) contenant tous les \( \supp(\phi_k)\) : pour tout \( k\), il existe \( k_i\in \eN\) tel que \( \supp(\phi_{k_i})\) n'est pas inclus dans \( K_i\).

		\spitem[\( \{ k_i \}\) n'est pas borné]
		%-----------------------------------------------------------
		Nous prouvons que l'ensemble des \( K_i\) n'est pas borné. Par l'absurde, nous supposons que \( k_i<n\) pour tout \( i\). Nous posons alors
		\begin{equation}
			A=\bigcup_{k<n}\supp(\phi_{k}).
		\end{equation}
		La partie \( A\) est un compact en tant que union finie de compacts\footnote{Lemme \ref{LEMooFJZDooSxYWVW}.}. Elle est donc contenue dans \( K_s\) pour un certain \( s\in \eN\). De plus \( \supp(\phi_{k_i})\subset A\) pour tout \( i\). Nous avons donc, pour tout \( i\) :
		\begin{equation}
			\supp(\phi_{k_i})\subset A\subset K_s.
		\end{equation}
		En posant \( i=s\), nous trouvons que \( \supp(\phi_{k_s})\subset K_s\), ce qui est faux. Nous en déduisons que \( (k_i)_{i\in \eN}\) n'est pas bornée.

		\spitem[Sous-suite des \( \phi_k\)]
		%-----------------------------------------------------------

		Vu que \( (k_i)\) est non bornée, nous pouvons en extraire une sous-suite \( (k'_i)_{i\in \eN}\) telle que \( k'_i\to \infty \). Nous considérons alors la sous-suite des \( \phi_k\) donnée par
		\begin{equation}
			\tilde \phi_i=\phi_{k'_i}.
		\end{equation}
		Et nous oublions immédiatement la suite initiale ainsi que le tilde. Donc pour tout \( i\), nous avons
		\begin{equation}
			\supp(\phi_i) \not\subset K_i.
		\end{equation}
		Pour tout \( n\), il existe \( x_n\not\in K_n\) tel que \( \phi_n(x_n)\neq 0\).

		\spitem[Une nouvelle seminorme]
		%-----------------------------------------------------------

		Si \( f\in C^{\infty}_c(\Omega)\), nous avons \( \supp(f)\subset K_r\) pour un certain \( r\). Si \( n>r\), nous avons \( K_r\subset K_n\) et \( x_n\not\in K_n\), donc \( x_n\not\in K_r\). Bref, nous avons alors \( f(x_n)=0\) pour tout \( n>r\).

		Tout cela pour dire que \( f(x_n)\neq 0\) pour seulement un nombre fini de \( n\). Il nous est donc loisible de poser
		\begin{equation}
			\begin{aligned}
				p\colon C^{\infty}_c(\Omega) & \to \eR                                                          \\
				f                            & \mapsto \max_{n\in \eN}\big| \frac{1}{ \phi_n(x_n)}f(x_n) \big|.
			\end{aligned}
		\end{equation}
		Il s'agit bien d'un max parce qu'il n'y a qu'un nombre fini de termes non nuls. Vérifions que c'est une seminorme en vérifiant les conditions de la définition \ref{DefPNXlwmi}.
		\begin{enumerate}
			\item
			      En tant que maximum de choses dans des valeurs absolues, nous avons \( p(f)\geq 0\).
			\item
			      Si \( \lambda\in \eR\),
			      \begin{equation}
				      p(\lambda f)=\max_n\big|   \frac{1}{ \phi_n(x_n)}(\lambda f)(x_n)  \big|=| \lambda |\max_n\big| \ldots \big|=| \lambda |p(f).
			      \end{equation}
			\item
			      Nous avons
			      \begin{subequations}
				      \begin{align}
					      p(f+g) & =\max_n\big| \frac{1}{ \phi_n(x_n)}\big( f(x_n)+g(x_n) \big) \big|                                                  \\
					             & \leq \max_n  \left[    \big| \frac{1}{ \phi_n(x_n)}f(x_n) \big| +\big| \frac{1}{ \phi_n(x_n)}g(x_n) \big|   \right] \\
					             & \leq \max_n| \ldots |+\max_n| \ldots |                                                                              \\
					             & =p(f)+g(f).
				      \end{align}
			      \end{subequations}
		\end{enumerate}
		Donc oui, \( p\) est une seminorme.

		\spitem[\( p\) est une bonne seminorme]
		%-----------------------------------------------------------
		Soit un compact \( K\). Nous devons prouver que \( p|_{C^{\infty}_c(K)}\) est continue. Soit \( N\in \eN\) tel que \( x_n\not\in K\) pour tout \( n\geq N\). Si \( f\in C^{\infty}_c(K)\), nous avons
		\begin{subequations}	\label{SUBEQSooTBSDooExlzPQ}
			\begin{align}
				p(f) & =\max_n\big| \frac{1}{ \phi_n(x_n)}f(x_n) \big|                \\
				     & =\max_{n\leq N}\big| \frac{1}{ \phi_n(x_n)}f(x_n) \big|        \\
				     & \leq \max_{n\leq N}\big| \frac{1}{ \phi_n(x_n)}\| f \|_K \big| \\
				     & =\| f \|_K\max_{n\leq N}\big| \frac{1}{ \phi_n(x_n)} \big|     \\
				     & = c_K\| f \|_K.
			\end{align}
		\end{subequations}
		où \( c_K\) est une constante qui dépend de \( K\).

		Supposons que \( f_k\stackrel{ C^{\infty}_c(K)}{\longrightarrow} f\). Pour prouver la continuité de \( p\), nous devons prouver que \( p(f)-p(f_k)\to 0\). Nous avons
		\begin{subequations}
			\begin{align}
				| p(f)-p(f_k) | & \leq p(f-f_k)       & \text{lem. \ref{LEMooHTOAooGmRGZL}}      \\
				                & \leq c\| f_k-f \|_K & \text{eq. \eqref{SUBEQSooTBSDooExlzPQ}}.
			\end{align}
		\end{subequations}
		La définition \ref{DEFooJJJYooIrekSp} donne la topologie sur \( C^{\infty}_c(K)\) par les seminormes
		\begin{equation}
			p_{K,m}(f-f_k)=\sup_{| \alpha |\leq m}\| \partial^{\alpha}(f-f_k) \|_K.
		\end{equation}
		La proposition \ref{PropQPzGKVk} nous dit que \( p_{K,m}(f-f_k)\to 0\) pour tout \( m\). En particulier pour \( m=0\) nous avons \( p_{K,0}(f-f_k)=\| f-f_k \|_K\to 0\).

		Au final nous avons
		\begin{equation}
			| p(f)-p(f_k) |\leq c_K\| f-f_k \|_K\to 0,
		\end{equation}
		et donc la continuité de \( p\) en \( f\).

		\spitem[La contradiction]
		%-----------------------------------------------------------

		Nous avons prouvé que \( p\) était une bonne seminorme. Nous devrions donc avoir \( p(\phi_k-\phi)\to 0\). Vérifions ça \ldots. Nous avons
		\begin{subequations}
			\begin{align}
				p(\phi_k-\phi) & =\max_n\big| \frac{1}{ \phi_n(x_n)}\big( \phi_k(x_n)-\phi(x_n) \big) \big|                                                 \\
				               & =\max_n\big| \frac{1}{ \phi_n(x_n)}\phi_k(x_n) \big|                       & \text{cf. justif.}	\label{SUBEQooBPBBooWiXATm} \\
				               & \geq \big| \frac{1}{ \phi_k(x_k)}\phi_k(x_k) \big|                                                                         \\
				               & = 1.
			\end{align}
		\end{subequations}
		Justification. Pour \eqref{SUBEQooBPBBooWiXATm}. Nous avons fait plein de choix pour avoir \( \phi(x_n)=0\) pour tout \( n\).

		Cela contredit la possibilité d'avoir \( p(\phi_k-\phi)\to 0\).

		\spitem[Précision]
		%-----------------------------------------------------------

		Nous avons trouvé une sous-suite non convergente des \( \phi_k\) originaux. Donc les \( \phi_k\) de départ ne convergent pas.

		\spitem[Première conclusion]
		%-----------------------------------------------------------

		Nous avons prouvé par l'absurde qu'il existe un compact \( K\subset \Omega\) tel que \( \supp(\phi)\subset K\) et \( \supp(\phi_k)\subset K\) pour tout \( k\).

		\spitem[\( \phi_k\stackrel{ C^{\infty}_c(K)}{\longrightarrow} \phi \)]
		%-----------------------------------------------------------
		Nous devons encore prouver que \( \phi_k\stackrel{ C^{\infty}_c(K)}{\longrightarrow} \phi\). Par la proposition \ref{PropQPzGKVk}, nous devons prouver que \( q(\phi_k-\phi)\to 0\) pour toutes les seminormes \( p_{k,m}\) de la définition \ref{DEFooJJJYooIrekSp}.

		Par hypothèse nous avons \( \phi_k\stackrel{ C^{\infty}_c(\Omega)}{\longrightarrow} \), et donc convergence \( p(\phi_k-\phi)\to 0\) pour toute bonne seminorme sur \( \Omega\). Oh, mais justement pour tout \( m\), l'application \( p_{K,m}\) est continue, et donc est une bonne seminorme. Nous avons donc bien
		\begin{equation}
			p_{K,m}(\phi_k-\phi)\to 0.
		\end{equation}
	\end{subproof}

	\begin{proofpart}
		Sens réciproque, \( \Leftarrow\).
	\end{proofpart}
	Nous supposons qu'il existe un compact \( K\subset \Omega\) tel que
	\begin{enumerate}
		\item \( \supp(\phi_k)\subset K\) pour tout \( k\),
		\item
		      \( \supp(\phi)\subset K\),
		\item
		      \( \phi_k\stackrel{ C^{\infty}_c(K)}{\longrightarrow} \phi\),
	\end{enumerate}
	et nous devons prouver que \( \phi_k\stackrel{ C^{\infty}_c(\Omega)}{\longrightarrow} \phi\), c'est à dire que \( q(\phi_k-\phi)\to 0\) pour toute bonne seminorme \( q\) sur \( \Omega\).

	Soit une bonne seminorme \( q\) sur \( \Omega\). En particulier, sa restriction à \( C^{\infty}_c(K)\) est continue. Par hypothèse, \( \phi_k\stackrel{ C^{\infty}_c(K)}{\longrightarrow} \phi\); par continuité de \( q\), nous avons \( q(\phi_k)\to q(\phi)\).
\end{proof}

\begin{theorem}[\cite{BIBooFBCEooQXpHce}]
	Soit \( T\), une application linéaire sur \( C^{\infty}_c(\Omega)\), c'est à dire \( T\in \aL\big( C^{\infty}_c(\Omega), \eC \big)\). Elle est continue si et seulement si pour tout compact \( K\subset \Omega\), il existe \( c>0\) et \( n\in \eN\) tels que
	\begin{equation}
		| T(\phi) |\leq c\max_{| \alpha |\leq n}\| \partial^{\alpha}\phi \|_{K}.
	\end{equation}
	pour tout \( \phi\in C^{\infty}_c(K)\).
\end{theorem}

\begin{proof}
	En deux parties
	\begin{proofpart}
		Sens direct, \( \Rightarrow\).
	\end{proofpart}
	Nous supposons que \( T\) est continue sur \( C^{\infty}_c(\Omega)\). Nous utilisons le corolaire \ref{CORooGMVHooNgYOaY} qui dit qu'il existe une seminorme \( p\) (parmi celles qui définissent la topologie de \( C^{\infty}_c(\Omega)\), c'est à dire une bonne seminorme) et \( M>0\) tels que
	\begin{equation}
		| T(\phi) |< Mp(\phi)
	\end{equation}
	pour tout \( \phi\in C^{\infty}_c(\Omega)\). Soit un compact \( K\). Étant donné que \( p\) est bonne, elle est continue sur \( C^{\infty}_c(K)\); nous pouvons donc appliquer le théorème \ref{THOooTKWYooYYiBNa} à la seminorme continue \( q=p|_{C^{\infty}_c(K)}\). Il existe une seminorme de \( C^{\infty}_c(K)\) (c'est à dire un \( p_{K,m}\)) et \( c>0\) tels que \( q(\phi)\leq c p_{K,m}(\phi)\).

	Pour tout \( \phi\in C^{\infty}_c(K)\) nous avons donc
	\begin{subequations}
		\begin{align}
			| T(\phi) |< Mp(\phi)                                           \\
			 & =Mq(\phi)                                                    \\
			 & \leq Mc p_{K,m}(\phi)                                        \\
			 & \leq Mc\max_{| \alpha |\leq m}\| \partial^{\alpha}\phi \|_K.
		\end{align}
	\end{subequations}
	La constante recherchée est donc \( Mc\).

	\begin{proofpart}
		Sens réciproque, \( \Leftarrow\).
	\end{proofpart}

	Soit une application linéaire \( T\) sur \( C^{\infty}_c(\Omega)\). Nous supposons que pour tout compact \( K\), il existe \( c_K>0\) et \( n_K\) tels que \( | T(\phi) |\leq c_K  \max_{| \alpha |<n_K}\| \partial^{\alpha}\phi \|_K   \).

	Soit une suite exhaustive de compacts \( (K_i)\) dans \( \Omega\). Nous avons des \( c_i\) et \( n_i\) tels que
	\begin{equation}		\label{EQooKRRTooKWTAap}
		| T(\phi) |\leq c_i\max_{| \alpha |\leq n_i}\| \partial^{\alpha}\phi \|_{K_i}
	\end{equation}
	pour tout \( i\).

	Pour chaque \( i\) nous considérons \( \phi_i\in C^{\infty}_c(K_i)\) tel que \( \phi_i=1\) sur \( K_{i-1}\). Cela existe par le lemme d'Urysohn\footnote{Proposition \ref{PROPooBOZIooAhKbPs}.} parce que \( K_{i-1}\) est inclus dans l'ouvert \( \Int(K_i)\).


	\begin{subproof}
		\spitem[Les fonctions \( \phi_i\) et \( \psi_i\)]
		%-----------------------------------------------------------
		Le lemme d'Urysohn\footnote{Proposition \ref{PROPooBOZIooAhKbPs}.}, appliqué à \( K_{i-1}\) inclus dans l'ouvert \( \Int(K_i)\) nous permet de considérer des fonctions \( \phi_i\) vérifiant :
		\begin{enumerate}
			\item
			      \( \supp(\phi_i)\subset\Int(K_i)\)
			\item
			      \( 0\leq \phi_i\leq 1\)
			\item
			      \( \phi_i=1\) sur \( K_{i-1}\).
		\end{enumerate}
		Nous posons aussi \( \phi_0=0\) et \( \psi_i=\phi_i-\phi_{i-1}\).

		\spitem[\( \sum_n\psi_n(x)=1\)]
		%-----------------------------------------------------------
		Soit \( x\in \Omega\). Soit \( i_0=\min\{ i\in \eN\tq x\in K_i \}\). Nous n'avons pas de garanties sur la valeur de \( \phi_{i_0}(x)\). Par contre, si \( i<i_0\), alors \( \supp(\phi_i)\subset K_i\) et par minimalité de \( i_0\), nous avons \( \phi_i(x)=0\).

		Si par contre \( i>i_0\), alors \( x\in K_{i-1}\), mais nous savons que \( \phi_i=1)\) sur \( K_{i-1}\). Donc pour \( i>i_0\) nous avons \( \phi_i(x)=1\).

		Nous avons donc :
		\begin{enumerate}
			\item
			      \( \psi_{i_0-k}(x)=\phi_{i_0-k}(x)-\phi_{i_0-k-1}(x)=0-0=0\).
			\item
			      \( \psi_{i_0}(x)=\phi_{i_0}(x)-\phi_{i_0-1}(x)=\phi_{i_0}(x)-0=\phi_{i_0}(x)\)
			\item
			      \( \psi_{i_0+1}(x)=\phi_{i_0+1}(x)-\phi_{i_0}(x)=1-\phi_{i_0}(x)\)
			\item
			      \( \psi_{i_0+k}(x)=1-1=0\) si \( k>1\).
		\end{enumerate}
		Tout ça pour dire que
		\begin{equation}		\label{EQooWUUAooLCgDMP}
			\sum_n\psi_n(x)=\psi_{i_0}(x)+\psi_{i_0+1}(x)=1.
		\end{equation}
		Il y a seulement deux termes non nuls dans la somme.

		\spitem[\( \supp(\psi_i)\subset K_i\setminus K_{i-2}\)]		\label{SPooCFEYooJySTba}
		%-----------------------------------------------------------
		Dans un premier temps nous prouvons que \( \supp(\psi_i)\subset K_i\). Si \( x\) est hors de \( K_i\), alors il est également hors de \( K_{i-1}\). Du coup\( \phi_i(x)=0\) et \( \phi_{i-1}(x)=0\). Nous avons alors \( \psi_i(x)=0\).

		Si par contre \( x\in K_{i-2}\), alors \( x\in K_{i-2}\subset K_{i-1}\). Vu que \( \phi_i=1\) sur \( K_{i-1}\) et \( \phi_{i-1}=1\) sur \( K_{i-2}\), nous avons \( \psi_i(x)=0\).

		\spitem[Quelques majorations]
		%-----------------------------------------------------------
		Un peu de calcul. Soit \( \phi\in C_c^{\infty}(\Omega)\). Nous avons
		\begin{subequations}
			\begin{align}
				| T(\phi) | & = | \sum_nT(\phi\psi_n) |                                                                                                                                                                          & \text{\( T\) est lin. et \eqref{EQooWUUAooLCgDMP}} \\
				            & \leq \sum_n| T(\phi\psi_n) |                                                                                                                                                                                                                            \\
				            & \leq \sum_n c_n\max_{| \alpha |\leq k_n}\| \partial^{\alpha}(\phi\psi_n) \|                                                                                                                        & \text{hyp. \eqref{EQooKRRTooKWTAap}}               \\
				            & \leq \sum_nc_n\max_{| \alpha |\leq k_n}\sum_{\beta\leq \alpha}\binom{ \alpha }{ \beta }\| (\partial^{\beta}\phi)(\partial^{\alpha-\beta}(\psi_n)) \|_{K_n}                                         & \text{lem. \ref{LEMooOLQTooEHJuBc}}                \\
				            & = \sum_nc_n\max_{| \alpha |\leq k_n}\sum_{\beta\leq \alpha}\binom{ \alpha }{ \beta }\| (\partial^{\beta}\phi)(\partial^{\alpha-\beta}(\psi_n)) \|_{K_n\setminus K_{n-2}}                           & \text{point \ref{SPooCFEYooJySTba}}                \\
				            & \leq \sum_n c_n\max_{| \alpha |\leq k_n}\sum_{\beta\leq \alpha}\binom{ \alpha }{ \beta }\| \partial^{\beta}\phi \|_{K_n\setminus K_{n-2}}\| \partial^{\alpha-\beta}\psi \|_{K_n\setminus K_{n-2}}.
			\end{align}
		\end{subequations}
		\spitem[Une bonne seminorme]
		%-----------------------------------------------------------
		Nous considérons l'application
		\begin{equation}
			\begin{aligned}
				p\colon C_c^{\infty}(\Omega) & \to \eR                                                                                                                                                                                \\
				\phi                         & \mapsto \sum_nc_n\max_{| \alpha |\leq k_n}\sum_{\beta\leq \alpha}\binom{ \alpha }{ \beta } \| \partial^{\alpha-\beta}\psi_n \|_{K_n}\| \partial^{\beta}\phi \|_{K_n\setminus K_{n-2}}.
			\end{aligned}
		\end{equation}
		Pour vérifier que ce \( p\) est une seminorme, il suffit de vérifier les conditions de la définition \ref{DefPNXlwmi}, et c'est facile. Pour prouver que c'est une bonne\footnote{Bonne seminorme, définition \ref{DEFooVSCRooLyYBzT}} seminorme, c'est un peu plus chaud, et il nous faudra utiliser le théorème \ref{THOooTKWYooYYiBNa}\ref{ITEMooBBNCooGwHrUI}.

		Soit un compact \( K\) dans \( \Omega\), et considérons la restriction \(p \colon C_c^{\infty}(K)\to \eR  \). Il existe un \( n_0\in \eN\) tel que \( K\subset K_{n_0}\). Pour \( \phi\in C_c^{\infty}(K)\), nous avons alors \( \supp(\phi)\subset K\subset K_{n_0}\), et donc
		\begin{equation}
			\| \partial^{\beta}\phi \|_{K_{n_0+2}\setminus K_{n_0}}=0.
		\end{equation}
		Nous avons donc
		\begin{subequations}
			\begin{align}
				p(\phi) & = \sum_{n\leq n_0+2} c_n\max_{| \alpha |\leq k_n}\sum_{\beta\leq \alpha}\binom{ \alpha }{ \beta }\| \partial^{\alpha-\beta}\psi_n \|_{K_n}\| \partial^{\beta}\phi \|_{K_n\setminus K_{n-2}}                                            \\
				        & \leq \max_{| \beta |\leq k_{n_0+2}}  \| \partial^{\beta}\phi \|_K\sum_{n\leq n_0+2}c_n\max_{| \alpha |\leq k_n}\sum_{\beta\leq \alpha}\binom{ \alpha }{\beta}\| \partial^{\alpha-\beta}\psi_n \|                                       \\
				        & = c'\max_{| \beta |\leq k_{n_0+2}}\| \partial^{\beta}\phi \|_K                                                                                                                                                                         \\
				        & = c' p_{K,k_{n_0+2}}(\phi)                                                                                                                                                                       & \text{eq. \eqref{EQooXGINooYJqmtD}}
			\end{align}
		\end{subequations}
		où
		\begin{equation}
			c'= \sum_{n\leq n_0+2}c_n\max_{| \alpha |\leq k_n}\sum_{\beta\leq \alpha}\binom{ \alpha }{\beta}\| \partial^{\alpha-\beta}\psi_n \|.
		\end{equation}
		Par le théorème \ref{THOooTKWYooYYiBNa}\ref{ITEMooBBNCooGwHrUI}, nous déduisons que \( p\) est continue.

		\spitem[Conclusion]
		%-----------------------------------------------------------
		Nous avons montré que \( | T(\phi) |\leq p(\phi)\) pour une bonne seminorme sur \( \Omega\). Le corolaire \ref{CORooGMVHooNgYOaY} conclu que \( T\) est continue.

	\end{subproof}
\end{proof}

%-------------------------------------------------------
\subsection{Espaces de fonctions et topologie inductive}
%----------------------------------------------------

\begin{probleme}		\label{PROBooQXUDooJwnzpn}
	Je ne suis pas tout à fait sûr que la proposition suivante \ref{PROPooUFJJooCoIdoI} soit vraie. Il y a peut-être une preuve dans \cite{TQSWRiz}.
\end{probleme}

\begin{proposition}		\label{PROPooUFJJooCoIdoI}
	La topologie sur \( \swD(\Omega)\) est la limite inductive\footnote{Définition \ref{DEFooFCLUooSGJIKJ}.} des \( \swD(K_i)\) où \( (K_i)\) est une suite exhaustive de compacts dans \( \Omega\).
\end{proposition}

%-------------------------------------------------------
\subsection{Convergence dans les fonctions test}
%----------------------------------------------------

\begin{lemma}[Convergence dans \( \swD(K)\)]    \label{LemXXwDjui}
	Si \( \alpha\) est un multiindice et si \( \varphi_n\stackrel{\swD(K)}{\longrightarrow}\varphi\), alors nous avons
	\begin{equation}
		\partial^{\alpha}\varphi_n\stackrel{unif}{\longrightarrow}\partial^{\alpha}\varphi.
	\end{equation}
\end{lemma}

\begin{proof}
	Quitte à considérer la suite \( \varphi_n-\varphi\) nous pouvons supposer \( \varphi_n\stackrel{\swD(K)}{\longrightarrow}0\). Nous avons
	\begin{equation}
		\| \partial^{\alpha}\varphi_n \|\leq \sum_{\mu\leq\alpha}\| \partial^{\mu}\varphi_n \|_{K,\infty}.
	\end{equation}
	Vu que le membre de droite tend vers zéro, nous avons
	\begin{equation}
		\lim_{n\to \infty} \| \partial^{\alpha}\varphi_n \|_{K,\infty}\to 0,
	\end{equation}
	ce qui revient à dire que \( \partial^{\alpha}\varphi_n\) converge uniformément sur \( K\) vers \( \partial^{\alpha}\varphi\).
\end{proof}

\begin{lemma}   \label{LemWEGpemo}
	Si une fonction \( f\colon \swD(\Omega)\to \eR\) est continue sur chacun des \( \swD(K)\) pour tout \( K\) compact dans \( \Omega\) alors est continue sur \( \swD(\Omega)\).
\end{lemma}

\begin{proof}
	Soit \( I\) ouvert dans \( \eR\); nous devons trouver un ouvert \( \mO\) dans \(  C^{\infty}(\Omega)\) tel que \( f^{-1}(I)=\swD(\Omega)\cap\mO\). Vu que \( f\) est continue sur chacun des \( \swD(K)\) avec \( K\) compact dans \( \Omega\), pour tout tel compact nous avons un ouvert \( \mO_K\) dans \( \swD(K)\) tel que \( f^{-1}(I)\cap \swD(K)=\mO_K\). En tant qu'union d'ouverts\footnote{Voir définition~\ref{DefTopologieGene}.}, l'ensemble
	\begin{equation}
		\mO=\bigcup_{ K\text{ compact de } \Omega}\mO_K
	\end{equation}
	est ouvert dans \(  C^{\infty}(\Omega)\). Si \( \phi\in f^{-1}(I)\), nous avons \( \phi\in\swD(K)\) pour un certain \( K\) compact de \( \Omega\), donc \( f^{-1}(I)\subset\mO\). À fortiori nous avons \( f^{-1}(I)\subset\mO\cap\swD(\Omega)\).

	Dans l'autre sens, si \( \phi\in\mO\), alors \( \phi\) est dans un des \( \mO_K\) et donc dans \( f^{-1}(I)\). Nous avons donc bien \( f^{-1}(I)=\swD(\Omega)\cap \mO\).
\end{proof}

\begin{theorem}[Convergence dans \( \swD(\Omega)\)\cite{TQSWRiz}]       \label{ThoXYADBZr}
	Soit \( (\varphi_n)_{n\in \eN}\) une suite dans \( \swD(\Omega)\) et \( \varphi\in\swD(\Omega)\). Nous avons \( \varphi_n\stackrel{\swD(\Omega)}{\longrightarrow}\varphi\) si et seulement si il existe \( K\) compact dans \( \Omega\) tel que \( \varphi_n\in\swD(K)\) pour tout \( n\) et \( \varphi_n\stackrel{\swD(K)}{\longrightarrow}\varphi\).
\end{theorem}

\begin{proof}
	Supposons que \( \varphi_n\stackrel{\swD(\Omega)}{\longrightarrow}\varphi\) et qu'il n'existe pas de compacts contenant tous les supports des \( \varphi_n\). Alors pour tout compact de \( \Omega\) il existe un \( n\) tel que le support de \( \varphi_n\) ne soit pas dans \( K\). Nous considérons une suite de compacts \( (K_i)\) tels que \( \Int(K_n)\subset K_{n+1}\) et \( \Omega=\bigcup_nK_n\). Une telle suite existe par le lemme~\ref{LemGDeZlOo}. Ensuite nous construisons des sous-suites de la façon suivante. D'abord \( L_1=K_1\) et \( n_1\in \eN\) est choisi de telle sorte que \( \varphi_{n_1}\) ait un support non contenu dans \( L_1\). Ensuite \( L_i\) est un compact de la suite \( (K_n)\) choisi plus loin que \( L_{i-1}\) et tel que \( \varphi_{n_{i-1}}\in \swD(L_i)\). Le nombre \( n_{i}\) est alors choisi plus grand que \( n_{i-1}\) de telle sorte que \( \varphi_{n_i}\notin\swD(L_i)\). Ce faisant, en posant \( \phi_i=\varphi_{n_i}\) nous avons
	\begin{equation}
		\phi_i\in\swD(L_{i+1})\setminus\swD(L_i)
	\end{equation}
	et \( \Int(L_n)\subset L_{n+1}\) et \( \Omega=\bigcup_nL_n\). Étant donné que \( (\phi_i)\) et une sous-suite de \( (\varphi_i)\) nous avons encore \( \phi_i\stackrel{\swD(\Omega)}{\longrightarrow}\varphi\).

	Soit \( i\in \eN\). Nous allons utiliser le corolaire \ref{CORooHTZVooFhgrSN} de la seconde forme géométrique du théorème de Hahn-Banach pour séparer les parties \( \{ \phi_i \}\) (compact) et \( \swD(L_i)\) (sous-espace vectoriel fermé de \( \swD(\Omega)\)) dans \( \swD(\Omega)\). Nous avons \( f_i\in \swD'(\Omega)\) telle que
	\begin{subequations}
		\begin{numcases}{}
			f_i(\phi_i)>\alpha\\
			f_i\big( \swD(L_i) \big)=0.
		\end{numcases}
	\end{subequations}

	Nous introduisons la fonction définie sur \( \swD(\Omega)\) par
	\begin{equation}    \label{EqJCqeXti}
		p(\phi)=\sum_{i=1}^{\infty}i\frac{ f_i(\phi) }{ | f_i(\phi_i) | }.
	\end{equation}
	Si \( \phi\in \swD(L_k)\), alors \( f_k(\phi)=0\) et même \( f_{l}(\phi)=0\) pour tout \( l\geq k\). Donc pour chaque \( k\), la somme définissant \( p\) est finie sur \( \swD(L_k)\). Nous en déduisons que \( p\) est continue sur chacun des \( \swD(L_k)\) et donc sur \( \swD(\Omega)\) par le lemme~\ref{LemWEGpemo}.

	L'image de la suite convergente \( \phi_k\stackrel{\swD(\Omega)}{\longrightarrow}\varphi\) par \( p\) doit être bornée parce que \( p\) est continue. Mais dans la somme  \eqref{EqJCqeXti}, tous les termes sont positifs et en particulier le terme \( i=k\) vaut \( k\), donc \( p(\phi_k)\geq k\), ce qui contredit le fait que l'image de la suite soit bornée. Nous en déduisons donc l'existence d'un compact \( K\) tel que \( \varphi_n\in \swD(K)\) pour tout \( n\).

	Nous devons encore prouver que \( \varphi_n\stackrel{\swD(K)}{\to}\varphi\) pour ce choix de \( K\). Vu que \( \varphi_n\stackrel{\swD(\Omega)}{\longrightarrow}\varphi\), le lemme~\ref{LemPESaiVw} nous dit que nous avons aussi \( \varphi_n\stackrel{ C^{\infty}(\Omega)}{\longrightarrow}\varphi\), ce qui signifie que pour tout \( K\) et \( m\) nous avons
	\begin{equation}
		p_{K,m}(\varphi_n-\varphi)\to 0.
	\end{equation}
	En particulier pour le \( K\) fixé plus haut nous avons \( p_m(\varphi_n-\varphi)\to 0\), c'est-à-dire que \( \varphi_n\stackrel{\swD(K)}{\longrightarrow}\varphi\).

\end{proof}

%\begin{definition}[Convergence dans \( \swD(\Omega)\)]
%    Nous avons \( f_n\stackrel{\swD'(\Omega)}{\longrightarrow}f\) si et seulement si \( f_n(\varphi)\to f(\varphi)\) pour tout \( \varphi\in\swD(\Omega)\).
%\end{definition}

\begin{proposition} \label{PropQAEVcTi}
	Soit un compact \( K\) de \( \Omega\).
	\begin{enumerate}
		\item
		      L'espace \( \swD(K)\) est complet.
		\item
		      L'espace \( \swD(K)\) est métrique.
	\end{enumerate}
\end{proposition}

\begin{proof}

	Nous allons d'abord montrer que \( \swD(K)\) est complet. Ensuite nous allons montrer que sa topologie peut être donnée par une distance.

	\begin{subproof}
		\spitem[Complet]
		Nous considérons une suite de Cauchy \( (\varphi_n)\) dans \( \swD(K)\) au sens de la définition~\ref{DefZSnlbPc}. Soient \( \epsilon>0\) et \( i\in \eN\); si \( k\) et \( l\) sont assez grands nous avons
		\begin{equation}
			\varphi_k-\varphi_l\in B_i(0,\epsilon).
		\end{equation}
		En particulier pour \( i=0\) nous avons l'inégalité
		\begin{equation}
			\| \varphi_k-\varphi_l \|_{\infty}\leq \epsilon,
		\end{equation}
		La suite \( (\varphi_n)\) est donc de Cauchy dans \( \big( C(K),\| . \|_{\infty} \big)\) et y converge donc par complétude, proposition~\ref{PropSYMEZGU}. Il existe donc une fonction \( \varphi\in C(K)\) telle que
		\begin{equation}
			\varphi_n\stackrel{unif}{\longrightarrow}\varphi.
		\end{equation}
		Notre jeu à présent est de prouver que \( \varphi\in\swD(K)\), c'est-à-dire qu'elle est de classe \(  C^{\infty}\).

		Soit un multiindice \( \alpha=\mu_1,\ldots, \mu_n,i\). Si \( k\) et \( l\) sont assez grands nous avons
		\begin{equation}
			\| \partial^{\alpha}(\varphi_k-\varphi_l) \|_{\infty}\leq \epsilon,
		\end{equation}
		c'est-à-dire que
		\begin{equation}
			\| \partial_i(\partial^{\mu}\varphi_k)-\partial_i(\partial^{\mu}\varphi_l) \|_{\infty}\leq \epsilon.
		\end{equation}
		Si nous notons \( \psi_k=\partial^{\mu}\varphi_k\) cela signifie que \( (\partial_i\psi_n)\) est une suite de Cauchy dans \( \big( C(K),\| . \|_{\infty} \big)\). Elle y converge donc et il existe une fonction \( g_i\in C(K)\) telle que
		\begin{equation}
			\partial_i\psi_n\stackrel{unif}{\longrightarrow}g_i.
		\end{equation}
		Dans ce cas le théorème~\ref{ThoSerUnifDerr} nous indique que \( g_i\) est de classe \( C^n\), c'est-à-dire que \( \varphi\in C^{n+1}(K)\).

		\spitem[Métrique]

		La proposition~\ref{PropLOwUvCO} nous dit que la topologie donnée par l'écart
		\begin{equation}
			d(\varphi_1,\varphi_2)=\sup_{k\geq 1}\min\{ \frac{1}{ k },p_{k-1}(\varphi_1-\varphi_2) \}
		\end{equation}
		est la même que celle de \( \swD(K)\). Il reste à montrer que cette formule est bien une distance au sens de la définition~\ref{DefMVNVFsX}.
		\begin{enumerate}
			\item
			      Nous avons bien \( d(\varphi_1,\varphi_2)\geq 0\) parce que tous les éléments du supremum et du minimum sont positifs.
			\item
			      Si \( d(\varphi_1,\varphi_2)=0\) alors pour tout \( k\) nous devons avoir \( p_{k-1}(\varphi_1-\varphi_2)=0\); en particulier pour \( k=1\) cela donne \( \varphi_1=\varphi_2\).
			\item
			      Nous avons
			      \begin{equation}
				      p_k(\varphi_1-\varphi_2)=p_k\big( -(\varphi_2-\varphi_1) \big)=p_k(\varphi_2-\varphi_1)
			      \end{equation}
			      en utilisant la propriété~\ref{ItemSHnimhDii} de la définition~\ref{DefPNXlwmi} de seminorme.
			\item
			      Nous avons
			      \begin{equation}
				      p_k(\varphi_1-\varphi_2)=p_k(\varphi_1-\varphi_3+\varphi_3-\varphi_2)\leq p_k(\varphi_1-\varphi_3)+p_k(\varphi_3-\varphi_2)
			      \end{equation}
			      en utilisant la propriété~\ref{ItemSHnimhDiii} de la définition~\ref{DefPNXlwmi}.
		\end{enumerate}
	\end{subproof}
\end{proof}
Notons que la proposition~\ref{PropLOwUvCO} nous dit que \( \swD(K)\) est complet tout autant pour la topologie des seminormes que pour celle de la distance que nous venons de décrire. Ces deux topologies sont les mêmes. Étant métrique et complet, l'espace \( \swD(\Omega)\) et donc de Baire par le théorème~\ref{ThoBBIljNM}. Ce qui est bien avec ces deux topologies identiques c'est qu'on peut utiliser la propriété de Baire même en ne parlant que des seminormes.

%+++++++++++++++++++++++++++++++++++++++++++++++++++++++++++++++++++++++++++++++++++++++++++++++++++++++++++++++++++++++++++
\section{Distributions}
%+++++++++++++++++++++++++++++++++++++++++++++++++++++++++++++++++++++++++++++++++++++++++++++++++++++++++++++++++++++++++++

Si \( \Omega\) est un ouvert de \( \eR^d\), alors l'ensemble \( \swD(\Omega)\) est contenu dans \(  C^{\infty}(\Omega)\). Nous allons commencer par définir une topologie sur \(  C^{\infty}(\Omega)\) et ensuite donner à \( \swD(\Omega)\) la topologie induite\footnote{Définition~\ref{DefVLrgWDB}.}.

\begin{definition}[Distribution]    \label{DefPZDtWVP}
	Une \defe{distribution}{distribution} sur un ouvert \( \Omega\) de \( \eR^d\) est une forme linéaire continue sur \(\swD(\Omega)= C^{\infty}_c(\Omega)\)\nomenclature[Y]{\( \swD(\Omega)\)}{Les fonctions \( C^{\infty}\) à support compact sur \( \Omega\)}. C'est donc un élément de \( \swD'(\Omega)\).
\end{definition}

Le théorème suivant donne quelques façons de vérifier qu'une forme linéaire soit continue. En particulier il nous dit que pour prouver qu'une forme linéaire est une distribution il suffit de prouver la continuité séquentielle.
\begin{theorem}[\cite{TQSWRiz,RIOTOaj}] \label{ThoVDDBnVn}
	Soit \( T\) une forme linéaire sur \( \swD(\Omega)\). Nous avons équivalence entre les points suivants.
	\begin{enumerate}
		\item
		      \( T\) est continue.
		\item   \label{ItemSPvoijoii}
		      Pour tout compact \( K\subset \Omega\) il existe \( m_K\in \eN\) et \( C_K\geq 0\) tels que pour tout \( \varphi\in\swD(K)\) nous ayons
		      \begin{equation}
			      \big| T(\varphi) \big|\leq C_K p_{m_K,K}(\varphi)
		      \end{equation}
		      où \( p_{m,K}\) est la seminorme donnée en \eqref{EQooZSQUooAJRIFe}.
		\item       \label{ITEMooBXFSooYtAXjy}
		      \( T\) est séquentiellement continue sur \( \swD(\Omega)\).
		\item
		      \( T\) est séquentiellement continue en \( 0\).
		\item
		      Pour tout compact \( K\subset \Omega\), la restriction de \( T\) à \( \swD(K)\) est continue.
	\end{enumerate}
\end{theorem}
Un certain nombre d'ouvrages prennent le point~\ref{ItemSPvoijoii} comme la définition d'une distribution.

\begin{definition}[Topologie sur \( \swD'(\Omega)\)]        \label{DefASmjVaT}
	Nous munissons l'espace \( \swD'(\Omega)\) de la \defe{topologie \( *\)-faible}{topologie!\( *\)-faible}, c'est-à-dire celle de la famille de seminormes
	\begin{equation}
		\begin{aligned}
			p_{\varphi}\colon \swD'(\Omega) & \to \eR                        \\
			T                               & \mapsto \big| T(\varphi) \big|
		\end{aligned}
	\end{equation}
	avec \( \varphi\in\swD(\Omega)\).
\end{definition}
Oui, c'est bien une famille de seminormes indicée par l'ensemble \( \swD(\Omega)\). Il n'y en a donc à priori pas du tout une quantité dénombrable.

\begin{proposition}[Convergence au sens des distributions]  \label{PropEUIsNhD}
	Nous avons \( T_n\stackrel{\swD'(\Omega)}{\longrightarrow}T\) si et seulement si \( T_n(\varphi)\to T(\varphi)\) pour tout \( \varphi\in\swD(\Omega)\).
\end{proposition}

\begin{proof}
	La convergence \( T_n\stackrel{\swD'(\Omega)}{\longrightarrow}T\) signifie que l'on ait \( p_{\varphi}(T_n-T)\to 0\) pour tout \( \varphi\in\swD(\Omega)\), ce qui en retour signifie que
	\begin{equation}
		\big| (T_n-T)(\varphi) \big|\to 0.
	\end{equation}
\end{proof}
Cette proposition suppose que l'on ait une distribution \( T\) qui vérifie \( T_n(\varphi)\to T(\varphi)\) et conclut qu'on a une convergence dans les distributions. Le théorème suivant est plus fort : il va seulement supposer que \( T_n(\varphi)\) converge dans \( \eC\) et va conclure que \( T\colon \varphi\mapsto \lim_{n\to \infty} T_n(\varphi)\) est une distribution.


\begin{definition}
	Si \( f\) est une fonction sur \( \eR^d\) telle que \( f\varphi\in L^1(\eR^d)\) pour tout \( \varphi\in \swD(\eR^d)\), alors nous définissons la distribution \( T_f\in\swD'(\eR^d)\) par
	\begin{equation}
		\langle T_f, \varphi\rangle =\int_{\eR^d}f(x)\varphi(x)dx.
	\end{equation}
\end{definition}

\begin{theorem}[\cite{GQYneyj}]
	Soit \( (T_n)\) une suite dans \( \swD'(\Omega)\) et nous supposons que pour tout \( \varphi\in\swD(\Omega)\) la suite \( \big( T_n(\varphi) \big)\) converge dans \( \eC\). Alors il existe \( T\in\swD'(\Omega)\) telle que \( T_n\stackrel{\swD'(\Omega)}{\longrightarrow}T\).
\end{theorem}

\begin{proposition}\label{PROPooYAJSooMSwVOm}
	L'application
	\begin{equation}
		\begin{aligned}
			i\colon L^2(\Omega) & \to \swD'(\Omega) \\
			f                   & \mapsto T_f
		\end{aligned}
	\end{equation}
	est une injection continue.
\end{proposition}

\begin{proof}
	Le fait que ce soit une injection est le fait que si \( T_f=T_g\) alors pour tout \( \phi\in \swD(\Omega)\) nous avons \( \langle f-g, \phi\rangle =0\), et cela implique que \( f-g\) est nulle presque partout en tant que fonction et est simplement nulle en tant que classe de fonction dans \( L^2\).

	En ce qui concerne la continuité, il suffit de la prouver en zéro (par linéarité). Soit donc \( f_n\stackrel{L^2}{\longrightarrow}0\) et montrons que \( T_{f_n}\stackrel{\swD'(\Omega)}{\longrightarrow} T_0\). Pour prouver cela, la proposition~\ref{PropEUIsNhD} nous indique qu'il est suffisant de tester \( T_n(\phi)\to 0\) pour tout \( \phi\in\swD(\Omega)\).

	Notons que si \( \phi\in \swD\) a fortiori \( \phi\in L^2\). Nous avons
	\begin{equation}
		T_{f_n}(\phi)=\int_{\Omega}f_n\phi\leq \| f_n\phi \|_{L^1}\leq \| f_n \|_2\| \phi \|_2\to 0
	\end{equation}
	où nous avons utilisé l'inégalité de Hölder de la proposition~\ref{ProptYqspT}.
\end{proof}

Cette proposition permet de donner un sens à des phrases du type «Soit une distribution \( T\). Si \( T\in L^2\), alors \ldots». Cela signifie qu'il existe \( u\in L^2\) tel que \( T=T_u\). Notons que dans ce cas, la distribution est définie sur \( L^2\) et non seulement sur \( \swD\).

%---------------------------------------------------------------------------------------------------------------------------
\subsection{Multiplication d'une distribution par une fonction}
%---------------------------------------------------------------------------------------------------------------------------

\begin{definition}  \label{DefZVRNooDXAoTU}
	Si \( T\in\swD'(\Omega)\) et si \( f\in  C^{\infty}(\Omega)\) nous définissons la distribution \( fT\) par
	\begin{equation}    \label{DefTDkrqkA}
		(fT)(\varphi)=T(f\varphi).
	\end{equation}
	Souvent écrit sous la forme plus compacte \( \langle fu, \phi\rangle =\langle u, f\phi\rangle \).
\end{definition}
\index{distribution!produit par une fonction}
\index{produit!distribution et fonction}
Cela a un sens parce que si \( \varphi\in\swD(\Omega)\) alors \( f\varphi\) est aussi dans \( \swD(\Omega)\).

Cette définition est motivée par ce que l'on ferait pour une distribution à densité. Si \( T\) est une distribution de densité notée également \( T\), nous avons \( T(\phi)=\int T(x)\phi(x)\) et donc
\begin{equation}
	(fT)(\phi)=\int (fT)(x)\phi(x)=\int T(x)f(x)\phi(x)=\int T(x)(f\phi)(x)=T(f\phi).
\end{equation}

En ce qui concerne les distributions tempérées, nous pouvons définir le produit avec une fonction \( f\in\swS(\Omega)\) par la même formule : si \( f,\varphi\in\swS(\Omega)\) alors le produit \( f\varphi\) est encore Schwartz. Notons toutefois que nous ne pouvons pas définir \( fT\) dans \( \swS'(\Omega)\) si \( f\) est seulement dans \(  C^{\infty}(\Omega)\).
% TODO : il faudrait prouver cette dernière affirmation.

%---------------------------------------------------------------------------------------------------------------------------
\subsection{Dérivée de distribution}
%---------------------------------------------------------------------------------------------------------------------------

\begin{propositionDef} \label{PropKJLrfSX}
	Soit \( T\) une distribution sur \( \Omega\) et \( \alpha\in \eN^d\). Alors la formule
	\begin{equation}
		(\partial^{\alpha}T)(\varphi)=(-1)^{| \alpha |}T(\partial^{\alpha}\varphi)
	\end{equation}
	définit une distribution \( \partial^{\alpha}T\).

	Cette distribution \( \partial^{\alpha}T\) sera la \defe{dérivée distributionnelle}{dérivée!distributionnelle} de \( T\). Notons que le même résultat est encore valide pour des distributions tempérées, et la démonstration est la même.
\end{propositionDef}

\begin{proof}
	La forme linéaire \( \partial^{\alpha}T\) sera continue si elle est séquentiellement continue par le théorème~\ref{ThoVDDBnVn}. Nous considérons donc une suite \( \varphi_n\stackrel{\swD(\Omega)}{\longrightarrow}\varphi\) et nous vérifions que
	\begin{equation}
		\lim_{n\to \infty} (\partial^{\alpha}T)(\varphi_n)=(\partial^{\alpha}T)(\varphi).
	\end{equation}
	D'abord \( T\) étant une distribution (et donc continue) nous pouvons la permuter avec la limite :
	\begin{equation}
		\lim_{n\to \infty} (\partial^{\alpha}T)(\varphi_n)=\lim_{n\to \infty} (-1)^{| \alpha |}T(\partial^{\alpha}\varphi_n)=(-1)^{| \alpha |}T\big( \lim_{n\to \infty} \partial^{\alpha}\varphi_n \big).
	\end{equation}
	Notons qu'à gauche la limite est une limite dans \( \eR\) tandis qu'à droite c'est une limite dans \( \swD(\Omega)\). Ensuite le lemme~\ref{LemXXwDjui} nous dit que l'hypothèse \( \varphi_n\stackrel{\swD(\Omega)}{\longrightarrow}\varphi\) signifie en particulier que nous avons un compact \( K\subset\Omega\) contenant tous les supports des \( \varphi_n\) et que \( \partial^{\alpha}\varphi_n\) converge uniformément (sur \( K\) et donc sur \( \Omega\)) vers \( \partial^{\alpha}\varphi\). Donc
	\begin{equation}
		\lim_{n\to \infty} (\partial^{\alpha}T)(\varphi_n)=(-1)^{| \alpha |}T\big( \lim_{n\to \infty} \partial^{\alpha}\varphi_n \big)=(-1)^{| \alpha |}T\big( \partial^{\alpha}\varphi \big)=(\partial^{\alpha}T)(\varphi),
	\end{equation}
	ce qui est la relation demandée.
\end{proof}

Le lemme suivant montre une compatibilité entre la dérivée des distributions, la dérivée faible et l'injection de \( L^2\) dans l'espace des distributions.

\begin{lemma}       \label{LEMooQRUOooWVjCAV}
	Soit \( \Omega\) un ouvert bornée de \( \eR^n\) et \( f\in L^2(\Omega)\). Alors nous avons
	\begin{equation}
		\partial_i(T_f)=T_{\partial_if}
	\end{equation}
	où la dérivée à droite est la dérivée faible définie en~\ref{DEFooBRFCooPncSCE}.
\end{lemma}

\begin{proof}
	En utilisant la définition de la dérivation de distribution, pour tout \( \phi\in \swD\) nous avons
	\begin{equation}
		\partial_i(T_f)\phi=-T_f(\partial_i\phi)=-\langle f, \partial_i\phi\rangle =\langle \partial_if, \phi\rangle =T_{\partial_if}(\phi).
	\end{equation}
	Nous avons utilisé la définition \eqref{EQooMRZUooFoqPqv} de la dérivée faible.
\end{proof}

Nous avons déjà vu dans l'exemple \ref{EXooRVGHooTWOCtF} que la fonction de Heaveside n'a pas de dérivée faible. Nous allons à présent voir que cette fonction a une dérivée au sens des distributions. Intuitivement, la fonction de Heaveside a une dérivée nulle partout sauf en \( x=0\) où sa dérivée serait infinie; nous nous attendons à un delta de Dirac.

\begin{proposition}     \label{PROPooVUDVooAlwZzB}
	La dérivée de la fonction de Heaveside
	\begin{equation}
		\begin{aligned}
			H\colon \eR & \to \eR^+                      \\
			x           & \mapsto \begin{cases}
				                      0 & \text{si } x\leq 0 \\
				                      1 & \text{si }x>0
			                      \end{cases}
		\end{aligned}
	\end{equation}
	est le delta de Dirac.
\end{proposition}

\begin{proof}
	Par définition, la dérivée de \( H\) au sens des distributions est la distribution \( H'\) qui fait
	\begin{equation}
		H'(\varphi)=-\langle H, \varphi'\rangle
	\end{equation}
	pour tout élément \( \varphi\in\swD(\eR)\). Un petit calcul :
	\begin{subequations}
		\begin{align}
			-\langle H, \varphi'\rangle & =-\int_{\eR}H(t)\varphi'(t)dt                                                          \\
			                            & =-\int_{0}^{\infty}\varphi'(t)dt       \label{SUBEQooMRMBooEGeDra}                     \\
			                            & =-\lim_{x\to \infty} \int_0^x\varphi'(t)dt     \label{SUBEQooZDSDooTbzgWD}             \\
			                            & =-\lim_{x\to \infty} \big( \varphi(x)-\varphi(0) \big)     \label{SUBEQooSUAHooOmjgcr} \\
			                            & =\varphi(0).
		\end{align}
	\end{subequations}
	Justifications :
	\begin{itemize}
		\item Pour \eqref{SUBEQooMRMBooEGeDra}, c'est que \( H(t)=0\) pour \( t\in \mathopen] -\infty , 0 \mathclose[\).
		\item Pour \eqref{SUBEQooZDSDooTbzgWD}, c'est le lemme \ref{LEMooMUHWooZPbMDb}.
		\item Pour \eqref{SUBEQooSUAHooOmjgcr}, c'est le corolaire \ref{CorMUIooXREleR}.
	\end{itemize}
\end{proof}

%---------------------------------------------------------------------------------------------------------------------------
\subsection{Ordre et support d'une distribution}
%---------------------------------------------------------------------------------------------------------------------------

\begin{definition}[support d'une distribution\cite{OEVAuEz}]        \label{DefVILMooBIYerO}
	Soit \( T\) une distribution. Le \defe{support}{support!distribution} de \( T\) est le complémentaire de l'union des ouverts \( \mO\) tels que \( T(\varphi)=0\) pour tout \( \varphi\) à support dans \( \mO\).
\end{definition}
\index{support!distribution}
%TODO : dans le cas d'une distribution définie par une densité, il faut voir le lien entre le support de la fonction et le support de la distribution. (j'imagine qu'ils doivent être égaux).

\begin{definition}  \label{DefXAHIooFeiRMB}
	Si \( T\) est une distribution sur \( \Omega\), nous disons que \( T\) est d'\defe{ordre}{ordre!distribution} inférieur ou égal à \( p\in \eN\) si pour tout compact \( K\) de \( \Omega\), il existe \( C_K\in \eR\) tel que pour tout \( \varphi\in \swD(K)\),
	\begin{equation}
		\big| \langle T, \varphi\rangle  \big|\leq C_K\max_{| \alpha |\leq p}\| \partial^{\alpha}\varphi \|_{\infty}.
	\end{equation}
	Ici \( \alpha\) est un multiindice.

	La distribution \( T\) est d'ordre \( p\) si elle est d'ordre inférieur ou égal à \( p\) mais pas à \( p-1\).
\end{definition}

Pour la proposition suivante, on peut se remémorer la définition~\ref{DefFGGCooTYgmYf} de la topologie sur \(  C^{\infty}(\Omega)\).
\begin{proposition}[\cite{RGWgOms}]		\label{PROPooSQPHooDPOrqK}
	Restriction entre \(  C^{\infty}\) et \( \swD\).
	\begin{enumerate}
		\item
		      Si \( T\in C^{\infty}(\Omega)'\), alors la restriction de \( T\) à \( \swD(\Omega)\) est une distribution à support\footnote{Définition~\ref{DefVILMooBIYerO}.} compact.
		\item
		      Si \( T\) est une distribution à support compact alors elle se prolonge de façon unique en une forme linéaire continue sur \(  C^{\infty}(\Omega)\).
	\end{enumerate}
\end{proposition}

\begin{proposition}[\cite{TQSWRiz}]     \label{PropZLUEooHcVxQj}
	Une distribution à support compact est d'ordre fini.
\end{proposition}

\begin{lemma}[\cite{PAXrsMn}]   \label{LemYHRDooOdSnnK}
	Soit \( u\in\swD'(\eR)\) et \( \phi\in\swD(\eR)\) tels que \( \supp(u)\cap\supp(\phi)=\emptyset\). Alors \( \langle u, \phi\rangle =0\).
\end{lemma}

\begin{proof}
	Soit \( x\notin\supp(u)\). Alors il existe un voisinage \( V_x\) de \( x\) tel que \( \langle u, \psi\rangle =0\) pour tout \( \psi\in\swD(V_x)\). En particulier, si \( x\in\supp(\phi)\), alors \( x\) n'est pas dans le support de \( u\) et les ensembles \( \{ V_x\tq x\in\supp(\phi) \}\) recouvrent \( \supp(\phi)\). Cependant \( \phi\) est à support compact et nous pouvons extraire un sous-recouvrement fini de \( \supp(\phi)\) : il existe \( x_1,\ldots, x_p\) tels que
	\begin{equation}
		\supp(\phi)\subset\bigcup_{i=1}^pV_{x_i}.
	\end{equation}
	Nous prenons une partition de l'unité\footnote{Théorème \ref{THOooQFCQooSlgLpz}.} subordonnée à ce recouvrement. C'est-à-dire des fonctions \( \chi_i\in\swD(V_{x_i})\) telles que pour tout \( x\in\supp(\phi)\),
	\begin{equation}
		\sum_{i=1}^p\chi_i(x)=1.
	\end{equation}
	En particulier nous avons \( \sum_i\chi_i(x)\phi(x)=\phi(x)\), et donc
	\begin{equation}
		\langle u, \phi\rangle =\langle u, \sum \chi_i\phi\rangle =\sum\langle u, \chi_i\phi\rangle =0
	\end{equation}
	parce que \( \supp(\chi_i\phi)\subset V_{x_i}\).
\end{proof}

\begin{lemma}[\cite{PAXrsMn}]
	Si \( u\) est une distribution d'ordre fini \( N\) sur \( \eR\), si \( \supp(u)=\{ x_0 \}\) et si
	\begin{equation}
		\phi(x_0)=\ldots=\phi^{(N)}(x_0)=0
	\end{equation}
	alors \( \langle u, \phi\rangle =0\).
\end{lemma}

\begin{proof}
	Les fonctions plateaux dont nous avons parlé dans la section~\ref{subsecOSYAooXXCVjv} nous permettent de considérer une fonction \( \chi\in\swD(\eR)\) vérifiant
	\begin{equation}
		\chi(x)=\begin{cases}
			1 & \text{si } x\in\overline{ B(0,1) } \\
			0 & \text{si } | x |>2
		\end{cases}
	\end{equation}
	Ensuite nous posons \( \chi_n(x)=\chi\big( n(x-x_0) \big)\). Par conséquent \( \chi_n(x_0)=\chi(0)=1\) et même \( \chi_n(x_0+\epsilon)=\chi(\epsilon)=1\) tant que \( \epsilon\) est plus petit que disons \( \frac{ 1 }{2n}\) pour être sûr. Nous en déduisons que la fonction \( 1-\chi_n\) s'annule sur un voisinage de \( x_0\) et que donc \( x_0\) n'est pas dans le support de \( 1-\chi_n\). Donc \( \supp(1-\chi_n)\cap\supp(u)=\emptyset\) et le lemme~\ref{LemYHRDooOdSnnK} est utilisable :
	\( \langle u, (1-\chi_n\phi)\rangle =0\), ou encore :
	\begin{equation}
		\langle u, \phi\rangle =\langle u, \chi_n\phi\rangle
	\end{equation}
	pour tout \( n\). Vu que le but est de prouver que \( \langle u, \phi\rangle =0\), nous allons prouver que
	\begin{equation}
		| \langle u, \chi_n\phi\rangle  |\stackrel{n\to\infty}{\longrightarrow}0.
	\end{equation}
	Dans ce dessein nous posons
	\begin{equation}
		\| f \|_n=\sup_{x\in\overline{ B(0,\frac{ 2 }{ n })}}\| f(x) \|
	\end{equation}
	et
	\begin{equation}
		\| f \|_{(p)}=\sup_{i\leq p}\| \partial^if \|_{\infty}.
	\end{equation}
	La distribution \( u\) est d'ordre fini \( N\), et nous en écrivons la définition~\ref{DefXAHIooFeiRMB} en prenant \( \supp(\chi_n\phi)\) en tant que \( K\) :
	\begin{equation}
		\big| \langle u, \chi_n\phi\rangle  \big|\leq C\max_{k\leq N}\| \partial^k(\chi_n\phi) \|_{\infty}.
	\end{equation}
	En remplaçant le maximum par une somme de \( k=0\) à \( k=N\), nous majorons. De plus le support de \( \chi_n\) étant contenu dans \( B_n=B(x_0,2/n)\) nous ne changeons rien en utilisant \( \| . \|_n\) au lieu de \( \| . \|_{\infty}\). Donc
	\begin{equation}
		| \langle u, \chi_n\phi\rangle  |\leq C\sum_{k=0}^N\| \partial^k(\chi_n\phi) \|_n\leq C\sum_{k=0}^N\sum_{i=0}^k\binom{ k }{ i }\| \partial^i\chi_n \|_n\| \partial^{k-i}\phi \|_n.
	\end{equation}
	Notons que la seconde inégalité est une inégalité du type \( \| fg \|\leq \| f \|\| g \|\). En dérivant un petit peu nous trouvons que
	\begin{equation}
		(\partial^i\chi_n)(x)=n^i(\partial^i\chi)\big( n(x-x_0) \big).
	\end{equation}
	Donc\quext{Dans \cite{PAXrsMn}, la dernière égalité vient avec une inégalité, et je comprends pas pourquoi.}
	\begin{equation}    \label{EqFKWTooFfZoSM}
		\| \partial^i\chi_n \|_n=\sup_{x\in B_n}n^i\big| (\partial^i\chi)\big( n(x-x_0) \big) \big|=n^i\sup_{y\in\mathopen[ -2 , 2 \mathclose]}\big| (\partial^i\chi)(y) \big|=n^i\| \partial^i\chi \|_{\infty}.
	\end{equation}
	Nous pouvons donc remplacer \( \| \partial^i\chi_n \|_n\) par \( n^i\| \partial^i\chi \|_{\infty}\).

	D'autre part nous voulons majorer \( \| \partial^{k-i}\phi \|_n\) par quelque chose ne dépendant ni de \( k\) ni de \( i\). Nous faisons le théorème des accroissements finis~\ref{ThoNAKKght} : \( \| \partial^l\phi \|_n\leq \frac{ 2 }{ n }\| \partial^{l+1}\phi \|_n\). Ce \( n\) au dénominateur est salutaire parce que nous avions un \( n^i\) apparu à cause du remplacement \eqref{EqFKWTooFfZoSM}. Nous faisons donc \( i+1\) fois le théorème des accroissements finis :
	\begin{equation}
		\| \partial^{k-i} \phi\|_n\leq \left( \frac{ 2 }{ n } \right)^{i+1}\| \partial^{k+1}\phi \|_n.
	\end{equation}
	Toutes ces majorations donnent
	\begin{subequations}
		\begin{align}
			\big| \langle u, \chi_n\phi\rangle  \big| & \leq C\sum_{k=0}^N\sum_{i=0}^k\binom{ k }{ i }n^i\underbrace{\| \partial^i\chi \|_{\infty}}_{\leq \| \chi \|_{(N)}}\left( \frac{ 2 }{ n } \right)^{i+1}  \underbrace{\|  \partial^{k+1}\phi \|_n}_{\leq \| \phi \|_{(N+1)}} \\
			                                          & \leq C\| \chi \|_{(N)}\| \phi \|_{(N+1)}\frac{1}{ n }\sum_{k=0}^N\sum_{i=0}^k\binom{ k }{ i }2^{i+1}                                                                                                                        \\
			                                          & =\frac{ C' }{ n }
		\end{align}
	\end{subequations}
	où \( C'\) est une constante qui dépend de \( \chi\), de \( \phi\) et de \( N\), mais pas de \( n\). Vu que \( \frac{ C' }{ n }\to 0\) nous avons bien
	\begin{equation}
		\langle u, \phi\chi_n\rangle=0,
	\end{equation}
	ce qu'il fallait démontrer.
\end{proof}


\begin{proposition}[\cite{PAXrsMn}]     \label{PropXXPLooSkgxOz}
	Soit \( u\in\swD(\eR)'\) avec \( \supp(u)=\{ x_0 \}\). Alors \( u=\sum_{i=0}^{N}a_i\partial^i\delta_{x_0}\) où \( N\) est l'ordre de \( u\).
\end{proposition}

\begin{proof}
	D'abord il faut préciser que l'ordre de \( u\) est fini parce que son support est compact (proposition~\ref{PropZLUEooHcVxQj}); nous notons \( N\) cet ordre.

	Soit \( \phi\in\swD(\eR)\).  Nous considérons \( \chi\in\swD(\eR)\) telle que
	\begin{equation}
		\chi(x)=\begin{cases}
			1 & \text{si } x\in\overline{ B(x_0,1) } \\
			0 & \text{si } | x-x_0 |>2.
		\end{cases}
	\end{equation}
	Encore une fois, \( 1-\chi\) s'annule sur un voisinage autour de \( x_0\), ce qui fait que
	\begin{equation}
		\supp(u)\cap\supp\big( (1-\chi)\phi \big)=\emptyset,
	\end{equation}
	et donc \( \langle u, (1-\chi)\phi\rangle =0\). Au final,
	\begin{equation}
		\langle u, \phi\rangle =\langle u, \chi\phi\rangle.
	\end{equation}
	C'est le moment de poser
	\begin{equation}
		\psi(x)=\chi(x)\big[   \phi(x)-\sum_{k=1}^N\frac{1}{ k! }(\partial^k\phi)(x_0)(x-x_0)^k \big]
	\end{equation}
	La fonction \( \psi\) ayant un support disjoint de celui de \( u\), nous avons aussi \( \langle u, \psi\rangle =0\), ce qui donne
	\begin{equation}
		\langle u, \phi\rangle =\langle u, \chi\phi\rangle =\langle u, \chi\sum_{k=0}^N\frac{1}{ k! }(\partial^k\phi)(x_0)(x-x_0)^k\rangle .
	\end{equation}
	En posant \( a_k=\frac{1}{ k! }\langle u, x\mapsto \chi(x)(x-x_0)^k\rangle \) nous avons alors
	\begin{equation}
		\langle u, \phi\rangle =\sum_{k=0}^Na_k(\partial^k\phi)(x_0)=\sum_k(-1)^ka_k(\partial^k\delta_{x_0})(\phi).
	\end{equation}
\end{proof}

%+++++++++++++++++++++++++++++++++++++++++++++++++++++++++++++++++++++++++++++++++++++++++++++++++++++++++++++++++++++++++++
\section{L'espace \texorpdfstring{\(   C^{\infty}(\eR,\swD'(\eR^d))\)}{C(R,D')}}
%+++++++++++++++++++++++++++++++++++++++++++++++++++++++++++++++++++++++++++++++++++++++++++++++++++++++++++++++++++++++++++
\label{SecTEgDVWO}

D'abord parlons un peu de continuité en recopiant la proposition~\ref{PropVKSNflB} dans notre contexte.
\begin{proposition}     \label{PropIPlKQBa}
	Soient \( I\) un intervalle ouvert de \( \eR\) et \( u\colon I\to \swD'(\eR^d)\) une fonction continue. Alors
	\begin{enumerate}
		\item   \label{ItemYAhnNhBi}
		      Pour tout \( \varphi\in\swD(\eR^d)\), l'application \( t\mapsto u_t(\varphi)\) est continue.
		\item
		      Pour tout \( \varphi\in\swD(\eR^d)\), nous avons la limite
		      \begin{equation}
			      \lim_{t\to t_0} u_t(\varphi)=u_{t_0}(\varphi).
		      \end{equation}
		\item
		      Nous avons la limite dans \( \swD'(\eR^d)\)
		      \begin{equation}
			      \lim_{t\to t_0} u_t=u_{t_0}.
		      \end{equation}
	\end{enumerate}
\end{proposition}
En ce qui concerne la définition de l'espace \( C^{\infty}(I,\swD'(\eR^d))\)\nomenclature[Y]{\(  C^{\infty}(I,\swD'(\eR^d))\)}{fonctions à valeurs dans les distributions}, c'est la définition~\ref{DefDZsypWu}. Grâce au point~\ref{ItemYAhnNhBi} de la proposition~\ref{PropIPlKQBa}, nous retenons que la propriété fondamentale d'une application \( T\in C^k\big( I,\swD'(\Omega) \big)\) est que pour tout \( \varphi\in\swD(\Omega)\), l'application
\begin{equation}
	\begin{aligned}
		I & \to \eC              \\
		t & \mapsto T_t(\varphi)
	\end{aligned}
\end{equation}
est de classe \( C^k\).

\begin{proposition} \label{PropOTlWzog}
	Soit \( I\), un intervalle ouvert de \( \eR\). Soit \( T\in C^0\big( I,\swD'(\Omega) \big)\) et \( \psi\in\swD(I\times \Omega)\). Alors l'application
	\begin{equation}
		t\mapsto T_t\big( \psi(t,.) \big)
	\end{equation}
	est continue sur \( I\).
\end{proposition}

\begin{proof}
	La fonction dont nous voulons prouver la continuité est une fonction \( \eR\to\eR\); il est donc loisible de se contenter de la continuité séquentielle. Soient \( t_0\in I\) et \( (t_j)\) une suite dans \( I\) convergeant vers \( t_0\). Nous posons \( U_j=T_{t_j}\) et \( \psi_j=\psi\big( t_j,. \big)\). Par hypothèse de continuité de \( (T_t)\) nous avons \( U_j\stackrel{\swD'(\Omega)}{\longrightarrow}T_{t_0}\). D'autre part le support de \( \psi\) étant compact nous avons \( \supp(\psi)\subset \mathopen[ c , d \mathclose]\times K\) où \( \mathopen[ c , d \mathclose]\subset I\) et \( K\) est compact dans \( \Omega\). Par conséquent nous avons aussi \( \supp(\psi_j)\subset K\).

	Afin d'alléger les notations notons \( \tilde \psi(x)=\psi(t_0,x)\). Pour tout multiindice \( \alpha\) et pour tout \( j\) nous avons
	\begin{equation}
		p_{\alpha}(\psi_i-\tilde \psi)=\Big|  \partial^{\alpha}\psi(t_j,x)-\partial^{\alpha}\psi(t_0,x)    \Big|\leq | t_j-t_0 |\sup_{\substack{t\in\mathopen[ c , d \mathclose]\\x\in K}}| \partial_t\partial^{\alpha}\psi(t,x) |\to 0.
	\end{equation}
	Nous avons donc la convergence
	\begin{equation}
		\psi_j\stackrel{\swD(K)}{\longrightarrow}\psi(t_0,.).
	\end{equation}

	Étant donné que \( U_j\stackrel{\swD'(\Omega)}{\longrightarrow}T_{t_0}\) et \( \psi_j\stackrel{\swD(\Omega)}{\longrightarrow}\tilde \psi\), le corolaire~\ref{CorPGwLluz}\ref{ItemAEOtOMLiii} nous donne la convergence
	\begin{equation}
		U_j(\psi_j)\stackrel{\eC}{\longrightarrow} T_{t_0}(\tilde \psi)
	\end{equation}
	Cela est bien la continuité de la fonction \( t\mapsto T_t\big( \psi(t,.) \big)\).
\end{proof}


\begin{proposition}[\cite{GQYneyj}] \label{PropLKtBsVi}
	Soit \( (T_t)\in C^0\big( I,\swD'(\Omega) \big)\). Nous définissons l'application \( T\colon \swD(\Omega)\to \eC\) par la formule
	\begin{equation}
		T(\psi)=\int_I T_t\big( \psi(t,.) \big)\,dt
	\end{equation}
	pour tout \( \psi\in\swD(I\times \Omega)\). Alors \( T\in \swD'(I\times \Omega)\).
\end{proposition}

\begin{proof}
	La proposition~\ref{PropOTlWzog} nous indique que la fonction \( t\mapsto T_t\big( \psi(t,.) \big)\) est continue. Étant donné qu'elle est seulement non nulle sur un compact, l'intégrale
	\begin{equation}
		\int_IT_t\big( \psi(t,.) \big)dt
	\end{equation}
	a un sens et est finie. L'application \( T\colon \swD(I\times \Omega)\to \eC\) ainsi définie est linéaire. Il reste à voir qu'elle est continue. Pour cela nous allons utiliser le théorème~\ref{ThoVDDBnVn}\ref{ItemSPvoijoii} qui nous dit que nous pouvons nous fixer un compact \( \mathopen[ c , d \mathclose]\times K\subset I\times\Omega\) et considérer \( \psi\in \swD\big( \mathopen[ c , d \mathclose]\times K \big)\).

	Soit, pour commencer, donnée une application \( \varphi\in\swD(K)\). L'application \( t\mapsto T_t(\varphi)\) est continue et non nulle sur le compact \( \mathopen[ c,d\mathclose]\) et il existe donc \( C_{\varphi}>0\) tel que
	\begin{equation}
		| T_t(\varphi) |\leq C_{\varphi}
	\end{equation}
	pour tout \( t\in\mathopen[ c , d \mathclose]\).

	Nous voulons utiliser le théorème de Banach-Steinhaus dans sa version~\ref{ThoNBrmGIg} sur la famille d'applications paramétrée par \( u\in\mathopen[ c , d \mathclose]\) :
	\begin{equation}        \label{EqBEKoqMb}
		\begin{aligned}
			U_u\colon \swD\big( \mathopen[ c , d \mathclose]\times K \big) & \to \eR                           \\
			\psi                                                           & \mapsto T_u\big( \psi(u,.) \big).
		\end{aligned}
	\end{equation}
	Commençons par prouver que cela est une application continue pour chaque \( u\). Ce sera le cas si la projection
	\begin{equation}
		\begin{aligned}
			\pr\colon \swD\big( \mathopen[ c , d \mathclose]\times K \big) & \to \swD(K)       \\
			\psi                                                           & \mapsto \psi(u,.)
		\end{aligned}
	\end{equation}
	est continue. Pour cela nous notons \( P_{kl}\) la seminorme sur \( \swD\big( \mathopen[ c , d \mathclose]\times K \big)\) donnée par
	\begin{equation}
		P_{k,l}(\psi)=\sum_{n\leq k}\sum_{| \alpha |\leq l}\sup_{\substack{t\in\mathopen[ c , d \mathclose]\\x\in K}}\big| \partial_t^n\partial^{\alpha}\psi(t,x) \big|.
	\end{equation}
	Nous montrons que \( \pr\) est séquentiellement continue; étant donné que les topologies sur \( \swD(K)\) et \( \swD\big( \mathopen[ c , d \mathclose]\times K \big)\) sont données par des métriques (proposition~\ref{PropQAEVcTi}), cela suffit pour assurer la continuité grâce à la proposition~\ref{PropXIAQSXr}. Montrons que si \( \psi_n\stackrel{\swD\big( \mathopen[ c , d \mathclose]\times K \big)}{\longrightarrow}0\), alors \( \pr(\psi_n)\stackrel{\swD(K)}{\longrightarrow}0\). Pour cela nous remarquons que
	\begin{subequations}
		\begin{align}
			p_j\big( \pr(\psi) \big) & =\sum_{| \alpha |\leq j}\sup_{x\in K}| \partial^{\alpha}\psi(u,x) |                                            \\
			                         & \leq \sum_{| \alpha |\leq j}\sup_{t\in\mathopen[ c , d \mathclose]}\sup_{x\in K}| \partial^{\alpha}\psi(t,x) | \\
			                         & =P_{0,j}(\psi).
		\end{align}
	\end{subequations}
	Par conséquent
	\begin{equation}
		p_j\big( \pr(\psi_n) \big)\leq P_{0,j}(\psi_n)\to 0
	\end{equation}
	où nous avons utilisé la proposition~\ref{PropQPzGKVk}. Utilisant cette même proposition à l'envers, nous déduisons que \( \pr(\psi_n)\stackrel{\swD(K)}{\longrightarrow}0\). Les applications \( U_u\) sont donc continues; elles sont également bornées parce que si \( \psi\in\swD\big( \mathopen[ c , d \mathclose]\times K \big) \) nous avons
	\begin{equation}
		\sup_{u \in\mathopen[ c , d \mathclose]}\big| U_u(\psi) \big|=\sup_u \big| T_u\big( \psi(u,.) \big) \big|,
	\end{equation}
	et la continuité déjà évoquée, sur le compact \( \mathopen[ c , d \mathclose]\), nous dit que cette quantité est finie. Le théorème de Banach-Steinhaus peut maintenant être appliqué et il existe \( C>0\) et \( k,l\in \eN\) tels que pour tout \( \psi\in\swD\big( \mathopen[ c , d \mathclose]\times K \big)\),
	\begin{equation}
		\big| U_u(\psi) \big|\leq C P_{k,l}(\psi)=C\sum_{| \alpha |\leq k}\sum_{n\leq l}\sup_{t,x}\big| \partial_t^n\partial^{\alpha}\psi(t,x) \big|\leq C\sum_{| \alpha |+n\leq k+l}\sup_{t,x}\big| \partial_t^n\partial^{\alpha}\psi(t,x) \big|.
	\end{equation}
	Quelques remarques
	\begin{itemize}
		\item Nous n'avons pas mis de maximum devant le supremum (alors que la conclusion \eqref{EqIFNGhtr} en demande) parce que dans le cas des seminormes \( P_{kl}\), c'est toujours celle avec \( k\) et \( l\) le plus grand possible qui sont les plus grandes parce qu'elles sont des sommes emboitées les unes les autres.
		\item La fusion de deux sommes est bien une majoration parce qu'il y a plus de termes dans la seconde que dans la première.
		\item La quantité la plus à droite est (à part le \( C\)) ce que nous pouvons noter \( P_{k+l}(\psi)\) : c'est bien une des seminormes associées à l'espace de dimension \( d+1\).
	\end{itemize}
	Nous majorons maintenant \( T(\psi)\) par
	\begin{equation}
		\big| T(\psi) \big|\leq \int_c^d\big| T_t\big( \psi(t,.) \big) \big|dt
		=\int_c^d\big|   U_t(\psi) \big|dt
		\leq C| d-c |P_{k+l}(\psi).
	\end{equation}
	Maintenant le théorème~\ref{ThoVDDBnVn}\ref{ItemSPvoijoii} appliqué à l'ouvert \( I\times \Omega\) et avec \( \psi\) au lieu de \( \varphi\) nous informe que \( T\in\swD(I\times K)\).
\end{proof}

%---------------------------------------------------------------------------------------------------------------------------
\subsection{Dérivation}
%---------------------------------------------------------------------------------------------------------------------------

Quelques propriétés de dérivation des fonctions \( I\to \swD(\Omega)\) seront directement énoncées et démontrées dans le cas des distributions tempérées. Les résultats~\ref{PropGKoPbko} et~\ref{PropUDkgksG} seront a fortiori valables si nous remplaçons \( \swS\) par \( \swD\).

%+++++++++++++++++++++++++++++++++++++++++++++++++++++++++++++++++++++++++++++++++++++++++++++++++++++++++++++++++++++++++++
\section{Une équation de distribution}
%+++++++++++++++++++++++++++++++++++++++++++++++++++++++++++++++++++++++++++++++++++++++++++++++++++++++++++++++++++++++++++

Nous allons étudier l'équation
\begin{equation}    \label{EqLLTPooJHUVvU}
	(x-x_0)^{\alpha}u=0
\end{equation}
pour \( u\in\swD'(\eR)\) et \( \alpha\in \eN\) est donné fixé. Notons tout de suite que \eqref{EqLLTPooJHUVvU} est un petit abus de notation pour dire qu'en vertu de la définition~\ref{DefZVRNooDXAoTU} du produit d'une distribution par une fonction, pour tout \( \phi\in\swD(\eR)\), nous avons \( u\Big( x\mapsto (x-x_0)^{\alpha}\phi(x) \Big)=0\).

\begin{lemma}[\cite{PAXrsMn}]       \label{LemWIGKooQpGXoI}
	Soit \( \alpha\in \eN\). Une solution à l'équation
	\begin{equation}        \label{EqKVNEooJNwsPc}
		(x-x_0)^{\alpha}u=0
	\end{equation}
	est une distribution à support dans \( \{ x_0 \}\) et d'ordre fini.
\end{lemma}

\begin{proof}
	Nous commençons par prouver que \( u\) est une solution de \eqref{EqKVNEooJNwsPc} si et seulement si\footnote{En réalité nous n'aurons besoin que de la condition nécessaire, en particulier pour le théorème~\ref{ThoRDUXooQBlLNb}.} \( \langle u, \phi\rangle =0\) pour tout \( \phi\in\swD\) telle que
	\begin{equation}    \label{EqYLIPooYByzwC}
		\phi(x_0)=\ldots=\partial^{\alpha-1}\phi(x_0)=0.
	\end{equation}
	\begin{subproof}
		\spitem[Condition nécéssaire]
		Supposons que \( u\) soit une solution. Alors le corolaire~\ref{CorQBXHooZVKeNG} du théorème de Hadamard donne \( \psi\in\swD(\eR)\) telle que
		\( \phi(x)=(x-x_0)^{\alpha}\psi(x)\).
		Dans ce cas, si \( u\) est solution de \eqref{EqKVNEooJNwsPc}, alors
		\begin{equation}
			0=\langle (x-x_0)^{\alpha}u, \psi\rangle =\langle u, (x-x_0)^{\alpha}\psi(x)\rangle =\langle u, \phi\rangle .
		\end{equation}
		Nous avons vu que si \( u\) est solution, alors \( \langle u, \phi\rangle =0\) pour tout \( \phi\) satisfaisant la condition \eqref{EqYLIPooYByzwC}.

		\spitem[Condition suffisante]
		Supposons maintenant l'inverse : \( u\) est une distribution s'annulant sur toute fonction \( \phi\in\swD'\) satisfaisant \eqref{EqYLIPooYByzwC}. Nous allons alors prouver que \( u\) est une solution. Soit donc \( \psi\in \swD\) et calculons
		\begin{equation}
			\langle (x-x_0)u, \psi\rangle =\langle u, (x-x_0)\psi\rangle =0
		\end{equation}
		parce que la fonction \( (x-x_0)\psi(x)\) vérifie la condition \eqref{EqYLIPooYByzwC}.
	\end{subproof}

	Nous passons maintenant au cœur de la preuve : nous supposons que \( u\) est une solution. Si le support de \( \phi\) est contenu dans \( \eR\setminus\{ x_0 \}\) alors \( \phi\) est nulle dans un voisinage de \( x_0\) (et donc \( \partial^k\phi=0\) pour tout \( k\)) et \( \langle u, \phi\rangle =0\). Autrement dit, pour  tout \( \phi\in\swD\big( \eR\setminus\{ x_0 \} \big)\) nous avons \( \langle u, \phi\rangle =0\), ce qui signifie que \( \supp(u)\cap\big( \eR\setminus\{ x_0 \} \big)=\emptyset\) ou encore que \( \supp(u)=\{ x_0 \}\).

	Maintenant que \( u\) a un support compact, la proposition~\ref{PropZLUEooHcVxQj} nous indique qu'elle est d'ordre fini.
\end{proof}

\begin{theorem}[\cite{PAXrsMn}]     \label{ThoRDUXooQBlLNb}
	Soit \( \alpha\in \eN\) et l'équation
	\begin{equation}        \label{EqDONTooKPfDWU}
		(x-x_0)^{\alpha}u=0
	\end{equation}
	pour \( u\in\swD'(\eR)\). Les solutions sont les combinaisons linéaires des dérivées de \( \delta_{x_0}\) jusqu'à la \( \alpha\)\ieme exclue.
\end{theorem}

\begin{proof}
	D'abord montrons que les \( \partial^i\delta_{x_0}\) sont des solutions. Avec les définition~\ref{DefZVRNooDXAoTU} et~\ref{PropKJLrfSX} des dérivées de distributions et de leur produits avec des fonctions\footnote{Comme souvent, dans l'expression suivante, il y a un abus de notation parce que \( x\) est une variable muette : il faudrait écrire «\( x\mapsto\)» au début de la grande parenthèse.},
	\begin{equation}
		(x-x_0)^{\alpha}\partial^i\delta_{x_0}(\phi)=\delta_{x_0}\Big( \partial^i\big( (x-x_0)^{\alpha}\phi(x) \big) \Big)
	\end{equation}
	Si \( i<\alpha\) alors dans chaque terme de Leibnitz\footnote{Lemme \ref{LEMooOLQTooEHJuBc}.}, il y aura un facteur \( (x-x_0)\), et la prise de \( \delta_{x_0}\) annulera. Si par contre \( i\geq \alpha\) alors il y aura le terme
	\begin{equation}
		\binom{ i }{ \alpha }\partial^{\alpha}\big( (x-x_0)^{\alpha} \big)\partial^{i-\alpha}\phi(x_0)=\binom{ i }{ \alpha }\alpha!(\partial^{i-\alpha}\phi)(x_0)
	\end{equation}
	qui est le seul terme contenant \( (\partial^{i-\alpha}\phi)(x_0)\). Il suffit alors de choisir \( \phi\in\swD(\eR) \) de sorte que
	\begin{equation}
		(\partial^k\phi)(x_0)=\begin{cases}
			0 & \text{si } k\neq i-\alpha \\
			1 & \text{si } k=i-\alpha
		\end{cases}
	\end{equation}
	et alors on est certain que le tout n'est pas nul, et donc que \( (x-x_0)^{\alpha}(\partial^i\delta_{x_0})\neq 0\).

	Jusqu'ici nous avons prouvé que \( \partial^i\delta_{x_0}\) est solution si et seulement si \( 0\leq i<\alpha\).

	Il faut encore prouver que les solutions sont toutes des combinaisons linéaires de dérivées de delta de Dirac centrées en \( x_0\). Pour cela nous invoquons d'abord le lemme~\ref{LemWIGKooQpGXoI} qui nous assure que \( u\) est d'ordre fini et de support \( \{ x_0 \}\). Ensuite la proposition~\ref{PropXXPLooSkgxOz} nous indique que \( u\) doit alors être une combinaisons linéaire de dérivées de Dirac.
\end{proof}

%+++++++++++++++++++++++++++++++++++++++++++++++++++++++++++++++++++++++++++++++++++++++++++++++++++++++++++++++++++++++++++ 
\section{Localisation, principe de recollement}
%+++++++++++++++++++++++++++++++++++++++++++++++++++++++++++++++++++++++++++++++++++++++++++++++++++++++++++++++++++++++++++

\begin{propositionDef}[\cite{BIBooPITOooZANjFn,MonCerveau}]     \label{PROPooNCHIooNOPfBt}
	Soient un ouvert \( \Omega\) de \( \eR^d\), un ouvert \( A\) dans \( \Omega\), et \( T\in\swD'(\Omega)\). Nous définissons
	\begin{equation}
		\begin{aligned}
			T|_A\colon \swD(A) & \to \eC              \\
			\phi               & \mapsto T(\bar \phi)
		\end{aligned}
	\end{equation}
	où  \(\bar \phi\colon \Omega\to \eC\) est définie par
	\begin{equation}
		\bar\phi(x)=\begin{cases}
			\phi(x) & \text{si }  x\in A \\
			0       & \text{sinon. }
		\end{cases}
	\end{equation}
	Alors \( T|_A\in \swD'(A)\).

	Cette distribution \( T|_A\) est la \defe{restriction}{restriction d'une distribution} de \( T\) à \( A\). Elle sera aussi souvent notée \( r_A(T)\).
\end{propositionDef}


\begin{proof}
	Nous allons prouver la séquentielle continuité de \( T|_A\). Ce sera suffisant par le théorème \ref{ThoVDDBnVn}\ref{ITEMooBXFSooYtAXjy}. Soit donc une suite \( \phi_n\stackrel{\swD(A)}{\longrightarrow}0\).

	\begin{subproof}
		\spitem[Débroussailler]
		Nous devons prouver que \( T|_A(\phi_n)\stackrel{\eC}{\longrightarrow}0\), c'est-à-dire que
		\begin{equation}
			T(\bar \phi_n)\stackrel{\eC}{\longrightarrow}0.
		\end{equation}
		\spitem[Notations]
		La topologie sur \( \swD(A)\) est celle de la définition \ref{DEFooVSCRooLyYBzT}. Fixons un peu de notations. Si \( K\) est compact dans \( A\) et si \( m\in\eN\), nous posons
		\begin{equation}        \label{EQooYZWHooVRrnlU}
			\begin{aligned}
				p_{K,m}\colon  C^{\infty}(A) & \to \eR                                                \\
				f                            & \mapsto \sum_{| \mu |\leq m}\| \partial^{\mu}f \|_{K}.
			\end{aligned}
		\end{equation}
		De même, si \( K\) est compact dans \( \Omega\) et si \( m\in \eN\) nous posons
		\begin{equation}        \label{EQooCNPIooSiqAzt}
			\begin{aligned}
				q_{K,m}\colon  C^{\infty}(\Omega) & \to \eR                                                \\
				f                                 & \mapsto \sum_{| \mu |\leq m}\| \partial^{\mu}f \|_{K}.
			\end{aligned}
		\end{equation}
		Les familles de seminormes \( (p_{K,m})\) et \( (q_{K,m})\) ne sont pas indexées par les mêmes ensembles et n'ont pas le même domaine. Bien que les formules \eqref{EQooYZWHooVRrnlU} et \eqref{EQooCNPIooSiqAzt} se ressemblent, elles n'ont rien à voir l'une avec l'autre.

		\spitem[La notion de convergence]

		La proposition \ref{PROPooRAZSooDttIbK} dit que\footnote{Pour rappel, \( \swD\) signifie \( C^{\infty}_c\).} \( f_n\stackrel{\swD(A)}{\longrightarrow}0\) si et seulement si \( p_{K_0,m}(f_n)\to 0\) pour tout \( m\) pour un certain compact \( K_0\) contenant tous les supports de \( f_n\).

		\spitem[\(  \bar\phi_n\stackrel{\swD(\Omega)}{\longrightarrow}0\) ]

		Nous utilisons le théorème \ref{ThoXYADBZr} dans les deux sens. D'abord, vu que \( \phi_n\stackrel{\swD(A)}{\longrightarrow}0\), il existe un compact \( K_0\) tel que \( \phi_n\in\swD(K_0)\) pour tout \( n\) et \( \phi_n\stackrel{\swD(K_0)}{\longrightarrow}0\), c'est-à-dire tel que
		\begin{equation}        \label{EQooRRMUooIznPeE}
			p_{K_0,m}(\phi_n)\to 0
		\end{equation}
		pour tout \( m\in \eN\).

		Le compact \( K_0\) vérifie également \( \bar\phi_n\in \swD(K_0)\) pour tout \( n\), vu que le support de \( \bar\phi_n\) est le même que celui de \( \phi_n\). De plus sur \( K_0\) nous avons \( \phi_n=\bar\phi_n\), donc
		\begin{equation}
			q_{K_0,m}(\bar\phi_n)=\sum_{| \mu |\leq m}\| \partial^{\mu}\bar\phi_n \|_{K_0}=\sum_{| \mu |\leq m}\| \partial^{\mu}\phi_n \|_{K_0}=p_{K_0,m}(\phi_n).
		\end{equation}
		En vertu de \eqref{EQooRRMUooIznPeE} nous avons donc \( q_{K_0,m}(\bar\phi_n)\to 0\), c'est-à-dire que \( \bar\phi_n\stackrel{\swD(K_0)}{\longrightarrow}0\). Donc le théorème \ref{ThoXYADBZr} (dans l'autre sens, cette fois) nous indique que \( \bar\phi_n\stackrel{\swD(\Omega)}{\longrightarrow}0\).


		\spitem[Conclusion]

		Vu que \( T\in \swD'(\Omega)\), nous avons \( T(\bar\phi_n)\stackrel{\eC}{\longrightarrow}0\). Donc
		\begin{equation}
			T|_A(\phi_n)=T(\bar\phi_n)\stackrel{\eC}{\longrightarrow}0.
		\end{equation}
		Cela prouve que \( T|_A\in \swD'(A)\) parce que \( \phi_n\) est une suite quelconque tendant vers \( 0\) dans \( \swD(A)\).
	\end{subproof}
\end{proof}


\begin{lemma}[\cite{MonCerveau}]       \label{LEMooCXIZooAbeMpF}
	Quelques propriétés de la restriction d'une distribution. Nous considérons un ouvert \( \Omega\) dans \( \eR^d\) ainsi qu'un ouvert \( A\subset \Omega\).
	\begin{enumerate}
		\item
		      L'application de restriction \( r_A\colon \swD'(\Omega)\to \swD'(A)\) est linéaire.
		\item       \label{ITEMooGXSKooYomqpg}
		      Si \( \phi\in  C^{\infty}(\eR^d)\) vérifie \( \supp(\phi)\subset A\), alors
		      \begin{equation}
			      T(\phi)=r_A(T)\phi.
		      \end{equation}
	\end{enumerate}
\end{lemma}


\begin{proposition}[Principe de recollement\cite{BIBooPITOooZANjFn, MonCerveau}]		\label{PROPooDVYNooKquEUl}
	Soient un ouvert \( \Omega\) de \( \eR^d\) ainsi que des ouverts \( \{ A_i \}_{i\in I}\) de \( \eR^d\) tels que \( \bigcup_{i\in I}A_i=\Omega\).

	Pour chaque \( i\in I\) nous supposons avoir un élément \( T_i\in\swD'(\Omega_i)\) tel que
	\begin{equation}
		r_{A_i\cap A_j}(T_i)=r_{A_i\cap A_j}(T_j)
	\end{equation}
	pour tout \( i,j\in I\) tels que \( A_i\cap A_j\neq \emptyset\).

	Alors il existe une unique distribution \( T\in \swD'(\Omega)\) telle que \( r_{A_i}(T)=T_i\) pour tout \( i\) dans \( I\).
\end{proposition}

\begin{proof}
	Nous prouvons l'unicité et l'existence séparément\footnote{Ce genre phrase semble ne servir à rien, mais elle sert à éviter qu'en l'environnement \info{description} qui suit ne soit tout moche.}.
	\begin{subproof}
		\spitem[Unicité]
		Soient \( T\) et \( U\) des distributions qui satisfont à la demande. Nous posons \( S=T-U\) et nous allons prouver que \( S=0\). Grâce à la linéarité de \( r_{A_i}\) (lemme \ref{LEMooCXIZooAbeMpF}),
		\begin{equation}
			r_{A_i}(S)=0
		\end{equation}
		pour tout \( i\) dans \( I\).

		Soit \( \phi\in \swD(\Omega)\); nous notons \( K=\supp(\phi)\subset\Omega\). Vu que les \( A_i\) recouvrent \( \Omega\), ils recouvrent \( K\). Il existe dont une partie finie \( J\) de \( I\) telle que \( K\subset\bigcup_{j\in J}A_j\). Et si nous prenions une partition de l'unité\footnote{Théorème \ref{THOooQFCQooSlgLpz}.} subordonnée à ces \( A_j\) ?

		Soit \( \{ \psi_j \}_{j\in J}\) une telle partition de l'unité. Vu que le support de \( \phi\) est borné, le support de \( \psi_k\phi\) est également borné. Et comme un support est toujours fermé, l'application \( \psi_k\phi\) est dans \( \swD(A_k)\). En utilisant consciencieusement le lemme \ref{LEMooCXIZooAbeMpF}\ref{ITEMooGXSKooYomqpg} nous avons
		\begin{equation}
			S(\phi)=S\big( \sum_{k=1}^n\psi_k\phi \big)=\sum_kS(\psi_k\phi)=\sum_{k=1}^n(r_{A_k}S)(\psi_k\phi)=0
		\end{equation}
		parce que \( r_{A_k}(S)=0\).
		\spitem[Existence, début]
		La preuve de l'existence va se faire en plusieurs étapes.

		\spitem[Définition de \( T_K\)]
		Soit un compact \( K\) dans \( \Omega\). Vu que les \( A_j\) forment un recouvrement de \( K\), il existe une partie finie \( J\) dans \( I\) telle que \( K\subset\bigcup_{j\in J}A_j\). Nous considérons une partition de l'unité\footnote{Théorème \ref{THOooQFCQooSlgLpz}.} \( \{ \psi_j \}_{j\in J}\) sur \( K\) subordonnée au recouvrement \( \{ A_j \}_{j\in J}\).

		Nous définissons alors
		\begin{equation}
			\begin{aligned}
				T_K\colon \swD(K) & \to \eC                               \\
				\phi              & \mapsto \sum_{j\in J}T_j(\psi_j\phi).
			\end{aligned}
		\end{equation}
		Nous supposons avoir fait, pour chaque compact \( K\) de \( \Omega\), un choix de sous-recouvrement fini (c'est-à-dire de partie \( J\)) et un choix de partition de l'unité.

		\spitem[\( T_K=T_{K'}  \) sur les intersection]
		Soient deux compacts \( K\), \( M\) dans \( \Omega\). Nous prouvons que si \( \phi\in\swD(K)\cap\swD(M)\), alors \( T_{K}(\phi)=T_{M}(\phi)\). Nous avons donc en main les objets suivants :
		\begin{equation}
			\begin{aligned}[]
				\big( K, J, \{ A_j \}_{j\in J}, \{ \chi_j \}_{j\in J} \big) \\
				\big( M, L, \{ A_l \}_{l\in L}, \{ \varphi_l \}_{l\in L} \big).
			\end{aligned}
		\end{equation}
		Nous considérons \( \phi\in\swD(K)\cap\swD(M)\). Vu que \(\{ \varphi_l \}_{l\in L} \) est une partition de l'unité sur \( M\) et que \( \phi\) a son support contenu dans \( M\), nous avons \( \phi=\sum_{l\in L}\varphi_l\phi\), et nous pouvons écrire
		\begin{equation}        \label{EQooXKYQooFUUovL}
			T_K(\phi)=\sum_{j\in J}T_j(\chi_j\phi)=\sum_{j\in J}\sum_{l\in L}T_j(\chi_j\varphi_l\phi).
		\end{equation}
		Nous savons que \( \supp(\chi_j\varphi_l\phi)\) est dans \( A_j\cap A_l\), donc
		\begin{equation}
			T_j(\chi_j\varphi_l\phi)=r_{A_j\cap A_l}(T_j)(\chi_j\varphi_l\phi).
		\end{equation}
		Mais l'hypothèse sur les \( T_i\) est que \( r_{A_j\cap A_l}(T_j)=r_{A_j\cap A_l}(T_l)\). Cela nous permet de continuer le calcul de \eqref{EQooXKYQooFUUovL} :
		\begin{subequations}
			\begin{align}
				T_K(\phi) & =\sum_{j\in J}\sum_{l\in L}r_{A_j\cap A_l}(T_j)(\chi_j\varphi_l\phi)
				=\sum_{jl}r_{A_j\cap A_l}(T_l)(\chi_j\varphi_l\phi)
				=\sum_{jl}T_l(\chi_j\varphi_l\phi)                                               \\
				          & =\sum_{l\in L}T_l\big( \varphi_l\sum_{j\in J}\chi_j\phi \big)
				=\sum_{l\in L}T_l(\varphi_l\phi)
				=T_M(\phi).
			\end{align}
		\end{subequations}
		Bien.
		\spitem[Ce qu'on pose]
		Si \( \phi\in \swD(\Omega)\), la valeur de \( T_K(\phi)\) ne dépend pas du choix du compact \( K\) dans \( \Omega\) contenant le support de \( \phi\). Nous pouvons donc poser
		\begin{equation}
			\begin{aligned}
				T\colon \swD(\Omega) & \to \eC           \\
				\phi                 & \mapsto T_K(\phi)
			\end{aligned}
		\end{equation}
		où \( K\) est un compact quelconque de \( \Omega\) contenant le support de \( \phi\).
		\spitem[\(  T\) est une distribution]
		Nous devons prouver que \( T\colon\swD(\Omega) \to \eC\) est une distribution, c'est-à-dire qu'elle est linéaire et continue. Pour la linéarité nous disons que c'est bon, et nous nous concentrons sur la continuité. Pour cela nous allons nous baser sur le critère du théorème \ref{ThoVDDBnVn}\ref{ItemSPvoijoii}.

		Soit un compact \( K\) dans \( \Omega\). Nous devons trouver des constantes \( m\) et \( C\) telles que \( |T(\phi)|\leq Cp_{m,K}(\phi)\) pour tout \( \phi\in \swD(K)\).

		Vu que \( \phi\in \swD(K)\) nous avons \( T(\phi)=T_K(\phi)=\sum_{j\in J}T_j(\chi_j\phi)\). Étant donné que \( T_j\) est continue sur \( \swD(A_j)\) et que \( \chi_j\phi\) est une fonction de classe \(  C^{\infty}\) à support dans \( A_j\), en posant \( M_j=\supp(\chi_j)\), il existe des constantes \( m_j\in \eN\) et \( C_j\geq 0\) telles que
		\begin{equation}
			| T_j(\chi_j\phi) |\leq C_jp_{m_j,M_j}(\chi_j\phi).
		\end{equation}
		Donc
		\begin{equation}
			| T(\phi) |\leq \sum_{j\in J}C_j| p_{m_j,M_j}(\chi_j\phi) |=\sum_{j\in J}C_j\sum_{| \alpha |\leq m_j}\| \partial^{\alpha}(\chi_j\phi) \|_{M_j}.
		\end{equation}
		Nous utilisons la formule de Leibnitz\footnote{Lemme \ref{LEMooOLQTooEHJuBc}.} pour décomposer \( \partial^{\alpha}(\chi_j\phi)\) :
		\begin{subequations}
			\begin{align}
				| T(\phi) | & \leq\sum_{j\in J}C_j\sum_{| \alpha |\leq m_j}\| \sum_{\beta\leq \alpha}\binom{ \alpha }{ \beta }\partial^{\alpha-\beta}\chi_j\partial^{\beta}\phi \|_{M_j}        \\
				            & \leq \sum_jC_j\sum_{| \alpha |\leq m_j}\sum_{\beta\leq \alpha}\binom{ \alpha }{ \beta }\| \partial^{\alpha-\beta}\chi_j \|_{M_j}\| \partial^{\beta}\phi \|_{M_j}.
			\end{align}
		\end{subequations}
		Pour chaque \( j\), il y a un nombre fini de multiindices \( \alpha\) en jeu, et pour chacun d'entre eux, un nombre fini de \( \beta\). Donc le coefficient \( \binom{ \alpha }{ \beta }\) se majore par une constante dépendante de \( j\); nous l'incluons dans \( C_j\). De même pour le coefficient \( \| \partial^{\alpha-\beta}\chi_j \|\); il se majore et s'absorbe dans \( C_j\). Nous avons donc
		\begin{equation}
			| T(\phi) |\leq \sum_{j\in J}C'_j\sum_{| \alpha |\leq m_j}\sum_{\beta\leq \alpha}\| \partial^{\beta}\phi \|_{M_j}.
		\end{equation}
		Dans ces sommes, chaque nombre de la forme \( \| \partial^{\mu}\phi \|_{M_j}\) avec \( | \mu |\leq m_j\) est pris un certain nombre de fois. On majore ce «certain nombre\footnote{Il n'est pas nécessaire de savoir exactement combien, mais si vous y tenez vous pouvez vous embarquer dans un petit peu de combinatoire.}» et on l'inclut dans la constante \( C'_j\). Cela donne
		\begin{equation}
			| T(\phi) |\leq \sum_{j\in J}C''_j\sum_{| \alpha |\leq m_j}\| \partial^{\alpha}\phi \|_{M_j}.
		\end{equation}
		L'ensemble \( M=\bigcup_{j\in J}M_j\) est un compact dans \( \Omega\) et bien entendu \( \| \partial^{\alpha}\phi \|_{M_j}\leq \| \partial^{\alpha}\phi \|_M\). Nous avons donc encore plein de majorations :
		\begin{subequations}
			\begin{align}
				| T(\phi) | & \leq \sum_{j\in J}C''_j\sum_{| \alpha |\leq m_j}\| \partial^{\alpha}\phi \|_M                          \\
				            & \leq C\sum_{j\in J}\sum_{| \alpha |\leq m_j}\| \partial^{\alpha}\phi \|_M  \label{SUBEQooOALIooTlhSXJ} \\
				            & \leq C\sum_{j\in J}\sum_{| \alpha |\leq m}\| \partial^{\alpha}\phi \|_M    \label{SUBEQooMNSNooVcTHXP} \\
				            & \leq C'\sum_{| \alpha |\leq m}\| \partial^{\alpha}\phi \|_M \label{SUBEQooKJYOooHNwQAe}                \\
				            & =C'p_{m,M}(\phi).
			\end{align}
		\end{subequations}
		Justifications :
		\begin{itemize}
			\item Pour \ref{SUBEQooOALIooTlhSXJ}. On définit \( C=\max_{j\in J}C''_j\).
			\item Pour \ref{SUBEQooMNSNooVcTHXP}. On définit \( m=\max_{j\in J}m_j\).
			\item Pour \ref{SUBEQooKJYOooHNwQAe}. Comme ce qui est dans la somme ne dépend plus de \( j\), nous supprimons la somme sur \( j\) et nous incluons le nombre de termes de la somme dans le changement \( C\to C'\).
		\end{itemize}
		Et maintenant on admire le fait qu'en ayant fait autant de majorations dans tous les sens, nous obtenons encore une inégalité qui ait un sens. Bref, nous avons trouvé un compact \( M\), un nombre \( m\) ainsi qu'une constante \( C'\) tels que
		\begin{equation}
			| T(\phi) |\leq C'p_{M,m}(\phi)
		\end{equation}
		pour tout \( \phi\in \swD(K)\). Notons que \( M\), \( m\) et \( C'\) dépendent fortement de \( K\), mais pas de \( \phi\). Cela prouve que \( T\) est continue par \ref{ThoVDDBnVn}\ref{ItemSPvoijoii}.
		\spitem[\(  T\) répond à la question]
		Il nous reste à prouver que \( T\) est bien la distribution recollée, c'est-à-dire que \( r_{A_i}(T)=T_i\) pour tout \( i\in I\). Le domaine de \( T_i\) et de \( r_{A_i}(T)\) est \( \swD(A_i)\). Pour être clair, les fonctions dans \( \swD(A_i)\) sont des fonctions définies sur \( \eR^d\), mais dont le support est compact dans \( A_i\).

		Utilisant la définition de la restriction, pour \( \phi\in \swD(A_i)\),
		\begin{equation}
			r_{A_i}(T)\phi=T(\bar \phi)
		\end{equation}
		où
		\begin{equation}
			\bar\phi(x)=\begin{cases}
				\phi(x) & \text{si } x\in A_i \\
				0       & \text{sinon. }
			\end{cases}
		\end{equation}
		Mais vu que \( \phi\in \swD(A_i)\) nous avons \( \phi=\bar\phi\) et donc pas de pièges de ce côté.

		Nous posons \( M=\supp(\phi)\) et nous avons
		\begin{equation}
			r_{A_i}(T)(\phi)=T(\bar\phi)=T(\phi)=T_M(\phi).
		\end{equation}
		Nous nommons \( \big( L, \{ \varphi_l \}_{l\in L} \big)\) les choix faits pour le compact \( M\) dans la construction de \( T_M\). Nous avons alors
		\begin{equation}
			T_M(\phi)=\sum_{l\in L}T_l(\varphi_p\phi).
		\end{equation}
		Mais le support de \( \varphi_k\phi\) est dans \( A_i\cap A_l\), de telle sorte que l'hypothèse faite sur les \( T_i\) aux intersections donne \( T_l(\varphi_l\phi)=T_i(\varphi_l\phi)\), et donc que
		\begin{equation}
			T_M(\phi)=\sum_{l\in L}T_l(\varphi_l\phi)=T_i\big( \sum_{l\in L}\varphi_l\phi \big)=T_i(\phi).
		\end{equation}
		La dernière égalité vient du fait que \( \{ \varphi_l \}_{l\in L}\) est une partition de l'unité sur (au moins) \( \supp(\phi)\).
	\end{subproof}
\end{proof}

%+++++++++++++++++++++++++++++++++++++++++++++++++++++++++++++++++++++++++++++++++++++++++++++++++++++++++++++++++++++++++++ 
\section{Permuter distributions, dérivées et intégrales}
%+++++++++++++++++++++++++++++++++++++++++++++++++++++++++++++++++++++++++++++++++++++++++++++++++++++++++++++++++++++++++++

\begin{proposition}[\cite{BIBooPITOooZANjFn}]     \label{PROPooCNYTooWCKHpV}
	Soient un ouvert \( \Omega\subset \eR^d\), une distribution \( T\in \swD'(\Omega)\) ainsi qu'une fonction \( \phi\in C^{\infty}(\Omega\times \eR^n,\eC)\) à support dans \( K\times \eR^n\) où \( K\) est compact dans \( \Omega\).

	Alors
	\begin{enumerate}
		\item       \label{ITEMooBIVOooHwGglM}
		      La fonction
		      \begin{equation}
			      y\mapsto T\big( \phi(.,y) \big)
		      \end{equation}
		      est de classe \(  C^{\infty}\) sur \( \eR^n\).
		\item
		      Pour toute liste d'indices \( \alpha\) et pour tout \( y_0\in \eR^n\), nous avons
		      \begin{equation}        \label{EQooYMXXooYkceTv}
			      \partial_y^{\alpha}\big( T\big( x\mapsto \phi(x,y) \big) \big)_{y=y_0}= T\big( x\mapsto(\partial_{y}^{\alpha}\phi)(x,y_0) \big).
		      \end{equation}
	\end{enumerate}
\end{proposition}

\begin{proof}
	Nous y allons par récurrence sur les dérivations, en commençant par \( \alpha=(i)\). Pourvu que le membre de gauche de \eqref{EQooYMXXooYkceTv} existe, il est donné par
	\begin{equation}        \label{EQooBHSSooXDrYQh}
		\partial_y^{(i)}\big[ T\big( \phi(x,y) \big) \big]_{y=y_0}=\lim_{t\to 0} \frac{ T\big( \phi(x,y_0+te_i) \big)-T\big( \phi(x,y) \big) }{ t }.
	\end{equation}
	Vu que \( T\) est linéaire, nous allons travailler sur
	\begin{equation}        \label{EQooXEFAooNzLBZx}
		T\Big(x\mapsto \frac{ \phi(x,y_0+te_i)-\phi(x,y_0)}{t} \Big).
	\end{equation}
	\begin{subproof}
		\spitem[Un développement]
		Nous commençons par fixer \( y_0\), \( x\) et \( t\) et écrire un développement de Taylor à l'ordre \( 1\) avec reste intégral\footnote{Donc \( m=2\) dans la proposition \ref{PropAXaSClx}.} pour la fonction \( v(y)=\phi(x,y)\) autour de \( y=y_0\) et dans la direction \( h=te_i\) :
		\begin{equation}        \label{EQooBMYMooAjrTGH}
			\phi(x,y_0+te_i)=\phi(x,y_0)+t\partial_y^{(i)}\phi(x,y_0)+r(x,y_0,te_i)
		\end{equation}
		où
		\begin{equation}        \label{EQooOWQZooMZXVoV}
			r(x,y_0,te_i)=\int_0^1(1-u)(d^2v)_{y_0+ute_i}(te_i)^2du.
		\end{equation}
		\spitem[Écrire proprement le reste]
		Nous allons maintenant dérouler les notations compliquées de différentielle seconde de l'expression \eqref{EQooOWQZooMZXVoV}. C'est la proposition \ref{PROPooFWZYooUQwzjW} qui nous dit comment faire. Le lemme \ref{LEMooVOTHooPJcrWH} est également à utiliser. D'abord
		\begin{equation}
			(d^2v)_{y_0+ute_i}(te_i)^2=\frac{ \partial^2v }{ \partial(te_i)\partial(te_i) }(y_0+ute_i)=t^2(\partial^2_{ii}v)(y_0+ute_i).
		\end{equation}
		Avec cela,
		\begin{equation}        \label{EQooLIBWooYJshbY}
			r(x,y_0,te_i)=t^2\int_0^1(1-u)(\partial_y^{(ii)}\phi)(x,y_0+ute_i)du.
		\end{equation}
		Pour rappel, la notation \( \partial_y^{(ii)}\phi\) signifie dériver deux fois selon la \( i\)\ieme composante de la variable \( y\) de \( \phi\).

		\spitem[Régularité du reste]
		Si vous avez peur d'étudier la régularité en \( x\) du reste à partir de l'intégrale \eqref{EQooLIBWooYJshbY}, vous pouvez simplement regarder le développement \eqref{EQooBMYMooAjrTGH} et dire que, \( t\) et \( y_0\) sont fixés. Les fonctions
		\begin{subequations}
			\begin{align}
				x & \mapsto \phi(x, y_0+te_i)           \\
				x & \mapsto \phi(x, y_0)                \\
				x & \mapsto (\partial^{(i)}\phi)(x,y_0)
			\end{align}
		\end{subequations}
		sont de classe \(  C^{\infty}\) à support compact dans \( K\). Donc \( x\mapsto r(x,y_0,te_i)\) est également \(  C^{\infty}\) à support dans \( K\).

		Nous allons d'ailleurs tellement fixer \( y_0\) et \( t\) que nous allons définir
		\begin{equation}
			\begin{aligned}
				s\colon \eR^d & \to \eC                                                         \\
				x             & \mapsto  t^2\int_0^1(1-u)(\partial_y^{(ii)}\phi)(x,y_0+ute_i)du
			\end{aligned}
		\end{equation}
		et voir \( s\in\swD(K)\). Nous pouvons donc parler de \( T(s)\) sans peurs.

		\spitem[Beaucoup de majorations]
		Vu que \( K\) est un compact dans \( \Omega\) et que \( T\) est dans \( \swD'(\Omega)\), le théorème \ref{ThoVDDBnVn}\ref{ItemSPvoijoii} dit qu'il existe des constantes \( m\) et \( C\) telles que \( | T(s) |\leq Cp_{m,K}(s)\). Nous avons alors les majorations suivantes :
		\begin{subequations}        \label{SUBEQSooYPLGooOFMSyJ}
			\begin{align}
				| T(s) | & \leq C p_{m,K}(s)                                                                                                                                     \\
				         & =C\sum_{| \mu |\leq m}\| \partial^{\mu}s \|_K                                                                                                         \\
				         & =\sum_{|\mu|\leq m}t^2\int_0^1(1-u)(\partial_x^{\mu}\partial_y^{ii}\phi)(x,y_0+tue_i)du       \label{SUBEQooKIETooZIpBBF}                             \\
				         & \leq\sum_{|\mu|\leq m}t^2\int_0^1(1-u)\| (\partial_x^{\mu}\partial_y^{ii}\phi)\|_{K\times \overline{ B(y_0,1) }} du       \label{SUBEQooDYJCooFAsAxE} \\
				         & =\frac{ Ct^2 }{2}\sum_{| \mu |\leq m}\| \partial_x^{\mu}\partial_y^{ii}\phi \|_{K\times \overline{ B(y_0,1) }}  \label{SUBEQooULTUooYlJHNO}           \\
				         & \leq C't^2 \max_{| \mu |\leq m}\| \partial_x^{\mu}\partial_y^{ii}\phi \|_{K\times \overline{ B(y_0,1) }}  \label{SUBEQooQENJooOgJuxj}
			\end{align}
		\end{subequations}
		Justifications:
		\begin{itemize}
			\item Pour \eqref{SUBEQooKIETooZIpBBF}. Pour évaluer \( \partial^{\mu}s\), nous utilisons la proposition \ref{PROPooJKXJooLxgEGd} pour permuter l'intégrale sur \( u\) avec la dérivée sur \( x\).
			\item Pour \eqref{SUBEQooDYJCooFAsAxE}. Nous supposons que \( | t |<1\), de telle sorte que \( | ut |<1\) et que \( y_0+tue_i\) reste dans \( \overline{ B(y_0,1) }\). Notez au passage qu'il n'est pas nécessaire de prendre la fermeture; c'est juste pour le plaisir de rester sur un compact.
			\item Pour \eqref{SUBEQooULTUooYlJHNO}. Évaluer l'intégrale : \( \int_0^1(1-u)du=1/2\).
			\item Pour \eqref{SUBEQooQENJooOgJuxj}. Le changement de constante \( C\to C'\) intègre le nombre de termes dans la somme qui a été remplacée par son maximum et le \( 1/2\).
		\end{itemize}
		\spitem[La dérivée (enfin)]
		Nous reprenons le calcul de la dérivée laissé en \eqref{EQooBHSSooXDrYQh} et \eqref{EQooXEFAooNzLBZx}. Nous avons
		\begin{subequations}
			\begin{align}
				T\left( x\mapsto\frac{ \phi(x,y_0+te_i)-\phi(x,y_0) }{ t } \right) & =T\left( \frac{ t(\partial_y^{(i)}\phi)(x,y_0)+s(x) }{ t }\right)                        \\
				                                                                   & =T\Big( x\mapsto (\partial_y^{(i)}\phi)(x,y_0) \Big)+T\left( \frac{ s(x) }{ t } \right).
			\end{align}
		\end{subequations}
		Le premier terme n'a plus de dépendance en \( t\). La limite \( t\to 0\) est donc sans problèmes. Pour le second terme, en prenant la majoration de \eqref{SUBEQSooYPLGooOFMSyJ} nous avons
		\begin{equation}
			| \frac{ s(x) }{ t } |\leq C't\max_{| \mu |\leq m}\| \partial_x^{\mu}\partial_y^{(ii)}\phi \|_{K\times \overline{ B(y_0,1) }}.
		\end{equation}
		La limite \( t\to 0\) de cela est zéro.
		\spitem[Conclusion]
		La limite du membre de droite de \eqref{EQooBHSSooXDrYQh} existe et vaut
		\begin{equation}
			T\Big[x\mapsto (\partial_y^{(i)}\phi)(x,y_0) \Big].
		\end{equation}
	\end{subproof}
	Nous avons prouvé que
	\begin{equation}
		\partial^{(i)}_y\Big[ T\big( x\mapsto\phi(x,y) \big) \Big]_{y=y_0}=T\Big[ x\mapsto (\partial^{(i)}_y\phi)(x,y_0) \Big].
	\end{equation}
	Une récurrence permet de passer de la liste d'indices \( (i)\) à une liste générale pour terminer la preuve du résultat \eqref{EQooYMXXooYkceTv}.
\end{proof}

\begin{theorem}[Permuter distribution et intégrale\cite{BIBooPITOooZANjFn}]
	Soit un ouvert \( \Omega\) dans \( \eR^n\). Soient \( T\in \swD'(\Omega)\) et \( \phi\in \swD(\Omega\times \eR^n)\). Alors
	\begin{equation}
		\int_{\eR^n}T\big( \phi(.,y) \big)dy=T\big( \int_{\eR^n}\phi(.,y)dy \big).
	\end{equation}
\end{theorem}

\begin{proof}
	Nous faisons une récurrence sur la valeur de \( n\) en partant de \ldots
	\begin{subproof}
		\spitem[\( n=1\)]
		Vu que le support de \( \phi\) est compact dans \( \Omega\times \eR\), la proposition \ref{PropGBZUooRKaOxy} dit qu'il existe un compact \( K\) dans \( \Omega\) et \( R>0\) tel que le support de \( \phi\) soit dans \( K\times \overline{ B(0,R) }\).

		Nous introduisons plein de fonctions intermédiaires.
		\begin{subproof}
			\spitem[La fonction \( \theta\)]


			Considérons une seconde la fonction
			\begin{equation}
				\begin{aligned}
					\theta\colon \Omega & \to \eC                        \\
					x                   & \mapsto \int_{\eR}\phi(x,z)dz.
				\end{aligned}
			\end{equation}
			Son support est dans \( K\). Est-elle de classe \(  C^{\infty}\) ? Oui. En effet, la proposition \ref{PROPooJKXJooLxgEGd} dit que l'on peut permuter les dérivées partielles et l'intégrale. Donc toutes les dérivées partielles de \( \theta\) existent (et sont donc continues). Le théorème \ref{THOooPZTAooTASBhZ} nous indique alors que \( \theta\) est de classe \( C^{\infty}\).

			\spitem[La fonction \( \xi\)]
			Le corolaire \ref{CORooHHZXooXmwGmC} nous permet de considérer \( \xi\in \swD\big( B(0,R) \big)\) telle que
			\begin{equation}
				\int_{\eR}\xi(x)dx=1.
			\end{equation}
			\spitem[La fonction \( \psi\)]
			Nous posons alors
			\begin{equation}
				\begin{aligned}
					\psi\colon \eR^d\times \eR & \to \eC                                        \\
					(x,y)                      & \mapsto \phi(x,y)-\xi(y)\int_{\eR}\phi(x,z)dz.
				\end{aligned}
			\end{equation}
			C'est une fonction \( \psi\in \swD(\Omega\times \eR)\) telle que
			\begin{equation}
				\supp(\psi)\subset K\times \overline{ B(0,R) }.
			\end{equation}

			Prouvons que pour tout \( x\in \Omega\) nous avons
			\begin{equation}
				\int_{\eR}\psi(x,y)dy=\int_{-R}^R\psi(x,y)dy=0.
			\end{equation}
			La première égalité est simplement le fait que \( \psi(x,y)=0\) lorsque \( y\notin \overline{ B(0,R) }\). Pour le fait que ce soit nul,
			\begin{equation}
				\int_{\eR}\psi(x,y)dy=\int_{\eR}\phi(x,y)dy-\int_{\eR}\xi(y)dy\int_{\eR}\phi(x,z)dz=0.
			\end{equation}

			\spitem[La fonction \( \sigma\)]
			Nous posons enfin
			\begin{equation}
				\begin{aligned}
					\sigma\colon \Omega\times \eR & \to \eC                              \\
					(x,y)                         & \mapsto \int_{-\infty}^y\psi(x,z)dz.
				\end{aligned}
			\end{equation}
			Nous avons \( \sigma\in C^{\infty}(\Omega\times \eR)\) et \( \supp(\sigma)\subset K\times \overline{ B(0,R) }\).
		\end{subproof}
		Nous pouvons maintenant entrer dans le dur.

		\begin{subproof}
			\spitem[Une première égalité]
			Nous montrons que
			\begin{equation}     \label{EQooDRISooWsWXnj}
				T\big( \sigma(.,y) \big)=\int_{-\infty}^yT\big( \psi(.,z) \big)dz.
			\end{equation}
			En voyant les deux côtés de cette égalité à prouver comme des fonctions de \( y\), nous allons montrer qu'elles sont dérivables, de même dérivée et égale en \( y<-R\).

			À gauche nous permutons \( T\) avec la dérivée par la proposition \ref{PROPooCNYTooWCKHpV}:
			\begin{equation}
				\partial_y\big( T\big( \sigma(.,y) \big) \big)_{y=y_0}=T\big( (\partial_y\sigma)(.,y_0) \big)=T\big( \psi(.,y_0) \big)
			\end{equation}
			parce que \( (\partial_y\sigma)(x,y_0)=\psi(x,y_0)\).

			À droite, la proposition \ref{PropJLnPpaw} donne la première dérivée:
			\begin{equation}
				\partial_y\big( \int_{-\infty}^yT\big( \psi(.,z) \big)dz \big)_{y=y_0}=T\big( \psi(.,y_0) \big).
			\end{equation}
			Notez que la proposition \ref{PROPooCNYTooWCKHpV}\ref{ITEMooBIVOooHwGglM} dit alors que les deux membres de \eqref{EQooDRISooWsWXnj} sont de classe \(  C^{\infty}\).

			Pour \( y<-R\), d'une part \( \sigma(.,y)=0\) et d'autre part, \( \int_{-\infty}^yT\big( \psi(.,z) \big)dz=0\) parce que l'intégrale est entièrement sur un domaine où \( \psi(.,y)=0\).

			Donc l'égalité \eqref{EQooDRISooWsWXnj} est établie.

			\spitem[Une seconde égalité]
			Nous prouvons maintenant que
			\begin{equation}     \label{EQooTKHVooEqhpHH}
				\int_{\eR}T\big( \psi(.,z) \big)dz=0.
			\end{equation}
			Nous nous souvenons que \( \psi(.,r)=0\) si \( r>R\). Donc
			\begin{subequations}
				\begin{align}
					\int_{\eR}T\big( \psi(.,z) \big)dz & =\int_{-\infty}^{\infty}T\big( \psi(.,z) \big)dz \\
					                                   & =\int_{-\infty}^RT\big( \psi(.,z) \big)dz        \\
					                                   & =T\big( \sigma(.,R) \big).
				\end{align}
			\end{subequations}
			La dernière égalité est \eqref{EQooDRISooWsWXnj}.

			\spitem[On ressort la définition de \( \psi\)]
			C'est le moment de nous souvenir la définition de \( \psi\) et de la substituer dans \eqref{EQooTKHVooEqhpHH} en tenant compte de la linéarité de \( T\). Nous avons :
			\begin{subequations}
				\begin{align}
					0 & =\int_{\eR}T\big( \psi(.,z) \big)dz                                                                                    \\
					  & =\int_{\eR}T\big( \phi(.,z) \big)dz-\int_{\eR}T\Big( \xi(z)\int_{\eR}\phi(.,s)ds \Big)dz                               \\
					  & =\int_{\eR}T\big( \phi(.,z) \big)dz-\int_{\eR}\xi(z)dz\,T\Big( \int_{\eR}\phi(.,s)ds \Big) \label{SUBEQooXSQMooEuUZUe} \\
					  & =\int_{\eR}T\big( \phi(.,z) \big)dz-T\Big( \int_{\eR}\phi(.,s)ds \Big).
				\end{align}
			\end{subequations}
			Justifications :
			\begin{itemize}
				\item Pour \eqref{SUBEQooXSQMooEuUZUe}. Le nombre \( \xi(z)\) sort de \( T\). Et \( T\big( \int_{\eR}\phi(.,s)ds \big)\) sort de l'intégrale en \( z\).
			\end{itemize}
			Le cas \( n=1\) est prouvé.
		\end{subproof}
		\spitem[La récurrence]
		Nous supposons que le théorème est prouvé jusqu'à \( n\) et nous le prouvons pour \( n+1\).

		Soit donc \( \phi\in\swD(\Omega\times \eR^{n+1})\). La fonction \( y\mapsto T\big( \phi(.,y) \big)\) est continue et à support compact. Donc Fubini \ref{ThoWTMSthY} s'applique et nous allons permuter les intégrales une à une avec \( T\). Avant cela nous définissons quelques fonctions intermédiaires.

		D'abord, pour \( t\in \eR\) nous posons
		\begin{equation}
			\begin{aligned}
				\tilde\phi_t\colon \Omega\times \eR^n & \to \eR                               \\
				(\omega,y)                            & \mapsto \phi\big( \omega,(y,t) \big).
			\end{aligned}
		\end{equation}
		Ensuite, nous posons
		\begin{equation}
			\begin{aligned}
				\bar \phi\colon \Omega\times \eR & \to \eR                                                                    \\
				(\omega,t)                       & \mapsto \int_{\eR^n}\tilde \phi_t(\omega, y_1,\ldots, y_n)dy_1\ldots dy_n.
			\end{aligned}
		\end{equation}
		Maintenant nous calculons :
		\begin{subequations}
			\begin{align}
				\int_{\eR^{n+1}}T\big( \phi(.,y) \big)dy & =\int_{\eR}\left[ \int_{\eR^n} T\big( \phi(.,y) \big)dy_1\ldots dy_n \right]dy_{n+1}       \label{SUBEQooBFEFooFVtHmz}                         \\
				                                         & =\int_{\eR}\left[ \int_{\eR^n}T\big( \tilde \phi_{y_{n+1}}(., y_1,\ldots, y_n) \big)dy_1\ldots dy_n \right]dy_{n+1}                            \\
				                                         & =\int_{\eR}\left[ T\big( \int_{\eR^n}\tilde \phi_{y_{n+1}}(.,y_1,\ldots, y_n) \big)dy_1\ldots dy_n \right]dy_{n+1} \label{SUBEQooROIEooVMbfii} \\
				                                         & =\int_{\eR}T\big( \bar\phi(.,y_{n+1}) \big)dy_{n+1}                                                                                            \\
				                                         & =T\big( \int_{\eR}\bar\phi(.,y_{n+1})dy_{n+1} \big)  \label{SUBEQooPFBOooMXUjXk}                                                               \\
				                                         & =T\big( \int_{\eR^{n+1}}\phi(.,y)dy \big)      \label{SUBEQooKYOPooPYHHbS}
			\end{align}
		\end{subequations}
		Justifications :
		\begin{itemize}
			\item Pour \eqref{SUBEQooBFEFooFVtHmz}. Théorème de Fubini \ref{ThoWTMSthY}.
			\item Pour \eqref{SUBEQooROIEooVMbfii}. Hypothèse de récurrence.
			\item Pour \eqref{SUBEQooPFBOooMXUjXk}. Encore la récurrence, avec \( n=1\).
			\item Pour \eqref{SUBEQooKYOPooPYHHbS}. Encore Fubini, mais dans l'autre sens.
		\end{itemize}

	\end{subproof}
\end{proof}

%+++++++++++++++++++++++++++++++++++++++++++++++++++++++++++++++++++++++++++++++++++++++++++++++++++++++++++++++++++++++++++
\section{Distributions tempérées}
%+++++++++++++++++++++++++++++++++++++++++++++++++++++++++++++++++++++++++++++++++++++++++++++++++++++++++++++++++++++++++++

L'espace de Schwartz\index{espace!de Schwartz}\footnote{Attention : ce Schwartz (avec un \emph{t}) est le Schwartz des distributions dont le prénom est Laurent. À ne pas confondre avec Schwarz (sans \emph{t}) dont le prénom est Cauchy.} \( \swS(\Omega)\) est défini dans la définition~\ref{DefHHyQooK}; sa topologie y est discutée.
\begin{definition}
	Une \defe{distribution tempérée}{distribution!tempérée} est une forme linéaire continue sur \( \swS(\eR^d)\). L'ensemble des distributions tempérées est noté \( \swS'(\eR^d)\)\nomenclature[Y]{\( \swS'(\eR^d)\)}{espace des distributions tempérées}. Si \( T\) est une telle distribution, nous notons \( \langle T, \varphi\rangle\) l'image de \( \varphi\) par \( T\).
\end{definition}

\begin{normaltext}
	Deux rappels.
	\begin{enumerate}
		\item
		      La topologique sur \( \swS(\eR^d)\) est celle de la définition~\ref{LEMDEFooZEFVooMMmiBr}.
		\item
		      La topologie sur \( \swD(\eR^d)\) est donnée par la définition \ref{DefFGGCooTYgmYf}.
	\end{enumerate}
\end{normaltext}




\begin{lemmaDef}      \label{DEFooUSTNooYEZfPN}
	L'application
	\begin{equation}
		\begin{aligned}
			\delta\colon \Fun(\eR^d) & \to \eC             \\
			\varphi                  & \mapsto \varphi(0).
		\end{aligned}
	\end{equation}
	est
	\begin{enumerate}
		\item
		      une distribution,
		\item
		      une distribution tempérée.
	\end{enumerate}
	Cette distribution est nommée \defe{distribution de Dirac}{distribution!de Dirac}.
\end{lemmaDef}

\begin{proof}
	Juste pour rappel, \( \Fun(X)\) est l'ensemble de toutes les fonctions sur \( X\). Pour prouver que \( \delta\) est une distribution, nous devons démontrer que \( \delta\colon \swD(\eR)\to \eC\) est continue. Et pour qu'elle soit une distribution tempérée, il faut démontrer que \( \delta\colon \swS(\eR)\to \eC\) est continue.

	Nous utilisons le théorème~\ref{ThoVDDBnVn}\ref{ItemSPvoijoii} pour démontrer la continuité de \( \delta\) sur \( \swD(\eR)\). Soit un compact \( K\subset \eR\) et \( \varphi\in \swD(K)\). En prenant \( m=0\) nous devons avons la majoration
	\begin{equation}
		| \delta(\varphi)|=\varphi(0) \leq \| \varphi \|_{K,\infty}=p_{0,K}(\varphi).
	\end{equation}

	Pour la continuité de \( \delta\) sur \( \swS(\eR^d)\), nous utilisons les résultats de~\ref{NORMooVQESooRwJShl}. Soit une suite \( \varphi_n\stackrel{\swS}{\to}0\). En particulier, \( p_{0,0}(\varphi_n)=\sup_x| \varphi_n(x) |\to 0\). Donc \( \varphi_n(0)\to 0\) comme il le faut.
\end{proof}

\begin{example}
	La \defe{valeur principale}{valeur!principale (distribution)} de la fonction \( x\mapsto \frac{1}{ x }\) est la distribution
	\begin{equation}
		\begin{aligned}
			T\colon \swS(\eR) & \to \eR                               \\
			\varphi           & \mapsto \lim_{\substack{\epsilon\to 0 \\\epsilon>0}}\int_{| x |>\epsilon}\frac{ \varphi(x) }{ x }.
		\end{aligned}
	\end{equation}
	Montrons que cela définit bien une distribution tempérée.

	D'abord l'intégrale existe pour tout \( \epsilon\), par exemple parce que pour les grands \( | x |\) nous avons par exemple \( | \varphi(x)|\leq \frac{1}{ x}\) et donc \( \varphi(x)/x\leq 1/x^2\) dont l'intégrale converge. Nous devons maintenant regarder la limite.

	Nous considérons une suite \( \epsilon_n\to 0\) et la suite
	\begin{equation}
		a_n=\int_{| x |\geq \epsilon_n}\frac{ \varphi(x) }{ x }dx.
	\end{equation}
	Nous montrons que cette suite converge dans \( \eR\) en montrant qu'elle est de Cauchy. Pour cela nous travaillons un peu la forme de \( \varphi\) :
	\begin{equation}
		\varphi(x)=\varphi(0)+\int_0^x\varphi'(t)dt=\varphi(0)+\int_0^1x\varphi'(x\theta)d\theta.
	\end{equation}
	Ce qui est dans l'intégrale est borné par \( K=\| M_x\varphi' \|_{\infty}\) qui est parfaitement fini parce que \( \varphi\) est à décroissance rapide. Lorsque nous calculons \( | a_m-a_n |\), le terme \( \varphi(0)/x\) donne une intégrale nulle parce que le domaine d'intégration \( \epsilon_m\leq | x |\leq \epsilon_n\) est symétrique alors que la fonction \( 1/x\) est impaire.
	\begin{equation}
		| a_m-a_n |\leq \big| \int_{\epsilon_m<| x |<\epsilon_n}K \big|=2| \epsilon_n-\epsilon_m |K
	\end{equation}
	Tout cela nous dit que \( T\) est bien définie. Nous devons encore étudier sa continuité.

	Soit \( \chi\) une fonction dans \(  C^{\infty}_c(\eR)\) valant \( 1\) sur \( \mathopen[ -1 , 1 \mathclose]\), paire et à valeurs dans \( \mathopen[ 0 , 1 \mathclose]\).
	%TODOooYKQPooAxYRlu : il faudrait montrer qu'il existe des fonctions C infini à support compact qui ne sont pas nulles partout. C'est fait autour du lemme de Borel.
	Pour tout \( \epsilon>0\) nous avons \( \int_{| x |>\epsilon}\frac{ \chi(x) }{ x }dx=0\).

	Nous avons aussi \( \varphi=\chi\varphi+(1-\chi)\varphi\), et donc
	\begin{subequations}
		\begin{align}
			\int_{| x |>\epsilon}\frac{ \varphi(x) }{ x }dx & =\int_{| \epsilon |>0}\chi(x)\frac{ \varphi(x)-\varphi(0) }{ x }dx+\int_{| \epsilon |>0}\big( 1-\chi(x) \big)\frac{ \varphi(x) }{ x }dx                                      \\
			                                                & =\int_{| \epsilon |>0}\chi(x)\int_0^1\underbrace{\varphi'(\theta x)}_{\leq \| \varphi' \|_{\infty}}d\theta+\int_{| x |\geq 1}\big( 1-\chi(x) \big)\frac{ \varphi(x) }{ x }dx \\
			                                                & \leq\| \varphi' \|_{\infty}\int_{| x |\geq \epsilon}\chi(x)dx+\| \varphi \|_{L^1}                                                                                            \\
			                                                & =C\| \varphi' \|_{\infty}+\| \varphi \|_{1}.
		\end{align}
	\end{subequations}
	Cela est valable pour toute fonction \( \varphi\in\swS(\eR)\). Mais nous savons que si \( \varphi_n\stackrel{\swS(\eR)}{\to}0\), alors \( \| \varphi_n \|_{\infty}\to 0\), \( \| \varphi'_n \|_{\infty}\to 0\) et \( \| \varphi_n \|_1\to 0\); donc si \( \varphi_n\stackrel{\swS(\eR)}{\to}0\), alors
	\begin{equation}
		T(\varphi_n)=\lim_{\substack{\epsilon\to 0\\\epsilon>0}}\int_{| x |>\epsilon}\frac{ \varphi(x) }{ x }\leq C\| \varphi_n' \|_{\infty}+\| \varphi_n \|_1\to 0.
	\end{equation}
\end{example}

%---------------------------------------------------------------------------------------------------------------------------
\subsection{Topologie}
%---------------------------------------------------------------------------------------------------------------------------

La topologie que nous mettons sur l'espace \( \swS'(\eR^d)\) est le même type que celle que nous mettons sur \( \swD'(\eR^d)\), c'est-à-dire celle des seminormes \( p_{\varphi}(T)=| T(\varphi) |\). La définition~\ref{DefASmjVaT} et la proposition~\ref{PropEUIsNhD} restent.

\begin{proposition} \label{PropQAuJstI}
	Nous avons \( T_n\stackrel{\swS'(\eR^d)}{\longrightarrow}T\) si et seulement si pour tout \( \varphi\in\swS(\eR^d)\) nous avons \( T_n(\varphi)\to T(\varphi)\).
\end{proposition}

%---------------------------------------------------------------------------------------------------------------------------
\subsection{Distributions associées à des fonctions}
%---------------------------------------------------------------------------------------------------------------------------

Si \( f\in L^1_{loc}(\eR^d)\) alors nous lui associons une distribution \( T_f\in \swD'(\eR^d)\) définie par la formule
\begin{equation}
	T_f(\varphi)=\int_{\eR^d}f(x)\varphi(x)dx.
\end{equation}

\begin{proposition}
	L'application ainsi définie
	\begin{equation}
		\begin{aligned}
			L^1_{loc}(\eR^d) & \to \swD'(\eR^d) \\
			f                & \mapsto T_f
		\end{aligned}
	\end{equation}
	est injective.
\end{proposition}

\begin{proof}
	Si \( T_f=0\) alors pour tout \( \varphi\in \swD\) nous avons \( \int_{\eR^d}f\varphi=0\). En vertu de la proposition~\ref{PropLGoLtcS} cela implique \( f=0\) presque partout.
\end{proof}

%---------------------------------------------------------------------------------------------------------------------------
\subsection{Composition avec une fonction}
%---------------------------------------------------------------------------------------------------------------------------

\begin{proposition}[\cite{GQYneyj}, page 113 et 32] \label{PropBQUOcyw}
	Soit \( T\in\swS'(\Omega)\) et \( f\in C^k(A\times \Omega)\) où \( A\) est ouvert dans \( \eR^d\). Nous posons
	\begin{equation}
		\begin{aligned}
			F\colon A & \to \eR                            \\
			\lambda   & \mapsto T\big( f(\lambda,.) \big).
		\end{aligned}
	\end{equation}
	Alors \( F\in C^k(A)\).
\end{proposition}
%TODO : une preuve.

%---------------------------------------------------------------------------------------------------------------------------
\subsection{Transformée de Fourier d'une distribution tempérée}
%---------------------------------------------------------------------------------------------------------------------------

\begin{definition}
	La \defe{transformée de Fourier}{transformée!Fourier!distribution tempérée} de la distribution tempérée \( T\in\swS'(\eR^d)\) est la distribution \( \hat T\) définie par
	\begin{equation}
		\hat T(\varphi)=T(\hat \varphi)
	\end{equation}
	pour tout \( \varphi\in\swS(\eR^d)\).
\end{definition}

\begin{lemma}
	Si \( f\in \swS(\eR^d)\), nous avons \( \hat T_f=T_{\hat f}\).
\end{lemma}

\begin{proof}
	En utilisant les définitions,
	\begin{equation}
		\hat T_f(\varphi)=T_f(\hat \varphi)=\int_{\eR^d}f(x)\hat \varphi(x)dx=\int_{\eR^d}f(x)\left[ \int_{\eR^d}\varphi(y) e^{-iyx}dy \right]dx
	\end{equation}
	où nous avons noté \( xy\) le produit scalaire \( x\cdot y\). Nous permutons les intégrales en utilisant le théorème de Fubini~\ref{ThoFubinioYLtPI} avec la fonction
	\begin{equation}
		(x,y)\mapsto f(x)\varphi(y) e^{-ixy}
	\end{equation}
	qui est parfaitement dans \( L^1(\eR^d\times \eR^d)\). Nous écrivons alors
	\begin{equation}
		\hat T_f(\varphi)=\int_{\eR^d}\left[ \int_{\eR^d}f(x)\varphi(y) e^{-iyx}dx \right]dy=\int_{\eR^d}\varphi(y)\hat f(y)dy=T_{\hat f}(\varphi).
	\end{equation}
\end{proof}

%---------------------------------------------------------------------------------------------------------------------------
\subsection{Convolution d'une distribution par une fonction}
%---------------------------------------------------------------------------------------------------------------------------

Nous savons que si \( \psi\in\swS(\eR^d)\) et si \( x\in\eR^d\) alors la fonction \( y\mapsto\psi(x-y)\) est encore une fonction dans \( \swS(\eR^d)\). Donc si \( T\in\swS'(\eR^d)\) nous pouvons considérer la fonction \( T*\psi=\psi*T\) définie par
\begin{equation}        \label{EQooOUXKooGHDSzL}
	(T*\psi)(x)=T\big( y\mapsto\psi(x-y) \big).
\end{equation}
Notons que \( T*\psi\) est bien une fonction et non une distribution.

Le but de la définition est d'avoir
\begin{equation}
	T_f*\psi=f*\psi.
\end{equation}
En effet
\begin{equation}
	(T_f*\psi)(x)=T_f\big( y\mapsto \psi(x-y) \big)=\int_{\eR^d}f(y)\psi(x-y)dy=(f*\psi)(x).
\end{equation}

\begin{example}
	La distribution de Dirac est le neutre pour le produit de convolution. En effet
	\begin{equation}
		(\delta*\psi)(x)=\delta\big( y\mapsto\psi(x-y) \big)=\psi(x),
	\end{equation}
	c'est-à-dire \( \delta*\psi=\psi\).
\end{example}

\begin{proposition}[\cite{OEVAuEz}] \label{PropZMKYMKI}
	Si \( T\in\swS'(\eR^d)\) et \( \psi\in\swS(\eR^d)\), alors la distribution associée à la fonction \( T*\psi\) est tempérée.
\end{proposition}

\begin{proof}
	En agissant sur \( \varphi\in\swD(\eR^d)\) nous avons
	\begin{subequations}
		\begin{align}
			T_{T*\psi}(\varphi) & =\int_{\eR^d}T\big( y\mapsto (t_x\psi)(y) \big)\varphi(x)dx                         \\
			                    & =\int_{\eR^d}T\big( y\mapsto \varphi(x)\psi(x-y) \big)dx                            \\
			                    & =T\left( y\mapsto\int_{\eR^d}\varphi(x)\psi(x-y)dx \right)     \label{SubEqSVDsVTS} \\
			                    & =T\big(y\mapsto (\varphi*\check\psi)(y)\big)                                        \\
			                    & =T(\varphi*\check\psi).
		\end{align}
	\end{subequations}

	Attention :
	\begin{probleme}
		Le passage à la ligne \eqref{SubEqSVDsVTS} n'est pas justifié.
		%TODOooPODMooXwToAR le faire
	\end{probleme}
\end{proof}

%---------------------------------------------------------------------------------------------------------------------------
\subsection{Approximation de la distribution de Dirac}
%---------------------------------------------------------------------------------------------------------------------------


\begin{lemma}[\cite{MonCerveau}]        \label{LEMooHEEOooFtKgfz}
	Soient des fonctions \( j_n\colon \eR\to \eR^+\) de classe \(  C^{\infty}\) telles que
	\begin{enumerate}
		\item
		      Pour chaque \( n\), la fonction \( x\mapsto j_n\big( | x | \big)\) est strictement décroissante et converge ponctuellement vers zéro.
		\item
		      Pour chaque \( x\), la suite \( n\mapsto j_n(x)\) est décroissante et converge vers \( 0\).
		\item       \label{ITEMooFQYXooEkUAIb}
		      Pour tout \( M>0\), la suite \( j_n\) converge vers zéro uniformément sur \( B(0,M)^c\).
		\item       \label{ITEMooFYCRooFeRRjE}
		      Pour tout \( \delta\) et \( \epsilon\), il existe un \( N\in \eN\) tel que \( | \int_{B(0,\delta)}j_n(x)dx-1 |\leq\epsilon\).
		\item
		      Pour tout \( n\), nous avons \( \int_{\eR}j_n=1\).
	\end{enumerate}
	Alors si \( u\in\swS(\eR)\) nous avons
	\begin{equation}
		\lim_{n\to \infty} \int_{\eR}u(x)j_n(x)dx=u(0).
	\end{equation}
\end{lemma}

\begin{proof}
	Nous posons
	\begin{subequations}
		\begin{align}
			I_n          & =\int_{\eR}j_nu            \\
			I_{\delta,n} & =\int_{B(0,\delta)}j_nu    \\
			Z_{\delta,n} & =\int_{B(0,\delta)}u(0)j_n
		\end{align}
	\end{subequations}
	Nous allons progressivement montrer qu'en prenant \( \delta\) assez petit et \( n\) assez grand, les quantités \( | I_n-I_{\delta,n} |\), \( | I_{\delta,n}-Z_{\delta,n}  |\) et \( | Z_{\delta,n}-u(0) |\) peuvent être simultanément majorées par \( \epsilon\).

	Soient \( \delta>0\) et \( \epsilon>0\); vu que \( u\in\swS(\eR)\), il existe \( M\) tel que \( \int_{| x |>M}| u |<\epsilon\). Soit \( N_1\in \eN\) tel que pour tout \( n>N_1\) nous avons \( | j_n(x) |<1\) dès que \( | x |>M\) (hypothèse~\ref{ITEMooFQYXooEkUAIb}). Alors
	\begin{equation}
		\int_{| x |>M}| j_n(x)u(x) |<\epsilon.
	\end{equation}
	De plus en posant \( s=\max\{ | u(x) |\tq \delta\leq | x |\leq M \}\) (qui existe parce que \( u\) est continue et prise sur un compact) nous pouvons considérer \( N_2\) tel que \( j_n(x)<\epsilon/s\) pour tout \( | x |>\delta\).

	Avec \( n>\max\{ N_1,N_2 \}\) nous avons
	\begin{subequations}
		\begin{align}
			| \int_{B(0,\delta)}j_nu-\int_{\eR}j_nu | & =| \int_{B(0,\delta)^c}j_nu |                                        \\
			                                          & \leq \int_{\delta\leq| x |\leq M}| j_nu |+\int_{| x |\geq M}| j_nu | \\
			                                          & \leq \epsilon(1+| M-\delta |).
		\end{align}
	\end{subequations}
	En redéfinissant le \( \epsilon\) nous avons donc montré que pour tout \( \epsilon\) et \( \delta\), il existe un \( N\in \eN\) tel que
	\begin{equation}
		| I_{\delta,n}-I_n |\leq \epsilon
	\end{equation}
	dès que \( n\geq N\).

	La fonction \( u\) est uniformément continue sur tout \( \overline{ B(0,\delta) }\), et nous pouvons donc choisir \( \delta\) tel que \( | u(0)-u(x) |\leq \epsilon\) pour tout \( x\in \overline{ B(0,\delta) }\). Pour ce \( \delta\), nous avons déjà trouvé un \( N\) tel que \( | I_{\delta,n}-I_n |\leq \epsilon\) dès que \( n>N\). Nous avons :
	\begin{subequations}
		\begin{align}
			| I_{\delta,n}-Z_{\delta,n} | & \leq \int_{B(0,\delta)}| u(x)-u(0) |j_n(x)dx \\
			                              & \leq \epsilon\int_{B(0,\delta)}j_n           \\
			                              & \leq \epsilon.
		\end{align}
	\end{subequations}
	Nous avons donc prouvé que pour tout \( \epsilon>0\), il existe un \( \delta\) et un \( N\) tels que
	\begin{subequations}
		\begin{numcases}{}
			| I_{\delta,n}-I_n |\leq \epsilon\\
			| I_{\delta,n}-Z_{\delta,n} |\leq \epsilon
		\end{numcases}
	\end{subequations}
	dès que \( n\geq N\).

	Enfin nous avons
	\begin{equation}
		| Z_{\delta,n}-u(0) |=u(0)\left( \int_{B(0,\delta)}j_n-1 \right),
	\end{equation}
	et par l'hypothèse~\ref{ITEMooFYCRooFeRRjE} nous pouvons choisir \( n\) assez grand pour que la parenthèse soit plus petite que~\( \epsilon\).

	Pour \( \epsilon\) donné, nous avons donc trouvé un \( \delta\) et un \( N\) tels que
	\begin{equation}
		| I_n-u(0) |\leq | I_n-I_{\delta,n} |+| I_{\delta,n}-Z_{\delta,n} |+| Z_{\delta,n}-u(0) |\leq 3\epsilon.
	\end{equation}
	En passant à la limite nous avons bien \( I_n\to u(0)\) dans \( \eR\).
\end{proof}

Il va sans dire que nous connaissons de telles fonctions. Nous en donnons une maintenant.

\begin{example}[\cite{ooKWQPooPhWNkI}]
	Nous introduisons la fonction \( f_{\epsilon}\) (\( \epsilon>0\)) donnée par
	\begin{equation}
		f_{\epsilon}(x)= e^{-\epsilon| x |}.
	\end{equation}
	Nous calculons la transformée de Fourier de \( f_{\epsilon}\) en divisant le domaine d'intégration :
	\begin{equation}
		\hat f_{\epsilon}(k)=\int_{\eR} e^{-ikx} e^{-\epsilon| x |}dx=\int_{-\infty}^0 e^{\epsilon x} e^{-ikx}dx+\int_{0}^{\infty} e^{-\epsilon x}\ e^{-ikx}dx
	\end{equation}
	En décomposant les parties imaginaires et réelles, et avec un peu de changement de variables, nous pouvons utiliser les intégrales \eqref{EQooNCVIooWqbbrH} et \eqref{EQooSAYUooSatbGc} pour obtenir
	\begin{equation}
		\hat f_{\epsilon}(k)=\frac{ 2\epsilon }{ k^2+\epsilon^2 }.
	\end{equation}
	Sachant que \( \arctan(x)\) est une primitive de \( \frac{1}{ x^2+1 }\) et avec encore un peu de changement de variables, nous avons\footnote{Et en écrivant correctement l'intégrale sur \( \eR\) comme une limite, etc.}
	\begin{equation}
		\int_{\eR}\hat f_{\epsilon}(k)dk=\int_{-\infty}^{\infty}\frac{ 2\epsilon }{ k^2+\epsilon^2 }=2[\arctan(x/\epsilon)]_{-\infty}^{\infty}=2\pi.
	\end{equation}
	Cela montre que si nous introduisons la fonction \( \delta_{\epsilon}\) donnée par
	\begin{equation}
		\delta_{\epsilon}(k)=\frac{1}{ \pi }\frac{ \epsilon }{ \epsilon^2+k^2 },
	\end{equation}
	alors nous avons une fonction qui tout en même temps ressemble à \( \hat f_{\epsilon}\) et vérifie
	\begin{equation}
		\int_{\eR}\delta_{\epsilon}(k)dk=1
	\end{equation}
	pour tout \( \epsilon\).

	Jusqu'ici nous avons montré que
	\begin{equation}        \label{EQooQOOOooBnfJNi}
		\int_{\eR} e^{-ikx} e^{-\epsilon| x |}dx=2\pi \delta_{\epsilon}(k).
	\end{equation}
	Pour chaque \( \epsilon>0\) nous avons \( \delta_{\epsilon}\in L^1(\eR)\).
\end{example}

\begin{proposition}[\cite{MonCerveau}]
	Soit \( g\in\swS(\eR)\). Alors nous avons
	\begin{equation}        \label{EQooBOJUooQGvMrk}
		\int_{\eR}\int_{\eR} g(x) e^{-ixy}dx\,dy=2\pi g(0).
	\end{equation}
\end{proposition}

\begin{proof}
	Soit \( u\in\swS(\eR)\); nous multiplions l'équation \eqref{EQooQOOOooBnfJNi} par \( u(k)\) et nous intégrons par rapport à \( k\) :
	\begin{equation}        \label{EQooTTQQooKxhxzl}
		\int_{\eR}u(k)\left[ \int_{\eR} e^{-ikx} e^{-\epsilon| x |}dx \right]dk=2\pi\int_{\eR}u(k)\delta_{\epsilon}(k)dk.
	\end{equation}
	Il s'agit de passer à la limite dans l'équation \eqref{EQooTTQQooKxhxzl}. Les intégrales à gauche peuvent être effectuées séparément parce qu'elles respectent le théorème de Fubini. En effet soit la fonction
	\begin{equation}
		f(k,x)=u(k) e^{-ikx} e^{-\epsilon| x |}
	\end{equation}
	qui est dans \( L^1(\eR\times \eR)\) en vertu du critère du corolaire~\ref{CorTKZKwP} et du fait que à la fois \( k\mapsto | u(k) | \) et \( x\mapsto  e^{-\epsilon| x |}\) sont dans \( L^1(\eR)\).

	Nous pouvons donc grouper et dégrouper les intégrales et en particulier les inverser. Si nous effectuons d'abord l'intégrale sur \( k\) nous trouvons
	\begin{equation}
		\int_{\eR}u(k)\left[ \int_{\eR} e^{-ikx} e^{-\epsilon| x |}dx \right]dk=\int_{\eR} e^{-\epsilon| x |}\int_{\eR}u(k) e^{-ikx}dk\,dx=\int_{\eR} e^{-\epsilon| x |}\hat u(x)dx.
	\end{equation}
	La fonction \( x\mapsto |  e^{-\epsilon| x |}\hat u(x) |\) est majorée (uniformément en \( \epsilon\)) par \( x\mapsto \hat u(x)\) qui est intégrable parce que la transformée de Fourier d'une fonction de \( \swS\) est dans \( \swS\) par la proposition~\ref{PropKPsjyzT}. Le théorème de la convergence dominée de Lebesgue~\ref{ThoConvDomLebVdhsTf} nous permet de permuter la limite \( \epsilon\to 0 \) avec l'intégrale et obtenir
	\begin{equation}
		\lim_{\epsilon\to 0}\int_{\eR}u(k)\int_{\eR} e^{-ikx} e^{-\epsilon| x |}dx\,dk=\int_{\eR}\hat u(x)dx=\int_{\eR}\int_{\eR}u(k) e^{-ikx}dk\,dx.
	\end{equation}
	Notons qu'en passant à la limite nous avons perdu le droit de permuter les intégrales.

	Nous devons encore prouver que
	\begin{equation}
		\lim_{\epsilon\to 0}\int_{\eR}u(k)\delta_{\epsilon}(k)dk=u(0).
	\end{equation}
	Cela n'est rien d'autre que le lemme~\ref{LEMooHEEOooFtKgfz} appliqué à la suite de fonctions \( j_n=\delta_{1/n}\).
\end{proof}

\begin{normaltext}
	Notons que les intégrales dans \eqref{EQooBOJUooQGvMrk} ne peuvent pas être permutées parce que \( \int_{\eR} e^{-ixy}dy\) n'existe pas. Il faut avouer que, malgré tous les conseils du type «attention : permuter des intégrales doit être fait avec prudence», ce n'est pas tous les jours que nous trouvons des intégrales qui ne peuvent pas être permutées, autrement que dans des exemples fait exprès.
\end{normaltext}

%---------------------------------------------------------------------------------------------------------------------------
\subsection{Peigne de Dirac}
%---------------------------------------------------------------------------------------------------------------------------

\begin{proposition}
	La formule
	\begin{equation}    \label{EqMEVmKvg}
		\Delta_a=\sum_{k\in \eZ}\delta_{ka}
	\end{equation}
	définit un élément de \( \swD'(\eR)\).
\end{proposition}
La forme linéaire \( \Delta_a\) est le \defe{peigne de Dirac}{peigne de Dirac} de pas \( a\).

\begin{proof}
	Nous utilisons le critère de continuité séquentielle en zéro du théorème~\ref{ThoVDDBnVn}. Soit une suite \( \varphi_n\to 0\) dans \( \swD(\eR)\). Par le théorème~\ref{ThoXYADBZr} il existe un compact \( K\) de \( \eR\) pour lequel \( \varphi_n\in\swD(K)\) pour tout \( n\) et \( \varphi_n\to0\) dans \( \swD(K)\). La somme~\ref{EqMEVmKvg} est donc finie et nous pouvons la permuter avec une limite :
	\begin{equation}
		\lim_{n\to \infty} \Delta_a(\varphi_n)=\sum_{k\in\eZ}\lim_{n\to \infty} \varphi_n(ka).
	\end{equation}
	La limite \( \varphi_n\to 0\) dans \( \swD(K)\) signifie que nous avons convergence uniforme de la fonction et de toutes ses dérivées vers \( 0\). En particulier \( \| \varphi_n \|_{\infty}\to 0\); disons que la somme (qui est finie) fasse \( s\) termes :
	\begin{equation}
		\sum_{k\in \eZ}\varphi_n(ka)\leq s\| \varphi_n \|_{\infty}.
	\end{equation}
	Le terme de droite tend vers zéro lorsque \( n\) tend vers l'infini.
\end{proof}
Donc \( \Delta_a\) est bien une distribution au sens de la définition~\ref{DefPZDtWVP}.

\begin{lemma}[\cite{CXCQJIt}]
	Le peigne de Dirac vérifie la relation
	\begin{equation}
		\Delta_a=\frac{1}{ a }\Delta_1\circ D_a
	\end{equation}
	où \( D_a\) est l'application \( D_a\colon \swD(\eR)\to \swD(\eR)\),
	\begin{equation}
		(D_af)(x)=af(ax).
	\end{equation}
\end{lemma}

\begin{proof}
	Pour \( \varphi\in\swD(\eR)\) nous avons
	\begin{equation}
		\Delta_a(\varphi)=\sum_{k\in \eZ}\varphi(ka)=\frac{1}{ a }\sum_{k\in \eZ}(D_a\varphi)(k)=\frac{1}{ a }\Delta_1(D_a\varphi).
	\end{equation}
\end{proof}

\begin{proposition}
	Le peigne de Dirac est une distribution tempérée.
\end{proposition}

Notez qu'il y a plus de fonctions dans \( \swS(\eR)\) que dans \( \swD(\eR)\); il est donc plus difficile de rentrer dans \( \swS'(\eR)\) que dans \( \swD'(\eR)\) : il est plus compliqué d'avoir existence de \( T(\varphi)\) pour tout \( \varphi\in\swS(\eR)\) que pour tout \( \varphi\in\swD(\eR)\).

\begin{proof}
	Soit \( \varphi\in\swS(\eR)\). Nous avons
	\begin{equation}
		|\Delta_a(\varphi)|=| \sum_k\varphi(ak) |=\left| \sum_k\frac{ (1+a^2k^2)\varphi(ak) }{ 1+a^2k^2 } \right| \leq \sup_{x\in \eR}\big| (1+x^2)\varphi(x) \big|\sum_k\frac{1}{ 1+a^2k^2 }.
	\end{equation}
	La somme \( \sum_k\frac{1}{ 1+a^2k^2 }\) est une somme convergente, et le supremum est borné par la proposition~\ref{PropCSmzwGv} en prenant \( Q(x)=1+x^2\). En effet sur \( \overline{ B(0,r) }\) la fonction \( x\mapsto (1+x^2)\varphi(x)\) est bornée par ce que c'est une fonction continue sur un compact, et à l'extérieur de \( B(0,r)\) cette fonction est alors bornée par \( 1\).
\end{proof}

Si aucune ambigüité n'est à craindre, nous noterons \( f\) la distribution \( T_f\).

\begin{example}
	La transformée de Fourier de la distribution de Dirac est la fonction constante : \( \hat \delta=1\). En effet si nous agissons sur une fonction test,
	\begin{equation}
		\hat \delta(\varphi)=\delta(\hat \varphi)=\hat \varphi(0)=\int_{\eR^d}\varphi(x)dx.
	\end{equation}
\end{example}

%+++++++++++++++++++++++++++++++++++++++++++++++++++++++++++++++++++++++++++++++++++++++++++++++++++++++++++++++++++++++++++
\section{L'espace \texorpdfstring{\(   C^{\infty}(\eR,\swS'(\eR^d))\)}{C(R,S')}}
%+++++++++++++++++++++++++++++++++++++++++++++++++++++++++++++++++++++++++++++++++++++++++++++++++++++++++++++++++++++++++++

Dans cette section nous notons \( I\) un ouvert de \( \eR\) et \( \Omega\) un ouvert de \( \eR^d\); si \( \psi\) est une fonction sur \( I\times \Omega\) nous allons noter \( \psi_t\colon \Omega\to \eR\) la fonction \( \psi_t(x)=\psi(t,x)\). C'est une notation plus légère que \( \psi(t,.)\).

%---------------------------------------------------------------------------------------------------------------------------
\subsection{Propriétés générales}
%---------------------------------------------------------------------------------------------------------------------------

La définition de l'espace \(  C^{\infty}(I,\swS'(\Omega))\) est encore la définition~\ref{DefDZsypWu} et les propriétés énoncées dans la proposition~\ref{PropIPlKQBa} sont encore bonnes ici.

D'abord parlons un peu de continuité en recopiant la proposition~\ref{PropVKSNflB} dans notre contexte.
\begin{proposition}     \label{PropBXFmvPs}
	Soient \( I\) un intervalle ouvert de \( \eR\) et \( u\colon I\to \swS'(\eR^d)\) une fonction continue. Alors
	\begin{enumerate}
		\item   \label{ItemFTvVUEW}
		      Pour tout \( \varphi\in\swS(\eR^d)\), l'application \( t\mapsto u_t(\varphi)\) est continue.
		\item
		      Pour tout \( \varphi\in\swS(\eR^d)\), nous avons la limite
		      \begin{equation}
			      \lim_{t\to t_0} u_t(\varphi)=u_{t_0}(\varphi).
		      \end{equation}
		\item
		      Nous avons la limite dans \( \swS'(\eR^d)\)
		      \begin{equation}
			      \lim_{t\to t_0} u_t=u_{t_0}.
		      \end{equation}
	\end{enumerate}
\end{proposition}

\begin{lemma}
	Nous avons \(  C^{\infty}(I,\swS'(\Omega))\subset C^{\infty}(I,\swD'(\Omega))\).
\end{lemma}

\begin{proof}
	Soit \( (T_t)\in C^{\infty}(I,\swS'(\Omega))\). Pour chaque \( t\) nous avons
	\begin{equation}
		T_t\in\swS'(\Omega)\subset\swD'(\Omega).
	\end{equation}
	Ensuite il suffit de dire que pour tout \( \varphi\in\swD(\Omega)\) la fonction
	\begin{equation}
		t\mapsto T_t(\varphi)
	\end{equation}
	est de classe \(  C^{\infty}\) parce que c'est le cas pour toute fonction dans \( \swS(\Omega)\). La proposition~\ref{PropIPlKQBa} (en changeant \( \swD\) en \( \swS\)) conclut que \( (T_t)\in C^{\infty}(I,\swD'(\Omega))\).
\end{proof}

\begin{proposition} \label{PropIIAcyDq}
	L'espace \( \swS(\Omega)\) est complet et métrisable.
\end{proposition}

\begin{proof}
	En ce qui concerne le métrisable nous reprenons la formule de l'écart \eqref{EqAAghiUR}. Dans notre cas pour l'écrire explicitement il faudrait une énumération de \( \eN^2\) à partir de \( 1\) (et non de zéro). Cette formule donne bien une distance parce que si \( d(\varphi_1-\varphi_2)=0\) alors en particulier \( p_{00}(\varphi_1-\varphi_2)=\| \varphi_1-\varphi_2 \|_{\infty}=0\) et donc \( \varphi_1=\varphi_2\).

	Nous montrons maintenant que \( \swS(\Omega)\) est complet en y considérant une suite de Cauchy \( (\varphi_n)\). Soit \( \epsilon>0\) et \( \alpha,\beta\in \eN\) ainsi que \( k,l\) assez grands pour que \( \varphi_k-\varphi_l\in B_{\alpha\beta}(0,\epsilon)\). En particulier pour \( \alpha=\beta=0\) nous avons \( \| \varphi_k-\varphi_l \|_{\infty}\leq \epsilon\), ce qui signifie que nous avons une suite vérifiant le critère de Cauchy uniforme~\ref{PropNTEynwq}. Elle converge donc uniformément vers une certaine fonction \( \varphi\) que la proposition~\ref{ThoUnigCvCont} nous assure être continue. Il existe donc \( \varphi\in C(\Omega)\) telle que
	\begin{equation}
		\varphi_k\stackrel{unif}{\longrightarrow}\varphi.
	\end{equation}
	Nous devons montrer que \( \varphi\in\swS(\Omega)\). Le fait que \( \varphi\) soit de classe \(  C^{\infty}\) s'obtient en utilisant les seminormes \( p_{0,\alpha}(\varphi)=\| \partial^{\alpha}\varphi \|_{\infty}\) de la même façon que dans la preuve que \( \swD(\Omega)\) était complet (proposition~\ref{PropQAEVcTi}). Nous obtenons en particulier que
	\begin{equation}    \label{EqSZyYkqk}
		\partial^{\alpha}\varphi_k\stackrel{unif}{\longrightarrow}\partial^{\alpha}\varphi
	\end{equation}
	pour tout multiindice \( \alpha\). Montrons encore que \( \varphi\) est à décroissance rapide : nous devons montrer que pour tout \( \alpha\) et \( \beta\) nous avons
	\begin{equation}
		p_{\alpha\beta}(\varphi)=\sup_{x\in \Omega}\big| x^{\beta}(\partial^{\alpha}\varphi)(x) \big|<\infty.
	\end{equation}
	Étant donné que \( (\varphi_n)\) est de Cauchy dans \( \swS(\Omega)\) nous avons (pour \( \epsilon\) fixé et \( k,l\) assez grands) :
	\begin{equation}
		\big| x^{\beta}(\partial^{\alpha}\varphi_k-\partial^{\alpha}\varphi_l)(x) \big|\leq \epsilon
	\end{equation}
	pour tout \( x\in\Omega\). En considérant \( l\) fixé et en prenant la limite \( k\to \infty\) et en utilisant la convergence uniforme \eqref{EqSZyYkqk} nous trouvons que
	\begin{equation}
		\big| x^{\beta}(\partial^{\alpha}\varphi-\partial^{\alpha}\varphi_l)(x) \big|\leq \epsilon
	\end{equation}
	Du coup nous pouvons faire la majoration
	\begin{equation}
		\sup_{x\in\Omega}\big| x^{\beta}(\partial^{\alpha}\varphi)(x) \big|\leq\sup_x\big| x^{\beta}(\partial^{\alpha}\varphi-\partial^{\alpha}\varphi_l)(x) \big|+\sup_x\big| (\partial^{\alpha}\varphi_l)(x) \big|\leq\epsilon+p_{\alpha\beta}(\varphi_l)<\infty
	\end{equation}
	du fait que \( p_{\alpha\beta}(\varphi_l)<\infty\) parce que \( \varphi_l\in\swS(\Omega)\).

	Donc \( \varphi\in\swS(\Omega)\) et ce dernier est alors complet.
\end{proof}

\begin{proposition}
	Soit \( (T_t)\in C^0\big( I,\swS'(\Omega) \big)\) et \( \psi\in\swS(I\times \Omega)\). Alors la fonction
	\begin{equation}    \label{EqULcaYjm}
		t\mapsto T_t(\psi_t)
	\end{equation}
	est continue sur \( I\).
\end{proposition}

\begin{proof}
	Soient \( t_0\in I\) et une suite convergente vers \( t_0\) : \( t_j\to t_0\) dans \( \eR\). Vu que \( (T_t)\) est continue en \( t\), elle est en particulier séquentiellement continue et nous avons
	\begin{equation}
		T_{t_j}\stackrel{\swS'(\Omega)}{\longrightarrow}T_{t_0}.
	\end{equation}
	Montrons que nous avons aussi \( \psi_{t_j}\stackrel{\swS(\Omega)}{\longrightarrow}\psi_{t_0}\). Pour cela nous utilisons les seminormes\footnote{Pas parce que nous en avons envie, mais bien parce qu'elles font partie de la définition de la convergence et de tous ces trucs.} \( p_{\alpha\beta}\) définies en \eqref{EqOWdChCu} :
	\begin{subequations}
		\begin{align}
			p_{\alpha\beta}(\psi_{t_j}-\psi_{t_0}) & =\sum_{x\in \Omega}\Big| x^{\beta}\big( \partial^{\alpha}\psi(t_j,x)-\partial^{\alpha}\psi(t_0,x) \big)       \Big|                                  \\
			                                       & \leq\sup_{x\in\Omega}\Big|  x^{\beta}| t_0-t_j |\sup_{t\in \mathopen[ t_0 , t_j \mathclose]}\big| \partial_t\partial^{\alpha}\psi(t,x) \big|   \Big| \\
			                                       & \leq| t_0-t_j |\sup_{x\in\Omega}\sup_{t\in I}\big| x^{\beta}\partial_t\partial^{\alpha}\psi(t,x) \big|                                               \\
			                                       & \leq| t_0-t_j |P_{(\alpha t),\beta}(\psi).
		\end{align}
	\end{subequations}
	Pour la première majoration nous avons utilisé le théorème des accroissements finie~\ref{val_medio_2}. Pour la dernière ligne nous avons noté \( P_{\alpha\beta}\) les seminormes de \( \swS(I\times \Omega)\) et \( (\alpha t)\) est le multiiindice \( \alpha\) à qui on a ajouté la variable \( t\) à la fin. Étant donné que \( P_{(\alpha t)\beta}(\psi)<\infty\) nous avons bien
	\begin{equation}
		p_{\alpha\beta}(\psi_{t_j}-\psi_{t_0})\to 0
	\end{equation}
	et donc \( \psi_{t_j}\stackrel{\swS(\Omega)}{\longrightarrow}\psi_{t_0}\).

	Étant donné que \( \swS(\Omega)\) est métrisable et complet, le corolaire~\ref{CorPGwLluz} nous dit que
	\begin{equation}
		T_{t_j}(\psi_{t_j})\stackrel{\eC}{\longrightarrow} T_{t_0}(\psi_{t_0}),
	\end{equation}
	ce qui est bien le critère de continuité séquentielle de la fonction \eqref{EqULcaYjm}.
\end{proof}

\begin{remark}  \label{RemZYVkHRT}
	La proposition~\ref{PropLKtBsVi} nous dit, a fortiori, que si \( (T_t)\in C^{\infty}\big( I,\swS'(\Omega) \big)\) alors la formule
	\begin{equation}
		\tilde T(\psi)=\int_IT_t(\psi_t)
	\end{equation}
	donne un élément \( \tilde T\in\swD'(I\times \Omega)\). Au cas où aucune confusion n'est à craindre, nous pourrons noter également \( T\) l'élément de \( \swD'(I\times \Omega)\) déduit de \( T\in C^{\infty}\big( I,\swS'(\Omega) \big)\).
\end{remark}

Notons que ce \( T\) ne sera pas toujours une distribution tempérée comme le montre l'exemple suivant.

\begin{example}
	En posant \( T_t(\varphi)= e^{t^2}\varphi(0)\) avec \( I=\eR\), l'intégrale
	\begin{equation}
		T(\psi)=\int_{\eR}T_t(\psi_t)=\int_{\eR} e^{t^2}\psi(t,0)dt
	\end{equation}
	ne converge pas pour tout \( \psi\in\swS(\eR\times \Omega)\). En effet par rapport à \( t\), la fonction \( \psi(t,0)\) décroît rapidement mais pas spécialement assez rapidement pour compenser \( e^{t^2}\).
\end{example}

%---------------------------------------------------------------------------------------------------------------------------
\subsection{Dérivation}
%---------------------------------------------------------------------------------------------------------------------------

\begin{proposition}[\cite{GQYneyj}] \label{PropGKoPbko}
	Soit \( T\in C^k\big( I,\swS'(\Omega) \big)\) et \( 0\leq l\leq k\). Pour tout \( t_0\in I\) l'application
	\begin{equation}    \label{EqZMDeZco}
		\begin{aligned}
			T_{t_0}^{(l)}\colon \swS(\Omega) & \to \eC                                                   \\
			\varphi                          & \mapsto \left(\frac{ d^l  }{ dt }T_t(\varphi)\right)(t_0)
		\end{aligned}
	\end{equation}
	est bien définie, est une distribution et de plus
	\begin{equation}
		t\mapsto T_t^{(l)}\in C^{k-l}\big( I,\swS'(\Omega) \big).
	\end{equation}
\end{proposition}
Attention que la formule \eqref{EqZMDeZco} est bonne si \( \varphi\in\swS(\Omega)\). Si par contre \( \psi\in\swS(I\times \Omega)\) et qu'on veut regarder \( u_t^{(1)}(\psi_t)\) alors il faut regarder la proposition~\ref{PropUDkgksG} et utiliser la formule \eqref{EqSCNYYhE} dans laquelle se trouve \( u_t^{(1)}(\psi_t)\).

\begin{proof}
	Pour \( k=0\) nous avons \( T^{(0)}_t=T_t\) et c'est bon. Pour le cas \( k=1\) et \( l=0\) c'est encore \( T_t^{(0)}=T_t\) qui fonctionne.

	Le premier cas non trivial à traiter est \( k=1\) et \( l=1\). Nous considérons \( t_0\in I\); par définition de la dérivée, pour tout \( \varphi\in\swS(\Omega)\), nous avons (pour peu que les limites existent) :
	\begin{equation}    \label{EqCTuAfXe}
		T^{(1)}_{t_0}(\varphi)=  \Dsdd{ T_t(\varphi) }{t}{t_0}=\lim_{j\to \infty} \frac{ T_{t_0+\epsilon_j}(\varphi)-T_{t_0}(\varphi) }{ \epsilon_j }= \lim_{j\to \infty} U_j(\varphi)
	\end{equation}
	où
	\begin{equation}
		U_j=\frac{1}{ \epsilon_j }\big( T_{t_0+\epsilon_j}-T_{t_0} \big).
	\end{equation}
	et \( (\epsilon_j)\) est une suite de réels tendant vers zéro.

	Vu que \( (T_t)\in C^k(I,\swS'(\Omega))\), l'application \( t\mapsto T_t(\varphi)\) est de classe \( C^k\) et en particulier l'expression \eqref{EqCTuAfXe} a une limite lorsque \( j\to \infty\). Donc \( T^{(1)}_{t_0}(\varphi)\) est bien définie. Le point~\ref{ItemAEOtOMLi} du corolaire~\ref{CorPGwLluz} nous dit que \( \lim_{j\to \infty} U_j(\varphi)= T_{t_0}^{(1)}(\varphi)\) et \( T_{t_0}^{(1)}\) est une distribution (linéaire et continue).

	Nous devons encore voir que \( t\mapsto T^{(1)}_t\) est une application \( C^0\big( I,\swS'(\Omega) \big)\). Cela est une conséquence du fait que \( (T_t)\) soit de classe \( C^1\), ce qui se traduit par le fait que l'application
	\begin{equation}
		t\mapsto \frac{ d }{ dt }\Big( T_t(\varphi) \Big)
	\end{equation}
	est continue (définition de la dérivée et point~\ref{ItemFTvVUEW} de la proposition~\ref{PropBXFmvPs} appliquée à la dérivée).

	Les cas \( k\geq 1\) se traitent par récurrence.
\end{proof}

\begin{proposition}[\cite{GQYneyj}] \label{PropUDkgksG}
	Soit \( (T_t)\in C^1\big( I,\swS'(\Omega) \big)\) et \( \psi\in\swS(I\times \Omega)\). Alors la fonction
	\begin{equation}
		t\mapsto T_t\big( \psi(t,.) \big)
	\end{equation}
	est de classe \( C^1\) sur \( I\) et
	\begin{equation}    \label{EqSCNYYhE}
		\frac{ d }{ dt }\Big( T_t\big( \psi(t,.) \big) \Big)=T_t^{(1)}\big( \psi(t,.) \big)+T_t\left( \frac{ \partial \psi }{ \partial t }(t,.) \right)
	\end{equation}
\end{proposition}

\begin{proof}
	Soient \( t_0\in I\) et \( \epsilon_j\to 0\) une suite réelle. Le membre de gauche de \eqref{EqSCNYYhE}, écrit en \( t_0\), donne
	\begin{equation}    \label{BJPHzwn}
		\spadesuit=\lim_{j\to \infty} \frac{ T_{t_0+\epsilon_j}\big( \psi(t_0+\epsilon_j,.) \big)-T_{t_0}\big( \psi(t_0,.) \big) }{ \epsilon_j }
	\end{equation}
	Afin d'alléger les notations nous allons écrire \( \psi_t=\psi(t,.)\). Dans le numérateur de \eqref{BJPHzwn} nous ajoutons et soustrayons la quantité \( T_{t_0+\epsilon_j}(\psi_{t_0})\) et nous découpons la limite en deux morceaux :
	\begin{equation}
		\spadesuit=\lim_{j\to \infty} \frac{ T_{t_0+\epsilon_j}(\psi_{t_0+\epsilon_j}-\psi_{t_0}) }{ \epsilon_j }+\lim_{j\to \infty} \frac{ (T_{t_0+\epsilon_j}-T_{t_0})(\psi_{t_0}) }{ \epsilon_j }
	\end{equation}
	Le second terme vaut
	\begin{equation}
		\frac{ d }{ dt }\Big( T_t(\psi_{t_0}) \Big)_{t=t_0}=T_{t_0}^{(1)}(\psi_{t_0})
	\end{equation}
	par la proposition~\ref{PropGKoPbko}. Occupons nous de l'autre morceau de \( \spadesuit\). Nous posons \( U_j=T_{t_0+\epsilon_j}\) et
	\begin{equation}
		\varphi_j=\frac{1}{ \epsilon_j }(\psi_{t_0+\epsilon_j}-\psi_{t_0}).
	\end{equation}
	Nous voulons utiliser le corolaire~\ref{CorPGwLluz}\ref{ItemAEOtOMLiii} pour obtenir
	\begin{equation}
		\lim_{j\to \infty} U_j(\varphi_j)=T_{t_0}\Big( \frac{ \partial \psi }{ \partial t }(t_0,.) \Big).
	\end{equation}
	D'une part \( (T_t)\) est de classe \(  C^{\infty}\) en \( t\) et nous avons donc la convergence \( U_j\stackrel{\swS'(\Omega)}{\longrightarrow}T_{t_0}\). Reste à prouver que
	\begin{equation}
		\varphi_j\stackrel{\swS(\Omega)}{\longrightarrow}\frac{ \partial \psi }{ \partial t }(t_0,.).
	\end{equation}
	Cela en remarquant bien que la variable de dérivation n'est pas celle par rapport à laquelle nous voulons la convergence Schwartz\footnote{Je ne sais pas si je me suis bien fait comprendre là.}. Soient \( \alpha\) et \( \beta\) des naturels et calculons un peu :
	\begin{equation}    \label{EqEBUYDRA}
		p_{\alpha\beta}\big( \varphi_j-\frac{ \partial \psi }{ \partial t }(t_0,.) \big)=\sup_{x\in\Omega}\left| x^{\beta}\partial^{\alpha}\Big( \frac{1}{ \epsilon_j }\big(\psi(t_0+\epsilon_j,x)-\psi(t_0,x)\big) -\frac{ \partial \psi }{ \partial t }(t_0,x) \Big)\right|
	\end{equation}
	Il est à présent l'heure d'utiliser un développement de Taylor avec le reste de la proposition~\ref{PropResteTaylorc} :
	\begin{equation}
		\psi(t_0+\epsilon_j,x)=\psi(t_0,x)+\epsilon_j\frac{ \partial \psi }{ \partial t }(t_0,x)+\frac{ \epsilon_j^2 }{2}\frac{ \partial^2\psi  }{ \partial t^2 }(\bar t,x)
	\end{equation}
	pour un certain \( \bar t\in\mathopen[ t_0 , t_0+\epsilon_j \mathclose]\). En mettant ça dans le calcul \eqref{EqEBUYDRA} nous restons avec
	\begin{equation}
		p_{\alpha\beta}\big( \varphi_j-\frac{ \partial \psi }{ \partial t }(t_0,.) \big)=\sup_{x\in\Omega}\left| x^{\beta}\partial^{\alpha}\Big( \epsilon_j\frac{ \partial^2\psi }{ \partial t^2 }(\bar t,x) \Big) \right| \leq \epsilon_j P_{\alpha,2;\beta,0}(\psi)
	\end{equation}
	où \( P_{\alpha,k;\beta,l}\) sont les seminormes de \( \swS(I\times \Omega)\) avec la notation plus ou moins évidente de prendre \( \alpha\) dérivations sur \( x\), \( k\) sur \( t\) puis de multiplier par \( x^{\beta}t^l\). Au final nous avons bien
	\begin{equation}
		\lim_{j\to \infty} p_{\alpha\beta}\big( \varphi_j-\frac{ \partial \psi }{ \partial t }(t_0,.) \big)=0
	\end{equation}
	et donc la convergence \( \varphi_j\stackrel{\swS(\Omega)}{\longrightarrow}\frac{ \partial \psi }{ \partial t }(t_0,.)\).
\end{proof}

\begin{lemma}   \label{LemWRoRPIX}
	Soit \( (T_t)\in C^1\big( I,\swS'(\Omega) \big)\) alors si \( \TF\) dénote la transformée de Fourier nous avons
	\begin{equation}
		\TF\big( T_t^{(1)} \big)=(\TF T)_t^{(1)}
	\end{equation}
	où \( (\TF T)\) est la famille de distributions \( (\TF T)_t=\TF T_t\).
\end{lemma}

\begin{proof}
	Pour la preuve il suffit de tester l'égalité sur une fonction \( \varphi\in\swS(\Omega)\) :
	\begin{equation}
		(\TF T_t^{(1)})(\varphi)=T_t^{(1)}(\TF \varphi)=\frac{ d }{ dt }\Big( T_t(\TF \varphi) \Big)=\frac{ d }{ dt }\Big( (\TF T_t)(\varphi) \Big)=(\TF T)_t^{(1)}(\varphi).
	\end{equation}
\end{proof}
