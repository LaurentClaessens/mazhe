% This is part of Mes notes de mathématique
% Copyright (c) 2011-2022
%   Laurent Claessens
% See the file fdl-1.3.txt for copying conditions.

%+++++++++++++++++++++++++++++++++++++++++++++++++++++++++++++++++++++++++++++++++++++++++++++++++++++++++++++++++++++++++++
\section{Extension de corps}
%+++++++++++++++++++++++++++++++++++++++++++++++++++++++++++++++++++++++++++++++++++++++++++++++++++++++++++++++++++++++++++
\label{SECooLQVJooTGeqiR}

\begin{lemma}       \label{LemobATFP}
    Soit \( \eL\) un corps\footnote{Définition \ref{DefTMNooKXHUd}.} fini et \( \eK\) un sous-corps de \( \eL\). Alors il existe \( s\in \eN\) tel que
    \begin{equation}        \label{EqUgqlJQ}
        \Card(\eL)=\Card(\eK)^s.
    \end{equation}
\end{lemma}

\begin{proof}
    Le corps \( \eL\) est un \( \eK\)-espace vectoriel de dimension finie. Si \( s\) est la dimension alors nous avons la formule \eqref{EqUgqlJQ} parce que chaque élément de \( \eL\) est un \( s\)-uple d'éléments de \( \eK\).
\end{proof}

\begin{definition}[\cite{ooOXISooAFtXsZ}]     \label{DEFooFLJJooGJYDOe}
    Soit \( \eK\) un corps commutatif. Une \defe{extension}{extension!de corps} de \( \eK\) est un couple \( (\eL,j)\) où \( \eL\) est un corps et \( j\colon \eK\to \eL\) est un morphisme de corps.
\end{definition}

    Nous identifions le plus souvent \( \eK\) avec \( j(\eK)\subset \eL\), mais il faut savoir que le corps \( \eL\) étendant \( \eK\) n'est pas toujours un sur-corps de \( \eK\). En particulier, l'ensemble \( \eL\) peut ne pas être une extension de l'ensemble \( \eK\).

\begin{lemmaDef}       \label{LemooOLIIooXzdppM}
    Si \( (\eL,i)\) est une extension de \( \eK\), alors \( \eL\) devient un espace vectoriel sur \( \eK\) si nous posons
    \begin{equation}
        \lambda\cdot x=i(\lambda)x
    \end{equation}
    pour tout \( \lambda\in \eK\) et \( x\in \eL\). La multiplication du membre de droite est celle du corps \( \eL\).
\end{lemmaDef}

\begin{definition}      \label{DefUYiyieu}
    Le \defe{degré}{degré!extension de corps} de \( \eL\) est la dimension de cet espace vectoriel. Il est noté \( [\eL:\eK]\)\nomenclature[A]{$ [\eL:\eK]$}{degré d'une extension de corps}; notons qu'il peut être infini.
\end{definition}

\begin{example}
    L'ensemble \( \eC\) est une extension de \( \eR\) et son degré est \( [\eC:\eR]=2\).
\end{example}

\begin{proposition}[Composition des degrés\cite{ooGIIFooVMVloY}]        \label{PROPooEGSJooBSocTf} \label{PropGWazMpY}
    Si \( \eL_2\) est une extension de \( \eL_1\) qui est elle-même une extension de \( \eK\), alors \( \eL_2\) est une extension de \( \eK\) et on a :
    \begin{equation}        \label{EQooOLLQooFdYtnh}
        [\eL_2:\eK]=[\eL_2:\eL_1][\eL_1:\eK].
    \end{equation}
    Dans ce cas, si \( \{ v_i \}_{i\in I}\) est une \( \eK\)-base de \( \eL_1\) et si \( \{ w_{\alpha} \}_{\alpha\in A}\) est une \( \eL_1\)-base de \( \eL_2\) alors \( \{ v_iw_{\alpha} \}_{\substack{i\in I\\\alpha\in A}}\) est une \( \eK\)-base de \( \eL_2\).
\end{proposition}

Notons que la formule \eqref{EQooOLLQooFdYtnh} n'est pas très instructive dans le cas des extensions non finies. La seconde partie, sur les bases, est en réalité nettement plus intéressante.

\begin{proof}
    Soit \( a\in \eL_2\). Puisque les \( w_{\alpha}\) forment une \( \eL_2\)-base nous avons une décomposition
    \begin{equation}
        a=\sum_{\alpha}a_{\alpha}w_{\alpha}
    \end{equation}
    pour des éléments \( a_{\alpha}\in \eL_1\). Mais les \( v_i\) forment une \( \eK\)-base de \( \eL_1\), donc chacun des \( a_{\alpha}\) peut être décomposé comme \( a_{\alpha}=\sum_ia_{\alpha i}v_iw_{\alpha}\). Donc :
    \begin{equation}
        a=\sum_{\alpha i}a_{\alpha i}v_iw_{\alpha},
    \end{equation}
    qui donne une décomposition de \( a\) en éléments de \( \{ v_iw_{\alpha} \}\) à coefficients dans \( \eK\). La partie proposée est donc génératrice.

    Pour prouver qu'elle est également libre, nous supposons avoir des éléments \( a_{\alpha i}\in \eK\) tels que
    \begin{equation}
        \sum_{\alpha i}a_{\alpha i}v_iw_{\alpha}=0.
    \end{equation}
    En récrivant sous la forme
    \begin{equation}
        \sum_{\alpha}\Big( \sum_ia_{\alpha i}v_i \Big)w_{\alpha}=0,
    \end{equation}
    nous reconnaissons une combinaison linéaire nulle des \( w_{\alpha}\) à coefficients dans \( \eL_1\). Les coefficients sont donc nuls : \( \sum_i a_{\alpha i}v_i=0\). C'est une combinaison linéaire nulle des \( v_i\) à coefficients dans \( \eK\). Comme les \( v_i\) forment une base, les coefficients sont nuls : \( a_{\alpha i}=0\).
\end{proof}

%---------------------------------------------------------------------------------------------------------------------------
\subsection{Extension et polynôme minimal}
%---------------------------------------------------------------------------------------------------------------------------

\begin{lemmaDef}[Polynôme minimal]    \label{DefCVMooFGSAgL}   
    Soit \( \eL\) une extension de \( \eK\) et \( a\in \eL\). Nous considérons la partie
    \begin{equation}
        I_a=\{ P\in \eK[X]\tq P(a)=0 \}
    \end{equation}
    que nous supposons non réduite à \( \{ 0 \}\)\footnote{La non trivialité de \( I_a\) est une vraie hypothèse. En effet si nous prenons \( \eK=\eQ\) et l'extension \( \eL=\eR\), alors il suffit de prendre un réel \( a\) non algébrique sur \( \eQ\) pour que \( I_a\) soit réduit au seul polynôme identiquement nul.}

    \begin{enumerate}
        \item       \label{ITEMooUNLCooIfYZry}
            La partie \( I_a\) est un idéal dans \( \eK[X]\),
        \item       \label{ITEMooDCDRooPDnnbu}
            la partie \( I_a\) est un idéal principal dans \( \eK[X]\),
        \item       \label{ITEMooXFYQooTuMzIu}
            l'idéal \( I_a\) possède un unique générateur unitaire.
    \end{enumerate}

    Cet unique générateur unitaire est le \defe{polynôme minimal}{polynôme!minimal!d'un élément d'une extension} de \( a\) sur \( \eK\).
\end{lemmaDef}

\begin{proof}
    En plusieurs parties.
    \begin{subproof}
    \item[Pour \ref{ITEMooUNLCooIfYZry}]
        Soit \( P\in I_a\) : \( P(a)=0\). Si \( Q\in \eK[X]\) alors la proposition \ref{PROPooGDQCooHziCPH} nous indique que
        \begin{equation}
            (PQ)(a)=P(a)Q(a)=0.
        \end{equation}
        Donc \( PQ\in I_a\). Comme de plus \( I_a\) est clairement vectoriel, \( I_a\) est un idéal.

        Notez que nous avons utilisé la règle du produit nul justifiée par le fait que \( \eK\) soit un corps\quext{Si vous connaissez un contre-exemple à cette proposition dans le cas où \( \eK\) serait remplacé par un anneau, écrivez-moi.} et donc soumis au point \ref{ITEMooQNTFooSRrVPK} de la proposition \ref{DEFooTAOPooWDPYmd}.
    \item[Pour \ref{ITEMooDCDRooPDnnbu}]
        Nous savons par le théorème \ref{ThoCCHkoU} que \( \eK[X]\) est un anneau principal. En particulier, tous ses idéaux sont principaux, c'est dans la définition \ref{DEFooGWOZooXzUlhK} d'un anneau principal.
    \item[Pour \ref{ITEMooXFYQooTuMzIu}]
        Le théorème~\ref{ThoCCHkoU}\ref{ITEMooASHKooZqkiCH} nous informe alors que \( I_a\) possède un unique générateur unitaire.
    \end{subproof}
\end{proof}

Si nous avons un corps et un élément dans une extension du corps, il n'est pas autorisé de dire «soit le polynôme minimal de cet élément dans le premier corps» parce qu'il n'existe peut-être pas de polynôme annulateur.

\begin{normaltext}
    Dans le cas des opérateurs sur un espace de dimension finie (par exemple les matrices), il existe toujours un polynôme minimal, comme nous le verrons dans le lemme \ref{LEMooQJQGooRcAxmJ}.
\end{normaltext}

\begin{example}
    Le polynôme minimal dépend du corps sur lequel on le considère. Par exemple le nombre imaginaire pur \( i\) accepte \( X-i\) comme polynôme minimal sur \( \eC\) et \( X^2+1\) sur \( \eQ[X]\).
\end{example}

\begin{proposition}[\cite{MonCerveau}]  \label{PropRARooKavaIT}
    Soit \( \eL\) une extension de \( \eK\) et \( a\in \eL\) dont le polynôme minimal sur \( \eK\) est \( \mu_a\in\eK[X]\). Alors
    \begin{enumerate}
        \item   \label{ItemDOQooYpLvXri}
            le polynôme \( \mu_a\) est irréductible\footnote{Définition~\ref{DefIrredfIqydS}.} sur \( \eK\);
        \item
            Le polynôme \( \mu_a\) est premier\footnote{Définition~\ref{DefDSFooZVbNAX}.} avec tout polynôme de \( \eK[X]\) non annulateur de \( a\).
    \end{enumerate}
\end{proposition}

\begin{proof}
    Une chose à la fois.
    \begin{enumerate}
        \item
            D'abord le polynôme \( \mu_a\) n'est pas inversible parce que seuls les éléments de \( \eK\) (ceux de degré zéro) peuvent être inversibles\footnote{Et d'ailleurs, le sont, mais ce n'est pas important ici.}. Mais ces polynômes sont constants et ne peuvent donc pas être des polynômes annulateurs de quoi que ce soit.

            Ensuite, supposons la décomposition \( \mu_a=PQ\) avec \( P,Q\in \eK[X]\). En évaluant cette égalité en \( a\) nous avons
            \begin{equation}
                0=P(a)Q(a).
            \end{equation}
            Puisque nous sommes sur un corps, nous avons la règle du produit nul\footnote{Parce qu'un corps est un anneau intègre par le lemme \ref{LemAnnCorpsnonInterdivzer} et qu'un anneau intègre est justement un anneau sur lequel nous avons la règle du produit nul, voir la définition \ref{DEFooTAOPooWDPYmd}.} et nous déduisons que soit \( P(a)\) soit \( Q(a)\) est nul, ou les deux. Pour fixer les idées, nous supposons \( P(a)=0\).

            Dans ce cas, \( P\) fait partie de l'idéal annulateur de \( a\), lequel idéal est engendré par \( \mu_a\). Donc il existe \( S\in \eK[X]\) tel que \( P=S\mu_a\). En récrivant \( \mu_a=PQ\) avec cela nous avons :
            \begin{equation}
                \mu_a=S\mu_aQ
            \end{equation}
            ou encore : \( SQ=1\), ce qui signifie que \( S\) et \( Q\) sont dans \( \eK\) et inversibles.

            Nous concluons que \( \mu_a\) ne peut pas être écrit sous forme de produit de deux non inversibles.
        \item
            Soit \( Q\) un polynôme non annulateur de \( a\). Soit aussi un diviseur commun \( P\) de \( Q\) et \( \mu_a\) dans \( \eK[X]\). Nous devons prouver que \( P\) est un inversible, c'est-à-dire un élément de \( \eK\) (le fait que \( P\) ne soit pas le polynôme nul est évident).
            Nous avons \( \mu_a=PR_1\) et \( Q=PR_2\) pour certains polynômes \( R_1,R_2\in \eK[X]\). Puisque \( \mu_a\) est irréductible par~\ref{ItemDOQooYpLvXri}, il n'est pas produit de deux non inversibles. En d'autres termes, soit \( P\) soit \( R_1\) est inversible. Si \( P \) n'est pas inversible, alors \( R_1\) est inversible; disons \( R_1=k\in \eK\). Alors
            \begin{equation}
                0=\mu_a(a)=P(a)k,
            \end{equation}
            donc \( P(a)=0\). Mais alors
            \begin{equation}
                Q(a)=P(a)R_2(a)=0,
            \end{equation}
            ce qui est contraire à l'hypothèse selon laquelle \( Q\) n'était pas annulateur de \( a\).

            Nous retenons donc que \( P\) est inversible, ce qu'il fallait montrer.
    \end{enumerate}
\end{proof}

\begin{definition}
    Deux éléments \( \alpha\) et \( \beta\) dans \( \eL\) sont dit \defe{conjugués}{conjugués!éléments d'une extension} s'ils ont même polynôme minimal. Par exemple \( i\) et \( -i\) sont conjugués dans \( \eC\) vu comme extension de \( \eQ\).
\end{definition}

%---------------------------------------------------------------------------------------------------------------------------
\subsection{Extensions algébriques et éléments transcendants}
%---------------------------------------------------------------------------------------------------------------------------

%///////////////////////////////////////////////////////////////////////////////////////////////////////////////////////////
\subsubsection{Éléments algébriques et transcendants}
%///////////////////////////////////////////////////////////////////////////////////////////////////////////////////////////


%TODOooAAOWooWNqqVO. Dans LEMooLVPLooEkWYDN, LEMooTZSSooZmwYji, et DEFooREUHooLVwRuw il faut modifier les énoncés pour être explicite que
% une extension L de K est bien une application i:K->L; nous n'avons pas spécialement que K est un sous-ensemble de L.
% Ça ajoute des niveaux d'indirections, mais on est névrosé des abus de notations et on assume.

\begin{definition}      \label{DEFooBBYGooWoOloR}
    L'ensemble \( A[X]\) devient un \( \eK\)-espace vectoriel avec la définition
    \begin{equation}
        (\lambda P)_k=\lambda P_k.
    \end{equation}
\end{definition}

Voici une définition d'un élément algébrique sur un corps. Une caractérisation plus «pratique» sera donnée dans le lemme \ref{LEMooTZSSooZmwYji}.
\begin{lemmaDef}[Élément algébrique et transcendant\cite{ooTGTKooFenWAc}] \label{LEMooLVPLooEkWYDN}
    Soit une extension \( \eL\) de \( \eK\) et \( \alpha\in \eL\). Nous considérons l'application
    \begin{equation}
        \begin{aligned}
            \varphi\colon \eK[X]&\to \eL \\
            P&\mapsto P(\alpha).
        \end{aligned}
    \end{equation}
    Alors
    \begin{enumerate}
        \item
            L'application \( \varphi\) est un morphisme d'anneaux\footnote{Définition \ref{DEFooSPHPooCwjzuz}.}.
        \item
            L'application \( \varphi\) est un morphisme de \( \eK\)-espace vectoriel.
    \end{enumerate}
    Si \( \varphi\) est injective, nous disons que \( \alpha\) est \defe{transcendant}{transcendant}. Sinon, nous disons qu'il est \defe{algébrique}{algébrique}.
\end{lemmaDef}

\begin{proof}
    Le fait que \( \varphi\) soit un morphisme d'anneaux est le lemme \ref{PROPooGDQCooHziCPH} déjà prouvé.

    Pour le morphisme de \( \eK\)-espace vectoriel, il faut seulement ajouter le calcul
    \begin{equation}
        \varphi(\lambda P)=(\lambda P)(\alpha)=\lambda P(\alpha)=\lambda \varphi(P).
    \end{equation}
    Notons la justification suivante qui n'est pas tout à fait triviale :
    \begin{equation}
        (\lambda P)(\alpha)=\sum_k(\lambda P)_k\alpha^k=\sum_k\lambda P_k\alpha^k=\lambda P(\alpha)
    \end{equation}
    qui utilise la définition \ref{DEFooBBYGooWoOloR}.
\end{proof}

\begin{example}
    L'injectivité de \( \varphi\) n'est pas automatique. Prenons par exemple \( \eL=\eQ[\sqrt{ 2 }]\) dans \( \eR\). Les polynômes dans \( \eQ[X]\) ont des degrés arbitrairement élevés en \( X\), tandis que les éléments de \( \eL\) n'ont pas de degré très élevés en \( \sqrt{ 2 }\) parce que \( \sqrt{ 2 }\sqrt{ 2 }=2\). L'ensemble \( \eQ[\sqrt{ 2 }]\) ne contient donc que des éléments de la forme \( a+b\sqrt{ 2 }\) avec \( a,b\in \eQ\).

    Si par contre \( x_0\in \eR\) n'est racine d'aucun polynôme (cela existe parce que \( \eR\) n'est pas dénombrable), alors \( \eQ[x_0]\) contient tous les \( \sum_{k=0}^Na_kx_0^k\) avec \( N\) arbitrairement grand. Et tous ces nombres sont différents.
\end{example}

Le lemme suivant donne une caractérisation d'élément algébrique moins abstraite que la définition \ref{LEMooLVPLooEkWYDN}.
\begin{lemma}       \label{LEMooTZSSooZmwYji}
    Soit \( \eK\), un corps et \( \eL\), une extension de \( \eK\). Un élément \( \alpha\in \eL\) est algébrique sur \( \eK\) si et seulement si existe un polynôme non nul \( P\in \eK[X]\) tel que \( P(\alpha)=0\).
\end{lemma}

\begin{proof}
    Nous considérons l'application \( \varphi\) de la définition \ref{LEMooLVPLooEkWYDN}. Si \( \varphi\) n'est pas injective, c'est qu'il existe un polynôme \( P\) dans \( \eK[X]\) tel que \( \varphi(P)=0\). Dans ce cas, \( P(\alpha)=0\).

    À l'inverse si il existe \( P\) non nul dans \( \eK[X]\) tel que \( P(\alpha)=0\), alors \( \varphi(P)=0\) et \( \varphi\) n'est pas injective.
\end{proof}

\begin{definition}      \label{DEFooYZOYooAesmnP}
    Un corps \( \eK\) est \defe{algébriquement clos}{corps algébriquement clos} si tout polynôme non constant à coefficients dans \( \eK\) contient au moins une racine dans \( \eK\).
\end{definition}

Nous verrons dans le théorème de d'Alembert \ref{THOooIRJYooBiHRyW} que \( \eC\) est un corps algébriquement clos.

\begin{definition}[Extension algébrique, clôture algébrique]      \label{DEFooREUHooLVwRuw}
    Soient un corps \( \eK\) et une extension \( \alpha\colon \eK\to \eL\).
    \begin{enumerate}
        \item
            L'extension \( \eL\) est une extension \defe{algébrique}{extension algébrique} de \( \eK\) si tous ses éléments sont algébriques\footnote{Définition \ref{LEMooLVPLooEkWYDN}.} sur \( \eK\), c'est-à-dire sont racines de polynômes à coefficients dans  \(\alpha(\eK)\), voir le lemme \ref{LEMooTZSSooZmwYji}.
        \item       \label{ITEMooEIWVooVjJRoR}
            L'extension \( \eL\) est \defe{algébriquement close}{extension algébriquement clos} si le corps \( \eL\) est algébriquement clos (définition \ref{DEFooYZOYooAesmnP}).
        \item
            L'extension \( \eL\) est une \defe{clôture algébrique}{clôture algébrique} du corps \( \eK\) si elle est une extension algébrique qui est algébriquement close. 
    \end{enumerate}
\end{definition}

\begin{normaltext}
    Donc une extension est algébrique si elle contient seulement des racines de polynômes; elle est close si elle contient au moins une racine de chaque polynôme. L'extension est une clôture algébrique si elle est les deux en même temps.
\end{normaltext}

\begin{example}
    Le corps \( \eR\) n'est pas une extension algébrique de \( \eQ\). En effet il existe seulement une infinité \emph{dénombrable} de polynômes dans \( \eQ[X]\) et donc une infinité dénombrable de racines de tels polynômes. Toute extension algébrique de \( \eQ\) est donc dénombrable. Voir aussi la proposition \ref{PROPooVPQFooScWvkS}.
\end{example}

\begin{lemma}       \label{LEMooEYRSooUREeDl}
    Un corps est algébriquement clos si et seulement si tous ses polynômes sont scindés\footnote{Définition \ref{DefCPLSooQaHJKQ}}.
\end{lemma}

\begin{proof}
    Si tout polynôme est scindé, tout polynôme possède des racines; c'est l'autre sens qui est plus consistant.

    Soit un corps algébriquement clos \( \eK\). Nous allons effectuer une récurrence sur le degré des polynômes. Si \( P\) est un polynôme de degré \( 1\), alors il est scindé.
    
    Nous supposons que tous les polynômes de degré \( n-1\) sont scindés. Soit un polynôme \( P\) de degré \( n\). Le corps étant algébriquement clos, le polynôme \( P\) a une racine que nous notons \( a_n\in \eK\). La proposition \ref{PROPooQCZSooVokxXQ} nous explique qu'il existe un polynôme \( Q\) de degré \( n-1\) tel que \( P=(X-a_n)Q\).

    Par hypothèse de récurrence, le polynôme \( Q\) est scindé : il existe \( \{ a_i \}_{i=1,\ldots, n-1}\) dans \( \eK\) tels que \( Q=\prod_{i=1}^{n-1}(X-a_i)\). Au final,
    \begin{equation}
        P=(X-a_n)Q=\prod_{k=1}^n(X-a_k)
    \end{equation}
    et \( P\) est scindé.
\end{proof}

\begin{lemma}       \label{LEMooYVHKooWhewKp}
    Soient un corps \( \eK\) et un polynôme \( P\in \eK[X]\). Nous supposons que \( P\) est scindé :
    \begin{equation}
        P=\prod_{k=1}^n(X-a_k).
    \end{equation}
    Si \( \alpha\) est une racine de \( P\), alors \( \alpha\) est l'un des \( a_k\).
\end{lemma}

\begin{proof}
    Dire que \( \alpha\) est une racine de \( P\) revient à dire que
    \begin{equation}        \label{EQooXHSCooVdMiva}
        \prod_{k=1}^n(\alpha-a_k)=0
    \end{equation}
    Un corps est toujours un anneau intègre (lemme \ref{LemAnnCorpsnonInterdivzer}), c'est-à-dire que la règle du produit nul est utilisable. Dans notre cas, le produit nul \eqref{EQooXHSCooVdMiva} nous indique que \( \alpha-a_k=0\) pour (au moins) un des \( k\). Donc effectivement \( \alpha\) est l'un des \( a_k\).
\end{proof}

\begin{lemma}
    Soient un corps algébriquement clos \( \eK\) ainsi qu'une extension algébrique \( \alpha\colon \eK\to \eL\). Alors \( \alpha(\eK)=\eL\).
\end{lemma}

\begin{proof}
    Nous allons montrer que tous les éléments de \( \eL\) sont dans l'image de \( \alpha\). Soit donc \( l\in \eL\). Puisque l'extension \( \alpha\colon \eK\to \eL\) est une extension algébrique, il existe un polynôme \( P\in \alpha(\eK)[X]\) tel que \( P(l)=0\).

    Étant donné que \( \alpha\) est injective, il est possible de considérer le polynôme \( Q=\alpha^{-1}(P)\), c'est-à-dire que, si \( P=\sum_ka_kX^k\), nous posons \( Q=\sum_k\alpha^{-1}(a_k)X^k\).

    Le corps \( \eK\) étant algébriquement clos, le polynôme \( Q\) est scindé (proposition \ref{LEMooEYRSooUREeDl}) :
    \begin{equation}
        Q=\prod_{k=1}^n(X-b_k)
    \end{equation}
    avec \( b_k\in \eK\). Nous avons alors aussi la factorisation
    \begin{equation}
        P=\prod_{k=1}^n(X-\alpha(b_k))
    \end{equation}
    dans \( \eL[X]\). Nous avons vu que \( l\) était une racine de \( P\). Donc \( l\) est un des \( \alpha(b_k)\) (lemme \ref{LEMooYVHKooWhewKp}). Cela prouve que \( l\in \alpha(\eK)\).
\end{proof}

%--------------------------------------------------------------------------------------------------------------------------- 
\subsection{Extension algébrique et polynôme minimal}
%---------------------------------------------------------------------------------------------------------------------------

\begin{proposition}[\cite{ooLIOMooBuCPUS}]      \label{PROPooALFJooDjmIcb}
    Soit une extension algébrique\footnote{Définition \ref{DEFooREUHooLVwRuw}.} \( \eL\) du corps $\eK$.
    \begin{enumerate}
        \item
            Pour tout \( a\in \eL\), il existe un polynôme \( P\in \eK[X]\) tel que \( P(a)=0\).
        \item       \label{ITEMooEFNFooKYqXDk}
            Le polynôme minimal de \( a\) dans \( \eK[X]\) est l'unique polynôme unitaire irréductible annulant \( a\).
    \end{enumerate}
\end{proposition}
\index{polynôme!minimal}

\begin{proof}
    Le premier point est seulement la définition~\ref{DEFooREUHooLVwRuw} d'une extension algébrique.

    L'idéal annulateur \( I_a=\{ P\in \eK[X]\tq P(a)=0 \}\) n'est pas réduit à \( \{ 0 \}\) parce que \( \eL\) est une extension algébrique. L'existence du polynôme minimal est le lemme~\ref{DefCVMooFGSAgL} et le fait qu'il soit irréductible est la proposition~\ref{PropRARooKavaIT}\ref{ItemDOQooYpLvXri}.

    Ce qui nous intéresse ici est l'unicité. Soit \( \mu_1\in \eK[X]\), un polynôme annulateur de \( a\) irréductible et unitaire. Puisque \( \mu_1\in I_a\) et que par définition, \( I_a=(\mu)\), il existe \( P\in \eK[X]\) tel que \( \mu_1=P\mu\). Comme \( \mu\) n'est pas inversible et que \( \mu_1\) est irréductible, \( P\) doit être inversible : \( \mu_1=k\mu\) pour un certain \( k\in \eK\).

    Puisque \( \mu\) et \( \mu_1\) sont unitaires, \( k=1\). Donc \( \mu_1=\mu\).
\end{proof}

\begin{lemma}       \label{LEMooHKTMooKEoOuK}
    Soient un corps \( \eK\), une extension \( \eL\) de \( \eK\) et \( \alpha\in \eL\), un élément algébrique\footnote{Définition \ref{LEMooLVPLooEkWYDN}.} sur \( \eK\). Si \( \mu\) est le polynôme minimal de \( \alpha\) sur \( \eK\) alors
    \begin{equation}
        \begin{aligned}
            \varphi\colon \eK[\alpha]&\to \eK[X]/(\mu) \\
            Q(\alpha)&\mapsto \bar Q
        \end{aligned}
    \end{equation}
    avec \( Q\in\eK[X]\) est un isomorphisme de corps et de \( \eK\)-espaces vectoriels.
    % position 118885898.
\end{lemma}

\begin{proof}
    D'abord, \( \alpha\) est algébrique, donc l'idéal annulateur \( I_{\alpha}\) n'est pas réduit à \( \{ 0 \}\), et l'existence d'un polynôme minimal est assurée par le lemme~\ref{DefCVMooFGSAgL}.

    Ensuite, le fait que \( \eK[X]/(\mu)\) soit un corps est le corolaire~\ref{CorsLGiEN}. Nous montrons à présent que \( \varphi\) est un isomorphisme (d'anneaux); cela suffit pour en déduire que \( \eK[\alpha]\) est également un corps.

    Ces préliminaires étant dits, nous commençons.
    \begin{subproof}
        \item[Bien définie]
            Nous devons prouver que \( \varphi\) est bien définie, c'est-à-dire que tout élément de \( \eK[\alpha]\) peut être écrit sous la forme \( Q(\alpha)\) pour un \( Q\in \eK[X]\), et que si \( Q_1(\alpha)=Q_2(\alpha)\) alors \( \bar{Q_1}=\bar{Q_2} \).

            Le fait que tous les éléments de \( \eK[\alpha]\) peuvent être écrits sous la forme \( Q(\alpha)\) est la proposition~\ref{PROPooPMNSooOkHOxJ}. Supposons que \( Q_1(\alpha)=Q_2(\alpha)\). Alors nous définissons \( R\in \eK[X]\) par \( Q_1=Q_2+R\), et en évaluant cette égalité en \( \alpha\) nous avons
            \begin{equation}
                Q_1(\alpha)=Q_2(\alpha)+R(\alpha),
            \end{equation}
            autrement dit \( R(\alpha)=0\). Donc \( R\) est dans l'idéal annulateur de \( \alpha\) et est donc dans \( (\mu)\), c'est-à-dire que dans le quotient \( \eK[X]/(\mu)\) nous avons \( \bar R=0\) et donc \( \bar{Q_1}=\bar{Q_2} \).

        \item[Surjective]

            Tout élément de \( \eK[X]/(\mu)\) est de la forme \( \bar Q\) pour un \( Q\in \eK[X]\). Or ces éléments sont ceux de l'ensemble d'arrivée de \( \varphi\).

        \item[Injective]

            Si \( \bar{Q_1}=\bar{Q_2} \), alors \( Q_1=Q_2+R\) avec \( R\) dans l'idéal engendré par \( \mu\), c'est-à-dire entre autres \( R(\alpha)=0\). Donc \( Q_1(\alpha)=Q_2(\alpha)\).

    \end{subproof}

    Nous devons encore montrer que nous avons là un morphisme de \( \eK\)-espaces vectoriels.
    \begin{enumerate}
        \item
            Si \( k\in \eK\) alors \( \varphi\big( kQ(\alpha) \big)=\overline{ kQ }\). Mais par définition de la structure d'espace vectoriel sur \( \eK[X]/(\mu)\), \( \overline{ kQ }=k\bar Q\) (vérifier que cette définition de la multiplication par un scalaire sur \( \eK[X]/(\mu)\) est correcte).
        \item
            Nous avons aussi \( \varphi\big( Q_1(\alpha)+Q_2(\alpha) \big)=\varphi\big( (Q_1+Q_2)(\alpha) \big)=\overline{ Q_1+Q_2 }=\bar{Q_1} +\bar{Q_2} \).
    \end{enumerate}
\end{proof}


%---------------------------------------------------------------------------------------------------------------------------
\subsection{Extensions et polynômes}
%---------------------------------------------------------------------------------------------------------------------------

Nous savons déjà depuis la définition~\ref{DEFooFYZRooMikwEL} ce qu'est \( A[X]\) pour tout anneau \( A\) et donc, à fortiori, pour un corps.

\begin{definition}  \label{DEFooQPZIooQYiNVh}
    Soit un corps commutatif\footnote{Sauf mention du contraire, tous les corps du Frido sont commutatifs.}. Nous notons \( \eK(X)\) le corps des fractions\footnote{Définition~\ref{DEFooGJYXooOiJQvP}.} de \( \eK[X]\).
\end{definition}

\begin{lemmaDef}        \label{DEFooZHBZooKlNfGZ}
    Si \( R\in \eK(X)\), avec \( R=P/Q\) et si \( \eL\) est une extension\footnote{Définition \ref{DEFooFLJJooGJYDOe}.} de \( \eK\) contenant l'élément \( \alpha\), alors nous définissons
    \begin{equation}
        R(\alpha)=P(\alpha)Q(\alpha)^{-1}.
    \end{equation}
    Cela est une bonne définition au sens où elle ne dépend pas du choix du représentant \( (P,Q)\) pris dans la classe \( P/Q\).
\end{lemmaDef}

\begin{proof}
    Supposons \( R=P_1/Q_1=P_2/Q_2\). Par définition des classes (définition~\ref{DEFooGJYXooOiJQvP}) nous avons
    \begin{equation}        \label{EQooKHVNooABuHaO}
        P_1Q_2=Q_1P_2.
    \end{equation}
    Puisque l'évaluation est un morphisme \( \eK[X]\to\eK\) \footnote{Lemme~\ref{PROPooGDQCooHziCPH}.
    % laisser ce saut de ligne
    Certes ce lemme ne parle que d'anneaux, mais à y bien penser, dans le passage de \eqref{EQooKHVNooABuHaO} à \eqref{EQooJAIGooRADgiD}, nous ne considérons que les structures d'anneaux sur \( \eK[X]\) et \( \eK\).} nous pouvons évaluer l'équation \eqref{EQooKHVNooABuHaO} en \( \alpha\) :
    \begin{equation}        \label{EQooJAIGooRADgiD}
        P_1(\alpha)Q_2(\alpha)=Q_1(\alpha)P_2(\alpha).
    \end{equation}
    Cette dernière est une égalité dans le corps \( \eK\). Nous pouvons donc la multiplier par \( Q_2(\alpha)^{-1}P_2(\alpha)^{-1}\) (et utiliser toutes les hypothèses de commutativité des anneaux et corps) pour obtenir
    \begin{equation}
        P_1(\alpha)Q_1(\alpha)^{-1}=P_2(\alpha)Q_2(\alpha)^{-1},
    \end{equation}
    c'est-à-dire
    \begin{equation}
        (P_1/Q_1)(\alpha)=(P_2/Q_2)(\alpha).
    \end{equation}
\end{proof}

\begin{propositionDef}[\cite{MonCerveau}]  \label{DEFooVSKGooMyeGel}
    Soient un corps \( \eK\), une extension \( (\eL,j_{\eL})\) de \( \eK\) et un élément \( \alpha\in\eL\). Nous définissons \( \eK(\alpha)_{\eL} \) comme étant l'intersection de tous les sous-corps de \( \eL\) contenant \( j_{\eL}(\eK)\) et \( \alpha\).

    Alors
    \begin{enumerate}
        \item
            \( \eK(\alpha)_{\eL}\) est un sous-corps de \( \eL\),
        \item
            \( \eK(\alpha)_{\eL}\) est une extension\footnote{Définition \ref{DEFooFLJJooGJYDOe}.} de \( \eK\).
    \end{enumerate}
\end{propositionDef}

\begin{proof}
    Nous commençons par prouver que \( \eK(\alpha)_{\eL}\) est bien un corps. Si \( a,b\in \eK(\alpha)_{\eL}\) alors il suffit de calculer \( ab\), \( a+b\) et \( a^{-1}\) dans n'importe quel sous-corps de \( \eL\) contenant \( \eK\) et \( \alpha\); nous avons une garantie que \( a\), \( b\), \( ab  \), \( a+b\) et \( a^{-1}\) sont dans tous les tels sous-corps.

    Pour prouver que \( \eK(\alpha)_\eL\) est bien une extension, nous devons trouver un morphisme de corps \( j\colon \eK\to \eK(\alpha)_{\eL}\). On constate que prendre \( j=j_{\eL}\) fonctionne parce que par définition, \( \eK(\alpha)_{\eL}\) est une partie de \( \eL\) contenant l'image de \( j_{\eL}\).
\end{proof}

\begin{lemma}       \label{LEMooHZLCooPLHkLS}
    Soit \( n\) tel que \( \sqrt{ n }\) ne soit pas un rationnel. Si \( \alpha\in \{ a+b\sqrt{ n } \}_{a,b\in \eQ}\), alors il existe un unique choix \( (x,y)\in \eQ^2\) tel que
    \begin{equation}
        \alpha=x+y\sqrt{ n }.
    \end{equation}
\end{lemma}

\begin{proof}
    L'existence est dans la définition de \( \alpha\). Il s'agit de voir l'unicité. Supposons \( x+y\sqrt{ n }=a+b\sqrt{ n }\) avec \( x,y,a,b\in \eQ\). Si \( b\neq y\) nous pouvons écrire
    \begin{equation}
        \sqrt{ n }=\frac{ x-a }{ b-y }.
    \end{equation}
    Comme \( \sqrt{ n }\) n'est pas un rationnel, une telle écriture est impossible. Donc \( b=y\). Nous avons alors \( x+y\sqrt{ n }=a+y\sqrt{ n }\) et du coup aussi \( x=a\).
\end{proof}

\begin{example}
    Nous avons 
    \begin{equation}
        \eQ(\sqrt{ 2 })_{\eR}=\{ a+b\sqrt{ 2 } \}_{a,b\in \eQ}
    \end{equation}
    où à droite nous calculons les sommes et les produits dans \( \eR\). Le tout est un sous-ensemble de \( \eR\) qui se révèle être un corps contenant \( \eQ\) et \( \sqrt{ 2 }\).

    En particulier, dans \( \eQ(\sqrt{ 2 })_{\eR}\) nous avons \( \sqrt{ 2 }\sqrt{ 2 }=2\).
\end{example}

\begin{lemma}   \label{LEMooKVPZooPqPrce}
    Les corps \( \eQ(\sqrt{ 2 })_{\eR}\) et \( \eQ(\sqrt{ 3 })_{\eR}\) ne sont pas isomorphes.
\end{lemma}

\begin{proof}
    Supposons l'existence d'un morphisme de corps\footnote{Définition \ref{DEFooSPHPooCwjzuz}. Oui, c'est un bête morphisme d'anneaux. Il n'y a pas plus de structure dans un corps que dans un anneau.}
    \begin{equation}
        \psi\colon \eQ(\sqrt{ 2 })_{\eR}\to \eQ(\sqrt{ 3 })_{\eR}.
    \end{equation}
    Nous notons «\( 1\)» à la fois le neutre de la multiplication dans \( \eQ(\sqrt{ 2 })_{\eR}\) et \( \eQ(\sqrt{ 3 })_{\eR}\) (qui s'avèrent être les mêmes en tant qu'élément de \( \eR\), mais ça n'a pas d'importance ici).

    Soit \( \alpha\in \eQ(\sqrt{ 2 })_{\eR}\) tel que \( \alpha^2-1=0\). Alors nous avons aussi
    \begin{equation}
        \psi(\alpha)^2-1=\psi(\alpha^2)-\psi(1)=\psi(\alpha^2-1)=\psi(0)=0.
    \end{equation}
    Donc \( \psi(\alpha)\) est un élément de \( \eQ(\sqrt{ 3 })_{\eR}\) qui est une racine de \( X^2-1\).

    Or un tel élément n'existe pas dans \( \eQ(\sqrt{ 3 })_{\eR}\) parce que nous savons que dans \( \eR\) entier, il n'y a que deux racines : \( \pm\sqrt{ 2 }\), et aucune des deux n'est dans \( \eQ(\sqrt{ 3 })_{\eR}\).
\end{proof}

\begin{example}      \label{EXooJRSUooYhAZkR}
    Est-ce que \( \eK(\alpha)_{\eL}\) dépend réellement de \( \eL\) ? Si \( \eL_2\) est une extension de \( \eL\) alors nous avons évidemment\footnote{Vérifiez-le tout de même.} \( \eK(\alpha)_{\eL_2}=\eK(\alpha)_{\eL}\).

    Nous commençons par construire un corps \( \eK\) un peu idiot qui, comme ensemble, est comme \( \eQ(\sqrt{ 2 })_{\eR}\), c'est-à-dire la partie
    \begin{equation}
        \{ a+b\sqrt{ 2 } \}_{a,b\in \eQ},
    \end{equation}
    de \( \eR\).

    Mais cette fois nous définissons la multiplication suivante :
    \begin{equation}
        (a+b\sqrt{ 2 })(c+d\sqrt{ 2 })=ac+3bd+(ad+bc)\sqrt{ 2 }.
    \end{equation}
    C'est un corps parce que tout élément non nul est inversible. En effet, l'équation
    \begin{equation}        \label{EQooIZLEooLPOBcC}
        (a+b\sqrt{ 2 })(x+y\sqrt{ 2 })=1
    \end{equation}
    donne 
    \begin{equation}
        \begin{pmatrix}
            a    &   3b    \\ 
            b    &   a    
        \end{pmatrix}\begin{pmatrix}
            x    \\ 
            y    
        \end{pmatrix}=\begin{pmatrix}
            1    \\ 
            0    
        \end{pmatrix}.
    \end{equation}
    Ce système a une unique solution si et seulement si \( \det\begin{pmatrix}
        a    &   3b    \\ 
        b    &   a    
    \end{pmatrix}=0\). Cela survient si et seulement si
    \begin{equation}
        a^2-3b^2=0.
    \end{equation}
    Les solutions de cette équation dans \( \eR\) sont \( a=\pm\sqrt{ 3 }| b |\). Dès que \( a\) ou \( b\) est non nul, cela ne peut pas satisfaire \( a,b\in \eQ\). Donc le déterminant est toujours non nul et il existe \( x,y\in \eQ\) tels que \eqref{EQooIZLEooLPOBcC} soit satisfaite.

    Tout cela nous a donné un corps \( \eK\) dont \( \eQ\) est un sous-corps et qui contient l'élément \( \sqrt{ 2 }\) de \( \eR\). Il n'est cependant pas un sous-corps de \( \eR\).

    Ce corps est isomorphe à \( \eQ(\sqrt{ 3 })_{\eR}\). En effet, nous montrons que
    \begin{equation}
        \begin{aligned}
            \psi\colon \eK  & \to \eQ(\sqrt{ 3 })_{\eR} \\
            a+b\sqrt{ 2 }   & \mapsto a+b\sqrt{ 3 } 
        \end{aligned}
    \end{equation}
    est un isomorphisme de corps. Pour le produit, nous avons
    \begin{subequations}
        \begin{align}
            \psi\big( (a+b\sqrt{ 2 })(c+d\sqrt{ 2 }) \big)&=\psi\big( ac+3bd+(ad+bc)\sqrt{ 2 } \big) \label{SUBEQooQSZBooHZDTKo}\\
            &=ac+3bd+(ad+bc)\sqrt{ 3 }\label{SUBEQooPEKHooNPcIjE}\\
            &=(a+b\sqrt{ 3 })(c+d\sqrt{ 3 })\label{SUBEQooIGBZooMwrmFe}\\
            &=\psi(a+b\sqrt{ 2 })\psi(c+d\sqrt{ 2 }).
        \end{align}
    \end{subequations}
    Remarques :
    \begin{itemize}
        \item L'application \( \psi\) est bien définie grâce au lemme \ref{LEMooHZLCooPLHkLS} couplé au théorème \ref{THOooYXJIooWcbnbm} appliqué à \( n=2\) et \( n=3\).
        \item Dans le membre de gauche de \eqref{SUBEQooQSZBooHZDTKo}, \( b\sqrt{ 2 }\) est un produit dans \( \eR\) (d'où l'importance du lemme \ref{LEMooHZLCooPLHkLS} qui permet de re-séparer les éléments de \( \eR\) partie rationnelle et partie multiple de \( \sqrt{ 2 }\)), et le produit entre \( (a+b\sqrt{ 2 })\) et \( (c+d\sqrt{ 2 })\) est un produit dans \( \eK\).
        \item
            Dans \eqref{SUBEQooPEKHooNPcIjE} et \eqref{SUBEQooIGBZooMwrmFe}, tous les produits sont dans \( \eR\).
    \end{itemize}

    En comparant avec le lemme \ref{LEMooKVPZooPqPrce}, nous avons alors
    \begin{equation}
        \eQ(\sqrt{ 2 })_\eK=\eQ(\sqrt{ 3 })_\eR\neq \eQ(\sqrt{ 2 })_{\eR}
    \end{equation}
\end{example}

\begin{normaltext}
    Nous allons encore enfoncer le clou sur le fait que \( \eK(\alpha)_{\eL}\) dépend de \( \eL\).

    Le fait est que si on y pense, l'objet \( \sqrt{ 2 }\) n'a aucun rapport avec \( \eQ\). En effet les objets de \( \eQ\) sont des classes d'équivalence de couples d'éléments de \( \eZ\), alors que l'élément \( \sqrt{ 2 }\) est une classe d'équivalence de suites de Cauchy dans \( \eQ\).

    Lorsque nous écrivons \( \eQ(\sqrt{ 2 })\), nous associons des objets de nature complètement différentes, et il n'y a aucune raison à priori de définir la multiplication entre eux d'une façon plutôt qu'une autre.

    Plus généralement, dans ZF (nous faisons semblant de suivre ZF tout en sachant que nous ne savons pas ce que c'est réellement\footnote{En lisant quelques pages de Wikipédia, vous pourrez briller en société, mais ne tentez pas le coup à l'agrégation.}), tout est ensemble. Peut-on dire ce que serait \( \eQ(I)\) si \( I\) est un ensemble quelconque ? Attention : en écrivant \( \eQ(I)\), nous entendons un corps dont \( I\) est un élément, pas un corps qui contiendrait comme éléments tous les éléments de \( I\).

    Si \( I\) est juste un ensemble, quelle définition donner de \( I^2\) ? Il y a plein de choix et rien ne se dégage clairement comme étant pertinent. Si par contre, en guise de \( I\) nous considérons l'ensemble \( \sqrt{ 2 }\) (oui, c'est un ensemble : un ensemble de suites de Cauchy dans \( \eQ\)), alors tout de suite nous nous disons que la bonne façon de faire est \( \sqrt{ 2 }^2=2\). Ce réflexe est juste conditionné par le fait que nous connaissons déjà par ailleurs le corps \( \eR\). Rien de plus.

    Donc oui, \( \eK(\alpha)_{\eL}\) dépend de \( \eL\), mais dans les cas particuliers où \( \eK\) est un sous-corps de \( \eC\), il y a une égalité implicite \( \eL=\eC\). Cela étant dit, il n'y a plus d'ambiguïté en écrivant \( \eQ(\sqrt{ 2 })\).
\end{normaltext}

\begin{definition}  \label{DefZCYIbve}
    Soit une extension\footnote{Définition \ref{DEFooFLJJooGJYDOe}.} de corps \( j\colon \eK\to \eL\). Soit \( A\subset \eL\).
    \begin{enumerate}
        \item       \label{ITEMooJEGUooMsDBhF}
            Nous notons \( \eK(A)_{\eL}\)\nomenclature[A]{$\eK(A)$}{corps contenant $\eK$ et $A$} le plus petit sous-corps de \( \eL\) contenant \( j(\eK)\) et \( A\). C'est l'intersection de tous les sous-corps de \( \eL\) contenant \( A\) et \( j(\eK)\).
        \item
            Nous notons \( \eK[A]_{\eL}\)\nomenclature[A]{$\eK[A]$}{anneau contenant $ \eK$ et $ A$} le plus petit sous-anneau de \( \eL\) contenant \( j(\eK)\) et \( A\). C'est l'intersection de tous les sous-anneaux de \( \eL\) contenant \( A\) et \( j(\eK)\).
    \end{enumerate}

    Nous disons que l'extension \( \eL\) de \( \eK\) est \defe{monogène}{monogène!extension de corps} ou \defe{\wikipedia{fr}{Extension_simple}{simple}}{extension!de corps!simple}\index{simple!extension de corps} si il existe \( \theta\in\eL\) tel que \( \eL=\eK(\theta)\). Un tel élément \( \theta\) est dit \defe{élément primitif}{primitif!élément d'une extension de corps} de \( \eL\). Il n'est pas nécessairement unique.
\end{definition}
    Le plus souvent, l'indice \( \eL\) dans \( \eK(A)_{\eL}\) et \( \eK[A]_{\eL}\) est omis parce que le contexte est clair\quext{Et je me demande si il est possible de trouver un cas tordu où \( \eK(A)_{\eL}\neq \eK(A)_\eM\). Par exemple lorsque \( A\) est dans \( \eL\) et \( \eM\), mais que \( \eL\) n'est pas inclus dans \( \eM\), ni \( \eM\) dans \( \eL\).}, et nous avons même très souvent \( \eK\subset \eL\) en tant qu'ensembles. Dans ce cas, l'application \( j\) est l'identité et elle sera omise.

\begin{remark}
    Les ensembles \( \eK(A)\) et \( \eK[A]\) sont aussi appelés respectivement corps \defe{engendré}{engendré!corps, extension} et anneau engendré par \( A\). Cependant il faut bien remarquer que ce sont les parties de \( \eL\) engendrées par \( A\). Il n'est pas question à priori de parler de corps engendré par \( A\) sans dire dans quel corps plus grand nous nous plaçons.
\end{remark}

\begin{example}
    Nous savons que \( \eR\) est une extension de \( \eQ\). Si \( a\in \eR\) alors \( \eQ(a)\) est le plus petit corps contenant \( \eQ\) et \( a\).
\end{example}

\begin{example}
    Nous avons déjà vu à l'occasion de la définition~\ref{DEFooFYZRooMikwEL} que \( A[X]\) est l'anneau de tous les polynômes de degré fini en \( X\). Cela rentre dans le cadre de la définition~\ref{DefZCYIbve} parce qu'un anneau contenant \( X\) doit contenir tous les \( X^n\).

    Notons que même si \( \eK\) est un corps, \( \eK[X]\) reste un anneau parce qu'un éventuel inverse de \( X\) n'est pas dedans\footnote{Lorsqu'on multiplie, les degrés montent toujours.}. Par contre, \( \eK(X)\) est un corps parce qu'il contient également les fractions rationnelles.
\end{example}

\begin{example} \label{ExLQhLhJ}
    Si nous prenons \( \eF_5\) et que nous l'étendons par \( i\), nous obtenons le corps \( \eK=\eF_5(i)\). Nous savons que tous les éléments \( a\in \eF_5\) sont racines de \( X^5-X\). Mais étant donné que \( i^5=i\), nous avons aussi \( x^5=x\) pour tout \( x\in \eF_5(i)\). Pour le prouver, utiliser le morphisme de Frobenius. Le polynôme \( X^5-X\) est donc le polynôme nul dans \( \eK\).

    Ceci est un cas très particulier parce que nous avons étendu \( \eF_p\) par un élément \( \alpha\) tel que \( \alpha^p=\alpha\). En général sur \( \eF_p(\alpha)\), le polynôme \( X^p-X\) n'est pas identiquement nul, et possède donc au maximum \( p\) racines. Pour \( x\in \eF_p(\alpha)\), nous avons \( x^p=x\) si et seulement si \( x\in \eF_p\).
\end{example}

Dans l'énoncé suivant, la notation \( R(\alpha)_{\eL}\) signifie que l'évaluation de \( R\) sur \( \alpha\) se fait en calculant dans le sur-corps \( \eL\) de \( \eK\).  Cette proposition semble indiquer que \( \eK(\alpha)\) est donné en termes de \( \eK(X)\), lequel est défini de façon très intrinsèque sans faire appel implicitement à un sur-corps de \( \eK\).

\begin{proposition}[\cite{MonCerveau}]     \label{PROPooYSFNooFGbbCi}
    Soit une extension \( \eL\) du corps \( \eK\) et \( \alpha\in \eL\). Alors nous avons les isomorphismes de corps suivants :
    \begin{enumerate}
        \item
            \( \eK(\alpha)_{\eL}=\Frac\big( \eK[\alpha]_{\eL} \big)\),
        \item       \label{ITEMooATPTooVXKdlK}
            \( \eK(\alpha)_{\eL}=\{ R(\alpha)_{\eL}\tq R\in \eK(X) \}\).
    \end{enumerate}
\end{proposition}

\begin{proof}
    Le corps \( \eK(\alpha)\) est un sous-corps de \( \eL\) contenant \( \eK[\alpha]\) comme sous-anneau. La proposition~\ref{PROPooIJBEooDjsoHr} nous dit alors que l'application suivante est un morphisme injectif de corps :
    \begin{equation}
        \begin{aligned}
            \epsilon\colon \Frac\big( \eK[\alpha] \big)&\to \eK(\alpha) \\
            P/Q&\mapsto PQ^{-1}.
        \end{aligned}
    \end{equation}
    Pour rappel, la notation \( P/Q\) est bien une notation pour la classe d'équivalence du couple \( (P,Q)\) pour la relation définie en~\ref{DEFooGJYXooOiJQvP}.

    Par ailleurs, la partie \( \epsilon\Big( \Frac\big( \eK[\alpha] \big) \Big) \) de \( \eL\) et est un corps contenant \( \eK\) et \( \alpha\). Donc ce corps fait partie des corps sur lesquels on prend l'intersection pour définir \( \eK(\alpha)\)\footnote{Pour rappel, la définition \ref{DefZCYIbve}\ref{ITEMooJEGUooMsDBhF} donne \( \eK(\alpha)\) comme une intersection.}. Cela prouve que
    \begin{equation}
        \eK(\alpha)\subset  \epsilon\Big( \Frac\big( \eK[\alpha] \big) \Big).
    \end{equation}
    L'application \( \epsilon\) est donc surjective sur \( \eK(\alpha)\). Comme elle était déjà injective, elle est bijective.

    Pour la seconde partie, veuillez lire la définition~\ref{DEFooLBIWooCPCaSY} de l'évaluation d'une fraction rationnelle sur un élément de l'anneau. Si \( R=P/Q\in \eK(X)\) et si \( \alpha\in \eL\), nous avons
    \begin{equation}
        R(\alpha)=P(\alpha)Q(\alpha)^{-1}.
    \end{equation}
    Tout sous-corps de \( \eL\) contenant \( \eK\) et \( \alpha\) doit contenir en particulier \( \{ P(\alpha)\tq P\in \eK[X] \} \), les inverses \( \{ P(\alpha)^{-1}\tq P\in \eK[X],\,P(\alpha)\neq 0 \}\) et les produits de ceux-ci. Donc tout sous-corps de \( \eL\) contenant \( \eK\) et \( \alpha\) contient \( \{ R(\alpha)\tq R\in \eK(X) \}\).

    Nous avons donc
    \begin{equation}
        \{ R(\alpha)\tq R\in \eK(X) \}\subset \eK(\alpha).
    \end{equation}
    Mais puisque \( \eK(\alpha)\) est lui-même un sous-corps de \( \eL\) contenant \( \eK\) et \( \alpha\), il est contenu dans \( \{ R(\alpha)\tq R\in \eK(X) \}\). D'où l'égalité.
\end{proof}

Pourquoi cela ne contredit pas l'exemple~\ref{EXooJRSUooYhAZkR} ? Lorsque nous écrivons
\begin{equation}
    \eK(\alpha)=\{ R(\alpha)\tq R\in \eK(X) \},
\end{equation}
certes \( \eK(X)\) est défini sans faire appel à un corps contenant \( \eK\). Mais l'évaluation \( R(\alpha)\), oui. Pour calculer \( R(\alpha)\), il faut écrire \( R=P/Q\) et calculer \( P(\alpha)Q(\alpha)^{-1}\). Tous les calculs de cette dernière expression doivent se faire dans un sur-corps de \( \eK\). Il suffit que le sur-corps en question soit un monceau de mauvaise foi comme celui de l'exemple~\ref{EXooJRSUooYhAZkR}, et en réalité \( \eK(\alpha)\) peut ne pas être ce que l'on croit.

Le corolaire suivant montre que les choses s'arrangent.

\begin{corollary}
    Soient un corps \( \eK\), une extension \( \eL_1\) de \( \eK\), un élément \( \alpha\in \eL_1\) et une extension \( \eL_2\) de \( \eL_1\). Alors
    \begin{equation}
        \eK(\alpha)_{\eL_1}=\eK(\alpha)_{\eL_2}.
    \end{equation}
\end{corollary}

\begin{proof}
    La proposition~\ref{PROPooYSFNooFGbbCi} nous dit que
    \begin{subequations}
        \begin{align}
            \eK(\alpha)_{\eL_1} &=\{ R(\alpha)_{\eL_1}\tq R\in\eK(X) \} \\
            \eK(\alpha)_{\eL_2} &=\{ R(\alpha)_{\eL_2}\tq R\in\eK(X) \}.
        \end{align}
    \end{subequations}
    Mais lorsque \( R\in \eK(X)\), le calcul de \( R(\alpha)\) est exactement le même dans \( \eL_1\) et dans \( \eL_2\) parce que \( \eL_2\) est un sur-corps de \( \eL_1\) et que les calculs effectifs de \( R(\alpha)=P(\alpha)Q(\alpha)^{-1}\) ne font intervenir que des quantités de \( \eK\) et des puissances de \( \alpha\).
\end{proof}

Ce que ce corolaire nous dit est que si le contexte fixe une extension de \( \eK\), nous pouvons faire tous les calculs dans cette extension, même si il y a des piles d'extensions à côté.

Typiquement, à chaque fois que nous considérons des sous-corps de \( \eC\), les extensions se feront dans \( \eC\) : pour tout \( \alpha\in \eC\), les corps \( \eQ(\alpha)\), \( \eR(\alpha)\) se calculent dans \( \eC\).


\begin{proposition}     \label{PROPooSYQWooFbfQtm}
    Soit un corps \( \eK\), une extension \( \eL\) et un élément \( \alpha\in \eL\). Nous considérons l'application
    \begin{equation}
        \begin{aligned}
            \varphi\colon \eK[X]&\to \eL \\
            P&\mapsto P(\alpha).
        \end{aligned}
    \end{equation}
    \begin{enumerate}
        \item       \label{ITEMooUZDQooOasiRQ}
            Si \( \alpha\) est transcendant, alors \( \eK[\alpha]=\eK[X]\) (isomorphisme d'anneaux).
        \item
            Si \( \alpha\) est transcendant, alors \( \eK(\alpha)_{\eL}=\eK(X)\) (isomorphisme de corps),
        \item
            Si \( \alpha\) est algébrique, alors \( \ker(\varphi)\) est un idéal possédant un unique générateur unitaire, lequel est le polynôme minimal\footnote{Définition~\ref{DefCVMooFGSAgL}.} de \( \alpha\) sur \( \eK\).
    \end{enumerate}
\end{proposition}

\begin{proof}
    Point par point.
    \begin{enumerate}
        \item
            Nous savons que \( \eK[\alpha]=\{ Q(\alpha)\tq Q\in \eK[X] \}\) (c'est la proposition~\ref{PROPooPMNSooOkHOxJ}). Donc \( \varphi\) est surjective sur \( \eK[\alpha]\), et est donc bijective. Elle est un isomorphisme\footnote{Les amateurs d'écriture inclusive ne seront, je l'espère, pas choqué par «\emph{elle} est \emph{un} isomorphisme»; c'est une tournure que je propose ici sur le modèle de l'immonde «\emph{elle} est \emph{un} ministre» ou, à peine moins grave, «\emph{il} est \emph{une} sommité».} parce que le lemme~\ref{LEMooLVPLooEkWYDN} dit déjà que c'est un morphisme.
        \item
            Nous supposons encore que \( \alpha\) est transcendant et nous considérons l'application
            \begin{equation}
                \begin{aligned}
                    \psi\colon \eK(X)&\to \eK(\alpha) \\
                    P&\mapsto R(\alpha).
                \end{aligned}
            \end{equation}
            Note : cette application n'est pas \( \varphi\). En effet \( \varphi\) n'est définie que sur \( \eK[X]\); le corps des fractions \( \eK(X)\) est nettement plus grand (classes d'équivalence de couples).

            Le fait que cette application soit surjective est la proposition~\ref{PROPooYSFNooFGbbCi}\ref{ITEMooATPTooVXKdlK}. Pour l'injectivité nous supposons que \( \psi(R)=0\), c'est-à-dire que \( R(\alpha)=0\). Nous considérons un représentant \( (P,Q)\) de \( R\); c'est-à-dire \( R=P/Q\). L'égalité \( R(\alpha)=0\) signifie \( P(\alpha)Q(\alpha)^{-1}=0\) (égalité dans \( \eL\)). Puisque \( \eL\) est un corps, c'est un anneau intègre et nous avons la règle du produit nul; soit \( P(\alpha)=0\), soit \( Q(\alpha)^{-1}=0\). La seconde possibilité est impossible parce que zéro n'est pas inversible. Donc \( P(\alpha)=0\). Donc \( \varphi(P)=0\) et \( \varphi\) étant injective, \( P=0\).

            Lorsque \( P=0\), la classe \( P/Q\) est nulle dans \( \eK(X)= \Frac\big(\eK[X]\big)\).

        \item

            C'est le lemme-définition~\ref{DefCVMooFGSAgL}.
    \end{enumerate}
\end{proof}

\begin{proposition}\label{PropXULooPCusvE}
    Soit un corps \( \eK\) et une extension \( \eL\). Soit \( P\in \eK[X]\) et  \( a\in \eL\), une racine de \( P\). Alors le polynôme minimal d'une racine divise\footnote{Définition~\ref{DefMPZooMmMymG}.} tout polynôme annulateur.

    Autrement dit, l'idéal engendré par le polynôme minimal est l'idéal des polynômes annulateurs.
\end{proposition}

\begin{proof}
    Nous considérons l'idéal
    \begin{equation}
        I=\{ Q\in \eK[X]\tq Q(a)=0 \}.
    \end{equation}
    Le fait que cela soit un idéal est simplement dû à la définition du produit : \( (PQ)(a)=P(a)Q(a)\). Par le théorème~\ref{ThoCCHkoU}, le polynôme minimal \( \mu_a\) de \( a\) est dans \( I\) et, qui plus est, le génère : \( I=(\mu_a)\). Par conséquent tout polynôme annulateur de \( a\) est divisé par \( \mu_a\).
\end{proof}

%///////////////////////////////////////////////////////////////////////////////////////////////////////////////////////////
\subsubsection{Extension algébrique, degré}
%///////////////////////////////////////////////////////////////////////////////////////////////////////////////////////////

\begin{proposition}
    Toute extension finie est algébrique.
\end{proposition}

\begin{proof}
    Soient un corps \( \eK\), une extension \( \eL\) de degré\footnote{Définition \ref{DefUYiyieu}.} \( n\) de \( \eK\) et \( a\in \eL\). Nous devons montrer qu'il existe un polynôme annulateur de \( a\) à coefficients dans \( \eK\).

    Soit la partie \( S=\{1,a,a^2,\ldots, a^n\}\) de \( \eL\). Si cette partie contient des éléments non distincts, alors c'est plié. En effet, si \( a^k=a^l\), alors le polynôme \( X^{k-l}\) est un polynôme annulateur de \( a\).

    Nous supposons donc que \( S\) contienne exactement \( n+1\) éléments distincts. Le lemme~\ref{LemytHnlD} nous assure que \( S\) est une partie liée : il existe des éléments \( k_i\in \eK\) tels que \( \sum_{i=0}^nk_ia^i=0\).

    Donc le polynôme \( \sum_ia_iX^i\) est un polynôme annulateur de \( a\).
\end{proof}

\begin{proposition}[Propriétés d'extensions algébriques\cite{MonCerveau}]   \label{PropURZooVtwNXE}
    Soit \( \eK\) un corps commutatif\footnote{Juste en passant nous rappelons que tous les corps considérés ici sont commutatifs} et \( a\) un élément algébrique\footnote{Définition \ref{LEMooLVPLooEkWYDN}.} sur \( \eK\), de polynôme minimal \( \mu_a\) de degré \( n\). Alors
    \begin{enumerate}
        \item\label{ItemJCMooDgEHajmi}
            En considérant l'application d'évaluation
            \begin{equation}
                \begin{aligned}
                    \varphi_a\colon \eK[X]&\to \eL \\
                    Q&\mapsto Q(a),
                \end{aligned}
            \end{equation}
            nous avons \( \eK[a]=\Image(\varphi_a)\).
        \item\label{ItemJCMooDgEHajiv}
            Une base de \( \eK[a]\) comme espace vectoriel sur \( \eK\) est donnée par \( \{ 1,a,a^2,\ldots, a^{n-1} \}\).
        \item\label{ItemJCMooDgEHajiii}
            Le degré de l'extension \( \eK[a]\) est égal au degré du polynôme minimal :
            \begin{equation}
                \big[ \eK[a]:\eK \big]=n.
            \end{equation}
         \item
            L'anneau \( \eK[a]\) est l'ensemble des polynômes en \( a\) de degré \( n-1\) à coefficient dans \( \eK\).
        \item\label{ItemJCMooDgEHaji}
            \( \eK(a)=\eK[a]\).
        \item   \label{ItemJCMooDgEHajii}
            Il existe un isomorphisme d'anneaux \( \varphi\colon \eK[a]\to \eK[X]/(\mu_a)\) tel que \( \varphi(k)=\bar k\) pour tout \( k\in \eK\).
            \( \eK[a]\simeq\eK[X]/(\mu_a)\) (isomorphisme d'anneau).
    \end{enumerate}
\end{proposition}
\index{extension!de corps!algébrique}
L'intérêt de~\ref{ItemJCMooDgEHajii} est qu'il permet de caractériser \( \eK[a]\) sans avoir recours à un sur-corps de \( \eK\). Le point~\ref{ItemJCMooDgEHajiii} indique que le degré d'une extension algébrique est égal au degré du polynôme minimal.


\begin{proof}
    \begin{enumerate}
        \item
            Nous avons \( \eK[a]\subset \Image(\varphi_a)\) parce que \( \Image(\varphi_a)\) est lui-même un sous-anneau de \( \eL\) contenant \( \eK\) et \( a\). Pour rappel, \( \eK[a]\) est l'intersection de tous les tels sous-anneaux.

            L'inclusion inverse est le fait que si \( Q\in \eK[X]\) alors \( Q(a)\in \eK[a]\) parce que \( \eK[a]\) est un anneau et contient donc tous les \( a^n\).
        \item
            La partie \( \{ 1,a,a^2,\ldots, a^{n-1} \}\) est libre parce qu'une combinaison linéaire de ces éléments est un polynôme de degré \( n-1\) en \( a\). Un tel polynôme ne peut pas être nul parce que nous avons mis comme hypothèse que le polynôme minimal de \( a\) est de degré \( n\).

            Rappelons qu'en vertu de la définition~\ref{DefCVMooFGSAgL}, le polynôme minimal \( \mu_a\) est unitaire; donc le polynôme \( \mu_a(X)-X^n\) est un polynôme de degré \( n-1\). Par conséquent en posant \( S(X)=X^n-\mu_a(X)\), le polynôme \( S\) est de degré \( n-1\) et vérifie \( a^n=S(a)\).

            En vertu du point~\ref{ItemJCMooDgEHajmi}, un élément de \( \eK[a]\) s'écrit \( Q(a)\) pour un certain \( Q\in\eK[X]\). Supposons que \( Q\) soit de degré \( p>n-1\); alors nous le décomposons en une partie contenant les termes de degré jusqu'à \( n-1\) et une partie contenant les autres :
            \begin{equation}
                Q(X)=Q_1(X)+X^nQ_2(X)
            \end{equation}
            où \( Q_1\) est de degré \( n-1\) et \( Q_2\) de degré \( p-n\). Nous évaluons cette égalité en \( a\) :
            \begin{equation}
                Q(a)=Q_1(a)+S(a)Q_2(a).
            \end{equation}
            Donc \( Q(a)\) est l'image de \( a\) par le polynôme \( Q_1+SQ_2\) qui est de degré \( p-1\). Par récurrence, \( Q(a)\) est l'image de \( a\) par un polynôme de degré \( n-1\).

            Notons que l'idée est très simple : il s'agit de remplacer récursivement tous les \( a^n\) par \( S(a)\).
    \item
        Conséquence immédiate de~\ref{ItemJCMooDgEHajiv}.
    \item
        Conséquence immédiate de~\ref{ItemJCMooDgEHajiv}.
    \item
        Un élément général non nul de \( \eK[a]\) est de la forme \( Q(a)\) avec \( Q\in\eK[X]\); il s'agit de lui trouver un inverse. Pour cela nous remarquons que les polynômes \( \mu_a(X)\) et \( Q(x)\) sont premiers entre eux, sinon \( \mu_a\) ne serait pas un polynôme minimal (voir la proposition~\ref{PropRARooKavaIT}). Donc le théorème de Bézout~\ref{ThoBezoutOuGmLB} affirme l'existence d'éléments \( U,V\in \eK[X]\) tels que
        \begin{equation}
            U\mu_a+VQ=1
        \end{equation}
        dans \( \eK[X]\). Nous évaluons cette égalité en \( a\) en tenant compte de \( \mu_a(a)=0\) dans \( \eK[a]\) :
        \begin{equation}
            U(a)\mu_a(a)+V(a)Q(a)=1
        \end{equation}
        dans \( \eK[a]\). Par conséquent \( V(a)Q(a)=1\), ce qui signifie que \( V(a)\) est l'inverse de \( Q(a)\).
        \item
            Nous considérons l'application
            \begin{equation}
                \begin{aligned}
                    \psi\colon \eK[X]/(\mu_a)&\to \eK[a] \\
                    \bar R&\mapsto R(a)
                \end{aligned}
            \end{equation}
            et nous montrons qu'elle convient. Pour cela, nous nous souvenons que la proposition~\ref{PropXULooPCusvE} nous enseigne que \( (\mu_a)\), l'idéal engendré par \( \mu_a\), est égal à l'idéal des polynômes annulateurs de \( a\) dans \( \eK[X]\). Le polynôme \( \mu_a\) divise tous les éléments de cet idéal; voir aussi la définition~\ref{DefSKTooOTauAR} de l'idéal \( (\mu_a)\). Cela étant mis au point, nous passons à la preuve.
            \begin{subproof}
            \item[\( \psi\) est bien définie]

                Si \( \bar R=\bar S\) alors \( R=S+Q\) avec \( Q\in(\mu_a)\), et par conséquent \( R(a)=S(a)+Q(a)\) avec \( Q(a)=0\).

            \item[Surjective]

                Nous savons que \( \eK[a]=\Image(\varphi_a)\). Si \( x\in \eK[a]\) alors il existe \( Q\in \eK[X]\) tel que \( x=Q(a)\). Dans ce cas nous avons aussi \( x=\psi(\bar Q)\).

            \item[Injective]

                Si \( \psi(\bar R)=0\) alors \( R(a)=0\), mais comme mentionné plus haut, \( \mu_a\) engendre l'idéal des polynômes annulateurs de \( a\). Donc \( R\in (\mu_a)\) et nous avons \( \bar R=0\) dans \( \eK[X]/(\mu_a)\).

            \end{subproof}

    \end{enumerate}
\end{proof}

\begin{example}
    Un fait connu est que \( \frac{1}{ \sqrt{2} }=\frac{ \sqrt{2} }{ 2 }\). Donc l'inverse de \( \sqrt{2}\) s'exprime bien comme un polynôme en \( \sqrt{2}\) à coefficients dans \( \eQ\), ce qui confirme le point~\ref{ItemJCMooDgEHaji} de la proposition~\ref{PropURZooVtwNXE}. Du point de vue de Bézout, \( \mu_{\sqrt{2}}(X)=X^2-2\), et nous cherchons des polynômes \( U\) et \( V\) tels que
    \begin{equation}
        U(X^2-2)+VX=1.
    \end{equation}
    cette égalité est réalisée par \( U=-\frac{ 1 }{2}\) et \( V=\frac{ 1 }{2}X\). Et effectivement \( V(\sqrt{2})\) est bien l'inverse de \( \sqrt{2}\) :
    \begin{equation}
        V(\sqrt{(2)})=\frac{ 1 }{2}\sqrt{2}.
    \end{equation}
\end{example}


\begin{proposition}[\cite{ooTGTKooFenWAc}]      \label{PROPooNGJWooYSpwVn}
    Soient un corps \( \eK\), une extension \( \eL\) de \( \eK\) et un élément \( \alpha\) de \( \eL\). Il y a équivalence entre les trois points suivants :
    \begin{enumerate}
        \item   \label{ITEMooYTEBooUuEfBz}
            \( \alpha\) est algébrique sur \( \eK\),
        \item   \label{ITEMooWMQTooLnepQl}
            \( \eK[\alpha]=\eK(\alpha)\),
        \item   \label{ITEMooAQIUooMVZojp}
            \( \eK[\alpha]\) est un \( \eK\)-espace vectoriel de dimension finie.
    \end{enumerate}
    Si ces affirmation sont vraies, alors \( [\eK(\alpha):\eK]\) est le degré du polynôme minimal de \( \alpha\) sur \( \eK\).
\end{proposition}

\begin{proof}
    Démonstration décomposée en plusieurs implications.
    \begin{subproof}
        \item[\ref{ITEMooYTEBooUuEfBz} implique~\ref{ITEMooWMQTooLnepQl}]

            Soit \( \alpha\) algébrique sur \( \eK\). Nous considérons le polynôme minimal de \( \alpha\) sur \( \eK\) (définition~\ref{DefCVMooFGSAgL}). Nous savons par le lemme~\ref{LEMooHKTMooKEoOuK} (qui fonctionne parce que \( \alpha\) est algébrique) que \( \eK[\alpha]=\eK[X]/(\mu)\) en tant qu'anneaux.

            Mais \( \eK[X]\) est un anneau principal et \( \mu\) en est un élément irréductible. Donc la proposition~\ref{PropomqcGe} dit que \( (\mu)\) est un idéal maximum; la proposition~\ref{PropoTMMXCx} avance encore un peu en disant que \( \eK[X]/(\mu)\) est un corps.

            Donc \( \eK[X]/(\mu)\) est un corps isomorphe à \( \eK[\alpha]\) en tant qu'anneaux. En conséquence de quoi \( \eK[\alpha]\) est un corps.

            Le corps \( \eK[\alpha]\) est un sous-corps de \( \eL\) contenant \( \eK\) et \( \alpha\); par définition nous avons donc \( \eK(\alpha)\subset \eK[\alpha]\). 

            Mais d'autre part, \( \eK[\alpha]\) est contenu dans tout sous-corps de \( \eL\) contenant \( \eK\) et \( \alpha\), donc il est inclus dans l'intersection de tout ces corps, donc \( \eK[\alpha]\subset \eK(\alpha)\).

            Nous avons donc l'égalité \( \eK[\alpha]=\eK(\alpha)\).

        \item[\ref{ITEMooWMQTooLnepQl} implique~\ref{ITEMooYTEBooUuEfBz}]

            Nous montrons que non-\ref{ITEMooYTEBooUuEfBz} implique non-\ref{ITEMooWMQTooLnepQl}. Nous disons donc que \( \alpha\) est transcendant sur \( \eK\); cela implique par la proposition~\ref{PROPooSYQWooFbfQtm}\ref{ITEMooUZDQooOasiRQ} que \( \eK[\alpha]=\eK[X]\) en tant qu'anneaux. Donc \( \eK[\alpha]\) n'est pas un corps parce que \( \eK[X]\) ne l'est pas.

            N'étant pas un corps, \( \eK[\alpha]\) ne peut pas être égal à \( \eK(\alpha)\) qui, lui, est un corps.

        \item[\ref{ITEMooYTEBooUuEfBz} implique~\ref{ITEMooAQIUooMVZojp}]

            L'élément \( \alpha\) est maintenant algébrique et nous considérons son polynôme minimal \( \mu\). Nous savons par le lemme~\ref{LEMooHKTMooKEoOuK} que \( \eK[\alpha]=\eK[X]/(\mu)\) en tant qu'espaces vectoriels. Or \( \eK[X]/(\mu)\) est de dimension finie \( \deg(\mu)\). Donc \( \eK[\alpha]\) est également de dimension finie.

        \item[\ref{ITEMooAQIUooMVZojp} implique~\ref{ITEMooYTEBooUuEfBz}]

            Nous démontrons la contraposée. En supposant que \( \alpha\) est transcendant nous avons \( \eK[\alpha]=\eK[X]\) par la proposition~\ref{PROPooSYQWooFbfQtm}. Or \( \eK[X]\) n'est pas de dimension finie sur \( \eK\), donc \( \eK[\alpha]\) non plus.

    \end{subproof}
\end{proof}

\begin{lemma}[\cite{SNDooCQYseS}]       \label{LEMooIPAXooCNGMQT}
    Soit \( \eL\) un corps commutatif et \( (\eK_i)_{i\in I}\) une famille de sous-corps de \( \eL\). Alors \( \bigcup_{i\in I}\eK_i\) est un sous-corps de \( \eL\).
\end{lemma}

\begin{lemma}
    Soit \( P\in\eK[X]\) un polynôme unitaire irréductible de degré \( n\). Il existe une extension \( \eL\) de \( \eK\) et \( a\in \eL\) telle que \( \eL=\eK(a)\) et \( P\) est le polynôme minimal de \( a\) dans \( \eL\).
\end{lemma}

\begin{proof}
    Nous prenons \( \eL=\eK[X]/(P)\) où \( (P)\) est l'idéal dans \( \eK[X]\) généré par \( P\). C'est un corps par le corolaire~\ref{CorsLGiEN}. Nous identifions \( \eK\) avec \( \phi(\eK)\) où
    \begin{equation}
        \phi\colon \eK[X]\to \eL
    \end{equation}
    est la projection canonique. Nous considérons également \( a=\phi(X)\).

    Nous avons alors \( P(a)=0\) dans \( \eL\). En effet \( P(a)=P\big( \phi(X) \big)\) est à voir comme l'application du polynôme \( P\) au polynôme \( X\), le résultat étant encore un élément de \( \eL\). En l'occurrence le résultat est \( P\) qui vaut \( 0\) dans \( \eL\).

    Le polynôme \( P\) étant unitaire et irréductible, il est minimum dans \( \eL\).

    Nous devons encore montrer que \( \eL=\eK(a)\). Le fait que \( \eK(a)\subset \eL\) est une tautologie parce qu'on calcule \( \eK(a)\) dans \( \eL\). Pour l'inclusion inverse soit \( Q(X)=\sum_iQ_iX^i\) dans \( \eK[X]\). Dans \( \eL\) nous avons évidemment \( Q=\sum_iQ_ia^i\).
\end{proof}

\begin{proposition}[\cite{ooLIOMooBuCPUS}] \label{PropyMTEbH}
    Soit \( \eK\), un corps et \( P\in \eK[X]\) un polynôme. Soient \( a\) et \( b\), deux racines de \( P\) dans (éventuellement) une extension \( \eL\) de \( \eK\). Si \( \mu_a\) et \( \mu_b\) sont les polynômes minimaux de \( a\) et \( b\) (dans \( \eK[X]\)) et si \( \mu_a\neq \mu_b\), alors \( \mu_a\mu_b\) divise \( P\) dans \( \eK[X]\).
\end{proposition}

\begin{proof}
    Nous considérons les idéaux
    \begin{subequations}
        \begin{align}
            I_a &=\{ Q\in \eK[X]\tq Q(a)=0 \};\\
            I_b &=\{ Q\in \eK[X]\tq Q(b)=0 \}.
        \end{align}
    \end{subequations}
    Même si \( Q(a)\) et \(Q(b)\) sont calculés dans \( \eL\), \( I_a, I_b\) sont des idéaux de \( \eK[X]\). Le polynôme \( \mu_a\) est par définition le générateur unitaire de \( I_a\), et comme \( a\) est une racine de \( P\), nous avons \( P\in I_a\) et il existe un polynôme \( Q\in \eK[X]\) tel que
    \begin{equation}    \label{EqvTPoSq}
        P=\mu_aQ.
    \end{equation}

    Montrons que \( \mu_a(b)\neq 0\). Pour cela, nous  supposons que \( \mu_a(b)=0\), c'est-à-dire que \( \mu_a\in I_b\). Il existe alors \( R\in \eK[X]\) tel que \( \mu_a=\mu_bR\). Mais par la proposition~\ref{PropRARooKavaIT}, le polynôme \( \mu_a\) est irréductible, donc soit \( \mu_b\), soit \( R\), est inversible. Comme les inversibles sont les éléments de \( \eK\) (polynômes de degré zéro), \( \mu_b\) n'est pas inversible (sinon il serait constant et ne pourait pas être annulateur de \( b\)). Donc \( R\) est inversible. Disons \( R=k\).

    Donc \( \mu_a=k\mu_b\). Mais puisque \( \mu_a\) et \( \mu_b\) sont unitaires, nous avons obligatoirement \( k=1\). Cela donnerait \( \mu_a=\mu_b\), ce qui est contraire aux hypothèses. Nous en déduisons que \( \mu_a(b)\neq 0\).

    Étant donné que \( \mu_a(b)\neq 0\), l'évaluation de \eqref{EqvTPoSq} en \( b\) montre que \( Q(b)=0\), de telle sorte que \( Q\in I_b\) et il existe un polynôme \( S\) tel que \( Q=\mu_bS\), c'est-à-dire tel que \( P=\mu_a\mu_bS\), ce qui signifie que \( \mu_a\mu_b\) divise \( P\).
\end{proof}

\begin{example}
    Soit \( P=(X^2+1)(X^2+2)\) dans \( \eR[X]\). Dans \( \eC\) nous avons les racines \( a=i\) et \( b=\sqrt{2}i\) dont les polynômes minimaux sont \( \mu_a=X^2+1\) et \( \mu_b=X^2+2\). Nous avons effectivement \( \mu_a\mu_b\) divise \( P\) dans \( \eR[X]\).

    Si par contre nous considérions les racines \( a=i\) et \( b=-i\), nous aurions \( \mu_a=\mu_b=X^2+1\), tandis que le polynôme \( \mu_a^2\) ne divise pas \( P\).
\end{example}




%---------------------------------------------------------------------------------------------------------------------------
\subsection{Racines de polynômes}
%---------------------------------------------------------------------------------------------------------------------------

\begin{corollary}[Factorisation d'une racine]   \label{CorDIYooEtmztc}
    Soit \( P\in \eK[X]\), un polynôme de degré \( n\) et \( \alpha\in \eK\) tel que \( P(\alpha)=0\). Alors il existe un polynôme \( Q\) de degré \( n-1\) tel que \( P(x)=(X-\alpha)Q\).
\end{corollary}
\index{factorisation!de polynôme}

\begin{proof}
    Il s'agit d'un cas particulier de la proposition~\ref{PropXULooPCusvE} : si \( \alpha\in \eK\) alors son polynôme minimal dans \( \eK\) est \( X-\alpha\); donc \( X-\alpha\) divise \( P\). Il existe un polynôme \( Q\) tel que \( P=(X-\alpha)Q\). Le degré est alors immédiat.
\end{proof}

Avant de lire l'énoncé suivant, allez relire la définition \ref{NORMooQFTJooLBcPxl} pour savoir ce qu'est un polynôme nul.
\begin{theorem}[Polynôme qui a tellement de racines qu'il s'annule]\label{ThoLXTooNaUAKR}
    Soit \( \eK\) un corps et \( P\in \eK[X]\) un polynôme de degré \( n\) possédant \( n+1\) racines distinctes \( \alpha_1\),\ldots, \( \alpha_{n+1}\), alors \( P=0\).
\end{theorem}
\index{racine!de polynôme}

\begin{proof}
    Si \( P\) est de degré \( 1\), il s'écrit \( P=aX+b\); si il a comme racines \( \alpha\) et \( \beta\), nous avons le système
    \begin{subequations}
        \begin{numcases}{}
            a\alpha+b=0\\
            a\beta+b=0.
        \end{numcases}
    \end{subequations}
    La différence entre les deux donne \( a(\alpha-\beta)=0\). Puisque \( \alpha\neq \beta\), la règle du produit nul (lemme~\ref{LemAnnCorpsnonInterdivzer}) nous donne \( a=0\). Maintenant que \( a=0\), l'annulation de \( b\) est alors immédiate.

    Nous faisons maintenant la récurrence en supposant le théorème vrai pour le degré \( n\) et en considérant un polynôme \( P\) de degré \( n+1\) possédant \( n+2\) racines distinctes. Puisque \( P(\alpha_1)=0\), le corolaire~\ref{CorDIYooEtmztc} nous donne un polynôme \( Q\) de degré \( n\) tel que
    \begin{equation}    \label{EqQGSooNdTWfz}
        P=(X-\alpha_1)Q.
    \end{equation}
    Étant donne que pour tout \( i\neq 1\) nous avons \( \alpha_i\neq \alpha_1\),
    \begin{equation}
        0=P(\alpha_i)=\underbrace{(\alpha_i-\alpha_1)}_{\neq 0}Q(\alpha_i),
    \end{equation}
    et la règle du produit nul donne \( Q(\alpha_i)=0\). Par conséquent le polynôme \( Q\) est de degré \( n\) et possède \( n+1\) racines distinctes; tous ses coefficients sont alors nuls par hypothèse de récurrence. Tous les coefficients du produit \eqref{EqQGSooNdTWfz} sont alors également nuls.
\end{proof}

\begin{example}\label{ExGRHooBNpjSP}
    Un polynôme à plusieurs variables peut s'annuler en une infinité de points sans être nul. Par exemple le polynôme \( X^2+Y^2-1\in\eR[X,Y]\) s'annule sur tout un cercle de \( \eR^2\) mais n'est pas nul, loin s'en faut.

    Nous verrons dans la proposition~\ref{PropTETooGuBYQf} une condition pour qu'un polynôme à plusieurs variables s'annule du fait qu'il ait «trop» de racines.
\end{example}

\begin{remark}
    L'intérêt du théorème~\ref{ThoLXTooNaUAKR} est que si l'on prouve qu'un polynôme s'annule sur un corps infini, alors il s'annulera sur n'importe quel autre corps. Nous aurons un exemple d'utilisation de cela dans le théorème de Cayley-Hamilton~\ref{ThoHZTooWDjTYI}.
\end{remark}

%---------------------------------------------------------------------------------------------------------------------------
\subsection{Corps de rupture}
%---------------------------------------------------------------------------------------------------------------------------

\begin{definition}      \label{DEFooVALTooDJJmJv}
    Soit \( P\in\eK[X]\) un polynôme irréductible. Une extension \( \eL\) de \( \eK\) est un \defe{corps de rupture}{corps!de rupture}\index{rupture!corps} pour \( P\) si il existe \( a\in \eL\) tel que \( P(a)=0\) et \( \eL=\eK(a)\).
\end{definition}

\begin{normaltext}      \label{NORMALooTPOIooVZAfUo}
    Nous insistons sur le fait que nous ne définissons le concept de corps de rupture que pour un polynôme irréductible à coefficients dans un corps. Les deux points sont importants : irréductible et à coefficient dans un corps.

    Nous discuterons brièvement le pourquoi de cela dans la section~\ref{SUBSECooEDMJooTXBfOu}.
\end{normaltext}

\begin{example}     \label{ExemGVxJUC}
    Soit \( \eK=\eQ\) et \( P=X^2-2\). On pose \( a=\sqrt{2}\) et \( \eL=\eQ(\sqrt{2})\subset\eR\). De cette façon \( P\) est scindé dans \( \eL \):
    \begin{equation}
        P=(X-\sqrt{2})(X+\sqrt{2}).
    \end{equation}
    Le corps \( \eQ(\sqrt{2})\) est donc un corps de rupture pour \( P\).
\end{example}

\begin{example}
    Dans l'exemple~\ref{ExemGVxJUC}, nous avions un corps de rupture dans lequel le polynôme \( P\) était scindé. Il n'en est pas toujours ainsi. Prenons
    \begin{equation}
        P=X^3-2
    \end{equation}
    et \( a=\sqrt[3]{2}\). Nous avons, certes, \( P(a)=0\) dans \( \eQ(\sqrt[3]{2})\), mais \( P\) n'est pas scindé parce qu'il y a deux racines complexes.
\end{example}

\begin{example}
    Nous considérons le corps \( \eZ/p\eZ\) où \( p\) est un nombre premier. Si \( s\in \eZ/p\eZ\) n'est pas un carré, alors le polynôme \(P= X^2+s\) est irréductible et un corps de rupture de \( P\) sur \( \eZ/p\eZ\) est donné par \( (\eZ/p\eZ)[X]/(X^2+s)\), c'est-à-dire l'ensemble des polynômes de degré \( 1\) en \( \sqrt{s}\). Le cardinal en est \( p^2\).
\end{example}

Comme nous allons abondamment parler du quotient \( \eK[X]/(P)\), nous nous permettons un petit lemme.
\begin{lemma}       \label{LEMooWYYFooXYacdF}
    Soit un corps \( \eK\) et \( P\in \eK[X]\) non constant. Alors \( \eK[X]/(P)\) est un corps si et seulement si \( P\) est irréductible.
\end{lemma}

\begin{proof}
    Nous utilisons le trio d'enfer dont il est question dans le thème~\ref{THEMEooZYKFooQXhiPD}. D'abord \( \eK[X]\) est un anneau principal par le lemme~\ref{LEMooIDSKooQfkeKp}. Donc \( \eK[X]/(P)\) sera un corps si et seulement si \( (P)\) est un idéal maximum (proposition~\ref{PROPooSHHWooCyZPPw}), et cela sera le cas si et seulement si \( (P)\) est engendré par un polynôme irréductible (proposition~\ref{PropomqcGe}).

    Il ne nous reste qu'à montrer que \( (P)\) est engendré par un irréductible si et seulement si \( P\) est irréductible. Il y a un sens dans lequel c'est évident.

    Soit un irréductible \( \mu\) tel que \( (P)=(\mu)\). En particulier \( \mu\in (P)\), c'est-à-dire qu'il existe \( Q\) tel que \( \mu=PQ\). Puisque \( \mu\) est irréductible, soit \( P\), soit \( Q\), est inversible. Si \( P\) est inversible, c'est-à-dire constant, c'est ce que nous avons exclu par hypothèse. Si par contre \( Q\) est inversible, alors \( P=k\mu\) pour un certain \( k\in \eK\), ce qui montre que \( P\) est irréductible autant que \( \mu\).
\end{proof}

\begin{proposition}[Existence d'un corps de rupture]        \label{PROPooUBIIooGZQyeE}
    Soit un corps \( \eK\) et un polynôme irréductible non constant \( P\). Alors
    \begin{enumerate}
        \item
            Le corps \( \eL=\eK[X]/(P)\) est un corps de rupture pour \( P\).
        \item
            L'élément \( \bar X\) de \( \eL\) est une racine de \( P\).
        \item
            \( \eL=\eK(\bar X)_{\eL}\)
    \end{enumerate}
\end{proposition}

\begin{proof}
    Commençons par nous convaincre que \( \eK[X]/(P)\) est une extension de \( \eK\) (définition~\ref{DEFooFLJJooGJYDOe}). Le fait que ce soit un corps est le lemme~\ref{LEMooWYYFooXYacdF}. Le morphisme \( j\colon \eK\to \eK[X]/(P)\) est simplement \( k\mapsto \bar k\) où à droite, \( \bar k\) voit \( k\) dans \( \eK[X]\) comme étant le polynôme constant. Notez qu'il est automatiquement injectif (lemme~\ref{LEMooWBOPooZnsZgQ}).

    Il faut maintenant voir que \( \eK[X]/(P)=\eK(\alpha)\) pour un certain \( \alpha\in \eK[X]/(P)\). Grâce à notre compréhension des notations acquise dans~\ref{SUBSUBSECooPNBYooWXEHrg}, nous savons que \( X\in\eK[X]\) et qu'il est donc parfaitement légitime de poser \( \alpha=\bar X\) dans \( \eK[X]/(P)\). Il s'agit simplement de l'ensemble \( \bar X=\{ X+QP\tq Q\in \eK[X] \}\) où \( X\) est une notation pour la suite \( (0,1,0,0,\ldots)\).

    Bref, nous notons \( \alpha=\bar X\) et nous démontrons que \( P(\alpha)=0\) et que \( \eK[X]/(P)=\eK(\alpha)\) (isomorphisme de corps).
    \begin{subproof}
        \item[\( P(\bar X)=0\)]

            C'est le moment de nous souvenir comment la notation des \( X\) fonctionne, et en particulier la pirouette autour de \eqref{EQooABULooFCEasf}. D'abord la définition du produit sur \( \eK[X]/(P)\) est \( \bar P\bar Q=\overline{ PQ }\); en particulier si \( P=\sum_ka_kX^k\), alors \( P(\bar X)=\sum_ka_k\bar X^k=\sum_ka_k\overline{ X^k }\), et
            \begin{equation}
                P(\bar X)=\overline{ P(X) }=\bar P=0.
            \end{equation}
        \item[L'égalité]

            Nous montrons à présent que \( \eK(\bar X)_{\eL}=\eL\). C'est-à-dire que \( \eL\) est bien engendrée par \( \eK\) et un seul élément. D'abord, \( \eL=\eK[X]/(P)\) contient bien évidemment \( \eK\) et \( \bar X\). Ensuite nous devons prouver que tout sous-corps de \( \eL\) contenant \( \eK\) et \( \bar X\) est en réalité \( \eL\) entier.

            Soit \( Q\in \eK[X]\), et montrons que \( \bar Q\) est dans tout sous-corps de \( \eL\) contenant \( \eK\) et \( \bar X\).

            Par le lemme~\ref{LEMooXFMAooMBgIrN} nous avons \( \bar Q=Q(\bar X)\). Et si un corps contient \( \eK\) et \( \bar X\), il doit contenir tous les polynômes en \( \bar X\) à coefficients dans \( \eK\). Donc un tel corps doit contenir \( Q(\bar X)\) et donc \( \bar Q\).

    \end{subproof}
\end{proof}

\begin{example}
    Soit le polynôme \( P=X^2+1\in \eZ[X]\). Dans le quotient \( \eZ[X]/(P)\) nous avons \( \bar X^2+1=0\) et donc \( \bar X^2=-1\). C'est-à-dire que \( \eZ[X]/(P)\) contient un élément dont le carré est \( -1\). Avouez que c'est bien ce à quoi nous nous attendions.

    Notons que \( -\bar X\) est également une racine de \( P\) dans \( \eZ[X]/(P)\).

    En calculant dans les polynômes à coefficients dans \( \eZ(\bar X)\) nous avons :
    \begin{equation}
        (X+\bar X)(X-\bar X)=X^2-\bar X^2=X^2+1,
    \end{equation}
    c'est-à-dire que \( P\) est bien factorisé, et que nous avons retrouvé la multiplication \( x^2+1=(x+i)(x-i)\).
\end{example}

\begin{normaltext}
    Il n'y a évidemment pas unicité d'un corps de rupture pour un polynôme donné. Une raison est qu'un polynôme peut accepter plusieurs racines complètement indépendantes. Le corps étendu par l'une ou l'autre racine donne deux corps de rupture différents. Par exemple dans \( \eQ[X]\), le polynôme
    \begin{equation}
        P=X^4-X^2-2
    \end{equation}
    a pour racines (dans \( \eC\)) les nombres \( \sqrt{ 2 }\) et \( i\). Donc on a deux corps de rupture complètement différents : \( \eQ(\sqrt{ 2 })\) et \( \eQ(i)\).
\end{normaltext}

\begin{normaltext}
    La proposition suivante donne une unicité du corps de rupture dans le cas d'un polynôme irréductible. Et nous comprenons pourquoi : un polynôme irréductible n'a fondamentalement qu'une seule racine «indépendante». Par exemple \( X^2-2\) a pour racines \( \pm\sqrt{ 2 }\). Autre exemple, le polynôme \( X^2+6X+13\) a pour racines, dans \( \eC\), les nombres complexes conjugués \( z=-3+2i\) et \( \bar z=-3-2i\).
\end{normaltext}

\begin{proposition}[\cite{ooUHHUooONXDDl}]          \label{PROPooVJACooNDmlfb}
    Soient un corps \( \eK\) et un polynôme irréductible \( P\in \eK[X]\). Alors toute extension \( \eL\) contenant une racine \( \alpha\) de \( P\) admet un unique morphisme de corps
            \begin{equation}
                \psi\colon \eK[X]/(P)\to \eL
            \end{equation}
            tel que \( \psi(\bar X)=\alpha\).

    Dans un tel cas,
    \begin{enumerate}
        \item
            l'image de \( \psi\) est $\eK(\alpha)_{\eL}$ ,
        \item       \label{ITEMooHRFHooWLIdWU}
            si \( \eL=\eK(\alpha)_{\eL}\) alors \( \psi\) est un isomorphisme.
    \end{enumerate}

\end{proposition}

\begin{proof}
    L'idéal annulateur de \( \alpha\) parmi les polynôme de \( \eK[X]\) n'est pas réduit à \( \{ 0 \}\) parce qu'il contient \( P\). Le lemme~\ref{DefCVMooFGSAgL} s'applique donc et nous avons \( \mu\), le polynôme minimal de \( \alpha\) dans \( \eK[X]\). Il divise \( P\) qui est irréductible, donc
    \begin{equation}
        P=\lambda \mu
    \end{equation}
    pour un certain \( \lambda\in \eK\).

    Nous posons
    \begin{equation}
        \begin{aligned}
            \psi\colon \eK[X]/(P)&\to \eL \\
            \bar Q&\mapsto Q(\alpha).
        \end{aligned}
    \end{equation}
    \begin{subproof}
        \item[Bien définie]
            Si \( \bar Q_1=\bar Q_2\) alors il existe un \( R\in \eK[X]\) tel que \( Q_1=Q_2+RP\). Mais alors \( \psi(\bar Q_1)=Q_1(\alpha)=Q_2(\alpha)+R(\alpha)P(\alpha)=Q_2(\alpha)\).
        \item[Injective]

            Si \( \psi(\bar Q_1)=\psi(\bar Q_2)\) alors \( Q_1-Q_2=R\) pour un certain \( R\in \eK[X]\) vérifiant \( R(\alpha)=0\). Nous avons alors un polynôme \( S\) tel que \( R=S\mu=\lambda^{-1}SP\). Donc \( \bar R=0\) et donc \( \bar Q_1=\bar Q_2\).

        \item[Morphisme]

            Laissé comme exercice; la paresse de l'auteur de ces lignes attend vos contributions.

        \item[La condition]

            Le morphisme \( \psi\) respecte de plus la condition
            \begin{equation}
                \psi(\bar X)=X(\alpha)=\alpha.
            \end{equation}

    \end{subproof}

    En ce qui concerne l'unicité, fixer \( \psi(\bar X)\) est suffisant pour fixer un morphisme. En effet si \( \psi(\bar X)=\alpha\), alors
    \begin{equation}
        \psi(\bar Q)=\psi\Big( \sum_ka_k\bar X^k \Big)=\sum_ka_k\psi(\bar X)^k=\sum_ka_k\alpha^k.
    \end{equation}

    Pour le second point de l'énoncé, il faut remarquer que \( \alpha\) est algébrique et non transcendant. Donc en utilisant les propositions~\ref{PROPooPMNSooOkHOxJ} et~\ref{PropURZooVtwNXE}\ref{ItemJCMooDgEHaji} nous trouvons
    \begin{equation}
        \Image(\psi)=\{ Q(\alpha)\tq Q\in \eK[X] \}=\eK[\alpha]=\eK(\alpha).
    \end{equation}

    Et finalement pour le dernier point, un morphisme de corps est toujours injectif. Si il est également surjectif, il sera bijectif.
\end{proof}

%---------------------------------------------------------------------------------------------------------------------------
\subsection{Pile d'extensions}
%---------------------------------------------------------------------------------------------------------------------------

\begin{lemma}[\cite{MonCerveau}]        \label{LEMooTURZooXnjmjT}
    Soient un corps \( \eK\), des extensions \( \eL_1\),\ldots, \( \eL_n\) et des éléments \( \alpha_i\in \eL_i\) tels que
    \begin{equation}    \label{EQooOCQSooFMkzTc}
        \eL_1=\eK(\alpha_1)_{\eL_1}
    \end{equation}
    et
    \begin{equation}
        \eL_k=\eL_{k-1}(\alpha_k)_{\eL_k}.
    \end{equation}
    Alors
    \begin{equation}
        \eL_n=\eK(\alpha_1,\ldots, \alpha_n)_{\eL_n}.
    \end{equation}
\end{lemma}

\begin{proof}
    Nous démontrons par récurrence sur \( n\). Le cas \( n=1\) est simplement l'hypothèse \eqref{EQooOCQSooFMkzTc}. 
    
    Supposons donc que le lemme soit correct pour \( n\), et étudions le cas \( n+1\). Nous avons, par définition et par hypothèse de récurrence :
    \begin{equation}
        \eL_{n+1}=\eL_n(\alpha_{n+1})_{\eL_{n+1}}=\Big( \eK(\alpha_1,\ldots, \alpha_n)_{\eL_n} \Big)(\alpha_{n+1})_{\eL_{n+1}}.
    \end{equation}
    Notre tâche sera donc de montrer que
    \begin{equation}\label{EQooIHMGooTlPcsd}
        \Big( \eK(\alpha_1,\ldots, \alpha_n)_{\eL_n} \Big)(\alpha_{n+1})=\eK(\alpha_1,\ldots,\alpha_{n+1})
    \end{equation}
    où nous n'écrivons plus les indices \( \eL_{n+1}\) partout.

    Le membre de gauche est un sous-corps de \( \eL_{n+1}\) contenant à la fois \( \eK \) et tous les \(\alpha_i \), si bien que
    \begin{equation}\label{EQooLLRHooHOjLfk}
        \eK(\alpha_1,\ldots,\alpha_{n+1})\subset \big( \eK(\alpha_1,\ldots, \alpha_n)_{\eL_n} \big)(\alpha_{n+1})_{\eL_{n+1}}.
    \end{equation}

    Il faut donc prouver l'inclusion inverse; c'est-à-dire montrer que tout élément \( x \) du corps \( \big( \eK(\alpha_1,\ldots, \alpha_n)_{\eL_n} \big)(\alpha_{n+1})\) est forcément dans tout sous-corps de \( \eL_{n+1}\) contenant \( \eK\) et les \( \alpha_i\). Un tel élément \( x \) est, par la proposition~\ref{PROPooYSFNooFGbbCi}\ref{ITEMooATPTooVXKdlK}, de la forme \( r(\alpha_{n+1})\) avec \( r\in \eK(\alpha_1,\ldots, \alpha_{n})(X)\), c'est-à-dire
            \begin{equation}
                P(\alpha_{n+1})Q(\alpha_{n+1})^{-1}
            \end{equation}
            avec \( P,Q\in \eK(\alpha_1,\ldots, \alpha_n)[X]\).

            Prouvons d'abord que si \( P\in \eK(\alpha_1,\ldots, \alpha_n)[X]\), alors \( P(\alpha_{n+1})\) est dans tout sous-corps de \( \eL_{n+1}\) contenant \( \eK\) et les \( \alpha_i\). Nous pouvons écrire \( P=\sum_ia_iX^i\) avec \( a_i\in \eK(\alpha_1,\ldots, \alpha_n)\), et donc
            \begin{equation}
                P(\alpha_{n+1})=\sum_ia_i\alpha_{n+1}^i.
            \end{equation}
            Tout corps contenant \( \eK\) et les \( \alpha_1\),\ldots, \( \alpha_n\) contient les \( a_i\). Par produit, tout corps contenant \( \eK\), \( \alpha_1\),\ldots,  \( \alpha_{n+1}\) contient les termes \( a_i\alpha_{n+1}^i\), et donc \( P(\alpha_{n+1})\) par somme.

        De la même façon, si un corps contient \( \eK\) et les \( \alpha_i\), (\( i=1,\ldots, n+1\)), alors il contient \( Q(\alpha_{n+1})\). Comme c'est un corps, il contient aussi son inverse \( Q(\alpha_{n+1})^{-1}\), et il contient aussi le produit
            \begin{equation}
                r(\alpha_{n+1})=P(\alpha_{n+1})Q(\alpha_{n+1})^{-1}.
            \end{equation}

    On vient ainsi de montrer que tout élément \( x \in  \big( \eK(\alpha_1,\ldots, \alpha_n)_{\eL_n} \big)(\alpha_{n+1})\) était dans tout sous-corps de \( \eL_{n+1} \) qui contient \( \eK \) et les \( \alpha_i\), (\( i=1,\ldots, n+1\)); en d'autres termes:
    \begin{equation}\label{EQooUFSMooJozpqL}
       \big( \eK(\alpha_1,\ldots, \alpha_n)_{\eL_n} \big)(\alpha_{n+1})_{\eL_{n+1}} \subset \eK(\alpha_1,\ldots,\alpha_{n+1}).
    \end{equation}
    Les inclusions \eqref{EQooLLRHooHOjLfk} et \eqref{EQooUFSMooJozpqL} prouvent l'égalité d'ensembles \eqref{EQooIHMGooTlPcsd} que nous voulions montrer.
\end{proof}

%--------------------------------------------------------------------------------------------------------------------------- 
\subsection{Clôture algébrique}
%---------------------------------------------------------------------------------------------------------------------------

Le concept de clôture algébrique a été défini dans \ref{DEFooREUHooLVwRuw}. Voici un lemme qui dit qu'une clôture algébrique est en quelque sorte une extension algébrique maximale.
\begin{lemma}[\cite{MonCerveau}]        \label{LEMooQSCGooMyCktA}
    Soient un corps \( \eK\) et une extension algébrique \( \eF\) de \( \eK\). Nous supposons que pour toute extension algébrique de \( \eL\) nous avons \( \eL=\eF\)

    Alors \( \eF\) est algébriquement clos\footnote{Définition \ref{DEFooREUHooLVwRuw}\ref{ITEMooEIWVooVjJRoR}.}.
\end{lemma}

\begin{proof}
    Soit un polynôme \( P\in \eK[X]\). Nous voudrions prouver que \( P\) a des racines dans \( \eF\). Pour cela, nous voyons \( P\) comme un polynôme sur \( \eF[X]\) et, grace à la proposition \ref{PROPooUBIIooGZQyeE} nous considérons un corps de rupture \( \eL\) pour \( P\). Puisque \( \eL\) est une extension de \( \eF\), nous avons \( \eL=\eF\). Donc \( \eF\) contient des racines de \( P\).
\end{proof}


\begin{normaltext}
    Nous avons défini le concept d'extension algébrique en \ref{DEFooREUHooLVwRuw}. Nous allons en construire un petit exemple très piéton.

    D'abord la proposition \ref{PROPooUHKFooVKmpte} nous donne l'existence et l'unicité d'un réel \( \sqrt{ 2 }\) strictement positif dont le carré est \( 2\). Ce réel est irrationnel par la proposition \ref{PropooRJMSooPrdeJb}. Cela étant posé, nous y allons.
\end{normaltext}

\begin{proposition}[\cite{BIBooAOINooYRFZFb}]
    Soit \( \eL=\{ a+b\sqrt{ 2 } \}_{a,b\in \eQ}\).
    \begin{enumerate}
        \item
            C'est un sous-corps de \( \eR\).
        \item   \label{ITEMooUSOAooZoBhla}
            Tout sous-corps de \( \eR\) contenant \( \eQ\) et \( \sqrt{ 2 }\) contient \( \eL\).
    \end{enumerate}
\end{proposition}

\begin{proof}
    Nous devons d'abord prouver que \( \eL\) est un corps en vérifiant d'une part que c'est un anneau (définition \ref{DefHXJUooKoovob}) et d'autre part le fait que tous les éléments non nuls sont inversibles.
    \begin{itemize}
        \item La partie \( \eL\) de \( \eR\) est stable pour l'addition : dès que \( a,b,a',b'\in \eQ\),
            \begin{equation}
                (a+b\sqrt{ 2 })+(a'+b'\sqrt{ 2 })=(a+a')+(b+b')\sqrt{ 2 }\in \eL.
            \end{equation}
        \item
            Les neutres \( 0\) et \( 1\) sont dans \( \eL\).
        \item
            Si \( \alpha\in \eL\), alors \( -\alpha\in \eL\) :
            \begin{equation}
                -(a+b\sqrt{ 2 })=-a-b\sqrt{ 2 }.
            \end{equation}
        \item
            La partie \( \eL\) est stable pour le produit parce que
            \begin{equation}
                (a+b\sqrt{ 2 })(a'+b'\sqrt{ 2 })=(aa'+2bb')+(ab'+ba')\sqrt{ 2 }.
            \end{equation}
        \item
            L'inverse d'un élément de \( \eL\) est dans \( \eL\). C'est le seul point pas tout à fait évident. D'abord, l'ensemble \( \eR\) est un corps par le théorème \ref{DefooFKYKooOngSCB}. Donc pour tout \( a,b\in \eR\), le nombre
            \begin{equation}
                \frac{1}{ a+b\sqrt{ 2 } }
            \end{equation}
            existe dans \( \eR\).
            
            D'abord \( a-b\sqrt{ 2 }\) n'est pas nul, parce que si il l'était, nous aurions \( \sqrt{ 2 }=-a/b\in \eQ\) alors que \( \sqrt{ 2 }\) n'est pas rationnel par la proposition \ref{PropooRJMSooPrdeJb}. Nous pouvons donc faire le coup de multiplier le numérateur et le dénominateur par le binôme conjugué :
            \begin{equation}
                \frac{1}{ a+b\sqrt{ 2 } }=\frac{ a-b\sqrt{ 2 } }{ (a+b\sqrt{ 2 })(a-b\sqrt{ 2 }) }=\frac{ a }{ a^2-2b^2 }-\frac{ b }{ a^2-2b^2 }\sqrt{ 2 }.
            \end{equation}
            Cela est un rationnel. Donc le éléments non nuls de \( \eL\) ont un inverse qui appartient également à \( \eL\).
    \end{itemize}
    Nous passons à la preuve du point \ref{ITEMooUSOAooZoBhla}. Si \( \eL'\) est un corps qui contient \( \eQ\) et \( \sqrt{ 2 }\), il doit contenir \( b\sqrt{ 2 }\) pour tout \( b\in \eQ\) et donc aussi tous les \( a+b\sqrt{ 2 }\) avec \( a,b\in \eQ\). Par conséquent, \( \eL'\) doit contenir au moins tout \( \eL\).
\end{proof}

\begin{proposition}
    Soit \( \eL=\{ a+b\sqrt{ 2 } \}_{a,b\in \eQ}\).
    \begin{enumerate}
        \item   \label{ITEMooOMDMooLNhlyh}
            C'est un espace vectoriel de dimension \( 2\) sur \( \eQ\).
        \item       \label{ITEMooWGGDooSbsesf}
            Si \( \alpha\in \eL\), alors il existe un polynôme \( P\in \eL[X]\) de degré \( 2\) ou moins tel que \( P(\alpha)=0\).
        \item   \label{ITEMooPNNYooPtKYwQ}
            Le corps \( \eL\) est une extension algébrique de \( \eQ\).
    \end{enumerate}
\end{proposition}

\begin{proof}
    En plusieurs parties.
    \begin{subproof}
        \item[\ref{ITEMooOMDMooLNhlyh}]
            Pour la dimension, notez que \( \{ 1,\sqrt{ 2 } \}\) est une partie libre et génératrice de \( \eL\).

        \item[\ref{ITEMooWGGDooSbsesf}]

            Soit \( \alpha\in \eL\). La partie \( \{ 1,\alpha,\alpha^2 \}\) est de cardinal \( 1\), \( 2\) ou \( 3\). Si c'est \( 1\) ou \( 2\), c'est que \( 1=\alpha\) ou \( 1=\alpha^2\) ou \( \alpha=\alpha^2\). Si par exemple \( 1=\alpha\) alors avec \( P=X-1\) nous avons \( P(\alpha)=0\).

            Si au contraire \( \{ 1,\alpha,\alpha^2 \}\) est de cardinal \( 3\), alors c'est une partie liée par la proposition \ref{PROPooEIQIooXfWDDV}. Il existe donc des rationnels \( a,b,c\) tels que \( a+b\alpha+c\alpha^2=0\), c'est-à-dire \( P(\alpha)=0\) avec \( P=cX^2+bX+a\).
        \item[\ref{ITEMooPNNYooPtKYwQ}]
            Nous venons de voir que tous les éléments de \( \eL\) sont des racines de polynômes de \( \eQ[X]\).
    \end{subproof}
\end{proof}

\begin{lemma}       \label{LEMooHWPHooZeWqns}
    Si \( \eK\) est un corps infini, alors \( \eK[X]\) est équipotent\footnote{Définition \ref{DEFooXGXZooIgcBCg}.} à \( \eK\).
\end{lemma}

\begin{proof}
    Notons provisoirement \( \eK_n[X]\) l'ensemble des polynômes de degré \( n\). Nous avons une surjection
    \begin{equation}    \label{EQooFGZVooKIMKRA}
        \begin{aligned}
            \varphi\colon \eK^{n+1}&\to \eK_n[X] \\
            (k_0,\ldots, k_n)&\mapsto \sum_{i=0}^nk_iX^i. 
        \end{aligned}
    \end{equation}
    Par récurrence sur le théorème\quext{J'ai quand même du mal à croire qu'il faille vraiment le lemme de Zorn pour prouver que \( \eK[X]\) est équipotent à \( \eK\). Si vous connaissez un moyen plus direct, écrivez-moi.} \ref{THOooDGOVooRdURVi}, nous avons \( \eK^{n+1}\approx \eK\). La surjection \eqref{EQooFGZVooKIMKRA} dit alors que
    \begin{equation}
        \eK_n[X]\preceq \eK^{n+1}\approx \eK.
    \end{equation}
    Mais puisqu'il y a une surjection \( \eK\to \eK_n[X]\), nous avons aussi \( \eK_n[X]\succeq \eK\). Le théorème \ref{THOooRYZJooQcjlcl} dit alors que \( \eK_n[X]\approx \eK\).

    Le lemme \ref{LEMooNKKDooUvSYPO} nous permet alors de conclure que
    \begin{equation}
        \eK[X]=\bigcup_{n=0}^{\infty}\eK_n[X]\approx \eK.
    \end{equation}
\end{proof}

\begin{proposition}[\cite{MonCerveau}]      \label{PROPooVPQFooScWvkS}
    Soit un corps \( \eK\). Une extension algébrique de \( \eK\) est
    \begin{enumerate}
        \item
            au plus dénombrable si \( \eK\) est fini,
        \item
            équipotente à \( \eK\) si \( \eK\) est infini.
    \end{enumerate}
\end{proposition}

\begin{proof}
    En deux parties.
    \begin{subproof}
        \item[Si \( \eK\) est fini]
            Un polynôme non nul possède toujours au maximum un nombre fini de racines (éventuellement zéro) par la proposition \ref{ThoLXTooNaUAKR}. Par ailleurs, chaque degré de polynôme ayant seulement un nombre fini de possibilités, l'ensemble \( \eK[X]\) est au plus dénombrable (proposition \ref{PROPooENTPooSPpmhY}).

            Pour \( P\in \eK[X]\) nous avons une surjection de \( \eN\) vers l'ensemble des racines de \( P\). Nous la notons \( \varphi_P\colon \eN\to \eL\), en posant par exemple \( \varphi_P(n)=1\) si \( P\) n'a pas de racines. Enfin nous posons
            \begin{equation}
                \begin{aligned}
                    \varphi\colon \eK[X]\times \eN&\to \eL \\
                    (P,n)&\mapsto \varphi_P(n). 
                \end{aligned}
            \end{equation}
            C'est la fonction qui à un polynôme \( P\) et un nombre \( n\) fait correspondre la \( n\)\ieme\ racine de \( P\).

            Comme \( \eL\) est une extension algébrique, \( \varphi\) est surjective.

            En termes de cardinalité, que \( \eK[X]\) soit fini ou dénombrable, dans les deux cas, \( \eK[X]\times \eN\) est dénombrable (proposition \ref{PROPooLPKUooAlsYJg}). Il existe donc une surjection d'un ensemble dénombrable vers \( \eL\). Le lemme \ref{LEMooDLWFooNAJbbq} conclut que \( \eL\) est fini ou dénombrable.
        \item[Si \( \eK\) est infini]
            Nous procédons de la même manière, mais nous devons faire appel à des résultats plus technologiques pour maitriser la cardinalité. Nous considérons à nouveau l'application
            \begin{equation}
                \begin{aligned}
                    \varphi\colon \eK[X]\times \eN&\to \eL \\
                    (P,n)&\mapsto \varphi_P(n).
                \end{aligned}
            \end{equation}
            Cette application est encore surjective : \( \eL\preceq \eK[X]\times \eN\). Le lemme \ref{LEMooHWPHooZeWqns} nous assure que \( \eK[X]\approx \eK\) parce que \( \eK\) est infini. Ensuite la proposition \ref{PROPooFKBEooKXqujV} nous dit que \( \eK[X]\times \eN\approx \eK[X]\). Donc
            \begin{equation}
                \eK\approx \eK[X]\approx \eK[X]\times \eN\succeq \eF.
            \end{equation}
            Mais \( \eF\) est une extension de \( \eK\). Donc il existe une injection \( \eK\to \eF\), c'est-à-dire \( \eK\preceq \eF\).

            Ayant \( \eK\preceq \eF\preceq \eK\), le théorème \ref{THOooRYZJooQcjlcl} implique que \( \eK\approx \eF\).
    \end{subproof}
\end{proof}

\begin{lemma}[\cite{MonCerveau}]
    Soient des corps \( \eK\) et \( \eL\) ainsi qu'un morphisme de corps \( \rho \colon \eK\to \eL\). Si \( P\in \eK[X]\) a une racine dans \( \eK\), alors le polynôme \( \rho(P)\) a une racine dans \( \eL\).
\end{lemma}

\begin{proof}
    Nous notons \( P=\sum_kP_kX^k\). Si \( a\in \eK\) est une racine de \( P\), alors \( \sum_kP_ka^k=0\). Nous appliquons \( \rho\) à cette égalité : \( \sum_k\rho(P_k)\rho(a)^k=0\), c'est-à-dire \( \rho(P)\big( \rho(a) \big)=0\). Donc \( \rho(a)\in \eL\) est une racine de \( \rho(P)\).
\end{proof}

\begin{lemma}[\cite{BIBooYYZPooMgFzkp}]     \label{LEMooIIKYooHMNqYn}
    Nous considérons un triplet \( (\eK,\eL,\eF)\) où
    \begin{enumerate}
        \item
            \( \eK\), \( \eL\) et \( \eF\) sont des corps
        \item
            il existe \( a\in \eL\) algébrique sur \( \eK\) tel que \( \eL=\eK(a)\) et un morphisme de corps \( \alpha\colon \eK\to \eL\).
        \item
            \( \eF\) est une extension algébriquement close de \( \eK\) : il existe un morphisme \( \beta\colon \eK\to \eF\).
    \end{enumerate}
\end{lemma}

Note : en pratique, les corps \( \eL\) et \( \eF\) sont le plus souvent des sur-corps de \( \eK\), de telle sorte que les applications \( \alpha\) et \( \beta\) sont l'identité. En particulier, la conclusion de ce lemme s'écrit le plus souvent \( \sigma|_{\eK}=\id\). Il faut juste savoir que le Frido est un névrosé des notations précises.

\begin{proof}
    Comme \( \eL\) est monogène, si \( \mu_a\in \eK[X]\) est le polynôme minimal de \( a\in \eL\), alors les points \ref{ItemJCMooDgEHaji} et \ref{ItemJCMooDgEHajii} de la proposition \ref{PropURZooVtwNXE} disent que \( \eL\simeq \eK[a]\simeq \eK[X]/(\mu_a)\). Pour référence ultérieure, nous considérons un isomorphisme
    \begin{equation}
        \varphi\colon \eL\to \eK[X]/(\mu_a).
    \end{equation}

    Les coefficients de \( \mu_a\) sont dans \( \eK\), donc nous pouvons voir \( \mu_a\in \eF[X]\). Plus précisément, si \( \mu_a=\sum_ka_kX^k\), nous définissons
    \begin{equation}
        \mu'_a=\sum_k\beta(a_k)X^k\in \eF[X].
    \end{equation}
    Comme \( \eF\) est algébriquement clos, le polynôme \( \mu'_a\) possède une racine (au moins) \( b\in \eF\) : \( \mu'_a(b)=0\).

    Nous posons
    \begin{equation}
        \begin{aligned}
            \sigma'\colon \eK[X]/(\mu_a)&\to \eF \\
            \overline{ \sum_k s_kX^k }&\mapsto \sum_k\beta(s_k)b^k. 
        \end{aligned}
    \end{equation}
    \begin{subproof}
    \item[\( \sigma'\) est bien définie]
    Si \( P=\sum_ks_kX^k\) et \( \mu_a=\sum_ka_kX^k\) (\( a_k, s_k\in \eK\)), alors
    \begin{equation}
        \sigma'(\overline{ P+\mu_a })=\sigma'\big( \overline{ \sum_k (a_k+s_k)X^k } \big)=\sum_k\beta(a_k)b^k+\sum_ks_kb^k=\mu'_a(b)+\sigma'(\overline{ P })=\sigma'(\overline{ P }).
    \end{equation}
\item[\( \sigma'\) est un morphisme de corps]
    À justifier.
    %TODOooQHHNooLXwVzd

\item[\( \varphi\big( \alpha(k) \big)=\bar k\)]
    Quand nous parlons de \( \bar k\), nous parlons de la classe du polynôme de degré zéro donné par \( k\in \eK\).
    %TODOooJHBWooSgqSDC : justifier ce point à partir de la forme explicite de \varphi.

\item[La réponse]
    Nous posons 
    \begin{equation}
    \sigma=\sigma'\circ \varphi.
    \end{equation} 

    Pour tout \( k\in \eK\), 
    \begin{equation}
        (\sigma'\varphi\alpha)(k)=\sigma'(\bar k)=\beta(k),
    \end{equation}
    c'est ce qu'il fallait.
    \end{subproof}
\end{proof}

\begin{lemma}[\cite{BIBooYYZPooMgFzkp}] \label{LEMooUULTooYcytat}
    Soit un corps \( \eK\) muni de deux extensions \( \alpha\colon \eK\to \eL\) et \( \beta\colon \eK\to \eF\). Nous supposons que
    \begin{enumerate}
        \item
            \( \eL\) est algébrique sur \( \eK\);
        \item
            \( \eF\) est algébriquement clos.
    \end{enumerate}

    Alors il existe un morphisme de corps \( \sigma\colon \eL\to \eF\) tel que \( \sigma\circ \alpha=\beta\).
\end{lemma}

\begin{proof}
    Nous allons utiliser le lemme de Zorn \ref{LemUEGjJBc} sur l'ensemble
    \begin{equation}
        \mA=\Big\{  (\eM,\varphi)  \tq
        \begin{cases}
            \eM\text{ est un sous-corps de } \eL\\
            \alpha(\eK)\subset \eM\\
            \varphi\colon \eM\to \eF\text{ est une extension de corps}\\
            \varphi\circ \alpha=\beta
        \end{cases}
    \Big\}.
    \end{equation}
    Nous ordonnons (partiellement) cet ensemble en disant que \( (\eM_1,\varphi_1)<(\eM_2,\varphi_2)\) si \( \eM_1\subset \eM_2\) et \( \varphi_2|_{\eM_1}=\varphi_1\). Il se fait que \( \mA\) est un ensemble inductif et que nous pouvons donc lui appliquer le lemme de Zorn.

    Soit un élément maximal \( (\eM,\varphi)\). Nous allons montrer que \( \eM=\eL\).
    
    Soit \( l\in \eL\). Puisque \( \eL\) est une extension algébrique de \( \eK\), il existe un polynôme \( P\) à coefficients dans \( \alpha(\eK)\) tel que \( P(l)=0\). Mais comme \( \alpha(\eK)\subset \eM\), ce polynôme est également à coefficients dans \( \eM\). Donc \( l\) est un élément algébrique sur \( \eM\).

    Nous pouvons donc considérer le triplet \( \big( \eM,\eM(l),\eF \big)\) qui vérifie les hypothèses du lemme \ref{LEMooIIKYooHMNqYn}. Il existe donc un morphisme de corps \( \sigma\colon \eM(l)\to \eF\) tel que \( \sigma|_{\eM}=\varphi\).

    Nous avons
    \begin{equation}
        \sigma\circ\alpha=\sigma|_{\sigma(\eK)}\circ \alpha=\sigma|_{\eM}\circ \alpha=\varphi\circ \alpha=\beta.
    \end{equation}
    Donc l'élément \( \big( \eM(l),\sigma \big)\) majore \( (\eM,\varphi)\) dans \( \mA\).

    Par maximalité, nous déduisons que \( \eM=\eL\). Donc le morphisme \( \varphi\colon \eL\to \eF\) vérifie \( \varphi\circ \alpha=\beta\), ce qu'il nous fallait.
\end{proof}

\begin{theorem}[Steinitz\cite{BIBooRTTUooSPwAKJ,BIBooFQQWooIuoZyf}]       \label{THOooEDQKooLEGlDv}
    À propos de clôture algébrique.
    \begin{enumerate}
        \item
            Tout corps possède une clôture algébrique.
        \item
            Si \( \alpha_1\colon \eK\to \eF_1\) et \( \alpha_2\colon \eK\to \eF_2\) sont deux clôtures algébriques du même corps \( \eK\), alors il existe un isomorphisme de corps \( \varphi\colon \eF_1\to \eF_2\) tel que \( \varphi\circ\alpha_1=\alpha_2\).
    \end{enumerate}
\end{theorem}

\begin{proof}
    Nous commençons par l'existence, en plusieurs points.
    \begin{subproof}
    \item[Un ensemble]
    Nous considérons un ensemble \( \Omega\) qui contient \( \eK\), qui est strictement surpotent\footnote{Définition \ref{DEFooXGXZooIgcBCg}.} à \( \eK\) et qui est infini non dénombrable si \( \eK\) est fini. Par exemple \( \mP(\eK)\cup \eK\) si \( \eK\) est infini et \( \eR\cup \eK\) si \( \eK\) est fini (voir le théorème de Cantor \ref{THOooJPNFooWSxUhd}).

    \item[L'ensemble pour Zorn]
    Nous considérons l'ensemble des extensions algébriques de \( \eK\) contenues dans \( \Omega\), c'est-à-dire
    \begin{equation}
        \mA=\Big\{  (\eL, +_{\eL}, \times_{\eL})  \tq
        \begin{cases}
            \eK\subset \eL\subset \Omega\\
            (\eL, +_{\eL}, \times_{\eL}) \text{ est une extension algébrique de } (\eK, +, \times).
        \end{cases}
    \Big\}
    \end{equation}
    Nous ordonnons \( \mA\) par l'inclusion : nous disons que
    \begin{equation}
        (\eL_1, +_{1}, \times_1)< (\eL_2,+_2,\times_2)
    \end{equation}
    lorsque $(\eL_2, +_2, \times_2)$ est un sur-coprs de $(\eL_1,+_1,\times_1)$ (en particulier \( \eL_1\subset \eL_2\)).

\item[\( \mA\) est inductif]
    Soit une partie \( \mF=\{ (\eL_i, +_i,\times_i) \}_{i\in I}\) de \( \mA\) que nous supposons être totalement ordonnée. Nous allons lui trouver un majorant dans \( \mA\). Nous posons \( \eL=\bigcup_{i\in I}\eL_i\), et si \( a\in \eL_1\), \( b\in \eL_j\), alors nous définissons
    \begin{subequations}
        \begin{numcases}{}
            a+_{\eL}b=a+_kb\\
            a\times_{\eL}b=a\times_kb
        \end{numcases}
    \end{subequations}
    où \( k\in I\) est sélectionné de telle façon à avoir \( (\eL_i,+_i,\times_i )<(\eL_k,+_k,\times_k)\) et \( (\eL_j,+_j,\times_j )<(\eL_k,+_k,\times_k)\). Comme tous les corps \( L_i\) sont des sous-corps les uns des autres, c'est une bonne définition.
\item[Lemme de Zorn]
    Nous utilisons le lemme de Zorn \ref{LemUEGjJBc}. Nous notons \( (\eF,+,\times)\) un élément maximal de \( \mA\). Puisque \( \eK\) en est un sous-corps, il n'y a pas d'ambiguïté de noter \( +\) et \( \times\) ses opérations.
\item[Stratégie pour la suite]
    Nous allons montrer que si \( \eE\) est une extension algébrique de \( \eF\), alors \( \eE=\eF\) (le but est d'utiliser le lemme \ref{LEMooQSCGooMyCktA}).

\item[Un peu de cardinalité]
    D'abord, comme \( \eF\) est algébrique sur \( \eK\), l'ensemble \( \eF\) est équipotent à \( \eK\) si \( \eK\) est infini, et au plus dénombrable, si \( \eK\) est fini; c'est la proposition \ref{PROPooVPQFooScWvkS}. En bref :
    \begin{itemize}
        \item Si \( \eK\) est infini, \( \eK\approx \eF\approx \eE\prec\Omega\).
        \item Si \( \eK\) est fini, \(  \eK\preceq \eF\preceq \eE\prec \Omega \) où \( \eE\) est au maximum dénombrable et \( \Omega\) est indénombrable.
    \end{itemize}
    Dans tous les cas, \( \Omega\) est strictement surpotent à \( \eF\), et le lemme \ref{LEMooIVCBooHWQiZB} permet de dire
    \begin{equation}
        \eE\setminus \eF\preceq \eE\prec \Omega\approx\Omega\setminus \eF.
    \end{equation}
\item[Quelques injections]
    Il existe donc une injection \( \varphi\colon \eE\setminus \eF\to \Omega\setminus \eF\). Nous posons 
    \begin{equation}
        \begin{aligned}
            f\colon \eE &\to \Omega \\
            x           &\mapsto \begin{cases}
                x             & \text{si } x\in \eF     \\
                \varphi(x)    & \text{si } x\notin \eF.
            \end{cases}
        \end{aligned}
    \end{equation}
    Nous montrons que \( f\) est injective. Soient \( x,y\in \eE\) tels que \( f(x)=f(y)\). Si \( x,y\in \eF\), alors \( x=f(x)=f(y)=y\). Si \(x \in \eF\) et \( y\notin \eF\), alors \( x=\varphi(y)\) alors que \( x\in \eF\) et \( \varphi(y)\in \Omega\setminus F\); ce cas est impossible. Enfin si \( x\) et \( y\) sont hors de \( \eF\), alors \( f(x)=\varphi(x)\) et \( f(y)=\varphi(y)\); donc \( \varphi(x)=\varphi(y)\) et \( x=y\) par injectivité de \( \varphi\).

    Nous avons donc bien une injection \( f\colon \eE\to \Omega\).

\item[La maximalité]

    Nous pouvons mettre sur \( f(\eE)\subset \Omega\) la structure de corps venant de \( \eE\). Comme \( f(\eF)=\eF\), le corps \( f(\eE)\) est une extension algébrique de \( \eF\). Par maximalité, \( f(\eE)=\eF\).

    Mais si \( x\in\eE\setminus \eF\), alors \( f(x)\in \Omega\setminus \eF\). Donc en réalité nous avons aussi \( \eE\subset \eF\).
\item[Conclusion]
    En conclusion \( \eE=\eF\) et le lemme \ref{LEMooQSCGooMyCktA} termine en disant que \( \eF\) est une clôture algébrique de \( \eK\).
    \end{subproof}
    Nous passons à la partie «unicité» de la clôture algébrique. Étant donné que \( \eF_1\) est une extension algébrique de \( \eK\) et que \( \eF_2\) est algébriquement clos, le lemme \ref{LEMooUULTooYcytat} nous donne un morphisme de corps \( \sigma\colon \eF_1\to \eF_2\) tel que \( \sigma\circ \alpha_1=\alpha_2\).

    Nous sommes donc dans la situation où \( \sigma\colon \eF_1\to \eF_2\) est une extension de corps où \( \eF_1\) est algébriquement clos et \( \eF_2\) est algébrique. Le lemme \ref{LEMooYVHKooWhewKp} conclut que \( \sigma(\eF_1)=\eF_2\), c'est-à-dire que \( \sigma\) est surjectif. En tant que morphisme de corps, \( \sigma\) était déjà injective; elle est donc bijective.

    Donc \( \sigma\colon \eF_1\to \eF_2\) est un isomorphisme de corps vérifiant \( \sigma\circ \alpha_1=\alpha_2\).
\end{proof}

\begin{normaltext}
    Bien que \( \eC\) soit une extension algébriquement close de \( \eQ\), l'ensemble \( \eC\) n'est pas une clôture algébrique de \( \eQ\). C'est ce que nous montrons maintenant.
\end{normaltext}

\begin{lemma}       \label{LEMooRDIZooRjWNMa}
    Le corps \( \eC\) n'est pas une clôture algébrique\footnote{Clôture algébrique, définition \ref{DEFooREUHooLVwRuw}.} de \( \eQ\).    
\end{lemma}

\begin{proof}
    Nous montrons qu'il existe des éléments de \( \eC\) qui ne sont pas des racines de polynômes à coefficients rationnels. L'ensemble \( \eQ\) est dénombrable par la proposition \ref{PROPooDHIAooZysvNs}. L'ensemble des polynômes de degré \( n\) à coefficients dans \( \eQ\) est en bijection avec les \( n\)-uples de rationnels, c'est-à-dire avec \( \eQ^n\) qui est également dénombrable par la proposition \ref{PROPooDMZHooXouDrQ}. Enfin l'ensemble des polynômes à coefficients sur \( \eQ\) est l'union des polynômes de degré fixés, donc dénombrable par la proposition \ref{PROPooENTPooSPpmhY}.

    Jusqu'ici nous avons prouvé que l'ensemble des polynômes à coefficients rationnels était dénombrable. Or chaque polynôme possède une quantité finie de racines par le corolaire \ref{CORooUGJGooBofWLr}. Donc la partie de \( \eC\) constituée des nombres qui sont des racines de polynômes à coefficients dans \( \eQ\) est dénombrable. Mais \( \eC\) n'est pas dénombrable, donc possède des éléments qui ne sont pas des racines de polynômes.
\end{proof}

%---------------------------------------------------------------------------------------------------------------------------
\subsection{Polynômes à plusieurs variables}
%---------------------------------------------------------------------------------------------------------------------------

Nous avons déjà vu \( A[X,Y]\) lorsque \( A\) est un anneau en la définition~\ref{DEFooZNHOooCruuwI}.

\begin{definition}      \label{DEFooRHRKooPqLNOp}
    Soit un corps \( \eK\). Le corps \( \eK(X_1,\ldots, X_n)\) est le corps des fractions de l'anneau \( \eK[X_1,\ldots, X_n]\).
\end{definition}

\begin{definition}  \label{DEFooOCPHooXneutp}
    Soient un corps \( \eK\) et une extension \( \eL\) de \( \eK\) contenant les éléments \( \alpha_1\),\ldots, \( \alpha_n\) de \( \eK\). Nous définissons \( \eK(\alpha_1,\ldots, \alpha_n)\) comme étant l'intersection de tous les sous-corps de \( \eL\) contenant \( \eK\) et les \( \alpha_i\).
\end{definition}

La proposition suivante est analogue à~\ref{PROPooYSFNooFGbbCi}\ref{ITEMooATPTooVXKdlK}.

\begin{lemma}[\cite{MonCerveau}]        \label{LEMooQEJHooAmSNxU}
    Soient un corps \( \eK\), une extension \( \eL\) et des éléments \( \alpha_1,\ldots, \alpha_n\) dans \( \eL\). Alors
    \begin{equation}
        \eK(\alpha_1,\ldots, \alpha_n)=\{ r(\alpha_1,\ldots, \alpha_n)\tq r\in \eK(X_1,\ldots, X_n) \}.
    \end{equation}
\end{lemma}

\begin{proof}
    Ce que nous avons à droite est un corps : par exemple pour l'inverse, si \( r=P/Q\) alors \( r(\alpha_1,\ldots,\alpha_n)=P(\alpha_1,\ldots, \alpha_n)Q(\alpha_1,\ldots, \alpha_n)^{-1}\). Cet élément a un inverse en la fraction \( (Q/P)(\alpha_1,\ldots, \alpha_n)\).

    Donc à droite nous avons un sous-corps de \( \eL\) contenant \( \eK\) ainsi que les \( \alpha_i\). Donc
    \begin{equation}
        \eK(\alpha_1,\ldots, \alpha_n)\subset \big\{ r(\alpha_1,\ldots, \alpha_n)\tq r\in \eK(X_1,\ldots, X_n) \big\}.
    \end{equation}

    D'autre part, tout corps contenant \( \eK\) et les \( \alpha_i\) doit contenir tous les \( P(\alpha_1,\ldots, \alpha_n)\) (\( P\in \eK[X_1,\ldots, X_n]\)), leurs inverses ainsi que leurs produits; bref doit contenir tous les \( r(\alpha_1,\ldots, \alpha_n)\) avec \( r\in\eK[X_1,\ldots, X_n]\).
\end{proof}
