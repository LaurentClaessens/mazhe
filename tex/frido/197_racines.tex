% This is part of (everything) I know in mathematics
% Copyright (c) 2011-2013,2016-2022
%   Laurent Claessens
% See the file fdl-1.3.txt for copying conditions.

%+++++++++++++++++++++++++++++++++++++++++++++++++++++++++++++++++++++++++++++++++++++++++++++++++++++++++++++++++++++++++++
\section{Le groupe des racines de l'unité dans les nombres complexes}
%+++++++++++++++++++++++++++++++++++++++++++++++++++++++++++++++++++++++++++++++++++++++++++++++++++++++++++++++++++++++++++
\label{SecGJOLooWdMYVl}

%---------------------------------------------------------------------------------------------------------------------------
\subsection{Le groupe}
%---------------------------------------------------------------------------------------------------------------------------

\begin{definition}      \label{DEFooDUWPooZaAByH}
    Une \defe{racine \( n\)\ieme\ de l'unité}{racine de l'unité} dans un anneau est une racine du polynôme \( X^n-1\).
\end{definition}

\begin{lemmaDef}        \label{LEMooSXFBooYJmRTK}
    Soit
    % Attention: ici le U n'est pas \gU parce que \gU est pour le
    % groupe unitaire. Il y a surement plein de fautes plus bas.
    % Mais ce n'est pas grave parce que la différence est à peine
    % visible.
    \begin{equation}        \label{EqIEAXooIpvFPe}
        U_n=\{  e^{2i\pi k/n}  \tq k=0,\ldots, n-1 \}
    \end{equation}
    \begin{enumerate}
        \item
            L'ensemble \( U_n\) est un groupe pour la multiplication.
        \item
            L'ensemble \( U_n\) est l'ensemble des racines \( n\)\ieme\ de l'unité dans \( \eC\).
    \end{enumerate}
    \nomenclature[A]{\( U_n\)}{Le groupe des racines \( n\)\ieme\ de l'unité}
\end{lemmaDef}

\begin{proof}
    Il est vite vérifié que tous les éléments de \( U_n\) sont des racines de l'unité parce que
    \begin{equation}
        \big(  e^{2i\pi k/n} \big)^n= e^{2i\pi k}=1,
    \end{equation}
    entre autres à cause du lemme~\ref{LEMooHOYZooKQTsXW}.

    Cela nous donne déjà \( n\) racines pour \( X^n-1\) dans \( \eC\). Le théorème~\ref{ThoLXTooNaUAKR} nous indique qu'il ne peut pas y en avoir plus.
\end{proof}

\begin{normaltext}
    En ce qui concerne les notations, dans \( U_n\), le «U» signifie «unité». Cela n'a à peu près rien à voir avec le «U» du groupe \( \SU(n)\); dans ce dernier, le «U» est pour «unitaire».
\end{normaltext}

Un des intérêts du groupe des racines est qu'il permet de factoriser \( X^n-1\), comme nous le verrons via les polynômes cyclotomiques dans le lemme~\ref{LemKYGBooAwpOHD}.

\begin{lemma}[\cite{UQerHHk}]       \label{LEMooBKTNooTmtUNQ}
    Un nombre complexe algébrique dont tous les conjugués sont de module \( 1\) est une racine de l'unité\footnote{Définition \ref{DEFooDUWPooZaAByH}.}.
\end{lemma}

\begin{lemma}       \label{LemWHQGooXyeJiw}
    L'ensemble \( U_n\) est un groupe cyclique\footnote{Définition~\ref{DefHFJWooFxkzCF}.} d'ordre \( n\) généré par \( \xi= e^{2i\pi/n}\).
\end{lemma}

\begin{proof}
    Il y a les trois propriétés à vérifier pour que ce soit un groupe.
    \begin{subproof}
        \item[Neutre]
            Le nombre \( 1\) est une racine de l'unité.
    \item[Inverse]
        Si \( \omega\in U_n\) alors \( \omega^n=1\) et donc \( \omega\omega^{n-1}=1\), ce qui signifie que \( \omega^{n-1}\) est un inverse de \( \omega\). Il reste à voir que \( \omega^{n-1}\in U_n\). En effet \( \big( \omega^{n-1} \big)^n=(\omega^n)^{n-1}=1^{n-1}=1  \).
    \item[Associativité]
        Cas particulier de l'associativité dans \( \eC\).
    \end{subproof}
    Le fait que ce soit un groupe cyclique contenant \( n\) éléments fait partie de la définition.
\end{proof}

Le lemme suivant donne les autres générateurs.
\begin{lemma}   \label{LemcFTNMa}
    Le nombre \( \xi^a\) est un générateur de \( U_n\) si et seulement si \( \pgcd(a,n)=1\).
\end{lemma}

\begin{proof}
    Si \( \pgcd(a,n)=1\) alors le théorème de Bézout~\ref{ThoBuNjam} nous fournit des entiers \( u\) et \( v\) tels que \( ua+vn=1\). Alors nous avons
    \begin{equation}
        e^{2i\pi /n}= e^{2(ua+vn)i\pi/n}=( e^{2ai\pi/n})^u,
    \end{equation}
    ce qui signifie que \( \xi\) est dans le groupe engendré par \( \xi^a\), et par conséquent tout \( U_n\) est engendré.

    Pour l'implication inverse, nous utilisons le théorème de Bézout dans le sens inverse. Soit \( \xi^a\) un générateur de \( U_n\). Alors il existe \( u\) tel que \( (\xi^a)^u=\xi\), donc \( \xi^{au-1}=1\), c'est-à-dire qu'il existe \( v\) tel que \( au-1=vn\). Cette dernière égalité implique que \( \pgcd(a,n)=1\).
\end{proof}

\begin{example}
    Une conséquence tout à fait extraordinaire de ce lemme est que le nombre \( 7\) est générateur de \( \eZ/12\eZ\) (parce que \( \pgcd(7,12)=1\)). Or en solfège\index{solfège}, une quinte fait \( 7\) demi-tons, et une gamme en fait 12. Le cycle des quintes est donc générateur de la gamme chromatique\cite{YDXsAM}. Ce fait est connu des musiciens\footnote{Même ceux qui ignorent le théorème de Bézout.} depuis des siècles.
\end{example}

\begin{proposition}[Intersection par deux]      \label{PROPooIOQEooGMcCJm}
    Les ensembles \( U_{\alpha}\) et \( U_{\beta}\) ont une intersection réduite à \( \{ 1 \}\) si et seulement si \( \alpha\) et \( \beta\) sont premiers entre eux.
\end{proposition}

\begin{proof}
    Nous rappelons qu'une racine \( \alpha\)\ieme\ de l'unité peut s'écrire sous la forme \( e^{2i\pi k/\alpha}\) avec \( 0\leq k<\alpha\).
    \begin{subproof}
    \item[Sens direct]
        Par contraposée, nous supposons que \( \alpha\) et \( \beta\) ne sont pas premiers entre eux, et nous notons \( d\) leur \( \pgcd\). Nous nommons \( \alpha=d\alpha'\) et \( \beta=d\beta'\). Pour trouver une intersection entre \( \gU_{\alpha}\) et \( \gU_{\beta}\) nous devons trouver une valeur de \( 0<k<\alpha\) telle que
        \begin{equation}
            ( e^{2i\pi k/\alpha})^{\beta}= e^{2i\pi k\beta/\alpha}=1,
        \end{equation}
        c'est-à-dire une valeur de \( k\) telle que \( k\beta/\alpha\) soit un entier. Mais \( k\beta/\alpha=k\beta'/\alpha'\) et par conséquent prendre \( k=\alpha'\) fonctionne. Surtout que par hypothèse \( d>1\) et donc \( k=\alpha'<\alpha\).
    \item[Sens réciproque]
        Supposons maintenant que \( \alpha\) et \( \beta \) soient premiers entre eux. Soit \( z\in U_{\alpha}\cap U_{\beta}\). Le fait que \( z\) soit une racine \( \alpha\)\ieme\ de l'unité implique qu'il existe un \( k<\alpha\) tel que \( z= e^{2i\pi k/\alpha}\). Mais si \( z\) est également une racine \( \beta\)\ieme\ de l'unité, alors \( z^{\beta}=1\), c'est-à-dire que \( k\beta/\alpha\) doit être un entier, soit \( l\) cet entier. Nous avons
        \begin{equation}
            k\beta=l\alpha.
        \end{equation}
        Si \( k>0\), comme le nombre \( \alpha\) divise \( k\beta\), cela conduirait via le lemme de Gauss~\ref{LemPRuUrsD} à dire que \( \alpha\) divise \( k\). Mais \( \alpha\) ne peut pas diviser \( k\) parce que nous avions supposé que \( k\) était strictement plus petit que \( \alpha\). Donc \( k = 0\) et \( z = 1\).
    \end{subproof}
\end{proof}

\begin{proposition}[Intersection : le cas général\cite{MonCerveau}]   \label{PropFDDHooEyYxBC}
    Soient des entiers positifs \( \alpha_1,\ldots, \alpha_p\). Nous avons
    \begin{equation}
        \bigcup_{i=1}^p U_{\alpha_i}=\{ 1 \}
    \end{equation}
    si et seulement si \( \pgcd(\alpha_1,\ldots, \alpha_p)=1\) (c'est-à-dire que les \( \alpha_i\) sont premiers dans leur ensemble).
\end{proposition}

\begin{proof}
    Nous décomposons les \( \alpha_i\) en facteurs premiers\footnote{Théorème~\ref{ThoAJFJooAveRvY}.} de la façon suivante : \( \alpha_i=\prod_{k\in \eN}p_k^{\alpha_i^{(k)}}\) où les \( p_k\) sont les nombres premiers.

    \begin{subproof}
    \item[Caractérisation par une décomposition en facteurs premiers]
        Les éléments \( z\) différents de \( 1\) dans \( U_{\alpha_1}\) s'écrivent sous la forme
        \begin{equation}
            z= e^{2i\pi k/\alpha_1}
        \end{equation}
        avec \( 0<k<\alpha_1\).

        Pour tout \( i\neq 1\), le fait que \( z\in U_{\alpha_i}\cap U_{\alpha_1}\) se traduit par le fait que \( \big(  e^{2i\pi k/\alpha_1} \big)^{\alpha_i}=1\), c'est-à-dire que \( \alpha_ik/\alpha_1\) est entier, donc que \( \alpha_1\) divise \( k\alpha_i\). Par conséquent il existera un élément différent de \( 1\) dans l'intersection des \( U_{\alpha_i}\) si et seulement si il existe un entier \( k\) strictement compris entre \( 0\) et \( \alpha_1\) pour lequel \( \alpha_1\) divise tous les \( k\alpha_i\).

        Un entier \( 0<k<\alpha_1\) convient si et seulement si pour tout \( l\), la puissance de \( p_l\) dans la décomposition de \( k\) est au moins égale à
        \begin{equation}
            \alpha_1^{(l)}-\alpha_i^{(l)}
        \end{equation}
        pour tout \( l\).
    \item[Sens direct]
        L'hypothèse \( \pgcd(\alpha_1,\ldots, \alpha_p)\neq 1\) implique qu'il existe un \( l\) pour lequel tous les \( \alpha_i^{(l)}\) sont non nuls. Nous construisons le \( k\) voulu en prenant pour tout \( p_i\) la même puissance que celle dans \( \alpha_1\), sauf pour \( p_l\) pour lequel nous prenons la puissance \(  \alpha_1^{(l)}-\min_i\{   \alpha_i^{(l)} \} \). Le minimum en question est strictement positif, ce qui donne un \( k\) strictement inférieur à \( \alpha_1\).
    \item[Sens réciproque]
        Si \( \pgcd(\alpha_1,\ldots, \alpha_p)=1\) alors pour tout \( l\), il existe un \( i\) tel que \( \alpha_i^{(l)}=0\). Donc pour tout \( l\), la puissance de \( p_l\) dans la décomposition de \( k\) est au moins \( \alpha_1^{(l)}\). Cela implique que \( k\geq \alpha_1\), ce qui est impossible.
    \end{subproof}
\end{proof}

\begin{definition}\label{DefLYGTooFPOYGZ}
    Les générateurs de \( U_n\) sont les \defe{racines primitives}{racine!de l'unité!primitive}\footnote{parce qu'en prenant les puissances successives de l'une d'entre elles, nous retrouvons toutes les racines de l'unité, voir aussi la définition~\ref{DefnPNCFO}.} de l'unité dans \( \eC\). Nous nommons \( \Delta_n\) leur ensemble :
\begin{equation}
    \Delta_n=\{  e^{2ki\pi/n}\tq 0\leq k\leq n-1,\pgcd(k,n)=1 \}.
\end{equation}
\end{definition}
Nous avons par exemple
\begin{subequations}
    \begin{align}
        \Delta_1  &=\{ 1 \}             \\
        \Delta_2  &=\{ e^{\pi i} \}     \\
        \Delta_4  &=\{ e^{\pi i/2}, e^{3\pi i/2} \}.
    \end{align}
\end{subequations}
Notons que \( 1\in \Delta_d\) seulement avec \( d=1\).

%---------------------------------------------------------------------------------------------------------------------------
\subsection{Fonction indicatrice d'Euler (première partie)}
%---------------------------------------------------------------------------------------------------------------------------

Nous introduisons ici la fonction indicatrice d'Euler et ses liens basiques avec les racines de l'unité. Pour les propriétés plus avancées, voir~\ref{subSecKGDFooAbETjs}.

%---------------------------------------------------------------------------------------------------------------------------
\subsection{Introduction par les racines de l'unité}
%---------------------------------------------------------------------------------------------------------------------------

\begin{definition}      \label{DEFooWYIGooRVBTil}
La fonction \( \varphi\) donnée par
\begin{equation}    \label{EqEulerGqPsvi}
    \varphi(n)=\Card(\Delta_n)
\end{equation}
est l'\defe{indicatrice d'Euler}{indicatrice d'Euler}\index{Euler!indicatrice}.
\end{definition}
Si \( p\) est un nombre premier, alors \( \varphi(p)=p-1\).

\begin{lemma}       \label{LemKcpjee}
    Nous avons
    \begin{equation}        \label{EqpZuIyL}
        U_n=\bigcup_{d\divides n}\Delta_d
    \end{equation}
    et l'union est disjointe. Nous avons aussi la formule
    \begin{equation}        \label{EqTPHqgJ}
        n=\sum_{d\divides n}\varphi(d).
    \end{equation}
\end{lemma}

\begin{proof}
    À l'application \( x\mapsto  e^{2i\pi x}\) près, nous pouvons considérer
    \begin{equation}
        \Delta_d=\left\{ \frac{ k }{ d }\tq k=0,\ldots, d-1, \pgcd(k,d)=1 \right\},
    \end{equation}
    c'est-à-dire l'ensemble des fractions irréductibles dont le dénominateur est \( d\). L'union des \( \Delta_d\) sera donc disjointe.

    Toujours à l'application \( x\mapsto  e^{2i\pi x}\) près, le groupe \( U_n\) est donné par
    \begin{equation}
        U_n=\left\{ \frac{ k }{ n }\tq k=0,\ldots, n-1 \right\}.
    \end{equation}
    L'égalité \eqref{EqpZuIyL} revient maintenant à dire que toute fraction de la forme \( \frac{ k }{ n }\) s'écrit de façon irréductible avec un dénominateur qui divise \( n\).

    La relation \eqref{EqTPHqgJ} consiste à prendre le cardinal des deux côtés de \eqref{EqpZuIyL}. Nous avons \( \Card(U_n)=n\) et l'union étant disjointe, à droite nous avons la somme des cardinaux.


    Pour chaque diviseur \( d\) de \( n\) nous considérons l'ensemble
    \begin{equation}
        \Phi_n(d)=\{ m\in \eN\tq \pgcd(m,n)=\frac{ n }{ d } \}.
    \end{equation}
    Étant donné que tous les entiers entre \( 1\) et \( n\) ont un pgcd avec \( n\) qui est automatiquement un quotient de \( n\) nous avons
    \begin{equation}
        \{ 1,\ldots, n \}=\bigcup_{d\divides n}\Phi_n(d)
    \end{equation}
    où l'union est disjointe. Par ailleurs nous savons que si \( \pgcd(a,b)=1\), alors \( \pgcd(ka,kb)=k\). Donc si \( m\in \Delta_d\), alors \( m\cdot \frac{ n }{ d }\) appartient à \( \Phi_n(d)\). En d'autres termes, \( a\mapsto \frac{ n }{ d }a\) est une bijection entre \( \Delta_d\) et \( \Phi_n(d)\).

    Nous avons donc \( \Card(\Phi_n(d))=\Card(\Delta_d)=\varphi(d)\) et finalement
    \begin{equation}
        \Card\{ 1,\ldots, n \}=\sum_{d\divides n}\Card(\Phi_n(d))=\sum_{d\divides n}\varphi(d).
    \end{equation}
\end{proof}

\begin{lemma}       \label{LEMooBEJOooDqTirj}
    Si \( p\) est un nombre premier, alors \( \varphi(p^n)=p^n-p^{n-1}\).
\end{lemma}

\begin{proof}
    Les éléments de \( \{ 1,\ldots,p^n \}\) qui ont un \( \pgcd\) différent de \( 1\) avec \( p^n\) sont des nombres qui s'écrivent sous la forme \( qp\) avec \( q\leq p^{n-1}\)\footnote{Corolaire~\ref{CORooQIMHooUzLUJY}.}. Il y a évidemment \( p^{n-1}\) tels nombres.

    Par conséquent le cardinal de \( P_{p^n}\) est \( \varphi(p^{n})=p^n-p^{n-1}\).
    %TODOooWJIYooYtATMi Il faut élucider ce qu'est P_n, voir 2754128708
\end{proof}

\begin{probleme}
    %2754128708
    \( P_n\) n'a pas été défini.

    Définition proposée (et vue par après): \( P_n = \{ m \in \eN \tq \pgcd(m,n) = 1 \}. \) À mettre donc en lien avec \( \Delta_d\).
\end{probleme}

%---------------------------------------------------------------------------------------------------------------------------
\subsection{Générateurs}
%---------------------------------------------------------------------------------------------------------------------------

\begin{proposition}     \label{PropZnmuphiGensn}
    Soit \( n\in\eN\setminus\{ 0 \}\) et le groupe (additif) \( \eZ/n\eZ\). L'élément \( [x]_n\) est un générateur de \( \eZ/n\eZ\) si et seulement si \( x\in P_n\). En particulier \( \eZ/n\eZ\) est un groupe contenant \( \varphi(n)\) générateurs.
\end{proposition}

\begin{proof}
    Nous avons \( \gr\big( [1]_n \big)=\eZ/n\eZ\). L'élément \( [x]_n\) sera générateur si et seulement si il génère \( [1]_n \), c'est-à-dire si il existe \( u\) tel que \( u[x]_n=[1]_n\). Cette dernière égalité étant une égalité de classes dans \( \eZ/n\eZ\), elle sera vraie si et seulement si il existe \( v\) tel que
    \begin{equation}
        ux+vn=1.
    \end{equation}
    Cela signifie entre autres que\footnote{Corolaire~\ref{CorgEMtLj}} \( x\eZ+n\eZ=\eZ\), et aussi que \( \pgcd(x,n)=1\) par le théorème de Bézout~\ref{ThoBuNjam}, et donc que \( x\in P_n\).
\end{proof}

\begin{corollary}\label{CORooMBLSooMHKmAq}
    Un groupe monogène d'ordre \( n\) possède \( \varphi(n)\) générateurs, où \( \varphi\) est la fonction indicatrice d'Euler définie en~\ref{DEFooWYIGooRVBTil}.
\end{corollary}

\begin{proof}
    Le théorème~\ref{THOooDOMZooOEYHAe} nous dit qu'un groupe monogène d'ordre \( n\) est isomorphe à \( \eZ/n\eZ\). La proposition~\ref{PropZnmuphiGensn} nous indique que \( \eZ/n\eZ\) possède \( \varphi(n)\) générateurs.
\end{proof}

%---------------------------------------------------------------------------------------------------------------------------
\subsection{Fonction indicatrice d'Euler (propriétés)}
%---------------------------------------------------------------------------------------------------------------------------
\label{subSecKGDFooAbETjs}

\begin{corollary}       \label{CorlvTmsf}
    L'indicatrice d'Euler est multiplicative : si \( p\) est premier avec \( q\), alors \( \varphi(pq)=\varphi(p)\varphi(q)\). De plus si \( p\) et \( q\) sont premiers entre eux,
    \begin{equation}
        \varphi(pq)=(p-1)(q-1).
    \end{equation}
\end{corollary}

\begin{proof}
    Nous savons que si \( p\) et \( q\) sont premiers entre eux, alors le théorème~\ref{ThoLnTMBy} nous donne l'isomorphisme de groupe
    \begin{equation}
        (\eZ/pq\eZ,+)\simeq(\eZ/p\eZ,+)\times(\eZ/q\eZ,+).
    \end{equation}
    Un élément \( (x,y)\) est générateur du produit si et seulement si \( x\) est générateur de \( \eZ/p\eZ\) et \( y\) est générateur de \( \eZ/q\eZ\). Par la proposition~\ref{PropZnmuphiGensn}, il y a \( \varphi(p)\varphi(q)\) tels éléments. Par ailleurs le nombre de générateurs de \( \eZ/pq\eZ\) est \( \varphi(pq)\), d'où l'égalité.

    Si \( p\) est premier, nous avons \( \varphi(p)=p-1\) parce que tous les entiers de \( \{ 1,\ldots, p-1 \}\) sont premiers avec \( p\).
\end{proof}

%---------------------------------------------------------------------------------------------------------------------------
\subsection{Décomposition en éléments simples}
%---------------------------------------------------------------------------------------------------------------------------
\label{SUBSECooSIYXooDDHUdD}


\begin{lemma}[\cite{BIBooOHJHooJHJyyp, MonCerveau}]     \label{LEMooABJMooJTUpgV}
    Soient des nombres complexes distincts \( a_1,\ldots, a_N\). Alors pour tout \( z\notin\{ a_i \}_{i=1,\ldots, N}\),
    \begin{equation}
        \prod_{i=1}^N  \frac{1}{z-a_i }=\sum_{i=1}^N\frac{ \lambda_i }{ z-a_i }
    \end{equation}
    où
    \begin{equation}
        \lambda_i=\prod_{j\neq i}\frac{1}{ a_i-a_j }.
    \end{equation}
\end{lemma}

\begin{proof}
    Nous posons \( P(z)=\prod_{i=1}^N(z-a_i)\), et nous calculons :
    \begin{subequations}
        \begin{align}
            \sum_{i=1}^N\frac{ \lambda_i }{ z-a_i } &=  \sum_{i=1}^N\frac{ \lambda_i\prod_{k\neq i}(z-a_k) }{ \prod_{k=1}^N(z-a_i) }\\
                                                    &=  \frac{1}{ P(z) }\sum_{i=1}^N\lambda_i\prod_{k\neq i}(z-a_k)
        \end{align}
    \end{subequations}
    Il s'agit maintenant de prouver que la somme de gauche vaut toujours \( 1\). Nous posons
    \begin{equation}
        S(z)=\sum_{i=1}^N\frac{ \prod_{k\neq i}(z-a_k) }{ \prod_{k\neq i}(a_i-a_k) }.
    \end{equation}
    Calculons \( S(a_l)\) :
    \begin{equation}
        S(a_l)=\sum_{i=1}^N\frac{ \prod_{k\neq i}(a_l-a_k) }{ \prod_{k\neq i}(a_i-a_k) }.
    \end{equation}
    Pour les termes \( i\neq l\), le numérateur est nul, car il contient le facteur \( a_l-a_l=0\). Donc la somme se réduit au seul terme \( i=l\) :
    \begin{equation}
        S(a_l)=\frac{ \prod_{k\neq l}(a_l-a_k) }{ \prod_{k\neq l}(a_l-a_k) }=1.
    \end{equation}
    Le polynôme \( S-1\) est donc un polynôme de degré \( N-1\) qui possède \( N\) racines distinctes. Le théorème \ref{ThoLXTooNaUAKR} implique que \( S-1=0\) et donc que \( S=1\) comme nous le voulions.
\end{proof}

Il est possible de décomposer une fraction rationnelle en fractions dites «simples». Si \( | z |<1\) nous avons par exemple la décomposition
\begin{equation}        \label{EqDWYBooJIMBAt}
    \frac{1}{ 1-z^r }=\sum_{\omega\in U_r}\frac{ A_{\omega} }{ \omega-z }
\end{equation}
où \( U_r\) est le groupe des racines \( r\)\ieme\ de l'unité défini en \eqref{EqIEAXooIpvFPe}. Les nombres \( A_{\omega}\) peuvent alors être déterminés en effectuant la somme. Le dénominateur commun sera \( 1-z^r\) tandis que les \( A_{\omega}\) sont déterminés en égalant le numérateur à \( 1\).

\begin{example}
    Pour décomposer la fraction \( \frac{1}{ 1-x^2 }\) nous savons que les racines sont \( \pm 1\). Donc nous écrivons
    \begin{equation}
        \frac{1}{ 1-x^2 }=\frac{ A }{ 1-x }+\frac{ B }{ 1+x }.
    \end{equation}
    Nous trouvons les valeurs de \( A\) et \( B\) en effectuant la somme :
    \begin{equation}
        \frac{ A(1+x)+B(1-x) }{ 1-x^2 }=\frac{ A+B+(A-B)x }{ 1-x^2 }.
    \end{equation}
    Les coefficients \( A\) et \( B\) doivent donc vérifier \( A+B=1\) et \( A-B=0\). Au final,
    \begin{equation}
        \frac{1}{ 1-x^2 }=\frac{1}{ 2(1-x) }+\frac{1}{ 2(1+x) }.
    \end{equation}
\end{example}

%+++++++++++++++++++++++++++++++++++++++++++++++++++++++++++++++++++++++++++++++++++++++++++++++++++++++++++++++++++++++++++
\section{Chiffrement RSA}
%+++++++++++++++++++++++++++++++++++++++++++++++++++++++++++++++++++++++++++++++++++++++++++++++++++++++++++++++++++++++++++
\label{SecEVaFYi}
\index{groupe!fini}
\index{groupe!permutation}
\index{groupe!partie génératrice}
\index{anneau!\( \eZ/n\eZ\)}
\index{nombre!premier}

Ce passage sur RSA provient en bonne partie de la page de Wikipédia\cite{ooRIFDooNxOehF}.

Alice veut envoyer un message à Bob. L'idée est que Bob va donner à Alice une clef publique qui va permettre de chiffrer le message, tandis que Bob va garder pour lui une clef privée qui permet de déchiffrer.

%---------------------------------------------------------------------------------------------------------------------------
\subsection{Mise en place par Bob}
%---------------------------------------------------------------------------------------------------------------------------

Bob se crée une paire de clef publique, clef privée de la façon suivante.
\begin{enumerate}
    \item
        Bob choisit deux nombres premiers distincts \( p,q\).
    \item
        Il calcule \( n=pq\) .
    \item
        Par le corolaire~\ref{CorlvTmsf}, l'indicatrice d'Euler \( \varphi(n)=(p-1)(q-1)\) est facile à calculer pour Bob.
    \item
        Bob choisit \( e\in \eN\) premier avec \( \varphi(n)\), puis \( d\) tel que \( ed\in[1]_{\varphi(n)}\).
\end{enumerate}
Maintenant la paire est : clef publique \( (n,e)\) et clef privée \( (n,d)\)\footnote{Le fait que \( e\) soit public et \( d\) soit privé est une convention. \( e\) comme  \emph{encryption} et \( d\) comme \emph{decryption}.}.

Bob envoie la paire \( (n,e)\) à Alice.

\begin{remark}
    Ici nous ne supposons pas que la communication soit sure. Une tierce personne peut intercepter le message. D'ailleurs en principe, les gens publient leur clef publique sur leurs sites, voire sur des sites dédiés. Le problème de l'identification reste à résoudre \wikipedia{en}{Key_signing_party}{à l'ancienne}.
\end{remark}

%---------------------------------------------------------------------------------------------------------------------------
\subsection{Chiffrement}
%---------------------------------------------------------------------------------------------------------------------------

Nous chiffrons en utilisant la clef publique \( (n,e)\). D'abord Alice se débrouille pour transformer son message en un nombre plus petit que \( n\). Soit \( M\) ce message. Alice code \( M\) en
\begin{equation}
    C=M^e\mod n.
\end{equation}
Tout le truc est que nous allons voir que l'application \( x\mapsto x^e\) est une bijection de \( \eF_n\), et que l'inverse est facile à calculer par Bob, et difficile pour les autres. Alice envoie \( C\) à Bob. Encore une fois, nous ne supposons pas que cette communication soit privée. Le nombre \( C\) peut être intercepté.

%---------------------------------------------------------------------------------------------------------------------------
\subsection{Déchiffrement}
%---------------------------------------------------------------------------------------------------------------------------

Nous allons montrer que \( M=C^d\mod n\), et donc que Bob, connaissant \( (n,d)\), peut déchiffrer. D'abord
\begin{equation}
    C^d=(M^e)^d=M^{ed},
\end{equation}
mais nous savons qu'il existe \( k\) tel que
\begin{equation}
    ed=1+k\varphi(n)=1+k(p-1)(q-1).
\end{equation}
L'étape astucieuse est de remarquer que
\begin{equation}    \label{EqreeHgn}
    M^{1+k(p-1)(q-1)}\in [M]_p\cap[M]_q.
\end{equation}
Pour montrer cela nous utilisons le petit théorème de Fermat~\ref{ThoOPQOiO}\ref{ITEMooRNIVooOIzqgc}.
\begin{itemize}
    \item Si \( M\) est premier avec \( p\), alors \( M^{p-1}\in[1]_p\).
    \item Si \( M\) n'est pas premier avec \( p\), alors \( M\) est multiple de \( p\) et on sait que \( M^{p-1}\in[0]_p=[M]_p\).
\end{itemize}
Dans les deux cas nous avons \eqref{EqreeHgn}. Le nombre \( M^{1+k\varphi(n)}-M\) est donc à la fois multiple de \( p\) et de \( q\).

Le lemme chinois~\ref{LemCtUeGA} nous dit immédiatement\footnote{C'est ici qu'il est important que \( p\) ne soit pas égal à \( q\). Si \( p=q\), alors le lemme chinois ne fonctionne pas.} qu'alors
\begin{equation}
    M^{1+k\varphi(n)}-M
\end{equation}
est un multiple de \( pq=n\), c'est-à-dire que
\begin{equation}
    C^d=M^{ed}\in [M]_n.
\end{equation}

Si on ne croit pas au lemme chinois, on peut utiliser le lemme de Gauss. Posons
\begin{equation}
    M^{1+k\varphi(n)}-M=ap=bq.
\end{equation}
Dans ce cas \( p\) divise \( bq\), mais \( q\) est premier avec \( p\), donc le lemme de Gauss~\ref{LemSdnZNX} nous enseigne\footnote{Ici aussi, si \( p=q\), ça ne marche pas.} que \( p\) divise \( b\).

%---------------------------------------------------------------------------------------------------------------------------
\subsection{Une imprudence à ne pas commettre}
%---------------------------------------------------------------------------------------------------------------------------

Nous avons pris deux cas selon que \( M\) soit ou non premier avec \( p\). Une question qui se pose est la suivante : est-ce que c'est une bonne idée d'envoyer un message qui ne soit pas premier avec \( p\) ?

Si nous savons que \( M\) n'est pas premier avec \( p\), alors nous avons \( M^e=l^ep^e\) et \( n=pq\) qui sont publics. Donc un calcul de PGCD permettrait de trouver \( p\).

Il faut cependant savoir que \label{PageAKTBooMDeQxY}
\begin{itemize}
    \item La probabilité que ça arrive est infime : vu que \( M\) est entre \( 0\) et \( n=pq\), les multiples de \( p\) possibles sont \( p,2p,\cdots pq\). Il y a donc une chance sur \( p\) que cela arrive. Typiquement avec des \( p\) de l'ordre de \( 10^{120}\), on peut utiliser RSA chaque milliseconde sur chaque atome de l'univers depuis le début des temps que ça ne se serait presque certainement pas encore produit.
    \item
        De toutes façons Alice ne sait pas vérifier si son message est premier avec \( p\) parce qu'elle ne connaît pas \( p\).
    \item
        En conclusion la partie de la preuve qui montre que \( M^{1+\varphi(n)}\in [M]_p\cap[M]_q\) dans le cas \( M\) non premier avec \( p\) est, à toutes fins pratiques, inutile parce que ce cas de figure ne se présentera jamais dans toutes l'histoire de l'univers, même pas avec une civilisation intelligente autour de chaque étoile.
\end{itemize}

\begin{probleme}\label{ProbGAYFooZATuYy}
    Est-ce que ces trois points sont corrects ?
\end{probleme}

%---------------------------------------------------------------------------------------------------------------------------
\subsection{Problèmes calculatoires}
%---------------------------------------------------------------------------------------------------------------------------

Pour implémenter RSA, il faut pouvoir faire (au moins) trois choses :
\begin{enumerate}
    \item
        Trouver de grands nombres premiers.
    \item
        Trouver des couples de Bézout.
    \item
        Calculer \( M^e\) lorsque \( e\) est très grand.
\end{enumerate}
En ce qui concerne le problème de trouver des nombres premiers, c'est compliqué, mais il faut savoir qu'il y en a plein. À \( 120\) chiffres, il y a environ autant de nombres premiers que d'atomes dans \( 10^{20}\) fois l'univers connu. Cela rend impossible toute tentative de factoriser un grand nombre en essayant toutes les possibilités. Même pas en science-fiction\footnote{Cela donne une idée des connaissances en math des klingons, dont le docteur Spock parvient à craquer le code mentalement en deux heures.}.

Trouver des nombres \( u\) et \( v\) tels que \( Au+Bv=\pgcd(A,B)\) est un problème expliqué en~\ref{subSecIpmnhO}.

En ce qui concerne le calcul de \( M^e\) lorsque \( e\) est grand, il n'est évidemment pas pensable de faire \( M\cdot M\cdot\ldots M\) avec \( e\) facteurs. Un truc pour calculer en moins d'étapes est l'\defe{exponentiation rapide}{exponentielle!rapide}. Si \( e=2k\) est pair, nous calculons
\begin{equation}
    M^e=(M^k)^2;
\end{equation}
si \( e=2k+1\) alors nous calculons
\begin{equation}
    M^e=M(M^k)^2.
\end{equation}
Le calcul prend alors seulement environ \(  \log_2(e)  \) étapes. Pour donner une idée,
\begin{equation}
    \log_2(10^{120})\simeq 400.
\end{equation}
Très raisonnable, mais un ordinateur reste indispensable.

%---------------------------------------------------------------------------------------------------------------------------
\subsection{La solidité de RSA}
%---------------------------------------------------------------------------------------------------------------------------

La solidité de la méthode repose sur deux conjectures (non démontrées !!) :
\begin{itemize}
    \item Pour déchiffrer il \emph{faut} connaitre \( p\) et \( q\).
    \item La difficulté de trouver \( p\) et \( q\) en partant de \( n=pq\) est exponentielle en \( n\).
\end{itemize}
Dans la méthode de déchiffrage proposée ici, \( p\) et \( q\) sont utilisés pour calculer \( d\) qui est solution de \( ed=[1]_{\varphi(n)}\). La seule formule connue pour calculer \( \varphi(n)\) est \( \varphi(n)=(p-1)(q-1)\). Si on trouve plus simple, alors RSA peut être craqué.

%---------------------------------------------------------------------------------------------------------------------------
\subsection{Note non mathématique pour doucher l'enthousiasme}
%---------------------------------------------------------------------------------------------------------------------------

Il est souvent dit\cite{ooTODUooAhFHQk} que différents systèmes de chiffrement peuvent aider à avoir des discussions «discrètes» dans les régimes totalitaires. La technologie au service de la démocratie, voilà qui enthousiasme la jeunesse\footnote{Cela dit, le navigateur Tor\cite{ooHBLCooYtBBfx}, qui est un pur produit de RSA, permet effectivement d'accéder en France aux sites bloqués pour apologie du terrorisme (mars 2015).}. La réalité est qu'il est souvent possible de craquer un système de chiffrement arbitrairement complexe, même sans connaitre le petit théorème de Fermat \ldots

\notbool{isBook}{
\begin{center}
    \includegraphics[width=10cm]{pictures_bitmap/security.png}        \\
    \url{http://xkcd.com/538/} \href{http://creativecommons.org/licenses/by-nc/2.5/}{ Creative Commons Attribution-NonCommercial 2.5 License}.
\end{center}%
}{}

\noindent \ldots tout dépend du contexte.

%+++++++++++++++++++++++++++++++++++++++++++++++++++++++++++++++++++++++++++++++++++++++++++++++++++++++++++++++++++++++++++
\section{Polynômes cyclotomiques}
%+++++++++++++++++++++++++++++++++++++++++++++++++++++++++++++++++++++++++++++++++++++++++++++++++++++++++++++++++++++++++++

%---------------------------------------------------------------------------------------------------------------------------
\subsection{Définitions et propriétés}
%---------------------------------------------------------------------------------------------------------------------------

\begin{definition}  \label{DefXGHooRAXlpp}
    Le \defe{polynôme cyclotomique}{polynôme!cyclotomique} d'indice \( n\) est le polynôme
    \begin{equation}    \label{EqLjGYKK}
        \phi_n(X)=\prod_{z\in\Delta_n}(X-z)
    \end{equation}
    où \( \Delta_n\) est l'ensemble des racines primitives de l'unité de la définition~\ref{DefLYGTooFPOYGZ} :
    \begin{equation}
        \Delta_n=\{  e^{2ik\pi/n}\tq 0\leq k\leq n-1\tq \pgcd(k,n)=1 \},
    \end{equation}
\end{definition}

Le polynôme \( \phi_n\) est un polynôme unitaire de degré \( \varphi(n)\) où \( \varphi\) est l'indicatrice d'Euler\footnote{Définie par l'équation~\ref{EqEulerGqPsvi}.} \( \varphi(n)\). Nous avons par exemple
\begin{subequations}
    \begin{align}
        \Delta_1  &=\{  1 \}  \\
        \Delta_2  &=\{ -1 \}  \\
        \Delta_3  &=\{  e^{2\pi i/3}, e^{4\pi i/3} \}
    \end{align}
\end{subequations}
et les premiers polynômes cyclotomiques sont donnés par
\begin{subequations}
    \begin{align}
        \phi_1(X) &=X-1       \\
        \phi_2(X) &=X+1       \\
        \phi_3(X) &=X^2+X+1.
    \end{align}
\end{subequations}
Pour le dernier nous avons utilisé le fait que \( e^{6\pi i/3}=1\) et \( e^{4\pi i/3+ e^{2\pi i/3}}=-1\).

\begin{lemma}   \label{LemKYGBooAwpOHD}
    Le polynôme \( X^n-1\) se factorise des diverses manières suivantes :
    \begin{equation}
        X^n-1=\prod_{z\in U_n}(X-z)=\prod_{d\divides n}\prod_{z\in \Delta_d}(X-z)=\prod_{d\divides n}\phi_d(X)
    \end{equation}
    où \( \gU_n\) est défini en \ref{LEMooSXFBooYJmRTK}.
\end{lemma}

\begin{proof}
    En ce qui concerne la première égalité, tous les éléments de \( U_n\) sont des racines simples de \( X^n-1\). Donc le théorème~\ref{ThoSVZooMpNANi} dit qu'il existe un nombre \( k\) (polynôme de degré zéro) tel que \( X^n-1=k\prod_{z\in U_n}(X-z)\). Vu le coefficient du terme de plus haut degré, ce \( k\) ne peut être que \( 1\).

    Pour la suite nous utilisons l'union disjointe \( U_n=\bigcup_{d\divides n}\Delta_d\) du lemme~\ref{LemKcpjee} et la définition \eqref{EqLjGYKK} des polynômes cyclotomiques.
\end{proof}

\begin{remark}
    Notons juste pour le plaisir que dans le produit \( \prod_{d\divides n}\prod_{z\in\Delta_d}\), il y a bien \( n\) termes parce que \( \Card(\Delta_d)=\varphi(d)\) et \( \sum_{d\divides n}\varphi(d)=n\) (définition~\ref{DefLYGTooFPOYGZ} et lemme~\ref{LemKcpjee}).
\end{remark}

\begin{proposition}
    Les polynômes cyclotomiques sont à coefficients entiers : \( \phi_n\in \eZ[X]\).
\end{proposition}

\begin{proof}
    Nous devons démontrer que les coefficients de \( \phi_n\) sont dans \( \eZ\) alors qu'ils sont à priori dans \( \eC\). Nous démontrons cela par récurrence. D'abord \( \phi_1(X)=X-1\), d'accord. Ensuite
    \begin{equation}
        X^{n+1}-1=\prod_{d\divides n+1}\phi_d(X)=\phi_{n+1}(X)\cdot\underbrace{\prod_{_{\substack{d\divides n+1\\d\leq n}}}\phi_d(X)}_{\in\eZ[X]\text{ par récurrence}}
    \end{equation}
    Le lemme~\ref{LemzwkYdn} conclut que \( \phi_{n+1}\in \eZ[X]\). Nous avons considéré \( \eZ\) comme sous anneau du corps \( \eC\).
\end{proof}

\begin{proposition}     \label{PropUImYnL}
    Soient \( 1\leq m\leq n\) deux entiers et
    \begin{equation}
        T(X)=\frac{ X^n-1 }{ X^m-1 }\in \eZ(X).
    \end{equation}
    Alors :
    \begin{enumerate}
        \item   \label{ItemhpDPKE}
            si \( m\divides n\) alors \( T\in \eZ[X]\),
        \item
            si \( m\divides n\) et si \( m<n\) alors \( \phi_n\) divise \( T\) dans \( \eZ[X]\).
    \end{enumerate}
\end{proposition}
\index{polynôme!cyclotomique!propriétés}
\index{racine!de l'unité!utilisation}

\begin{proof}
    Nous prouvons point par point.
    \begin{enumerate}
        \item
            Si \( m\) divise \( n\) alors les diviseurs de \( n\) sont l'union des diviseurs de \( m\) et des diviseurs de \( n\) qui ne divisent pas \( m\). Soit
            \begin{equation}
                Q=\{\text{diviseurs de } n\text{ ne divisant pas } m \}.
            \end{equation}
            Nous avons alors
            \begin{equation}
                X^n-1=\prod_{d\divides n}\phi_d(X)=\prod_{d\divides m}\phi_d(X)\cdot\prod_{q\in Q}\phi_q(X)=(X^m-1)\cdot\prod_{q\in Q}\phi_q(X).
            \end{equation}
            Nous avons donc
            \begin{equation}
                T(X)=\frac{ X^n-1 }{ X^m-1 }=\prod_{q\in Q}\phi_q(X)\in \eZ[X].
            \end{equation}

        \item
            Nous venons de montrer que
            \begin{equation}
                T=\prod_{q\in Q}\phi_q\in \eZ[X].
            \end{equation}
            Étant donné que \( m<n\) nous avons \( n\in Q\) et donc
            \begin{equation}
                T=\phi_n\cdot\prod_{q\in Q\setminus\{ n \}}\phi_q.
            \end{equation}
            Par conséquent \( \phi_n\) divise \( T\) dans \( \eZ[X]\).
        \end{enumerate}
\end{proof}

\begin{corollary}   \label{CorTVUooErJiAC}
    Si \( p\) est premier alors le polynôme cyclotomique \( \phi_p\) a une bonne tête :
    \begin{equation}
        \phi_p(X)=1+X+\cdots +X^{p-1}.
    \end{equation}
\end{corollary}

\begin{proof}
    Nous utilisons la formule du lemme~\ref{LemKYGBooAwpOHD} en remarquant que seuls \( p\) et \( 1\) divisent \( p\) :
    \begin{equation}
        X^p-1=\prod_{d\divides p}\phi_d(X)=\phi_1(X)\phi_p(X)=(X-1)\phi_p(X).
    \end{equation}
    Nous pouvons simplifier par \( X-1\) en utilisant la formule du lemme~\ref{LemISPooHIKJBU}\ref{ItemLTBooAcyMtNii} :
    \begin{equation}
        1+X+\cdots +X^{p-1}=\phi_p(X)
    \end{equation}
\end{proof}

\begin{proposition}[Irréductibilité des polynômes cyclotomiques\cite{DVEIity}]      \label{PropoIeOVh}
    Les polynômes cyclotomiques sont irréductibles sur \( \eQ\).
\end{proposition}
\index{polynôme!cyclotomique!irréductibilité}
\index{Anneau!\( \eZ/n\eZ\)!polynôme cyclotomique}
\index{nombre premier!polynôme cyclotomique}
\index{racine!de l'unité}
\index{corps!de rupture!polynôme cyclotomique}

\begin{proof}
    Pour rappel, nous savons déjà que pour tout \( n\in\eN\), \( \phi_n\in \eZ[X]\). Puisque les racines de \( \phi_n\) sont les racines primitives de l'unité, nous devons montrer que toutes les racines primitives de l'unité ont même polynôme minimal (qui sera alors \( \phi_n\)); en effet comme ces polynômes divisent \( \phi_n\), s'ils sont distincts, la proposition~\ref{PropyMTEbH} s'applique et le produit des polynômes minimaux diviserait \( \phi_n\). Dans le cas inverse, \( \phi_n\) est polynôme minimal des racines primitives de l'unité et est donc irréductible. Soit donc \( \xi\), une telle racine primitive. Une autre racine primitive est de la forme \( \xi^l\) où \( l\) est un nombre premier tel que \( \pgcd(l,n)=1\).

    Soient \( f\) et \( g\), les polynômes minimaux dans \( \eZ[X]\) de \( \xi\) et \( \xi^l\). Nous allons montrer que \( f=g\) et donc que \( f=g=\phi_n\). Supposons par l'absurde que \( f\neq g\). Dans ce cas ils seraient des facteurs irréductibles distincts de \( \phi_n\) et il existerait un polynôme \( h\) tel que \( \phi_n=fgh\). A priori, \( h\in \eQ[X]\) parce que nous sommes justement en train de prouver que \( \phi_n\) est irréductible dans \( \eQ[X]\). Quoi qu'il en soit, le lemme de Gauss~\ref{LemEfdkZw} nous montre que \( h\in \eZ[X]\) parce que \( \phi_n\), \( f\) et \( g\) ont des coefficients entiers. Nous avons
    \begin{equation}
        f(\xi)=g(\xi^l)=0.
    \end{equation}
    Considérons le polynôme \( \psi(X)=g(X^l)\). Ce polynôme \( \psi\) est dans \( \eZ[X]\) et \( \psi\) est annulateur de \( \xi\), donc \( f\) divise \( \psi\) en tant que polynôme minimal de \( \xi\). Il y a un polynôme unitaire à coefficients entiers (lemme de Gauss forever) \( k\) tel que
    \begin{equation}
        \psi=fk
    \end{equation}
    Nous considérons maintenant les projections sur \( \eF_l[X]\) : étant donné que \( \phi_n=fgh\), nous savons que \( \bar f\bar g\) divise \( \bar\phi_n\). En même temps, \( \bar f\) divise \( \bar \psi\). En utilisant le morphisme de Frobenius (c'est ici que la projection sur \( \eF_l\) joue), nous avons aussi
    \begin{equation}
        \bar\psi(X)=\bar g(X^l)=\bar g(X)^l.
    \end{equation}
    Par conséquent dire que \( \bar f\) divise \( \bar\psi\) revient à dire que \( \bar f(X)\) divise \( \bar g(X)^l\). En particulier tout facteur irréductible de \( \bar f\) divise \( \bar g\). Un facteur irréductible de \( \bar f\) serait donc à la fois dans \( \bar f\) et dans \( \bar g\) et donc deux fois (au moins) dans \( \bar\phi_n\) parce que \( \bar f\bar g\) divise \( \phi_n\). Dans un corps de décomposition de ce facteur, \( \phi_n\) aurait une racine double, alors que ce n'est pas le cas. Contradiction. Nous concluons que \( f=g\).
\end{proof}

Le corolaire suivant va être utilisé pour déterminer les polygones constructibles à la règle et au compas, théorème de Gauss-Wantzel~\ref{ThoTWAooEsLjJu}.
\begin{corollary}   \label{CorKRTooTJtyvP}
    Soit \( p\) un nombre premier et \( \alpha\) un entier non nul. Nous posons \( q=p^{\alpha}\). Alors le polynôme minimal de \(  e^{2 i\pi/q}\) sur \( \eQ\) est le polynôme cyclotomique \( \phi_q\).
\end{corollary}

\begin{proof}
    Le polynôme \( \phi_q\) est irréductible par la proposition~\ref{PropoIeOVh}, il est unitaire par définition et contient le monôme \( X- e^{2i\pi/q}\), donc il est annulateur. Annulateur, irréductible et unitaire, la proposition~\ref{PROPooALFJooDjmIcb}\ref{ITEMooEFNFooKYqXDk} en fait le polynôme minimal de \(   e^{2i\pi/q}  \).
\end{proof}

%TODO : comprendre pourquoi cette démonstration est plus compliquée.
%La preuve suivante est celle donnée par la taverne de l'irlandais et par Arnaud Girand. Je me demande pourquoi elle
%est plus compliquée que celle que je donne au dessus et qui vient de Wikipédia.
%\begin{proposition}[Irréductibilité des polynômes cyclotomiques\cite{KXjFWKA}]
    %Les polynômes cyclotomiques sont irréductibles sur \( \eQ\).
%\end{proposition}
%
%\begin{proof}
    %L'anneau \( \eQ[X]\) est un anneau factoriel par le théorème~\ref{ThoBUEDrJ}; il existe donc un unique \( r\)-uplet de polynômes irréductibles \( G_1,\ldots, G_r\in \eQ[X]\) tel que
    %\begin{equation}
        %\phi_n=\prod_{i=1}^rG_i.
    %\end{equation}
    %Pour chaque \( i\) nous considérons \( \alpha_i\in \eN^*\) tel que \( \alpha_iG_i\in \eZ[X]\) (par exemple \( \alpha_i\) est le \( \ppcm\) des dénominateurs des coefficients de \( G_i\)). Nous avons alors
    %\begin{equation}
        %\big(  \prod_{i=1}^r\alpha_i \big)\phi_n=\prod_{i=1}^r\alpha_iG_i.
    %\end{equation}
    %D'autre part le polynôme \( \phi_n\) étant unitaire, son contenu est \( 1\) et donc
    %\begin{equation}
        %\prod_i\alpha_i=c\Big( \big( \prod_i\alpha_i \big)\phi_n \Big)=\prod_ic(\alpha_iG_i)
    %\end{equation}
    %par le lemme~\ref{LemHULrVaF}. Nous posons à présent
    %\begin{equation}    \label{EqKNKxqXI}
        %F_i=\pm\frac{ \alpha_iG_i }{ c(\alpha_iG_i) }.
    %\end{equation}
    %Les polynômes \( F_i\) sont dans \( \eZ[X]\) parce que les \( \alpha_iG_i\) y sont. De plus nous avons
    %\begin{equation}
        %\prod_{i=1}^rF_i=\phi_n.
    %\end{equation}
    %Montrons que \( F_i\) est unitaire. Si \( b_i\) est le coefficient dominant de \( F_i\), alors nous avons \( \prod_ib_i=1\). Vu que les \( b_i\) sont dans \( \eZ\), il n'y a pas des tonnes de solutions : \( b_i= \pm 1\). Nous fixons donc le choix de \( \pm\) dans \eqref{EqKNKxqXI} de telle sorte à avoir \( b_i=1\).
%
    %Nous allons prouver maintenant que si \( k\) est premier avec \( n\) et si \( \xi\) est une racine de \( F_1\), alors \( F_1(\xi^k)=0\). Nous allons prouver cela par récurrence sur la longueur de la décomposition de \( k\) en nombres premiers. C'est-à-dire que nous posons \( k=p_1\ldots p_s\) (les \( p_i\) non spécialement distincts) et que nous faisons la récurrence sur \( s\).
%
    %Pour \( s=1\). Soit \( \xi\) une racine de \( F_1\) et \( p\) un nombre premier tel %que \( p\notdivides n\). Vu que \( \xi\) est une racine de \( \phi_n\), c'est une racine %\( n\)\ieme\ de l'unité. Vu que nous demandons \( \pgcd(n,p)=1\), le nombre \( \xi^p\) %est également une racine primitive de l'unité, et il existe donc \( i\in\{ 1,\ldots, r %\}\) tel que \( F_i(\xi^p)=0\). Nous considérons maintenant les polynômes \( F_1(X)\) et %\( F_i(X^p)\), et nous montrons qu'ils ne sont pas premiers entre eux sur \( \eQ\). S'ils l'étaient, le sieur Bézout nous donnerait \( U,V\in \eQ[X]\) tels que
    %\begin{equation}
        %U(X)F_1(X)+V(X)F_i(X^p)=1.
    %\end{equation}
    %En évaluant cela en \( X=\xi\), nous obtenons \( 0=1\) qui est une contradiction. Mais étant donné que \( F_1\) est irréductible sur \( \eQ\), il doit être lui-même le facteur commun entre \( F_1\) et \( F_i\). Nous avons donc
    %\begin{equation}
        %F_1(X)\divides F_i(X^p)
    %\end{equation}
    %dans \( \eQ[X]\).
   %
%\end{proof}
%

\begin{theorem} \label{ThojCJpFW}
    Soit \( P\in \eZ[X]\) un polynôme unitaire irréductible non constant tel que toutes les racines dans \( \eC\) soient de module \( \leq 1\). Alors soit \( P=X\), soit \( P\) est un polynôme cyclotomique.
\end{theorem}

\begin{proof}
    Le polynôme \( P=X\) vérifie les conditions. Pour la suite, nous supposons que \( P\neq X\).

    Nous notons \( P=\sum_ia_iX^i\). Étant donné que \( P\) est irréductible et différent de \( X\), nous avons \( a_0\neq 0\) (sinon \( x=0\) serait une racine). Nous allons montrer que les racines de \( P\) sont toutes des racines \( N\)-ièmes de l'unité (avec le même \( N\) pour toutes).

    Soient \( \{ \xi_i \}_{i=1,\ldots, d}\) les racines de \( P\); on a
    \begin{equation}
        P=\prod_{i=1}^d(X-\xi_i)
    \end{equation}
    avec \( \prod_{i=1}^d\xi_i=a_0\). Par hypothèse, \( | \xi_i |\leq 1\) et donc \( 0<| a_0 |\leq 1\). Puisque \( P\in \eZ[X]\) nous avons donc \( a_0=1\) et \( | \xi_i |=1\), pour tout \( i\).

    Nous introduisons les polynômes
    \begin{equation}
        g_q(X)=\prod_{i=1}^d\big( X-(\xi_i)^q \big),
    \end{equation}
    et en particulier \( g_1=P\), et nous développons
    \begin{equation}
        g_q(X)=X^n+C_{1,q}X^{n-1}+\cdots +C_{n,q}
    \end{equation}
    où
    \begin{equation}
        C_{k,q}=(-1)^k\sum_{1\leq i_1<\ldots<i_k\leq d}(\xi_{i_1}\ldots \xi_{i_k})^q.
    \end{equation}
    Nous introduisons aussi les polynômes
    \begin{equation}
        F_{k,q}(X_1,\ldots, X_n)=(-1)^k\sum_{1\leq i_1<\ldots< i_k\leq d}(X_{i_1}\ldots X_{i_k})^q
    \end{equation}
    qui sont des polynômes symétriques. Ils vérifient deux propriétés. La première est que
    \begin{equation}
        C_{r,q}=F_{r,q}(\xi_1,\ldots, \xi_n),
    \end{equation}
    et la seconde est que les polynômes \( F_{r,1}\) sont les polynômes symétriques élémentaires à un coefficient près. Le théorème~\ref{TholReBiw} nous donne alors des polynômes \( G_{k,q}\in \eZ[X_1,\ldots, X_n]\) tels que
    \begin{equation}
        F_{k,q}(X_1,\ldots, X_n)=G_{k,q}\big( F_{1,1}(X_1,\ldots, X_n),\ldots, F_{k,1}(X_1,\ldots, X_n) \big).
    \end{equation}
    Nous savons que
    \begin{equation}
        | C_{k,q} |\leq \sum_{1\leq i_1<\ldots<i_k<d}1={d\choose k}.
    \end{equation}
    Donc \( g_q\) fait partie de l'ensemble fini des polynômes dans \( \eZ[q]\) dont tous les coefficients sont bornés en valeur absolue par
    \begin{equation}
        \max_{k=1,\ldots, d}{d\choose k}.
    \end{equation}
    Il existe un certain nombre d'ensembles \( \{ \xi_i \}\) qui sont racines de polynômes vérifiant les conditions du théorème. À chacun de ces ensembles est associé une suite de polynômes \( g_q\) et donc des coefficients \( C_{k,q}\). Ce que nous avons vu est que l'ensemble de tous les coefficients \( C_{k,q}\) possibles (pour un choix donné des \( \{ \xi_i \}\)) est fini, en particulier, comme \( C_{1,q}=\sum_i\xi_i^q\), pour chaque \( k\), l'ensemble
    \begin{equation}
        \{ \xi_k^q\tq q\in \eN \}.
    \end{equation}
    Par le principe des tiroirs, il existe \( q_1\) et \( q_2\) tels que \( \xi_k^{q_1}=\xi_k^{q_2}\). Ici, \( q_1\) et \( q_2\) dépendent de \( k\) et nous notons \( N_k=q_1-q_2\); nous avons donc \( \xi_k^{N_k}=1\).

    En posant \( N=\ppcm(N_1,\ldots, N_d)\), nous avons
    \begin{equation}
        \xi_k^N=1
    \end{equation}
    pour tout \( k\).

    Mais \( P\) est irréductible dans \( \eZ[X]\); si il a \( \pm 1\) comme racines, alors c'est que \( P=X+1\) ou \( P=X-1\) et ce sont des polynômes cyclotomiques. Si \( P\) n'a pas \( \pm 1\) parmi ses racines, alors \( P\) n'a pas de racines dans \( \eQ\) parce que \( \pm 1\) sont les seules racines de \( X^N-1\) dans \( \eQ\).

    Par conséquent \( P\) est un facteur irréductible de \( X^N-1\) dans \( \eQ[X]\). Mais étant donné que
    \begin{equation}
        X^N-1=\prod_{d\divides N}\phi_d(X),
    \end{equation}
    les polynômes cyclotomiques sont les seuls facteurs irréductibles de \( X^N-1\). Donc \( P\) est un polynôme cyclotomique.
\end{proof}

%---------------------------------------------------------------------------------------------------------------------------
\subsection{Nombres premiers}
%---------------------------------------------------------------------------------------------------------------------------

\begin{lemma}[\cite{naKXuR}]    \label{LemiAqLEn}
    Soit \( n\geq 1\). Il existe un nombre premier \( p\) et un entier \( a\) tels que
    \begin{enumerate}
        \item
            \( p\) divise \( \phi_n(a)\),
        \item
            \( p\) ne divise aucun de \( \phi_d(a)\) avec \( d\divides n\) et \( d\neq n\).
    \end{enumerate}
    De tels \( p\) et \( a\) vérifient automatiquement
    \begin{enumerate}
        \item
            \( p\) divise \( a^n-1\),
        \item
            \( p\) ne divise aucun des \( a^d-1\) pour \( d\divides n\), \( d\neq n\).
    \end{enumerate}
\end{lemma}

\begin{proof}
    Nous posons
    \begin{equation}
        B(X)=\prod_{_{\substack{d\divides n\\d\neq n}}}\phi_d(X),
    \end{equation}
    et nous commençons par montrer que \( \phi_n\) est premier avec \( B\). Nous avons \( X^n-1=B\phi_n\), donc \( B\) et \( \phi_n\) n'ont pas de racine commune (même pas dans \( \eC\)) parce que ce serait une racine double de \( X^n-1\). Notons que par définition~\ref{EqLjGYKK}, les polynômes cyclotomiques sont scindés (dans \( \eC\)), donc en particulier les polynômes \( \phi_n\) et \( B\) sont scindés, et donc premiers entre eux, dans \( \eC\) et a fortiori dans \( \eQ\). Par Bézout (corolaire~\ref{CorimHyXy}), il existe \( U,V\in\eQ[X]\) tels que
    \begin{equation}
        U\phi_n+VB=1.
    \end{equation}
    Si nous prenons \( a\in \eZ\) tel que \( U'=aU\) et \( V'=aV\) soient tous deux dans \( \eZ[X]\), alors nous avons
    \begin{equation}    \label{EqCpNMEi}
        U'\phi_n+V'B=a,
    \end{equation}
    égalité dans \( \eZ[X]\). Quitte à prendre un multiple assez grand de \( a\), nous pouvons choisir \( a\) de telle sorte que \( | \phi_n(a) |\geq 2\). Nous prenons alors un nombre premier \( p\) divisant \( \phi_n(a)\).

    Montrons que le \( a\) et le \( p\) ainsi construits satisfont aux exigences.

    Puisque \( X^n-1=B\phi_n\), si \( p\) divise \( \phi_n(a)\), il divise automatiquement \( a^n-1\) et donc \( [a^n]_p=1\), ce qui signifie entre autres que \( a\) et \( p\) sont premiers entre eux. Évaluons l'équation \eqref{EqCpNMEi} en~\( a\) :
    \begin{equation}
        U'(a)\phi_n(a)+V'(a)B(a)=a.
    \end{equation}
    Le nombre \( p\) ne divisant pas \( a\), mais divisant \( \phi_n(a)\), il ne peut pas diviser \( B(a)\)\footnote{C'est pour pouvoir dire ça que l'on a choisi \( V'\in \eZ[X]\) de telle sorte que \( V'(a)\) soit dans \( \eZ\)}. Étant donné que \( p\) ne divise pas \( B(a)\), il ne divise aucun des \( \phi_d(a)\) avec \( d\divides n\) et \( d\neq n\).

    Nous passons maintenant à la seconde partie de la preuve. Nous supposons avoir \( a\) et \( p\) tels que \( p\) soit un nombre premier divisant \( \phi_n(a)\) et tels que \( p\) ne divise aucun des \( \phi_d(a)\) avec \( d\divides n\), \( d\neq n\). Le fait de diviser \( \phi_n(a)\) entraine le fait de diviser \( a^n-1\) parce que \( \phi_n\) est un des facteurs de \( X^n-1\). Soit maintenant \( d\neq n\) divisant \( n\); nous avons
    \begin{equation}    \label{EqwTWcCu}
        X^d-1=\prod_{d'\divides d}\phi_{d'},
    \end{equation}
    et cela est une partie du produit
    \begin{equation}
        \prod_{\substack{d\divides n\\d\neq n}}\phi_d.
    \end{equation}
    Puisque \( p\) ne divise aucun des \( \phi_d(a)\) de ce dernier produit, a fortiori, il ne divise pas le produit~\ref{EqwTWcCu}, et donc pas \( a^d-1\).
\end{proof}

\begin{lemma}       \label{LemrZnmpG}
    Si \( n\geq 1\), alors il existe un nombre premier dans \( [1]_n\), c'est-à-dire un nombre premier de la forme \( 1+kn\) avec \( k\in \eN\setminus\{ 0 \}\).
\end{lemma}

\begin{proof}
    Soient \( n\geq 1\) et \( p,a\) les nombres donnés par le lemme~\ref{LemiAqLEn}. Puisque \( p\) divise \( \phi_n(a)\), \( p\) divise \( a^n-1\) et donc \( [a]_p\) a un ordre qui divise \( n\) dans \( (\eZ/p\eZ)\setminus\{ 0 \}\) parce que \( [a]_p^n=[1]_p\).

    Prenons \( d\neq n\) divisant \( n\). Nous savons que
    \begin{equation}
        a^d-1=\prod_{d'\divides d}\phi_{d'}(a).
    \end{equation}

    Par construction de \( a\) et \( p\), nous avons
    \begin{equation}
        [\phi_{d'}(a)]_p\neq 0
    \end{equation}
    Comme \( \eZ/p\eZ\) est intègre, le produit est également non nul, c'est-à-dire
    \begin{equation}
        \big[ \prod_{d'\divides d}\phi_{d'}(a) \big]_p\neq 0,
    \end{equation}
    et donc \( [a]_p^a\neq 1\). Nous avons donc montré que si \( d\neq n\) divise \( n\), alors nous avons en même temps
    \begin{equation}
        [a]_p^n=1
    \end{equation}
    et
    \begin{equation}
        [a]_p^d\neq 1.
    \end{equation}
    Cela prouve que \( [a]_p\) est d'ordre exactement \( n\). Oui, mais l'ordre de \( [a]_p\) doit diviser l'ordre du groupe \( \eZ/p\eZ\) qui est \( p-1\), donc \( n\) divise \( p-1\) et nous écrivons \( p=kn+1\) avec \( k\) entier.
\end{proof}

\begin{theorem}[Forme faible du théorème de Dirichlet \cite{fJhCTE}]    \label{ThoxwTjcl}
    Pour tout \( n\geq 1\), il existe une infinité de nombres premiers dans \( [1]_n\).
\end{theorem}
\index{nombre!premier}
\index{Dirichlet!théorème (sur les nombres premiers)}
\index{théorème!Dirichlet!forme faible}
\index{anneau!\( \eZ/n\eZ\)}
\index{racine!de l'unité}

\begin{proof}
    Le lemme~\ref{LemrZnmpG} nous donne déjà l'existence de nombres premiers dans \( [1]_n\). Il faut maintenant voir qu'il y en a une infinité. Nous supposons qu'il y en ait seulement un nombre fini : \( p_1,\ldots, p_r\), et nous notons
    \begin{equation}
        N=np_1\ldots p_r.
    \end{equation}
    Nous utilisons maintenant le lemme~\ref{LemrZnmpG} avec ce \( N\), c'est-à-dire qu'on a un nombre premier de la forme
    \begin{equation}
        p=1+kN=1+knp_1\ldots p_r.
    \end{equation}
    C'est un nombre premier plus grand que tous les \( p_i\), et de la forme \( 1+\lambda n\). Cela contredit l'exhaustivité de la liste \( p_1,\ldots, p_r\).
\end{proof}

%+++++++++++++++++++++++++++++++++++++++++++++++++++++++++++++++++++++++++++++++++++++++++++++++++++++++++++++++++++++++++++
\section{Corps finis}
%+++++++++++++++++++++++++++++++++++++++++++++++++++++++++++++++++++++++++++++++++++++++++++++++++++++++++++++++++++++++++++
\label{SecCorpsFinizkAcbS}

Si vous cherchez des choses à propos de RSA, c'est à la section~\ref{SecEVaFYi}.

%---------------------------------------------------------------------------------------------------------------------------
\subsection{Théorème de Wedderburn}
%---------------------------------------------------------------------------------------------------------------------------

\begin{theorem}[Théorème de Wedderburn\cite{SQxrsoL}]    \label{ThoMncIWA}
    Tout corps fini est commutatif.
\end{theorem}
\index{groupe!fini}
\index{théorème!Wedderburn}
\index{action!de groupe!Wedderburn}
\index{nombre!complexe!norme \( 1\)}
\index{groupe!fini!Wedderburn}
\index{corps!fini!Wedderburn}

\begin{proof}
    Soit \( \eK\) un corps fini et \( Z\), le centre de \( \eK\). Ce dernier est un corps fini et un sous-corps de \( \eK\). Si \( q=\Card(Z)\) alors par le lemme~\ref{LemobATFP} nous avons
    \begin{equation}
        \Card(\eK)=q^n
    \end{equation}
    pour un certain \( n\).

    Nous supposons maintenant que \( \eK\) est non commutatif. Dans ce cas \( Z\neq \eK\) et nous avons \( n\geq 2\). Nous considérons aussi
    \begin{equation}
        Z_x=\{ a\in \eK\tq ax=xa \}.
    \end{equation}
    Le centre \( Z\) est un sous-corps de \( Z_x\), donc il existe \( d(x)\) tel que
    \begin{equation}
        \Card(Z_x)=q^{d(x)}.
    \end{equation}
    De la même manière, \( Z_x\) est un sous-corps de \( \eK\), donc il existe \( m(x)\) tel que
    \begin{equation}
        \Card(\eK)=\Card(Z_x)^{m(x)}.
    \end{equation}
    En mettant bout à bout, nous avons
    \begin{equation}
        q^n=\Card(Z_x)^{m(x)}=q^{d(x)m(x)},
    \end{equation}
    et par conséquent \( n=d(x)m(x)\). Le point important à retenir est que \( d(x)\) divise \( n\) pour tout \( x\in \eK\).

    Nous considérons maintenant l'action adjointe du groupe \( \eK\setminus\{ 0 \}\) sur lui-même :
    \begin{equation}
        \varphi(k)x=kxk^{-1}.
    \end{equation}
    Nous notons \( \mO_x\) l'orbite de \( x\in \eK\setminus\{ 0 \}\) pour cette action, et \( \Fix(x)\) son stabilisateur. Nous avons
    \begin{equation}
        Z_y=\Fix(y)\cup\{ 0 \}
    \end{equation}
    parce que \( Z_y\) et \( \Fix(y)\) ont les mêmes définitions, sauf que \( \Fix(y)\) est dans \( \eK\setminus\{ 0 \}\) alors que \( Z_y\) est dans \( \eK\). Nous avons donc
    \begin{equation}
        \Card\big( \Fix(y) \big)=\Card(Z_y)-1=q^{d(y)}-1.
    \end{equation}
    Nous avons \( \Card(\mO_x)=1\) si et seulement si \( \mO_x=\{ x \}\) si et seulement si \( \Fix(x)=\eK\setminus\{ 0 \}\) si et seulement si \( z\in Z\setminus\{ 0 \}\). Soient \( z_0,\ldots, z_{q-1}\) les éléments de \( Z\) avec \( z_0=0\). Ce sont les éléments qui auront une orbite réduite à un point. Les orbites qui coupent \( Z\setminus\{ 0 \}\) sont
    \begin{equation}
        \{ z_1 \},\ldots, \{ z_{q-1} \}
    \end{equation}
    et il y en a \( q-1\). Soient \( \mO_{y_1},\ldots, \mO_{y_r}\), les autres orbites. Nous utilisons l'équation des classes \eqref{EqkgGmoq} :
    \begin{equation}
        \Card(\eK^*)=\Card(Z^*)+\sum_{i=1}^{r}\frac{ \Card(\eK^*) }{ \Card(\Fix(y_i)) },
    \end{equation}
    mais \( \Card(Z^*)=q-1\), \( \Card(\eK^*)=q^n-1\) et \( \Card\big( \Fix(y_i) \big)=q^{d(y_i)}-1\), donc
    \begin{equation}        \label{EqBPBDzE}
        q^n-1=(q-1)+\sum_{i=1}^{r}\frac{ q^n-1 }{ q^{d(y_i)}-1 }.
    \end{equation}
    Nous considérons la fraction rationnelle
    \begin{equation}        \label{EqATGciu}
        F(X)=(X^n-1)-\sum_{i=1}^{r}\frac{ X^n-1 }{ X^{d(y_i)}-1 }.
    \end{equation}
    Étant donné que \( d(y_i)\) divise \( n\), nous avons, contrairement aux apparences, que \( F\in \eZ[X]\) par la proposition~\ref{PropUImYnL}\ref{ItemhpDPKE}.

    Nous pouvons exploiter un peu mieux la proposition~\ref{PropUImYnL} en remarquant que \( d(y_i)<n\) parce que sinon \( \Card(Z_{y_i})=\Card(\eK)\), ce qui signifierait que \( y_i\in Z\), ce qui nous avions exclu. Par conséquent le polynôme cyclotomique \( \phi_n\) divise
    \begin{equation}
        \frac{ X^n-1 }{ X^{d(y_i)}-1 }
    \end{equation}
    dans \( \eZ[X]\). Le polynôme cyclotomique \( \phi_n\) divise également \( X^n-1\) et par conséquent \( \phi_n\) divise \( F\). Il existe donc \( Q\in \eZ[X]\) tel que \( F=Q\phi_n\). En particulier en évaluant en \( q\) :
    \begin{equation}    \label{eqmoLdJy}
        F(q)=Q(q)\phi_n(q)=q-1.
    \end{equation}
    En effet nous avons \( F(q)=q-1\) par construction : comparer \eqref{EqBPBDzE} avec \eqref{EqATGciu}. Évidemment \( q\neq 1\) parce que si \( q=1\) alors \( \Card(\eK)=1\) et le théorème est trivial. Par ailleurs \( Q(q)\) est un entier (parce que \( Q\in \eZ[X]\) et \( q\in \eN\)) et \( Q(q)\neq 0\), parce qu'à droite de \eqref{eqmoLdJy} nous avons \( q-1\neq 0\). Nous avons donc \( | Q(q) |\geq 1\) et donc
    \begin{equation}
        | \phi_n(q) |\leq q-1.
    \end{equation}
    Par définition du polynôme cyclotomique nous avons
    \begin{equation}
        | \phi_n(q) |=\prod_{z\in\Delta_n}| q-z |.
    \end{equation}
    Étant donné que ce produit doit être inférieur à \( q-1\), au moins un des termes doit l'être : il existe \( z_0\in \Delta_n\) tel que \( | z_0-q |\leq q-1\). Étant donné que \( n\geq 2\) nous avons \( z_0\neq 1\).

    Mais d'autre part, comme indiqué sur la figure~\ref{LabelFigtrigoWedd}, la distance entre \( z_0\) et \( q\) doit être strictement plus grande que \( q-1\) parce que \( q-1\) est le minimum de la distance entre le cercle trigonométrique et \( q\), et n'est atteint qu'en \( z=1\).
    \newcommand{\CaptionFigtrigoWedd}{Nous devons avoir \( | z_0-q |>q-1\).}
    \input{auto/pictures_tex/Fig_trigoWedd.pstricks}

    Nous avons ainsi obtenu une contradiction, et nous concluons que le corps \( \eK\) est commutatif.
\end{proof}

%---------------------------------------------------------------------------------------------------------------------------
\subsection{Existence, unicité}
%---------------------------------------------------------------------------------------------------------------------------

Nous avons déjà défini le corps fini \( \eF_p\) lorsque \( p\) est un nombre premier dans la section~\ref{subseccorpspremhBlYIv}. Le théorème suivant sert à définir \( \eF_{p^n}\)\nomenclature[A]{\( \eF_{p^n}\)}{corps fini à \( p^n\) éléments} lorsque \( p\) est premier.
\begin{theorem}     \label{ThoOzgSfy}
    Soit \( p\) un nombre premier, soit \( n\in \eN\setminus\{ 0 \}\) et \( q=p^n\). Alors il existe un unique corps \( \eK\) de cardinal \( q\). Ce corps est le corps de décomposition du polynôme \( X^q-X\) sur \( \eF_p\).
\end{theorem}

\begin{proof}
    Montrons l'unicité. Soit \( \eK\) un corps fini de cardinal \( q=p^n\). Le groupe multiplicatif \( \eK^*\) est de cardinal \( q-1\), et par le corolaire~\ref{CorpZItFX} tous les éléments de \( \eK^*\) vérifient \( g^{q-1}=e\), c'est-à-dire que dans \( \eK[X]\), les éléments de \( \eK^*\) sont des racines du polynôme
    \begin{equation}
        X^{q-1}-1
    \end{equation}
    Par conséquent \( \eK\) est un corps de décomposition pour le polynôme \( Q(X)=X^q-X=X(X^{q-1}-1)\) parce que \( Q(X)=0\) dans \( \eK\). Il est unique par la proposition~\ref{PropTMkfyM}.

    Montrons maintenant que le corps de décomposition de \( P=X^q-X\) sur \( \eF_p\) est un corps de cardinal \( q\). Pour ce faire nous considérons \( \eK\) ce corps de décomposition et \(\eE\), l'ensemble des racines de \( P\) dans \( \eK\). Nous allons montrer que \( \eE=\eK\) et que \( \eE\) est un corps contenant \( q\) éléments.

    Montrons que \( \eE\) est un corps. Pour \( \alpha,\beta\in \eE\) nous avons
    \begin{equation}
        (\alpha\beta)^q=\alpha^q\beta^q=\alpha\beta
    \end{equation}
    parce que \( \alpha^q=\alpha\). Le produit \( \alpha\beta\) est donc encore dans \( \eE\). Pour la somme,
    \begin{equation}
        (\alpha+\beta)^q=(\alpha+\beta)^{p^n}=\Big( (\alpha+\beta)^p \Big)^{p^{n-1}}=(\alpha^p+\beta^p)^{p^{n-1}}=\ldots=\alpha^{p^n}+\beta^{p^n}=\alpha+\beta.
    \end{equation}
    En ce qui concerne l'inverse,
    \begin{equation}
        (\alpha^{-1})^q=(\alpha^q)^{-1}=\alpha^{-1}.
    \end{equation}
    Donc \( \eE\) est un corps. Évidemment \( \eE\) est un corps de décomposition de \( P\) au sens où \( \eE\) est une extension de \( \eF_p\) sur lequel \( P\) est scindé (parce qu'il est scindé sur \( \eK\) et \( \eE\) est le sous-corps de \( \eK\) contenant les racines de \( P\)) et tel que \( \eE=\eF_p(\{ \alpha_i \})\) où les \( \alpha_i\) sont les racines de \( P\). Notons que \( \eF_p\subset \eE\) parce que dans \( \eF_p\) on a \( x^q=x\).

    Par unicité, nous avons \( \eK=\eE\). Nous devons montrer que \( P\) possède exactement \( q\) racines distinctes, afin d'avoir \( \Card(\eE)=q\). Pour cela remarquons que
    \begin{equation}
        P'(X)=qX^{q-1}-1=-1
    \end{equation}
    dans \( \eF_p\). En effet \( P\in\eF_p\) et \( q=0\) dans \( \eF_p\). Par conséquent \( P'\) ne s'annule pas et \( P\) n'a pas de racine double. Toutes les racines étant simples, il y en a exactement \( q\).
\end{proof}

Le théorème~\ref{ThoOzgSfy} ne permet pas de \emph{construire} le corps à \( q=p^n\) éléments. Nous allons maintenant voir un certain nombre de résultats donnant des façons de le construire. Ces résultats proviennent de \cite{MichelMerlecorpsfinis,GabrielPeyre,RodierCorpsFinis} et de \wikipedia{fr}{Théorème_de_l'élément_primitif}{wikipedia}

\begin{proposition}[\cite{RodierCorpsFinis}]     \label{PropnfebjI}
    Soit \( \eK\) un corps fini. Alors le groupe multiplicatif \( \eK\setminus\{ 0 \}\) est cyclique.
\end{proposition}

\begin{proof}
    Soit \( \eK\) un corps ayant \( q\) éléments. Le groupe \( \eK\setminus\{ 0 \}\) en a \( q-1\); ergo l'ordre des éléments de \( \eK\setminus\{ 0 \}\) sont des diviseurs de \( q-1\); c'est le corolaire~\ref{CorpZItFX}. Soit \( d\) un diviseur de \( q-1\) et
    \begin{subequations}
        \begin{align}
            H_d\setminus\{ 0 \} &=\{ x\,\text{d'ordre }      d  \text{ dans } \eK\setminus\{ 0 \} \}          \\
            H_d                 &=\{    \text{racines de } X^d-1\text{ dans } \eK \}.
        \end{align}
    \end{subequations}
%TODO: je trouve ca lourd de remplacer ^*  par  \setminus\{ 0 \}
    Ici le polynôme \( X^d-1\) est vu dans \( \eK[X]\). Notons que nous avons automatiquement \( H^*_d\subset H_d\), mais l'inclusion inverse n'est pas assurée parce que les éléments d'ordre \( d/2\) par exemple sont aussi dans \( H_d\). Supposons \( H^*_d\neq \emptyset\) et considérons \( a\in H^*_d\). Alors l'application
    \begin{equation}
        \begin{aligned}
            \phi\colon \eZ/d\eZ & \to H_d         \\
                            n   & \mapsto a^n
        \end{aligned}
    \end{equation}
    est un isomorphisme d'anneaux. En effet étant donné que \( a\in H^*_d\subset H_d\), l'ensemble \( H_d\) contient le groupe cyclique engendré par \( a\). Ce dernier contient, par construction, \( d\) éléments. Mais \( \Card(H_d)\leq d\) parce que \( H_d\) est l'ensemble des racines d'un polynôme de degré \( d\). Par conséquent \( \Card(H_d)=d\) et l'ensemble \( H_d\) est bien engendré par \( a\) et \( \phi\) est bien un isomorphisme. Par conséquent tous les éléments de \( H^*_d\) sont des générateurs de \( H_d\).

    Inversement soit \( x\) un générateur de \( H_d\). L'ordre de \( H_d\) étant \( d\), l'ordre de \( x\) doit être un diviseur de \( d\). Supposons donc que \( x\) soit d'ordre \( d/k\). Dans ce cas nous devrions avoir \( \Card(H_d)=d/k\), ce qui contredit l'isomorphisme \( \phi\).

    En conclusion, \( H^*_d\) est l'ensemble des générateurs du groupe \( H_d\). Le nombre de générateurs de \( \eZ/d\eZ\) étant \( \varphi(d)\) par la proposition~\ref{PropZnmuphiGensn}, et \( H_d\) étant isomorphe à \( \eZ/d\eZ\) nous avons
    \begin{equation}
        \Card(H^*_d)=\varphi(d).
    \end{equation}

    Par conséquent si \( H^*_d\) n'est pas vide, son cardinal est \( \varphi(d)\). Nous avons
    \begin{subequations}
        \begin{align}
            q-1 &   =   \Card(\eK^*)                                  \\
                &   =   \Card\big( \bigcup_{d\divides q-1}H^*_d \big) \\
                &   =   \sum_{d\divides q-1}\Card(H^*_d)              \\
                & \leq  \sum_{d\divides q-1}\varphi(d)
                    =   q-1
        \end{align}
    \end{subequations}
    où nous avons utilisé le lemme~\ref{LemKcpjee}. Par conséquent pour tout \( d\) divisant \( q-1\) nous avons \( \Card(H^*_d)=\varphi(d)\) et il y a au moins un élément d'ordre \( q-1\) dans \( \eK\). Cet élément engendre \( \eK^*\) parce que \( \eK^*\) contient exactement \( q-1\) éléments. Par conséquent \( \eK\) est cyclique.
\end{proof}

\begin{corollary}   \label{CorpRUndR}
    Si \( p\) est un nombre premier, alors
    \begin{equation}
        (\eZ/p\eZ)^*\simeq\eZ/(p-1)\eZ.
    \end{equation}
    L'isomorphisme est un isomorphisme de groupes (abéliens). À gauche multiplicatif et à droite additif.
\end{corollary}
\index{groupe!fini}
\index{corps!fini}
\index{anneau!\( \eZ/n\eZ\)}
\index{nombre!premier}
\index{isomorphisme!\( (\eZ/p\eZ)^*\simeq\eZ/(p-1)\eZ\)}

\begin{proof}
    La proposition~\ref{PropnfebjI} nous enseigne que le groupe multiplicatif d'un corps fini est cyclique et donc isomorphe à un certain \( \eZ/n\eZ\). Donc \( (\eZ/p\eZ)^*\) est un groupe cyclique d'ordre \( p\), et donc isomorphe à \( \eZ/n\eZ\) avec \( n=p-1\).
\end{proof}

Lorsque \( \eK\) est un corps les éléments du groupe \( \eK^*\) sont les \defe{éléments primitifs}{primitif!élément d'un corps} de \( \eK\).
\begin{proposition}     \label{propQRcUlq}
    Soit \( \eK\) un corps contenant \( q\) éléments. Alors
    \begin{enumerate}
        \item
            \( x^q=x\) pour tout \( x\in \eK\),
        \item
            \( X^q-X=\prod_{a\in \eK}(X-a)\).
    \end{enumerate}
\end{proposition}

\begin{proof}
    Le groupe \( \eK\setminus\{ 0 \}\) ayant \( q-1\) éléments, ses éléments vérifient \( a^{q-1}=1\) par le corolaire~\ref{CorpZItFX} et par conséquent \( a^q=aa^{q-1}=a \).

    Soit \( a\in \eK\). Étant donné que \( a^q-a=0\), le polynôme \( (X-a)\) divise \( X^q-X\) dans \( \eK[X]\). Par conséquent
    \begin{equation}
        \prod_{a\in \eK}(X-a)
    \end{equation}
    divise également \( X^q-X\). Les polynômes \( X^q-X\) et \( \prod_{a\in \eK}(X-a)\) étant deux polynômes unitaires de même degré, le fait que l'un divise l'autre montre qu'ils sont égaux.
\end{proof}

\begin{example}
    Soit \( \eK=\eQ\) et \( \eL=\eQ(\sqrt{2},\sqrt{3})\). Afin de montrer que \( \eL=\eQ(\alpha)\) avec \( \alpha=\sqrt{2}+\sqrt{3}\) nous devons montrer que \( \sqrt{2}\) et \( \sqrt{3}\) sont des polynômes en \( \alpha\).
\end{example}

Une conséquence du fait que \( x^q=x\) est qu'il ne faut pas regarder le théorème~\ref{ThoLXTooNaUAKR} trop rapidement en disant «s'il s'annule partout, alors c'est le polynôme nul». En effet dans un corps fini, «partout» n'est pas forcément très grand.

% Le saut de ligne au milieu de la phrase est utile pour séparer deux références.
\begin{example}\label{exVQBooBMPLkD}
    Si \( \eF_3=\eZ/3\eZ\) est le\footnote{Le singulier est justifié par le théorème~\ref{ThoOzgSfy}, mais ça n'a pas d'importance ici.} corps à \( 3\)
    éléments, alors le polynôme \( P(X)=X^3-X\) s'évalue à zéro pour tout \( x\in \eF_3\) (proposition~\ref{propQRcUlq}.) mais il n'est pas le polynôme nul.
\end{example}

%---------------------------------------------------------------------------------------------------------------------------
\subsection{Symboles de Legendre et carrés}
%---------------------------------------------------------------------------------------------------------------------------

Source : \cite{RecQuadVento}.

Nous disons que \( a\in \eF_p\) est un \defe{carré}{carré!dans un corps fini} si il existe \( b\in \eF_p\) tel que \( a=b^2\).

\begin{definition}
    Soit \( n\in \eN\) et \( p>2\) un nombre premier. Le \defe{symbole de Legendre}{symbole!de Legendre}\index{Legendre!symbole} est défini par
    \begin{equation}
        \left( \frac{ n }{ p } \right)=\begin{cases}
            0   &   \text{si } p\text{ divise } n                     \\
            1   &   \text{si } n\text{ est un carré dans } \eF_p      \\
            -1  &   \text{sinon}.
        \end{cases}
    \end{equation}
\end{definition}

Note : \( -1\) peut être un carré, et pas que dans \( \eC\). Par exemple dans \( \eF_5\) nous avons \( 4=-1\) et donc \( -1\) est un carré.

\begin{proposition} \label{PropcGsJjk}
    Soit un nombre premier \( p>2\). Le corps \( \eF_p^*\) contient autant de carrés que de non carrés. De plus pour tout \( n\in \eN\) nous avons
    \begin{equation}    \label{Eqbcugos}
        \left(\frac{n}{p}\right)=n^{(p-1)/2}\mod p.
    \end{equation}
\end{proposition}

\begin{proof}
    Nous considérons l'application
    \begin{equation}
        \begin{aligned}
            \psi\colon \eF^*_p  & \to \eF^*_p     \\
                        x       & \mapsto x^2.
        \end{aligned}
    \end{equation}
    C'est un morphisme de groupes multiplicatifs et \( \ker\psi=\{ -1,1 \}\). Étant donné que \( p>2\), nous avons alors
    \begin{equation}
        \Card(\ker\psi)=2
    \end{equation}
    parce que \( 1\neq -1\). Évidemment l'ensemble des carrés dans \( \eF^*_p\) est l'image de \( \psi\). Le premier théorème d'isomorphisme~\ref{ThoPremierthoisomo}\ref{ItemWLCLdk} nous permet alors de conclure que
    \begin{equation}
        \Card(\Image(\psi))=\frac{ \Card(\eF^*_p) }{2}.
    \end{equation}
    Ceci prouve la première assertion.

    Par le petit théorème de Fermat (théorème~\ref{ThoOPQOiO}), nous avons \( x^{p-1}=1\) pour tout \( x\in \eF^*_p\). Les \( (p-1)\) éléments de \( \eF^*_p\) sont donc tous racines d'un des deux polynômes
    \begin{equation}
        X^{(p-1)/2}=\pm 1.
    \end{equation}
    Mais chacun des deux ne peut avoir, au maximum, que \( (p-1)/2\) solutions. Ils ont donc chacun exactement \( (p-1)/2\) racines.

    Nous pouvons maintenant prouver la formule \eqref{Eqbcugos}. D'abord si \( n=0\), elle est évidente. Si \( n\) est un carré dans \( \eF_p\), nous posons \( n=x^2\) et nous avons
    \begin{equation}
        n^{(p-1)/2}=n^{p-1}=1=\left(\frac{n}{p}\right).
    \end{equation}
    Si \( n\) n'est pas un carré, c'est que \( n\) n'est pas une racine de \( X^{(p-1)/2}=1\). Le nombre \( n\) est alors une racine de \( X^{(p-1)/2}=-1\). Nous avons alors
    \begin{equation}
        n^{(p-1)/2}=-1=\left(\frac{n}{p}\right).
    \end{equation}
\end{proof}

\begin{corollary}   \label{CoruJosNz}
    Si \( a,b\in \eN\) et si \( p>2\) est un nombre premier, alors
    \begin{equation}
        \left(\frac{ab}{p}\right)=\left(\frac{a}{p}\right)\left(\frac{b}{p}\right).
    \end{equation}
\end{corollary}

\begin{proof}
    Par la formule \eqref{Eqbcugos},
    \begin{equation}
        \left(\frac{ab}{p}\right)=(ab)^{(p-1)/2}=a^{(p-1)/2}b^{(p-1)/2}=\left(\frac{a}{p}\right)\left(\frac{b}{p}\right).
    \end{equation}
\end{proof}

Soit un nombre premier \( q>2\) et \( \eA\), un anneau de caractéristique \( p\). Si \( \alpha\in \eA\) vérifie
\begin{equation}
    1+\alpha+\cdots+\alpha^{q-1}=0,
\end{equation}
nous définissons la \defe{somme de Gauss}{Gauss!somme de} par
\begin{equation}
    \tau=\sum_{x\in \eF_q}\left(\frac{i}{q}\right)\alpha^i=\sum_{x=1}^{q-1}\left(\frac{x}{q}\right)\alpha^i.
\end{equation}
Notons que la somme de Gauss dépend de \( q\) et du \( \alpha\) choisis.

\begin{proposition} \label{PropciRUov}
    Les sommes de Gauss vérifient les propriétés suivantes.
    \begin{enumerate}
        \item
            \( \tau^2=\left(\frac{-1}{q}\right)q\). Nous allons noter \( \epsilon(q)=\left(\frac{-1}{q}\right)\).
        \item
            Si \( \eA\) est de caractéristique \( p\geq 3\) et si \( p\neq q\) alors
            \begin{equation}    \label{EqxBNpJz}
                \tau^p=\left(\frac{p}{q}\right)\tau.
            \end{equation}
        \item
            Si \( \eA\) est de caractéristique \( p\) et si \( q\) est premier avec \( p\), alors \( \tau\) est inversible dans \( \eA\).
    \end{enumerate}
\end{proposition}

\begin{proof}
    D'abord nous notons que
    \begin{equation}
        \alpha^q-1=(\alpha-1)(1+\alpha+\cdots+\alpha^{q-1})=0
    \end{equation}
    par définition de \( \alpha\). Nous calculons
    \begin{subequations}
        \begin{align}
            \epsilon(q)\tau^2 &=\epsilon(q)\sum_{x,y\in \eF_q}\left(\frac{x}{q}\right)\left(\frac{y}{q}\right)\alpha^{x+y}        \\
                              &=\sum_{x,y\in \eF_q}\left(\frac{-xy}{q}\right)\alpha^{x+y}.                    \label{EqlObFeo}    \\
                              &=\sum_{z\in \eF_q}\sum_{y\in \eF_q}\left(\frac{-(z-y)y}{q}\right)\alpha^{z}    \label{EqWyIhhk}    \\
                              &=\sum_{z\in \eF_q}s_z\alpha^z                                                  \label{EqWoIszS}
        \end{align}
    \end{subequations}
    Justifications :
    \begin{itemize}
        \item
            Pour obtenir \eqref{EqlObFeo} nous avons utilisé le corolaire~\ref{CoruJosNz}.
        \item
            \eqref{EqWyIhhk} est un changement de variable \( z=x+y\) dans la somme sur \( x\).
        \item
            Pour \eqref{EqWoIszS} nous avons posé
            \begin{equation}
                s_z=\sum_{y\in \eF_q}\left(\frac{-(z-y)y}{q}\right).
            \end{equation}
    \end{itemize}
    Nous avons
    \begin{equation}
        s_0=\sum_{y\in \eF_q}\left(\frac{y^2}{q}\right).
    \end{equation}
    Dans cette somme, tous les termes sont égaux à \( 1\), sauf celui avec \( y=0\) qui vaut zéro. Nous avons donc \( s_0=q-1\). Voyons maintenant \( s_y\) avec \( y\neq 0\). L'application
    \begin{equation}
        \begin{aligned}
            \eF^*_q & \to \eF_q\setminus\{ 1 \}   \\
                  k & \mapsto 1-zy^{-1}
        \end{aligned}
    \end{equation}
    étant une bijection nous pouvons effectuer le changement de variables \( t=y^{-1}z-1\) pour la somme sur \( y\) en notant \( y^{-1}\) l'inverse de \( y\) dans \( \eF^*_q\), nous trouvons alors
    \begin{subequations}
        \begin{align}
            \sum_{y\in \eF_q}\left(\frac{y(z-y)}{q}\right)  &=\sum_{y\in \eF_q}\left(\frac{y^2(y^{-1}z-1)}{q}\right)    \\
                &=\sum_{y\in \eF_q}\left(\frac{y^{-1}z-1}{q}\right)                                     \\
                &=\sum_{t\in \eF_q\setminus\{ 1 \}}\left(\frac{t}{q}\right)                             \\
                &=\underbrace{\sum_{t\in \eF_q}\left(\frac{y}{q}\right)}_{=0}-\left(\frac{1}{1}\right)  \\
                &=-1
        \end{align}
    \end{subequations}
    parce qu'il  y a autant de carrés que de non carrés dans \( \eF_q^*\) (proposition~\ref{PropcGsJjk}). En résumé nous avons
    \begin{equation}
        \epsilon(q)\tau^2=\sum_{z\in \eF_q}s_z\alpha^z
    \end{equation}
    où
    \begin{equation}
        s_z=\begin{cases}
            q-1   &   \text{si } z=0    \\
            -1    &   \text{sinon}.
        \end{cases}
    \end{equation}
    Cela donne
    \begin{equation}
        \epsilon(q)\tau^2=(q-1)-\underbrace{(\alpha+\cdots +\alpha^{q-1})}_{=-1}=q
    \end{equation}
    où nous avons utilisé l'hypothèse sur \( \alpha\). Donc \( \epsilon(q)\tau^2=q\), et étant donné que \( \epsilon(q)=\pm 1\) nous concluons
    \begin{equation}
        \tau^2=\epsilon(q)q.
    \end{equation}

    Nous prouvons maintenant la seconde partie. Comme \( \eA\) est de caractéristique \( p\), en utilisant le fait que le morphisme de Frobenius est un morphisme,
    \begin{equation}
        \tau^p=\left( \sum_{x\in \eF_q}\left(\frac{x}{q}\right)\alpha^x \right)^p=\sum_{x\in \eF_q}\left(\frac{x}{q}\right)^p\alpha^{px}.
    \end{equation}
    Étant donné que \( \left(\frac{x}{q}\right)=\pm 1\) et que \( p\) est impair, nous avons
    \begin{equation}
        \left(\frac{x}{q}\right)^p=\left(\frac{x}{q}\right).
    \end{equation}
    Du coup nous avons
    \begin{equation}
        \left(\frac{p}{q}\right)\tau^p=\sum_{x\in \eF_p}\left(\frac{xp}{q}\right)\alpha^{px}.
    \end{equation}
    Mais \( p\) étant inversible dans \( \eF_q\), l'application \( x\mapsto px\) est une bijection et nous pouvons sommer sur \( px\) au lieu de \( x\) :
    \begin{equation}
        \left(\frac{p}{q}\right)\tau^p=\sum_{x\in \eF_p}\left(\frac{x}{q}\right)\alpha^x=\tau.
    \end{equation}
    Nous trouvons alors que
    \begin{equation}
        \tau^p=\left(\frac{p}{q}\right)\tau.
    \end{equation}

    Étant donné la formule du \( \tau^2\) que nous venons de démontrer, nous avons \( \tau^2=\pm q\). Les nombres \( p\) et \( q\) étant premiers entre eux, le théorème de Bézout (théorème~\ref{ThoBuNjam}) nous donne \( a\) et \( b\) tels que
    \begin{equation}
        ap+ba=1.
    \end{equation}
    Cela montre que \( b\) est un inverse de \( q\) modulo \( p\). Donc \( \tau^2\) est inversible, et il en découle que \( \tau\) lui-même est inversible.
\end{proof}

\begin{theorem}[Loi de réciprocité quadratique]\index{loi!réciprocité quadratique}  \label{ThoMiEiUm}
    Soient deux nombres premiers distincts \( p,q\geq 3\). Alors
    \begin{equation}
        \left(\frac{p}{q}\right)=(-1)^{\frac{ (p-1)(q-1) }{ 4 }}\left(\frac{q}{p}\right).
    \end{equation}
\end{theorem}
\index{corps!fini}

\begin{proof}
    Soit \( \phi_q\) le polynôme \( 1+X+\cdots+X^{q-1}\) et l'anneau
    \begin{equation}
        \eA=\eF_p[X]/(\phi_q).
    \end{equation}
    C'est un anneau de caractéristique \( p\) parce que son unité est le polynôme constant \( 1\). Nous nommons \( \alpha=X/(\phi_q)\), c'est-à-dire que \( \phi_q(\alpha)=0\) dans \( \eA\), et nous pouvons considérer la somme de Gauss
    \begin{equation}
        \tau=\sum_{i\in \eF_q}\left(\frac{i}{q}\right)\alpha^i.
    \end{equation}
    Notons que ceci est un élément de \( \eA\) et plus précisément un polynôme de degré zéro dans \( \eA\), et encore plus précisément, une classe d'un tel polyôme. Donc les coefficients de \( \alpha\) doivent être compris comme des éléments de \( \eF_p\).
    Nous savons (proposition~\ref{PropciRUov}) que
    \begin{equation}
        \tau^2=\left(\frac{-1}{q}\right),
    \end{equation}
    et en utilisant la formule \eqref{Eqbcugos} nous trouvons
    \begin{equation}
        \left(\frac{\tau^2}{p}\right)=(\tau^2)^{(p-1)/2}\mod p=\tau^{p-1}\mod p
    \end{equation}
    En réalité sur cette dernière ligne, nous ne devrions pas préciser le «\(\mod p\)» parce que, comme mentionné plus haut, ce sont des éléments de \( \eF_p\). En utilisant cela, ainsi que \eqref{EqxBNpJz}, nous avons
    \begin{equation}
        \underbrace{\left(\frac{\tau^2}{p}\right)}_{\tau^{p-1}}\tau=\tau^p=\left(\frac{p}{q}\right)\tau
    \end{equation}
    Puisque \( \tau\) est inversible, nous écrivons
    \begin{equation}
        \left(\frac{\tau^2}{p}\right)=\left(\frac{p}{q}\right)
    \end{equation}
    Nous utilisons maintenant la formule \eqref{Eqbcugos} sur le membre de gauche avec \( n=\tau^2=\left(\frac{-1}{q}\right)\) :
    \begin{equation}
        \left(\frac{p}{q}\right)=\left(\frac{-1}{q}\right)^{\frac{ q-1 }{2}}\left(\frac{q}{p}\right).
    \end{equation}
    Toujours avec la même formule nous pouvons substituer \( \left(\frac{-1}{q}\right)\) par \( (-1)^{(q-1)/2}\) et obtenir
    \begin{equation}
        \left(\frac{p}{q}\right)=(-1)^{\frac{ (q-1) }{2}\frac{ (p-1) }{2}}.
    \end{equation}
\end{proof}

\begin{lemma}\label{Lemoabzrn}
    Si \( p\) est un nombre premier \( p\geq 3\), alors le symbole de Legendre \( x\mapsto\left(\frac{x}{p}\right)\) est l'unique morphisme non trivial de \( \eF^*_p\) dans \( \{ -1,1 \}\).
\end{lemma}

\begin{proof}
    Le fait que le symbole de Legendre soit non trivial est simplement le fait qu'il y ait des carrés et des non carrés dans \( \eF_p^*\); voir la proposition~\ref{PropcGsJjk}. Pour l'unicité, soit \( \alpha\colon \eF^*_p\to \{ -1,1 \}\) un morphisme surjectif (c'est-à-dire non trivial). Étant donné que
    \begin{equation}
        \eF^*_p=\ker(\alpha)\cup-\ker(\alpha),
    \end{equation}
    le groupe \( \eF_p^*/\ker(\alpha)\) ne contient que deux éléments : \( [1]\) et \( [-1]\). Autrement dit, \( \ker(\alpha)\) est d'indice \( 2\) dans \( \eF_p^*\).

    Or \( \eF_p^*\) ne possède qu'un seul sous-groupe d'indice \( 2\). En effet soit \( S\) un tel sous-groupe et \( a\), un générateur de \( \eF_p^*\) (qui est cyclique par la proposition~\ref{PropnfebjI}), alors \( a^2\in S\) par le lemme~\ref{PropubeiGX}. Par conséquent \( S\) contient le groupe des puissances paires de \( a\). Le groupe \( S\) ne peut rien contenir de plus parce qu'il est d'indice \( 2\) et que l'ordre de \( \eF_p^*\) est pair.

    Bref, le sous-groupe \( \ker(\alpha)\) est l'unique sous-groupe d'indice \( 2\) dans \( \eF_p^*\). Mais la proposition~\ref{PropcGsJjk} nous indique que \( | (\eF_p^*)^2 |=\frac{ p-1 }{2}\), c'est-à-dire que le groupe des carrés est d'indice \( 2\). Nous avons donc, par l'unicité,
    \begin{equation}
        \ker(\alpha)=(\eF_p^*)^2.
    \end{equation}
    Au final, pour \( y\in \eF_p^*\),
    \begin{equation}
        \alpha(y)=\begin{cases}
            1   &   \text{si } y\text{ est un carré}        \\
            -1  &   \text{sinon.}
        \end{cases}
    \end{equation}
    Ce qui est bien la définition des symboles de Legendre.
\end{proof}

\begin{proposition}
    Pour \( p\) premier nous avons
    \begin{equation}
        \left(\frac{2}{p}\right)=\begin{cases}
            1   &   \text{si } p\in [1]_8\text{ ou } p\in [7]_8       \\
            -1  &   \text{sinon}.
        \end{cases}
    \end{equation}
\end{proposition}

\begin{proof}
    Soit le polynôme
    \begin{equation}
        X^4+1\in \eF_p[X]
    \end{equation}
    et \( \alpha\), une racine dans une extension\quext{Dans la source que je suivais (je ne sais plus où), on parlait ici de «fermeture» de \( \eF_pj\) et non d'extension. Il me semble que parler simplement d'extension suffit. Vous confirmez ?} de \( \eF_p\)\footnote{Voir par exemple la proposition~\ref{PROPooDPOYooFHcqkU} pour l'existence d'une extension comme il faut.}. Nous posons \( \theta=\alpha+\alpha^{-1}\) et nous calculons
    \begin{equation}
        \theta^2=(\alpha+\alpha^{-1})(\alpha+\alpha^{-1})=\alpha^2+2+(\alpha^2)^{-1}=\alpha^2+2-\alpha^2=2
    \end{equation}
    parce que \( \alpha^2\) étant \( -1\), nous avons \( (\alpha^2)^{-1}=-\alpha^2\). Bref, \( \theta^2=2\).

    Dire que \( 2\) est un carré modulo \( p\) revient à dire que \( \theta\) est dans \( \eF_p\). C'est-à-dire que pour calculer le symbole de Legendre \( \left(\frac{2}{p}\right)\), nous étudions pour quels \( p\), l'élément \( \theta\) est vraiment dans \( \eF_p\) et non seulement dans l'extension \( \eF_p(\alpha)\). En tenant compte de l'exemple~\ref{ExLQhLhJ}, il faut distinguer deux cas : \( \alpha^p=\alpha\) et \( \alpha^p\neq \alpha\). Autrement dit, si \( \alpha^k=\alpha\) pour un certain nombre premier \( k\), alors le cas \( p=k\) est à traiter à part. La liste des puissances de \( \alpha\) est :
    \begin{equation}
        1,\alpha,\alpha^2,\alpha^3,-1,-\alpha,-\alpha^2,-\alpha^3,1,\alpha,\ldots
    \end{equation}
    Nous avons donc automatiquement \( \alpha^{9k}=\alpha\), mais \( p=9k\) est exclu parce que \( p\) est premier. Nous devons donc vérifier si une des propriétés
    \begin{subequations}
        \begin{align}
            \alpha^2  &=\alpha\\
            \alpha^3  &=\alpha\\
            -\alpha   &=\alpha\\
            -\alpha^3 &=\alpha
        \end{align}
    \end{subequations}
    est possible. Il est aisément vérifiable, au cas par cas, que ces possibilités sont toutes incompatibles avec \( \alpha^4=-1\). Nous avons donc certainement \( \alpha^p\neq \alpha\) et compte tenu de l'exemple~\ref{ExLQhLhJ}, l'équation \( x^p=x\) caractérise les éléments de \( \eF_p\) dans \( \eF_p(\alpha)\).

    L'équation \( X^2=2\) a exactement deux solutions qui sont \( \pm\theta\). Nous avons donc \( 2\in \eF_p^2\) si et seulement si \( \theta\in \eF_p\) si et seulement si \( \theta^p=\theta\). Nous avons réduit notre problème à déterminer pour quels \( p\) nous avons \( \theta^p=\theta\). D'abord nous avons, par le morphisme de Frobenius,
    \begin{equation}
        \theta^p=(\alpha+\alpha^{-1})^p=\alpha^p+\alpha^{-p}.
    \end{equation}
    Nous pouvons maintenant conclure facilement. Un nombre premier étant impair (sauf \( p=2\) qui peut être traité à part), \( p\) est automatiquement dans un des ensembles \( [1]_8\), \( [3]_8\), \( [5]_8\) ou \( [7]_8\). Nous avons quatre petites vérifications à faire. Dans tous les cas \( \alpha^{8k}=1\). Si \( p=1+8k\), alors
    \begin{equation}
        \theta^p=\alpha^{1+8k}+(\alpha^{-1})^{1+8k}=\alpha+\alpha^{-1}=\theta,
    \end{equation}
    donc \( 2\) est un carré dans \( \eF_p\). Si \( p\in[3]_8\), alors \( \theta^p=\alpha^3+\alpha^{-3}\). Si cela était égal à \( \alpha+\alpha^{-1}\), alors nous aurions
    \begin{equation}
        \alpha^6+1=\alpha^4+\alpha^2,
    \end{equation}
    et donc \( \alpha^2=1\), ce qui est impossible. Les vérifications pour \( p\in [5]_8\) et \( p\in [7]_8\) sont du même style.

\end{proof}

%---------------------------------------------------------------------------------------------------------------------------
\subsection{Théorème de Chevalley-Warning}
%---------------------------------------------------------------------------------------------------------------------------

\begin{lemma}
    Soit \( \eK\) un corps de caractéristique \( p\) et de cardinal \( q\). Pour \( m\in \eN\) nous définissons
    \begin{equation}
        S_m=\sum_{x\in \eK}x^m.
    \end{equation}
    Alors nous avons
    \begin{equation}
        S_m\mod p=\begin{cases}
            -1  &   \text{si } m\geq 1\text{ et } m\text{ divisible par } q-1   \\
            0   &   \text{sinon}.
        \end{cases}
    \end{equation}
\end{lemma}

\begin{proof}
    Si \( m=0\), alors \( x^0=1\) et \( S_m=q\). Par conséquent \( S_m\mod p=0\) parce que la caractéristique d'un corps divise son ordre (proposition~\ref{PropGExaUK}).

    Nous prenons maintenant \( m\geq 1\) et nous voyons séparément les cas où \( q-1\) divise \( m\) ou non. Si \( q-1\) divise \( m\), alors pour tout \( x\neq 0\) nous avons
    \begin{equation}
        x^m=x^{k(q-1)}=1
    \end{equation}
    parce que \( \eK^*\) est cyclique et \( x^{q-1}=1\) par le petit théorème de Fermat (théorème~\ref{ThoOPQOiO}). Par conséquent nous avons
    \begin{equation}
        \sum_{x\in \eK}x^m=\sum_{x\in \eK^*}1=q-1.
    \end{equation}

    Si le nombre \( m\geq 1\) n'est pas divisible par \( q-1\) alors nous prenons un générateur \( y\) du groupe \( \eK^*\). Un tel élément vérifie \( y^m\neq 1\). En effet, si \( y\) vérifiait \( y^m=1\) alors cela signifierait que l'ordre de \( \eK^*\) est un diviseur de \( m\), ce qui n'est pas le cas ici parce que l'ordre de \( \eK^*\) est \( q-1\). Pour un tel \( y\), l'application
    \begin{equation}
        \begin{aligned}
            \varphi\colon \eK^* & \to \eK^*       \\
                            x   & \mapsto yx
        \end{aligned}
    \end{equation}
    est une bijection\footnote{Notons que nous n'avons pas réellement besoin que \( y\) soit un générateur. Nous n'utilisons seulement le fait que \( y^m\neq 1\) et \( y\neq 0\).}. En ce qui concerne l'injectivité, \( ya=yb\) implique \( a=b\). En ce qui concerne la surjectivité, si \( a\) est un générateur, si \( z=a^l\) et si \( y=a^k\), alors
    \begin{equation}
        z=\varphi(a^{l-k}).
    \end{equation}
    Nous pouvons maintenant faire le calcul.
    \begin{equation}
        S_m=\sum_{x\in \eK^*}x^n=\sum_{x\in \eK^*}(yx)^m=y^m\sum_{x\in \eK^*}x^m=y^mS_m.
    \end{equation}
    Étant donné que \( y^m\neq 1\), la seule solution est \( S_m=0\).
\end{proof}

\begin{theorem}[Chevalley-Warninig\cite{ooHDDGooOhEgma}]\index{théorème!Chevalley-Warning}        \label{ThoLTcYKk}
    Soit \( \eK\) un corps fini de cardinal \( q\) et de caractéristique \( p\). Soient \( P_1,\ldots, P_r\) des éléments de \( \eK[X_1,\ldots, X_n]\) tels que \( \sum_{i=1}^r\deg(P_i)<n\). Nous considérons l'ensemble des zéros communs à tous les polynômes :
    \begin{equation}
        V=\{ x\in \eK^n\tq P_1(x)=\ldots=P_r(x)=0 \}.
    \end{equation}
    Alors \( \Card(V)=0\mod p\).
\end{theorem}
\index{corps!fini}
\index{polynôme!à plusieurs indéterminées}
\index{polynôme!symétrique}

\begin{proof}
    Nous considérons le polynôme
    \begin{equation}
        P=\prod_{i=1}^r(1-P_i^{q-1}).
    \end{equation}
    Montrons que
    \begin{equation}
        P(x)=\begin{cases}
            1    &   \text{si } x\in V\\
            0    &    \text{sinon}.
        \end{cases}
    \end{equation}
    La première ligne est facile : étant donné que tous les \( P_i(x)\) sont nuls pour \( x\in V\), nous avons \( P(x)=1\). Si \( x\) n'est pas dans \( V\), alors nous avons un \( i\) tel que \( P_i(x)\in \eK^*\). Mais dans ce cas (toujours la cyclicité de \( \eK^*\)) nous avons \( P_i(x)^{q-1}=1\) et donc le produit est nul.

    En utilisant l'hypothèse sur le degré des \( P_i\), nous trouvons
    \begin{equation}
        \deg(P)=\sum_{i=1}^r(q-1)\deg(P_i)<n(q-1).
    \end{equation}

    Pour un polynôme \( Q\in \eK[X_1,\ldots, X_n]\), nous définissons
    \begin{equation}
        \int Q=\sum_{x\in \eK^n}Q(x).
    \end{equation}
    Nous avons immédiatement
    \begin{equation}
        \int P=\sum_{x\in \eK^n}P(x)=\sum_{x\in V}1=\Card(V)\mod p.
    \end{equation}
    Nous insistons sur le «modulo \( p\)» parce que dans la formule \( P(x)=1\), le membre de droite est le \( 1\) de \( \eK\); il est donc automatiquement modulo la caractéristique de \( \eK\).

    Il nous reste à prouver que \( \int P=0\). Pour cela nous décomposons
    \begin{equation}        \label{EqHnUVlM}
        P=\sum_m c_mX_1^{m_1}\ldots X_n^{m_n}
    \end{equation}
    où la somme s'étend sur les \( m\in \eN^n\) tels que \( c_m\neq 0\). Nous avons
    \begin{subequations}
        \begin{align}
            \int P  & =\sum_{\in \eK^n}\sum_mc_m x_1^{m_1}\ldots x_n^{m_n}                          \\
                    & =\sum_{m}c_m\left( \sum_{x\in \eK^n}x_1^{m_1}\ldots x_n^{m_n} \right)         \\
                    & =\sum_m c_m S_{m_1}\ldots S_{m_n}.
        \end{align}
    \end{subequations}
    Le terme de plus haut degré dans la décomposition \eqref{EqHnUVlM} est celui du \( m\) tel que \( \sum_im_i\) est le plus grand. Vu que ce degré est plus petit que \( n(q-1)\), pour chacun des \( m\) rentrant dans la somme, nous avons
    \begin{equation}
        \sum_{i=1}^nm_i<n(q-1).
    \end{equation}
    En particulier pour tout \( m\in \eN^n\), il existe \( i\) tel que \( m_i<q-1\), et dans ce cas \( S_{m_i}=0\). Donc tous les termes de la somme
    \begin{equation}
        \sum_{m\in \eN^n}c_mS_{m_1}\ldots S_{m_n}
    \end{equation}
    ont un facteur nul.
\end{proof}

\begin{corollary}       \label{CorfuHNKz}
    Soit \( P_i\) des polynômes à \( n\) variables avec \( \sum_{i=1}^r\deg(P_i)<n\). Si les \( P_i\) n'ont pas de termes constants, alors ils ont un zéro commun non trivial.
\end{corollary}

\begin{proof}
    Nous reprenons les notations du théorème~\ref{ThoLTcYKk}. Étant donné que les \( P_i\) n'ont pas de termes constants, \( 0\in V\), mais \( \Card(V)=0\mod p\). Par conséquent nous devons avoir \( \Card(V)>p\).
\end{proof}

\begin{example}
    Nous considérons les polynômes
    \begin{subequations}
        \begin{align}
            P_1(x,y,t,u)  & = xy+x+ux   \\
            P_2(x,y,t,u)  & = x+y-3t.
        \end{align}
    \end{subequations}
    La somme de leurs degrés est \( 3\) et ce sont des polynômes à \( 4\) variables. Nous devons donc avoir, en vertu du corolaire~\ref{CorfuHNKz}, d'autres racines que la racine triviale \( (x,y,t,u)=(0,0,0,0)\).

    Le corolaire nous donne aussi une borne inférieure du nombre de racines à chercher : plus que la caractéristique du corps sur lequel nous travaillons. Nous pouvons dire cela sans avoir la moindre idée de la façon dont on pourrait résoudre le système \( P_1=P_2=0\).
\end{example}

%---------------------------------------------------------------------------------------------------------------------------
\subsection{Contenu d'un polynôme}
%---------------------------------------------------------------------------------------------------------------------------

\begin{lemma}[de Gauss\cite{KXjFWKA,OGaUHGn}]   \label{LemHULrVaF}
    Soient \( P,Q\in \eZ[X]\). Alors
    \begin{equation}
        c(PQ)=c(P)c(Q)
    \end{equation}
    où \( c\) est le contenu, définition~\ref{DefContenuPolynome}.
\end{lemma}
\index{lemme!de Gauss!contenu de polynôme}

\begin{proof}
    Afin de fixer les notations, nous posons \( P=\sum_ia_iX^i\) et \( Q=\sum_jb_jY^j\).

    \begin{subproof}
        \item[Pour les polynômes primitifs]

            Nous commençons par supposer que \( c(P)=c(Q)=1\). Dans ce cas si \( c(PQ)\neq 1\), nous considérons un nombre premier \( p\) divisant \( c(PQ)\). Puisque le contenu de \( P\) et de \( Q\) sont \( 1\), le nombre \( p\) ne peut pas diviser tous leurs coefficients. Nous définissons \( i_0\) de façon que \( a_{i_0}\) soit le premier à ne pas être divisible par \( p\), et \( j_0\) de telle façon que \( b_{j_0}\) soit le premier à ne pas être divisible par \( p\). Autrement dit :
            \begin{equation}
                p\divides a_0, p\divides a_1, \ldots, p\divides a_{i_0-1}, p\notdivides a_{i_0}
            \end{equation}
            et de façon similaire pour \( j_0\). Donc \( p\) ne divise ni \( a_{i_0}\), ni \( b_{j_0}\). Nous nous demandons alors avec malice quel est le coefficient de \( X^{i_0+j_0}\) dans \( PQ\). La réponse est :
            \begin{equation}
                a_{i_0}b_{j_0}+\sum_{\substack{i+j=i_0+j_0\\i<i_0\text{ ou }j<j_0}}a_ib_j.
            \end{equation}
            Par définition \( p\) divise soit \( a_i\) soit \( b_j\) pour chacun des termes de la grande somme. Comme \( p\) ne divise pas \( a_{i_0}b_{j_0}\), il ne divise pas le coefficient de \( X^{i_0+j_0}\) dans \( PQ\), alors que nous étions partis en disant que \( p\) divisait tous les coefficients de \( PQ\).

            Nous concluons donc que \( c(PQ)=1\).

        \item[Cas général]

            Si \( P\) et \( Q\) sont maintenant des polynômes sans condition particulière dans \( \eZ[X]\), nous considérons \( P_1=\frac{ P }{ c(P) }\) et \( Q_1=\frac{ Q }{ c(Q) }\); ces deux polynômes sont primitifs et nous avons alors, en utilisant la première partie :
            \begin{equation}
                c(P_1Q_1)=1.
            \end{equation}
            Étant donné que
            \begin{equation}
                P_1Q_1=\frac{1}{ c(P)c(Q) }PQ,
            \end{equation}
            nous avons
            \begin{equation}
                c(PQ)=c(P)c(Q)c(P_1Q_1)=c(P)c(Q).
            \end{equation}
    \end{subproof}
\end{proof}

%---------------------------------------------------------------------------------------------------------------------------
\subsection{Théorème de l'élément primitif}
%---------------------------------------------------------------------------------------------------------------------------

% TODO : fusionner cette définition avec celle du degré.
\begin{definition}
    Soit \( \eK\) un corps. Une extension \( \eL\) de \( \eK\) est dite \defe{finie}{extension!de corps!finie} si \( \eL\) est un espace vectoriel de dimension finie sur \( \eK\).
\end{definition}
Notez que la définition d'extension finie ne suppose ni que \( \eK\), ni que \( \eL\), soient finis en tant qu'ensembles.

\begin{theorem}[de l'élément primitif]\index{théorème!élément primitif}
    Si \( \eK\) est un corps fini, toute extension finie de \( \eK\) est simple\footnote{Définition~\ref{DefZCYIbve}.}.

    Si \( \eK\) est un corps quelconque alors toute extension séparable finie est simple.
\end{theorem}

\begin{proof}
    Nous ne donnons la preuve que dans le cas où \( \eK\) est fini. Dans ce cas nous savons par la proposition~\ref{PropnfebjI} que le groupe \( \eK^*\) est cyclique. Si de plus \( \eL\) est une extension finie alors \( \eL\) est fini en tant qu'ensemble. Par conséquent \( \eL^*\) est un groupe cyclique. Si \( \alpha\) est un générateur de \( \eL\) alors \( \eL=\eK(\alpha)\) et l'extension est donc simple.

    Une preuve de l'assertion dans le cas où \( \eK\) est infini peut être trouvée sur wikipédia.
\end{proof}

\begin{proposition}
    L'ordre d'un polynôme \( P\) vérifie les propriétés suivantes :
    \begin{enumerate}
        \item
            L'ordre de \( P\) est l'ordre multiplicatif de ses racines
        \item
            L'ordre de \( P\) divise \( p^n-1\).
    \end{enumerate}
\end{proposition}

\begin{lemma}       \label{LemZrUUOz}
    Soit \( p\) un nombre premier et \( P\) un polynôme irréductible unitaire de degré \( n\). Si \( \alpha,\beta\in \eF_p[X]/P\), alors \( (\alpha+\beta)^p=\alpha^p+\beta^p\).
\end{lemma}

\begin{proof}
    La preuve est exactement la preuve classique :
    \begin{equation}
        (\alpha+\beta)^p=\sum_k{k\choose p} \alpha^k\beta^{p-k}
    \end{equation}
    où les coefficients binomiaux sont dans \( \eF_p\) et donc nuls pour les \( k\) différents de \( p\) et de \( 0\).
\end{proof}
Cette proposition est encore vraie avec \( \alpha,\beta\in\eF_{p^n}\) et \( (\alpha+\beta)^{p^n}\).
%TODO : dire plus précisément, et prouver.

\begin{lemma}
    Si \( \alpha\in \eF_q\) est une racine d'ordre \( k\) de \( P\) (de degré \( n\)) alors les racines de \( X^k-1\) sont \( \{ \alpha^i\tq i=0,\ldots, k-1 \}\).
\end{lemma}

Nous serions donc intéressés à construire \( \eF_{q}\) comme quotient de \( \eF_p[X]\) par un polynôme primitif. Le théorème suivant donne une description abstraite de \( \eF_q\) qui va nous servir de point de départ pour la construction.
\begin{theorem}[Théorème de l'élément primitif]    \label{ThoqSludu}
    Soit \( p\) un nombre premier, \( n\in \eN\) et \( q=p^n\). Soit \( \eK\) un corps à \( q\) éléments. Alors
    \begin{enumerate}
        \item
            Il existe \( \alpha\in \eK\) tel que \( \eK=\eF_p[\alpha]\).
        \item
            Il existe une polynôme irréductible \( P\in\eF_p[X]\) de degré \( n\) tel que
            \begin{equation}        \label{EqWlMhhm}
                \begin{aligned}
                    \phi\colon \eF_p[X]/(P) & \to \eK       \\
                                \bar X      & \mapsto \alpha
                \end{aligned}
            \end{equation}
            soit un isomorphisme de corps.
    \end{enumerate}
    Soit \( \alpha\) et \( P\) choisis pour avoir les propriétés citées plus haut. Alors nous avons les propriétés suivantes.
    \begin{enumerate}
        \item
            \( P\) est primitif\footnote{Définition~\ref{DEFooDVOOooKaPZQC}.}.
        \item
            \( P\) est scindé dans \( \eK\).
        \item
            L'ensemble des racines de \( P\) est \( \{ \alpha,\alpha^p,\ldots, \alpha^{p^{n-1}} \}\).
        \item
            Le polynôme \( P\) divise \( X^q-X\) dans \( \eF_p[X]\).
    \end{enumerate}
\end{theorem}
\index{théorème!élément primitif}

\begin{proof}
    Le corps \( \eK\) étant fini, il est cyclique par la proposition~\ref{PropnfebjI}. Si \( \alpha\) un générateur de \( \eK^*\) alors
    \begin{equation}
        \eK=\eF_p[\alpha].
    \end{equation}
    Soit \( \ell\) le plus grand entier tel que l'ensemble
    \begin{equation}
        \{ 1,\alpha,\cdots,\alpha^{\ell-1} \}\subset\eK
    \end{equation}
    soit libre. Pour rappel \( \eK\) est un espace vectoriel sur \( \eF_p\). Il existe des \( a_i\in \eF_p\) tels que
    \begin{equation}
        \alpha^{\ell}+a_{\ell-1}\alpha^{\ell-1}+\cdots+a_0=0.
    \end{equation}
    De façon équivalente, il existe un polynôme unitaire \( P\in\eF_p[X]\) de degré \( \ell\) tel que \( P(\alpha)=0\). Étant donné que \( \alpha\) est générateur de \( \eK\),
    \begin{equation}
        \eK=\Span\{ 1,\alpha,\ldots, \alpha^{\ell-1} \}
    \end{equation}
    parce que \( \eK\) est généré par les puissances de \( \alpha\) alors que les puissances de \( \alpha\) plus hautes que \( \ell-1\) peuvent être générées par \( 1,\alpha,\ldots, \alpha^{\ell-1}\). L'espace \( \eK\) est donc un \( \eF_p\)-espace vectoriel de dimension \( \ell\); par conséquent
    \begin{equation}
        \Card(\eK)=p^n=q
    \end{equation}
    et \( \ell=n\).

    Montrons que \( P\) est irréductible dans \( \eF_p\). Si \( P\) était réductible dans \( \eF_p\), l'élément \( \alpha\in \eK\) serait une racine d'un des facteurs, c'est-à-dire qu'il serait racine d'un polynôme de degré inférieur à \( n\), ce qui contredirait le fait que
    \begin{equation}
        \{ \alpha^{\ell-1},\ldots, 1 \}
    \end{equation}
    soit libre.

    Montrons que l'application
    \begin{equation}
        \begin{aligned}
            \phi\colon \eF_p[X]/(P) & \to \eK     \\
                        \bar X      & \mapsto \alpha
        \end{aligned}
    \end{equation}
    est un isomorphisme. Pour l'injectivité, deux éléments \( Q_1,Q_2\in \eF_p[X]/(P)\) s'écrivent
    \begin{subequations}
        \begin{align}
            Q_1 &=\sum_{k=0}^{n-1}a_k\bar X^k     \\
            Q_2 &=\sum_{k=0}^{n-1}b_k\bar X^k.
        \end{align}
    \end{subequations}
    Dans ce cas si \( \phi(Q_1)=\phi(Q_2)\) alors
    \begin{equation}
        \phi(Q_1)=\sum_{k=0}^{n-1}a_k\alpha^k=\phi(Q_2)=\sum_{k=0}^{n-1}b_k\alpha^k.
    \end{equation}
    Mais l'ensemble \( \{ 1,\alpha,\ldots, \alpha^{n-1} \}\) étant libre sur \( \eF_p\), cela implique \( a_k=b_k\). La surjectivité de \( \phi\) provient du fait que \( \alpha\) génère \( \eK\).

    Nous passons maintenant à la seconde partie de la démonstration. Soient \( \alpha\in \eK\) tel que \( \eK=\eF_p[\alpha]\) et \( P\in \eF_p[X]\) un polynôme irréductible de degré \( n\) tel que \( \alpha\mapsto \bar X\) soit un isomorphisme entre \( \eK\) et \( \eF_p[X]/(P)\).

    Le polynôme \( P\) est primitif parce que \( \alpha\) est d'ordre \( p^n\) dans \( \eK\) alors que \( \bar X\mapsto \alpha\) est un isomorphisme. Par conséquent \( \bar X\) est d'ordre \( p^n\) dans \( \eF_p[X]/P\).

    Nous commençons par prouver que l'ensemble
    \begin{equation}        \label{EqAcsQHL}
        \{ \alpha,\alpha^p,\alpha^{p^2},\ldots, \alpha^{p^{n-1}} \}
    \end{equation}
    est l'ensemble des racines distinctes de \( P\). Pour cela nous posons
    \begin{equation}
        P(X)=\sum_{k=0}^na_kX^k
    \end{equation}
    avec \( a_k\in\eF_p\). D'abord \( \alpha\) est une racine de \( P\). En effet
    \begin{equation}        \label{EqbTAmKG}
        P(\bar X)=\sum_ka_k\bar X^k=0
    \end{equation}
    parce que cette somme est calculée dans \( \eF_p[X]/(P)\). En appliquant l'isomorphisme \( \phi\) à l'égalité \eqref{EqbTAmKG} nous trouvons
    \begin{equation}
        0=\phi\big( P(\bar X) \big)=\sum_ka_k\phi(\bar X^k)=\sum_ka_k\alpha^k.
    \end{equation}
    Donc \( \alpha\) est bien une racine de \( P\) dans \( \eF_p[X]\). Nous devons montrer qu'il en est de même pour les autres puissances dans l'ensemble \eqref{EqAcsQHL}. Étant donné que pour tout \( x\) dans \( \eF_p\) nous avons \( x^p=x\), nous avons aussi
    \begin{equation}
        P(X^p)=\sum_ka_k(X^p)^k=\sum_ka_k^p(X^p)^k=\sum_k(a_kX^k)^p
    \end{equation}
    alors que nous savons que \( x\mapsto x^p\) est un automorphisme de \( \eF_p\) par la proposition~\ref{PropFrobHAMkTY}. Par conséquent
    \begin{equation}
        P(X^p)=\sum_k(a_kX^k)^p=\left( \sum_k a_kX^k\right)^p=P(X)^p.
    \end{equation}
    Nous avons montré que si \( \beta\) est une racine de \( P\), alors \( \beta^p\) est également une racine de \( P\). Nous savons déjà que \( \alpha\) est une racine de \( P\), et que \( \alpha\) est également générateur de \( \eK\), c'est-à-dire que \( \alpha\) est d'ordre \( q-1\). Les puissances
    \begin{equation}
        \alpha,\alpha^p,\alpha^{p^2},\ldots, \alpha^{p^{n-1}}
    \end{equation}
    sont donc distinctes (\( \alpha^{p^n}=\alpha^q=1\)) et sont toutes des racines de \( P\). Étant donné que \( P\) est de degré \( n\) il ne peut pas y avoir d'autres racines. Nous concluons que l'ensemble
    \begin{equation}
        \{ \alpha,\alpha^p,\alpha^{p^2},\ldots, \alpha^{p^{n-1}}\}
    \end{equation}
    est l'ensemble des racines distinctes de \( P\) dans \( \eK\). Le polynôme \( P\) est alors scindé dans \( \eK[X]\).

    Le dernier point du théorème est de montrer que \( P\) divise \( X^q-X\). Pour cela nous allons montrer que toutes les racines de \( P\) sont des racines de \( X^q-X\). Soit \( \beta\) une racine de \( P\); il s'écrit \( \beta=\alpha^k\) pour un certain \( k\). Étant donné que \( \alpha^{q-1}=e=\alpha^{p^n-1}\),
    \begin{subequations}
        \begin{align}
            \beta^q &=(\alpha^{p^n})^k                                \\
                    &=\left( \alpha^{p^n-1}\alpha \right)^k           \\
                    &=\left( \alpha^{q-1}\alpha \right)^k             \\
                    &=\alpha^k                                        \\
                    &=\beta.
        \end{align}
    \end{subequations}
    Cela signifie que \( \beta^q=\beta\) et donc que \( \beta\) est racine de \( X^q-X\).
\end{proof}

\begin{corollary}
    Le corps fini à \( q=p^n\) éléments est de caractéristique \( p\).
\end{corollary}

\begin{proof}
    Nous considérons le corps fini \( \eK\) à \( q\) éléments sous la forme \( \eK=\eF_p[X]/P\) comme indiqué par l'équation \eqref{EqWlMhhm}. Soit \( 1_q\) la classe du polynôme \( 1\) modulo \( P\), nous considérons le morphisme
    \begin{equation}
        \begin{aligned}
            \mu\colon \eZ&\to \eF_q \\
            n&\mapsto n1_q.
        \end{aligned}
    \end{equation}
    Le noyau de cette application est \( \ker\mu=\eZ_p\) parce que \( p1_q=0\), les coefficients étant à comprendre dans \( \eF_p\).
\end{proof}

\begin{definition}  \label{DefnPNCFO}
    Soient \( P\), un polynôme de degré \( n\), et \( p\), un nombre premier. Un élément \( \alpha\in \eF_p[X]/(P)\) est une \defe{racine primitive}{racine!primitive}\index{primitif!racine} si les puissances de \( \alpha\) parcourent tout le groupe multiplicatif \( (\eF_p[X]/P)^*\).
\end{definition}

\begin{lemma}       \label{Lembcerei}
    Soit \( p\) un nombre premier et \( P\), un polynôme de degré \( n\). Si \( \alpha\in \eF_p[X]/P\) est une racine primitive de \( P\) alors les autres racines de \(P\) sont également primitives.
\end{lemma}

\begin{proof}
    Soit \( \alpha\in \eF_p[X]/P\) une racine primitive de \( P\). L'élément \( \alpha^p\) est également une racine parce que si \( P=\sum_ka_kX^k\),
    \begin{equation}
        P(\alpha^p)=\sum_k(a_k\alpha^k)^p=\big( \sum_ka_k\alpha^k \big)^p=0
    \end{equation}
    où nous avons utilisé le fait que \( a_k^p=a_k\) étant donné que \( a_k\in\eF_p\). Par hypothèse \( \alpha\) est une racine primitive; cela implique que les éléments \( \alpha,\alpha^p,\alpha^{p^2},\ldots,\alpha^{p^n-1}\) sont distincts dans \( \eF_p[X]/P\). Ces éléments constituent donc \emph{toutes} les racines de \( P\).

    Soit \( \beta=\alpha^{p^k}\) une racine de \( P\). Montrons que \( \alpha\) est une puissance de \( \beta\). Étant donné que \( (\eF_p[X]/P)^*\) est un groupe à \( p^n-1\) éléments, le corolaire~\ref{CorpZItFX} indique que \( \alpha^{p^n}=\alpha\). En particulier avec \( r=p^{n-k}\) nous avons
    \begin{equation}
        \beta^r=\alpha^{rp^k}=\alpha^{p^n}=\alpha.
    \end{equation}
    Par suite toutes les puissances de \( \alpha\) sont des puissances de \( \beta\), ce qui implique que \( \beta\) est générateur du groupe cyclique \( (\eF_p[X]/P)^*\).
\end{proof}

\begin{lemma}       \label{LemkzWjse}
    Soit \( p\) un nombre premier et \( n\), un entier. Un polynôme de degré \( d\), irréductible dans \( \eF_p[X]\), divise \( X^{p^n}-X\) si et seulement si, \( d\) divise \( n\).
\end{lemma}

\begin{theorem}
    Soient \( P\) et \( Q\) deux polynômes irréductibles de degré \( n\) dans \( \eF_p[X]\). Alors les quotients \( \eF_p[X]/P\) et \( \eF_p[X]/Q\) sont isomorphes en tant que corps.
\end{theorem}
En guise de démonstration de ce théorème, nous allons démontrer la proposition suivante.
\begin{proposition}      \label{PropCRPjZsp}
    Si \( \eK\) et \( \eL\) sont deux corps à \( q=p^n\) éléments, alors ils sont isomorphes.
\end{proposition}

\begin{proof}
    Soit \( a\) un élément primitif de \( \eK\) et \( P\) son polynôme minimal. Nous savons que \( \eK\simeq \eF_p[X]/P\) par le théorème de l'élément primitif~\ref{ThoqSludu}. L'élément \( a\) est en particulier une racine de \( X^q-X\). Par ailleurs \( P\) divise \( X^q-X\) par le lemme~\ref{LemkzWjse}.

    Nous avons aussi
    \begin{equation}
        X^q-X=\prod_{b\in \eL}(X-b)
    \end{equation}
    par la proposition~\ref{propQRcUlq}. Étant donné que \( P\) divise \( X^q-X\), un des éléments de \( \eL\) annule \( P\). Soit \( b\in \eL\) tel que \( P(b)=0\). Soit \( Q\) le polynôme minimal de \( b\). Par définition nous savons que \( Q\) divise \( P\), mais \( P\) étant irréductible et unitaire, nous avons immédiatement \( P=Q\). En particulier
    \begin{equation}
        \eF_p[X]/P\simeq\eF_p[X]/Q\simeq \eK.
    \end{equation}
    Nous montrons maintenant que \( \eF_p[X]/Q\simeq \eL\) par l'application
    \begin{equation}
        \begin{aligned}
            \phi\colon \eF_p[X]/Q&\to \eL \\
            \bar X&\mapsto b
        \end{aligned}
    \end{equation}
    qui se prolonge en \( R(\bar X)\mapsto R(b)\) pour tout \( R\in \eF_p[X]\). Cette application est bien définie parce que \( Q(b)=0\). Elle est injective parce que \( R(b)=0\) ne peut pas avoir lieu avec \( R\in \eF_p[X]/Q\) parce que \( Q\) est le polynôme minimal de \( b\). La surjectivité vient alors du fait que les deux corps ont le même nombre d'éléments.
\end{proof}

%---------------------------------------------------------------------------------------------------------------------------
\subsection{Construction de $\eF_{p^n}$}
%---------------------------------------------------------------------------------------------------------------------------

Le théorème \ref{ThoOzgSfy} nous indique que, pour tout \( q\in \eN\), il existe un unique corps possédant \( q\) éléments. Ce corps est noté \( \eF_{q}\).

Le théorème~\ref{ThoqSludu} nous incite à chercher à écrire \( \eF_q\) sous la forme
\begin{equation}
    \eF_q=\eF_p[X]/(P)
\end{equation}
pour un certain polynôme irréductible \( P\in\eF_p[X]\).

%///////////////////////////////////////////////////////////////////////////////////////////////////////////////////////////
\subsubsection{La version du faignant}
%///////////////////////////////////////////////////////////////////////////////////////////////////////////////////////////

Nous pouvons construire le corps à \( p^n\) éléments en prenant le quotient de \( \eF_p[X]\) par n'importe quel polynôme irréductible de degré \( n\). Le résultat est le suivant.
\begin{proposition} \label{PropHfrNCB}
    Soit \( P\) un polynôme unitaire irréductible dans \( \eF_p[X]\). Nous posons \( \eK=\eF_p[X]/(P)\). Alors
    \begin{enumerate}
        \item
            \( \eK\) est un corps à \( q\) éléments.
        \item
            \( \alpha=\bar X\) est une racine de \( P\) dans \( \eK\).
        \item   \label{ItemiEFRTg}
            \( \eK=\eF_p[\alpha]\).
    \end{enumerate}
\end{proposition}

\begin{proof}
    \begin{enumerate}
        \item
            En vertu du corolaire~\ref{CorsLGiEN}, \( \eK\) est un corps. Il est aussi un espace vectoriel de dimension \( n\) sur \( \eF_p\), et contient donc \( p^n=q\) éléments.
        \item
            Nous avons \( P(\bar X)=0\) par construction de \( \eK=\eF_p[X]/(P)\).
        \item
            En tant que quotient de \( \eF_p[X]\), les éléments de \( \eK\) sont des polynômes en \( \bar X\).
    \end{enumerate}
\end{proof}

%///////////////////////////////////////////////////////////////////////////////////////////////////////////////////////////
\subsubsection{La version plus élaborée}
%///////////////////////////////////////////////////////////////////////////////////////////////////////////////////////////

Construire \( \eF_q\) comme quotient de \( \eF_p[X]\) par un polynôme irréductible quelconque ne donne pas d'information sur les générateurs de \( \eF_q^*\), et en particulier il n'est pas toujours vrai que \( \bar X\) est générateur.

\begin{example}
    Construisons \( \eF_4\). Le polynôme \( X^2+X+1\) est irréductible dans \( \eF_2\) parce qu'il n'a pas de racines (c'est vite vu : dans \( \eF_2\) il n'y a que deux candidats). Donc \( \eF_4=\eF_2[X]/(X^2+X+1)\).
\end{example}

\begin{remark}
    Le corps \( \eF_2\) n'est pas un sous-corps de \( \eC\) parce que leurs caractéristiques ne sont pas les mêmes. Une conséquence est que les racines de polynômes peuvent être très différentes. Par exemple le polynôme \( X^2+1\) accepte \( x=1\) comme racine dans \( \eF_2\) tandis qu'il a pour racines \( \pm i\) dans \( \eC\).

    En changeant de corps, les racines peuvent donc complètement changer. Ce n'est pas juste qu'il y a des racines dans l'un et pas dans l'autre.
\end{remark}

\begin{example}     \label{ExemWUdrcs}
    Cherchons à construire \( \eF_{16}\) comme quotient de \( \eF_2\) par un polynôme de degré \( 4\).
    \begin{verbatim}
----------------------------------------------------------------------
| Sage Version 4.7.1, Release Date: 2011-08-11                       |
| Type notebook() for the GUI, and license() for information.        |
----------------------------------------------------------------------
sage: x=polygen(GF(2))
sage: -x-1
x + 1
sage: Q=x**15-1
sage: Q.factor()
(x + 1) * (x^2 + x + 1) * (x^4 + x + 1) * (x^4 + x^3 + 1)
            * (x^4 + x^3 + x^2 + x + 1)
    \end{verbatim}
    Les polynômes candidats à avoir des racines génératrices sont donc au nombre de \( 3\):
    \begin{subequations}
        \begin{align}
            P_1 & =X^4+X+1              \\
            P_2 & =X^4+X^3+1            \\
            P_3 & =X^4+X^3+X^2+X+1.
        \end{align}
    \end{subequations}
    Dans le quotient \( \eF_2[X]/P_3\), l'élément \( \bar X\) n'est pas générateur. En effet nous avons \( X^4=X^3+X^2+X+1\) et par conséquent les puissances successives de \( X\) sont
    \begin{subequations}
        \begin{align}
            & X                        \\
            & X^2                      \\
            & X^3                      \\
            & X^4  =X^3+X^2+X+1        \\
            & 1.
        \end{align}
    \end{subequations}
    La classe de \( X\) dans \( \eF_2[X]/P_3\) n'est donc pas génératrice du groupe \( (\eF_2[X]/P_3)^*\).

    Le polynôme \( P_1=X^4+X+1\) par contre est primitif parce que les puissances de \( X\) dans \( \eF_2[X]/P_1\) sont
    \begin{subequations}
        \begin{align}
            & X               \\
            & X^2             \\
            & X^3             \\
            & X+1             \\
            & X^2+X           \\
            & X^3+X^2         \\
            & X+1+X^3         \\
            & X^2+1           \\
            & X^3+X           \\
            & X+1+X^2         \\
            & X^2+X+X^3       \\
            & X^3+X^2+X+1     \\
            & 1+X^2+X^2       \\
            & 1+X^3           \\
            & 1
        \end{align}
    \end{subequations}
    Cela fait \( 15\) puissances distinctes, ce qui prouve que \( P_1\) est primitif. Nous verrons plus loin comment alléger un peu la vérification de la primitivité de \( P_1\).
\end{example}

\begin{proposition}[\cite{MonCerveau}]            \label{PropNsLqWb}
    Soient un nombre premier \( p\), un entier non nul \( n\in \eN^*\) ainsi qu' un polynôme \( P\) irréductible unitaire primitif dans \( \eF_p[X]\). Nous considérons \( \eK=\eF_p[X]/P\) et \( \alpha=\bar X\in \eK\). En notant \( q=p^n\) nous avons
    \begin{enumerate}
        \item
            Les racines de \( P\) sont \( \{ \alpha,\alpha^p,\ldots, \alpha^{p^{n-1}} \}\) et \( \alpha^q=\alpha\).
        \item
            \( P\) est le polynôme minimal de \( \alpha\).
        \item
            \( P\) est scindé dans \( \eK\).
        \item
            \( P\) divise \( X^q-X\) dans \( \eK\).
        \item
            La famille \( \{1, \alpha,\alpha^2,\ldots, \alpha^{n-1} \}\) est une base de \( \eK\) en tant qu'espace vectoriel sur \( \eF_p\).
        \item
            En tant qu'ensemble,
            \begin{equation}
                \eF_q=\{0, \alpha,\alpha^2,\alpha^3,\ldots, \alpha^{q-1} \},
            \end{equation}
            et les \( \alpha^k\) sont distincts pour \( k=1,\ldots, q-1\).
    \end{enumerate}
\end{proposition}

\begin{proof}
    La plupart des assertions sont des corolaires ou des paraphrases de résultats contenus dans les propositions précédentes.
    \begin{enumerate}
        \item
            L'assertion à propos des racines de \( P\) est contenue dans le lemme~\ref{Lembcerei}. D'autre part le groupe \( (\eF_p[X]/P)^*\) est cyclique d'ordre \( q-1\). Par conséquent le corolaire~\ref{CorpZItFX} indique que \( \alpha^{q-1}=1\) et donc \( \alpha^q=\alpha\).
        \item
            Soit \( \tilde P\) un polynôme annulateur de \( \alpha\). Nous voyons que si \( \beta\) est racine de \( \tilde P\) alors \( \beta^p\) est également racine de \( \tilde P\) en utilisant les techniques habituelles. Par conséquent toutes les racines de \( P\) sont racines de \( \tilde P\), ce qui implique que \( \tilde P\) est de degré au moins égal à celui de \( P\).
        \item
            Possédant \( n\) racines distinctes dans \( \eK\), le polynôme \( P\) est scindé.
        \item
            D'après le lemme~\ref{propQRcUlq} un polynôme irréductible de degré \( n\) divise le polynôme \( X^{p^n}-X\). Une autre façon de montrer ce point est de remarquer que le polynôme \( P\) est scindé et que toutes ses racines sont également racines de \( X^q-X\).
        \item
            Une combinaison linéaire nulle entre les éléments de \( \{ 1,\alpha,\alpha^2,\ldots, \alpha^{n-1} \}\) serait un polynôme annulateur de degré \( n-1\) de \( \alpha\). Cet ensemble est donc libre. Par ailleurs un ensemble libre de \( n\) éléments dans un espace vectoriel de dimension \( n\) est générateur.
        \item
            Si \( \alpha^l=\alpha^k\) avec \( k<l\) et \( k,l\leq q\) alors nous avons \( \alpha^r=1\) avec \( r=l-k<q\), ce qui contredirait la primitivité de \( P\). Les éléments \( 0,\alpha,\ldots, \alpha^{q-1}\) étant distincts et au nombre de \( q\), ils forment tout l'ensemble \( \eF_q\).
    \end{enumerate}
\end{proof}

%---------------------------------------------------------------------------------------------------------------------------
\subsection{Exemple : étude de \texorpdfstring{$\eF_{16}$}{F16}}
%---------------------------------------------------------------------------------------------------------------------------

Dans cette section nous voulons construire \( \eF_{16}\). Nous considérons donc \( p=2\) et \( n=4\). Des polynôme irréductibles de degré \( 4\) dans \( \eF_2[X]\) ne sont pas très difficiles à trouver. Par exemple \( X^4+X^3+X^2+X+1\).

Si vous en voulez d'autres, en voici.

\begin{probleme}
    Le lemme suivant me semble douteux. Écrivez-moi si vous avez une preuve ou un contre-exemple.
\end{probleme}

\begin{lemma}[\cite{MonCerveau}]        \label{LEMooTBROooANstIL}
    Soit un polynôme de degré \( 4\) dans \( \eF_2[X]\). Si il vérifie
    \begin{enumerate}
        \item
            le terme constant est non nul,
        \item
            il y un nombre impair de termes non nuls,
    \end{enumerate}
    alors il est irréductible.
\end{lemma}

Les polynômes primitifs par contre doivent être trouvés parmi les diviseurs irréductibles de \( X^{15}-1\). Montrons que
\begin{equation}
    P=X^4+X^3+1
\end{equation}
est primitif. Nous posons \( \omega=\bar X\in \eF_2[X]/P\). L'ordre de \( \omega\) dans le groupe \( (\eF_2[X]/P)^* \) doit être un diviseur de \( 15\) et donc peut être seulement \( 1\), \( 3\), \( 5\) ou \( 15\). Le fait que l'ordre ne soit ni \( 1\) ni \( 3\) est trivial parce que le degré de \( P\) est \( 4\). Montrons que l'ordre de \( \omega\) n'est pas \( 5\) non plus :
\begin{equation}
    \omega^5=\omega^4\omega=(\omega^3+1)\omega=\omega^4+\omega=\omega^3+\omega+1\neq 1.
\end{equation}
Dans ce calcul nous avons abondamment utilisé le fait que \( -1=1\).

À partir de maintenant nous posons \( \eK=\eF_2[X]/P\). Les racines de \( P\) sont \( \omega,\omega^2,\omega^4\) et \( \omega^8\). En effet si \( \beta\) est une racine de \( P\), alors \( \beta^2\) est une racine en vertu de
\begin{equation}
    P(\beta^2)=(\beta^2)^4+(\beta^2)^3+1=(\beta^4)^2+(\beta^3)^2+1^2=(\beta^4+\beta^3+1)^2=0.
\end{equation}
Ici nous avons implicitement utilisé le lemme~\ref{LemZrUUOz}. D'autre part \( P\) ne peut pas avoir plus de \( 4\) racines.

\begin{proposition}
    L'ensemble \( \{ \omega,\omega^2,\omega^4,\omega^8 \}\) est une base de \( \eF_{16}\) sur \( \eF_2\).
\end{proposition}

\begin{proof}
    Nous savons que \( \{ 1,\omega,\omega^2,\omega^3 \}\) est une base. En effet cet ensemble est libre (sinon \( \omega\) aurait un polynôme annulateur de degré \( 3\)) et générateur parce que l'espace engendré par \( 4\) vecteurs indépendants sur \( \eF_2\) contient \( 2^4=16\) éléments.

    Nous posons \( e_0=1\), \( e_1=\omega\), \( e_2=\omega^2\), \( e_3=\omega^3\) et \( f_1=\omega\), \( f_2=\omega^2\), \( f_3=\omega^4\), \( f_4=\omega^8\). En utilisant le calcul modulo \( \omega^4+\omega^3+1=0\) et \( 2=0\) nous trouvons
    \begin{subequations}
        \begin{align}
            f_1 &=\omega                \\
            f_2 &=\omega^2              \\
            f_3 &=\omega^3+1            \\
            f_4 &=\omega^3+\omega^2+\omega.
        \end{align}
    \end{subequations}
    Ensuite nous montrons que les vecteurs \( e_i\) peuvent être construits comme combinaisons linéaires des vecteurs \( f_j\) :
    \begin{subequations}
        \begin{align}
            f_1+f_2+f_3+f_4 &=e_0       \\
            f_1             &=e_1       \\
            f_2             &=e_2       \\
            f_1+f_2+f_4     &=e_3.
        \end{align}
    \end{subequations}
    Les quatre vecteurs \( f_j\) forment donc bien un base parce qu'ils sont générateurs d'un espace de dimension \( 4\).
\end{proof}

\begin{example}
    \begin{enumerate}
        \item
            Résoudre dans \( \eF_{16}\) l'équation \( x^5=a\) en discutant éventuellement en fonction de la valeur de \( a\).
        \item
            Montrer qu'il existe quatre éléments \( \gamma\in\eF_{16}\) tels que pour chacun d'eux l'ensemble \( B_{\gamma}=\{ \gamma,\gamma^2,\gamma^4,\gamma^8 \}\) est une base de \( \eF_{16}\) sur \( \eF_2\) telle que le produit de deux éléments de \( B_{\gamma}\) est, soit un élement de \( B_{\gamma}\), soit \( 1\).
    \end{enumerate}

    C'est parti !
    \begin{enumerate}
        \item
            Si \( a=0\), alors \( x=0\) est la seule solution. Si \( a\neq 0 \) alors \( a\) est une puissance de \( \omega\); nous posons \( a=\omega^l\). Nous cherchons \( x\) sous la forme \( x=\omega^k\). L'équation à résoudre pour \( k\) est
            \begin{equation}
                \omega^{5k}=\omega^l
            \end{equation}
            où \( l\) est donné. Cette équation revient à
            \begin{equation}
                5k=l\mod 15.
            \end{equation}
            Si \( l\) n'est pas un multiple de \( 5\), alors il n'y a pas de solution. Il n'y a des solutions que pour \( l=0,5,10\) et elles sont :
            \begin{equation}
                k=\begin{cases}
                    3,6,9,12    &   \text{si } l\text{=0}   \\
                    1           &   \text{si } l\text{=5}   \\
                    2           &   \text{si } l\text{=10}
                \end{cases}
            \end{equation}
        \item
            Nous cherchons \( \gamma\) sous la forme \( \gamma=\omega^k\). Parmi les nombreuses contraintes liées à l'énoncé, nous devons avoir
            \begin{equation}
                \gamma^5=1,\gamma,\gamma^2,\gamma^4,\gamma^8.
            \end{equation}
            Les possibilités \( \gamma^5=\gamma,\gamma^2,\gamma^4,\gamma^5\) ne sont pas bonnes parce qu'elles impliqueraient que \( B_{\gamma}\) n'est pas une base. Reste à explorer \( \gamma^5=1\).

            Étant donné le premier point, nous restons avec les possibilités
            \begin{equation}
                \gamma=1,\omega^3,\omega^6,\omega^9,\omega^{12}.
            \end{equation}
            Évidemment \( \gamma=1\) ne produit pas une base. Avec \( \gamma=\omega^3\) nous trouvons
            \begin{equation}
                B_{\gamma}=\{ \omega^3,\omega^6,\omega^{12},\omega^{24} \}=\{ \omega^3,\omega^6,\omega^{12},\omega^9 \}
            \end{equation}
            où nous avons utilisé le fait que \( \omega^k=\omega^{k\mod 15}\). En utilisant le fait que \( \omega^4=\omega^3+1\) nous trouvons
            \begin{subequations}
                \begin{align}
                    \omega^5      &=\omega^3+\omega+1                 \\
                    \omega^6      &=\omega^3+\omega^2+\omega+1        \\
                    \omega^9      &=\omega^2+1                        \\
                    \omega^{12}   &=\omega+1.
                \end{align}
            \end{subequations}
            L'ensemble \( B_{\gamma}\) est alors formé des éléments
            \begin{subequations}
                \begin{align}
                    f_1 &=\omega^3                          \\
                    f_2 &=\omega^3+\omega^2+\omega+1        \\
                    f_3 &=\omega+1                          \\
                    f_4 &=\omega^2+1.
                \end{align}
            \end{subequations}
            Il est assez simple de vérifier que c'est une base en remarquant que \( f_1+f_2+f_2+f_4=1\).

            Les possibilités \( \gamma=\omega^6,\omega^9,\omega^{12}\) produisent les mêmes ensembles \( B_{\gamma}\).
    \end{enumerate}
\end{example}

%---------------------------------------------------------------------------------------------------------------------------
\subsection{Polynômes irréductibles sur $\eF_q$}
%---------------------------------------------------------------------------------------------------------------------------

\begin{definition}  \label{DefWXBkOxg}
    La \defe{fonction de Möbius}{fonction!de Möbius} est la fonction \( \mu\colon \eN^*\to \{ -1,0,1 \}\) définie par
    \begin{equation}
        \mu(n)=\begin{cases}
            0   &   \text{si } n\text{ est divisible par un carré différent de } 1\text{,}                \\
            1   &   \text{si } n\text{ est le produit d'un nombre pair de nombres premiers distincts,}    \\
            -1  &   \text{si } n\text{ est le produit d'un nombre impair de nombres premiers distincts,}
        \end{cases}
    \end{equation}
\end{definition}

\begin{proposition}[\cite{POkXeBE}]     \label{PROPooOVYJooFvmxyj}
    Si \( m\) et \( n\) sont strictement positifs et premiers entre eux, alors
    \begin{equation}
        \mu(mn)=\mu(m)\mu(n).
    \end{equation}
    De plus nous avons
    \begin{equation}
        \sum_{d\divides n}\mu(d)=\begin{cases}
            1    &   \text{si } n=1     \\
            0    &   \text{si } n>1
        \end{cases}
    \end{equation}
\end{proposition}

\begin{proposition}[Formule d'inversion de Möbius\cite{POkXeBE}]    \label{PropLBZoIoO}
    Soient \( f,g\colon \eN\to \eC\) telles que pour tout \( n\geq 1\),
    \begin{equation}
        g(n)=\sum_{d\divides n}f(d).
    \end{equation}
    Alors
    \begin{equation}
        f(n)=\sum_{d\divides n}\mu\left( \frac{ n }{ d } \right)g(d)
    \end{equation}
    où \( \mu\) est la fonction de Möbius pour tout \( n\geq 1\).
\end{proposition}
\index{formule!inversion Möbius}

\begin{lemma}[\cite{ERWMpWo}]   \label{LemRGuWqNu}
    Soient \( P,Q\in \eK[X]\) ayant une racine commune dans une extension \( \eL\) de \( \eK\). Si \( P\) est irréductible, alors \( P\divides Q\).
\end{lemma}

\begin{proof}
    Si \( P\) ne divise pas \( Q\), alors \( P\) et \( Q\) sont premiers entre eux parce que dans la décomposition en irréductibles de \( Q\), il n'y a pas de \( P\) tandis que dans celle de \( P\), il n'y a que \( P\). Par conséquent, il existe \( a,b\in \eK\subset\eL\) tels que\footnote{Théorème de Bézout,~\ref{ThoBezoutOuGmLB}.} \( aP+bQ=1\). Cette dernière égalité est encore valable dans \( \eL\) et donc rend impossible l'existence d'une racine commune.
\end{proof}

\begin{proposition}[\cite{ERWMpWo,KXjFWKA}] \label{PropVFNOvzZ}
    Soit \( p\) un nombre premier, \( n\geq 1\) et \( r\in \eN^*\). Nous notons \( q=p^r\), \( A(n,q)\), l'ensemble des polynômes unitaires irréductibles de degré \( n\) sur \( \eF_q\). Nous notons aussi \( I(n,q)=\Card\big( A(n,q) \big)\). Alors :
    \begin{enumerate}
        \item
            Le polynôme \( X^{q^n}-X\) se décompose en irréductibles de la façon suivante :
            \begin{equation}
                X^{q^n}-X=\prod_{d\divides n}\,\prod_{p\in A(d,q)}P.
            \end{equation}
        \item
            Le nombre d'irréductibles est donné par
            \begin{equation}
                I(n,q)=\frac{1}{ n }\sum_{d\divides n}\mu\left( \frac{ n }{ d } \right)q^d
            \end{equation}
            où \( \mu\) est la fonction de Möbius (définition~\ref{DefWXBkOxg}).
        \item
            Nous avons l'équivalence de suite
            \begin{equation}
                I(n,q)\sim_{n\to\infty}\frac{ q^n }{ n }.
            \end{equation}
    \end{enumerate}
\end{proposition}
\index{polynôme!irréductible!sur $ \eF_q$}

\begin{proof}
    \begin{enumerate}
        \item
            Soit un diviseur \( d\) de \( n\) et \( P\in A(d,q)\). Montrons que \( P\) divise \( X^{q^n}-X\). Nous considérons le corps \( \eK=\eF_q[X]/(P)\), qui est une extension de degré \( \deg(P)\) de \( \eF_q\) parce qu'il s'agit des polynômes de degré au maximum \( \deg(P)\) à coefficients dans \( \eF_q\). Ce corps possède donc \( q^d\) éléments et est isomorphe à \( \eF_{q^d}\) par la proposition~\ref{PropCRPjZsp}. Par construction dans \( \eK\), l'élément \( \alpha=[X]\) (la classe de \( X\) dans le quotient par \( P\)) est une racine de \( P\). Cet élément est également une racine de \( X^{q^d}-X\) parce que tout élément de \( \eF_{q^d}\) est une racine de ce polynôme. Ce dernier point est la proposition~\ref{propQRcUlq}.

            Nous sommes donc dans la situation où \( P\) et \( X^{q^d}-X\) ont une racine commune dans l'extension \( \eF_q[X]/(P)\). Nous en déduisons que \( \alpha\) est aussi une racine de \( X^{q^n}-X\). En effet en utilisant le fait que \( \alpha^{q^d}=\alpha\), nous avons
            \begin{equation}
                \alpha^{q^n}=\alpha^{q^{kd}}=\alpha^{q^dq^{(k-1)d}}=\left( \alpha^{q^d} \right)^{q^{(k-1)d}}=\alpha^{q^{(k-1)d}},
            \end{equation}
            donc par récurrence, on a encore \( \alpha^{q^n}=\alpha\), et \( \alpha\) est racine de \( X^{q^n}-X\). Puisque \( P\) est irréductible, le lemme~\ref{LemRGuWqNu} nous indique que \( P\) divise \( X^{q^n}-X\). Nous en déduisons que \( P\) divise \( X^{q^n}-X\).

            Étant donné que tous les éléments de \( A(d,q)\) divisent \( X^{q^n}-X\) et sont irréductibles, leur produit divise encore \( X^{q^n}-X\) :
            \begin{equation}
                \prod_{d\divides n}\prod_{P\in A(d,q)}P\divides X^{q^n}-X.
            \end{equation}

            Nous devons à présent montrer que tous les facteurs irréductibles de \( X^{q^n}-X\) sont dans un \( A(d,q)\) avec \( d\divides n\). Soit donc \( P\) un facteur irréductible de \( X^{q^n}-X\) de degré \( d\geq 1\). Nous posons encore \( \eK=\eF_q[X]/(P)\) et nous utilisons la propriété de multiplication sur les degrés (proposition~\ref{PropGWazMpY}) :
            \begin{equation}
                [\eF_{q^n}:\eK][\eK:\eF_q]=[\eF_{q^n}:\eF_q]=n,
            \end{equation}
            donc \( [\eK:\eF_q]\), qui vaut \( \deg(P)\) est un diviseur de \( n\).

            Étant donné que \( X^{q^n}-X\) n'a que des racines simples sur \( \eF_{q^n}\) (à nouveau la proposition~\ref{propQRcUlq}), dans sa décomposition en irréductibles sur \( \eF_q\), il n'a pas de facteur carré; il n'a donc qu'une fois chacun des \( P\in A(d,q)\) avec \( d\divides n\). Autrement dit, tous les facteurs irréductibles de \( X^{q^n}-X\) sont dans le produit \( \prod_{d\divides n}\prod_{P\in A(d,q)}P\) et donc \( X^{q^n}-X\) divise ce gros produit :
            \begin{equation}
                X^{q^n}-X\divides \prod_{d\divides n}\prod_{P\in A(d,q)}P.
            \end{equation}
            Ayant déjà obtenu la divisibilité inverse et les polynômes étant unitaires, nous avons égalité.

        \item
            Nous passons au degré dans l'expression que nous venons de démontrer :
            \begin{equation}
                q^n=\sum_{d\divides n}d\Card\big( A(d,q) \big)=\sum_{d\divides n}dI(d,q).
            \end{equation}
            Nous pouvons utiliser la formule d'inversion de Möbius (proposition~\ref{PropLBZoIoO}) pour les fonctions \( g(n)=q^n\) et \( f(n)=dI(n,q)\). Nous écrivons alors
            \begin{equation}
                f(n)=\sum_{d\divides n}\mu\left( \frac{ n }{ d } \right)q^d,
            \end{equation}
            ou encore
            \begin{equation}
                I(n,q)=\frac{1}{ n }\sum_{d\divides n}\mu\left( \frac{ n }{ d } \right)q^d,
            \end{equation}
            ce qu'il fallait.

        \item
            Nous posons
            \begin{equation}
                r_n=\sum_{\substack{d\divides n\\d<n}}\mu\left( \frac{ n }{ d } \right)q^d,
            \end{equation}
            mais sachant que les diviseurs de \( n\), outre \( n\) lui-même, sont tous plus petits ou égaux à \( n/2\) et qu'en valeur absolue, la fonction de Möbius est toujours plus petite ou égale à\quext{Dans \cite{KXjFWKA}, ma dernière inégalité arrive comme une égalité.} \( 1\),
            \begin{equation}
                | r_n |\leq\sum_{d=1}^{\lfloor n/2\rfloor}q^d=\frac{ q-q^{\lfloor n/2\rfloor} }{ 1-q }=q\frac{ q^{\lfloor n/2}-1 }{ q-1 }\leq \frac{ q^{\lfloor n/2 \rfloor+1} }{ q-1 }.
            \end{equation}
            D'autre part en reprenant la formule déjà prouvée,
            \begin{equation}
                I(n,q)=\frac{1}{ n }\sum_{d\divides n}\mu\left( \frac{ n }{ d } \right)q^d=\frac{1}{ n }\left( r_n+\mu\left( \frac{ n }{ n } \right)q^n \right)=\frac{ r_n+q^n }{ n }.
            \end{equation}
            Au numérateur, le plus haut degré en \( n\) est \( q^n\) parce que \( r_n\) est en \( q^{\lfloor n/2\rfloor}\). Donc nous avons bien l'équivalence de suite pour \( n\to \infty\) :
            \begin{equation}
                \frac{ q^n+r_n }{ n }\sim_{n\to\infty}\frac{ q^n }{ n }.
            \end{equation}
    \end{enumerate}
\end{proof}

%---------------------------------------------------------------------------------------------------------------------------
\subsection{Matrices}
%---------------------------------------------------------------------------------------------------------------------------

\begin{proposition}
    Nous avons
    \begin{equation}
        | \GL(n,\eF_p) |=(p^n-1)(p^n-p)\ldots (p^n-p^{n-1}).
    \end{equation}
\end{proposition}

\begin{proof}
    Par construction il existe une bijection entre \( \GL(n,\eF_p)\) et l'ensemble des bases de \( \eF_p^n\). Nous devons donc seulement compter le nombre de bases. Pour le premier vecteur de base nous avons le choix entre les \( p^n-1\) éléments non nuls de \( \eF_p^n\). Pour le second nous avons le choix entre \( p^n-p\) éléments, et ainsi de suite.
\end{proof}

\begin{lemma}   \label{LemcDOTzM}
    Soit \( \eK\) un corps fini autre que \( \eF_2\)\quext{Je ne comprends pas très bien à quel moment joue cette hypothèse.}, soit un groupe abélien \( M\) et un morphisme \( \varphi\colon \GL(n,\eK)\to M\). Alors il existe un unique morphisme \( \delta\colon \eK^*\to M\) tel que \( \varphi=\delta\circ\det\).
\end{lemma}

\begin{proof}
    D'abord le groupe dérivé de \( \GL(n,\eK)\) est \( \SL(n,\eK)\) parce que les éléments de \( D\big( \GL(n,\eK) \big)\) sont de la forme \( ghg^{-1}h^{-1}\) dont le déterminant est \( 1\).

    De plus le groupe \( \SL(n,\eK)\) est normal dans \( \GL(n,\eK)\). Par conséquent \( \GL(n,\eK)/\SL(n,\eK)\) est un groupe et nous pouvons définir l'application relevée
    \begin{equation}
        \tilde \varphi\colon \frac{ \GL(n,\eK) }{ \SL(n,\eK) }\to M
    \end{equation}
    vérifiant \( \varphi=\tilde \varphi\circ\pi\) où \( \pi\) est la projection.

    Nous pouvons faire la même chose avec l'application
    \begin{equation}
        \det\colon \GL(n,\eK)\to \eK^*
    \end{equation}
    qui est un morphisme de groupes dont le noyau est \( \SL(n,\eK)\). Cela nous donne une application
    \begin{equation}
        \tilde \det\colon \frac{ \GL(n,\eK) }{ \SL(n,\eK) }\to \eK^*
    \end{equation}
    telle que \( \det=\tilde \det\circ\pi\). Cette application \( \tilde \det\) est un isomorphisme. En effet elle est surjective parce que le déterminant l'est et elle est injective parce que son noyau est précisément ce par quoi on prend le quotient. Par conséquent \( \tilde \det \) possède un inverse et nous pouvons écrire
    \begin{equation}
        \varphi=\tilde \varphi\circ\tilde \det^{-1}\circ\tilde \det\circ\pi.
    \end{equation}
    Étant donné que \( \tilde \det\circ\pi=\det\), nous avons alors \( \varphi=\delta\circ\det\) avec \( \delta=\tilde \varphi\circ\tilde \det^{-1}\). Ceci conclut la partie existence de la preuve.

    En ce qui concerne l'unicité, nous considérons \( \delta'\colon \eK^*\to M\) telle que \( \varphi=\delta'\circ\det\). Pour tout \( u\in \GL(n,\eK)\) nous avons \( \delta'(\det(u))=\varphi(u)=\delta(\det(u))\). L'application \( \det\) étant surjective depuis \( \GL(n,\eK)\) vers \( \eK^*\), nous avons \( \delta'=\delta\).
\end{proof}

\begin{theorem}
    Soit \( p\geq 3\) un nombre premier et \( V\), un \( \eF_p\)-espace vectoriel de dimension finie \( n\). Pour tout \( u\in\GL(V)\) nous avons
    \begin{equation}
        \epsilon(u)=\left(\frac{\det(u)}{p}\right).
    \end{equation}
\end{theorem}
Ici \( \epsilon\) est la signature de \( u \) vue comme une permutation des éléments de \( \eF_p\).

\begin{proof}
    Commençons par prouver que
    \begin{equation}
        \epsilon\colon \GL(V)\to \{ -1,1 \}.
    \end{equation}
    est un morphisme. Si nous notons \( \bar u\in S(V)\) l'élément du groupe symétrique correspondant à la matrice \( u\in \GL(V)\), alors nous avons \( \overline{ uv }=\bar u\circ\bar v\), et la signature étant un homomorphisme (proposition~\ref{ProphIuJrC}),
    \begin{equation}
        \epsilon(uv)=\epsilon(\bar u\circ\bar v)=\epsilon(\bar u)\epsilon(\bar v).
    \end{equation}
    Par ailleurs \( \{ -1,1 \}\) est abélien, donc le lemme~\ref{LemcDOTzM} s'applique et nous pouvons considérer un morphisme \( \delta\colon \eF_p^*\to \{ -1,1 \}\) tel que \( \epsilon=\delta\circ\det\).

    Nous allons utiliser le lemme~\ref{Lemoabzrn} pour montrer que \( \delta\) est le symbole de Legendre. Pour cela il nous faudrait trouver un \( x\in \eF_p^*\) tel que \( \delta(x)=-1\). Étant donné que \( \det\) est surjective, nous cherchons ce \( x\) sous la forme \( x=\det(u)\). Par conséquent nous aurions
    \begin{equation}
        \delta(x)=(\delta\circ\det)(u)=\epsilon(u),
    \end{equation}
    et notre problème revient à trouver une matrice \( u\in\GL(V)\) dont la permutation associée soit de signature \( -1\).

    Soit \( n=\dim V\); en conséquence de la proposition~\ref{PropHfrNCB}\ref{ItemiEFRTg}, l'espace \( \eE_q=\eF_{p^n}\) est un \( \eF_p\)-espace vectoriel de dimension \( n\) et est donc isomorphe en tant qu'espace vectoriel à \( V\). Étant donné que \( \eF_q\) est un corps fini, nous savons que \( \eF_q^*\) est un groupe cyclique à \( q-1\) éléments. Soit \( y\), un générateur de \( \eF_q^*\) et l'application
    \begin{equation}
        \begin{aligned}
            \beta\colon \eF_q & \to \eF_q    \\
                          x   & \mapsto yx.
        \end{aligned}
    \end{equation}
    Cela est manifestement \( \eF_p\)-linéaire (ici \( y\) et \( x\) sont des classes de polynômes et \( \eF_p\) est le corps des coefficients). L'application \( \beta\) fixe zéro et à part zéro, agit comme le cycle
    \begin{equation}
        (1,y,y^2,\ldots, y^{q-2}).
    \end{equation}
    Nous savons qu'un cycle de longueur \( n\) est de signature \( (-1)^{n+1}\). Ici le cycle est de longueur \( q-1\) qui est pair (parce que \( p\geq 3\)) et par conséquent, l'application \( \beta\) est de signature \( -1\).
\end{proof}

%+++++++++++++++++++++++++++++++++++++++++++++++++++++++++++++++++++++++++++++++++++++++++++++++++++++++++++++++++++++++++++
\section{Constructions à la règle et au compas}
%+++++++++++++++++++++++++++++++++++++++++++++++++++++++++++++++++++++++++++++++++++++++++++++++++++++++++++++++++++++++++++

\begin{definition}[\cite{POUoocLUrO}]
    Soit \( E\) une partie de \( \eR^2\). Un point de \( \eR^2\) est constructible en une étape à partir de \( E\) si il est un point de \( E\) ou une intersection de deux objets parmi
    \begin{itemize}
        \item les droites passant par deux points distincts de \( E\);
        \item les cercles centrés en un point de \( E\) et dont le rayon est la distance entre deux points de \( E\).
    \end{itemize}
    Nous notons \( C_1(E)\) l'ensemble des points constructibles en une étape à partir de \( E\).

    Les points constructibles en \( n\) étapes à partir de \( E\) sont définis par récurrence : \( C_{n+1}(E)=C_1\big( C_n(E) \big)\). Enfin un point de \( \eR^2\) est \defe{constructible}{constructible!point} à partir de \( E\) si il appartient à
    \begin{equation}
        C(E)=\bigcup_{n=1}^{\infty}C_n(E).
    \end{equation}

    Un réel est \defe{constructible}{constructible!réel} si il est l'abscisse d'un point constructible.
\end{definition}
Pour toute la suite, nous allons considérer les points et réels constructibles à partir de l'ensemble \( E=\{ (0,0),(0,1) \}\).

%---------------------------------------------------------------------------------------------------------------------------
\subsection{Quelques constructions}
%---------------------------------------------------------------------------------------------------------------------------

\begin{proposition}[\cite{VFWooabXoA}]  \label{PropIMFooDWAyoH}
    Les nombres rationnels sont tous constructibles.
\end{proposition}

\begin{proof}
    Si le réel \( r\) est constructible, alors \( kr\) est également constructible pour tout \( k\in \eZ\). Nous devons donc seulement pouvoir construire le nombre \( 1/n\) pour tout \( n\in \eN^*\).

    \begin{center}
        \input{auto/pictures_tex/Fig_EHDooGDwfjC.pstricks}
    \end{center}

    La méthode pour construire le nombre \( 1/n\) est la suivante. Soit \( [AB]\) un segment de longueur \( 1\) (par exemple \( A=(0,0)\) et \( B=(1,0)\)) et un point \( C\) tel que \( [AC]\) ait une longueur \( n\). Nous plaçons sur \( [AC]\) le point \( K\) situé à une distance \( 1\) de \( A\) en pointant le compas en \( A\) et en traçant le cercle de rayon \( [AB]\).

    La droite passant par \( K\) et parallèle à \( (BC)\) coupe \( [AB]\) en un point \( L\). Maintenant le segment \( [AL]\) a une longueur \( 1/n\) par le théorème de Thalès\footnote{Théorème \ref{THOooFMMLooLmAnAd}.}.
\end{proof}

\begin{example}[Multiplication à la règle et au compas\cite{POUoocLUrO}]    \label{ExGROooIosiBt}
    Soient \( x\) et \( y\) deux nombres constructibles. Montrons qu'il est possible de construire le nombre \( xy\). La construction est la suivante :

    \begin{center}
        \input{auto/pictures_tex/Fig_UYJooCWjLgK.pstricks}
    \end{center}

    \begin{itemize}
        \item On trace deux droites sécantes en \( A\).
        \item Sur la première nous plaçons le point \( Y\) à distance \( y\) de \( A\) et le point \( P\) à distance \( 1\) de \( A\).
        \item Sur la seconde on place le point \( X\) à distance \( x\) de \( A\).
        \item On trace la droite \( (PX)\)
        \item Puis la parallèle à \( (PX)\) passant par \( Y\).
        \item Le point d'intersection entre cette dernière droite et \( (AX)\) est le point \( B\).
    \end{itemize}
    La longueur \( AB\) est égale à \( xy\).
\end{example}

\begin{example}[Racine carrée à la règle et au compas\cite{POUoocLUrO}]  \label{ExTYMooSMCvSr}

    Nous supposons que le nombre \( x\) est constructible, et nous voulons une construction qui donne un segment de longueur \( \sqrt{x}\). Nous traçons un segment \( [BC]\) dont la longueur correspond à la plus grande des valeurs entre \( x\) et \( 1\), puis le cercle de diamètre \( BC\), ensuite le point \( H\) sur \( [BC]\) tel que \( BH\) corresponde à la plus petite des valeurs entre \( x\) et \( 1\), enfin la perpendiculaire à \( (BC)\) menée par \( H\), qui rencontre le cercle en un point \( A\). D'après le théorème de Thalès sur le cercle\footnote{Théorème \ref{THOooGFTWooACQLFJ}.}, le triangle \( ABC\) est rectangle en \( A\).

    Les triangles \( ABC\) et \( ABH\) sont donc semblables parce qu'ils sont rectangles avec un angle (autre que l'angle droit) égal. Nous avons donc proportionnalité des longueurs des côtés :
    \begin{equation}
        \frac{ AB }{ BH }=\frac{ BC }{ AB },
    \end{equation}
    ce qui donne \( AB^2=BC\times BH=x\) (\( BC\) et \( BH\) valent respectivement \( 1\) et \( x\) ou le contraire).

    \begin{center}
        \input{auto/pictures_tex/Fig_QIZooQNQSJj.pstricks}
    \end{center}
\end{example}

\begin{example}[Duplication d'un angle] \label{ExAHCooELGGPa}
    Si un angle \( \alpha\) est constructible, nous allons construire les angles \( 2\alpha\), \( 3\alpha\), etc. Pour cela nous considérons un cercle de centre \( O\) et les points \( A\) et \( I\) sur le cercle tels que \( \widehat{AOI}=\alpha\). Le cercle de centre \( A\) et de rayon \( AI\) intersecte le cercle de départ en les points \( I\) et \( B\).

    Le point \( A\) est à égale distance de \( B\) et \( I\); le point \( O\) également. Donc la droite \( (OA)\) est médiatrice du segment \( [BI]\). Par conséquent elle est la hauteur du triangle isocèle \( OBI\). L'angle \( \widehat{BOA}\) est alors le même que \( \widehat{AOI}\); par conséquent \( \widehat{BOI}=2\alpha\).

    \begin{center}
        \input{auto/pictures_tex/Fig_WHCooNZAmYB.pstricks}
    \end{center}

    En traçant le cercle de centre \( B\) et de rayon \( BA\), nous continuons et nous construisons \( 3\alpha\).
\end{example}

L'exemple suivant qui permet d'additionner des angles repose sur le fait que deux cordes de mêmes longueurs sous-tendent des angles égaux, et est une adaptation simple de la duplication d'angle.

\begin{example}[Addition d'angles]  \label{ExOVDooXnWPDl}
    Quitte à soustraire ou additionner un certain nombre de fois \unit{90}{\degree}, nous supposons que les deux angles donnés sont entre \unit{0}{\degree} et \unit{90}{\degree}.

    Soient \( A,B,I\) sur un cercle de centre \( O\), et nous notons \( \alpha=\widehat{OAI}\), \( \beta=\widehat{BOI}\). Nous traçons le cercle de centre \( B\) et de rayon \( AI\); il intersecte le cercle en des points \( K_1\) et \( K_2\). Les angles \( \widehat{K_1OB}\) et \( \widehat{K_2OB}\) sont tous deux égaux à \( \alpha\).

    Les angles \( \widehat{K_1OA}\) et \( \widehat{K_2OA}\) sont égaux à \( \beta-\alpha \) et \( \beta+\alpha\).
\end{example}

\begin{normaltext}
    Notez que l'exemple \ref{ExOVDooXnWPDl} donne un moyen de construire \( \alpha+\beta\) et \( \alpha-\beta\) ensemble. Il ne permet pas de savoir lequel est \( \alpha+\beta\) et lequel est \( \alpha-\beta\).
\end{normaltext}

%---------------------------------------------------------------------------------------------------------------------------
\subsection{Nombres constructibles}
%---------------------------------------------------------------------------------------------------------------------------

\begin{theorem}[Wantzel\cite{HQZoogglNj}] \label{ThoRHFooZsLbqd}
    Le réel \( a\) est constructible si et seulement si il existe une suite finie de corps \( \eL_i\) tels que
    \begin{enumerate}
        \item
            \( \eL_0=\eQ\),
        \item
            \( \eL_{i+1}\) est un extension quadratique\footnote{C'est-à-dire une extension finie de degré \( 2\).} de \( \eL_i\)
        \item
            \( a\in \eL_n\).
    \end{enumerate}
\end{theorem}

\begin{proof}
    Soit \( \eK\) un corps de nombres constructibles (par exemple \( \eQ\)); nous notons \( E_{\eK}\) l'ensemble des points de \( \eR^2\) dont les coordonnées sont dans \( \eK\). Ce sont des points forcément constructibles.

    \begin{subproof}
        \item[Intersection de droites]
            Si \( A,B\in E_{\eK}\) alors la droite \( (AB)\) a pour équation \( ax+by+c=0\) avec \( a,b,c\in \eK\), et le point d'intersection entre deux droites est donné par la solution du système
            \begin{subequations}
                \begin{numcases}{}
                    ax+by+c=0\\
                    a'x+b'y+c'=0,
                \end{numcases}
            \end{subequations}
            dont les solutions sont encore dans \( \eK\).

        \item[Intersection droite-cerle]
            L'équation d'un cercle est de la forme
            \begin{equation}
                (x-u)^2+(y-v)^2=r^2
            \end{equation}
            où \( (u,v)\in E_{\eK}\) et \( r\) est la distance entre deux points de \( E_{\eK}\); donc \( r^2\in \eK\). En développant et en redéfinissant \( u,v\) nous voyons que tous les cercles à considérer ont une équation de la forme
            \begin{equation}
                x^2+y^2+ux+vy+t=0
            \end{equation}
            avec \( u,v,t\in \eK\). Il s'agit de voir où sont les solutions du système
            \begin{subequations}
                \begin{numcases}{}
                    ax+by+c=0\\
                    x^2+y^2+ux+vy+t=0.
                \end{numcases}
            \end{subequations}
            Si \( a\neq 0\) alors nous pouvons faire la substitution \( x=-(c+by)/a\) et obtenir l'équation suivante pour \( y\) :
            \begin{equation}
                \left( \frac{ -c-by }{ a } \right)^2+y^2+u\left( \frac{ -c-by }{ a } \right)+vy+t=0.
            \end{equation}
            Cela est de la forme \( P(y)=0\) où \( P\) est un polynôme du second degré à coefficients dans \( \eK\).

            Si \( a=0\) nous substituons \( y\) au lieu de \( x\) et le résultat est le même.

            C'est le moment de relire la proposition~\ref{PropURZooVtwNXE} qui nous assure que si le réel \( \alpha\) est une solution de \( P(y)=0\) hors de \( \eK\), alors l'extension \( \eK(\alpha)\) est de degré \( 2\) parce que le polynôme minimal de \( \alpha\) est de degré \( 2\) :
            \begin{equation}
                \big[ \eK[\alpha]:\eK \big]=2.
            \end{equation}
            De plus \( \eK[\alpha]=\eK(\alpha)\).

        \item[Intersection cercle-cercle]
            Le système
            \begin{subequations}
                \begin{numcases}{}
                    x^2+y^2+ux+vy+y=0\\
                    x^2+y^2+ax+by+c=0
                \end{numcases}
            \end{subequations}
            est équivalent au système (en substituant la seconde équation par la différence entre les deux)
            \begin{subequations}
                \begin{numcases}{}
                    x^2+y^2+ux+vy+y=0\\
                    (u-a)x+(v-b)y+t-c=0
                \end{numcases}
            \end{subequations}
            qui est à nouveau une intersection entre un cercle et une droite.
    \end{subproof}
    Passons à la conclusion. Si \( \alpha\) est un nombre constructible, alors il apparaît dans les coordonnées d'un point de \( C_m(E_{\eQ})\) pour un certain \( m\). Nous supposons (pour la récurrence) que pour chaque réel constructible en \( m-1\) étapes possède sa pile d'extension quadratiques en partant de \( \eQ\). Le nombre \( \alpha\) vérifie donc
    \begin{equation}
        a\alpha^2+b\alpha+c=0
    \end{equation}
    avec \( a,b,c\) trois réels constructibles en \( m-1\) étapes. L'hypothèse de récurrence donne donc des piles d'extensions
    \begin{equation}
        \begin{aligned}[]
            \eL_0   &=\eQ       &\eL'_0     &=\eQ         &\eL''_0      &=\eQ             \\
            L_{i+1} &=\eL_i(a_i)&\eL'_{i+1} &=\eL'_i(b_i) &\eL''_{i+1}  &=\eL''_i(c_i)    \\
            a       &\in \eL_n  &   b       &\in \eL'_{n'}&   c         &\in \eL''_{n''}.
        \end{aligned}
    \end{equation}
    Nous considérons donc la suite d'extensions de \( \eQ\) qui consiste à étendre successivement par les nombres \( a_1,\ldots, a_n,b_1,\ldots, b_{n'},c_1,\ldots, c_{n''}\) tout en excluant les doublons : il est possible que par exemple \( b_3\) soit déjà dans \( \eQ(a_1,\ldots, a_n)\). Cela nous fournit une suite d'extensions
    \begin{equation}
        \begin{aligned}[]
            \eM_0     &=\eQ             \\
            \eM_{i+1} &=\eM_i(\alpha_i) \\
            a,b,c     &\in \eM_n.
        \end{aligned}
    \end{equation}
    Ici le \( n\) n'est pas spécialement le même que celui plus haut. Maintenant, \( \alpha\) est dans un extension quadratique de \( \eM_n\).

    Pour la réciproque nous supposons avoir une tour d'extensions quadratiques \( \eL_0=\eQ\), \( \eL_i\) et \( \alpha\in \eL_n\) et nous voulons prouver que \( \alpha\) est constructible. Nous y allons par récurrence : si \( n=0\) alors \( \alpha\) est rationnel et il est constructible par la proposition~\ref{PropIMFooDWAyoH}.

    Supposons que tous les points à coordonnées dans \( \eL_i\) sont constructibles. Alors nous allons prouver que les éléments de \( \eL_{i+1}\) le sont également. Vu que \( \alpha\) est solution d'une équation de degré \( 2\) à coefficients dans \( \eL_i\), nous avons
    \begin{equation}
        a\alpha^2+b\alpha+c=0
    \end{equation}
    pour certains \( a,b,c\in\eL_i\) et donc
    \begin{equation}
        \alpha=\frac{-b\pm\sqrt{b^2-4ac} }{ 2a }.
    \end{equation}
    Les exemples~\ref{ExGROooIosiBt} et~\ref{ExTYMooSMCvSr} montrent que les produits et les racines carrées de nombres constructibles sont constructibles. Donc \( \alpha\) est constructible.
\end{proof}

%---------------------------------------------------------------------------------------------------------------------------
\subsection{Polygones constructibles}
%---------------------------------------------------------------------------------------------------------------------------

\begin{definition}
    Un angle \( \alpha\) est \defe{constructible}{constructible!angle} si le nombre \( \cos(\alpha)\) est constructible.
\end{definition}
La raison est qu'il suffit de prendre la perpendiculaire à l'axe horizontal.

\begin{lemma}[\cite{KXjFWKA}]   \label{LemMAHooXcOCpr}
    Soient \( m\) et \( n\), deux nombres premiers entre eux. L'angle \( \frac{ 2\pi }{ mn }\) est constructible si et seulement si les angles \( \frac{ 2\pi }{ m }\) et \( \frac{ 2\pi }{ n }\) sont constructibles.
\end{lemma}

\begin{proof}
    \begin{subproof}
        \item[Sens direct]
            Il suffit de pouvoir multiplier un angle par un entier, ce qui est fait dans l'exemple~\ref{ExAHCooELGGPa}.
        \item[Sens réciproque]
            Le théorème de Bézout~\ref{ThoBuNjam} nous donne \( a,b\in \eZ\) tels que \( an+bm=1\). Cela donne immédiatement
            \begin{equation}
                \frac{1}{ mn }=\frac{ a }{ m }+\frac{ b }{ n },
            \end{equation}
            et donc
            \begin{equation}
                \frac{ 2\pi }{ mn }=a\frac{ 2\pi }{ m }+b\frac{ 2\pi }{ n },
            \end{equation}
            ce qui fait que l'angle \( 2\pi/mn\) est une combinaison entière d'angles constructibles. Il est donc constructible par l'exemple~\ref{ExOVDooXnWPDl}.
    \end{subproof}
\end{proof}

\begin{lemma}   \label{LemUKNooSBzDyY}
    Soit un entier \( n\geq 3\) ayant
    \begin{equation}
        n=\prod_{i=1}^kp_i^{\alpha_i}
    \end{equation}
    comme décomposition en facteurs premiers. Les polynôme régulier à \( n\) côtés est constructible si et seulement si les angles \( \frac{ 2\pi }{ p_i^{\alpha_i} }\) sont constructibles.
\end{lemma}

\begin{proof}
    Le polygone régulier à \( n\) côtés est constructible si et seulement si l'angle \( \frac{ 2\pi }{ n }\) est constructible, c'est-à-dire si l'angle
    \begin{equation}
        \frac{ 2\pi }{ \prod_{i=1}^kp_i^{\alpha_i} }
    \end{equation}
    est constructible. Par récurrence sur le lemme~\ref{LemMAHooXcOCpr}, cet angle est constructible si et seulement si les angles \( \frac{ 2\pi }{ p_i^{\alpha_i} }\) sont constructibles.
\end{proof}

\begin{theorem}[Gauss-Wantzel\cite{KXjFWKA}]    \label{ThoTWAooEsLjJu}
    Soit \( \alpha\in \eN^*\).
    \begin{enumerate}
        \item   \label{ItemFSEooONDFrSi}
            L'angle \( \frac{ 2\pi }{ 2^{\alpha} } \) est constructible.
        \item\label{ItemFSEooONDFrSii}
            Si \( p\) est premier \( p\neq 2\) alors l'angle \( \frac{ 2\pi }{ p^{\alpha} } \) est constructible si et seulement si \( \alpha=1\) et \( p\) est un \defe{nombre de Fermat}{nombre!de Fermat}, c'est-à-dire de la forme \( 1+2^{(2^{\beta})}\) pour \( \beta\in \eN\).
        \item\label{ItemFSEooONDFrSiii}
            Le polygone régulier à \( n\) côtés est constructible si et seulement si \( n \) est le produit d'une puissance de \( 2\) et d'un nombre fini de nombres de Fermat premiers distincts.
    \end{enumerate}
\end{theorem}
\index{théorème!Gauss-Wantzel}
\index{extension!de corps!utilisation}

\begin{proof}
    Le point~\ref{ItemFSEooONDFrSi} est une construction de bissectrice, et le point~\ref{ItemFSEooONDFrSiii} consistera à remettre en place différents morceaux. Le gros de la preuve est donc consacré à~\ref{ItemFSEooONDFrSii}.
    \begin{subproof}
        \item[Sens direct]
            Nous supposons que l'angle \( \frac{ 2\pi }{ p^{\alpha} }\) est constructible; alors le nombre \( \cos\big( 2\pi/p^{\alpha} \big)\) l'est également et le théorème de Wantzel~\ref{ThoRHFooZsLbqd} nous indique que
            \begin{equation}
                \left[ \eQ\Big( \cos(2\pi/p^{\alpha}) \Big):\eQ \right]=2^m
            \end{equation}
            pour un certain \( m\). Posons \( q=p^{\alpha}\) et \( \omega= e^{2i\pi/q}\). Grâce au corolaire~\ref{CorKRTooTJtyvP}, nous savons que le polynôme minimal de \( \omega\) est le polynôme cyclotomique \( \phi_q\) dont le
            degré est\footnote{Voir juste en dessous de la définition~\ref{DefXGHooRAXlpp}.}         % en deux lignes pour vérifier les références
            \begin{equation}
                \varphi(q)=\varphi(p^{\alpha})=p^{\alpha-1}(p-1).
            \end{equation}
            Par conséquent
            \begin{equation}
                \big[ \eQ(\omega):\eQ \big]=p^{\alpha-1}(p-1).
            \end{equation}
            Mais par ailleurs
            \begin{equation}    \label{EqJDBooHURUQa}
                \omega+\omega^{-1}=2\cos(2\pi/q),
            \end{equation}
            donc \( \cos(2\pi/q)\in \eQ(\omega)\). Et en multipliant \eqref{EqJDBooHURUQa} par \( \omega\) nous trouvons le polynôme annulateur suivant pour \( \omega\) :
            \begin{equation}
                \omega^2-2\cos\left( \frac{ 2\pi }{ q } \right)\omega+1=0.
            \end{equation}
            Cela signifie que \( \omega\) est de degré \( 2\) dans \( \eQ\big( \cos(2\pi/q) \big)\), c'est-à-dire
            \begin{equation}    \label{EqSJXooRCLJyt}
                \big[ \eQ(\omega):\eQ\big( \cos(2\pi/q) \big) \big]=2.
            \end{equation}
            En remettant bout à bout et en utilisant la propriété multiplicative des degrés des extensions\footnote{Proposition~\ref{PropGWazMpY}.},
            \begin{equation}
                \underbrace{\big[ \eQ(\omega):\eQ \big]}_{p^{\alpha-1}(p-1)}=\underbrace{\Big[ \eQ(\omega):\eQ\big( \cos(2\pi/p^{\alpha}) \big) \Big]}_{2}\underbrace{\Big[ \eQ\big( \cos(2\pi/p^{\alpha}) \big):\eQ \Big]}_{2^m},
            \end{equation}
            donc \( p^{\alpha-1}(p-1)=2^{m+1}\), mais comme \( p\) est premier et impair, \( \alpha=1\) et \( p=2^{m+1}+1\). Par ailleurs \( m+1\) est un entier et nous nous proposons de mettre en facteur la puissance de \( 2\) dans son développement en facteurs premiers : \( m+1=\lambda 2^{\beta}\) avec \( \beta\in \eN\) et un certain nombre impair \( \lambda\in \eN^*\). Nous avons :
            \begin{equation}
                p=2^{\lambda 2^{\beta}}+1=\left( 2^{2^{\beta}} \right)^{\lambda}+1,
            \end{equation}
            mais le lemme~\ref{LemISPooHIKJBU}\ref{ItemLTBooAcyMtN} nous indique alors que \( 1+2^{2^{\beta}}\) divise \( 1+\left( 2^{2^{\beta}} \right)^{\lambda}=p\). Mais vu que \( p\) est premier, il ne peut être divisé que par \( p\) lui-même et donc \( \lambda\) soit être égal à \( 1\) et nous avons
            \begin{equation}
                p=1+2^{2^{\beta}},
            \end{equation}
            ce qui signifie que \( p\) est un nombre de Fermat premier.
        \item[Sens réciproque]
            Nous supposons que \( p\) est un nombre premier de la forme \( p=1+2^{2^{\beta}}\) et nous devons prouver que l'angle \( \frac{ 2\pi }{ p }\) est constructible. Pour cela nous posons tout de suite \( n=2^{\beta}\) et \( \omega= e^{2i\pi/p}\). Comme dans la première partie nous nous souvenons que le polynôme minimal de \( \omega\) est le polynôme cyclotomique \( \phi_p\) de degré \( p-1\); donc
            \begin{equation}
                [\eQ(\omega):\eQ]=p-1.
            \end{equation}
            \begin{subproof}
                \item[Un groupe d'automorphismes]
                    Nous considérons le groupe \( G=\Aut_{\eQ}\big( \eQ(\omega) \big)\) des automorphismes du corps \( \eQ(\omega)\) agissant sur \( \eQ\) comme l'identité. Tous les éléments de \( \eQ(\omega)\) étant des polynômes en \( \omega\) (proposition~\ref{PropURZooVtwNXE}), un élément \( g\in G\) est uniquement déterminé par \( g(\omega)\), et de plus \( g(0)=0\) ainsi que \( \phi_p(\omega)=0\) et que \( \phi_p\) commute avec \( g\), donc
                    \begin{equation}
                        \phi_p\big( g(\omega) \big)=g\big( \phi_p(\omega) \big)=g(0)=0,
                    \end{equation}
                    ce qui signifie que \( g(\omega)\) est une racine du polynôme cyclotomique \( \phi_p\). Par définition~\ref{DefXGHooRAXlpp}, les racines sont \( \{ \omega,\omega^2,\ldots, \omega^{p-1} \}\), ce qui signifie que l'action de \( g\) consiste à élever \( \omega\) à une certaine puissance entre \( 1\) et \( p-1\). Le groupe \( G\) est donc d'ordre \( | G |=p-1\) et a pour éléments
                    \begin{equation}
                        g_k\colon \omega\mapsto \phi^k
                    \end{equation}
                    avec \( k=1,\ldots, p-1\). L'application\footnote{Pour rappel, la notation \( [k]_p\) est la classe de l'entier \( k\) modulo \( p\), qui est un élément de \( \eZ/p\eZ\).}
                    \begin{equation}
                        \begin{aligned}
                            \psi\colon G&\to (\eZ/p\eZ)^* \\
                            g_k&\mapsto [k]_p .
                        \end{aligned}
                    \end{equation}
                    est un morphisme surjectif entre deux groupes finis de même cardinal, donc c'est un isomorphisme. Le corolaire~\ref{CorpRUndR} nous donne de plus l'isomorphisme \( (\eZ/p\eZ)^*\simeq \eZ/(p-1)\eZ\), ce qui fait que \( G\) a un élément d'ordre \( p-1\) (parce que \( \eZ/(p-1)\eZ\) en a un). Nous notons \( g_0\) cet élément.
                \item[La tour d'extensions]
                    À partir de cet élément \( g_0\in G\) nous définissons avec \( 0\leq i\leq n\) :
                    \begin{equation}
                        \eK_i=\{ z\in \eQ(\omega)\tq g_0^{2^i}(z)=z \}.
                    \end{equation}
                    Ces ensembles sont bien des corps parce que \( g_0\) est un morphisme : \( g_0(zz')=g_0(z)g_0(z')\); on en déduit immédiatement que \( g_0^2(zz')=g_0^2(z)g_0^2(z')\).
                \item[\( \eK_n=\eQ(\omega)\)]
                    Cela est dû au fait que \( g_0\) est par définition d'ordre \( 2^n\), ce qui signifie que \( g_0^{2^n}(z)=z\) pour tout \( z\in \eQ(\omega)\).
                \item[\( \eK_0=\eQ\)]
                    Puisque les éléments de \( G\) laissent \( \eQ\) invariant nous avons forcément \( \eQ\subset \eK_0\). Nous nous attelons maintenant à prouver l'inclusion inverse. Nous savons par la proposition~\ref{PropURZooVtwNXE}\ref{ItemJCMooDgEHajiv} que \( \{ \omega^i \}_{i=0,\ldots, p-2}\) est une base\quext{Ici \cite{KXjFWKA} parle de \( \eQ(\omega)/\eQ\). Dans ce cas il me semble qu'il faille faire partir les valeurs de \( i\) de \( 0\) et non de \( 1\).} de \( \eQ(\omega)\). Vu que \( g_0\colon \eQ(\omega)\to \eQ(\omega)\) est un isomorphisme, l'image d'une base est une base :
                    \begin{equation}
                        \{ g_0(\omega)^i \}_{i=0,\ldots, p-2}
                    \end{equation}
                    est également une base de \( \eQ(\omega)\). Soit \( z\in \eK_0\) et décomposons-le dans cette base\footnote{Il s'agit d'une base en tant qu'espace vectoriel sur \( \eQ\), donc les \( \lambda\i\) sont dans \( \eQ\).} :
                    \begin{equation}    \label{EqTMZooCUySZb}
                        z=\sum_{i=0}^{p-2}\lambda_i g_0(\omega)^i;
                    \end{equation}
                    nous appliquons \( g_0\) à cette égalité en tenant compte du fait que \( g_0(z)=z\) :
                    \begin{equation}    \label{EqGGDooMAngMs}
                        z=\sum_{i=0}^{p-2}\lambda_ig_0(\omega)^{i+1}.
                    \end{equation}
                    En identifiant les coefficients (et en remarquant que le dernier terme dans \eqref{EqTMZooCUySZb} est le premier dans \eqref{EqGGDooMAngMs}) nous voyons que tous les coefficients sont égaux :
                    \begin{equation}
                        z=\lambda_0g_0(\omega+\cdots +\omega^{p-1})=\lambda_0g_0\big( \phi_p(\omega)-1 \big)=-\lambda_0\in \eQ.
                    \end{equation}
                    Dans cette chaine d'égalités nous avons utilisé le fait que \( \phi_p(X)=1+X+\cdots +X^{p-1}\) (corolaire~\ref{CorTVUooErJiAC}) et que \( \phi_p(\omega)=0\) par définition.
                \item[\( \eK_i\subset \eK_{i+1}\) strictement]
                    Nous avons une inclusion pour la simple raison que si \( z\in \eK_i\), alors \( g_0^{2^i}(z)=z\). Par conséquent :
                    \begin{equation}
                        g_0^{2^{i+1}}(z)=g_0^{2\times 2^i}(z)=(g_0^{2^i})^2(z)=z.
                    \end{equation}
                    Afin de voir que l'inclusion est stricte nous montrons que l'élément
                    \begin{equation}\label{EqTBLooWpeGgk}
                        x=\sum_{k=0}^{2^{n-i}-1}g_0^{k2^{i+1}}(\omega)
                    \end{equation}
                    est dans \( \eK_{i+1}\setminus \eK_i\). Dans la base \( \{ 1,\omega,\ldots, \omega^{p-1} \}\) l'élément \( x\) a une composante \( \omega\) avec le terme \( k=0\), mais si on lui applique \( g_0^{2^i}\) nous obtenons
                    \begin{equation}
                        g_0^{2^i}(x)=\sum_{k=0}^{2^{n-i-1}-1}g_0^{2^i(1+2k)}(\omega).
                    \end{equation}
                    Une composante \( \omega\) pour cela demanderait d'avoir \( 2^i(1+2k)=\lambda 2^n\) avec \( \lambda\in \eN\); cela demanderait
                    \begin{equation}
                        k=\lambda 2^{n-i-1}-\frac{ 1 }{2}
                    \end{equation}
                    ce qui est impossible. Donc \( x\notin \eK_{i}\).

                    Nous montrons que \( x\in \eK_{i+1}\) de la même manière :
                    \begin{equation}    \label{eqXHSooGvfNul}
                        g_0^{2^{i+1}}(x)=\sum_{k=0}^{2^{n-i-1}-1}g^{2^{i+1}(k+1)}(\omega).
                    \end{equation}
                    Soit \( k_0\) entre \( 0\) et \( 2^{n-i-1}-1\); nous voulons trouver un \( k\) entre \( 0\) et \( 2^{n-i-1}-1\) tel que \( k_02^{i+1}=2^{i+1}(k+1)\); cela est très simple : il suffit de prendre \( k=k_0-1\) tant que \( k_0\neq 0\). Si \( k_0=0\) alors cela correspond au terme \( \omega\) dans \eqref{EqTBLooWpeGgk}, et il se trouve dans \eqref{eqXHSooGvfNul} avec \( k=2^{n-i-1}-1\). Donc \( g_0^{2^{i+1}}(x)=x\) et \( x\in \eK_{i+1}\).
                \item[Les degrés dans la tour]
                    Par définition de \( n\) et de \( \omega\), et par la propriété multiplicative des degrés\footnote{Proposition~\ref{PropGWazMpY}} nous avons :
                    \begin{equation}
                        2^n=p-1=\big[ \eQ(\omega):\eQ \big]=\underbrace{\big[ \eQ(\omega):\eK_{n-1} \big]\ldots\big[ \eK_1:\eQ \big]}_{ n\text{ facteurs}}.
                    \end{equation}
                    Un produit de \( n\) facteurs entiers tous strictement plus grands que zéro doit valoir \( 2^n\). Ils doivent donc tous valoir \( 2\) et nous avons en particulier
                    \begin{equation}
                        \big[ \eK_{n-1}:\eQ \big]=2^{n-1}.
                    \end{equation}
                \item[\( \cos(2\pi/p)\in \eK_{n-1}\)]
                    Soit \( f=g_0^{2^n-1}\); puisque l'action de \( g_0\) (comme de tous les éléments de \( G\)) est de décaler les puissances de \( \omega\), il existe \( \lambda\in \eN\) tel que \( f(\omega)=\omega^{\lambda}\). Mais \( f^2=\id\) donc
                    \begin{equation}
                        \omega=f^2(\omega)=\omega^{2\lambda},
                    \end{equation}
                    ce qui donne \( 1=\omega^{\lambda^2-1}\) ou encore que \( p\) divise \( \lambda^2-1\) parce que \( \omega\) est une racine primitive \( p\)\ieme de l'unité. Par conséquent \( [\lambda^2-1]_p=0\) ce qui donne \( [\lambda]_p=\pm[1]_p\), mais comme \( f\neq \id\) nous avons
                    \begin{equation}
                        [\lambda]_p=-[1]_p.
                    \end{equation}
                    Cela fait \( f(\omega)=\omega^{-1}\).

                    Par ailleurs \( \eQ(\omega)\) est un corps, donc il contient
                    \begin{equation}
                        \cos\left( \frac{ 2\pi }{ p } \right)=\frac{ 1 }{2}(\omega+\omega^{-1}).
                    \end{equation}
                    Nous en déduisons que \( \cos(2\pi/p)\) est un point fixe de \( f=g_0^{2^-1}\) :
                    \begin{equation}
                        f\big( \cos(2\pi/p) \big)=\frac{ 1 }{2}\big( f(\omega)+f(\omega^{-1}) \big)=\cos(2\pi/p).
                    \end{equation}
                    Être un point fixe de \( g_0^{2^{n-1}}\) signifie être un élément de \( \eK_{n-1}\) :
                    \begin{equation}
                        \cos\left( \frac{ 2\pi }{ p } \right)\in \eK_{n-1}.
                    \end{equation}
                \item[Questions de degrés]
                    Nous avons alors les (non)inclusions suivantes :
                    \begin{equation}
                        \eQ\left( \cos(2\pi/p) \right)\subset \eK_{n-1}\varsubsetneq \eK_n=\eQ(\omega),
                    \end{equation}
                    ce qui fait que
                    \begin{equation}    \label{EqGEKooJBAQbg}
                        1<\big[ \eQ(\omega):\eK_{n-1} \big]\leq\big[ \eQ(\omega):\eQ\big( \cos(2\pi/p) \big) \big]=2
                    \end{equation}
                    La première inégalité est le fait que \( \eQ(\omega)\) n'est pas égal à \( \eK_{n-1}\). La dernière égalité se démontre de la même manière que \eqref{EqSJXooRCLJyt} (ici \( \alpha=1\)). Les inégalités de \eqref{EqGEKooJBAQbg} sont donc en réalité des égalités :
                    \begin{equation}
                        \big[ \eQ(\omega):\eK_{n-1} \big]=\big[ \eQ(\omega):\eQ\big( \cos(2\pi/p) \big) \big]=2.
                    \end{equation}
                    Cela, combiné au fait que \( \eQ\big( \cos(2\pi/p) \big)\subset \eK_{n-1}\) donne
                    \begin{equation}
                        \eK_{n-1}=\eQ\big( \cos(2\pi/p) \big).
                    \end{equation}
                    Mais nous savons déjà que \( \big[ \eK_{n-1}:\eQ \big]=2^{n-1}\), donc
                    \begin{equation}
                        \big[ \eQ\big( \cos(2\pi/p) \big):\eQ \big]=2^{n-1}
                    \end{equation}
                    et le nombre \( \cos(2\pi/p)\) est bien sûr le sommet d'une tour d'extensions quadratiques partant de \( \eQ\). Il est donc constructible par le théorème de Wantzel~\ref{ThoRHFooZsLbqd}.
            \end{subproof}
    \end{subproof}
    Il est maintenant l'heure de conclure en prouvant le point~\ref{ItemFSEooONDFrSiii}. D'abord si \( n\) est un produit de nombres premiers de Fermat distincts, alors \( n=p_1\ldots p_m\) et l'angle \( \frac{ 2\pi }{ n }\) est constructible si et seulement si chacun des angles \( \frac{ 2\pi }{ p_k }\) est constructible (lemme~\ref{LemUKNooSBzDyY}), ce qui est le cas d'après la partie~\ref{ItemFSEooONDFrSii}. Si \( n\) est le produit d'une puissance de \( 2\) avec un produit de nombres premiers de Fermat, le polygone à \( n\) côtés est tout autant constructible : il suffit de bissecter les angles.

    À l'inverse si le polynôme à \( n\) avec \( n\) impair côtés est constructible, alors \( n=\prod_ip_i^{\alpha_i}\), dont l'angle est constructible si et seulement si \( \alpha=1\) et \( p_i\) est un nombre premier de Fermat. Et si le polygone à \( n\) côtés (\( n\) pair) est constructible, alors \( n=2^{\lambda}\prod_ip_i^{\alpha_i}\). Ce polygone est constructible si et seulement si le polygone à \( \prod_ip_i^{\alpha_i}\) est constructible : il suffit de regrouper les côtés par deux.
\end{proof}

\begin{remark}
    D'après Wikipédia\cite{PLQooFQtjyR}, les seuls nombres de Fermat connus pour être premiers sont
    \begin{subequations}
        \begin{align}
            F_0 &=3     \\
            F_1 &=5     \\
            F_2 &=17    \\
            F_3 &=257   \\
            F_4 &=65537.
        \end{align}
    \end{subequations}
    Pour les autres, essentiellement on ne sait pas. Il n'est même pas sûr qu'il y en ait d'autres. Le problème des polygones constructibles n'est donc pas encore tout à fait terminé.
\end{remark}

\begin{remark}
    Les angles \( \frac{ 2\pi }{ p^{\alpha} }\) avec \( p\) premier et \( \alpha>1\) ne sont pas constructibles. Il n'est donc pas possible de trisecter l'angle \( \frac{ 2\pi }{ 3 }\). Voilà qui règle un des vieux problèmes de l'antiquité.
\end{remark}
