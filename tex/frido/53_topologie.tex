% This is part of Mes notes de mathématique
% Copyright (c) 2008-2018
%   Laurent Claessens, Carlotta Donadello
% See the file fdl-1.3.txt for copying conditions.

%+++++++++++++++++++++++++++++++++++++++++++++++++++++++++++++++++++++++++++++++++++++++++++++++++++++++++++++++++++++++++++
\section{Éléments généraux de topologie}
%+++++++++++++++++++++++++++++++++++++++++++++++++++++++++++++++++++++++++++++++++++++++++++++++++++++++++++++++++++++++++++

%---------------------------------------------------------------------------------------------------------------------------
\subsection{Définitions de base}
%---------------------------------------------------------------------------------------------------------------------------

\begin{definition}		\label{DefTopologieGene}
Soit \( X \), un ensemble et \( \mT \), une partie de l'ensemble de ses parties qui vérifie les propriétés suivantes.
\begin{enumerate}
\item
  Les ensembles \( \emptyset \) et \( X \) sont dans \( \mT \),
\item
  Une union quelconque\footnote{Par «quelconque» nous entendons vraiment quelconque : c'est à dire indicée par un ensemble qui peut autant être \( \eN\) que \( \eR\) qu'un ensemble encore considérablement plus grand.} d'éléments de \( \mT\) est dans \( \mT\).
\item
  Une intersection \emph{finie} d'éléments de \( \mT\) est dans \( \mT\).
\end{enumerate}
Un tel choix \( \mT \) de sous-ensembles de \( X \) est une  \defe{topologie}{topologie} sur \( X \), et les éléments de \( \mT \) sont appelés des \defe{ouverts}{ouvert}. On dit aussi que \( (X,\mT) \) (voire simplement \( X \) lorsqu'il n'y a pas d'ambiguïté) est un  \defe{\href{http://fr.wikipedia.org/wiki/Espace_topologique}{espace topologique}}{espace topologique}.
\end{definition}

\begin{definition}		\label{DefFermeVoisinage}
Si \(X \) est un espace topologique, un sous-ensemble \( F \) de \( X \) est dit \defe{fermé}{fermé} si son complémentaire, \( F^c \), est ouvert.

Si \(a \in X\), on dit que \(V \subset X\) est un \defe{voisinage}{voisinage} de \(a\) s'il existe un ouvert \(\mO \in \mT\) tel que \(a \in \mO\) et \(\mO \subset V\).
\end{definition}

\begin{lemma}   \label{LemQYUJwPC}
    Union et intersection de fermés.
    \begin{enumerate}
        \item
            Une intersection quelconque de fermés est fermée.
        \item       \label{ItemKJYVooMBmMbG}
            Une union finie de fermés est fermée.
    \end{enumerate}
\end{lemma}

\begin{proof}
    Soit \( \{ F_i \}_{i\in I} \) est un ensemble de fermés; nous avons
    \begin{equation}
        \left( \bigcap_{i\in I}F_i \right)^c=\bigcup_{i\in I}F_i^c.
    \end{equation}
    Le membre de droite est une union d'ouverts, c'est donc un ouvert; donc l'intersection qui apparaît dans le membre de gauche est le complémentaire d'un ouvert: c'est donc un fermé.

    De la même manière, le complémentaire d'une union finie de fermés est une intersection finie de complémentaires de fermés, et est donc ouvert\footnote{Un bon exercice est d'écrire ces unions et intersections, pour se convaincre que ça fonctionne.}.
\end{proof}

Dans un espace topologique, nous avons une caractérisation très importante des ouverts.
\begin{theorem}		\label{ThoPartieOUvpartouv}
    Une partie d'un espace topologique est ouverte si et seulement si elle contient un voisinage\footnote{Définition~\ref{DefFermeVoisinage}.} ouvert de chacun de ses éléments.
\end{theorem}

\begin{proof}
    Soit \( X\) un espace topologique et \( A\subset X\). Le sens direct est évident : $A$ lui-même est un ouvert autour de $x\in A$, qui est inclus dans $A$.

Pour le sens inverse, nous supposons que \( A\) contienne un ouvert autour de chacun de ses points. Pour chaque $x\in A$, nous considérons l'ensemble $\mO_x\subset A$, un ouvert autour de $x$. Nous avons que
\begin{equation}	\label{EqAUniondesOx}
	A=\bigcup_{x\in A}\mO_x.
\end{equation}
En effet $A\subset\bigcup_{x\in A}\mO_x$ parce que tous les éléments de $A$ sont dans un des $\mO_x$, par construction. D'autre part, $\bigcup_{x\in A}\mO_x\subset A$ parce que chacun des $\mO_x$ est compris dans $A$.

L'union du membre de droite de \eqref{EqAUniondesOx} est une union d'ouverts et est donc un ouvert. Cela prouve que $A$ est un ouvert.

\end{proof}
Une utilisation typique de ce théorème est faite dans le lemme~\ref{LemMESSExh}.

Quelques exemples particuliers de topologies.

\begin{example}\label{DefTopologieGrossiere}
  Pour un ensemble \( X \) quelconque, on considère l'ensemble \( \mT = \{ \emptyset; X\} \). Avec cet ensemble, on confère à \(X \) une structure d'espace topologique - même si elle nous apprend peu de choses\dots La topologie ainsi posée sur \(X \) est appelée \defe{topologie grossière}{topologie!grossière}.
\end{example}

\begin{example}\label{DefTopologieDiscrete}
  Pour un ensemble \( X \) quelconque, on considère l'ensemble \( \mT \) constitué de toutes les parties de \( X \). Avec cet ensemble, on confère à nouveau une structure d'espace topologique à \(X \); toutes les parties sont des ouverts, et aussi des fermés. La topologie ainsi posée sur \(X \) est appelée \defe{topologie discrète}{topologie!discrète}.
\end{example}

\begin{example}\label{DefTopologieEngendree}
  Soit \( X \) un ensemble, et \( \mT_0 \) un sous-ensemble de parties de \( X \). On construit alors l'ensemble \( \mT \) par
  \begin{equation}
    \label{EqTopologieEngendree}
    \mT = \bigl\{\bigcup_{\alpha \in A} \bigcap_{i=1}^{n_a} \mO_{\alpha, i} \bigr\}.
  \end{equation}
  Alors \(\mT \) est une topologie\footnote{Ce n'est pas un résultat évident: l'annoncer à un jury nécessite d'en avoir écrit la preuve.} sur \(X\), qu'on appelle \defe{topologie engendrée}{topologie!engendrée par une famille} par \( \mT_0 \).
\end{example}

\begin{example} [Toutes les topologies d'un ensemble à 3 éléments]
  On pose \( X = \{1, 2, 3\} \). Alors on peut munir \( X \) de 29 topologies différentes\footnote{Remercions Erwann Aubry d'en avoir fourni la liste exhaustive!  \url{https://math.unice.fr/~eaubry/Enseignement/L3/rappelstopo.pdf}}; saurez-vous les retrouver toutes?
\end{example}

\begin{example} [Topologie induite] \label{DefVLrgWDB}
  Soit un espace topologique \( (X, \mT) \), et soit \( Y \subset X \). Alors on peut munir \( Y \) de la topologie constituée des \( Y \cap \mO \), pour \( \mO \in \mT \): c'est ce qu'on appelle la \defe{topologie induite}{topologie!induite}.
\end{example}

%---------------------------------------------------------------------------------------------------------------------------
\subsection{Adhérence, fermeture, intérieur, point d'accumulation et isolé}
%---------------------------------------------------------------------------------------------------------------------------

\subsubsection{Intérieur}
%///////////////////////

\begin{definition}      \label{DEFooSVWMooLpAVZRInt}
    Soient un espace topologique \( X\) et une partie \( A\) de \( X\).
    \begin{enumerate}
        \item
            Un point \( x\in X\) est \defe{intérieur}{point intérieur} à \( A\) s'il est contenu dans un ouvert inclus à \( A\). L'ensemble des points intérieurs de \( A\) est noté $\Int(A)$.\nomenclature[T]{$\Int(A)$}{intérieur de \( A\)}
        \item
            L'\defe{intérieur}{intérieur} de \( A\), notée \( \mathring A\), est l'union de tous les ouverts de \( X\) contenus dans \( A\).
    \end{enumerate}
\end{definition}
\begin{remark}\label{RemIntOuvert}
  Pour tout \( A \subset X\), l'ensemble \( \mathring A\) est un ouvert, comme union quelconque d'ouverts.

  Par ailleurs, on a  \( \mathring A = \Int A \): en effet, \( x \in \Int A \) si et seulement s'il existe un ouvert \( O \) contenant \( x \) et inclus à \( A \), si et seulement si \( x \) est dans l'union de tous les ouverts contenus dans \( A \), si et seulement si \( x \in \mathring A \).
\end{remark}

\subsubsection{Adhérence et fermeture}
%///////////////////////

Disons-le tout de suite : «adhérence» et «fermeture» sont synonymes.
\begin{definition}      \label{DEFooSVWMooLpAVZR}
    Soient un espace topologique \( X\) et une partie \( A\) de \( X\).
    \begin{enumerate}
        \item
            Un point \( x\in X\) est \defe{adhérent}{point adhérent} à \( A\) si tout ouvert de \( X\) contenant \( x\) a une intersection non vide avec \( A\). L'ensemble des points d'adhérence de \( A\) est noté $\Adh(A)$.\nomenclature[T]{$\Adh(A)$}{adhérence de \( A\)}
        \item
            L'\defe{adhérence}{adhérence} de \( A\), notée \( \bar A\), est l'intersection de tous les fermés de \( X\) contenant \( A\).
    \end{enumerate}
\end{definition}

\begin{definition}      \label{DEFooGHUUooZKTJRi}
    Soient un espace topologique \( X\) et une partie \( A\) de \( X\). Un point \( s\in X \) est un \defe{point d'accumulation}{point d'accumulation} de \( A\) si tout ouvert contenant \( s\) contient au moins un élément de \( A\setminus\{ s \}\).
\end{definition}

Quelle est la différence entre un point d'accumulation et un point d'adhérence ? La différence est que tous les points de \( A\) sont des points d'adhérence de \( A\), parce que tout voisinage de \( a\in A\) contient au moins \( a\) lui-même, alors que certains points de \( A\) peuvent ne pas être des points d'accumulation de \( A\). Voir l'exemple \ref{EXooWOYQooJolaTV}.

\begin{definition}      \label{DEFooXIOWooWUKJhN}
    Soient un espace topologique \( X\) et une partie \( A\) de \( X\). Un point \( s\in A \) est un \defe{point isolé}{point isolé} de \( A\) si il existe un voisinage ouvert \( \mO\) de \( s\) dans \( X\) tel que \( A\cap\mO=\{ s \}\).
\end{definition}

\begin{lemma}       \label{LEMooILNCooOFZaTe}
    L'adhérence de \( A\) est l'ensemble des points adhérents :
    \begin{equation}
        \Adh(A)=\bar A.
    \end{equation}
    Par ailleurs, on a le lien
    \begin{equation}
      (\mathring A)^c = \widebar{A^c}.
    \end{equation}
\end{lemma}

\begin{proof}
    Commençons par prouver la dernière égalité d'ensembles. On a les équivalences entre les éléments suivants, pour tout $x \in X$:
    \begin{itemize}
    \item $x$ n'est pas dans $\mathring A$;
    \item il n'y a aucun ouvert contenant $x$ et inclus à $A$;
    \item tout ouvert contenant $x$ a une intersection non-vide avec $A^c$;
    \item $x$ est dans $\widebar{A^c}$.
    \end{itemize}
    Nous allons à présent montrer l'égalité d'ensembles \( \Adh(A)=\bar A \) en prouvant la double inclusion per contraposée.
    \begin{subproof}
        \item[Si \( x\in \bar A\) alors \( x\in\Adh(A)\)]
            Si \( x\) n'est pas dans \( \bar A\) alors nous avons un fermé \( F\) contenant \( A\) et pas \( x\). Le complémentaire \( F^c\) est un ouvert qui contient \( x\) et dont l'intersection avec \( A\) est vide. Donc \( x\) n'est pas dans \( \Adh(A)\).

        \item[Si \( x\in\bar A\) alors \( x\in \Adh(A)\)]

            Si \( x\) n'est pas dans \( \Adh(A)\) alors il existe un ouvert \( \mO\) contenant \( x\) et n'intersectant pas \( A\). Le complémentaire \( \mO^c\) est un fermé qui contient \( A\) et qui ne contient pas \( x\).

            Vu que \( \bar A\) est l'intersection de tous les fermés contenant \( A\), nous avons \( \bar A\subset\mO^c\) et donc \( x\) n'est pas dans \( \bar A\).
    \end{subproof}
\end{proof}

\begin{remark}\label{RemAdhFerme}
  Comme corollaire du lemme précédent, on obtient que pour \( A \subset X \), l'ensemble \( \bar A \) est fermé: c'est en effet le complémentaire d'un ouvert, précisément l'intérieur de \( A^c \).
\end{remark}

\begin{definition}\label{DefEnsembleDense}
  Soit \( X \) un espace topologique. Un sous-ensemble \( A \) de \( X \) est \defe{dense}{dense} dans \( X \) si \( \bar A = X\). 
\end{definition}



%---------------------------------------------------------------------------------------------------------------------------
\subsection{Suites et convergence}
%---------------------------------------------------------------------------------------------------------------------------

\begin{normaltext}
    À propos de notations. La pire notation possible pour une suite est \( (a_n)_n\). What on the f*** vient faire le second indice \( n\) ? Il peut être raisonnable d'écrire \( (a_n)_{n\in I}\) lorsqu'on veut dire dans quel ensemble se déplace \( n\). Si nous parlons de \emph{suite}, il faut une sérieuse raison de prendre autre chose que \( \eN\) comme ensemble d'indices.

    Une suite étant une fonction, de la même façon qu'on ne devrait pas dire «la fonction \( f(x)\)», mais «la fonction \( f\)» ou «la fonction \( x\mapsto f(x)\)», nous devrions simplement écrire \( a\) pour désigner la suite dont les éléments sont \( a_n\). 

    Par conséquent, il est parfaitement légal, et même conseillé, d'écrire «\( a+b\)» pour la somme des suites \( a\) et \( b\). Et il est tout aussi légal d'écrire «\( \lim a\)» au lieu de \( \lim_{n\to \infty} a_n\).

    Le hic est que nous écrivons souvent \( x\) la limite de la suite \( n\mapsto x_n\). Dans ce cas, nous sommes évidemment obligé d'écrire l'indice \( n\) pour parler de la suite.

    Tout cela pour dire qu'il faut être souple avec les notations.
\end{normaltext}

Dès que nous avons une topologie nous avons une notion de convergence.
\begin{definition}[Convergence de suite] \label{DefXSnbhZX}
    Une suite $(x_n)$ d'éléments de $E$ \defe{converge}{convergence!de suite} vers un élément $y$ de $E$ si pour tout ouvert $\mO$ contenant $y$, il existe un $K\in \eN$ tel que $k>K$ implique $x_k\in\mO$.
\end{definition}
\index{limite!de suite!espace topologique}

\begin{proposition}[\cite{MonCerveau}]      \label{PROPooBBNSooCjrtRb}
    Une suite contenue dans un fermé ne peut converger que vers un élément de ce fermé.
\end{proposition}

\begin{proof}
    Soient un espace topologique \( X\) et un fermé \( F\) dans \( X\). Nous supposons que la suite \( (x_k)\) soit contenue dans \( F\). Nous allons prouver qu'aucun élément de \( F^c\) ne peut être limite.

    Soit \( a \in F^c\). Vu que le complémentaire de \( F\) est un ouvert, et vu le théorème \ref{ThoPartieOUvpartouv}, il existe un ouvert \( \mO_a\) contenant \( a\), et contenu dans \( F^c\). Le voisinage \( \mO_a\) de \( a\) ne contient donc aucun élément de la suite \( (x_k)\), qui ne peut donc pas converger vers \( a\).
\end{proof}

\begin{corollary}\label{CorLimAbarA}
  Soit \( A \) un sous-ensemble d'un espace topologique \(X \). Toute suite d'éléments de \(A \) qui converge, admet pour limite un élément de \( \bar A \).
\end{corollary}
\begin{proof}
  Une fois la suite \( (x_n) \) fixée, il suffit de remarquer que tous les \( x_n \) sont dans \( \bar A \), et puis d'appliquer la proposition~\ref{PROPooBBNSooCjrtRb}. 
\end{proof}

%---------------------------------------------------------------------------------------------------------------------------
\subsection{Pour des limites uniques : séparabilité}
%---------------------------------------------------------------------------------------------------------------------------

Notons que l'on a parlé d'\emph{une} limite de suite jusqu'à présent: en effet, s'il existe deux éléments distincts $x$ et $y$ tels que tout ouvert contenant $x$ contient $y$, alors la définition \ref{DefXSnbhZX} dit que toute suite convergeant vers $x$ converge aussi vers $y$\dots


\begin{example} \label{EXooSHKAooZQEVLB}
    Oui, il y a moyen de converger vers plusieurs points distincts si l'espace n'est pas super cool. Nous pouvons par exemple\cite{EJVQuas} considérer la droite réelle munie de sa topologie usuelle et y ajouter un point $0'$ (qui clone le réel $0$) dont les voisinages sont les voisinages de $0$ dans lesquels nous remplaçons $0$ par $0'$. Dans cet espace, la suite $(1/n)$ converge à la fois vers $0$ et $0'$.

    En fait, on «voit» le problème: on ne peut pas distinguer d'un point de vue topologique le $0$ et le $0'$.
\end{example}

Pour assurer l'unicité de la limite d'une suite, ou plus généralement d'une fonction en un point (voir la proposition~\ref{PropFObayrf}), on pose la définition suivante.
\begin{definition}[Espace topologique séparé]  \label{DefYFmfjjm}\label{DefWEOTrVl}
    Si deux points distincts admettent toujours deux voisinages disjoints\footnote{Définition~\ref{DefEnsemblesDisjoints}.}, nous disons que l'espace est \defe{séparé}{espace!séparé} ou \defe{Hausdorff}{Hausdorff}.
\end{definition}


Attention, cette notion est à ne pas confondre avec :
\begin{definition}[Espace topologique séparable]  \label{DefUADooqilFK}
    Un espace topologique est \defe{séparable}{séparable!espace topologique} s'il possède une partie dénombrable\footnote{Définition~\ref{DefEnsembleDenombrable}.} dense\footnote{Définition~\ref{DefEnsembleDense}.}.
\end{definition}

\begin{proposition}\label{PropUniciteLimitePourSuites}
  Dans un espace séparé, si une suite converge, alors sa limite est unique.
\end{proposition}
\begin{proof}
  Supposons que la suite \( (x_k)\) converge vers deux éléments distincts \( x \) et \( y \). L'espace étant séparé, il existe deux ouverts \( \mO_x \) et \( \mO_y \), disjoints, contenant respectivement \( x \) et \( y \). La suite convergeant à la fois vers \( x \) et \( y \), il existe \( k_x \) et \( k_y \), tels que, si \( k \geq \max\{k_x, k_y\} \), l'élément  \( x_k \) est (à la fois) dans  \( \mO_x \) et \( \mO_y \). Cela est en contradiction avec le fait que ces deux ensembles sont disjoints.
\end{proof}

\begin{normaltext}
  Donc, on pourra parler, avec des espaces séparés, de «la limite d'une suite». On notera \( x_n\to a\), ou \(\lim_{n\to \infty} x_n = a \), pour signifier que la suite \( (x_n) \) converge vers \( a \). 
\end{normaltext}

%----------------------------------------------------------------------------------------------------------------------
\subsection{Compacité}
%----------------------------------------------------------------------------------------------------------------------

\begin{definition}  \label{DefJJVsEqs}
  Une partie $A$ d'un espace topologique est \defe{compacte}{compact} s'il vérifie la propriété de Borel-Lebesgue : pour tout recouvrement de $A$ par des ouverts (c'est-à-dire une collection d'ouverts dont la réunion contient $A$) on peut tirer un recouvrement fini.
\end{definition}
\begin{remark}
    Certaines sources (dont \wikipedia{fr}{Compacité_(mathématiques)}{wikipédia}) disent que pour être compact il faut aussi être séparé\footnote{Définition~\ref{DefWEOTrVl}.}. Pour ces sources, un espace qui ne vérifie que la propriété de Borel-Lebesgue est alors dit \defe{quasi-compact}{quasi-compact}\index{compact!quasi}.
\end{remark}

\begin{definition}
    Une partie d'un espace topologique est \defe{relativement compact}{compact!relativement}\index{relativement!compact} si son adhérence est compacte.
\end{definition}

\begin{definition}  \label{DefEIBYooAWoESf}
    Un espace topologique est \defe{localement compact}{compact!localement} si tout élément possède un voisinage compact.
\end{definition}

\begin{definition}[Séquentiellement compact]
    Nous disons qu'un espace topologique est \defe{séquentiellement compact}{compact!séquentiellement} si toute suite admet une sous-suite convergente.
\end{definition}

\begin{probleme}
  Étudier les implications - équivalences entre ces définitions.
\end{probleme}

\begin{definition}      \label{DefFCGBooLpnSAK}
    Un espace topologique est \defe{dénombrable à l'infini}{dénombrable!à l'infini} s'il est réunion dénombrable de compacts.
\end{definition}

%---------------------------------------------------------------------------------------------------------------------------
\subsection{Base d'une topologie}
%---------------------------------------------------------------------------------------------------------------------------

\begin{definition}[Base de topologie]   \label{DefQELfbBEyiB}
    Une famille \( \mB\) d'ouverts de \( X\) est une \defe{base de la topologie}{base!de topologie} de \( X\) si pour tout \( x\in X\) et pour tout voisinage \( V\) de \( x\), il existe \( A\in \mB\) tel que \( x\in A\subset V\).
\end{definition}

\begin{proposition} \label{PropMMKBjgY}
    Si \( \mB\) est une base de la topologie de \( X\) alors tout ouvert de \( X\) est une union d'éléments de \( \mB\).
\end{proposition}

\begin{proof}
    Soit \( \mO\) un ouvert de \( X\); pour chaque \( x\in\mO\) nous considérons un ouvert \( U(x)\) tel que \( x\in U(x)\subset \mO\) (possible par le théorème~\ref{ThoPartieOUvpartouv}). Nous prenons alors \( B(x)\in\mB\) tel que
    \begin{equation}
        x\in B(x)\subset U(x)\subset \mO.
    \end{equation}
    Alors nous avons \( \mO=\bigcup_{x\in \mO}B(x)\).
\end{proof}
Notons toutefois que nous sommes loin d'avoir une union dénombrable en général.

%---------------------------------------------------------------------------------------------------------------------------
\subsection{Topologie produit}
%---------------------------------------------------------------------------------------------------------------------------

\begin{definition}[Produit d'espaces topologiques, thème~\ref{THEMEooYRIWooDXZnhX}]      \label{DefIINHooAAjTdY}
    Soient \( X_1\),\ldots, \( X_n\) des espaces topologiques. Leur \defe{produit}{produit!espaces topologiques}\index{topologie!produit} est l'ensemble
    \begin{equation}
        X=\prod_{i=1}^nX_i
    \end{equation}
    muni de la topologie engendrée par les produits \(A_1\times \cdots\times A_n\), avec \( A_i\in X_i \) ouverts de chacun des ensembles.
\end{definition}

\begin{proposition}[\cite{MonCerveau}]      \label{PROPooNRRIooCPesgO}
    La convergence d'une suite pour la topologie de l'espace produit implique la convergence des suites «composante par composante».
\end{proposition}

\begin{proof}
    Pour simplifier les notations, nous allons considérer le produit de deux espaces. Soit donc \( (x_k,y_k)\stackrel{X\times Y}{\longrightarrow}(x,y)\) et des ouverts \( \mO_1\) dans \( X\) autour de \( x\) et \( \mO_2\) autour de \( y\) dans \( Y\). La partie \( \mO_1\times \mO_2\) est ouverte dans \( X\times Y\). Donc il existe \( K\) tel que \( k>K\) implique \( (x_k,y_k)\in \mO_1\times \mO_2\).

    Nous avons prouvé que pour tout ouvert \( \mO_1\) autour de \( x\) il existe \( K\) tel que \( k>K\) implique \( x_k\in \mO_1\). Donc \( x_k\stackrel{X}{\longrightarrow}x\). Idem pour \( y\).
\end{proof}
% TODO : faire la réciproque.

%---------------------------------------------------------------------------------------------------------------------------
\subsection{Limite et continuité d'une fonction}
%---------------------------------------------------------------------------------------------------------------------------

\begin{definition}[Limite d'une fonction, thème~\ref{THEMEooGVCCooHBrNNd}]\label{DefYNVoWBx}
    Soient \( X\) et \( Y\) des espaces topologiques, \( A\subset X\) et \( a\in\bar A\). Soit encore une fonction \( f\colon A\to Y\). L'élément \( y\in Y\) est une \defe{limite}{limite!d'une fonction} de \( f\) en \( a\) si pour tout voisinage \( V\) de \( y\) (pour la topologie de \( Y\)), il existe un voisinage \( W\) de \( a\) dans \( X\) tel que
    \begin{equation}
        f\big( W\cap A\setminus\{a\} \big)\subset V.
    \end{equation}

    Si un tel élément est unique\footnote{Rappelons que ce n'est pas toujours le cas, mais que ça l'est si l'espace topologique est séparé -- définition~\ref{DefYFmfjjm}.}, alors nous disons que cet élément est la \defe{limite}{limite!d'une fonction} de \( f\) et nous notons
    \begin{equation}
        \lim_{x\to a} f(x)=y.
    \end{equation}
\end{definition}

\begin{remark}
    Nous ne saurions trop insister sur le fait que la valeur de \( f\) en \( a\) n'intervient pas dans la définition de la limite de \( f\) en \( a\). Il n'est même pas pas nécessaire que \( f\) soit définie en \( a\) pour que l'on puisse parler de limite de \( f\) en \( a\). Par exemple nous avons
    \begin{equation}
        \lim_{x\to 1} \frac{ x^2-1 }{ x-1 }=2,
    \end{equation}
    alors que la fonction n'est pas définie en \( x=1\).

    Plus généralement, un peu par principe, toutes les fois que la notion de limite apporte une information, le point où l'on prend la limite est spécial. Sinon on ne calculerait pas la limite, mais on regarderait directement la valeur de la fonction. Cela est typiquement le cas lorsque nous verrons les dérivées. En effet, regardons (en faisans du semblant d'anticiper) la définition  \eqref{DEFooOYFZooFWmcAB}. Dans la formule
    \begin{equation}
        f'(a)=\lim_{x\to a} \frac{ f(x)-f(a) }{ x-a },
    \end{equation}
    la fonction sur laquelle nous prenons la limite n'est \emph{jamais} définie en \( x=a\).

    Cela est intimement liée à ce que je raconte dans~\ref{SUBSECooVHKCooYRFgrb}.
\end{remark}

\begin{proposition}[Unicité de la limite pour un espace séparé]\label{PropFObayrf}
    Soient \( X\) un espace topologique, \( A\) une partie de \( X\) et \( Y\) un espace topologique séparé\footnote{Définition~\ref{DefYFmfjjm}.}. Nous considérons une fonction \( f\colon A\to Y\). Si \( a\in\bar A\), alors \( f\) admet au plus une limite en \( a\).
\end{proposition}
\index{limite!unicité}

\begin{proof}
    Soient \( y\) et \( y'\) des limites de \( f\) en \( a\), ainsi que des voisinages \( V\) et \( V'\) de \( y\) et \( y'\). Nous prenons également les voisinages \( W\) et \( W'\) correspondants :
    \begin{subequations}
        \begin{numcases}{}
            f(W\cap A)\subset V\\
            f(W'\cap A)\subset V'.
        \end{numcases}
    \end{subequations}
    Quitte à prendre des sous-ensembles nous pouvons supposer que \( W\) et \( W'\) sont ouverts. Il s'ensuit alors que:
    \begin{itemize}
      \item l'ensemble \( W\cap W'\) est un ouvert contenant \( a\) et intersecte donc \( A\);
      \item l'ensemble \( (W\cap W')\cap A\) est donc non vide;
      \item et donc, \( f(W\cap W'\cap A) \) est aussi non vide.
    \end{itemize}
    Mais
    \begin{equation}
            f(W\cap W'\cap A)\subset f(W\cap A)\subset V,
    \end{equation}
    et
    \begin{equation}
            f(W\cap W'\cap A)\subset f(W'\cap A)\subset V',
    \end{equation}
    d'où \( V \) et \( V'\) ont une intersection. Puisque ces ensembles sont arbitraires, nous avons prouvé que tout voisinage de \( y\) et tout voisinage de \( y'\) ont une intersection non vide; étant donné que \( Y\) est séparé, nous devons avoir \( y=y'\).
\end{proof}

La définition suivante est \emph{la} définition de la continuité dans tous les cas.
\begin{definition}[Fonction continue]\label{DefOLNtrxB}
    Une fonction \( f\colon X\to Y\) est \defe{continue au point}{continue!fonction!en un point} \( a\in X\) si \( f(a)\) est une limite\footnote{Définition~\ref{DefYNVoWBx}.} de \( f\) en \( a\).

    Une fonction \( f\colon X\to Y\) est \defe{continue}{continue!fonction entre espaces topologiques} si pour tout ouvert \( \mO\) de \( Y\), l'ensemble \( f^{-1}(\mO)\) est ouvert dans \( X\).
\end{definition}
La proposition~\ref{PropQZRNpMn} donnera des détails sur ce qu'il se passe lorsque l'espace est métrique.

\begin{theorem} \label{ThoESCaraB}
    Une fonction \( f\colon X\to Y\) est une fonction continue si et seulement si elle est continue en chacun des points de \( X\).
\end{theorem}

\begin{proof}
    En deux parties.
    \begin{subproof}
    \item[Sens direct]
        Nous supposons que \( f\) est une fonction continue. Soient \( a\in X\) et \( V\) un voisinage de \( f(a)\). Nous considérons \( \mO\), un voisinage ouvert de \( f(a)\) contenu dans \( V\); l'ensemble \( f^{-1}(\mO)\) est alors un ouvert contenant \( a\), et l'image de \( f^{-1}(\mO)\) par \( f\) est bien entendu contenue dans \( V\).

    \item[Sens inverse]

        Soit \( \mO\) un ouvert de \( Y\). Pour prouver que \( f^{-1}(\mO)\) est un ouvert de \( X\), nous allons considérer un élément \( a\in f^{-1}(\mO)\) et montrer qu'il existe un voisinage ouvert de \( a\) contenu dans \( f^{-1}(\mO)\); le théorème~\ref{ThoPartieOUvpartouv} nous assurera alors que \( f^{-1}(\mO)\) est ouvert.

        L'ensemble \( \mO\) est un voisinage ouvert de \( f(a)\) parce que \( a\) a été choisi dans \( f^{-1}(\mO)\). Donc la continuité de \( f\) en \( a\) nous assure qu'il existe un voisinage \( W\) de \( a\) tel que \( f(W)\subset\mO\). En prenant un ouvert contenant \( a\) à l'intérieur de \( W\) nous avons un voisinage ouvert de \( a\) contenu dans \( f^{-1}(\mO)\).
    \end{subproof}
\end{proof}

\begin{remark}
    À cause de l'éventuelle non unicité de la limite, deux fonctions continues et égales sur un sous-ensemble dense ne sont pas spécialement égales. Ce sera vrai sur les espaces métriques et plus généralement pour les espaces séparés. Voir l'exemple~\ref{EXooSHKAooZQEVLB} et la proposition~\ref{PropFObayrf}.
\end{remark}

\begin{definition}
    Un \defe{homéomorphisme}{homéomorphisme} est une application bijective continue entre deux espaces topologiques dont la réciproque est continue. Deux espaces topologiques $X$ et $Y$ pour lesquels il existe un homéomorphisme entre $X$ et $Y$, sont dits \defe{isomorphes}{isomorphisme!d'espaces topologiques}.
\end{definition}


%+++++++++++++++++++++++++++++++++++++++++++++++++++++++++++++++++++++++++++++++++++++++++++++++++++++++++++++++++++++++++++
\section{Topologie, distance et structure d'espace vectoriel}
%+++++++++++++++++++++++++++++++++++++++++++++++++++++++++++++++++++++++++++++++++++++++++++++++++++++++++++++++++++++++++++

%---------------------------------------------------------------------------------------------------------------------------
\subsection{Topologie métrique}
%---------------------------------------------------------------------------------------------------------------------------

\begin{definition}  \label{DefMVNVFsX}
    Si $E$ est un ensemble, une \defe{distance}{distance} sur $E$ est une application $d\colon E\times E\to \eR$ telle que pour tout $x,y\in E$,
    \begin{enumerate}

    \item
    $d(x,y)\geq 0$

    \item
    $d(x,y)=0$ si et seulement si $x=y$,

    \item
    $d(x,y)=d(y,x)$

    \item
    $d(x,y)\leq d(x,z)+d(z,y)$.

    \end{enumerate}
    La dernière condition est l'\defe{inégalité triangulaire}{inégalité!triangulaire}.

    Un couple $(E,d)$ formé d'un ensemble et d'une distance est un \defe{espace métrique}{espace!métrique}.
\end{definition}

La définition-théorème suivante donne une topologie sur les espaces métriques en partant des boules.

\begin{theoremDef}     \label{ThoORdLYUu}
    Soit \( (E,d)\) un espace métrique. Nous définissons les \defe{boules ouvertes}{boule!ouverte} par
    \begin{equation}        \label{EQooYCWSooIhibvd}
        B(x,r)=\{ y\in E\tq d(x,y)<r \}.
    \end{equation}
    pout tout \( x\in E\) et \( r>0\).
    Alors en posant
    \begin{equation}        \label{EqGDVVooDZfwSf}
        \mT=\big\{  \mO\subset E  \tq\forall x\in \mO,\exists r>0\tq B(x,r)\subset \mO \big\}
    \end{equation}
    nous définissons une topologie sur \( E\).

    Cette topologie sur \( E\) est la \defe{topologie métrique}{topologie!métrique} de \( (E,d)\). En présence d'une distance, sauf mention explicite du contraire, c'est toujours cette topologie-là que nous utiliserons.
\end{theoremDef}

\begin{proof}
    D'abord \( \emptyset\in\mT\) parce que tout élément de l'ensemble vide \ldots heu \ldots enfin parce que d'accord hein\footnote{Pour qui ne seraient pas d'accord, allez ajouter \( \emptyset\) dans la définition des ouverts et puis c'est tout.}. Ensuite si \( (A_i)_{i\in I}\) sont des éléments de \( \mT\) et si \( x\in\bigcup_{i\in I}A_i\) alors il existe \( k\in I\) tel que \( x\in A_k\). Par hypothèse il existe une boule \( B(x,r)\subset A_k\subset\bigcup_{i\in I}A_i\).

    Enfin si \( (A_i)_{i\in\{ 1,\ldots, n \}}\) sont des éléments de \( \mT\) alors pour tout \( i\) il existe \( r_i>0\) tel que \( B(x,r_i)\subset A_i\). En prenant \( r=\min\{ r_i \}_{i=1,\ldots, n}\) nous avons $B(x,r)\subset\bigcap_{i=1}^nA_i.$
\end{proof}

\begin{remark}  \label{RemQDRooKnwKk}
    Quatre remarques à propos de cette définition.
    \begin{enumerate}
    \item
      Cette définition est faite exprès pour respecter le théorème~\ref{ThoPartieOUvpartouv}. Même si, a priori, on aurait dû utiliser la topologie engendrée faite à l'exemple \ref{DefTopologieEngendree}\dots\ mais on peut montrer que les deux topologies sont les mêmes.
    \item      \label{ITEMooUIHJooXAFaIz}
      Par construction, les boules ouvertes sont une base de la topologie (définition~\ref{DefQELfbBEyiB}) des espaces métriques.
    \item       \label{ITEMooUIHJooXAFaJa}
      Si \( V\) est un voisinage de \( x\), alors il existe \( r\) tel que \( B(x,r)\subset V\).
    \item
      Tout espace métrique est séparé. En effet, si deux éléments \( x \) et \( y \) sont distincts, alors en posant \( r = d(x , y) / 3 > 0 \), les boules \( B(x,r) \) et \( B(y,r)\) sont disjointes. Très pratique pour les limites : elles sont uniques, grâce aux propositions~\ref{PropUniciteLimitePourSuites} et \ref{PropFObayrf}!
    \end{enumerate}
\end{remark}

\begin{normaltext}
    Si vous avez un peu de temps, vous pouvez vérifier que si \( \eK\) est un corps totalement ordonné, alors avec toutes les définitions de~\ref{DefKCGBooLRNdJf}, en posant \( d(x,y)=| x-y |\) nous avons une distance sur \( \eK\).

    De plus, les boules définies en~\ref{DefKCGBooLRNdJf} sont alors les mêmes que celles définies en \eqref{EQooYCWSooIhibvd}, ce qui donne à tout corps totalement ordonné une structure d'espace topologique.
\end{normaltext}

\begin{definition}\label{DefEnsembleBorne}
  Soit \( (X, d) \) un espace métrique. Un sous-ensemble $A \subset X$ est \defe{borné}{borné} s'il existe une boule de $X$ contenant $A$.
\end{definition}

\begin{proposition}
  Toute réunion finie d'ensembles bornés est un ensemble borné. Toute partie d'un ensemble borné est un ensemble borné.
\end{proposition}

\begin{definition}
    Si \( (X,d_X)\) et \( (Y,d_Y)\) sont des espaces métriques, une \defe{isométrie}{isométrie d'espaces métriques} est une application bijective \( f\colon X\to Y\) telle que pour tout \( x,y\in X\) nous ayons
    \begin{equation}        \label{EQooVUOXooKJntMN}
        d_Y\big( f(x),f(y) \big)=d_X(x,y).
    \end{equation}
\end{definition}

\begin{remark}
    Une application vérifiant \eqref{EQooVUOXooKJntMN} est automatiquement injective. En pratique, il ne faut donc vérifier que la surjectivité.
\end{remark}

\begin{example}[Manque de surjectivité]
    Si \( X=\mathopen[ 0 , \infty \mathclose[\) et \( f(x)=x+1\) alors \( f\) vérifie \eqref{EQooVUOXooKJntMN} pour la distance \( d(x,y)=| x-y |\), mais n'est pas surjective.
\end{example}

\begin{propositionDef}[Groupe des isométries]
    Si \( (X,d)\) est un espace métrique,
    \begin{enumerate}
        \item
            l'ensemble des isométries de \( X\), noté \( \Isom(X)\)\nomenclature[Y]{$\Isom(X)$}{Le groupe des isométries de \( X\)} est un groupe pour la composition\index{isométrie!groupe}\index{groupe!des isométries!espace métrique}.
        \item
            Ce groupe agit fidèlement\footnote{Si vous ne savez pas ce que c'est, alors vous avez zappé la définition~\ref{DefuyYJRh}.} sur \( X\).
    \end{enumerate}
\end{propositionDef}
\begin{proposition}\label{PropLYMgVMJ}
    Une isométrie entre deux espaces métriques est continue.
\end{proposition}

\begin{proof}
    Soient \( f\colon X\to Y\) une application isométrique et \( \mO\) un ouvert de \( Y\). Soit \( a\in f^{-1}(\mO)\); si \( d(a,b)<r\), alors \( d\big( f(a),f(b) \big)<r\) et donc \( b\in f^{-1}\big( B(f(a),r) \big)\). Donc autour de chaque point de \( f^{-1}(\mO)\) nous pouvons trouver une boule ouverte contenue dans \( f^{-1}(\mO)\), ce qui prouve que \( f^{-1}(\mO)\) est ouvert.
\end{proof}

\begin{example}
    Si \( X\) est un ensemble, nous pouvons écrire la \defe{distance discrète}{distance discrète} :
    \begin{equation}
        d(x,y)=\begin{cases}
            0    &   \text{si } x=y\\
            1    &    \text{si } x\neq y\text{.}
        \end{cases}
    \end{equation}
    La topologie résultante est la topologie discrète, côtoyée dans l'exemple~\ref{DefTopologieDiscrete}\footnote{Vérifiez-le tout de même!}.

    Pour cette métrique, le groupe des isométries est le groupe symétrique de \( X\), c'est à dire le groupe de toutes les bijections de \( X\) sur lui-même.
\end{example}

%---------------------------------------------------------------------------------------------------------------------------
\subsection{Caractérisations séquentielles}
%---------------------------------------------------------------------------------------------------------------------------

\begin{definition}  \label{DefENioICV}
    Si \( X\) et \( Y \) sont deux espaces topologiques, une fonction \( f\colon X\to \eR\) est \defe{séquentiellement continue}{continuité!séquentielle} en un point \( a\) si pour toute suite convergente \( x_n\to a\) dans \( X\) nous avons \( f(x_n)\to f(x)\) dans \( Y\).
\end{definition}

%TODO : il y a un contre-exemple à faire à la page http://www.les-mathematiques.net/phorum/read.php?14,787368,787582

\begin{proposition}[Caractérisation séquentielle de la limite\cite{MonCerveau}]     \label{PROPooJYOOooZWocoq}
    Soient deux espaces topologiques \( X\) et \( Y\) ainsi qu'une fonction \( f\colon X\to Y\). Soit \( a\in X\) et \( \ell\in Y\). Si
    \begin{equation}\label{EqLimooJYOOooZWocoqG}
        \lim_{x\to a} f(x)=\ell,
    \end{equation}
    alors, pour toute suite \( (x_k) \) telle que \( x_k \to a \), on a
    \begin{equation}\label{EqLimooJYOOooZWocoqS}
        \lim f(x_k)=\ell.
    \end{equation}
    
    De plus, si \( X\) et \( Y\) sont des espaces \emph{métriques}, alors \ref{EqLimooJYOOooZWocoqG} est équivalent à \ref{EqLimooJYOOooZWocoqS} pour toute suite \( (x_k) \) convergeant vers \( a\), et une des deux limites existe si et seulement si l'autre existe.
\end{proposition}

\begin{proof}
    En deux parties.
    \begin{subproof}
    \item[Sens direct, espaces topologiques]
        Nous considérons une suite \( (x_k)\) qui converge vers \( a\) dans \( X\). Soient \( V\) un voisinage de \( \ell \) et \( W\) un voisinage de \( a\) tels que \( f(W)\subset V\) (définition~\ref{DefYNVoWBx} de la continuité en un point). Par la convergence \( a_k\to a\),  il existe \( N\) tel que pour tout \( k>N\), \( a_k\in W\), et donc tel que \( f(a_k)\in V\), ce qui donne la continuité séquentielle de \( f\).

    \item[Réciproque]
        Pour la réciproque, nous passons par la contraposée. C'est à dire que nous supposons que \( \ell\) n'est pas une limite de \( f\) pour \( x\to a\). Il existe un \( \epsilon\) tel que pour tout \( \delta\), il existe un \( x\) vérifiant \( \| x-a \|<\delta\) et \( \| f(x)-\ell \|>\epsilon\).

        Nous construisons à présent une suite de la manière suivante. Pour \( \delta=\frac{1}{ n }\) nous considérons \( x_n\) tel que \( \| x_n-a \|<\delta\) et \( \| f(x_n)-\ell \|>\epsilon\). Cette suite converge vers \( a\), mais la suite \( f(x_n)\) ne converge manifestement pas vers \( \ell\) : elle ne rentre jamais dans la boule \( B(\ell,\epsilon)\).
    \end{subproof}
\end{proof}

Une fonction continue est séquentiellement continue. Dans les espaces métriques la proposition suivante montre que la réciproque est également vraie et la continuité est équivalente à la continuité séquentielle. Cela n'est cependant pas vrai pour n'importe quel espace topologique.

\begin{corollary}[Caractérisation séquentielle de la continuité en un point\cite{MonCerveau}]		\label{PropFnContParSuite}
    Le lien entre continuité et continuité séquentielle.

    \begin{enumerate}
        \item
    Une application continue est séquentiellement continue (quels que soient les espaces de départ et d'arrivée).

\item\label{ItemWJHIooMdugfu}

    Si \( X\) et \( Y\) sont des espaces métriques, alors une fonction \( f\colon X\to Y\) est continue en un point si et seulement si elle est séquentiellement continue en ce point.
    \end{enumerate}
\end{corollary}

\begin{proof}
    Il suffit de reprendre la preuve précédente en remplaçant \( \ell \) par \( f(a) \).
    \begin{subproof}
    \item[Sens direct]
      Recopie telle quelle, modulo la modification.
    \item[Sens réciproque, espaces métriques]
      Paraphrasons la preuve précédente. Nous supposons que \( X\) et \( Y\) sont métriques. Si \( f\) n'est pas continue en \( a\), il existe \( \epsilon>0\) tel que pour tout \( \delta>0\), il existe \( x\) tel que \( \| x-a \|\leq\delta\) et \( \| f(x)-f(a) \|>\epsilon\). Nous considérons un tel \( \epsilon\) et pour chaque \( n\geq1\in \eN\) nous considérons un \( x_n\) correspondant à \( \delta=\frac{1}{ n }\). Cela nous donne une suite \( x_n\to a\) dans \( X\) mais \( \| f(x_n) -f(a)\|\) reste plus grand que \( \epsilon\). Cela montre que \( f\) n'est pas non plus séquentiellement continue.
    \end{subproof}
\end{proof}

Les espaces métriques ont une propriété importante que la \wikipedia{fr}{Espace_séquentiel}{fermeture séquentielle} est équivalente à la fermeture.
\begin{proposition}[Caractérisation séquentielle d'un fermé]    \label{PropLFBXIjt}
    Soient \( X\) un espace métrique et \( F\subset X\). L'ensemble \( F\) est fermé si et seulement si toute suite contenue dans \( F\) et convergeant dans \( X\) converge vers un élément de \( F\).
\end{proposition}
\index{fermeture séquentielle}
\index{séquentiellement fermé}

\begin{proof}
   Une suite contenue dans un fermé ne peut converger que vers un élément de ce fermé: c'était la proposition \ref{PROPooBBNSooCjrtRb}. Le point le plus important est donc l'autre sens: si toute suite d'éléments de \( F \) converge dans \( F \) alors \( F \) est fermé.
    
   Par contraposée, supposons que \( X\setminus F\) ne soit pas ouvert. Alors il existe \( x\in X\setminus F\) pour lequel tout voisinage intersecte \( F\). En prenant \( x_k\in B(x,\frac{1}{ k })\), nous construisons une suite contenue dans \( F\), convergeant vers \( x\) qui n'est pas dans \( F \).
\end{proof}


\begin{lemma}		\label{LemLimAbarA}
	Soit $(x_n)$ une suite convergente contenue dans un ensemble $A\subset V$. Alors la limite $x_n$ appartient à $\bar A$.
\end{lemma}
Ce lemme est précisément la version «espace vectoriel normé» du corollaire \ref{CorLimAbarA}; mais, donnons-en une preuve tout de même.
\begin{proof}
	Supposons que nous ayons une partie $A$ de $V$, et une suite $(x_n)$ dont la limite $\ell$ se trouve hors de $\bar A$. Dans ce cas, il existe un $r>0$ tel que\footnote{Une autre manière de dire la même chose : si $\ell\notin\bar A$, alors $d(\ell,A)>0$.} $B(\ell,r)\cap A=\emptyset$. Si tous les éléments $x_n$ de la suite sont dans $A$, il n'y en a donc aucun tel que $d(x_n,\ell)=\| x_n-\ell \|<r$. Cela contredit la notion de convergence $x_n\to \ell$.
\end{proof}

\begin{corollary}		\label{CorAdhEstLim}
	Soit $a$ un point de l'adhérence d'une partie $A$ de $V$. Alors il existe une suite d'éléments dans $A$ qui converge vers $a$.
\end{corollary}

\begin{proof}
	Si $a\in A$, alors nous pouvons prendre la suite constante $x_n=a$. Si $a$ n'est pas dans $A$, alors $a$ est dans $\partial A$, et pour tout $n$, il existe un point de $A$ dans la boule $B(a,\frac{1}{ n })$. Si nous nommons $x_n$ ce point, la suite ainsi construite est une suite contenue dans $A$ et qui converge vers $a$ (ce dernier point est laissé à la sagacité du lecteur ou de la lectrice).
\end{proof}

En termes savants, ce corollaire signifie que la fermeture $\bar A$ est composé de $A$ plus de toutes les limites de toutes les suites contenues dans $A$.

\begin{proposition} \label{PropXIAQSXr}[Caractérisation séquentielle de la continuité\cite{MonCerveau}]	
    Soient \( X\) et \( Y\) deux espaces métriques. Soit \( f\colon X\to Y\) une application séquentiellement continue (en tout \( a \in X \)). Alors \( f\) est continue.
\end{proposition}

\begin{proof}
    Soit \( \mO\) un ouvert de \( Y\); nous allons voir que le complémentaire de \( f^{-1}(\mO)\) est fermé dans \( E\). Pour cela nous considérons une suite convergente \( x_k\stackrel{E}{\longrightarrow} x\) avec \( x_k\in\complement f^{-1}(\mO)\) pour tout \( k\). Nous allons montrer que \( x\in \complement f^{-1}(\mO)\) et la caractérisation séquentielle\footnote{Proposition~\ref{PropLFBXIjt}.} de la fermeture conclura que \( \complement f^{-1}(\mO)\) est fermé.

    Pour tout \( k\), nous avons \( f(x_k)\in\complement \mO\), mais \( \mO\) est ouvert et \( f(x_k)\stackrel{Y}{\longrightarrow}f(x)\) parce que \( f\) est séquentiellement continue. Par conséquent \( f(x)\in\complement \mO\) et \( x\in\complement f^{-1}(\mO)\).
\end{proof}

%TODO : il y a ici trois théorèmes sur la continuité séquentielle. Il faut sans doute les fusionner.

\begin{proposition} \label{PropCJGIooZNpnGF}
    Si \( X\) et \( Y\) sont deux espaces métriques et \( f,g\colon X\to Y\) sont deux fonctions continues égales sur une partie dense de \( X\) alors \( f=g\).
\end{proposition}
\index{fonction!continue!égales}

\begin{proof}
    Les fonctions \( f\) et \( g\) sont séquentiellement continues (proposition~\ref{PropFnContParSuite}\ref{ItemWJHIooMdugfu}). Soient \( A\) un ensemble dense dans \( X\) sur lequel \( f\) et \( g\) sont égales, et \( x\notin A\). Vu que \( A\) est dense, il existe une suite \( a_n\) dans \( A\) telle que \( a_n\to x\). La séquentielle continuité de \( f\) et \( g\) donnent
    \begin{subequations}
        \begin{align}
            f(a_n)\to f(x)\\
            g(a_n)\to g(x),
        \end{align}
    \end{subequations}
    mais pour tout \( n\), \( f(a_n)=g(a_n)\). Par unicité de la limite\footnote{Proposition~\ref{PropFObayrf}.} dans \( Y\), \( f(x)=g(x)\).
\end{proof}

%---------------------------------------------------------------------------------------------------------------------------
\subsection{Espace vectoriel topologique}
%---------------------------------------------------------------------------------------------------------------------------

\begin{definition}\label{DefEVTopologique}
  Un espace vectoriel \( V\) sur le corps \( \eK\) muni d'une topologie est un \defe{espace vectoriel topologique}{espace vectoriel!topologique} si
    \begin{enumerate}
        \item
            la somme de deux vecteurs est une application continue\footnote{Naturellement, l'espace \(V \times V \) est muni de la topologie produit.} \( V\times V\to V \); et
        \item
            la multiplication par un scalaire est une application continue\footnote{Naturellement, l'espace \(\eK \times V \) est muni (lui aussi) de la topologie produit.} \( \eK\times V\to V\).
    \end{enumerate}
\end{definition}
On le redit quand même: le corps\footnote{Définition~\ref{DefTMNooKXHUd}} lui-même doit avoir sa topologie. Dans la grande majorité des cas, ce corps est \( \eR\) ou \( \eC\) muni de la topologie usuelle.

\begin{definition}      \label{DEFooGTOZooRcvGHg}
    Une distance \( d\) sur un espace vectoriel topologique \( V\) est dite \defe{compatible}{distance!compatible} avec la topologie si la topologie induite\footnote{Définition~\ref{ThoORdLYUu}.} de \( d\) est celle de \( V\).

    Une distance \( d\) sur un espace vectoriel \( V\) est dite \defe{invariante}{distance!invariante} si pour tout \( x,y,u\in V\) nous avons
    \begin{equation}
        d(x+u,y+u)=d(x,y).
    \end{equation}
\end{definition}
Notons que lorsque nous parlons d'une distance compatible avec un espace vectoriel topologique, nous parlons de compatibilité avec la topologie, pas avec la structure vectorielle.

%---------------------------------------------------------------------------------------------------------------------------
\subsection{Norme; espace vectoriel normé}
%---------------------------------------------------------------------------------------------------------------------------
\label{SECooWKJNooKOqpsx}

La valeur absolue est essentielle pour introduire les notions de limite et de continuité pour les fonctions d'une variable. En fait, pour $\eR$ muni de sa topologie usuelle, on a:
\begin{proposition}
  Une fonction $f\colon \eR\to \eR$ est continue au point $a \in \eR$ lorsque pour tout $\varepsilon > 0$, il existe un $\delta > 0$ tel que
  \begin{equation}
    | x-a |\leq\delta \Rightarrow | f(x)-f(a) |\leq \varepsilon.
  \end{equation}
\end{proposition}
La quantité $| x-a |$ donne la «distance» entre $x$ et $a$; la définition de la continuité signifie que pour tout $\varepsilon$, il existe un $\delta$ tel que si $a$ et $x$ sont au plus à la distance $\delta$ l'un de l'autre, alors $f(x)$ et $f(a)$ ne seront éloignés au plus d'une distance $\varepsilon$.

La valeur absolue, dans $\eR$, nous sert donc à mesurer des distances entre les nombres. Les principales propriétés de la valeur absolue sont :
\begin{enumerate}

	\item
		$| x |=0$ implique $x=0$,
	\item
		$| \lambda x |=| \lambda | |x |$,
	\item
		$| x+y |\leq | x |+| y |$

\end{enumerate}
pour tout $x,y\in\eR$ et $\lambda\in\eR$.

Afin de donner une notion de limite pour les fonctions de plusieurs variables, nous devons trouver un moyen de définir les notions de «taille» d'un vecteur et de distance entre deux points de $\eR^n$, avec $n>1$. La notion de «taille» doit satisfaire propriétés analogues à celles de la valeur absolue.

La première notion de «taille» pour un vecteur de $\eR^2$ que nous vient à l'esprit est la longueur du segment entre l'origine et l'extrémité libre du vecteur. Cela peut être calculée à l'aide du théorème de Pythagore :
\begin{equation}
  \textrm{taille de } (a,b) = \sqrt{a^2+b^2}.
\end{equation}
Nous pouvons introduire une notion de distance entre les éléments de $\eR^2$ de façon similaire :
\begin{equation}
	d\big((a_x,a_y),(b_x,b_y)\big)=\sqrt{  (a_x-b_x)^2+(a_y-b_y)^2  }.
\end{equation}
Cette définition a l'air raisonnable; est-elle mathématiquement correcte ? Peut-elle jouer le rôle de la valeur absolue dans $\eR^2$ ? Est-elle la seule définition possibles de «taille» et distance en $\eR^2$ ?

Nous voulons formaliser les notions de «taille» et de distance dans $\eR^n$, et plus généralement dans un espace vectoriel $V$ de dimension finie. Pour cela nous nous inspirons des propriétés de la valeur absolue.

\begin{definition}[\cite{BrunelleMatricielle}, thème~\ref{THEMEooUJVXooZdlmHj}]  \label{DefNorme}	\label{DefEVNetDistance}
    Soit \( E\) un espace vectoriel (pas spécialement de dimension finie) sur le corps \( \eK\) (\( =\eR\) ou \( \eC\)). Une  \defe{norme}{norme} sur $E$ est une application $N\colon E\to \eR^+$ telle que
	\begin{enumerate}
		\item
            \( N(x)=0\) si et seulement si \( x=0\);
		\item\label{ItemDefNormeii}
			$N(\lambda x)=| \lambda |N(x)$ pour tout $\lambda\in\eR$ et $x\in E$;
		\item\label{ItemDefNormeiii}
			$N(x+y)\leq N(x)+N(y)$
	\end{enumerate}
    pour tout $x,y\in E$ et pour tout $\lambda\in\eK$.

    La propriété~\ref{ItemDefNormeiii} est appelée \defe{inégalité triangulaire}{inégalité!triangulaire}.

    Un espace vectoriel muni d'une norme est un \defe{espace vectoriel normé}{espace vectoriel normé}.
\end{definition}
En prenant $\lambda=-1$ dans la propriété~\ref{ItemDefNormeii}, nous trouvons immédiatement que $N(-x)=N(x)$.

\begin{proposition}		\label{PropNmNNm}
	Toute norme $N$ sur l'espace vectoriel $E$ vérifie l'inégalité
	\begin{equation}
		\big| N(x)-N(y) \big|\leq N(x-y)
	\end{equation}
	pour tout $x,y\in E$.
\end{proposition}

\begin{proof}
	Nous avons, en utilisant le point~\ref{ItemDefNormeiii} de la définition~\ref{DefNorme},
	\begin{subequations}
		\begin{align}
			N(x)&=N(x-y+y)\leq N(x-y)+N(y),	\label{subEqNNNxxyyya}\\
			N(y)&=N(y-x+x)\leq N(y-x)+N(x).	\label{subEqNNNxxyyyb}
		\end{align}
	\end{subequations}
	Supposons d'abord que $N(x)\geq N(y)$. Dans ce cas, en utilisant \eqref{subEqNNNxxyyya},
	\begin{equation}
		\big| N(x)-N(y) \big|=N(x)-N(y)\leq N(x-y)+N(y)-N(y)=N(x-y).
	\end{equation}
	Si par contre $N(x)\leq N(y)$, alors nous utilisons \eqref{subEqNNNxxyyyb} et nous trouvons
	\begin{equation}
		\big| N(x)-N(y) \big|=N(y)-N(x)\leq N(y-x)+N(x)-N(x)=N(y-x).
	\end{equation}
	Dans les deux cas, nous avons retrouvé l'inégalité annoncée.
\end{proof}
Cette proposition signifie aussi que
\begin{equation}	\label{EqNleqNNleqNvqlqbs}
	-N(x-y)\leq N(x)-N(y)\leq N(x-y).
\end{equation}

\begin{normaltext}	
Afin de suivre une notation proche de celle de la valeur absolue, à partir de maintenant, la norme d'un vecteur $v$ sera notée $\| v\|$ au lieu de $N(v)$. La proposition~\ref{PropNmNNm} s'énoncera donc
\begin{equation}
\big| \| x \|-\| y \| \big|\leq \| x-y \|.
\end{equation}
Un espace vectoriel $E$ muni d'une norme est, on l'a déjà dit, un \defe{espace vectoriel normé}{normé!espace vectoriel}; on le notera $(E,\| . \|)$ pour distinguer la norme fixée.
\end{normaltext}

\begin{lemmaDef}[Distance induite par une norme]        \label{LEMooWGBJooYTDYIK}
    Soit un espace vectoriel normé \( (E,\| . \|)\). Nous posons
    \begin{equation}        \label{EQooZYJRooAHnsIG}
        d(x,y)=\| x-y \| .
    \end{equation}
    Alors
    \begin{enumerate}
        \item       \label{ITEMooLITDooPeReOk}
            \( d\) est invariante par translations : $d(a,b)=d(a+u,b+u)$
        \item
            \( d\) est une distance\footnote{Définition~\ref{DefMVNVFsX}.} sur \( E\).
    \end{enumerate}
    C'est la \defe{distance induite}{distance!associée à une norme} par la norme.
\end{lemmaDef}

\begin{proof}
    Le fait que la formule \eqref{EQooZYJRooAHnsIG} soit invariante par translations est immédiat. En ce qui concerne le fait que ce soit une distance, le seul point délicat à vérifier est l'inégalité triangulaire. Mais, pour tous \( x, y, z \in E\), on a
    \begin{equation}
            d(x,y)=\| x-y \| = \| x-z+z-y \|  \leq\| x - z \|+\| z - y\| =d(x,z)+d(z,y).
    \end{equation}
\end{proof}

Un espace vectoriel normé est alors immédiatement un espace vectoriel topologique\footnote{Définition~\ref{DefEVTopologique}.}.

%---------------------------------------------------------------------------------------------------------------------------
\subsection{Quelques exemples de normes sur \texorpdfstring{$\eR^n$}{Rn}}
%---------------------------------------------------------------------------------------------------------------------------

Il est possible de définir de nombreuses normes sur $\eR^n$. Citons-en quelques-unes.

\begin{propositionDef}      \label{PROPooCLZRooIRxCnZ}
    Les formules suivantes définissent des normes sur \( \eR^n\).
    \begin{enumerate}
        \item
    Les normes $\| . \|_{L^p}$ ($p\in\eN$) sont définies de la façon suivante :
    \begin{equation}		\label{EqDeformeLp}
        \| x \|_{L^p}=\Big( \sum_{i=1}^n| x_i |^p\Big)^{1/p},
    \end{equation}
    pour tout $x=(x_1,\ldots,x_n)\in\eR^n$.
\item
    La norme $L^2$ est la \defe{norme euclidienne}{norme!euclidienne}.
\item
    Nous définissons également la \defe{norme supremum}{norme!supremum} par
    \begin{equation}
	    \| x \|_{\infty}=\max_i| x_i |.
    \end{equation}
    \end{enumerate}
\end{propositionDef}

\begin{proof}
    Point par point\quext{Preuve non terminée}.
    \begin{enumerate}
        \item
        \item
    Le fait que \( x\mapsto\| x \|_{L^2}\) soit une norme provient de la propriété suivante :
    \begin{equation}
        \sqrt{ (a+b)^2 }\leq \sqrt{ a^2 }+\sqrt{ b^2 },
    \end{equation}
    laquelle se démontre en passant au carré :
    \begin{equation}        \label{EQooRYNYooTzZpPz}
        (a+b)^2=a^2+b^2+2ab\leq a^2+b^2+2| ab |=\big( \sqrt{ a^2 }+\sqrt{ b^2 } \big)^2.
    \end{equation}
\item
    \end{enumerate}
\end{proof}

Parmi ces normes, celles qui seront le plus souvent utilisées dans ces notes sont
\begin{equation}
	\begin{aligned}[]
		\| x \|_{L^1}&=\sum_{i=1}^n| x_i |,\\
		\| x \|_{L^2}&=\Big( \sum_{i=1}^n| x_i |^2 \Big)^{1/2}.
	\end{aligned}
\end{equation}

\newcommand{\CaptionFigDistanceEuclide}{La \emph{norme} euclidienne induit la \emph{distance} euclidienne. D'où son nom. Le point $C$ est construit aux coordonnées $(A_x,B_y)$.}
\input{auto/pictures_tex/Fig_DistanceEuclide.pstricks}

Soient $A=(A_x,A_y)$ et $B=(B_x,B_y)$ deux éléments de $\eR^2$. La distance\footnote{Ne pas confondre «distance» et «norme».} euclidienne entre $A$ et $B$ est donnée par $\| A-B \|_2$. En effet, sur la figure~\ref{LabelFigDistanceEuclide}, la distance entre les points $A$ et $B$ est donnée par
\begin{equation}
	| AB |^2=| AC |^2+| CB |^2=| A_x-B_x |^2+| A_y-B_y |^2,
\end{equation}
par conséquent,
\begin{equation}
	| AB |=\sqrt{| A_x-B_x |^2+| A_y-B_y |^2}=\| A-B \|_2.
\end{equation}

\begin{remark}
	Si $A$, $B$ et $C$ sont trois points dans le plan $\eR^2$, alors l'inégalité triangulaire $| AB |\leq| AC |+| CB |$ est précisément la propriété~\ref{ItemDefNormeiii} de la norme (définition~\ref{DefNorme}). En effet l'inégalité triangulaire s'exprime de la façon suivante en terme de la norme $\| . \|_2$ :
	\begin{equation}	\label{EqNDeuxAmBNNdd}
		\| A-B \|_2\leq \| A-C \|_2+\| C-B \|_2.
	\end{equation}
	En notant $u=A-C$ et $v=C-B$, l'équation \eqref{EqNDeuxAmBNNdd} devient exactement la propriété de définition de la norme :
	\begin{equation}
		\| u+v \|_2\leq \| u \|_2+\| v \|_2.
	\end{equation}
	Ceci explique pourquoi cette propriété des normes est appelée «inégalité triangulaire».
\end{remark}

%+++++++++++++++++++++++++++++++++++++++++++++++++++++++++++++++++++++++++++++++++++++++++++++++++++++++++++++++++++++++++++
\section{Suites de Cauchy, métrique et espaces complets}
%+++++++++++++++++++++++++++++++++++++++++++++++++++++++++++++++++++++++++++++++++++++++++++++++++++++++++++++++++++++++++++

%---------------------------------------------------------------------------------------------------------------------------
\subsection{Généralités}
%---------------------------------------------------------------------------------------------------------------------------

\begin{definition}[Suite de \( \tau\)-Cauchy, espace vectoriel topologique\cite{TQSWRiz,ooMKWJooLSkGfh}]   \label{DefZSnlbPc}
    Soit \( E\) un espace vectoriel topologique. Une suite \( (x_k)\) dans \( E\) est une \defe{suite \( \tau\)-Cauchy}{suite!de Cauchy} si pour tout voisinage \( \mU\) de \( 0\) il existe \( N\in \eN\) tel que \( x_k-x_l\in\mU\) pour tout \( k,l\geq N\).
\end{definition}

\begin{definition}[Espace \( \tau\)-complet]      \label{DEFooVQDBooNxprFU}
    Nous disons qu'une partie \( A\) d'un espace vectoriel topologique est \defe{\( \tau\)-complet}{complet!espace topologique} si toute suite \(  \tau\)-Cauchy d'éléments de \( A\) converge\footnote{Définition~\ref{DefXSnbhZX}.} vers un élément de \( A\).
\end{definition}

\begin{definition}[Suite de Cauchy, espace métrique]      \label{DEFooXOYSooSPTRTn}
    Une suite \( (a_k)\) dans un espace métrique \( (V,d)\) est \defe{de Cauchy}{suite!de Cauchy} si pour tout \( \epsilon\in \eR\), il existe \( N\) tel que si \( n,m\geq N\) alors \( d(a_n,a_m)<\epsilon\).
\end{definition}

Notons qu'ici, même si l'espace \( V\) n'a rien à voir avec \( \eR\), nous prenons \( \epsilon\) dans \( \eR\) et la distance à valeurs dans \( \eR\). Cela semble une évidence, mais il faut se rendre compte que \( \eR\) commence à prendre une place centrale dans nos constructions. Ce n'était pas le cas du temps où nous parlions de suites de Cauchy et de complétude dans des corps totalement ordonnés (définitions~\ref{DefKCGBooLRNdJf}). Dans ce contexte, le \( \epsilon\) était pris dans le corps lui-même.

\begin{definition}[Métrique complète]       \label{DEFooHBAVooKmqerL}
    Soit \( (E,d)\) un espace métrique. Nous disons que la métrique \( d\) est \defe{complète}{complet!métrique} si toute suite de Cauchy dans \( (E,d)\) converge dans \( E\).
\end{definition}

\begin{normaltext}
    Ces définitions méritent quelques remarques.
    \begin{enumerate}
        \item
            Dans le cas des espaces vectoriels topologiques, nous définissons les notions de suite \( \tau\)-Cauchy et d'espace topologique \( \tau\)-complet. Nous ajoutons le préfixe \( \tau\) pour indiquer que ce sont des notions topologiques.
        \item
            Dans le cas des espaces métriques, nous définissons la notion de \emph{métrique} complète. C'est bien la métrique qui est complète, et non l'espace. En effet nous allons voir dans l'exemple \ref{EXooNMNVooXyJSDm} que le même espace topologique peut accepter plusieurs distances différentes (donnant la même topologie) donnant lieu à des suites de Cauchy différentes.
        \item
            Si un espace vectoriel a une topologie issue d'une distance, rien ne dit que ses suites \( \tau\)-Cauchy et ses suites de Cauchy sont les mêmes. Ce sont deux notions a priori séparées. Si \( V\) est un espace vectoriel topologique que l'on peut munir de deux distances \( d_1, d_2\) donnant toutes deux la topologie, dire que \( V\) est \( \tau\)-complet, dire que \( d_1\) est complète et dire que \( d_2\) est complète sont trois choses différentes. Même si les trois topologies sont identiques.
        \item
            Nous allons bien entendu voir que dans de larges gammes d'exemples, les notions de suite de Cauchy et \( \tau\)-Cauchy coincident.
    \end{enumerate}
\end{normaltext}

\begin{example}[La complétude n'est pas une propriété topologique\cite{ooSCDYooWutzzr}]     \label{EXooNMNVooXyJSDm}
    Le fait pour un espace d'être complet n'est pas une propriété topologique, mais une propriété métrique. Plus exactement, il existe des espaces topologiques isomorphes, mais dont l'un est complet et l'autre non.

    Nous considérons la distance suivante sur \( \eN\) :
    \begin{equation}
        d_1(x,y)=\Bigl| \frac{1}{ x }-\frac{1}{ y } \Bigr|.
    \end{equation}
    Pour vérifier que cette formule définit bien une distance (définition~\ref{DefMVNVFsX}), le seul point non immédiat est l'inégalité triangulaire :
    \begin{equation}
        d_1(x,y)=\Bigl| \frac{1}{ x }-\frac{1}{ y } \Bigr|\leq\Bigl| \frac{1}{ x }-\frac{1}{ z } \Bigr|+\Bigl| \frac{1}{ z }-\frac{1}{ y } \Bigr|=d_1(x,z)+d_1(z,y).
    \end{equation}

    Au niveau de la topologie induite par cette distance, c'est la topologie discrète. En effet, soit \( x\in \eN\) et \( \epsilon>0\); nous voulons déterminer la boule \( B(x,\epsilon)\) en résolvant l'équation
    \begin{equation}
        \Bigl| \frac{1}{ x }-\frac{1}{ y } \Bigr|<\epsilon
    \end{equation}
    pour \( y\in \eN\). Nous trouvons que $\frac 1 y > \frac 1 x  - \epsilon$ et $\frac 1 y < \frac 1 x + \epsilon$, soit
    \begin{subequations}
        \begin{numcases}{}
            y > \frac 1 {\frac 1 x  + \epsilon}\\
            y < \frac 1 {\frac 1 x - \epsilon}.
        \end{numcases}
    \end{subequations}
    Si \( \epsilon \) est assez petit, la seule solution entière est \( y=x\). Les ouverts sont donc toutes les parties parce que tous les singletons sont ouverts.

    Si \( d\) est la distance usuelle sur \( \eN\) (\( d(x,y)=| x-y |\)), nous avons donc un isomorphisme d'espaces topologiques
    \begin{equation}
        (\eN,d)\simeq (\eN,d_1).
    \end{equation}
    Nous pouvons même donner un isomorphisme explicite : \( f(n)=n\).

    La suite \( (x_n)=n\) est une suite de Cauchy dans \( (\eN,d_1)\) parce que si \( \epsilon>0\) est donné, il suffit de prendre \( N\) assez grand pour avoir \( \frac{1}{ N }<\epsilon\) (possible par le lemme~\ref{LemooHLHTooTyCZYL}) nous avons, pour \( n,m>N\) :
    \begin{equation}
        \Bigl| \frac{1}{ n }-\frac{1}{ m } \Bigr|<\frac{1}{ n }<\frac{1}{ N }<\epsilon.
    \end{equation}
    Or cette suite ne converge pas. Soit en effet un candidat limite \( k\). Calculons
    \begin{equation}
        d_1(x_n,k)= \Bigl| \frac{1}{ n }-\frac{1}{ k } \Bigr |\to \frac{1}{ k }\neq 0.
    \end{equation}
    L'espace \( (\eN,d_1)\) n'est pas complet.

    Notons que cette suite n'est pas de Cauchy dans \( (\eN,d)\).

    En résumé :
    \begin{enumerate}
        \item
            Les espaces topologiques \( (\eN,d)\) et \( (\eN,d_1)\) sont isomorphes.
        \item
            Ils ont les mêmes notions de suites convergentes : une suite convergente pour l'un est convergente pour l'autre.
        \item
            Ils n'ont pas les mêmes notions de suites de Cauchy.
        \item
            Dans \(  (\eN,d_1)  \), il existe des suites de Cauchy qui ne convergent pas (pas complet).
        \item
            L'espace \( (\eN,d)\) est complet, mais \( (\eN,d_1)\) n'est pas complet.
        \item
            Le fait pour un espace topologique métrique d'être complet n'est pas intrinsèque à sa topologie : la complétude est une propriété de la distance. La complétude est une propriété de la métrique, et non de la topologie qui s'en suit.
    \end{enumerate}
\end{example}

%---------------------------------------------------------------------------------------------------------------------------
\subsection{Cas d'équivalence}
%---------------------------------------------------------------------------------------------------------------------------
\begin{lemma}       \label{LEMooIAHSooFkXjvr}
    Soit un espace vectoriel topologique\footnote{Définition~\ref{DefEVTopologique}.} \( V\) et une distance \( d\colon V\times V\to \eR^+\) compatible\footnote{Définition~\ref{DEFooGTOZooRcvGHg}.} avec la topologie de \( V\). Si \( d\) est invariante\footnote{Définition~\ref{DEFooGTOZooRcvGHg}.}, alors les suites de Cauchy pour \( d \) et les suites \( \tau\)-Cauchy sont les mêmes.
\end{lemma}

\begin{proof}
    Nous avons deux implications à prouver. 
    \begin{subproof}
    \item[Cauchy pour \( d\) implique \( \tau\)-Cauchy]
        Soit \( (x_n)\), une suite de Cauchy dans \( V\) pour \( d\), et un voisinage \( U\) de \( 0\). Vu que \( d\) est compatible avec la topologie de \( V\), il existe une boule ouverte \( B(0,\epsilon)\) incluse à \( U\). Soit \( N>0\) tel que \( m,n>N\) implique \( d(x_n,x_m)<\epsilon\). Par invariance de la métrique, nous avons aussi
        \begin{equation}
            d(0,x_m-x_n)<\epsilon,
        \end{equation}
        c'est à dire \( x_m-x_n\in B(0,\epsilon)\subset U\). La suite \( (x_n)\) est donc \( \tau\)-Cauchy.
    \item[\( \tau\)-Cauchy implique Cauchy pour \( d\)]
        Soit $(x_n)$, une suite \( \tau\)-Cauchy dans \( V\) et \( \epsilon>0\). Vu que \( B(0,\epsilon)\) est un voisinage de \( 0\) dans \( V\), il existe \( N\) tel que \( m,n>N\) implique \( x_n-x_m\in B(0,\epsilon)\). Cela signifie que \( d(0,x_n-x_m)<\epsilon\) et toujours par invariance, que \( d(x_n,x_m)<\epsilon\).
    \end{subproof}
\end{proof}

\begin{definition}[Espace vectoriel topologique métrisable\cite{ooOFEPooVFgTXm}]
    Un espace vectoriel topologique est \defe{métrisable}{métrisable} si il existe une distance compatible avec la topologie.
\end{definition}

\begin{proposition}[\cite{ooCGEHooVTyTuY}]      \label{PROPooXWBTooCvGLOj}
    Soit un espace topologique métrisable \( X\).
    \begin{enumerate}
        \item
            Tout fermé de \( X\) est une intersection dénombrable d'ouverts.
        \item
            Tout ouvert de \( X\) est une union dénombrable de fermés.
    \end{enumerate}
\end{proposition}

Pour nous simplifier la vie dans la preuve, introduisons la notation qui suit, que nous emploierons par la suite.
\begin{definition}
	Si $A$ est une partie de l'espace métrique $(X,d)$ et si $x\in X$, nous disons que la \defe{distance}{distance!point et ensemble} entre $A$ et $x$ est le nombre
	\begin{equation}		\label{EqdefDistaA}
		d(x,A)=\inf_{a\in A}d(x,a).
	\end{equation}
\end{definition}
%The result is on the figure~\ref{LabelFigDistanceEnsemble}
\newcommand{\CaptionFigDistanceEnsemble}{La distance entre $x$ et $A$ est donnée par la distance entre $x$ et $p$. Les distances entre $x$ et les autres points de $A$ sont plus grandes que $d(x,p)$.}
\input{auto/pictures_tex/Fig_DistanceEnsemble.pstricks}

\begin{proof}
    Soit une métrique \( d\) compatible avec la topologie de \( X\) et un fermé \( A\). Nous posons
    \begin{equation}
        V_n=\{ x\in X\tq d(x,A)<\frac{1}{ n } \}.
    \end{equation}
    Et juste pour faire simple nous notons \( V_0=X\).
    \begin{subproof}
        \item[Les parties \( V_n\) sont ouvertes]
            Soit \( x\in V_n\). Trouvons un voisinage de \( x\) contenu dans \( V_n\) -- encore le théorème~\ref{ThoPartieOUvpartouv}. Si \( y\in B(x,\epsilon)\) alors il existe \( a\in A\) tel que \( d(x,a)<\frac{1}{ n }\) (ici les inégalités strictes sont importantes) et donc
            \begin{equation}
                d(y,a)\leq d(y,x)+d(x,a)<\epsilon+\frac{1}{ n }.
            \end{equation}
            Nous pouvons donc choisir \( \epsilon\) de telle sorte que \( d(y,a)<\epsilon\) pour tout \( a\in B(x,\epsilon)\). Cela prouve que \( V_n\) est ouvert.
        \item[\( A\) est l'intersection des \( V_n\)]
            Nous avons évidemment \( A\subset V_n\) pour tout \( n\). Et d'autre part, si \( a\in\bigcap_{n\in \eN} V_n\) alors \( d(a,A)<\frac{1}{ n }\) pour tout \( n\). Cela implique \( d(a,A)=0\), c'est à dire \( a\in A\).
        \end{subproof}
    En ce qui concerne la seconde partie, nous passons au complémentaire. Si \( \mO\) est ouvert, \( \mO^c\) est fermé et
    \begin{equation}
        \mO^c=\bigcap_{n\in \eN}V_n,
    \end{equation}
    ce qui donne immédiatement
    \begin{equation}
        \mO=\bigcup_{n\in \eN}V_n^c
    \end{equation}
    où les \( V_n^c\) sont fermés.
\end{proof}

\begin{probleme}
Il faut sans doute mettre un peu plus de rigueur dans la preuve ci-dessus (surtout si on veut la resservir à un jury). Notamment, justifier proprement le fait que $d(a, A ) =0$ pour $A$ fermé implique $a \in A$.
\end{probleme}


\begin{corollary}       \label{CORooTWFYooCNMieM}
    Si \( X\) est un espace topologique métrisable, alors \( X\) accepte une base dénombrable de topologie autour de chaque point.
\end{corollary}

\begin{proof}
    Il s'agit seulement de remarquer que les singletons sont fermés et d'appliquer la proposition~\ref{PROPooXWBTooCvGLOj}.
\end{proof}

\begin{theorem}[\cite{ooMKWJooLSkGfh}]      \label{THOooAGBXooZnvQLK}
    Si $V$ est un espace vectoriel topologique possédant en tout point une base de topologie dénombrable, alors il existe une distance \( d\) sur \( V\) telle que
    \begin{enumerate}
        \item
            \( d\) est compatible avec la topologie de \( V\),
        \item
            \( d\) est invariante par translation.
    \end{enumerate}
\end{theorem}
Une preuve est donnée dans \cite{ooMKWJooLSkGfh} et je vous préviens : c'est pas simple.

\begin{proposition}     \label{PROPooPRLBooGtsRjr}
    Un espace vectoriel topologique est métrisable si et seulement si il possède en tout point une base dénombrable de topologie.
\end{proposition}

\begin{proof}
    Il s'agit seulement de mettre bout à bout les corollaires~\ref{CORooTWFYooCNMieM} et théorème~\ref{THOooAGBXooZnvQLK}.
\end{proof}

Tout ceci nous mène à donner une large classe d'espaces vectoriels topologiques sur lesquelles les notions de suites de Cauchy pour une distance et \( \tau\)-Cauchy coïncident.

\begin{theoremDef}     \label{THOooGQZSooAmQolf}
    Soit \( V\) un espace vectoriel topologique métrisable\footnote{i.e. admet une base dénombrable de topologie, voir la proposition~\ref{PROPooPRLBooGtsRjr}}, alors il admet une métrique \( d\) compatible avec la topologie telle que une suite dans \( V\) est de Cauchy pour \( d\) si et seulement si elle est \( \tau\)-Cauchy.

    Une \defe{suite de Cauchy}{Cauchy!suite} dans un espace vectoriel métrique \( (E,d)\) est une suite \( \tau\)-Cauchy ou de Cauchy pour \( d \).
\end{theoremDef}

\begin{proof}
    Soit \( d\) une métrique sur \( V\) satisfaisant au théorème~\ref{THOooAGBXooZnvQLK}. Vu qu'elle est invariante par translation, les suites \( d\)-Cauchy sont exactement les suites \( \tau\)-Cauchy par le lemme~\ref{LEMooIAHSooFkXjvr}.
\end{proof}

\begin{remark}  \label{REMooUFQYooUVCCjs}
    Même si \( V\) est métrisable, si on choisit la métrique n'importe comment, on ne peut rien espérer.
\end{remark}

\begin{normaltext}
    Sur les espaces vectoriels topologiques métrisables, nous pouvons donc parler de suite de Cauchy sans préciser si nous parlons de \( \tau\)-Cauchy ou de \( d\)-Cauchy, parce que nous sous-entendons avoir choisi une métrique non seulement compatible avec la topologie, mais également invariante par translation.

    Il reste cependant à traiter le cas d'un espace vectoriel topologique non métrisable. Dans ce cas, il n'y a pas de métrique, et la question de l'équivalence des définitions ne se pose pas.
\end{normaltext}

Le théorème suivant donne la complétude de \( \eR\) et le critère de Cauchy pour les définitions métriques et topologiques usuelles. Lorsqu'on dit que \( \eR\) est complet, le plus souvent nous parlons de ce théorème, et non de~\ref{THOooUFVJooYJlieh} qui en est un lemme indispensable mais qui parle de notions différentes, bien que très liées.
\begin{theorem}[Complétude de \( \eR\), critère de Cauchy\cite{RWWJooJdjxEK}]       \label{THOooNULFooYUqQYo}
    Nous avons :
    \begin{enumerate}
        \item
            L'espace métrique \( (\eR,d)\) est complet (définition~\ref{DEFooHBAVooKmqerL}).
        \item
            Une suite dans \( \eR\) est convergente (définition~\ref{DefXSnbhZX}) si et seulement si elle est de Cauchy (définition~\ref{THOooGQZSooAmQolf}).
    \end{enumerate}
\end{theorem}
\index{complet!$\eR$!espace métrique}
\index{critère!de Cauchy}

\begin{proof}
    Tout ce théorème se base sur le fait que la définition de suite de Cauchy dans \( (\eR,d)\) et de suite convergente dans \( (\eR,d)\) coïncident avec les définitions correspondantes dans \( \eR\) vu comme simple corps ordonné (définitions~\ref{DefKCGBooLRNdJf}).

    Donc si \( (x_n)\) est de Cauchy dans \( (\eR,d)\), elle est de Cauchy dans le corps ordonné \( (\eR,\leq)\). Donc le théorème~\ref{THOooUFVJooYJlieh} nous dit que \( (x_n)\) est convergente dans \( (\eR,\leq)\). Et donc convergente dans \( (\eR,d)\).

    Toutes les autres affirmations se prouvent de la même manière.
\end{proof}

Si vous n'êtes pas sûr ou si vous ne voulez pas étudier les notations de convergence et de suites de Cauchy dans les corps, vous pouvez simplement recopier la démonstration du théorème~\ref{THOooUFVJooYJlieh} en remplaçant partout \( \eQ\) par \( \eR\), et aussi en remplaçant les \( | x-y |\) par \( d(x,y)\).

\begin{normaltext}
    Nous pouvons également mettre une structure d'espace métrique sur \( \eC\) en posant
    \begin{equation}
        d(z,z')=| z-z' |.
    \end{equation}
\end{normaltext}

\begin{proposition}
    L'espace métrique \( (\eC,d)\) est complet.
\end{proposition}

\begin{proof}
    Commençons par nous rendre compte que pour tout \( z\in \eC\) nous avons \( | \real(z) |\leq | z |\). C'est bon ? Vous vous en êtes rendu compte ? Ok. Continuons.

    Soit une suite de Cauchy \( (z_k)\) dans \( \eC\) et \( \epsilon>0\). Si \( x_k=\real(z_j)\), nous avons
    \begin{equation}
        | x_k-x_l |=| \real(z_k-z_l) |\leq | z_k-z_l |.
    \end{equation}
    Vu que \( (z_k)\) est de Cauchy, il existe un \( N\) tel que si \( k,l\geq N\),
    \begin{equation}
        | x_k-x_l |\leq | z_k-z_l |\leq \epsilon.
    \end{equation}

    Donc la suite des parties réelles converge par la complétude de \( (\eR,d)\) du théorème~\ref{THOooNULFooYUqQYo}. Notez que le \( d\) ici n'est pas tout à fait le même, et que la démonstration fonctionne parce que la distance prise sur \( \eR\) est la restriction à \( \eR\) de la distance prise sur \( \eC\). Notons \( x\) la limite de \( (x_k)\).

    De la même manière la suite des parties imaginaires \( y_k=\imag(z_k)\) converge vers un réel que nous notons \( y\). Avec tout cela, la suite \( z_k\) converge dans \( \eC\) vers \( x+iy\). En effet pour \( \epsilon\) donné et pour un \( k\) suffisament grand,
    \begin{equation}
        | z_k-(x+iy) |=\big| \real(z_k)-x+i(\imag(z_k)-y) \big|\leq | x_k-x |+| y_k-y |\leq \epsilon.
    \end{equation}
\end{proof}

%---------------------------------------------------------------------------------------------------------------------------
\subsection{Espace topologique métrique}
%---------------------------------------------------------------------------------------------------------------------------

Dans les espaces vectoriels topologiques métriques, il n'y a pas d'ambiguïté.
\begin{proposition}[Caractérisations avec la distance \( d \)]     \label{PropooUEEOooLeIImr}
    Soit \( (E,d)\) un espace vectoriel topologique métrique.
    \begin{enumerate}
        \item   \label{ItemooROYMooAQCXnj}
            Une suite \( (x_n)\) dans \( E\) est convergente\footnote{Définition~\ref{DefXSnbhZX}.} vers \( x\) si et seulement si pour tout \( \epsilon\in \eR\) il existe \( N_{\epsilon}\) tel que pour tout \( n\geq N_{\epsilon}\) nous avons \( d(x_n,x)\leq \epsilon\).
        \item
            Une suite \( (x_n)\) dans \( E\) est de Cauchy\footnote{Définition~\ref{DefZSnlbPc}.} si pour tout \( \epsilon\in \eR\), il existe un \( N_{\epsilon}\) tel que si \( p,q\geq N_{\epsilon}\), nous avons \( d(x_p,x_q)\leq \epsilon\).
    \end{enumerate}
\end{proposition}

\begin{proof}
   En ce qui concerne la convergence :
    \begin{subproof}
        \item[Sens direct]

            Nous supposons que \( x_k\to x\) dans \( E\). Soit \( \epsilon>0\); vu que \( B(x,\epsilon)\) est un ouvert contenant \( x\), il existe un \( N_{\epsilon}>0 \) tel que \( k>N_{\epsilon}\) implique \( x_k\in B(x,\epsilon)\). Cela signifie \( d(x,x_k)\leq \epsilon\).

        \item[Réciproque]

            Nous supposons que pour tout \( \epsilon>0\), il existe \( N_{\epsilon}>0\) tel que si \( k>N_{\epsilon}\) alors \( x_k\in B(x,\epsilon)\). Soit un ouvert \( \mO\) autour de \( x\). Nous sommes dans un espace métrique; ergo la topologie est donné par le théorème~\ref{ThoORdLYUu} et en particulier la liste des ouverts est donnée par \eqref{EqGDVVooDZfwSf}. Il existe donc une boule \( B(x,\epsilon)\) incluse à \( \mO\). Pour tout \( k>N_{\epsilon}\) nous avons alors \( x_k\in B(x,\epsilon)\subset\mO\).
    \end{subproof}
    En ce qui concerne les suites de Cauchy :
    \begin{subproof}
    \item[Sens direct]
        Si \( (x_n)\) est une suite de Cauchy et si \( \epsilon>0\) est donné, alors \( B(0,\epsilon)\) est un voisinage de \( 0\) et il existe \( N_{\epsilon}\) tel que si \( p,q\geq N_{\epsilon}\) alors \( x_p-x_q\in B(0,\epsilon)\). Posons \( u=x_p-x_q\); en utilisant l'invariance par translation (lemme~\ref{LEMooWGBJooYTDYIK}\ref{ITEMooLITDooPeReOk}) nous avons
        \begin{equation}
            d(u,0)=d(x_p-x_q,0)=d(x_p,x_q).
        \end{equation}
        Par conséquent \( d(x_p,x_q)\leq \epsilon\).
    \item[Réciproque]
        Soit \( \mO\) un voisinage de \( 0\). Il existe \( \epsilon\) tel que \( B(0,\epsilon)\subset \mO\). Par hypothèse il existe \( N_{\epsilon}\) tel que \( d(x_p,x_q)\leq \epsilon\) dès que \( p,q\geq N_{\epsilon}\). En utilisant encore l'invariance par translation nous avons
        \begin{equation}
            d(x_p,x_q)=d(x_p-x_q,0),
        \end{equation}
        et comme cela est plus petit que \( \epsilon\), nous avons \( x_p-x_q\in B(0,\epsilon)\subset\mO\).
    \end{subproof}
\end{proof}

\begin{proposition}[\cite{IRWFPQB}]
    Toute suite convergente dans un espace métrique est de Cauchy.
\end{proposition}

\begin{proof}
    Nous utilisons les caractérisations de la proposition~\ref{PropooUEEOooLeIImr} des suites convergentes et de Cauchy.

    Soit un espace métrique \( (X,d)\) et \( x_n\to\ell\) une suite convergente. Si \( \epsilon>0\), la proposition~\ref{PropooUEEOooLeIImr}\ref{ItemooROYMooAQCXnj}, dit qu'il existe \( N\) tel que pour tout \( n>N\) nous ayons \( d(x_n,\ell)<\epsilon\). Par conséquent si \( n,m>N\) alors
    \begin{equation}
        d(x_n,x_m)\leq d(x_m,\ell)+d(l,x_m)\leq 2\epsilon.
    \end{equation}
    Cela prouve que \( (x_n)\) est de Cauchy.
\end{proof}

%---------------------------------------------------------------------------------------------------------------------------
\subsection{Les boules, une base de topologie}
%---------------------------------------------------------------------------------------------------------------------------

\begin{proposition} \label{PropNBSooraAFr}
    Un espace métrique séparable\footnote{Qui possède une partie dense dénombrable, définition~\ref{DefUADooqilFK}.} accepte une base de topologie dénombrable.

     Soit \( A\) dense et dénombrable dans l'espace métrique séparable \( (E,d)\). Si \( \{ a_i \}_{i\in \eN}\) est une énumération de \( A\) et \( \{ r_i \}_{i\in \eN}\) une énumération de \( \eQ\), alors
    \begin{equation}
        \mB=\{ B(a_i,r_j) \}_{i,j\in \eN}
    \end{equation}
    est une base de la topologie de \( E\).
\end{proposition}
\index{base!de topologie!espace métrique}
\index{espace!métrique!base de topologie}
\index{base!de topologie!dénombrable}

\begin{proof}
    Soient \( x\in E\) et \( V\) un voisinage de \( x\). Ce dernier contient une boule \( B(x,r)\) et quitte à prendre \( r\) un peu plus petit nous supposons que \( r\in \eQ\) (densité de \( \eQ\) dans \( \eR\), proposition~\ref{PropooUHNZooOUYIkn}).

    Soit \( a\in A\) avec \( \| a-x \|<\frac{ r }{ 3 }\) (existe par densité de \( A\) dans \( E\)); nous avons \( B(a,\frac{ 2r }{ 3 })\subset B(x,r)\) parce que si \( y\in B( a,\frac{ 2r }{ 3 } )\) alors
    \begin{equation}
        \| y-x \|\leq \| y-a \|+\| a-x \|<\frac{ 2 }{ 3 }r+\frac{ 1 }{ 3 }r=r.
    \end{equation}
    La seconde inégalité est stricte parce que les boules sont ouvertes. Le tout montre que \( y\in B(x,r)\). Par ailleurs \( x\in B(a,\frac{ 2 }{ 3 }r)\) et nous avons trouvé un élément de \( \mB\) contenant \( x\) tout en étant inclus dans \( V\). Cela prouve que \( \mB\) est bien une base de la topologie de \( E\).
\end{proof}


\begin{remark}      \label{RemIPVLooHUXyeW}
    Il est vite vu que les cubes ouverts forment aussi une base de la topologie de \( \eR^n\). Cela est à mettre en rapport avec le fait que toutes les normes sont équivalentes sur \( \eR^n\) (proposition~\ref{ThoNormesEquiv}).

    % position 13268

    Voir aussi le corollaire~\ref{CorTHDQooWMSbJe} qui donnera tout ouvert comme union de pavés presque disjoints.
\end{remark}

\begin{lemma}   \label{LemDUJXooWsnmpL}
    Soient \( (X_1,d_1)\) et \( (X_2,d_2)\) des espaces métriques séparables. Alors \( X_1\times X_2\) admet une base dénombrable de topologie constituée de produits de boule de \( X_1\) par des boules de \( X_2\). Plus précisément si $A_i$ est dénombrable et dense dans \( X_i\) alors l'ensemble des produits
    \begin{equation}
        \big\{ B(y_1,r_1)\times B(y_2,r_2)\big\}_{\substack{y_i\in A_i\\r_i\in \eQ^+}}
    \end{equation}
    est une base de topologie pour \( X_1\times X_2\).
\end{lemma}

\begin{proof}
    Soit \( \mO\) un ouvert de \( X_1\times X_2\) et \( (x_1,x_2)\in \mO\). Par définition de la topologie produit\footnote{Définition~\ref{DefIINHooAAjTdY}.}, il existe \( r_1,r_2\in \eQ^+\) tels que \( B(x_1,r_1)\times B(x_2,r_2)\subset\mO\). Les parties \( A_i\) étant denses, il existe \( y_i\in B(x_i,r_i/2)\cap A_i\). Avec ces choix nous avons $x_i\in B(y_i,\frac{ r_i }{2})$. Nous avons donc
    \begin{equation}
        (x_1,x_2)\in B(y_1,\frac{ r_1 }{ 2 })\times B(y_2,\frac{ r_2 }{2}).
    \end{equation}
    Il est facile de voir que \( B(y_i,r_i/2)\subset B(x_i,r_i)\). En effet si \( z_i\in B(y_i,r_i/2)\) alors
    \begin{equation}
        d_i(z_i,x_i)\leq d(z_i,y_i)+d(y_i,x_i)\leq \frac{ r_i }{2}+\frac{ r_i }{2}=r_i.
    \end{equation}
    Au final,
    \begin{equation}
        (x_1,x_2)\in B(y_1,\frac{ r_1 }{ 2 })\times B(y_2,\frac{ r_2 }{2})\subset \mO.
    \end{equation}
\end{proof}

\begin{lemma}   \label{LemOWVooZKndbI}
    Une partie \( K\) d'un espace topologique est compacte si et seulement si de tout recouvrement par des ouverts d'une base de topologie nous pouvons extraire un sous-recouvrement fini.
\end{lemma}
Remarquons que la partie qui est réellement à prouver est que, si \og ça marche \fg{} pour des ouverts d'une base de topologie, alors \og ça marche\fg{} pour tous types d'ouverts.
\begin{proof}
    Soit \( K\) une partie d'un espace topologique et \( \{ \mO_i \}_{i\in I}\) un recouvrement de \( K\) par des ouverts. Chacun des \( \mO_i\) est une union d'éléments de la base de topologie par la proposition~\ref{PropMMKBjgY}: disons \( \mO_i = \bigcup_{j \in J_i} A_{(i,j)} \). Soit \( J = \{ j = (i, j_i) | i \in I, j_i \in J_i \} \); alors nous obtenons  \( \bigcup_{j\in J}A_j=\bigcup_{i\in I}\mO_i\).

    Par hypothèse nous pouvons extraire un ensemble fini \( J_0\subset J\) tel que \( K\subset\bigcup_{j\in J_0}A_j\). Par construction chacun des \( A_j\) est inclus dans (au moins) un des \( \mO_i\). Le choix d'un élément de \( I\) pour chacun des éléments de \( J_0\) donne une partie finie \( I_0\) de \( I\) telle que \( K\subset\bigcup_{j\in J_0}A_j\subset\bigcup_{i\in I_0}\mO_i\).
\end{proof}