% This is part of Mes notes de mathématique
% Copyright (c) 2008-2021
%   Laurent Claessens, Carlotta Donadello
% See the file fdl-1.3.txt for copying conditions.

%+++++++++++++++++++++++++++++++++++++++++++++++++++++++++++++++++++++++++++++++++++++++++++++++++++++++++++++++++++++++++++
\section{Produit fini d'espaces vectoriels normés}
%+++++++++++++++++++++++++++++++++++++++++++++++++++++++++++++++++++++++++++++++++++++++++++++++++++++++++++++++++++++++++++
\label{sec_prod}

Dans cette sections nous parlons de produits finis d'espaces. Cela ne signifie pas que chacun des espaces soient séparément de dimension finie.

%---------------------------------------------------------------------------------------------------------------------------
\subsection{Distance et norme produit}
%---------------------------------------------------------------------------------------------------------------------------

\begin{propositionDef}[Distance produit]    \label{DefZTHxrHA}
	Si \( (E_1,d_1)\),\ldots, \( (E_n,d_n)\) sont des espaces métriques alors la formule
	\begin{equation}
		d(x,y)=\max_{i=1,\ldots, n}d_i(x_i,y_i)
	\end{equation}
	définit une distance sur le produit cartésien \( E=E_1\times\ldots\times E_n\). Elle est la \defe{distance produit}{distance produit}.
\end{propositionDef}

La définition de la norme sur un produit d'espaces vectoriels normés découle immédiatement de la définition de la distance~\ref{DefZTHxrHA} :
\begin{lemmaDef}[\cite{ooALKGooMAzKpz}]  \label{DefFAJgTCE}
	Soient $V$ et $W$ deux espaces vectoriels sur \( \eK\). Si \( (v_1,w_1)\) et \( (v_2,w_2)\) sont des éléments de \( V\times W\) et si \( \lambda\) est un élément de \( \eK\), alors les opérations suivantes donnent une structure d'espace vectoriel au produit \( V\times W\):
	\begin{itemize}
		\item \( (v_1,w_1)+(v_2,w_2)=(v_1+v_2,w_1+w_2)\)
		\item \( \lambda(v_1,w_1)=(\lambda v_1,\lambda w_1)\).
	\end{itemize}
\end{lemmaDef}

\begin{proof}
	Il faut seulement faire les vérifications d'usage.
\end{proof}

\begin{lemmaDef}        \label{LEMooFQMSooLmdIvD}
	Soient deux espaces vectoriels normés \( V\) et \( W\).
	\begin{enumerate}
		\item
		      L'opération
		      \begin{equation}	\label{EqNormeVxWmax}
			      \|(v,w) \|_{V\times W}=\max\{\|v\|_{V},\|w\|_W\}.
		      \end{equation}
		      est une norme\footnote{Définition \ref{DefNorme}.} sur \( V\times W\).
		\item
		      La topologie de cette norme\footnote{Topologie associée à une norme : c'est la topologie associée à la distance correspondante, définition \ref{DEFooPMVFooPSYVNQ}.} est la même que la topologie produit\footnote{Topologie produit, définition \ref{DefIINHooAAjTdY}.} sur \( V\times W\).
	\end{enumerate}
\end{lemmaDef}

\begin{proof}
	En plusieurs parties.
	\begin{subproof}
		\item[Norme]
		On doit vérifier les trois conditions de la définition~\ref{DefNorme}.
		\begin{itemize}
			\item Soit $(v,w)$ dans $V\times W$ tel que $\|(v,w)\|_{V\times W}=\max\{\|v\|_{V},\|w\|_W\}=0$. Alors $\|v\|_V=0$ et $\|w\|_W=0$, donc $v=0_V$ et $w=0_W$. Cela implique $(v,w)=(0_v,0_w)=0_{V\times W}$.
			\item Pour tout $a$ dans $\eR$ et $(v,w)$ dans $V\times W$, la norme $\|a (v,w)\|_{V\times W}$ se calcule de la façon suivante :
			      \begin{equation}
				      \|a (v,w)\|_{V\times W}= \max\{ \| av \|_V,\| aw \|_W \} =|a|\max\{\|v\|_{V},\|w\|_W\}=|a|\|(v,w)\|_{V\times W}.
			      \end{equation}
			\item Soient $(v_1,w_1)$ et $(v_2,w_2)$ dans $V\times W$.
			      \begin{equation}
				      \begin{aligned}
					      \|(v_1,w_1)+(v_2,w_2)\|_{V\times W} & =\max\{\|v_1+v_2\|_{V},\|w_1+w_2\|_W\}                         \\
					                                          & \leq \max\{\|v_1\|_V+\|v_2\|_{V},\|w_1\|_W+\|w_2\|_W\}         \\
					                                          & \leq\max\{\|v_1\|_V,\|w_1\|_W\}+ \max\{\|v_2\|_{V},\|w_2\|_W\} \\
					                                          & =\|(v_1,w_1)\|_{V\times W}+\|(v_2,w_2)\|_{V\times W}.
				      \end{aligned}
			      \end{equation}
		\end{itemize}
		\item[Équivalence]

		Dans cette preuve, nous considérons la «topologie de \( V\times W\)» comme étant la topologie produit et «la topologie métrique de \( V\times W\)» la topologie de la norme produit.
		\begin{subproof}
			\item[Dans un sens]
			La définition \ref{DefIINHooAAjTdY} de la topologie produit dit qu'une prébase de \( V\times W\) est donnée par
			\begin{equation}
				\big\{   B(v,r)\times B(w,s)\tq v\in V;w\in W;r,s>0   \big\}.
			\end{equation}
			Nous prouvons maintenant que la partie \( S= B(v_0,r)\times B(w_0,s)\) est un ouvert de l'espace \( \big( V\times W,\| . \|_{V\times W} \big)\). Pour cela nous prouvons que tout élément de \( S\) contient un voisinage métrique contenu dans \( S\).

			Soit \( (v_1,w_1)\in S\). Nous posons
			\begin{equation}
				d\big( (v_1,w_1), (v_0,w_0) \big)=\delta<\max\{ r,s \}.
			\end{equation}
			Nous considérons \( \epsilon>0\) et nous montrons que si \( \epsilon\) est assez petit, \( B\big( (v_1,w_1),\epsilon \big)\subset S\). Pour cela nous considérons \( (v,w)\in B\big( (v_1,w_1),\epsilon \big)\) et nous calculons un tout petit peu :
			\begin{subequations}
				\begin{align}
					d\big( (v,w),(v_0,w_0) \big) & \leq d\big( (v,w),(v_1,w_1) \big)+d\big( (v_1,w_1),(v_0,w_0) \big) \\
					                             & <\epsilon+\delta.
				\end{align}
			\end{subequations}
			Si \( \epsilon\) est assez petit, le tout reste plus petit que \( \max\{ r,s \}\).

			Donc \( S\) est bien un ouvert métrique par le théorème \ref{ThoPartieOUvpartouv}. Vu que la topologie métrique contient une prébase de la topologie produit, tout ouvert de la topologie produit est un ouvert de la topologie métrique.
			\item[Dans l'autre sens]
			Soient un ouvert métrique \( \mO\) ainsi que \( (v_0,w_0)\in \mO\); il existe \( r>0\) tel que
			\begin{equation}
				B\big( (v_0,w_0),r \big)\subset \mO.
			\end{equation}
			Nous affirmons que \( B(v_0,r)\times B(w_0,r)\) est contenu dans \( \mO\), de telle sorte que \( \mO\) soit un ouvert de la topologie produit. Pour \( (v_1,w_1)\in B(v_0,r)\times B(w_0,r)\) nous avons
			\begin{equation}
				d\big( (v_1,w_1),(v_0,w_0) \big)=\max\{ \| v_1-v_0 \|,\| w_1-w_0 \| \}<r
			\end{equation}
			parce que \( v_1\in B(v_0,r)\) et \( w_1\in B(w_0,r)\).

			Donc tout élément de \( \mO\) admet un voisinage «produit» contenu dans \( \mO\); donc \( \mO\) est ouvert pour le topologie produit.
		\end{subproof}
	\end{subproof}
\end{proof}

\begin{normaltext}
	En particulier, pour la topologie de la norme maximum, la convergence d'une suite implique la convergence «composante par composante» par la proposition~\ref{PROPooNRRIooCPesgO}.
\end{normaltext}


%+++++++++++++++++++++++++++++++++++++++++++++++++++++++++++++++++++++++++++++++++++++++++++++++++++++++++++++++++++++++++++
\section{Topologie réelle en dimension $n$}
%+++++++++++++++++++++++++++++++++++++++++++++++++++++++++++++++++++++++++++++++++++++++++++++++++++++++++++++++++++++++++++

Nous considérons sur \( \eR\) la topologie donnée par la valeur absolue, et sur \( \eR^n\) celle de la topologie produit ou du maximum, qui sont identiques par le lemme \ref{LEMooFQMSooLmdIvD}.

En particulier, nous n'avons pas encore la norme donnée par \( \| x \|=\sqrt{ \sum_ix_i^2 }\), parce qu'elle demande la racine carré, définie en \ref{DEFooGQTYooORuvQb}.

%---------------------------------------------------------------------------------------------------------------------------
\subsection{Ouverts et fermés}
%---------------------------------------------------------------------------------------------------------------------------

\begin{proposition}\label{PROPooEQYJooBbPiAj}
	Une partie \( A\) de \( \eR^n\) est ouverte si et seulement si pour tout \( a\in A\) il existe \( r>0\) tel que \( B(a,r)\subset A\).
\end{proposition}
Cette proposition est évidemment à mettre en rapport avec le théorème~\ref{ThoPartieOUvpartouv}.

\begin{lemma}   \label{LemMESSExh}
	Pour tout $x \in \eR^n$ et tout $r >0$ la boule\footnote{Définition \ref{ThoORdLYUu}.} \( B(x,r)\) est ouverte.
\end{lemma}

\begin{proof}
	Afin de prouver que la boule est ouverte, nous prenons un point $p\in B(x,r)$, et nous allons montrer qu'il existe une boule autour de $p$ qui est contenue dans $B(x,r)$.

	Étant donné que $p\in B(x,r)$, nous avons $d(p,x)<r$. Prouvons que la boule $B\big(p,r-d(p,x)\big)$ est contenue dans $B(x,r)$. Pour cela, nous prenons $p'\in B\big(p,r-d(p,x)\big)$, et nous essayons de prouver que $p'\in B(x,r)$. En effet, en utilisant l'inégalité triangulaire,
	\begin{equation}
		d(x,p')\leq d(x,p)+d(p,p')\leq d(x,p)+r-d(p,x)=r.
	\end{equation}
\end{proof}

%---------------------------------------------------------------------------------------------------------------------------
\subsection{Intérieur, adhérence et frontière}
%---------------------------------------------------------------------------------------------------------------------------

\begin{normaltext}
	La notion d'intérieur est déjà faite dans la définition \ref{DEFooSVWMooLpAVZRInt}.

	La notion d'adhérence a déjà été définie en~\ref{DEFooSVWMooLpAVZR}, et précisé par le lemme~\ref{LEMooILNCooOFZaTe}. Dans le cas de \( \eR^n\) dans lequel les boules forment une base de la topologie nous pouvons encore préciser de la façon suivante:
	\begin{equation}
		x \in \Adh A \iffdefn \forall \epsilon > 0, B(x,\epsilon) \cap A \neq \emptyset
	\end{equation}
\end{normaltext}

\begin{proposition}
	Pour $A \subset \eR^n$, nous avons
	\begin{equation*}
		\Int A \subseteq A  \subseteq \Adh A
	\end{equation*}
\end{proposition}

\begin{definition}      \label{DEFooACVLooRwehTl}
	La \defe{frontière}{frontière} ou le \defe{bord}{bord} de $A$ est défini par $\partial A = \Adh A \setminus \Int A$. 
\end{definition}

\begin{lemma}       \label{LEMooMPZWooGrqYIX}
    Une partie $A$ d'un espace topologique est ouverte si $A = \Int A$, et fermée si $A = \Adh A$.
\end{lemma}

\begin{lemma}[Caractérisation équivalente de la frontière]      \label{LEMooEUYEooYcUfKr}
	Soient \( X\) un espace topologique et \( S\subset X\). Un point \( x\in X\) est dans \( \partial S\) si et seulement si tout voisinage de \( x\) contient un point de \( S\) et un point de \( S^c\).
\end{lemma}

\begin{proof}
	Supposons que tout voisinage de \( x\) contienne un point de \( S\) et un point de \( S^c\). Alors \( x\in Adh(S)\) (définition~\ref{DEFooSVWMooLpAVZR}), mais pas dans l'intérieur de \( S\) parce que \( x\) ne possède pas de voisinage contenu dans \( S\). Donc \( x\in \partial S\).

	À l'inverse, si \( x\in\partial S\) alors \( x\) est dans l'adhérence de \( S\) et tout voisinage de \( x\) contient un point de \( S\). Mais \( x\) n'est pas dans l'intérieur de \( S\) et tout voisinage de \( x\) contient un point qui n'est pas dans \( S\), aka un point de \( S^c\).
\end{proof}

\begin{corollary}
	Un ensemble et son complémentaire ont même frontière.
\end{corollary}

\begin{proof}
	Conséquence du lemme~\ref{LEMooEUYEooYcUfKr}. Les points de \( \partial(S^c)\) sont caractérisés par le fait que tout voisinage contient un point de \( S^c\) et un point de \( (S^c)^c=S\).
\end{proof}

\begin{example}
	Soit \( X=\mathopen[ 0 , 1 \mathclose]\) muni de la topologie de la distance \( | x-y |\) (définition~\ref{ThoORdLYUu}). Les points \( 0\) et \( 1\) \emph{ne sont pas} dans la frontière de $X$. En effet une boule ouverte autour de \( 1\) est un ensemble de la forme
	\begin{equation}
		B(1,r)=\{ x\in X\tq | x-1 |<r \}=\mathopen] 1-r , 1 \mathclose]
	\end{equation}
	où nous avons supposé \( r<1\).

	Les points \( 0\) et \( 1\) sont par contre sur la frontière de \( \mathopen[ 0 , 1 \mathclose]\) lorsque cet ensemble est vu comme partie de l'espace métrique \( \eR\).
\end{example}

\begin{lemma}[Passage de douane\cite{ooDKEWooFqlDyN,ooWBUCooAdPjMK}]        \label{LEMooLKWEooItGnkP}
	Dans un espace topologique, toute partie connexe qui rencontre à la fois une partie \( A\) et son complémentaire rencontre nécessairement la frontière de \( A\).
\end{lemma}

\begin{proof}
	Nommons \( \gamma\) la partie connexe qui intersecte \( A\) et \( A^c\). Les ouverts \( \Int(A)\) et \( X\setminus \bar A\) ne peuvent pas recouvrir \( \gamma\) parce que ce sont deux ouverts disjoints alors que \( \gamma\) est connexe (voir la définition~\ref{DefIRKNooJJlmiD} de la connexité). Donc \( \gamma\) doit contenir des points qui sont dans \( \bar A\) mais pas dans \( \Int(A)\). C'est-à-dire des points de \( \partial A\).
\end{proof}

On vérifiera que les notations et les dénominations sont cohérentes en prouvant la proposition suivante.
\begin{proposition}Pour $\epsilon > 0$,
	\begin{enumerate}
		\item l'adhérence de $B(x,\epsilon)$ est $\bar B(x,\epsilon)$,
		\item l'intérieur de $\bar B(x,\epsilon)$ est $B(x,\epsilon)$,
		\item la boule ouverte $B(x,\epsilon)$ est un ouvert,
		\item la boule fermée $\bar B(x,\epsilon)$ est un fermé.
	\end{enumerate}
\end{proposition}

Nous avons également les liens suivants entre intérieur, adhérence, ouvert, fermé et passage au complémentaire (noté ${}^c$)~:
\begin{proposition}
	Si $A \subset \eR^n$ et $A^c = \eR^n\setminus A$, nous
	avons
	\begin{enumerate}
		\item $(\Int A)^c = \Adh (A^c)$ et $(\Adh A)^c = \Int
			      (A^c)$,
		\item $A$ est ouvert si et seulement si $A^c$ est fermé,
		\item $\Int A$ est le plus grand ouvert contenu dans $A$,
		\item $\Adh A$ est le plus petit fermé contenant $A$,
	\end{enumerate}
\end{proposition}

\begin{example} \label{ExBFLooUNyvbw}
	Il n'est en général pas vrai que \( \overline{ A\cap B }=\bar A\cap \bar B\). Par exemple si \( A=\mathopen[ 0 , 1 [\) et \( B=\mathopen] 1 , 2 \mathclose]\). Dans ce cas, \( A\cap B=\emptyset\) alors que \( \bar A\cap\bar B=\{ 1 \}\).
\end{example}

%---------------------------------------------------------------------------------------------------------------------------
\subsection{Point d'accumulation, point isolé}
%---------------------------------------------------------------------------------------------------------------------------

Les définitions de point d'accumulation et de point isolé sont \ref{DEFooGHUUooZKTJRi} et \ref{DEFooXIOWooWUKJhN}. Nous voyons maintenant ce que ces définitions donnent dans le cas de l'espace topologique \( \eR\).

\begin{lemma}
	Soit $D\subset\eR$. Un point $a\in D$ est isolé dans $D$ si et seulement si il existe $\varepsilon>0$ tel que
	\begin{equation}
		\mathopen[ a-\varepsilon , a+\varepsilon \mathclose]\cap D=\{ a \}.
	\end{equation}
	Autrement dit, il existe un intervalle autour de $a$ dans lequel $a$ est le seul élément de $D$.
\end{lemma}

\begin{lemma}
	Un point $a\in \eR$ est un point d'accumulation de $D$ si pour tout $\varepsilon>0$,
	\begin{equation}
		\Big( \mathopen[ a-\varepsilon , a+\varepsilon \mathclose]\setminus\{ a \} \Big)\cap D\neq\emptyset.
	\end{equation}
	Autrement dit, quel que soit l'intervalle autour de  $a$ que l'on considère, le point $a$ n'est pas tout seul dans $D$.
\end{lemma}

\begin{example}
	Prenons $D=\mathopen[ 0 , 1 [\cup\mathopen] 2 , 3 \mathclose]$. Cet ensemble n'a pas de point isolé, et l'ensemble de ses points d'accumulation est $\mathopen[ 0 , 1 \mathclose]\cup\mathopen[ 2,3  \mathclose]$.

	Notez que les points $1$ et $2$ sont des points d'accumulation de $D$ qui ne font pas partie de $D$. Il est possible d'être un point d'accumulation de $D$ sans être dans $D$, mais pour être un point isolé dans $D$, il faut être dans $D$.
\end{example}

\begin{example}
	Soit $D=\{ \frac{1}{ n }\}_{n\in\eN}$. Tous les points de cet ensemble sont des points isolés (vérifier !).  Aucun point de $D$ n'est point d'accumulation. Cependant $0$ est un point d'accumulation.
\end{example}

\begin{example}     \label{EXooWOYQooJolaTV}
	Soit \( D=\mathopen] 1 , 2 \mathclose[\cup\{ 12 \}\). Le point \( 12\) est adhérence, mais pas d'accumulation parce que le voisinage \( \mathopen] 9 , 14 \mathclose[\) n'intersectionne pas \( D\setminus \{ 12 \}\).
\end{example}

%---------------------------------------------------------------------------------------------------------------------------
\subsection{Limite de suite}
%---------------------------------------------------------------------------------------------------------------------------

\begin{definition}[Limite d'une suite dans $\eR^m$]
	Une suite de points $(x_n)$ dans $\eR^m$ est dite \defe{convergente}{convergence!suite dans $\eR^m$} si il existe un élément $\ell\in\eR^m$ tel que
	\begin{equation}	\label{EqCondLimSuite}
		\forall\varepsilon>0,\,\exists N\in \eN\tq\,\forall n\geq N,\,\| x_n-\ell \|<\varepsilon.
	\end{equation}
	Dans ce cas, nous disons que $\ell$ est la \defe{limite}{limite!suite dans $\eR^m$} de la suite $(x_n)$ et nous écrivons $\lim x_n=\ell$ ou plus simplement $x_n\to \ell$.
\end{definition}
Notez aussi la similarité avec la définition~\ref{PropLimiteSuiteNum}.

\begin{remark}
	Nous n'écrivons pas «$\lim_{n\to\infty}x_n$» parce que, lorsqu'on parle de suites, la limite est \emph{toujours} lorsque $n$ tend vers l'infini. Il n'y a aucun intérêt à chercher par exemple $\lim_{n\to 4}x_n$ parce que cela vaudrait $x_4$ et rien d'autre.

	Ceci est une différence importante avec les limites de fonctions.
\end{remark}

\begin{lemma}[Unicité de la limite]
	Il ne peut pas y avoir deux nombres différents qui satisfont à la condition \eqref{EqCondLimSuite}. En d'autres termes, si $\ell$ et $\ell'$ sont deux limites de la suite $(x_n)$, alors $\ell=\ell'$.
\end{lemma}

\begin{proof}
	Soit $\varepsilon>0$. Nous considérons $N$ tel que
	\begin{equation}
		\| x_n-\ell \|<\varepsilon
	\end{equation}
	pour tout $n\geq N$, et $N'>0$ tel que
	\begin{equation}
		\| x_n-\ell' \|<\epsilon
	\end{equation}
	pour tout $n>N'$. Maintenant, nous prenons $n$ plus grand que $N$ et $N'$ de telle façon que les deux équations pour $x_n$ soient vérifiées en même temps. Alors
	\begin{equation}
		\| \ell-\ell' \|=\| \ell-x_n+x_n-\ell' \|\leq\| \ell-x_n \|+\| x_n-\ell' \|<2\varepsilon.
	\end{equation}
	Cela prouve que $\| \ell-\ell' \|=0$.
\end{proof}
Le théorème de Bolzano-Weierstrass~\ref{ThoBWFTXAZNH} dit que dans le cas métrique, la compacité séquentielle est équivalente à la compacité.

%TODO : le théorème sur l'équivalence des normes sur les espaces vectoriels normés devrait être énoncé comme le fait que si N1 et N2 sont deux normes sur V, alors
%       nous avons un isomorphisme d'espace topologique (V,N1) ~ (V,N2). L'isomorphisme étant donné par l'identité.
