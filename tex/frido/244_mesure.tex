% This is part of Mes notes de mathématique
% Copyright (c) 2011-2020, 2023-2025
%   Laurent Claessens, Carlotta Donadello
% See the file fdl-1.3.txt for copying conditions.


%+++++++++++++++++++++++++++++++++++++++++++++++++++++++++++++++++++++++++++++++++++++++++++++++++++++++++++++++++++++++++++
\section{Mesure à densité}
%+++++++++++++++++++++++++++++++++++++++++++++++++++++++++++++++++++++++++++++++++++++++++++++++++++++++++++++++++++++++++++


%---------------------------------------------------------------------------------------------------------------------------
\subsection{Théorème de Radon-Nikodym}
%---------------------------------------------------------------------------------------------------------------------------

\begin{definition}[\cite{PersoFeng}]
	Soient \( \mu\) et \( \nu\) deux mesures sur l'espace mesurable \( (\Omega,\tribA)\). Nous disons que la mesure \( \mu\) est \defe{dominée}{dominée!mesure} par \( \nu\) si pour tout ensemble mesurable \( A\), \( \nu(A)=0\) implique \( \mu(A)=0\).

	Nous disons que la mesure signée\footnote{Définition \ref{DefBTsgznn}.} \( \nu\) est \defe{absolument continue}{mesure absolument continue} par rapport à la mesure positive \( \mu\) si pour tout \( A\in \mA\), nous  \( \mu(A)=0\,\Rightarrow \nu(A)=0\). Dans ce cas nous notons
	\begin{equation}
		\nu\ll \mu.
	\end{equation}
	\nomenclature[Y]{\( \mu\ll\nu\)}{La mesure \( \mu\) est absolument continue par rapport à la mesure \( \nu\)}
\end{definition}

\begin{definition}[\cite{BIBooQMABooVJScYf}]		\label{DEFooRZARooTbtJac}
	Deux mesures signées \( \mu\) et \( \nu\) sur \( (\Omega,\tribA)\) sont \defe{mutuellement singulières}{mesures mutuellement singulières} si il existe \( E,F\in \tribA\) tels que
	\begin{subequations}
		\begin{numcases}{}
			E\cap F=\emptyset\\
			\mu(A)=\mu(A\cap E)\\
			\nu(A)=\nu(A\cap F).
		\end{numcases}
	\end{subequations}
	Nous notons ce fait par
	\begin{equation}
		\mu\perp\nu.
	\end{equation}
	\nomenclature[Y]{\( \mu\perp\nu\)}{mesures mutuellement singulières}
\end{definition}

\begin{lemma}[\cite{MonCerveau}]		\label{LEMooDIGPooJqEwGl}
	Soient deux mesures positives \( \lambda\) et \( \mu\). Nous avons \( \lambda\perp\mu \) si et seulement si il existe des parties mesurables \( E'\), \( F'\) telles que
	\begin{enumerate}
		\item
		      $E'\cup F'=\Omega$
		\item
		      \( E'\cap F'=\emptyset\)
		\item
		      \( \mu(F')=\lambda(E')=0\).
	\end{enumerate}
\end{lemma}

\begin{proof}
	En deux parties.
	\begin{subproof}
		\spitem[\( \Rightarrow\)]
		%-----------------------------------------------------------
		Nous supposons que \( \mu\perp\lambda\), c'est-à-dire qu'il existe \( E,F\) tels que \( E\cap F=\emptyset\), \( \mu(A)=\mu(A\cap E)\) et \( \lambda(A)=\lambda(A\cap F)\) pour tout mesurables \( A\).

		Nous posons \( R=\Omega\setminus(E\cup F)\). C'est la partie de \( \Omega\) sur laquelle ni \( \mu\) ni \( \lambda\) n'agit. Il suffit de poser \( F'=F\cup R\) et \( E'=E\) pour avoir ce qu'il faut. Par exemple,
		\begin{equation}
			\mu(F')=\mu(F\cup R)=\mu\big( (F\cup R)\cap E \big)=\mu(\emptyset)=0.
		\end{equation},
		et de même pour les autres vérifications.

		\spitem[\( \Leftarrow\)]
		%-----------------------------------------------------------
		Il suffit de poser \( E=E'\) et \( F=F'\) et de vérifier. Par exemple, vu que l'union \( E\cup F=\Omega\) est disjointe,
		\begin{equation}
			\mu(A)=\mu(A\cap E)+\mu(A\cap F)=\mu(A\cap E).
		\end{equation}
	\end{subproof}
\end{proof}

\begin{lemma}[\cite{BIBooZJADooSngWXY}]		\label{LEMooVOJDooFgbwSE}
	Soit un espace mesurable \( (\Omega,\tribA)\). Nous considérons
	\begin{enumerate}
		\item
		      des mesures positives finies \( (\lambda_n)_{n\in \eN}\),
		\item
		      une mesure positive \( \mu\)
	\end{enumerate}
	telles que \( \lambda_n\perp\mu\) pour tout \( n\). Alors
	\begin{equation}
		\sum_{n=1}^{\infty}\lambda_n\perp\mu.
	\end{equation}
\end{lemma}

\begin{proof}
	Nous utilisons la caractérisation du lemme \ref{LEMooDIGPooJqEwGl} pour des mesures mutuellement singulières.

	Pour chaque \( n\) nous avons des mesurables \( A_n\) et \( B_n\) tels que \( A_n\cup B_n=\Omega\), \( A_n\cap B_n=\emptyset\) et \( \mu(A_n)=\lambda_n(B_n)=0\).

	Nous posons \( A=\bigcup_{n=1}^{\infty}A_n\) et \( B=\Omega\setminus A\). Nous avons bien entendu \( A\cup B=\Omega\) et \( A\cap B=\emptyset\).
	\begin{subproof}
		\spitem[\( \lambda_k(B)=0\) pour tout \( k\)]
		%-----------------------------------------------------------
		Pour chaque \( k\) nous avons \( B\subset B_k\) parce que si \( x\in B\), alors en particulier \( x\in \Omega\setminus A_k=B_k\). Nous avons donc
		\begin{equation}
			\Omega\setminus A=B\subset B_k=\Omega\setminus A_k.
		\end{equation}
		Donc, par croissance de la mesure,
		\begin{equation}
			\lambda_k(B)=\lambda_k(\Omega\setminus A)\leq \lambda_k(\Omega\setminus A_k)=\lambda_k(B_k)=0.
		\end{equation}

		\spitem[\( \lambda(B)=0\)]
		%-----------------------------------------------------------
		Il s'agit de sommer : \( \lambda(B)=\sum_{n=1}^{\infty}\lambda_k(B)=0\).

		\spitem[\( \mu(A)=0\)]
		%-----------------------------------------------------------
		Nous avons
		\begin{equation}
			\mu(A)=\mu\Big( \bigcup_{n=1}^{\infty}A_n \Big)\leq \sum_{n=1}^{\infty}\mu(A_n)=0.
		\end{equation}
	\end{subproof}
\end{proof}

\begin{lemma}[\cite{MonCerveau}]		\label{LEMooYYIIooCUfGxA}
	Soient \( \lambda\) et \( \nu\) des mesures positives \( \sigma\)-finies ainsi qu'une mesure positive \( \mu\) telle que \( \lambda\perp \mu\) et \( \nu\perp \mu\). Alors
	\begin{equation}
		(\lambda+\nu)\perp\mu.
	\end{equation}
\end{lemma}

\begin{proof}
	Nous considérons des mesurables disjoints \( (X_n)\) comme dans le lemme \ref{LEMooXOPMooMmIbmD}. Pour chaque \( n\) nous avons les mesures finies \( \lambda_n\) et \( \nu_n\) données par
	\begin{subequations}
		\begin{align}
			\lambda_n(A) & =\lambda(A\cap X_n) \\
			\nu_n(A)     & =\nu(A\cap X_n).
		\end{align}
	\end{subequations}
	Ces mesures vérifient \( \lambda_n\perp \mu\) et \( \nu_n\perp \mu\). De plus les \( (X_n)\) étant disjoints, nous avons \( \lambda(A)=\sum_n\lambda_n(A)\) et \( \nu(A)=\sum_n\nu_n(A)\). Nous pouvons donc utiliser le lemme \ref{LEMooVOJDooFgbwSE} :
	\begin{equation}		\label{EQooWXCKooGrqShO}
		\sum_{n=1}^{\infty}(\lambda_n+\nu_n)\perp\mu.
	\end{equation}
	Par associativité de la somme (tous les termes sont positifs), pour chaque mesurable \( A\) nous avons
	\begin{equation}
		\sum_n(\lambda_n+\nu_n)(A)=\sum_n\lambda_n(A)+\sum_n\nu_n(A)=\lambda(A)+\nu(A)=(\lambda+\nu)(A).
	\end{equation}
	Cela pour dire que le membre de gauche de \eqref{EQooWXCKooGrqShO} est bien \( \lambda+\nu\).
\end{proof}


\begin{definition}
	La mesure \( \mu\) est \defe{portée}{portée!mesure} par la partie \( E\in\tribA\) si pour tout \( A\in\tribA\),
	\begin{equation}
		\mu(A)=\mu(A\cap E).
	\end{equation}
\end{definition}

\begin{definition}
	Soit une mesure signée \( \mu\) et une partie mesurable \( A\).
	\begin{enumerate}
		\item
		      La partie \( A\) est \defe{négative}{partie négative} si \( \mu(A')\leq 0\) pour tout \( A'\) mesurable dans \( A\).
		\item
		      La partie \( A\) est \defe{nulle}{partie nulle} si \( \mu(A')= 0\) pour tout \( A'\) mesurable dans \( A\).
		\item
		      La partie \( A\) est \defe{positive}{partie positive} si \( \mu(A')\geq0\) pour tout \( A'\) mesurable dans \( A\).
	\end{enumerate}
\end{definition}


\begin{proposition}[\cite{BIBooQMABooVJScYf}]		\label{PROPooBLNXooRRxxVv}
	Soit un espace mesuré \( (\Omega,\tribA,\mu)\) où \( \mu\) est une mesure signée. Si une partie \( D\in \tribA\) vérifie \( \mu(D)\leq 0\), alors il existe une partie négative \( A\subset D\) telle que \( \mu(A)\leq \mu(D)\).
\end{proposition}

\begin{proof}
	Nous définissons des parties \( A_i\) par récurrence en commençant par poser \( A_0=D\). Ensuite, si \( A_n\) est défini, nous posons
	\begin{equation}
		t_n=\sup\{ \mu(B)\tq B\subset A_n, B\in\tribA \}.
	\end{equation}
	Nous avons \( t_n\geq \mu(\emptyset)=0\). Par définition du supremum, il existe \( B_n\) mesurable dans \( A_n\) vérifiant \( \mu(B_n)\geq t_n/2\geq 0\). Nous posons alors
	\begin{equation}
		A_{n+1}=A_n\setminus B_n.
	\end{equation}
	Enfin nous posons
	\begin{equation}
		A=D\setminus \bigcup_{n\geq 0}B_n.
	\end{equation}
	Vérifions que \( A\) possède les propriétés demandées.
	\begin{subproof}
		\spitem[\( A\subset D\)]
		%-----------------------------------------------------------
		Par définition.
		\spitem[\( \mu(A)\leq \mu(D)\)]
		%-----------------------------------------------------------
		\begin{subproof}
			\spitem[Les \( B_i\) sont disjoints]
			%-----------------------------------------------------------
			Nous avons
			\begin{equation}
				B_n\subset A_n= A=D\setminus \bigcup_{n\geq 0}B_n,
			\end{equation}
			et donc \( B_n\cap B_k=\emptyset\) pour tout \( k<n\). Cela montre que les \( B_i\) sont disjoints.
			\spitem[\( \mu(A)\leq \mu(D)\)]
			%-----------------------------------------------------------
			Étant donné que \( B_n\subset D\) pour tout \( n\), l'égalité \( A=D\setminus \bigcup_{n\geq 0} B_n\) peut être écrite
			\begin{equation}
				D=A\cup\bigcup_{n\geq 0}B_n
			\end{equation}
			où à droite nous avons une union disjointe. Bref, au niveau des mesures nous avons
			\begin{equation}
				\mu(A)=\mu(D)-\sum_{n\geq 0}\mu(B_n)\leq \mu(D)
			\end{equation}
			parce que \( \mu(B_n)\geq 0\) pour tout \( n\).
		\end{subproof}
		\spitem[\( A\subset A_n\)]
		%-----------------------------------------------------------
		Par définition \( A=D\setminus \bigcup_{n\geq 0}B_n\) alors que \( A_n=D\setminus\bigcup_{n=0}^{n-1}B_n\). Donc en partant de \( D\), pour construire \( A\), on enlève plus de choses que pour construire \( A_n\). Donc \( A\subset A_n\).
		\spitem[\( A\) est une partie négative]
		%-----------------------------------------------------------
		Supposons que \( A\) n'est pas une partie négative : il existe \( A'\) mesurable dans \( A\) tel que \( \mu(A')>0\). En particulier,
		\begin{equation}
			A'\subset A\subset A_n.
		\end{equation}
		La partie \( A'\) est donc dans l'ensemble sur lequel on prend le supremum pour construire \( t_n\). En d'autres termes, \( t_n\geq \mu(A')>0\), et donc
		\begin{equation}
			\mu(B_n)\geq \frac{ t_n }{ 2 }\geq\frac{ \mu(A') }{2}.
		\end{equation}
		Et avec tout ça,
		\begin{equation}		\label{EQooYNJFooKJCfFT}
			\mu(A)=\mu(D)-\sum_{n\geq 0}\mu(B_n)\leq \mu(D)-\sum_{n\geq 0}\frac{ \mu(A) }{2}=-\infty.
		\end{equation}
		\spitem[Conclusion]
		%-----------------------------------------------------------
		Nous avons dit, dans la définition d'une mesure négative que \( \mu\) ne prenait jamais la valeur \( -\infty\). L'inégalité \eqref{EQooYNJFooKJCfFT} est donc une contradiction et nous concluons que \( A\) est une partie négative pour \( \mu\).
	\end{subproof}
\end{proof}

\begin{lemma}[\cite{BIBooQMABooVJScYf}]		\label{LEMooVFEWooOyXcPS}
	Soient une mesure positive \( \mu\) et une mesure signée \( \nu\). Si \( \nu\perp\mu\) et si \( \nu\ll \mu\), alors \( \nu=0\)
\end{lemma}

\begin{proof}
	Vu que \( \mu\perp\nu\), il existe \( E\in\tribA\) tel que
	\begin{subequations}
		\begin{numcases}{}
			\mu(A)=\mu(A\cap E)\\
			\nu(A)=\nu(A\cap E^c)
		\end{numcases}
	\end{subequations}
	pour tout \( A\in \tribA\).

	Soit \( A\in\tribA\). Nous avons \( \mu(A\cap E^c)=0\) parce que \( \mu(A)=\mu(A\cap E)+\mu(A\cap E^c)\) et \( \mu(A\cap E)=\mu(A)\). Étant donné que \( \nu\ll\mu\) nous avons aussi \( \nu(A\cap E^c)=0\). Donc
	\begin{equation}
		\nu(A)=\nu(A\cap E^c)=0.
	\end{equation}
\end{proof}

\begin{lemma}		\label{LEMooABVVooMvxlRo}
	Soient une mesure positive \( \mu\) ainsi que deux mesures signées \( \nu_1,\nu_2\) telles que \( \nu_1\perp\mu\) et \( \nu_2\perp \mu\). Alors \( \nu_1+\nu_2\perp\mu\).
\end{lemma}

\begin{proof}
	Nous avons des parties \( E_1,F_1,E_2,F_2\in \tribA\) telles que
	\begin{subequations}
		\begin{numcases}{}
			\nu_1(A)=\nu_1(A\cap F_1)\\
			\mu(A)=\mu(A\cap E_1),
		\end{numcases}
	\end{subequations}
	et
	\begin{subequations}
		\begin{numcases}{}
			\nu_2(A)=\nu_2(A\cap F_2)\\
			\mu(A)=\mu(A\cap E_2),
		\end{numcases}
	\end{subequations}
	et \( E_1\cap F_1=\emptyset\) et \( E_2\cap F_2=\emptyset\). Nous posons \( E_0=E_1\cap E_2\) et \( F_0=F_1\cup F_2\), et nous allons prouver que \( \nu_1+\nu_2\perp \mu\) vérifiant que le couple \( (E_0,F_0)\) vérifie les conditions de la définition \ref{DEFooRZARooTbtJac}.

	\begin{subproof}
		\spitem[\( E_0\cap F_0=\emptyset\)]
		%-----------------------------------------------------------
		Nous avons :
		\begin{equation}
			E_0\cap F_0=(E_1\cap E_2)\cap(F_1\cup F_2)=(E_1\cap E_2\cap F_1)\cup(E_1\cap E_2\cap F_2)=\emptyset.
		\end{equation}

		\spitem[\( \mu(A)=\mu(A\cap E_0)\)]
		%-----------------------------------------------------------
		Nous prouvons maintenant que \( \mu(A)=\mu(A\cap E_0)\) pour tout \( A\in\tribA\). Nous écrivons d'abord
		\begin{equation}	\label{EQooTMOKooEulFmx}
			\mu(A)=\mu(A\cap E_1)=\mu(A\cap E_1\cap E_2)+\mu\big( A\cap(E_1\setminus E_2) \big),
		\end{equation}
		et, de la même façon,
		\begin{equation}
			\mu(A)=\mu(A\cap E_2)=\mu(A\cap E_2\cap E_1)+\mu\big( A\cap(E_2\setminus E_1) \big).
		\end{equation}
		En faisant la différence entre les deux, \( 0=\mu\big( A\cap(E_1\setminus E_2) \big)-\mu\big( A\cap(E_2\setminus E_1) \big)\), c'est-à-dire
		\begin{equation}
			\mu\big( A\cap(E_1\setminus E_2) \big)=\mu\big( A\cap(E_2\setminus E_1) \big).
		\end{equation}
		En particulier pour \( A=\Omega\) nous avons
		\begin{equation}
			\mu(E_1\setminus E_2)=\mu(E_2\setminus E_1).
		\end{equation}
		Nous savons, par définition de \( E_1\) que \( \mu(A)=\mu(A\cap E_1)\) pour tout \( A\in\tribA\). Nous écrivons cette égalité pour \( A=(E_2\setminus E_1)\) :
		\begin{equation}
			\mu(E_1\setminus E_2)=\mu\big( (E_2\setminus E_1)\cap E_1 \big)=\mu(\emptyset)=0.
		\end{equation}
		Nous en déduisons que \( \mu(E_2\setminus E_1)=0\) et donc que \( \mu\big( A\cap(E_2\setminus E_1) \big)=0\) pour tout \( A\in\tribA\). Et enfin nous repartons de \eqref{EQooTMOKooEulFmx} :
		\begin{equation}
			\mu(A)=\mu(A\cap E_1)=\underbrace{\mu(A\cap E_1\cap E_2)}_{=\mu(A\cap E_0)}+\underbrace{\mu\big( A\cap(E_1\setminus E_2) \big)}_{=0},
		\end{equation}
		autrement dit, \( \mu(A)=\mu(A\cap E_0)\), comme nous devions prouver.

		\spitem[\( \nu_i(A)=\nu_i(A\cap F_0)\)]
		%-----------------------------------------------------------
		Nous allons prouver que pour \( i=1,2\) nous avons \( \nu_i(A)=\nu_i(A\cap F_0)\). Nous commençons par écrire \( A\cap F_0\) comme union disjointe en partant de la proposition \ref{LEMooHPUNooPmViwi} :
		\begin{subequations}
			\begin{align}
				A\cap F_0 & =(A\cap F_1)\cup (A\cap F_2)                                                       \\
				          & =(A\cap F_1)\cup\big( (A\cap F_2)\setminus (A\cap F_1) \big). & \text{cf. justif.}
			\end{align}
		\end{subequations}
		Justification : pour tout ensembles \( X\) et \( Y\) nous avons \( X\cup Y=X\cup(Y\setminus X)\).

		Ayant écrit \( A\cap F_0\) comme une union disjointe, nous pouvons calculer \( \nu_1\) dessus. Nous posons \( R= (A\cap F_2)\setminus (A\cap F_1) \), et nous calculons un peu :
		\begin{equation}
			\nu_1(A\cap F_0)=\nu_1(A\cap F_1)+\nu_1(R)=\nu_1(A)+\nu_1(R\cap F_1)=\nu_1(A)+\nu_1(\emptyset)=\nu_1(A).
		\end{equation}
		Le même jeu en permutant les rôles de \( 1\) et \( 2\) donne \( \nu_2(A\cap F_0)=\nu_2(A)\).
		\spitem[Conclusion]
		%-----------------------------------------------------------
		Nous avons \( \nu_1(A)=\nu_1(A\cap F_0)\) et \( \nu_2(A)=\nu_2(A\cap F_0)\). En faisant la somme, c'est bon.
	\end{subproof}
\end{proof}

\begin{propositionDef}[Décomposition de Hahn\cite{BIBooQMABooVJScYf}]		\label{PROPooIBLHooTMfEJW}
	Soit une mesure signée \( \mu\) sur \( (\Omega,\tribA)\). Il existe des parties mesurables \( N,P\subset \Omega\) telles que
	\begin{enumerate}
		\item
		      \( N\cup P=\Omega\),
		\item
		      \( N\cap P=\emptyset\),
		\item
		      \( N\) est négatif pour \( \mu\),
		\item
		      \( P\) est positive pour \( \mu\).
	\end{enumerate}

	Un tel couple \( (N,P)\) est une \defe{décomposition de Hahn}{décomposition de Hahn} pour \( \mu\).

	De plus si \( (N,P)\) et \( (N',P')\) sont deux décompositions de Hahn pour \( \mu\), alors les parties\footnote{La partie \( A\Delta B\) est la différence symétrique (définition \ref{DefBMLooVjlSG}). Dire que \( N\Delta N'\) est nulle pour \( \mu\) signifie que pratiquement tout est dans l'intersection. Nous avons donc unicité de la décomposition à partie nulle près.} \( N\Delta N'\) et \( P\Delta P'\) sont nulles pour \( \mu\).
\end{propositionDef}

\begin{proof}

	D'abord l'existence puis l'«unicité».

	\begin{center}
		Existence
	\end{center}

	Nous posons d'abord \( N_0=\emptyset\), et nous construisons des parties \( N_k\) par récurrence. Nous posons
	\begin{equation}
		s_n=\inf\{ \mu(D)\tq D\subset \Omega\setminus N_n,D\in\tribA \}.
	\end{equation}
	Notes : il est possible d'avoir \( s_n=-\infty\), et comme \( D=\emptyset\) fonctionne dans l'infimum, nous avons \( s_n\leq 0\). Vu que \( s_n\) est un infimum et qu'il est négatif, il existe une partie mesurable \( D_n\subset \Omega\setminus N_n\) telle que \( \mu(D_n)\leq \frac{ s_n }{2}\). Pour cette partie nous avons
	\begin{equation}
		\mu(D_n)\leq \frac{ s_n }{2}\leq 0.
	\end{equation}
	La proposition \ref{PROPooBLNXooRRxxVv} dit qu'il existe une partie mesurable \( A_n\subset D_n\) qui est négative pour \( \mu\) et qui vérifie \( \mu(A_n)\leq \mu(D_n)\). Nous posons alors, pour notre récurrence :
	\begin{equation}
		N_{n+1}=N_n\cup A_n.
	\end{equation}
	Nous posons ensuite
	\begin{subequations}
		\begin{numcases}{}
			N=\bigcup_{n\geq 0}A_n\\
			P=\Omega\setminus N,
		\end{numcases}
	\end{subequations}
	et nous prouvons que \( (N,P)\) fonctionne. Nous avons immédiatement que \( N\cup P=\Omega\) et que \( N\cap P=\emptyset\).

	\begin{subproof}
		\spitem[\( N\) est négative pour \( \mu\)]
		%-----------------------------------------------------------
		Nous avons, pour tout \( n\) que \( N_n=\emptyset\cup A_1\cup\ldots\cup A_{n-1}\). Donc \( N_n\subset N\) pour tout \( n\geq 1\).

		Notons aussi que les \( A_n\) sont disjoints. En effet
		\begin{equation}
			A_n\subset D_n\subset \Omega\setminus N_n=\Omega\setminus(A_1\cup\ldots \cup A_{n-1}).
		\end{equation}
		Donc \( A_n\) n'intersecte aucun des \( A_k\) avec \( k<n\).

		Nous pouvons maintenant que \( N\) est une partie négative. Soit \( B\subset N\). Notons que chaque \( A_n\) est négatif, donc \( \mu(A_n\cap B)\leq 0\) parce que, évidemment, \( A_n\cap B\subset B\). Étant donné que l'union \( N=\bigcup_{n\geq 0}A_n\) est disjointe, nous avons
		\begin{equation}
			\mu(B)=\sum_{n\geq 0}\mu(B\cap A_n)\leq 0.
		\end{equation}
		Donc \( m(B)\leq 0\) et nous avons montré que \( N\) est une partie négative.

		\spitem[\( P\) est positive pour \( \mu\)]
		%-----------------------------------------------------------
		Nous y allons par l'absurde. Supposons avoir \( D\subset P\) avec \( \mu(D)<0\). Nous avons déjà vu que \( N_n\subset N\), donc $\Omega\setminus N\subset \Omega\setminus N_n$. En ce qui concerne \( D\),
		\begin{equation}
			D\subset P=\Omega\setminus N\subset \Omega\setminus N_n.
		\end{equation}
		Donc \( s_n\leq \mu(D)<0\). Nous avons alors le calcul
		\begin{subequations}
			\begin{align}
				\mu(N) & =\sum_{n\geq 0}\mu(A_n)                                                                \\
				       & \leq \sum_{n\geq 0}\mu(D_n)           & \text{cf. justif.}		\label{ITEMooMRFYooZjBiOM} \\
				       & \leq \sum_{n\geq 0}\frac{ \mu(D) }{2} & \text{cf.justif. } \label{SUBEQooJRGSooDvMOhI} \\
				       & =-\infty.
			\end{align}
		\end{subequations}
		Justifications.
		\begin{itemize}
			\item
			      Pour \eqref{ITEMooMRFYooZjBiOM}. Nous avons \( \mu(A_n)\leq \mu(D_n)\) par choix de \( A_n\).
			\item
			      Pour \eqref{SUBEQooJRGSooDvMOhI}. Parce que\quext{Je n'ai vu ce passage correctement justifié nulle part. Voir ma question sur le Wikipédia anglophone\cite{BIBooKJIAooDAFaZo}, qui fait la même «faute» que \cite{BIBooQMABooVJScYf}. Écrivez-moi si vous comprenez leur démarche.}
			      $\mu(D_n)\leq \frac{ s_n }{2}\leq \frac{ \mu(D) }{2}$.
		\end{itemize}
		Vu que \( \mu\) ne prend pas la valeur \( -\infty\), nous avons une contradiction. Nous en déduisons que \( P\) est positive pour \( \mu\).
	\end{subproof}

	\begin{center}
		«Unicité»
	\end{center}

	Supposons avoir des couples \( (N,P)\) et \( (N',P')\). Nous avons d'abord
	\begin{equation}
		P\setminus P'=P\cap N'
	\end{equation}
	parce que si \( x\in P\setminus P'\), alors \( x\in P\) (ça c'est facile), et \( x\in P\setminus P'\subset \Omega\setminus P'=N'\). Dans l'autre sens si \( x\in P\cap N'\), alors \( x\in N'\subset\Omega\cap P'\). De même nous avons \( P'\setminus P=P'\cap N\).

	En utilisant la formule \eqref{EQooMWTNooKXfFvU} de la différence symétrique,
	\begin{equation}		\label{EQooUZFVooDqAnPF}
		P\Delta P'=(P\setminus P')\cup (P'\setminus P)=(P\cap N')\cup(P'\cap N).
	\end{equation}
	Nous trouvons de la même façon
	\begin{equation}		\label{EQooLPBIooTExEkK}
		N\Delta N'=(P\cap N')\cup(P'\cap N).
	\end{equation}
	En ce qui concerne les mesures, \( P\cap N'\subset P\) et \( P\) est positif, donc \(\mu(P\cap N')\geq 0 \). Mais \( P\cap N'\subset N'\) et \( N'\) est négatif, donc \( \mu(P\cap N'))\leq 0\).

	Nous en déduisons que \( \mu(P\cap N')=0\). Pour la même raison, \( \mu(P'\cap N)=0 \). Nous déduisons alors de \eqref{EQooUZFVooDqAnPF} et \eqref{EQooLPBIooTExEkK} que \( \mu(P\Delta P')=\mu(N\Delta N')=0\).
\end{proof}

\begin{lemma}
	Soit une décomposition de Hahn \( (N,P)\) pour la mesure signée \( \mu\).  Nous posons
	\begin{subequations}
		\begin{align}
			\mu_-(A) & =-\mu(A\cap N) \\
			\mu_+(A) & =\mu(A\cap P).
		\end{align}
	\end{subequations}
	Alors
	\begin{enumerate}
		\item
		      \( \mu_-\) et \( \mu_+\) sont des mesures positives.
		\item
		      \( \mu_-\) prend ses valeurs dans \( \mathopen[ 0,\infty\mathclose[\)
	\end{enumerate}
\end{lemma}

\begin{proof}
	Vu que \( N\) est négative pour \( \mu\), pour toute partie mesurable \( A\), nous avons \( \mu(A\cap N)\leq 0\) et donc
	\begin{equation}
		\mu_-(A)=-\mu(A\cap N)\geq 0.
	\end{equation}
	Par ailleurs \( \mu\) prend ses valeurs dans \( \mathopen] -\infty,\infty\mathclose]\). De cette façon \( -\mu\) ne peut pas descendre jusqu'à \( -\infty\).
\end{proof}

\begin{definition}[Décomposition de Jordan]		\label{DEFooGICVooKZWLrB}
	Soit une mesure signée \( \mu\) sur \( (\Omega,\tribA)\). Un couple de mesures positives \( (\mu_-,\mu_+)\) est une \defe{décomposition de Jordan}{décomposition de Jordan} de \( \mu\) si il existe une partie mesurable \( E\) telle que
	\begin{enumerate}
		\item
		      \( \mu_-(E)=0\)
		\item
		      \( \mu_+(\Omega\setminus E)=0\)
		\item
		      \( \mu=\mu_+-\mu_-\).
	\end{enumerate}
\end{definition}

\begin{lemma}[\cite{MonCerveau}]		\label{LEMooTVFRooAPdbuP}
	Soit une mesure signée \( \mu\) et une décomposition de Jordan \( (\mu_-,\mu_+)\). Si \( E\) est un mesurable vérifiant les conditions de la définition \ref{DEFooGICVooKZWLrB}, alors \( (\Omega\setminus E, E)\) est une décomposition de Hahn.
\end{lemma}

\begin{proof}
	Supposons que \( A\subset E\). Alors \( \mu_-(A)=0\) parce que \( \mu_-(E)=0\) et \( \mu_-\) est une mesure positive. Nous avons alors
	\begin{equation}
		\mu(A)=\mu_+(A)-\mu_-(A)=\mu_+(A)\geq 0.
	\end{equation}
	Cela prouve que \( E\) est une partie positive pour \( \mu\). Le même raisonnement montre que \( \Omega\setminus E\) est une partie négative pour \( \mu\).
\end{proof}

\begin{proposition}[Décomposition de Jordan\cite{BIBooQMABooVJScYf}]		\label{DEFooDYFPooITNRlI}
	Soit une mesure mesure signée \( \mu\) sur \( (\Omega,\tribA)\).
	\begin{enumerate}
		\item
		      Il existe une unique décomposition de Jordan \( (\mu_-, \mu_+)\) pour \( \mu\).
		\item		\label{ITEMooDMVXooTXBFnM}
		      Si \( (N,P)\) est une décomposition de Hahn, alors la décomposition de Jordan de \( \mu\) est donnée par
		      \begin{subequations}
			      \begin{align}
				      \mu_-(A)=-\mu(A\cap N) \\
				      \mu_+(A)=\mu(A\cap P).
			      \end{align}
		      \end{subequations}
	\end{enumerate}
\end{proposition}
\index{décomposition de Jordan}

\begin{proof}
	Nous prouvons l'existence directement en prouvant le point \ref{ITEMooDMVXooTXBFnM}. Vérifions les trois points de la définition \ref{DEFooDYFPooITNRlI}. Nous posons \( E=P\), et nous vérifions
	\begin{equation}
		\mu_-(E)=-\mu(E\cap N)=-\mu(E\cap N)=-\mu(\emptyset)=0.
	\end{equation},
	et de même \( \mu_+(\Omega\setminus E)=\mu_+(N)=\mu(A\cap N)=0\). Enfin, vu que \( \Omega=N\cup P\) est une union disjointe, nous avons $\mu(A)=\mu(A\cap N)+\mu(A\cap P)=-\mu_-(A)+\mu_+(A)$.

	Il nous reste à prouver l'unicité de la décomposition de Jordan. Nous supposons que \( (\mu_-,\mu_+)\) et \( (\nu_-,\nu_+)\) sont des décompositions de Jordan de \( \mu\). Nous avons donc des mesurables \( E\) et \( F\) tels que \( \mu=\nu_+-\nu_-=\mu_+-\mu_-\) et
	\begin{subequations}
		\begin{numcases}{}
			\mu_-(E)=0\\
			\mu_+(\Omega\setminus E)=0\\
			\nu_-(F)=0\\
			\nu_+(\Omega\setminus F)=0.
		\end{numcases}
	\end{subequations}
	Le lemme \ref{LEMooTVFRooAPdbuP} nous indique que \( (\Omega\setminus E,E)\) et \( (\Omega\setminus F,F)\) sont des décompositions de Hahn pour \( \mu\). Aussi, pour prendre des notations plus intuitives, nous posons
	\begin{equation}
		\begin{aligned}[]
			N      & =\Omega\setminus E & P      & =E     \\
			N'     & =\Omega\setminus F & P'     & =F     \\
			\mu'_- & =\nu_-             & \mu'_+ & =\nu_+
		\end{aligned}
	\end{equation}

	La partie unicité de la décomposition de Hahn \ref{PROPooIBLHooTMfEJW} implique que \( N\Delta N'\) et \( P\Delta P'\) sont des parties nulles pour \( \mu\). Étant donné que \( N'\setminus N\subset N\Delta N'\), la partie \( N'\setminus N\) est également nulle pour \( \mu\). Idem pour \( N\setminus N'\). Enfin nous avons l'union disjointe
	\begin{equation}
		N'=(N'\cap N)\cup(N'\setminus N),
	\end{equation}
	et nous pouvons calculer
	\begin{subequations}
		\begin{align}
			\mu'_-(A) & = - \mu(A\cap N')                                                                                                                            \\
			          & = - \mu\big((A\cap N')\cap (N'\cap N)\big)-\mu\big(  (A\cap N')\cap (N'\setminus N)  \big)                                                   \\
			          & = - \mu(A\cap N\cap N') - \mu\big(  A\cap (N'\setminus N) \big)                                                                              \\
			          & = - \mu\big(   A\cap N\cap N' \big)-\mu\big(  A\cap (N\setminus N')  \big)                 & \text{cf.justiif.}  \label{SUBEQooFQVAooNNxDHX} \\
			          & = - \mu(A\cap N)                                                                           & \text{cf. justif}	\label{SUBEQooKCMDooWeFCXv}   \\
			          & = \mu_-(A).
		\end{align}
	\end{subequations}
	Justifications.
	\begin{itemize}
		\item
		      Pour \eqref{SUBEQooFQVAooNNxDHX}. Nous avons
		      \begin{equation}
			      \mu\big( A\cap (N\setminus N') \big)=\mu\big( A\cap (N'\setminus N) \big)=0
		      \end{equation}
		      parce que \( N\setminus N'\) et \( N'\setminus N\) sont des parties nulles pour \( \mu\). Bref le passage \eqref{SUBEQooFQVAooNNxDHX} consiste à remplacer un truc nul par un autre.
		\item
		      Pour \eqref{SUBEQooKCMDooWeFCXv}. Nous avons l'union disjointe \( N=(N'\cap N)\cup(N\setminus N')\).
	\end{itemize}
	Nous avons prouvé que \( \mu'_-(A)=\mu_-(A)\), et donc que \( \mu'_-=\mu_-\). Nous faisons de même pour \( \mu'_+=\mu_+\).
\end{proof}


\begin{theorem}[Radon-Nikodym\cite{BIBooQMABooVJScYf,MonCerveau}]		\label{THOooKSISooPAqZcp}
	Soit un espace mesurable \( (\Omega,\tribA)\). Nous considérons
	\begin{enumerate}
		\item
		      \( \mu\), une mesure positive \( \sigma\)-finie.
		\item
		      \( \nu\), une mesure signée\footnote{Mesure signée, définition \ref{DefBTsgznn}.} ou non telle que \( | \nu |\) est \( \sigma\)-finie.
	\end{enumerate}
	Alors
	\begin{enumerate}
		\item
		      Il existe un unique couple de mesures signées \( \nu_a,\nu_s\) tel que
		      \begin{subequations}
			      \begin{numcases}{}
				      \nu=\nu_a+\nu_s\\
				      \nu_a\ll\mu\\
				      \nu_s\perp\mu.
			      \end{numcases}
		      \end{subequations}
		\item
		      Il existe une unique (à égalité presque partout près) fonction mesurable et intégrable\footnote{Intégrale ne signifie pas \( L^1\). Nous demandons juste que l'intégrale existe, si elle vaut \( \infty\), c'est bon pour nous.} \(f \colon \Omega\to \eR\cup\{ +\infty \}  \) telle que
		      \begin{equation}
			      \nu_a(A)=\int_Afd\mu
		      \end{equation}
		      pour tout \( A\in\tribA\).
		\item
		      Si \( \nu\geq 0\) alors \( \nu_a\geq 0\), \( \nu_s\geq 0\) et \( f\geq 0\).
	\end{enumerate}
\end{theorem}
\index{théorème!Radon-Nikodym}

\begin{proof}
	Nous commençons par l'unicité.

	\begin{proofpart}
		Unicité des mesures et de la fonction
	\end{proofpart}

	Séparément.
	\begin{subproof}
		\spitem[Unicité des mesures]
		%-----------------------------------------------------------
		Nous supposons avoir deux tels couples \( (\nu_a,\nu_s)\) et \( (\nu'_a,\nu'_s)\). Nous avons donc \( \nu_a+\nu_s=\nu\) et \( \nu'_a+\nu'_s=\nu\). En faisant la différence,
		\begin{equation}
			\nu_a-\nu'_a=(\nu-\nu_s)-(\nu-\nu'_s)=\nu'_s-\nu_s.
		\end{equation}
		D'autre part, vu que \( \nu_s\perp\mu\) et \( \nu'_s\perp \mu\), le lemme \ref{LEMooABVVooMvxlRo} donne \( \nu_s'-\nu_s\perp\mu\). Enfin \( \nu_a\ll\mu\) et \( \nu_a'\ll \mu\), donc \( \nu_a-\nu_a'\ll\mu\). Au final nous avons
		\begin{subequations}
			\begin{numcases}{}
				\nu_a-\nu'_a\ll\mu\\
				\nu'_s-\nu_s\perp\mu\\
				\nu_a-\nu'_a=\nu_s'-\nu_s.
			\end{numcases}
		\end{subequations}
		Tout cela avec le lemme \ref{LEMooVFEWooOyXcPS} nous donne \( \nu_a-\nu'_a=\nu_s-\nu'_s=0\).

		\spitem[Unicité de la fonction]
		%-----------------------------------------------------------

		Soient deux fonctions \(f,g \colon\Omega \to \eR\cup\{ \infty \}  \) vérifiant la contrainte. Pour toute partie \( A\in\tribA\), nous avons
		\begin{equation}
			\nu(A)=\int_Afd\mu=\int_Agd\mu.
		\end{equation}
		Nous écrivons cette égalité pour la partie \( A=\{ f>g \}\) :
		\begin{equation}
			\int_{\{ f>g \}}(f-g)d\mu=0.
		\end{equation}
		Vu que sur la partie \( \{ f>g \}\) nous avons \( f-g>0\), nous avons \( \mu\big( \{ f>g \} \big)=0\) par le lemme \ref{LEMooLXVGooUDuQzc}. Nous trouvons de même que \( \mu\big( \{ f<g \} \big)=0\). Au final, nous avons \( f=g\) presque partout.
	\end{subproof}

	\begin{proofpart}
		Existence, \( \mu\) positive finie, \( \nu\) positive et finie
	\end{proofpart}

	Nous commençons par prouver la partie existence en supposant que \( \mu\) est finie et que \( \nu\) est positive et finie. Nous introduisons l'ensemble suivant :
	\begin{equation}
		\mH=\Big\{
		f \colon \Omega\to \eR^+\cup\{ \infty \} \tq
		\begin{cases}
			f                       \text{est mesurable } \\
			\int_Afd\mu\leq \nu(A)  \text{ pour tout  }A\in\tribA
		\end{cases}
		\Big\}
	\end{equation}
	L'ensemble \( \mH\) n'est pas vide parce que la fonction \( f=0\) est dedans (ici nous utilisons l'hypothèse supplémentaire que \( \nu\geq 0\)). Nous notons
	\begin{equation}
		\alpha=\sup\{ \int_{\Omega}fd\mu\tq f\in\mH \}.
	\end{equation}
	Nous avons \( 0\leq \alpha\leq \mu(\Omega)<\infty\). En vertu du lemme \ref{LEMooTRJZooSTuHWs} nous pouvons prendre une suite \( (f_n)\) dans \( \mH\) telle que
	\begin{equation}
		\alpha=\lim{n\to\infty }\int_{\Omega}f_nd\mu
	\end{equation}
	et
	\begin{equation}
		\int_{\Omega}f_nd\mu\leq \int_{\Omega}f_{n+1}d\mu.
	\end{equation}

	\begin{subproof}
		\spitem[Les fonctions \( g_n\)]
		%-----------------------------------------------------------

		Pour chaque \( n\geq 1\) nous notons
		\begin{equation}
			\begin{aligned}
				g_n\colon \Omega & \to \eR                                  \\
				x                & \mapsto \max\{ f_1(x),\ldots, f_n(x) \}.
			\end{aligned}
		\end{equation}
		Nous nous fixons provisoirement un \( n\).

		\spitem[\( g_n\in \mH\)]
		%-----------------------------------------------------------

		Pour chaque \( x\in\Omega\), il existe un \( i\in\{ 1,\ldots,n \}\) tel que \( g_n(x)=f_i(x)\). Pour \( A\in \tribA\) nous posons
		\begin{subequations}
			\begin{align}
				A_1     & =\{ x\in\Omega\tq g_n(x)=f_1(x) \}                                       \\
				A_2     & =\{ x\in\Omega\tq g_n(x)=f_2(x) \}\setminus A_1                          \\
				        & \vdots                                                                   \\
				A_{k+1} & =\{ x\in\Omega\tq g_n(x)=f_{k+1}(x) \}\setminus (A_1\cup\ldots \cup A_k) \\
				        & \vdots                                                                   \\
				A_n     & =\{ x\in\Omega\tq g_n(x)=f_n(x) \}\setminus (A_1\cup\ldots \cup A_{n-1})
			\end{align}
		\end{subequations}
		Ces ensembles sont disjoints et \( A=\bigcup_{i=1}^nA_n\), de telle sorte que
		\begin{subequations}
			\begin{align}
				\int_Ag_nd\mu & =\sum_{i=1}^n\int_{A_i}g_nd\mu                                             \\
				              & =\sum_{i=1}^n\int_{A_i}f_id\mu                                             \\
				              & \leq \sum_{i=1}^n\nu(A_i)   \text{cf. justif}		\label{SUBEQooIULRooCXJmzd} \\
				              & =\nu(A).
			\end{align}
		\end{subequations}
		Justification de \eqref{SUBEQooIULRooCXJmzd}. Vu que \( f_i\in \mH\), nous avons \( \int_{A_i}f_i\leq \nu(A_i)\). Bref, nous avons prouvé que \( \int_Ag_nd\mu\leq \nu(A)\), et donc que \( g_n\in \mH\).
		\spitem[\( g_n\) est mesurable]
		%-----------------------------------------------------------
		Les applications \( f_n\) sont mesurables, et \( g_n\) est définie comme un maximum sur les \( f_i(x)\). Donc \( g_n\) est mesurable (lemme \ref{LEMooMGUOooMGwknZ}).
		Nous ne fixons plus le \( n\). La suite \( (g_n)\) est croissante. Donc la limite \( f=\lim_{n\to\infty}g_n\) existe, et elle est mesurable par la proposition \ref{PropooMFIBooJzaleK}.

		\spitem[\( \int_{\alpha}fd\mu=\alpha\)]		\label{SPITEMooGLPCooBjBAdm}
		%-----------------------------------------------------------

		Le théorème de la convergence monotone \ref{ThoRRDooFUvEAN} va plus loin et nous dit que \( f\) est intégrable et
		\begin{equation}		\label{EQooBGAOooCdaIID}
			\int_Afd\mu=\lim_{n\to\infty }\int_Ag_nd\mu\leq \nu(A).
		\end{equation}
		Nous avons donc encore \( f\in\mH\) et donc aussi
		\begin{equation}
			\int_{\Omega}fd\mu\leq \alpha.
		\end{equation}
		Mais cette inégalité tient aussi dans l'autre sens :
		\begin{subequations}
			\begin{align}
				\int_{\Omega}fd\mu & =\lim\int_{\Omega}g_nd\mu     \\
				                   & \geq \lim\int_{\Omega}f_nd\mu \\
				                   & =\alpha.
			\end{align}
		\end{subequations}
		Nous avons donc prouvé que \( \int_{\Omega}fd\mu=\alpha<\infty\), et donc que \( f\in L^1(\Omega,\tribA,\mu)\).

		\spitem[Résumé pour l'instant]
		%-----------------------------------------------------------
		Nous avons construit une fonction \( f\in L^1(\Omega,\tribA, \mu)\) telle que pour tout \( A\in\tribA\),
		\begin{equation}
			\int_Afd\mu\leq \nu(A).
		\end{equation}
		\spitem[Les mesures]
		%-----------------------------------------------------------
		Nous définissons à présent les applications \( \nu_a\) et \( \nu_s\) en espérant qu'elles fonctionneront. D'abord nous posons
		\begin{equation}		\label{EQooVKTXooCwZdha}
			\nu_a(A)=\int_Afd\mu.
		\end{equation}
		Et ensuite nous posons \( \nu_s=\nu-\nu_a\). Il nous reste à prouver que \( f, \nu_a\) et \( \nu_s\) satisfont à toutes les conditions.
		\spitem[\( \nu_a\ll \mu\)]
		%-----------------------------------------------------------

		Si \( \mu(A)=0\) nous avons
		\begin{equation}
			\nu_a(A)=\int_Afd\mu=0
		\end{equation}
		par le lemme \ref{LEMooVZXFooZAQWKY}.

		\spitem[\( \nu_s\) est positive]
		%-----------------------------------------------------------

		Nous avons
		\begin{equation}
			\nu_s(A)=\nu(A)-\nu_a(A)=\nu(A)-\int_Afd\mu\geq 0,
		\end{equation}
		par l'inégalité \eqref{EQooBGAOooCdaIID}.

		\spitem[\( \nu_s\) est finie]
		%-----------------------------------------------------------

		Nous savons que \( \nu_a\) et finie et que \( \int_{\Omega}fd\mu=\alpha<\infty\). Donc \( \nu_s\) est finie.

		\spitem[\( \nu_s\perp\mu\)]
		%-----------------------------------------------------------

		Nous introduisons la mesure signée \( \lambda_n=\nu_s-\frac{1}{  n}\mu\), et nous considérons la décomposition de Hahn pour chacun des \( \lambda_n\) (proposition \ref{PROPooIBLHooTMfEJW}) : les couples \( (N_n, P_n)\). Nous considérons aussi les fonctions \( h_n=f+\frac{1}{ n}\mtu_{P_n}\).

		\begin{subproof}
			\spitem[\( h_n\in \mH\)]
			%-----------------------------------------------------------
			La fonction \( h_n\) est mesurables parce que \( f\) et \( P_n\) le sont. En termes d'intégrale nous avons
			\begin{subequations}
				\begin{align}
					\int_Ah_nd\mu & =\int_Afd\mu +\frac{1}{ n}\mu(A\cap P_n)                                                                                                                                  \\
					              & =\nu_a(A)+\nu_s(A\cap P_n)-\lambda_n(A\cap P_n)                                                                         & \text{cf. justif}		\label{SUBEQooAVCMooXXSXZO}  \\
					              & = \nu(A)-\nu_s(A)+\nu_s(A\cap P_n)-\lambda_n(A\cap P_n)                                                                                                                   \\
					              & =\nu(A)-\big( \nu_s(A\cap P_n)+\nu_s(A\cap N_n) \big)+\nu_s(A\cap P_n)                       \label{SUEQooUJUDooLFrBcR}                                                   \\
					              & \nonumber\quad-\lambda_n(A\cap P_n)                                                                                     & \text{cf. justif.}                              \\
					              & =\nu(A) -\nu_s(A\cap N_n)-\lambda_n(A\cap P_n)                                                                                                                            \\
					              & \leq \nu(A)                                                                                                             & \text{cf. justif.}		\label{SUBEQooACEXooWCHuag}
				\end{align}
			\end{subequations}
			Justifications.
			\begin{itemize}
				\item
				      Pour \eqref{SUBEQooAVCMooXXSXZO}. La définition \eqref{EQooVKTXooCwZdha} de \( \nu_a)\).
				\item
				      Pour \eqref{SUEQooUJUDooLFrBcR}. Parce que \( P_n\cup N_n=\Omega\) est une union disjointe.
				\item
				      Pour \eqref{SUBEQooACEXooWCHuag}. Parce que \( \nu_s\) est une mesure positive. Donc \( \nu_s(A\cap N_n)\geq 0\). Et aussi parce que, par construction, \( \lambda_n\) est positive sur \( P_n\).
			\end{itemize}
			Bref, nous avons déjà prouvé que \( \int_Ah_nd\mu\leq \nu(A)\). Cela signifie que \( h_n\in \mH\).
			\spitem[\( \mu(P_n)=0\)]
			%-----------------------------------------------------------
			Nous avons vu en \ref{SPITEMooGLPCooBjBAdm} que \( \int_{\Omega}fd\mu=\alpha\). Nous pouvons alors faire le calcul suivant :
			\begin{subequations}
				\begin{align}
					\alpha & \geq \int_{\Omega}h_nd\mu                                    & \text{cf.justi.}	\label{SUBEQooUHXDooDJIXSI} \\
					       & = \int_{\Omega}df\mu+\frac{1}{ n}\int_{\Omega}\mtu_{P_n}d\mu                                                \\
					       & =\alpha+\frac{1}{ n}\mu(P_n).
				\end{align}
			\end{subequations}
			Justification pour \eqref{SUBEQooUHXDooDJIXSI} : parce que \( h_n\in\mH\) et que \( \alpha\) est le supremum.

			En soustrayant \( \alpha\) des deux côtés, nous avons \( 0\geq \frac{1}{ n}\mu(P_n)\) pour tout \( n\). Comme \( \mu\) est une mesure positive, nous en déduisons que \( \mu(P_n)=0\).
			\spitem[Définition de \( P\)]
			%-----------------------------------------------------------
			Nous posons \( P=\bigcup_{n=1}^{\infty}P_n\).

			\spitem[\( \mu(P)=0\)]
			%-----------------------------------------------------------
			Vu que \( \mu(P_n)=0\) pour tout \( n\), nous avons \( \mu(P)\leq \sum_n\mu(P_n)=0\).

			\spitem[\( \nu_s(\Omega\setminus P)=0\)]
			%-----------------------------------------------------------

			Nous avons \( \Omega\setminus P\subset \Omega\setminus P_n=N_n\). Donc pour tout \( n\) nous avons
			\begin{equation}
				\lambda_n(\Omega\setminus P)\leq \lambda_n(N_n)=0.
			\end{equation}
			En utilisant la définition de \( \lambda_n\) nous avons
			\begin{equation}
				0\geq \lambda_n(\Omega\setminus P)=\nu_s(\Omega\setminus P)-\frac{1}{ n}\mu(\Omega\setminus P),
			\end{equation}
			et donc, pour tout \( n\),
			\begin{equation}
				\nu_s(\Omega\setminus P)\leq \frac{1}{ n}\mu(\Omega\setminus P).
			\end{equation}
			En faisant \( n\to\infty\) nous trouvons \( \nu_s(\Omega\setminus P)\leq 0\). Et comme \( \nu_s\) est une mesure positive, nous en déduisons que \( \nu_s(\Omega\setminus P)=0\).
		\end{subproof}
		Nous avons fini de prouver que \( \nu_s\perp\mu\).
	\end{subproof}
	Nous avons donc terminé la première partie de la preuve, c'est-à-dire la partie où nous supposions que \( \mu\) est finie et \( \nu\) est positive et finie. Nous passons à présent à un cas un peu plus général.


	\begin{proofpart}
		Existence, \( \mu\) positive \( \sigma\)-finie, \( \nu\) positive et \( \sigma\)-finie
	\end{proofpart}

	Vu que \( \mu\) et \( \nu\) sont \( \sigma\)-finie, le lemme \ref{LEMooXOPMooMmIbmD} dit qu'il existe des \( X_n\) mesurables tels que \( \Omega=\bigcup_{n\geq 1}X_n\), \( \mu(X_n)<\infty\) et\( \nu(X_n)<\infty\). Nous supposons de plus que ce \( X_n\) sont disjoints\footnote{Si ce n'est pas le cas, poser \( Y_{n+1}=X_{n+1}\setminus Y_n\).}

	Nous considérons à présent les mesures finies \( \mu_n\) et \( \nu_n\) définies par
	\begin{subequations}
		\begin{align}
			\mu_n(A) & =\mu(A\cap X_n) \\
			\nu_n(A) & =\nu(A\cap X_n) \\
		\end{align}
	\end{subequations}
	Par la partie déjà faite, nous pouvons faire une décomposition pour chaque \( n\) : il existe des mesures \( \nu_{n,a}\), \( \nu_{n,s}\) et une fonction \( f_n\in L^1(\Omega,\tribA,\mu)\) telles que
	\begin{subequations}		\label{SUBEQSooLONIooAksRgK}
		\begin{numcases}{}
			\nu_{n,a}(A)=\int_Af_{n}d\mu_n \\
			\nu_n=\nu_{n,a}+\nu_{n,s}\\
			\nu_{n,a}\ll \mu_n\\
			\nu_{n,s}\perp \mu_n.
		\end{numcases}
	\end{subequations}

	Nous allons directement changer les \( f_n\) pour qu'ils soient plus faciles à traiter. Nous posons\quext{Cette partie est de l'invention personnelle. J'espère que c'est correct. Redoublez de vigilance.}
	\begin{equation}
		g_n(x)=\begin{cases}
			f_n(x) & \text{si } x\in X_n \\
			0      & \text{sinon. }
		\end{cases}
	\end{equation}
	Pour ces fonctions nous avons, parce que \( \mu_n\) est nulle en dehors de \( X_n\),
	\begin{equation}
		\int_Ag_nd\mu_n=\int_{A\cap X_n}g_nd\mu_n=\int_{A\cap X_n}f_nd\mu_n=\int_Af_nd\mu_n=\nu_{n,a}(A).
	\end{equation}
	Autrement dit, les fonctions \( g_n\) fonctionnent aussi bien que les fonctions \( f_n\). Pour ne pas alourdir les notations, nous gardons \eqref{SUBEQSooLONIooAksRgK}, mais nous supposons que \( f_n\) est nulle en dehors de \( X_n\).

	Notez aussi que pour tout \( A\in \tribA\) nous avons l'union disjointe \( A=\bigcup A\cap X_n\), de telle sorte que
	\begin{equation}
		\mu(A)=\sum_n\mu(A\cap X_n)=\sum_n\mu_n(A),
	\end{equation}
	et nous pouvons écrire
	\begin{equation}
		\mu=\sum_{n\geq 0}\mu_n.
	\end{equation}

	Nous posons
	\begin{subequations}
		\begin{numcases}{}
			f=\sum_{n\geq 1}f_n\\
			\nu_a=\sum_{n\geq 1}\nu_{n,a}\\
			\nu_s=\sum_{n\geq 1}\nu_{s,a},
		\end{numcases}
	\end{subequations}
	et nous prouvons que cela fait l'affaire. Notons que les fonctions \( f_n\) sont \( L^1\); cela ne signifie pas que \( f\) soit \( L^1\), mais nous ne le demandons pas. La fonction \( f\) est certainement intégrable parce qu'elle est positive. Au pire \( \int_Afd\mu=\infty\) pour certains \( A\). C'est la vie.

	Nous devons vérifier les points suivants :
	\begin{itemize}
		\item
		      \( \nu_a(A)=\int_Afd\mu\)
		\item
		      $\nu=\nu_a+\nu_s$
		\item
		      \( \nu_a\ll \mu\)
		\item
		      \( \nu_s\perp\mu\).
	\end{itemize}
	\begin{subproof}
		\spitem[\( \int_Adf\mu=\nu(A)\)]
		%-----------------------------------------------------------
		Voici un calcul :
		\begin{subequations}
			\begin{align}
				\int_Afd\mu & =\int_A\sum_nf_nd\mu                                                                                 \\
				            & =\sum_k\int_{A\cap X_k}\big( \sum_nf_n \big)d\mu_k & \text{cf.justif.}	\label{SUBEQooWILAooLqholj}   \\
				            & = \sum_k\sum_n\int_{A\cap X_k}f_nd\mu_k            & \text{cf. justif}\label{SUQEQooMIVJooBVexgH}    \\
				            & = \sum_k\int_{A\cap X_k}f_kd\mu_k                  & \text{cf. justif.}		\label{SUBEQooNKQZooUMJeUL} \\
				            & = \sum_k \nu_{k,a}(A)                                                                                \\
				            & =\nu_a(A).
			\end{align}
		\end{subequations}
		Justifications.
		\begin{itemize}
			\item
			      Pour \eqref{SUBEQooWILAooLqholj}. Les \( X_k\) sont disjoints et leur union fait \( \Omega\). Nous pouvons donc tranquillement couper l'intégrale en morceaux\quext{Ce découpage est également de l'invention personnelle. Quadruplez de vigilance.}.
			\item
			      Pour \eqref{SUQEQooMIVJooBVexgH}.
			      Nous permutons la somme et l'intégrale avec le corolaire \ref{CorNKXwhdz} :
			\item
			      Pour \eqref{SUBEQooNKQZooUMJeUL}. Vu que \( f_n=0\) sur \( X_k\) tant que \( k\neq n\), tous les termes de la somme sur \( n\) sont nuls sauf le terme \( n=k\).
		\end{itemize}
		Voila déjà une belle première vérification de faite.

		\spitem[\( \nu=\nu_a+\nu_s\)]
		%-----------------------------------------------------------

		La proposition \ref{PROPooQCOBooAMnljj} nous permet de fusionner le sommes.
		\begin{equation}
			\begin{aligned}[]
				(\nu_a+\nu_s)(A) & =\sum_n\nu_{n,a}(A)+\sum_n\nu_{n,s}(A)                              \\
				                 & =\sum_n\Big( \nu_{n,a}(A)+\nu_{n,s}(A) \Big)=\sum_n\nu_n(A)=\nu(A).
			\end{aligned}
		\end{equation}

		\spitem[\( \nu_a\ll\mu\)]
		%-----------------------------------------------------------

		Supposons que \( \mu(A)=0\). Vu que l'union \( \Omega=\bigcup_nX_n\) est disjointe, nous avons
		\begin{equation}
			0=\mu(A)=\sum_{n\geq 1}\mu(A\cap X_n)=\sum_n\mu_n(A).
		\end{equation}
		Nous en déduisons que \( \mu_n(A)=0\) pour tout \( n\) parce que \( \mu_n\) est une mesure positive. Étant donné que \( \mu_{n,a}\ll \mu_n\), nous avons \( \nu_{n,a}(A)=0\) pour chaque \( n\). En faisant la somme,
		\begin{equation}
			\nu_a(A)=\sum_n\nu_{n,a}(A)=0.
		\end{equation}

		\spitem[\( \nu_s\perp \mu\)]
		%-----------------------------------------------------------
		Nous décomposons en petites parties.
		\begin{subproof}
			\spitem[\( \nu_{n,s}\perp\mu_k\) pour tout \( n\) et \( k\)]
			%-----------------------------------------------------------
			Nous vérifions la définition \ref{DEFooRZARooTbtJac} avec \( E=X_k\) et \( F=X_n\), qui sont disjoints. Nous avons d'une part que \( \mu_k(A)=\mu(A\cap X_k)\), et d'autre part que \( \mu_k(A\cap X_k)=\mu(A\cap X_k\cap X_k)=\mu(A\cap X_k)\). Donc \( \mu_k(A)=\mu_k(A\cap X_k)\).

			Et d'autre part\quext{Invention personnelle, octuplez de vigilance.},
			\begin{subequations}
				\begin{align}
					\nu_{n,s}(A) & = \nu_n(A)-\nu_{n,a}(A)                                                                       \\
					             & = \nu(A\cap X-n)-\int_A f_nd\mu_n                                                             \\
					             & = \nu(A\cap X-n)-\int_{A\cap X_n} f_nd\mu_n & \text{\( f_n=0\) sur \( \Omega\setminus X_n\).} \\
					             & = \nu(A\cap X_n)-\nu_{n,a}(A\cap X_n).
				\end{align}
			\end{subequations}
			Le membre de droite ne change pas si nous prenons \( A\cap X_n\) au lieu de \( A\). Donc le membre de gauche non plus. Nous en déduisons que \( \nu_{n,s}(A)=\nu_{n,s}(A\cap X_n)\).
			\spitem[\( \nu_{n,s}\perp \mu\)]
			%-----------------------------------------------------------
			Appliquer le lemme \ref{LEMooVOJDooFgbwSE} à \( \mu=\sum_n\mu_n\).

			\spitem[\( \nu_{s}\perp \mu\)]
			%-----------------------------------------------------------
			Appliquer le lemme \ref{LEMooVOJDooFgbwSE} à \( \nu_s=\sum_n\nu_{n,s}\).
		\end{subproof}
	\end{subproof}


	\begin{proofpart}
		Existence, \( \mu\) positive \( \sigma\)-finie, \( \nu\) signée, \( | \nu |\) \( \sigma\)-finie
	\end{proofpart}

	Nous sommes enfin dans le cas général. Vu que \( | \nu |\) est \( \sigma\)-finie, les deux parties de la décomposition de Jordan\footnote{Proposition \ref{DEFooDYFPooITNRlI}.} \( (\nu_-,\nu_+)\) de \( \nu\) sont des mesures positives \( \sigma\)-finies. Nous pouvons donc appliquer tout ce que nous venons de voir aux mesures \( \nu_-\) et \( \nu_+\) séparément : nous avons des mesures positives \( \nu_{+,a}\), \( \nu_{+,s}\), \( \nu_-,a\) et \( \nu_-,s\) telles que
	\begin{subequations}
		\begin{numcases}{}
			\nu_{+,a}+\nu_{+,s}=\nu_+\\
			\nu_{+,a}\ll \mu\\
			\nu_{+,s}\perp\mu\\
			\nu_{-,a}+\nu_{-,s}=\nu_-\\
			\nu_{-,a}\ll \mu\\
			\nu_{-,s}\perp\mu.
		\end{numcases}
	\end{subequations}
	Nous posons alors
	\begin{subequations}
		\begin{align}
			\nu_a & =\nu_{+,a}-\nu_{-,a}  \\
			\nu_s & =\nu_{+,s}-\nu_{-,s},
		\end{align}
	\end{subequations}
	et nous vérifions que toutes les conditions sont respectée. D'abord
	\begin{equation}
		\nu_a+\nu_s=\nu_{+,a}-\nu_{-,a}+\nu_{+,s}-\nu_{-,s}=\nu_+-\nu_-=\nu.
	\end{equation}
	Ensuite si \( \nu_{+,a}\ll \mu\) et \( \nu_{-,a}\ll\mu\), nous avons \( \nu_a=\nu_{+,a}-\nu_{-,a}\ll \mu\) parce que le fait d'être dominée passe à la somme finie\footnote{Je n'ai pas dit que ça ne passait pas aux sommes dénombrables. Je suis prêt à parier que non, mais j'y ai pas vraiment réfléchi.}. Et enfin, le lemme \ref{LEMooYYIIooCUfGxA} assure que \( \nu_a\perp\mu\).
\end{proof}


\begin{corollary}   \label{CorZDkhwS}
	Si \( \mu\) est une mesure \( \sigma\)-finie dominée par la mesure \( \sigma\)-finie \( m\), alors \( \mu\) possède une unique fonction de densité.
\end{corollary}

\begin{corollary}       \label{CorDomDens}
	Soient \( \mu\) et \( m\), deux mesures positives \( \sigma\)-finies sur \( (\Omega,\tribA)\). Alors \( m\) domine \( \mu\) si et seulement si \( \mu\) possède une densité par rapport à \( m\).
\end{corollary}

\begin{proof}
	Si \( \mu\) est dominée par \( m\), alors la décomposition \( \mu=\mu+0\) satisfait le théorème de Radon-Nikodym. Par conséquent il existe une fonction \( f\) telle que
	\begin{equation}
		\mu(A)=\int_Afdm.
	\end{equation}
	Cette fonction est alors une densité pour \( \mu\) par rapport à \( m\).

	Pour la réciproque, nous supposons que \( \mu\) a une densité \( f\) par rapport à \( m\), et que \( A\) est un ensemble de \( m\)-mesure nulle :
	\begin{equation}
		m(A)=\int_{\Omega}\mtu_Adm=0.
	\end{equation}
	Cela signifie que la fonction \( \mtu_A\) est \( m\)-presque partout nulle. La fonction produit \( \mtu_Af\) est également nulle \( m\)-presque partout, et par conséquent
	\begin{equation}
		\mu(A)=\int_{\Omega}\mtu_Afdm=0.
	\end{equation}
\end{proof}

\begin{probleme}
	Est-ce que la démonstration de cela ne demande pas la convergence monotone d'une façon ou d'une autre ?
\end{probleme}

%---------------------------------------------------------------------------------------------------------------------------
\subsection{Mesure complexe}
%---------------------------------------------------------------------------------------------------------------------------

\begin{definition}[Mesure complexe\cite{TLRRooOjxpTp}] \label{DefGKHLooYjocEt}
	Si \( (\Omega,\tribA)\) est un espace mesurable, une \defe{mesure complexe}{mesure!complexe} est une application \( \mu\colon \tribA\to \eC\) telle que
	\begin{enumerate}
		\item
		      \( \mu(\emptyset)=0\),
		\item
		      La \( \sigma\)-additivité : si les éléments \( A_i\in\tribA\) sont disjoints, alors \( \sum_i\mu(A_i)=\mu(\bigcup_iA_i)\).
	\end{enumerate}
\end{definition}
Notons que la série \( \sum_i\mu(A_i)\) est alors nécessairement absolument convergente. En effet changer l'ordre de la somme ne change pas l'union, et donc ne change pas la valeur de la somme. Si \( \sigma\colon \eN\to \eN\) est une permutation,
\begin{equation}
	\sum_i\mu(A_{\sigma(i)})=\mu\big( \bigcup_iA_{\sigma(i)} \big)=\mu\big( \bigcup_iA_i \big)=\sum_i\mu(A_i).
\end{equation}
Le théorème~\ref{PROPooOGOEooBkZoEJ} dit alors que la somme doit être absolument convergente.


\begin{theorem}[Radon-Nikodym complexe\footnote{L'histoire du nom de ce théorème est intéressante. Lorsque monsieur et madame Rèmederdonnukodym apprirent que leurs amis, les Rèmedelaboulechevelue avaient appelé leur fils Théo, ils décidèrent d'en faire autant. C'est en souvenir de ces circonstances que monsieur Nikodym (prénommé Radon) décida de faire des math.}]\label{ThoZZMGooKhRYaO}
	Soit \( \mu\) une mesure positive sur \( (\Omega,\tribA)\) et \( \nu\) une mesure complexe. Alors
	\begin{enumerate}
		\item
		      Il existe un unique couple de mesures complexes \( \nu_a\), \( \nu_s\) sur \( (\Omega,\tribA)\) tel que
		      \begin{enumerate}
			      \item
			            \( \nu=\nu_a+\nu_s\)
			      \item
			            \( \nu_a\ll\mu\)
			      \item
			            \( \nu_s\perp \mu\).
		      \end{enumerate}
		\item
		      Ces mesures satisfont alors \( \nu_a\perp\nu_s\).
		\item
		      Il existe une fonction intégrable \( h\colon \Omega\to \eC\) telle que \( \nu_a=h\mu\).
		\item
		      La fonction \( h\) est unique à \( \mu\)-équivalence près.
		\item   \label{ItemDIXOooFqOkgGv}
		      Si de plus \( \nu\ll \mu\) alors \( \nu=h\mu\).
	\end{enumerate}
\end{theorem}
\index{théorème!Radon-Nikodym!complexe}
\begin{proof}
	No proof.
\end{proof}

\begin{remark}  \label{RemSYRMooZPBhbQ}
	Le point~\ref{ItemDIXOooFqOkgGv} est souvent utilisé sous la forme
	\begin{equation}
		\nu(A)=\int_{\Omega}\mtu_A(\omega)h(\omega)d\mu(\omega)=\int_{A}h(\omega)d\mu(\omega).
	\end{equation}
\end{remark}

%---------------------------------------------------------------------------------------------------------------------------
\subsection{Théorème d'approximation}
%---------------------------------------------------------------------------------------------------------------------------

\begin{theorem}[Théorème d'approximation, thème \ref{THEMEooKLVRooEqecQk}\cite{YHRSDGc}]     \label{ThoAFXXcVa}
	Soit un espace topologique métrique \( (\Omega,d)\). Nous considérons sa tribu des boréliens \( \Borelien(\Omega)\) ainsi qu'une mesure \( \mu\) sur \( \big(\Omega,\Borelien(\Omega)\big)\).

	Soient un ouvert \( W\subset \Omega\) tel que \( \mu(W)<\infty\) et un un borélien \( A\) tel que \( A\subset W\). Soit aussi \( \epsilon>0\).

	Il existe un fermé \( F\subset W\) et une fonction  \( f\in C^0(\Omega,\eR)\) vérifiant
	\begin{enumerate}
		\item
		      \( F\subset A\subset W\),
		\item       \label{ITEMooOZVJooSViuds}
		      \( f|_F=1\),
		\item       \label{ITEMooIEFSooHXYZrK}
		      \( f|_{W^c}=0\)
		\item       \label{ITEMooSOQVooBbvfgy}
		      \( \| f-\mtu_A \|_{L^1}<\epsilon\)
	\end{enumerate}
\end{theorem}

\begin{proof}
	Par le lemme \ref{LEMooZDFVooFUgFGZ}, il existe un fermé \( F\) et un ouvert \( V\) tels que
	\begin{equation}
		F\subset A\subset V\subset W
	\end{equation}
	et \( \mu(V\setminus F)<\epsilon\). Nous posons alors
	\begin{equation}
		f(x)=\frac{ d(x,V^c) }{ d(x,V^c)+d(x,F) }.
	\end{equation}
	Le dénominateur de cette expression ne s'annule jamais parce que si \( d(x,V^c)=0\), c'est que \( x\in V^c\). Mais alors \( x\) n'est pas dans \( V\) et donc pas dans \( F\) non plus. La partie \( F\) étant fermée, \( d(x,F)>0\) par lemme \ref{LEMooEQIZooLpsbOe}. De plus la fonction \( f\) est continue par le lemme \ref{LEMooCFGTooIfdcfk}.

	\begin{subproof}
		\spitem[Pour \ref{ITEMooOZVJooSViuds}]
		Si \( x\in F\), alors \( d(x,F)=0\), et \( f\) devient
		\begin{equation}
			f(x)=\frac{ d(x,V^c) }{ d(x,V^c) }=1
		\end{equation}
		\spitem[Pour \ref{ITEMooIEFSooHXYZrK}]
		Si \( x\in W^c\), alors \( x\in V^c\) et \( d(x,V^c)=0\) si bien que \( f(x)=0\).
		\spitem[Pour \ref{ITEMooSOQVooBbvfgy}]
		Les premiers points montrent que
		\begin{equation}
			\mtu_F\leq f\leq \mtu_V.
		\end{equation}
		Mais nous avons aussi, par ailleurs,
		\begin{equation}
			\mtu_{F}\leq \mtu_A\leq \mtu_V.
		\end{equation}
		Ces deux encadrement, par le lemme \ref{LEMooEGYLooCGrDrl} donnent l'encadrement
		\begin{equation}
			| f-\mtu_A |\leq \mtu_V-\mtu_F.
		\end{equation}
		En ce qui concernent les intégrales nous avons alors
		\begin{subequations}
			\begin{align}
				\int_{\Omega}| \mtu_A-f | & \leq \int_{\Omega}(\mtu_V-\mtu_F)d\mu      \\
				                          & =\mu(V)-\mu(F) \label{SUBEQooVJDXooFtCelQ} \\
				                          & <\epsilon.
			\end{align}
		\end{subequations}
		Pour \eqref{SUBEQooVJDXooFtCelQ}, c'est le lemme \ref{LemooPJLNooVKrBhN}.
	\end{subproof}
\end{proof}

%+++++++++++++++++++++++++++++++++++++++++++++++++++++++++++++++++++++++++++++++++++++++++++++++++++++++++++++++++++++++++++
\section{Produit de mesures}
%+++++++++++++++++++++++++++++++++++++++++++++++++++++++++++++++++++++++++++++++++++++++++++++++++++++++++++++++++++++++++++

Si vous cherchez le produit d'une mesure par une fonction, c'est la définition \ref{PropooVXPMooGSkyBo}.

\begin{lemma}[Propriété des sections\cite{NBoIEXO}] \label{LemAQmWEmN}
	Soient \( \tribA_1\) et \( \tribA_2\) des tribus sur les ensembles \( \Omega_1\) et \( \Omega_2\). Si \( A\in\tribA_1\otimes\tribA_2\) alors pour tout \( x\in \Omega_1\) et \( y\in\Omega_2\), les ensembles
	\begin{subequations}    \label{subEqCTtPccK}
		\begin{align}
			A_1(y)=\{ x\in\Omega_1\tq (x,y)\in A \} \\
			A_2(x)=\{ y\in\Omega_2\tq (x,y)\in A \}
		\end{align}
	\end{subequations}
	sont mesurables.
\end{lemma}
\index{section!propriété des}

\begin{proof}
	Soit \( y\in\Omega_2\); nous allons prouver le résultat pour \( A_1(y)\). Pour cela nous notons
	\begin{equation}
		S=\{ A\in \tribA_1\otimes\tribA_2\tq \forall y\in\Omega_2, A_1(y)\in\tribA_1 \},
	\end{equation}
	et nous allons noter que \( S\) est une tribu contenant les rectangles. Par conséquent, \( S\) sera égal à \( \tribA_1\otimes \tribA_2\).

	\begin{subproof}
		\spitem[Les rectangles]

		Considérons le rectangle \( A=X\times Y\) et si \( y\in \Omega_2\) alors
		\begin{equation}
			A_1(y)=\{ x\in \Omega_1\tq (x,y)\in X\times Y \}.
		\end{equation}
		Donc soit \( y\in Y\) alors \( A_1(y)=X\in\tribA_1\), soit \( y\notin Y\) et alors \( A_1(y)=\emptyset\in\tribA_1\).

		\spitem[Tribu : ensemble complet]

		Nous avons \( \Omega_1\times \Omega_2\in S\) parce que c'est un rectangle.

		\spitem[Tribu : complémentaire] Soit \( A\in S\). Montrons que \( A^c\in S\). Nous avons d'abord
		\begin{equation}
			(A^c)_1(y)=\{ x\in \Omega_1\tq (x,y)\in A^c \}.
		\end{equation}
		D'autre part
		\begin{equation}
			A_1(y)^c=\{ x\in\Omega_1\tq (x,y)\notin A \}=\{ x\in \Omega_1\tq (x,y)\in A^c \}=(A^c)_1(y).
		\end{equation}
		Vu que \( \tribA_1\) est une tribu et que par hypothèse \( A_1(y)\in\tribA_1\), nous avons aussi \( A_1(y)^c\in S\), et donc \( (A^c)_1(y)\in \tribA_1\), ce qui prouve que \( A^c\in S\).

		\spitem[Tribu : union dénombrable] Soit une suite \( A_n\in S\). Nous avons
		\begin{subequations}
			\begin{align}
				(\bigcup_nA_n)_1(y) & =\{ x\in\Omega_1\tq (x,y)\in \bigcup_nA_n \} \\
				                    & =\bigcup_n\{ x\in\Omega_1\tq (x,y)\in A_n \} \\
				                    & =\bigcup_n(A_n)_1(y),
			\end{align}
		\end{subequations}
		et ce dernier ensemble est dans \( \tribA_1\) parce que c'est une union dénombrable d'éléments de \( \tribA_1\).

	\end{subproof}
	Nous avons donc prouvé que \( S\) est une tribu contenant les rectangles, donc \( S\) contient au moins \( \tribA_1\otimes \tribA_2\).
\end{proof}

\begin{corollary}
	Si \( f\colon \Omega_1\times \Omega_2\to \eR\) est une fonction mesurable\footnote{Définition~\ref{DefQKjDSeC}.} sur \( X\times Y\) alors pour chaque \( y\) dans \( \Omega_2\), la fonction
	\begin{equation}
		\begin{aligned}
			f_y\colon X & \to \eR        \\
			x           & \mapsto f(x,y)
		\end{aligned}
	\end{equation}
	est mesurable.
\end{corollary}

\begin{proof}
	Soit \( \mO\) un ensemble mesurable de \( \eR\) (i.e. un borélien), et \( y\in \Omega_2\). Nous avons
	\begin{equation}
		f_y^{-1}(\mO)=\{ x\in X\tq f(x,y)\in \mO \}=A_1(y)
	\end{equation}
	où
	\begin{equation}
		A=\{ (x,y)\in \Omega_1\times \Omega_2\tq f(x,y)\in \mO \}=f^{-1}(\mO).
	\end{equation}
	Ce dernier est mesurable parce que \( f\) l'est.
\end{proof}

\begin{theorem}[\cite{NBoIEXO}\footnote{Modèle non contractuel : des notations et la définition de \( \lambda\)-système peuvent varier entre la référence et le présent texte.}]    \label{ThoCCIsLhO}
	Soient \( (\Omega_i,\tribA_i,\mu_i)\) (\( i=1,2\)) deux espaces mesurés \( \sigma\)-finis. Soit \( A\in\tribA_1\otimes \tribA_2\). Alors les fonctions\footnote{Voir la notation du lemme~\ref{subEqCTtPccK}.}
	\begin{subequations}
		\begin{align}
			x\mapsto\mu_2\big( A_2(x) \big) \\
			y\mapsto\mu_1\big( A_1(y) \big)
		\end{align}
	\end{subequations}
	sont mesurables et
	\begin{equation}    \label{EqRKXwsQJ}
		\int_{\Omega_1}\mu_2\big( A_2(x) \big)d\mu_1(x)=\int_{\Omega_2}\mu_1\big( A_1(y) \big)d\mu_2(y).
	\end{equation}
\end{theorem}

\begin{proof}
	Nous supposons d'abord que \( \mu_1\) et \( \mu_2\) sont finies et nous notons \( \tribD\) le sous-ensemble de \( \tribA_1\otimes \tribA_2\) sur lequel le théorème est correct. Nous allons commencer par prouver que \( \tribD\) est un \( \lambda\)-système.

	\begin{subproof}
		\spitem[\( \lambda\)-système : différence ensembliste]
		Soient \( A,B\in\tribD\) avec \( A\subset B\). Nous avons
		\begin{subequations}
			\begin{align}
				(B\setminus A)_1(y) & =\{ x\in \Omega_1\tq(x,y)\in B\setminus A \}                               \\
				                    & =\{ x\in \Omega_1\tq(x,y)\in B\}\setminus\{ x\in \Omega_1\tq(x,y)\in  A \} \\
				                    & =B_1(y)\setminus A_1(y).
			\end{align}
		\end{subequations}
		Vu que \( A_1(y)\subset B_1(y)\) et que les mesures sont finies le lemme~\ref{LemPMprYuC} nous donne
		\begin{equation}
			\mu_1\big( (B\setminus A)_1(y) \big)=\mu_1\big( B_1(y) \big)-\mu_1\big( A_1(y) \big),
		\end{equation}
		et similairement pour \( 1\leftrightarrow 2\). Les deux fonctions (de \( y\)) à droite étant mesurables, nous avons la mesurabilité de la fonction \( y\mapsto \mu_1\big( (B\setminus A)_1(y) \big)\).

		Prouvons la formule intégrale en nous rappelant que la formule \eqref{EqRKXwsQJ} est supposée correcte pour \( A\) et \( B\) séparément :
		\begin{subequations}
			\begin{align}
				\int_{\Omega_2}\mu_1\big( (B\setminus A)_1(y) \big)d\mu_2(y) & =\int_{\Omega_2}\mu_1\big( B_1(y) \big)d\mu_2(y)-\int_{\Omega_2}\mu_1\big( A_1(y) \big)d\mu_2(y) \\
				                                                             & =\int_{\Omega_1}\mu_2\big( B_2(x) \big)d\mu_1(x)-\int_{\Omega_1}\mu_2\big( A_2(x) \big)d\mu_1(x) \\
				                                                             & =\int_{\Omega_1}\mu_2\big( (B\setminus A)_2(x) \big)d\mu_1(x).
			\end{align}
		\end{subequations}


		\spitem[\( \lambda\)-système : limite de suite croissante]

		Soit \( (A_n)\) une suite croissante dans \( \tribD\); nous posons \( B_n=A_n\setminus A_{n-1}\) et \( A_0=\emptyset\) de telle sorte à travailler avec une suite d'ensembles disjoints qui satisfait \( \bigcup_nA_n=\bigcup_nB_n\). Vu que la suite est croissante nous avons \( A_{n-1}\subset A_n\) et donc \( B_n\in\tribD\) par le point déjà fait sur la différence ensembliste. Nous avons :
		\begin{subequations}
			\begin{align}
				(\bigcup_nB_n)_1(y) & =\{ x\in \Omega_1\tq (x,y)\in\bigcup_nB_n \} \\
				                    & =\bigcup_n\{ x\in\Omega_1\tq (x,y)\in B_n \} \\
				                    & =\bigcup_n (B_n)_1(y).
			\end{align}
		\end{subequations}
		Par conséquent, par la propriété~\ref{ItemQFjtOjXiii} d'une mesure nous avons
		\begin{equation}
			\mu_1\big( (\bigcup_nB_n)_1(y) \big)=\sum_n\mu_1\big( (B_n)_1(y) \big).
		\end{equation}
		En tant que somme de fonctions positives et mesurables, la fonction
		\begin{equation}
			y\mapsto\sum_n\mu_1\big( (B_n)_1(y) \big)
		\end{equation}
		est mesurable par la proposition~\ref{PropFYPEOIJ}. Il faut encore vérifier la formule intégrale. Le gros du boulot est de permuter une somme et une intégrale par le corolaire~\ref{CorNKXwhdz} :
		\begin{subequations}
			\begin{align}
				\int_{\Omega_2}\sum_n\mu_1\big( (B_n)_1(y) \big)d\mu_2(y) & =\sum_n\int_{\Omega_2}\mu_1\big( (B_n)_1(y) \big)d\mu_2(y)     \\
				                                                          & =\sum_n\int_{\Omega_1}\mu_2\big( (B_n)_2(x) \big)d\mu_1(x)     \\
				                                                          & =\int_{\Omega_1}\sum_n\mu_2\big( (B_n)_2(x) \big)d\mu_1(x)     \\
				                                                          & =\int_{\Omega_1}\mu_2\big( (\bigcup_nB_n)_1(y) \big)d\mu_1(x).
			\end{align}
		\end{subequations}
	\end{subproof}
	Maintenant que \( \tribD\) est un \( \lambda\)-système contenant les rectangles, le lemme~\ref{LemLUmopaZ} dit que la tribu engendrée par \( \tribD\) (c'est-à-dire \( \tribA_1\otimes \tribA_2\)) est le \( \lambda\)-système \( \tribD\) lui-même.

	La preuve est finie dans le cas de mesures finies. Nous commençons maintenant à prouver dans le cas où les mesures \( \mu_1\) et \( \mu_2\) sont seulement \( \sigma\)-finies. Nous considérons des suites croissantes \( \Omega_{i,n}\to\Omega_i\) d'ensembles mesurables et de mesure finie : \( \mu_i(\Omega_{i,n})<\infty\). D'abord remarquons que
	\begin{equation}\label{EqNFuBzBF}
		\mu_2\Big( (A\cap \Omega_{1,j}\times E_{2,j})_2(x) \Big)=\mu_2\Big( A_2(x)\cap \Omega_{2,j} \Big)\mtu_{\Omega_{1,j}}.
	\end{equation}
	En effet,
	\begin{subequations}
		\begin{align}
			\heartsuit & =(A\cap\Omega_{1,j}\times E_{2,j})_2(x)                                                                                  \\
			           & =\{ y\in\Omega_2\tq (x,y)\in A\cap \Omega_{1,j}\times E_{2,j} \}                                                         \\
			           & =\{ y\in \Omega_2\tq (x,y)\in A\times \Omega_{2,j} \}\cap\{ y\in\Omega_2\tq (x,y)\in \Omega_{1,j}\times \Omega_{2,j} \}.
		\end{align}
	\end{subequations}
	Si \( y\in \Omega_{1,j}\) alors \( \{ y\in \Omega_2\tq (x,y)\in \Omega_{1,j}\times \Omega_{2,j} \}=\Omega_{2,j}\) et dans ce cas
	\begin{equation}
		\heartsuit=\{ y\in \Omega_2\tq (x,y)\in A\times \Omega_{2,j} \}\cap \Omega_{2,j}=A_2(x)\cap E_{2,j}.
	\end{equation}
	Et inversement, si \( x\notin \Omega_{1,j}\) alors \( \heartsuit=\emptyset\). Dans les deux cas nous avons \eqref{EqNFuBzBF}.

	Les ensembles \( A\cap \Omega_{1,j}\times \Omega_{2,j}\) étant de mesure finie, nous pouvons leur appliquer la première partie :
	\begin{equation}
		\int_{\Omega_1}\mu_2\Big( (A\cap\Omega_{1,j}\times \Omega_{2,j})_2(x) \Big)d\mu_1(x)=\int_{\Omega_2}\mu_1\Big( (A\cap\Omega_{1,j}\times \Omega_{2,j})_1(y) \Big)d\mu_2(u),
	\end{equation}
	ou encore
	\begin{equation}
		\int_{\Omega_1}\mu_2\Big( A_2(x)\cap \Omega_{2,j} \Big)\mtu_{\Omega_{1,j}}(x)d\mu_1(x)=\int_{\Omega_2}\mu_1\Big( A_1(y)\cap \Omega_{1,j} \Big)\mtu_{\Omega_{2,j}}(y)d\mu_2(y).
	\end{equation}
	Ce que nous avons dans ces intégrales sont (par rapport à \( j\)) des suites croissantes de fonction positives; nous pouvons donc permuter une limite et une intégrale. En sachant que si \( k\to \infty\), alors
	\begin{subequations}
		\begin{align}
			\mtu_{1,j}(x)\to 1 \\
			\mu_2\big( A_2(x)\cap \Omega_2,j \big)\to\mu_2\big( A_2(x) \big),
		\end{align}
	\end{subequations}
	nous trouvons le résultat demandé.
\end{proof}

\begin{theoremDef}[\cite{FubiniBMauray,MesIntProbb}]   \label{ThoWWAjXzi}
	Soient \( \mu_i\) des mesures \( \sigma\)-finies sur\footnote{La tribu produit est la définition \ref{DefTribProfGfYTuR}.} \( (\Omega_i,\tribA_i)\) (\( i=1,2\)).
	\begin{enumerate}
		\item		\label{ITEMooMMAAooVnHmzc}

		      Il existe une et une seule mesure, notée \( \mu_1\otimes \mu_2\), sur \( (\Omega_1\times\Omega_2,\tribA_1\otimes\tribA_2)\) telle que
		      \begin{equation}    \label{EqOIuWLQU}
			      (\mu_1\otimes\mu_2)(A_1\times A_2)=\mu_1(A_1)\mu_2(A_2)
		      \end{equation}
		      pour tout \( A_1\in \tribA_1\) et \( A_2\in\tribA_2\).
		\item
		      Cette mesure est donnée par la formule\footnote{Voir les notations du lemme~\ref{LemAQmWEmN}.}
		      \begin{equation}   \label{EqDFxuGtH}
			      (\mu_1\otimes \mu_2)(A)=\int_{\Omega_1}\mu_2\big( A_2(x) \big)d\mu_1(x)=\int_{\Omega_2}\mu_1\big( A_1(y) \big)d\mu_2(y)
		      \end{equation}
		      où \( A_1(y)\) et \( A_2(x)\) sont définis en \eqref{subEqCTtPccK}.

		      Cette mesure est la \defe{mesure produit}{mesure!produit} de \( \mu_1\) par \( \mu_2\).
		\item
		      La mesure \( \mu_1\otimes \mu_2\) ainsi définie est \( \sigma\)-finie.
		\item		\label{ITEMooUOATooQpdcND}
		      Le produit de mesures est associatif : si \( \mu_1\), \( \mu_2\) et \( \mu_3\) sont des mesures, alors \( \sigma\)-finies, alors
		      \begin{equation}
			      (\mu_1\otimes \mu_2)\otimes \mu_3=\mu_1\otimes (\mu_2\otimes \mu_3).
		      \end{equation}
		      %TODOooFAKJooWLsMIy Prouver l'associativité.
	\end{enumerate}
\end{theoremDef}
\index{mesure!produit}

\begin{proof}
	La partie «existence» sera divisée en deux parties : l'une pour prouver que les formules \eqref{EqDFxuGtH} donnent une mesure et une pour montrer que cette mesure vérifie la condition \eqref{EqOIuWLQU}.
	\begin{subproof}
		\spitem[Unicité]

		L'ensemble des rectangles de \( \Omega_1\times \Omega_2\) engendre la tribu \( \tribA_1\otimes\tribA_2\), est fermé par intersection et contient une suite croissante d'ensembles \( P_n\times R_n\) de mesure finie (\( \mu(P_n\times R_n)<\infty\)) telle que \( P_n\times R_n\to \Omega_1\times \Omega_2\). Cette suite est donnée par le fait que \( \mu_1\) et \( \mu_2\) sont \( \sigma\)-finies. En effet si \( (X_n)\) et \( (Y_n)\) sont des recouvrements dénombrables de \( \Omega_1\) et \( \Omega_2\) par des ensembles de mesure finie, en posant \( P_n=\bigcup_{k=1}^nX_k\) et \( R_n=\bigcup_{k=1}^nY_k\) nous avons bien une suite croissante de rectangles qui tendent vers \( \Omega_1\times \Omega_2\). Avec ces rectangles en main, le théorème~\ref{ThoJDYlsXu} donne l'unicité.

		\spitem[Les formules définissent une mesure]
		Le théorème~\ref{ThoCCIsLhO} dit que ces formules ont un sens et que l'égalité entre les deux intégrales est correcte. Nous prouvons à présent qu'elles déterminent effectivement une mesure sur \( (\Omega_1\times\Omega_2,\tribA_1\otimes \tribA_2)\).

		Pour tout \( A\in \tribA_1\otimes \tribA_2\), \( \mu(A)\geq 0\) parce que \( \mu\) est donnée par l'intégrale d'une fonction positive.

		En ce qui concerne la condition d'unions dénombrable disjointe, soient \( A^{(i)}\) des éléments disjoints de \( \tribA_1\otimes \tribA_2\); nous commençons par remarquer que
		\begin{subequations}
			\begin{align}
				\left( \bigcup_{i=1}^{\infty}A^{(i)} \right)_2(x) & =\{ y\in\Omega_2\tq (x,y)\in\bigcup_{i=1}^{\infty}A^{(i)} \}  \\
				                                                  & =\bigcup_{i=1}^{\infty}\{ y\in\Omega_2\tq (x,y)\in A^{(i)} \} \\
				                                                  & =\bigcup_{i=1}^{\infty}A^{(i)}_2(x).
			\end{align}
		\end{subequations}
		Par conséquent,
		\begin{subequations}
			\begin{align}
				\mu\left( \bigcup_{i=1}^{\infty}A^{(i)} \right) & =\int_{\Omega_1}\mu_2\left(    \Big( \bigcup_{i=1}^{\infty}A^{(i)} \Big)_2(x)     \right)d\mu_1(x) \\
				                                                & =\int_{\Omega_1}\sum_{i=1}^{\infty}\mu_2\big( A^{(i)}_2(x) \big)d\mu_1(x)                          \\
				                                                & =\int_{\Omega_1}\lim_{n\to \infty} \sum_{i=1}^{n}\mu_2\big( A^{(i)}_2(x) \big)d\mu_1(x).
			\end{align}
		\end{subequations}
		où nous avons utilisé l'additivité de la mesure \( \mu_2\). À ce niveau, il serait commode de permuter la somme et l'intégrale. Pour ce faire nous considérons la suite (croissante) de fonctions
		\begin{equation}
			f_n(x)=\sum_{i=1}^n\mu_2\big( A_2^{(i)}(x) \big).
		\end{equation}
		Nous pouvons permuter la limite et l'intégrale grâce au théorème de la convergence monotone~\ref{ThoRRDooFUvEAN}; ensuite la somme se permute avec l'intégrale en tant que somme finie :
		\begin{subequations}
			\begin{align}
				\mu\left( \bigcup_{i=1}^{\infty}A^{(i)} \right) & =\lim_{n\to \infty} \sum_{i=1}^n\int_{\Omega_1}\big( A_2^{(i)}(x) \big)d\mu_1(x) \\
				                                                & =\lim_{n\to \infty} \sum_{i=1}^n\mu(A^{(i)})                                     \\
				                                                & =\sum_{i=1}^{\infty}\mu( A^{(i)} ).
			\end{align}
		\end{subequations}

		\spitem[Elles vérifient la condition]
		Prouvons que les formules \eqref{EqDFxuGtH} se réduisent à \eqref{EqOIuWLQU} dans le cas des rectangles. Soit donc \( A=X_1\times X_2\) avec \( X_i\in\tribA_i\). Alors
		\begin{equation}
			A_1(y)=\{ x\in\Omega_1\tq (x,y)\in X_1\times X_2 \}
		\end{equation}
		et
		\begin{equation}
			\mu_1\big( A_1(y) \big)=\mtu_{X_2}(y)\mu_1(X_1),
		\end{equation}
		donc
		\begin{subequations}
			\begin{align}
				(\mu_1\otimes\mu_2)(A) & =\int_{\Omega_2}\mu_1\big( A_1(y) \big)d\mu_2(y) \\
				                       & =\int_{\Omega_2}\mu_1(X_1)\mtu_{X_2}(y)d\mu_2(y) \\
				                       & =\mu_1(X_1)\int_{\Omega_2}\mtu_{X_2}(y)d\mu_2(y) \\
				                       & =\mu_1(X_1)\mu_2(X_2).
			\end{align}
		\end{subequations}
		Pour cela nous avons utilisé le fait que l'intégrale de la fonction caractéristique d'un ensemble mesurable est la mesure de cet ensemble.
	\end{subproof}
\end{proof}

\begin{definition}[Produit d'espaces mesurés]  \label{DefUMlBCAO}
	Si \( (\Omega_i,\tribA_i,\mu_i)\) sont deux espaces mesurés, l'\defe{espace produit}{produit!espaces mesurés} est l'ensemble \( \Omega_1\times \Omega_2\) muni de la tribu produit \( \tribA_1\otimes \tribA_2\) de la définition~\ref{DefTribProfGfYTuR} et de la mesure produit \( \mu_1\otimes \mu_2\) définie par le théorème~\ref{ThoWWAjXzi}.
\end{definition}

\begin{remark}
	Il n'est pas garanti que la tribu \( \tribA_1\otimes\tribA_2\) soit la tribu la plus adaptée à l'ensemble \( S_1\times S_2\). Dans le cas de \( \eR^N\), il se fait que c'est le cas : en prenant des produits des boréliens sur \( \eR\) on obtient bien les boréliens sur \( \eR^N\), voir proposition~\ref{CorWOOOooHcoEEF}.
\end{remark}

%+++++++++++++++++++++++++++++++++++++++++++++++++++++++++++++++++++++++++++++++++++++++++++++++++++++++++++++++++++++++++++
\section{Tribu et mesure de Lebesgue sur \texorpdfstring{\(  \eR^d\)}{Rd}}
%+++++++++++++++++++++++++++++++++++++++++++++++++++++++++++++++++++++++++++++++++++++++++++++++++++++++++++++++++++++++++++

\begin{definition}[Mesure de Lebesgue]      \label{DEFooSWJNooCSFeTF}
	En plusieurs étapes.
	\begin{enumerate}
		\item
		      D'abord nous avons la mesure \( \lambda_N\) sur \( \eR^N\) définie sur
		      \begin{equation}
			      \big( \eR^N,\Borelien(\eR)\otimes\ldots\otimes\Borelien(\eR) \big)
		      \end{equation}
		      comme le produit \( \lambda\otimes\ldots\otimes \lambda\) via la définition~\ref{DefUMlBCAO} (et grâce au fait que le produit est associatif, théorème \ref{ThoWWAjXzi}\ref{ITEMooUOATooQpdcND}).
		\item
		      Ensuite nous nous souvenons du corolaire~\ref{CorWOOOooHcoEEF} qui donne \( \lambda_N\) comme une mesure sur
		      \begin{equation}
			      \big( \eR^N,\Borelien(\eR^N) \big).
		      \end{equation}
		\item
		      Et enfin nous considérons la complétion de la mesure \( \lambda_N\) (théorème~\ref{thoCRMootPojn}), que nous notons encore \( \lambda_N\).
	\end{enumerate}
\end{definition}

\begin{proposition}[\cite{ooRCYWooNAeaTA}]     \label{PropSKXGooRFHQst}
	Tout ouvert de \( \eR^n\) est une union dénombrable et disjointe de cubes semi-ouverts.
\end{proposition}

\begin{proof}
	Nous allons même montrer que ces cubes peuvent être choisis sur un quadrillage.

	Soit \( G\) un ouvert de \( \eR^n\). Soit \( \{ Q_i^{1} \}_{i\in \eN}\) un découpage de \( \eR^n\) en cubes semi-ouverts de côté \( 1\) et dont les sommets sont en les coordonnées entières. Ils sont de la forme
	\begin{equation}
		\prod_{i=1}^n\mathopen[ n_i , n_i+1 \mathclose[
	\end{equation}
	où les \( n_i\) sont des entiers. Ce sont des cubes disjoints. Nous considérons ensuite pour chaque \( k>1\) le découpage \( \{ Q_i^{(k)} \}_{i\in\eN}\) de \( \eR^n\) en cubes de côtés \( 2^{-k}\) qui consiste à découper en \( 2\) les côtés des cubes du découpage \( Q^{(k-1)}\). Ces cubes forment encore un découpage dénombrable de \( \eR^n\) en des cubes disjoints. Ils sont de la forme
	\begin{equation}
		\prod_{i=1}^n\mathopen[ \frac{ n_i }{ 2^k } , \frac{ n_i+1 }{ 2^k } \mathclose[
	\end{equation}
	où les \( n_i\) sont encore entiers. Ensuite nous considérons \( \mE\) l'union de tous les \( Q_i^{(k)}\) contenus dans \( G\).

	Montrons que \( \mE=G\). D'abord \( \mE\subset G\) parce que \( \mE\) est une union d'ensembles contenus dans \( G\). Ensuite si \( x\in G\), il existe une boule de rayon \( r\) autour de \( x\) contenue dans \( G\); alors un des ensembles \( Q_i^{(k)}\) avec \( 2^{-j}<\frac{ r }{2}\) est contenue dans \( B(x,r)\) et donc dans \( \mE\).

	Bien entendu l'union qui donne \( \mE\) n'est pas satisfaisante par ce que les \( Q_i^{(k+1)}\) sont contenus dans les \( Q_i^{(k)}\); les intersections sont donc loin d'être vides.

	Nous faisons ceci :
	\begin{subequations}
		\begin{align}
			R^{(0)}   & =\{ Q_i^{(1)} \text{contenu dans } G \}                                \\
			R^{(k+1)} & =\{ Q_i^{(k+1)}\text{contenus dans } G\text{ et pas dans } R^{(k)} \}.
		\end{align}
	\end{subequations}
	En fin de compte l'union de tous les ensembles contenus dans les \( R^{(k)}\) forment encore \( \eR^n\), mais sont d'intersection vide.
\end{proof}

Les cubes dont il est question dans cette preuve, de côtés \( 2^{-k}\) sont souvent appelés des cubes \defe{dyadiques}{dyadique}.

%---------------------------------------------------------------------------------------------------------------------------
\subsection{Propriétés d'unicité}
%---------------------------------------------------------------------------------------------------------------------------

\begin{corollary}       \label{CorMPDAooDJRrom}
	La mesure \( \lambda_N\) est l'unique mesure sur \(   (\eR^N,  \Borelien(\eR^N) )   \) à satisfaire
	\begin{equation}
		\mu\big( \prod_{i=1}^N\mathopen[ a_i , b_i \mathclose] \big)=\prod_{i=1}^N| a_i-b_i |
	\end{equation}
\end{corollary}

\begin{proof}
	Par définition de la mesure produit, \( \lambda_N\) est l'unique mesure sur \(   (\eR^N,  \Borelien(\eR)\otimes\ldots\otimes\Borelien(\eR) )   \) à satisfaire la condition. La proposition~\ref{CorWOOOooHcoEEF} conclut.
\end{proof}

Vu que les compacts de \( \eR^n\) sont les fermés bornés (théorème~\ref{ThoXTEooxFmdI}), et que tout borné est dans un tel produit d'intervalle, la mesure de Lebesgue est une mesure de Borel (définition~\ref{DefFMTEooMjbWKK}\ref{ItemTTPTooStDcpw}).

\begin{theorem}[\cite{PMTIooJjAmWR}]        \label{THOooTMWHooThsDHj}
	La mesure de Lebesgue est invariante par translation. Autrement dit si \( A\) est mesurable dans \( \eR^n\) et si \( a\in \eR^n\) alors \( A+a\) est mesurable et
	\begin{equation}
		\lambda_N(A+a)=\lambda_N(A).
	\end{equation}
\end{theorem}

\begin{proof}
	Nous supposons que \( A\) est borélien; sinon il l'est à ensemble négligeable près. Nous nommons \( \mu\) la mesure donnée par
	\begin{equation}
		\mu(A)=\lambda_N(A+a).
	\end{equation}
	Vu que
	\begin{equation}
		\mu\big( \prod_{n=1}^N\mathopen[ r_n , s_n \mathclose[ \big)=\lambda_N\big( \prod_i\mathopen[ r_n+a_n , s_n+a_n [ \big)=\prod_i| s_n-r_n |.
	\end{equation}
	Vu qu'il y a unicité de la mesure vérifiant cette propriété (corolaire~\ref{CorMPDAooDJRrom}), nous avons \( \mu=\lambda_N\).
\end{proof}

Pour la suite nous notons \( Q_0\) le cube unité de \( \eR^N\) : \( Q_0=\big( \mathopen[ 0 , 1 \mathclose[ \big)^N\).

\begin{theorem}[\cite{PMTIooJjAmWR}]        \label{ThoCABFooHbUzWc}
	Nous notons  \( Q_0=\big( \mathopen[ 0 , 1 \mathclose[ \big)^N\). Soit \( \mu\) une mesure positive sur \( \eR^N\) telle que
	\begin{enumerate}
		\item
		      \( \mu\) soit invariante par translation (des boréliens),
		\item
		      \( \mu(Q_0)=1\).
	\end{enumerate}
	Alors \( \mu=\lambda_N\).
\end{theorem}

\begin{proof}
	Pour simplifier l'écriture nous faisons \( N=2\). Notre but est de prouver que \( \mu(  \mathopen[ 0 , r \mathclose[\times \mathopen[ 0 , r' \mathclose[ )=rr'\) pour tout \( r,r'\in \eR\).

	\begin{subproof}
		\spitem[Longueur =\( 1/J\)]
		Soient \( J,K\) des entiers. Nous pouvons diviser le cube \( Q_0\) en rectangles de côtés \( 1/J\) et \( 1/K\) :
		\begin{equation}
			Q_0=\bigcup_{\substack{1\leq j\leq J\\1\leq k\leq K}}\mathopen[ \frac{ j-1 }{ J } , \frac{ j }{ J } \mathclose[\times \mathopen[ \frac{ k-1 }{ K } , \frac{ k }{ K } \mathclose[
		\end{equation}
		où l'union est disjointe. En ce qui concerne la mesure nous commençons par utiliser la sous-additivité :
		\begin{equation}
			\mu(Q_0)=\sum_{\substack{1\leq j\leq J\\1\leq k\leq K}}\mu\left(  \mathopen[ \frac{ j-1 }{ J } , \frac{ j }{ J } \mathclose[\times \mathopen[ \frac{ k-1 }{ K } , \frac{ k }{ K } \mathclose[      \right).
		\end{equation}
		Nous utilisons ensuite, sur chacun des termes séparément l'invariance par translation selon les vecteurs \( (\frac{ j-1 }{ J },0)\) et \( ( 0,\frac{ k-1 }{ K } )\) :
		\begin{equation}
			1=\mu(Q_0)=\sum_{\substack{1\leq j\leq J\\1\leq k\leq K}}\mu\left(  \mathopen[ 0,\frac{1}{ J } \mathclose[\times \mathopen[0,\frac{1}{ K }\mathclose[      \right)=JK\mu\left(  \mathopen[ 0,\frac{1}{ J } \mathclose[\times \mathopen[0,\frac{1}{ K }\mathclose[      \right),
		\end{equation}
		et donc
		\begin{equation}
			\mu\left(  \mathopen[ 0,\frac{1}{ J } \mathclose[\times \mathopen[0,\frac{1}{ K }\mathclose[      \right)=\frac{1}{ J }\times \frac{1}{ K }.
		\end{equation}
		\spitem[Longueur \( L/K\)]

		Soient \( L,M\) des entiers et calculons :
		\begin{subequations}
			\begin{align}
				\mu\left( \mathopen[ \frac{ 0 }{ J } , \frac{ L }{ J } \mathclose[\times \mathopen[ \frac{ 0 }{ K } , \frac{ M }{ K } \mathclose[ \right) & =\sum_{\substack{0\leq l\leq L-1                                                                                                               \\0\leq m\leq M-1}}\mu\left(   \mathopen[    \frac{ l }{ J },\frac{ l+1 }{ J }  \mathclose[\times \mathopen[ \frac{ m }{ K } , \frac{ m+1 }{ K } \mathclose[      \right)\\
				                                                                                                                                          & =LM\mu\left(  \mathopen[ \frac{ 0 }{ J } , \frac{ 1 }{ J } \mathclose[\times \mathopen[ \frac{ 0 }{ K } , \frac{ 1 }{ K } \mathclose[  \right) \\
				                                                                                                                                          & =LM\times \frac{1}{ J }\times \frac{1}{ K }.
			\end{align}
		\end{subequations}
		Nous avons donc, pour tout \( J,K,L,M\) :
		\begin{equation}
			\mu\left( \mathopen[ 0 , \frac{ L }{ J } \mathclose[\times \mathopen[ 0, \frac{ M }{ K } \mathclose[ \right)=\frac{ L }{ J }\times \frac{ M }{ K },
		\end{equation}
		c'est-à-dire que pour tout \( r,s\in \eQ^+\) nous avons
		\begin{equation}
			\mu\big(   \mathopen[ 0 , r \mathclose[\times \mathopen[ 0 , s \mathclose[ \big)=rs.
		\end{equation}
		\spitem[Longueur réelle]
		Nous passons au cas de longueur réelle. Soit \( a>0\) et une suite croissante de rationnels \( r_n\to a\). Une telle suite existe par la proposition~\ref{PropSLCUooUFgiSR}. L'intervalle \( \mathopen[ 0 , a \mathclose[\) s'écrit sous la forme d'une union croissante \( \mathopen[ 0 , a \mathclose[=\bigcup_{n\geq 1}\mathopen[ 0 , r_n \mathclose[\); le lemme~\ref{LemAZGByEs}\ref{ItemJWUooRXNPci} peut être utilisé et nous avons
		\begin{equation}
			\mu\big( \mathopen[ 0 , a \mathclose[ \big)=\mu\left( \bigcup_{n\geq 1}\mathopen[ 0 , r_n \mathclose[ \right)=\lim_{n\to \infty} \mu\big( \mathopen[ 0 , r_n \mathclose[ \big)=\lim_{n\to \infty} r_n=a.
		\end{equation}
	\end{subproof}

	Enfin, si \( a,a'\in \eR\), l'invariance par translation donne
	\begin{equation}
		\mu\big( \mathopen[ a , a' \mathclose[ \big)=\mu\big( \mathopen[ 0 , a'-a \mathclose[ \big)=a'-a.
	\end{equation}
	Par unicité de la mesure ayant cette propriété, nous avons \( \mu=\lambda_N\).
\end{proof}

\begin{corollary}       \label{CorKGMRooHWOQGP}
	Si \( \mu\) est une mesure positive sur \( \eR^N\) invariante par translation et telle que \( \mu(Q_0)=C<\infty\) alors \( \mu=C\lambda_N\).
\end{corollary}

\begin{proof}
	Si \( C>0\) nous considérons la mesure \( \frac{1}{ C }\mu\) qui vérifie \( (\frac{1}{ C }\mu)(Q_0)=1\). En conséquence du théorème~\ref{ThoCABFooHbUzWc}, \( \frac{1}{ C }\mu=\lambda_N\) et \( \mu=C\lambda_N\).

	Si au contraire \( C=0\) alors nous pouvons paver \( \eR^N\) avec des cubes \( Q_i\) de côté \( 1\) qui ont tous mesure \( 0\). Par conséquent, \( \eR^N=\bigcup_{i=1}^{\infty}Q_i\), donc \( \mu(\eR^N)=\sum_i\mu(Q_i)=0\). Par conséquent \( \mu=0\) parce que toute partie de \( \eR^N\) a une mesure au maximum égale à celle de \( \eR^N\).
\end{proof}

%---------------------------------------------------------------------------------------------------------------------------
\subsection{Régularité}
%---------------------------------------------------------------------------------------------------------------------------

Les différentes notions de régularité pour une mesure sont données dans la définition~\ref{DefFMTEooMjbWKK}. Ce sont essentiellement des questions de compatibilité entre la mesure et la topologie.
\begin{proposition}
	La mesure de Lebesgue est une mesure de Radon sur tout ouvert de \( \eR^N\).
\end{proposition}

\begin{proof}
	Soit \( V\) un ouvert de \( \eR^N\). C'est localement compact et \( \sigma\)-compact\footnote{Définition \ref{DEFooSOLWooWlSiUn}.}. Il suffit de prouver que \( \lambda_N\) est de Borel sur \( V\) pour que le théorème~\ref{PropNCASooBnbFrc} conclue à la régularité de la mesure de Lebesgue.

	Soit \( K\) un compact de \( V\). Par la proposition~\ref{PropGBZUooRKaOxy} c'est également un compact de \( \eR^N\). Par conséquent \( K\) est dans un pavé fermé de \( \eR^N\) du type
	\begin{equation}
		K\subset \prod_{n=1}^N\mathopen[ a_n , b_n \mathclose]
	\end{equation}
	et donc en passant par le corolaire~\ref{CorMPDAooDJRrom},
	\begin{equation}
		\lambda_N(K)\leq \prod_{i=1}^N(b_n-a_n)<\infty.
	\end{equation}
	Nous avons démontré que \( \lambda_N\) reste fini sur tout compact de \( V\).
\end{proof}
