% This is part of Mes notes de mathématique
% Copyright (c) 2011-2024
%   Laurent Claessens
% See the file fdl-1.3.txt for copying conditions.

%+++++++++++++++++++++++++++++++++++++++++++++++++++++++++++++++++++++++++++++++++++++++++++++++++++++++++++++++++++++++++++ 
\section{Sur \( \mathopen\lbrack -T , T \mathclose\lbrack\)}
%+++++++++++++++++++++++++++++++++++++++++++++++++++++++++++++++++++++++++++++++++++++++++++++++++++++++++++++++++++++++++++

Pour rappel, les éléments de \( L^2\) sont des classes de fonctions à valeurs dans \( \eC\).

\begin{proposition}     \label{PROPooHNJZooGfRCfU}
	Les fonctions
	\begin{equation}
		\begin{aligned}
			e_n\colon \mathopen[ -T , T \mathclose] & \to \eC                                        \\
			t                                       & \mapsto \frac{1}{ \sqrt{ 2T } } e^{\pi int/T}.
		\end{aligned}
	\end{equation}
	forment une base hilbertienne\footnote{Définition \ref{DEFooADQXooFoIhTG}.} de \( L^2\big( \mathopen[ -T , T \mathclose[ \big)\).
\end{proposition}

\begin{proof}
	C'est un cas particulier du théorème \ref{THOooAVWIooDhnjpN}.
\end{proof}

%--------------------------------------------------------------------------------------------------------------------------- 
\subsection{Le cas dans \( \mathopen[ 0 , 2\pi \mathclose]\)}
%---------------------------------------------------------------------------------------------------------------------------

En pratique, nous n'allons pas souvent travailler avec des fonctions sur intervalle symétrique \( \mathopen[ -T , T \mathclose]\), mais le plus souvent nous serons sur \( \mathopen[ 0 , 2\pi \mathclose]\).

Nous notons ici une conséquence du théorème~\ref{ThoGVmqOro} dans le cas de l'espace \( L^2\). La proposition suivante est une petite partie du corolaire~\ref{CorQETwUdF}, qui sera d'ailleurs démontré de façon indépendante.

\begin{proposition}
	Si nous avons une suite de réels \( (a_k)\) telle que \( \sum_{k=0}^{\infty}| a_k |^2<\infty\) alors la suite
	\begin{equation}
		f_n(x)=\sum_{k=0}^na_k e^{ikx}
	\end{equation}
	converge dans \( L^2\big( \mathopen] 0 , 2\pi \mathclose[ \big)\).
\end{proposition}

\begin{proof}
	Quitte à séparer les parties réelles et imaginaires, nous pouvons faire abstraction du fait que nous parlons d'une série de fonctions à valeurs dans \( \eC\) au lieu de \( \eR\).

	Un simple calcul est :
	\begin{equation}    \label{EqHVdJxZT}
		\| f_n-f_m \|^2\leq\int_0^{2\pi}\sum_{k=n}^m| a_k |^2dx\leq 2\pi\sum_{k=n}^m| a_k |^2.
	\end{equation}
	Par hypothèse le membre de droite est \( | s_m-s_n |\) où \( s_k\) dénote la suite des sommes partielles de la série des \( | a_k |^2\). Cette dernière est de Cauchy (parce que convergente dans \( \eR\)) et donc la limite \( n\to\infty\) (en gardant \( m>n\)) est zéro. Donc la suite des \( f_n\) est de Cauchy dans \( L^2\) et donc converge dans \( L^2\).
\end{proof}

\begin{normaltext}
	Adaptons tout cela pour l'espace \( L^2\big( \mathopen[ 0 , 2\pi \mathclose] \big)\). Nous posons
	\begin{equation}        \label{EQooBFKDooMkCZOt}
		\langle f, g\rangle =\int_0^{2\pi}f(t)\overline{ g(t) }dt
	\end{equation}
	et
	\begin{equation}        \label{EQooKMYOooLZCNap}
		e_n(t)=\frac{1}{ \sqrt{ 2\pi } } e^{int}.
	\end{equation}
\end{normaltext}

\begin{normaltext}
	Attention que \( e_n(x)\) n'est pas exactement \(  e^{inx}\) : il y a un coefficient. Lorsque ça a un sens, la théorie de Fourier permet d'écrire
	\begin{equation}
		f(x)=\sum_{n\in \eZ}c_n(f) e^{inx}.
	\end{equation}
	Ici les \( c_n(f)\) sont les coefficients de Fourier de \( f\). Ce développement n'est pas le même que
	\begin{equation}
		f(x)=\sum_{n\in \eZ}a_n(f)e_n(x).
	\end{equation}
	Dans cette dernière égalité, les \( a_n(f)\) ne sont pas les coefficients de Fourier, mais \( a_n=\langle f, e_n\rangle \). Le lien entre les deux est fondamentalement l'objet du corolaire \ref{CordgtXlC}.
\end{normaltext}


L'importance du système trigonométrique défini en \ref{DEFooGCZAooFecAHB} est d'être une base de \( L^2\big( \mathopen[ 0 , 2\pi \mathclose] \big)\), comme précisé dans le lemme suivant.
\begin{lemma}       \label{LEMooBJDQooLVPczR}
	Le système trigonométrique \( \{ e_n \}_{n\in \eZ}\) est une base hilbertienne\footnote{Définition \ref{DEFooADQXooFoIhTG}.} de \( L^2\big( \mathopen[ 0 , 2\pi \mathclose] \big)\).
\end{lemma}

\begin{proof}
	Cas particulier du théorème \ref{THOooAVWIooDhnjpN}.
\end{proof}

Note : le théorème~\ref{ThoDPTwimI} donne aussi la densité, mais sera démontré plus tard, indépendamment. Voir aussi les thèmes~\ref{THEooPUIIooLDPUuq} et~\ref{THEMooNMYKooVVeGTU}.

Pour un élément donné \( f\in L^2\big( \mathopen[ 0 , 2\pi \mathclose] \big)\), nous définissons\nomenclature[Y]{\( S_nf\)}{somme partielle de série de Fourier}
\begin{equation}
	S_nf=\sum_{k=-n}^n\langle f, e_k\rangle e_k
\end{equation}
et nous avons le théorème suivant, qui récompense les efforts consentis à propos de la densité des polynômes trigonométriques dans \( L^2\).

\begin{theorem} \label{ThoYDKZLyv}
	Soit \( f\in L^2\big( \mathopen[ 0 , 2\pi \mathclose] \big)\). Nous avons égalité\footnote{Notons que la somme sur \( \eZ\) dans \eqref{EqXMMRpSN} est commutative; il n'est donc pas besoin d'être plus précis.}
	\begin{equation}    \label{EqXMMRpSN}
		f=\sum_{n\in \eZ}c_n(f)e_n
	\end{equation}
	dans \( L^2\).

	Nous avons aussi la convergence
	\begin{equation}    \label{EqRBWKsYP}
		S_nf\stackrel{L^2}{\to} f.
	\end{equation}
\end{theorem}

\begin{proof}
	Le système trigonométrique \( \{ e_n \}_{n\in \eZ}\) est total pour l'espace de Hilbert \( L^2\big( \mathopen[ 0 , 2\pi \mathclose] \big)\) (sans périodicité particulière). Donc le point~\ref{ItemQGwoIxi} du théorème~\ref{ThoyAjoqP} nous donne l'égalité demandée.

	La convergence \eqref{EqRBWKsYP} est une reformulation de l'égalité \eqref{EqXMMRpSN}.
\end{proof}

\begin{normaltext}
	Obtenir la convergence \( L^2\) ne demande pas d'hypothèses de périodicité : la convergence \eqref{EqRBWKsYP} est automatique du fait que le système trigonométrique soit total. Ce n'est cependant pas plus qu'une convergence \( L^2\) et elle ne demande pas \( f(0)=f(2\pi)\), même si pour chacun des \( e_k\) nous avons \( e_k(0)=e_k(2\pi)\).

	Si \( f(2\pi)\neq f(0)\), alors il existe tout de même une suite \( (f_n)\) convergente vers \( f\) au sens \( L^2\) telle que \( f_n(0)=f_n(2\pi)\). Cela ne contredit en rien le fait que \( e_k(0)=e_k(2\pi)\) parce que dans \( L^2\), la valeur d'un point seul n'a pas d'importance.

	Si nous voulons une vraie convergence ponctuelle ou uniforme \( (S_nf)(x)\to f(x)\), alors il faut ajouter des hypothèses sur la continuité de \( f\), sa périodicité ou le comportement des coefficients \( c_n\). Voir aussi le thème~\ref{THMooHWEBooTMInve}.
\end{normaltext}

\begin{example}     \label{EXooQDWUooLtuIOm}
	Si \( f\in L^2\big( \mathopen[ 0 , 2\pi \mathclose] \big)\) est (la classe de) une fonction à valeurs réelles, alors on peut la développer avec nettement moins de termes. D'abord nous savons que \( e_{-n}=\overline{ e_n }\), et donc
	\begin{equation}
		\langle f, e_n\rangle =\overline{ \langle f, e_{-n}\rangle  },
	\end{equation}
	ce qui donne
	\begin{equation}
		f=\sum_{n\in\eZ}\langle f, e_n\rangle e_n
		=\sum_{n>0}\langle f, e_n\rangle e_n +\sum_{n<0}\overline{ \langle f, e_n\rangle e_n }+\langle f, e_0 \rangle e_0
		=\sum_{n\in \eN}\Re\big( \langle f, e_n\rangle e_n \big).
	\end{equation}
	Notez que \( f\) étant supposée réelle et \( e_0\) étant la fonction constante (réelle) \( 1/\sqrt{ 2\pi }\), le terme \( n=0\) est bien réel.

	Or
	\begin{equation}        \label{EQooMWJNooSjPCpR}
		\Re\big( \langle f, e_n\rangle e_n \big)=\frac{1}{ (2\pi)^{3/2} }\cos(nx)\int_0^{2\pi}f(t)\cos(nt)dt-\frac{1}{ (2\pi)^{3/2} }\sin(nx)\int_0^{2\pi}f(t)\sin(nt)dt.
	\end{equation}

	Considérons la fonction impaire \( \tilde f\in L^2\big( [-2\pi,2\pi] \big)\) créée à partir de \( f\). Elle se développe de même et nous avons la même formule \eqref{EQooMWJNooSjPCpR} à part quelques coefficients et le fait que les intégrales sont entre \( -2\pi\) et \( 2\pi\). Vu que \( \tilde f\) est impaire, l'intégrale avec \( \cos(nt)\) s'annule et
	\begin{equation}
		\tilde f(x)=\sum_{n\in \eN}c_n\sin(nx)
	\end{equation}
	pour certains coefficients réels \( c_n\). Cette égalité est à considérer dans \( L^2\), c'est-à-dire presque partout et en particulier presque partout sur \( \mathopen[ 0 , 2\pi \mathclose]\).

	Donc les fonctions réelles sur \( \mathopen[ 0 , 2\pi \mathclose]\) peuvent être écrites sous la forme d'une série de seulement des sinus.

	Note : en choisissant \( \tilde f\) paire, nous aurions eu une série de cosinus.
\end{example}


%+++++++++++++++++++++++++++++++++++++++++++++++++++++++
\section{Des inégalités}
%+++++++++++++++++++++++++++++++++++++++++++++++++++++++


Avant d'entrer dans le vif du sujet, nous nous fendons d'une petite étude de fonction. Soit
\begin{equation}
	\begin{aligned}
		\phi\colon \mathopen[ 0 , 1 \mathclose] & \to \eR                            \\
		x                                       & \mapsto \frac{ (1+x)^r }{ 1+x^r }.
	\end{aligned}
\end{equation}
Un peu de calcul montre que
\begin{equation}
	\frac{ \phi'(x) }{ \phi(x) }=\frac{ r(1-x^{r-1}) }{ (1+x^r)(1+x) }.
\end{equation}

%-------------------------------------------------------
\subsection{c-r inégalité}
%----------------------------------------------------

\begin{lemma}       \label{LEMooFKKEooDTypUd}
	Soient \( a,b>0\) et \( r\geq 1\). Nous avons les inégalités
	\begin{equation}
		a^r+b^r\leq (a+b)^r\leq 2^{r-1}(a^r+b^r).
	\end{equation}
\end{lemma}

\begin{proof}
	Notons d'abord que si \( r=1\), nous avons juste de égalités; pas de problèmes. Nous supposons donc que \( r>1\). Pour la première inégalité, nous posons \( f(x)=a^r+x^r\) et \( g(x)=(a+x)^r\). Nous avons \( f(0)=g(0)=a^r\), et, en utilisant la fonction \( f_{\alpha}\) définite par \( f_{\alpha}(x)=x^{^\alpha}\), nous avons
	\begin{subequations}
		\begin{align}
			f'(x) & =rx^{r-1} =rf_{r-1}(x)       \\
			g'(x) & =r(a+x)^{r-1}=rf_{r-1}(a+x).
		\end{align}
	\end{subequations}
	Vu que \( r>1\), la fonction \( f_{r-1}\) est strictement croissante sur les positifs par la proposition \ref{PROPooUOFKooYyGwIr}. Donc pour \( x\geq 0\) nous avons
	\begin{equation}
		f'(x)=rf_{r-1}(x)\leq rf_{r-1}(a+x)=g'(x).
	\end{equation}
	Nous avons donc \( f(b)\leq g(b)\) comme souhaité.

	Nous passons à la seconde inégalité. Le lemme \ref{LEMooSXTXooZOmtKq} nous dit que la fonction \( f\colon x\mapsto x^r \) est convexe. Donc
	\begin{equation}
		f\left( \frac{ a }{2}+\frac{ b }{2} \right)\leq\frac{ 1 }{2}f(a)+\frac{ 1 }{2}f(b).
	\end{equation}
	De là nous déduisons
	\begin{equation}
		\frac{ (a+b)^r }{ 2^r }\leq \frac{ 1 }{2}(a^r+b^r),
	\end{equation}
	c'est-à-dire la seconde inégalité.
\end{proof}

Comme semble indiquer le code suivant, la seconde inégalité du lemme \ref{LEMooFKKEooDTypUd} semble vraie pour tous les complexes.

\lstinputlisting{tex/frido/codeSnip_5.py}


\begin{proposition}[\cite{BIBooCPELooXNADgx}]	\label{PROPooPMIGooRAjROX}
	Soient \( \alpha,\beta\in \eC\) et \( r\geq 1\).Nous avons
	\begin{equation}
		\left|  \frac{ \alpha+\beta }{ 2 } \right|^r\leq \frac{ | \alpha |^r+| \beta |^r }{2}.
	\end{equation}
\end{proposition}

\begin{proof}
	Nous prouvons en montant progressivement en généralité.
	\begin{subproof}
		\spitem[\( \alpha=1\) et \( \beta=t\geq 1\)]	\label{SPITEMooUDCCooSxNxhv}
		%-----------------------------------------------------------
		Nous commençons avec \( \alpha=1\) et en supposant que \( \beta\) est un réel plus grand ou égal à \( 1\). Nous devons prouver que
		\begin{equation}
			\left(  \frac{ 1+t }{2}  \right)^r\leq \frac{ 1+t^r }{2}.
		\end{equation}
		Nous posons
		\begin{equation}
			\begin{aligned}
				f\colon \mathopen[ 1,\infty\mathclose[ & \to \eR                                                   \\
				t                                      & \mapsto \frac{ 1+t^r }{2}-\left(\frac{ 1+t }{2}\right)^r,
			\end{aligned}
		\end{equation}
		et nous prouvons que \( f(t)\geq 0\) pour tout \( t\geq 1\). Pour cela nous calculons la dérivée :
		\begin{equation}
			f'(t)=\frac{ r }{2}\left(  t^{r-1}-\Big( \frac{ 1+t }{2} \Big)^{r-1}   \right)\geq 0.
		\end{equation}
		Nous avons utilisé le fait que \( x\mapsto x^{r-1}\) est croissante et que \( t\geq 1\). Vu que \( f(1)=1-1=0\), la proposition \ref{PROPooKZPZooWjIsWg}\ref{ITEMooKMDXooPBnVoi} donne
		\begin{equation}
			f(t)\geq f(1)=0
		\end{equation}
		pour tout \( t\geq 1\)

		\spitem[\( \alpha=z\in \eC\) avec \( | z |\geq 1\) et \( \beta=1\)]		\label{SPITEMooLSGLooXiItES}
		%-----------------------------------------------------------

		Nous avons
		\begin{subequations}
			\begin{align}
				\left| \frac{ z+1 }{2} \right|^r & \leq \left( \frac{ | z |+1 }{ 2 } \right)^r & \text{pcq.} | z+1 |\leq | z |+1              \\
				                                 & \leq \frac{ 1+| z |^2 }{2}                  & \text{par point \ref{SPITEMooUDCCooSxNxhv}}.
			\end{align}
		\end{subequations}

		\spitem[Cas général]
		%-----------------------------------------------------------

		Notons que si \( \beta=0\) c'est facile. Nous supposons que \( \beta\neq 0\) et que \( | \alpha |\geq | \beta |\). En posant \( z=\alpha/\beta\) nous avons \( | z |\geq 1\) et nous pouvons calculer un peu :
		\begin{subequations}
			\begin{align}
				\left|\frac{ \alpha+\beta }{2}\right|^r & =\left| \frac{ \beta\frac{ \alpha+\beta }{ \beta } }{2}  \right|                                                  \\
				                                        & =| \beta |^r\left|\frac{ z+1 }{2}\right|                                                                          \\
				                                        & \leq | \beta |^r\frac{ 1+| z |^r }{ 2 }                          & \text{par le point \ref{SPITEMooLSGLooXiItES}} \\
				                                        & =\frac{ | \beta |^r+| \alpha |^r }{2}.
			\end{align}
		\end{subequations}
	\end{subproof}
\end{proof}

\begin{proposition}[\cite{BIBvitali2}]	\label{PROPooHEWFooJpfzEL}
	Si \( a,b\geq 0\) et si \( 0<r\leq 1\), alors
	\begin{equation}
		(a+b)^r\leq a^r+b^r.
	\end{equation}
\end{proposition}

\begin{proof}
	Nous avons un calcul :
	\begin{subequations}
		\begin{align}
			(a+b)^r-a^r & =\int_a^{a+b}rt^{r-1}dt & \text{tho. \ref{ThoRWXooTqHGbC}}              \\
			            & =\int_0^br(s+a)^{r-1}ds & \text{ch.var. $s=t-a$}                        \\
			            & \leq \int_0^brs^{r-1}ds & \text{cf. justif.}		\label{SUBEQooUCTOooBULeqL} \\
			            & =b^r.
		\end{align}
	\end{subequations}
	Pour \eqref{SUBEQooUCTOooBULeqL}, nous savons que \( r< 1\) (le cas \( r=0\) est facile), de telle sorte que \( r-1< 0\). La fonction \( s\mapsto (s+a)^{r-1}\) est donc décroissante.
\end{proof}


\begin{proposition}[c-r inégalité\cite{BIBvitali2}]	\label{PROPooTIJYooJGxphQ}
	Soit un espace mesuré \( (\Omega,\tribA,\mu)\). Soient \( f,g\in L^r(\Omega,\tribA,\mu)\) des fonctions à valeurs réelles\footnote{Par défaut, les éléments de \( L^p\) sont à valeurs complexe. Et oui, je sais que ce sont des classes de fonctions, et donc elles peuvent être à valeurs complexes sur une partie de mesure nulle.}. En posant \( c_r=1\) pour \( 0<r\leq 1\) et \( c_r=2^{r-1}\) si \( r\geq 1\), nous avons
	\begin{enumerate}
		\item
		      Pour tout \( \omega\in\Omega\),
		      \begin{equation}		\label{EQooWIQOooLWPioQ}
			      \big|  f(\omega)+g(\omega)  \big|^r\leq  c_r| f(\omega) |^r+c_r| g(\omega) |^r.
		      \end{equation}
		\item
		      Nous avons
		      \begin{equation}		\label{EQooAJTKooGNxHqt}
			      \int_{\Omega}| f+g |^rd\mu\leq c_r\int_{\Omega}| f |^rd\mu+c_r\int_{\Omega}| g |^rd\mu.
		      \end{equation}
	\end{enumerate}
\end{proposition}

\begin{proof}
	L'inégalité \eqref{EQooWIQOooLWPioQ} provient de propositions \ref{PROPooHEWFooJpfzEL} et \ref{PROPooPMIGooRAjROX} (suivant la valeur de \( r\)). L'inégalité \eqref{EQooAJTKooGNxHqt} demande maintenant d'intégrer sur \( \Omega\).
\end{proof}

%-------------------------------------------------------
\subsection{Inégalité de Hanner}
%----------------------------------------------------

Nous allons démontrer les inégalités de Hanner dans le théorème \ref{THOooZRRYooBTBQKW}. Vu que ce sera un peu longuet, nous faisons un lemme.
\begin{lemma}       \label{LEMooDHRCooQiSpyC}
	Soient \( z_1,z_2\in \eC\). Nous avons
	\begin{equation}        \label{EQooMUXVooSpGSyG}
		| z_1+z_2 |^p+| z_1-z_2 |^p\geq \big( | z_1 |+| z_2 | \big)^p+\big| | z_1 |-| z_2 | \big|^p.
	\end{equation}
\end{lemma}

\begin{proof}
	Soient \( z_1,z_2\in \eC\). Nous posons
	\begin{equation}        \label{EQooJKYZooFzbETG}
		d=| z_1+z_2 |^p+| z_1-z_2 |^p.
	\end{equation}
	Pour \( | z_1 |\) et \( | z_2 |\) fixés, nous nous demandons quel est le minimum possible de \( d\).

	Si \( | z_1 |=0\), alors le minimum est \( 2| z_2 |^p\) et si \( | z_2 |=0\) alors il est \( 2| z_1 |^p\). Pour les autres cas, nous posons \( | z_1 |=a>0\) ainsi que \( b\in \eR\) et \( \theta\in \eR\) tels que\footnote{Proposition \ref{PROPooRFMKooURhAQJ}}
	\begin{equation}
		z_2=z_1a^{-1}b e^{i\theta}.
	\end{equation}
	Nous avons déjà que \( z_1+z_2=z_1(1+a^{-1}b e^{i\theta})\) et donc
	\begin{equation}
		| z_1+z_2 |=a| 1+a^{-1}b e^{i\theta} |=| a+b e^{i\theta} |
	\end{equation}
	parce que \( a>0\). De plus,
	\begin{equation}
		| a+b e^{i\theta} |^2= (a+b e^{i\theta})(a+b e^{-i\theta})=a^2+b^2+2ab\cos(\theta)
	\end{equation}
	parce que \(  e^{i\theta}+ e^{-i\theta}=\cos(\theta)\). Nous posons
	\begin{equation}
		d(\theta)=| a+b e^{i\theta} |^p+| a-b e^{i\theta} |^p.
	\end{equation}
	En développant,
	\begin{equation}
		d(\theta)=\big(a^2+b^2+2ab\cos(\theta)\big)^{p/2}+\big(a^2+b^2-2ab\cos(\theta)\big)^{p/2}.
	\end{equation}
	Trouvons le minimum de cette fonction de \( \theta\). D'abord sa dérivée :
	\begin{subequations}
		\begin{align}
			d'(\theta) & =pab\sin(\theta)\big[ \big( a^2+b^2-2ab\cos(\theta) \big)^{p/2-1}-\big( a^2+b^2+2ab\cos(\theta) \big)^{p/2-1}  \big] \\
			           & =pab\sin(\theta)s(\theta).
		\end{align}
	\end{subequations}
	Nous avons \( s(\theta)=0\) pour \( \theta=\pi/2\) et \( \theta=3\pi/2\). Il faut surtout remarquer que \( 1<p<2\), ce qui donne \( \frac{ p }{2}-1<0\). La fonction \( x\mapsto x^{p/2-1}\) est donc décroissante. Cela pour dire que
	\begin{equation}
		s(0)=\left( | a-b |^2 \right)^{p/2-1}-\left( | a+b |^2 \right)^{p/2-1}>0.
	\end{equation}
	De la même façon, \( s(\pi)=-s(0)<0\). Cela permet d'écrire un petit tableau de signe de \( d'\), et de conclure que \( d(\theta)\) a un minimum en \( 0\) et en \( \pi\). Calcul fait, nous avons
	\begin{equation}
		d(0)=d(\pi)=| a+b |^p+| a-b |^p.
	\end{equation}
	En reliant à \eqref{EQooJKYZooFzbETG} nous avons l'inégalité
	\begin{equation}        \label{EQooVHQOooJcheCR}
		| z_1+z_2 |^p+| z_1-z_2 |^p\geq (a+b)^p-| a-b |^p.
	\end{equation}
	Nous rappelons que \( a=| z_1 |\) et que \( z_2=z_1a^{-1}b e^{i\theta}\). Notons au passage que \( | z_2 |=b\), donc que ce que nous dit l'équation \eqref{EQooVHQOooJcheCR} est que
	\begin{equation}
		| z_1+z_2 |^p+| z_1-z_2 |^p\geq \big( | z_1 |+| z_2 | \big)^p+\big| | z_1 |-| z_2 | \big|^p.
	\end{equation}
\end{proof}

Encore dans la catégorie des lemmes pour les inégalités de Hanner, nous avons celui-ci.
\begin{lemma}[\cite{MonCerveau,ooKGWWooAybolH}]     \label{LEMooTCNEooADpNai}
	La fonction
	\begin{equation}
		\begin{aligned}
			\eta\colon \mathopen] 0 , \infty \mathclose[ & \to \eR                               \\
			a                                            & \mapsto (a^{1/p}+1)^p+| a^{1/p}-1 |^p
		\end{aligned}
	\end{equation}
	est strictement convexe.
\end{lemma}

\begin{proof}
	La fonction \( \eta\) est une fonction de classe \(  C^{\infty}\) sur \( \mathopen] 0 , \infty \mathclose[\setminus\{ 1 \}\). Quelle est sa régularité en \( a=1\) ? À cause de la valeur absolue, il n'est pas clair qu'elle y soit dérivable. En tout cas, la fonction \( x\mapsto| x-1 |\) n'est pas dérivable en \( x=1\), mais peut-être que les exposants aident à lisser. Nous y reviendrons.

		Afin de  suivre les calculs nous introduisons quelques fonctions :
		\begin{subequations}
			\begin{align}
				so(x) & =1+x^{1/p} \\
				di(x) & =1-x^{1/p} \\
				dj(x) & =x^{1/p}-1
			\end{align}
		\end{subequations}
		Pour les dérivées, nous avons
		\begin{subequations}
			\begin{align}
				so'(x) & =\frac{1}{ p }x^{1/p-1} \\
				di'(x) & =-so'(x)                \\
				dj'(x) & =so'(x).
			\end{align}
		\end{subequations}
		Nous divisons les cas selon \( a<1\) ou \( a>1\).
		\begin{subproof}

			\spitem[Pour \( a<1\)]
			%----------------------------------------------
			Nous avons
			\begin{equation}
				\eta(a)=so(a)^p+di(a)^p,
			\end{equation}
			et la première dérivée donne :
			\begin{equation}        \label{EQooCLXZooXClOwd}
				\eta'(a)=p\,so'(a)\big( so(a)^{p-1}-di(a)^{p-1} \big).
			\end{equation}
			Pour la seconde dérivée nous trouvons d'abord
			\begin{equation}
				\begin{aligned}[]
					\eta''(a) & =\left( \frac{ 1-p }{ p } \right)a^{\frac{ 1 }{ p }-2}\big( so(a)^{p-1}-di(a)^{p-1} \big) \\
					          & \quad+\frac{ p-1 }{ p }a^{\frac{ 2 }{ p }-2}\big( so(a)^{p-2}+di(a)^{p-2} \big).
				\end{aligned}
			\end{equation}
			À partir de là, le truc est de substituer les expressions suivantes :
			\begin{subequations}
				\begin{align}
					so(a)^{p-1} & =so(a)^{p-2}so(a)=so(a)^{p-2}+so(a)^{p-2}a^{1/p} \\
					di(a)^{p-1} & =di(a)^{p-2}-a^{1/p}di(a)^{p-2}.
				\end{align}
			\end{subequations}
			Plein de trucs se simplifient et nous obtenons
			\begin{equation}
				\eta''(a)=\frac{ p-1 }{ p }a^{\frac{1}{ p }-2}\big( di(a)^{p-1}-so(a)^{p-2} \big).
			\end{equation}

			\spitem[Pour \( a>1\)]
			%----------------------------------------------
			Les calculs sont essentiellement les mêmes, en partant de
			\begin{equation}
				\eta(a)=so(a)^p+dj(a)^p.
			\end{equation}
			Les résultats sont :
			\begin{equation}    \label{EQooAJLHooGWjPlz}
				\eta'(a)=p\,so'(a)\big( so(a)^{p-1}+dj(a)^{p-1} \big),
			\end{equation}
			et
			\begin{equation}
				\eta''(a)=\frac{ p-1 }{ p }a^{\frac{1}{ p }-2}\big( dj(a)^{p-2}-so(a)^{p-2} \big).
			\end{equation}

			\spitem[Résumé pour \( a\neq 1\)]
			%-----------------------------------------------------------

			Au final, nous avons pour tout \( a\neq 1\) :
			\begin{equation}
				\eta''(a)=\frac{ p-1 }{ p }a^{\frac{1}{ p }-2}\big( | 1-a^{1/p} |^{p-2}-(1+a^{1/p})^{p-2} \big).
			\end{equation}
			Ce qu'il se passe en \( a=1\) est encore une question ouverte que nous traitons maintenant.

			\spitem[Pour \( a=1\)]
			%-----------------------------------------------------------

			Les limites des expressions \eqref{EQooCLXZooXClOwd} et \eqref{EQooAJLHooGWjPlz} en \( a=1\) sont vite calculées et c'est \( 2^{p-1}\) dans les deux cas. Donc la dérivée admet une prolongation continue en \( a=1\). Nous allons prouver que la fonction \( \eta\) est en réalité dérivable en \( a=1\) et que la dérivée vaut \( 2^{p-1}\).

			Nous nous concentrons sur la partie difficile donnée par \( f(x)=| x^{1/p}-1 |^p\). Elle est donnée par
			\begin{equation}
				f(x)=\begin{cases}
					di(x)^p & \text{si } x<1  \\
					dj(x)^p & \text{si } x>1  \\
					0       & \text{si } x=1.
				\end{cases}
			\end{equation}
			Si \( f'(1)\) existe, alors elle est égale à la limite

			\begin{equation}
				f'(1)=\lim_{\epsilon\to 0}\frac{ f(1)-f(1-\epsilon) }{ \epsilon }.
			\end{equation}
			Les deux limites à calculer sont :
			\begin{equation}
				\lim_{\epsilon\to 0^+}\frac{ \big( (1+\epsilon)^{1/p}-1 \big)^p }{ \epsilon }
			\end{equation}
			et
			\begin{equation}
				\lim_{\epsilon\to 0^-}\frac{ \big( 1-(1+\epsilon)^{1/p} \big)^p }{ \epsilon }.
			\end{equation}
			La première se traite par la règle de l'Hospital\footnote{Proposition \ref{PROPooBZHTooHmyGsy}}, et le résultat est zéro. Pour la seconde, il faut juste transformer
			\begin{equation}
				\lim_{\epsilon\to 0^+}\frac{ \big( (1+\epsilon)^{1/p}-1 \big)^p }{ \epsilon }=\lim_{h\to 0^+} \frac{ \big( 1-(1-h)^{1/p} \big)^p }{ -h },
			\end{equation}
			qui se traite également par la règle de l'Hospital. Le résultat est également zéro.

			Donc \( \eta\) est dérivable en \( a=1\) et la dérivée vaut \(\eta'(1)= 2^{p-1}\).
		\end{subproof}
		En récapitulant, nous avons \( \eta''>0\) sur \( \mathopen] 0  , \infty \mathclose[\setminus\{ 1 \}\), donc \( \eta'\) est croissante sur cette partie (proposition \ref{PropGFkZMwD}). Vu que \( \eta'\) est continue sur \( \mathopen] 0 , \infty \mathclose[\), elle est même croissante (strictement) sur tout \( \mathopen] 0 , \infty \mathclose[\).

		La proposition \ref{PropYKwTDPX} conclut que \( \eta\) est strictement convexe sur \( \mathopen] 0 , \infty \mathclose[\).
\end{proof}

Toujours dans la catégorie des lemmes pour les inégalités de Hanner, nous avons celui-ci.
\begin{lemma}[\cite{ooKGWWooAybolH}]
	Soit \( 1<p<2\). La fonction
	\begin{equation}
		\begin{aligned}
			\xi\colon \eR^+\times \eR^+ & \to \eR                                                     \\
			(a,b)                       & \mapsto \big( a^{1/p}+b^{1/p} \big)^p+| a^{1/p}-b^{1/p} |^p
		\end{aligned}
	\end{equation}
	est convexe.

	Pour rappel, les conventions de données en \ref{REMooOCXLooKQrDoq} donnent \( \eR^+=\mathopen[ 0 , \infty \mathclose[\).
\end{lemma}

\begin{proof}
	La fonction \( \xi\) vérifie facilement les conditions suivante :
	\begin{itemize}
		\item \( \xi(a,b)=\xi(b,a)\),
		\item \( \xi(0,0)=0\),
		\item \( \xi(ta,tb)=t\xi(a,b)\) pour tout \( t\geq 0\).
	\end{itemize}
	Nous posons
	\begin{equation}
		\begin{aligned}
			\eta\colon \eR^+ & \to \eR           \\
			a                & \mapsto \xi(a,1).
		\end{aligned}
	\end{equation}
	Le lemme \ref{LEMooTCNEooADpNai} dit que \( \eta\) est strictement convexe, et le lemme \ref{LEMooNUDOooVfVPkw} conclut que \( \xi\) est convexe.
\end{proof}

\begin{lemma}[\cite{BIBooGPACooYtOhPP}]     \label{LEMooWIPYooMZqjbn}
	Soit \( 1<p<2\). Nous considérons les fonctions
	\begin{equation}
		\begin{aligned}
			\alpha\colon \mathopen[ 0 , 1 \mathclose] & \to \eR                         \\
			x                                         & \mapsto (1+x)^{p-1}+(1-x)^{p-1}
		\end{aligned}
	\end{equation}
	et
	\begin{equation}
		\begin{aligned}
			\beta\colon \mathopen[ 0 , 1 \mathclose] & \to \eR                                             \\
			x                                        & \mapsto x^{1-p}\big( (1+x)^{p-1}-(1-x)^{p-1} \big).
		\end{aligned}
	\end{equation}
	Soient \( A,B\in \eR\). Nous avons
	\begin{equation}
		\alpha(x)| A |^p+\beta(x)| B |^p\leq | A+B |^p+| A-B |^p.
	\end{equation}
\end{lemma}

\begin{proof}
	Plusieurs étapes.
	\begin{subproof}
		\spitem[\( \beta(x)\leq \alpha(x)\)]
		Nous avons \( \alpha(1)=\beta(1)=2^{p-1}\). Pour les autres valeurs de \( x\), nous allons raisonner avec la dérivée. La valeur de \( \alpha'(x)\) est facile à calculer
		\begin{equation}
			\alpha'(x)=(p-1)(x+1)^{p-2}-(p-1)(1-x)^{p-2}.
		\end{equation}
		Pour \( \beta'(x)\) c'est un peu plus lourd. En substituant \( (1+x)^{p-1}=(1+x)^{p-2}(1+x)\) et \( (1-x)^{p-1}=(1-x)^{p-2}(1-x)\) nous pouvons regrouper les termes en \( (1+x)^{p-2}\) et \( (1-x)^{p-2}\). Après un peu de travail,
		\begin{equation}
			\beta'(x)=\frac{ p-1 }{ x^p }\big( (1-x)^{p-2}-(1+x)^{p-2} \big).
		\end{equation}
		Cela nous permet de calculer \( \alpha'-\beta'\) :
		\begin{equation}
			\alpha'(x)-\beta'(x)=(p-1)\big( 1+\frac{1}{ x^p } \big)\big( (1+x)^{p-2}-(1-x)^{p-2} \big).
		\end{equation}
		Vu que \( 1<p<2\), le nombre \( p-2\) est strictement négatif; afin de travailler avec des exposants positifs, nous écrivons
		\begin{equation}
			\alpha'(x)-\beta'(x)=\underbrace{(p-1)}_{>0}\underbrace{\big( 1+\frac{1}{ x^p } \big)}_{>0}\underbrace{\left( \frac{1}{ (1+x)^{2-p}}-\frac{1}{ (1-x)^{2-p} }  \right)}_{<0}.
		\end{equation}
		Nous avons \( \alpha'(x)-\beta'(x)<0\) pour tout \( x\in \mathopen] 0 , 1 \mathclose]\). Du fait qu'en plus nous ayons \( \alpha(1)=\beta(1)\), nous déduisons que \( \alpha(x)\geq \beta(x)\).

		\spitem[Une petite étude de fonction]
		%-----------------------------------------------------------
		Soit \( R\in \mathopen[ 0 , 1 \mathclose]\). Nous considérons la fonction
		\begin{equation}
			\begin{aligned}
				F\colon \mathopen[ 0 , 1 \mathclose] & \to \eR                        \\
				x                                    & \mapsto \alpha(x)+R^p\beta(x).
			\end{aligned}
		\end{equation}
		Nous montrons maintenant que cette fonction a un maximum global pour \( x=R\). D'abord sa dérivée :
		\begin{equation}
			F'(x)=\underbrace{(p-1)}_{>0}\underbrace{\Big( (1-x)^{p-1}-(1+x)^{p-2} \Big)}_{<0}\Big( 1-\left( \frac{ R }{ x } \right)^p \Big)
		\end{equation}
		Nous avons
		\begin{itemize}
			\item \( F'(x)=0\) pour \( x=R\),
			\item \( F'(x)<0\) pour \( x>R\),
			\item \( F'(x)>0\) pour \( x<R\).
		\end{itemize}
		Donc \( x=R\) est bien un maximum global.
		\spitem[Pause]
		Nous avons les petits résultats utiles pour commencer à prouver. Petite pause avant de commencer; pas de panique, ça ne va pas être trop violent.
		\spitem[Pour \( 0<B<A\)]
		Nous devons prouver que
		\begin{equation}        \label{EQooEPKRooBYJDSF}
			\alpha(x)A^p+\beta(x)B^p\leq (A+B)^p+(A-B)^p.
		\end{equation}
		En divisant par \( A^p\) et en posant \( R=B/A\), l'inéquation \eqref{EQooEPKRooBYJDSF} est équivalente à
		\begin{equation}
			\alpha(x)+\beta(x)R^p\leq (1+R)^p+(1-R)^p
		\end{equation}
		où \( R\in \mathopen] 0 , 1 \mathclose[\) parce que nous avons supposé \( 0<B<A\). Nous avons (il y a un petit calcul pour \( F(R)\))
		\begin{equation}
			(1+R)^p+(1-R)^p=F(R)\geq F(x)=\alpha(x)+\beta(x)R^p.
		\end{equation}
		ok.
		\spitem[Pour \( 0<A<B\)]
		Lorsque \( 0<A<B\) nous avons
		\begin{subequations}
			\begin{align}
				\alpha(x)| A |^p+\beta(x)| B |^p & =\alpha(x)A^p+\beta(x)B^p                                      \\
				                                 & \leq \alpha(x)B^p+\beta(x)A^p      \label{SUBEQooSHNUooCoWMFB} \\
				                                 & \leq (B+A)^p+(B-A)^p       \label{SUBEQooBPYVooPsAjbq}         \\
				                                 & =| A+B |^p+| A-B |^p.
			\end{align}
		\end{subequations}
		Justification :
		\begin{itemize}
			\item Pour \eqref{SUBEQooSHNUooCoWMFB}, c'est parce que \( \alpha(x)>\beta(x)\); alors en mettant le plus grand de \( A\) et \( B\) devant le \( \alpha\) au lieu du \( \beta\), nous majorons.
			\item Pour \eqref{SUBEQooBPYVooPsAjbq}, c'est l'inégalité dans le cas \( 0<B<A\), mais en inversant les noms de \( A\) et \( B\).
		\end{itemize}
		\spitem[Pour \( 0<A=B\)]
		Toutes les expressions sont continues par rapport à \( B\) (fixons \( x\) et \( A\)). Nous avons prouvé pour \( B<A\) et pour \( B>A\). Par continuité, l'inégalité est encore valide pour \( A=B\).
		\spitem[Pour \( A<0\), \( B>0\)]
		En posant \( A'=-A\) nous avons \( A'>0\) et nous pouvons écrire
		\begin{equation}
			| A+B |^p+| A-B |^p=| -A'+B |^p+| -A'-B |^p=| B-A' |^p+| B+A' |^p\geq \alpha(x)| A' |^p+\beta(x)| B |^p.
		\end{equation}
		Nous avons utilisé, avec \( A'\) et \( B\) le cas déjà prouvé \( A',B>0\).
		\spitem[Pour \( A>0\), \( B<0\)]
		Celui-là, je vous le laisse.
		\spitem[Pour \( A<0\), \( B<0\)]
		Posez \( A'=-A\) et \( B'=-B\) et hop.
	\end{subproof}
\end{proof}


\begin{theorem}[Inégalités de Hanner\cite{ooKGWWooAybolH,BIBooGPACooYtOhPP}]       \label{THOooZRRYooBTBQKW}
	Soit un espace mesuré \(  (\Omega,\tribA,\mu)\). Soit \( 1<p<2\) et \( f,g\in L^p(\Omega,\tribA,\mu)\); nous avons
	\begin{equation}
		\big( \| f \|_p+\| g \|_p \big)^p+\Big| \| f \|_p-\| g \|_p \Big|^p
		\leq \| f+g \|_p^p+\| f-g \|_p^p
		%           \leq 2\| f \|_p^p+2\| g \|_p^p.        
		% Je laisse tomber cette partie parce qu'elle est -je crois- inutile pour le théorème de Weienersmith
	\end{equation}
	Il y a égalité si et seulement si \( f(t) \) et \( g(t)\) sont colinéaires pour presque tout \( t\).
\end{theorem}

\begin{proof}
	Nous supposons que \( \| f \|_p\geq \| g \|_p\) pour fixer les idées. De toutes façons, la symétrie des formules nous fait passer de ce cas à l'autre sans difficulté.

	Soit \( x\in \mathopen[ 0 , 1 \mathclose]\). Nous écrivons l'inégalité du lemme \ref{LEMooWIPYooMZqjbn} pour \( A=| f(\omega) | \) et \( B=| g(\omega) |\) :
	\begin{equation}
		\alpha(x)| f(\omega) |^p+\beta(x)| g(\omega) |^p\leq \big| f(\omega)+g(\omega) \big|^p+\big| f(\omega)-g(\omega) \big|^p.
	\end{equation}
	Nous intégrons cela par rapport à \( \omega\) sur \( \Omega\) :
	\begin{equation}
		\alpha(x)\| f \|_p^p+\beta(x)\| g \|_p^p\leq \| f+g \|^p_p+\| f-g \|_p^p.
	\end{equation}
	Et là vient l'idée qu'on se demande ce qui est passé par l'esprit du mec qui a tout combiné : nous évaluons cela pour \( x=\frac{ \| g \|_p }{ \| f \|_p }\), ce qui est permis parce que nous avons supposé \( \| f \|_p\geq \| g \|_p \). Faites le calcul, collectez les termes identiques, vous obtiendrez
	\begin{equation}
		\big( \| f \|_p+\| g \|_p \big)^p+\big( \| f \|_p-\| g \|_p \big)^p\leq \| f+g \|^p_p+\| f-g \|_p^p.
	\end{equation}
	Et vu que \( \| f \|_p\geq \| g \|_p\), nous pouvons gratuitement faire
	\begin{equation}
		\| f \|_p-\| g \|_p=\big| \| f \|_p-\| g \|_p \big|.
	\end{equation}
	Fini pour Hanner.
\end{proof}


%--------------------------------------------------------------------------------------------------------------------------- 
\subsection{Inégalités de Clarkson}
%---------------------------------------------------------------------------------------------------------------------------

\begin{lemma}[\cite{BIBooUNSIooQCLkzT}]     \label{LEMooWEODooLHeVrP}
	Si \( p\geq 2\) et si \( a,b\in \eC\), alors
	\begin{equation}
		\left| \frac{ a+b }{2} \right|^p+\left| \frac{ a-b }{2} \right|^p\leq \frac{ 1 }{2}\big( | a |^p+| b |^p \big).
	\end{equation}
\end{lemma}

\begin{proof}
	Nous prouvons l'inégalité en montant petit à petit en généralité.
	\begin{subproof}
		\spitem[Avec \( x>0\)]
		Soit \( x\geq 0\). Nous montrons dans cette partie l'inégalité
		\begin{equation}        \label{EQooDJBNooEyfNtq}
			x^p+1\leq (x+1)^{p/2}.
		\end{equation}
		Pour cela nous considérons la fonction
		\begin{equation}
			\begin{aligned}
				f\colon \mathopen[ 0 , \infty \mathclose[ & \to \eR                      \\
				t                                         & \mapsto (t^2+1)^{p/2}-t^p-1.
			\end{aligned}
		\end{equation}
		Nous avons \( f(0)=0\), mais aussi, en utilisant les règle de dérivation\footnote{Par exemple celle de la proposition \ref{PROPooKIASooGngEDh}.} nous trouvons vite
		\begin{equation}
			f'(t)=p(t^2+1)^{p/2-1}t-pt^{p-1}.
		\end{equation}
		Vu que \( (t^2+1)^{p/2-1}\geq t^{p-2}\), le signe de \( f'(t)\) est toujours strictement positif pour \( t>0\). La proposition \ref{PropGFkZMwD} fait que \( f\) est strictement croissante et que \( f(t)>0\) pour tout \( t>0\).

		\spitem[Avec \( x,y\geq 0\)]
		Soient \( x,y\geq 0\) dans \( \eR\). Nous prouvons dans cette partie que
		\begin{equation}        \label{EQooGFGMooSiDfKX}
			(x^2+y^2)^{p/2}\geq x^p+y^p.
		\end{equation}
		Il s'agit d'appliquer l'inégalité \eqref{EQooDJBNooEyfNtq} à \( x/y\) :
		\begin{equation}
			\left( \left( \frac{ x }{ y } \right)^2+1 \right)^{p/2}\geq \left( \frac{ x }{ y } \right)^p+1.
		\end{equation}
		En multipliant par \( y^p\) et en simplifiant un peu, nous trouvons le résultat \eqref{EQooGFGMooSiDfKX}.
		\spitem[Avec \( a,b\in \eC\)]
		Nous appliquons l'inégalité \eqref{EQooGFGMooSiDfKX} à \( x=| \frac{ a+b }{ 2 } |\) et \( y=| \frac{ a-b }{2} |\). Cela donne :
		\begin{subequations}
			\begin{align}
				\left| \frac{ a+b }{2} \right|^p+\left| \frac{ a-b }{2} \right|^p & \leq \left( \left| \frac{ a+b }{2} \right|^2+\left| \frac{ a-b }{2} \right|^2 \right)^{p/2} \\
				                                                                  & =\left( \frac{ 2| a |^2+2| b |^2 }{ 4 } \right)^{p/2}                                       \\
				                                                                  & \leq\frac{ 1 }{2}| a |^p+\frac{ 1 }{2}| b |^p.
			\end{align}
		\end{subequations}
		La dernière ligne est la convexité de la fonction \( t\mapsto t^{p/2}\) (lemme \ref{LEMooSXTXooZOmtKq}).
	\end{subproof}
\end{proof}

\begin{lemma}       \label{LEMooFGKXooZCHNln}
	Si \( 1<p<2\), alors l'exposant conjugué \( q\) vérifie \( q>2\).
\end{lemma}

\begin{proof}
	Nous considérons \( q\) en fonction de \( p\), sur le domaine \( 1<p<2\) :
	\begin{equation}
		q(p)=\frac{ p }{ p-1 }.
	\end{equation}
	Donc\footnote{Dire que \( q(1)=\infty\) est un abus de notations pour parler de la limite \( p\to 1\) avec \( p>1\).} \( q(1)=\infty\) et \( q(2)=2\). Nous étudions ensuite la dérivée :
	\begin{equation}
		q'(p)=-\frac{1}{ (p-1)^2 }<0.
	\end{equation}
	C'est donc une fonction strictement décroissante. Vues les valeurs aux bornes, nous voyons que \( q(p)>2\) sur tout son domaine.
\end{proof}

\begin{lemma}[\cite{BIBooVHQSooTrLCzQ}]         \label{LEMooMKIXooVOYaxI}
	Soit \( 1<p<2\). Pour \( x,y\in \eR\) nous avons
	\begin{equation}
		| x+y |^q+| x-y |^q\leq 2\big( | x |^p+| y |^p \big)^{q-1}.
	\end{equation}
\end{lemma}

\begin{proof}
	Nous considérons l'exposant conjugué \( q\) et \( p\), c'est-à-dire \( q\) tel que \( \frac{1}{ p }+\frac{1}{ q }=1\). Nous considérons la fonction
	\begin{equation}
		\begin{aligned}
			f\colon \eR\times \mathopen[ 0 , 1 \mathclose] & \to \eR                                                                          \\
			(\alpha,z)                                     & \mapsto (1+\alpha^{1-q}z)(1+\alpha z)^{q-1}+(1-\alpha^{1-q}z)(1-\alpha z)^{q-1}.
		\end{aligned}
	\end{equation}
	Cette fonction vérifie
	\begin{equation}        \label{EQooRFZQooJvdocT}
		f(1,z)=(1+z)^q+(1-z)^q,
	\end{equation}
	ainsi que
	\begin{subequations}        \label{EQooISBRooHMiPRE}
		\begin{align}
			f(z^{p-1},z) & =\big( 1+z^{(p-1)(1-q)} \big)(1+z^p)^{q-1}+\big( 1-z^{(p-1)(1-q)} \big)(1+z^p)^{q-1} \\
			             & =2(1+z^p)^{q-1}.
		\end{align}
	\end{subequations}

	Nous montrons maintenant que \( (\partial_{\alpha}f)(\alpha,z)\leq 0\) pour tout \( \alpha\in\mathopen] 0 , 1 \mathclose[\) et pour tout \( z\in \mathopen] 0 , 1 \mathclose[\). C'est du calcul :
	\begin{equation}
		\frac{ \partial f }{ \partial \alpha }(\alpha,z)=(1-q)z\big[ \alpha^{-q}(1+\alpha z)^{q-1}-(1+\alpha^{1-q}z)(1+\alpha z)^{q-1}-\alpha^{-q}(1-\alpha z)^{q-1}+(1-\alpha^{1-q}z)(1-\alpha z)^{q-2} \big].
	\end{equation}
	Maintenant nous factorisons \( (1+\alpha z)^{q-2}\) grâce à la décomposition \( (1+\alpha z)^{q-1}=(1+\alpha z)(1+\alpha z)^{q-2}\). Notez que le lemme \ref{LEMooFGKXooZCHNln} donne \( q>2\), et donc pas de problèmes avec la puissance \( q-2\). Nous continuons le calcul
	\begin{subequations}
		\begin{align}
			\frac{ \partial f }{ \partial \alpha }(\alpha,z) & =(1-q)z(1+\alpha z)^{q-2}\big[ \alpha^{-q}(1+\alpha z)-(1+\alpha^{1-q}z) \big]       \\
			\nonumber                                        & \quad+(1-q)z(1-\alpha z)^{q-2}\big[ -\alpha^{-q}(1-\alpha z)+(1-\alpha^{1-q}z) \big] \\
			                                                 & =(1-q)z(1+\alpha z)^{q-2}\big[ \alpha^{-q}+\alpha^{-q+1}z-1-\alpha^{1-q}z \big]      \\
			\nonumber                                        & \quad+(1-q)z(1-\alpha z)^{q-2}\big[ -\alpha^{-q}+\alpha^{-q+1}+1-\alpha^{1-q}z \big] \\
			                                                 & =(1-q)z(1+\alpha z)^{q-2}[\alpha^{-q}-1]+(1-q)z(1-\alpha z)^{q-2}[1-\alpha^{-q}]     \\
			                                                 & =(1-q)z(\alpha^{-q}-1)\big[ (1+\alpha z)^{q-2}-(1-\alpha z)^{q-2} \big].
		\end{align}
	\end{subequations}
	Vu que \( q>2\), la fonction \( x\mapsto x^{q-2}\) est strictement croissante sur les positifs\footnote{Proposition \ref{PROPooUOFKooYyGwIr}.}. Et vu que \( \alpha z<1\), le crochet est strictement positif. Par ailleurs, \( z>0\), \( (1-q)<0\) et \( (\alpha^{-q}-1)>0\) donc nous avons prouvé que \( (\partial_{\alpha}f)(\alpha,z)\leq 0\).

	Donc \( f\) est décroissante par rapport à \( \alpha\). Vu que \( z\in \mathopen[ 0 , 1 \mathclose]\) et que \( p>1\), nous avons \( z^{p-1}<1\) et donc \( f(1,z)\leq f(z^{p-1},z)\). Nous y substituons les valeurs calculées en \eqref{EQooRFZQooJvdocT} et \eqref{EQooISBRooHMiPRE} :
	\begin{equation}        \label{EQooFQJAooPCYtMG}
		(1+z)^q+(1-z)^q\leq 2(1+z^p)^{q-1}.
	\end{equation}

	Nous pouvons maintenant facilement prouver notre inégalité.
	\begin{subproof}
		\spitem[Pour \( 0<x<y\)]
		Si \( 0<x<y\), nous avons \( x/y\in \mathopen[ 0 , 1 \mathclose]\) et nous pouvons appliquer l'inégalité \eqref{EQooFQJAooPCYtMG} à \( z=x/y\). Nous avons successivement :
		\begin{subequations}
			\begin{align}
				\left(1+\frac{ x }{ y }\right)^q+\left( 1-\frac{ x }{ y } \right)^q   & \leq \left( 1+\frac{ x^p }{ y^p } \right)^{q-1}    \\
				\left( \frac{ x+y }{ y } \right)^q+\left( \frac{ y-x }{ y } \right)^q & \leq 2\left( \frac{ y^p+x^p }{ y^p } \right)^{q-1} \\
				y^{-q}(x+y)^q+y^{-q}(y-x)^q                                           & \leq 2y^{-p(q-1)}(y^p+x^p)                         \\
				(x+y)^q+(y-x)^q                                                       & \leq 2y^{-p(q-1)+q}(y^p+x^p).
			\end{align}
		\end{subequations}
		Nous avons utilisé la proposition \ref{PROPooDWZKooNwXsdV} sur la composition de puissances. Maintenant il suffit de remarquer que \( q=p(q-1)\) pour avoir le résultat.

		\spitem[\( x<0\) et \( y>0\)]
		En posant \( x'=-x\) nous avons
		\begin{subequations}
			\begin{align}
				| x+y |^q+| x-y |^q & =| -x'+y |^q+| -x'-y |^q           \\
				                    & =| x'-y |^q+| x'+y |^q             \\
				                    & \leq 2\big( | x' |^p+| y |^p \big) \\
				                    & \leq 2\big( | x |^p+| y |^p \big).
			\end{align}
		\end{subequations}
		\spitem[Les autres cas]
		% -------------------------------------------------------------------------------------------- 
		Je vous prie de faire la liste, et d'adapter.
	\end{subproof}
\end{proof}


Pour d'autres preuves du lemme suivant, voir \cite{BIBooHJQOooJsInho}.
\begin{lemma}[\cite{BIBooKDOKooFbAlfz}]       \label{LEMooLTROooVusGte}
	Soient \( a,b\in \eC\) ainsi que \( 1<p<2\). Nous notons \( q\) l'exposant conjugué de \( p\). Nous avons l'inégalité
	\begin{equation}
		| a+b |^q+| a-b |^q\leq 2\big( | a |^p+| b |^p \big)^{q-1}.
	\end{equation}
\end{lemma}

\begin{proof}
	Commençons doucement avec le cas \( b=0\). À gauche nous gardons \( 2| a |^q\), et pour le membre de droite nous remarquons que
	\begin{equation}
		q-1=\frac{ 1 }{ p-1 },
	\end{equation}
	de telle sorte que
	\begin{equation}
		(| a |^p)^{q-1}=| a |^{p/(p-1)}=|a|^q.
	\end{equation}

	Gardez en tête que, par le lemme \ref{LEMooFGKXooZCHNln}, \( q>2\); ce sera utile.

	Nous commençons le vrai combat. Vu que \( | a-b |=| b-a |\) nous pouvons supposer \( | a |\geq | b |\) pour fixer les idées. En utilisant le lemme \ref{LEMooOQKNooGZlJHf}, il existe \( t_0\in\mathopen[ 0 , \frac{ \pi }{2} \mathclose]\) tel que
	\begin{equation}
		\begin{aligned}[]
			| a+b |^2 & =| a |^2+| b |^2+2| a | |b |\cos(t_0) \\
			| a-b |^2 & =| a |^2+| b |^2-2| a | |b |\cos(t_0) \\
		\end{aligned}
	\end{equation}

	Nous considérons la fonction
	\begin{equation}
		\begin{aligned}
			f\colon \mathopen[ 0 , 2\pi \mathclose] & \to \eR                                                                                                              \\
			t                                       & \mapsto \Big( | a |^2+| b |^2+2| a | |b |\cos(t) \Big)^{q/2}+  \Big( | a |^2+| b |^2-2| a | |b |\cos(t) \Big)^{q/2}.
		\end{aligned}
	\end{equation}
	Elle vérifie
	\begin{equation}
		f(t_0)=(| a+b |^2)^{q/2}+(| a-b |^2)^{q/2}.
	\end{equation}
	Vu que \( | a+b |\) et \( | a-b |\) sont positifs, nous pouvons «simplifier» le carré et la racine carré, de telle sorte que
	\begin{equation}
		f(t_0)=| a+b |^q+| a-b |^q.
	\end{equation}
	Nous cherchons un maximum pour \( f\) sur \( \mathopen[ 0 , \frac{ \pi }{2} \mathclose]\). Pour cela, nous prenons d'abord la dérivée :
	\begin{equation}
		f'(t)=-q| a | |b |\sin(t)\Big[   \big( | a |^2+| b |^2+2| a | |b |\cos(t) \big)^{q/2-1}-\big( | a |^2+| b |^2-2| a | |b |\cos(t) \big)^{q/2-1}    \Big].
	\end{equation}
	Notez que \( q>2\), donc la fonction \( x\mapsto x^{q/2-1}\) est croissante.

	Pour \( t\in\mathopen[ 0 , \pi/2 \mathclose]\), nous avons \( \cos(t)\geq -\cos(t)\) ainsi que \( \sin(t)\geq 0\). Donc \( f'(t)\leq 0\). De la même manière, nous avons \( f'(t)\geq 0\) pour \( t\in\mathopen[ \pi/2 , \pi \mathclose]\).

	Par le lien entre dérivée et croissance (proposition \ref{PROPooKZPZooWjIsWg}), nous savons que le maximum de \( f\) sur \( \mathopen[ 0 , \pi \mathclose]\) est atteint en \( 0\) ou en \( \pi\).

	Nous avons, en utilisant la supposition \( | a |\geq | b |\):
	\begin{subequations}
		\begin{align}
			f(0)=f(\pi) & =\big( | a |^2+| b |^2+2| a | |b | \big)^{q/2}+\big( | a |^2+| b |^2-2| a | |b | \big)^{q/2} \\
			            & =\big( \big| | a |+| b |\big|^2 \big)^{q/2}+\Big( \big| | a |-| b | \big|^2 \Big)            \\
			            & =\big| | a |+| b | \big|^q+\big| | a |-| b | \big|^q.
		\end{align}
	\end{subequations}
	En particulier \( f(t_0)\leq f(0)\) et donc
	\begin{equation}
		| a+b |^q+| a-b |^q\leq \big| | a |+| b | \big|^q+\big| | a |-| b | \big|^q\leq 2\big( | a |^p+| b |^p \big)^{q-1}.
	\end{equation}
	La dernière inégalité est le lemme \ref{LEMooMKIXooVOYaxI} appliqué aux réels \( | a |\) et \( | b |\).
\end{proof}

\begin{proposition}[Inégalité de Clarkson\cite{BIBooVHQSooTrLCzQ}]      \label{PROPooJDOQooWsGlkr}
	Soient \( f,g\in L^p(\Omega,\tribA,\mu)\).
	\begin{enumerate}
		\item
		      Si \( p\geq 2\), alors
		      \begin{equation}        \label{EQooBWDJooGXzdxz}
			      \| \frac{ f+g }{2} \|_p^p+\| \frac{ f-g }{2} \|_p^p\leq \frac{ 1 }{2}\Big( \| f \|_p^p+\| g \|_p^p \Big).
		      \end{equation}
		\item
		      Si \( 1<p<2\) et si \( q\) est l'exposant conjugué de \( p\), alors
		      \begin{equation}        \label{EQooXMWBooYrvaoV}
			      \| f+g \|_p^q+\| f-g \|_p^q\leq 2\Big( \| f \|_p^p +\| g \|_p^p \Big)^{q-1},
		      \end{equation}
		      ou
		      \begin{equation}        \label{EQooZCWDooBnaMom}
			      \| \frac{ f+g }{2} \|_p^q+\| \frac{ f-g }{2} \|_p^q\leq 2^{1-q}\big( \| f \|_p^p+\| g \|_p^p \big)^{q-1}.
		      \end{equation}
	\end{enumerate}
\end{proposition}

\begin{proof}
	En deux parties.
	\begin{subproof}
		\spitem[Pour \( p\geq 2\)]
		Soient \( f,g\in L^p(\Omega,\tribA,\mu)\); ce sont des fonctions à valeurs dans \( \eC\). Pour chaque \( \omega\in \Omega\) nous considérons les nombres complexes \( f(\omega)\) et \( g(\omega)\); nous pouvons écrire l'inégalité du lemme \ref{LEMooWEODooLHeVrP} :
		\begin{equation}        \label{EQooIGNKooKFUpKO}
			\left| \frac{ f(\omega)+g(\omega) }{2} \right|^p+\left| \frac{ f(\omega)-g(\omega) }{2} \right|^p\leq \frac{ 1 }{2}\big( | f(\omega) |^p+| g(\omega) |^p \big).
		\end{equation}
		Nous avons les substitutions évidentes \( f(\omega)+g(\omega)=(f+g)(\omega)\) et \( f(\omega)-g(\omega)=(f-g)(\omega)\). En intégrant alors \eqref{EQooIGNKooKFUpKO} sur \( \Omega\) nous trouvons l'inégalité demandée.
		\spitem[Pour \( 1<p<2\)]
		Il s'agit de faire la même chose, en utilisant l'inégalité de Clarkson du lemme \ref{LEMooLTROooVusGte}.

		Pour obtenir \eqref{EQooZCWDooBnaMom}, il s'agit simplement de multiplier et diviser le member de gauche de \eqref{EQooXMWBooYrvaoV} par \( 2^q\).
	\end{subproof}
\end{proof}



%+++++++++++++++++++++++++++++++++++++++++++++++++++++++++++++++++++++++++++++++++++++++++++++++++++++++++++++++++++++++++++ 
\section{Théorème de la projection normale}
%+++++++++++++++++++++++++++++++++++++++++++++++++++++++++++++++++++++++++++++++++++++++++++++++++++++++++++++++++++++++++++

%--------------------------------------------------------------------------------------------------------------------------- 
\subsection{Espace uniformément convexe}
%---------------------------------------------------------------------------------------------------------------------------

\begin{definition}[Espace uniformément convexe\cite{BIBooPYEZooTxohAd}]     \label{DEFooOPQBooBhufew}
	Un espace de Banach \( B\) est \defe{uniformément convexe}{uniformément convexe} si il existe une fonction \( \delta\colon \mathopen] 0 , \infty \mathclose[\to \eR^+\) telle que si
	\begin{enumerate}
		\item
		      \( \| x \|\leq \| y \|\leq 1\),
		\item
		      \( \| x-y \|\geq \epsilon\),
	\end{enumerate}
	alors
	\begin{equation}
		\| \frac{ x+y }{ 2 } \|\leq \| y \|-\delta(\epsilon).
	\end{equation}
\end{definition}

\begin{lemma}[\cite{MonCerveau}]
	Si \( B\) est un espace de Banach uniformément convexe, alors pour tout \( k>0\), il existe une fonction \( \delta_k\colon \mathopen] 0 , \infty \mathclose[\to \eR^2\) telle que si
	\begin{enumerate}
		\item
		      \( \| x \|\leq \| y \|\leq k\),
		\item
		      \( \| x-y \|\geq \epsilon\),
	\end{enumerate}
	alors
	\begin{equation}
		\| \frac{ x+y }{ 2 } \|\leq \| y \|-\delta_k(\epsilon).
	\end{equation}
\end{lemma}

\begin{proof}
	Nous posons \( x'=x/k\) et \( y'=y/k\). Nous avons alors
	\begin{equation}
		\| x'-y' \|=\frac{ \| x-y \| }{ k }>\frac{ \epsilon }{ k }.
	\end{equation}
	L'uniforme convexité de \( B\) dit alors que
	\begin{equation}
		\| \frac{ x'+y' }{2} \|>\| y' \|-\delta(\epsilon/k).
	\end{equation}
	En multipliant cette inégalité par \( k\) nous trouvons
	\begin{equation}
		\| \frac{ x+y }{2} \|>\| y \|-k\delta(\epsilon/k).
	\end{equation}
	Donc en posant \( \delta_k(\epsilon)=k\delta(\epsilon/k)\), nous avons le résultat escompté.
\end{proof}

\begin{definition}[Projection normale\cite{BIBooPYEZooTxohAd}]      \label{DEFooMYYLooJyACPL}
	Soient un espace de Banach \( B\) ainsi que \( V\subset B\). Soit \( a\in B\). La fonction
	\begin{equation}
		\begin{aligned}
			f\colon V & \to \eR        \\
			x         & \mapsto d(x,a)
		\end{aligned}
	\end{equation}
	possède un infimum\footnote{Toute fonction à valeurs positives possède un infimum, c'est la proposition \ref{DefSupeA}.} \( m\). Si \( x\in V\) est tel que \( d(x,a)=m\), alors \( x\) est une \defe{projection normale}{projection normale} de \( a\) sur \( V\).
\end{definition}

\begin{proposition}[\cite{BIBooPYEZooTxohAd}]       \label{PROPooDKXVooUoYPgz}
	Soient un espace de Banach \( B\) et un sous-espace vectoriel \( V\subset B\). Si une projection normale de \( a\in B\) sur \( V\) existe, alors elle est unique.
\end{proposition}

\begin{proof}
	Soient deux projections normales \( b,b'\) de \( a\) sur \( V\).

	Si \( m=0\), alors \( \| a-b \|=0\) et \( \| a-b' \|=0\), ce qui donne \( a=b\) et \( a=b'\). Donc d'accord pour \( b=b'\).

	Si \( m>0\) alors nous utilisons l'inégalité \( \| x+y \|\leq \| x \|+\| y \|\) sous la forme
	\begin{equation}        \label{EQooQWJWooVaWMCL}
		\| a-\frac{ b+b' }{ 2 } \|=\| \frac{ a-b }{2}+\frac{ a-b' }{2} \|\leq \| \frac{ a-b }{2} \|+\| \frac{ a-b' }{2} \|=\frac{ m }{ 2 }+\frac{ m }{2}=m.
	\end{equation}
	Mais \( \frac{ b+b' }{2}\in V\), donc
	\begin{equation}
		\| a-\frac{ b+b' }{2} \|\geq m.
	\end{equation}
	Nous en déduisons que dans \eqref{EQooQWJWooVaWMCL}, toutes les inégalités sont des égalités et en particulier
	\begin{equation}
		\| \frac{ b+b' }{2}-a \|=m.
	\end{equation}
	Nous avons donc les deux égalités suivantes :
	\begin{equation}
		2m=\| a-b \|+\| a-b' \|
	\end{equation}
	et
	\begin{equation}
		2m=\| b+b'-2a \|.
	\end{equation}
	Cela donne
	\begin{equation}
		\| a-b \|+\| a-b' \|=\| (a-b)+(a-b') \|.
	\end{equation}
	Vu que \( B\) est strictement convexe, cela n'est possible que si \( a-b=a-b'\), ce qui signifie que \( b=b'\).
\end{proof}

\begin{theorem}[\cite{BIBooVHQSooTrLCzQ,BIBooPYEZooTxohAd}]     \label{THOooOOVVooMhzHqd}
	Si \( B\) est un espace de Banach uniformément convexe, si \( V\subset B\) est un sous-espace vectoriel fermé et si \( a\in B\), alors \( a\) admet une unique projection normale\footnote{Définition \ref{DEFooMYYLooJyACPL}.} sur \( V\).
\end{theorem}

\begin{proof}
	En deux parties.
	\begin{subproof}
		\spitem[Unicité]
		Soient deux projections normales \( b\) et \( b'\) de \( a\) sur \( V\). Nous avons \( \| a-b \|=\| a-b' \|=m\). Si \( b\neq b'\), il existe \( \epsilon>0\) tel que
		\begin{equation}
			\| b-b' \|>\epsilon>0.
		\end{equation}
		En posant
		\begin{equation}
			\begin{aligned}[]
				x & =\frac{ 1 }{2}\frac{ b-a }{ m }, & y=\frac{ 1 }{2}\frac{ b'-a }{ m },
			\end{aligned}
		\end{equation}
		nous avons \( \| x \|=\| y \|=\frac{ 1 }{2}<1\). L'uniforme convexité de \( B\) donne alors
		\begin{equation}
			\| \frac{ x+y }{2} \|\leq \| y \|-\delta(\epsilon).
		\end{equation}
		Mais
		\begin{equation}
			x+y=\frac{ \frac{ 1 }{2}(b+b')-a }{ m }
		\end{equation}
		et
		\begin{equation}
			\| x+y \|\leq 2\| y \|-2\delta(\epsilon)=1-2\delta(\epsilon)<1.
		\end{equation}
		Nous avons donc prouvé que
		\begin{equation}
			\| \frac{ 1 }{2}(b+b')-a \|<m,
		\end{equation}
		ce qui est impossible parce que cela dirait que \( \frac{ b+b' }{2}\) est une «meilleure» projection normale que \( b\) et \( b'\).

		\spitem[Existence]
		Soient \( b_k\) dans \( V\) tels que \( \| a-b_k \|\to m\). Nous supposons (quitte à passer à une sous-suite) que
		\begin{equation}
			\| a-b_{k+1} \|\leq \| a-b_k \|.
		\end{equation}

		\begin{subproof}
			\spitem[La suite $(b_k)$ converge]

			Nous supposons qu'elle ne converge pas. Elle n'est donc pas de Cauchy parce que \( B\) est de Banach\footnote{Définition \ref{DefVKuyYpQ}.} et donc complet. Il existe \( \epsilon>0\) tel que pour tout \( N\in \eN\) il existe \( p,q>N\) tels que
			\begin{equation}
				\| b_p-b_q \|>\epsilon.
			\end{equation}

			Nous effectuons quelque choix.
			\begin{enumerate}
				\item
				      nous choisissons \( q>p\) de telle sorte que \( \| a-b_p \|\leq\| a-b_q \|\),
				\item
				      nous choisissons \( N\) assez grand pour avoir
				      \begin{equation}
					      \| a-b_p \|\leq \| a-b_q \|<m.
				      \end{equation}
			\end{enumerate}

			Nous avons \( \| (a-b_p)-(a-b_q) \|=\| b_q-b_p \|>\epsilon\), ce qui avec l'uniforme convexité donne
			\begin{equation}
				\frac{ \| (a-b_p)+(a-b_q) \| }{2}\leq \| a-b_q \|-\delta(\epsilon).
			\end{equation}
			Donc
			\begin{equation}
				m\leq \| a-\frac{ b_p-b_q }{ 2 } \|=\| \frac{ (a-b_p)+(a-b_q) }{2} \|\leq \| a-b_q \|-\delta(\epsilon)<m-\delta(\epsilon)<m.
			\end{equation}
			Cela signifie que \( m<m\), ce qui est impossible.
			\spitem[Conclusion]
			La suite \( (b_k)\) converge dans \( B\). Vu que \( V\) est fermé, la limite est dans \( V\). Cette limite, que nous nommons \( b\), vérifie
			\begin{equation}
				\| a-b \|\leq \| a-b_k \|
			\end{equation}
			pour tout \( k\). Mais comme nous avons \( m\leq \| a-b_k \|\to m\), nous avons \( \| a-b \|=m\), c'est-à-dire que \( b\) est une projection normale de \( a\) sur \( V\).
		\end{subproof}
	\end{subproof}
\end{proof}



%--------------------------------------------------------------------------------------------------------------------------- 
\subsection{Uniforme convexité des espaces de Lebesgue}
%---------------------------------------------------------------------------------------------------------------------------

\begin{proposition}[\cite{BIBooRISHooBcPPKQ}]     \label{PROPooFNLJooDlyIKV}
	Si \( 1<p<\infty\), l'espace \( L^p(\Omega,\tribA, \mu)\) est uniformément convexe\footnote{Définition \ref{DEFooOPQBooBhufew}.}.
\end{proposition}

\begin{proof}
	En deux parties.

	\begin{subproof}
		\spitem[\( 1<p\leq 2\)]
		Nous montrons que la fonction \( \delta(\epsilon)=2^{-q}\epsilon^q\) fonctionne.

		Soient \( f,g\in L^p\) telles que \( \| f \|_p\leq \| g \|_p\leq 1\) et \( \| f-g \|_p\geq \epsilon\). Nous commençons par écrire l'inégalité de Clarkson \eqref{EQooXMWBooYrvaoV} :
		\begin{equation}        \label{EQooOWVEooGGfCpy}
			\| \frac{ f+g }{2} \|_p^q+\| \frac{ f-g }{2} \|_p^q\leq 2^{1-q}\big( \| f \|_p^p+\| g \|_p^p \big)^{q-1}.
		\end{equation}
		Par hypothèse, \( \| f \|_p\) et \( \| g \|_p\) sont plus petites que \( 1\). Vu que \( p>1\), nous avons
		\begin{equation}
			\| f \|_p^p+\| g \|_p^p\leq 1+1=2.
		\end{equation}
		En remplaçant dans le membre de droite de \eqref{EQooOWVEooGGfCpy} nous avons
		\begin{equation}
			\| \frac{ f+g }{2} \|_p^q+\| \frac{ f-g }{2} \|_p^q\leq 2^{1-q}2^{q-1}=1,
		\end{equation}
		et donc
		\begin{equation}        \label{EQooKARVooDrOuJI}
			\| \frac{ f+g }{2} \|_p^q\leq 1-\| \frac{ f-g }{2} \|_p^q.
		\end{equation}

		Par ailleurs nous avons supposé \( \| f-g \|_p\geq \epsilon\). Donc aussi\quext{Ici j'ai un coefficient un peu différent que celui de \cite{BIBooRISHooBcPPKQ}. Écrivez-moi pour confirmer ou infirmer mes calculs.}
		\begin{equation}        \label{EQooCGDDooWtDokf}
			\| \frac{ f-g }{2} \|_p^q\geq 2^{-q}\epsilon^q.
		\end{equation}

		Et par un autre ailleurs,
		\begin{equation}        \label{EQooOFWYooLVrNDc}
			\| \frac{ f+g }{2} \|_p=\frac{ 1 }{2}\| f+g \|_p\leq \frac{ 1 }{2}\big( \| f \|_p+\| g \|_p \big)\leq 1.
		\end{equation}
		Vu que nous avons \( q\geq 2\), cela donne aussi
		\begin{equation}        \label{EQooGMPRooGiLSss}
			\| \frac{ f+g }{2} \|_p\leq \| \frac{ f+g }{2} \|^q.
		\end{equation}

		Avec les inégalités \eqref{EQooCGDDooWtDokf} et \ref{EQooGMPRooGiLSss} nous finissons l'inégalité \eqref{EQooKARVooDrOuJI} :
		\begin{equation}
			\| \frac{ f+g }{2} \|_p\leq \| \frac{ f+g }{2} \|_p^q\leq 1-2^{-q}\epsilon^q\leq \| g \|_p-\delta(\epsilon).
		\end{equation}
		Okay, c'est bon.

		\spitem[\( 2\leq p<\infty\)]
		Il s'agit de faire la même chose en partant de Clarkson \eqref{EQooBWDJooGXzdxz}. Le résultat est que la fonction \( \delta(\epsilon)=(\epsilon/2)^p\), ça fonctionne.
	\end{subproof}
\end{proof}

%--------------------------------------------------------------------------------------------------------------------------- 
\subsection{Théorème de la projection normale}
%---------------------------------------------------------------------------------------------------------------------------

\begin{proposition}     \label{PROPooTZMRooCvQtGg}
	Si \( 1<p<\infty\), et si \( V\) est un sous-espace vectoriel fermé de \( L^p(\Omega,\tribA, \mu)\), alors la projection normale\footnote{Définition \ref{DEFooMYYLooJyACPL}.} de \( a\in L^p\) sur \( V\) existe et est unique.
\end{proposition}

\begin{proof}
	La proposition \ref{PROPooFNLJooDlyIKV} nous indique que l'espace \( L^p(\Omega,\tribA, \mu)\) est uniformément convexe. Or le théorème \ref{THOooOOVVooMhzHqd} nous indique que les espaces uniformément convexes vérifient la présente proposition.
\end{proof}

Nous pouvons donner une preuve directe, sans passer par l'uniforme convexité, dans les cas \( p\geq 2\).
\begin{theorem}[Théorème de la projection normale\cite{BIBooRYTOooYjaNkX}] \label{THOooRJFUooQivDKm}
	Nous considérons \( p\geq 2\). Soit un sous-espace vectoriel fermé \( W\subset L^p(\Omega,\tribA,\mu)\) et \( u_0\in L^p\). Nous notons
	\begin{equation}
		d(u_0,W)=\inf_{w\in W}d(u_0,W).
	\end{equation}
	Alors il existe \( w_0\in W\) tel que \( \| u_0-w_0 \|=d(u_0,W)\).
\end{theorem}

\begin{proof}
	Nous allons séparer trois cas : \( p=2\) et \( p>2\).
	\begin{subproof}
		\spitem[\( p=2\)]
		Pour \( p=2\), nous savons que \( L^2\) est un espace de Hilbert\footnote{Lemme \ref{LemIVWooZyWodb}.}, et nous avons déjà le théorème de la projection \ref{ThoProjOrthuzcYkz}.
		\spitem[\( p>2\)]
		Pour chaque \( x\in \Omega\) nous avons \( f(x), g(x)\in \eC\) et donc l'identité du parallélogramme\footnote{Théorème \ref{ThoAYfEHG} en remarquant que \( (z_1,z_2)\mapsto z_1\bar z_2\) est un produit scalaire hermitien sur \( \eC\).} :
		\begin{equation}        \label{EQooUBFEooDUjLnb}
			\big| f(x)-g(x) \big|^2+\big| f(x)+g(x) \big|^2=2| f(x) |^2+2| g(x) |^2.
		\end{equation}
		Vu que \( p>2\), la fonction \( s\colon x\mapsto  x^{p/2}\) est convexe (lemme \ref{LEMooSXTXooZOmtKq}). Calcul :
		\begin{subequations}
			\begin{align}
				| f(x)-g(x) |^p+| f(x)+g(x) |^p & =\big( | f(x)-g(x) |^2 \big)^{p/2}+\big( | f(x)+g(x) |^2 \big)^{p/2}                    \\
				                                & =s\big( | \ldots |^2 \big)+s\big( | \ldots |^2 \big)                                    \\
				                                & \leq \big( | f(x)-g(x) |^2+| f(x)+g(x) |^2 \big)^{p/2}     \label{SUBEQooRHAEooHkYNLH}  \\
				                                & =\big( 2| f(x) |^2+2| g(x) |^2 \big)^{p/2}                 \label{SUBEQooQFSLooJkoeqN}  \\
				                                & =2^{p/2}\big( | f(x) |^2+| g(x) |^2 \big)^{p/2}                                         \\
				                                & \leq  2^{p/2}2^{p/2-1}\big( | f(x) |^p+| g(x) |^p \big)     \label{SUBEQooQSUHooXKaWwO} \\
				                                & =2^{p-1}\big( | f(x) |^p+| g(x) |^p \big)
			\end{align}
		\end{subequations}
		Justifications :
		\begin{itemize}
			\item Pour \eqref{SUBEQooRHAEooHkYNLH} : la convexité de \( s\).
			\item Pour \eqref{SUBEQooQFSLooJkoeqN} : la relation \eqref{EQooUBFEooDUjLnb}.
			\item Pour \eqref{SUBEQooQSUHooXKaWwO} : la seconde inégalité du lemme \ref{SUBEQooQSUHooXKaWwO}.
		\end{itemize}
		Nous isolons \( | f(x)-g(x) |^p\) :
		\begin{subequations}
			\begin{align}
				| f(x)-g(x) |^p & \leq 2^{p-1}\big( | f(x) |^p+| g(x) |^p \big)-| f(x)+g(x) |^p                                       \\
				                & =2^p\left( \frac{ | f(x) |^p+| g(x) |^p }{2}-\left| \frac{ | f(x) |+| g(x) | }{2} \right|^p \right)
			\end{align}
		\end{subequations}
		Cette inégalité étant valable pour tout \( x\), nous pouvons intégrer sur \( \Omega\) et découper l'intégrale en petits morceaux :
		\begin{equation}        \label{EQooVNHSooPXjFNC}
			\| f-g \|^p_p\leq 2^p\left( \frac{ \| f \|_p^p+\| g \|_p^p }{2}- \| \frac{ f+g }{2} \|_p^p \right).
		\end{equation}
		Voilà une bonne chose de prouvée. Nous pouvons maintenant passer au vif du sujet.

		Soit une suite \( w_j\) dans \( W\) telle que \( \| u_0-w_j \|\to d(u_0,W)\). Trois choses à savoir sur cette suite :
		\begin{enumerate}
			\item
			      Une telle suite existe parce que \( d(u_0,W)\) est défini comme un infimum.
			\item
			      Rien ne garantit qu'elle converge.
			\item
			      Même si elle convergeait, rien ne garantirait que la limite soit encore dans \( W\).
		\end{enumerate}
		Le troisième point est facile à régler : vu que \( W\) est fermé par hypothèse, une suite convergente contenue dans \( W\) a sa limite dans \( W\). Nous allons régler la convergence de \( w_j\) en prouvant qu'elle est de Cauchy.

		Remarquons que \( W\) est vectoriel, donc \( (w_j+w_k)/2\) est dans \( W\) pour tout \( j\) et \( k\); donc
		\begin{equation}
			\| \frac{ w_j+w_k }{2}-u_0 \|\geq d(u_0,W).
		\end{equation}
		En tenant compte de cela, nous écrivons l'inégalité \eqref{EQooVNHSooPXjFNC} avec \( f=w_j-u_0\) et \( g=w_k-u_0\) :
		\begin{equation}
			\| f-g \|_p^p=\| w_j-w_k \|_p^p\leq 2^p\left( \frac{ \| w_j-u_0 \|^p+\| w_k-u_0 \|^p }{2}-d(u_0,W) \right).
		\end{equation}
		Soit \( \epsilon>0\) et \( 0<\epsilon_1,\epsilon_2<\epsilon\) tels que \( \epsilon_1+\epsilon_2<\epsilon\). Il existe un \( N\) tel que si \( j,k>N\) alors \( \| w_j-u_0 \|^p\leq d(u_0,W)^p+\epsilon_1\) et \( \| w_k-u_0 \|^p\leq d(u_0,W)^p+\epsilon_2\). Pour de telles valeurs de \( j\) et \( k\), nous avons
		\begin{equation}
			\| w_j-w_k \|_p\leq 2\left( \frac{ \epsilon_1+\epsilon_2 }{2} \right)<2\epsilon^{1/p}.
		\end{equation}
		Donc la suite \( (w_j)\) est de Cauchy.

		L'espace \( L^p\) étant complet par le théorème \ref{ThoUYBDWQX}, nous en déduisons que \( (w_j)\) converge dans \( L^p\). Mais comme \( W\) est fermé, nous avons \( w_j\stackrel{L^p}{\longrightarrow}w\in W\).

		En termes de normes, nous avons
		\begin{equation}
			\| w-u_0 \|=\lim_j\| w_j-u_0 \|=d(W,u_0).
		\end{equation}
	\end{subproof}
\end{proof}
