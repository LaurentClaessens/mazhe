% This is part of Mes notes de mathématique
% Copyright (c) 2011-2023
%   Laurent Claessens
% See the file fdl-1.3.txt for copying conditions.

%+++++++++++++++++++++++++++++++++++++++++++++++++++++++++++++++++++++++++++++++++++++++++++++++++++++++++++++++++++++++++++ 
\section{Sur \( \mathopen\lbrack -T , T \mathclose\lbrack\)}
%+++++++++++++++++++++++++++++++++++++++++++++++++++++++++++++++++++++++++++++++++++++++++++++++++++++++++++++++++++++++++++

Pour rappel, les éléments de \( L^2\) sont des classes de fonctions à valeurs dans \( \eC\).

\begin{proposition}     \label{PROPooHNJZooGfRCfU}
	Les fonctions
	\begin{equation}
		\begin{aligned}
			e_n\colon \mathopen[ -T , T \mathclose] & \to \eC                                        \\
			t                                       & \mapsto \frac{1}{ \sqrt{ 2T } } e^{\pi int/T}.
		\end{aligned}
	\end{equation}
	forment une base hilbertienne\footnote{Définition \ref{DEFooADQXooFoIhTG}.} de \( L^2\big( \mathopen[ -T , T \mathclose[ \big)\).
\end{proposition}

\begin{proof}
	C'est un cas particulier du théorème \ref{THOooAVWIooDhnjpN}.
\end{proof}

%--------------------------------------------------------------------------------------------------------------------------- 
\subsection{Le cas dans \( \mathopen[ 0 , 2\pi \mathclose]\)}
%---------------------------------------------------------------------------------------------------------------------------

En pratique, nous n'allons pas souvent travailler avec des fonctions sur intervalle symétrique \( \mathopen[ -T , T \mathclose]\), mais le plus souvent nous serons sur \( \mathopen[ 0 , 2\pi \mathclose]\).

Nous notons ici une conséquence du théorème~\ref{ThoGVmqOro} dans le cas de l'espace \( L^2\). La proposition suivante est une petite partie du corolaire~\ref{CorQETwUdF}, qui sera d'ailleurs démontré de façon indépendante.

\begin{proposition}
	Si nous avons une suite de réels \( (a_k)\) telle que \( \sum_{k=0}^{\infty}| a_k |^2<\infty\) alors la suite
	\begin{equation}
		f_n(x)=\sum_{k=0}^na_k e^{ikx}
	\end{equation}
	converge dans \( L^2\big( \mathopen] 0 , 2\pi \mathclose[ \big)\).
\end{proposition}

\begin{proof}
	Quitte à séparer les parties réelles et imaginaires, nous pouvons faire abstraction du fait que nous parlons d'une série de fonctions à valeurs dans \( \eC\) au lieu de \( \eR\).

	Un simple calcul est :
	\begin{equation}    \label{EqHVdJxZT}
		\| f_n-f_m \|^2\leq\int_0^{2\pi}\sum_{k=n}^m| a_k |^2dx\leq 2\pi\sum_{k=n}^m| a_k |^2.
	\end{equation}
	Par hypothèse le membre de droite est \( | s_m-s_n |\) où \( s_k\) dénote la suite des sommes partielles de la série des \( | a_k |^2\). Cette dernière est de Cauchy (parce que convergente dans \( \eR\)) et donc la limite \( n\to\infty\) (en gardant \( m>n\)) est zéro. Donc la suite des \( f_n\) est de Cauchy dans \( L^2\) et donc converge dans \( L^2\).
\end{proof}

\begin{normaltext}
	Adaptons tout cela pour l'espace \( L^2\big( \mathopen[ 0 , 2\pi \mathclose] \big)\). Nous posons
	\begin{equation}        \label{EQooBFKDooMkCZOt}
		\langle f, g\rangle =\int_0^{2\pi}f(t)\overline{ g(t) }dt
	\end{equation}
	et
	\begin{equation}        \label{EQooKMYOooLZCNap}
		e_n(t)=\frac{1}{ \sqrt{ 2\pi } } e^{int}.
	\end{equation}
\end{normaltext}

\begin{normaltext}
	Attention que \( e_n(x)\) n'est pas exactement \(  e^{inx}\) : il y a un coefficient. Lorsque ça a un sens, la théorie de Fourier permet d'écrire
	\begin{equation}
		f(x)=\sum_{n\in \eZ}c_n(f) e^{inx}.
	\end{equation}
	Ici les \( c_n(f)\) sont les coefficients de Fourier de \( f\). Ce développement n'est pas le même que
	\begin{equation}
		f(x)=\sum_{n\in \eZ}a_n(f)e_n(x).
	\end{equation}
	Dans cette dernière égalité, les \( a_n(f)\) ne sont pas les coefficients de Fourier, mais \( a_n=\langle f, e_n\rangle \). Le lien entre les deux est fondamentalement l'objet du corolaire \ref{CordgtXlC}.
\end{normaltext}


L'importance du système trigonométrique défini en \ref{DEFooGCZAooFecAHB} est d'être une base de \( L^2\big( \mathopen[ 0 , 2\pi \mathclose] \big)\), comme précisé dans le lemme suivant.
\begin{lemma}       \label{LEMooBJDQooLVPczR}
	Le système trigonométrique \( \{ e_n \}_{n\in \eZ}\) est une base hilbertienne\footnote{Définition \ref{DEFooADQXooFoIhTG}.} de \( L^2\big( \mathopen[ 0 , 2\pi \mathclose] \big)\).
\end{lemma}

\begin{proof}
	Cas particulier du théorème \ref{THOooAVWIooDhnjpN}.
\end{proof}

Note : le théorème~\ref{ThoDPTwimI} donné aussi la densité, mais sera démontré plus tard, indépendamment. Voir aussi les thèmes~\ref{THEooPUIIooLDPUuq} et~\ref{THEMooNMYKooVVeGTU}.

Pour un élément donné \( f\in L^2\big( \mathopen[ 0 , 2\pi \mathclose] \big)\), nous définissons\nomenclature[Y]{\( S_nf\)}{somme partielle de série de Fourier}
\begin{equation}
	S_nf=\sum_{k=-n}^n\langle f, e_k\rangle e_k
\end{equation}
et nous avons le théorème suivant, qui récompense les efforts consentis à propos de la densité des polynômes trigonométriques dans \( L^2\).

\begin{theorem} \label{ThoYDKZLyv}
	Soit \( f\in L^2\big( \mathopen[ 0 , 2\pi \mathclose] \big)\). Nous avons égalité\footnote{Notons que la somme sur \( \eZ\) dans \eqref{EqXMMRpSN} est commutative; il n'est donc pas besoin d'être plus précis.}
	\begin{equation}    \label{EqXMMRpSN}
		f=\sum_{n\in \eZ}c_n(f)e_n
	\end{equation}
	dans \( L^2\).

	Nous avons aussi la convergence
	\begin{equation}    \label{EqRBWKsYP}
		S_nf\stackrel{L^2}{\to} f.
	\end{equation}
\end{theorem}

\begin{proof}
	Le système trigonométrique \( \{ e_n \}_{n\in \eZ}\) est total pour l'espace de Hilbert \( L^2\big( \mathopen[ 0 , 2\pi \mathclose] \big)\) (sans périodicité particulière). Donc le point~\ref{ItemQGwoIxi} du théorème~\ref{ThoyAjoqP} nous donne l'égalité demandée.

	La convergence \eqref{EqRBWKsYP} est une reformulation de l'égalité \eqref{EqXMMRpSN}.
\end{proof}

\begin{normaltext}
	Obtenir la convergence \( L^2\) ne demande pas d'hypothèses de périodicité : la convergence \eqref{EqRBWKsYP} est automatique du fait que le système trigonométrique soit total. Ce n'est cependant pas plus qu'une convergence \( L^2\) et elle ne demande pas \( f(0)=f(2\pi)\), même si pour chacun des \( e_k\) nous avons \( e_k(0)=e_k(2\pi)\).

	Si \( f(2\pi)\neq f(0)\), alors il existe tout de même une suite \( (f_n)\) convergente vers \( f\) au sens \( L^2\) telle que \( f_n(0)=f_n(2\pi)\). Cela ne contredit en rien le fait que \( e_k(0)=e_k(2\pi)\) parce que dans \( L^2\), la valeur d'un point seul n'a pas d'importance.

	Si nous voulons une vraie convergence ponctuelle ou uniforme \( (S_nf)(x)\to f(x)\), alors il faut ajouter des hypothèses sur la continuité de \( f\), sa périodicité ou le comportement des coefficients \( c_n\). Voir aussi le thème~\ref{THMooHWEBooTMInve}.
\end{normaltext}

\begin{example}     \label{EXooQDWUooLtuIOm}
	Si \( f\in L^2\big( \mathopen[ 0 , 2\pi \mathclose] \big)\) est (la classe de) une fonction à valeurs réelles, alors on peut la développer avec nettement moins de termes. D'abord nous savons que \( e_{-n}=\overline{ e_n }\), et donc
	\begin{equation}
		\langle f, e_n\rangle =\overline{ \langle f, e_{-n}\rangle  },
	\end{equation}
	ce qui donne
	\begin{equation}
		f=\sum_{n\in\eZ}\langle f, e_n\rangle e_n
		=\sum_{n>0}\langle f, e_n\rangle e_n +\sum_{n<0}\overline{ \langle f, e_n\rangle e_n }+\langle f, e_0 \rangle e_0
		=\sum_{n\in \eN}\Re\big( \langle f, e_n\rangle e_n \big).
	\end{equation}
	Notez que \( f\) étant supposée réelle et \( e_0\) étant la fonction constante (réelle) \( 1/\sqrt{ 2\pi }\), le terme \( n=0\) est bien réel.

	Or
	\begin{equation}        \label{EQooMWJNooSjPCpR}
		\Re\big( \langle f, e_n\rangle e_n \big)=\frac{1}{ (2\pi)^{3/2} }\cos(nx)\int_0^{2\pi}f(t)\cos(nt)dt-\frac{1}{ (2\pi)^{3/2} }\sin(nx)\int_0^{2\pi}f(t)\sin(nt)dt.
	\end{equation}

	Considérons la fonction impaire \( \tilde f\in L^2\big( [-2\pi,2\pi] \big)\) créée à partir de \( f\). Elle se développe de même et nous avons la même formule \eqref{EQooMWJNooSjPCpR} à part quelques coefficients et le fait que les intégrales sont entre \( -2\pi\) et \( 2\pi\). Vu que \( \tilde f\) est impaire, l'intégrale avec \( \cos(nt)\) s'annule et
	\begin{equation}
		\tilde f(x)=\sum_{n\in \eN}c_n\sin(nx)
	\end{equation}
	pour certains coefficients réels \( c_n\). Cette égalité est à considérer dans \( L^2\), c'est-à-dire presque partout et en particulier presque partout sur \( \mathopen[ 0 , 2\pi \mathclose]\).

	Donc les fonctions réelles sur \( \mathopen[ 0 , 2\pi \mathclose]\) peuvent être écrites sous la forme d'une série de seulement des sinus.

	Note : en choisissant \( \tilde f\) paire, nous aurions eu une série de cosinus.
\end{example}

%+++++++++++++++++++++++++++++++++++++++++++++++++++++++++++++++++++++++++++++++++++++++++++++++++++++++++++++++++++++++++++ 
\section{Théorème de la projection normale}
%+++++++++++++++++++++++++++++++++++++++++++++++++++++++++++++++++++++++++++++++++++++++++++++++++++++++++++++++++++++++++++

%--------------------------------------------------------------------------------------------------------------------------- 
\subsection{Espace uniformément convexe}
%---------------------------------------------------------------------------------------------------------------------------

\begin{definition}[Espace uniformément convexe\cite{BIBooPYEZooTxohAd}]     \label{DEFooOPQBooBhufew}
	Un espace de Banach \( B\) est \defe{uniformément convexe}{uniformément convexe} si il existe une fonction \( \delta\colon \mathopen] 0 , \infty \mathclose[\to \eR^+\) telle que si
	\begin{enumerate}
		\item
		      \( \| x \|\leq \| y \|\leq 1\),
		\item
		      \( \| x-y \|\geq \epsilon\),
	\end{enumerate}
	alors
	\begin{equation}
		\| \frac{ x+y }{ 2 } \|\leq \| y \|-\delta(\epsilon).
	\end{equation}
\end{definition}

\begin{lemma}[\cite{MonCerveau}]
	Si \( B\) est un espace de Banach uniformément convexe, alors pour tout \( k>0\), il existe une fonction \( \delta_k\colon \mathopen] 0 , \infty \mathclose[\to \eR^2\) telle que si
	\begin{enumerate}
		\item
		      \( \| x \|\leq \| y \|\leq k\),
		\item
		      \( \| x-y \|\geq \epsilon\),
	\end{enumerate}
	alors
	\begin{equation}
		\| \frac{ x+y }{ 2 } \|\leq \| y \|-\delta_k(\epsilon).
	\end{equation}
\end{lemma}

\begin{proof}
	Nous posons \( x'=x/k\) et \( y'=y/k\). Nous avons alors
	\begin{equation}
		\| x'-y' \|=\frac{ \| x-y \| }{ k }>\frac{ \epsilon }{ k }.
	\end{equation}
	L'uniforme convexité de \( B\) dit alors que
	\begin{equation}
		\| \frac{ x'+y' }{2} \|>\| y' \|-\delta(\epsilon/k).
	\end{equation}
	En multipliant cette inégalité par \( k\) nous trouvons
	\begin{equation}
		\| \frac{ x+y }{2} \|>\| y \|-k\delta(\epsilon/k).
	\end{equation}
	Donc en posant \( \delta_k(\epsilon)=k\delta(\epsilon/k)\), nous avons le résultat escompté.
\end{proof}

\begin{definition}[Projection normale\cite{BIBooPYEZooTxohAd}]      \label{DEFooMYYLooJyACPL}
	Soient un espace de Banach \( B\) ainsi que \( V\subset B\). Soit \( a\in B\). La fonction
	\begin{equation}
		\begin{aligned}
			f\colon V & \to \eR        \\
			x         & \mapsto d(x,a)
		\end{aligned}
	\end{equation}
	possède un infimum\footnote{Toute fonction à valeurs positives possède un infimum, c'est la proposition \ref{DefSupeA}.} \( m\). Si \( x\in V\) est tel que \( d(x,a)=m\), alors \( x\) est une \defe{projection normale}{projection normale} de \( a\) sur \( V\).
\end{definition}

\begin{proposition}[\cite{BIBooPYEZooTxohAd}]       \label{PROPooDKXVooUoYPgz}
	Soient un espace de Banach \( B\) et un sous-espace vectoriel \( V\subset B\). Si une projection normale de \( a\in B\) sur \( V\) existe, alors elle est unique.
\end{proposition}

\begin{proof}
	Soient deux projections normales \( b,b'\) de \( a\) sur \( V\).

	Si \( m=0\), alors \( \| a-b \|=0\) et \( \| a-b' \|=0\), ce qui donne \( a=b\) et \( a=b'\). Donc d'accord pour \( b=b'\).

	Si \( m>0\) alors nous utilisons l'inégalité \( \| x+y \|\leq \| x \|+\| y \|\) sous la forme
	\begin{equation}        \label{EQooQWJWooVaWMCL}
		\| a-\frac{ b+b' }{ 2 } \|=\| \frac{ a-b }{2}+\frac{ a-b' }{2} \|\leq \| \frac{ a-b }{2} \|+\| \frac{ a-b' }{2} \|=\frac{ m }{ 2 }+\frac{ m }{2}=m.
	\end{equation}
	Mais \( \frac{ b+b' }{2}\in V\), donc
	\begin{equation}
		\| a-\frac{ b+b' }{2} \|\geq m.
	\end{equation}
	Nous en déduisons que dans \eqref{EQooQWJWooVaWMCL}, toutes les inégalités sont des égalités et en particulier
	\begin{equation}
		\| \frac{ b+b' }{2}-a \|=m.
	\end{equation}
	Nous avons donc les deux égalités suivantes :
	\begin{equation}
		2m=\| a-b \|+\| a-b' \|
	\end{equation}
	et
	\begin{equation}
		2m=\| b+b'-2a \|.
	\end{equation}
	Cela donne
	\begin{equation}
		\| a-b \|+\| a-b' \|=\| (a-b)+(a-b') \|.
	\end{equation}
	Vu que \( B\) est strictement convexe, cela n'est possible que si \( a-b=a-b'\), ce qui signifie que \( b=b'\).
\end{proof}

\begin{theorem}[\cite{BIBooVHQSooTrLCzQ,BIBooPYEZooTxohAd}]     \label{THOooOOVVooMhzHqd}
	Si \( B\) est un espace de Banach uniformément convexe, si \( V\subset B\) est un sous-espace vectoriel fermé et si \( a\in B\), alors \( a\) admet une unique projection normale\footnote{Définition \ref{DEFooMYYLooJyACPL}.} sur \( V\).
\end{theorem}

\begin{proof}
	En deux parties.
	\begin{subproof}
		\spitem[Unicité]
		Soient deux projections normales \( b\) et \( b'\) de \( a\) sur \( V\). Nous avons \( \| a-b \|=\| a-b' \|=m\). Si \( b\neq b'\), il existe \( \epsilon>0\) tel que
		\begin{equation}
			\| b-b' \|>\epsilon>0.
		\end{equation}
		En posant
		\begin{equation}
			\begin{aligned}[]
				x & =\frac{ 1 }{2}\frac{ b-a }{ m }, & y=\frac{ 1 }{2}\frac{ b'-a }{ m },
			\end{aligned}
		\end{equation}
		nous avons \( \| x \|=\| y \|=\frac{ 1 }{2}<1\). L'uniforme convexité de \( B\) donne alors
		\begin{equation}
			\| \frac{ x+y }{2} \|\leq \| y \|-\delta(\epsilon).
		\end{equation}
		Mais
		\begin{equation}
			x+y=\frac{ \frac{ 1 }{2}(b+b')-a }{ m }
		\end{equation}
		et
		\begin{equation}
			\| x+y \|\leq 2\| y \|-2\delta(\epsilon)=1-2\delta(\epsilon)<1.
		\end{equation}
		Nous avons donc prouvé que
		\begin{equation}
			\| \frac{ 1 }{2}(b+b')-a \|<m,
		\end{equation}
		ce qui est impossible parce que cela dirait que \( \frac{ b+b' }{2}\) est une «meilleure» projection normale que \( b\) et \( b'\).

		\spitem[Existence]
		Soient \( b_k\) dans \( V\) tels que \( \| a-b_k \|\to m\). Nous supposons (quitte à passer à une sous-suite) que
		\begin{equation}
			\| a-b_{k+1} \|\leq \| a-b_k \|.
		\end{equation}

		\begin{subproof}
			\spitem[La suite $(b_k)$ converge]

			Nous supposons qu'elle ne converge pas. Elle n'est donc pas de Cauchy parce que \( B\) est de Banach\footnote{Définition \ref{DefVKuyYpQ}.} et donc complet. Il existe \( \epsilon>0\) tel que pour tout \( N\in \eN\) il existe \( p,q>N\) tels que
			\begin{equation}
				\| b_p-b_q \|>\epsilon.
			\end{equation}

			Nous effectuons quelque choix.
			\begin{enumerate}
				\item
				      nous choisissons \( q>p\) de telle sorte que \( \| a-b_p \|\leq\| a-b_q \|\),
				\item
				      nous choisissons \( N\) assez grand pour avoir
				      \begin{equation}
					      \| a-b_p \|\leq \| a-b_q \|<m.
				      \end{equation}
			\end{enumerate}

			Nous avons \( \| (a-b_p)-(a-b_q) \|=\| b_q-b_p \|>\epsilon\), ce qui avec l'uniforme convexité donne
			\begin{equation}
				\frac{ \| (a-b_p)+(a-b_q) \| }{2}\leq \| a-b_q \|-\delta(\epsilon).
			\end{equation}
			Donc
			\begin{equation}
				m\leq \| a-\frac{ b_p-b_q }{ 2 } \|=\| \frac{ (a-b_p)+(a-b_q) }{2} \|\leq \| a-b_q \|-\delta(\epsilon)<m-\delta(\epsilon)<m.
			\end{equation}
			Cela signifie que \( m<m\), ce qui est impossible.
			\spitem[Conclusion]
			La suite \( (b_k)\) converge dans \( B\). Vu que \( V\) est fermé, la limite est dans \( V\). Cette limite, que nous nommons \( b\), vérifie
			\begin{equation}
				\| a-b \|\leq \| a-b_k \|
			\end{equation}
			pour tout \( k\). Mais comme nous avons \( m\leq \| a-b_k \|\to m\), nous avons \( \| a-b \|=m\), c'est-à-dire que \( b\) est une projection normale de \( a\) sur \( V\).
		\end{subproof}
	\end{subproof}
\end{proof}

%--------------------------------------------------------------------------------------------------------------------------- 
\subsection{Des inégalités}
%---------------------------------------------------------------------------------------------------------------------------

Avant d'entrer dans le vif du sujet, nous nous fendons d'une petite étude de fonction. Soit
\begin{equation}
	\begin{aligned}
		\phi\colon \mathopen[ 0 , 1 \mathclose] & \to \eR                            \\
		x                                       & \mapsto \frac{ (1+x)^r }{ 1+x^r }.
	\end{aligned}
\end{equation}
Un peu de calcul montre que
\begin{equation}
	\frac{ \phi'(x) }{ \phi(x) }=\frac{ r(1-x^{r-1}) }{ (1+x^r)(1+x) }.
\end{equation}

\begin{lemma}       \label{LEMooFKKEooDTypUd}
	Soient \( a,b>0\) et \( r>1\). Nous avons les inégalités
	\begin{equation}
		a^r+b^r\leq (a+b)^r\leq 2^{r-1}(a^r+b^r).
	\end{equation}
\end{lemma}

\begin{proof}
	Pour la première inégalité, nous posons \( f(x)=a^r+x^r\) et \( g(x)=(a+x)^r\). Nous avons \( f(0)=g(0)=a^r\), et, en utilisant la fonction \( f_{\alpha}\) définite par \( f_{\alpha}(x)=x^{^\alpha}\), nous avons
	\begin{subequations}
		\begin{align}
			f'(x) & =rx^{r-1} =rf_{r-1}(x)       \\
			g'(x) & =r(a+x)^{r-1}=rf_{r-1}(a+x).
		\end{align}
	\end{subequations}
	Vu que \( r>1\), la fonction \( f_{r-1}\) est strictement croissante sur les positifs par la proposition \ref{PROPooUOFKooYyGwIr}. Donc pour \( x\geq 0\) nous avons
	\begin{equation}
		f'(x)=rf_{r-1}(x)\leq rf_{r-1}(a+x)=g'(x).
	\end{equation}
	Nous avons donc \( f(b)\leq g(b)\) comme souhaité.

	Nous passons à la seconde inégalité. Le lemme \ref{LEMooSXTXooZOmtKq} nous dit que la fonction \( f\colon x\mapsto x^r \) est convexe. Donc
	\begin{equation}
		f\left( \frac{ a }{2}+\frac{ b }{2} \right)\leq\frac{ 1 }{2}f(a)+\frac{ 1 }{2}f(b).
	\end{equation}
	De là nous déduisons
	\begin{equation}
		\frac{ (a+b)^r }{ 2^r }\leq \frac{ 1 }{2}(a^r+b^r),
	\end{equation}
	c'est-à-dire la seconde inégalité.
\end{proof}

Nous allons démontrer les inégalités de Hanner dans le théorème \ref{THOooZRRYooBTBQKW}. Vu que ce sera un peu longuet, nous faisons un lemme.
\begin{lemma}       \label{LEMooDHRCooQiSpyC}
	Soient \( z_1,z_2\in \eC\). Nous avons
	\begin{equation}        \label{EQooMUXVooSpGSyG}
		| z_1+z_2 |^p+| z_1-z_2 |^p\geq \big( | z_1 |+| z_2 | \big)^p+\big| | z_1 |-| z_2 | \big|^p.
	\end{equation}
\end{lemma}

\begin{proof}
	Soient \( z_1,z_2\in \eC\). Nous posons
	\begin{equation}        \label{EQooJKYZooFzbETG}
		d=| z_1+z_2 |^p+| z_1-z_2 |^p.
	\end{equation}
	Pour \( | z_1 |\) et \( | z_2 |\) fixés, nous nous demandons quel est le minimum possible de \( d\).

	Si \( | z_1 |=0\), alors le minimum est \( 2| z_2 |^p\) et si \( | z_2 |=0\) alors il est \( 2| z_1 |^p\). Pour les autres cas, nous posons \( | z_1 |=a>0\) ainsi que \( b\in \eR\) et \( \theta\in \eR\) tels que\footnote{Proposition \ref{PROPooRFMKooURhAQJ}}
	\begin{equation}
		z_2=z_1a^{-1}b e^{i\theta}.
	\end{equation}
	Nous avons déjà que \( z_1+z_2=z_1(1+a^{-1}b e^{i\theta})\) et donc
	\begin{equation}
		| z_1+z_2 |=a| 1+a^{-1}b e^{i\theta} |=| a+b e^{i\theta} |
	\end{equation}
	parce que \( a>0\). De plus,
	\begin{equation}
		| a+b e^{i\theta} |^2= (a+b e^{i\theta})(a+b e^{-i\theta})=a^2+b^2+2ab\cos(\theta)
	\end{equation}
	parce que \(  e^{i\theta}+ e^{-i\theta}=\cos(\theta)\). Nous posons
	\begin{equation}
		d(\theta)=| a+b e^{i\theta} |^p+| a-b e^{i\theta} |^p.
	\end{equation}
	En développant,
	\begin{equation}
		d(\theta)=\big(a^2+b^2+2ab\cos(\theta)\big)^{p/2}+\big(a^2+b^2-2ab\cos(\theta)\big)^{p/2}.
	\end{equation}
	Trouvons le minimum de cette fonction de \( \theta\). D'abord sa dérivée :
	\begin{subequations}
		\begin{align}
			d'(\theta) & =pab\sin(\theta)\big[ \big( a^2+b^2-2ab\cos(\theta) \big)^{p/2-1}-\big( a^2+b^2+2ab\cos(\theta) \big)^{p/2-1}  \big] \\
			           & =pab\sin(\theta)s(\theta).
		\end{align}
	\end{subequations}
	Nous avons \( s(\theta)=0\) pour \( \theta=\pi/2\) et \( \theta=3\pi/2\). Il faut surtout remarquer que \( 1<p<2\), ce qui donne \( \frac{ p }{2}-1<0\). La fonction \( x\mapsto x^{p/2-1}\) est donc décroissante. Cela pour dire que
	\begin{equation}
		s(0)=\left( | a-b |^2 \right)^{p/2-1}-\left( | a+b |^2 \right)^{p/2-1}>0.
	\end{equation}
	De la même façon, \( s(\pi)=-s(0)<0\). Cela permet d'écrire un petit tableau de signe de \( d'\), et de conclure que \( d(\theta)\) a un minimum en \( 0\) et en \( \pi\). Calcul fait, nous avons
	\begin{equation}
		d(0)=d(\pi)=| a+b |^p+| a-b |^p.
	\end{equation}
	En reliant à \eqref{EQooJKYZooFzbETG} nous avons l'inégalité
	\begin{equation}        \label{EQooVHQOooJcheCR}
		| z_1+z_2 |^p+| z_1-z_2 |^p\geq (a+b)^p-| a-b |^p.
	\end{equation}
	Nous rappelons que \( a=| z_1 |\) et que \( z_2=z_1a^{-1}b e^{i\theta}\). Notons au passage que \( | z_2 |=b\), donc que ce que nous dit l'équation \eqref{EQooVHQOooJcheCR} est que
	\begin{equation}
		| z_1+z_2 |^p+| z_1-z_2 |^p\geq \big( | z_1 |+| z_2 | \big)^p+\big| | z_1 |-| z_2 | \big|^p.
	\end{equation}
\end{proof}

Encore dans la catégorie des lemmes pour les inégalités de Hanner, nous avons celui-ci.
\begin{lemma}[\cite{MonCerveau,ooKGWWooAybolH}]     \label{LEMooTCNEooADpNai}
	La fonction
	\begin{equation}
		\begin{aligned}
			\eta\colon \mathopen] 0 , \infty \mathclose[ & \to \eR                               \\
			a                                            & \mapsto (a^{1/p}+1)^p+| a^{1/p}-1 |^p
		\end{aligned}
	\end{equation}
	est strictement convexe.
\end{lemma}

\begin{proof}
	La fonction \( \eta\) est une fonction de classe \(  C^{\infty}\) sur \( \mathopen] 0 , \infty \mathclose[\setminus\{ 1 \}\). Quelle est sa régularité en \( a=1\) ? À cause de la valeur absolue, il n'est pas clair qu'elle y soit dérivable. En tout cas, la fonction \( x\mapsto| x-1 |\) n'est pas dérivable en \( x=1\), mais peut-être que les exposants aident à lisser. Nous y reviendrons.

		Afin de  suivre les calculs nous introduisons quelques fonctions :
		\begin{subequations}
			\begin{align}
				so(x) & =1+x^{1/p} \\
				di(x) & =1-x^{1/p} \\
				dj(x) & =x^{1/p}-1
			\end{align}
		\end{subequations}
		Pour les dérivées, nous avons
		\begin{subequations}
			\begin{align}
				so'(x) & =\frac{1}{ p }x^{1/p-1} \\
				di'(x) & =-so'(x)                \\
				dj'(x) & =so'(x).
			\end{align}
		\end{subequations}
		Nous divisons les cas selon \( a<1\) ou \( a>1\).
		\begin{subproof}

			\spitem[Pour \( a<1\)]
			%----------------------------------------------
			Nous avons
			\begin{equation}
				\eta(a)=so(a)^p+di(a)^p,
			\end{equation}
			et la première dérivée donne :
			\begin{equation}        \label{EQooCLXZooXClOwd}
				\eta'(a)=p\,so'(a)\big( so(a)^{p-1}-di(a)^{p-1} \big).
			\end{equation}
			Pour la seconde dérivée nous trouvons d'abord
			\begin{equation}
				\begin{aligned}[]
					\eta''(a) & =\left( \frac{ 1-p }{ p } \right)a^{\frac{ 1 }{ p }-2}\big( so(a)^{p-1}-di(a)^{p-1} \big) \\
					          & \quad+\frac{ p-1 }{ p }a^{\frac{ 2 }{ p }-2}\big( so(a)^{p-2}+di(a)^{p-2} \big).
				\end{aligned}
			\end{equation}
			À partir de là, le truc est de substituer les expressions suivantes :
			\begin{subequations}
				\begin{align}
					so(a)^{p-1} & =so(a)^{p-2}so(a)=so(a)^{p-2}+so(a)^{p-2}a^{1/p} \\
					di(a)^{p-1} & =di(a)^{p-2}-a^{1/p}di(a)^{p-2}.
				\end{align}
			\end{subequations}
			Plein de trucs se simplifient et nous obtenons
			\begin{equation}
				\eta''(a)=\frac{ p-1 }{ p }a^{\frac{1}{ p }-2}\big( di(a)^{p-1}-so(a)^{p-2} \big).
			\end{equation}

			\spitem[Pour \( a>1\)]
			%----------------------------------------------
			Les calculs sont essentiellement les mêmes, en partant de
			\begin{equation}
				\eta(a)=so(a)^p+dj(a)^p.
			\end{equation}
			Les résultats sont :
			\begin{equation}    \label{EQooAJLHooGWjPlz}
				\eta'(a)=p\,so'(a)\big( so(a)^{p-1}+dj(a)^{p-1} \big),
			\end{equation}
			et
			\begin{equation}
				\eta''(a)=\frac{ p-1 }{ p }a^{\frac{1}{ p }-2}\big( dj(a)^{p-2}-so(a)^{p-2} \big).
			\end{equation}

			\spitem[Résumé pour \( a\neq 1\)]
			%-----------------------------------------------------------

			Au final, nous avons pour tout \( a\neq 1\) :
			\begin{equation}
				\eta''(a)=\frac{ p-1 }{ p }a^{\frac{1}{ p }-2}\big( | 1-a^{1/p} |^{p-2}-(1+a^{1/p})^{p-2} \big).
			\end{equation}
			Ce qu'il se passe en \( a=1\) est encore une question ouverte que nous traitons maintenant.

			\spitem[Pour \( a=1\)]
			%-----------------------------------------------------------

			Les limites des expressions \eqref{EQooCLXZooXClOwd} et \eqref{EQooAJLHooGWjPlz} en \( a=1\) sont vite calculées et c'est \( 2^{p-1}\) dans les deux cas. Donc la dérivée admet une prolongation continue en \( a=1\). Nous allons prouver que la fonction \( \eta\) est en réalité dérivable en \( a=1\) et que la dérivée vaut \( 2^{p-1}\).

			Nous nous concentrons sur la partie difficile donnée par \( f(x)=| x^{1/p}-1 |^p\). Elle est donnée par
			\begin{equation}
				f(x)=\begin{cases}
					di(x)^p & \text{si } x<1  \\
					dj(x)^p & \text{si } x>1  \\
					0       & \text{si } x=1.
				\end{cases}
			\end{equation}
			Si \( f'(1)\) existe, alors elle est égale à la limite

			\begin{equation}
				f'(1)=\lim_{\epsilon\to 0}\frac{ f(1)-f(1-\epsilon) }{ \epsilon }.
			\end{equation}
			Les deux limites à calculer sont :
			\begin{equation}
				\lim_{\epsilon\to 0^+}\frac{ \big( (1+\epsilon)^{1/p}-1 \big)^p }{ \epsilon }
			\end{equation}
			et
			\begin{equation}
				\lim_{\epsilon\to 0^-}\frac{ \big( 1-(1+\epsilon)^{1/p} \big)^p }{ \epsilon }.
			\end{equation}
			La première se traite par la règle de l'Hospital\footnote{Proposition \ref{PROPooBZHTooHmyGsy}}, et le résultat est zéro. Pour la seconde, il faut juste transformer
			\begin{equation}
				\lim_{\epsilon\to 0^+}\frac{ \big( (1+\epsilon)^{1/p}-1 \big)^p }{ \epsilon }=\lim_{h\to 0^+} \frac{ \big( 1-(1-h)^{1/p} \big)^p }{ -h },
			\end{equation}
			qui se traite également par la règle de l'Hospital. Le résultat est également zéro.

			Donc \( \eta\) est dérivable en \( a=1\) et la dérivée vaut \(\eta'(1)= 2^{p-1}\).
		\end{subproof}
		En récapitulant, nous avons \( \eta''>0\) sur \( \mathopen] 0  , \infty \mathclose[\setminus\{ 1 \}\), donc \( \eta'\) est croissante sur cette partie (proposition \ref{PropGFkZMwD}). Vu que \( \eta'\) est continue sur \( \mathopen] 0 , \infty \mathclose[\), elle est même croissante (strictement) sur tout \( \mathopen] 0 , \infty \mathclose[\).

		La proposition \ref{PropYKwTDPX} conclut que \( \eta\) est strictement convexe sur \( \mathopen] 0 , \infty \mathclose[\).
\end{proof}

Toujours dans la catégorie des lemmes pour les inégalités de Hanner, nous avons celui-ci.
\begin{lemma}[\cite{ooKGWWooAybolH}]
	Soit \( 1<p<2\). La fonction
	\begin{equation}
		\begin{aligned}
			\xi\colon \eR^+\times \eR^+ & \to \eR                                                     \\
			(a,b)                       & \mapsto \big( a^{1/p}+b^{1/p} \big)^p+| a^{1/p}-b^{1/p} |^p
		\end{aligned}
	\end{equation}
	est convexe.

	Pour rappel, les conventions de données en \ref{REMooOCXLooKQrDoq} donnent \( \eR^+=\mathopen[ 0 , \infty \mathclose[\).
\end{lemma}

\begin{proof}
	La fonction \( \xi\) vérifie facilement les conditions suivante :
	\begin{itemize}
		\item \( \xi(a,b)=\xi(b,a)\),
		\item \( \xi(0,0)=0\),
		\item \( \xi(ta,tb)=t\xi(a,b)\) pour tout \( t\geq 0\).
	\end{itemize}
	Nous posons
	\begin{equation}
		\begin{aligned}
			\eta\colon \eR^+ & \to \eR           \\
			a                & \mapsto \xi(a,1).
		\end{aligned}
	\end{equation}
	Le lemme \ref{LEMooTCNEooADpNai} dit que \( \eta\) est strictement convexe, et le lemme \ref{LEMooNUDOooVfVPkw} conclut que \( \xi\) est convexe.
\end{proof}

\begin{lemma}[\cite{BIBooGPACooYtOhPP}]     \label{LEMooWIPYooMZqjbn}
	Soit \( 1<p<2\). Nous considérons les fonctions
	\begin{equation}
		\begin{aligned}
			\alpha\colon \mathopen[ 0 , 1 \mathclose] & \to \eR                         \\
			x                                         & \mapsto (1+x)^{p-1}+(1-x)^{p-1}
		\end{aligned}
	\end{equation}
	et
	\begin{equation}
		\begin{aligned}
			\beta\colon \mathopen[ 0 , 1 \mathclose] & \to \eR                                             \\
			x                                        & \mapsto x^{1-p}\big( (1+x)^{p-1}-(1-x)^{p-1} \big).
		\end{aligned}
	\end{equation}
	Soient \( A,B\in \eR\). Nous avons
	\begin{equation}
		\alpha(x)| A |^p+\beta(x)| B |^p\leq | A+B |^p+| A-B |^p.
	\end{equation}
\end{lemma}

\begin{proof}
	Plusieurs étapes.
	\begin{subproof}
		\spitem[\( \beta(x)\leq \alpha(x)\)]
		Nous avons \( \alpha(1)=\beta(1)=2^{p-1}\). Pour les autres valeurs de \( x\), nous allons raisonner avec la dérivée. La valeur de \( \alpha'(x)\) est facile à calculer
		\begin{equation}
			\alpha'(x)=(p-1)(x+1)^{p-2}-(p-1)(1-x)^{p-2}.
		\end{equation}
		Pour \( \beta'(x)\) c'est un peu plus lourd. En substituant \( (1+x)^{p-1}=(1+x)^{p-2}(1+x)\) et \( (1-x)^{p-1}=(1-x)^{p-2}(1-x)\) nous pouvons regrouper les termes en \( (1+x)^{p-2}\) et \( (1-x)^{p-2}\). Après un peu de travail,
		\begin{equation}
			\beta'(x)=\frac{ p-1 }{ x^p }\big( (1-x)^{p-2}-(1+x)^{p-2} \big).
		\end{equation}
		Cela nous permet de calculer \( \alpha'-\beta'\) :
		\begin{equation}
			\alpha'(x)-\beta'(x)=(p-1)\big( 1+\frac{1}{ x^p } \big)\big( (1+x)^{p-2}-(1-x)^{p-2} \big).
		\end{equation}
		Vu que \( 1<p<2\), le nombre \( p-2\) est strictement négatif; afin de travailler avec des exposants positifs, nous écrivons
		\begin{equation}
			\alpha'(x)-\beta'(x)=\underbrace{(p-1)}_{>0}\underbrace{\big( 1+\frac{1}{ x^p } \big)}_{>0}\underbrace{\left( \frac{1}{ (1+x)^{2-p}}-\frac{1}{ (1-x)^{2-p} }  \right)}_{<0}.
		\end{equation}
		Nous avons \( \alpha'(x)-\beta'(x)<0\) pour tout \( x\in \mathopen] 0 , 1 \mathclose]\). Du fait qu'en plus nous ayons \( \alpha(1)=\beta(1)\), nous déduisons que \( \alpha(x)\geq \beta(x)\).

		\spitem[Une petite étude de fonction]
		%-----------------------------------------------------------
		Soit \( R\in \mathopen[ 0 , 1 \mathclose]\). Nous considérons la fonction
		\begin{equation}
			\begin{aligned}
				F\colon \mathopen[ 0 , 1 \mathclose] & \to \eR                        \\
				x                                    & \mapsto \alpha(x)+R^p\beta(x).
			\end{aligned}
		\end{equation}
		Nous montrons maintenant que cette fonction a un maximum global pour \( x=R\). D'abord sa dérivée :
		\begin{equation}
			F'(x)=\underbrace{(p-1)}_{>0}\underbrace{\Big( (1-x)^{p-1}-(1+x)^{p-2} \Big)}_{<0}\Big( 1-\left( \frac{ R }{ x } \right)^p \Big)
		\end{equation}
		Nous avons
		\begin{itemize}
			\item \( F'(x)=0\) pour \( x=R\),
			\item \( F'(x)<0\) pour \( x>R\),
			\item \( F'(x)>0\) pour \( x<R\).
		\end{itemize}
		Donc \( x=R\) est bien un maximum global.
		\spitem[Pause]
		Nous avons les petits résultats utiles pour commencer à prouver. Petite pause avant de commencer; pas de panique, ça ne va pas être trop violent.
		\spitem[Pour \( 0<B<A\)]
		Nous devons prouver que
		\begin{equation}        \label{EQooEPKRooBYJDSF}
			\alpha(x)A^p+\beta(x)B^p\leq (A+B)^p+(A-B)^p.
		\end{equation}
		En divisant par \( A^p\) et en posant \( R=B/A\), l'inéquation \eqref{EQooEPKRooBYJDSF} est équivalente à
		\begin{equation}
			\alpha(x)+\beta(x)R^p\leq (1+R)^p+(1-R)^p
		\end{equation}
		où \( R\in \mathopen] 0 , 1 \mathclose[\) parce que nous avons supposé \( 0<B<A\). Nous avons (il y a un petit calcul pour \( F(R)\))
		\begin{equation}
			(1+R)^p+(1-R)^p=F(R)\geq F(x)=\alpha(x)+\beta(x)R^p.
		\end{equation}
		ok.
		\spitem[Pour \( 0<A<B\)]
		Lorsque \( 0<A<B\) nous avons
		\begin{subequations}
			\begin{align}
				\alpha(x)| A |^p+\beta(x)| B |^p & =\alpha(x)A^p+\beta(x)B^p                                      \\
				                                 & \leq \alpha(x)B^p+\beta(x)A^p      \label{SUBEQooSHNUooCoWMFB} \\
				                                 & \leq (B+A)^p+(B-A)^p       \label{SUBEQooBPYVooPsAjbq}         \\
				                                 & =| A+B |^p+| A-B |^p.
			\end{align}
		\end{subequations}
		Justification :
		\begin{itemize}
			\item Pour \eqref{SUBEQooSHNUooCoWMFB}, c'est parce que \( \alpha(x)>\beta(x)\); alors en mettant le plus grand de \( A\) et \( B\) devant le \( \alpha\) au lieu du \( \beta\), nous majorons.
			\item Pour \eqref{SUBEQooBPYVooPsAjbq}, c'est l'inégalité dans le cas \( 0<B<A\), mais en inversant les noms de \( A\) et \( B\).
		\end{itemize}
		\spitem[Pour \( 0<A=B\)]
		Toutes les expressions sont continues par rapport à \( B\) (fixons \( x\) et \( A\)). Nous avons prouvé pour \( B<A\) et pour \( B>A\). Par continuité, l'inégalité est encore valide pour \( A=B\).
		\spitem[Pour \( A<0\), \( B>0\)]
		En posant \( A'=-A\) nous avons \( A'>0\) et nous pouvons écrire
		\begin{equation}
			| A+B |^p+| A-B |^p=| -A'+B |^p+| -A'-B |^p=| B-A' |^p+| B+A' |^p\geq \alpha(x)| A' |^p+\beta(x)| B |^p.
		\end{equation}
		Nous avons utilisé, avec \( A'\) et \( B\) le cas déjà prouvé \( A',B>0\).
		\spitem[Pour \( A>0\), \( B<0\)]
		Celui-là, je vous le laisse.
		\spitem[Pour \( A<0\), \( B<0\)]
		Posez \( A'=-A\) et \( B'=-B\) et hop.
	\end{subproof}
\end{proof}


\begin{theorem}[Inégalités de Hanner\cite{ooKGWWooAybolH,BIBooGPACooYtOhPP}]       \label{THOooZRRYooBTBQKW}
	Soit un espace mesuré \(  (\Omega,\tribA,\mu)\). Soit \( 1<p<2\) et \( f,g\in L^p(\Omega,\tribA,\mu)\); nous avons
	\begin{equation}
		\big( \| f \|_p+\| g \|_p \big)^p+\Big| \| f \|_p-\| g \|_p \Big|^p
		\leq \| f+g \|_p^p+\| f-g \|_p^p
		%           \leq 2\| f \|_p^p+2\| g \|_p^p.        
		% Je laisse tomber cette partie parce qu'elle est -je crois- inutile pour le théorème de Weienersmith
	\end{equation}
	Il y a égalité si et seulement si \( f(t) \) et \( g(t)\) sont colinéaires pour presque tout \( t\).
\end{theorem}

\begin{proof}
	Nous supposons que \( \| f \|_p\geq \| g \|_p\) pour fixer les idées. De toutes façons, la symétrie des formules nous fait passer de ce cas à l'autre sans difficulté.

	Soit \( x\in \mathopen[ 0 , 1 \mathclose]\). Nous écrivons l'inégalité du lemme \ref{LEMooWIPYooMZqjbn} pour \( A=| f(\omega) | \) et \( B=| g(\omega) |\) :
	\begin{equation}
		\alpha(x)| f(\omega) |^p+\beta(x)| g(\omega) |^p\leq \big| f(\omega)+g(\omega) \big|^p+\big| f(\omega)-g(\omega) \big|^p.
	\end{equation}
	Nous intégrons cela par rapport à \( \omega\) sur \( \Omega\) :
	\begin{equation}
		\alpha(x)\| f \|_p^p+\beta(x)\| g \|_p^p\leq \| f+g \|^p_p+\| f-g \|_p^p.
	\end{equation}
	Et là vient l'idée qu'on se demande ce qui est passé par l'esprit du mec qui a tout combiné : nous évaluons cela pour \( x=\frac{ \| g \|_p }{ \| f \|_p }\), ce qui est permis parce que nous avons supposé \( \| f \|_p\geq \| g \|_p \). Faites le calcul, collectez les termes identiques, vous obtiendrez
	\begin{equation}
		\big( \| f \|_p+\| g \|_p \big)^p+\big( \| f \|_p-\| g \|_p \big)^p\leq \| f+g \|^p_p+\| f-g \|_p^p.
	\end{equation}
	Et vu que \( \| f \|_p\geq \| g \|_p\), nous pouvons gratuitement faire
	\begin{equation}
		\| f \|_p-\| g \|_p=\big| \| f \|_p-\| g \|_p \big|.
	\end{equation}
	Fini pour Hanner.
\end{proof}

%--------------------------------------------------------------------------------------------------------------------------- 
\subsection{Inégalités de Clarkson}
%---------------------------------------------------------------------------------------------------------------------------

\begin{lemma}[\cite{BIBooUNSIooQCLkzT}]     \label{LEMooWEODooLHeVrP}
	Si \( p\geq 2\) et si \( a,b\in \eC\), alors
	\begin{equation}
		\left| \frac{ a+b }{2} \right|^p+\left| \frac{ a-b }{2} \right|^p\leq \frac{ 1 }{2}\big( | a |^p+| b |^p \big).
	\end{equation}
\end{lemma}

\begin{proof}
	Nous prouvons l'inégalité en montant petit à petit en généralité.
	\begin{subproof}
		\spitem[Avec \( x>0\)]
		Soit \( x\geq 0\). Nous montrons dans cette partie l'inégalité
		\begin{equation}        \label{EQooDJBNooEyfNtq}
			x^p+1\leq (x+1)^{p/2}.
		\end{equation}
		Pour cela nous considérons la fonction
		\begin{equation}
			\begin{aligned}
				f\colon \mathopen[ 0 , \infty \mathclose[ & \to \eR                      \\
				t                                         & \mapsto (t^2+1)^{p/2}-t^p-1.
			\end{aligned}
		\end{equation}
		Nous avons \( f(0)=0\), mais aussi, en utilisant les règle de dérivation\footnote{Par exemple celle de la proposition \ref{PROPooKIASooGngEDh}.} nous trouvons vite
		\begin{equation}
			f'(t)=p(t^2+1)^{p/2-1}t-pt^{p-1}.
		\end{equation}
		Vu que \( (t^2+1)^{p/2-1}\geq t^{p-2}\), le signe de \( f'(t)\) est toujours strictement positif pour \( t>0\). La proposition \ref{PropGFkZMwD} fait que \( f\) est strictement croissante et que \( f(t)>0\) pour tout \( t>0\).

		\spitem[Avec \( x,y\geq 0\)]
		Soient \( x,y\geq 0\) dans \( \eR\). Nous prouvons dans cette partie que
		\begin{equation}        \label{EQooGFGMooSiDfKX}
			(x^2+y^2)^{p/2}\geq x^p+y^p.
		\end{equation}
		Il s'agit d'appliquer l'inégalité \eqref{EQooDJBNooEyfNtq} à \( x/y\) :
		\begin{equation}
			\left( \left( \frac{ x }{ y } \right)^2+1 \right)^{p/2}\geq \left( \frac{ x }{ y } \right)^p+1.
		\end{equation}
		En multipliant par \( y^p\) et en simplifiant un peu, nous trouvons le résultat \eqref{EQooGFGMooSiDfKX}.
		\spitem[Avec \( a,b\in \eC\)]
		Nous appliquons l'inégalité \eqref{EQooGFGMooSiDfKX} à \( x=| \frac{ a+b }{ 2 } |\) et \( y=| \frac{ a-b }{2} |\). Cela donne :
		\begin{subequations}
			\begin{align}
				\left| \frac{ a+b }{2} \right|^p+\left| \frac{ a-b }{2} \right|^p & \leq \left( \left| \frac{ a+b }{2} \right|^2+\left| \frac{ a-b }{2} \right|^2 \right)^{p/2} \\
				                                                                  & =\left( \frac{ 2| a |^2+2| b |^2 }{ 4 } \right)^{p/2}                                       \\
				                                                                  & \leq\frac{ 1 }{2}| a |^p+\frac{ 1 }{2}| b |^p.
			\end{align}
		\end{subequations}
		La dernière ligne est la convexité de la fonction \( t\mapsto t^{p/2}\) (lemme \ref{LEMooSXTXooZOmtKq}).
	\end{subproof}
\end{proof}

\begin{lemma}       \label{LEMooFGKXooZCHNln}
	Si \( 1<p<2\), alors l'exposant conjugué \( q\) vérifie \( q>2\).
\end{lemma}

\begin{proof}
	Nous considérons \( q\) en fonction de \( p\), sur le domaine \( 1<p<2\) :
	\begin{equation}
		q(p)=\frac{ p }{ p-1 }.
	\end{equation}
	Donc\footnote{Dire que \( q(1)=\infty\) est un abus de notations pour parler de la limite \( p\to 1\) avec \( p>1\).} \( q(1)=\infty\) et \( q(2)=2\). Nous étudions ensuite la dérivée :
	\begin{equation}
		q'(p)=-\frac{1}{ (p-1)^2 }<0.
	\end{equation}
	C'est donc une fonction strictement décroissante. Vues les valeurs aux bornes, nous voyons que \( q(p)>2\) sur tout son domaine.
\end{proof}

\begin{lemma}[\cite{BIBooVHQSooTrLCzQ}]         \label{LEMooMKIXooVOYaxI}
	Soit \( 1<p<2\). Pour \( x,y\in \eR\) nous avons
	\begin{equation}
		| x+y |^q+| x-y |^q\leq 2\big( | x |^p+| y |^p \big)^{q-1}.
	\end{equation}
\end{lemma}

\begin{proof}
	Nous considérons l'exposant conjugué \( q\) et \( p\), c'est-à-dire \( q\) tel que \( \frac{1}{ p }+\frac{1}{ q }=1\). Nous considérons la fonction
	\begin{equation}
		\begin{aligned}
			f\colon \eR\times \mathopen[ 0 , 1 \mathclose] & \to \eR                                                                          \\
			(\alpha,z)                                     & \mapsto (1+\alpha^{1-q}z)(1+\alpha z)^{q-1}+(1-\alpha^{1-q}z)(1-\alpha z)^{q-1}.
		\end{aligned}
	\end{equation}
	Cette fonction vérifie
	\begin{equation}        \label{EQooRFZQooJvdocT}
		f(1,z)=(1+z)^q+(1-z)^q,
	\end{equation}
	ainsi que
	\begin{subequations}        \label{EQooISBRooHMiPRE}
		\begin{align}
			f(z^{p-1},z) & =\big( 1+z^{(p-1)(1-q)} \big)(1+z^p)^{q-1}+\big( 1-z^{(p-1)(1-q)} \big)(1+z^p)^{q-1} \\
			             & =2(1+z^p)^{q-1}.
		\end{align}
	\end{subequations}

	Nous montrons maintenant que \( (\partial_{\alpha}f)(\alpha,z)\leq 0\) pour tout \( \alpha\in\mathopen] 0 , 1 \mathclose[\) et pour tout \( z\in \mathopen] 0 , 1 \mathclose[\). C'est du calcul :
	\begin{equation}
		\frac{ \partial f }{ \partial \alpha }(\alpha,z)=(1-q)z\big[ \alpha^{-q}(1+\alpha z)^{q-1}-(1+\alpha^{1-q}z)(1+\alpha z)^{q-1}-\alpha^{-q}(1-\alpha z)^{q-1}+(1-\alpha^{1-q}z)(1-\alpha z)^{q-2} \big].
	\end{equation}
	Maintenant nous factorisons \( (1+\alpha z)^{q-2}\) grâce à la décomposition \( (1+\alpha z)^{q-1}=(1+\alpha z)(1+\alpha z)^{q-2}\). Notez que le lemme \ref{LEMooFGKXooZCHNln} donne \( q>2\), et donc pas de problèmes avec la puissance \( q-2\). Nous continuons le calcul
	\begin{subequations}
		\begin{align}
			\frac{ \partial f }{ \partial \alpha }(\alpha,z) & =(1-q)z(1+\alpha z)^{q-2}\big[ \alpha^{-q}(1+\alpha z)-(1+\alpha^{1-q}z) \big]       \\
			\nonumber                                        & \quad+(1-q)z(1-\alpha z)^{q-2}\big[ -\alpha^{-q}(1-\alpha z)+(1-\alpha^{1-q}z) \big] \\
			                                                 & =(1-q)z(1+\alpha z)^{q-2}\big[ \alpha^{-q}+\alpha^{-q+1}z-1-\alpha^{1-q}z \big]      \\
			\nonumber                                        & \quad+(1-q)z(1-\alpha z)^{q-2}\big[ -\alpha^{-q}+\alpha^{-q+1}+1-\alpha^{1-q}z \big] \\
			                                                 & =(1-q)z(1+\alpha z)^{q-2}[\alpha^{-q}-1]+(1-q)z(1-\alpha z)^{q-2}[1-\alpha^{-q}]     \\
			                                                 & =(1-q)z(\alpha^{-q}-1)\big[ (1+\alpha z)^{q-2}-(1-\alpha z)^{q-2} \big].
		\end{align}
	\end{subequations}
	Vu que \( q>2\), la fonction \( x\mapsto x^{q-2}\) est strictement croissante sur les positifs\footnote{Proposition \ref{PROPooUOFKooYyGwIr}.}. Et vu que \( \alpha z<1\), le crochet est strictement positif. Par ailleurs, \( z>0\), \( (1-q)<0\) et \( (\alpha^{-q}-1)>0\) donc nous avons prouvé que \( (\partial_{\alpha}f)(\alpha,z)\leq 0\).

	Donc \( f\) est décroissante par rapport à \( \alpha\). Vu que \( z\in \mathopen[ 0 , 1 \mathclose]\) et que \( p>1\), nous avons \( z^{p-1}<1\) et donc \( f(1,z)\leq f(z^{p-1},z)\). Nous y substituons les valeurs calculées en \eqref{EQooRFZQooJvdocT} et \eqref{EQooISBRooHMiPRE} :
	\begin{equation}        \label{EQooFQJAooPCYtMG}
		(1+z)^q+(1-z)^q\leq 2(1+z^p)^{q-1}.
	\end{equation}

	Nous pouvons maintenant facilement prouver notre inégalité.
	\begin{subproof}
		\spitem[Pour \( 0<x<y\)]
		Si \( 0<x<y\), nous avons \( x/y\in \mathopen[ 0 , 1 \mathclose]\) et nous pouvons appliquer l'inégalité \eqref{EQooFQJAooPCYtMG} à \( z=x/y\). Nous avons successivement :
		\begin{subequations}
			\begin{align}
				\left(1+\frac{ x }{ y }\right)^q+\left( 1-\frac{ x }{ y } \right)^q   & \leq \left( 1+\frac{ x^p }{ y^p } \right)^{q-1}    \\
				\left( \frac{ x+y }{ y } \right)^q+\left( \frac{ y-x }{ y } \right)^q & \leq 2\left( \frac{ y^p+x^p }{ y^p } \right)^{q-1} \\
				y^{-q}(x+y)^q+y^{-q}(y-x)^q                                           & \leq 2y^{-p(q-1)}(y^p+x^p)                         \\
				(x+y)^q+(y-x)^q                                                       & \leq 2y^{-p(q-1)+q}(y^p+x^p).
			\end{align}
		\end{subequations}
		Nous avons utilisé la proposition \ref{PROPooDWZKooNwXsdV} sur la composition de puissances. Maintenant il suffit de remarquer que \( q=p(q-1)\) pour avoir le résultat.

		\spitem[\( x<0\) et \( y>0\)]
		En posant \( x'=-x\) nous avons
		\begin{subequations}
			\begin{align}
				| x+y |^q+| x-y |^q & =| -x'+y |^q+| -x'-y |^q           \\
				                    & =| x'-y |^q+| x'+y |^q             \\
				                    & \leq 2\big( | x' |^p+| y |^p \big) \\
				                    & \leq 2\big( | x |^p+| y |^p \big).
			\end{align}
		\end{subequations}
		\spitem[Les autres cas]
		% -------------------------------------------------------------------------------------------- 
		Je vous prie de faire la liste, et d'adapter.
	\end{subproof}
\end{proof}


Pour d'autres preuves du lemme suivant, voir \cite{BIBooHJQOooJsInho}.
\begin{lemma}[\cite{BIBooKDOKooFbAlfz}]       \label{LEMooLTROooVusGte}
	Soient \( a,b\in \eC\) ainsi que \( 1<p<2\). Nous notons \( q\) l'exposant conjugué de \( p\). Nous avons l'inégalité
	\begin{equation}
		| a+b |^q+| a-b |^q\leq 2\big( | a |^p+| b |^p \big)^{q-1}.
	\end{equation}
\end{lemma}

\begin{proof}
	Commençons doucement avec le cas \( b=0\). À gauche nous gardons \( 2| a |^q\), et pour le membre de droite nous remarquons que
	\begin{equation}
		q-1=\frac{ 1 }{ p-1 },
	\end{equation}
	de telle sorte que
	\begin{equation}
		(| a |^p)^{q-1}=| a |^{p/(p-1)}=|a|^q.
	\end{equation}

	Gardez en tête que, par le lemme \ref{LEMooFGKXooZCHNln}, \( q>2\); ce sera utile.

	Nous commençons le vrai combat. Vu que \( | a-b |=| b-a |\) nous pouvons supposer \( | a |\geq | b |\) pour fixer les idées. En utilisant le lemme \ref{LEMooOQKNooGZlJHf}, il existe \( t_0\in\mathopen[ 0 , \frac{ \pi }{2} \mathclose]\) tel que
	\begin{equation}
		\begin{aligned}[]
			| a+b |^2 & =| a |^2+| b |^2+2| a | |b |\cos(t_0) \\
			| a-b |^2 & =| a |^2+| b |^2-2| a | |b |\cos(t_0) \\
		\end{aligned}
	\end{equation}

	Nous considérons la fonction
	\begin{equation}
		\begin{aligned}
			f\colon \mathopen[ 0 , 2\pi \mathclose] & \to \eR                                                                                                              \\
			t                                       & \mapsto \Big( | a |^2+| b |^2+2| a | |b |\cos(t) \Big)^{q/2}+  \Big( | a |^2+| b |^2-2| a | |b |\cos(t) \Big)^{q/2}.
		\end{aligned}
	\end{equation}
	Elle vérifie
	\begin{equation}
		f(t_0)=(| a+b |^2)^{q/2}+(| a-b |^2)^{q/2}.
	\end{equation}
	Vu que \( | a+b |\) et \( | a-b |\) sont positifs, nous pouvons «simplifier» le carré et la racine carré, de telle sorte que
	\begin{equation}
		f(t_0)=| a+b |^q+| a-b |^q.
	\end{equation}
	Nous cherchons un maximum pour \( f\) sur \( \mathopen[ 0 , \frac{ \pi }{2} \mathclose]\). Pour cela, nous prenons d'abord la dérivée :
	\begin{equation}
		f'(t)=-q| a | |b |\sin(t)\Big[   \big( | a |^2+| b |^2+2| a | |b |\cos(t) \big)^{q/2-1}-\big( | a |^2+| b |^2-2| a | |b |\cos(t) \big)^{q/2-1}    \Big].
	\end{equation}
	Notez que \( q>2\), donc la fonction \( x\mapsto x^{q/2-1}\) est croissante.

	Pour \( t\in\mathopen[ 0 , \pi/2 \mathclose]\), nous avons \( \cos(t)\geq -\cos(t)\) ainsi que \( \sin(t)\geq 0\). Donc \( f'(t)\leq 0\). De la même manière, nous avons \( f'(t)\geq 0\) pour \( t\in\mathopen[ \pi/2 , \pi \mathclose]\).

	Par le lien entre dérivée et croissance (proposition \ref{PROPooKZPZooWjIsWg}), nous savons que le maximum de \( f\) sur \( \mathopen[ 0 , \pi \mathclose]\) est atteint en \( 0\) ou en \( \pi\).

	Nous avons, en utilisant la supposition \( | a |\geq | b |\):
	\begin{subequations}
		\begin{align}
			f(0)=f(\pi) & =\big( | a |^2+| b |^2+2| a | |b | \big)^{q/2}+\big( | a |^2+| b |^2-2| a | |b | \big)^{q/2} \\
			            & =\big( \big| | a |+| b |\big|^2 \big)^{q/2}+\Big( \big| | a |-| b | \big|^2 \Big)            \\
			            & =\big| | a |+| b | \big|^q+\big| | a |-| b | \big|^q.
		\end{align}
	\end{subequations}
	En particulier \( f(t_0)\leq f(0)\) et donc
	\begin{equation}
		| a+b |^q+| a-b |^q\leq \big| | a |+| b | \big|^q+\big| | a |-| b | \big|^q\leq 2\big( | a |^p+| b |^p \big)^{q-1}.
	\end{equation}
	La dernière inégalité est le lemme \ref{LEMooMKIXooVOYaxI} appliqué aux réels \( | a |\) et \( | b |\).
\end{proof}

\begin{proposition}[Inégalité de Clarkson\cite{BIBooVHQSooTrLCzQ}]      \label{PROPooJDOQooWsGlkr}
	Soient \( f,g\in L^p(\Omega,\tribA,\mu)\).
	\begin{enumerate}
		\item
		      Si \( p\geq 2\), alors
		      \begin{equation}        \label{EQooBWDJooGXzdxz}
			      \| \frac{ f+g }{2} \|_p^p+\| \frac{ f-g }{2} \|_p^p\leq \frac{ 1 }{2}\Big( \| f \|_p^p+\| g \|_p^p \Big).
		      \end{equation}
		\item
		      Si \( 1<p<2\) et si \( q\) est l'exposant conjugué de \( p\), alors
		      \begin{equation}        \label{EQooXMWBooYrvaoV}
			      \| f+g \|_p^q+\| f-g \|_p^q\leq 2\Big( \| f \|_p^p +\| g \|_p^p \Big)^{q-1},
		      \end{equation}
		      ou
		      \begin{equation}        \label{EQooZCWDooBnaMom}
			      \| \frac{ f+g }{2} \|_p^q+\| \frac{ f-g }{2} \|_p^q\leq 2^{1-q}\big( \| f \|_p^p+\| g \|_p^p \big)^{q-1}.
		      \end{equation}
	\end{enumerate}
\end{proposition}

\begin{proof}
	En deux parties.
	\begin{subproof}
		\spitem[Pour \( p\geq 2\)]
		Soient \( f,g\in L^p(\Omega,\tribA,\mu)\); ce sont des fonctions à valeurs dans \( \eC\). Pour chaque \( \omega\in \Omega\) nous considérons les nombres complexes \( f(\omega)\) et \( g(\omega)\); nous pouvons écrire l'inégalité du lemme \ref{LEMooWEODooLHeVrP} :
		\begin{equation}        \label{EQooIGNKooKFUpKO}
			\left| \frac{ f(\omega)+g(\omega) }{2} \right|^p+\left| \frac{ f(\omega)-g(\omega) }{2} \right|^p\leq \frac{ 1 }{2}\big( | f(\omega) |^p+| g(\omega) |^p \big).
		\end{equation}
		Nous avons les substitutions évidentes \( f(\omega)+g(\omega)=(f+g)(\omega)\) et \( f(\omega)-g(\omega)=(f-g)(\omega)\). En intégrant alors \eqref{EQooIGNKooKFUpKO} sur \( \Omega\) nous trouvons l'inégalité demandée.
		\spitem[Pour \( 1<p<2\)]
		Il s'agit de faire la même chose, en utilisant l'inégalité de Clarkson du lemme \ref{LEMooLTROooVusGte}.

		Pour obtenir \eqref{EQooZCWDooBnaMom}, il s'agit simplement de multiplier et diviser le member de gauche de \eqref{EQooXMWBooYrvaoV} par \( 2^q\).
	\end{subproof}
\end{proof}

%--------------------------------------------------------------------------------------------------------------------------- 
\subsection{Uniforme convexité des espaces de Lebesgue}
%---------------------------------------------------------------------------------------------------------------------------

\begin{proposition}[\cite{BIBooRISHooBcPPKQ}]     \label{PROPooFNLJooDlyIKV}
	Si \( 1<p<\infty\), l'espace \( L^p(\Omega,\tribA, \mu)\) est uniformément convexe\footnote{Définition \ref{DEFooOPQBooBhufew}.}.
\end{proposition}

\begin{proof}
	En deux parties.

	\begin{subproof}
		\spitem[\( 1<p\leq 2\)]
		Nous montrons que la fonction \( \delta(\epsilon)=2^{-q}\epsilon^q\) fonctionne.

		Soient \( f,g\in L^p\) telles que \( \| f \|_p\leq \| g \|_p\leq 1\) et \( \| f-g \|_p\geq \epsilon\). Nous commençons par écrire l'inégalité de Clarkson \eqref{EQooXMWBooYrvaoV} :
		\begin{equation}        \label{EQooOWVEooGGfCpy}
			\| \frac{ f+g }{2} \|_p^q+\| \frac{ f-g }{2} \|_p^q\leq 2^{1-q}\big( \| f \|_p^p+\| g \|_p^p \big)^{q-1}.
		\end{equation}
		Par hypothèse, \( \| f \|_p\) et \( \| g \|_p\) sont plus petites que \( 1\). Vu que \( p>1\), nous avons
		\begin{equation}
			\| f \|_p^p+\| g \|_p^p\leq 1+1=2.
		\end{equation}
		En remplaçant dans le membre de droite de \eqref{EQooOWVEooGGfCpy} nous avons
		\begin{equation}
			\| \frac{ f+g }{2} \|_p^q+\| \frac{ f-g }{2} \|_p^q\leq 2^{1-q}2^{q-1}=1,
		\end{equation}
		et donc
		\begin{equation}        \label{EQooKARVooDrOuJI}
			\| \frac{ f+g }{2} \|_p^q\leq 1-\| \frac{ f-g }{2} \|_p^q.
		\end{equation}

		Par ailleurs nous avons supposé \( \| f-g \|_p\geq \epsilon\). Donc aussi\quext{Ici j'ai un coefficient un peu différent que celui de \cite{BIBooRISHooBcPPKQ}. Écrivez-moi pour confirmer ou infirmer mes calculs.}
		\begin{equation}        \label{EQooCGDDooWtDokf}
			\| \frac{ f-g }{2} \|_p^q\geq 2^{-q}\epsilon^q.
		\end{equation}

		Et par un autre ailleurs,
		\begin{equation}        \label{EQooOFWYooLVrNDc}
			\| \frac{ f+g }{2} \|_p=\frac{ 1 }{2}\| f+g \|_p\leq \frac{ 1 }{2}\big( \| f \|_p+\| g \|_p \big)\leq 1.
		\end{equation}
		Vu que nous avons \( q\geq 2\), cela donne aussi
		\begin{equation}        \label{EQooGMPRooGiLSss}
			\| \frac{ f+g }{2} \|_p\leq \| \frac{ f+g }{2} \|^q.
		\end{equation}

		Avec les inégalités \eqref{EQooCGDDooWtDokf} et \ref{EQooGMPRooGiLSss} nous finissons l'inégalité \eqref{EQooKARVooDrOuJI} :
		\begin{equation}
			\| \frac{ f+g }{2} \|_p\leq \| \frac{ f+g }{2} \|_p^q\leq 1-2^{-q}\epsilon^q\leq \| g \|_p-\delta(\epsilon).
		\end{equation}
		Okay, c'est bon.

		\spitem[\( 2\leq p<\infty\)]
		Il s'agit de faire la même chose en partant de Clarkson \eqref{EQooBWDJooGXzdxz}. Le résultat est que la fonction \( \delta(\epsilon)=(\epsilon/2)^p\), ça fonctionne.
	\end{subproof}
\end{proof}

%--------------------------------------------------------------------------------------------------------------------------- 
\subsection{Théorème de la projection normale}
%---------------------------------------------------------------------------------------------------------------------------

\begin{proposition}     \label{PROPooTZMRooCvQtGg}
	Si \( 1<p<\infty\), et si \( V\) est un sous-espace vectoriel fermé de \( L^p(\Omega,\tribA, \mu)\), alors la projection normale\footnote{Définition \ref{DEFooMYYLooJyACPL}.} de \( a\in L^p\) sur \( V\) existe et est unique.
\end{proposition}

\begin{proof}
	La proposition \ref{PROPooFNLJooDlyIKV} nous indique que l'espace \( L^p(\Omega,\tribA, \mu)\) est uniformément convexe. Or le théorème \ref{THOooOOVVooMhzHqd} nous indique que les espaces uniformément convexes vérifient la présente proposition.
\end{proof}

Nous pouvons donner une preuve directe, sans passer par l'uniforme convexité, dans les cas \( p\geq 2\).
\begin{theorem}[Théorème de la projection normale\cite{BIBooRYTOooYjaNkX}] \label{THOooRJFUooQivDKm}
	Nous considérons \( p\geq 2\). Soit un sous-espace vectoriel fermé \( W\subset L^p(\Omega,\tribA,\mu)\) et \( u_0\in L^p\). Nous notons
	\begin{equation}
		d(u_0,W)=\inf_{w\in W}d(u_0,W).
	\end{equation}
	Alors il existe \( w_0\in W\) tel que \( \| u_0-w_0 \|=d(u_0,W)\).
\end{theorem}

\begin{proof}
	Nous allons séparer trois cas : \( p=2\) et \( p>2\).
	\begin{subproof}
		\spitem[\( p=2\)]
		Pour \( p=2\), nous savons que \( L^2\) est un espace de Hilbert\footnote{Lemme \ref{LemIVWooZyWodb}.}, et nous avons déjà le théorème de la projection \ref{ThoProjOrthuzcYkz}.
		\spitem[\( p>2\)]
		Pour chaque \( x\in \Omega\) nous avons \( f(x), g(x)\in \eC\) et donc l'identité du parallélogramme\footnote{Théorème \ref{ThoAYfEHG} en remarquant que \( (z_1,z_2)\mapsto z_1\bar z_2\) est un produit scalaire hermitien sur \( \eC\).} :
		\begin{equation}        \label{EQooUBFEooDUjLnb}
			\big| f(x)-g(x) \big|^2+\big| f(x)+g(x) \big|^2=2| f(x) |^2+2| g(x) |^2.
		\end{equation}
		Vu que \( p>2\), la fonction \( s\colon x\mapsto  x^{p/2}\) est convexe (lemme \ref{LEMooSXTXooZOmtKq}). Calcul :
		\begin{subequations}
			\begin{align}
				| f(x)-g(x) |^p+| f(x)+g(x) |^p & =\big( | f(x)-g(x) |^2 \big)^{p/2}+\big( | f(x)+g(x) |^2 \big)^{p/2}                    \\
				                                & =s\big( | \ldots |^2 \big)+s\big( | \ldots |^2 \big)                                    \\
				                                & \leq \big( | f(x)-g(x) |^2+| f(x)+g(x) |^2 \big)^{p/2}     \label{SUBEQooRHAEooHkYNLH}  \\
				                                & =\big( 2| f(x) |^2+2| g(x) |^2 \big)^{p/2}                 \label{SUBEQooQFSLooJkoeqN}  \\
				                                & =2^{p/2}\big( | f(x) |^2+| g(x) |^2 \big)^{p/2}                                         \\
				                                & \leq  2^{p/2}2^{p/2-1}\big( | f(x) |^p+| g(x) |^p \big)     \label{SUBEQooQSUHooXKaWwO} \\
				                                & =2^{p-1}\big( | f(x) |^p+| g(x) |^p \big)
			\end{align}
		\end{subequations}
		Justifications :
		\begin{itemize}
			\item Pour \eqref{SUBEQooRHAEooHkYNLH} : la convexité de \( s\).
			\item Pour \eqref{SUBEQooQFSLooJkoeqN} : la relation \eqref{EQooUBFEooDUjLnb}.
			\item Pour \eqref{SUBEQooQSUHooXKaWwO} : la seconde inégalité du lemme \ref{SUBEQooQSUHooXKaWwO}.
		\end{itemize}
		Nous isolons \( | f(x)-g(x) |^p\) :
		\begin{subequations}
			\begin{align}
				| f(x)-g(x) |^p & \leq 2^{p-1}\big( | f(x) |^p+| g(x) |^p \big)-| f(x)+g(x) |^p                                       \\
				                & =2^p\left( \frac{ | f(x) |^p+| g(x) |^p }{2}-\left| \frac{ | f(x) |+| g(x) | }{2} \right|^p \right)
			\end{align}
		\end{subequations}
		Cette inégalité étant valable pour tout \( x\), nous pouvons intégrer sur \( \Omega\) et découper l'intégrale en petits morceaux :
		\begin{equation}        \label{EQooVNHSooPXjFNC}
			\| f-g \|^p_p\leq 2^p\left( \frac{ \| f \|_p^p+\| g \|_p^p }{2}- \| \frac{ f+g }{2} \|_p^p \right).
		\end{equation}
		Voilà une bonne chose de prouvée. Nous pouvons maintenant passer au vif du sujet.

		Soit une suite \( w_j\) dans \( W\) telle que \( \| u_0-w_j \|\to d(u_0,W)\). Trois choses à savoir sur cette suite :
		\begin{enumerate}
			\item
			      Une telle suite existe parce que \( d(u_0,W)\) est défini comme un infimum.
			\item
			      Rien ne garantit qu'elle converge.
			\item
			      Même si elle convergeait, rien ne garantirait que la limite soit encore dans \( W\).
		\end{enumerate}
		Le troisième point est facile à régler : vu que \( W\) est fermé par hypothèse, une suite convergente contenue dans \( W\) a sa limite dans \( W\). Nous allons régler la convergence de \( w_j\) en prouvant qu'elle est de Cauchy.

		Remarquons que \( W\) est vectoriel, donc \( (w_j+w_k)/2\) est dans \( W\) pour tout \( j\) et \( k\); donc
		\begin{equation}
			\| \frac{ w_j+w_k }{2}-u_0 \|\geq d(u_0,W).
		\end{equation}
		En tenant compte de cela, nous écrivons l'inégalité \eqref{EQooVNHSooPXjFNC} avec \( f=w_j-u_0\) et \( g=w_k-u_0\) :
		\begin{equation}
			\| f-g \|_p^p=\| w_j-w_k \|_p^p\leq 2^p\left( \frac{ \| w_j-u_0 \|^p+\| w_k-u_0 \|^p }{2}-d(u_0,W) \right).
		\end{equation}
		Soit \( \epsilon>0\) et \( 0<\epsilon_1,\epsilon_2<\epsilon\) tels que \( \epsilon_1+\epsilon_2<\epsilon\). Il existe un \( N\) tel que si \( j,k>N\) alors \( \| w_j-u_0 \|^p\leq d(u_0,W)^p+\epsilon_1\) et \( \| w_k-u_0 \|^p\leq d(u_0,W)^p+\epsilon_2\). Pour de telles valeurs de \( j\) et \( k\), nous avons
		\begin{equation}
			\| w_j-w_k \|_p\leq 2\left( \frac{ \epsilon_1+\epsilon_2 }{2} \right)<2\epsilon^{1/p}.
		\end{equation}
		Donc la suite \( (w_j)\) est de Cauchy.

		L'espace \( L^p\) étant complet par le théorème \ref{ThoUYBDWQX}, nous en déduisons que \( (w_j)\) converge dans \( L^p\). Mais comme \( W\) est fermé, nous avons \( w_j\stackrel{L^p}{\longrightarrow}w\in W\).

		En termes de normes, nous avons
		\begin{equation}
			\| w-u_0 \|=\lim_j\| w_j-u_0 \|=d(W,u_0).
		\end{equation}
	\end{subproof}
\end{proof}

%+++++++++++++++++++++++++++++++++++++++++++++++++++++++++++++++++++++++++++++++++++++++++++++++++++++++++++++++++++++++++++
\section{Théorèmes de Hahn-Banach}
%+++++++++++++++++++++++++++++++++++++++++++++++++++++++++++++++++++++++++++++++++++++++++++++++++++++++++++++++++++++++++++

%--------------------------------------------------------------------------------------------------------------------------- 
\subsection{Applications \( \eR\)-linéaires et \( \eC\)-linéaires}
%---------------------------------------------------------------------------------------------------------------------------

\begin{lemma}[\cite{BIBooFMTKooRcLRXg,BIBooZBJFooCSKKgD}]        \label{LEMooBZHIooSQJSnM}
	Soit un espace vectoriel \( X\) sur \( \eC\). Nous considérons\footnote{Voir la définition \ref{DEFooULVAooXJuRmr} pour les notations.} l'application
	\begin{equation}        \label{EQooLYYGooJfKIfu}
		\begin{aligned}
			\psi\colon \aL_{\eR}(X,\eR) & \to \aL_{\eC}(X,\eC)                      \\
			f                           & \mapsto \big[ x\mapsto f(x)-if(ix) \big].
		\end{aligned}
	\end{equation}
	\begin{enumerate}
		\item
		      Cette application est bien définie.
		\item
		      Elle est une bijection.
		\item		\label{ITEMooNVKBooIGhzWM}
		      Elle vérifie \( \| \psi(f) \|_{\aL_{\eC}(X,\eC)}=\| f \|_{\aL_{\eR}(X,\eR)}\).
		\item		\label{ITEMooTFWOooIIhcnZ}
		      Si \( g\in \aL_{\eC}(X,\eC)\) se décompose en \( g(x)=u(x)+iv(x)\) où \( u\) et \( v\) sont à valeurs réelles, alors \( g=\psi(u)\).
	\end{enumerate}
\end{lemma}

\begin{proof}
	En plusieurs parties.
	\begin{subproof}
		\spitem[Bien définie]
		%----------------------------------------------------------------------------------------------
		Nous devons avant tout prouver que \eqref{EQooLYYGooJfKIfu} a un sens : si \( f\in \aL_{\eR}(X,\eR)\), alors nous devons prouver que \( \psi(f)\in\aL_\eC(X,\eC)\), c'est-à-dire que \( \psi(f)\) est \( \eC\)-linéaire. Soient donc \( x,y\in X\) et \( \alpha,\beta\in \eR\) de telle sorte que \( \alpha+\beta i\) soit un élément générique de \( \eC\). Prouver que
		\begin{equation}
			\psi(f)(x+y)=\psi(f)(x)+\psi(f)(y)
		\end{equation}
		est un simple calcul. Ensuite
		\begin{subequations}
			\begin{align}
				\psi(f)\big( (\alpha+i\beta)x \big) & =f\big( (\alpha+i\beta)x \big)-if\big( i(\alpha+i\beta)x \big) \\
				                                    & =\alpha f(x)+\beta f(ix)-i\alpha f(ix)-i(-\beta)f(x)           \\
				                                    & =(\alpha+i\beta)\big( f(x)-if(ix) \big)                        \\
				                                    & =(\alpha+i\beta)\psi(f)(x).
			\end{align}
		\end{subequations}

		\spitem[\( \psi\) est injective]
		%----------------------------------------------------------------------------------------------
		Soient \( f,g\in \aL_{\eR}(X,\eR)\) telles que \( \psi(f)=\psi(g)\). Nous avons l'égalité
		\begin{equation}        \label{EQooOFWHooIJutnV}
			f(x)-if(ix)=g(x)-ig(ix),
		\end{equation}
		qui est une égalité dans \( \eC\). En sachant que les nombres \( f(x)\), \( f(ix)\), \( g(x)\) et \( g(ix)\) sont des réels, nous séparons les parties réelles et imaginaires dans \eqref{EQooOFWHooIJutnV} et nous trouvons \( f(x)=g(x)\) et \( f(ix)=g(ix)\). Chacune de ces deux égalités nous assurent que \( f=g\).

		\spitem[\( \psi\) est surjective]
		%----------------------------------------------------------------------------------------------
		Soit \( g\in \aL_{\eC}(X,\eC)\). Nous séparons ses parties réelles et imaginaires :
		\begin{equation}
			g(x)=g_1(x)+ig_2(x).
		\end{equation}
		Avec \( g_1,g_2\in \aL_{\eR}(X,\eR)\). Nous allons prouver que \( g=\psi(g_1)\).

		Ensuite, utilisant la \( \eC\)-linéarité de \( g\) pour calculer \( g(ix)\) de deux façon différentes. D'une part
		\begin{equation}        \label{EQooRMJAooQlTJLY}
			g(ix)=g_1(ix)+ig_2(ix),
		\end{equation}
		et d'autre part
		\begin{equation}        \label{EQooCJQHooFVJXGa}
			g(ix)=i\big( g_1(x)+ig_2(x) \big).
		\end{equation}
		En égalisant les parties réelles et imaginaires de \eqref{EQooRMJAooQlTJLY} et de \eqref{EQooCJQHooFVJXGa},
		\begin{subequations}
			\begin{align}
				g_1(x) & =g_2(ix)   \\
				g_2(x) & =-g_1(ix),
			\end{align}
		\end{subequations}
		et en particulier \( g_1(ix)=g_2(-x)\).

		Nous pouvons maintenant calculer
		\begin{subequations}
			\begin{align}
				\psi(g_1)(x) & =g_1(x)-ig_1(ix) \\
				             & =g_1(x)-ig_2(-x) \\
				             & =g_1(x)+ig_2(x)  \\
				             & =g(x).
			\end{align}
		\end{subequations}

		\spitem[Norme]
		%-----------------------------------------------------------
		Nous prouvons maintenant le point \ref{ITEMooNVKBooIGhzWM} qui prétend que \( \| \psi(f) \|_{\aL_{\eC}(X,\eC)}=\| f \|_{\aL_{\eR}(X,\eR)}\). Nous allons montrer les inégalités dans les deux sens.

		D'abord nous prouvons que \( \| f \|\leq \| \psi(f) \|\). Pour cela, nous considérons \( x\in X\), et nous écrivons
		\begin{equation}
			| f(x) |\leq | f(x)-if(ix) |=| \psi(f)x |
		\end{equation}
		dont la première inégalité est vraie parce que \( f(x)\) et \( f(ix)\) sont réels. Avec cette inégalité nous avons tout de suite
		\begin{equation}
			\sup_{\| x \|=1}| f(x) |\leq \sup_{\| x \|=1}| \psi(f)x |.
		\end{equation}

		Nous passons maintenant à l'autre sens. Soit \( x\in X\). Nous commençons par trouver \( \alpha\in S^1\) (c'est-à-dire \( | \alpha |=1\)) tel que \( f(\alpha ix)=0\). Si \( f(ix)=0\), alors \( \alpha=1\) fait l'affaire. Sinon nous considérons l'application
		\begin{equation}
			\begin{aligned}
				s\colon \mathopen[ 0,2\pi\mathclose[ & \to \eR                            \\
				t                                    & \mapsto f\big( \varphi(t)ix \big).
			\end{aligned}
		\end{equation}
		où \( \varphi\) est le difféomorphisme usuel\footnote{Par exemple de la proposition \ref{PROPooXELTooYKjDav}.}. Pour \( t=0\) nous avons \( s(0)=f(ix)\) et, utilisant la \( \eR\)-linéarité de \( f\), pour \( t=\pi\) nous avons \( s(\pi)=-f(ix)\). Le théorème des valeurs intermédiaires \ref{ThoValInter} appliqué à \( s\) donne un \( t\in \mathopen] 0,2\pi\mathclose[\) tel que \( s(t_0)=0\). En posant \( \alpha=\varphi(t_0)\) nous avons le résultat.

		Armé de ce \( \alpha\), nous montrons que \( \| \psi(f) \|\leq \| f \|\). Nous avons
		\begin{subequations}
			\begin{align}
				| \psi(f)x | & =| \alpha || \psi(f)x |                                                    \\
				             & =| \psi(f)(\alpha x) |         & \text{\( \psi(f)\) est \( \eC\)-linéaire} \\
				             & =| f(\alpha x)-if(i\alpha x) |                                             \\
				             & =| f(\alpha x) |.
			\end{align}
		\end{subequations}
		Nous avons prouvé que pour tout \( x\in X\) tel que \( \| x \|=1\), il existe \( y\in X\) tel que \( \| y \|=1\) et \( | \psi(f)x |=| f(y) |\). En l'occurrence, \( y=\alpha x\). Notez que \( \alpha\) est une fonction de \( x\). Nous avons donc
		\begin{equation}
			\sup_{\| y \|=1}| f(y) |\geq \sup_{\| x \|=1}| \psi(f)x |,
		\end{equation}
		autrement dit \( \| f \|\geq \| \psi(f) \|\).

		\spitem[Décomposition]
		%-----------------------------------------------------------
		Nous prouvons le point \ref{ITEMooTFWOooIIhcnZ}. Supposons \( g(x)=u(x)+iv(x)\). Nous avons \( g(ix)=ig(x)\) par \( \eC\)-linéarité de \( g\). Donc
		\begin{equation}
			u(ix)+iv(ix)=i\big( u(x)+iv(x) \big),
		\end{equation}
		c'est-à-dire \( u(ix)+iv(ix)=iu(x)-v(x)\). En égalisant les partie imaginaires, \( v(x)=-u(ix)\). En remettant dans l'expression de \( g\),
		\begin{equation}
			g(x)=u(x)+iv(x)=u(x)+i\big( -u(ix) \big)=u(x)-iu(ix)=\psi(u)x,
		\end{equation}
		ce qu'il fallait démontrer.
	\end{subproof}
\end{proof}

\begin{lemma}[\cite{BIBooFMTKooRcLRXg}]     \label{LEMooUFMFooEXecXE}
	Soient un espace vectoriel \( X\) sur \( \eC\) ainsi qu'une seminorme\footnote{Définition \ref{DefPNXlwmi}.} \( p\). Soit \( f\in\aL_{\eR}(X,\eR)\). Nous avons
	\begin{equation}
		| \psi(f)(x) |\leq p(x)\quad\forall x\in X
	\end{equation}
	si et seulement si
	\begin{equation}
		| f(x) |\leq p(x)\quad\forall x\in X
	\end{equation}
\end{lemma}

\begin{proof}
	En deux parties.
	\begin{subproof}
		\spitem[\( \Rightarrow\)]
		Nous supposons que \( f\in \aL_{\eR}(X,\eR)\) vérifie \( | \psi(f)(x) |\leq p(x)\) pour tout \( x\). L'observation à faire est \( f(x)=\real\big( \psi(f)(x) \big)\). Nous avons alors\quext{Dans \cite{BIBooFMTKooRcLRXg}, l'auteur sépare le calcul en deux parties : une majoration pour \( f(x)\) et une pour \( -f(x)\). Je ne vois pas où est le mal à la faire d'un seul coup.}
		\begin{equation}
			| f(x) |=| \real\big( \psi(f)(x) \big) |\leq | \psi(f)(x) |\leq p(x).
		\end{equation}
		\spitem[\( \Leftarrow\)]
		Nous supposons que \( | f(x) |\leq p(x)\) pour tout \( x\). Soit \( x\in X\). Si \( \psi(f)(x)=0\), alors nous avons bien \( | \psi(f)(x) |\leq p(x)\) parce qu'une seminorme est toujours positive. Nous supposons donc que \( \psi(f)(x)\neq 0\).

		La décomposition polaire (proposition \ref{PROPooRFMKooURhAQJ}) du nombre complexe \( \psi(f)x\) donne un \( \theta\in \eR\) tel que \( \psi(f)x= e^{i\theta}| \psi(f)x |\). Donc
		\begin{equation}        \label{EQooFYVFooOvOTHz}
			| \psi(f)(x) |= e^{-i\theta}\psi(f)(x)=\psi(f)( e^{-i\theta}x).
		\end{equation}
		Ces égalités montrent entre autres que \( \psi(f)( e^{-i\theta}x)\in \eR^+\); et en particulier, il est égal à sa partie réelle. Vu la définition \eqref{EQooLYYGooJfKIfu}, ça nous dit que
		\begin{equation}
			\psi(f)( e^{-i\theta}x)=f( e^{-i\theta}x).
		\end{equation}
		Nous pouvons alors continuer les égalités \eqref{EQooFYVFooOvOTHz} en mettant des normes partout :
		\begin{equation}
			| \psi(f)(x) |=| \psi(f)( e^{-i\theta}x) |=| f( e^{-i\theta}x) |\leq p( e^{-i\theta}x)=|  e^{-i\theta} |p(x)=p(x).
		\end{equation}
	\end{subproof}
\end{proof}

%--------------------------------------------------------------------------------------------------------------------------- 
\subsection{Hahn-Banach, théorème d'extension dominée}
%---------------------------------------------------------------------------------------------------------------------------

\begin{theorem}[Hahn-Banach, extension dominée cas réel\cite{brezis,TQSWRiz}]		\label{THOooXALCooFrkvDo}
	Soit \( E\), un espace vectoriel réel et une application \( p\colon E\to \eR\) satisfaisant\footnote{Ce n'est pas tout à fait une seminorme (définition \ref{DefPNXlwmi}) à cause de la valeur absolue. Le théorème \ref{THOooVQLJooWuBMoZ} parlera pour les vraies seminormes, et pour les espaces sur \( \eC\).}
	\begin{enumerate}
		\item
		      \( p(\lambda x)=\lambda p(x)\) pour tout \( x\in E\) et pour tout \( \lambda>0\),
		\item
		      \( p(x+y)\leq p(x)+p(y)\) pour tout \( x,y\in E\).
	\end{enumerate}
	Soit de plus \( M\subset E\) un sous-espace vectoriel muni d'une application \( g\colon M\to \eR\) vérifiant \( g(m)\leq p(m)\) pour tout \( m\in M\). Alors il existe \( f\in\aL(E,\eR)\) telle que
	\begin{enumerate}
		\item
		      L'application \( f\) étend \( g\): \( f(m)=g(m)\) pour tout \( m\in M\),
		\item
		      L'application \( f\) reste dominée par \( p\) : \( f(x)\leq p(x)\) pour tout \( x\in E\).
	\end{enumerate}
\end{theorem}
\index{théorème!Hahn-Banach}

\begin{proof}
	Si \( h\) est une application linéaire définie sur un sous-espace de \( E\), nous notons \( D_h\) ledit sous-espace.

	\begin{subproof}
		\spitem[Un ensemble inductif]

		Nous considérons \( P\), l'ensemble des fonctions linéaires suivant
		\begin{equation}
			P=\Big\{  h\colon D_h\to \eR\tq
			\begin{cases}
				M\subset D_h                     \\
				h(m)=g(m)     & \forall m\in M   \\
				h(x)\leq p(x) & \forall x\in D_h
			\end{cases}
			\Big\}
		\end{equation}
		Cet ensemble est non vide parce que \( g\) est dedans. Nous le munissons de la relation d'ordre \( h_1\leq h_2\) si et seulement si \( D_{h_1}\subset D_{h_2}\) et \( h_2\) prolonge \( h_1\). Nous montrons à présent que \( P\) est un ensemble inductif. Soit un sous-ensemble totalement ordonné \( Q\subset P\); nous définissons une fonction \( h\) de la façon suivante. D'abord \( D_h=\sup_{l\in Q}D_l\) et ensuite
		\begin{equation}
			\begin{aligned}
				h\colon D_h & \to \eR                            \\
				x           & \mapsto l(x) & \text{si } x\in D_l
			\end{aligned}
		\end{equation}
		Cela est bien défini parce que si \( x\in D_l\cap D_{l'}\) alors, vu que \( Q\) est totalement ordonné (i.e. \( l\leq l'\) ou \( l'\leq l\)), on a obligatoirement \( D_l\subset D_{l'}\) et \( l'\) qui prolonge \( l\) (ou le contraire). Donc \( h\) est un majorant de \( Q\) dans \( P\) parce que \( h\geq l\) pour tout \( l\in Q\). Cela montre que \( P\) est inductif (définition~\ref{DefGHDfyyz}). Le lemme de Zorn~\ref{LemUEGjJBc} nous dit alors que \( P\) possède un élément maximal \( f\) qui va être la réponse à notre théorème.

		\spitem[Le support de \( f\)]

		La fonction \( f\) est dans \( P\); donc \( f(x)\leq p(x)\) pour tout \( x\in D_h\) et \( f(m)=g(m)\) pour tout \( m\in M\). Pour terminer nous devons montrer que \( D_f=E\). Supposons donc que \( D_f\neq E\) et prenons \( x_0\notin D_f\). Nous allons contredire la maximalité de \( f\) en considérant la fonction \( h\) donnée par \( D_h=D_f+\eR x_0 \) et
		\begin{equation}
			h(x+tx_0)=f(x)+t\alpha
		\end{equation}
		où \( \alpha\) est une constante que nous allons fixer plus tard.

		Nous commençons par prouver que \( h\) est dans \( P\). Nous devons prouver que
		\begin{equation}    \label{EqOIXrlFe}
			h(x+tx_0)=f(x)+t\alpha\leq p(x+tx_0)
		\end{equation}
		Pour cela nous allons commencer par fixer \( \alpha\) pour avoir les relations suivantes :
		\begin{subequations}    \label{EqMDNkcQk}
			\begin{numcases}{}
				f(x)+\alpha\leq p(x+x_0)    \label{EqDYmRWEY}\\
				f(x)-\alpha\leq p(x-x_0)
			\end{numcases}
		\end{subequations}
		pour tout \( x\in D_f\). Ces relations sont équivalentes à demander \( \alpha \) tel que
		\begin{subequations}
			\begin{numcases}{}
				\alpha\leq p(x+x_0)-f(x)\\
				\alpha\geq f(x)-p(x-x_0)
			\end{numcases}
		\end{subequations}
		Nous nous demandons donc si il existe un \( \alpha\) qui satisfasse
		\begin{equation}
			\sup_{y\in D_f}\big( f(y)-p(y-x_0) \big)\leq \alpha\leq \inf_{z\in D_f}\big( p(z+x_0)-f(z) \big).
		\end{equation}
		Ou encore nous devons prouver que pour tout \( y,z\in D_f\),
		\begin{equation}
			p(z+x_0)-f(x)\geq f(y)-p(y-x_0)\geq 0.
		\end{equation}
		Par les propriétés de \( p\) et de \( f\),
		\begin{equation}
			p(z+x_0)+p(y-x_0)-f(z)-f(y)\geq p(z+y)-f(z+y)\geq 0.
		\end{equation}
		La dernière inégalité est le fait que \( f\in P\). Un choix de \( \alpha\) donnant les inéquations \eqref{EqMDNkcQk} est donc possible.

		À partir des inéquations \eqref{EqMDNkcQk} nous obtenons la relation \eqref{EqOIXrlFe} de la façon suivante. Si \( t>0\) nous multiplions l'équation \eqref{EqDYmRWEY} par \( t\) :
		\begin{equation}
			tf(x)+t\alpha\leq tp(x+x_0).
		\end{equation}
		Et nous écrivons cette relation avec \( x/t\) au lieu de \( x \) en tenant compte de la linéarité de \( f\) :
		\begin{equation}
			f(x)+t\alpha\leq  tp\big( \frac{ x }{ t }+x_0 \big)=p(x+tx_0).
		\end{equation}
		Avec \( t<0\), c'est similaire, en faisant attention au sens des inégalités.

		Nous avons donc construit \( h\colon D_h\to \eR\) avec \( h\in P\), \( D_f\subset D_h\) et \( h(x)=f(x)\) pour tout \( x\in D_f\). Cela pour dire que \( h>f\), ce qui contredit la maximalité de \( f\). Le domaine de \( f\) est donc \( E\) tout entier.

		La fonction \( f\) est donc une fonction qui remplit les conditions.

	\end{subproof}
\end{proof}


\begin{lemma}[\cite{BIBooMGFTooLkfVxy}]
	Soient un espace vectoriel complexe \( X\), et une application linéaire \(f \colon X\to \eC  \). Il existe une application \( u\in \aL_\eR(X,\eR)\) telle que
	\begin{enumerate}
		\item
		      Pour tout \( x\in X\), nous avons
		      \begin{equation}
			      f(x)=u(x)-iu(ix)
		      \end{equation}
		\item
		      Nous avons égalité des normes\quext{Pas encore démontrée\ldots}		%TODOooNUHNooMFTptV démonter l'égalité des normes.
		      \begin{equation}
			      \| f \|_{\aL_\eC(X,\eC)}=\| u \|_{\aL_\eR(X,\eR)}.
		      \end{equation}
	\end{enumerate}
\end{lemma}

\begin{proof}
	Nous décomposons, pour chaque \( x\in X\), l'élément \( f(x)\) en parties réelles et imaginaires :
	\begin{equation}		\label{EQooZJZIooBQxffg}
		f(x)=u(x)+iv(x).
	\end{equation}
	Il ne reste plus qu'à trouver le lien entre \( u\) et \( v\). Pour cela, nous écrions \( f(ix)=if(x)\) en substituant \( f(x)\) et \( f(ix)\) par leurs valeurs en termes de \( u\) et \( v\) :
	\begin{equation}
		u(ix)+iv(ix)=iu(x)-v(x).
	\end{equation}
	En égalisant les parties réelles, nous avons \( u(ix)=-v(x)\) et donc bien
	\begin{equation}	\label{EQooZUTLooVzwRDM}
		f(x)=u(x)+iv(x)=u(x)-iu(ix).
	\end{equation}

	Nous prouvons maintenant que \( u\) est \( \eR\)-linéaire. Il suffit de considérer \( \lambda\in\eR\) et d'écrire \( f(\lambda x)=\lambda f(x)\) en substituant \eqref{EQooZJZIooBQxffg} pour \( f(\lambda x)\) et \( f(x)\) :
	\begin{equation}
		u(\lambda x)+iv(\lambda x)=\lambda\big( u(x)+iv(x) \big).
	\end{equation}
	En égalisant encore les parties réelles, nous avons \( u(\lambda x)=\lambda u(x)\). Nous faisons de même pour prouver que \( u(x+y)=u(x)+u(y)\).
\end{proof}

\begin{theorem}[Hahn-Banach complexe\cite{BIBooMGFTooLkfVxy}]		\label{THOooVQLJooWuBMoZ}
	Soient un espace vectoriel \( X\) sur \( \eK\) (qui est \( \eR\) ou \( \eC\)), et une seminorme \(p \colon X\to \eR  \). Soient un sous-espace vectoriel \( M\) de \( X\) ainsi qu'une application linéaire \( \ell\in\aL(M,\eK)\) telle que
	\begin{equation}
		| \ell(m) |\leq p(m)\;\forall m\in m.
	\end{equation}

	Alors il existe une extension linéaire \( f\in\aL(X,\eK)\) telle que
	\begin{enumerate}
		\item
		      \( f(m)=\ell(m)\) pour tout \( m\in M\),
		\item
		      \( | f(x) |\leq p(x)\) pour tout \( x\in X\).
	\end{enumerate}
\end{theorem}

\begin{proof}
	Nous considérons \( X\) comme espace vectoriel réel (de dimension double de celle de \( X\) comme espace vectoriel réel). Là-dedans, l'application \( p\) vérifie \( p(\lambda x)=\lambda p(x)\) pour tout \( \lambda>0\). D'autre part nous décomposons l'application \(\ell \colon M\to \eC  \) en parties réelles et imaginaires :
	\begin{equation}
		\ell(x)=u(x)+iv(x),
	\end{equation}
	où \(u \colon M\to \eR  \) est une application vérifiant, pour tout \( m\in M\),
	\begin{equation}
		u(m)\leq | u(m) |\leq | \ell(m) |\leq p(m).
	\end{equation}
	Donc le théorème en version réelle \ref{THOooXALCooFrkvDo} s'applique et il existe une extension \(R \colon X\to \eR  \) de \( u\) qui vérifie \( R\leq p\) sur \( X\). Nous posons alors
	\begin{equation}
		f(x)=R(x)-iR(ix).
	\end{equation}
	En vertu du lemme \ref{LEMooBZHIooSQJSnM}, cela est une application \( \eC\)-linéaire sur \( X\), et nous avons \( f=\ell\) sur \( M\) parce que leurs parties réelles sont égales.

	De plus nous avons \( R\leq p\) et donc \( | f |\leq p\) par le lemme \ref{LEMooKTSKooIteFGq}.
\end{proof}

%-------------------------------------------------------
\subsection{Hyperplan séparateur}
%----------------------------------------------------


\begin{definition}[Hyperplan qui sépare]
	Soit \( E\) un espace vectoriel topologique ainsi que \( A\), \( B\) des sous-ensembles de \( E\). Nous disons que l'hyperplan d'équation \( f=\alpha\) \defe{sépare au sens large}{hyperplan!séparer!au sens large} les parties \( A\) et \( B\) si \( f(x)\leq \alpha\) pour tout \( x\in A\) et \( f(x)\geq \alpha\) pour tout \( x\in B\).

	La séparation est \defe{au sens strict}{hyperplan!sépare!au sens strict} si il existe \( \epsilon>0\) tel que
	\begin{subequations}
		\begin{align}
			f(x)\leq \alpha-\epsilon &  & \text{pour tout } x\in A  \\
			f(x)\geq \alpha+\epsilon &  & \text{pour tout } x\in B.
		\end{align}
	\end{subequations}
\end{definition}

\begin{theorem}[Hahn-Banach, première forme géométrique\cite{TQSWRiz}]  \label{ThoSAJjdZc}
	Soit \( E\) un espace vectoriel topologique et \( A\), \( B\) deux convexes non vides disjoints de \( E\). Si \( A\) est ouvert, il existe un hyperplan fermé qui sépare \( A\) et \( B\) au sens large.
\end{theorem}

\begin{theorem}[Hahn-Banach, seconde forme géométrique] \label{ThoACuKgtW}
	Soient un espace vectoriel topologique localement convexe\footnote{Définition~\ref{DefPJokvAa}.} \( E\) ainsi que des parties \( A\), \( B\) qui sont
	\begin{enumerate}
		\item
		      des convexes
		\item
		      non vides
		\item
		      disjoints
		\item
		      \( A\) compact et \( B\) soit fermé.
	\end{enumerate}
	Alors il existe un hyperplan fermé qui sépare strictement \( A\) et \( B\).
\end{theorem}

\begin{proof}
	Vu que \( B\) est fermé, \( A\) est dans l'ouvert \( E\setminus B\). Donc si \( a\in A\), il existe un voisinage ouvert convexe de \( a\) inclus dans \( E\setminus B\). Soit \( U_a\) un voisinage ouvert et convexe de \( 0\) tel que \( (a+U_a)\cap B=\emptyset\).

	Vu que la fonction \( (x,y)\mapsto x+y\) est continue, nous pouvons trouver un ouvert convexe \( V_a\) tel que \( V_a+V_a\subset U_a\). L'ensemble \( a+V_a\) est alors un voisinage ouvert de \( a\) et bien entendu \( \bigcup_a(a+V_a)\) recouvre \( A\) qui est compact. Nous en extrayons un sous-recouvrement fini, c'est-à-dire que nous considérons \( a_1,\ldots, a_n\in A\) tels que
	\begin{equation}
		A\subset \bigcup_{i=1}^n(a_i+V_{a_i}).
	\end{equation}
	Nous posons alors
	\begin{equation}
		V=\bigcap_{i=1}^nV_{a_i}.
	\end{equation}
	Cet ensemble est non vide parce et il contient un voisinage de zéro parce que c'est une intersection finie de voisinages de zéro. Soit \( x\in A+V\). Il existe \( i\) tel que
	\begin{equation}
		x\in a_i+U_{a_i}+V\subset a_i+V_{a_i}+V_{a_i}\subset a_i+U_{a_i}\subset E\setminus B.
	\end{equation}
	Donc \( (A+V)\cap B=\emptyset\). L'ensemble \( A+V\) est alors un ouvert convexe disjoint de \( B\). Par la première forme géométrique du théorème de Hahn-Banach~\ref{ThoSAJjdZc} nous avons un hyperplan qui sépare \( A+V\) de \( B\) au sens large : il existe \( f\in E'\setminus\{ 0 \}\) tel que \( f(a)+f(v)\leq f(b)\) pour tout \( a\in A\), \( v\in V\) et \( b\in B\).

	Il suffit donc de trouver un \( v\in V\) tel que \( f(v)\neq 0\) pour avoir la séparation au sens strict. Cela est facile : \( V\) étant un voisinage de zéro et \( f\) étant linéaire, si elle était nulle sur \( V\), elle serait nulle sur \( E\).
\end{proof}

\begin{corollary}[\cite{TQSWRiz}]		\label{CORooHTZVooFhgrSN}
	Soit un espace vectoriel localement convexe. Soit un sous-espace vectoriel \( F\) tel que \( \bar F\neq E\). Alors il existe une application \( f\in E'\setminus\{ 0 \}\) telle que \( f=0\) sur \( F\).
\end{corollary}

\begin{proof}
	Soit \( a\in E\) avec \( a\notin \bar F\). Vu que le singleton \( \{ a \}\) est compact et que \( \bar F\) est fermé, le théorème de Hahn-Banach \ref{ThoACuKgtW} dit qu'il existe \( f\in E'\setminus\{ 0 \}\) et \( \alpha\neq 0\) tel que \( f<\alpha\) sur \( \bar F\) et \( f(a)>\alpha\).

	Étant donné que \( F\) est vectoriel et que \( f\) est linéaire, avoir \( f(x)<\alpha\) pour tout \( x\) dans \( F\) implique d'avoir \( f(x)=0\) pour tout \( x\in F\). En effet si \( f(x)\neq 0\), il existe un \( \lambda\) tel que \( f(\lambda x)>\alpha\) (\( \eR\) est archimédien, théorème \ref{ThoooKJTTooCaxEny}).
\end{proof}

%--------------------------------------------------------------------------------------------------------------------------- 
\subsection{Prolongement de fonctionnelles (dimension finie)}
%---------------------------------------------------------------------------------------------------------------------------

Nous allons prouver quelques lemmes qui permettent de prolonger des fonctionnelles d'un sous-espace vers un sous-espace contenant un nombre fini de dimensions en plus.

\begin{lemma}[\cite{BIBooMXTXooOsgBRx}]     \label{LEMooHWSJooGVmIPV}
	Soient un espace vectoriel réel normé \( X\) ainsi qu'un sous-espace \( M\). Soient \( \ell\in\aL(M,\eR)\) de norme finie, et \( x_1\in X\setminus M\). On pose \( M_1=\Span\{ M, x_1 \}\). Alors il existe \( \ell_1\in\aL(M_1,\eR)\) tel que
	\begin{enumerate}
		\item
		      \( \ell_1(x)=\ell(x)\) pour tout \( x\in M\),
		\item
		      \( \| \ell_1 \|_{\aL(M_1,\eR)}=\| \ell \|_{\aL(M,\eR)}\)
	\end{enumerate}
\end{lemma}

\begin{proof}
	Si \( \ell=0\), alors c'est facile : on prend \( \ell_1=0\). Nous commençons par supposer que \( \| \ell \|=1\); nous ferons le cas général ensuite.
	\begin{subproof}
		\spitem[Deux fonctions]
		Nous considérons les fonctions
		\begin{equation}
			\begin{aligned}
				f_+\colon M & \to \eR                     \\
				z           & \mapsto \| x_1+z \|-\ell(z)
			\end{aligned}
		\end{equation}
		et
		\begin{equation}
			\begin{aligned}
				f_-\colon M & \to \eR                       \\
				z           & \mapsto -\| x_1+z \|-\ell(z).
			\end{aligned}
		\end{equation}
		Pour tout \( z_1,z_2\in M\) nous avons
		\begin{subequations}
			\begin{align}
				f_+(z_1)-f_-(z_2) & =\| x_1+z_1 \|-\ell(z_1)+\| x_1+z_2 \|+\ell(z_2) \\
				                  & =\| x_1+z_1 \|+\| x_1+z_2 \|-\ell(z_1-z_2)       \\
				                  & \geq \| (x_1+z_1)-(x_1+z_2) \|-\ell(z_1-z_2)     \\
				                  & =\| z_1-z_2  \|-\ell(z_1-z_2)                    \\
				                  & \geq 0
			\end{align}
		\end{subequations}
		parce que \( \| \ell \|=1\). Nous en déduisons que \( f_+(z_1)\geq f_-(z_2)\).

		\spitem[Un entre les deux]

		En posant
		\begin{subequations}
			\begin{align}
				c_- & =\sup_{z\in M}\big( -\| x_1+z \|-\ell(z) \big)  \\
				c_+ & =\inf_{z\in M}\big(  \| x_1+z \|-\ell(z) \big),
			\end{align}
		\end{subequations}
		nous avons \( c_-\leq c_+\). Nous choisissons \( c_1\in \mathopen[ c_- , c_+ \mathclose]\).
		\spitem[La définition]
		Si \( x\in M_1=\Span{M,x_1}\), alors il existe \( y\in M\) et \( \lambda\in \eR\) tels que
		\begin{equation}
			x=\lambda x_1+y.
		\end{equation}
		Nous posons alors
		\begin{equation}        \label{EQooVTFTooFyfXbK}
			\ell_1(x)=\lambda c_1+\ell(y).
		\end{equation}
		Voilà qui définit notre \( \ell_1\). Nous devons prouver qu'elle satisfait \( \| \ell_1 \|=1\) et \( \ell_1(x)=\ell(x)\) pour tout \( x\in M\). La seconde condition est facile : si \( x\in M\), alors \( \lambda=0\) dans \eqref{EQooVTFTooFyfXbK} et nous avons bien \( \ell_1(x)=\ell(x)\).

		Pour la condition sur la norme, nous allons devoir un peu travailler.
		\spitem[\( | \ell_1(x) |\leq \| x \|\) pour tout \( x\in M_1\)]
		Soit \( x\in M_1\). Nous avons \( x=\lambda x_1+y\) avec \( \lambda\in \eR\) et \( y\in M\). i

		\begin{subproof}
			\spitem[Si \( \lambda=0\)]
			Alors \( x\in M\) et \( |\ell_1(x)|=|\ell(x)|\leq \| x \|\) parce que \( \| \ell \|=1\).
			\spitem[Si \( \lambda\neq 0\)]
			Soient \( z_1,z_2\in M\). Par définition de \( c_1\), \( c_+\) et \( c_-\) nous avons les inégalités
			\begin{equation}
				-\| x_1+z_1 \|-\ell(z_1)\leq c_-\leq c_1\leq c_+\leq \| x_1+z_2 \|-\ell(z_2).
			\end{equation}
			Nous écrivons ces inégalités pour \( z_1=z_2=y/\lambda\) :
			\begin{equation}        \label{EQooGYXSooIffZoL}
				-\| x_1+\frac{ y }{ \lambda } \|-\ell(y/\lambda)\leq c_1\leq \| x_1+\frac{ y }{ \lambda } \|-\ell(y/\lambda).
			\end{equation}
			\spitem[Si \( \lambda>0\)]
			Nous multiplions \eqref{EQooGYXSooIffZoL} par \( \lambda\) et nous profitons de la linéarité de \( \ell\) :
			\begin{equation}
				-\| \lambda x_1+y \|-\ell(y)\leq \lambda c_1\leq \| \lambda x_1+y \|+\ell(y),
			\end{equation}
			donc
			\begin{equation}        \label{EQooYRULooBebNTq}
				-\| \lambda x_1+y \|\leq \underbrace{\lambda c_1+\ell(y)}_{=\ell_1(\lambda x_1+y)}\leq \| \lambda x_1+y \|
			\end{equation}
			Nous en déduisons que
			\begin{equation}
				| \ell(\lambda x_1+y) |\leq \| \lambda x_1+y \|,
			\end{equation}
			ce qu'il fallait.
			\spitem[Si \( \lambda<0\)]
			Le calcul est le même, mais il faut faire attention à bien reverser les inégalités au bon moment, et en manipulant bien les valeur absolues. Nous avons par exemple
			\begin{equation}
				\lambda\|  x_1+\frac{ y }{ \lambda } \|=-\big\| | \lambda |x_1+\frac{ | \lambda |y }{ \lambda } \big\|=-\| -\lambda x_1-y \|=-\| \lambda x_1+y \|.
			\end{equation}
			En multipliant encore \eqref{EQooGYXSooIffZoL} par \( \lambda\), nous trouvons
			\begin{equation}
				-\lambda\| x_1+\frac{ y }{ \lambda } \|-\ell(y)\geq \lambda c_1\geq \lambda\| x_1+\frac{ y }{ \lambda } \|-\ell(y).
			\end{equation}
			qui devient
			\begin{equation}
				\| \lambda x_1+y \|\geq \lambda c_1+\ell(y)\geq -\| \lambda x_1+y \|,
			\end{equation}
			qui revient au même que \eqref{EQooYRULooBebNTq}.
		\end{subproof}
		\spitem[Première conclusion]
		Nous avons prouvé que \( | \ell_1(x) |\leq \| x \|\) pour tout \( x\) dans \( M_1\). Cela signifie que \( \| \ell_1 \|_{M_1'}\leq 1\). Pour prouver que \( \| \ell_1 \|=1\) nous prouvons l'inégalité inverse :
		\begin{equation}
			\| \ell_1 \|_{M_1'}=\sup_{x\in M_1}\frac{ | \ell_1(x) | }{ \| x \| }\geq\sup_{x\in M}\frac{ | \ell_1(x) | }{ \| x \| }=\sup_{x\in M}\frac{ | \ell(x) | }{ \| x \| }=\| \ell \|_{M}=1.
		\end{equation}
		Nous en déduisons que \( \| \ell_1 \|=1\) et cela termine la preuve dans le cas où \( \|\ell \|=1\).
	\end{subproof}
	Maintenant, si \( \| \ell \|=a\neq 1\), nous considérons la forme linéaire \( f=\ell/a\) qui satisfait \( \| f \|=1\). Par la partie déjà prouvée, nous définissons une extension \( f_1\colon M_1\to \eR\) telle que \( \| f_1 \|=1\).

	Il suffit alors de poser \( \ell_1=af_1\), et nous avons le résultat.
\end{proof}

Ce lemme est également valable pour les complexes.

\begin{lemma}[\cite{MonCerveau}]       \label{LEMooBYEGooRswAmh}
	Soit un espace vectoriel normé \( X\) sur \( \eC\). Soit \( \ell\in \aL_{\eC}(X,\eC)\) de norme finie. Soient un sous-espace \( M\) de \( X\) ainsi qu'un élément \( a\in X\setminus M\). Il existe \( \ell_1\in\aL_{\eC}\big( \Span_{\eC}(M, a), \eC \big)\) tel que
	\begin{enumerate}
		\item
		      \( \ell_1(x)=\ell(x)\) pour tout \( x\in M\),
		\item
		      \( \| \ell_1 \|=\| \ell \|\).
	\end{enumerate}
\end{lemma}

\begin{proof}
	Nous considérons l'application \( \psi\) du lemme \ref{LEMooBZHIooSQJSnM}. En posant \( f=\psi^{-1}(\ell)\), nous avons $f\in\aL_{\eR}(M,\eR)$ vérifiant \( \| \ell \|=\| f \|\).

	L'espace \( X\) peut être vu comme vectoriel sur \( \eR\). Le lemme \ref{LEMooHWSJooGVmIPV} permet de prolonger \( f\) à
	\begin{equation}
		f_1\in\aL_{\eR}\big( \Span_{\eR}(M,a),\eR \big)
	\end{equation}
	vérifiant \( \| \ell \|=\| f \|=\| f_1 \|\).

	Vu que \( M\) est déjà un espace vectoriel sur \( \eC\), l'espace \( \Span_{\eC}(M,a)\) qui nous intéresse est donné par
	\begin{equation}
		\Span_{\eC}(M,a)=\Span_{\eR}(M,a,ia).
	\end{equation}
	Nous pouvons donc utiliser une deuxième fois le lemme \ref{LEMooHWSJooGVmIPV} avec le vecteur \( ia\), et définir une extension
	\begin{equation}
		f_2\in\aL_{\eR}\big( \Span_{\eR}(M,a,ia),\eR \big)=\aL_{\eR}\big( \Span_{\eC}(M,a),\eR \big)
	\end{equation}
	vérifiant \( \| \ell \|=\| f \|=\| f_1 \|=\| f_2 \|\).

	En utilisant à nouveau le lemme \ref{LEMooBZHIooSQJSnM}, nous avons encore une extension
	\begin{equation}
		\psi(f_2)\in\aL_{\eC}\big( \Span_{\eC}(M,a),\eC \big)
	\end{equation}
	vérifiant \( \| \ell \|=\| f \|=\| f_1 \|=\| f_2 \|=\| \psi(f_2) \|\).

	Et voilà ! La fonctionnelle \( \psi(f_2)\) est celle que nous voulions.
\end{proof}

%--------------------------------------------------------------------------------------------------------------------------- 
\subsection{Prolongement de fonctionnelles (dimension infinie)}
%---------------------------------------------------------------------------------------------------------------------------

Les lemmes \ref{LEMooHWSJooGVmIPV} et \ref{LEMooBYEGooRswAmh} permettent de prolonger une forme linéaire une dimension réelle ou complexe à la fois. Rien ne nous permet de prolonger d'une infinité de dimensions d'un seul coup. Le théorème de Hahn-Banach va nous permettre de faire une infinité de dimension d'un coup à l'aide du lemme de Zorn.

\begin{theorem}[Hahn-Banach\cite{BIBooMXTXooOsgBRx}]        \label{THOooTZSSooBKfxXE}
	Soit \( X\), un espace vectoriel normé sur \( \eK\) (\(=\eR\) ou \( \eC\)). Soient un sous-espace vectoriel \( M\), et une fonctionnelle linéaire \( \ell\in\aL_{\eK}(M,\eK)\) de norme finie. Il existe \( f\in\aL_{\eK}(X,\eK)\) telle que
	\begin{enumerate}
		\item
		      \( f(x)=\ell(x)\) pour tout \( x\in M\),
		\item
		      \( \| f \|_{\aL(X,\eK)}=\| \ell \|_{\aL(M,\eK)}\).
	\end{enumerate}
\end{theorem}
\index{prolongement de fonctionnelle linéaire}

\begin{proof}
	Nous considérons l'ensemble \( \mE\) des paires \( (N,f)\) telles que
	\begin{enumerate}
		\item
		      \( N\) est un sous-espace de \( X\) contenant \( M\),
		\item
		      \( f\in \aL(N,\eK)\)
		\item
		      \( \| f \|=\| \ell \|\)
		\item
		      \( f(x)=\ell(x)\) pour tout \( x\in M\).
	\end{enumerate}
	Cet ensemble n'est pas vide parce que \( (M, \ell)\in \mE\). Nous mettons un ordre partiel sur \( \mE\) en posant \( (N,f)\leq (N',f')\) si et seulement si \( N\subset N'\) et \( f'|_{N}=f\).
	\begin{subproof}
		\spitem[\( \mE\) est inductif]
		Nous commençons par prouver que \( (\mE,\leq)\) est un ensemble inductif\footnote{Définition \ref{DefGHDfyyz}.}. Soit une partie \( \mF\) totalement ordonnée dans \( \mE\).

		\begin{subproof}
			\spitem[L'espace vectoriel]


			Nous commençons par poser
			\begin{equation}
				Y=\bigcup_{(N,f)\in\mF}N.
			\end{equation}
			Et nous prouvons que c'est un espace vectoriel. Soient \( x,y\in Y\). Supposons \( x\in N_1\) et \( y\in N_2\). Vu que \( \mF\) est totalement ordonné, nous avons \( N_1\subset N_2\) (ou le contraire). Donc \( x+y\in N_2\subset Y\). De même pour \( \lambda x\) avec \( \lambda\in \eK\) et \( x\in Y\).

			\spitem[La fonctionnelle]
			Nous devons trouver une fonctionnelle \( g\) sur \( Y\). Soit \( y\in Y\). Commençons par prouver que l'ensemble
			\begin{equation}
				\{ f(y)\tq (N,f)\in \mF, y\in N \}
			\end{equation}
			est un singleton. Soient \( (N_1,f_1), (N_2,f_2)\in\mF\) avec \( y\in N_1\cap N_2\). Nous supposons que \( (N_1,f_1)\leq (N_2,f_2)\) (sinon c'est le contraire). Alors \( f_2(y)=f_2|_{N_1}(y)=f_1(y)\). Nous pouvons donc définir
			\begin{equation}
				\begin{aligned}
					g\colon Y & \to \eK      \\
					y         & \mapsto f(y)
				\end{aligned}
			\end{equation}
			où \( (N,f)\in \mF\) est tel que \( y\in N\).
			\spitem[\( g\) est linéaire]
			Soient \( x,y\in Y\). Nous supposons que \( x\in N_1\) et \( y\in N_2\) avec \( N_1\subset N_2\). Donc \( x,y,x+y\in N_2\) et nous avons \( g(x)=f_2(x)\), \( g(y)=f_2(y)\) et \( g(x+y)=f_2(x+y)\). La linéarité de \( f_2\) fait alors le boulot. Même raisonnement pour \( g(\lambda x)=\lambda g(x)\).
			\spitem[\( g\) se restreint à \( \ell\)]
			Soit \( x\in M\). Nous avons \( g(x)=f(x)\) pour un couple \( (N,f)\in \mF\) vérifiant \( x\in N\). Vu que \( f\) prolonge \( \ell\) nous avons \( g(x)=f(x)=\ell(x)\).
			\spitem[Norme de \( g\)]
			Nous devons voir que \( \| g \|_{\aL(Y,\eK)}=\| \ell \|_{\aL(M,\eK)}\). L'inégalité dans un sens est facile pour qui comprend la norme opérateur\footnote{Si ce n'est pas votre cas, vous ne devriez franchement pas être en train de lire ces lignes. Ce n'est que mon avis; après tout, vous faites comme vous le sentez.}. Étant donné que \( M\subset Y\) nous avons
			\begin{equation}
				\| g \|_{\aL(Y,\eK)}=\sup_{y\in Y}\frac{ | g(y) | }{ \| y \| }\geq\sup_{y\in M}\frac{ | g(y) | }{ \| y \| }=\sup_{y\in M}\frac{ | \ell(y) | }{ \| y \| }=\| \ell \|_{\aL(M,\eK)}.
			\end{equation}
			L'inégalité dans l'autre sens n'est pas trop compliquée non plus. Prenons \( x\in Y\) vérifiant \( \| x \|=1\). Nous considérons \( (N,f)\in \mE\) tel que \( x\in N\). Alors
			\begin{equation}
				| g(x) |=| f(x) |\leq\| f \|=\| \ell \|.
			\end{equation}
			Donc \( \| g \|\leq \| \ell \|\).
			\spitem[Conclusion pour le moment]
			Nous avons prouvé que \( (Y,g)\) est un majorant de \( \mF\). Donc \( (\mE,\leq)\) est un ensemble inductif.
		\end{subproof}
		\spitem[Lemme de Zorn]
		Vu que \( (\mE,\leq)\) est un ensemble inductif non vide, il possède un élément maximal par lemme de Zorn \ref{LemUEGjJBc}. Nous nommons \( (Y,f)\) un tel élément maximal.
		\spitem[Fin de la preuve]
		Nous devons prouver que \( Y=X\), de telle sorte que \( f\in\aL(Y,\eK)\) soit définie sur \( X\). Supposons que \( Y\neq X\). Dans ce cas nous considérons \( x_1\in X\setminus Y\). Suivant que \( \eK\) est \( \eR\) ou \( \eC\), nous utilisons le lemme \ref{LEMooHWSJooGVmIPV} ou \ref{LEMooBYEGooRswAmh} pour construire la paire
		\begin{equation}
			\big( \Span_{\eK}(Y,x_1),f_1 \big)
		\end{equation}
		qui majore \( (Y,f)\). Contradiction avec la maximalité de \( (Y,f)\). Donc \( Y=X\).
	\end{subproof}
\end{proof}

\begin{proposition}[\cite{MonCerveau}]          \label{PROPooFJPXooWrjbuH}
	Soit un espace de Banach \( E\) sur \( \eK\)(\( =\eR\) ou \( \eC\)). Soit \( a\in E\). Nous avons
	\begin{equation}
		\| a \|=\max_{\substack{\varphi\in E'\\\| \varphi \|=1}}| \varphi(a) |.
	\end{equation}
\end{proposition}

\begin{proof}
	Soit \( \varphi\in E'\) tel que \( \| \varphi \|=1\). Si \( | \varphi(a) |>\| a \|\) alors
	\begin{equation}
		\| \varphi \|=\sup_{x\in E}\frac{ | \varphi(x) | }{ \| x \| }\geq \frac{ | \varphi(a) | }{ \| a \| }>1
	\end{equation}
	Nous en déduisons que pour tout \( \varphi\in B_{E'}(0,1)\), \( | \varphi(a) |\leq \|a  \|\).

	Maintenant nous construisons une application linéaire \( \varphi\in E'\) telle que \( | \varphi(a) |=\| a \|\) et \( \| \varphi \|=1\). Pour cela nous considérons le sous-espace \( M=\Span\{ a \}\) et l'application
	\begin{equation}
		\begin{aligned}
			f\colon M & \to \eK                 \\
			\lambda a & \mapsto \lambda\| a \|.
		\end{aligned}
	\end{equation}
	Cette application vérifie \( f(a)=\| a \|\) et \( \| f\|=1\) parce que
	\begin{equation}
		\| f \|=\sup_{x\in M}\frac{ | f(x) | }{ \| x \| }=\sup_{\lambda\in\eK}\frac{ | f(\lambda a) | }{ \| \lambda a \| }=\frac{ | \lambda |\| a \| }{ | \lambda |\| a \| }=1.
	\end{equation}

	Le théorème de Hahn-Banach \ref{THOooTZSSooBKfxXE} prolonge \( f\) en un élément \( \varphi\in E'\) vérifiant \( \| \varphi \|=\| f \|=1\). Cette fonctionnelle vérifie donc aussi \( \varphi(a)=f(a)=\| a \|\).
\end{proof}

\begin{corollary}[\cite{MonCerveau}]        \label{CORooOBDHooJpiBrs}
	Voici trois façons différentes de dire la même chose, par ordre décroissant de frime.
	\begin{enumerate}
		\item
		      Le dual d'un espace vectoriel normé sépare les points.
		\item
		      Les fonctionnelles bornées d'un espace vectoriel normé séparent les points.
		\item
		      Si \( X\) est un espace vectoriel normé sur \( \eK\) (\( =\eR\) ou \( \eC\)), et si \( x,y\in X\), alors il existe une application \( f\in\aL(X,\eK)\) telle que \( f(x)\neq f(y)\).
	\end{enumerate}
\end{corollary}

\begin{proof}
	Le théorème de la base incomplète \ref{THOooOQLQooHqEeDK} nous permet de considérer une base \( \{ e_i \}_{i\in I}\) de \( X\) telle que \( e_0=x\) et \( e_1=y\).

	Évacuons quelque objections.
	\begin{itemize}
		\item Si \( x\) et \( y\) sont colinéaires, on complète seulement \( \{ x \}\) et ça ne changera rien pour la suite\footnote{Soyez quand même \randomGender{attentifs}{attentives} à ne pas vous laisser enfumer.}.
		\item En écrivant «\( e_0\)» et «\( e_1\)»  ne prétends pas que \( I\) soit un ensemble de nombres. C'est juste une facilité d'écriture pour éviter de dire «il existe \( \alpha,\beta\in I\) tels que \( x=e_{\alpha}\) et \( y=e_{\beta}\)».
	\end{itemize}
	Nous considérons l'application linéaire
	\begin{equation}
		\begin{aligned}
			\ell\colon \Span_{\eK}(e_0,e_1) & \to \eK         \\
			\alpha e_1+\beta e_2            & \mapsto \alpha.
		\end{aligned}
	\end{equation}
	Cette application est parfaitement bornée. Le théorème de Hahn-Banach \ref{THOooTZSSooBKfxXE} nous permet de considérer une extension \( f\) sur \( X\). Cette extension a la même norme et est donc bornée (donc continue par la proposition \ref{PROPooQZYVooYJVlBd}). C'est donc un élément du dual de \( X\).

	Comme \( f=\ell\) sur \( \Span(e_0,e_1)\), nous avons \( f(x)\neq f(y)\).
\end{proof}

%+++++++++++++++++++++++++++++++++++++++++++++++++++++++++++++++++++++++++++++++++++++++++++++++++++++++++++++++++++++++++++
\section{Théorème de Tietze}
%+++++++++++++++++++++++++++++++++++++++++++++++++++++++++++++++++++++++++++++++++++++++++++++++++++++++++++++++++++++++++++

\begin{definition}
	Si \( E\) et \( F\) sont des espaces normés, une application \( f\colon E\to F\) est \defe{presque surjective}{presque!surjective} si il existe \( \alpha\in\mathopen] 0 , 1 \mathclose[\) et \( C>0\) tels que pour tout \( y\in \overline{ B_F(0,1) }\), il existe \( x\in\overline{ B_E(0,C) }\) tel que \( \| y-f(x) \|\leq \alpha\).
\end{definition}

\begin{lemma}[\cite{KXjFWKA}]   \label{LemBQLooRXhJzK}
	Soient \( E\) et \( F\) des espaces de Banach et \( f\in\cL(E,F)\)\footnote{L'ensemble des applications linéaires continues}. Si \( f\) est presque surjective, alors
	\begin{enumerate}
		\item   \label{ItemTSOooYkxvBui}
		      \( f\) est surjective
		      \item\label{ItemTSOooYkxvBuii}
		      pour tout \( y\in \overline{ B_F(0,1) }\), il existe \( x\in\overline{ B_E(0,\frac{ C }{ 1-\alpha }) }\) tel que \( y=f(x)\).
	\end{enumerate}
\end{lemma}
Le point~\ref{ItemTSOooYkxvBuii} est une précision du point~\ref{ItemTSOooYkxvBui} : il dit quelle est la taille de la boule de \( E\) nécessaire à obtenir la boule unité dans \( F\).

\begin{proof}
	Soit \( y\in \overline{ B_F(0,1) }\). Nous allons construire \( x\in B\big( 0,\frac{ C }{ 1-\alpha } \big)\) qui donne \( f(x)=y\). Ce \( x\) sera la limite d'une série que nous allons construire par récurrence. Pour \( n=1\) nous utilisons la presque surjectivité pour considérer \( x_1\in\overline{ B_E(0,C) } \) tel que \( \| y-f(x_1) \|\leq \alpha\). Ensuite nous considérons la récurrence
	\begin{equation}
		x_n\in \overline{ B_E(0,C) }
	\end{equation}
	tel que
	\begin{equation}
		\big\| y-\sum_{i=1}^n\alpha^{i-1}f(x_i) \big\|\leq \alpha^n
	\end{equation}
	Pour montrer que cela existe nous supposons que la série est déjà construite jusqu'à \( n>1\) :
	\begin{equation}
		\frac{1}{ \alpha^n }\Big( y-\sum_{i=1}^n\alpha^{i-1}f(x_i) \Big)\in \overline{ B_F(0,1) }
	\end{equation}
	À partir de là, par presque surjectivité il existe un \( x_{n+1}\in \overline{ B_E(0,C) }\) tel que
	\begin{equation}
		\big\| \frac{ y-\sum_{i=1}^n\alpha^{i-1}f(x_i) }{ \alpha^n }-f(x_{n+1}) \big\|\leq \alpha.
	\end{equation}
	En multipliant par \( \alpha^{n}\), le terme \( \alpha^nf(x_{n+1})\) s'intègre bien dans la somme :
	\begin{equation}
		\big\| y-\sum_{i=1}^{n+1}\alpha^{i-1}f(x_i) \big\|\leq \alpha^{n+1}.
	\end{equation}
	Nous nous intéressons à une éventuelle limite à la somme des \( \alpha^{n-1}x_n\). D'abord nous avons la majoration \( \| \alpha^{n-1}x_n \|\leq \alpha^{n-1}C\), et vu que par la définition de la presque surjectivité \( 0<\alpha<1\), la série
	\begin{equation}
		\sum_{n=1}^{\infty}\alpha^{n-1}x_n
	\end{equation}
	converge absolument\footnote{Définition~\ref{DefVFUIXwU}.} parce que la suite des normes est une suite géométrique de raison \( \alpha\). Vu que \( E\) est de Banach, la convergence absolue implique la convergence simple (la suite des sommes partielles est de Cauchy et Banach est complet). Nous posons
	\begin{equation}
		x=\sum_{n=1}^{\infty}\alpha^{n-1}x_n\in E,
	\end{equation}
	et en termes de normes, ça vérifie
	\begin{equation}
		\| x \|\leq\sum_{n=1}^{\infty}\alpha^{n-1}\| x_n \|\leq C\sum_{n=1}^{\infty}\alpha^{n-1}=\frac{ C }{ 1-\alpha }.
	\end{equation}
	Donc c'est bon pour avoir \( x\in B\big( 0,\frac{ C }{ 1-\alpha } \big)\). Nous devons encore vérifier que \( y=f(x)\). Pour cela nous remarquons que
	\begin{equation}
		\| y-f\Big( \sum_{n=1}^N\alpha^{n-1}x_n \Big) \|\leq \alpha^N.
	\end{equation}
	Nous pouvons prendre la limite \( N\to \infty\) et permuter \( f\) avec la limite (par continuité de \( f\)). Vu que \( 0<\alpha<1\) nous avons
	\begin{equation}
		\| y-f(x) \|=0.
	\end{equation}
\end{proof}

\begin{theorem}[Tietze\cite{KXjFWKA,ytMOpe,BIBooCWPNooQHQtRb}]   \label{ThoFFQooGvcLzJ}
	Soit un espace métrique \( (X,d)\) et un fermé \( Y\subset X\). Soit \( g_0\in C^0(Y,\eR)\). Alors \( g_0\) admet un prolongement continu sur \( X\).
\end{theorem}

\begin{proof}
	Soit l'opération de restriction
	\begin{equation}
		\begin{aligned}
			T\colon (C^0_b(X,\eR),\| . \|_{\infty}) & \to (C^0_b(Y,\eR),\| . \|_{\infty}) \\
			f                                       & \mapsto f|_Y.
		\end{aligned}
	\end{equation}
	L'application \( T\) est évidemment linéaire. Elle est de plus bornée pour la norme opérateur usuelle donnée par la proposition~\ref{DefNFYUooBZCPTr} parce que \( \| T(f) \|\leq \| f \|<\infty\). L'application \( T\) est alors continue par la proposition~\ref{PROPooQZYVooYJVlBd}.

	\begin{subproof}
		\spitem[Presque surjection]

		Soit \( g\in C^0_b(Y,\eR)\) avec \( \| g \|_{\infty}\leq 1\). Nous posons
		\begin{subequations}
			\begin{align}
				Y^+=\{ x\in Y\tq \frac{1}{ 3 }\leq g(x)\leq 1 \} \\
				Y^-=\{ x\in Y\tq -1\leq g(x)\leq -\frac{1}{ 3 } \}.
			\end{align}
		\end{subequations}
		Nous considérons alors
		\begin{equation}
			\begin{aligned}
				f\colon X & \to \eR                                                              \\
				x         & \mapsto \frac{1}{ 3 }\frac{ d(x,Y^-)-d(x,Y^+) }{ d(x,Y^-)+d(x,Y^+) }
			\end{aligned}
		\end{equation}
		Vu qu'en valeur absolue le dénominateur est plus grand que le numérateur nous avons \( \| f \|_{\infty}\leq \frac{1}{ 3 }\). Notons que
		\begin{itemize}
			\item Si \( x\in Y^+\) alors \( f(x)=\frac{1}{ 3 }\) et \( g(x)\in\mathopen[ \frac{1}{ 3 } , 1 \mathclose]\);
			\item Si \( x\in Y^-\) alors \( f(x)=-\frac{1}{ 3 }\) et \( g(x)\in\mathopen[-1,-\frac{1}{ 3 } \mathclose]\);
			\item Si \( x\) n'est ni dans \( Y^+\) ni dans \( Y^-\) alors nous avons\footnote{Nous rappelons que \( \| g \|=1\), donc \( g(x)\) est forcément ente \( -1\) et \( 1\).} \( g(x)\in\mathopen[ -\frac{1}{ 3 } , \frac{1}{ 3 } \mathclose]\) et donc \( \big| f(x)-g(x) \big|\leq \big| f(x) \big|+\big| g(x) \big|\leq \frac{ 2 }{ 3 }\).
		\end{itemize}
		Dans les trois cas nous avons \( \big| f(x)-g(x) \big|\in\mathopen[ 0 , \frac{ 2 }{ 3 } \mathclose]\) pour tout \( x\in X\). Cela prouve que
		\begin{equation}
			\| T(f)-g \|_{Y,\infty}\leq \frac{ 2 }{ 3 }.
		\end{equation}
		En résumé nous avons pris \( g\) dans la boule \( \overline{ B(0,1) }\) de \( \big( C^0_b(Y,\eR), \| . \|_{\infty} \big)\) et nous avons construit une fonction \( f\) dans la boule \( \overline{ B(0,\frac{1}{ 3 }) }\) de \( \big( C^0_b(X,\eR),\| . \|_{\infty} \big)\) telle que \( \| T(f)-g \|_{\infty}\leq \frac{ 2 }{ 3 }\). L'application \( T\) est donc une presque surjection avec \( \alpha=\frac{2}{ 3 }\) et \( C=\frac{ 1 }{ 3 }\).

		\spitem[Extension avec \( \| . \|_{\infty}\leq 1\)]		\label{SPITEMooSTXMooIdockk}
		%-----------------------------------------------------------
		Nous montrons que si \(g \in C^0(Y,\eR)  \) vérifie \( \| g \|_{\infty}\leq 1\), alors il existe une extension \( f\in C^0(X,\eR)\) vérifiant \( \| f \|_{\infty}\leq 1\).

		La proposition~\ref{PropSYMEZGU} nous assure que les espaces \( C^0_b(X,\eR)\) et \( C_b^0(Y,\eR)\) sont de Banach (complets), et le lemme~\ref{LemBQLooRXhJzK} nous dit alors que \( T\) est surjective et que pour tout \( g\in\overline{ B(0,1) }\), il existe
		\begin{equation}
			f\in\overline{ B\left( 0,\frac{ 1/3 }{ 1-\frac{ 2 }{ 3 } } \right) }=\overline{ B(0,1) }.
		\end{equation}
		telle que \( g=T(f)\).

		Pour le redire avec des mots plus simples, nous avons prouvé que si \(g \colon Y\to \eR  \) vérifie \( \| g \|_{\infty}\leq 1\), il existe une extension \(f \colon X\to \eR  \) telle que \( \| f \|_{\infty}\leq 1\).

		\spitem[Extension avec \( | . |<1\)]		\label{SPITEMooZBGTooKbyuch}
		%-----------------------------------------------------------
		Nous montrons que si \(g \in C^0(Y,\eR)  \) vérifie \( | g |< 1\), alors il existe une extension \( f\in C^0(X,\eR)\) vérifiant \( | f |<1\).

		Soit \( g\in B_{C^0_b(Y)}(0,1) \) et son prolongement \( h\in \overline{ B_{C_b^0(X)}(0,1) }\) du point \ref{SPITEMooSTXMooIdockk}. Si \( | h |<1\) alors nous avons le résultat en posant \( f=g\).

		Si au contraire il existe \( z\in X\) tel que \( | h(z)|=1 \), nous posons
		\begin{equation}		\label{EQooIVXAooZQZxlq}
			Z=\{ x\in X\tq | h(x) |=1 \}.
		\end{equation}
		Nous avons \( Y\cap Z=\emptyset\) parce que nous avons \( h=g\) sur \( Y\) et nous avons choisi \( \| g \|_{\infty}<1\). Par ailleurs \( Y\) est fermé par hypothèse et \( Z\) est fermé parce que \( h\) est continue; par conséquent \( Y\cap Z\) est fermé, donc\footnote{Si vous avez l'intention de dire que \( \overline{ Y\cap Z }=\bar Y\cap\bar Z=Y\cap Z=\emptyset\), allez d'abord voir l'exemple~\ref{ExBFLooUNyvbw}. Ici c'est correct parce que \( Y\) et \( Z\) sont fermés.}
		\begin{equation}
			\bar Y\cap\bar Z=Y\cap Z=\emptyset.
		\end{equation}
		Nous posons
		\begin{equation}
			\begin{aligned}
				u\colon X & \to \eR^+                                \\
				x         & \mapsto \frac{ d(x,Z) }{ d(x,Y)+d(x,Z) }
			\end{aligned}
		\end{equation}
		Le dénominateur n'est pas nul parce qu'il faudrait \( d(x,Y)=d(x,Z)=0\), ce qui demanderait \( x\in \bar Y\cap\bar Z\), ce qui n'est pas possible. Nous posons \( f=uh\). Si \( x\in Y\) alors \( u(x)=1\), donc \( f\) est encore un prolongement de \( g\). De plus \( f\) est encore continue, et donc encore un bon candidat. Enfin si \( x\) est hors de \( Y\) alors \( d(x,Y)>0\) (strictement parce que \( Y\) est fermé) et donc \( 0<u(x)<1\), ce qui donne \( | f(x) |<| h(x) |\leq 1\).

		Et voila.

		\spitem[Démonstration du théorème]
		%-----------------------------------------------------------

		Nous sommes maintenant prêts à prouver le théorème. Soit \( g_0\in C^0(Y,\eR)\). Nous allons nous ramener au cas de la boule unité ouverte en utilisant un homéomorphisme\footnote{Existence dans le lemme \ref{LEMooHJEKooARZMil}.} \( \phi\colon \eR\to \mathopen] -1 , 1 \mathclose[\).

		L'application \( g=\phi\circ g_0\) vérifie \( | g(y) |<1\) pour tout \( y\in Y\). Par le point \ref{SPITEMooZBGTooKbyuch}, il existe une extension \( f\in C^0(X,\eR)\) vérifiant \( | f(x) |<1\) pour tout \( x\in X\). Attention : cette application \( f\) prolonge \( g\) et non \( g_0\).

		Nous posons finalement \( f_0=\phi^{-1}\circ f\). Cela est encore une application dans \( C^0(X,\eR)\), mais cette fois elle prolonge \( g_0\). En effet si \( y\in Y\), en tenant compte de \( f(y)=g(y)\), nous avons
		\begin{equation}
			f_0(y)=(\phi^{-1}\circ f)(y)=\phi^{-1}\big( f(y) \big)=\phi^{-1}\big( g(y) \big)=\phi^{-1}\big( (\phi\circ g_0)(y) \big)=g_0(y).
		\end{equation}
	\end{subproof}
\end{proof}

%-------------------------------------------------------
\subsection{Pas de bicontinues entre dimensions différentes}
%----------------------------------------------------




\begin{theorem}[\cite{BIBooJZMNooSVbblj}]		\label{THOooLGJMooIYzOBD}
	Soit une application continue et injective \(f \colon \overline{B(0,1)}\to \eR^n  \). Alors
	\begin{equation}
		f(0)\in \Int\Big( f\big( \overline{B(0,)} \big) \Big).
	\end{equation}
\end{theorem}

\begin{proof}
	Soit une application continue et injective \(f_0 \colon \overline{B(0,1)}\to \eR^n  \). Soit \( \epsilon<0\). Si nous supposons que \( f_0(0)\) n'est pas à l'intérieur de \( f_0\big( \overline{B(0,1)} \big)\), alors il existe un \( c\in \eR^n\) hors de \( f_0\big( \overline{B(0,1)} \big)\) et tel que \( \| c-f_0(0) \|<\epsilon\).

	Nous posons maintenant \( f(x)=f_0(x)-c\). Nous avons
	\begin{subequations}
		\begin{numcases}{}
			\| f(0) \|<\epsilon\\
			0\notin f\big( \overline{B(0,1)} \big).
		\end{numcases}
	\end{subequations}

	Vu que \( f\) est continue et que \( \overline{B(0,1)}\) est compact, la partie \( f\big( \overline{B(0,1)} \big)\) est compacte (théorème \ref{ThoImCompCotComp}). L'application restreinte \(f \colon \overline{(B(0,1))}\to f_0\big( \overline{B(0,1)} \big)  \) est une bijection continue entre compacts. Son inverse est alors également continue (lemme \ref{LEMooPLGTooATIGov}).

	Mais comme \( f\big( \overline{B(0,1)} \big)\) est un fermé de \( \eR^n\), le théorème de Tietze (\ref{ThoFFQooGvcLzJ}) prolonge son inverse en une application continue
	\begin{equation}
		G \colon \eR^n\to \eR^n.
	\end{equation}
	Nous avons \( G\big( f(0) \big)=0\). Nous choisissons \( \epsilon\) assez petit pour que \( \| G(y) \|<0.1\) dès que \( \| y-f(0) \|\leq 2\epsilon\). Comme \( \| f(0) \|<\epsilon\), nous avons aussi
	\begin{equation}
		\| y \|\leq \epsilon \Rightarrow \| y-f(0) \|\leq 2\epsilon \Rightarrow \| G(y) \|<0.1.
	\end{equation}
	Autrement dit,
	\begin{equation}
		G\Big( \overline{B(0,\epsilon)} \Big)\subset B(0,\frac{1}{ 10}).
	\end{equation}
	Nous posons
	\begin{equation}
		\Sigma_1=\{ y\in f\big( \overline{B(0,1)} \big)\tq \| y \|\geq \epsilon \},
	\end{equation}
	ainsi que
	\begin{equation}
		\Sigma_2=\{ y\in \eR^n\tq \| y \|=\epsilon \},
	\end{equation}
	et \( \Sigma=\Sigma_1\cup \Sigma_2\).

	\begin{enumerate}
		\item

		      La partie \( \Sigma_2\) est compacte en tant que fermée et bornée\footnote{Borel-Lebesgue \ref{ThoXTEooxFmdI}.}. La partie \( \Sigma_1\) est compacte en tant que intersection entre le compact \( f\big( \overline{B(0,1)} \big)\) et le fermé \( \| y \|\geq \epsilon\) (corolaire \ref{CORooSSFFooNkNmlS}). Finalement la partie \( \Sigma\) est également compacte\footnote{Union de compacts, lemme \ref{LEMooFJZDooSxYWVW}.}
		\item
		      La partie \( \Sigma\) ne contient pas \( f(0)\) parce que \( \| f(0)<\epsilon \|\) tandis que les éléments \( y\in \Sigma\) vérifient \( \| y \|\geq \epsilon\).

		\item
		      La partie \( \Sigma_1\) ne contient pas \( 0\). En effet, vu que \( G\) étend \( f^{-1}\), si \( x\in G(\Sigma_1)\), nous avons \( f(x)\in \Sigma_1\) et en particulier \( \| f(x) \|\geq \epsilon\). Vu que \( \| f(0) \|<\epsilon\), nous avons \( x\neq 0\).

		\item
		      La partie \( \Sigma_1\) est compacte, \( G\) est continue dessus et ne prend pas la valeur \( 0\). Il existe donc \( \delta>0\) tel que
		      \begin{equation}
			      G(\Sigma_1)>\delta.
		      \end{equation}
		      Nous considérons un tel \( \delta\) qui vérifie de plus \( \delta<1/10\).
	\end{enumerate}

	Nous introduisons l'application
	\begin{equation}
		\begin{aligned}
			\phi\colon f\big( \overline{B(0,1)} \big) & \to \Sigma                                   \\
			y                                         & \mapsto \max\big( \epsilon/\| y \|,1 \big)y.
		\end{aligned}
	\end{equation}
	Prouvons que \( \phi\) prend effectivement ses valeurs dans \( \Sigma\). Il y a deux possibilités : soit \( \| y \|<\epsilon\), soit \( \| y \|\geq \epsilon\). Si \( \| y \|<\epsilon\), alors \( \phi(y)=\frac{ \epsilon }{ \| y \| }y\), ce qui donne \( \| \phi(y) \|=\epsilon\), et donc \( \phi(y)\in \Sigma_2\). Si \( \| y \|\geq \epsilon\), alors le max est \( 1\) et nous avons \( \phi(y)=y\). Dans ce cas, nous avons d'une part \( \| \phi(y) \|=\| y \|\geq \epsilon\) et d'autre part \( \phi(y)=y\in f\big( \overline{B(0,1)} \big)\); donc \( \phi(y)\in \Sigma_1\).

	Le théorème de Stone-Weierstrass \ref{THOooJGAJooBcTtDH} nous permet de considérer un polynôme (à \( n\) variables) \(P \colon \eR^n\to \eR^n  \) vérifiant
	\begin{equation}
		\| P-G \|_{\Sigma}<\delta.
	\end{equation}
	Un tel polynôme ne s'annule pas sur \( \Sigma_1\) parce que \( \| G \|_{\Sigma_1}>\delta\).

	Le polynôme \( P\) est de classe \( C^{\infty}\) (proposition \ref{PROPooMGFBooHWGXyC}). Vu que \( \Sigma_2\) est de mesure nulle, la proposition \ref{PROPooSEJJooIfGUkW} dit que \( P(\Sigma_2)\) est de mesure nulle dans \( \eR^n\).

	À l'aide de la proposition \ref{PROPooIESAooBsThho}, nous choisissons \( s\in \eR^n\) tel que \( P_2(x)=P(x+s)\) ne s'annule ni sur \( \Sigma_1\) ni sur \( \Sigma_2\). Vu que \( P\) est uniformément continue sur \( \Sigma\) (théorème de Heine \ref{PROPooBWUFooYhMlDp}), il existe \( M\) (majorant de la norme de la dérivée sur \( \Sigma\)) tel que
	\begin{equation}
		\| P(x)-P_2(x) \|=\| P(x+s)-P(x) \|_{\Sigma}\leq sM.
	\end{equation}
	En choisissant \( s<\epsilon'/M\), nous avons
	\begin{equation}
		\| P-P_2 \|_{\Sigma}<\epsilon'.
	\end{equation}
	En prenant \( \epsilon'\) assez petit\footnote{Plus petit que la différence entre \( \delta\) et \( \| P-G \|_{\Sigma}\).}, nous avons encore
	\begin{equation}		\label{EQooIVQLooVjVzsA}
		\| P_2-G \|_{\Sigma}<\delta.
	\end{equation}

	Nous introduisons maintenant
	\begin{equation}
		\begin{aligned}
			\tilde G\colon f\big( \overline{B(0,1)} \big) & \to \eR^n       \\
			\tilde G                                      & =P_2\circ \phi.
		\end{aligned}
	\end{equation}

	Nous allons prouver que \( \| G(y)-\tilde  G(y) \|\leq 0.3\) pour tout \( y\in f\big( \overline{B(0,1)} \big)\).
	\begin{subproof}
		\spitem[Si \( \| y \| > \epsilon\)]
		%-----------------------------------------------------------

		Si \( \| y \|>\epsilon\) et \( y\in b\big( \overline{B(0,1)} \big)\), alors
		\begin{equation}
			\tilde G(y)=(P_2\circ \phi)(y)=P\big( \min(\epsilon/\| y \|,1)y \big).
		\end{equation}
		Vu que \( \| y \|>\epsilon\), le max est \( 1\) et nous avons \( \tilde G(y)=P_2(y)\), et donc, en utilisant \eqref{EQooIVQLooVjVzsA},
		\begin{equation}
			\| G(y)-\tilde G(y) \|=\| G(y)-P_2(y) \|<\delta.
		\end{equation}
		Quitte à choisir \( \delta\) plus petit, nous pouvons supposer \( \delta<0.3\).

		\spitem[Si \( \| y \|\leq \epsilon\)]
		%-----------------------------------------------------------
		Nous considérons maintenant le cas \( \| y \|\leq\epsilon\) et \( y\in b\big( \overline{B(0,1)} \big)\).


		Montrons que \( \| G(y)-\tilde G(y) \|<\delta\) pour tout \( y\in\big( \overline{B(0,1)} \big)\) tel que \( \| y \|>\epsilon\). Nous avons
		\begin{subequations}
			\begin{align}
				\| G(y)-\tilde G(y) \| & = \| G(y)-P_2\big( \phi(y) \big) \|                                                    \\
				                       & \leq \| G(y)-G\big( \phi(y) \big) \|+\| G\big( \phi(y) \big)-P_2\big( \phi(y) \big) \| \\
				                       & \leq \| G(y) \| + \| G\big( \phi(y) \big) \|+\| (G-P_2)\big( \phi(y) \big) \|.
			\end{align}
		\end{subequations}
		Vu que \( \| y \|\leq \epsilon\), nous avons \( \phi(y)=\frac{ \epsilon }{ \| y \| }y\), et donc \( \| \phi(y) \|=\epsilon\). De ce fait, \( \phi(y)\in \Sigma_2\) et nous avons \( \| (G-P_2)\big( \phi(y) \big) \|\leq \delta\).

		Si nous choisissons \( \delta<0.1\) nous avons donc encore
		\begin{equation}
			\| G(y)-\tilde G(y) \|\leq 2\times 0.1+\delta\leq 0.3.
		\end{equation}
	\end{subproof}
	Bref, pour tout \( y\in f\big( \overline{B(0,1)} \big)\) nous avons
	\begin{equation}		\label{EQooMIVFooPwKeSB}
		\| G(y)-\tilde  G(y) \|\leq 0.3.
	\end{equation}

	Ça va être le moment de demander au théorème de Brouwer d'intervenir.

	\begin{subproof}
		\spitem[Définition de \( g\)]
		%-----------------------------------------------------------

		Vu que \( \tilde G\) est définie sur \( f\big( \overline{B(0,1)} \big)\), nous pouvons considérer l'application
		\begin{equation}
			\begin{aligned}
				g\colon \overline{B(0,1)} & \to \eR^n                           \\
				x                         & \mapsto x-\tilde g\big( f(x) \big).
			\end{aligned}
		\end{equation}
		Notons pour \( x\in \overline{B(0,1)}\), nous avons \( x=G\big( f(x) \big)\), et donc nous pouvons aussi écrire
		\begin{equation}		\label{EQooJWNRooKIffex}
			g(x)=(G\circ f)(x)-(\tilde  G\circ f)(x).
		\end{equation}
		\spitem[\( g(x)\in \overline{B(0,1)}\) ]
		%-----------------------------------------------------------

		Nous avons déjà vu en \eqref{EQooMIVFooPwKeSB} que \( \| G-\tilde G \|_{f\big( \overline{B(0,1)} \big)}< 0.3<1\). L'équation \eqref{EQooJWNRooKIffex} dit alors que \( g(x)\in \overline{B(0,1)}\).

		\spitem[Brouwer]
		%-----------------------------------------------------------
		Nous pouvons appliquer le théorème de Brouwer \eqref{ThoRGjGdO} à la fonction \( g\). Il existe \( x_0\in \overline{B(0,1))}\) tel que \( g(x_0)=x_0\), c'est-à-dire \( x_0-\tilde G\big( f(x_0) \big)=x_0\). Nous en déduisons que
		\begin{equation}
			\tilde G\big( f(x_0) \big)=0.
		\end{equation}

		\spitem[La contradiction]
		%-----------------------------------------------------------
		Rappel : \( \tilde G=P_2\circ \phi\). Si \( y\in \overline{B(0,1)}\), alors \( \phi(y)\in \Sigma\), et nous avons tout fait pour que \( P_2\) ne s'annule pas sur \( \Sigma\).

		Et voila la contradiction.

	\end{subproof}
\end{proof}

\begin{remark}  % TOTOooVTLSooSNLVBD justifier ça. C'est une conséquence de THOooLGJMooIYzOBD
	Il n'existe pas de bijection bicontinues d'un ouvert de \( \eR^m\) vers un ouvert de \( \eR^n\) si \( m\neq n\). Il n'y a donc pas de notion de difféomorphismes entre ouverts de dimensions différentes.

	Je crois que c'est une conséquence facile de \ref{THOooLGJMooIYzOBD}. Écrivez-moi si vous pouvez rédiger une preuve.
\end{remark}


%+++++++++++++++++++++++++++++++++++++++++++++++++++++++++++++++++++++++++++++++++++++++++++++++++++++++++++++++++++++++++++
\section{Dualité, réflexivité et théorème de représentation de Riesz}
%+++++++++++++++++++++++++++++++++++++++++++++++++++++++++++++++++++++++++++++++++++++++++++++++++++++++++++++++++++++++++++

Dans la suite \( E'\) est le dual topologique, c'est-à-dire l'espace des formes linéaires et continues sur \( E\). Nous notons également \( V''\) le dual de \( (V',\| . \|)\). Certes en tant qu'ensembles, \( (V',*)\) et \( (V',\| . \|) \) sont identiques, mais comme ils n'ont pas la même topologie, les duaux ne sont pas les mêmes.

Bref, \( V''\) est l'ensemble des applications linéaires continues \( (V',\| . \|)\to \eC\). Et lorsque nous disons \( \eC\) ici, ça peut aussi bien être \( \eR\) selon le contexte.

De plus nous considérons que \( V''\) la norme opérateur qui dérive de la norme de \( V'\), laquelle dérive de la norme vectorielle sur \( V\).

\begin{propositionDef}[Espace réflexif]      \label{PROPooMAQSooCGFBBM}
	Soit un espace vectoriel normé \( V\) sur \( \eR\) ou \( \eC\). Nous considérons l'application
	\begin{equation}
		\begin{aligned}
			J\colon V   & \to V''       \\
			J(x)\varphi & = \varphi(x).
		\end{aligned}
	\end{equation}
	\begin{enumerate}
		\item       \label{ITEMooNVVSooNFXgnE}
		      L'application \( J\) est bien définie : \( J(x)\colon V'\to \eC\) est continue.
		\item       \label{ITEMooKURHooZZWpbu}
		      L'application \( J\) est continue.
		\item       \label{ITEMooTFYVooKhMOjp}
		      Elle est injective.
	\end{enumerate}

	Lorsque \( J\) est bijective, l'espace \( V\) est dit \defe{réflexif}{réflexif}.
\end{propositionDef}

\begin{proof}
	Point par point.
	\begin{subproof}
		\spitem[\ref{ITEMooNVVSooNFXgnE}]
		Nous commençons par montrer que \( J(x)\colon (V',\| . \|)\to \eC\) est continue pour chaque \( x\in V\). Soit une suite \( \varphi_k\stackrel{\| . \|}{\longrightarrow}0\). Nous avons :
		\begin{equation}
			J(x)\varphi_k=\varphi_k(x)\leq \| \varphi_k \|\| x \|\to 0
		\end{equation}
		où vous aurez noté l'utilisation du lemme~\ref{LEMooIBLEooLJczmu}.  Cela prouve que \( J(x)\) est continue et donc que \( J\) est bien à valeurs dans \( V''\).
		\spitem[\ref{ITEMooKURHooZZWpbu}]

		Soit une suite \( x_k\stackrel{V}{\longrightarrow}0\), et étudions \( \| J(x_k) \|\) pour la norme dans \( V''\). Nous posons \( x'_k=x_k/\| x_k \|\) et nous calculons (encore une fois, nous écrivons «\( \eC\)», mais ça pourrait être \( \eR\))
		\begin{equation}
			\| J(x_k) \|=\sup_{\| \varphi \|=1}| J(x_k)\varphi |_{\eC}=\sup_{\| \varphi \|=1}| \varphi(x_k) |=\| x_k \|\sup_{\| \varphi \|=1}| \varphi(x'_k) |\leq \| x_k \|\to 0.
		\end{equation}
		La dernière inégalité pourrait être sans doute une égalité\quext{Écrivez-moi si vous en êtes certain.}, mais nous n'en avons pas besoin ici.
		\spitem[\ref{ITEMooTFYVooKhMOjp}]
		Soient \( x\neq y\) dans \( V\). Le corolaire \ref{CORooOBDHooJpiBrs} nous permet de considérer un élément \( \varphi\in V'\) tel que \( \varphi(x)\neq \varphi(y)\). Nous avons alors
		\begin{equation}
			J(x)\varphi\neq J(y)\varphi,
		\end{equation}
		et donc \( J(x)\neq J(y)\). Cela prouve que \( J\) est injective.
	\end{subproof}
\end{proof}

\begin{lemma}[\cite{MonCerveau}]       \label{LEMooWEMFooEHIaxY}
	Soient des espaces de Banach \( V\) et \( W\). Si \( \alpha\colon V\to W\) est une bijection linéaire et isométrique, alors l'application définie par
	\begin{equation}        \label{EQooTSVHooEQGuNw}
		\begin{aligned}
			\beta\colon V'  & \to W'                             \\
			\beta(\varphi)y & =\varphi\big( \alpha^{-1}(y) \big)
		\end{aligned}
	\end{equation}
	pour tout \( \varphi\in V'\) et \( y\in W\) est une bijection linéaire isométrique.
\end{lemma}

\begin{proof}
	En plusieurs points.
	\begin{subproof}
		\spitem[\( \alpha^{-1}\) est continue]
		Le théorème d'isomorphisme de Banach \ref{ThofQShsw} implique que \( \alpha^{-1}\) est continue parce que \( \alpha\) est bijective, continue et linéaire entre deux espaces de Banach.
		\spitem[\( \beta\) prend ses valeurs dans \( W'\)]
		Nous devons prouver que, si \( \varphi\in V'\), alors \( \beta(\varphi)\) est bien un élément de \( W'\). Autrement dit, nous devons prouver que \(  \beta(\varphi)\colon W\to \eC   \) est continue. Considérons une suite \( y_k\stackrel{W}{\longrightarrow}0\). Alors, vu que \( \alpha\) est continue, nous avons aussi \( \alpha^{-1}(y_k)\stackrel{V}{\longrightarrow}0\), et donc
		\begin{equation}
			\varphi\big( \alpha^{-1}(y_k) \big)\stackrel{\eC}{\longrightarrow}0
		\end{equation}
		parce que \( \varphi\) est continue. Donc \( \beta(\varphi)\) est continue.
		\spitem[\( \beta\) est linéaire]
		Parce que \( \alpha\) l'est.
		\spitem[\( \beta\) est isométrique]
		Nous devons prouver que la norme de l'application linéaire \( \beta(\varphi)\colon W\to \eC\) est la même que celle de \( \varphi\). Vu que \( \alpha\colon V\to W\) est une bijection isométrique, nous avons
		\begin{equation}
			\{ \alpha^{-1}(y)\tq \| y \|=1,y\in W \}=\{ x\in V\tq \|x  \|=1 \}.
		\end{equation}
		Donc nous pouvons faire le calcul suivant :
		\begin{equation}
			\| \beta(\varphi) \|=\sup_{\substack{y\in W\\\| y \|=1}}\| \beta(\varphi)y \|
			=\sup_{\substack{y\in W\\\| y \|=1}}| \varphi\big( \alpha^{-1}(y) \big) |
			=\sup_{\substack{x\in V\\\| x \|=1}}| \varphi(x) |
			=\| \varphi \|.
		\end{equation}
	\end{subproof}
\end{proof}

La proposition suivante dit que la notion d'être réflexif passe aux isomorphismes d'espaces vectoriels normés.
\begin{proposition}[\cite{MonCerveau}]      \label{PROPooVRQKooLdmajh}
	Soient des espaces vectoriels normés \( V\) et \( W\) ainsi qu'une bijection linéaire isométrique \( \alpha\colon V\to W\). L'espace \( W\) est réflexif\footnote{Définition \ref{PROPooMAQSooCGFBBM}.} si et seulement si \( V\) l'est.
\end{proposition}

\begin{proof}
	Nous allons prouver que si \( W\) est réflexif, alors \( V\) est réflexif. Pour l'implication inverse, il suffira de noter que \( \alpha^{-1}\colon W\to V\) est une bijection linéaire isométrique.

	En plusieurs points.
	\begin{subproof}
		\spitem[Quelques applications]
		Nous notons \( J_V\) l'application de la proposition \ref{PROPooMAQSooCGFBBM} :
		\begin{equation}
			\begin{aligned}
				J_V\colon V   & \to V''       \\
				J_V(v)\varphi & = \varphi(v),
			\end{aligned}
		\end{equation}
		et \( J_W\) l'application correspondante pour \( W\). Notre hypothèse est que \( J_W\) est bijective.

		Nous posons
		\begin{equation}
			\begin{aligned}
				\beta\colon V'  & \to W'                              \\
				\beta(\varphi)w & =\varphi\big( \alpha^{-1}(w) \big),
			\end{aligned}
		\end{equation}
		et
		\begin{equation}
			\begin{aligned}
				\gamma\colon V''      & \to W''                                 \\
				\gamma(\sigma)\varphi & =\sigma\big( \beta^{-1}(\varphi) \big).
			\end{aligned}
		\end{equation}
		L'application \( \beta\) est une bijection linéaire isométrique par le lemme \ref{LEMooWEMFooEHIaxY}; l'application \( \gamma\) l'est aussi, par le même lemme appliqué à \( \beta\colon V'\to W'\).

		\spitem[Le \( S\)]
		Nous posons
		\begin{equation}
			\begin{aligned}
				S\colon V   & \to V''                                  \\
				S(v)\varphi & =J_W\big( \alpha(v) \big)\beta(\varphi).
			\end{aligned}
		\end{equation}
		Nous allons montrer que \( S=J_V\) et que \( S\) est bijectif.
		\spitem[\( S=J_V\)]
		Cela est un calcul :
		\begin{subequations}
			\begin{align}
				J_W\big( \alpha(v) \big)\beta(\varphi) & =\beta(\varphi)\big( \alpha(v) \big)                 \\
				                                       & =\varphi\Big( \alpha^{-1}\big( \alpha(v) \big) \Big) \\
				                                       & =\varphi(v)                                          \\
				                                       & =J_V(v)\varphi.
			\end{align}
		\end{subequations}
		\spitem[\( S\) est injective]
		L'application \( J_V\) est toujours injective par la proposition \ref{PROPooMAQSooCGFBBM}.
		\spitem[\( S\) est surjective]
		Soi \( \sigma\in V''\). Nous devons prouver l'existence de \( v\in V\) tel que \( S(v)=\sigma\). Vu que \( J_W\colon W\to W''\) est surjective et que \( \gamma(\sigma)\in W''\), il existe \( w\in W\) tel que
		\begin{equation}
			J_W(w)=\gamma(\sigma).
		\end{equation}
		Un simple calcul montre alors que \( S\big( \alpha^{-1}(w) \big)=\sigma\) :
		\begin{subequations}
			\begin{align}
				S\big( \alpha^{-1}(w) \big)\varphi & =J_W(w)\beta(\varphi)                                   \\
				                                   & =\gamma(\sigma)\big( \beta(\varphi) \big)               \\
				                                   & =\sigma\Big( \beta^{-1}\big( \beta(\varphi) \big) \Big) \\
				                                   & =\sigma(\varphi).
			\end{align}
		\end{subequations}
		Nous avons donc bien que \( S(v)=\sigma\) en posant \( v=\alpha^{-1}(w)\).
		\spitem[Conclusion]
		L'application \( S=J_V\) est surjective; l'espace \( V\) est donc réflexif par la proposition \ref{PROPooMAQSooCGFBBM}.
	\end{subproof}
\end{proof}

Voici déjà un bel énoncé. Pour des espaces mesurés \( (\Omega,\tribA,\mu)\) plus généraux, voir l'arme totale en le théorème \ref{ThoLPQPooPWBXuv}.

\begin{proposition}[\cite{PAXrsMn}, thème~\ref{THEMEooULGFooPscFJC}] \label{PropOAVooYZSodR}
	Soit \( 1<p<2\) et \( q\) tel que \( \frac{1}{ p }+\frac{1}{ q }=1\). L'application
	\begin{equation}
		\begin{aligned}
			\Phi\colon L^q\big( \mathopen[ 0 , 1 \mathclose] \big) & \to  L^p\big( \mathopen[ 0 , 1 \mathclose] \big)' \\
			\Phi_g(f)                                              & = \int_{\mathopen[ 0 , 1 \mathclose]}f\bar g.
		\end{aligned}
	\end{equation}
	est une isométrie linéaire surjective.
\end{proposition}

\begin{proof}
	Pour la simplicité des notations nous allons noter \( L^2\) pour \( L^2\big( \mathopen[ 0 , 1 \mathclose] \big)\), et pareillement pour \( L^p\).
	\begin{subproof}
		\spitem[\( \Phi_g\) est un élément de \( (L^p)'\)]

		Si \( f\in L^p\) et \( g\in L^q\) nous devons prouver que \( \Phi_g(f)\) est bien définie. Pour cela nous utilisons l'inégalité de Hölder\footnote{Proposition~\ref{ProptYqspT}.} qui dit que \( fg\in L^1\); par conséquent la fonction \( f\bar g\) est également dans \( L^1\) et nous avons
		\begin{equation}
			| \Phi_g(f) |\leq\int_{\mathopen[ 0 , 1 \mathclose]}| f\bar g |=\| fg \|_1\leq \| f \|_p\| g \|_q.
		\end{equation}
		En ce qui concerne la norme de l'application \( \Phi_g\) nous avons tout de suite
		\begin{equation}
			\| \Phi_g \|=\sup_{\| f\|_p=1}\big| \Phi_g(f) \big|\leq \| g \|_q.
		\end{equation}
		Cela signifie que l'application \( \Phi_g\) est bornée et donc continue par la proposition~\ref{PROPooQZYVooYJVlBd}. Nous avons donc bien \( \Phi_g\in (L^p)'\).

		\spitem[Isométrie]

		Afin de prouver que \( \| \Phi_g \|=\| g \|_q\) nous allons trouver une fonction \( f\in L^p\) telle que \( \frac{ | \Phi_g(f) | }{ \| f \|_p }=\| g \|_q\).  De cette façon nous aurons prouvé que \( | \Phi_g |\geq \| g \|_q\), ce qui conclurait que \( | \Phi_g |=\| g \|_q\).

		Nous posons \( f=g| g |^{q-2}\), de telle sorte que \( | f |=| g |^{q-1}\) et
		\begin{equation}
			\| f \|_p=\left( \int| g |^{p(q-1)} \right)^{1/p}=\left( \int | g |^q \right)^{1/p}=\| g \|_q^{q/p}
		\end{equation}
		où nous avons utilisé le fait que \( p(q-1)=q\). La fonction \( f\) est donc bien dans \( L^p\). D'autre part,
		\begin{equation}
			\Phi_g(f)=\int f\bar g=\int g| g |^{q-2}\bar g=\int | g |^q=\| g \|_q^q.
		\end{equation}
		Donc
		\begin{equation}
			\frac{ | \Phi_g(f) | }{ \| f \|_p }=\| g \|_q^{q-\frac{ q }{ p }}=\| g \|_q
		\end{equation}
		où nous avons encore utilisé le fait que \( q-\frac{ q }{ p }=\frac{ q(p-1) }{ p }=1\).

		\spitem[Surjectif]

		Soit \( \ell\in (L^p)'\); c'est une application \( \ell\colon L^p\to \eC\) dont nous pouvons prendre la restriction à \( L^2\) parce que la proposition~\ref{PropIRDooFSWORl} nous indique que \( L^2\subset L^p\). Nous nommons \( \phi\colon L^2\to \eC\) cette restriction.

		\begin{subproof}

			\spitem[\( \phi\in (L^2)'\)]

			Nous devons montrer que \( \phi\) est continue pour la norme sur \( L^2\). Pour cela nous montrons que sa norme opérateur (subordonnée à la norme de \( L^2\) et non de \( L^p\)) est finie :
			\begin{equation}
				\sup_{f\in L^2}\frac{ | \phi(f) | }{ \| f \|_{2} }\leq \sup_{f\in L^2}\frac{ | \ell(f) | }{ \| f \|_p }<\infty.
			\end{equation}
			Nous avons utilisé l'inégalité de norme \( \| f \|_p\leq \| f \|_2\) de la proposition~\ref{PropIRDooFSWORl}\ref{ItemWSTooLcpOvXii}.

			\spitem[Utilisation du dual de \( L^2\)]

			Étant donné que \( L^2\) est un espace de Hilbert (lemme~\ref{LemIVWooZyWodb}) et que \( \phi\in (L^2)'\), le théorème~\ref{ThoQgTovL} nous donne un élément \( g\in L^2\) tel que \( \phi(f)=\Phi_g(f)\) pour tout \( f\in L^2\).

			Nous devons prouver que \( g\in L^q\) et que pour tout \( f\in L^p\) nous avons \( \ell(f)=\Phi_g(f)\).

			\spitem[\( g\in L^q\)]

			Nous posons \( f_n=g| g |^{q-2}\mtu_{| g |<n}\). Nous avons d'une part
			\begin{equation}    \label{EqEBUooOnlRHj}
				\Phi_g(f_n)=\int_0^1f_n\bar g=\int_{| g |<n}| g |^q.
			\end{equation}
			Et d'autre part comme \( f_n\in L^2\) nous avons aussi \( \phi(f_n)=\Phi_g(f_n)\) et donc
			\begin{subequations}
				\begin{align}
					0\leq \Phi(f_n)= \phi(f_n) & \leq \| \ell \|\| f_n \|_p                                   \\
					                           & =\| \ell \|\left( \int_{| g |<n}| g |^{(q-1)p} \right)^{1/p} \\
					                           & =\| \ell \|\left( \int_{| g |<n}| g |^q \right)^{1/p}.
				\end{align}
			\end{subequations}
			où nous avons à nouveau tenu compte du fait que \( p(q-1)=q\). En combinant avec \eqref{EqEBUooOnlRHj} nous trouvons
			\begin{equation}
				\int_{| g |<n}| g |^q\leq \| \ell \|\left( \int_{| g |<n}| g |^q \right)^{1/p},
			\end{equation}
			et donc
			\begin{equation}
				\left( \int_{| g |<n}| g |^{q} \right)^{1-\frac{1}{ p }}\leq \| \ell \|,
			\end{equation}
			c'est-à-dire
			\begin{equation}
				\Big( \int_{| g |<n}| g |^q \Big)^{1/q}\leq \| \ell \|.
			\end{equation}

			Si ce n'était pas encore fait nous nous fixons un représentant de la classe \( g\) (qui est dans \( L^2\)), et nous nommons également \( g\) ce représentant. Nous posons alors
			\begin{equation}
				g_n=| g |^q\mtu_{| g |<n}
			\end{equation}
			qui est une suite croissante de fonctions convergeant ponctuellement vers \( | g |^q\). Le théorème de Beppo-Levi~\ref{ThoRRDooFUvEAN} nous permet alors d'écrire
			\begin{equation}
				\lim_{n\to \infty} \int_{| q |<n}| g |^q=\int_{0}^1| g |^q.
			\end{equation}
			Mais comme pour chaque \( n\) nous avons \( \int_{| g |<n}| g |^q\leq \| \ell \|^q\), nous conservons l'inégalité à la limite et
			\begin{equation}
				\int_0^1| g |^q\leq \| \ell \|^q.
			\end{equation}
			Cela prouve que \( g\in L^p\).

			\spitem[\( \ell(f)=\Phi_g(f)\)]

			Soit \( f\in L^p\). En vertu de la densité de \( L^2\) dans \( L^p\) prouvée dans le corolaire~\ref{CorFZWooYNbtPz} nous pouvons considérer une suite \( (f_n)\) dans \( L^2\) telle que \( f_n\stackrel{L^p}{\longrightarrow}f\). Pour tout \( n\) nous avons
			\begin{equation}
				\ell(f_n)=\Phi_g(f_n).
			\end{equation}
			Mais \( \ell\) et \( \Phi_g\) étant continues sur \( L^p\) nous pouvons prendre la limite et obtenir
			\begin{equation}
				\ell(f)=\Phi_g(f).
			\end{equation}
		\end{subproof}
	\end{subproof}
\end{proof}

\begin{lemma}[\cite{MathAgreg}] \label{LemHNPEooHtMOGY}
	Soit \( (\Omega,\tribA,\mu)\) un espace mesuré fini. Soit \( g\in L^1(\Omega)\) et \( S\) fermé dans \( \eC\). Si pour tout \( E\in \tribA\) nous avons
	\begin{equation}
		\frac{1}{ \mu(E) }\int_Egd\mu\in S,
	\end{equation}
	alors \( g(x)\in S\) pour presque tout \( x\in \Omega\).
\end{lemma}

\begin{proof}
	Soit \( D=\overline{ B(a,r) }\) un disque fermé dans le complémentaire de \( S\) (ce dernier étant fermé, le complémentaire est ouvert). Posons \( E=g^{-1}(D)\). Prouvons que \( \mu(E)=0\) parce que cela prouverait que \( g(x)\in D\) pour seulement un ensemble de mesure nulle. Mais \( S^c\) pouvant être écrit comme une union dénombrable de disques fermés\footnote{Tout ouvert peut être écrit comme union dénombrable d'éléments d'une base de topologie par la proposition~\ref{DEFooLEHPooIlNmpi} et \( \eC\) a une base dénombrable de topologie par la proposition~\ref{PropNBSooraAFr}.}, nous aurions \( g(x)\in S^c\) presque nulle part.

	Vu que \( \frac{1}{ \mu(E) }\int_Ea=a\) nous avons
	\begin{subequations}
		\begin{align}
			\big| \frac{1}{ \mu(E) }gd\mu-a \big|=\big| \frac{1}{ \mu(E) }\int_E(g-a) \big|\leq  \frac{1}{ \mu(E) }\int_E| g-a |\leq\frac{1}{ \mu(E) }\mu(E)r=r.
		\end{align}
	\end{subequations}
	Donc
	\begin{equation}
		\frac{1}{ \mu(E) }\int_Egd\mu\in D,
	\end{equation}
	ce qui est une contradiction avec le fait que \( D\subset S^c\).
\end{proof}

Dans toute la partie d'analyse fonctionnelle, sauf mention du contraire, nous considérons dans \( L^p\) des fonctions à valeurs complexes, et donc les éléments du dual sont des applications linéaires continues à valeurs dans \( \eC\). La raison est que nous allons utiliser les résultats concernant \( L^p\) dans la partie sur les transformations de Fourier, tandis que les transformations de Fourier demandent naturellement de travailler sur les complexes.

\begin{theorem}[Théorème de représentation de Riesz, thème~\ref{THEMEooULGFooPscFJC}, \cite{MathAgreg,TLRRooOjxpTp,LRBWftc,ooRCYWooNAeaTA}]
	\label{ThoLPQPooPWBXuv}
	Soit un espace mesuré \( (\Omega,\tribA,\mu)\). Soit \( q\) tel que \( \frac{1}{ p }+\frac{1}{ q }=1\) avec la convention que \( q=\infty\) si \( p=1\). Alors l'application
	\begin{equation}
		\begin{aligned}
			\Phi\colon L^q & \to (L^p)'                \\
			\Phi_g(f)      & =\int_{\Omega}f\bar gd\mu
		\end{aligned}
	\end{equation}
	est une bijection isométrique\footnote{Pour rappel, la norme sur le dual est la norme opérateur \ref{DefNFYUooBZCPTr}.} dans les cas suivants :
	\begin{enumerate}
		\item       \label{ITEMooSQQBooWSFBmX}
		      si \( 1<p<\infty\) et \( (\Omega,\tribA,\mu)\) est un espace mesuré quelconque,
		\item       \label{ITEMooCQGJooOWzjoV}
		      si \( p=1\) et \( (\Omega,\tribA, \mu)\) est \( \sigma\)-fini.
	\end{enumerate}
\end{theorem}
\index{dual!de \( L^p\)}

\begin{proof}
	Par petits bouts.
	\begin{subproof}
		\spitem[\( \Phi\) est injective]
		Nous commençons par prouver que \( \Phi\) est injectif. Soient \( g,g'\in L^q\) tels que \( \Phi_g=\Phi_{g'}\). Alors pour tout \( f\in L^p\) nous avons
		\begin{equation}
			\int_{\Omega}f(\bar g-\bar g')d\mu=0.
		\end{equation}
		Soient des parties \( A_i\) de mesures finies telles que \( \Omega=\bigcup_{i=1}^{\infty}A_i\). Étant donné que \( \mu(A_i)\) est fini, nous avons \( \mtu_{A_i}\in L^p(\Omega)\) et donc
		\begin{equation}
			\int_{A_i}(\bar g-\bar g')d\mu=\int_{\Omega}\mtu_{A_i}(x)(\bar g-\bar g')(x)d\mu(x)=0.
		\end{equation}
		La proposition~\ref{PropRERZooYcEchc} nous dit alors que \( g-g'=0\) dans \( L^q(A_i)\). Pour chaque \( i\), la partie \( N_i=\{ x\in A_i\tq (g-g')(x)\neq 0 \}\) est de mesure nulle.

		Vu que \( \Omega\) est l'union de tous les \( A_i\), la partie de \( \Omega\) sur laquelle \( g-g'\) est non nulle est l'union des \( N_i\) et donc de mesure nulle parce que une réunion dénombrable de parties de mesure nulle est de mesure nulle. Donc \( g-g'=0\) presque partout dans \( \Omega\), ce qui signifie \( g-g'=0\) dans \( L^q(\Omega)\).

		\spitem[La suite]

		La partie difficile est de montrer que \( \Phi\) est surjective.

		Soit \( \phi\in L^p(\Omega)'\). Si \( \phi=0\), c'est bien dans l'image de \( \Phi\); nous supposons donc que non. Nous allons commencer par prouver qu'il existe une (classe de) fonction \( g\in L^1(\Omega)\) telle que \( \Phi_g(f)=\phi(f)\) pour tout \(f\in L^{\infty}(\Omega,\mu)\); nous montrerons ensuite que \( g\in L^q\) et que le tout est une isométrie.

		\spitem[Une mesure complexe]

		Si \( E\in\tribA\) nous notons \( \nu(E)=\phi(\mtu_E)\). Nous prouvons maintenant que \( \nu\) est une mesure complexe\footnote{Définition~\ref{DefGKHLooYjocEt}.} sur \( (\Omega,\tribA)\). Le seul point pas facile est de démontrer est l'additivité dénombrable. Il est déjà facile de voir que \( A\) et \( B\) sont disjoints, \( \nu(A\cup B)=\nu(A)+\nu(B)\). Soient ensuite des ensembles \( A_n\) deux à deux disjoints et posons \( E_k=\bigcup_{i\leq k}A_i\) pour avoir \( \bigcup_kA_k=\bigcup_kE_k\) avec l'avantage que les \( E_k\) soient emboîtés. Cela donne
		\begin{equation}
			\| \mtu_E-\mtu_{E_k} \|_p=\mu(E\setminus E_k)^{1/p},
		\end{equation}
		mais vu que \( 1\leq p<\infty\), avoir \( x_k\to 0\) implique d'avoir \( x_k^{1/p}\to 0\). Prouvons que \( \mu(E\setminus E_k)\to 0\). En vertu du lemme~\ref{LemPMprYuC} nous avons pour chaque \( k\) :
		\begin{equation}
			\mu(E\setminus E_k)=\mu(E)-\mu(E_k),
		\end{equation}
		et vu que \( E_k\to E\) est une suite croissante, le lemme~\ref{LemAZGByEs}\ref{ItemJWUooRXNPci}, sachant que \( \mu\) est une mesure « normale », donne
		\begin{equation}
			\lim_{n\to \infty} \mu(E_k)=\mu\big( \bigcup_kE_k \big).
		\end{equation}
		Donc effectivement \( \mu(E_k)\to \mu(E)\) et donc oui : \( \mu(E\setminus E_k)\to 0\). Jusqu'à présent nous avons
		\begin{equation}
			\lim_{k\to \infty} \| \mtu_E-\mtu_{E_k} \|_p=0,
		\end{equation}
		c'est-à-dire \( \mtu_{E_k}\stackrel{L^p}{\longrightarrow}\mtu_E\). La continuité de \( \phi\) sur \( L^p\) donne alors
		\begin{equation}
			\lim_{k\to \infty} \nu(E_k)=\lim_{k\to \infty} \phi(\mtu_{E_k})=\phi(\lim_{k\to \infty} \mtu_{E_k})=\phi(\mtu_E)=\nu(E).
		\end{equation}
		Par additivité finie de \( \nu\) nous avons
		\begin{equation}
			\nu(E_k)=\sum_{i\leq k}\nu(A_i)
		\end{equation}
		et en passant à la limite, \( \sum_{i=1}^{\infty}\nu(A_i)=\nu(\bigcup_{i}A_i)\). L'application \( \nu\) est donc une mesure complexe.

		\spitem[Mesure absolument continue]

		En prime, si \( \mu(E)=0\) alors \( \nu(E)=0\) parce que
		\begin{equation}
			\mu(E)=0\Rightarrow \| \mtu_E \|_p=0\Rightarrow \mtu_E=0\text{ (dans } L^p\text{)}\Rightarrow\phi(\mtu_E)=0
		\end{equation}

		\spitem[Utilisation de Radon-Nikodym]

		Nous sommes donc dans un cas où \( \nu\ll\mu\) et nous utilisons le théorème de Radon-Nikodym~\ref{ThoZZMGooKhRYaO} sous la forme de la remarque~\ref{RemSYRMooZPBhbQ} : il existe une fonction intégrable \( g\colon \Omega\to \eC\)\footnote{On peut écrire, pour utiliser de la notation compacte que \(  g\in L^1(\Omega,\eC)\).} telle que pour tout \( A\in\tribA\),
		\begin{equation}
			\nu(A)=\int_A\bar gd\mu.
		\end{equation}
		C'est-à-dire que
		\begin{equation}
			\phi(\mtu_A)=\int_A\bar gd\mu=\int_{\Omega}\bar g\mtu_Ad\mu.
		\end{equation}
		Nous avons donc exprimé \( \phi\) comme une intégrale pour les fonctions caractéristiques d'ensembles.

		\spitem[Pour les fonctions étagées]

		Par linéarité si \( f\) est mesurable et étagée nous avons aussi
		\begin{equation}
			\phi(f)=\int f\bar gd\mu=\Phi_g(f).
		\end{equation}

		\spitem[Pour \( f\in L^{\infty}(\Omega)\)]

		Une fonction \( f\in L^{\infty}\) est une fonction presque partout bornée. Nous supposons que \( f\) est presque partout bornée par \( M\). Par ailleurs cette \( f\) est limite uniforme de fonctions étagées : \( \| f_k-f \|_{\infty}\to 0\) en posant \( f_k=f\mtu_{| f |\leq k}\). Pour chaque \( k \) nous avons l'égalité
		\begin{equation}    \label{EqPDCJooGNjuAO}
			\Phi_g(f_k)=\phi(f_k).
		\end{equation}
		Par ailleurs la fonction \( f_k\bar g\) est majorée par la fonction intégrable \( M\bar g\) et le théorème de la convergence dominée~\ref{ThoConvDomLebVdhsTf} nous donne
		\begin{equation}
			\lim_{k\to \infty} \Phi_g(f_k)=\lim_{k\to \infty} \int f_k\bar g=\int f\bar g=\Phi_g(f).
		\end{equation}
		Et la continuité de \( \phi\) sur \( L^p\) couplée à la convergence \( f_k\stackrel{L^p}{\longrightarrow}f\) donne \( \lim_{k\to \infty} \phi(f_k)=\phi(f)\). Bref prendre la limite dans \eqref{EqPDCJooGNjuAO} donne
		\begin{equation}
			\Phi_g(f)=\phi(f)
		\end{equation}
		pour tout \( f\in L^{\infty}(\Omega)\).

		\spitem[La suite \ldots]

		Voici les prochaines étapes.
		\begin{itemize}
			\item Nous avons \( \int f\bar g=\phi(f)\) tant que \( f\in L^{\infty}\). Nous allons étendre cette formule à \( f\in L^p\) par densité. Cela terminera de prouver que notre application est une bijection.
			\item Ensuite nous allons prouver que \( \| \phi \|=\| \Phi_g \|\), c'est-à-dire que la bijection est une isométrie.
		\end{itemize}

		\spitem[De \( L^{\infty}\) à \( L^p\)]

		Soit \( f\in L^p\). Si nous avions une suite \( (f_n) \) dans \( L^{\infty}\) telle que \( f_n\stackrel{L^p}{\longrightarrow}f\) alors \( \lim \phi(f_n)=\phi(f)\) par continuité de \( \phi\). La difficulté est de trouver une telle suite de façon à pouvoir permuter l'intégrale et la limite :
		\begin{equation}    \label{EqLYYAooUQnbfV}
			\lim_{n\to \infty} \int_{\Omega}f_n\bar g=\int_{\Omega}\lim_{n\to \infty} f_n\bar g=\int_{\Omega}f\bar g=\Phi_g(f).
		\end{equation}
		Nous allons donc maintenant nous atteler à la tâche de trouver \( f_n\in L^{\infty}\) avec \( f_n\stackrel{L^p}{\longrightarrow}f\) et telle que \eqref{EqLYYAooUQnbfV} soit valide.

		Nous allons d'abord supposer que \( f\in L^p\) est positive à valeurs réelles. Nous avons alors par le théorème \ref{THOooXHIVooKUddLi} qu'il existe une suite croissante de fonction étagées (et donc \( L^{\infty}\)) telles que \( f_n\to f\) ponctuellement. De plus étant donné que \( | f_n |\leq | f |\), la proposition~\ref{PropBVHXycL} nous dit que \( f_n\stackrel{L^p}{\longrightarrow}f\). Pour chaque \( n\) nous avons
		\begin{equation}
			\int_{\Omega}f_n\bar g=\phi(f_n).
		\end{equation}
		Soit \( g^+\) la partie réelle positive de \( \bar g\). Alors nous avons la limite croissante ponctuelle \( f_ng^+\to fg^+\) et le théorème de la convergence monotone~\ref{ThoRRDooFUvEAN} nous permet d'écrire
		\begin{equation}
			\lim_{n\to \infty} \int f_ng^+=\int fg^+.
		\end{equation}
		Faisant cela pour les trois autres parties de \( \bar g\) nous avons prouvé que si \( f\in L^p\) est réelle et positive,
		\begin{equation}
			\int f\bar g=\phi(f),
		\end{equation}
		c'est-à-dire que \( \Phi_g(f)=\phi(f)\).

		Refaisant le tout pour les trois autres parties de \( f\) nous montrons que
		\begin{equation}
			\Phi_g(f)=\phi(f)
		\end{equation}
		pour tout \( f\in L^p(\Omega)\). Nous avons donc égalité de \( \phi\) et \( \Phi_g\) dans \(  (L^p)' \) et donc bijection entre \( (L^p)'\) et \( L^q\).

		\spitem[Isométrie : mise en place]

		Nous devons prouver que cette bijection est isométrique. Soit \( \phi\in (L^p)'\) et \( g\in L^q\) telle que \( \Phi_g=\phi\). Il faut prouver que
		\begin{equation}
			\| g \|_q=\| \phi \|_{(L^p)'}.
		\end{equation}

		\spitem[ \( \| \phi \|\leq \| g \|_q\) ]

		Nous savons que \( \phi(f)=\int f\bar g\), et nous allons écrire la définition de la norme dans \( (L^p)'\) :
		\begin{subequations}
			\begin{align}
				\| \phi \|_{(L^p)'} & =\sup_{\| f \|_p=1}\big| \phi(f) \big|                   \\
				                    & =\sup| \int f\bar g |                                    \\
				                    & \leq\sup\underbrace{\int| f\bar g |}_{=\| f\bar g \|_1}.
			\end{align}
		\end{subequations}
		Il s'agit maintenant d'utiliser l'inégalité de Hölder~\ref{ProptYqspT} :
		\begin{equation}
			\| \phi \|\leq \sup_{\| f \|_p=1}\| f \|_p\| \bar g \|_q=\| g \|_q.
		\end{equation}

		L'inégalité dans l'autre sens sera démontrée en séparant les cas \( p=1\) et \( 1<p<\infty\).

		\spitem[Si \( p=1\), une formule]
		Si \( E\) est un ensemble mesurable de mesure finie, alors
		\begin{equation}
			| \int_Egd\mu |=\big| \phi(\mtu_E) \big|.
		\end{equation}
		Mais le fait que \( \mu(E)<\infty\) donne que \( \mtu_E\in L^1(\Omega)\). Donc \( \mtu_E\in L^{\infty}\cap L^1\); nous pouvons alors écrire \( \phi(\mtu_E)=\int_{\Omega}\mtu_E\bar gd\mu\) et donc
		\begin{equation}    \label{EqUPCTooJvoKKI}
			| \int_{\Omega}\mtu_E\bar gd\mu |=|\int_E\bar gd\mu |=\big| \phi(\mtu_E) \big|\leq \| \phi \|_{(L^1)'}\| \mtu_E \|_1=\| \phi \|\mu(E).
		\end{equation}
		Nous écrivons cela dans l'autre sens :
		\begin{equation}
			\| \phi \|\geq \frac{1}{ \mu(E) }| \int_{\Omega}\mtu_E\bar gd\mu |=| \frac{1}{ \mu(E) }\int_E\bar gd\mu |.
		\end{equation}
		Si nous prenons \( S=\{ t\in \eC\tq | t |\leq \| \phi \| \}\), c'est un fermé vérifiant que
		\begin{equation}        \label{EQooMRLGooYPEjUo}
			\frac{1}{ \mu(E) }\int_E\bar gd\mu\in S.
		\end{equation}

		Voilà une petite formule qui va nous aider à utiliser le lemme \ref{LemHNPEooHtMOGY}. Nous ne pouvons cependant pas l'utiliser immédiatement parce que l'appartenance \eqref{EQooMRLGooYPEjUo} n'est vraie que pour les parties de mesure finie.

		\spitem[Si \( p=1\), conclusion\cite{MonCerveau}]

		Pour utiliser le lemme~\ref{LemHNPEooHtMOGY}, nous utilisons l'hypothèse que \( \Omega\) est \( \sigma\)-fini. Soient des mesurables \( A_i\) de mesure fine tels que \( \bigcup_{i\in \eN}A_i=\Omega\).

		Pour chaque \( i\) nous considérons la restriction \( g_i\colon A_i\to \eC\) de \( g\) à \( A_i\). Par le point précédent, elle vérifie
		\begin{equation}
			\frac{1}{ \mu(A_i) }\int_{A_i}\bar g_id\mu=\frac{1}{ \mu(A_i) }\int_{A_i}\bar gd\mu\in S.
		\end{equation}
		En appliquant le lemme \ref{LemHNPEooHtMOGY} à l'espace restreint \( (A_i,\tribA_i,\mu_i)\), nous concluons \( \bar g_i\in S\) presque partout, ce qui signifie que \( \| g_i \|_{\infty}\in S\). Nous en concluons que
		\begin{equation}
			\| g_i \|_{\infty}\leq \| \phi \|
		\end{equation}
		où, dans ce contexte, \( \| g_i \|_{\infty}\) signifie \( \sup_{x\in A_i}| g_i(x) |\).

		Nous avons alors
		\begin{equation}
			\| g \|_{\infty}=\sup_{x\in \Omega}| g(x) |=\sup_{i\in \eN}\| g_i \|_{\infty}\leq \| \phi \|.
		\end{equation}
		Une petite justification pour cela ? Prenons une suite \( x_k\) telle que \( | g(x_k) |\to \| g \|_{\infty}\). Vu que les \( A_i\) recouvrent \( \Omega\), existe un naturel \( i(k)\) tel que \( x_k\in A_{i(k)}\). Nous avons alors
		\begin{equation}
			| g(x_k) |\leq \| g_{i(k)} \|_{\infty}\leq \| \phi \|.
		\end{equation}
		Cela pour conclure que \( g\in L^{\infty}\).

		Notons que cet argument ne tient pas avec \( p> 1\) parce que l'équation \eqref{EqUPCTooJvoKKI} terminerait sur \( \| \phi \|\mu(E)^{1/p}\). Du coup l'ensemble \( S\) à prendre serait \( S=\{ t\in \eC\tq | t |\leq \| \phi \|\mu(E)^{1/p-1} \}\) et nous sommes en dehors des hypothèses du lemme parce qu'il n'y a pas d'ensemble \emph{indépendant} de \( E\) dans lequel l'intégrale \( \frac{1}{ \mu(E) }\int_{E}\bar gd\mu\) prend ses valeurs.

		\spitem[\( 1<p<\infty\)]

		La fonction
		\begin{equation}
			\alpha(x)=\begin{cases}
				\frac{ g(x) }{ | g(x) | } & \text{si } g(x)\neq 0 \\
				1                         & \text{si } g(x)=0
			\end{cases}
		\end{equation}
		a la propriété de faire \( \alpha g=| g |\) en même temps que \( | \alpha(x) |=1\) pour tout \( x\). Nous définissons
		\begin{equation}
			E_n=\{ x\tq | g(x) |\leq n \}
		\end{equation}
		et
		\begin{equation}
			f_n=\mtu_{E_n}| g^{q-1} |\alpha.
		\end{equation}
		Ce qui est bien avec ces fonctions c'est que\footnote{C'est ici que nous utilisons le lien entre \( p\) et \( q\). En l'occurrence, de \( 1/p+1/q=1\) nous déduisons \( q(p-1)=p\).}
		\begin{equation}
			| f_n |^p=| g^{p(q-1)} || \alpha |^p=| g |^q
		\end{equation}
		sur \( E_n\). Dans \( E_n\) nous avons \( | f_n |=| g^{q-1} |\leq n^{q-1}\) et dans \( E_n\) nous avons \( f_n=0\). Au final, \( f_n\in L^{\infty}\). Par ce que nous avons vu plus haut, nous avons alors
		\begin{equation}
			\phi(f_n)=\Phi_g(f_n).
		\end{equation}
		Par ailleurs,
		\begin{equation}
			f_n\bar g=\mtu_{E_n}| g^{q-1} |\frac{ g }{ | g | }\bar g,
		\end{equation}
		donc\quext{Dans \cite{MathAgreg}, cette équation arrive sans modules, ce qui me laisse entendre que \( \phi(f_n)\) est réel et positif pour pouvoir écrire que \( \phi(f_n)\leq \| \phi \|\| f_n \|_p\), mais je ne comprends pas pourquoi.}
		\begin{subequations}
			\begin{align}
				\left|\int_{E_n}| g |^qd\mu\right| & =|\int_{\Omega}f_n\bar gd\mu|                       \\
				                                   & =|\phi(f_n)|                                        \\
				                                   & \leq \| \phi \|\| f_n \|_p                          \\
				                                   & =\| \phi \|\left( \int_{E_n}| f_n |^p \right)^{1/p} \\
				                                   & =\| \phi \|\left( \int_{E_n}| g |^q \right)^{1/p}.
			\end{align}
		\end{subequations}
		Nous avons de ce fait une inégalité de la forme \( A\leq \| \phi \|A^{1/p}\) et donc aussi \( A^{1/p}\leq \| \phi \|^{1/p}A^{1/p^2}\), et donc \( A\leq \| \phi \|\| \phi \|^{1/p}A^{1/p^2}\). Continuant ainsi à injecter l'inégalité dans elle-même, pour tout \( k\in \eN\) nous avons :
		\begin{equation}
			\left| \int_{E_n}| g |^qd\mu \right| \leq\| \phi \|^{1+\frac{1}{ p }+\cdots+\frac{1}{ p^k }}\left( \int_{E_n}| g |^qd\mu \right)^{1/p^k}.
		\end{equation}
		Nous pouvons passer à la limite \( k\to \infty\). Sachant que \( p>1\) nous savons \( A^{1/k}\to 1\) et
		\begin{equation}
			1+\frac{1}{ p }+\cdots+\frac{1}{ p^k }\to\frac{ p }{ p-1 }=q.
		\end{equation}
		Nous avons alors
		\begin{equation}
			\int_{E_n}| g |^qd\mu\leq \| \phi \|^q.
		\end{equation}
		L'intégrale s'écrit tout aussi bien sous la forme \( \int_{\Omega}| g  |^q\mtu_{E_n}\). La fonction dans l'intégrale est une suite croissante de fonctions mesurables à valeurs dans \( \mathopen[ 0 , \infty \mathclose]\). Nous pouvons alors permuter l'intégrale et la limite \( n\to \infty\) en utilisant la convergence monotone (théorème~\ref{ThoRRDooFUvEAN}) qui donne alors \( \int_{\Omega}| g |^q\leq \| \phi \|^q\) ou encore
		\begin{equation}
			\| g \|_q\leq \| \phi \|.
		\end{equation}

		Ceci achève de prouver que l'application \( \phi\mapsto \Phi_g\) est une isométrie, et donc le théorème.
	\end{subproof}
\end{proof}

\begin{theorem}     \label{THOooXMVTooBAbyvr}
	Soit un espace mesuré \( (\Omega,\tribA,\mu)\). Pour \( f,g\in\Fun(\Omega,\eC)\), nous écrivons
	\begin{equation}
		\Phi_g(f)=\int_{\Omega}f\bar g
	\end{equation}
	pour toutes les combinaisons de \( f\) et \( g\) pour lesquelles l'intégrale a un sens.
	\begin{enumerate}
		\item   \label{ITEMooNCVEooTyNsoJ}
		      Si \( 1<p<\infty\), alors \( L^p(\Omega,\tribA,\mu)\) est réflexif\footnote{Définition \ref{PROPooMAQSooCGFBBM}.}.
		\item   \label{ITEMooTQDJooFShTiA}
		      Si \( (\Omega,\tribA,\mu)\) est \( \sigma\)-finie, alors
		      \begin{enumerate}
			      \item\label{ITEMooHMMZooMQxWgB}
			      L'application \( \Phi\colon L^{\infty}(\Omega,\tribA,\mu)\to L^1(\Omega,\tribA,\mu)'\) est une bijection isométrique.
			      \item       \label{ITEMooBFFZooNxoHER}
			            L'application \( \Phi\colon L^1(\Omega,\tribA,\mu)\to L^{\infty}(\Omega,\tribA,\mu)'\) est une injection isométrique.
		      \end{enumerate}
	\end{enumerate}
	Note : nous verrons dans \ref{PROPooXNRRooUdgFPr} que l'application \( \Phi\colon L^1(\eR^d)\to L^{\infty}(\eR^d)'\) n'est pas surjective.
\end{theorem}
\index{dual!de \( L^p(\Omega)\)}

\begin{proof}
	En plusieurs parties, en notant toujours \( p\) et \( q\) les exposants conjugués, c'est-à-dire \( \frac{1}{ p }+\frac{1}{ q }=1\).
	\begin{subproof}
		\spitem[Pour \ref{ITEMooNCVEooTyNsoJ}]

		Nous considérons les applications
		\begin{equation}
			\begin{aligned}
				\alpha_1\colon L^p & \to (L^q)'            \\
				\alpha_1(g)f       & =\int_{\Omega}f\bar g
			\end{aligned}
		\end{equation}
		et
		\begin{equation}
			\begin{aligned}
				\alpha_2\colon L^q & \to (L^p)'             \\
				\alpha_2(g)f       & =\int_{\Omega}g\bar f.
			\end{aligned}
		\end{equation}
		Ce sont des bijections linéaires par le théorème de représentation de Riesz \ref{ThoLPQPooPWBXuv}\footnote{Notez que la conjugaison complexe dans \( \alpha_2\) n'est pas à la même place que dans \( \alpha_1\). L'application \( \alpha_1\) est exactement la même que le \( \Phi\) de représentation de Riesz alors que \( \alpha_2\) est un tout petit peu modifiée. La raison de ce changement n'est pas très profonde : c'est seulement pour que les choses tombent juste à la fin.}. L'espace  \( L^p\) est de Banach par le théorème de Riesz-Fischer \ref{ThoGVmqOro}; le lemme \ref{LEMooWEMFooEHIaxY} s'applique donc. Nous considérons alors les applications correspondantes
		\begin{equation}
			\beta_1\colon (L^p)'\to (L^q)''
		\end{equation}
		et
		\begin{equation}
			\beta_2\colon (L^q)'\to (L^p)'',
		\end{equation}
		qui sont également des bijections linéaires. Nous allons montrer que la bijection
		\begin{equation}
			\beta_2\circ \alpha_1\colon L^p\to (L^p)''
		\end{equation}
		est l'application \( J\) de la définition \ref{PROPooMAQSooCGFBBM}.

		Soient \( f\in L^p\) et \( \varphi\in (L^p)'\). Il existe \( g\in L^q\) tel que \( \varphi=\alpha_2(g)\). Nous pouvons donc calculer d'une part, en développant \( \beta_2\) par sa définition \eqref{EQooTSVHooEQGuNw},
		\begin{equation}
			(\beta_2\circ \alpha_1)(f)\varphi=\beta_2\big( \alpha_1(f) \big)\varphi
			=\alpha_1(f)\big( \alpha_2^{-1}(\varphi) \big)
			=\alpha_1(f)(g)
			=\int_{\Omega}g\bar f,
		\end{equation}
		et d'autre part
		\begin{equation}
			J(f)\varphi=\varphi(f)
			=\alpha_2(g)f
			=\int_{\Omega}g\bar f.
		\end{equation}
		\spitem[Pour \ref{ITEMooHMMZooMQxWgB}]
		Il s'agit du théorème \ref{ThoLPQPooPWBXuv}\ref{ITEMooCQGJooOWzjoV}.
		\spitem[Pour \ref{ITEMooBFFZooNxoHER}\cite{MonCerveau}]
		Il nous reste à couvrir le cas de \( (L^{\infty})'\). Pour \( g\in L^1\) nous prouvons que \( \Phi_g\in (L^{\infty})'\).

		\begin{subproof}
			\spitem[\( \Phi_g(f)\) est bien définie]
			Nous prouvons d'abord que si \( f\in L^{\infty}\), alors l'intégrale \( \int_{\Omega}f\bar g\) est bien définie. Par définition du supremum essentiel\footnote{Voir les définitions \ref{DEFooIQOOooLpJBqi} et \ref{DEFooXUKHooXYrlYq}.}, il existe \( M>0\) tel que \( | f(x) |<M\) pour tout \( x\) hors d'une partie \( A\) de mesure nulle. Nous avons alors
			\begin{equation}
				\int_{\Omega}|f\bar g|=\int_{\Omega\setminus A}| f\bar g |\leq M\int_{\Omega\setminus A}| f |= M\int_{\Omega}| f |<\infty.
			\end{equation}
			\spitem[\( \Phi_g\) est continue]
			Soit une suite \( f_k\stackrel{L^{\infty}}{\longrightarrow}f\) ainsi que \( g\in L^1\). Pour chaque \( k\), il existe une partie de mesure nulle \( A_k\) et un nombre \( M_k=\| f_k \|_{L^{\infty}}\) tel que \( | f_k(x) |<\| f_k \|_{L^{\infty}}\) pour tout \( x\) hors de \( A_k\). Nous avons alors
			\begin{equation}
				| \Phi_g(f_k) |\leq \int_{\Omega\setminus A_k}| f_k\bar g |d\mu\leq \| f_k \|_{L^{\infty}}\int_{\Omega\setminus A_k}| g |\leq \| f_k \|_{L^{\infty}}\| g \|_1.
			\end{equation}
			Vu que par hypothèse \( f_k\to 0\) dans \( L^{\infty}\), nous avons \( \| f_k \|_{L^{\infty}}\to 0\), et donc aussi
			\begin{equation}
				|\Phi_g(f_k)|\to 0.
			\end{equation}
		\end{subproof}
	\end{subproof}
\end{proof}

\begin{normaltext}
	Les gens qui n'ont peur d'aucun abus de notations écrivent le théorème \ref{THOooXMVTooBAbyvr}\ref{ITEMooHMMZooMQxWgB} en disant simplement que \( L^1=L^{\infty}\) et le démontrent de la façon suivante. Le théorème \ref{ThoLPQPooPWBXuv}\ref{ITEMooSQQBooWSFBmX} nous indique que
	\begin{equation}        \label{EQooJVDEooNGYEtg}
		(L^p)'=L^q.
	\end{equation}
	Vu que \( 1<p<\infty\), nous avons aussi \( 1<q<\infty\) et donc \( (L^q)'=L^p\). En prenant le dual des deux côtés de \eqref{EQooJVDEooNGYEtg},
	\begin{equation}
		(L^p)''=(L^q)'=L^p.
	\end{equation}
	À ce moment, un second abus vient en aide et nous disons qu'un espace \( V\) est réflexif quand \( V=V''\). Et voilà.
\end{normaltext}


\begin{proposition} \label{PropUKLZZZh}
	Soit \( f\in L^p(\Omega)\) telle que
	\begin{equation}
		\int_{\Omega}f\varphi=0
	\end{equation}
	pour tout \( \varphi\in \swD(\Omega)\). Alors \( f=0\) presque partout.
\end{proposition}

\begin{proof}
	Nous considérons la forme linéaire \( \Phi_f\in (L^q)'\) donnée par
	\begin{equation}
		\begin{aligned}
			\Phi_f\colon L^p & \to \eC                 \\
			u                & \mapsto \int_{\Omega}fu
		\end{aligned}
	\end{equation}
	Par hypothèse cette forme est nulle sur la partie dense \(  C^{\infty}_c(\Omega)\). Si \( (\varphi_n)\) est une suite dans \(  C^{\infty}_c(\Omega)\) convergente vers \( u\) dans \( L^p\), nous avons pour tout \( n\) que
	\begin{equation}
		0=\Phi_f(\varphi_n)
	\end{equation}
	En passant à la limite, nous voyons que \( \Phi_f\) est la forme nulle. Elle est donc égale à \( \Phi_0\). La partie « unicité » du théorème de représentation de Riesz~\ref{ThoLPQPooPWBXuv} nous indique alors que \( f=0\) dans \( L^p\) et donc \( f=0\) presque partout.
\end{proof}

\begin{proposition} \label{PropLGoLtcS}
	Si \( f\in L^1_{loc}(I)\) est telle que
	\begin{equation}
		\int_If\varphi'=0
	\end{equation}
	pour tout \( \varphi\in  C^{\infty}_c(I)\), alors il existe une constante \( C\) telle que \( f=C\) presque partout.
\end{proposition}

\begin{proof}
	Soit \( \psi\in C^{\infty}_c(I)\) une fonction d'intégrale \( 1\) sur \( I\). Si \( w\in C^{\infty}_c(I)\) alors nous considérons la fonction
	\begin{equation}
		h=w-\psi\int_Iw,
	\end{equation}
	qui est dans \(  C^{\infty}_c(I)\) et dont l'intégrale sur \( I\) est nulle. Par la proposition~\ref{PropHFWNpRb}, la fonction \( h\) admet une primitive dans \(  C^{\infty}_c(I)\); et nous notons \( \varphi\) cette primitive. L'hypothèse appliquée à \( \varphi\) donne
	\begin{equation}
		0=\int_If\varphi'=\int_If\left( w-\psi\int_Iw \right)=\int_Ifw-\underbrace{\left( \int_If(x)\psi(x)dx \right)}_C\left( \int_Iw(y)dy \right)=\int_Iw(f-C).
	\end{equation}
	L'annulation de la dernière intégrale implique par la proposition~\ref{PropUKLZZZh} que \( f-C=0\) dans \( L^2\), c'est-à-dire \( f=C\) presque partout.
\end{proof}

\begin{proposition}[\cite{BIBooFDGQooYferue}]     \label{PROPooXNRRooUdgFPr}
	L'application
	\begin{equation}
		\begin{aligned}
			\Phi\colon L^1(\eR^d) & \to L^{\infty}(\eR^d)'        \\
			\Phi_g(f)             & =\int_{\Omega}f\bar gd\lambda
		\end{aligned}
	\end{equation}
	n'est pas surjective.
\end{proposition}

\begin{proof}
	Nous allons construire un élément de \( (L^{\infty})'\) qui n'est pas dans l'image de \( L^1\).

	\begin{subproof}
		\spitem[Mise en place du décor]
		Nous considérons l'espace vectoriel normé \( \big( L^{\infty}(\eR^d), N_{\infty} \big)\) défini en \ref{DEFooIQOOooLpJBqi} et \ref{DEFooXUKHooXYrlYq}. Dedans, nous considérons le sous-espace \( D\) des classes des fonctions dans \( \swD(\eR^d)\); nous aurions très envie de le noter \( \big( \swD(\eR^d), N_{\infty} \big)\), mais nous allons le noter \( \big( D,N_{\infty} \big)\) parce que \( \swD\) est un espace de fonctions tandis que nous considérons un espace de classes de fonctions.

		Si \( f\in D\), alors \( f\) est une classe de fonctions contenant un unique représentant continu\footnote{Existence parce que les éléments de \( D\) sont des classes d'éléments de \( \swD\) qui sont \(  C^{\infty}\). Unicité par la proposition \ref{PropNCMToWI}.}; nous le notons \( \tilde f\).

		\spitem[Une première fonctionnelle]

		Avec ça nous posons\footnote{Dans \cite{BIBooFDGQooYferue}, l'auteur ne définit pas \( L^{\infty}\) comme un espace de classes de fonctions, et ces complications disparaissent.}
		\begin{equation}
			\begin{aligned}
				\phi_0\colon D & \to \eC              \\
				f              & \mapsto \tilde f(0).
			\end{aligned}
		\end{equation}
		\spitem[\( \phi_0\) est linéaire]
		Ça ne devrait pas poser de problèmes.
		\spitem[\( \phi_0\) est continue]
		Soit \( f_k\stackrel{N_{\infty}}{\longrightarrow}0\). Nous devons prouver que \( \tilde f_k(0)\stackrel{\eC}{\longrightarrow}0\). Supposons le contraire et considérons \( \epsilon>0\) ainsi que \( k\) tels que \( \tilde f_k(0)>\epsilon\).

		Par continuité de \( \tilde f_k\), il existe un \( \delta>0\) tel que \( \tilde f_k(x)>\epsilon/2\) pour tout \( x\in B(0,\delta)\). Avec cela nous avons \( N_{\infty}(f_k)>\epsilon/2\), et une impossibilité d'avoir\footnote{Je vous laisse emballer ce raisonnement dans «si pour tout \( N\), il existe \( k>N\) tel que».} \( f_k\stackrel{N_{\infty}}{\longrightarrow}0\).
		\spitem[\( \phi_0\) est de norme finie]
		C'est parce qu'elle est continue.
		\spitem[Utilisation de Hahn-Banach]
		Le théorème de Hahn-Banach \ref{THOooTZSSooBKfxXE} donne une extension continue
		\begin{equation}
			\phi\colon L^{\infty}(\eR^d)\to \eC.
		\end{equation}
		Cela est un élément de \( (L^{\infty})'\) et nous allons montrer qu'il n'est pas dans \( \Phi(L^1)\).
		\spitem[Par l'absurde]
		Supposons qu'il existe \( u\in L^1(\eR^d)\) telle que \( \phi=\Phi_u\). Pour tout \( f\in L^{\infty}\), nous avons
		\begin{equation}
			\phi(f)=\Phi_u(f)=\int_{\eR^d}f\bar u.
		\end{equation}
		Si \( f\) est la classe d'une fonction de \( \swD(\eR^d)\) s'annulant en \( 0\), alors \( \phi(f)=0\). De telles fonctions non identiquement nulles existent par le lemme \ref{LEMooFFPVooDKGUAp}\footnote{Ce lemme donne un exemple \( f\) sur \( \eR\). Si vous voulez vraiment un exemple dans \( \eR^d\), prenez \( g(x)=f(\| x \|)\).}.


		Soient \( \varphi\in \swD(\eR^d)\), un compact \( K\) ne contenant pas \( 0\) et \( \chi_K\) sa fonction indicatrice. Nous avons \( \chi_K\varphi\in \mL^{\infty}(\eR^d)\), et \( (\chi_K\varphi)(0)=0\). En passant aux classes, \( \phi\big( [\chi_Kf] \big)=0\). Nous avons :

		\begin{equation}
			0=\int_{\eR^d}u\overline{ (\chi_K\varphi) }=\int_{\eR^d}\overline{ \varphi }(u\overline{ \chi_K }).
		\end{equation}
		Vu que cela est valable pour tout \( \varphi\in \swD(\eR^d)\), la proposition \ref{PropUKLZZZh} dit que \( u\chi_K=0\) presque partout.

		\spitem[Le coup du compact]

		Soit une suite de compacts \( K_n\) recouvrant \( \eR^d\setminus\{ 0 \}\). Par exemple
		\begin{equation}
			K_n=\overline{ B(0,n) }\setminus B(0,\frac{1}{ n }).
		\end{equation}
		Pour chacun des \( K_n\), nous avons \( u\chi_{K_n}=0\) presque partout. Il existe donc une partie de mesure nulle \( N_k\) telle que \( u\chi_{K_n}\) est nulle à part sur \( N_k\). Au total, \( u\) est non nulle seulement sur \( \bigcup_{k}N_k\) et peut-être en \( 0\).

		Bref, \( u\) est non nulle sur une partie de mesure nulle par le lemme \ref{LemIDITgAy}.

		\spitem[Conclusion]

		La fonction \( u\) est nulle presque partout. Donc \( \phi=\Phi_u=0\). Nous savons pourtant que \( \phi\) n'est pas nulle parce qu'il existe des fonctions dans \( \swD(\eR^d)\) qui ne s'annulent pas en \( 0\).

		Cela est une contradiction. Donc \( \phi\) n'est pas dans l'image de \( \Phi\) tout en étant dans \( L^{\infty}(\eR^d)'\).

	\end{subproof}
\end{proof}

Dans \cite{ooHGADooNGZnbt}, il est dit que « la preuve [du lemme suivant], un peu fastidieuse mais en rien ingénieuse, est laissée en exercice ». La preuve est donc de moi; elle est un tout petit peu ingénieuse mais en rien fastidieuse. J'espère ne pas m'être trompé et me demande bien ce que l'auteur avait en tête. Ma preuve s'appuie sur la proposition \ref{PROPooLIGIooPrHYlb} dont la preuve ne me paraît pas non plus «fastidieuse mais en rien ingénieuse».

\begin{lemma}[\cite{ooHGADooNGZnbt,MonCerveau}]        \label{LEMooLDQRooEGWDlm}
	Soit \( r>0\). Il existe \( \delta>0\) tel que pour tout \( s,t\in \eC\) vérifiant \( | s |\leq 1\), \( | t |\leq 1\) et \( | s-t |\geq r\) nous ayons
	\begin{equation}
		\left| \frac{ s+t }{ 2 } \right|^p\leq (1-\delta)\frac{ | s |^p+| t |^p }{2}.
	\end{equation}
\end{lemma}

\begin{proof}
	Soit \( r>0\). La partie de \( \eC^2\) donnée par
	\begin{equation}
		D=\{ (s,t)\in \eC^2\tq | s |\leq 1, | t |\leq 1,| s-t |\geq r \}
	\end{equation}
	est compacte. En effet elle est bornée (par la sphère de rayon \( \sqrt{ 2 }\)) et fermée comme intersection de fermés\footnote{Lemme \ref{LemQYUJwPC} suivit du théorème de Borel-Lebesgue \ref{ThoXTEooxFmdI}.}. Nous considérons la fonction \( \Delta\colon D\to \eR\) donnée par
	\begin{equation}
		\left| \frac{ s+t }{2} \right|^p=\Delta(s,t)\frac{ | s |^p+| t |^p }{2}.
	\end{equation}
	Si vous voulez une expression explicite,
	\begin{equation}
		\Delta(s,t)=\frac{ 2^{p-1}| s+t |^p }{ | s |^p+| t |^p }.
	\end{equation}
	Cela est bien défini et continu sur \( D\) parce que le complémentaire \( D^c\) (qui est ouvert) contient \( (0,0)\) et donc aussi un voisinage de \( (0,0)\).

	La proposition \ref{PROPooLIGIooPrHYlb} nous dit que la fonction \( z\mapsto | z |^p\) est strictement convexe. En prenant la définition \ref{DEFooKCFPooLwKAsS} de la stricte convexité avec \( \theta=\frac{ 1 }{2}\), nous trouvons que
	\begin{equation}
		\Delta(s,t)<1
	\end{equation}
	pour tout \( (s,t)\in D\). Vu que par ailleurs \( \Delta\) est une fonction continue sur le compact \( D\), elle atteint un minimum dans \( D\). Soit \( \Delta_0\) ce minimum qui vérifie forcément \( \Delta_0<1\).

	En posant \( 1-\delta=\Delta_0\) nous avons le résultat.
\end{proof}
