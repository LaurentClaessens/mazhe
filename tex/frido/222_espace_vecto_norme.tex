% This is part of Le Frido
% Copyright (c) 2008-2022
%   Laurent Claessens
% See the file fdl-1.3.txt for copying conditions.

%+++++++++++++++++++++++++++++++++++++++++++++++++++++++++++++++++++++++++++++++++++++++++++++++++++++++++++++++++++++++++++
\section{Applications multilinéaires}
%+++++++++++++++++++++++++++++++++++++++++++++++++++++++++++++++++++++++++++++++++++++++++++++++++++++++++++++++++++++++++++

\begin{definition}[Application multilinéaire]       \label{DefFRHooKnPCT}
	Une application $T: \eR^{m_1}\times \ldots \times\eR^{m_k}\to\eR^p $ est dite \defe{\( k\)-linéaire}{application!multilinéaire} si pour tout $X=(x_1, \ldots,x_k)$ dans $ \eR^{m_1}\times \ldots \times\eR^{m_k}$ les applications $x_i\mapsto T(x_1, \ldots, x_i,\ldots,x_k)$ sont linéaires pour tout $i$ dans $\{1,\ldots,k\}$, c'est-à-dire
	\begin{equation}
		\begin{aligned}[]
			T(\cdot,x_2, \ldots, x_i,\ldots,x_k)     & \in \mathcal{L}(\eR^{m_1}, \eR^p), \\
			T(x_1,\cdot, \ldots, x_i,\ldots,x_k)     & \in \mathcal{L}(\eR^{m_2}, \eR^p), \\
			                                         & \vdots                             \\
			T(x_1, \ldots, x_i,\ldots,x_{k-1},\cdot) & \in \mathcal{L}(\eR^{m_k}, \eR^p). \\
		\end{aligned}
	\end{equation}
	En particulier lorsque $k=2$, nous parlons d'applications \defe{bilinéaires}{bilinéaire}. Vous pouvez deviner ce que sont les applications \emph{tri}linéaire ou \emph{quadri}linéaire.
\end{definition}

L'ensemble des applications $k$-linéaires de $ \eR^{m_1}\times \ldots \times\eR^{m_k}$ dans $\eR^p$ est noté $\mathcal{L}(\eR^{m_1}\times \ldots \times\eR^{m_k}, \eR^p)$ ou $\mathcal{L}(\eR^{m_1}, \ldots,\eR^{m_k}; \eR^p)$.

\begin{example}
	Soit $A$ une matrice avec $m$ lignes et $n$ colonnes. L'application bilinéaire de $\eR^m\times \eR^n$ dans $\eR$ associée à $A$ est définie par
	\[
		T_A(x,y)= x^TAy=\sum_{i,j}a_{i,j}x_i y_j, \qquad \forall x\in \eR^m, \, y \in \eR^n.
	\]
\end{example}

Nous énonçons la proposition suivante dans le cas d'espaces vectoriels normés\footnote{Sans hypothèses sur la dimension.} parce que nous allons l'utiliser dans ce cas, mais le cas particulier \( E_i=\eR^{m_i}\) et \( F=\eR^p\) est important.
\begin{proposition} \label{PropUADlSMg}
	Soient des espaces vectoriels normés \( E_i\) et \( F\). Une application \( n\)-linéaire
	\begin{equation}
		T\colon E_1\times\ldots\times E_n\to F
	\end{equation}
	est est continue si et seulement si il existe un réel $L\geq 0$ tel que
	\begin{equation}\label{limitatezza}
		\|T(x_1, \ldots,x_n)\|_F\leq L \|x_1\|_{F_1}\cdots\|x_n\|_{F_n}, \qquad \forall x_i\in E_i.
	\end{equation}
\end{proposition}

\begin{proof}
	Pour simplifier l'exposition nous nous limitons au cas $n=2$ et nous notons $T(x,y)=x*y$

	Supposons que l'inégalité \eqref{limitatezza} soit satisfaite.
	\begin{equation}\label{LimImplCont}
		\begin{aligned}
			\|x*y-x_0*y_0\| & =\|(x-x_0)*y-x_0*(y-y_0)\|                \\
			                & \leq \|(x-x_0)*y\|+\|x_0*(y-y_0)\|        \\
			                & \leq L\|x-x_0\|\|y\| + L\|x_0\|\|y-y_0\|.
		\end{aligned}
	\end{equation}
	Si $x\to x_0$ et $y\to y_0$  on voit que $T$ est continue en passant à la limite aux deux côtés de l'inégalité \eqref{LimImplCont}.

	Soit $T$ continue en $(0,0)$. Évidemment\footnote{Dans la formule suivante, les trois zéros sont les zéros de trois espaces différents.} $0*0=0$, donc il existe $\delta>0$ tel que si $x\in B_{E_1}(0,\delta)$ et $y\in B_{E_2}(0,\delta)$ alors $\|x*y\|\leq 1$. En particulier si \( (x,y)\in B_{E_1\times E_2}(0,\delta)\) nous sommes dans ce cas. Soient maintenant  $x\in E_1\setminus\{ 0 \}$  et $y\in E_2\setminus\{ 0\}$
	\begin{equation}
		x*y =\left(\frac{\|x\|}{\delta}\frac{\delta x}{\|x\|}\right)*\left(\frac{\|y\|}{\delta}\frac{\delta y}{\|y\|}\right)
		=\frac{\|x\|\|y\|}{\delta^2} \left(\frac{\delta x}{\|x\|}\right)*\left(\frac{\delta y}{\|y\|}\right).
	\end{equation}
	On remarque que $\delta x/\|x\|_m$ est dans la boule de rayon $\delta$ centrée en $0_m$ et que $\delta y/\|y\|_n$ est dans la boule de rayon $\delta$ centrée en $0_n$. On conclut
	\[
		x*y\leq \frac{\|x\|_m\|y\|_n}{\delta^2}.
	\]
	Il faut prendre \( L=1/\delta^2\).
\end{proof}

La norme de \( T\) est alors définie comme la plus petite constante \( L\) qui fait fonctionner la proposition~\ref{PropUADlSMg}.
\begin{definition}  \label{DefKPBYeyG}
	La norme sur l'espace $\aL(E_1\times \cdots\times E_n, F)$ des applications $k$-linéaires et continues est
	\begin{equation}
		\|T\|_{E_1\times \ldots\times E_n}=\sup\{ \|T(u_1, \ldots,u_k)\|_{F}\,\vert\,\|u_i\|_{E_i}\leq 1, i=1,\ldots, k \}.
	\end{equation}
\end{definition}
Nous avons donc automatiquement
\begin{equation}    \label{EqYLnbRbC}
	\| T(u,v) \|\leq \| T \|\| u \|\| v \|.
\end{equation}
Et nous notons que cette norme est uniquement définie pour les applications linéaires continues. Ce n'est pas très grave parce qu'alors nous définissons \( \| T \|=\infty\) si \( T\) n'est pas continue. Cela pour retrouver le principe selon lequel on est continue si et seulement si on est borné.

\begin{proposition}\label{isom_isom}
	On définit les fonctions
	\begin{equation}
		\begin{array}{rccc}
			\omega_g: & \mathcal{L}(\eR^{m}\times\eR^{n}, \eR^p) & \to & \mathcal{L}(\eR^{m}, \mathcal{L}(\eR^{n}, \eR^p)), \\
			\omega_d: & \mathcal{L}(\eR^{m}\times\eR^{n}, \eR^p) & \to & \mathcal{L}(\eR^{n}, \mathcal{L}(\eR^{m}, \eR^p)),
		\end{array}
	\end{equation}
	par
	\[
		\omega_g(T)(x)=T(x,\cdot), \qquad \forall x\in\eR^m,
	\]
	et
	\[
		\omega_d(T)(y)=T(\cdot, y), \qquad \forall y\in\eR^n.
	\]
	Les fonctions $\omega_g$ et $\omega_d$ sont des isomorphismes qui préservent les normes.
\end{proposition}

%+++++++++++++++++++++++++++++++++++++++++++++++++++++++++++++++++++++++++++++++++++++++++++++++++++++++++++++++++++++++++++
\section{Séries}
%+++++++++++++++++++++++++++++++++++++++++++++++++++++++++++++++++++++++++++++++++++++++++++++++++++++++++++++++++++++++++++
\label{SECooYCQBooSZNXhd}

Pour une somme indexée par un ensemble infini, nous aurons la définition plus générale \ref{DefIkoheE}.
\begin{definition}\label{DefGFHAaOL}
	Soit \( (a_k)\) une suite dans un espace vectoriel normé \( (V,\| . \| )\). La suite des \defe{sommes partielles}{somme!partielle} associée est la suite \( (s_k)\) définie par
	\begin{equation}
		s_k=\sum_{i=0}^ka_i
	\end{equation}
	La \defe{série}{série!dans un espace vectoriel normé} associée est la limite des sommes partielles
	\begin{equation}
		\sum_{n=0}^{\infty}a_k=\lim_{k\to \infty} \sum_{k=0}^na_k
	\end{equation}
	si elle existe.

	Si une telle limite existe nous disons que \( \sum_{k=0}^{\infty}a_k\) \defe{converge}{série convergente} dans \( V\). Si la limite de la suite des sommes partielles n'existe pas nous disons que la série \defe{diverge}{série divergente}.
\end{definition}

\begin{remark}
	Si la limite de la suite des sommes partielles n'existe pas dans \( V\), alors elle peut parfois exister dans des extensions de \( V\). Par exemple une série de rationnels convergeant vers \( \sqrt{2}\) dans \( \eR\) ne converge pas dans \( \eQ\). Autre exemple : avec une bonne topologie sur \( \bar \eR\), une série peut ne pas converger dans \( \eR\) mais converger vers \( \pm\infty\) dans \( \bar \eR\).
\end{remark}

Dans le cas des espaces de fonctions, nous avons une norme importante : la norme uniforme définie par \( \| f \|_{\infty}=\sup\{ f(x) \}\) où le supremum est pris sur l'ensemble de définition de \( f\).

\begin{lemma}       \label{LEMooHUZEooSyPipb}
	Soit une suite \( (a_k)\) dans un espace métrique complet\footnote{Définition \ref{DEFooHBAVooKmqerL}.} dont la série converge.

	\begin{enumerate}
		\item       \label{ITEMooPFSQooDhKFGL}
		      Pour tout \( N\) nous avons
		      \begin{equation}
			      \sum_{k=0}^{\infty}a_k=\sum_{k=0}^Na_k+\sum_{k=N+1}^{\infty}a_k.
		      \end{equation}
		\item       \label{ITEMooQNHMooUPjupB}
		      La suite des queues de série converge vers \( 0\), c'est-à-dire que
		      \begin{equation}
			      \lim_{N\to \infty} \sum_{k=N}^{\infty}a_k=0.
		      \end{equation}
	\end{enumerate}
\end{lemma}

\begin{proof}
	Voici un petit calcul :
	\begin{subequations}
		\begin{align}
			\lim_{n\to \infty} \sum_{k=0}^na_k & =\lim_{n\to \infty} \big( \sum_{k=0}^Na_k+\sum_{k=N+1}^{n}a_k \big)      \label{SUBEQooZRSHooSjismK}           \\
			                                   & =\lim_{n\to \infty} \sum_{k=0}^{N}a_k+\lim_{n\to \infty} \sum_{k=N+1}^{n}a_k       \label{SUBEQooTLVKooQfYXam} \\
			                                   & =\sum_{k=0}^Na_k+\sum_{k=N+1}^{\infty}a_k.
		\end{align}
	\end{subequations}
	Justifications :
	\begin{itemize}
		\item Pour \eqref{SUBEQooZRSHooSjismK}. Pour chaque \( n\), la somme est finie et nous pouvons la décomposer. Si vous voulez vraiment couper les cheveux en quatre, vous devez fixer un \( \epsilon\), et un \( n\) de telle sorte à avoir \( n>N\), parce que \( N\) est fixé dans l'énoncé du lemme.
		\item Pour \eqref{SUBEQooTLVKooQfYXam}. Nous sommes dans un cas \( \lim_{n\to \infty}(u_n+v_n) \) où \( (u_n)\) est constante et où \( (u_n+v_n)\) converge. Nous pouvons donc permuter limite et somme\footnote{Pour rappel, la proposition \ref{PROPooICZMooGfLdPc} demande la convergence des deux suites pour fonctionner.}.
	\end{itemize}
	Voilà que \ref{ITEMooPFSQooDhKFGL} est prouvé.

	Nous écrivons \( s_n=\sum_{k=0}^na_k\); l'hypothèse est que la suite \( (s_n)\) est une suite convergente dans un espace métrique. Elle est donc de Cauchy par la proposition \ref{PROPooZZNWooHghltd}.

	Soit \( \epsilon>0\). Il existe \( N\in \eN\) tel que pour tout \( p,q>N\), nous ayons \( \| s_p-s_q \|\leq \epsilon\). Soit \( p>N\). Pour tout \( n\geq 0\) nous avons
	\begin{equation}
		\epsilon>\| s_{p+n}-s_{p+1} \|=\| \sum_{k=p}^{p+n}a_k \|.
	\end{equation}
	En prenant la limite \( n\to \infty\) nous avons
	\begin{equation}
		\| \sum_{k=p}^{\infty}a_k \|\leq \epsilon.
	\end{equation}
	Nous avons donc démontré qu'il existe \( N\) tel que si \( p>N\), alors \( \| \sum_{k=p}^{\infty}a_k \|\leq \epsilon\). Cela signifie exactement que \( \lim_{n\to \infty} \sum_{k=n}^{\infty}a_k=0\).
\end{proof}

%---------------------------------------------------------------------------------------------------------------------------
\subsection{Les trois types de convergence}
%---------------------------------------------------------------------------------------------------------------------------

Trois notions de convergence à ne pas confondre :
\begin{enumerate}
	\item
	      La convergence absolue,
	\item
	      la convergence normale. C'est la même que la convergence absolue, mais dans le cas particulier d'un espace de fonctions muni de la norme uniforme.
	\item
	      la convergence uniforme.
\end{enumerate}
Voici les définitions.

\begin{definition}[Convergence absolue] \label{DefVFUIXwU}
	Nous disons que la série \( \sum_{n=0}^{\infty}a_n\) dans l'espace vectoriel normé \( V\) \defe{converge absolument}{convergence absolue} si la série \( \sum_{n=0}^{\infty}\| a_n \|\) converge dans \( \eR\).
\end{definition}

\begin{definition}[Convergence normale] \label{DefVBrJUxo}
	Une série de fonctions \( \sum_{n\in \eN}u_n \) converge \defe{normalement}{convergence normale} si la série de nombres \( \sum_n\| u_n \|_{\infty}\) converge. C'est-à-dire si la série converge absolument pour la norme \( \| f \|_{\infty}\).
\end{definition}

\begin{definition}[Convergence uniforme]        \label{DEFooPABSooPMXMOV}
	La somme \( \sum_nf_n\) \defe{converge uniformément}{convergence uniforme!série de fonctions} vers la fonction \( F\) si la suite des sommes partielles converge uniformément, c'est-à-dire si
	\begin{equation}        \label{EqLNCJooVCTiIw}
		\lim_{N\to \infty} \| \sum_{n=1}^Nf_n-F \|_{\infty}=0.
	\end{equation}
\end{definition}

\begin{proposition} \label{PropAKCusNM}
	Une série absolument convergente dans un espace de Banach\footnote{Un espace vectoriel normé complet. Typiquement \( \eR\).} y converge au sens usuel.
\end{proposition}

\begin{proof}
	Soit \( (a_k)\) une suite dans un espace vectoriel normé complet dont la série converge absolument. Nous allons montrer que la suite des sommes partielles est de Cauchy. Cela suffira à montrer sa convergence par hypothèse de complétude.

	Nous avons
	\begin{equation}
		\| s_p-s_l \|=\| \sum_{k=l+1}^{p}a_k\|  \leq\sum_{k=l+1}^p\| a_k \|=\bar s_p-\bar s_l
	\end{equation}
	où \( \bar s_n=\sum_{k=0}^n \| a_k \|\) est la suite des sommes partielles de la série des normes (qui converge). Vu que la suite \( (\bar s_n)\) converge dans \( \eR\), elle y est de Cauchy par la proposition~\ref{PROPooTFVOooFoSHPg}. Donc il existe un \( N\) tel que \( p,l>N\) implique
	\begin{equation}
		\| s_p-s_l \|=\bar s_p-\bar s_l\leq \epsilon.
	\end{equation}
	Cela signifie que \( (s_n)\) est une suite de Cauchy et donc convergente.
\end{proof}

\begin{example}[Si l'espace n'est pas complet\cite{MonCerveau}]
	Dans un espace qui n'est pas complet, il est possible de construire un série qui converge absolument sans converger au sens usuel.

	Nous allons trouver dans \( \eQ\) une série qui converge simplement vers \( \sqrt{ 2 }\) (et donc ne converge pas dans \( \eQ\)) mais absolument vers \( 4\).

	La base est que si \( A,B\in \eQ\) avec \( A<B\) il est possible de résoudre
	\begin{subequations}
		\begin{numcases}{}
			r_1+r_2=A\\
			| r_1 |+| r_2 |=B
		\end{numcases}
	\end{subequations}
	pour \( r_1,r_2\in \eQ\). Ce n'est pas très compliqué : la solution est \( r_1=(A+B)/2\) et \( r_2=(A-B)/2\).

	Nous considérons l'espace \( \eQ\) qui n'est pas complet dans \( \eR\). Soit une série \( (a_k)\) dans \( \eQ\) qui converge vers \( \sqrt{ 2 }\) (convergence dans \( \eR\)) nous nommons \( (s_k)\) la suite des ses sommes partielles. Soit aussi la suite \( (b_k)\) qui converge vers \( 4\) (zéro serait encore plus facile mais bon, juste pour faire un peu de généralité).

	Nous supposons que \( a_k<b_k\) pour tout \( k\) et que les deux suites sont constituées de rationnels positifs. Nous nommons \( (s_k)\) et \( (s'_k)\) les sommes partielles. En particulier \( s_n<s'_n\) et ce sont des suites croissantes.

	Nous savons comment trouver \( r_1,r_2\in \eQ\) tels que \( r_1+r_2=s_1\) et \( | r_1 |+| r_2 |=s'_1\). Par récurrence, si nous savons \( r_1,\ldots, r_k\) tels que
	\begin{subequations}
		\begin{numcases}{}
			r_1+\ldots +r_k=s_n\\
			|r_1|+\ldots +|r_k|=s'_n
		\end{numcases}
	\end{subequations}
	(avec, soit dit en passant \( k=2n\)), alors nous pouvons trouver des rationnels \( r_{k+1}\), \( r_{k+2}\) tels que
	\begin{subequations}
		\begin{numcases}{}
			r_1+\ldots +r_k+r_{k+1}+r_{k+2}=s_{n+1}\\
			|r_1|+\ldots +|r_k|+|r_{k+1}|+|r_{k+2}|=s'_{n+1},
		\end{numcases}
	\end{subequations}
	en effet il s'agit de résoudre
	\begin{subequations}
		\begin{numcases}{}
			r_{k+1}+r_{k+2}=s_{n+1}-r_1-\ldots-r_k=s_{n+1}-s_n>0\\
			| r_{k+1} |+| r_{k+2} |=s'_{n+1}-| r_1 | -\ldots -| r_k |=s'_{n+1}-s'_n>0.
		\end{numcases}
	\end{subequations}
	Cela se résout comme ci-dessus. Au final nous pouvons construire une suite \( (r_k)\) dans \( \eQ\) telle que
	\begin{equation}
		\sum_{k=0}^{2n}r_k=s_n
	\end{equation}
	et
	\begin{equation}
		\sum_{k=0}^{2n}| r_k |=s'_n.
	\end{equation}
\end{example}

\begin{remark}
	Nous savons que sur les espaces vectoriels de dimension finie toutes les normes sont équivalentes (théorème~\ref{DefEquivNorm}). La notion de convergence de série ne dépend alors pas du choix de la norme. Il n'en est pas de même sur les espaces de dimension infinie. Une série peut converger pour une norme mais pas pour une autre.
\end{remark}
Lorsque nous verrons la convergence de séries, nous verrons que la convergence normale est la convergence absolue pour la norme uniforme.

\begin{lemma}       \label{LemCAIPooPMNbXg}
	Si \( E\) et \( F\) sont des espaces de Banach\quext{Je crois qu'il ne faut pas que \( E\) soit complet.}, l'espace \( \aL(E,F)\) est également de Banach.
\end{lemma}

\begin{proof}
	Soit \( (u_n)\) une suite de Cauchy dans \( \aL(E,F)\); si \( x\in E\) il existe \( N\) tel que si \( l,m>N\) alors \( \| u_l-u_m \|<\epsilon\), c'est-à-dire que pour tout \( \| x \|=1\) on a \( \| u_l(x)-u_n(x) \|<\epsilon\). Cela signifie que \( u_n(x)\) est une suite de Cauchy dans l'espace complet \( F\). Cette suite converge et nous pouvons définir l'application \( u\colon E\to F\) par
	\begin{equation}
		u(x)=\lim_{n\to \infty} u_n(x).
	\end{equation}
	Il suffit maintenant de prouver que \( u\) est linéaire, ce qui est une conséquence directe de la linéarité de la limite :
	\begin{equation}
		u(\alpha x+\beta y)=\lim_{n\to \infty} \big( \alpha u_n(x)+\beta u_n(y) \big).
	\end{equation}
\end{proof}

\begin{proposition}  \label{PROPooYDFUooTGnYQg}
	Si une série converge dans un espace complet, la norme de son terme général converge vers \( 0\).
\end{proposition}

\begin{proof}
	Soit une suite \( (a_n)\) dont la série converge vers \( s\). Soit \( \epsilon>0\). La suite des sommes partielles \( (s_n)\) est de Cauchy et converge vers \( s\) : \( s_n\to s\). En particulier il existe un \( N\) tel que si \( n>N\), nous avons \( \| s_n-s_{n-1} \|<\epsilon\). Pour de telles valeurs de \( n\) nous avons :
	\begin{equation}
		\| a_n \|=\| s_n-s_{n-1} \|\leq \epsilon.
	\end{equation}
	Cela prouve que \( a_n\to 0\).
\end{proof}

Dans le même ordre d'idée nous avons la convergence des queues de suites.

\begin{lemma}       \label{LEMooFUCOooCOqLRj}
	Si \( \sum_{k=0}^{\infty}a_k\) est finie, alors
	\begin{equation}
		\lim_{n\to \infty} \sum_{k=n}^{\infty}a_k=0.
	\end{equation}
\end{lemma}

\begin{proposition}     \label{PROPooUEBWooUQBQvP}
	Si la série converge alors la somme est associative :
	\( \sum_k (a_k+b_k) = \sum_k a_k + \sum_k b_k \).
\end{proposition}

\begin{proof}
	Associativité. Supposons que \( \sum_ka_k\) et \( \sum_kb_k\) convergent tous deux. Alors nous avons pour tout \( N\) :
	\begin{equation}
		\sum_{k=0}^N(a_k+b_k)=\sum_{k=0}^Na_k+\sum_{k=0}^Nb_k.
	\end{equation}
	Mais si deux limites existent alors la somme commute avec la limite. C'est le cas pour la limite \( N\to \infty\), donc
	\begin{equation}
		\lim_{N\to \infty} \sum_{k=1}^{\infty}(a_k+b_k)=\lim_{N\to \infty} \sum_{k=0}^{\infty}a_k+\lim_{N\to \infty} \sum_{k=0}^{\infty}b_k.
	\end{equation}
\end{proof}


%---------------------------------------------------------------------------------------------------------------------------
\subsection{Séries dans une algèbre normée}
%---------------------------------------------------------------------------------------------------------------------------

Nous allons parler d'exponentielle de matrice. Avant cela, quelques propriétés qui sont valables sur des algèbres normées. Le principal exemple que nous avons en tête est \( \eA=\eM(n,\eC)\).

\begin{proposition}[Distributivité de la somme infinie\cite{MonCerveau}]      \label{PROPooMZZQooEhQsgQ}
	Soit une algèbre normée \( \eA\). Soient une suite d'éléments \( A_k\in \eA\) et un élément \( B\). Nous supposons que la somme \( \sum_{k=0}^{\infty}A_k\) converge. Alors
	\begin{equation}
		B\sum_kA_k=\sum_k(BA_k).
	\end{equation}
\end{proposition}

\begin{proof}
	Soit \( N\in \eN\). Nous avons:
	\begin{subequations}
		\begin{align}
			\| \sum_{k=0}^NBA_k-B\sum_{k=0}^{\infty}A_k \| & =\| B\sum_{k=N+1}^{\infty}A_k \|            \label{SUBEQooDTNAooWpXOKP} \\
			                                               & \leq \| B \|\| \sum_{k=N+1}^{\infty}A_k \|  \label{SUBEQooJPSJooAqXtOJ}
		\end{align}
	\end{subequations}
	Justifications:
	\begin{itemize}
		\item Pour \eqref{SUBEQooDTNAooWpXOKP}. Linéarité du produit matriciel.
		\item Pour \eqref{SUBEQooJPSJooAqXtOJ}. La norme est une norme d'algèbre\footnote{Définition \ref{DefJWRWQue}. Pour rappel, la norme opérateur en est une par le lemme \ref{LEMooFITMooBBBWGI}.}.
	\end{itemize}
	À droite, la limite \( N\to \infty\) donne zéro car \( \| B \|\) est un simple nombre, et \( \| \sum_{k=N+1}^{\infty}A_k \|\) est une queue de suite convergente par hypothèse.

	Nous avons donc bien convergence
	\begin{equation}
		\lim_{N\to \infty}\sum_{k=0}^{N}BA_k=B\sum_{k=0}^{\infty}A_k.
	\end{equation}
\end{proof}

\begin{proposition}[Produit de Cauchy dans une algèbre normée\cite{MonCerveau}]      \label{PROPooFMEXooCNjdhS}
	Soient une algèbre normée \( \eA\), un élément \( A\in \eA\), ainsi que des séries convergentes \( \sum_{k=0}^{\infty}a_kA^k\) et \( \sum_{l=0}^{\infty}b_lA^l\). Alors
	\begin{equation}
		\left( \sum_ka_kA^k \right)\left( \sum_lb_lA^l \right)=\sum_{n=0}^{\infty}\big( \sum_{m=0}^na_mb_{n-m} \big)A^n.
	\end{equation}
\end{proposition}

\begin{proof}
	Un calcul :
	\begin{subequations}
		\begin{align}
			\left( \sum_ka_kA^k \right)\left( \sum_lb_lA^l \right) & =\sum_k\big( \sum_lb_lA^l \big)a_kA^k       \label{SUBEQooFAECooWFCaNW}                                           \\
			                                                       & = \sum_k\big( \sum_lb_la_kA^{l+k} \big)   \label{SUBEQooDZTHooMwmKxJ}                                             \\
			                                                       & = \lim_{K\to\infty} \sum_{k=0}^K\big( \lim_{L\to \infty} \sum_{l=0}^Lb_la_kA^{k+l} \big)                          \\
			                                                       & = \lim_{K\to \infty} \lim_{L\to \infty} \sum_{k=0}^K\sum_{l=0}^Lb_la_kA^{k+l}         \label{SUBEQooISSHooJsyMTv} \\
			                                                       & = \lim_{K\to \infty} \lim_{L\to \infty} \sum_{n=0}^{K+L}\sum_{m=0}^na_mb_{n-m}A^n     \label{SUBEQooAWUQooZCHIXH} \\
			                                                       & = \lim_{K\to \infty} \sum_{n=0}^{\infty}\sum_{m=0}^na_mb_{n-m}A^m                     \label{SUBEQooUVOBooSPGjrA} \\
			                                                       & = \sum_{n=0}^{\infty}\sum_{m=0}^na_mb_{n-m}A^m                                        \label{SUBEQooCGRGooGIDCYv}
		\end{align}
	\end{subequations}
	Justifications :
	\begin{itemize}
		\item Pour \eqref{SUBEQooFAECooWFCaNW}, la proposition \ref{PROPooMZZQooEhQsgQ} nous permet d'entrer l'élément \( \sum_lb_lA^l\in \eA\) dans la somme sur \( k\).
		\item
		      Pour \eqref{SUBEQooDZTHooMwmKxJ}, c'est la même chose.
		\item
		      Pour \eqref{SUBEQooISSHooJsyMTv}, la somme sur \( k\) étant finie (pour chaque \( K\)), elle commute avec la limite sur \( L\).
		\item
		      Pour \eqref{SUBEQooAWUQooZCHIXH}, c'est une manipulation de sommes finies. On regroupe les termes selon les puissances de \( A\).
		\item
		      Pour \eqref{SUBEQooUVOBooSPGjrA}, c'est effectuer la limite sur \( L\) pour \( K\) fixé.
		\item
		      Pour \eqref{SUBEQooCGRGooGIDCYv}, l'expression dans la limite sur \( K\) ne dépend pas de \( K\). Donc nous pouvons simplement supprimer la limite.
	\end{itemize}
\end{proof}


%+++++++++++++++++++++++++++++++++++++++++++++++++++++++++++++++++++++++++++++++++++++++++++++++++++++++++++++++++++++++++++
\section{Série réelle}
%+++++++++++++++++++++++++++++++++++++++++++++++++++++++++++++++++++++++++++++++++++++++++++++++++++++++++++++++++++++++++++
\label{secseries}

La notion de série formalise le concept de somme infinie\footnote{La définition d'une somme infinie est la définition \ref{DefHYgkkA}.}. L'absence de certaines propriétés de ces objets (problèmes de commutativité et même d'associativité) incite à la prudence et montre à quel point une définition précise est importante.


\subsection{Critères de convergence absolue}

Étant donné le terme général d'une série, il est souvent --dans les cas qui nous intéressent-- difficile de déterminer la somme de la série. L'exemple de la série géométrique est particulier\footnote{Voir la proposition \ref{PROPooWOWQooWbzukS}.}, puisqu'on connaît une formule pour chaque somme partielle, mais pour l'exemple des séries de Riemann il n'y a aucune formule simple pour un $\alpha$ général. D'où l'intérêt d'avoir des critères de convergence ne nécessitant aucune connaissance de l'éventuelle limite de la série.

\begin{lemma}[Critère de comparaison]   \label{LemgHWyfG}
	Soient $\sum_i a_i$ et $\sum_j
		b_j$ deux séries à termes positifs vérifiant
	\begin{equation*}
		0 \leq a_i \leq b_i
	\end{equation*}
	alors
	\begin{enumerate}
		\item si $\sum_i a_i$ diverge, alors $\sum_j b_j$ diverge,
		\item si $\sum_j b_j$ converge, alors $\sum_i a_i$ converge
		      (absolument).
	\end{enumerate}
\end{lemma}

\begin{proposition}[Critère d'équivalence\cite{TrenchRealAnalisys}]
	Soient $\sum_i a_i$ et $\sum_j b_j$ deux séries à termes positifs. Supposons l'existence de la limite (éventuellement infinie) suivante
	\begin{equation}
		\limite i \infty \frac{a_i}{b_i} = \alpha
	\end{equation}
	avec \( \alpha\in \eR\cup\{ +\infty \}\). Alors
	\begin{enumerate}
		\item si $\alpha \neq 0$ et $\alpha\neq \infty$, alors
		      \begin{equation}
			      \sum_i a_i \text{~converge} \ssi \sum_j b_j\text{~converge,}
		      \end{equation}
		\item si $\alpha = 0$ et $\sum_j b_j$ converge, alors $\sum_i a_i$ converge (absolument),
		\item si $\alpha = +\infty$ et $\sum_j b_j$ diverge, alors $\sum_i a_i$ diverge.
	\end{enumerate}
\end{proposition}

\begin{proof}
	\begin{enumerate}
		\item
		      Le fait que la suite $a_n/b_n$ converge vers $\alpha$ signifie que tant sa limite supérieure que sa limite inférieure convergent vers $\alpha$. En particulier la suite $\frac{ a_n }{ b_n }$ est bornée vers le haut et vers le bas. À partir d'un certain rang $N$, il existe $M$ tel que
		      \begin{equation}
			      \frac{ a_n }{ b_n }<M
		      \end{equation}
		      et il existe $m$ tel que
		      \begin{equation}
			      \frac{ a_n }{ b_n }>m.
		      \end{equation}
		      Nous avons donc $a_n<Mb_n$ et $a_n>mb_n$. La série de $(a_n)$ converge donc si et seulement si la série de $(b_n)$ converge.
		\item
		      Si $\alpha=0$, cela signifie que pour tout $\epsilon$, il existe un rang tel que $\frac{ a_n }{ b_n }<\epsilon$, et donc tel que $a_n<\epsilon b_k$. La suite de $(a_i)$ converge donc dès que la suite de $(b_i)$ converge.
		\item
		      Pour tout $M$, il existe un rang dans la suite à partir duquel on a $\frac{ a_i }{ b_i }>M$, et donc $a_k>Mb_k$. Si la série de $(b_k)$ diverge, la série de $(a_k)$ doit également diverger.
	\end{enumerate}
\end{proof}

\begin{proposition}[Critère du quotient\cite{KeislerElemCalculus}]     \label{PropOXKUooQmAaJX}
	Soit $\sum_i a_i$ une série. Supposons l'existence de la limite (éventuellement infinie) suivante
	\begin{equation}
		\limite i \infty \abs{\frac{a_{i+1}}{a_i}} = L
	\end{equation}
	avec \( L\in \eR\cup\{ +\infty \}\).  Alors
	\begin{enumerate}
		\item si \(L < 1\), la série converge absolument,
		\item si \(L > 1\), la série diverge,
		\item si \(L = 1\) le critère échoue : il existe des exemples de convergence et des exemples de divergence.
	\end{enumerate}
\end{proposition}
\index{critère du quotient}

\begin{proof}
	\begin{enumerate}
		\item
		      Soit $b$ tel que $L<b<1$. À partir d'un certain rang $K$, on a $\left| \frac{ a_{i+1} }{ a_i } \right| <b$. En particulier,
		      \begin{equation}
			      | a_{K+1} |<b| a_K |,
		      \end{equation}
		      et pour $a_{K+2}$ nous avons
		      \begin{equation}
			      | a_{K+2} |<b| a_{K+1} |<b^2| a_K |.
		      \end{equation}
		      Au final,
		      \begin{equation}
			      | a_{K+n} |<b^n| a_K |.
		      \end{equation}
		      Étant donné que la série $\sum_{n\geq K}b^n$ converge (parce que $b<1$), la queue de suite $\sum_{i\geq K}a_i$ converge, et par conséquent la suite au complet converge.
		\item
		      Si $L>1$, on a
		      \begin{equation}
			      | a_K |<| a_{K+1} |<| a_{K+2} |<\ldots
		      \end{equation}
		      Il est donc impossible que la suite $(a_i)$ converge vers zéro. La série ne peut donc pas converger.
		\item
		      Par exemple la suite harmonique $a_n=\frac{1}{ n }$ vérifie $L=1$, mais la série ne converge pas. Par contre, la suite $a_n=\frac{ 1 }{ n^2 }$ vérifie aussi le critère avec $L=1$ tandis que la série $\sum_n\frac{1}{ n^2 }$ converge.
	\end{enumerate}
\end{proof}


\begin{proposition}[Critère de la racine\cite{TrenchRealAnalisys}]
	Soit $\sum_i a_i$ une série, et considérons
	\begin{equation*}
		\limsup_{i \rightarrow \infty} \sqrt[i]{\abs{a_i}} = L
	\end{equation*}
	avec \( L\in \eR\cup\{ +\infty \}\). Alors
	\begin{enumerate}
		\item si \(L < 1\), la série converge absolument,
		\item si \(L > 1\), la série diverge,
		\item si \(L = 1\) le critère échoue.
	\end{enumerate}
\end{proposition}

\begin{proof}
	\begin{enumerate}
		\item
		      Si $L<1$, il existe un $r\in \mathopen] 0 , 1 \mathclose[$ tel que $| a_n |^{1/n}<r$ pour les grands $n$. Dans ce cas, $| a_n |<r^{n}$, et la série converge absolument parce que la série $\sum_nr^n$ converge du fait que $r<1$.
		\item
		      Si $L>1$, il existe un $r>1$ tel que $| a_n |^{1/n}>r>1$. Cela fait que $| a_n |$ prend des valeurs plus grandes que $n$ pour une infinité de termes. Le terme général $a_n$ ne peut donc pas être une suite convergente. Par conséquent la suite diverge au sens où elle ne converge pas.

	\end{enumerate}
\end{proof}

%---------------------------------------------------------------------------------------------------------------------------
\subsection{Critères de convergence simple}
%---------------------------------------------------------------------------------------------------------------------------

Les critères de comparaison, d'équivalence, du quotient et de la racine sont des critères de convergence absolue. Pour conclure à une convergence simple qui n'est pas une convergence absolue, le critère d'Abel sera notre outil principal.

\subsubsection{Critère d'Abel}

\begin{proposition}[Critère d'Abel]
	Soit la série $\sum_i c_iz_i$ avec
	\begin{enumerate}
		\item $(c_i)$ est une suite réelle décroissante qui tend vers zéro,
		\item $(z_i)$ est une suite dans $\eC$ dont la suite des sommes partielles est bornée dans $\eC$, c'est-à-dire qu'il existe un $M>0$ tel que pour tout $n$,
		      \begin{equation}
			      \left| \sum_{i=1}^nz_i \right| \leq M.
		      \end{equation}
		      Alors la série $\sum_ic_iz_i$ est convergente.
	\end{enumerate}
\end{proposition}
Remarquons que ce critère ne donne pas de convergence absolue.

%---------------------------------------------------------------------------------------------------------------------------
\subsection{Quelques séries usuelles}
%---------------------------------------------------------------------------------------------------------------------------
\label{SUBSECooDTYHooZjXXJf}

\begin{example}[Série harmonique]       \label{EXooDVQZooEZGoiG}
	La \defe{série harmonique}{série!harmonique} est
	\begin{equation}
		\sum_{i=k}^\infty \frac{1}{ k }=+\infty.
	\end{equation}
\end{example}

\begin{propositionDef}[Série géométrique]      \label{PROPooWOWQooWbzukS}
	La \defe{série géométrique}{série!géométrique} de raison $q \in \eC$ est
	\begin{equation}    \label{EqZQTGooIWEFxL}
		\sum_{i=0}^\infty q^i.
	\end{equation}
	\begin{enumerate}
		\item       \label{ITEMooAFAMooGuXqBm}
		      Elle converge si et seulement si \( | q |<1\).
		\item       \label{ITEMooBJHBooBMEmiG}
		      Si \( | q |<1\) alors
		      \begin{equation}    \label{EqRGkBhrX}
			      \sum_{n=0}^{\infty}q^n=\frac{ 1 }{ 1-q }.
		      \end{equation}
		\item
		      Quand la série géométrique converge, la convergence est absolue.
		\item
		      Si la somme commence en \( n=1\) au lieu de \( n=0\) alors
		      \begin{equation}        \label{EqPZOWooMdSRvY}
			      \sum_{n=1}^{\infty}q^n=\frac{ q }{ 1-q }.
		      \end{equation}
	\end{enumerate}
\end{propositionDef}

\begin{proof}
	La somme partielle est déjà donnée dans le lemme \ref{LEMooAFSCooWEVlvp} :
	\begin{equation}
		S_N=\sum_{n=0}^Nq^n=\frac{ 1-q^{N+1} }{ 1-q }.
	\end{equation}
	En vertu de \eqref{EQooATTQooRpJeCo}, la limite \( \lim_{N\to \infty} S_N\) existe si et seulement si \( | q |\leq 1\) et dans ce cas nous avons le résultat parce que \( q^{N+1}\to 0\).

	Pour le dernier point, il s'agit seulement du calcul
	\begin{equation}
		\sum_{n=1}^{\infty}q^n=\frac{1}{ 1-q }-1=\frac{ q }{ 1-q }.
	\end{equation}
\end{proof}

Un cas particulier de la formule \eqref{EqASYTiCK} est le calcul de \( \sum_{j=1}^{N}q^{-j}\) bien utile lorsque l'on joue avec des nombres binaires (voir l'exemple~\ref{EXEMooRHENooGwumoA}). Nous avons
\begin{equation}        \label{EQooFMBAooEJkHWT}
	\sum_{j=1}^Nq^{-j}=\sum_{j=0}^Nq^{-j}-1=\frac{ 1-q^{-N} }{ q-1 }.
\end{equation}

La série de Riemann est très liée aux intégrales impropres de la proposition \ref{PropBKNooPDIPUc}.
\begin{proposition}[Série de Riemann] \label{PROPooFPVZooGnsqrs}      \label{EXooCTYNooCjYQvJ}
	Pour \( \alpha \in \eR\), la \defe{série de Riemann}{série!Riemann}
	\begin{equation}        \label{EqSerRiem}
		\sum_{k=1}^\infty \frac{ 1 }{ k^{\alpha}}
	\end{equation}
	converge (absolument, puisque réelle et positive) si et seulement si $\alpha > 1$, et diverge sinon.
\end{proposition}

\begin{example}[Série exponentielle] \label{ExIJMHooOEUKfj}
	La série exponentielle est la série (pour \( t\in \eR\))
	\begin{equation}
		\exp(t)=\sum_{k=0}^{\infty}\frac{ t^k }{ k! }.
	\end{equation}
	Nous montrons qu'elle converge pour tout \( t\in \eR\). Si \( a_k=t^k/k!\) alors \( \frac{ a_{k+1} }{ a_k }=\frac{ t }{ k }\) dont la limite \( k\to \infty\) est zéro (quel que soit \( t\)). En vertu du critère du quotient~\ref{PropOXKUooQmAaJX} la série exponentielle converge (absolument) pour tout \( t\in \eR\).

	Pour tout savoir de l'exponentielle et de ses variations, voir le thème~\ref{THEMEooKXSGooCsQNoY}.
\end{example}
\index{exponentielle!convergence}

\begin{example}[Série arithmético-géométrique\cite{QXuqdoo}]
	Une \defe{suite arithmético-géométrique}{suite!arithmético-géométrique} est une suite vérifiant pour tout \( n\) la relation
	\begin{equation}
		u_{n+1}=au_n+b
	\end{equation}
	avec \( a\) et \( b\) non nuls. Si elle possède une limite, cette dernière doit résoudre \( l=al+b\), et donc être donnée par
	\begin{equation}
		l=\frac{ b }{ 1-a }.
	\end{equation}

	Comportement amusant : la limite peut exister pour certains valeurs de \( a_0\) et pas pour d'autres. Mais elle ne dépend pas de \( a_0\) parmi ceux pour lesquelles la limite existe.

	Il n'est pas très compliqué de trouver le terme général de la suite en fonction de \( a\) et de \( b\). Il suffit de considérer la suite \( v_n=u_n-r\), et de remarquer que cette suite est géométrique :
	\begin{equation}
		v_{n+1}=av_n.
	\end{equation}
	Par conséquent \( v_n=a^nv_0\), ce qui donne pour la suite \( (u_n)\) la formule
	\begin{equation}
		u_n=a^n(u_0-r)+r.
	\end{equation}
\end{example}

\begin{lemma}[\cite{BIBooTIZHooGeFZri}]     \label{LEMooKDHPooPlFTIT}
	Nous avons :
	\begin{equation}
		\sum_{k=1}^N\frac{1}{ k(k+1) }=\frac{ N }{ N+1 }.
	\end{equation}
	et
	\begin{equation}
		\sum_{k=1}^{\infty}\frac{1}{ k(k+1) }=1.
	\end{equation}
\end{lemma}

\begin{proof}
	Nous posons
	\begin{subequations}
		\begin{align}
			f(n) & =\sum_{k=1}^n\frac{1}{ k(k+1) } \\
			g(n) & =\frac{ n }{ n+1 }
		\end{align}
	\end{subequations}
	et nous montrons par récurrence que \( f(n)=g(n)\). Pour \( n=1\) nous avons \( f(1)=g(1)=\frac{ 1 }{2}\).

	Nous supposons que \( f(n)=g(n)\) et nous prouvons que \( f(n+1)=g(n+1)\). Facile :
	\begin{subequations}
		\begin{align}
			f(n+1) & =f(n)+\frac{1}{ (n+1)(n+2) }              \\
			       & =\frac{ n }{ n+1 }+\frac{1}{ (n+1)(n+2) } \\
			       & =\frac{ n(n+2)+1 }{ (n+1)(n+2) }          \\
			       & =\frac{ n^2+2n+1 }{ (n+1)(n+2) }          \\
			       & =\frac{ (n+1)^2 }{ (n+1)(n+2) }           \\
			       & =\frac{ n+1 }{ n+2 }                      \\
			       & =g(n+1).
		\end{align}
	\end{subequations}
	En ce qui concerne la seconde formule, par définition\footnote{Définition d'une série, \ref{DefGFHAaOL}.}
	\begin{equation}
		\sum_{k=1}^{\infty}\frac{1}{ k(k+1) }=\lim_{n\to \infty} \sum_{k=1}^n\frac{1}{ k(k+1) }=\lim_{n\to \infty}\frac{ n }{ n+1 } =1.
	\end{equation}
\end{proof}

%---------------------------------------------------------------------------------------------------------------------------
\subsection{Séries alternées}
%---------------------------------------------------------------------------------------------------------------------------

\begin{theorem}[Critère des séries alternées\cite{ooXFPIooCLUvzV}]      \label{THOooOHANooHYfkII}
	Si \( (a_n)_{n\in \eN}\) est une suite positive décroissante à limite nulle, alors
	\begin{enumerate}
		\item
		      Si nous notons \( (S_n)\) la suite des sommes partielles, les sous-suites \( (S_{2n})\) et \( (S_{2n+1})\) sont adjacentes\footnote{Définition \ref{DEFooDMZLooDtNPmu}.}.
		\item
		      La série \( \sum_n(-1)^na_n\) converge.
		\item       \label{ITEMooWEPWooXhLMYL}
		      Si nous considérons le reste
		      \begin{equation}
			      R_n=\sum_{k=n+1}^{\infty}(-1)^ka_k,
		      \end{equation}
		      nous avons
		      \begin{subequations}
			      \begin{align}
				      \signe(R_n)=(-1)^{n+1} \\
				      | R_n |\leq a_{n+1}.
			      \end{align}
		      \end{subequations}
	\end{enumerate}
\end{theorem}

\begin{proof}
	En termes de notations, nous allons écrire \( (S_n)\) la suite des sommes partielles de \( \sum_{k=0}^{\infty}(-1)^ka_k\). Nous notons \( (S_{2n})\) la suite des termes pairs de cette suite. C'est donc la suite \( n\mapsto S_{2n}\).
	Nous divisons en plusieurs morceaux.
	\begin{subproof}
		\item[\( S_{2n}\) est croissante]
		Nous avons simplement
		\begin{equation}
			S_{2n+2}-S_{2n}=a_{2n+2}-a_{2n+1}\leq 0.
		\end{equation}
		\item[\( (S_{2n+1})\) est décroissante]
		Même calcul.
		\item[Les suites \( (S_{2n})\) et \( S_{2n+1}\) sont adjacentes] Nous avons simplement
		\begin{equation}
			S_{2n+1}-S_{2n}=a_{2n+1}\to 0.
		\end{equation}
		Nous concluons par le théorème des suites adjacentes \ref{THOooZJWLooAtGMxD} que les sous-suites des termes pairs et impairs sont convergentes et convergent vers la même limite.
	\end{subproof}
	C'est le moment d'utiliser la proposition \ref{PROPooXOOCooGMqJNe} qui convaincra la lectrice que \( (S_n)\) converge vers la même limite, que nous notons \( S\). Le théorème des suites adjacentes nous dit encore que
	\begin{equation}
		S_{2n+1}\leq S\leq S_{2n}
	\end{equation}
	et donc que \( R_{2n}=S-S_{2n}\leq 0\). Cela donne la majoration
	\begin{equation}
		| R_{2n} |=| S-S_n |=S_{2n}-S\leq S_{2n}-S_{2n+1}=a_{2n+1}.
	\end{equation}
	Nous faisons le même genre de majorations pour \( R_{2n+1}\).
\end{proof}

%---------------------------------------------------------------------------------------------------------------------------
\subsection{Moyenne de Cesàro}
%---------------------------------------------------------------------------------------------------------------------------

\begin{definition}      \label{DEFooLVRLooTeowkn}
	Si \( a\colon \eN\to V \) est une suite dans l'espace vectoriel \( V\), alors sa \defe{moyenne de Cesàro}{moyenne!de Cesàro}\index{Cesàro!moyenne} est la limite (si elle existe) de la suite
	\begin{equation}
		\sigma_n(a)=\frac{1}{ n }\sum_{k=1}^na_k.
	\end{equation}
	En un mot, c'est la limite des moyennes partielles.
\end{definition}

\begin{lemma}       \label{LemyGjMqM}
	Si la suite \( (a_n)\) converge vers la limite \( \ell\) alors la suite admet une moyenne de Cesàro qui vaudra \( \ell\).
\end{lemma}

\begin{proof}
	Soit \( \epsilon>0\) et \( N\in \eN\) tel que \( | a_n-\ell |<\epsilon\) pour tout \( n>N\). En remarquant que
	\begin{equation}
		\frac{1}{ n }\sum_{k=1}^nk-\ell=\frac{1}{ n }\sum_{k=1}^n(a_k-\ell),
	\end{equation}
	nous avons
	\begin{subequations}
		\begin{align}
			| \frac{1}{ n }\sum_{k=1}^na_k-\ell | & \leq| \frac{1}{ n }\sum_{k=1}^N| a_k-\ell | |+\big| \frac{1}{ n }\sum_{k=N+1}^n\underbrace{| a_k-\ell |}_{\leq \epsilon} \big| \\
			                                      & \leq \epsilon+\frac{ n-N-1 }{ n }\epsilon                                                                                      \\
			                                      & \leq 2\epsilon.
		\end{align}
	\end{subequations}
	Dans ce calcul nous avons redéfini \( N\) de telle sorte que le premier terme soit inférieur à \( \epsilon\).
\end{proof}

%---------------------------------------------------------------------------------------------------------------------------
\subsection{Écriture décimale d'un réel}
%---------------------------------------------------------------------------------------------------------------------------

Nous avons déjà vu la fonction \eqref{EQooWWTUooHAnSEv} qui permet d'écrire des naturels dans une base \( b\geq 2\) donnée. Nous allons maintenant construire une fonction du même type, pour la partie décimale d'un réel.

\begin{normaltext}      \label{NORMALooTZWYooPMgOIm}
	Soit \( b\geq 2\) un entier qui sera la base dans laquelle nous allons écrire les nombres. Nous considérons l'ensemble \( \eD_b\)\nomenclature[Y]{\( \eD_b\)}{l'ensemble des écritures décimales en base \( b\)} des suites dans \( \{ 0,1,\ldots, b-1 \}\) qui n'ont pas une queue de suite uniquement formée de \( b-1\). Autrement dit une suite \( (c_n)\) est dans \( \eD_b\) lorsque pour tout \( N\), il existe \( k>N\) tel que \( c_k\neq b-1\). Associé à cet ensemble nous considérons la fonction
	\begin{equation}    \label{EqXXXooOTsCK}
		\begin{aligned}
			\varphi_b\colon \eD_b & \to \mathopen[ 0 , 1 [                          \\
			c                     & \mapsto \sum_{n=1}^{\infty}\frac{ c_n }{ b^n }.
		\end{aligned}
	\end{equation}
\end{normaltext}

\begin{lemma}
	La fonction \( \varphi_b\) est bien définie au sens où elle converge et prend ses valeurs dans \( \mathopen[ 0 , 1 [\).
\end{lemma}

\begin{proof}
	Tout se base sur la somme de la série géométrique \eqref{EqRGkBhrX} sous la forme
	\begin{equation}    \label{EqWZGooXJgwl}
		\sum_{k=0}^{\infty}\frac{1}{ b^k }=\frac{ b }{ b-1 }.
	\end{equation}
	La somme \eqref{EqXXXooOTsCK} est donc majorée par \( \sum_n\frac{ b-1 }{ b^n }\) qui converge.

	Pour prouver que l'image de \( \varphi_b\) est bien \( \mathopen[ 0 , 1 [\), nous savons qu'au moins un des \( c_n\) (en fait une infinité) est plus petit que \( b-1\), donc nous avons la majoration stricte\footnote{Notez que la somme \eqref{EqXXXooOTsCK} commence à un tandis que la série géométrique \eqref{EqWZGooXJgwl} commence à zéro.}
	\begin{equation}
		\varphi_b(c)<\sum_{n=1}^{\infty}\frac{ b-1 }{ b^n }=(b-1)\left( \sum_{n=1}^{\infty}\frac{1}{ b^n }-1 \right)=1
	\end{equation}
\end{proof}

Le fait d'introduire l'ensemble \( \eD\) au lieu de l'ensemble de toutes les suites est justifié par la proposition suivante. Elle explique pourquoi un nombre possède au maximum deux écritures décimales distinctes et que ces deux sont obligatoirement de la forme, par exemple en base \( 10\) :
\begin{equation}
	0.34599999999\ldots=0.34600000\ldots
\end{equation}
mais qu'un nombre commençant par \( 0.347\) ne peut pas être égal. C'est pour cela que dans la définition de \( \eD_b\) nous avons exclu les suites qui terminent par tout des \( b-1\).

La proposition suivante complète ce qui est déjà dit dans le lemme \ref{LEMooIQBXooUEtdoy}.
\begin{proposition} \label{PropSAOoofRlQR}
	Soit la fonction
	\begin{equation}
		\begin{aligned}
			\varphi\colon \{ 0,\ldots, b-1 \}^{\eN} & \to \mathopen[ 0 , 1 [                          \\
			x                                       & \mapsto \sum_{n=1}^{\infty}\frac{ x_n }{ b^n }.
		\end{aligned}
	\end{equation}
	Si \( \varphi(x)=\varphi(y)\) et si \( n_0\) est le plus petit entier tel que \( x_{n_0}\neq y_{n_0}\) alors soit
	\begin{equation}
		x_{n_0}-y_{n_0}=1
	\end{equation}
	et \( x_n=0\), \( y_n=b-1\) pour tout \( n>n_0\), soit le contraire : \( y_{n_0}-x_{n_0}=1\) avec \( y_n=0\) et \( x_n=b-1\) pour tout \( n>n_0\).
\end{proposition}

\begin{proof}
	Nous nous basons sur la formule (facilement dérivable depuis \eqref{EqWZGooXJgwl}) suivante :
	\begin{equation}
		\sum_{k=n_0+1}^{\infty}\frac{1}{ b^k }=\frac{1}{ b^{n_0+1} }\frac{ b }{ b-1 }.
	\end{equation}
	Nous avons
	\begin{equation}
		0=\varphi(x)-\varphi(y)=\frac{ x_{n_0}-y_{n_0} }{ b^{n_0} }+\sum_{n=n_0+1}^{\infty}\frac{ x_n-y_n }{ b^n }\geq \frac{ x_{n_0}-y_{n_0} }{ b^{n_0} }-\sum_{n=n_0+1}^{\infty}\frac{ b-1 }{ b^n }=\frac{ x_{n_0}-y_{n_0}-1 }{ b^{n_0} }.
	\end{equation}
	Le dernier terme étant manifestement positif\footnote{C'est ici qu'intervient la subdivision entre le cas \( x_{n_0}-y_{n_0}=1\) ou le contraire. En effet si «ce dernier terme était manifestement \emph{négatif}», il aurait fallu majorer avec de \( 1-b\) au lieu de \( 1-b\).}, il est nul et nous avons \( x_{n_0}-y_{n_0}=1\).

	Nous avons donc maintenant
	\begin{equation}    \label{EqHWQoottPnb}
		0=\varphi(x)-\varphi(y)=\frac{1}{ b^{n_0} }+\sum_{n=n_0+1}^{\infty}\frac{ x_n-y_n }{ b^n }.
	\end{equation}
	Nous majorons la dernière somme de la façon suivante, en supposant que \( | x_n-y_n |\neq b-1\) pour un certain \( n>n_0\) :
	\begin{equation}
		\left| \sum_{n=n_0+1}^{\infty}\frac{ x_n-y_n }{ b^n } \right| \leq\sum_{n=n_0+1}^{\infty}\frac{ | x_n-y_n | }{ b^n }<\sum_{n=n_0+1}^{\infty}\frac{ b-1 }{ b^n }=\frac{1}{ b^{n_0} }.
	\end{equation}
	Étant donné cette inégalité stricte, l'équation \eqref{EqHWQoottPnb} ne peut pas être correcte (valoir zéro). Nous avons donc \( | x_n-b_n |=b-1\) pour tout \( n>n_0\). Donc pour chaque \( n>n_0\) nous avons soit \( x_n=0\) et \( y_n=b-1\), soit \( a_n=b-1\) et \( b_n=0\). Pour conclure il faut encore prouver que le choix doit être le même pour tout \( n\).

	Nous nous mettons dans le cas \( x_{n_0}-y_{n_0}=1\); dans ce cas nous avons bien l'égalité \eqref{EqHWQoottPnb} sans petites nuances de signes. Nous écrivons
	\begin{equation}
		\sum_{n=n_0+1}^{\infty}\frac{ x_n-y_n }{ b^n }=(b-1)\sum_{n=n_0+1}^{\infty}\frac{ (-1)^{s_n} }{ b^n }
	\end{equation}
	où \( s_n\) est pair ou impair suivant que \( x_n=0\), \( y_n=b-1\) ou le contraire. Si un des \( (-1)^{s_n}\) est pas \( -1\) alors nous avons l'inégalité stricte
	\begin{equation}
		(b-1)\sum_{n=n_0+1}^{\infty}\frac{ (-1)^{s_n} }{ b^n }>(b-1)\sum_{n=n_0+1}^{\infty}\frac{-1}{ b^n }=-\frac{1}{ b^{n_0} }.
	\end{equation}
	Dans ce cas il est impossible d'avoir \( \varphi(x)-\varphi(y)=0\). Nous en concluons que \( (-1)^{s_n}\) est toujours \( -1\), c'est-à-dire \( x_n-y_n=1-b\), ce qui laisse comme seule possibilité \( x_n=0\) et \( y_n=b-1\).
\end{proof}

\begin{theorem} \label{ThoRXBootpUpd}
	L'application \( \varphi_b\colon \eD_b\to \mathopen[ 0 , 1 [\) est bijective.
\end{theorem}

\begin{proof}
	En ce qui concerne l'injection, nous savons de la proposition~\ref{PropSAOoofRlQR} que si \( \varphi_b(x)=\varphi_b(y)\) pour \( x,y\in\{ 0,\ldots, b-1 \}^{\eN}\), alors soit \( x\) soit \( y\) a une queue de suite composée uniquement de \( b-1\), ce qui est exclu dans \( \eD_b\). Nous en déduisons que \( \varphi_b\) est bien injective en prenant \( \eD_b\) comme ensemble départ.

	La partie lourde est la surjectivité. Nous prenons \( x\in \mathopen[ 0 , 1 [\) et nous allons construire par récurrence une suite \( a\in \eD_b\) telle que \( \varphi_b(a)=x\). Si il existe \( a_1\in\{ 0,\ldots, b-1 \}\) tel que \( x=a_1/b\) alors nous prenons la suite \( (a_1,0,\ldots, )\) et nous avons évidemment \( \varphi(a)=x\). Sinon il existe \( a_1\in\{ 0,\ldots, b-1 \}\) tel que
	\begin{equation}
		\frac{ a_1 }{ b }<x<\frac{ a_1+1 }{ b }
	\end{equation}
	parce que les autres possibilités pour \( x\) sont dans l'ensemble \( \mathopen[ 0 , 1 \mathclose[\setminus\{ \frac{ k }{ b } \}_{k=0,\ldots, b-1}\) que nous subdivisons en
	\begin{equation}
		\mathopen] 0 , \frac{1}{ b } \mathclose[\cup\mathopen] \frac{1}{ b } , \frac{ 2 }{ b } \mathclose[\cup\ldots\cup\mathopen] \frac{ b-1 }{ b } , 1 \mathclose[.
	\end{equation}
	Pour la récurrence nous supposons avoir trouvé \( a_1,\ldots, a_n\) tels que
	\begin{equation}
		\sum_{k=1}^n\frac{ a_k }{ b^k }< x<\sum_{k=1}^{n-1}\frac{ a_k }{ b^k }+\frac{ a_n+1 }{ b^n }.
	\end{equation}
	Encore une fois si il existe \( a_{n+1}\in\{ 0,\ldots, b-1 \}\) tel que \( \sum_{k=1}^{n+1}\frac{ a_k }{ b^k }=x\) alors nous prenons ce \( a_{n+1}\) et nous complétons la suite avec des zéros pour avoir \( \varphi(a)=x\). Sinon
	%nous subdivisions l'intervalle \( \mathopen]  \frac{ a_n }{ b^n }, \frac{ a_n }{ b^n }+\frac{ a_n+1 }{ b^n } \mathclose[\) (auquel nous retranchons les \( b\) nombres déjà traités) en
	%       \begin{equation}
	%       \mathopen] \frac{ a_n }{ b^n } , \frac{ a_n }{ b^n }+\frac{1}{ b^{n+1} } \mathclose[ \cup \mathopen] \frac{ a_n }{ b^n }+\frac{1}{ b^{n+1} } , \frac{ a_n }{ b^n }+\frac{2}{ b^{n+1} } \mathclose[\cup\ldots\cup\mathopen] \frac{ a_n }{ b^n }+\frac{ b-1 }{ b^{n+1} } , \frac{ a_n }{ b^n }+\frac{ 1 }{ b^n } \mathclose[.
	%       \end{equation}
	, pour simplifier les notations nous notons \( x'=x-\sum_{k=1}^{n}\frac{ a_k }{ b^k }\) et nous avons
	\begin{equation}
		0<x'<\frac{ a_n+1 }{ b^n }.
	\end{equation}
	Le nombre \( x'\) est forcément dans un des intervalles
	\begin{equation}
		\mathopen] \frac{ s }{ b^{n+1} } , \frac{ s+1 }{ b^{n+1} } \mathclose[
	\end{equation}
	avec \( s\in\{ 0,\ldots, b-1 \}\). Nous prenons le \( s\) correspondant à \( x'\) comme \( a_{n+1}\). Dans ce cas nous avons
	\begin{equation}
		\sum_{k=1}^{n+1}\frac{ a_k }{ b^k }< x<\sum_{k=1}^{n+1}\frac{ a_k }{ b^k }+\frac{1}{ b^{n+1} }.
	\end{equation}
	Note : les deux inégalités sont strictes. La première parce que si il y avait égalité, nous nous serions déjà arrêté en complétant avec des zéros. La seconde parce que
	\begin{equation}
		\sum_{k=n+2}^{\infty}\frac{ a_k }{ b^k }\leq \sum_{k=n+2}^{\infty}\frac{ b-1 }{ b^k }=\frac{1}{ b^{n+1} }
	\end{equation}
	où l'égalité n'est possible que si \( a_k=b-1\) pour tout \( k\geq n+2\). Dans ce cas nous aurions eu
	\begin{equation}
		x=\sum_{k=1}^{n}\frac{ a_k }{ b^k }+\frac{ a_{n+1}+1 }{ b^{n+1} }
	\end{equation}
	et nous aurions choisi le nombre \( a_{n+1}\) autrement et complété la suite par des zéros à partir de là. Notons que cela prouve au passage que la suite que nous sommes en train de construire est bien dans \( \eD_b\) parce qu'elle ne contiendra pas de queue de suite composée de \( b-1\).

	Ceci termine la construction par récurrence de la suite \( a\in \eD_b\). Par construction nous avons pour tout \( N\geq 1\),
	\begin{equation}
		\sum_{k=1}^N\frac{ a_k }{ b^k }\leq x\leq \sum_{k=1}^N\frac{ a_k }{ b^k }+\frac{1}{ b^{N+1} },
	\end{equation}
	autrement dit : \( \varphi_b(a_1,\ldots, a_N)\in B(x,\frac{1}{ b^{N+1} })\). Nous avons donc bien convergence
	\begin{equation}
		\lim_{N\to \infty} \varphi_b(a_1,\ldots, a_N)=x
	\end{equation}
	et l'application \( \varphi_b\) est surjective.
\end{proof}

L'application \( \varphi_b^{-1}\colon \mathopen[ 0 , 1 [\to \eD_b\) est la \defe{décomposition décimale}{décomposition décimale} en base \( b\) des nombres de \( \mathopen[ 0 , 1 [\).

Tout cela nous permet de montrer entre autres que \( \eR\) n'est pas dénombrable. Vu qu'il y a une bijection entre \( \mathopen[ 0 , 1 [\) et \( \eD_b\), il suffit de prouver que \( \eD_b\) est non dénombrable. De plus il suffit de démontrer que \( \eD_b\) est non dénombrable pour un entier \( b\geq 2\) donné.

\begin{proposition}[\cite{KZIoofzFLV}]  \label{PropNNHooYTVFw}
	Il n'existe pas de surjection \( \eN\to \eD_b\). Autrement dit \( \eD_b\) est non dénombrable.
\end{proposition}

\begin{proof}
	Nous prenons \( b\neq 2\) pour des raisons qui seront claires plus tard. Soit \( f\colon \eN\to \eD_b\). Pour \( i\in \eN\) nous notons
	\begin{equation}
		f(n)=(c_i^{(n)})_{i\geq 1},
	\end{equation}
	et nous définissons la suite
	\begin{equation}
		c_k=\begin{cases}
			0 & \text{si } c_k^{(k)}\neq 0 \\
			1 & \text{si } c_k^{(k)}=0.
		\end{cases}
	\end{equation}
	C'est une suite dans \( \eD_b\) parce que \( b\neq 2\) et que la suite ne contient que des \( 0\) et des \( 1\). Mais nous n'avons \( f(n)=c\) pour aucun \( n\in \eN\) parce que nous avons \( c_n\neq f(n)_n\).

	Si \( b=2\) alors nous savons que \( \eD_2\sim\mathopen[ 0 , 1 [\sim \eD_3\). Donc \( \eD_2\sim \eD_3\) et \( \eD_2\) ne peut pas plus être mis en bijection avec \( \eN\) que \( \eD_3\).
\end{proof}

\begin{remark}
	Le cas de la base \( b=2\) doit être fait à part parce que rien n'empêche d'avoir une queue de \( 1\). Il y a alors toutefois moyen de se débrouiller en construisant la suite \( c\) de façon plus subtile. Si \( b=2\) et \( n\in \eN\) alors \( f(n)\) est une suite de \( 0\) et \( 1\) contenant une infinité de \( 0\) (parce qu'il n'y a pas de queue de suite ne contenant que des \( 1\)). Nous construisons alors \( c\) de la façon suivante : d'abord nous recopions \( f(0)\) jusqu'à son \emph{deuxième} zéro que nous changeons en \( 1\); nommons \( n_0\) le rang de ce deuxième zéro. Ensuite nous recopions les éléments de \( f(1) \) à partir du rang \( n_0+1\) jusqu'au second zéro que nous changeons en \( 1\), etc.

	Le fait de prendre le deuxième zéro nous garantit que la suite \( c\) n'aura pas de queue de suite ne contenant que des \( 1\).

	Notons que cette construction s'adapte à tout \( b\); il suffit de prendre le second terme qui n'est pas \( b-1\) et le remplacer par \( b-1\).
\end{remark}

\begin{corollary}
	L'ensemble \( \mathopen[ 0 , 1 [\) n'est pas dénombrable.
\end{corollary}

\begin{proof}
	L'ensemble \( \mathopen[ 0 , 1 [\) est en bijection avec \( \eD_b\) que nous venons de prouver n'être pas dénombrable.
\end{proof}

%---------------------------------------------------------------------------------------------------------------------------
\subsection{Théorème de Banach-Steinhaus}
%---------------------------------------------------------------------------------------------------------------------------

\begin{lemma}[\cite{BIBooZUTUooNMvrdQ}]     \label{LEMooPIPLooMppGSO}
	Soient des espaces vectoriels normés \( X\) et \( Y\) ainsi qu'une application linéaire bornée \( T\colon X\to Y\). Pour tout \( a\in X\) et pour tout \( r>0\) nous avons
	\begin{equation}
		\sup_{x\in B(a,r)}\| Tx \|\geq r\| T \|
	\end{equation}
\end{lemma}

\begin{proof}
	Nous commençons avec \( a=0\). En utilisant la définition \ref{DefNFYUooBZCPTr} de la norme opérateur,
	\begin{equation}
		\| T \|=\sup_{x\in X}\frac{ \| Tx \| }{ \| x \| }=\sup_{x\in B(0,r)}\frac{ \| Tx \| }{ \| x \| }\leq \frac{1}{ r }\sup_{x\in B(0,r)}\| Tx \|.
	\end{equation}
	Donc
	\begin{equation}
		\sup_{x\in B(0,r)}\| Tx \|\geq r\| T \|.
	\end{equation}

	Il y a maintenant une astuce. Nous considérons un maximum :
	\begin{subequations}
		\begin{align}
			\max\{ \| T(a+x),\| T(a-x) \| \| \} & \geq \frac{ 1 }{2}\big( \| T(a+x) \|+\| T(a-x) \| \big) \label{SUBEQooPJPMooDkqRHs} \\
			                                    & \geq \frac{ 1 }{2}\big( \| T(a+x)-T(a+x) \| \big)      \label{SUBEQooEZUUooVlKtfn}  \\
			                                    & =\frac{ 1 }{2}\| T(2x) \|                                                           \\
			                                    & =\| Tx \|.
		\end{align}
	\end{subequations}
	Justifications :
	\begin{itemize}
		\item Pour \eqref{SUBEQooPJPMooDkqRHs}, la moyenne est plus petite que le maximum.
		\item Pour \eqref{SUBEQooEZUUooVlKtfn}, inégalité triangulaire : \( \| \alpha-\beta \|\leq \| \alpha \|+\| \beta \|\).
	\end{itemize}
	Si maintenant \( y\in B(a,r)\), nous avons \( y=a+x\) pour un certain \( x\in B(0,r)\), donc
	\begin{subequations}
		\begin{align}
			\sup_{y\in B(a,r)}\| Ty \| & =\sup_{x\in B(0,r)}\| T(a+x) \|                                                            \\
			                           & =\sup_{x\in B(0,r)}\max\{ \| T(a+x) \|, \| T(a-x) \| \}        \label{SUBEQooACJSooTHCAWs} \\
			                           & \geq \sup_{x\in B(0,r)}\| Tx \|                                                            \\
			                           & \geq r\| T \|.
		\end{align}
	\end{subequations}
	Pour \eqref{SUBEQooACJSooTHCAWs}, l'ensemble sur lequel nous prenons le supremum n'est pas modifié fondamentalement si nous regroupons les éléments deux à deux en prenant le maximum : les éléments exclus sont majorés.
\end{proof}

Une version avec des seminormes sera le théorème \ref{ThoNBrmGIg}.
\begin{theorem}[Théorème de Banach-Steinhaus\cite{BIBooZUTUooNMvrdQ}]       \label{THOooJHVNooIDDxyT}
	Soient un espace de Banach\footnote{Définition \ref{DefVKuyYpQ}.} \( X\) et un espace vectoriel normé \( Y\). Soit une famille \( \mF\) d'opérateurs linéaire bornés. Si pour tout \( x\in  X\),
	\begin{equation}
		\sup_{T\in\mF}\| Tx \|<\infty,
	\end{equation}
	alors
	\begin{equation}
		\sup_{T\in \mF}\| T \|<\infty.
	\end{equation}
\end{theorem}

\begin{proof}
	Nous supposons que \( \sup_{T\in\mF}\| T \|=\infty\), de telle sorte que nous pouvons choisir une suite \( (T_n)\) dans \( \mF\) telle que \( \| T_n \|\to \infty\). Cette suite peut diverger arbitrairement vite, et nous fixerons exactement cela plus tard.

	Soit par ailleurs une suite \( \alpha_n>0\) d'éléments petits et tels que \( \alpha_n\to 0\). Nous supposons que \( \sum_{n=0}^{\infty}\alpha_n<\infty\).

	Si \( a\in X\), le lemme \ref{LEMooPIPLooMppGSO} dit que
	\begin{equation}
		\sup_{x\in B(a,\alpha_n)}\| T_nx \|\geq \| T_n \|\alpha_n.
	\end{equation}
	En posant \( x_0=0\), nous construisons une suite \( (x_n)\) par récurrence en imposant
	\begin{enumerate}
		\item
		      \( x_n\in B(x_{n=1}, \alpha_n)\)
		\item
		      \( \| T_nx_n \|\geq \| T_n \|\alpha_n\).
	\end{enumerate}
	En utilisant une série télescopique et l'inégalité triangulaire \( \| x_k-x_{k+1} \|\leq \alpha_n\) à chaque étage,
	\begin{equation}
		\| x_p-x_q \|\leq \sum_{k=p}^q\alpha_k\leq \sum_{k=p}^{\infty}\alpha_k.
	\end{equation}
	Mais puisque la somme des \( \alpha_n\) converge, la suite des queues de somme converge vers zéro\footnote{Lemme \ref{LEMooHUZEooSyPipb}\ref{ITEMooQNHMooUPjupB}.} : \( \lim_{p\to \infty}\sum_{k=p}^{\infty}\alpha_n=0\). Cela implique que \( (x_n)\) est une suite de Cauchy\footnote{Proposition \ref{PROPooZZNWooHghltd}.}. Vu que \( X\) est de Banach, la suite \( (x_n)\) a une limite dans \( X\). Soit \( x\) cette limite.

	Nous avons \( \beta_n=\| x_n-x \|\to 0\). Il y aurait moyen de calculer \( \beta_n\) en fonction de \( \alpha_n\) (surtout si nous avions donné une forme explicite à \( \alpha_n\)), mais c'est sans importance ici. L'important est que c'est une suite qui tend vers zéro.

	Nous avons
	\begin{equation}
		x\in B(x_n,\beta_n),
	\end{equation}
	et donc il existe \( a_n\in B(0,\beta_n)\) tel que \( x=x_n+a_n\). Avec cela, pour chaque \( n\) nous avons :
	\begin{subequations}
		\begin{align}
			\| T_nx \| & =\| T_n(x_n+a_n) \|                                                   \\
			           & \geq\| T_nx_n \|-\| T_na_n \|                                         \\
			           & \geq \| T_nx_n \|-\| T_n \|\beta_n    \label{SUBEQooPLVQooChVCLU}     \\
			           & \geq \| T_n \|\alpha_n-\| T_n \|\beta_n =\| T_n \|(\alpha_n-\beta_n).
		\end{align}
	\end{subequations}
	Pour \ref{SUBEQooPLVQooChVCLU}, nous avons utilisé \( \| T_na_n \|\leq \| T_n \|\beta_n\). En résumé,
	\begin{equation}
		\| T_nx \|\geq \| T_n \|(\alpha_n-\beta_n).
	\end{equation}
	Il suffit de choisir \( \| T_n \|\) suffisamment rapidement croissant pour que\footnote{Le point important ici est que \( \alpha_n\) (et donc \( \beta_n\)) est choisi sans référence à \( \| T_n \|\).}
	\begin{equation}
		\| T_n \|(\alpha_n-\beta_n)\to \infty,
	\end{equation}
	et nous avons \( \| T_nx \|\to \infty\), qui est contraire aux hypothèses.
\end{proof}

\begin{theorem}[Théorème de Banach-Steinhaus\cite{KXjFWKA,VPvwAaQ}] \label{ThoPFBMHBN}
	Soit \( E\) un espace de Banach\footnote{Définition~\ref{DefVKuyYpQ}.} et \( F\) un espace vectoriel normé. Nous considérons une partie \( H\subset \aL_c(E,F)\) (espace des fonctions linéaires continues). Alors \( H\) est uniformément borné si et seulement si il est simplement borné.
\end{theorem}
\index{théorème!Banach-Steinhaus}
\index{application!linéaire!théorème de Banach-Steinhaus}

\begin{proof}
	Si \( H\) est uniformément borné, il est borné; pas besoin de rester longtemps sur ce sens de l'équivalence. Supposons donc que \( H\) soit borné. Pour chaque \( k\in \eN^*\) nous considérons l'ensemble
	\begin{equation}
		\Omega_k=\{ x\in E\tq \sup_{f\in H}\| f(x) \|>k \}.
	\end{equation}

	\begin{subproof}
		\item[Les \( \Omega_k\) sont ouverts]
		Soit \( x_0\in \Omega_k\); nous avons alors une fonction \( f\in H\) telle que \(  \| f(x_0) \|>k \), et par continuité de \( f\) il existe \( \rho>0\) tel que \( \| f(x) \|>k\) pour tout \( x\in B(x_0,\rho)\). Par conséquent \( B(x_0,\rho)\subset \Omega_k\) et \( \Omega_k\) est ouvert par le théorème~\ref{ThoPartieOUvpartouv}.

		\item[Les \( \Omega_k\) ne sont pas tous denses dans \( E\)]
		Nous supposons que les ensembles \( \Omega_k\) soient tous dense dans \( E\). Le théorème de Baire~\ref{ThoBBIljNM} nous indique que \( E\) est un espace de Baire (parce que de Banach) et donc que
		\begin{equation}
			\overline{ \bigcap_{k\in \eN}\Omega_k }=E.
		\end{equation}
		En particulier l'intersection des \( \Omega_k\) n'est pas vide. Soit \( x_0\in \bigcap_{k\in \eN}\Omega_k\). Nous avons alors
		\begin{equation}
			\sup_{f\in H}\| f(x) \|=\infty,
		\end{equation}
		ce qui est contraire à l'hypothèse. Donc les ouverts \( \Omega_k\) ne sont pas tous denses dans \(E\).

		\item[La majoration]
		Il existe \( k\geq 0\) tel que \( \Omega_k\) ne soit pas dense dans \( E\), et nous voulons prouver que \( \{ \| f \|\tq f\in H \}\) est un ensemble borné. Soit donc \( k\geq 0\) tel que \( \Omega_k\) ne soit pas dense dans \( E\); il existe un \( x_0\in E\) et \( \rho>0\) tels que
		\begin{equation}
			B(x_0,\rho)\cap \Omega_k=\emptyset.
		\end{equation}
		Si \( x\in B(x_0,\rho)\) alors \( x\) n'est pas dans \( \Omega_k\) et donc
		\begin{equation}
			\sup_{f\in H}\| f(x) \|\leq k.
		\end{equation}
		Afin d'évaluer \( \| f \|\) nous devons savoir ce qu'il se passe avec les vecteurs sur une boule autour de \( 0\). Pour tout \( x\in B(0,\rho)\) et pour tout \( f\in H\), la linéarité de \( f\) donne
		\begin{equation}
			\| f(x) \|=\| f(x+x_0)-f(x_0) \|\leq \| f(x+x_0)+f(x_0) \|\leq 2k.
		\end{equation}
		Par continuité nous avons alors \( \| f(x) \|\leq 2k\) pour tout \( x\in \overline{ B(0,\rho) }\). Si maintenant \( x\in F\) vérifie \( \| x \|=1\) nous avons
		\begin{equation}
			\| f(x) \|=\frac{1}{ \rho }\| f(\rho x) \|\leq \frac{ 2k }{ \rho },
		\end{equation}
		et donc \( \| f \|\leq \frac{ 2k }{ \rho }\), ce qui montre que \( 2k/\rho\) est un majorant de l'ensemble \( \{ \| f \|\tq f\in H \}\).
	\end{subproof}

\end{proof}
Une application du théorème de Banach-Steinhaus est l'existence de fonctions continues et périodiques dont la série de Fourier ne converge pas. Ce sera l'objet de la proposition~\ref{PropREkHdol}.

%---------------------------------------------------------------------------------------------------------------------------
\subsection{Convergence forte}
%---------------------------------------------------------------------------------------------------------------------------

Lorsque nous avons une suite d'opérateurs linéaires, nous pouvons considérer la convergence d'une suite pour la norme opérateur : \( A_k\to A\) lorsque \( \| A_k-A \|\to 0\).

\begin{definition}[\cite{ooAGRZooTyUUVy}]       \label{DEFooNREQooElLvec}
	Soient un espace vectoriel \( E\) et un espace vectoriel normé \( V\). Nous disons que la suite d'opérateur \( T_k\colon E\to V\) \defe{converge fortement}{convergence forte} vers l'opérateur $T$ si pour tout \( x\in E\) nous avons
	\begin{equation}
		\| T_kx-Tx \|\to 0.
	\end{equation}
\end{definition}

Cette notion s'appelle \emph{forte} par opposition à la convergence \emph{faible} dont nous ne parlerons pas. Elle est cependant moins forte que la convergence en norme dont nous avons déjà parlé.

\begin{proposition}     \label{PROPooRFBLooUjSirP}
	Soient des espaces vectoriels normés \( E\) et \( F\) et une suite d'opérateurs \( T_k\colon E\to F\) convergeant vers \( T\)\footnote{Sans précisions, ce sera toujours la convergence en norme.}. Alors cette suite converge également fortement.
\end{proposition}

\begin{proof}
	Soit \( x\in E\) que nous supposons non nul. Soit \( \lambda\in \eC\) tel que \( x=\lambda y\) avec \( \| y \|=1\). Nous avons
	\begin{equation}
		\| T_kx-Tx \|=| \lambda |\| T_ky-Ty \|\leq | \lambda |\sup_{\| z \|=1}\| T_kz-Tz \|=| \lambda |\| T_k-T \|\to 0.
	\end{equation}
	La dernière étape est la convergence en norme \( T_k\to T\).
\end{proof}

\begin{proposition}
	Soient \( E\) et \( F\), des espaces vectoriels normés de dimension finie. Soit une suite \( (A_n)\) d'applications linéaires \( E\to F\). Si elle converge fortement vers \( A\), alors elle converge en norme vers \( A\).
\end{proposition}

\begin{proof}
	En plusieurs coups.
	\begin{subproof}
		\item[Si une sous-suite converge]
		Commençons par montrer que si \( (B_n)\) est une sous-suite de \( (A_n)\) qui converge vers \( B\), alors \( B=A\). Autrement dit, \( A\) est le seul candidat limite pour \( A_n\).

		Soit \( \| x \|=1\). Nous avons
		\begin{equation}
			\| B_nx-Bx \|\leq \| B_n-B \|\| x \|=\| B_n-B \|,
		\end{equation}
		mais pour la sous-suite \( (B_n)\) nous avons supposé \( \| B_n-B \|\to 0\). Donc \( \| B_nx-Bx \|\to 0\), ce qui signifie que \( B_nx\to Bx\). Mais par hypothèse, \( B_nx\to Ax\). Par unicité de la limite, \( Bx=Ax\) pour tout \( x\) de norme \( 1\). Pour les autres \( x\), c'est la linéarité qui conclu.

		\item[Utilisation de deux gros résultats]
		Par l'hypothèse de convergence, pour chaque \( x\) nous avons \( \sup_n\| A_nx \|<\infty\). Le théorème de Banach-Steinhaus \ref{THOooJHVNooIDDxyT} nous indique alors que l'ensemble \( \mF=\{ A_n \}_{n\in \eN}\) est borné. Il existe donc \( M > 0\) tel que \( \| A_n \|< M\) pour tout \( n\).

		Nous utilisons à présent l'hypothèse de dimension finie en disant que l'espace des applications linéaires \( E\to F\) est de dimension finie, de telle sorte que ses boules fermées soient compactes.

		Donc la suite \( (A_n)\) est contenue dans un compact.

		\item[Les sous-suite convergentes]
		La suite \( (A_n)\) est contenue dans un compact. Toutes ses sous-suites sont dans ce compact et possèdent donc une sous-suite convergente (théorème \ref{ThoBWFTXAZNH}). Toutes ces sous-sous-suites convergent nécessairement vers \( A\) par ce que nous avons dit dans la première étape de la preuve. Le lemme \ref{LEMooSJKMooKSiEGq} nous dit alors que \( A_n\to A\).
	\end{subproof}
\end{proof}


%+++++++++++++++++++++++++++++++++++++++++++++++++++++++++++++++++++++++++++++++++++++++++++++++++++++++++++++++++++++++++++
\section{Application ouverte}
%+++++++++++++++++++++++++++++++++++++++++++++++++++++++++++++++++++++++++++++++++++++++++++++++++++++++++++++++++++++++++++

\begin{definition}[application ouverte]
	Soient deux espaces topologiques \( X\) et \( Y\). Une application \( f\colon X\to Y\) est \defe{ouverte}{application ouverte} si l'image de tout ouvert de \( X\) par \( f\) est un ouvert de \( Y\).

	Nous disons que \( f\) est ouverte en \( a\in X\) si l'image de tout ouvert contenant \( a\) est ouverte.
\end{definition}

\begin{proposition}     \label{PROPooXGEGooHoMsne}
	Une application bijective est ouverte si et seulement si son inverse est continue.
\end{proposition}

\begin{proof}
	Ce n'est seulement que la définition, mais pour le sport nous démontrons le sens direct.

	Soit donc une application \( f\colon X\to Y\) bijective et ouverte entre les espaces topologiques \( X\) et \( Y\). Prouvons que \( f^{-1}\colon Y\to X\) est continue. Pour cela nous considérons un ouvert \( \mO\) dans \( X\), et nous prouvons que \( (f^{-1})^{-1}(\mO)\) est ouvert dans \( Y\). Par définition de l'inverse, \( (f^{-1})^{-1}(\mO)=f(\mO)\) et vu que \( f\) est ouverte, \( f(\mO)\) est ouvert.
\end{proof}

\begin{lemma}       \label{LEMooHHIPooEpGfCg}
	Une application \( f\colon X\to Y\) est ouverte si et seulement si pour tout \( x\in X\) et pour tout voisinage \( U\) de \( x\), la partie \( f(U)\) est un voisinage de \( f(x)\).
\end{lemma}

\begin{proof}
	La preuve suit celle de la proposition \ref{PROPooOXBCooIzLaPe}. Le sens direct est un à fortiori.

	Dans l'autre sens. Soit un ouvert \( \mO\) de \( X\). Pour prouver que \( f(\mO)\) est ouvert, nous considérons \( y\in f(\mO)\), ainsi que \( x\in\mO\) tel que \( f(x)=y\). Vu que \( \mO\) est un voisinage de \( x\), la partie \( f(\mO)\) est un voisinage de \( y=f(x)\).

	Il existe donc un ouvert \( V\) de \( Y\) tel que \( y=f(x)\in V\subset f(\mO)\). Donc la partie \( f(\mO)\) contient un ouvert autour de chacun de ses points, et elle est ouverte par le théorème \ref{ThoPartieOUvpartouv}.
\end{proof}


\begin{lemma}[\cite{BIBooNJJUooDaGnPZ}]
	Une application linéaire entre espaces vectoriels topologiques est ouverte si et seulement si elle est ouverte en \( 0\).
\end{lemma}

\begin{proof}
	Le sens direct est un à fortiori.

	Soit un ouvert \( \mO\) et \( a\in \mO\). La partie \( \mO-a\) est ouverte et contient \( 0\). Donc \( f(\mO-a)\) est un ouvert parce que \( f\) est ouverte en \( 0\). Nous en déduisons, par linéarité, que \( f(\mO)-f(a)\) est ouvert et donc que \( f(\mO)\) est ouverte.
\end{proof}

\begin{lemma}[\cite{BIBooNJJUooDaGnPZ}]
	Soient des espaces vectoriels normés \( E\) et \( F\). Une application linéaire ouverte \( f\colon E\to F\) est surjective.
\end{lemma}

\begin{proof}
	Soit un ouvert \( B(0,r)\) dans \( E\). Puisque \( f\) est ouverte, la partie \( f\big( B(0,r) \big)\) est ouverte dans \( F\), et contient donc une boule \( B_F(0,r')\) pour un certain \( r'>0\).

	Soit \( v\in F\). Nous considérons
	\begin{equation}
		v'=r'\frac{ v }{ 2\| v \| }.
	\end{equation}
	Nous avons \( \| v' \|=r'/2\), et donc \( v'\in B_F(0,r')\). Il existe donc \( x\in E\) (et même dans \( B_E(0,r)\)) tel que \( f(x)=v'\). Nous avons alors
	\begin{equation}
		f\big( \frac{ 2\| v \| }{ r' }x \big)=v,
	\end{equation}
	ce qui prouve que \( v\) est dans l'image de \( f\), et donc que \( f\) est surjective.
\end{proof}

\begin{theorem}[théorème de l'application ouverte\cite{BIBooNJJUooDaGnPZ, BIBooYJXXooTvzpDW,BIBooMKAVooDLCzUX}]     \label{THOooATZKooXHWCRD}
	Soient des espaces de Banach\footnote{Espace de Banach : vectoriel, normé, complet. Définition \ref{DefVKuyYpQ}.} \( E\) et \( F\). Si l'application \( f\colon E\to F\) est linéaire, surjective et continue, alors elle est ouverte.
\end{theorem}

\begin{proof}
	En plusieurs étapes.
	\begin{subproof}
		\item[Une union de fermés]
		Soit \( y\in F\). Comme \( f\) est surjective, il existe \( x\in E\) tel que \( y=f(x)\). Soit \( n\in \eN\) tel que \( x\in B(0,n)\). Nous avons alors
		\begin{equation}
			y\in f\big( B(0,1) \big)\subset \overline{ f\big( B(0,n) \big) }
		\end{equation}
		En notant
		\begin{equation}
			F_n=\overline{ f\big( B_E(0,n) \big) },
		\end{equation}
		nous avons
		\begin{equation}
			F=\bigcup_{n=0}^{\infty}F_n.
		\end{equation}
		\item[Théorème de Baire]
		Le théorème \ref{ThoBBIljNM} nous indique que \( F\) est un espace de Baire. Le lemme \ref{LEMooTOJDooQDtWUC} nous dit alors qu'il existe un \( n\) tel que \( F_n\) soit d'intérieur non vide. Mettons \( F_N\) d'intérieur non vide.
		\item[Boule unité]
		Puisque \( F_N\) est d'intérieur non vide, il existe \( y\in F_N\) et \( \eta>0\) tels que \( B_F(y,\eta)\subset F_N\). Nous avons aussi
		\begin{equation}
			B_F(0,\eta)=B_F(y,\eta)-y,
		\end{equation}
		et comme \( y\in F_N\) nous avons \( B_F(0,\eta)\subset F_N-F_N\), et vu qu'en plus \( -F_N=F_N\), nous avons
		\begin{equation}
			B_F(0,\eta)\subset 2F_N=\overline{ f\big( B_E(0,2N) \big) }.
		\end{equation}
		Nous avons ensuite
		\begin{equation}
			B_F(0,1)=\frac{1}{ \eta }B_F(0,\eta)\subset\frac{1}{ \eta }\overline{ f\big( B_E(0,2N) \big) }=\overline{ f\big( B_E(0,2N/\eta) \big) }.
		\end{equation}
		Ceci pour dire qu'il existe un \( M\in \eR\) tel que
		\begin{equation}
			B_F(0,1)\subset \overline{ f\big( B_E(0,M) \big) }.
		\end{equation}
		Nous avons de même que
		\begin{equation}        \label{EQooCMSPooYtzAuC}
			B_F\big( 0,\frac{1}{ 2^n } \big)\subset \overline{ f\big( B_E(0,M/2^n) \big) }.
		\end{equation}
		Nous voudrions maintenant avoir la même inclusion sans la fermeture.

		\item[Une suite par récurrence]
		Soit \( z\in B_F(0,1)\). Nous allons définir par récurrence une suite \( (x_n)\) dans \( E\) telle que
		\begin{subequations}        \label{SUBEQSooLJEMooOaFncH}
			\begin{numcases}{}
				x_n\in B_E\big( 0,\frac{ M }{ 2^{n-1} } \big)\\
				\| z-f(x_1+\ldots +x_n) \|<\frac{1}{ 2^n }.
			\end{numcases}
		\end{subequations}
		\begin{subproof}
			\item[Le premier élément]
			Puisque \( z\in B_F(0,1)\subset\overline{ f\big( B_E(0,M) \big) }\), nous avons
			\begin{equation}
				B(z,\frac{ 1 }{2})\cap f\big( B_E(0,M) \big)\neq \emptyset.
			\end{equation}
			Nous pouvons donc considérer \( x_1\in B_E(0,M)\) tel que \( f(x_1)\in B(z,\frac{ 1 }{2})\).

			Ce \( x_1\) vérifie les conditions \eqref{SUBEQSooLJEMooOaFncH}.
			\item[La récurrence]
			En utilisant l'hypothèse de récurrence et \eqref{EQooCMSPooYtzAuC},
			\begin{equation}
				z-f(x_1+\ldots +x_n)\in B_F\big( 0,\frac{1}{ 2^n } \big)\subset\overline{ f\big( B_E(0,M/2^n) \big) },
			\end{equation}
			de telle sorte que
			\begin{equation}
				B_F\big( z-f(x_1,\ldots, x_n),\frac{1}{ 2^{n+1} } \big)\cap f\big( B_E(0,M/2^n) \big)\neq \emptyset.
			\end{equation}
			Nous pouvons donc considérer \( x_{n+1}\in B_E(0,M/2^n)\) tel que
			\begin{equation}
				f(x_{n+1})\in B_F\big( z-f(x_1+\ldots +x_n),\frac{1}{ 2^{n+1} } \big).
			\end{equation}
			Donc
			\begin{equation}
				z-f(x_1+\ldots +x_n)-f(x_{n+1})\in B_F\big( 0,\frac{1}{ 2^{n+1} } \big).
			\end{equation}
			Nous avons donc bien
			\begin{equation}
				\| z-f(x_1+\ldots +x_{n+1}) \|<\frac{1}{ 2^{n+1} }.
			\end{equation}
		\end{subproof}
		\item[Convergence]
		Nous avons, pour tout \( n\), \( \| x_n \|<\frac{ M }{ 2^{n-1} }\). Donc la série
		\begin{equation}
			\sum_{n=1}^{\infty}\| x_n \|\leq \sum_{n=1}^{\infty}\frac{ M }{ 2^{n-1} }
		\end{equation}
		converge. Autrement dit, la série des \( x_n\) converge absolument\footnote{Définition \ref{DefVFUIXwU}.}. Puisque \( E\) est une espace de Banach, la proposition \ref{PropAKCusNM} nous dit que \( \sum_{n=1}^{\infty}x_n\) converge dans \( E\). Nous posons
		\begin{equation}
			x=\sum_{n=1}^{\infty}x_n.
		\end{equation}
		En utilisant la série géométrique de la proposition \ref{PROPooWOWQooWbzukS}\ref{ITEMooBJHBooBMEmiG}, nous trouvons
		\begin{equation}
			\| x \|\leq \sum_{k=1}^{\infty}\| x_k \|\leq M\sum_{k=1}^{\infty}\frac{1}{ 2^{n+1} }=M\sum_{k=0}^{\infty}\frac{1}{ 2^k }=2M.
		\end{equation}
		\item[Passage à la limite]
		Nous avons \( x\in B_E(0,2M)\), et
		\begin{equation}
			\lim_{n\to \infty} \| z-f(x_1+\ldots +x_n) \|=0.
		\end{equation}
		Puisque \( \| . \|\), \( t\mapsto z-t\) et \( f\) sont continue\footnote{Oui, la continuité de \( f\) est une hypothèse en plus de sa linéarité parce que nous n'avons pas d'hypothèses sur la dimension de \( E\) et \( F\).}, nous pouvons rentrer la limite de partout et écrire
		\begin{equation}
			\| z-f(x) \|=0,
		\end{equation}
		ce qui signifie que \( z=f(x)\). Comme \( z\) est un élément arbitraire de \( B_F(0,1)\) nous avons prouvé que
		\begin{equation}
			B_F(0,1)\subset f\big( B_E(0,2M) \big).
		\end{equation}
		Nous avons donc aussi que pour tout \( r>0\), il existe \( r'\) tel que
		\begin{equation}
			B_F\big( 0, r \big)\subset f\big( B_E(0,r) \big).
		\end{equation}
		En l'occurrence, \( r'=r/2M\).
		\item[Passage aux voisinages]
		Nous montrons que l'image de tout voisinage de \( x\in E\) contient un voisinage de \( f(x)\) dans \( F\). Soit \( x\in E\) ainsi qu'un voisinage \( V\) de \( x\). Il existe \( r>0\) tel que \( B(x,r)\subset V\). Vu que \( f\) est linéaire,
		\begin{equation}
			f\big( B(x,r) \big)=f(x)+f\big( B(0,r) \big),
		\end{equation}
		et il existe un \( r'\) tel que \( B_F(0,r')\subset f\big( B_E(0,r) \big)\). Cela pour dire que
		\begin{equation}
			f(x)+B_F(0,r')\subset f\big( B(0,r) \big)\subset f(V).
		\end{equation}
		Vu que \( f(x)+B_F(0,r')\) est un ouvert autour de \( f(x)\), nous avons prouvé que \( f(V)\) contient un ouvert autour de \( f(x)\), c'est-à-dire que \( f(V)\) est un voisinage de \( f(x)\).
		\item[Conclusion]
		Le lemme \ref{LEMooHHIPooEpGfCg} conclu que \( f\) est ouverte.
	\end{subproof}

\end{proof}
