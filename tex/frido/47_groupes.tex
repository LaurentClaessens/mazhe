% This is part of Mes notes de mathématique
% Copyright (c) 2011-2019
%   Laurent Claessens
% See the file fdl-1.3.txt for copying conditions.


%+++++++++++++++++++++++++++++++++++++++++++++++++++++++++++++++++++++++++++++++++++++++++++++++++++++++++++++++++++++++++++
\section{Produit semi-direct de groupes}
%+++++++++++++++++++++++++++++++++++++++++++++++++++++++++++++++++++++++++++++++++++++++++++++++++++++++++++++++++++++++++++

\begin{definition}
    Une \defe{suite exacte}{suite!exacte} est une suite d'applications comme suit :
    \begin{equation}
        \xymatrix{%
        \cdots \ar[r]^{f_i}&A_i\ar[r]^{f_{i+1}}& A_{i+1}\ar[r]^{f_{i+2}}&\ldots
           }
    \end{equation}
    où pour chaque \( i\), les applications \( f_i\) et \( f_{i+1}\) vérifient \( \ker(f_{i+1})=\Image(f_i)\). Lorsque les ensembles \( A_i\) sont des groupes, alors nous demandons de plus que les \( f_i\) soient de homomorphismes.
\end{definition}

Très souvent nous sommes confrontés à des suites exactes de la forme
\begin{equation}
    \xymatrix{%
    1 \ar[r]& A\ar[r]^f&G\ar[r]^g&B\ar[r]&1
       }
\end{equation}
où \( G\), \( A\) et \( B\) sont des groupes, \( 1\) est l'identité. La première flèche est l'application \( \{ 1 \}\to A\) qui à \( 1\) fait correspondre \( 1\). La dernière est l'application \( B\to 1\) qui à tous les éléments de \( B\) fait correspondre \( 1\). Le noyau de \( f\) étant l'image de la première flèche (c'est-à-dire \( \{ 1 \}\)), l'application \( f\) est injective. L'image de \( g\) étant le noyau de la dernière flèche (c'est-à-dire \( B\) en entier), l'application \( g\) est surjective.

\begin{definition}     \label{DEFooKWEHooISNQzi}
    Soient \( N\) et \( H\) deux groupes et un morphisme de groupes \( \phi\colon H\to \Aut(N)\). Le \defe{produit semi-direct}{produit!semi-direct} de \( N\) et \( H\) relativement à \( \phi\), noté \( N\times_{\phi}H\)\nomenclature[R]{\( N\times_{\phi}H\)}{produit semi-direct} est l'ensemble \( N\times H\) muni de la loi (que l'on vérifiera être de groupe)
    \begin{equation}\label{EqDRgbBI}
        (n,h)\cdot (n',h')=(n\phi_h(n'),hh').
    \end{equation}
\end{definition}
Attention à l'ordre quelque peu contre-intuitif. Lorsque nous notons \( N\times_{\phi}H\), c'est bien \( \phi\colon H\to \Aut(N)\), c'est-à-dire \( H\) qui agit sur \( N\) et non le contraire.

Lorsque \( N\) et \( H\) sont des sous-groupes d'un même groupe, le plus souvent \( \phi\) est l'action adjointe définie en~\ref{DEFooCORTooEeOLPT}.

Le théorème suivant permet de reconnaitre un produit semi-direct lorsqu'on en voit un.
\begin{theorem}[\cite{MathAgreg}]       \label{THOooZNYTooPhnIdE}
    Soit une suite exacte de groupes
    \begin{equation}
    \xymatrix{%
    1 \ar[r]        & N\ar[r]^i&G\ar[r]^s&H\ar[r]&1
       }
    \end{equation}
    Si il existe un sous-groupe \( \tilde H\) de \( G\) à partir duquel \( s\) est un isomorphisme, alors
    \begin{equation}
        G\simeq i(N)\times_{\sigma}\tilde H
    \end{equation}
    où \( \sigma\) est l'action adjointe\footnote{Le fait que \( H\) agisse sur \( i(N)\) fait partie du théorème.} de \( \tilde H\) sur \( i(N)\).
\end{theorem}

\begin{proof}
    Nous posons \( \tilde N=i(N)\) et nous allons subdiviser la preuve en petits pas.

    \begin{enumerate}
        \item  \( \tilde N\) est normal dans \( G\). En effet étant donné que la suite est exacte nous avons \( \tilde N=\ker(s)\). Le noyau d'un morphisme est toujours un sous-groupe normal.

        \item \( \tilde N\cap\tilde H=\{ e \}\). L'application \( s\) étant un isomorphisme depuis $\tilde H$, il n'y a pas d'éléments de \( \tilde H\) dans \( \ker(s)\) autre que $e$.

        \item\label{ItemzIaXGM} \( G=\tilde N\tilde H\). Nous considérons \( g\in G\) et \( h\in \tilde H\) tel que \( s(g)=s(h)\). L'existence d'un tel \( h\) est assurée par le fait que \( s\) est surjective depuis \( \tilde H\). Du coup nous avons \( e=s(gh^{-1})\), c'est-à-dire \( gh^{-1}\in \ker (s)=\tilde N\). Nous avons donc bien la décomposition \( g=(gh^{-1})h\), et donc \( G=\tilde N\tilde H\).

        \item\label{ItemUGFjle} L'écriture \( g=nh\) avec \( n\in \tilde N\) et \( h\in \tilde H\) est unique. Si \( nh=n'h'\), alors \( n=n'h'h^{-1}\), ce qui signifierait que \( h'h^{-1}\in\tilde N\). Mais étant donné que \( \tilde H\cap\tilde N=\{ e \}\), nous obtenons \( h=h'\) et par suite \( n=n'\).

        \item   \label{ItemUZlrKo}
            L'application
            \begin{equation}
                \begin{aligned}
                    \phi\colon G&\to \tilde N\times \tilde H \\
                    nh&\mapsto (n,h)
                \end{aligned}
            \end{equation}
            est une bijection. C'est une conséquence des points~\ref{ItemzIaXGM} et~\ref{ItemUGFjle}.

        \item
            Si sur \( \tilde N\times \tilde H\) nous mettons le produit
            \begin{equation}
                (n,h)\cdot(n',h')=(n\sigma_hn',hh')
            \end{equation}
            où \( \sigma\) est l'action adjointe du groupe sur lui-même, c'est-à-dire \( \sigma_x(y)=xyx^{-1}\), alors \( \phi\) est un isomorphisme. Si \( g,g'\in G\) s'écrivent (de façon unique par le point~\ref{ItemUZlrKo}) \( g=nh\) et \( g'=n'h'\) alors
            \begin{subequations}
                \begin{align}
                    \phi(nhn'h')&=\phi(n\underbrace{hn'h^{-1}}_{\in \tilde N}hh')\\
                    &=\phi\big( (nhn'h^{-1})(hh') \big)\\
                    &=(nhn'h^{-1},hh')\\
                    &=(n,h)\cdot(n',h')\\
                    &=\phi(nh)\phi(n'h').
                \end{align}
            \end{subequations}
    \end{enumerate}
\end{proof}

\begin{corollary}\label{CoroGohOZ}
    Soit \( G\) un groupe, et \( N,H\) des sous-groupes de \( G\) tels que
    \begin{enumerate}
        \item
            \( H\) normalise \( N\) (c'est-à-dire que \( hnh^{-1}\in N\) pour tout \( h\in H\) et \( n\in N\)\footnote{Ou encore que \( H\) agit sur \( N\) par automorphismes internes.}),
        \item
            \( H\cap N=\{ e \}\),
        \item
            \( HN=G\).
    \end{enumerate}
    Alors l'application
    \begin{equation}
        \begin{aligned}
            \psi\colon N\times_{\sigma}H&\to G \\
            (n,h)&\mapsto nh
        \end{aligned}
    \end{equation}
    est un isomorphisme de groupes.
\end{corollary}
Dans les hypothèses, l'ordre entre \( N\) et \( H\) est important lorsqu'on dit que c'est \( N\) qui agit sur \( H\); mais l'hypothèse \( NH=G\) est équivalente à \( HN=G\) (passer à l'inverse pour s'en assurer).

Insistons encore un peu sur la notation : dans \( N\times_{\sigma}H\), c'est \( H\) qui agit sur \( N\) par \( \sigma\).

%+++++++++++++++++++++++++++++++++++++++++++++++++++++++++++++++++++++++++++++++++++++++++++++++++++++++++++++++++++++++++++
\section{Groupe de torsion}
%+++++++++++++++++++++++++++++++++++++++++++++++++++++++++++++++++++++++++++++++++++++++++++++++++++++++++++++++++++++++++++

Soit \( G\) un groupe. Un élément \( g\in G\) est un \defe{élément de torsion}{element@élément!de torsion} si il est d'ordre fini. La \defe{torsion}{torsion!d'un groupe} de \( G\) est l'ensemble de ses éléments de torsion. Nous disons qu'un groupe est un \defe{groupe de torsion}{groupe!de torsion} si tous ses éléments sont de torsion.

\begin{example}
    Le groupe additif \( \eQ/\eZ\) est un groupe de torsion parce que si \( [x]=[p/q]\), alors \( q[x]=[p]=[0]\).
\end{example}

\section{Famille presque nulle}
%+++++++++++++++++++++++++++++++++++++++++++++++++++++++++++++++++++++++++++++++++++++++++++++++++++++++++++++++++++++++++++

Soit \( (G,+)\) un groupe abélien et \( \mF=\{ g_i \}_{i\in I}\) une famille d'éléments de \( G\) indicés par un ensemble \( I\). Le \defe{support}{support!famille d'éléments} de \( \mF\) est l'ensemble \( \{ i\in I\tq g_i\neq 0 \}\). La famille est dite \defe{presque nulle}{presque!nulle} si le support est fini.

Nous disons que \( \mF\) est une \defe{suite}{suite} si \( I=\eN\).

