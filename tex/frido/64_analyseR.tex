% This is part of Mes notes de mathématique
% Copyright (c) 2008-2024
%   Laurent Claessens
% See the file fdl-1.3.txt for copying conditions.

%+++++++++++++++++++++++++++++++++++++++++++++++++++++++++++++++++++++++++++++++++++++++++++++++++++++++++++++++++++++++++++
\section{Intervalles}
%+++++++++++++++++++++++++++++++++++++++++++++++++++++++++++++++++++++++++++++++++++++++++++++++++++++++++++++++++++++++++++

\begin{definition}[Intervalle]      \label{DEFooKRRYooZlwiEo}
	Une partie \( I\) de \( \eR\) est un \defe{intervalle}{intervalle} si pour tout \( a,b\in I\) nous avons \( t\in I\) dès que \( a\leq t\leq b\).
\end{definition}

\begin{proposition}     \label{PROPooJJRZooACUmWN}
	À propos d'intervalles.
	\begin{description}
		\item
		      Un intervalle\footnote{Définition \ref{DEFooKRRYooZlwiEo}.} est ouvert si il est de la forme \( \mathopen] a , b \mathclose[\) avec éventuellement \( a=-\infty\) ou \( b=+\infty\).
		\item
		      Un intervalle est fermé si il est de la forme \( \mathopen[ a , b \mathclose]\) ou \( \mathopen] -\infty , b \mathclose]\) ou \( \mathopen[ a , +\infty [\) avec \( a,b\in \eR\).
	\end{description}
\end{proposition}

\begin{remark}
	L'ensemble \( \eR\) ne contient pas \( +\infty\) et \( -\infty\). L'intervalle \( [-\infty, 5]\) par exemple, n'est pas une partie de \( \eR\).
\end{remark}

\begin{example}
	\begin{enumerate}
		\item
		      Les ensembles \( \mathopen] 3 , 7 \mathclose[\) et \( \mathopen] -\infty , \pi \mathclose[\) sont des intervalles ouverts.
		\item
		      Les ensembles \( \mathopen[ 10 , 15 \mathclose]\) et \( \mathopen[ -1 , +\infty [\) sont des intervalles fermés.
		\item
		      L'ensemble \( \mathopen] -4 , -2 \mathclose[\cup\mathopen] 2 , 9 \mathclose[\) n'est pas un intervalle (il y a un «trou» entre \(- 2\) et \( 2\)).
		\item
		      L'ensemble \( \eR\) lui-même est un intervalle; par convention, il est à la fois ouvert et fermé.
	\end{enumerate}
	Un intervalle peut n'être ni ouvert ni fermé; par exemple \( \mathopen] 4 , 8 \mathclose]\). Cet intervalle est «ouvert en \( 4\) et fermé en \( 8\)» .
\end{example}

\begin{definition}[Fonction, domaine, image, graphe]
	Soient \( X\) et \( Y\) deux ensembles. Une \defe{fonction}{fonction} \( f\) définie sur \( X\) et à valeurs dans \( Y\) est une partie de \( X\times Y\) telle que pour tout \( x\in X\), il existe un unique \( y\in Y\) tel que \( (x,y)\in f\).
\end{definition}

\begin{normaltext}
	Notons qu'il n'est pas demandé que pour tout \( y\), il existe \( x\) tel que \( (x,y)\in f\). Autrement dit, la notation «\( f\colon X\to Y\)» ne suppose pas que \( f\) est surjective sur \( Y\). Mais elle doit être définie sur tout \( X\).

	Nous écrivons \( y=f(x)\) pour dire \( (x,y)\in f\).
	\begin{itemize}
		\item La partie de \( X\) qui contient tous les \( x\) sur lesquels \( f\) peut opérer est dite \defe{domaine}{domaine} de \( f\). Le domaine de \( f\) est indiqué par \( \Dom f\).
		\item L'élément de \( y\in Y\) associé par \( f\) à un élément \( x\in \Dom f\) (c'est-à-dire \( f(x) = y\))  est appellé \defe{image}{image} de \( x\) par \( f\). L'\defe{image}{fonction!image} de la fonction \( f\) est la partie de \( Y\) qui contient les images de tous les éléments de \( \Dom f\). L'image de \( f\) est indiquée par \( \Im f\).
		\item Le \defe{graphe}{graphe} de \( f\) est l'ensemble de tous les couples \( (x, f(x))\) pour \( x\in \Dom f\). Le graphe de \( f\) est une partie de l'ensemble noté \( X\times Y\) et il est indiqué par \( \Graph f\). Dans ce chapitre \( X = \eR\) et \( Y = \eR\), donc le graphe de \( f\) est contenu dans le plan cartésien.
	\end{itemize}

\end{normaltext}

\begin{definition}[Fonction croissante, décroissante et monotone]
	Soit \( f\colon \eR\to \eR\) une fonction définie sur un intervalle \( I\subset \eR\).
	\begin{enumerate}
		\item
		      La fonction \( f\) est \defe{croissante}{fonction!croissante} sur \( I\) si pour tout \( x<y\) dans \( I\) nous avons \( f(x)\leq f(y)\). Elle est \emph{strictement} croissante si \( f(x)<f(y)\) dès que \( x<y\).
		\item
		      La fonction \( f\) est \defe{décroissante}{fonction!décroissante} sur \( I\) si pour tout \( x<y\) dans \( I\) nous avons \( f(x)\geq f(y)\). Elle est \emph{strictement} décroissante si \( f(x)>f(y)\) dès que \( x<y\).
		\item
		      La fonction \( f\) est dite \defe{monotone}{fonction!monotone} sur \( I\) si elle est, soit croissante, soit décroissante, sur \( I\).
	\end{enumerate}
\end{definition}

\begin{example}
	La fonction \( x\mapsto x^2\) est décroissante sur l'intervalle \( \mathopen] -\infty , 0 \mathclose]\) et croissante sur l'intervalle \( \mathopen[ 0 , \infty \mathclose[\). Elle n'est par contre ni croissante ni décroissante sur l'intervalle \( \mathopen[ -4 , 3 \mathclose]\).
\end{example}

%+++++++++++++++++++++++++++++++++++++++++++++++++++++++++++++++++++++++++++++++++++++++++++++++++++++++++++++++++++++++++++
\section{Application réciproque}
%+++++++++++++++++++++++++++++++++++++++++++++++++++++++++++++++++++++++++++++++++++++++++++++++++++++++++++++++++++++++++++

%---------------------------------------------------------------------------------------------------------------------------
\subsection{Définitions}
%---------------------------------------------------------------------------------------------------------------------------

Les définitions d'injection, surjection, bijection et d'application réciproque sont les définitions~\ref{DEFooBFCQooPyKvRK} et~\ref{DEFooTRGYooRxORpY}.

\begin{example}     \label{EXooCWYHooLEciVj}
	\begin{enumerate}
		\item
		      La fonction \( x\mapsto x^2\) n'est pas une bijection de \( \eR\) vers \( \eR\) parce qu'il n'existe aucun \( x\) tel que \( x^2=-1\).
		\item
		      Nous verrons un peu plus tard (\ref{PROPooXQYFooPxoEHE}) que la fonction
		      \begin{equation}
			      \begin{aligned}
				      f\colon \mathopen[ 0 , +\infty [ & \to \mathopen[ 0 , +\infty [ \\
				      x                                & \mapsto x^2
			      \end{aligned}
		      \end{equation}
		      est une bijection. Notez que c'est la même fonction que celle de l'exemple précédent. Seul l'intervalle sur lequel nous nous plaçons a changé.
		\item
		      La fonction
		      \begin{equation}
			      \begin{aligned}
				      f\colon \eR & \to \mathopen[ 0 , \infty \mathclose[ \\
				      x           & \mapsto x^2
			      \end{aligned}
		      \end{equation}
		      n'est pas une bijection parce qu'il existe plusieurs \( x\) pour lesquels \( f(x)=4\).
	\end{enumerate}
	En conclusion : il est très important de préciser les domaines des fonctions considérées.
\end{example}

\begin{remark}
	Dire que la fonction \( f\colon I\to J\) est bijective, c'est dire que l'équation \( f(x)=y\) d'inconnue \( x\) peut être résolue de façon univoque pour tout \( y\in J\).
\end{remark}

\begin{lemma}       \label{LEMooSDMMooYYDDLs}
	Toute fonction strictement monotone sur un intervalle \( I\) est injective.
\end{lemma}

\begin{example}
	Trouvons la fonction réciproque de la fonction affine \( f\colon \eR\to \eR\), \( x\mapsto 3x-2\). Si \( y\in \eR\), le nombre \( f^{-1}(y)\) est la valeur de \( x\) pour laquelle \( f(x)=y\). Il s'agit donc de résoudre
	\begin{equation}
		3x-2=y
	\end{equation}
	par rapport à \( x\). La solution est \( x=\frac{ y+2 }{ 3 }\) et donc nous écrivons
	\begin{equation}
		f^{-1}(y)=\frac{ y+2 }{ 3 }.
	\end{equation}
\end{example}

%---------------------------------------------------------------------------------------------------------------------------
\subsection{Graphe de la fonction réciproque}
%---------------------------------------------------------------------------------------------------------------------------

Par définition le graphe de la fonction \( f\) est l'ensemble des points de la forme \( (x,y)\) vérifiant \( y=f(x)\). Afin de déterminer le graphe de la bijection réciproque nous pouvons faire le raisonnement suivant.

Le point \( (x_0,y_0)\) est sur le graphe de \( f\)

\noindent\( \Leftrightarrow\)

La relation \( f(x_0)=y_0\) est vérifiée

\noindent\( \Leftrightarrow\)

La relation \( x_0=f^{-1}(y_0)\) est vérifiée

\noindent\( \Leftrightarrow\)

Le point \( (y_0,x_0)\) est sur le graphe de \( f^{-1}\).

\begin{Aretenir}
	Dans un repère orthonormal, le graphe de la bijection réciproque est obtenu à partir du graphe de \( f\) en effectuant une symétrie par rapport à la droite d'équation \( y=x\).
\end{Aretenir}

Le dessin suivant montre le cas de la courbe de la fonction carré comparé à celle de la racine carrée.
\begin{center}
	\input{auto/pictures_tex/Fig_CELooGVvzMc.pstricks}
\end{center}

%+++++++++++++++++++++++++++++++++++++++++++++++++++++++++++++++++++++++++++++++++++++++++++++++++++++++++++++++++++++++++++
\section{Topologie sur l'ensemble des réels}
%+++++++++++++++++++++++++++++++++++++++++++++++++++++++++++++++++++++++++++++++++++++++++++++++++++++++++++++++++++++++++++
\label{SECooGKHYooMwHQaD}

Nous allons à présent donner la topologie sur \( \eR\) et ainsi résoudre les questions laissées en suspens lors de la construction des réels, voir~\ref{NormooHRDZooRGGtCd}.

Afin de pouvoir étudier la topologie des espaces métriques, il faut connaître quelques propriétés des réels, parce que nous allons étudier la fonction «distance» qui est une fonction continue à valeurs dans les réels.

La valeur absolue de la définition~\ref{DefKCGBooLRNdJf}\ref{ItemooWUGSooRSRvYC} permet de définir une norme sur \( \eR\).
\begin{lemma}       \label{LEMooBNAPooBTtXnX}
	L'application
	\begin{equation}
		x\mapsto | x |
	\end{equation}
	est une norme\footnote{Définition \ref{DefNorme}.} sur \( \eR\).
\end{lemma}

\begin{proof}
	Grâce au lemme \ref{LemooANTJooYxQZDw} et à la remarque \ref{RemooJCAUooKkuglX}, on a, pour tous \(x,\ y,\ \lambda \in \eR \):
	\begin{enumerate}
		\item \( | x |=0\) implique \( x=0\),
		\item \( | \lambda x |=| \lambda | |x |\),
		\item \( | x+y |\leq | x |+| y |\),
	\end{enumerate}
	et donc, les conditions de la définition \ref{DefNorme} sont immédiatement vérifiées.
\end{proof}

\begin{definition}[Topologie sur \( \eR\) et sur \( \eQ\)]      \label{DEFooNYGIooVGHSIA}
	Le lemme \ref{LEMooBNAPooBTtXnX} donne une norme sur \( \eR\) et \( \eQ\) à partir de la valeur absolue. La définition \ref{ThoORdLYUu} donne alors une structure d'espace topologique. Hors cas rarissimes qui seront signalés, nous utiliserons toujours cette topologie sur \( \eR\) et sur \( \eQ\).
\end{definition}

\begin{proposition}     \label{PropooUHNZooOUYIkn}
	Les rationnels sont denses dans les réels\footnote{Pour les topologies usuelles données en la définition \ref{DEFooNYGIooVGHSIA}.}.
\end{proposition}
\index{densité!de \( \eQ\) dans \( \eR\)}

\begin{proof}
	Soient \( x\in \eR\) et \( \epsilon\in \eR^+\). Nous devons prouver l'existence d'un rationnel dans \( B(x,\epsilon)\). Le lemme~\ref{LemooHLHTooTyCZYL} dit qu'il existe un rationnel dans \( \mathopen] x-\epsilon/2 , x+\epsilon/2 \mathclose[\) et donc dans \( B(x,\epsilon)\).
\end{proof}

\begin{proposition}[\cite{MonCerveau}] \label{PropSLCUooUFgiSR}
	Quel que soit le réel \( x\), il existe une suite croissante de rationnels convergente vers \( x\).
\end{proposition}

\begin{proof}
	Soient \( x\in \eR\) et \( \delta\in \eR\); comme \( x-\delta\) et \( x\) sont des réels, le lemme~\ref{LemooHLHTooTyCZYL} donne un élément \( q_{\delta}\in \eQ \) tel que
	\begin{equation}
		x-\delta<q_{\delta}<x.
	\end{equation}
	Il suffit alors de pêcher parmi ces \( q_{\delta}\) pour trouver une suite croissante, et on montrera que cette suite converge vers \( x \).

	Soit \( x_0\) un rationnel plus petit que \( x\). Nous posons \( \delta_0=x-x_0\) et ensuite :
	\begin{subequations}
		\begin{numcases}{}
			\delta_i=x-x_i\\
			x_{i+1}=q_{\delta_i/2} \in \eQ.
		\end{numcases}
	\end{subequations}
	Ainsi nous avons pour tout \( i\) les inégalités
	\begin{equation}
		x_i=x-\delta_i<x-\frac{ \delta_i }{ 2 }<x_{i+1}<x.
	\end{equation}
	La suite \( (x_i) \) est donc une suite de rationnels, croissante et toujours plus petite que \( x\). Mais nous avons à chaque étape \( \delta_{i+1}<\frac{ \delta_i }{ 2 }\), ce qui implique que la suite des  \( \delta_i \) converge vers \( 0 \). Soit \( \epsilon>0\). Il existe \( k_0\) tel que pour tout \( k > k_0 \), \( \delta_k<\epsilon\). Pour un tel \( k \), nous avons alors
	\begin{equation}
		x_{k+1}\in B(x,\frac{ \delta_k }{ 2 })\subset B(x,\epsilon).
	\end{equation}
	Tous les \( x_k \), pour \( k > k_0 + 1 \), sont tels que \( |x - x_k| < \epsilon \): la suite des \( x_k \) converge donc vers \( x \).
\end{proof}

%---------------------------------------------------------------------------------------------------------------------------
\subsection{Compacité pour les réels}
%---------------------------------------------------------------------------------------------------------------------------

\begin{proposition}     \label{PROPooBFSAooKSugMj}
	Les parties compactes\footnote{Définition \ref{DefJJVsEqs}.} de \( \eR\) sont fermées et bornées.
\end{proposition}

\begin{proof}
	Prouvons d'abord qu'un ensemble compact est borné. Pour cela, supposons que \( K\) est un compact non borné vers le haut\footnote{Nous laissons à titre d'exercice le cas où \( K\) est borné par le haut et pas par le bas.}.

	Nous posons \( A_0=\mathopen] -\infty , 1 \mathclose[\) et \( A_n=\mathopen] n-1 , n+1 \mathclose[\). Ces ouverts recouvrent \( \eR\) et donc \( K\). Hélas tout choix d'un nombre fini des \( A_n\) est borné, et ne recouvre donc pas \( K\). Les \(  \{ A_i \}_{i=0,\ldots, }  \) forment un recouvrement de \( K\) par des ouverts dont on ne peut pas extraire un sous-recouvrement fini.

	Cela prouve que \( K\) doit être borné.

	Pour prouver que \( K\) est fermé, nous allons prouver que le complémentaire est ouvert. Et pour cela, nous allons prouver que si le complémentaire n'est pas ouvert, alors nous pouvons construire un recouvrement de \( K\) dont on ne peut pas extraire de sous-recouvrement fini.

	Si \( \eR\setminus K\) n'est pas ouvert, il possède un point, disons \( x\), tel que tout voisinage de \( x\) intersecte \( K\). Soit \( B(x,\epsilon_1)\), un de ces voisinages, et prenons \( k_1\in K\cap B(x,\epsilon_1)\). Ensuite, nous prenons \( \epsilon_2\) tel que \( k_1\) ne soit pas dans \( B(x,\epsilon_2)\), et nous choisissons \( k_2\in K\cap B(x,\epsilon_2)\). De cette manière, nous construisons une suite de \( k_i\in K\) tous différents et de plus en plus proches de \( x\). Prenons un recouvrement quelconque par des ouverts de la partie de \( K\) qui n'est pas dans \( B(x,\epsilon_1)\). Les nombres \( k_i\) ne sont pas dans ce recouvrement.

	Nous ajoutons à ce recouvrement les ensembles \( \mO=]k_i,k_{i+1}[\). Le tout forme un recouvrement (infini) par des ouverts dont il n'y a pas moyen de tirer un sous-recouvrement fini, pour exactement la même raison que la première fois.
\end{proof}

\begin{theorem}[Borel-Lebesgue]   \label{ThoBOrelLebesgue}
	Un intervalle de \( \eR\) est compact si et seulement si il est de la forme \( \mathopen[ a , b \mathclose]\).
\end{theorem}

\begin{proof}
	Tous les intervalles de \( \eR\) sont listés dans la proposition \ref{PROPooHPMWooQJXCAS}. Un compact est fermé et borné (proposition \ref{PROPooBFSAooKSugMj}). Donc les intervalles dont une borne est \( \pm\infty\) ne sont pas compacts. Parmi les intervalles \( \mathopen] a , b \mathclose[\), \( \mathopen] a , b \mathclose]\), \( \mathopen[ a , b \mathclose[\) et \( \mathopen[ a , b \mathclose]\), seul le dernier est fermé. Nous avons prouvé que si un intervalle est compact, alors il est de la forme \( \mathopen[ a , b \mathclose]\).

	Nous prouvons à présent l'implication inverse : tous les intervalles de la forme \( \mathopen[ a , b \mathclose]\) sont compacts.

	Soit \( \Omega\), un recouvrement du segment \( [a,b]\) par des ouverts, c'est-à-dire que
	\begin{equation}
		[a,b]\subseteq\bigcup_{\mO\in\Omega}\mO.
	\end{equation}
	Nous notons par \( M\) le sous-ensemble de \( [a,b]\) des points \( m\) tels que l'intervalle \( [a,m]\) peut être recouvert par un sous-ensemble fini de \( \Omega\). C'est-à-dire que \( M\) est le sous-ensemble de \( [a,b]\) sur lequel le théorème est vrai. Le but est maintenant de prouver que \( M=[a,b]\).
	\begin{description}
		\item[\( M\) est non vide] En effet, \( a\in M\) parce qu'il existe un ouvert \( \mO\in\Omega\) tel que \( a\in\mO\). Donc \( \mO\) tout seul recouvre l'intervalle \( [a,a]\).
		\item[\( M\) est un intervalle] Soient \( m_1\), \( m_2\in M\). Le but est de montrer que si \( m'\in[m_1,m_2]\), alors \( m'\in M\). Il y a un sous-recouvrement fini de l'intervalle \( [a,m_2]\) (par définition de \( m_2\in M\)). Ce sous-recouvrement fini recouvre évidemment aussi \( [a,m']\) parce que \( [a,m']\subseteq [a,m_2]\), donc \( m'\in M\).
		\item[\( M\) est un ensemble ouvert] Soit \( m\in M\). Le but est de prouver qu'il y a un ouvert autour de \( m\) qui est contenu dans \( M\). Admettons que \( \Omega'\) soit un sous-recouvrement fini qui contienne l'intervalle \( [a,m]\). Dans ce cas, on a un ouvert \( \mO\in\Omega'\) tel que \( m\in\mO\). Tous les points de \( \mO\) sont dans \( M\), puisqu'ils sont tous recouverts par \( \Omega'\). Donc \( \mO\) est un voisinage de \( m\) contenu dans \( M\).
		\item[\( M\) est un ensemble fermé] \( M\) est un intervalle qui commence en \( a\), en contenant \( a\), et qui finit on ne sait pas encore où. Il est donc soit de la forme \( [a,m]\), soit de la forme \( [a,m[\). Nous allons montrer que \( M\) est de la première forme en démontrant que \( M\) contient son supremum \( s\). Ce supremum est un élément de \( [a,b]\), et donc il est contenu dans un des ouverts de \( \Omega\). Disons \( s\in\mO_s\). Soit \( c\), un élément de \( \mO_s\) strictement plus petit que \( s\); étant donné que \( s\) est supremum de \( M\), cet élément \( c\) est dans \( M\), et donc on a un sous-recouvrement fini \( \Omega'\) qui recouvre \( [a,c]\). Maintenant, le sous-recouvrement constitué de \( \Omega'\) et de \( \mO_s\) est fini et recouvre \( [a,s]\).
	\end{description}
	Nous pouvons maintenant conclure : le seul intervalle non vide de \( [a,b]\) qui soit à la fois ouvert et fermé est \( [a,b]\) lui-même (proposition \ref{PropHSjJcIr}), ce qui prouve que \( M=[a,b]\),
	et donc que \( [a,b]\) est compact\footnote{Si vous n'aimez pas le coup du fermé et ouvert, le lemme \ref{LemOACGWxV} donne une autre preuve.}.
\end{proof}

\begin{lemma}[\cite{JUwQXOF}]\label{LemOACGWxV}
	Si \( a<b\in \eR\) alors le segment \( \mathopen[ a , b \mathclose]\) est compact\footnote{Définition~\ref{DefJJVsEqs}}.
\end{lemma}
\index{compact!intervalle \( \mathopen[ a , b \mathclose]\)}

\begin{proof}
	Soit \( \{ \mO_i \}_{i\in I}\) un recouvrement de \( \mathopen[ a , b \mathclose]\) par des ouverts. Nous posons
	\begin{equation}
		M=\{ x\in\mathopen[ a , b \mathclose]\tq \mathopen[ a , x \mathclose] \text{ admet un sous-recouvrement fini extrait de } \{ \mO_i \}_{i\in I} \}.
	\end{equation}
	Notre but est de prouver que \( b\in M\).
	\begin{subproof}

		\spitem[\( a\) est dans \( M\)]
		Le point \( a\) est naturellement dans un des \( \mO_i\). L'intervalle \( \mathopen[ a , a \mathclose]\) est donc recouvert par un seul des \( \mO_i\).

		\spitem[\( M\) est un intervalle]
		Soient \( m\in M\) et \( m'\in\mathopen[ a , m [\). Le sous-recouvrement fini qui recouvre \( \mathopen[ a , m \mathclose]\) recouvre a fortiori \( \mathopen[ a , m' \mathclose]\).

		\spitem[Les trois possibilités restantes]
		À ce niveau de la preuve, il reste trois possibilités pour \( M\) soit il est de la forme \( \mathopen[ a , c \mathclose]\) ou \( \mathopen[ a , c [\) avec \( c<b\), soit il est de la forme \( \mathopen[ a , b \mathclose]\). Nous allons maintenant éliminer les deux premiers cas.

		\spitem[Ce que \( M\) n'est pas]
		D'abord \( M\) n'est pas de la forme \( \mathopen[ a , c [\) avec \( c<b\). Par l'absurde, commençons par considérer \( \mO_{i_0}\) un ouvert du recouvrement qui contient \( c\); choisissons  \(m \in \mO_{i_0}\) tel que \( m<c\). Alors \( m \in M \), et, si nous joignons \( \mO_{i_0}\) à un recouvrement fini de \( \mathopen[ a , m \mathclose]\) alors nous avons un recouvrement fini de \( \mathopen[ a , c \mathclose]\). On en déduit \( c\in M\).

		Ensuite \( M\) n'est pas de la forme \( \mathopen[ a , c \mathclose]\) avec \( c<b\). En effet si on a un recouvrement fini de \( \mathopen[ a , c \mathclose]\) par des ouverts, alors un de ces ouverts contient \( c\) et donc contient des éléments de \( \mathopen[ a , b \mathclose]\) plus grands que \( c\).
	\end{subproof}
	Nous déduisons que \( M=\mathopen[ a , b \mathclose]\) et qu'il est possible d'extraire un sous-recouvrement fini recouvrant \( \mathopen[ a , b \mathclose]\).
\end{proof}

\begin{lemma}[\cite{MonCerveau}]\label{LemCKBooXkwkte}
	Si \( K_1\) et \( K_2\) sont des compacts dans \( \eR\) alors \( K_1\times K_2\) est compact dans \( \eR^2\).
\end{lemma}

\begin{proof}
	Soit \( \{ \mO_i \}_{i\in I}\) un recouvrement de \( K_1\times K_2\) par des ouverts; grâce au lemme~\ref{LemOWVooZKndbI} nous pouvons supposer que ce sont des carrés. Pour chaque \( x\in K_1\), l'ensemble \( \{ x \}\times K_2\) est compact et donc recouvert par un nombre fini des \( \mO_i\). Soit \( R_x\) un ensemble fini des \( \mO_i\) recouvrant \( \{ x \}\times K_2\).

	Comme \( R_x\) est une collection finie de carrés, nous pouvons considérer \( m_x\), le minimum des rayons. L'ensemble \( K_1\) est recouvert par les boules \( B(x,m_x)\) et il existe donc une collection finie de \( \{ x_i \}_{i\in A}\) tels que \( B(x_i,m_{x_i})\) recouvre \( K_1\).

	Alors \( \{ R_{x_i} \}_{i\in A}\) recouvre \( K_1\times K_2\) parce que \( R_{x_i}\) recouvre l'ensemble \( B(x_i,m_{x_i})\times \{ K_2 \}\).
\end{proof}

\begin{definition}[Partie inductive\cite{BIBooQGHRooJxIhEw,BIBooNCHGooOsxdWf}]  \label{DEFooCNCJooWgCCrF}
	Soient \( a<b\) dans \( \eR\). Une partie \( S\) de \( \mathopen[ a , b \mathclose]\) est \defe{inductive}{partie inductive} si
	\begin{enumerate}
		\item
		      \( a\in S\),
		\item
		      Si \( a\leq x<b\) et si \( x\in S\) alors il existe \( y>x\) tel que \( \mathopen[ x , y \mathclose]\subset S\).
		\item
		      Si \( a<x\leq b\) et si \( \mathopen[ a , x \mathclose[\subset S\), alors \( x\in S\).
	\end{enumerate}
\end{definition}

\begin{proposition}[\cite{BIBooQGHRooJxIhEw}]       \label{PROPooQVQKooIjWBrk}
	L'unique partie inductive\footnote{Définition \ref{DEFooCNCJooWgCCrF}.} de \( \mathopen[ a , b \mathclose]\) est \( \mathopen[ a , b \mathclose]\) lui-même.
\end{proposition}

\begin{proof}
	Le fait que \( \mathopen[ a , b \mathclose]\) soit inductif dans \( \mathopen[ a , b \mathclose]\) est immédiat. Nous prouvons que c'est la seule. Soit \( S\) inductif dans \( \mathopen[ a , b \mathclose]\). Nous posons \( S'=\mathopen[ a , b \mathclose]\setminus S\). Si \( S'\) est non vide, nous pouvons considérer le nombre \( \inf(S')\).

	\begin{subproof}
		\spitem[Nous avons\( \inf(S')>a\)]
		% -------------------------------------------------------------------------------------------- 
		D'abord nous savons que \( \inf(S')\geq a\) parce que \( S'\) est une partie de \( \mathopen[ a , b \mathclose]\). De plus, vu que \( a\in S\) il existe \( y>a\) tel que \( \mathopen[ a , y \mathclose]\subset S\). Donc \( S'\subset\mathopen] y , b \mathclose]\), de telle sorte que \( \inf(S')\geq y>a\).
		\spitem[Si \( \inf(S')\in S\)]
		% -------------------------------------------------------------------------------------------- 
		Si \( \inf(S')=b\) alors \( S=\mathopen[ a , b \mathclose]\) et la preuve est faite. Supposons \( \inf(S')<b\). Le nombre \( \inf(S')\) vérifie \( a<\inf(S')<b\) et \( \inf(S')\in S\). Il existe donc \( y>\inf(S')\) tel que \( \mathopen[ \inf(S') , y \mathclose]\subset S\). Nous avons alors \( \mathopen[ a , y \mathclose]\subset S\) et donc \( \inf(S')>y\). Contradiction.
		\spitem[Si \( \inf(S')\in S'\)]
		% -------------------------------------------------------------------------------------------- 
		Dans ce cas \( \mathopen[ a , \inf(S') \mathclose[\subset S\), et vu que \(S \) est inductif, \( \inf(S')\in S\), contradiction.
	\end{subproof}
	Toutes les possibilités (à part \( S'=\emptyset\)) portant à des contradictions, nous déduisons que \( S'=\emptyset\) et donc que \( S=\mathopen[ a , b \mathclose]\).
\end{proof}

%---------------------------------------------------------------------------------------------------------------------------
\subsection{Conséquence: les fermés bornés sont compacts}
%---------------------------------------------------------------------------------------------------------------------------

\begin{theorem}[Théorème de Borel-Lebesgue] \label{ThoXTEooxFmdI}
	Une partie d'un espace vectoriel normé réel de dimension finie est compacte si et seulement si elle est fermée et bornée.
\end{theorem}
\index{théorème!Borel-Lebesgue}
\index{compact!fermé et borné}

\begin{proof}
	Sens direct.
	\begin{subproof}
		\spitem[Compact implique borné]
		En effet si \( K\) est non borné dans \( E\) alors \( K\) contient une suite \( (x_n)\) avec \( \| x_n \|>n\). Les boules \( B_i(x_i,\frac{ 1 }{3})\) sont disjointes. On pose \( \mO_0=\complement\bigcup_i\overline{ B(x_i,\frac{1}{ 5 }) }\), qui est ouvert comme complément d'un fermé. Pour \( i\geq 1\) nous posons \( \mO_i=B(x_i,\frac{1}{ 4 })\). Nous avons
		\begin{equation}
			K\subset\bigcup_{i\in \eN}\mO_i
		\end{equation}
		mais puisque \( x_i\) est uniquement dans \( \mO_i\), nous ne pouvons pas extraire de sous-recouvrement fini.
		\spitem[Compact implique fermé]
		C'est le lemme \ref{LemnAeACf}\ref{ITEMooAZWVooLyPDeY}.
	\end{subproof}
	Sens réciproque.
	\begin{subproof}
		\spitem[Un intervalle fermé et borné est compact dans \( \eR\)]
		C'est le lemme~\ref{LemOACGWxV}.
		\spitem[Un produit de segments est compact]
		Le produit de deux compacts de \( \eR\) est un compact dans \( \eR^2\) par le lemme~\ref{LemCKBooXkwkte}.
		\spitem[Un fermé et borné est compact]
		Soit \( K\) fermé et borné. Puisque \( K\) est borné, il est contenu dans un produit de segments. L'ensemble \( K\) est donc compact parce que fermé dans un compact, lemme~\ref{LemnAeACf}.
	\end{subproof}
\end{proof}

\begin{example}[Compacité de la boule unité]
	La boule unité fermée \( \overline{ B(0,1) }\) d'un espace vectoriel normé de dimension finie est compacte parce que fermée et bornée. En dimension infinie, cela n'est plus le cas. Certes la boule unité est encore fermée et bornée, mais elle n'est plus compacte. En effet nous allons donner un recouvrement par des ouverts duquel il ne sera pas possible d'extraire un sous-recouvrement fini.

	Autour de chacune des extrémités des vecteurs de base, nous considérons la boule \( A_i=B(e_i,\frac{1}{ 3 })\). Ensuite nous considérons aussi l'ouvert
	\begin{equation}
		B(0,1)\setminus\bigcup_i\overline{ B(e_i,\frac{1}{ 4 })}.
	\end{equation}
	Le tout recouvre \( B(0,1)\) mais toutes les premières boules sont nécessaires.
\end{example}
\index{compact!boule unité}

Le théorème de Bolzano-Weierstrass \ref{THOooRDYOooJHLfGq} nous permettra de prouver plus simplement la non compacité en dimension infinie. Voir l'exemple~\ref{ExEFYooTILPDk}.

%---------------------------------------------------------------------------------------------------------------------------
\subsection{Suites et limites dans les réels}
%---------------------------------------------------------------------------------------------------------------------------

\subsubsection{Limites, convergence}
%////////////////////////////////

\begin{proposition}     \label{PROPooOSXCooJWXkWH}
	Une suite \( (x_n)\) dans un espace vectoriel normé \( E\) est convergente\footnote{Définition \ref{DefXSnbhZX}.} si et seulement si il existe un élément \( \ell\in E\) tel que
	\begin{equation}		\label{EQooJRAMooEphgem}
		\forall \varepsilon>0,\,\exists N\in\eN\tq n\geq N\Rightarrow \| x_n-\ell \|<\varepsilon.
	\end{equation}
	Dans ce cas, \( \ell\) est la limite de la suite \( (x_n)\).
\end{proposition}
\index{convergence!suite numérique}
\index{limite!suite numérique}
\index{convergence!dans un espace vectoriel normé}

\begin{proof}
	En deux parties.
	\begin{subproof}
		\spitem[Sens direct]
		Si \( x_n\to \ell\) et si \( \epsilon>0\) il existe \( N_{\epsilon}\) tel que pour tout \( n\geq N\) nous avons \( x_n\in B(\ell,\epsilon)\) (parce que cette boule est un ouvert contenant \( \ell\)). Considérant la définition d'une boule, cette condition s'écrit bien \( \| x_n-\ell \|<\epsilon\).

		\spitem[Sens inverse]
		Dans l'autre sens, soit \( \mO\) un ouvert contenant \( \ell\). Par définition de la topologie, il existe \( \epsilon>0\) tel que \( B(\ell,\epsilon)\subset \mO\). La condition \eqref{EQooJRAMooEphgem} nous assure qu'il existe \( N_{\epsilon} \) tel que pour tout \( n\geq N_{\epsilon}\) nous ayons
		\begin{equation}
			x_n\in B(\ell,\epsilon)\subset\mO,
		\end{equation}
		ce qui assure que la suite \( (x_n)\) converge vers \( \ell\) pour la topologie métrique de \( E\).
	\end{subproof}
\end{proof}

Une façon équivalente d'exprimer le critère \eqref{EQooJRAMooEphgem} est de dire que pour tout \( \epsilon\) positif, il existe un rang \( N\in\eR\) tel que l'intervalle \( \mathopen[ \ell-\epsilon , \ell+\epsilon \mathclose]\) contient tous les termes \( x_n\) au-delà de \( N\).

Il est à noter que le rang \( N\) dont il est question dans la définition de suite convergente dépend de~\( \epsilon\).

%---------------------------------------------------------------------------------------------------------------------------
\subsection{Opérations sur les limites}
%---------------------------------------------------------------------------------------------------------------------------

\begin{proposition}[\cite{MonCerveau}]     \label{PROPooIQOAooJPMoDD}
	Soient des suites à valeurs réelles \( (a_i)\) et \( (b_j)\). Si elles sont convergentes, alors la suite \( ab\) est convergente et
	\begin{equation}
		\big( \lim_ia_i \big)\big( \lim_jb_j \big)=\lim_i(a_ib_i).
	\end{equation}
\end{proposition}

\begin{proof}
	Nous nommons \( a\) et \( b\) les limites des suites \( (a_i)\) et \( (b_j)\). Soit \( \epsilon>0\) ainsi que \( i\in \eN\). Nous avons la majoration
	\begin{subequations}
		\begin{align}
			| a_ib_i-ab | & \leq | a_ib_i-a_ib |  + | a_ib-ab |      \\
			              & \leq | a_i | |b_i-b | + | b | | a_i-a |.
		\end{align}
	\end{subequations}
	Comme la suite \( (a_i)\) est convergente, elle est bornée\footnote{Par \ref{PROPooUXDJooCrWBbd}. Attention : soyez capable d'adapter au cas présent.}. Nous pouvons donc majorer \( | a_i |\) par \( R>0\) qui ne dépend pas de \( i\). Soit \( \eta>0\) tel que \( (R+b)\eta<\epsilon\). Alors en prenant \( i\) assez grand pour que \( | b_i-b |<\eta\) et \( | a_i-a |<\eta\), nous avons bien
	\begin{equation}
		| a_ib_i-ab |\leq (R+b)\eta<\epsilon.
	\end{equation}
\end{proof}

\begin{proposition}     \label{PROPooICZMooGfLdPc}
	Soient des suites \( (x_n)\) et \( (y_n)\) dans un espace vectoriel normé \( E\). Si \( x_n\stackrel{E}{\longrightarrow}x\) et \( y_n\stackrel{E}{\longrightarrow}y\), alors
	\begin{equation}
		x_n+y_n\stackrel{E}{\longrightarrow}x+y.
	\end{equation}
\end{proposition}

\begin{proof}
	Soit \( \epsilon>0\). Nous considérons \( N\) tel que si \( n\geq N\), alors \( \| x_n-x \|\leq \epsilon\) et \( \| y_n-y \|\leq \epsilon\). En utilisant l'inégalité \ref{DefNorme}\ref{ItemDefNormeiii},
	\begin{equation}
		\| x_y+y_n-(x+y) \|\leq \| x_n-x \|+\| y_n-y \|\leq 2\epsilon.
	\end{equation}
	Donc la suite \( (x_n+y_n)\) converge vers \( x+y\).
\end{proof}

\begin{lemma}       \label{LEMooGKIPooWgpFTB}
	La fonction
	\begin{equation}
		\begin{aligned}
			f\colon \eR\times \eR & \to \eR     \\
			(x,y)                 & \mapsto x+y
		\end{aligned}
	\end{equation}
	est continue.
\end{lemma}

\begin{proof}
	Pour rappel, la topologie considérée sur \( \eR^n\) est celle de la définition \ref{DefFAJgTCE}. En vertu de la proposition \ref{PropXIAQSXr}, il est suffisant de prouver la continuité séquentielle. Soit donc une suite convergente
	\begin{equation}
		(x_n,y_n)\stackrel{\eR\times \eR}{\longrightarrow}(x,y).
	\end{equation}
	Nous devons prouver que
	\begin{equation}
		f(x_n,y_n)\stackrel{\eR}{\longrightarrow}f(x,y).
	\end{equation}
	La proposition \ref{PROPooNRRIooCPesgO} nous permet de déduire la convergence composante par composante : \( x_n\stackrel{\eR}{\longrightarrow}x\) et \( y_n\stackrel{\eR}{\longrightarrow}x\). En permutant somme et limite (proposition \ref{PROPooICZMooGfLdPc}) nous avons le calcul
	\begin{equation}
		f(x_n,y_n)=x_n+y_n\stackrel{\eR}{\longrightarrow}x+y=f(x,y).
	\end{equation}
	D'où la convergence demandée.
\end{proof}

\subsection{Exemples}
%//////////////////////////

\begin{lemma}       \label{LEMooNDSKooMsexOq}
	Quelques suites usuelles.
	\begin{enumerate}
		\item
		      La suite \( x_n=\frac{1}{ n }\) converge vers \( 0\).
		\item
		      La suite \( x_n=(-1)^n\) ne converge pas.
	\end{enumerate}
\end{lemma}

%---------------------------------------------------------------------------------------------------------------------------
\subsection{Limites infinies}
%---------------------------------------------------------------------------------------------------------------------------

Deux limites pour voir comment ça fonctionne.
\begin{lemma}       \label{LEMooWCRSooWXVvcc}
	Si \( r>1\) nous avons :
	\begin{enumerate}
		\item
		      \( \lim_{n\to \infty} r^n=\infty\).
		\item
		      \( \lim_{n\to \infty} \frac{ r^n }{ n }=\infty\).
	\end{enumerate}
\end{lemma}

\begin{proof}
	Puisque \( r>1\) nous pouvons écrire \( r=1+\delta\) avec \( \delta>0\). La formule du binôme de Newton \eqref{EqNewtonB} nous donne
	\begin{equation}
		(1+\delta)^n=\sum_{k=0}^n{n\choose k}\delta^k>{n\choose 1}\delta=n\delta.
	\end{equation}
	La proposition \ref{ThoooKJTTooCaxEny} (\( \eR\) est archimédien) nous indique que \( n\delta\) est arbitrairement grand lorsque \( n\) est grand, quelle que soit \( \delta>0\). Cela finit la preuve de la première limite.

	Pour la seconde, nous posons \( a_n=\frac{ r^n }{ n }\). Nous avons
	\begin{equation}
		\frac{ a_{n+1} }{ a_n }=\frac{ n }{ n+1 }r.
	\end{equation}
	Comme \( \frac{ n }{ n+1 }\to 1\), la suite \( \frac{ n }{ n+1 }r\) tend vers \( r>1\), et en particulier pour tout \( \delta>0\) tel que \( r>1+\delta\), il existe \( N\in \eN\) tel que, pour tout \( n > N \),
	\begin{equation}
		\frac{ n }{ n+1 }r>1+\delta.
	\end{equation}
	Soit maintenant \( k\in \eN\). En utilisant un produit télescopique,
	\begin{equation}
		a_{N+k}=a_N\frac{ a_{N+1} }{ a_N }\frac{ a_{N+2} }{ a_{N+1} }\cdots\frac{ a_{N+k} }{ a_{N+k-1} }>a_N(1+\delta)^{k-1}.
	\end{equation}
	Or \( (1+\delta)^{k-1}\) tend vers \( \infty\) lorsque \( k\to \infty\) par le premier point. Donc nous avons \( \lim_{n\to \infty} r^n/n=\infty\).
\end{proof}

\begin{definition}      \label{DEFooEWRTooKgShmT}
	Nous disons que deux suites \( (u_n)\) et \( (v_n)\) sont \defe{équivalentes}{equivalence@équivalence!de suites} si il existe une application \( \alpha\colon \eN\to \eR\) telle que
	\begin{enumerate}
		\item
		      pour tout \( n\) à partir d'un certain rang, \( u_n=v_n\alpha(n)\)
		\item
		      \( \alpha(n)\to 1\).
	\end{enumerate}
\end{definition}




%---------------------------------------------------------------------------------------------------------------------------
\subsection{Topologie et matrices}
%---------------------------------------------------------------------------------------------------------------------------

\begin{lemmaDef}[Topologie sur les matrices]      \label{DEFooCQHDooYpUAhG}
	Si \( \eK\) est un corps valué\footnote{Définition \ref{DEFooBWXXooAkBBRS}.}, alors l'opération
	\begin{equation}
		\begin{aligned}
			\| . \|_{\eM}\colon \eM(n\times m,\eK) & \to \eR^+                             \\
			M                                      & \mapsto  \max_{kl}\| M_{kl} \|_{\eK}.
		\end{aligned}
	\end{equation}
	est une norme\footnote{Définition \ref{DefNorme}.}.

	Cette norme est appelée \defe{norme maximum}{norme maximum} et nous considérons sur \( \eM(n\times m, \eK)\) la topologie associée à cette norme\footnote{Définition \ref{DEFooPMVFooPSYVNQ}.}
\end{lemmaDef}

\begin{proposition}     \label{PROPooOEETooPhqWuf}
	La multiplication matricielle
	\begin{equation}
		\begin{aligned}
			m\colon \eM(n,s)\times \eM(s,l) & \to \eM(n,l) \\
			(A,B)                           & \mapsto AB
		\end{aligned}
	\end{equation}
	est une opération continue\footnote{Topologies de la norme \ref{DEFooCQHDooYpUAhG}.}.
\end{proposition}

\begin{proof}
	D'après la proposition \ref{DefZTHxrHA}, nous pouvons tout faire avec la métrique produit. Supposons que \( (A_k,B_k)  \stackrel{ \eM(n,s)\times \eM(s,l) }{\longrightarrow} \eM(n,l) \), et montrons que \( \| m(A_k,B_k)-m(A,B) \|\to 0\). Notre hypothèse nous dit que
	\begin{equation}
		\| (A_k-A,B_k-B) \|\to 0,
	\end{equation}
	c'est à dire (métrique produit)
	\begin{equation}
		\max\big\{  \| A_k-A \|,\| B_k-B \| \big\}\to 0,
	\end{equation}
	et donc \( A_k\to A\) et \( B_k\to B\) séparément. La définition \ref{DEFooCQHDooYpUAhG} nous dit alors que pour tout \( (ij)\) nous avons
	\begin{equation}
		\begin{aligned}[]
			(A_k)_{ij} & \to A_{ij}  \\
			(B_k)_{ij} & \to B_{ij}.
		\end{aligned}
	\end{equation}

	Nous avons
	\begin{subequations}		\label{SUBEQSooHPHLooXdtxYY}
		\begin{align}
			\Big( m(A_k,B_k)-m(A,B) \Big)_{ij} & =(A_kB_k)_{ij}-(AB)_{ij}                               \\
			                                   & = \sum_l\Big( (A_k)_{il}(B_k)_{lj}-A_{il}B_{lj} \Big).
		\end{align}
	\end{subequations}
	Soit \( \epsilon>0\). Par la proposition \ref{PROPooIQOAooJPMoDD}, pour chaque triple \( (i,j,l)\), il existe un \( u_{ijk}\in \eN\) tel que si \( k\geq u_{ijl}\) nous avons
	\begin{equation}
		| (A_k)_{il}(B_k)_{lj}-A_{il}B_{lj} |<\epsilon.
	\end{equation}
	Nous pouvons reprendre \eqref{SUBEQSooHPHLooXdtxYY} avec un tel \( k\). Nous avons :
	\begin{equation}
		\| (A_knB_k)-(A,B) \|<\sum_l\epsilon<s\epsilon.
	\end{equation}
\end{proof}

%---------------------------------------------------------------------------------------------------------------------------
\subsection{Suites croissantes et bornées}
%---------------------------------------------------------------------------------------------------------------------------

Une suite est dite \defe{contenue}{} dans un ensemble \( A\) si \( x_n\in A\) pour tout \( n\). Une suite est \defe{bornée supérieurement}{bornée!suite} si il existe un \( M\) tel que \( x_n\leq M\) pour tout \( n\). De la même manière, la suite est bornée inférieurement si il existe un \( m\) tel que \( x_n\geq m\) pour tout \( n\).

Le lemme suivant est souvent utilisé pour prouver qu'une suite est convergente. Une version pour les fonctions \( f\colon \eR\to \eR\) sera la proposition \ref{PROPooGQHKooWgykjW}.
\begin{lemma}[\cite{MonCerveau}]       \label{LemSuiteCrBorncv}
	Une suite croissante et bornée supérieurement converge. Une suite décroissante bornée inférieurement est convergente.
\end{lemma}

\begin{proof}
	Supposons que \( (x_n)\) soit une suite croissante non convergente. En particulier, par le théorème \ref{THOooNULFooYUqQYo}, cette suite n'est pas de Cauchy et il existe \( \epsilon>0\) tel que pour tout \( N\in \eN\), il existe \( p>N\) vérifiant
	\begin{equation}
		| x_p-x_N |>\epsilon.
	\end{equation}
	Vu que \( p>N\), et vu que la suite est croissante, nous pouvons récrire cette condition sous la forme \( x_p\geq x_N+\epsilon\).

	Nous définissons ainsi une application \( p\colon \eN\to \eN\) telle que
	\begin{equation}
		x_{p(N)}>x_N+\epsilon.
	\end{equation}
	Une telle application n'est pas du tout unique, mais nous en considérons une telle.

	Il est vite vu par récurrence que
	\begin{equation}
		x_{p^k(0)}\geq x_0+k\epsilon.
	\end{equation}

	La suite \( n\mapsto x_{p^n(0)}\) est donc une sous-suite qui tend vers l'infini. Or cela n'est pas possible parce que la suite \( (x_n)\) est bornée. Donc contradiction, donc \( (x_n)\) est convergente.
\end{proof}

Une erreur courante est de croire que la borne est la limite : le lemme n'affirme pas ça. Par contre il est vrai que la borne donne \ldots hum \ldots une borne inférieure (ou supérieure) pour la limite.

\begin{proposition}     \label{PropCvRpComposante}
	Une suite \( (x_n)\) dans \( \eR^m\) est convergente dans \( \eR^m\) si et seulement si les suites de chaque composante sont convergentes dans \( \eR\). Dans ce cas nous avons
	\begin{equation}
		\lim x_n=\Big( \lim(x_n)_1,\lim (x_n)_2,\ldots,\lim (x_n)_m \Big)
	\end{equation}
	où \( (x_n)_k\) dénote la \( k\)-ième composante de \( (x_n)\).
\end{proposition}

\begin{example}
	La suite \( x_n=\big( \frac{1}{ n },1-\frac{1}{ n } \big)\) converge vers \( (0,1)\) dans \( \eR^2\). En effet, en utilisant la proposition~\ref{PropCvRpComposante}, nous devons calculer séparément les limites
	\begin{equation}
		\begin{aligned}[]
			\lim\frac{1}{ n }               & =0  \\
			\lim\big( 1-\frac{1}{ n } \big) & =1.
		\end{aligned}
	\end{equation}
\end{example}

\begin{example}
	Étant donné que la suite \( (-1)^n\) n'est pas convergente, la suite \( x_n=\big( (-1)^n,\frac{1}{ n } \big)\) n'est pas convergente dans \( \eR^2\).
\end{example}

%---------------------------------------------------------------------------------------------------------------------------
\subsection{Suites adjacentes}
%---------------------------------------------------------------------------------------------------------------------------

\begin{definition}[\cite{ooZZNWooSIipwW}]         \label{DEFooDMZLooDtNPmu}
	Les suites \( (a_n)\) et \( (b_n)\) sont \defe{adjacentes}{suites adjacentes} si l'une est croissante, l'autre décroissante et si \( a_n-b_n\to 0\).
\end{definition}

\begin{theorem}[Théorème des suites adjacentes]   \label{THOooZJWLooAtGMxD}
	Nous considérons des suites adjacentes \( (a_n)\) et \( (b_n)\) avec \( (a_n)\) croissante et \( (b_n)\) décroissante. Alors
	\begin{enumerate}
		\item
		      \( b_n\geq a_n\) pour tout \( n\),
		\item
		      \( a_n\leq b_q\) pour tout \( n\) et \( q\). C'est-à-dire que toute la suite \( a\) est plus petite que toute la suite \( b\).
		\item
		      les suites \( a\) et \( b\) sont convergentes,
		\item
		      les suites \( a\) et \( b\) convergent vers la même limite, notée \( \ell\),
		\item
		      nous avons \( a_n\leq \ell\leq b_n\) pour tout \( n\).
	\end{enumerate}
\end{theorem}

\begin{proof}
	La suite \( n\mapsto b_n-a_n\) est décroissante parce que \( b_n-a_n\geq b_{n+1}-a_{n+1}\). Comme en plus \( b_a-a_n\to 0\) nous avons
	\begin{equation}
		b_n-a_n\geq 0
	\end{equation}
	pour tout \( n\in \eN\). De plus \( a_n\leq b_0\) pour tout \( n\) parce que si \( a_N>b_0\) alors, \( b\) étant décroissante, \( a_N>b_0\geq b_N\) qui est contraire à ce que nous venons de prouver. La suite \( a\) étant croissante et majorée, elle est convergente\footnote{Proposition \ref{LemSuiteCrBorncv}.}; notons \( \ell\) sa limite.

	La suite \( b\) peut maintenant être écrite par
	\begin{equation}
		b_n=(b_n-a_n)+a_n
	\end{equation}
	qui est une somme de deux suites convergentes. Elle est donc convergente et sa limite est la somme des limites\footnote{Proposition \ref{PROPooICZMooGfLdPc}.}, donc
	\begin{equation}
		\lim_{n\to \infty} b_n=\lim_{n\to \infty} (b_n-a_n)+\lim_{n\to \infty} a_n = 0+\ell=\ell.
	\end{equation}
	Voilà. Donc les suites \( a\) et \( b\) convergent et ont la même limite.

	Pour tout \( n,q\in \eN\) nous avons l'inégalité \( a_n\leq b_q\). En prenant la limite \( n\to \infty\) nous trouvons
	\begin{equation}
		\ell\leq b_q
	\end{equation}
	pour tout \( q\). Et de la même façon, \( b_n\geq a_q\) donne \( \ell\geq a_q\). L'un avec l'autre donne
	\begin{equation}
		a_q\leq \ell\leq b_q
	\end{equation}
	pour tout \( q\in \eN\).
\end{proof}

\begin{proposition}[\cite{ooXFPIooCLUvzV}]      \label{PROPooXOOCooGMqJNe}
	Soit une suite \( (a_n)\) dans \( \eR\).  Nous supposons que les suites extraites \( (a_{2n})\) et \( (a_{2n+1})\) convergent vers la même limite notée \( \ell\).

	Alors \( a_n\to \ell\).
\end{proposition}

\begin{proof}
	Soit \( \epsilon>0\). Il existe \( N_1\) tel que \( | a_{2n}-\ell |\leq \epsilon\) dès que \( n\geq N_1\). Il existe également \( N_2\) dès que \( | a_{2n+1}-\ell |\leq \epsilon\) dès que \( n\geq N_2\).

	Nous posons \( N=\max\{ 2N_1,2N_2+2 \}\) et nous avons, pour tout \( n\geq N\) :
	\begin{equation}
		| a_n-\ell |\leq \epsilon,
	\end{equation}
	c'est-à-dire que \( a\to \ell\).
\end{proof}

%---------------------------------------------------------------------------------------------------------------------------
\subsection{Limite supérieure et inférieure}
%---------------------------------------------------------------------------------------------------------------------------

\begin{lemmaDef}      \label{ooMVZAooVVCOnP}
	Soit \( (a_n)\) une suite dans \( \bar \eR\). Les limites suivantes existent dans \( \bar \eR\)
	\begin{equation}
		\limsup_{n\to\infty}a_n=\lim_{n\to \infty}\big( \sup_{k\geq n}a_k \big)
	\end{equation}
	et
	\begin{equation}
		\liminf_{n\to \infty}a_n=\lim_{n\to\infty}\big( \inf_{k\geq n}a_k \big).
	\end{equation}
	Elles sont nommées \defe{limite supérieure}{limite!supérieure} et \defe{limite inférieure}{limite inférieure} de la suite \( (a_n)\).
\end{lemmaDef}
\nomenclature[Y]{\( \limsup a_n\)}{limite supérieure}
\nomenclature[Y]{\( \liminf a_n\)}{limite inférieure}

\begin{proof}
	Pour la limite supérieure, l'ensemble des \( k\geq n\) est de plus en plus petit lorsque \( n\) grandit. Donc les ensembles \( A_n=\{ a_k\tq k\geq n \}\) sont emboîtés et la suite \( n\to \sup A_n\) est une suite décroissante. Elle a donc une limite dans \( \bar \eR\).
\end{proof}

\begin{normaltext}      \label{ooEEQJooRMFzVR}
	En ce qui concerne les suites d'ensembles, utiles en théorie des probabilités, nous définissons de même. Si les \( A_n\) sont des parties d'un ensemble \( \Omega\), nous définissons la \defe{limite supérieure}{limite!supérieure} et la \defe{limite inférieure}{limite!inférieure} de la suite \( A_n\) par
	\begin{equation}
		\limsup_{n\to\infty}A_n=\bigcap_{n\geq 1}\bigcup_{k\geq n}A_k
	\end{equation}
	et
	\begin{equation}
		\liminf_{n\to\infty}A_n=\bigcup_{n\geq 1}\bigcap_{k\geq n}A_k
	\end{equation}

	Nous avons
	\begin{equation}
		\limsup A_n=\{ \omega\in\Omega\tq \omega\in A_n\text{ pour une infinité de } n \}.
	\end{equation}
\end{normaltext}

\begin{lemma}     \label{ooAQTEooYDBovS}
	Nous avons les formules pratiques suivantes :
	\begin{subequations}
		\begin{align}
			\limsup a_n & =\inf_{n\geq 1}\big( \sup_{k\geq n}a_k \big)  \\
			\liminf a_n & =\sup_{n\geq 1}\big( \inf_{k\geq n}a_k \big).
		\end{align}
	\end{subequations}
\end{lemma}

\begin{proof}
	La suite \( n\mapsto \sup_{k\geq n}a_k\) est une suite décroissante, donc la limite est l'infimum. Même argument pour l'autre.
\end{proof}

\begin{lemma}       \label{ooIQIKooXWwAmM}
	La suite \( (a_n)\) dans \( \eR\) converge si et seulement si
	\begin{equation}
		\limsup a_n=\liminf a_n.
	\end{equation}
	Dans ce cas, \( \lim a_n=\limsup a_n=\liminf a_n\).
\end{lemma}

\begin{proof}
	Nous commençons par supposer que \( \limsup a_n=\liminf a_n=l\), et nous prouvons que \( \lim a_n\) existe et vaut \( l\). Soit \( \epsilon>0\). Il existe \( N\) tel que si \( n\geq N\) nous avons
	\begin{equation}
		\big| \sup_{k\geq n}a_k-l \big|<\epsilon
	\end{equation}
	et
	\begin{equation}
		\big| \inf_{k\geq n}a_k-l \big|<\epsilon.
	\end{equation}
	Si \( i\geq N\), alors\footnote{Voir le lemme \ref{LEMooQXDCooPEABBm}\ref{ITEMooXJGVooSebiip}.} \( a_i\leq \sup_{k\geq N}(a_k)\leq \ell+\epsilon\), et \( a_i\geq\inf_{k\geq N}(a_k)\geq \ell-\epsilon\). Cela signifie que \( a_n\in B(l,\epsilon)\), c'est-à-dire \( a_k\to l\) par la proposition \ref{PROPooOSXCooJWXkWH}.

	Dans l'autre sens, nous supposons que \( \lim_n a_n=l\) et nous prouvons que la limite supérieure est égale à \( l\)\footnote{Je vous laisse faire la démonstration correspondante pour la limite inférieure. Contactez-moi si ça pose un problème.}. Soit \( \epsilon>0\) et \( N_{\epsilon}\) tel que \( | a_n-l |<\epsilon\) pour tout \( n\geq N_{\epsilon}\). Si \( n\geq N_{\epsilon}\) nous avons
	\begin{equation}
		\big| \sup_{k\geq n}a_k-l \big|\leq \epsilon
	\end{equation}
	et donc la limite de \( \sup_{k\geq n}a_k\) lorsque \( n\to \infty\) est bien \(l\).
\end{proof}

\begin{lemma}       \label{LEMooHGJVooCbgOEK}
	Soit une suite \( (a_i)\) dans \( \eR\). Notons \( L= \limsup_i(a_i)\).
	Pour tout \( \epsilon>0\), l'ensemble
	\begin{equation}
		S_{\epsilon}=\{ n\in \eN\tq a_n\geq L+\epsilon \}
	\end{equation}
	est fini.
\end{lemma}

\begin{proof}
	Nous y allons par récurrence. Juste pour le sport, nous allons au passage montrer en détail comment on utilise le théorème \ref{THOooEJPYooZFVnez}.


	Supposons que \( S_{\epsilon}\) est infini. Alors pour tout \( n\), la partie \( S_{\epsilon}\setminus\{ 0,\ldots, n \}\) est non vide (lemme \ref{LEMooRWUDooTgoRXH}) et nous pouvons considérer l'application
	\begin{equation}
		\begin{aligned}
			g\colon S_{\epsilon} & \to S_{\epsilon}                                                \\
			n                    & \mapsto \min\big( S_{\epsilon}\setminus\{ 0,\ldots, n \} \big).
		\end{aligned}
	\end{equation}
	Nous prenons \( b>1\) dans \( S_{\epsilon}\) et considérons la fonction \( f\colon \eN\to S_{\epsilon}\) donnée par le théorème \ref{THOooEJPYooZFVnez}.

	L'application \( f\) est strictement croissante parce que \( f(n+1)=g\big( f(n) \big)\in \eN\setminus\{ 0,\ldots, f(n) \}\). En particulier \( f(n)>n\) parce que nous avons décidé de commencer avec \( f(0)=b>1\).

	Nous sommes maintenant armés pour contredire la définition \ref{ooMVZAooVVCOnP} de la limite supérieure. Soit \( n\in \eN\). Vu que \( f(n)\in S_{\epsilon}\) nous avons
	\begin{equation}
		a_{f(n)}\geq L+\epsilon,
	\end{equation}
	et donc \( \sup_{k\geq n}a_k\geq a_{f(n)}\geq L+\epsilon\) parce que \( f(n)\geq n\).

	Nous avons prouvé que \( \sup_{k\geq n}a_k\geq L+\epsilon\) pour tout \( n\), donc
	\begin{equation}
		\lim_{n\to \infty} \big( \sup_{k\geq n}a_k \big)\geq L+\epsilon>L.
	\end{equation}
	Voila. Donc si \( S_{\epsilon}\) est infini, \( \limsup_ia_i\geq L+\epsilon>L\).
\end{proof}

\begin{lemma}       \label{LEMooMTRDooBMxFmn}
	Si \( (a_n)\) est une suite réelle,
	\begin{equation}
		\limsup_{n\to\infty}(a_n)=-\liminf_{n\to\infty}(-a_n).
	\end{equation}
\end{lemma}

\begin{proof}
	En utilisant la proposition \ref{PROPooGOPOooYyZSuU}, nous avons
	\begin{equation}
		\limsup(a_n)=\lim_{n\to\infty}\Big( \sup_{k\geq n}(a_k) \Big)=\lim_{n\to \infty} \Big( -\inf_{k\geq n}(-a_k) \Big)=-\liminf(-a_n).
	\end{equation}
\end{proof}

\begin{proposition}[\cite{MonCerveau}]	\label{PROPooUOYTooGwGZHz}
	Si \( a_n\to a\), alors
	\begin{equation}
		\liminf_n| a_n |=| a |.
	\end{equation}
\end{proposition}

%---------------------------------------------------------------------------------------------------------------------------
\subsection{Ouverts, voisinage, topologie}
%---------------------------------------------------------------------------------------------------------------------------

Lorsque \( x\in E\), nous rappelons qu'un voisinage\footnote{Définition~\ref{DEFVoisinageooGHZCooLRcpXY}.} de \( x\) est n'importe quel sous-ensemble de \( E\) qui contient une boule ouverte centrée en \( x\). La proposition \ref{ThoPartieOUvpartouv} nous dit qu'un ensemble est ouvert si il contient un voisinage de chacun de ses points. Au passage, rappelons que l'ensemble vide est ouvert.

Pour rappel, la proposition \ref{PROPooZXTXooEMLgMn} dit que l'ensemble des boules ouvertes d'un espace métrique génère la topologie de l'espace.

Nous rappelons qu'une partie \( A\) d'un espace métrique est dite bornée\footnote{Définition~\ref{DefEnsembleBorne}.} si il existe une boule\footnote{À titre d'exercice, convainquez-vous que l'on peut dire boule \emph{ouverte} ou \emph{fermée} au choix sans changer la définition.} qui contient \( A\).

Mais revenons à \( \eR \)\dots
\begin{lemma}  \label{LemSupOuvPas}
	Une partie ouverte de \( \eR\) ne contient pas son supremum.
\end{lemma}

\begin{proof}
	Soit \( \mO\), un ensemble ouvert et \( s\), son supremum. Si \( s\) était dans \( \mO\), on aurait un voisinage \( B=B(s,r)\) de \( s\) contenu dans \( \mO\). Le point \( s+r/2\) est alors à la fois dans \( \mO\) et plus grand que \( s\), ce qui contredit le fait que \( s\) soit un supremum de \( \mO\).
\end{proof}

Par le même genre de raisonnement, on montre que l'union et l'intersection de deux ouverts, sont encore des ouverts.

\begin{remark}
	L'intersection d'une infinité d'ouverts n'est pas spécialement un ouvert comme le montrent les parties \( \{ \mO_k \}_{k\in \eN^*}\) donnés par
	\begin{equation}
		\mO_k=]1,2+\frac{ 1 }{ k }[.
	\end{equation}
	Tous les ensembles \( \mO_k\) contiennent le point \( 2\) qui est donc dans l'intersection. Mais nous allons montrer que pour tout \( \epsilon>0\), il existe \( n\) tel que \( 2+\epsilon\notin\mO_n\). Il suffit de prendre \( n\) tel que \( \frac{1}{ n }<\epsilon\) (lemme \ref{LemooHLHTooTyCZYL}\ref{ITEMooCVDSooAjimCL}).
\end{remark}

\begin{proposition}     \label{PROPooANIOooIJHelX}
	Quelles que soient les parties \( A\) et \( B\) de \( \eR\), nous avons
	\begin{equation}
		\sup(A\cap B)\leq\sup A\leq\sup(A\cup B).
	\end{equation}
\end{proposition}

\begin{proof}
	En deux parties.
	\begin{subproof}
		\spitem[\(  \sup(A\cap B)\leq \sup(A)\) ]
		Soit \( s=\sup(A)\). En particulier, \( s\) est un majorant de \( A\). Si \( x\in A\cap B\), alors \( x\in A\) et \( s\geq x\). Donc \( s\) est également un majorant de \( A\cap B\). Le lemme \ref{LEMooSSVKooDPhSkq} conclut que \( s\geq \sup(A\cap B)\).

		\spitem[\(  \sup(A)\leq \sup(A\cup B)\) ]
		Soit \( s=\sup(A\cup B)\). Par définition, \( s\) est un majorant de \( A\cup B\). A fortiori, \( s\) est un majorant de \( A\) et donc est plus grand ou égal à \( \sup(A)\).
	\end{subproof}
\end{proof}

%---------------------------------------------------------------------------------------------------------------------------
\subsection{Intervalles et connexité}
%---------------------------------------------------------------------------------------------------------------------------

Nous allons déterminer tous les sous-ensembles connexes\footnote{Définition~\ref{DefIRKNooJJlmiD}.} de \( \eR\). Pour cela nous relisons d'abord la notion d'intervalle donnée en~\ref{DefEYAooMYYTz} ainsi que la proposition \ref{PROPooHPMWooQJXCAS} qui liste tous les intervalles de \( \eR\). La partie \( I\subset \eR\) est un intervalle si pour tout \( a,b\in I\), tout nombre entre \( a\) et \( b\) est également dans \( I\). Cette définition englobe tous les exemples connus d'intervalles ouverts, fermés avec ou sans infini : \( [a,b]\), \( [a,b[\), \( ]-\infty,a]\), \ldots L'ensemble \( \eR\) lui-même est un intervalle.

Si \( I\) est un intervalle, les nombres \( \inf(I)\) et \( \sup(I)\)\footnote{Qui existent par la proposition~\ref{DefSupeA}, quitte à poser \( \pm\infty\) comme infimum et supremum lorsque \( I\) n'est pas borné.} sont les \defe{extrémités}{extrémité!d'un intervalle} de \( I\).

\begin{definition}      \label{DefLISOooDHLQrl}
	Étant donnés deux points \( a\) et \( b\) dans \( \eR^p\) on appelle \defe{segment}{segment!dans \( \eR^p\)} d'extrémités \( a\) et \( b\), et on note \( [a,b]\), l'image de \( [0,1]\) par l'application \( s: [0,1]\to \eR^p\), \( s(t)= (1-t)a+tb\).  On pose \( ]a,b[=s\left(]0,1[\right)\), et  \( ]a,b]=s\left(]0,1]\right)\).
\end{definition}
Il faut observer que le segment \( [a,b]\) est une courbe orientée : certes en tant que ensembles, \( [a,b]=[b,a]\), mais si nous regardons la fonction de \( t\) correspondante à \( [b,a]\), nous voyons qu'elle va dans le sens inverse de celle qui correspond à \( [a,b]\). Nous approfondirons ces questions lorsque nous parlerons d'arcs paramétrés autour de la section~\ref{SecArcGeometrique}.

Le segment \( [b,a]\) est l'image de l'application \( r\colon [0,1]\to \eR^p\) donnée par \( r(t)=(1-t)b+ta\).

\begin{proposition} \label{PropInterssiConn}
	Une partie de \( \eR\) est connexe si et seulement si c'est un intervalle\footnote{Définition \ref{DefEYAooMYYTz}.}.
\end{proposition}
\index{connexité!et intervalles}

\begin{proof}
	La preuve est en deux parties. D'abord nous démontrons que si un sous-ensemble de \( \eR\) est connexe, alors c'est un intervalle; et ensuite nous démontrons que tout intervalle est connexe.

	Afin de prouver qu'un ensemble connexe est toujours un intervalle, nous allons prouver que si un ensemble n'est pas un intervalle, alors il n'est pas connexe. Prenons \( A\), une partie de \( \eR\) qui n'est pas un intervalle. Il existe donc \( a\), \( b\in A\) et un \( x_0\) entre \( a\) et \( b\) qui n'est pas dans \( A\). Comme le but est de prouver que \( A\) n'est pas connexe, il faut couper \( A\) en deux ouverts disjoints. L'élément \( x_0\) qui n'est pas dans \( A\) est le bon candidat pour effectuer cette coupure. Prenons \( M\), un majorant de \( A\) et \( m\), un minorant de \( A\), et définissons
	\begin{align*}
		\mO_1 & =]m,x_0[  \\
		\mO_2 & =]x_0,M[.
	\end{align*}
	Si \( A\) n'a pas de minorant, nous remplaçons la définition de \( \mO_1\) par \( ]-\infty,x_0[\), et si \( A\) n'a pas de majorant, nous remplaçons la définition de \( \mO_2\) par \( ]x_0,\infty[\). Dans tous les cas, ce sont deux ensembles ouverts dont l'union recouvre tout \( A\). En effet, \( \mO_1\cup \mO_2\) contient tous les nombres entre un minorant de \( A\) et un majorant sauf \( x_0\), mais on sait que \( x_0\) n'est pas dans \( A\). Cela prouve que \( A\) n'est pas connexe.

	Jusqu'à présent nous avons prouvé que si un ensemble n'est pas un intervalle, alors il ne peut pas être connexe. Pour remettre les choses à l'endroit, prenons un ensemble connexe, et demandons-nous si il peut être autre chose qu'un intervalle ? La réponse est \emph{non} parce que si il était autre chose, il ne serait pas connexe.

	Prouvons à présent que tout intervalle est connexe. Pour cela, nous refaisons le coup de la contraposée. Nous allons donc prendre une partie \( A\) de \( \eR\), supposer qu'elle n'est pas connexe et prouver qu'elle n'est alors pas un intervalle. Nous avons deux ouverts disjoints \( \mO_1\) et \( \mO_2\) tels que \( A\subset \mO_1\cup \mO_2\). Notons \( A_1 = A \cap \mO_1 \) et \( A_2 = A \cap \mO_2 \); et prenons \( a\in A_1\) et \( b\in A_2\). Pour fixer les idées, on suppose que \( a<b\). Maintenant, le jeu est de montrer qu'il existe un point \( x_0\) entre \( a\) et \( b\) qui ne soit pas dans \( A\) (cela montrerait que \( A\) n'est pas un intervalle). Nous allons prouver que c'est le cas du point
	\[
		x_0=\sup\{ x\in\mO_1\tq x<b \}.
	\]
	Étant donné que l'ensemble \( \mA=\{ x\in\mO_1\tq x<b \}\) est ouvert\footnote{C'est l'intersection entre l'ouvert \( \mO_1\) et l'ouvert \( \{x\tq x<b \}\).}, le point \( x_0\) n'est pas dans l'ensemble par le lemme~\ref{LemSupOuvPas}. Nous avons donc
	\begin{itemize}
		\item soit \( x_0\) n'est pas dans \( \mO_1\),
		\item soit \( x_0\leq b\),
		\item soit les deux en même temps.
	\end{itemize}
	Nous allons montrer qu'un tel \( x_0\) ne peut pas être dans \( A\). D'abord, remarquons que \( \sup\mA\leq\sup\mO_1\) parce que \( \mA\) est une intersection de \( \mO\) avec quelque chose. Ensuite, il n'est pas possible que \( x_0\) soit dans \( \mO_2\) parce que tout élément de \( \mO_2\) possède un voisinage contenu dans \( \mO_2\). Un point de \( \mO_2\) est donc toujours strictement plus grand que le supremum de \( \mO_1\).

	Maintenant, en remarquant que si \( x_0\leq b\), alors \( x_0=b\) sinon \( b\) serait un majorant de \( \mA\) plus petit que \( x_0\), ce qui n'est pas possible puisque \( x_0\) est le supremum de \( \mA\) et donc le plus petit majorant. Oui mais si \( x_0=b\), c'est que \( x_0\in\mO_2\), ce qu'on vient de montrer être impossible. Nous voilà déjà débarrassés des deuxièmes et troisièmes possibilités.

	Si la première possibilité est vraie, alors \( x_0\) n'est pas dans \( A\) parce qu'on a aussi prouvé que \( x_0\notin\mO_2\). Or n'être ni dans \( \mO_1\) ni dans \( \mO_2\) implique de ne pas être dans \( A\). Ce point \( x_0=\sup\mA\) est donc hors de \( A\).

	Oui, mais comme \( a\in\mA\), on a obligatoirement \( x_0\geq a\). Mais par construction, on a aussi \( x_0\leq b\) (ici, l'inégalité est même stricte, mais ce n'est pas important). Donc
	\[
		a\leq x_0\leq b
	\]
	avec \( a\), \( b\in A\), et \( x_0\notin A\). Cela finit de prouver que \( A\) n'est pas un intervalle.
\end{proof}

Le lemme suivant dit que si on recouvre un intervalle avec des ouverts, alors on peut ordonner ces ouverts de telle sorte qu'ils s'enchainent bien : on peut sauter de l'un à l'autre en passant par les intersections. C'est donc un lemme qui permet de passer du local au global.

\begin{lemma}       \label{LEMooNMGWooTfQDeO}
	Soient un intervalle \( I\) de \( \eR\) ainsi qu'un recouvrement \( \{ \mO_i \}_{i=1,\ldots, n}\) de \( I\) par des ouverts connexes tels que \( \mO_i\cap I\neq\emptyset\) pour tout \( i\)\footnote{Il est cependant possible que les \( \mO_i\) ne soient pas inclus dans \( I\).}. Alors il existe une bijection \( \psi\colon \{ 1,\ldots, n \}\to \{ 1,\ldots, n \}\) telle que
	\begin{enumerate}
		\item
		      \begin{equation}
			      \bigcup_{i=1}^m\mO_{\psi(i)}
		      \end{equation}
		      est connexe pour tout \( m\).
		\item
		      \begin{equation}
			      \mO_{\psi(m)}\cap\bigcup_{i=1}^{m-1}\mO_{\psi(i)}\neq \emptyset.
		      \end{equation}
	\end{enumerate}
\end{lemma}

\begin{proof}
	Nous allons construire \( \psi\) par récurrence; plus précisément nous allons construire des applications \( \psi_k\colon \{ 1,\ldots, k \}\to \{ 1,\ldots, n \}\) telle que
	\begin{enumerate}
		\item
		      \( \psi_k\) est injective.
		\item
		      Si \( i\leq k\) alors \( \psi_k(i)=\psi_i(i)\).
		\item
		      La partie
		      \begin{equation}        \label{EQooAMAGooDHHvGR}
			      \bigcup_{i=1}^m\mO_{\psi_k(i)}
		      \end{equation}
		      est connexe pour tout \( m\leq k\).
		\item
		      Nous avons
		      \begin{equation}        \label{EQooOOZUooKVlDPi}
			      \mO_{\psi_k(m)}\cap\bigcup_{i=1}^{m-1}\mO_{\psi_k(i)}\neq \emptyset
		      \end{equation}
		      pour tout \( m\leq k\).
	\end{enumerate}

	Nous commençons en douceur par
	\begin{equation}
		\begin{aligned}
			\psi_1\colon \{ 1 \} & \to \{ 1,\ldots, n \} \\
			1                    & \mapsto 1.
		\end{aligned}
	\end{equation}
	Ai-je besoin de vous prouver que c'est injectif ?

	Nous supposons que les applications \( \psi_i\) sont correctement définies pour \( i\leq k\), et nous construisons \( \psi_{k+1}\). Nous posons
	\begin{subequations}
		\begin{align}
			A & =\psi_k\big( \{ 1,\ldots, k \} \big) \\
			B & =\{ 1,\ldots, n \}\setminus A        \\
			P & =\bigcup_{i=1}^k\mO_{\psi_k(i)}      \\
			Q & =\bigcup_{i=k+1}^n\mO_{\psi_k(i)}
		\end{align}
	\end{subequations}
	En tant qu'unions d'ouverts, les parties \( P\) et \( Q\) sont ouvertes dans \( \eR\). Elles recouvrent l'intervalle \( I\) qui est connexe par la proposition \ref{PropInterssiConn}. De plus \( P\cap I\neq \emptyset\) et \( Q\cap I\neq \emptyset\); donc, par définition de la connexité nous avons \( P\cap Q\neq\emptyset\).

	Il existe donc \( i_0\in B\) tel que \( P\cap\mO_{i_0}\neq \emptyset\). Nous posons
	\begin{equation}
		\begin{aligned}
			\psi_{k+1}\colon \{ 1,\ldots, k+1 \} & \to \{ 1,\ldots, n \}                    \\
			i                                    & \mapsto \begin{cases}
				                                               \psi_k(i) & \text{si } i\neq k+1 \\
				                                               i_0       & \text{si } i=k+1.
			                                               \end{cases}
		\end{aligned}
	\end{equation}
	Vérifions que ce \( \psi_{k+1}\) vérifie les conditions.
	\begin{enumerate}
		\item
		      \( \psi_{k+1}\) est injective. Soient \( i,j\) tels que \( \psi_{k+1}(i)=\psi_{k+1}(j)\). Si \( i=k+1\) et \( j\neq k+1\) alors \( \psi_{k+1}(i)=i_0\in B\), alors que \( \psi_{k+1}(j)=\psi_k(j)\in A\). Donc le cas \( i=k+1\), \( j\neq k+1\) n'est pas possible.

		      Si \( i,j\neq k+1\), alors \( \psi_{k+1}(i)=\psi_k(i)\) et \( \psi_{k+1}(j)=\psi_k(j)\). L'injectivité de \( \psi_k\) implique que \( i=j\).
		\item
		      Si \( i\leq k\), nous avons \( \psi_{k+1}(i)=\psi_k(i)=\psi_i(i)\) en utilisant la récurrence.
		\item
		      Nous séparons les cas \( m=k+1\) et \( m\neq k+1\). Si \( m\neq k+1\) alors tous les \( \psi_{k+1}\) dans \eqref{EQooAMAGooDHHvGR}\footnote{Nous sommes en train de parler de cette équation avec \( k+1\) au lieu de \( k\), parce que nous sommes dans un processus de récurrence. Il est donc normal de dire qu'il y a des \( \psi_{k+1}\) dans cette équation.} sont des \( \psi_k\) et la récurrence fonctionne. Si \( m=k+1\) alors
		      \begin{equation}
			      \bigcup_{i=1}^{k+1}\mO_{\psi_{k+1}(i)}=\bigcup_{i=1}^k\mO_{\psi_k(i)}\cup\mO_{\psi_{k+1}(k+1)}=P\cup\mO_{i_0}.
		      \end{equation}
		      Le nombre \( i_0\) a été choisi pour avoir \( \mO_{i_0}\cap P\neq \emptyset\). Comme \( \mO_{i_0} \) et \( P\) sont des connexes, la proposition \ref{PropIWIDzzH}\ref{ITEMooLVSSooTGstBz} implique que \( P\cup\mO_{i_0}\) est connexe.
		\item
		      Encore une fois, si \( m\neq k+1\), tous les \( \psi_{k+1}\) de \eqref{EQooOOZUooKVlDPi} deviennent des \( \psi_k\) et la récurrence fonctionne. Avec \( m=k+1\) nous avons
		      \begin{equation}
			      \mO_{\psi_{k+1}(k+1)}\cap\bigcup_{i=1}^k\mO_{\psi_{k+1}(i)}=\mO_{i_0}\cap P.
		      \end{equation}
		      Cette intersection est non vide, par choix du \( i_0\).
	\end{enumerate}
	Quand tous les \( \psi_k\) (\( k=1,\ldots, n\)) sont construits, en posant \( \psi=\psi_n\) nous avons le résultat annoncé.
\end{proof}

\begin{theorem}[Théorème des bornes atteintes]\label{ThoMKKooAbHaro}
	Une fonction à valeurs réelles continue sur un compact est bornée et atteint ses bornes.

	C'est-à-dire qu'il existe \( x_0\in K\) tel que \( f(x_0)=\inf\{ f(x)\tq x\in K \}\) ainsi que \( x_1\) tel que \( f(x_1)=\sup\{ f(x)\tq x\in K \}\).
\end{theorem}
\index{compact!et fonction continue}

\begin{proof}
	Soient un espace topologique compact \( K\) et une fonction continue \( f\colon K\to \eR\). Alors le théorème~\ref{ThoImCompCotComp} indique que \( f(K)\) est compact. Par conséquent \( f(K)\) est un fermé borné de \( \eR\) par le théorème de Borel-Lebesgue~\ref{ThoXTEooxFmdI}. Puisque \( f(K)\) est borné, la fonction \( f\) est bornée.

	De plus \( f(K)\) étant fermé, son infimum est un minimum et son supremum est un maximum : il existe \( x\in K\) tel que \( f(x)=\sup f(K)\) et il existe \( y\in K\) tel que \( f(y)=\inf f(K)\).
\end{proof}


\begin{definition}[propriété de valeurs intermédiaires\cite{BIBooCNVHooXZImOd}]		\label{DEFooIVFZooLLlahl}
	Un espace topologique \( X\) a la \defe{propriété de valeurs intermédiaires}{propriété des valeurs intermédiaires} si pour toute fonctions continue \(f \colon X\to \eR  \), si pour tout \( a,b\in X\), nous avons \(\mathopen[ r_1,r_2\mathclose]\subset f(X) \).
\end{definition}

\begin{proposition}[\cite{BIBooCNVHooXZImOd}]		\label{PROPooGURQooAwKNUJ}
	Un espace topologique a la propriété des valeurs intermédiaires si et seulement si il est connexe.
\end{proposition}

\begin{proof}
	Les deux sens.
	\begin{subproof}
		\spitem[\( \Rightarrow\)]
		%-----------------------------------------------------------
		Soit un espace topologique \( X\) que nous supposons ne pas être connexe. Nous considérons des ouverts \( A\) et \( B\) tels que \( A\cup B=X\) et \( A\cap B=\emptyset\). Ensuite nous prenons la fonction
		\begin{equation}
			\begin{aligned}
				f\colon X & \to \eR                        \\
				x         & \mapsto \begin{cases}
					                    1  & \text{si } x\in A \\
					                    -1 & \text{si }x\in B.
				                    \end{cases}
			\end{aligned}
		\end{equation}
		Cette fonction ne vérifie manifestement pas la propriété des valeurs intermédiaires. Et pourtant elle est continue. En effet si \( \mO\) est un ouvert de \( \eR\), alors \( f^{-1}(\mO)\) peut valoir \( \emptyset\), \( A\), \( B\) ou \( X\) suivant que \( 1\) ou \( -1\) sont ou non dans \( \mO\). Or toutes ces parties sont ouvertes. Donc \( f^{-1}(\mO)\) est ouvert.

		\spitem[\( \Leftarrow\)]
		%-----------------------------------------------------------
		Nous supposons maintenant que \( X\) est connexe, nous considérons une fonction continue \(f \colon X\to \eR  \), et nous montrons qu'elle vérifie la propriété des valeurs intermédiaires. Le lemme \ref{LemConncontconn} dit que \( f(X)\) est un connexe de \( \eR\). La proposition \ref{PropInterssiConn} dit alors que \( f(X)\) est un intervalle de \( \eR\).
	\end{subproof}
\end{proof}


%-------------------------------------------------------
\subsection{Théorème de Bolzano-Weierstrass}
%----------------------------------------------------


Le théorème suivant est essentiellement inutile pour les raisons suivantes :
\begin{itemize}
	\item
	      Il est un cas particulier du théorème~\ref{ThoBWFTXAZNH} qui donne pour tout espace métrique, l'équivalence entre la compacité et la compacité séquentielle.
	\item
	      Il est un cas particulier du théorème \ref{THOooRDYOooJHLfGq} qui le donne pour tous les espaces compacts.
\end{itemize}
Bref, nous ne le laissons que pour le lecteur qui n'aurait pas en tête d'autres définitions de «compact» à part «fermé borné».

% Pour les raisons invoquées, il ne faut pas faire de références vers ce théorème. Le label ici ne sert qu'à le mettre dans l'index thématique.
\begin{theorem}[Théorème de Bolzano-Weierstrass]        \label{ThoBolzanoWeierstrassRn}
	Toute suite contenue dans un compact de \( \eR^m\) admet une sous-suite convergente.
\end{theorem}

\begin{proof}
	Nous rappelons qu'une partie compacte de \( \eR^m\) est fermée et bornée par le théorème de Borel-Lebesgue~\ref{ThoXTEooxFmdI}.

	Soit \( (x_n)\) une suite contenue dans une partie bornée de \( \eR^m\). Considérons \( (a_n)\), la suite réelle des premières composantes des éléments de \( (x_n)\) : pour chaque \( n\in\eN\), le nombre \( a_n\) est la première composante de \( x_n\). Étant donné que la suite \( (x_n)\) est bornée, il existe un \( M\) tel que \( \| x_n \|<M\). La croissance de la fonction racine carrée donne
	\begin{equation}
		| a_n |\leq\| x_n \|\leq M.
	\end{equation}
	La suite \( (a_n)\) est donc une suite réelle bornée et donc contient une sous-suite convergente par le théorème correspondant dans \( \eR\) :  \ref{ThoBWFTXAZNH}. Soit \( a_{I_1}\) une sous-suite convergente de \( a_n\). Nous considérons maintenant \( x_{I_1}\), c'est-à-dire la suite de départ dont on a enlevé tous les éléments qu'il faut pour qu'elle converge en ce qui concerne la première composante.

	Si nous considérons la suite \( b_{I_1}\) des \emph{secondes} composantes de \( x_{I_1}\), nous en extrayons, de la même façon que précédemment, une sous-suite convergente, c'est-à-dire que nous avons un \( I_2\subset I_1\) tel que \( b_{I_2}\) est convergent. Notons que \( a_{I_2}\) est une sous-suite de la (sous) suite convergente \( x_{I_1}\), et donc \( a_{I_2}\) est encore convergente.

	En continuant ainsi, nous construisons une sous-sous-sous-suite \( x_{I_3}\) telle que la suite des \emph{troisièmes} composantes est convergente. Lorsque nous avons effectué cette procédure \( m\) fois, la suite \( x_{I_m}\) est une suite dont toutes les composantes convergent, et donc est une suite convergente par la proposition~\ref{PropCvRpComposante}.

	Le tableau suivant donne un petit schéma de la façon dont nous procédons. Les \( \bullet\) sont les éléments de la suite que nous gardons, et les \( \times\) sont ceux que nous «jetons».
	\begin{equation}
		\begin{array}{lccccccccccc}
			x_{\eN} & \bullet & \bullet & \bullet & \bullet & \bullet & \bullet & \bullet & \bullet & \bullet & \bullet & \ldots \\
			x_{I_1} & \times  & \bullet & \bullet & \times  & \bullet & \times  & \times  & \bullet & \bullet & \bullet & \ldots \\
			x_{I_2} & \times  & \bullet & \times  & \times  & \bullet & \times  & \times  & \bullet & \bullet & \times  & \ldots \\
			\vdots                                                                                                               \\
			x_{I_m} & \times  & \times  & \times  & \times  & \bullet & \times  & \times  & \times  & \bullet & \times  & \ldots
		\end{array}
	\end{equation}
	La première ligne, \( x_{\eN}\), est la suite de départ.
\end{proof}

\begin{corollary}   \label{CorFHbMqGGyi}
	Si une suite est croissante et bornée alors elle est convergente.
\end{corollary}

\begin{proof}
	Nous nommons \( (x_n)\) la suite et nous prenons un majorant \( M\). Toute la suite est alors contenue dans le compact \( \mathopen[ x_0 , M \mathclose]\), ce qui donne une sous-suite \( (x_{\alpha(n)})\) convergente par le théorème de Bolzano-Weierstrass~\ref{THOooRDYOooJHLfGq}. Si \( \ell\) est la limite de cette sous-suite alors nous avons \( \ell\geq x_n\) pour tout \( n\).

	Pour tout \( \epsilon>0\) il existe \( K\) tel que si \( n>K\) alors \( | \ell-x_{\alpha(n)} |<\epsilon\). Comme \( \ell\) majore la suite nous avons même
	\begin{equation}
		x_{\alpha(n)}+\epsilon>\ell.
	\end{equation}
	Puisque la suite est croissante pour tout \( m>\alpha(K)\) nous avons \( x_m+\epsilon>\ell\), ce qui signifie \( | x_m-\ell |<\epsilon\).
\end{proof}
Nous aurons une version pour les fonctions croissantes et bornées en la proposition~\ref{PropMTmBYeU}.

La proposition suivante dit que la notion d'ensemble non dénombrable ne prend pas réellement de force entre \( \eR\) et \( \eR^n\) : il n'y a pas moyen de caser \( \eR\) dans \( \eR^n\) de façon à ce qu'il y tienne à son aise.

\begin{proposition}
	Une partie non dénombrable de \( \eR^n\) possède un point d'accumulation\footnote{Définition \ref{DEFooGHUUooZKTJRi}.}.
\end{proposition}

\begin{proof}
	Soit une partie \( A\subset \eR^n\) sans point d'accumulation. Nous allons prouver que \( A\) est dénombrable.

	Soient les compacts \( K_n=\overline{ B(0,n) }\). La partie \( A\cap K_n\) est finie; sinon elle aurait une partie en bijection avec \( \eN\) (proposition~\ref{PROPooUIPAooCUEFme}) et donc une suite. Or une suite dans un compact possède un point d'accumulation par le théorème~\ref{THOooRDYOooJHLfGq}.

	Donc tous les \( A\cap K_n\) sont finis. Puisque \( A=\bigcup_nA\cap K_n\), l'ensemble \( A\) est une réunion dénombrable d'ensembles finis. Il est donc dénombrable.
\end{proof}

%---------------------------------------------------------------------------------------------------------------------------
\subsection{Recouvrement par des intervalles ouverts}
%---------------------------------------------------------------------------------------------------------------------------

Soit un ensemble \( E\) et un ensemble \( \mA\) de parties de \( E\). Soit \( A\in \mA\). Nous aimerions savoir quels sont les éléments de \( \mA\) qui sont atteignables en partant de \( A\) et en ne «sautant» que d'intersection en intersection.

Nous notons \( \mA=\{ B_i \}_{i\in I}\) où \( I\) est un ensemble d'indices (un ensemble quelconque).
\begin{subequations}
	\begin{align}
		s_1(A)      & =\{  i\in I\tq B_i\cap A\neq \emptyset   \} \\
		\sigma_1(A) & =\bigcup_{B\in s_1(A)}B.
	\end{align}
\end{subequations}
Et ensuite :
\begin{subequations}
	\begin{align}
		s_{k+1}(A)      & =\{ i\in I\tq B_i\cap \sigma_k(A)\neq \emptyset \} \\
		\sigma_{k+1}(A) & =\bigcup_{B\in s_{k+1}(A)}B
	\end{align}
\end{subequations}

\begin{lemma}       \label{LEMooJHMTooXwxSAa}
	Soient un intervalle \( A\) de \( \eR\) et \( \mA=\{ I_i \}_{i=1,\ldots, N}\) un recouvrement de \( A\) par des intervalles ouverts. Si \( I_1\cap A\neq \emptyset\) alors
	\begin{enumerate}
		\item
		      \( \sigma_{N}=\sigma_{N+1}\)
		\item
		      \( A\subset \sigma_N(I_1)\).
	\end{enumerate}
\end{lemma}

\begin{proof}
	Si \( \sigma_{k+1}=\sigma_k\), alors tous les \( \sigma_{k+l}\) sont identiques. De plus si \( \sigma_{k+1}\neq \sigma_k\), alors \( \sigma_{k+1}\) contient au moins un élément de plus que \( \sigma_k\). Donc \( \Card(\sigma_k)\geq k\) et en particulier \( N\leq \Card(\sigma_N)\leq N\). Cela prouve le premier point.

	L'ensemble \( \sigma_N(I_1)\) est une union d'ouverts et est donc un ouvert. Quitte à renuméroter, nous écrivons
	\begin{equation}
		\sigma_N(I_1)=I_1\cup \ldots \cup I_n.
	\end{equation}
	L'ensemble
	\begin{equation}
		\tau=\bigcup_{k=n+1}^NI_k
	\end{equation}
	est ouvert et est disjoint de \( \sigma_N(I_1)\) parce que si \( I_l\) (\( l\geq n+1\)) intersectait \( \sigma_N(I_1)\), nous aurions \( l\in s_{N+1}\) ou encore \( I_l\subset \sigma_{N+1}\setminus\sigma_N\).

	Donc \( \tau\) et \( \sigma_N\) sont deux ouverts disjoints qui recouvrent \( A\). Puisque \( A\) est un intervalle, il est connexe\footnote{Définition \ref{DefIRKNooJJlmiD} et proposition \ref{PropInterssiConn}.}. Donc, soit \( A\subset \tau\), soit \( A\subset \sigma_N\). Comme \( I_1\cap A\neq \emptyset\) nous sommes dans le cas \( A\subset \sigma_N\).
\end{proof}

\begin{lemma}       \label{LEMooGHPTooKgFvGb}
	Soit \( x\in \eR\). Si \( \mA=\{ I_s \}_{s\in S}\) est un ensemble d'intervalles contenant \( x\), alors \( I=\bigcup_{s\in S}I_s\) est un intervalle\footnote{Définition \ref{DefEYAooMYYTz}.}.
\end{lemma}

\begin{proof}
	Soient \( a,b\in I \) (nous supposons \( a<b\)). Nous devons prouver que \( \mathopen[ a , b \mathclose]\subset I\). Pour cela nous considérons \( y\in \mathopen[ a , b \mathclose]\); il y a deux possibilités : soit \( y<x\) soit \( y>x\) (si \( y=x\) alors \( y\in I_s\)).

	Si \( y<x\), alors \( a\leq y<x\) et donc \( y=\in I\). Si \( y>x\), alors \( x<y\leq b\) et \( y\in I\).
\end{proof}

\begin{proposition}[\cite{BIBooVUFIooNEETXD,BIBooFEKBooFhPatO}]         \label{PROPooTXMBooZaSKFF}
	Un ouvert de \( \eR\) peut s'écrire comme union au plus dénombrable d'intervalles ouverts disjoints.

	Plus précisément, si \( \mO\) est un ouvert de \( \eR\), il existe un ensemble \( \mF=\{ I_s \}_{s\in S}\) où
	\begin{enumerate}
		\item
		      Chaque \( I_s\) est un intervalle ouvert contenu dans \( \mO\),
		\item
		      Pour \( s,t\in S\), si \( I_s\neq I_t\), alors \( I_s\cap I_t=\emptyset\).
		\item
		      \( S\) est dénombrable,
	\end{enumerate}
\end{proposition}   \index{intervalle}

\begin{proof}
	Pour \( x\in \mO\), nous définissons \( J_x\) comme étant l'union de tous les intervalles ouverts contenus dans \( \mO\) et contenant \( x\). Les \( J_x\) ne sont pas vides parce qu'ils contiennent toujours une boule centrée en \( x\)\footnote{C'est la définition \ref{EqGDVVooDZfwSf} de la topologie métrique.}.

	En tant qu'union d'intervalles, \( J_x\) est un intervalle par le lemme \ref{LEMooGHPTooKgFvGb}. De plus, \( J_x\) est ouvert parce que toute union d'ouverts est ouverte\footnote{C'est dans la définition \ref{DefTopologieGene} d'une topologie.}.

	Nous notons \( \mA\) l'ensemble des intervalles ouverts contenus dans \( \mO\), et
	\begin{equation}
		\mA_x=\{ I\in\mA\tq x\in I \}.
	\end{equation}

	\begin{subproof}
		\spitem[Si \( y\in J_x\), alors \( J_x=J_y\)]
		%---------------------------------------------------------------
		Puisque \( y\in J_x\), nous pouvons considérer \( J=\mA_x\cap \mA_y\). Nous avons
		\begin{equation}
			J_y=\bigcup_{I\in\mA_y}I\subset \bigcup_{I\in \mA_y}\underbrace{(I\cup J)}_{\in\mA_x}\subset \bigcup_{I\in\mA_x}I=J_x.
		\end{equation}
		L'inclusion dans l'autre sens s'obtient en écrivant la même équation en échangeant \( x\) et \( y\).
		\spitem[Les \( J_x\) sont disjoints]
		%---------------------------------------------------------------
		Nous prouvons à présent que pour \( x,y\in \mO\), nous avons \( J_x=J_y \) ou \( J_x\cap J_y=\emptyset\). En effet si \( a\in J_x\cap J_y\), alors \( J_a=J_x\) et \( J_a=J_y\). Donc \( J_x=J_y\).
		\spitem[Dénombrable]
		%---------------------------------------------------------------
		C'est le moment d'écrire \( \mF=\{ J_x \}_{x\in \mO}\). Comme tout intervalle contient au moins un rationnel (proposition \ref{PropooUHNZooOUYIkn}), nous avons aussi
		\begin{equation}
			\mF=\{ J_x \}_{x\in \mO}=\{ J_q \}_{q\in \eQ\cap\mO}.
		\end{equation}
		Cet ensemble \( \mF\) vérifie les conditions demandées.
	\end{subproof}
\end{proof}
