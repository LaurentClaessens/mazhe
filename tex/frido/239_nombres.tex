% This is part of Mes notes de mathématique
% Copyright (c) 2011-2024
%   Laurent Claessens
% See the file fdl-1.3.txt for copying conditions.

%+++++++++++++++++++++++++++++++++++++++++++++++++++++++++++++++++++++++++++++++++++++++++++++++++++++++++++++++++++++++++++ 
\section{Groupes}
%+++++++++++++++++++++++++++++++++++++++++++++++++++++++++++++++++++++++++++++++++++++++++++++++++++++++++++++++++++++++++++

%---------------------------------------------------------------------------------------------------------------------------
\subsection{Définition, unicité du neutre}
%---------------------------------------------------------------------------------------------------------------------------

La définition d'un groupe est la définition \ref{DEFooBMUZooLAfbeM}.

\begin{lemmaDef}[Unicités]  \label{LEMooECDMooCkWxXf}
	Dans un groupe, l'inverse et le neutre sont uniques. Plus précisément, si \( G\) est un groupe nous avons :
	\begin{enumerate}
		\item
		      il existe un unique élément \( e\in G\) tel que \( e g=g e=g\) pour tout \( g\in G\),
		\item       \label{ITEMooOIWTooYqmMPP}
		      pour tout \( g\in G\), il existe un unique élément \( h\in  G\) tel que \(g h=h g=e \).
	\end{enumerate}
	Le \( e\) ainsi défini est nommé \defe{neutre}{neutre!dans un groupe} de \( G\). Le \( h\) tel que \( g h=h g=e\) est nommé l'\defe{inverse}{inverse!dans un groupe} de \( g\) et est noté \( g^{-1}\).
\end{lemmaDef}

\begin{proof}
	Chaque point séparément.
	\begin{enumerate}
		\item
		      Supposons que \( e_1\) et \( e_2\) vérifient la propriété. Nous avons pour tout \( g\in G\) : \( e_1g=ge_1=g\). En particulier pour \( g=e_2\) nous écrivons \( e_1e_2=e_2e_1=e_2\). Mais en partant dans l'autre sens : \( e_2g=ge_2=g\) avec \( g=e_1\) nous avons \( e_2e_1=e_1e_2=e_1\). En égalant ces deux valeurs de \( e_2e_1\) nous avons \( e_1=e_2\).

		      Pour la suite de la preuve nous écrivons \( e\) l'unique neutre de \( G\).

		\item
		      Supposons que \( k_1\) et \( k_2\) soient deux inverses de \( g\). On considère alors le produit \( k_1 g k_2 \). Puisque \(k_1 g = e \), on a \( k_1 g k_2 = e k_2 = k_2 \); mais, comme \(g k_2 = e \), on a aussi \( k_1 g k_2 = k_1 e = k_1 \). Le produit est donc à la fois égal à \( k_1 \) et à \( k_2 \), et donc \( k_1 = k_2 \).
	\end{enumerate}
\end{proof}

\begin{lemma}       \label{LEMooWYLRooNOdZnp}
	Soient deux groupes \( G\) et \( H\). Si \( \alpha\colon G\to H\) est un morphisme de groupes\footnote{Définition \ref{DEFooBEHTooMeCOTX}.}, alors
	\begin{enumerate}
		\item
		      \( \alpha(e_G)=e_H\).
		\item
		      \( \alpha(g^{-1})=\alpha(g)^{-1}\).
	\end{enumerate}
\end{lemma}

\begin{lemma}       \label{LEMooBIBFooBHxFYC}
	Si \( G\) est un groupe et si \( h\in G\), alors les applications
	\begin{equation}
		\begin{aligned}
			L_h\colon G & \to G      \\
			g           & \mapsto hg
		\end{aligned}
	\end{equation}
	et
	\begin{equation}
		\begin{aligned}
			R_h\colon G & \to G      \\
			g           & \mapsto gh
		\end{aligned}
	\end{equation}
	sont des bijections.
\end{lemma}

\begin{proof}
	D'abord si \( L_h(g_1)=L_h(g_2)\), alors \( hg_1=hg_2\) et en multipliant à gauche par \( h^{-1}\) nous avons \( g_1=g_2\); donc \( L_h\) est injective. Ensuite \( L_h\) est surjective parce que si \( g\in G\), alors \( g=L_h(h^{-1} g)\).

	Pour l'application \( R_h\), la preuve est une simple adaptation.
\end{proof}

%--------------------------------------------------------------------------------------------------------------------------- 
\subsection{Groupe ordonné}
%---------------------------------------------------------------------------------------------------------------------------

\begin{definition}[\cite{BIBooZIFGooBrNwWU}]        \label{DEFooEUHFooYvhnLQ}
	Soient un groupe \( (G,+)\), ainsi qu'une relation d'ordre \( \leq\) sur \( G\). Nous disons que la relation d'ordre est \defe{compatible}{} avec la structure de groupe si pour tout \( x,y,z\in G\), si \( x\leq y\) alors \( x+z\leq y+z\) et \( z+x\leq z+y\). Dans ce cas, le triple \( (G,+,\leq)\) est un \defe{groupe ordonné}{groupe ordonné}.

	Si \( (G,\leq)\) est totalement ordonné, nous disons que le groupe est totalement ordonné.
\end{definition}

%---------------------------------------------------------------------------------------------------------------------------
\subsection{Classes de conjugaison}
%---------------------------------------------------------------------------------------------------------------------------

\begin{definition}[classe de conjugaison]       \label{DEFooOLXPooWelsZV}
	Soit un groupe \( G\) et un élément \( g\in G\). La \defe{classe de conjugaison}{classe!de conjugaison} de \( g\) est la partie
	\begin{equation}
		C_g=\{ kgk^{-1}\tq k\in G \}.
	\end{equation}
\end{definition}

\begin{lemma}       \label{LEMooQYBJooYwMwGM}
	Un groupe est commutatif si et seulement si ses classes de conjugaison sont des singletons.
\end{lemma}

\begin{proof}
	Supposons que \( G\) soit commutatif. Alors
	\begin{equation}
		C_g=\{ kgk^{-1}\tq k\in G \}=\{ g \}.
	\end{equation}
	Donc les classes de conjugaison sont des singletons.

	Dans l'autre sens, si les classes sont des singletons, on a \( kgk^{-1}=g\) pour tous \( k,g\in G\). Cela signifie immédiatement que \( G\) est commutatif.
\end{proof}

\begin{definition}[centralisateur\cite{Kropholler}]         \label{defGroupeCentre}
	Soient un groupe \( G\), un sous-groupe \( H\) et un élément \( h\in H\). Le \defe{centralisateur}{centralisateur} de \( h\) dans \( G\) est l'ensemble des éléments de \( G\) qui commutent avec \( h\) :
	\begin{equation}
		Z_G(h)=\{z\in G\tq hz=zh\}.
	\end{equation}
	Le centralisateur de \( H\) dans \( G\) est l'ensemble des éléments de \( G\) qui commutent avec tous les éléments de \( H\) :
	\begin{equation}
		Z_G(H)=\bigcap_{h\in H}Z_G(h).
	\end{equation}
	Le \defe{centre}{centre!d'un groupe} d'un groupe \( G\) est l'ensemble des éléments de \( G\) qui commutent avec tous les autres:
	\begin{equation}
		Z_G=Z_G(G)=\{ z\in G\tq gz=zg , \forall g\in G \}.
	\end{equation}
\end{definition}

\begin{definition}[normalisateur\cite{Kropholler}]          \label{DEFooZTSMooBislIy}
	Soient un groupe \( G\) et un sous-groupe \( H\). Le \defe{normalisateur}{normalisateur} de \( H\) dans \( G\) est
	\begin{equation}
		\mN_G(H)=\{ g\in G\tq gH=Hg \}.
	\end{equation}
\end{definition}

\begin{definition}[Sous-groupe normal]                      \label{DEFooNIIMooFkZgvX}
	Un sous-groupe \( N\) de \( G\) est \defe{normal}{normal!sous-groupe} ou \defe{distingué}{sous-groupe distingué}\index{distingué!sous-groupe} si pour tout \( g\in G\) et pour tout \( n\in N\), \( gng^{-1}\in N\). Autrement dit lorsque \( gNg^{-1}\subset N\).

	Lorsque \( N\) est normal dans \( G\) il est parfois noté \( N\normal G\)\nomenclature[R]{\(N \normal G\)}{Le sous-groupe \( N\) est normal dans \( G\)}.
\end{definition}

\begin{definition}      \label{DEFooUXXTooCCLmQe}
	Un sous-groupe \( H\) de \( G\) est un sous-groupe \defe{caractéristique}{sous-groupe!caractéristique}\index{caractéristique!sous-groupe} si \( \alpha(H)\subset H\) pour tout automorphisme\footnote{Automorphisme de groupe, définition \ref{DEFooBEHTooMeCOTX}.} \( \alpha\) de \( G\).
\end{definition}

\begin{lemma}[\cite{BIBooOHKHooAcewiw}]
	Si \( H\) est un sous-groupe caractéristique de \( G\), alors \( \alpha(H)=H\) pour tout automorphisme \( \alpha\) de \( G\).
\end{lemma}

\begin{proof}
	Si \( \alpha\) est un automorphisme de \( G\), alors \( \alpha^{-1}\) est encore un automorphisme de \( G\). En particulier \( \alpha^{-1}(H)\subset H\).

	Soit \( h\in H\). Nous devons prouver que \( h\in \alpha(H)\). Pour cela :
	\begin{equation}
		h=\alpha\big( \alpha^{-1}(h) \big)\in \alpha\big( \alpha^{-1}(H) \big)\subset\alpha(H).
	\end{equation}
\end{proof}

\begin{definition}[Groupe simple]                 \label{DefGroupeSimple}
	Un groupe est dit \defe{simple}{groupe simple}\index{groupe!simple} si il est non trivial et si les seuls sous-groupes normaux qu'il admet sont lui-même et le sous-groupe réduit à l'élément neutre.
\end{definition}

%+++++++++++++++++++++++++++++++++++++++++++++++++++++++++++++++++++++++++++++++++++++++++++++++++++++++++++++++++++++++++++ 
\section{Anneaux}
%+++++++++++++++++++++++++++++++++++++++++++++++++++++++++++++++++++++++++++++++++++++++++++++++++++++++++++++++++++++++++++

\begin{definition}      \label{DEFooKWKGooIOwGTA}
	Un \defe{isomorphisme d'anneaux}{isomorphisme!d'anneaux} est un morphisme d'anneaux\footnote{Définition \ref{DEFooSPHPooCwjzuz}.}, bijectif.
\end{definition}

La distributivité de la partie \ref{ITEMooGMNOooSTGiXw} de la définition \ref{DefHXJUooKoovob} ne traite que de l'addition; pas de la soustraction. Voici une lemme qui dit que ça fonctionne quand même.
\begin{lemma}[\cite{BIBooZFPUooIiywbk}]     \label{LEMooVPYUooRzexke}
	Soient un anneau \( A\) ainsi que \( a,b,c\in A\). Alors
	\begin{equation}
		a(b-c)=ab-ac.
	\end{equation}
\end{lemma}

\begin{proof}
	Nous avons le calcul suivant :
	\begin{subequations}
		\begin{align}
			a(b-c)+ac & =a\big( (b-c)+c \big)     \label{SUBEQooKCOWooFeOHUM} \\
			          & =ab.       \label{SUBEQooMLLOooNRmIYM}
		\end{align}
	\end{subequations}
	Justifications :
	\begin{itemize}
		\item Pour \ref{SUBEQooKCOWooFeOHUM}. Distributivité.
		\item Pour \ref{SUBEQooMLLOooNRmIYM}. Parce que \( (b-c)+c=b\).
	\end{itemize}
	Nous avons donc \( a(b-c)+ac=ab\) et donc l'égalité demandée en ajoutant \( -ac\) des deux côtés.
\end{proof}

\begin{lemma}       \label{LEMooVUSMooWisQpD}
	Pour tout élément \( a\) d'un anneau nous avons \( a\cdot 0=0\).
\end{lemma}

\begin{proof}
	L'élément \( 0\) est le neutre de l'addition. Il peut être écrit \( 1-1\), et en utilisant la distributivité sous la forme du lemme \ref{LEMooVPYUooRzexke},
	\begin{equation}
		a\cdot 0=a\cdot (1-1)=a-a=0.
	\end{equation}
	Notons que la dernière égalité s'écrit en détail \( a-a=a+(-a)\) qui donne le neutre de l'addition.
\end{proof}

\begin{proposition}     \label{PROPooNCCGooXjVyVt}
	Dans un anneau\footnote{Définition \ref{DefHXJUooKoovob}.} non nul, le neutre pour l'addition est distinct du neutre pour la multiplication.
\end{proposition}
\begin{proof}
	Supposons par contraposée que dans un anneau \( A\), nous ayons \( 1 = 0 \). Alors, pour tout \( a \in A \), on a \( a = 1a = 0a = (1 - 1)a = a - a=0 \), \( a = 0\). Autrement dit \( a=0\) pour tout \( a\in A\) et l'anneau est nul.
\end{proof}

\begin{lemma}[\cite{MonCerveau}]        \label{LEMooLTERooVKgqjn}
	Un peu d'arithmétique. Soit un anneau \( A\) et un élément \( a\in A\).
	\begin{enumerate}
		\item       \label{ITEMooUGHCooOPgoeR}
		      \( 1\times 1=1\).
		\item       \label{ITEMooJMBSooVgvVwg}
		      \( (-1)\times a=-a\).
		\item       \label{ITEMooXJGMooKNLlHU}
		      \( -(-a)=a\).
		\item       \label{ITEMooYMRKooHVYYKU}
		      \( (-1)\times (-1)=1\).
	\end{enumerate}
\end{lemma}

\begin{proof}
	En plusieurs parties.
	\begin{subproof}
		\spitem[Pour \ref{ITEMooUGHCooOPgoeR}]
		La définition de \( 1\) est que \( 1\times a=a\) pour tout \( a\). En particulier pour \( a=1\) nous avons le résultat.
		\spitem[Pour \ref{ITEMooJMBSooVgvVwg}]
		Nous avons
		\begin{equation}
			(-1)\times a + a= a\times \big( (-1)+1 \big)=a\times 0=0.
		\end{equation}
		Nous avons utilisé le fait que la multiplication était distributive et que le zéro était absorbant (lemme \ref{LEMooVUSMooWisQpD}).

		\spitem[Pour \ref{ITEMooXJGMooKNLlHU}]
		Nous avons \( -a+a=0\) par définition de la notation \( -a\). Donc \( a\) est bien l'inverse de \( -a\) pour l'addition.

		\spitem[Pour \ref{ITEMooYMRKooHVYYKU}]
		En utilisant les points \ref{ITEMooJMBSooVgvVwg} et \ref{ITEMooXJGMooKNLlHU} nous avons
		\begin{equation}
			(-1)\times (-1)=-(-1)=1.
		\end{equation}
	\end{subproof}
\end{proof}

Soit \( X\) un ensemble et un anneau \( (A, +, \times)\). Nous considérons \( \Fun(X,A)\)\nomenclature[A]{\( \Fun(X,Y)\)}{les applications de \( X\) vers \( Y\)} l'ensemble des applications \( X\to A\). Cet ensemble devient un anneau avec les définitions
\begin{subequations}
	\begin{align}
		(f+g)(x)=f(x)+g(x) \\
		(fg)(x)=f(x)g(x).
	\end{align}
\end{subequations}
C'est la \defe{structure canonique}{structure d'anneau canonique} d'anneau sur \( \Fun(X,A)\).

\begin{definition}
	Le \defe{centralisateur}{centralisateur} de \( x\in A\) dans \( A\) est l'ensemble
	\begin{equation}
		\{ y\in A\tq xy=yx \},
	\end{equation}
	le \defe{centre}{centre!d'un anneau} de \( A\) est
	\begin{equation}
		\{ y\in A\tq xy=yx,\forall x\in A \}.
	\end{equation}
\end{definition}

%-------------------------------------------------------
\subsection{Idéal dans un anneau}
%----------------------------------------------------

\begin{definition}[Idéal dans un anneau]  \label{DefooQULAooREUIU}
	Un sous-ensemble \( I\subset A\) est un \defe{idéal à gauche}{idéal!dans un anneau} si
	\begin{enumerate}
		\item
		      \( I\) est un sous-groupe pour l'addition,
		\item
		      pour tout \( a\in A\), \( aI\subset I\).
	\end{enumerate}
	De même nous disons que \( I\subset A\) est une \defe{idéal à droite}{idéal à droite} lorsque \( I\) est un sous-groupe pour l'addition et \( Ia\subset I\) pour tout \( a\in A\).

	Lorsqu'un ensemble est idéal à gauche et à droite, nous disons que c'est un \defe{idéal bilatère}{idéal!bilatère}. Lorsque nous parlons d'idéal sans précision, nous parlons d'idéal bilatère.
\end{definition}


\begin{lemma}		\label{LEMooMAHXooXSowdn}
	Les seuls idéaux d'un corps sont \( \{ 0 \}\) et le corps lui-même.
\end{lemma}

%-------------------------------------------------------
\subsection{Idéal engendré}
%----------------------------------------------------


\begin{proof}
	Soient un corps \( \eK\) et un idéal \( I\) dans \( A\). Si \( I=\{ 0 \}\), c'est un idéal, pas de problèmes. Si \( I\neq\{ 0 \}\), alors \( 0\in I\) parce qu'un idéal doit contenir le neutre de l'addition.

	Soit \( x\neq 0\) dans \( I\). Alors pour tout \( a\in \eK\) nous avons \( ax\in I\). En particulier avec \( a=x^{-1}\) nous voyons que \( 1\in I\). De là, \( I=\eK\) parce que si \( x\in \eK\), nous avons \( x=x\cdot 1\in xI\subset I\).
\end{proof}

\begin{propositionDef}[Idéal engendré par un élément]  \label{DefSKTooOTauAR}
	Soit un anneau \( A\) et une partie \( S\subset A\). Nous notons \( (S)\) l'intersection de tous les idéaux de \( A\) contenant \( S\). C'est un idéal.

	Nous nommons \( (S)\)\nomenclature[A]{\( (S)\)}{idéal engendré par \( S\)} l'\defe{idéal engendré}{idéal engendré} par \( S\),
\end{propositionDef}

\begin{proof}
	Nous prouvons que \( (S)\) est un idéal. En deux parties.
	\begin{subproof}
		\spitem[Somme]
		%-----------------------------------------------------------
		Soient \( a,b\in (S)\). Nous devons prouver que \( a+b\in (S)\) c'est à dire que que si \( I\) est un idéal de \( A\), alors \( a+b\in I\). Soit donc un idéal \( I\). Vu que \( (S)\subset I\) nous avons \( a,b\in I\). Et comme \( I\) est un idéal, \( a+b\in I\).
		\spitem[\( a(S)\subset (S)\)]
		%-----------------------------------------------------------
		Soit \( a\in A\). Nous prouvons que \( a(S)\subset (S)\). Soit \( x\in (S)\) et un idéal \( I\) de \( A\). Nous avons \( x\in (S)\subset I\). Donc \( ax\in I\).
	\end{subproof}
\end{proof}

\begin{proposition}[\cite{BIBooKGBRooAujjRR,MonCerveau}]	\label{PROPooDTYUooJPzPZV}
	Soient un anneau \( A\) et une partie \( S\subset A\). L'idéal engendré par \( S\) est la partie
	\begin{equation}
		P=\{ \sum_{i=1}^ra_is_ib_i \tq a_i,b_i\in A, s_i\in S \}.
	\end{equation}
\end{proposition}

\begin{proof}
	Nous commençons par remarquer que \( P\) est un idéal. Ensuite nous prouvons
	\begin{subproof}
		\spitem[\( P\subset (S)\)]
		%-----------------------------------------------------------
		Il faut montrer que \( P\) est inclus à tous les idéaux contenant \( S\). Soit \( I\) un tel idéal. Soit un élément \( \sum_{i=1}^ra_is_ib_i\) dans \( P\). Vu que \( s_i\in S\subset I\), nous avons \( s_i\in I\) et donc \( a_is_ib_i\in I\).

		\spitem[\( (S)\subset P\)]
		%-----------------------------------------------------------
		Nous savons que \( P\) est un idéal contenant \( S\). Comme \( (S)\) est l'intersection de tous les idéaux contenant \( S\), en particulier \( (S)\subset P\).
	\end{subproof}
\end{proof}


\begin{lemma}[\cite{MonCerveau}]	\label{LEMooHSVYooRjhGUU}
	Soient un anneau commutatif \( A\) ainsi que \( S\subset A\). Les parties
	\begin{equation}
		I=\bigcap_{s\in S}sA
	\end{equation}
	et
	\begin{equation}
		J=\bigcup_{\alpha\text{ fini dans }S}\sum_{s\in \alpha}sA
	\end{equation}
	sont des idéaux.
	%TODOooGIZEooXLjfFT. Prouver ça.
\end{lemma}


\begin{proposition}[\cite{BIBooXLOMooVnXMbS, MonCerveau}]	\label{PROPooMMHPooZYzvdK}
	Soit un anneau commutatif \( A\). Soit une partie \( S\) de \( A\).
	\begin{enumerate}
		\item		\label{ITEMooJKGMooCqWYOq}
		      \( m\) est un ppcm de \( S\) si et seulement si il est générateur de l'idéal\footnote{Lemme \ref{LEMooHSVYooRjhGUU}.} \( I=\bigcap_{s\in S}sA\).
		\item		\label{ITEMooZUCVooRIpnhU}
		      Si \( \delta\) est un générateur de l'idéal \( J=\bigcup_{\alpha\text{ fini dans }S}\sum_{s\in \alpha}sA\), alors \( \delta\in\pgcd(S)\).
	\end{enumerate}
\end{proposition}

\begin{proof}
	En plusieurs parties.
	\begin{subproof}
		\spitem[Pour \ref{ITEMooJKGMooCqWYOq}]
		%-----------------------------------------------------------
		En deux parties.
		\begin{subproof}
			\spitem[\( \Rightarrow\)]
			%-----------------------------------------------------------
			\spitem[\( \Leftarrow\)]
			%-----------------------------------------------------------
		\end{subproof}
		\spitem[Pour \ref{ITEMooZUCVooRIpnhU}]
		%-----------------------------------------------------------

	\end{subproof}
\end{proof}


\begin{proposition}[\cite{MonCerveau}]	\label{PROPooFDJXooYbXEpo}
	Soient un anneau \( A\) ainsi que \( a,b,x\in A\).
	\begin{enumerate}
		\item
		      La partie \( (a)+(b)\) est un idéal.
		\item
		      Si \( a\in (x)\), alors \( (a)\subset (x)\).
	\end{enumerate}
	%TODOooMDEUooUIpqlg. Prouver ça.
\end{proposition}

%-------------------------------------------------------
\subsection{Autres trucs sur les idéaux}
%----------------------------------------------------


\begin{propositionDef}      \label{PROPooGXMRooTcUGbi}
	Soit \( A\), un anneau, \( I\) un idéal bilatère\footnote{Définition~\ref{DefooQULAooREUIU}.} de \( A\). Nous considérons la relation d'équivalence \( x\sim y\) si et seulement si \( x-y\in I\). Sur le quotient\footnote{Définition \ref{DEFooRHPSooHKBZXl}.}
	\begin{equation}
		A/\sim=A/I,
	\end{equation}
	nous mettons les opérations
	\begin{enumerate}
		\item
		      \( [x]+[y]=[x+y]\)
		\item
		      \( [x][y]=[xy]\).
	\end{enumerate}
	Nous avons alors les résultats suivants :
	\begin{enumerate}
		\item       \label{ITEMooEJPEooRKAqmS}
		      Les opérations sont bien définies,
		\item       \label{ITEMooYBEGooTlHgNz}
		      l'ensemble \( A/I\), muni de ces opérations, est un anneau. Le neutre pour l'addition est \( [0]\), l'inverse de \( [a]\) est \( [-a]\) que nous noterons \( -[a]\).
		\item       \label{ITEMooLNRLooMkoWXZ}
		      la surjection canonique \( \pi\colon A\to A/I\) est un morphisme.
	\end{enumerate}
	Cet anneau est appelé \defe{anneau quotient}{anneau!quotient par un idéal}.
\end{propositionDef}

\begin{proof}
	En plusieurs parties.
	\begin{subproof}
		\spitem[Pour \ref{ITEMooEJPEooRKAqmS}]
		Nous savons que, par définition,
		\begin{equation}
			\bar x=\{ x+i\tq i\in I \}.
		\end{equation}
		Calculons le produit de représentants génériques de \( \bar x\) et de \( \bar y\) :
		\begin{equation}
			(x+i_1)(y+i_2)=xy+xi_2+yi_1+i_1i_2.
		\end{equation}
		Puisque \( I\) est un idéal, nous avons \( xi_2+yi_1+i_1i_2\in I\) et donc bien
		\begin{equation}
			(x+i_1)(y+i_2)\in \overline{ xy }.
		\end{equation}
		\spitem[Pour \ref{ITEMooYBEGooTlHgNz}]
		Il s'agit de vérifier les conditions de la définition \ref{DefHXJUooKoovob}.

		D'abord \( A/I\) est un groupe de neutre \( [0]\). En effet, vu que \( (A,+)\) est un groupe commutatif de neutre \( 0\), nous avons
		\begin{enumerate}
			\item Neutre : $[a]+[0]=[a+0]=[a]$.
			\item Associativité :
			      $[a]+([b]+[c])=[a]+[b+c]=[a+b+c]=[a+b]+[c]=\big( [a]+[b] \big)+[c]$.
			\item Inversibilité : l'inverse de \( [a]\) est \( [-a]\) parce que \( [a]+[-a]= [a-a]=[0] \).
		\end{enumerate}
		Nous pouvons noter \( -[a]\) l'élément \( [-a]\). Le groupe \( A/I\) est commutatif:
		\begin{equation}
			[a]+[b]=[a+b]=[b+a]=[b]+[a].
		\end{equation}
		Donc \( (A/I,+)\) est un groupe commutatif de neutre \( [0]\).

		L'associativité de \( A\) donne l'associativité dans \( A/I\) :
		\begin{equation}
			\big( [a][b] \big)[c]=[ab][c]=[abc]=[a][bc]=[a]\big( [b][c] \big).
		\end{equation}
		Et enfin pour la distributivité,
		\begin{equation}
			[a]\big( [b]+[c] \big)=[a][b+c]=[a(b+c)]=[ab+ac]=[ab]+[ac]=[a][b]+[a][c].
		\end{equation}
		Nous avons prouvé que \( A/I\) est un anneau de neutre \( [0]\) et d'unité \( [1]\).
		\spitem[Pour \ref{ITEMooLNRLooMkoWXZ}]
		Nous devons vérifier les trois conditions de la définition \ref{DEFooSPHPooCwjzuz}. Cela est immédiat parce que \( \pi(x)=[x]\).
	\end{subproof}
\end{proof}

%-------------------------------------------------------
\subsection{Diviseurs}
%----------------------------------------------------


\begin{definition}[Diviseurs dans un anneau]\label{DiviseursAnneau}
	Soient \( a, b \in A \). On dit que \( a\) divise \( b\), ou que \( a\) est un \defe{diviseur (à gauche)}{diviseur!dans un anneau} de \( b\) si il existe \( c \in A \) tel que \( ac = b \). On dit que c'est un diviseur de \( b\) à droite si \( ca = b \) pour un certain \( c \in A \).
\end{definition}
Un cas particulier est le cas des diviseurs de zéro. L'absence de tels diviseurs dans un anneau est une propriété intéressante: on dit dans ce cas que l'anneau est intègre. Définition \ref{DEFooTAOPooWDPYmd}.

Un élément \( a\in A\) est \defe{régulier à droite}{régulier à droite} si \( ba=0\) implique \( b=0\). Il est régulier à gauche si \( ab=0\) implique \( b=0\).

\begin{definition}          \label{DefrYwbct}
	Soient \( A\) un anneau commutatif et \( S\subset A\). Nous disons que \( \delta\in A\) est un \defe{PGCD}{pgcd!dans un anneau intègre} de \( S\) si
	\begin{enumerate}
		\item
		      \( \delta\) divise\footnote{Définition \ref{DiviseursAnneau}.} tous les éléments de \( S\).
		\item       \label{ITEMooVCKGooWDXZOj}
		      si \( d\) divise également tous les éléments de \( S\), alors \( d\) divise \( \delta\).
	\end{enumerate}
	Nous disons que \( \mu\in A\) est un \defe{PPCM}{ppcm!dans un anneau intègre} de \( S\) si
	\begin{enumerate}
		\item
		      \( S\divides \mu\),
		\item
		      si \( S\divides m\), alors \( \mu\divides m\).
	\end{enumerate}
	Si \( P\) et \( Q\) sont des polynômes, ce que nous notons \( \pgcd(P,Q)\) est l'unique polynôme unitaire dans \( \pgcd\big( \{ P,Q \} \big)\). Voir \ref{NORMooUJDJooWfijxT}.
\end{definition}

\begin{normaltext}
	On parle d'existence de pgcd dans \ref{}.
\end{normaltext}

\begin{remark}
	Au sens de la définition \ref{DefrYwbct}, le pgcd n'est pas unique. Dans \( \eZ\) par exemple les nombres \( 4\) et \( -4\) sont tous deux pgcd de \( \{4,16  \}\).

	Dans \( \eZ\) cependant, nous modifions implicitement la définition et nous n'acceptons que les positifs, de telle sorte que l'unique pgcd soit effectivement le plus grand pour l'ordre usuel sur \( \eZ\).

	Pour l'unicité dans \( \eZ\), voir \ref{LEMooBJVJooFyuFeN}.
\end{remark}


%---------------------------------------------------------------------------------------------------------------------------
\subsection{Élément irréductible et premier}
%---------------------------------------------------------------------------------------------------------------------------

\begin{definition}[\cite{ooWBLYooLYwALS}]       \label{DEFooZCRQooWXRalw}
	Soit un anneau commutatif \( A\). Un élément \( p\in A\) est \defe{premier}{élément premier} si il est
	\begin{enumerate}
		\item
		      non nul,
		\item
		      non inversible,
		\item       \label{ITEMooPMTTooCVHPIm}
		      si \( p\) divise un produit \( ab\), alors il divise soit \( a\) soit \( b\) (ou les deux).
	\end{enumerate}
\end{definition}

\begin{definition}[Élément irréductible\cite{ooWUNIooXKxRya}]  \label{DeirredBDhQfA}
	Un élément d'un anneau commutatif est \defe{irréductible}{irréductible!dans un anneau} si il n'est ni inversible, ni le produit de deux éléments non inversibles. \index{polynôme irréductible}
\end{definition}

\begin{normaltext}
	Nous allons voir dans la section \ref{SECooSWGKooEeOZTO} que le concept d'élément irréductible n'est vraiment utile que dans le cas des anneaux intègres.
\end{normaltext}

\begin{example}
	Un corps n'a pas d'élément irréductible parce qu'à part zéro, tous les éléments sont inversibles. Mais \( 0\) n'est pas irréductible parce qu'il peut être écrit comme produit d'éléments non inversibles : \( 0=0\cdot 0\).
\end{example}

\begin{lemma}[\cite{MonCerveau}]		\label{LEMooJBUJooScsiGc}
	Si \( p\) est irréductible et si \( u\) est inversible, alors \( pu\) est irréductible.
\end{lemma}

\begin{proof}
	D'abord \( pu\) n'est pas inversible parce que \( p\) ne l'est pas.

	Ensuite supposons que \( pu=ab\). Vu que \( u\) est inversible, nous avons \( p=a(bu^{-1})\). Comme \( p\) est irréductible, soit \( a\), soit \( bu^{-1}\) est inversible.

	Si c'est \( a\), c'est gagnée. Sinon, soit \( k\) un inverse de \( bu^{-1}\) : \( bu^{-1}k=1\). Nous voyons que \( u^{-1}k\) est un inverse de \( b\). Donc \( b\) est inversible.
\end{proof}

\begin{proposition}     \label{PROPooKDWQooTtScrN}
	Les éléments irréductibles de l'anneau \( \eZ\) sont les nombres premiers\footnote{Nombre premier, définition \ref{DEFooZCRQooWXRalw}.}.
\end{proposition}

\begin{proof}
	Les seuls inversibles de \( \eZ\) sont \( \pm 1\).

	Si \( p\) est premier et \( p=ab\) avec \( a,b\in \eZ\), alors nous avons soit \( a=\pm 1\) soit \( b=\pm 1\). Donc \( p\) n'est pas le produit de deux éléments non inversibles.

	Dans le sens inverse, supposons que \( p\) soit irréductible dans \( \eZ\). D'abord \( p\) ne peut pas être \( \pm 1\) parce que \( \pm 1\) sont inversibles. Ensuite supposons que \( p=ab\). Vu que \( p\) est irréductible, nous avons \( a=\pm1\) ou \( b=\pm1\). Autrement dit, dans \( p=ab\), soit \( a\) soit \( b\) est un inversible.
\end{proof}

%---------------------------------------------------------------------------------------------------------------------------
\subsection{Anneau intègre}
%---------------------------------------------------------------------------------------------------------------------------

\begin{definition}[Éléments nilpotents, unipotents]  \label{DEFooHRRYooTmbUTH}
	On dit que \( a \in A \) est \defe{nilpotent}{nilpotent} si il existe \( n \in \eN \) tel que \( a^n = 0 \). Il est dit \defe{unipotent}{unipotent} si \( a-1\) est nilpotent, c'est-à-dire si \( (a-1)^n =0\) pour un certain \( n \in \eN \).
\end{definition}

\begin{definition}[Éléments inversibles]        \label{DEFooCIHVooAhpJxy}
	Un élément \( a \in A \) est dit \defe{inversible}{élément!inversible!dans un anneau} si il existe \( b \in A \) tel que \( ab = 1 \).

	L'ensemble \( U(A)\)\nomenclature[A]{\( U(A)\)}{ensemble des inversibles} des éléments inversibles de \( A\) est un groupe pour la multiplication. Nous notons \( A^*=A\setminus\{ 0 \}\).
\end{definition}

Conformément à la définition \ref{DiviseursAnneau} de diviseur, nous posons la définition suivante pour les diviseurs de zéro.
\begin{definition}[diviseur de zéro\cite{ooTNKJooSCSCZQ}]		\label{DEFooCIYLooFkhVOc}
	Un élément \( a\neq 0\) est un \defe{diviseur de zéro à gauche}{diviseur!de zéro} si il existe \( x\neq 0\) tel que \( ax=0\). L'élément \( a\) est un diviseur de zéro \defe{à droite}{diviseur!de zéro à droite} si il existe \( y\neq 0\) tel que \( ya=0\).

	Nous disons que \( a\) est un \defe{diviseur de zéro}{diviseur de zéro} si il est un diviseur de zéro à gauche ou à droite.
\end{definition}

\begin{propositionDef}[Anneau intègre\cite{MonCerveau}]           \label{DEFooTAOPooWDPYmd}
	Soit \( A\) un anneau non réduit à \( \{ 0 \}\). Les assertions suivantes sont équivalentes:
	\begin{enumerate}
		\item       \label{ITEMooMXMKooXMYpkN}
		      \( A\) ne possède pas de diviseurs de zéro\footnote{Définition \ref{DEFooCIYLooFkhVOc}.}.
		\item       \label{ITEMooLAJCooFwxXrV}
		      La règle du produit nul s'applique dans \( A\): pour tous \( a, b \in A \), si \( ab=0\), alors \( a = 0\) ou \( b = 0\).
		      \index{règle du produit nul}
		\item       \label{ITEMooQNTFooSRrVPK}
		      On peut simplifier par un même élément non-nul, deux expressions produit dans \( A\) qui sont égales: pour tous \( a, b, c \in A \) avec \( a \neq 0 \), si \( ab = ac \), alors \( b = c \).
	\end{enumerate}
	Un anneau non réduit à \( \{ 0 \}\) qui vérifie ces propriétés est dit \defe{intègre}{anneau intègre}.
\end{propositionDef}

\begin{proof}
	En trois implications.
	\begin{subproof}
		\spitem[\ref{ITEMooMXMKooXMYpkN} implique \ref{ITEMooLAJCooFwxXrV}]

		Si \( ab=0\) avec \( b\neq 0\) alors \( a\) est un diviseur de zéro. Vu que nous supposons que \( A\) n'a pas de diviseurs de zéros, \( a\) est nul. De même, si \( a\neq 0\) \( b\) devrait être nul.
		\spitem[\ref{ITEMooLAJCooFwxXrV} implique \ref{ITEMooQNTFooSRrVPK}]

		Si \( ab=ac\), alors \( a(b-c)=0\) et l'hypothèse dit que soit \( a=0\), soit \( b-c=0\). Donc si \( a\neq 0\), alors \( b-c=0\).
		\spitem[\ref{ITEMooQNTFooSRrVPK} implique \ref{ITEMooMXMKooXMYpkN}]
		Si \( A=\{ 0 \}\), le point \ref{ITEMooQNTFooSRrVPK} n'est pas applicable.

		Si \( a\neq 0\) et \( ax=0\), alors nous avons aussi \( ax=a\times 0\). Par propriété de simplification, \( x=0\). Donc \( a\) n'est pas un diviseur de zéro à gauche. Nous prouvons de la même façon qu'il n'y a pas de diviseurs de zéro à droite.
	\end{subproof}
\end{proof}

\begin{lemma}		\label{LEMooIKNMooMfvQnu}
	Un corps est un anneau intègre\footnote{Définition \ref{DEFooTAOPooWDPYmd}.}.
\end{lemma}

\begin{proof}
	Nous vérifions la définition \ref{DEFooTAOPooWDPYmd}, et nous nommons \( \eK\) le corps considéré.
	\begin{subproof}
		\spitem[Pour \ref{ITEMooMXMKooXMYpkN}]
		%-----------------------------------------------------------
		Soit \( a\in \eK\). Si \( ax=0\) avec \( x\neq 0\) alors en multipliant par \( x^{-1}\) nous trouvons \( a=0\). Donc \( a\) n'est pas un diviseur de zéro non nul.

		\spitem[Pour \ref{ITEMooLAJCooFwxXrV}]
		%-----------------------------------------------------------
		Idem à ce que nous venons de faire. Si dans \( ab=0\) l'un des deux est non nul, en multipliant par son inverse, nous trouvons que l'autre est nul.

		\spitem[Pour \ref{ITEMooQNTFooSRrVPK}]
		%-----------------------------------------------------------
		Si \( ab=ac\) avec \( a\neq 0\), alors il suffit de multiplier à gauche par \( a^{-1}\) (qui existe parce que nous sommes dans un corps) pour obtenir \( b=c\).
	\end{subproof}
\end{proof}
Conséquence : dans un corps nous avons toujours la règle du produit nul, et l'élément nul n'est jamais inversible.


\begin{proposition}[\cite{BIBooNVSKooJdnbyO}]		\label{PROPooZBTIooRhAhvg}
	Soit un anneau unitaire et intègre \( A\). Soit \( p\in A\).
	\begin{enumerate}
		\item		\label{ITEMooJWRYooHndNpV}
		      \( p\) est premier si et seulement si l'idéal \( pA\) est premier.
		\item		\label{ITEMooGHGCooRkJilg}
		      \( p\) est irréductible si et seulement si il n'existe pas d'idéal principal \( I\) tel que \( pA\subsetneq I\subsetneq A\).
	\end{enumerate}
\end{proposition}

\begin{proof}
	En plusieurs parties.
	\begin{subproof}
		\spitem[\ref{ITEMooJWRYooHndNpV}\( \Rightarrow\)]
		%-----------------------------------------------------------

		Supposons que \( p\) est premier. Soient \( a,b\in A\) tels que \( ab\in pA\). En particulier \( p\divides ab\), et \( p\) étant premier\footnote{Définition \ref{DEFooZCRQooWXRalw}.}, nous avons soit \( p\divides a\) soit \( p\divides b\). Supposons que \( p\divides a\). Il existe \( k\in A\) tel que \( pk=a\), et donc
		\begin{equation}
			a=pk\in pA.
		\end{equation}
		De même si \( p\divides b\) nous avons \( b\in pA\).

		\spitem[\ref{ITEMooJWRYooHndNpV}\( \Leftarrow\)]
		%-----------------------------------------------------------

		Nous supposons que l'idéal \( pA\) est premier, et nous prouvons que \( p\) est premier. Soient \( a,b\in A\) tels que \( p\divides ab\). Il existe \( k\in A\) tel que \( pk=ab\). Donc \( ab\in pA\). L'idéal \( pA\) étant premier, nus avons soit \( a\in pA\) soit \( b\in pA\). Donc soit \( a\divides a\) soit \( b\divides p\).

		\spitem[\ref{ITEMooGHGCooRkJilg}\( \Rightarrow\)]
		%-----------------------------------------------------------

		Supposons que \( p\) est irréductible. Soit un idéal principal \( I\) vérifiant \( pA\subset I\). Nous allons montrer que soit \( I=pA\) soit \( I=A\). Vu que \( I\) est principal, il existe \( a\in A\) tel que \( I=aA\). Nous avons \( pA\subset aA\), et en particulier \( p\in aA\). Notons \( b\in A\) un élément tel que \( ab=p\).

		Vu que \( p\) est irréductible, il n'est pas le produit de deux non inversibles. Donc soit \( a\) soit \( b\) est inversible.

		Si \( a\) est inversible, alors \( I=A\).

		Si \( b\) est inversible, alors \( a=pb^{-1}\), de telle sorte que \( a\in pA\). De ce fait \( I=pA\).

		\spitem[\ref{ITEMooGHGCooRkJilg}\( \Leftarrow\)]
		%-----------------------------------------------------------

		Enfin, nous supposons qu'il n'existe pas d'idéal principal \( I\) tel que \( pA\subsetneq I\subsetneq A\), et nous montrons que \( p\) est irréductible. Soient \( a,b\in A\) tels que \( p=ab\).

		Nous avons \( p\in aA\) et donc \( pA\subset aA\). Donc soit \( aA=pA\) soit \( aA=A \).

		Si \( aA=pA\), alors il existe \( k\in A\) tel que \( pk=a\). En multipliant l'égalité \( ab=p\) par \( k \) nous trouvons \( abk=pk=a\). Vu que \( A\) est intègre, nous pouvons simplifier par \( a\) et trouver \( bk=1\), de telle sorte que \( b\) soit inversible.


		Si \( aA=A\), alors \( a\) est inversible parce que \( 1\in A=aA\).
	\end{subproof}
\end{proof}

\begin{lemma}[\cite{MonCerveau}]		\label{LEMooSFHMooQoKsPV}
	Soient un anneau intègre \( A\), et une partie \( S\subset A\). Si un des \( \pgcd\) de \( S\) est inversible\footnote{Définition \ref{DEFooCIHVooAhpJxy}.}, alors ils le sont tous.
\end{lemma}

\begin{proof}
	Pour rappel, les pgcd d'une partie de \(A\) sont définis dans \ref{DefrYwbct}. Soit un \( \pgcd\) inversible de \( S\), ainsi qu'un autre \( \pgcd\) que nous nommons \( \delta'\). Vu que \( \delta'\) divise tous les éléments de \( S\), il est divisé par \( \delta\) : \( \delta\divides\delta'\). Réciproquement, \( \delta'\divides \delta\).

	Soient \( x\) et \( y\) définis par \( \delta=x\delta'\) et \( \delta'=y\delta\). Nous avons
	\begin{equation}
		\delta=x\delta'=xy\delta.
	\end{equation}
	Comme l'anneau \( A\) est intègre, nous pouvons simplifier par \( \delta\) et voir \( xy=1\), ce qui signifie que \( x\) et \( y\) sont inversibles. Donc si \( \delta\) est inversible, alors \( \delta'=y\delta\) est inversible.
\end{proof}

\begin{lemma}[\cite{MonCerveau,BIBooZXSRooOqGHBA}]		\label{LEMooMHZQooIcSNSf}
	Soient un anneau intègre, \( A\), une partie \( S\subset A\) et un élément \( a\in A\). Nous avons\footnote{Définition du pgcd: \ref{DefrYwbct}.}
	\begin{equation}
		\pgcd(aS)=a\pgcd(S).
	\end{equation}
\end{lemma}

\begin{proof}
	Deux inclusions à prouver.
	\begin{subproof}
		\spitem[\( \pgcd(aS)\subset a\pgcd(S)\)]
		%-----------------------------------------------------------
		Soit un pgcd \( \delta\) de \( aS\). Nous devons trouver un \( \delta'\in\pgcd(S)\) tel que \( \delta=a\delta'\). En termes de notations, nous notons \( S=\{ s_i \}_{i\in I}\). Pour chaque \( i\) nous avons \( \delta\divides as_i\) : il existe \( x_i\in A\) tel que
		\begin{equation}
			\delta x_i=as_i.
		\end{equation}
		Vu que \( a\) divise tous les éléments de \( aS\), il divise n'importe quel pgcd de \( aS\), et en particulier \( a\divides \delta\) : il existe \( \delta'\in A\) tel que \( \delta=a\delta'\). Nous montrons que \( \delta'\in\pgcd(S)\).

		Nous savons que \( \delta x_i=as_i\). En remplaçant \( \delta\) par \( a\delta'\), \( a\delta'x_i=as_i\). Vu que nous sommes dans un anneau intègre, nous pouvons simplifier par \( a\) (définition \ref{DEFooTAOPooWDPYmd}\ref{ITEMooQNTFooSRrVPK}) :
		\begin{equation}
			\delta'x_i=s_i.
		\end{equation}
		Donc \( \delta'\) divise tous les éléments de \( S\), et vérifie la première condition pour être un pgcd de \( S\). Pour la seconde condition, nous supposons que \( d\) divise tous les éléments de \( S\). Nous avons \( d\divides S\), donc \( ad\divides aS\). Et comme \( \delta\) est un pgcd de \( aS\), nous déduisons que \( ad\) divise \( \delta\). Il existe \( y\in A\) tel que
		\begin{equation}
			ady=\delta.
		\end{equation}
		Nous remplaçons \( \delta\) par sa valeur \( a\delta'\) : \( ady=a\delta'\). Encore une fois nous simplifions par \( a\) et nous trouvons \( dy=\delta'\), c'est-à-dire que \( \delta\) divise \( \delta'\).

		\spitem[\(a \pgcd(S)\subset \pgcd(aS)\)]
		%-----------------------------------------------------------

		Soit un pgcd \( \delta\) de \( S\). Nous voulons que \( a\delta\in \pgcd(aS)\). Vu que \( \delta\in\pgcd(S)\) nous avons \( \delta x_i=s_i\) pour tout \( i\), et donc aussi \( a\delta x_i=as_i\), de telle sorte que \( a\delta\) divise tous les éléments de \( aS\).

		Soit maintenant \( d\in A\) divisant tous les éléments de \( aS\). Nous devons prouver que \( d\divides a\delta\).

		\begin{subproof}
			\spitem[Travail préliminaire]
			%-----------------------------------------------------------

			Nous considérons \( \delta'\in \pgcd(aS)\). Vu que \( \delta\divides s\) pour tout \( s\in S\), nous avons aussi \( a\delta\divides as\) pour tout \( s\in S\). Comme \( \delta'\) est un est un pgcd de \( aS\), nous avons donc
			\begin{equation}
				a\delta\divides\delta'.
			\end{equation}
			Soit \( u\in A\) tel que \( \delta'=a\delta u\).

			En utilisant la première partie de la preuve, nous avons
			\begin{equation}
				\delta'\in\pgcd(aS)\subset a\pgcd(S).
			\end{equation}
			Donc il existe \( \delta_1\in\pgcd(S)\) tel que \( \delta'=a\delta_1\). En écrivant l'égalité \( \delta' =a\delta u\) avec cette valeur de \( \delta'\), nous trouvons
			\begin{equation}		\label{EQooVHSSooDdVUeW}
				a\delta_1=a\delta u,
			\end{equation}
			et donc \( \delta_1=\delta u\) parce que \( A\) est intègre. Vu que \( \delta_1\in\pgcd(S)\), nous avons aussi \( \delta u\in\pgcd(S)\). En particulier \( \delta u\) divise tous les éléments de \( S\), et donc divise \( \delta\) qui est un pgcd de \( S\) : \( \delta u\divides \delta\). En multipliant par \( a\),
			\begin{equation}
				\delta'=a\delta u\divides a \delta.
			\end{equation}

			\spitem[Résumé]
			%-----------------------------------------------------------
			Nous avons considéré \( \delta\in\pgcd(S)\) et nous sommes en train de prouver que \( a\delta\in\pgcd(aS)\). Nous avons déjà prouvé que si \( \delta'\in \pgcd(aS)\), alors nous avons \( \delta'\divides a\delta\).

			Nous posons \( y\in A\) tel que \( \delta' y=a\delta\).

			\spitem[Et enfin]
			%-----------------------------------------------------------
			Soit \( d\) divisant tous les éléments de \( a\delta\). Donc \( d\divides \delta'\) : il existe \( x\in A\) tel que \( dx=\delta'\). En multipliant par \( y\),
			\begin{equation}
				dxy=\delta' y=a\delta.
			\end{equation}
			Nous avons montré que \( d\) divise \( a\delta\), ce qu'il nous fallait.
		\end{subproof}
	\end{subproof}
\end{proof}

\begin{lemma}[\cite{BIBooSJZQooOytVhm}]		\label{LEMooZSUNooUmYmgt}
	Soient un anneau intègre \( A\) et une partie \( S\subset A\). Si \( \delta\in\pgcd(S)\), alors \( \pgcd(S/\delta)\) ne contient que des inversibles.
\end{lemma}

\begin{proof}
	En utilisant le lemme \ref{LEMooMHZQooIcSNSf}, nous avons
	\begin{equation}
		\pgcd(S)=\pgcd\big( \delta(S/\delta) \big)=\delta\pgcd(S/\delta).
	\end{equation}
	Soit \( d\in\pgcd(S/\delta)\). Notre but est de montrer que \( d\) est inversible. D'abord
	\begin{equation}
		\delta d\in \delta\pgcd(S/\delta)=\pgcd(S).
	\end{equation}
	Donc \( \delta d\) divise tous les éléments de \( S\). Étant donné que \( \delta\) est un pgcd de \( S\), la définition \ref{DefrYwbct}\ref{ITEMooVCKGooWDXZOj} nous dit que \( \delta d\) divise \( \delta\). Il existe donc \( d'\in A\) tel que \( \delta dd'=\delta\), c'est à dire
	\begin{equation}
		\delta(dd'-1)=0.
	\end{equation}
	Étant donné que \( \delta\neq 0\) et que \( A\) est intègre nous avons \( dd'-1=0\) (définition \ref{DEFooTAOPooWDPYmd}\ref{ITEMooLAJCooFwxXrV}), c'est-à-dire \( dd'=1\). Cela montre que \( d'\) est un inverse de \( d\), et donc que \( d\) est inversible.
\end{proof}

\begin{lemma}[\cite{MonCerveau}]		\label{LEMooZKASooKstTuK}
	Soit un anneau intègre \( A\). Si \( \delta\in\pgcd(S)\) et si \( u\in A\) est inversible, alors \( \delta u\in\pgcd(S)\).
\end{lemma}

\begin{proof}
	Soit \( s\in S\). Si \( \delta x=s\), alors \( u\delta (u^{-1} x)=s\). Donc \( u\delta\) divise tous les éléments de \( S\). De plus si \( d\divides S\), alors \( d\divides \delta\). Dans ce cas il existe \( y\) tel que \( dy=\delta\). Nous avons alors aussi
	\begin{equation}
		dxu=\delta u,
	\end{equation}
	de telle sorte que \( d\) divise \( \delta u\).
\end{proof}

%-------------------------------------------------------
\subsection{Positivité}
%----------------------------------------------------

\begin{definition}[\cite{MonCerveau}]	\label{DEFooZRMFooCtzMov}
	Soit un anneau \( A\). Une positivité sur \( A\) est une partie \( A^+\subset A\) telle que
	\begin{enumerate}
		\item
		      Pour tout \( a\neq 0\) dans \( A\), nous avons\footnote{\emph{xor} est le ou exclusif. L'un ou l'autre est vrai, mais les deux en même temps.} \( a\in A^+ \text{ xor } -a\in A^+\).
		\item
		      Pour tout \( a,b\in A^+\), nous avons \( a+b\in A^+\).
	\end{enumerate}
	Oui, il faudrait dire une \emph{stricte} positivité.
\end{definition}

\begin{proposition}[\cite{MonCerveau}]	\label{PROPooKLOPooBgQqhM}
	Si \( A^+\) est une positivité sur \( A\), alors en posant \( a\leq b\) si et seulement si \( b-a\in A^+\cup\{ 0 \}\), nous avons une relation d'ordre\footnote{Définition \ref{DefooFLYOooRaGYRk}.}.
\end{proposition}

\begin{proof}
	Nous vérifions les conditions de la définition \ref{DefooFLYOooRaGYRk}.
	\begin{subproof}
		\spitem[Réflexivité]
		%-----------------------------------------------------------
		Vu que \( a-a=0\) et que \( 0\in A^\cup\{ 0 \}\), nous avons \( a\leq a\).
		\spitem[Antisymétrie]
		%-----------------------------------------------------------
		Si \( a\leq b\) et \( b\leq a\), nous avons
		\begin{subequations}
			\begin{numcases}{}
				b-a\in A^+\cup\{ 0 \}\\
				a-b\in A^+\cup\{ 0 \}.
			\end{numcases}
		\end{subequations}
		Entre \( a-b\) et \( b-a\), au maximum un des deux peut être dans \( A^+\). Donc soit \( a-b=0\), soit \( b-a=0\) (ou le deux en même temps, ce qui est le cas).

		\spitem[Transitive]
		%-----------------------------------------------------------
		Supposons que \( a\leq b\) et \( b\leq c\). Nous avons alors
		\begin{subequations}
			\begin{numcases}{}
				b-a\in A^+\cup\{ 0 \}\\
				c-b\in A^+\cup\{ 0 \}.
			\end{numcases}
		\end{subequations}
		Si l'un des deux est nul, c'est facile. Nous supposons qu'ils sont tous les deux dans \( A^+\). Alors leur somme est dans \( A^+\) : \( (b-a)+(c-b)\in A^+\). Cela donne tout de suite \( c-a\in A^+\) et donc \( a\leq c\) comme demandé.
	\end{subproof}
\end{proof}

%--------------------------------------------------------------------------------------------------------------------------- 
\subsection{Fonction puissance}
%---------------------------------------------------------------------------------------------------------------------------

Voici une première définition de la fonction puissance. Il y en aura d'autres, de plus en plus générales. Voir le thème \ref{THEMEooBSBLooWcaQnR}.
\begin{definition}\label{DEFooGVSFooFVLtNo}
	Si \( A\) est un anneau, si \( a\in A\) et si \( n\in \eN\), nous définissons \( a^n\) par récurrence :
	\begin{enumerate}
		\item
		      \( a^0=1\) (l'unité pour la multiplication dans \( A\)),
		\item       \label{ITEMooOUIPooGjAgQb}
		      \( a^{k+1}=a\cdot a^{k}\).
	\end{enumerate}
	Si vous n'êtes pas \randomGender{sûr}{sure} de vous, ne citez pas le théorème \ref{THOooEJPYooZFVnez}. Il est indispensable pour faire fonctionner cette définition, mais vous pouvez faire comme si vous n'avez rien vu.
\end{definition}

Le lemme suivant dit que le point \ref{ITEMooOUIPooGjAgQb} de la définition \ref{DEFooGVSFooFVLtNo} aurait pu être écrit \( a^k\cdot a\) au lieu de \( a\cdot a^k\).
\begin{lemma}[\cite{MonCerveau}]        \label{LEMooWPARooYLZlzr}
	Si \( A\) est un anneau, si \( a\in A\) et si \( n\in \eN\), alors
	\begin{equation}
		a^n=a\cdot a^{n-1}=a^{n-1}\cdot a.
	\end{equation}
\end{lemma}

\begin{proof}
	Cela se prouve par récurrence. Pour \( n=1\) c'est l'égalité \( a=a^0a\) qui est correcte parce que par définition \( a^0=1\).

	Supposons que le résultat soit bon pour \( n\) et voyons ce que ça donne pour \( n+1\) :
	\begin{subequations}
		\begin{align}
			a^{n+1} & =aa^n        & \text{Définition de } a^{n+1}            \\
			        & =a(a^{n-1}a) & \text{hypothèse de récurrence pour } a^n \\
			        & =(aa^{n-1})a & \text{associativité}                     \\
			        & =a^na        & \text{Définition de } a^n.
		\end{align}
	\end{subequations}
\end{proof}

\begin{proposition}[\cite{MonCerveau}]	\label{PROPooPLLSooUpiLKa}
	Soit un naturel \( b\in\eN\setminus\{ 0 \}\). Soit un naturel \( m>0\). Il existe \( n\in \eN\) tel que \( b^n>m\).
	%TODOooDFWXooMCkcsJ le démontrer
\end{proposition}
