% This is part of Mes notes de mathématique
% Copyright (c) 2011-2025
%   Laurent Claessens
% See the file fdl-1.3.txt for copying conditions.

%+++++++++++++++++++++++++++++++++++++++++++++++++++++++++++++++++++++++++++++++++++++++++++++++++++++++++++++++++++++++++++ 
\section{Groupes}
%+++++++++++++++++++++++++++++++++++++++++++++++++++++++++++++++++++++++++++++++++++++++++++++++++++++++++++++++++++++++++++

%---------------------------------------------------------------------------------------------------------------------------
\subsection{Définition, unicité du neutre}
%---------------------------------------------------------------------------------------------------------------------------

La définition d'un groupe est la définition \ref{DEFooBMUZooLAfbeM}.

\begin{lemmaDef}[Unicités]  \label{LEMooECDMooCkWxXf}
	Dans un groupe, l'inverse et le neutre sont uniques. Plus précisément, si \( G\) est un groupe nous avons :
	\begin{enumerate}
		\item
		      il existe un unique élément \( e\in G\) tel que \( e g=g e=g\) pour tout \( g\in G\),
		\item       \label{ITEMooOIWTooYqmMPP}
		      pour tout \( g\in G\), il existe un unique élément \( h\in  G\) tel que \(g h=h g=e \).
	\end{enumerate}
	Le \( e\) ainsi défini est nommé \defe{neutre}{neutre!dans un groupe} de \( G\). Le \( h\) tel que \( g h=h g=e\) est nommé l'\defe{inverse}{inverse!dans un groupe} de \( g\) et est noté \( g^{-1}\).
\end{lemmaDef}

\begin{proof}
	Chaque point séparément.
	\begin{enumerate}
		\item
		      Supposons que \( e_1\) et \( e_2\) vérifient la propriété. Nous avons pour tout \( g\in G\) : \( e_1g=ge_1=g\). En particulier pour \( g=e_2\) nous écrivons \( e_1e_2=e_2e_1=e_2\). Mais en partant dans l'autre sens : \( e_2g=ge_2=g\) avec \( g=e_1\) nous avons \( e_2e_1=e_1e_2=e_1\). En égalant ces deux valeurs de \( e_2e_1\) nous avons \( e_1=e_2\).

		      Pour la suite de la preuve nous écrivons \( e\) l'unique neutre de \( G\).

		\item
		      Supposons que \( k_1\) et \( k_2\) soient deux inverses de \( g\). On considère alors le produit \( k_1 g k_2 \). Puisque \(k_1 g = e \), on a \( k_1 g k_2 = e k_2 = k_2 \); mais, comme \(g k_2 = e \), on a aussi \( k_1 g k_2 = k_1 e = k_1 \). Le produit est donc à la fois égal à \( k_1 \) et à \( k_2 \), et donc \( k_1 = k_2 \).
	\end{enumerate}
\end{proof}

\begin{lemma}[\cite{BIBooFZWZooQQUYRW}]       \label{LEMooWYLRooNOdZnp}
	Soient deux groupes \( G\) et \( H\). Si \( \alpha\colon G\to H\) est un morphisme de groupes\footnote{Définition \ref{DEFooBEHTooMeCOTX}.}, alors
	\begin{enumerate}
		\item
		      \( \alpha(e_G)=e_H\).
		\item
		      \( \alpha(g^{-1})=\alpha(g)^{-1}\).
	\end{enumerate}
\end{lemma}

\begin{proof}
	Pour le premier point, soit \( g\in G\). Nous avons
	\begin{equation}
		\alpha(g)=\alpha(ge)=\alpha(g)\alpha(e).
	\end{equation}
	En multipliant les deux côtés par \( \alpha(g)^{-1}\) nous trouvons \( e=\alpha(e)\).

	Pour le second point, nous vérifions que \( \alpha(g^{-1})\alpha(g)=e\). C'est le cas parce que
	\begin{equation}
		\alpha(g^{-1})\alpha(g)=\alpha(g^{-1}g)=\alpha(e)=e.
	\end{equation}
\end{proof}

\begin{lemma}       \label{LEMooBIBFooBHxFYC}
	Si \( G\) est un groupe et si \( h\in G\), alors les applications
	\begin{equation}
		\begin{aligned}
			L_h\colon G & \to G      \\
			g           & \mapsto hg
		\end{aligned}
	\end{equation}
	et
	\begin{equation}
		\begin{aligned}
			R_h\colon G & \to G      \\
			g           & \mapsto gh
		\end{aligned}
	\end{equation}
	sont des bijections.
\end{lemma}

\begin{proof}
	D'abord si \( L_h(g_1)=L_h(g_2)\), alors \( hg_1=hg_2\) et en multipliant à gauche par \( h^{-1}\) nous avons \( g_1=g_2\); donc \( L_h\) est injective. Ensuite \( L_h\) est surjective parce que si \( g\in G\), alors \( g=L_h(h^{-1} g)\).

	Pour l'application \( R_h\), la preuve est une simple adaptation.
\end{proof}


%-------------------------------------------------------
\subsection{Ordre d'un groupe et d'un élément}
%----------------------------------------------------


L'ordre d'un groupe et l'ordre d'un élément d'un groupe sont deux choses différentes.

\begin{definition}[Ordre d'un groupe]    \label{DEFooKWBCooMlmpCP}
	Soit un groupe \( G\).
	\begin{enumerate}
		\item
		      Si \( G\) est un ensemble fini, l'\defe{ordre}{ordre d'un groupe} de \( G\) est son cardinal\footnote{Définition \ref{PROPooJLGKooDCcnWi}.}, et nous le notons \( | G |\).
		\item
		      Si l'ensemble \( G\) est infini, nous disons que \( | G |=\infty\) et qu'il est d'ordre infini.
	\end{enumerate}
	Oui : nous pourrions simplement toujours dire «cardinalité» et écrire \( \Card(G)\). Au lieu de ça, dans le cas particulier des groupes, il y a une tradition de dire «ordre» et d'écrire \( | G |\).
\end{definition}

\begin{definition}[Ordre d'un élément]      \label{DEFooKSTVooOObpgC}
	L'\defe{ordre}{ordre!d'un élément} d'un élément \( g\) de \( G\) est le naturel
	\begin{equation}
		\min\{ n\in\eN\setminus\{ 0 \}\tq g^n=e \},
	\end{equation}
	si il existe; dans le cas contraire, nous disons que l'ordre de \( g\) est infini.
\end{definition}

\begin{normaltext}
	Nous verrons que le corolaire~\ref{CorpZItFX} au théorème de Lagrange dira que l'ordre d'un élément divise l'ordre du groupe.
\end{normaltext}


%--------------------------------------------------------------------------------------------------------------------------- 
\subsection{Groupe ordonné}
%---------------------------------------------------------------------------------------------------------------------------

\begin{definition}[\cite{BIBooZIFGooBrNwWU}]        \label{DEFooEUHFooYvhnLQ}
	Soient un groupe \( (G,+)\), ainsi qu'une relation d'ordre \( \leq\) sur \( G\). Nous disons que la relation d'ordre est \defe{compatible}{} avec la structure de groupe si pour tout \( x,y,z\in G\), si \( x\leq y\) alors \( x+z\leq y+z\) et \( z+x\leq z+y\). Dans ce cas, le triple \( (G,+,\leq)\) est un \defe{groupe ordonné}{groupe ordonné}.

	Si \( (G,\leq)\) est totalement ordonné, nous disons que le groupe est totalement ordonné.
\end{definition}

%---------------------------------------------------------------------------------------------------------------------------
\subsection{Classes de conjugaison}
%---------------------------------------------------------------------------------------------------------------------------

\begin{definition}[classe de conjugaison]       \label{DEFooOLXPooWelsZV}
	Soit un groupe \( G\) et un élément \( g\in G\). La \defe{classe de conjugaison}{classe!de conjugaison} de \( g\) est la partie
	\begin{equation}
		C_g=\{ kgk^{-1}\tq k\in G \}.
	\end{equation}
\end{definition}

\begin{lemma}       \label{LEMooQYBJooYwMwGM}
	Un groupe est commutatif si et seulement si ses classes de conjugaison sont des singletons.
\end{lemma}

\begin{proof}
	Supposons que \( G\) soit commutatif. Alors
	\begin{equation}
		C_g=\{ kgk^{-1}\tq k\in G \}=\{ g \}.
	\end{equation}
	Donc les classes de conjugaison sont des singletons.

	Dans l'autre sens, si les classes sont des singletons, on a \( kgk^{-1}=g\) pour tous \( k,g\in G\). Cela signifie immédiatement que \( G\) est commutatif.
\end{proof}

\begin{definition}[centralisateur\cite{Kropholler}]         \label{defGroupeCentre}
	Soient un groupe \( G\), un sous-groupe \( H\) et un élément \( h\in H\). Le \defe{centralisateur}{centralisateur} de \( h\) dans \( G\) est l'ensemble des éléments de \( G\) qui commutent avec \( h\) :
	\begin{equation}
		Z_G(h)=\{z\in G\tq hz=zh\}.
	\end{equation}
	Le centralisateur de \( H\) dans \( G\) est l'ensemble des éléments de \( G\) qui commutent avec tous les éléments de \( H\) :
	\begin{equation}
		Z_G(H)=\bigcap_{h\in H}Z_G(h).
	\end{equation}
	Le \defe{centre}{centre!d'un groupe} d'un groupe \( G\) est l'ensemble des éléments de \( G\) qui commutent avec tous les autres:
	\begin{equation}
		Z_G=Z_G(G)=\{ z\in G\tq gz=zg , \forall g\in G \}.
	\end{equation}
\end{definition}

\begin{definition}[normalisateur\cite{Kropholler}]          \label{DEFooZTSMooBislIy}
	Soient un groupe \( G\) et un sous-groupe \( H\). Le \defe{normalisateur}{normalisateur} de \( H\) dans \( G\) est
	\begin{equation}
		\mN_G(H)=\{ g\in G\tq gH=Hg \}.
	\end{equation}
\end{definition}

\begin{definition}[Sous-groupe normal]                      \label{DEFooNIIMooFkZgvX}
	Un sous-groupe \( N\) de \( G\) est \defe{normal}{normal!sous-groupe} ou \defe{distingué}{sous-groupe distingué}\index{distingué!sous-groupe} si pour tout \( g\in G\) et pour tout \( n\in N\), \( gng^{-1}\in N\). Autrement dit lorsque \( gNg^{-1}\subset N\).

	Lorsque \( N\) est normal dans \( G\) il est parfois noté \( N\normal G\)\nomenclature[R]{\(N \normal G\)}{Le sous-groupe \( N\) est normal dans \( G\)}.
\end{definition}

\begin{definition}      \label{DEFooUXXTooCCLmQe}
	Un sous-groupe \( H\) de \( G\) est un sous-groupe \defe{caractéristique}{sous-groupe!caractéristique}\index{caractéristique!sous-groupe} si \( \alpha(H)\subset H\) pour tout automorphisme\footnote{Automorphisme de groupe, définition \ref{DEFooBEHTooMeCOTX}.} \( \alpha\) de \( G\).
\end{definition}

\begin{lemma}[\cite{BIBooOHKHooAcewiw}]
	Si \( H\) est un sous-groupe caractéristique de \( G\), alors \( \alpha(H)=H\) pour tout automorphisme \( \alpha\) de \( G\).
\end{lemma}

\begin{proof}
	Si \( \alpha\) est un automorphisme de \( G\), alors \( \alpha^{-1}\) est encore un automorphisme de \( G\). En particulier \( \alpha^{-1}(H)\subset H\).

	Soit \( h\in H\). Nous devons prouver que \( h\in \alpha(H)\). Pour cela :
	\begin{equation}
		h=\alpha\big( \alpha^{-1}(h) \big)\in \alpha\big( \alpha^{-1}(H) \big)\subset\alpha(H).
	\end{equation}
\end{proof}

\begin{definition}[Groupe simple]                 \label{DefGroupeSimple}
	Un groupe est dit \defe{simple}{groupe simple}\index{groupe!simple} si il est non trivial et si les seuls sous-groupes normaux qu'il admet sont lui-même et le sous-groupe réduit à l'élément neutre.
\end{definition}


%--------------------------------------------------------------------------------------------------------------------------- 
\subsection{Permutations, groupe symétrique}
%---------------------------------------------------------------------------------------------------------------------------

Nous donnons ici quelques éléments à propos du groupe symétrique. Beaucoup de choses supplémentaires sont reportées à la section \ref{SECooZFYQooFfopMa}. Voir aussi le thème \ref{THEMEooQEEWooXDhvhv}.


\begin{definition}      \label{DEFooJNPIooMuzIXd}
	Soit un ensemble \( E\). Une \defe{permutation}{permutation} de l'ensemble \( E\) est une bijection \( E\to E\). Le \defe{groupe symétrique}{groupe!symétrique} de \( E\) est le groupe des bijections \( E\to E\); il est noté \( S_E\).

	Le \defe{groupe symétrique}{groupe!symétrique} \( S_n\)\nomenclature[R]{\( S_n\)}{le groupe symétrique} est le groupe des permutations de l'ensemble \( \{ 1,\ldots,n \}\). C'est donc l'ensemble des bijections \( \{ 1,\ldots, n \}\to\{ 1,\ldots, n \}\).
\end{definition}

\begin{definition}      \label{DEFooSupportPermutation}
	Le \defe{support}{support!d'une permutation} d'une permutation \( \sigma\) est l'ensemble constitué des éléments modifiés par \( \sigma\):
	\begin{equation*}
		\supp \sigma = \{ i \in \{1,\ldots,n \} \tq \sigma(i) \neq i\}.
	\end{equation*}
\end{definition}

\begin{definition}[\cite{PDFpersoWanadoo}]      \label{DEFooMVFKooMpXMQy}
	Soient une permutation \( \sigma\in E\) ainsi que \( a\in E\). La \( \sigma\)-\defe{orbite}{orbite d'un élément sous une permutation} de \( a\) est l'ensemble
	\begin{equation}
		\Omega_{\sigma}(a)=\{ \sigma^i(a) \}_{i\in \eN}.
	\end{equation}
\end{definition}

\begin{lemma}[\cite{ooJBHPooToyYYI}]        \label{LEMooSGWKooKFIDyT}
	Le groupe symétrique \( S_n\) est un ensemble fini contenant \( n!\) éléments :
	\begin{equation}
		\Card(S_n)=n!.
	\end{equation}
\end{lemma}

\begin{lemma}[\cite{ooJFLYooKMbycW}]        \label{LEMooUPBOooWbwMTx}
	Deux résultats.
	\begin{enumerate}
		\item
		      Tout groupe est isomorphe à un sous-groupe d'un groupe symétrique.
		\item
		      Tout groupe fini d'ordre \( n\) est isomorphe à un sous-groupe de \( S_n\).
	\end{enumerate}
\end{lemma}

\begin{proof}
	Soit, pour \( g\in G\) donné, l'application
	\begin{equation}
		\begin{aligned}
			\tau_g\colon G & \to G       \\
			x              & \mapsto gx.
		\end{aligned}
	\end{equation}
	Commençons par prouver que cela est une bijection.  D'une part, \( \tau_g(x)=y\) pour \( x=g^{-1} y\) (surjection) et, d'autre part, \( \tau_g(x)=\tau_g(y)\) implique \( gx=gy\) et donc \( x=y\) (injection).

	Nous avons donc \( \tau_g\in S_G\). L'application
	\begin{equation}
		\begin{aligned}
			\varphi\colon G & \to S_G        \\
			g               & \mapsto \tau_g
		\end{aligned}
	\end{equation}
	est un morphisme de groupe\footnote{Définition \ref{DEFooBEHTooMeCOTX}.} :
	\begin{equation}
		\varphi(gh)x=ghx=\varphi(g)(hx)=\varphi(g)\Big( \varphi(h)x \Big)=\Big( \varphi(g)\circ \varphi(h) \Big)x.
	\end{equation}
	\randomGender{Il est injectif}{Elle est injective} parce que si \( \tau_g=\tau_h\) alors \( gx=hx\) pour tout \( x\). En particulier \( g=h\). Donc \( \varphi\colon G\to \Image(\varphi)\) est un isomorphisme entre \( G\) et un sous-groupe de \( S_G\) (proposition \ref{PROPooJHHPooIpciPA}).

	Un groupe fini de cardinal \( n\) est isomorphe à un sous-groupe de \( S_G\); or \( S_G\) est isomorphe à un des \( S_n\).
\end{proof}


%--------------------------------------------------------------------------------------------------------------------------- 
\subsection{Décomposition en cycles}
%---------------------------------------------------------------------------------------------------------------------------

\begin{definition}[cycle\cite{PDFpersoWanadoo}]		\label{DEFooRWYSooYmKwph}
	Soit \( E\) un ensemble de cardinal\footnote{Définition \ref{PROPooJLGKooDCcnWi}.} \( n\). Soit un entier \( 1\leq k\leq n\). Un élément \( \sigma\in S_E\) est un \( k\)-\defe{cycle}{cycle} si il ne possède qu'une seule orbite\footnote{Définition \ref{DEFooMVFKooMpXMQy}.} non réduite à un élément et qu'elle est de cardinal \( k\).
\end{definition}

\begin{lemma}[\cite{MonCerveau}]        \label{LEMooADNGooDZpdTb}
	Soient un \( k\)-cycle \( \sigma\) et \( a\in \Omega\). Alors
	\begin{equation}
		\Omega_{\sigma}(a)=\{ a,\sigma(a),\ldots, \sigma^{k-1}(a) \}
	\end{equation}
	et \( \sigma^k(a)=a\).

	En particulier, les éléments \( \sigma^q(a)\) avec \( q=0,\ldots, k-1\) sont tous distincts.
\end{lemma}

\begin{proof}
	Soit \( l\) le plus grand entier tel que les \( \sigma^i(a)\) avec \( 0\leq i\leq l\) soient tous distincts, et notons \( A=\{ \sigma^i(a) \}_{i=0,\ldots, l}\). Cet ensemble satisfait
	\begin{itemize}
		\item \( \Card(A)=l+1\)
		\item \( A\subset \Omega_{\sigma}(a)\), et donc \( \Card(A)\leq \Card(\Omega_{\sigma}(a))=k\) par le lemme \ref{LEMooVFPNooVmdUXY}\ref{ITEMooYJSZooXQXkOX}.
	\end{itemize}
	Que vaut \( \sigma^{l+1}(a)\) ? Par maximalité de \( l\), \( \sigma^{l+1}(a)\) est un des \( \sigma^i(a)\) avec \( i\leq l\). Par injectivité de \( \sigma\), nous avons donc forcément \( \sigma^{l+1}(a)=a\).

	Donc pour tout \( i>l\) il existe \( j\leq l\) tel que \( \sigma^i(a)=\sigma^j(a)\) (parce que \( \sigma^i(a)=\sigma^{i-l-1}(a)\)). Nous en déduisons que
	\begin{equation}
		\Omega_{\sigma}(a)=\{ \sigma^i(a) \}_{0\leq i\leq l}=A.
	\end{equation}
	Le cardinal de \( \Omega_{\sigma}(a)\) étant \( k\) par hypothèse nous avons \( k=l+1\), et donc \( l=k-1\).
\end{proof}

\begin{lemma}      \label{LEMooANVHooOQiTwY}
	Si \( \sigma\) est une cycle de longueur \( k\), et si \( b\in \Omega_{\sigma}(a)\), alors
	\begin{equation}
		\Omega_{\sigma}(a)=\Omega_{\sigma}(b)=\{ \sigma^i(b) \}_{i=0,\ldots, k-1}.
	\end{equation}
\end{lemma}

\begin{proof}
	Comme \( b\in \Omega_{\sigma}(a)\), il existe \( l\leq k-1\) tel que \( b=\sigma^l(a)\). Pour tout \( i\) nous avons \( \sigma^i(a)=\sigma^{k-l+1}(b)\), et donc \( \Omega_{\sigma}(a)\subset\Omega_{\sigma}(b)\).

	Mais pour tout \( i\) nous avons aussi \( \sigma^i(b)=\sigma^{l+1}(a)\) et donc \( \Omega_{\sigma}(b)\subset\Omega_{\sigma}(a)\).

	Nous avons donc montré que \( \Omega_{\sigma}(a)=\Omega_{\sigma}(b)\). La seconde égalité est le lemme \ref{LEMooADNGooDZpdTb} appliqué à \( b\).
\end{proof}

\begin{lemma}[\cite{MonCerveau}]       \label{LEMooMIHGooQfALbc}
	Soient un ensemble fini \( E\), une permutation \( \sigma\in S_E\) ainsi que \( a\in E\). Si \( b\in \Omega_{\sigma}(a)\), alors \( \Omega_{\sigma}(b)=\Omega_{\sigma}(a)\).
\end{lemma}


\begin{lemma}[\cite{PDFpersoWanadoo}]		\label{LEMooGWZNooXFezrg}
	Tout \( k\)-cycle est d'ordre\footnote{Définition \ref{DEFooKSTVooOObpgC}.} \( k\).
\end{lemma}

\begin{proof}
	Soit le cycle \( \{ a,\sigma(a),\ldots, \sigma^{k-1}(a) \}\). Tous les \( \sigma^i(a)\) avec \( i\leq k-1\) sont distincts et \( \sigma^k(a)=a\). Donc \( \sigma^k\) est l'identité, et l'ordre de \( \sigma\) est plus petit ou égal à \( k\).

	Si \( i\leq k-1\), alors \( \sigma^i(a)\neq a\) parce que les éléments du cycle sont distincts. Donc \( \sigma^i\neq \id\) pour \( i\leq k-1\). Nous en déduisons que l'ordre de \( \sigma\) est \( k\).
\end{proof}


\begin{lemma}[\cite{PDFpersoWanadoo,BIBooONYNooQolWiC}]       \label{LEMooQLSAooBrXDXw}
	Tout élément du groupe symétrique \( S_n\) peut être décomposé en un nombre fini de cycles de supports disjoints.

	Cette décomposition est unique à l'ordre près de l'écriture des cycles.

	Plus précisément, si \( \sigma\) est une permutation, alors il existe un unique ensemble fini \( \{ \omega_i \}_{i\in I}\) de cycles de supports disjoints tels que\footnote{Ici \( I\) est un ensemble fini et vu que les supports sont disjoints, le produit est commutatif.} \( \sigma=\prod_{i\in I}\omega_i\).
	%TODOooLWGBooZFcosV: l'unicité n'est pas prouvée, et il serait bien de réexprimer en termes de l'ensemble fini dont on parle dans le ``plus précisément''.
	% Il est déjà dans la liste. Il ne faut pas l'ajouter à nouveau.
\end{lemma}

\begin{proof}
	Soit \( \sigma\in S_E\). Si les éléments \( \{ a,\sigma(a), \ldots \sigma^k(a)\}\) sont distincts, alors soit \( \sigma^{k+1}(a)\) est distincts des autres, soit \( \sigma^{k+1}(a)=a\). Il n'est en effet pas possible d'avoir \( \sigma^{k+1}(a)=\sigma^l(a)\) avec \( l<k\) parce que ça contredirait l'injectivité de \( \sigma\).

	Soit donc \( a\in E\). Nous considérons le cycle \( \big( a,\sigma(a),\ldots, \sigma^k(a) \big)\) où \( k\) est maximum tel que tous les éléments sont distincts.

	Soit ce cycle contient tous les éléments de \( E\), soit il existe un élément \( b\) hors de ce cycle. Dans le second cas, nous considérons le cycle commençant par \( b\).

	Et ça continue\ldots
\end{proof}

\begin{lemma}[\cite{PDFpersoWanadoo}]       \label{LEMooWXXLooIzrwJT}
	Deux cycles de support disjoint commutent.
	%TODOooLIFRooRdFIuo. Prouver ça.
\end{lemma}

\begin{lemma}[\cite{BIBooEPQHooBvhumd}]     \label{LEMooVVPWooMkRjyR}
	Tout cycle de longueur \( r\) est le produit de \( r-1\) transpositions.
\end{lemma}

\begin{proof}
	Il suffit de vérifier que
	\begin{equation}
		(a_1,\ldots, a_r)=(a_1, a_r)(a_1, a_{r-1})\ldots (a_1, a_2).
	\end{equation}
\end{proof}

\begin{lemma}[\cite{MonCerveau}]        \label{LEMooGGLUooUSzuAx}
	Soit une permutation \( \sigma\in S_E\). Soit un cycle \( c\) et une permutation \( s\) de supports disjoints telles que \( \sigma=s\circ c\). Alors
	\begin{enumerate}
		\item       \label{LEMooUHWTooFptoZU}
		      Si \( a\in\supp(\sigma)\), alors pour tout \( q\in \eN\) nous avons \( c^q(a)=\sigma^q(a)\).
		\item       \label{ITEMooHSDLooIAKYZA}
		      Si \( a\in\supp(c)\), alors
		      \begin{equation}
			      \Omega_{\sigma}(a)=\Omega_c(a)=\supp(c).
		      \end{equation}
		\item
		      Si \( a\in\supp(c)\), alors
		      \begin{equation}
			      c(x)=\begin{cases}
				      \sigma(x) & \text{si } x\in\Omega_{\sigma}(a) \\
				      x         & \text{sinon. }
			      \end{cases}
		      \end{equation}
	\end{enumerate}
\end{lemma}

\begin{proof}
	En plusieurs parties.
	\begin{subproof}
		\spitem[Si \( a\in \supp(c)\), alors \( c(a)=\sigma(a)\)]
		% -------------------------------------------------------------------------------------------- 
		Soit \( a\in \supp(c)\). Nous savons que \( c(a)\neq a\), et vu que \( c\) est injective, nous devons aussi avoir \( c\big( c(a) \big)\neq c(a)\). Donc \( a\) et \( c(a)\) sont dans \( \supp(c)\). Étant donné que les supports de \( c\) et de \( s\) sont disjoints, nous déduisons que \( c(a)\) n'est pas dans le support de \( s\), et donc que
		\begin{equation}
			\sigma(a)=(s\circ c)(a)=s\big( c(a) \big)=c(a).
		\end{equation}
		\spitem[\( \sigma^q(a)=c^q(a)\)]
		% -------------------------------------------------------------------------------------------- 
		Juste une récurrence sur le point précédent : si \( b\in\supp(a)\), alors \( \sigma(b)=c(b)\in\supp(a)\).
		\spitem[Si \( a\in \supp(c)\), alors \( \Omega_{\sigma}(a)=\Omega_c(a)\)]
		% -------------------------------------------------------------------------------------------- 
		Utilisant le point précédent, ainsi que la définition \ref{DEFooMVFKooMpXMQy} d'une orbite,
		\begin{equation}
			\Omega_{\sigma}(a)=\{ \sigma^q(a) \}=\{ c^q(a) \}=\Omega_c(a).
		\end{equation}
		\spitem[\( \supp(c)\subset\Omega_c(a)\)]
		% -------------------------------------------------------------------------------------------- 
		Comme toujours, \( a\) est un élément de \( \supp(c)\). Nous considérons \( b\in\supp(c)\) et nous montrons que \( b\in \Omega_c(a)\). Étant donné que \( b\in\supp(c)\), nous avons \( c(b)\neq b\), de telle sorte que \( \Omega_c(b)\) contienne au moins deux éléments distincts.

		Même chose pour \( a\) : l'ensemble \( \Omega_c(a)\) contient au moins \( a\) et \( c(a)\). Vu que \( c\) est un cycle, il n'existe qu'une seule orbite non triviale. Donc \( \Omega_c(a)=\Omega_c(b)\). En particulier \( b\in\Omega_c(b)=\Omega_c(a)\).
		\spitem[\( \Omega_c(a)\subset\supp(c)\)]
		% -------------------------------------------------------------------------------------------- 
		Soit \( b\in \Omega_c(a)\). Le lemme \ref{LEMooANVHooOQiTwY} nous permet de dire que \( \Omega_c(a)=\Omega_c(b)\). Comme \( \Omega_c(b)\) contient au moins deux éléments (parce qu'il est égal à \( \Omega_c(a)\) et que \( a\) est dans le support de \( c\)), nous savons que \( c(b)\neq b\) et donc que \( b\in\supp(c)\).
		\spitem[La formule pour \( c(x)\)]
		% -------------------------------------------------------------------------------------------- 
		Si \( x\in \Omega_{\sigma)}(a)\), alors \( c(x)=\sigma(x)\) par le point \ref{LEMooUHWTooFptoZU}. Si \( x\) n'est pas dans \( \Omega_c(a)=\supp(c)\), alors \( x\) n'est pas dans le support de \( c\) et donc \( c(x)=x\).
	\end{subproof}
\end{proof}

Le lemme suivant permet d'extraire le cycle de \( \sigma\) associé à un élément de \( E\).
\begin{lemma}[\cite{MonCerveau}]        \label{LEMooFFTBooCZsaFu}
	Soient un ensemble fini \( E\) ainsi que \( \sigma\in S_E\), et \( a\in E\) tel que \( \sigma(a)\neq a\). Nous posons
	\begin{equation}
		\begin{aligned}
			c\colon E & \to E                                                  \\
			x         & \mapsto \begin{cases}
				                    \sigma(x) & \text{si } x\in \Omega_{\sigma}(a) \\
				                    x         & \text{sinon }
			                    \end{cases}
		\end{aligned}
	\end{equation}
	Alors
	\begin{enumerate}
		\item
		      Si \( b\in \Omega_{\sigma}(a)\), nous avons \( \Omega_{c}(b)=\Omega_{\sigma}(a)\).
		\item
		      Si \( b\notin \Omega_{\sigma}(a)\), nous avons \( \Omega_{c}(b)=\{ b \}\).
		\item
		      \( c\) est un cycle.
	\end{enumerate}
\end{lemma}

\begin{proof}
	Si \( b\in \Omega_{\sigma}(a)\), alors \( \sigma^q(b)\in \Omega_{\sigma}(a)\) pour tout \( q\in \eN\), et donc
	\begin{equation}
		c^q(b)=\sigma^q(b)\in\Omega_{\sigma}(a)
	\end{equation}
	pour tout \( q\). Donc nous avons
	\begin{equation}
		\Omega_c(b)=\{ c^q(b)\tq q\in \eN \}=\{ \sigma^q(b)\tq q\in \eN \}=\Omega_{\sigma}(b)=\Omega_{\sigma}(a).
	\end{equation}
	La dernière égalité est le lemme \ref{LEMooMIHGooQfALbc}.

	Si \( b\notin\Omega_{\sigma}(a)\), alors \( c(b)=b\) et \( \Omega_c(b)=\{ b \}\).

	Nous avons prouvé que \( c\) a une seule orbite de taille plus grands ou égale à \( 2\). Donc \( c\) est un cycle.
\end{proof}


\begin{theorem}[\cite{BIBooONYNooQolWiC}]
	Soit un ensemble fini \( E\) de cardinal au moins deux. Soit une permutation \( \sigma\in S_E\).
	\begin{enumerate}
		\item Il existe des cycles \( c_1,\ldots, c_m\) à support disjoints tels que \( \sigma=c_1\circ\ldots \circ c_m\).
		\item
		      Cette décomposition est unique à l'ordre près.
	\end{enumerate}
\end{theorem}

\begin{proof}
	Plusieurs points.
	\begin{subproof}
		\spitem[Existence]
		Nous choisissons des éléments \( \{ a_i \}_{i=1,\ldots, p}\) tels que les \( \Omega_{\sigma}(a_i)\) forment une partition de \( E\) en sous-ensembles disjoints. En posant \( l_k=\min\{ r\tq \sigma^r(a_k)=a_k \}\), nous avons
		\begin{equation}
			\Omega_{\sigma}(a_k)=\{ \sigma^q(a_k) \}_{q=1,\ldots, l_k-1}
		\end{equation}
		et tous les \( \sigma^q(a_k)\) sont distincts pour \( q=1,\ldots, l_k-1\).

		Posons
		\begin{equation}
			\begin{aligned}
				c_k\colon E & \to E                                                    \\
				x           & \mapsto \begin{cases}
					                      \sigma(x) & \text{si } x\in \Omega_{\sigma}(a_k) \\
					                      x         & \text{sinon. }
				                      \end{cases}
			\end{aligned}
		\end{equation}
		Le lemme \ref{LEMooGGLUooUSzuAx} dit que \( c_k\) est un cycle. Vu que \( c(a_k)=\sigma(a_k)\), le cycle \( c\) est un \( l_k\)-cycle.

		Nous montrons à présent que \( \sigma=c_1\circ\ldots c_p\). Soit \( x\in E\). Il existe un \( k\in\{1,\ldots, p \}\) tel que
		\begin{equation}
			x\in \Omega_{\sigma}(a_k)=\Omega_{c_k}(a_k)=\Omega_{c_k}(x),
		\end{equation}
		la dernière égalité est parce que \( x\in \Omega_{c_k}(a_k)\). Nous en déduisons que \( c_k(x)\neq x\). D'autre part si \( l\neq k\), alors \( x\) n'est pas dans \( \Omega_{\sigma}(a_l)\), et donc \( c_l(x)=x\). Au final,
		\begin{equation}
			(c_1\circ\ldots c_p)(x)=c_k(x)=\sigma(x).
		\end{equation}
		\spitem[Unicité]
		% -------------------------------------------------------------------------------------------- 
		Nous supposons avoir \( \sigma=c_1\circ\ldots\circ c_p=\gamma_1\circ\gamma_q\) où les \(  c_i\) et les \( \gamma_j\) sont deux ensembles de cycles de supports disjoints. Nous avons
		\begin{equation}
			\supp(\sigma)=\bigcup_{i=1}^p\supp(c_i)=\bigcup_{j=1}^q\supp(\gamma_j).
		\end{equation}
		Montrons que si \( \supp(c_i)\cap\supp(\gamma_j)\neq \emptyset\), alors \( \supp(c_i)=\supp(\gamma_j)\). En effet si \( a\in\supp(c_i)\cap\supp(\gamma_j)\), alors
		\begin{equation}
			\supp(c_i)=\Omega_{c_i}(a)=\Omega_{\sigma}(a)=\Omega_{\gamma_j}(a)=\supp(\gamma_j).
		\end{equation}
		Et comme les \( \supp(\gamma_j)\) sont disjoints, l'ensemble \( \supp(c_i)\) n'a d'intersection qu'avec un et un seul des \( \supp(\gamma_j)\). Cela définit donc une application
		\begin{equation}
			\begin{aligned}
				u\colon \{ 1,\ldots, p \} & \to \{ 1,\ldots, q \}                                              \\
				i                         & \mapsto \text{l'unique \( j\) tel que} \supp(c_i)=\supp(\gamma_j).
			\end{aligned}
		\end{equation}
		Autrement dit, l'application \( u\) permet d'écrire
		\begin{equation}
			\supp(c_i)=\supp\big( \gamma_{u(i)} \big).
		\end{equation}

		L'application \( u\) est injective. En effet si \( u(i)=u(l)\), nous avons
		\begin{subequations}
			\begin{align}
				\supp(c_i) & =\supp\big( \gamma_{u(i)} \big) \\
				\supp(c_l) & =\supp\big( \gamma_{u(l)} \big) \\
				u(i)       & =u(l).
			\end{align}
		\end{subequations}
		Donc \( \supp(c_i)=\supp(c_l)\). Et comme les supports sont disjoints, \( i=l\).

		L'application \( u\) est surjective. En effet, soit \( j\in \{ 1,\ldots, q \}\). Soit \( a\in \supp(\gamma_j)\). Il existe un \( i\) tel que \( a\in\supp(c_i)\). Nous avons alors \( a\in\supp(\gamma_j)\cap\supp(c_i)\), autrement dit \( u(i)=j\).

		Maintenant l'application \( u\colon \{ 1,\ldots, p \}\to \{ 1,\ldots, q \}\) est bijective. Nous en déduisons que \( p=q\). Concluons en montrant que \( c_i=\gamma_{u(i)}\). Soit \( a\in\supp(c_i)=\supp\big( \gamma_{u(i)} \big)\).

		Nous avons
		\begin{equation}
			c_i(x)=\begin{cases}
				\sigma(x) & \text{si } x\in\Omega_{\sigma}(a) \\
				x         & \text{sinon }
			\end{cases}
		\end{equation}
		et
		\begin{equation}
			\gamma_j(x)=\begin{cases}
				\sigma(x) & \text{si } x\in\Omega_{\sigma}(a) \\
				x         & \text{sinon, }
			\end{cases}
		\end{equation}
		et donc \( c_i=\gamma_j\).
	\end{subproof}
\end{proof}


\begin{lemma}[\cite{Combes}]        \label{LemmvZFWP}
	Soit \( \sigma=(i_1,\ldots, i_k)\in S_n\), un cycle de longueur \( k\) et \( \theta\in S_n\). Alors
	\begin{equation}
		\theta\sigma\theta^{-1}=\big( \theta(i_1),\ldots, \theta(i_k) \big).
	\end{equation}
	Tous les cycles de longueur \( k\) sont conjugués entre eux.
\end{lemma}

\begin{proposition}[Classes de conjugaison et structure en cycles\cite{UXMTXxl}] \label{PropEAHWXwe}
	Une classe de conjugaison\footnote{Définition \ref{DEFooOLXPooWelsZV}.} dans \( S_n\) est formée des permutations ayant une décomposition en cycles disjoints de même structure. Autrement dit, deux permutations \( \sigma\) et \( \sigma'\) sont conjuguées si et seulement si le nombre \( k_i\) de cycles de longueur \( i\) dans \( \sigma\) est le même que le nombre \( k'_i\) de cycles de longueur \( i\) dans \( \sigma'\).
\end{proposition}

\begin{proof}
	Soit \( \sigma=c_1\ldots c_m\) la décomposition de \( \sigma\) en cycles \( c_i\) de supports disjoints. Si \( \tau\) est une permutation, alors
	\begin{equation}
		\sigma'=\tau\sigma\tau^{-1}=(\tau c_1\tau^{-1})\ldots (\tau c_m\tau^{-1}),
	\end{equation}
	mais \( \tau c_i\tau^{-1}\) est un cycle de même longueur que \( c_i\), puisque le lemme~\ref{LemmvZFWP} nous dit que si \( \sigma=(a_1,\ldots, a_k)\), alors \( \tau c\tau^{-1}=\big( \tau(a_1),\ldots, \tau(a_k) \big)\). Notons encore que les cycles \( \tau c_i\tau^{-1}\) restent à support disjoints.

	Donc tous les éléments de la classe de conjugaison de \( \sigma\) sont des permutations de même structure que \( \sigma\).

	Réciproquement, si \( \sigma'=c'_1\ldots c'_m\) est une décomposition de \( \sigma'\) en cycles disjoints tels que la longueur des \( c_i\) est la même que la longueur des \( c'_i\), alors il suffit de construire des permutations \( \tau_i\) telles que \( \tau_i c_i\tau_i^{-1}=c_i'\), à travers le lemme~\ref{LemmvZFWP}. Comme les supports des \( c_i\) et des \( c'_i\) sont disjoints, la permutation \( \tau_1\ldots \tau_m\) conjugue \( \sigma\) et \( \sigma'\).
\end{proof}

\begin{example}     \label{EXooQAXRooBsPURs}
	Voyons les classes de conjugaison de \( S_3\). Étant donné que ce groupe agit par définition sur un ensemble à \( 3\) éléments, aucun élément de \( S_3\) ne possède un cycle de plus de \( 3\) éléments. Il y a donc seulement des cycles de longueur deux ou trois (à part les triviaux). Aucun élément de \( S_3\) n'a une décomposition en cycles disjoints contenant deux cycles de deux ou un cycle de deux et un de trois.

	En résumé il y a trois classes de conjugaison dans \( S_3\). La première est celle contenant seulement l'identité. La seconde est celle contenant les cycles de longueur deux et la troisième contient les cycles de longueur \( 3\).

	Ce sont donc
	\begin{subequations}
		\begin{align}
			C_1 & =\{ \id \}               \\
			C_2 & =\{ (1,2),(1,3),(2,3) \} \\
			C_3 & =\{ (1,2,3),(2,1,3) \}.
		\end{align}
	\end{subequations}
\end{example}

\begin{definition}[transposition]      \label{DEFooXNAFooGTbTTJ}
	Une \defe{transposition}{transposition} est une permutation\footnote{Une permutation est une bijection, définition \ref{DEFooJNPIooMuzIXd}.} qui échange deux éléments de \( E\). Plus précisément, une bijection \( \sigma\colon E\to E\) est une transposition si il existe \( a,b\in E\) tels que
	\begin{equation}
		\sigma(x)=\begin{cases}
			a & \text{si } x=b \\
			b & \text{si } x=a \\
			x & \text{sinon. }
		\end{cases}
	\end{equation}
\end{definition}

\begin{example} \label{ExVYZPzub}
\end{example}


\begin{proposition}[\cite{MonCerveau}]	\label{PROPooQRZYooKVZTCJ}
	Les classes de conjugaison de \( S_4\) sont :
	\begin{enumerate}
		\item		\label{ITEMooZIYYooCqpicx}
		      Le cycle vide qui représente la classe constituée de l'identité seule.
		\item \label{ITEMooVDLGooFOkgzc}
		      Les transpositions (de type \( (a,b)\)) qui sont au nombre de \( 6\).
		\item	\label{ITEMooWKUIooTedZAA}
		      Les doubles transpositions (de type \( (a,b)(c,d)\)) qui sont au nombre de \( 6\).
		\item		\label{ITEMooMHRWooSkUveJ}
		      Les \( 3\)-cycles qui sont au nombre de \( 8\).
		\item		\label{ITEMooCBYNooZmFsBd}
		      Les \( 4\)-cycles qui sont au nombre de \( 6\).

		\item       \label{ITEMooGCMYooKZgFHX}
		      Les doubles transpositions, du type \( (a,b)(c,d)\) qui sont au nombre de \( 3\).
	\end{enumerate}
	\index{classe de conjugaison!dans \( S_4\) }
\end{proposition}

\begin{proof}
	Nous savons que les classes de conjugaison dans \( S_4\) sont caractérisées par la structure des décompositions en cycles (proposition~\ref{PropEAHWXwe}). Avec \( 4\) éléments, il n'y a pas des tonnes de cycles qu'on peut faire.

	On peut faire des cycles de taille \( 1\), \( 2\), \( 3\) ou \( 4\). Les seules façons de faire \( 4\) avec ces nombres sont :
	\begin{enumerate}
		\item
		      \( 1+1+1+1\)
		\item
		      \( 2+1+1\)
		\item
		      \( 2+2\)
		\item
		      \( 3+1\)
		\item
		      \( 4\)
	\end{enumerate}
	Comptons combien il y a de possibilités dans chaque cas.
	\begin{subproof}
		\spitem[Pour \ref{ITEMooZIYYooCqpicx}]
		%-----------------------------------------------------------
		À propos de la possibilité \( 1+1+1+1=4\). Il y a une seule possibilité de fixer tous les éléments.
		\spitem[Pour \ref{ITEMooVDLGooFOkgzc}]
		%-----------------------------------------------------------
		À propos de \( 2+1+1=4\). Il y a \( 4\) possibilités pour \( a\), et \( 3 \) pour \( b\). Mais \( (a,b)=(b,a)\), donc il faut diviser par deux. Donc \( 6\) en tout.
		\spitem[Pour \ref{ITEMooWKUIooTedZAA}]
		%-----------------------------------------------------------
		À propos de \( 2+2=4\). Nous cherchons les possibilités pour  \( (a,b)(c,d)\). Nous venons de voir qu'il y a \( 6\) possibilités pour \( (a,b)\). Les deux autres sont fixés.
		\spitem[Pour \ref{ITEMooMHRWooSkUveJ}]
		%-----------------------------------------------------------
		Pour savoir \href{http://www.toujourspret.com/techniques/expression/chants/C/cantique_des_etoiles.php}{quel est leur nombre} nous commençons par remarquer qu'il y a \( 4\) façons de prendre \( 3\) nombres parmi \( 4\) et ensuite \( 2\) façons de les arranger. Il y a donc \( 8\) éléments dans cette classe de conjugaison.
		\spitem[Pour \ref{ITEMooCBYNooZmFsBd}]
		%-----------------------------------------------------------
		Le premier est arbitraire (parce que c'est cyclique). Pour le second il y a \( 3\) possibilités, et deux possibilités pour le troisième; le quatrième est alors automatique. Cette classe de conjugaison contient donc \( 6\) éléments.
		\spitem[Pour \ref{ITEMooGCMYooKZgFHX}]
		%-----------------------------------------------------------
		Dans ce cas, tous les nombres sont permutés, et l'image de \( 1\) détermine la double transposition. Il y a \( 3\) images possibles, et donc \( 3\) éléments dans cette classe.
	\end{subproof}
\end{proof}

\begin{proposition} \label{PropPWIJbu}
	Tout élément de \( S_n\) peut être écrit sous la forme d'un produit fini de transpositions.

	Si \( E \) est un ensemble fini, tout élément de \( S_E\) pour être écrit sous forme d'un produit fini de transposition de \( E\).
	%TODOooAFZSooPMWvJb prouver cette deuxième partie
\end{proposition}

\begin{proof}
	Un élément de \( S_n\) se décompose en un nombre fini de cycles par le lemme \ref{LEMooQLSAooBrXDXw} et chacun des cycles peut être décomposé en un nombre fini de transpositions par le lemme \ref{LEMooVVPWooMkRjyR}.
\end{proof}

Cette décomposition n'est pas à confondre avec celle en cycles de support disjoints. Par exemple \( (1,2,3)=(1,3)(1,2)\).

%-------------------------------------------------------
\subsection{Parité d'une permutation, signature}
%----------------------------------------------------

La notion de parité d'une permutation est la clef pour savoir quelles positions du jeu de taquin sont possibles ou impossibles\cite{BIBooGLIWooAggcqh,BIBooCDDXooYWCtNZ}.

\begin{definition}[\cite{BIBooCSDMooMcXgXz}]	\label{DEFooYDUHooKIXGNW}
	Soit une permutation \( \sigma\in S_E\) où \( \Card(E)=n\). Si \( k\) est le nombre de \( \sigma\)-orbites dans \( E\), nous définissons la \defe{signature}{signature d'une permutation} par
	\begin{equation}
		\epsilon(\sigma)=(-1)^{n-k}.
	\end{equation}
\end{definition}

\begin{proposition}[\cite{BIBooCSDMooMcXgXz}]	\label{PROPooWCHBooTgMokj}
	Soient une transposition \( \tau\) et une permutation \( \sigma\) de l'ensemble fini \( E\). Nous avons
	\begin{equation}
		\epsilon(\sigma\tau)=-\epsilon(\sigma).
	\end{equation}
\end{proposition}

\begin{proof}
	Soient \( a,b\in E\) tels que \( \tau(a)=b\) et \( \tau(b)=a\). Nous allons comparer la liste des orbites de \( \sigma\) avec celle de \( \sigma\tau\). Il y a deux possibilités : soit \( a\) et \( b\) sont dans la même orbite de \( \sigma\), soit ils sont dans des orbites séparées.

	Nous notons\footnote{Parce que nous adorons à la fois les notations et les listes énumérées.}
	\begin{enumerate}
		\item
		      \( I\) l'ensemble des orbites de \( \sigma\) ne contenant ni \( a\) ni \( b\).
		\item
		      \( I'\) l'ensemble des orbites de \( \tau\sigma\) ne contenant ni \( a\) ni \( b\).
		\item
		      \( \mO_a\) la \( \sigma\)-orbite de \( a\), \( \mO_b\) la \( \sigma\)-orbite de \( b\).
		\item
		      \( \mO'_a\) la \( \sigma\tau\)-orbite de \( a\), \( \mO'_b\) la \( \sigma\tau\)-orbite de \( b\).
	\end{enumerate}
	Il est tout à fait possible à priori d'avoir \( \mO_a=\mO_b\) ou \( \mO'_a=\mO'_b\). Je vous laisse méditer un instant sur le fait que \( \Card\{ \mO_a,\mO_b \}\) vaut \( 1\) ou  \( 2\) et que le nombre d'orbites de \( \sigma\) est \( \Card(I)+\Card\{ \mO_a,\mO_b \}\).

	\begin{subproof}
		\spitem[\( I=I'\)]
		%-----------------------------------------------------------

		Soit une orbite \( \mO\) de \( \sigma\) ne contenant ni \( a\) ni \( b\). Alors pour tout \( y\in \mO\) nous avons \( (\sigma\tau)(y)=\sigma(y)\), et donc \( \mO\) est également une orbite de \( \sigma\tau\). Cela prouve que \( I=I'\).

		\spitem[Si \( \mO_a=\mO_b\)]
		%-----------------------------------------------------------
		Nous supposons que \( a\) et \( b\) sont dans la même orbite de \( \sigma\) que nous écrivons
		\begin{equation}
			\mO=\{ a,x_1,\ldots,x_r,b,x_{r+1},\ldots,x_s \}
		\end{equation}
		avec \( \sigma(x_s)=a\). Étudions la \( \sigma\tau\)-orbite de \( a\). Nous avons \( (\sigma\tau)(a)=\sigma(b)=x_{r+1}\), et \( (\sigma\tau)(x_s)=\sigma(x_s)=a\), donc
		\begin{equation}
			\mO'_a=\{ a,x_{r+1},\ldots,x_s \}
		\end{equation}
		ne contient pas \( b\). De même la \( (\sigma\tau)\)-orbite de \( b\) est
		\begin{equation}
			\mO'_b=\{ b,x_1,\ldots,x_r \}
		\end{equation}
		et ne contient pas \( a\). Autrement dit, \( \mO'_a\neq \mO'_b\).

		\spitem[Si \( \mO_a\neq\mO_b\)]
		%-----------------------------------------------------------
		Disons que  \( \mO_a=\{ a,x_1,\ldots,x_r \}\) et \( \mO_b=\{ b,y_1,\ldots,y_s \}\) avec \( \sigma(y_s)=b\) et \( \sigma(x_r)=a\). Nous étudions la \( (\sigma\tau)\)-orbite de \( a\). Nous avons \( (\sigma\tau)(a)=\sigma(b)=y_1\), et ensuite \( (\sigma\tau)(y_1)=\sigma(y_1)=y_2\), et ainsi de suite jusqu'à \( (\sigma\tau)(y_s)=\sigma(y_s)=b\), et \( (\sigma\tau)(b)=\sigma(a)=x_1\). Bref nous avons
		\begin{equation}
			\mO_a'=\{ a,y_1,\ldots,y_s,b,x_1,\ldots,x_r \},
		\end{equation}
		et donc \( \mO_a'=\mO_b'\).
	\end{subproof}

	Nous avons prouvé que si \( \Card\{ \mO_a,\mO_b \}=1\), alors \( \Card\{ \mO_a',\mO_b' \}=2\) et que si \( \Card\{ \mO_a,\mO_b \}=2\), alors \( \Card\{ \mO_a',\mO_b' \}=1\). Donc \( \sigma\tau\) a une orbite en plus ou une orbite en moins que \( \sigma\) suivant les cas. Dans les deux cas, \( \epsilon(\sigma\tau)=-\epsilon(\sigma)\).
\end{proof}

\begin{theorem}[\cite{BIBooCSDMooMcXgXz}]	\label{THOooUILDooRUAZfW}
	À propos de signature.
	\begin{enumerate}
		\item		\label{ITEMooOLVOooWJKSDY}
		      Soient des transpositions \( \tau_1,\ldots,\tau_s\). Nous avons
		      \begin{equation}
			      \epsilon(\tau_1\ldots \tau_s)=(-1)^s.
		      \end{equation}
		\item		\label{ITEMooWRONooJQkhzG}
		      L'application
		      \begin{equation}
			      \epsilon \colon S_E\to \big( \{ -1,1 \},\times \big)
		      \end{equation}
		      est un morphisme de groupe surjectif.
	\end{enumerate}
\end{theorem}

\begin{proof}
	En plusieurs parties.
	\begin{subproof}
		\spitem[Pour \ref{ITEMooOLVOooWJKSDY}]
		%-----------------------------------------------------------
		D'abord si \( \tau\) est une transposition, alors \( \epsilon(\tau)=-1\) parce que \( \tau \) possède \( n-1\) orbite dans \( \{ 1,\ldots,n \}\). Ensuite une récurrence en utilisant la proposition \ref{PROPooWCHBooTgMokj} donne le résultat.


		\spitem[Pour \ref{ITEMooWRONooJQkhzG}]
		%-----------------------------------------------------------
		L'application \( \epsilon\) est surjective parce que \(\epsilon(\tau)=-1 \) et \( \epsilon(\id)=1\). En ce qui concerne le morphisme, si \( \sigma,\sigma'\in S_E\), ils peuvent être décomposés en transpositions\footnote{Par la proposition \ref{PropPWIJbu}.} : \( \sigma=\tau_1\ldots\tau_s\) et \( \sigma'=\tau_{s+1}\ldots \tau_{s+r}\). Nous avons alors
		\begin{equation}
			\epsilon(\sigma\sigma')=\epsilon(\tau_1\ldots\tau_s\tau_{s+1}\ldots \tau_{s+r})=(-1)^{s+r}=(-1)^s(-1)^r=\epsilon(\sigma)\epsilon(\sigma').
		\end{equation}
	\end{subproof}
\end{proof}

\begin{propositionDef}[parité d'une permutation]\label{PROPooKRHEooAxtmRv}
	À propos de décomposition ne permutations.
	\begin{enumerate}
		\item
		      Si une permutation peut être écrite sous forme d'un produit d'un nombre pair de transpositions, alors toute décomposition en transpositions sera en quantité paire.
		\item
		      Si une permutation peut être écrite sous forme d'un produit d'un nombre impair de transpositions, alors toute décomposition en transpositions sera en quantité impaire.
	\end{enumerate}
	Une permutation qui se décompose en une quantité paire de transpositions est une \defe{permutation paire}{permutation paire} (et \defe{impaire}{permutation impaire} sinon).
\end{propositionDef}

\begin{proof}
	Si \( \sigma=\tau_1\ldots \tau_s\) et \( \sigma=\tau'_1\ldots \tau'_r\), alors le théorème \ref{THOooUILDooRUAZfW}\ref{ITEMooOLVOooWJKSDY} dit que \( \epsilon(\sigma)=(-1)^s\) et \( \epsilon(\sigma)=(-1)^r\), ce qui signifie que \( (-1)^r=(-1)^s\) et donc que \( r\) et \( s\) ont la même parité.
\end{proof}

\begin{proposition}       \label{PROPooXIQOooCcoekP}
	La signature de \( \sigma\in S_E\) est donné par
	\begin{equation}
		\begin{aligned}
			\epsilon\colon S_E & \to \{ -1,1 \}                                   \\
			\sigma             & \mapsto \begin{cases}
				                             1  & \text{si \( \sigma\) est paire }    \\
				                             -1 & \text{si \( \sigma\) est impaire. }
			                             \end{cases}
		\end{aligned}
	\end{equation}
\end{proposition}

\begin{proof}
	Si \( \sigma\) est paire, alors, par la définition \ref{PROPooKRHEooAxtmRv}, \( \sigma=\tau_1\ldots\tau_s\) avec \( s\) pair. Dans ce cas\footnote{Par le théorème \ref{THOooUILDooRUAZfW}\ref{ITEMooOLVOooWJKSDY}}, \( \epsilon(\sigma)=(-1)^s=1\). Même chose si \( \sigma\) est impaire.
\end{proof}

\begin{lemma}  \label{LEMooWGRXooHWyzLC}
	Nous disons qu'un élément \( \sigma \in S_n\) est une \defe{inversion}{inversion!dans le groupe symétrique} pour les nombres \( i<j\) si \( \sigma(i)>\sigma(j)\). Soit \( N_\sigma\) le nombre d'inversions que \( \sigma\in S_n\) possède (c'est le nombre de couples \( (i,j)\) avec \( i<j\) tels que \( \sigma(i)>\sigma(j)\)). Nous avons
	\begin{equation}
		\epsilon(\sigma)=(-1)^{N_\sigma}
	\end{equation}
	où \( \epsilon\) est la signature\footnote{Définition \ref{DEFooYDUHooKIXGNW}.} dans \( S_n\).
	%TODOooJYQSooXIWGuU. Prouver ça.
\end{lemma}

\begin{lemma}[\cite{PDFpersoWanadoo}]       \label{LemhxnkMf}
	Un \( k\)-cycle\footnote{Définition \ref{DEFooRWYSooYmKwph}.} est une permutation impaire si \( k\) est pair et paire si \( k\) est impair.
\end{lemma}

\begin{proof}
	Il faut remonter à la définition \ref{DEFooYDUHooKIXGNW} de la signature. Si \( \sigma\) est un \( k\)-cycle, alors il possède une orbite de cardinal \( k\) et \( n-k\) orbites singleton. Donc \( \epsilon(\sigma)=(-1)^{n-(n-k+1)}=(-1)^{k-1}\). Donc la parité de la signature est toujours opposée à la parité de \( k\).
\end{proof}

\begin{proposition}[\cite{Combes}]  \label{ProphIuJrC}
	Soit \( S_n\) le groupe symétrique.
	\begin{enumerate}
		\item       \label{ITEMooBQKUooFTkvSu}
		      L'application \( \epsilon\colon S_n\to \{ 1,-1 \}\) est l'unique homomorphisme surjectif de \( S_n\) sur \( \{ -1,1 \}\).
		\item
		      Si \( s=t_1\cdots t_k\) est le produit de \( k\) transpositions, alors \( \epsilon(s)=(-1)^k\).
	\end{enumerate}
\end{proposition}

\begin{proof}
	La seule partie à prouver est l'unicité. Le reste est dans le théorème \ref{THOooUILDooRUAZfW}.

	Soit un homomorphisme surjectif \( \varphi\colon S_n\to \{ -1,1 \}\) et \( \tau\), une transposition telle que \( \varphi(\tau)=-1\) (qui existe parce que sinon \( \varphi\) ne serait pas surjectif\footnote{Nous utilisons ici le fait que tous les éléments de \( S_n\) sont des produits de transpositions, proposition~\ref{PropPWIJbu}.}). Si \( \tau'\) est une autre transposition, il existe \( \sigma\in S_n\) tel que \( \tau'=\sigma\tau\sigma^{-1}\) (lemme~\ref{LemmvZFWP}). Dans ce cas, \( \varphi(\tau')=\varphi(\tau)=-1\), et si \( \sigma=(\tau_1\ldots \tau_k) \),
	\begin{equation}
		\varphi(\sigma)=(-1)^k=\epsilon(\sigma).
	\end{equation}
\end{proof}

\begin{corollary}       \label{CORooZLUKooBOhUPG}
	Si \( \sigma\in S_n\), alors
	\begin{equation}
		\epsilon(\sigma)=\epsilon(\sigma^{-1}).
	\end{equation}
\end{corollary}

\begin{proof}
	Comme énoncé par la proposition \ref{ProphIuJrC}, \( \epsilon\) est un homomorphisme, donc
	\begin{equation}
		\epsilon(\sigma)\epsilon(\sigma^{-1})=\epsilon(\sigma\sigma^{-1})=\epsilon(\id)=1.
	\end{equation}
	Puisque \( \epsilon(\sigma)\) et \( \epsilon(\sigma^{-1})\) ne peuvent valoir que \( \pm1\), ils doivent être tous les deux égaux à \( 1\) ou tous les deux à \( -1\) pour que le produit soit \( 1\).
\end{proof}

%-------------------------------------------------------
\subsection{Permutation un peu ordonnées}
%----------------------------------------------------

\begin{lemma}[\cite{MonCerveau}]		\label{LEMooEOTGooPslULz}
	Deux énoncés.
	\begin{enumerate}
		\item		\label{ITEMooQZDRooUYjcgX}
		      Soit \( \pi\in S_k\) satisfaisant \( \pi(1)=1\). Nous définissons \( \tau\in S_{k-1}\) par \( \tau(i)=\pi(i+1)-1 \).

		      Alors \( \epsilon(\tau)=\epsilon(\pi)\).
		\item		\label{ITEMooBUGZooVCVhKz}
		      L'application
		      \begin{equation}
			      \begin{aligned}
				      \varphi\colon \{ \pi\in S_k\tq \pi(1)=1 \} & \to S_{k-1}                             \\
				      \pi                                        & \mapsto \Big[  \tau(i)=\pi(i+1)-1 \Big]
			      \end{aligned}
		      \end{equation}
		      est une bijection.
	\end{enumerate}
\end{lemma}

\begin{proof}
	Nous nommons \( \pi'\) la restriction de \( \pi\) à \( \{ 2,\ldots,k \}\). Cela est encore une bijection, de telle sorte que \( \pi'\) puisse être écrite sous forme d'un produit de \( n\) transpositions de \( \{ 2,\ldots,k \}\) (proposition \ref{PropPWIJbu}) :
	\begin{equation}
		\pi=\sigma_1\circ\ldots\circ\sigma_n.
	\end{equation}
	Les \( \sigma_i\) étant des transpositions de \( \{ 2,\ldots,k \}\), elles sont des transpositions de \( \{ 1,\ldots,k \}\). Tout cela pour dire que \( \pi\) peut être écrite comme produit de transpositions ne faisant pas intervenir \( 1\).

	Nous introduisons la bijection
	\begin{equation}
		\begin{aligned}
			\psi\colon \{ 2,\ldots,k \} & \to \{ 1,\ldots,k-1 \} \\
			i                           & \mapsto i-1.
		\end{aligned}
	\end{equation}
	En termes de \( \psi\), la définition de \( \tau\) s'écrit \( \psi(\pi)=\psi\circ \pi\circ\psi^{-1}\), et nous avons
	\begin{subequations}
		\begin{align}
			\tau & =\psi\circ \pi\circ\psi^{-1}                                                                           \\
			     & =\psi\circ\sigma_1\circ\ldots\circ\sigma_n\circ\psi^{-1}                                               \\
			     & =\psi\circ\sigma_1\circ\psi^{-1}\circ\psi\circ\sigma_2\circ\ldots\circ\psi\circ\sigma_n\circ\psi^{-1}.
		\end{align}
	\end{subequations}
	Bref, en notant \( \sigma'_i=\psi\sigma_i\psi^{-1}\) nous avons \( \tau=\sigma'_1\circ\ldots\circ\sigma_n\).

	Il suffit maintenant de remarquer que \( \sigma'_i\) est une transposition dans \( \{ 1,\ldots,k-1 \}\). On vérifie pour cela que si \( \sigma=(a,b)\) alors \( \sigma'=\big( \psi(a),\psi(b) \big)\). En effet, pour \( i=1,\ldots,k-1\), il y a trois possibilités : soit \( i=\psi(a)\), soit \( i=\psi(b)\) soit \( i\) n'est ni l'un ni l'autre.

	Si \( i=\psi(a)\), alors \( \sigma'(i)=(\psi\sigma\psi^{-1})\big( \psi(a) \big)=\psi\sigma(a)=\psi(b)\). Même vérification pour montrer que \( \sigma'\big( \psi(b) \big)=\psi(a)\). Si \( i\) n'est ni \( \psi(a)\) ni \( \psi(b)\), alors \( \psi^{-1}(i)\) n'est ni \( a\) ni \( b\) et dans ce cas
	\begin{equation}
		(\psi\sigma\psi^{-1})(i)=\psi\sigma\big( \psi^{-1}(i) \big)=\psi\big( \psi^{-1}(i) \big)=i.
	\end{equation}

	Donc la décomposition \( \tau=\sigma'_1\circ\ldots\circ\sigma_n'\) est une décomposition de \( \tau\) en \( n\) transpositions. Autrement dit le nombre de transpositions est le même pour \( \tau\) que pour \( \pi\).

	En particulier les signatures de \( \tau\) et de \( \pi\) sont les mêmes. Cela finit la preuve du point \ref{ITEMooQZDRooUYjcgX}.

	Prouvons le point \ref{ITEMooBUGZooVCVhKz}.
	\begin{subproof}
		\spitem[Injectif]
		%-----------------------------------------------------------
		Supposons que \( \varphi(\pi)=\varphi(\sigma)\). Alors pour tout \( i=1,\ldots,k-1\) nous avons
		\begin{equation}
			\pi(i+1)-1=\sigma(i+1)-1,
		\end{equation}
		c'est-à-dire \( \pi(j)=\sigma(j)\) pour tout \( j=2,\ldots,k\). Vu que \( \pi(1)=\sigma(1)\) nous avons bien \( \pi=\sigma\).

		\spitem[Surjectif]
		%-----------------------------------------------------------
		Soit \( \tau\in S_{k-1}\). En posant
		\begin{equation}
			\pi(i)=\begin{cases}
				1           & \text{si } i=1 \\
				\tau(i-1)+1 & \text{sinon, }
			\end{cases}
		\end{equation}
		nous avons bien \( \tau=\varphi(\pi)\).
	\end{subproof}
\end{proof}

\begin{definition}[\cite{MonCerveau, BIBooBVCRooAwDAqk}]		\label{DEFooDFBEooFElghU}
	Nous notons
	\begin{equation}
		S_{(k,l)}=\{ \pi\in S_{k+l}\tq \pi(1)<\ldots \pi(k),\,\pi(k+1)<\ldots<\pi(k+l) \},
	\end{equation}
	et
	\begin{equation}
		A_{(k,l)}=\{ \pi\in S_{(k,l)}\tq \pi(1)=1 \}.
	\end{equation}
\end{definition}

\begin{lemma}[\cite{MonCerveau}]		\label{LEMooCKJAooBIAyVs}
	Deux énoncés.
	\begin{enumerate}
		\item		\label{ITEMooCNWMooIbnHnz}
		      Soit \( \pi\in A_{(k,l)}\). En posant \( \tau(i)=\pi(i+1)-1\) nous avons \( \tau\in S_{(k-1,l)}\).
		\item		\label{ITEMooKNKRooWbLYrF}
		      L'application
		      \begin{equation}
			      \begin{aligned}
				      \varphi\colon A_{(k,l)} & \to S_{(k-1,l)}                          \\
				      \pi                     & \mapsto \Big[  \tau(i)=\pi(i+1)-1  \Big]
			      \end{aligned}
		      \end{equation}
		      est une bijection.
		\item
		      En ce qui concerne la signature, \( \epsilon\big( \varphi(\pi) \big)=\epsilon(\pi)\).
	\end{enumerate}
\end{lemma}

\begin{proof}
	Pour prouver \ref{ITEMooCNWMooIbnHnz}, soit d'abord \( 1\leq i<j\leq k-1\). Nous avons
	\begin{equation}
		\tau(i)=\pi(i+1)-1<\pi(i+j)-1=\tau(j)
	\end{equation}
	parce que \( i+1\) et \( j+1\) sont entre \( 1\) et \( k\). Le même raisonnement tient pour \( k+1\leq i<j\leq k+l\).

	Pour \ref{ITEMooBUGZooVCVhKz}. Même preuve que la partie correspondante du lemme \ref{LEMooEOTGooPslULz}.
\end{proof}




%+++++++++++++++++++++++++++++++++++++++++++++++++++++++++++++++++++++++++++++++++++++++++++++++++++++++++++++++++++++++++++ 
\section{Symbole de sommation}
%+++++++++++++++++++++++++++++++++++++++++++++++++++++++++++++++++++++++++++++++++++++++++++++++++++++++++++++++++++++++++++

%--------------------------------------------------------------------------------------------------------------------------- 
\subsection{Somme à valeurs dans un groupe commutatif}
%---------------------------------------------------------------------------------------------------------------------------

Si \( S\) est un ensemble fini, nous savons de la proposition \ref{PROPooJLGKooDCcnWi} qu'il existe un unique \( N\in \eN\) pour lequel il existe une bijection \( \varphi\colon \{ 0,\ldots, N \}\to S\). Cette bijection n'est à priori pas unique.

\begin{lemmaDef}[\cite{MonCerveau}]       \label{DEFooLNEXooYMQjRo}
	Soient un groupe commutatif \( (G,+)\) ainsi qu'un ensemble fini \( I\) contenant \( n\) éléments. Soit une application \( f\colon I\to G \). Si \( \sigma_1,\sigma_2\colon \{1,\ldots, n \}\to I\) sont deux bijections, alors\footnote{Pour rappel, le symbole \( \sum_{i=1}^n\) est défini par \ref{DEFooNEVNooJlmJOC}.}
	\begin{equation}
		\sum_{i=1}^nf\big( \sigma_1(i) \big)=\sum_{i=1}^nf\big( \sigma_2(i) \big).
	\end{equation}
	La valeur commune est notée
	\begin{equation}
		\sum_{i\in I}f(i)
	\end{equation}
\end{lemmaDef}

\begin{proof}
	Nous commençons par considérer une transposition \( \sigma\) (qui permute \( k\) et \( l\) avec \( k<l\)). Nous avons
	\begin{subequations}
		\begin{align}
			\sum_{i=1}^nf(i) & =\sum_{i=1}^{k-1}f(i)+f(k)+\sum_{i=k+1}^{l-1}f(i)+f(l)+\sum_{i=l+1}^nf(i) \\
			                 & =\sum_{i=1}^{k-1}f(i)+f(l)+\sum_{i=k+1}^{l-1}f(i)+f(k)+\sum_{i=l+1}^nf(i) \\
			                 & =\sum_{i=1}^nf\big( \sigma(i) \big).
		\end{align}
	\end{subequations}
	Pour cela nous avons utilisé le fait que \( G\) est commutatif pour permuter \( f(l)\in G\) et \( f(k)\in G\) avec \( \sum_{i=k+1}^{l-1}f(i)\in G\).

	Une permutation quelconque est un produit de telles transpositions (proposition \ref{PropPWIJbu}). Donc pour toute permutation \( \sigma\) nous avons
	\begin{equation}
		\sum_{i=1}^nf\big( \sigma(i) \big)=\sum_{i=1}^nf(i).
	\end{equation}
\end{proof}

La définition \ref{DEFooLNEXooYMQjRo} donne lieu à un certain nombre de remarques.
\begin{enumerate}
	\item
	      Elle donne la somme sur un ensemble fini. Un problème avec les ensembles infinis (outre la convergence) est l'ordre de sommation. Si vous voulez sommer sur \( \eZ\), dans quel ordre le faire ?
	\item
	      Pour aller plus loin, et sommer sur des ensembles infinis, rendez-vous dans le thème \ref{THEMEooMKLBooLGFCdx}.
\end{enumerate}

\begin{proposition}     \label{PROPooJBQVooNqWErk}
	Soient un groupe commutatif \( (G,+)\), un ensemble fini \( I\), une application \( f\colon I\to G\) et une bijection \( \sigma\colon I\to I\). Alors
	\begin{equation}
		\sum_{i\in I}f(i)=\sum_{i\in I}f\big( \sigma(i) \big).
	\end{equation}
\end{proposition}

\begin{proof}
	Par définition \ref{DEFooLNEXooYMQjRo}, si \(\varphi \colon \{ 1,\ldots,n \}\to G  \) est une bijection, alors
	\begin{equation}
		\sum_{i\in I}f(i)=\sum_{i=1}^n(f\circ\varphi)(i)
	\end{equation}
	et
	\begin{equation}
		\sum_{i\in I}(f\circ\sigma)(i)=\sum_{i=1}^n(f\circ\varphi\circ \sigma)(i).
	\end{equation}
	Vu que \( \varphi\) et \( \varphi\circ\sigma\) sont des bijections, les deux sont égaux par le lemme \ref{DEFooLNEXooYMQjRo}.
\end{proof}

Si nous avons une application \( L\colon S\to S\), nous notons
\begin{equation}
	\sum_{s\in S}f\big( L(s) \big)=\sum_{s\in S}(f\circ L)(s).
\end{equation}
Cette façon d'écrire donne une interprétation pour la notation \( \sum_{g\in G}f(hg)\) qui arrive dans la proposition \ref{PROPooWJQQooFINSEc}. Il s'agit de considérer l'application \( L_h\) du lemme \ref{LEMooBIBFooBHxFYC}, de considérer\footnote{Le fait que \( L_h\) soit une bijection n'a pas d'importance ici.}
\begin{equation}        \label{EQooQQBEooFDOBVG}
	\sum_{g\in G}f(hg)=\sum_{g\in G}(f\circ L_h)(g)
\end{equation}
et de faire tourner la définition \ref{DEFooLNEXooYMQjRo}. La même chose tient pour définir \( \sum_{g\in G}(gh)\) à l'aide de \( R_h\).


\begin{lemma}[Changement de variables dans une somme\cite{MonCerveau}]		\label{LEMooGAMAooOAFhrc}
	Soient deux ensembles finis \( I,J\) ainsi qu'une bijection \(\varphi \colon I\to J  \). Soient un groupe abélien \( G\) et une application \(f \colon I\to G  \). Alors
	\begin{equation}
		\sum_{i\in I}f(i)=\sum_{j\in J}f\big( \varphi^{-1}(j) \big).
	\end{equation}
\end{lemma}

\begin{lemma}		\label{LEMooNDBYooAGEkmw}
	Soit un ensemble \( A\) fini pouvant être écrit comme une union disjointe \( A=\bigcup_{k=1}^nA_k\); nous supposons que les \( A_i\) sont non vides. Soient un groupe commutatif \( (G,+)\) et une application \( f\colon A\to G\). Alors
	\begin{equation}
		\sum_{a\in A}f(a)=\sum_{k=1}^n\sum_{a\in A_k}f(a).
	\end{equation}
\end{lemma}


\begin{proof}
	Le lemme \ref{LEMooTUIRooEXjfDY} nous indique que les parties \( A_k\) sont des ensembles finis. Nous notons
	\begin{enumerate}
		\item
		      \( N_0=0\), et \( N_k=\Card(A_k)\),
		\item
		      \( S_k=\sum_{i=1}^kN_k\).
		\item
		      \( \varphi_k\colon \{ 1,\ldots, N_k \}\to A_k\), une bijection (l'existence est dans la proposition \ref{PROPooJLGKooDCcnWi}).
	\end{enumerate}
	Nous avons \( \Card(A)=S_n\) par le lemme \ref{LEMooVFPNooVmdUXY}\ref{ITEMooSWJCooEpBVkG}. Nous définissons une belle bijection comme il faut :
	\begin{equation}
		\begin{aligned}
			\alpha\colon \{ 1,\ldots, S_n \} & \to A                        \\
			i                                & \mapsto \varphi_{k+1}(i-S_k)
		\end{aligned}
	\end{equation}
	pour \( i\in\mathopen] S_k , S_{k+1} \mathclose]\).

	\begin{subproof}
		\spitem[\( \alpha\) est bien définie]
		Puisque \( i>S_k\) et \( i\leq S_{k+1}\) nous avons \( i-S_k\in \{ 1,\ldots, N_{k+1} \}\), et donc \( \varphi_{k+1}\) s'applique bien à \( i-S_k\).
		\spitem[\( \alpha\) est injective]
		Supposons que \( \alpha(i)=\alpha(j)\). Si \( i\in \mathopen] S_k , S_{k+1} \mathclose]\) et \( j\in \mathopen] S_l , S_{l+1} \mathclose]\), alors \( \alpha(i)=\varphi_{k+1}(i-S_k)\in A_{k+1}\) et \( \alpha(j)=\varphi_{l+1}(j-S_l)\in A_{l+1}\). Vu que les \( A_i\) sont disjoints, nous avons \( k=l\), et donc
		\begin{equation}
			\varphi_{k+1}(i-S_k)=\varphi_{k+1}(j-S_k).
		\end{equation}
		Étant donné que \( \varphi_{k+1}\) est injective, nous avons \( i-S_k=j-S_k\), ce qui montre que \( i=j\).
		\spitem[\( \alpha\) est surjective]
		Soit \( a\in A\). Il existe \( k\) tel que \( a\in A_k\). Nous avons donc un \( s\in\{ 1,\ldots, N_k \}\) tel que \( a=\varphi_k(s)\). En posant \( i=s+S_k\), nous avons bien \( a=\alpha(s+S_k)\) parce que \( s+S_k\in \mathopen] S_{k-1} , S_k \mathclose]\).
	\end{subproof}
	Vu que \( \alpha\) est une bijection, nous avons l'égalité
	\begin{equation}
		\sum_{a\in A}f(a)=\sum_{i=1}^{S_n}(f\circ \alpha)(i).
	\end{equation}

	Nous avons encore besoin d'introduire une bijection. Nous posons
	\begin{equation}
		\begin{aligned}
			\beta_k\colon \mathopen] S_{k-1} , S_k \mathclose] & \to A_k                       \\
			i                                                  & \mapsto \varphi_k(i-S_{k-1}).
		\end{aligned}
	\end{equation}
	C'est une bijection parce que \( \varphi_k\) en est une, et que \( i\mapsto i-S_{k-1}\) est une bijection de \( \mathopen] S_{k-1} , S_k \mathclose]\).

	Nous pouvons maintenant terminer :
	\begin{subequations}
		\begin{align}
			\sum_{a\in A}f(a) & =\sum_{i=1}^{S_n}(f\circ \alpha)(i)                                                                            \\
			                  & =\sum_{k=1}^n\left( \sum_{i=S_{k-1}-1}^{S_k}(f\circ \alpha)(i) \right)        \label{SUBEQooNVKWooZqBAau}      \\
			                  & =\sum_{k=1}^n\left( \sum_{i\in \mathopen] S_{k-1} , S_k \mathclose]}f\big( \varphi_k(i-S_{k-1}) \big)  \right) \\
			                  & =\sum_{k=1}^n\left( \sum_{i\in \mathopen] S_{k-1} , S_k \mathclose]}f\big( \beta_k(i) \big) \right)            \\
			                  & =\sum_{i=1}^n\left( \sum_{a\in A_k}f(a) \right).
		\end{align}
	\end{subequations}
	Justifications :
	\begin{itemize}
		\item Pour \eqref{SUBEQooNVKWooZqBAau}. Associativité de la somme.
	\end{itemize}
\end{proof}


\begin{proposition}[\cite{MonCerveau}]     \label{PROPooWJQQooFINSEc}
	Soient un groupe fini \( G\) et une fonction \( f\colon G\to A\) où \( A\) est un anneau commutatif. Alors
	\begin{equation}
		\sum_{g\in G}f(g)=\sum_{g\in G}f(gh)=\sum_{g\in G}f(hg)
	\end{equation}
	pour tout \( h\in G\).
\end{proposition}

\begin{proof}
	Nous avons une bijection \( \varphi\colon \{ 0,\ldots,  N \}\to G\) garantie par la proposition \ref{PROPooJLGKooDCcnWi}. Sa définition est
	\begin{equation}
		\sum_{g\in G}f(g)=\sum_{i=0}^Nf\big( \varphi(i) \big).
	\end{equation}
	Par ailleurs, le lemme \ref{LEMooBIBFooBHxFYC} donne une bijection \( L_h\colon G\to G\) et permet de considérer la composée
	\begin{equation}
		\begin{aligned}
			\varphi'\colon \{ 0,\ldots,  N \} & \to G \\
			\varphi'=L_h\circ \varphi.
		\end{aligned}
	\end{equation}
	La proposition \ref{DEFooLNEXooYMQjRo} nous permet d'utiliser la bijection \( \varphi'\) au lieu de \( \varphi\) pour exprimer la somme \( \sum_{g\in G}\). Ensuite un jeu de notation utilisant \eqref{EQooQQBEooFDOBVG} donne
	\begin{equation}
		\begin{aligned}[]
			 & \sum_{g\in G}f(g)=\sum_{i=0}^Nf\big( \varphi(i) \big)=\sum_{i=0}^Nf\big( \varphi'(i) \big)=\sum_{i=0}^N(f\circ L_h\circ \varphi)(i) \\
			 & \quad=\sum_{i=0}^N(f\circ L_h)\big( \varphi(i) \big)=\sum_{g\in G}(f\circ L_h)(g)=\sum_{g\in G}f(hg).
		\end{aligned}
	\end{equation}
	En ce qui concerne \( \sum_{g\in G}f(gh)\), c'est la même chose, en utilisant \( R_h\) au lieu de \( L_h\).
\end{proof}

\begin{lemma}       \label{LEMooKSVWooIFsfwm}
	Soit un groupe totalement ordonné\footnote{Définition \ref{DEFooEUHFooYvhnLQ}.} \( (A,+,\leq)\). Soient deux suites \( (a_i)\) et \( (b_i)\) dans \( G\) telles que \( a_i\leq b_i\) pour tout \( i\). Alors pour tout \( n\) nous avons
	\begin{equation}
		\sum_{i=0}^na_i\leq \sum_{i=0}^nb_i.
	\end{equation}
\end{lemma}

Tout cela nous permet de définir une somme sympathique et bien connue.
\begin{lemma}
	Soit \( n\in \eN\). Nous avons
	\begin{equation}
		\sum_{k=0}^nk=\frac{ n(n+1) }{ 2 }.
	\end{equation}
\end{lemma}

\begin{proof}
	La preuve est pratiquement immédiate par récurrence. Nous allons donner une preuve plus «constructive», qui formalise l'idée classique d'écrire la somme à l'endroit et à l'envers.


	Nous notons \( S\) la somme \( \sum_{k=0}^nk\). Le lemme \ref{DEFooLNEXooYMQjRo} dit que si les \( \sigma_i\colon \{ 0,\ldots, n \}\to \{ 0,\ldots, n \}\) sont des bijections, alors \( \sum_{k=0}^nf\big( \sigma_1(k) \big)=\sum_{k=0}^nf\big( \sigma_2(k) \big)\). Nous sommes intéressé au cas \( f(i)=i\).

	En prenant \( \sigma_1(k)=k\) et \( \sigma_2(k)=n-k\), nous avons
	\begin{equation}
		S=\sum_{k=0}^nk=\sum_{k=0}^n(n-k).
	\end{equation}
	Donc
	\begin{equation}
		2S=\sum_{k=0}^n\big( k+(n-k) \big)=\sum_{k=0}^nn=n\sum_{k=0}^n1=n(n+1).
	\end{equation}
	En divisant par deux, nous obtenons le résultat annoncé.
\end{proof}


%-------------------------------------------------------
\subsection{Symbole de produit}
%----------------------------------------------------

\begin{normaltext}      \label{NORMooDBOFooQCwbOY}
	Si \( (G,\cdot)\) est un groupe et si \( H\subset G\), nous notons le produit des éléments de \( H\) par
	\begin{equation}
		\prod_{g\in H }g=\sum_{g\in H}g
	\end{equation}
	où à droite, c'est la somme déjà définie. La différence entre \( \prod\) et \( \sum\) est que nous utilisons \( \prod\) pour les groupes notés «multiplicativement» comme \( (G,\cdot)\) alors que nous utilisons \( \sum\) lorsque le groupe est noté «additivement» comme \( (G,+)\).

	Dans le cas d'un anneau \( (A,+,\cdot)\), la distinction est importante pour savoir quelle opération est sous-entendue.

	La définition \ref{DEFooNEVNooJlmJOC}\ref{ITEMooIPDTooEhOxea} signifie qu'une somme vide vaut zéro : \( \sum_{x\in \emptyset}x=0\). Vu que zéro est la façon usuelle de noter le neutre pour une opération notée «\( +\)», lorsque l'opération est notée \( \cdot\) nous avons
	\begin{equation}        \label{EQooCSDSooTxdfzO}
		\prod_{x\in\emptyset}x=1
	\end{equation}
	parce que \( 1\) est la façon usuelle de noter le neutre d'une opération notée «\( \cdot\)».

	Notez que \eqref{EQooCSDSooTxdfzO} n'est pas une nouvelle définition ou une nouvelle convention. C'est seulement l'égalité \( \sum_{x\in\emptyset x}x=0\), avec des notations adaptées à un groupe dont l'opération est notée multiplicativement.
\end{normaltext}

\begin{proposition}     \label{PROPooQMUDooQQVRIe}
	Si \( E\) est un ensemble fini et si \( G\) est un groupe commutatif, alors pour toute fonction \( f\colon E\to G\) et pour toute permutation\footnote{Une permutation est une bijection, définition \ref{DEFooJNPIooMuzIXd}.} \( \sigma\) de \( E\),
	\begin{equation}
		\prod_{i\in E}f(i)=\prod_{i\in E}f\big( \sigma(i) \big)
	\end{equation}
\end{proposition}

\begin{proof}
	C'est exactement la proposition \ref{DEFooLNEXooYMQjRo}, sauf qu'ici la loi de groupe est notée multiplicativement au lieu d'additivement.
\end{proof}



%+++++++++++++++++++++++++++++++++++++++++++++++++++++++++++++++++++++++++++++++++++++++++++++++++++++++++++++++++++++++++++ 
\section{Anneaux}
%+++++++++++++++++++++++++++++++++++++++++++++++++++++++++++++++++++++++++++++++++++++++++++++++++++++++++++++++++++++++++++

Nous avons déjà parlé d'anneaux dans la définition \ref{DefHXJUooKoovob}.

\begin{definition}      \label{DEFooKWKGooIOwGTA}
	Un \defe{isomorphisme d'anneaux}{isomorphisme!d'anneaux} est un morphisme d'anneaux\footnote{Définition \ref{DEFooSPHPooCwjzuz}.}, bijectif.
\end{definition}

Soit \( X\) un ensemble et un anneau \( (A, +, \times)\). Nous considérons \( \Fun(X,A)\)\nomenclature[A]{\( \Fun(X,Y)\)}{les applications de \( X\) vers \( Y\)} l'ensemble des applications \( X\to A\). Cet ensemble devient un anneau avec les définitions
\begin{subequations}
	\begin{align}
		(f+g)(x)=f(x)+g(x) \\
		(fg)(x)=f(x)g(x).
	\end{align}
\end{subequations}
C'est la \defe{structure canonique}{structure d'anneau canonique} d'anneau sur \( \Fun(X,A)\).

\begin{definition}
	Le \defe{centralisateur}{centralisateur} de \( x\in A\) dans \( A\) est l'ensemble
	\begin{equation}
		\{ y\in A\tq xy=yx \},
	\end{equation}
	le \defe{centre}{centre!d'un anneau} de \( A\) est
	\begin{equation}
		\{ y\in A\tq xy=yx,\forall x\in A \}.
	\end{equation}
\end{definition}

%-------------------------------------------------------
\subsection{Idéal dans un anneau}
%----------------------------------------------------

\begin{definition}[Idéal dans un anneau]  \label{DefooQULAooREUIU}
	Un sous-ensemble \( I\subset A\) est un \defe{idéal à gauche}{idéal!dans un anneau} si
	\begin{enumerate}
		\item
		      \( I\) est un sous-groupe pour l'addition,
		\item
		      pour tout \( a\in A\), \( aI\subset I\).
	\end{enumerate}
	De même nous disons que \( I\subset A\) est une \defe{idéal à droite}{idéal à droite} lorsque \( I\) est un sous-groupe pour l'addition et \( Ia\subset I\) pour tout \( a\in A\).

	Lorsqu'un ensemble est idéal à gauche et à droite, nous disons que c'est un \defe{idéal bilatère}{idéal!bilatère}. Lorsque nous parlons d'idéal sans précision, nous parlons d'idéal bilatère.
\end{definition}

\begin{definition}[Idéaux premiers entre eux\cite{BIBooSALPooIrQwPd}]	\label{DEFooZFYSooJGZndS}
	Deux définitions.
	\begin{enumerate}
		\item
		      Deux idéaux \( I\) et \( J\) de l'anneau \( A\) sont \defe{premiers entre eux}{idéaux premiers entre eux} si \( I+J=A\).
		\item
		      Des idéaux \( I_1\),\ldots, \( I_n\) sont \defe{premiers dans leur ensemble}{idéaux premiers dans leur ensemble} si \( I_1+\ldots+I_n=A\).
	\end{enumerate}
\end{definition}


\begin{lemma}		\label{LEMooMAHXooXSowdn}
	Les seuls idéaux d'un corps sont \( \{ 0 \}\) et le corps lui-même.
	%TODOooTIZHooMPxRYF. Prouver ça.
\end{lemma}

%-------------------------------------------------------
\subsection{Idéal engendré}
%----------------------------------------------------


\begin{proof}
	Soient un corps \( \eK\) et un idéal \( I\) dans \( A\). Si \( I=\{ 0 \}\), c'est un idéal, pas de problèmes. Si \( I\neq\{ 0 \}\), alors \( 0\in I\) parce qu'un idéal doit contenir le neutre de l'addition.

	Soit \( x\neq 0\) dans \( I\). Alors pour tout \( a\in \eK\) nous avons \( ax\in I\). En particulier avec \( a=x^{-1}\) nous voyons que \( 1\in I\). De là, \( I=\eK\) parce que si \( x\in \eK\), nous avons \( x=x\cdot 1\in xI\subset I\).
\end{proof}

\begin{propositionDef}[Idéal engendré par un élément]  \label{DefSKTooOTauAR}
	Soit un anneau \( A\) et une partie \( S\subset A\). Nous notons \( (S)\) l'intersection de tous les idéaux de \( A\) contenant \( S\). C'est un idéal.

	Nous nommons \( (S)\)\nomenclature[A]{\( (S)\)}{idéal engendré par \( S\)} l'\defe{idéal engendré}{idéal engendré} par \( S\),
\end{propositionDef}

\begin{proof}
	Nous prouvons que \( (S)\) est un idéal. En deux parties.
	\begin{subproof}
		\spitem[Somme]
		%-----------------------------------------------------------
		Soient \( a,b\in (S)\). Nous devons prouver que \( a+b\in (S)\) c'est à dire que que si \( I\) est un idéal de \( A\), alors \( a+b\in I\). Soit donc un idéal \( I\). Vu que \( (S)\subset I\) nous avons \( a,b\in I\). Et comme \( I\) est un idéal, \( a+b\in I\).
		\spitem[\( a(S)\subset (S)\)]
		%-----------------------------------------------------------
		Soit \( a\in A\). Nous prouvons que \( a(S)\subset (S)\). Soit \( x\in (S)\) et un idéal \( I\) de \( A\). Nous avons \( x\in (S)\subset I\). Donc \( ax\in I\).
	\end{subproof}
\end{proof}

\begin{proposition}[\cite{BIBooKGBRooAujjRR,MonCerveau}]	\label{PROPooDTYUooJPzPZV}
	Soient un anneau \( A\) et une partie \( S\subset A\). L'idéal engendré par \( S\) est la partie
	\begin{equation}
		P=\{ \sum_{i=1}^ra_is_ib_i \tq a_i,b_i\in A, s_i\in S \}.
	\end{equation}
\end{proposition}

\begin{proof}
	Nous commençons par remarquer que \( P\) est un idéal. Ensuite nous prouvons
	\begin{subproof}
		\spitem[\( P\subset (S)\)]
		%-----------------------------------------------------------
		Il faut montrer que \( P\) est inclus à tous les idéaux contenant \( S\). Soit \( I\) un tel idéal. Soit un élément \( \sum_{i=1}^ra_is_ib_i\) dans \( P\). Vu que \( s_i\in S\subset I\), nous avons \( s_i\in I\) et donc \( a_is_ib_i\in I\).

		\spitem[\( (S)\subset P\)]
		%-----------------------------------------------------------
		Nous savons que \( P\) est un idéal contenant \( S\). Comme \( (S)\) est l'intersection de tous les idéaux contenant \( S\), en particulier \( (S)\subset P\).
	\end{subproof}
\end{proof}


\begin{proposition}[\cite{MonCerveau}]	\label{PROPooFDJXooYbXEpo}
	Soient un anneau \( A\) ainsi que \( a,b,x\in A\).
	\begin{enumerate}
		\item		\label{ITEMooVFXMooPljoHf}
		      La partie \( (a)+(b)\) est un idéal.
		\item	\label{ITEMooMXHAooHPnAdu}
		      Si \( a\in (x)\), alors \( (a)\subset (x)\).
	\end{enumerate}
\end{proposition}

\begin{proof}
	En deux parties.
	\begin{subproof}
		\spitem[Pour \ref{ITEMooVFXMooPljoHf}]
		%-----------------------------------------------------------
		Soient \( u,v\in (a)+(b)\). Nous avons \( u=u_1+u_2\) et \( v=v_1+v_2\) avec \( u_1,v_1\in (a)\) et \( u_2,v_2\in (b)\). Donc
		\begin{equation}
			u+v=(u_1+v_1)+(u_2+v_2)\in (a)+(b)
		\end{equation}
		parce que \( u_1+v_1\in (a)\) du fait que \( (a)\) est un idéal.
		\spitem[Pour \ref{ITEMooMXHAooHPnAdu}]
		%-----------------------------------------------------------
		Si \( s=\in (x)\), alors la proposition \ref{PROPooDTYUooJPzPZV} dit qu'il existe \( a,b\in A\) tels que \( s=axb\). Donc un élément général de \( (s)\) est de la forme \( \alpha s\beta\) avec \( \alpha,\beta\in A\). Mais \( \alpha s\beta=\alpha axb\beta\in (x)\).
	\end{subproof}
\end{proof}

\begin{lemma}[\cite{}]	\label{LEMooSEIAooYVIvPK}
	Soit un anneau \( A\).
	\begin{enumerate}
		\item
		      Toute union d'idéaux est un idéal.
		\item
		      Une somme finie d'idéaux est un idéal.
	\end{enumerate}
	%TODOooIXGMooXlWont. Prouver ça.
\end{lemma}


\begin{lemma}[\cite{MonCerveau}]	\label{LEMooHSVYooRjhGUU}
	Soient un anneau commutatif \( A\) ainsi que \( S\subset A\). Les parties
	\begin{equation}
		I=\bigcap_{s\in S}sA
	\end{equation}
	et
	\begin{equation}
		J=\bigcup_{\alpha\text{ fini dans }S}\sum_{s\in \alpha}sA
	\end{equation}
	sont des idéaux.
\end{lemma}

\begin{proof}
	Nous faisons pour \( J\). Soient \( x,y\in J\). Il existe \( \alpha,\beta\) finis dans \( S\) et des élements \( a_s,b_s\in A\) pour chaque \( s\in \alpha\cup\beta\) tels que \( x=\sum_{s\in \alpha}sa_s\) et \( y=\sum_{s\in\beta}sb_s\).

	Nous posons \( a_s=0\) quand \( s\not\in \alpha\) et \( b_s=0\) lorsque \( s\not\in\beta\) et ensuite \( \gamma=\alpha\cup\beta\). Nous avons alors
	\begin{equation}
		x+y=\sum_{s\in\gamma}s(a_s+b_s)\in J.
	\end{equation}

	De plus si \( a\in A\), alors \( xa=\sum_{s\in\alpha}sAa\subset \sum_{s\in A}sA\).
\end{proof}


\begin{proposition}[\cite{BIBooXLOMooVnXMbS, MonCerveau}]	\label{PROPooMMHPooZYzvdK}
	Soit un anneau commutatif \( A\). Soit une partie \( S\) de \( A\).
	\begin{enumerate}
		\item		\label{ITEMooJKGMooCqWYOq}
		      Un élément \( \mu\) est un ppcm de \( S\) si et seulement si il est générateur de l'idéal\footnote{Lemme \ref{LEMooHSVYooRjhGUU}.} \( I=\bigcap_{s\in S}sA\).
		\item		\label{ITEMooZUCVooRIpnhU}
		      Si \( \delta\) est un générateur de l'idéal \( J=\bigcup_{\alpha\text{ fini dans }S}\sum_{s\in \alpha}sA\), alors \( \delta\in\pgcd(S)\).
	\end{enumerate}
\end{proposition}

\begin{proof}
	En plusieurs parties.
	\begin{subproof}
		\spitem[Pour \ref{ITEMooJKGMooCqWYOq}]
		%-----------------------------------------------------------
		En deux parties.
		\begin{subproof}
			\spitem[\( \Rightarrow\)]
			%-----------------------------------------------------------
			Soit \( m\in \ppcm(S)\). Nous allons montrer que \( \mu A=I\), et ce sera suffisant par la proposition \ref{PROPooDTYUooJPzPZV} et le fait que \( A\) est commutatif.
			\begin{subproof}
				\spitem[\( \mu A\subset I\)]
				%-----------------------------------------------------------
				Vu que \( \mu\in \ppcm(S)\), pour tout \( s\in S\), il existe \( a_s\in A\) tel que \( \mu=s a_s\). Donc \( \mu A\subset sA\) pour tout \( s\in S\). Autrement dit, \( \mu A\subset \bigcup_{s\in S}sA=I\).
				\spitem[\( I\subset \mu A\)]
				%-----------------------------------------------------------
				Soit \( m\in I\). Donc \( m\) est un multiple de tout élément de \( S\). Vu que \( \mu\) est un ppcm de \( S\), il existe \( a\in A\) tel que \( m=\mu a\). Et donc \( m\in \mu A\), et donc \( I\subset \mu A\).
			\end{subproof}
			\spitem[\( \Leftarrow\)]
			%-----------------------------------------------------------
			Soit un générateur \( \mu\) de \( I\), c'est à dire que \( \mu A=I\).
			\begin{subproof}
				\spitem[\( S\divides \mu\).]
				%-----------------------------------------------------------
				Soit \( s\in S\). Vu que \( \mu\in I\), nous avons \( \mu\in\bigcap_{s\in S}sA\), et donc il existe \( a\in A\) tel que \( \mu =sa\), c'est à dire que \( s\divides \mu\). Bref, tout élément de \( S\) divise \( \mu\).
				\spitem[Si \( S\divides m\), alors \( \mu\divides m\)]
				%-----------------------------------------------------------
				Nous supposons que \( m\)  est un multiple de tout élément de \( S\). Donc pour tout \( s\in S\), il existe \( a_s\) tel que \( m=sa_s\). Donc \( m\in I\), et donc \( \mu\) divise \( m\).
			\end{subproof}
		\end{subproof}
		\spitem[Pour \ref{ITEMooZUCVooRIpnhU}]
		%-----------------------------------------------------------
		Nous considérons un générateur \( \delta\) de \( J\), et nous prouvons que \( \delta\in\pgcd(S)\).
		\begin{subproof}
			\spitem[\( \delta\) divise \( S\)]
			%-----------------------------------------------------------
			Soit \( s\in S\). Nous avons \( s\in J\) et donc il existe \( a_s\in A\) tel que \( \delta a_s=s\). Donc \( \delta\divides s\).

			\spitem[Si \( d\divides S\) alors \( d\divides \delta\)]
			%-----------------------------------------------------------
			Vu que \( \delta\in J\), il existe une partie fine \( \alpha\subset S\) et des \( b_s\in A\) tels que \( \delta=\sum_{s\in \alpha}sb_s\). En même temps, nous savons que \( s=da_s\). Donc
			\begin{equation}
				\delta=\sum_{s\in \alpha}sb_s=\sum_{s\in \alpha}da_sb_s=d\sum_{s\in \alpha}a_sb_s,
			\end{equation}
			c'est à dire que \( d\) divise \( \delta\).
		\end{subproof}
	\end{subproof}
\end{proof}

%-------------------------------------------------------
\subsection{Autres trucs sur les idéaux}
%----------------------------------------------------


\begin{propositionDef}      \label{PROPooGXMRooTcUGbi}
	Soit \( A\), un anneau, \( I\) un idéal bilatère\footnote{Définition~\ref{DefooQULAooREUIU}.} de \( A\). Nous considérons la relation d'équivalence \( x\sim y\) si et seulement si \( x-y\in I\). Sur le quotient\footnote{Définition \ref{DEFooRHPSooHKBZXl}.}
	\begin{equation}
		A/\sim=A/I,
	\end{equation}
	nous mettons les opérations
	\begin{enumerate}
		\item
		      \( [x]+[y]=[x+y]\)
		\item
		      \( [x][y]=[xy]\).
	\end{enumerate}
	Nous avons alors les résultats suivants :
	\begin{enumerate}
		\item       \label{ITEMooEJPEooRKAqmS}
		      Les opérations sont bien définies,
		\item       \label{ITEMooYBEGooTlHgNz}
		      l'ensemble \( A/I\), muni de ces opérations, est un anneau. Le neutre pour l'addition est \( [0]\), l'inverse de \( [a]\) est \( [-a]\) que nous noterons \( -[a]\).
		\item       \label{ITEMooLNRLooMkoWXZ}
		      la surjection canonique \( \pi\colon A\to A/I\) est un morphisme.
	\end{enumerate}
	Cet anneau est appelé \defe{anneau quotient}{anneau!quotient par un idéal}.
\end{propositionDef}

\begin{proof}
	En plusieurs parties.
	\begin{subproof}
		\spitem[Pour \ref{ITEMooEJPEooRKAqmS}]
		Nous savons que, par définition,
		\begin{equation}
			\bar x=\{ x+i\tq i\in I \}.
		\end{equation}
		Calculons le produit de représentants génériques de \( \bar x\) et de \( \bar y\) :
		\begin{equation}
			(x+i_1)(y+i_2)=xy+xi_2+yi_1+i_1i_2.
		\end{equation}
		Puisque \( I\) est un idéal, nous avons \( xi_2+yi_1+i_1i_2\in I\) et donc bien
		\begin{equation}
			(x+i_1)(y+i_2)\in \overline{ xy }.
		\end{equation}
		\spitem[Pour \ref{ITEMooYBEGooTlHgNz}]
		Il s'agit de vérifier les conditions de la définition \ref{DefHXJUooKoovob}.

		D'abord \( A/I\) est un groupe de neutre \( [0]\). En effet, vu que \( (A,+)\) est un groupe commutatif de neutre \( 0\), nous avons
		\begin{enumerate}
			\item Neutre : $[a]+[0]=[a+0]=[a]$.
			\item Associativité :
			      $[a]+([b]+[c])=[a]+[b+c]=[a+b+c]=[a+b]+[c]=\big( [a]+[b] \big)+[c]$.
			\item Inversibilité : l'inverse de \( [a]\) est \( [-a]\) parce que \( [a]+[-a]= [a-a]=[0] \).
		\end{enumerate}
		Nous pouvons noter \( -[a]\) l'élément \( [-a]\). Le groupe \( A/I\) est commutatif:
		\begin{equation}
			[a]+[b]=[a+b]=[b+a]=[b]+[a].
		\end{equation}
		Donc \( (A/I,+)\) est un groupe commutatif de neutre \( [0]\).

		L'associativité de \( A\) donne l'associativité dans \( A/I\) :
		\begin{equation}
			\big( [a][b] \big)[c]=[ab][c]=[abc]=[a][bc]=[a]\big( [b][c] \big).
		\end{equation}
		Et enfin pour la distributivité,
		\begin{equation}
			[a]\big( [b]+[c] \big)=[a][b+c]=[a(b+c)]=[ab+ac]=[ab]+[ac]=[a][b]+[a][c].
		\end{equation}
		Nous avons prouvé que \( A/I\) est un anneau de neutre \( [0]\) et d'unité \( [1]\).
		\spitem[Pour \ref{ITEMooLNRLooMkoWXZ}]
		Nous devons vérifier les trois conditions de la définition \ref{DEFooSPHPooCwjzuz}. Cela est immédiat parce que \( \pi(x)=[x]\).
	\end{subproof}
\end{proof}


%---------------------------------------------------------------------------------------------------------------------------
\subsection{Élément irréductible et premier}
%---------------------------------------------------------------------------------------------------------------------------

\begin{definition}[\cite{ooWBLYooLYwALS}]       \label{DEFooZCRQooWXRalw}
	Soit un anneau commutatif \( A\). Un élément \( p\in A\) est \defe{premier}{élément premier} si il est
	\begin{enumerate}
		\item
		      non nul,
		\item
		      non inversible,
		\item       \label{ITEMooPMTTooCVHPIm}
		      si \( p\) divise un produit \( ab\), alors il divise soit \( a\) soit \( b\) (ou les deux).
	\end{enumerate}
\end{definition}

\begin{definition}[Élément irréductible\cite{ooWUNIooXKxRya}]  \label{DeirredBDhQfA}
	Un élément d'un anneau commutatif est \defe{irréductible}{irréductible!dans un anneau} si il n'est ni inversible, ni le produit de deux éléments non inversibles. \index{polynôme irréductible}
\end{definition}

\begin{normaltext}
	Nous allons voir dans la section \ref{SECooSWGKooEeOZTO} que le concept d'élément irréductible n'est vraiment utile que dans le cas des anneaux intègres.
\end{normaltext}

\begin{example}
	Un corps n'a pas d'élément irréductible parce qu'à part zéro, tous les éléments sont inversibles. Mais \( 0\) n'est pas irréductible parce qu'il peut être écrit comme produit d'éléments non inversibles : \( 0=0\cdot 0\).
\end{example}

\begin{lemma}[\cite{MonCerveau}]		\label{LEMooJBUJooScsiGc}
	Si \( p\) est irréductible et si \( u\) est inversible, alors \( pu\) est irréductible.
\end{lemma}

\begin{proof}
	D'abord \( pu\) n'est pas inversible parce que \( p\) ne l'est pas.

	Ensuite supposons que \( pu=ab\). Vu que \( u\) est inversible, nous avons \( p=a(bu^{-1})\). Comme \( p\) est irréductible, soit \( a\), soit \( bu^{-1}\) est inversible.

	Si c'est \( a\), c'est gagnée. Sinon, soit \( k\) un inverse de \( bu^{-1}\) : \( bu^{-1}k=1\). Nous voyons que \( u^{-1}k\) est un inverse de \( b\). Donc \( b\) est inversible.
\end{proof}

\begin{proposition}     \label{PROPooKDWQooTtScrN}
	Les éléments irréductibles de l'anneau \( \eZ\) sont les nombres premiers\footnote{Nombre premier, définition \ref{DEFooZCRQooWXRalw}.}.
\end{proposition}

\begin{proof}
	Les seuls inversibles de \( \eZ\) sont \( \pm 1\).

	Si \( p\) est premier et \( p=ab\) avec \( a,b\in \eZ\), alors nous avons soit \( a=\pm 1\) soit \( b=\pm 1\). Donc \( p\) n'est pas le produit de deux éléments non inversibles.

	Dans le sens inverse, supposons que \( p\) soit irréductible dans \( \eZ\). D'abord \( p\) ne peut pas être \( \pm 1\) parce que \( \pm 1\) sont inversibles. Ensuite supposons que \( p=ab\). Vu que \( p\) est irréductible, nous avons \( a=\pm1\) ou \( b=\pm1\). Autrement dit, dans \( p=ab\), soit \( a\) soit \( b\) est un inversible.
\end{proof}

%---------------------------------------------------------------------------------------------------------------------------
\subsection{Anneau intègre}
%---------------------------------------------------------------------------------------------------------------------------

\begin{definition}[Éléments nilpotents, unipotents]  \label{DEFooHRRYooTmbUTH}
	On dit que \( a \in A \) est \defe{nilpotent}{nilpotent} si il existe \( n \in \eN \) tel que \( a^n = 0 \). Il est dit \defe{unipotent}{unipotent} si \( a-1\) est nilpotent, c'est-à-dire si \( (a-1)^n =0\) pour un certain \( n \in \eN \).
\end{definition}

\begin{definition}[Éléments inversibles]        \label{DEFooCIHVooAhpJxy}
	Un élément \( a \in A \) est dit \defe{inversible}{élément!inversible!dans un anneau} si il existe \( b \in A \) tel que \( ab = 1 \).

	L'ensemble \( U(A)\)\nomenclature[A]{\( U(A)\)}{ensemble des inversibles} des éléments inversibles de \( A\) est un groupe pour la multiplication. Nous notons \( A^*=A\setminus\{ 0 \}\).
\end{definition}

Conformément à la définition \ref{DiviseursAnneau} de diviseur, nous posons la définition suivante pour les diviseurs de zéro.
\begin{definition}[diviseur de zéro\cite{ooTNKJooSCSCZQ}]		\label{DEFooCIYLooFkhVOc}
	Un élément \( a\neq 0\) est un \defe{diviseur de zéro à gauche}{diviseur!de zéro} si il existe \( x\neq 0\) tel que \( ax=0\). L'élément \( a\) est un diviseur de zéro \defe{à droite}{diviseur!de zéro à droite} si il existe \( y\neq 0\) tel que \( ya=0\).

	Nous disons que \( a\) est un \defe{diviseur de zéro}{diviseur de zéro} si il est un diviseur de zéro à gauche ou à droite.
\end{definition}

\begin{propositionDef}[Anneau intègre\cite{MonCerveau}]           \label{DEFooTAOPooWDPYmd}
	Soit \( A\) un anneau non réduit à \( \{ 0 \}\). Les assertions suivantes sont équivalentes:
	\begin{enumerate}
		\item       \label{ITEMooMXMKooXMYpkN}
		      \( A\) ne possède pas de diviseurs de zéro\footnote{Définition \ref{DEFooCIYLooFkhVOc}.}.
		\item       \label{ITEMooLAJCooFwxXrV}
		      La règle du produit nul s'applique dans \( A\): pour tous \( a, b \in A \), si \( ab=0\), alors \( a = 0\) ou \( b = 0\).
		      \index{règle du produit nul}
		\item       \label{ITEMooQNTFooSRrVPK}
		      On peut simplifier par un même élément non-nul, deux expressions produit dans \( A\) qui sont égales: pour tous \( a, b, c \in A \) avec \( a \neq 0 \), si \( ab = ac \), alors \( b = c \).
	\end{enumerate}
	Un anneau non réduit à \( \{ 0 \}\) qui vérifie ces propriétés est dit \defe{intègre}{anneau intègre}.
\end{propositionDef}

\begin{proof}
	En trois implications.
	\begin{subproof}
		\spitem[\ref{ITEMooMXMKooXMYpkN} implique \ref{ITEMooLAJCooFwxXrV}]

		Si \( ab=0\) avec \( b\neq 0\) alors \( a\) est un diviseur de zéro. Vu que nous supposons que \( A\) n'a pas de diviseurs de zéros, \( a\) est nul. De même, si \( a\neq 0\) \( b\) devrait être nul.
		\spitem[\ref{ITEMooLAJCooFwxXrV} implique \ref{ITEMooQNTFooSRrVPK}]

		Si \( ab=ac\), alors \( a(b-c)=0\) et l'hypothèse dit que soit \( a=0\), soit \( b-c=0\). Donc si \( a\neq 0\), alors \( b-c=0\).
		\spitem[\ref{ITEMooQNTFooSRrVPK} implique \ref{ITEMooMXMKooXMYpkN}]
		Si \( A=\{ 0 \}\), le point \ref{ITEMooQNTFooSRrVPK} n'est pas applicable.

		Si \( a\neq 0\) et \( ax=0\), alors nous avons aussi \( ax=a\times 0\). Par propriété de simplification, \( x=0\). Donc \( a\) n'est pas un diviseur de zéro à gauche. Nous prouvons de la même façon qu'il n'y a pas de diviseurs de zéro à droite.
	\end{subproof}
\end{proof}

\begin{lemma}		\label{LEMooIKNMooMfvQnu}
	Un corps est un anneau intègre\footnote{Définition \ref{DEFooTAOPooWDPYmd}.}. Tout corps satisfait la règle du produit nul.
\end{lemma}

\begin{proof}
	Nous vérifions la définition \ref{DEFooTAOPooWDPYmd}, et nous nommons \( \eK\) le corps considéré.
	\begin{subproof}
		\spitem[Pour \ref{ITEMooMXMKooXMYpkN}]
		%-----------------------------------------------------------
		Soit \( a\in \eK\). Si \( ax=0\) avec \( x\neq 0\) alors en multipliant par \( x^{-1}\) nous trouvons \( a=0\). Donc \( a\) n'est pas un diviseur de zéro non nul.

		\spitem[Pour \ref{ITEMooLAJCooFwxXrV}]
		%-----------------------------------------------------------
		Idem à ce que nous venons de faire. Si dans \( ab=0\) l'un des deux est non nul, en multipliant par son inverse, nous trouvons que l'autre est nul.

		\spitem[Pour \ref{ITEMooQNTFooSRrVPK}]
		%-----------------------------------------------------------
		Si \( ab=ac\) avec \( a\neq 0\), alors il suffit de multiplier à gauche par \( a^{-1}\) (qui existe parce que nous sommes dans un corps) pour obtenir \( b=c\).
	\end{subproof}
\end{proof}
Conséquence : dans un corps nous avons toujours la règle du produit nul, et l'élément nul n'est jamais inversible.


\begin{proposition}[\cite{BIBooNVSKooJdnbyO}]		\label{PROPooZBTIooRhAhvg}
	Soit un anneau unitaire et intègre \( A\). Soit \( p\in A\).
	\begin{enumerate}
		\item		\label{ITEMooJWRYooHndNpV}
		      \( p\) est premier si et seulement si l'idéal \( pA\) est premier.
		\item		\label{ITEMooGHGCooRkJilg}
		      \( p\) est irréductible si et seulement si il n'existe pas d'idéal principal \( I\) tel que \( pA\subsetneq I\subsetneq A\).
	\end{enumerate}
\end{proposition}

\begin{proof}
	En plusieurs parties.
	\begin{subproof}
		\spitem[\ref{ITEMooJWRYooHndNpV}\( \Rightarrow\)]
		%-----------------------------------------------------------

		Supposons que \( p\) est premier. Soient \( a,b\in A\) tels que \( ab\in pA\). En particulier \( p\divides ab\), et \( p\) étant premier\footnote{Définition \ref{DEFooZCRQooWXRalw}.}, nous avons soit \( p\divides a\) soit \( p\divides b\). Supposons que \( p\divides a\). Il existe \( k\in A\) tel que \( pk=a\), et donc
		\begin{equation}
			a=pk\in pA.
		\end{equation}
		De même si \( p\divides b\) nous avons \( b\in pA\).

		\spitem[\ref{ITEMooJWRYooHndNpV}\( \Leftarrow\)]
		%-----------------------------------------------------------

		Nous supposons que l'idéal \( pA\) est premier, et nous prouvons que \( p\) est premier. Soient \( a,b\in A\) tels que \( p\divides ab\). Il existe \( k\in A\) tel que \( pk=ab\). Donc \( ab\in pA\). L'idéal \( pA\) étant premier, nus avons soit \( a\in pA\) soit \( b\in pA\). Donc soit \( a\divides a\) soit \( b\divides p\).

		\spitem[\ref{ITEMooGHGCooRkJilg}\( \Rightarrow\)]
		%-----------------------------------------------------------

		Supposons que \( p\) est irréductible. Soit un idéal principal \( I\) vérifiant \( pA\subset I\). Nous allons montrer que soit \( I=pA\) soit \( I=A\). Vu que \( I\) est principal, il existe \( a\in A\) tel que \( I=aA\). Nous avons \( pA\subset aA\), et en particulier \( p\in aA\). Notons \( b\in A\) un élément tel que \( ab=p\).

		Vu que \( p\) est irréductible, il n'est pas le produit de deux non inversibles. Donc soit \( a\) soit \( b\) est inversible.

		Si \( a\) est inversible, alors \( I=A\).

		Si \( b\) est inversible, alors \( a=pb^{-1}\), de telle sorte que \( a\in pA\). De ce fait \( I=pA\).

		\spitem[\ref{ITEMooGHGCooRkJilg}\( \Leftarrow\)]
		%-----------------------------------------------------------

		Enfin, nous supposons qu'il n'existe pas d'idéal principal \( I\) tel que \( pA\subsetneq I\subsetneq A\), et nous montrons que \( p\) est irréductible. Soient \( a,b\in A\) tels que \( p=ab\).

		Nous avons \( p\in aA\) et donc \( pA\subset aA\). Donc soit \( aA=pA\) soit \( aA=A \).

		Si \( aA=pA\), alors il existe \( k\in A\) tel que \( pk=a\). En multipliant l'égalité \( ab=p\) par \( k \) nous trouvons \( abk=pk=a\). Vu que \( A\) est intègre, nous pouvons simplifier par \( a\) et trouver \( bk=1\), de telle sorte que \( b\) soit inversible.


		Si \( aA=A\), alors \( a\) est inversible parce que \( 1\in A=aA\).
	\end{subproof}
\end{proof}

\begin{lemma}[\cite{MonCerveau}]		\label{LEMooSFHMooQoKsPV}
	Soient un anneau intègre \( A\), et une partie \( S\subset A\). Si un des \( \pgcd\) de \( S\) est inversible\footnote{Définition \ref{DEFooCIHVooAhpJxy}.}, alors ils le sont tous.
\end{lemma}

\begin{proof}
	Pour rappel, les pgcd d'une partie de \(A\) sont définis dans \ref{DefrYwbct}. Soit un \( \pgcd\) inversible de \( S\), ainsi qu'un autre \( \pgcd\) que nous nommons \( \delta'\). Vu que \( \delta'\) divise tous les éléments de \( S\), il est divisé par \( \delta\) : \( \delta\divides\delta'\). Réciproquement, \( \delta'\divides \delta\).

	Soient \( x\) et \( y\) définis par \( \delta=x\delta'\) et \( \delta'=y\delta\). Nous avons
	\begin{equation}
		\delta=x\delta'=xy\delta.
	\end{equation}
	Comme l'anneau \( A\) est intègre, nous pouvons simplifier par \( \delta\) et voir \( xy=1\), ce qui signifie que \( x\) et \( y\) sont inversibles. Donc si \( \delta\) est inversible, alors \( \delta'=y\delta\) est inversible.
\end{proof}

\begin{lemma}[\cite{MonCerveau,BIBooZXSRooOqGHBA}]		\label{LEMooMHZQooIcSNSf}
	Soient un anneau intègre, \( A\), une partie \( S\subset A\) et un élément \( a\in A\). Nous avons\footnote{Définition du pgcd: \ref{DefrYwbct}.}
	\begin{equation}
		\pgcd(aS)=a\pgcd(S).
	\end{equation}
\end{lemma}

\begin{proof}
	Deux inclusions à prouver.
	\begin{subproof}
		\spitem[\( \pgcd(aS)\subset a\pgcd(S)\)]
		%-----------------------------------------------------------
		Soit un pgcd \( \delta\) de \( aS\). Nous devons trouver un \( \delta'\in\pgcd(S)\) tel que \( \delta=a\delta'\). En termes de notations, nous notons \( S=\{ s_i \}_{i\in I}\). Pour chaque \( i\) nous avons \( \delta\divides as_i\) : il existe \( x_i\in A\) tel que
		\begin{equation}
			\delta x_i=as_i.
		\end{equation}
		Vu que \( a\) divise tous les éléments de \( aS\), il divise n'importe quel pgcd de \( aS\), et en particulier \( a\divides \delta\) : il existe \( \delta'\in A\) tel que \( \delta=a\delta'\). Nous montrons que \( \delta'\in\pgcd(S)\).

		Nous savons que \( \delta x_i=as_i\). En remplaçant \( \delta\) par \( a\delta'\), \( a\delta'x_i=as_i\). Vu que nous sommes dans un anneau intègre, nous pouvons simplifier par \( a\) (définition \ref{DEFooTAOPooWDPYmd}\ref{ITEMooQNTFooSRrVPK}) :
		\begin{equation}
			\delta'x_i=s_i.
		\end{equation}
		Donc \( \delta'\) divise tous les éléments de \( S\), et vérifie la première condition pour être un pgcd de \( S\). Pour la seconde condition, nous supposons que \( d\) divise tous les éléments de \( S\). Nous avons \( d\divides S\), donc \( ad\divides aS\). Et comme \( \delta\) est un pgcd de \( aS\), nous déduisons que \( ad\) divise \( \delta\). Il existe \( y\in A\) tel que
		\begin{equation}
			ady=\delta.
		\end{equation}
		Nous remplaçons \( \delta\) par sa valeur \( a\delta'\) : \( ady=a\delta'\). Encore une fois nous simplifions par \( a\) et nous trouvons \( dy=\delta'\), c'est-à-dire que \( \delta\) divise \( \delta'\).

		\spitem[\(a \pgcd(S)\subset \pgcd(aS)\)]
		%-----------------------------------------------------------

		Soit un pgcd \( \delta\) de \( S\). Nous voulons que \( a\delta\in \pgcd(aS)\). Vu que \( \delta\in\pgcd(S)\) nous avons \( \delta x_i=s_i\) pour tout \( i\), et donc aussi \( a\delta x_i=as_i\), de telle sorte que \( a\delta\) divise tous les éléments de \( aS\).

		Soit maintenant \( d\in A\) divisant tous les éléments de \( aS\). Nous devons prouver que \( d\divides a\delta\).

		\begin{subproof}
			\spitem[Travail préliminaire]
			%-----------------------------------------------------------

			Nous considérons \( \delta'\in \pgcd(aS)\). Vu que \( \delta\divides s\) pour tout \( s\in S\), nous avons aussi \( a\delta\divides as\) pour tout \( s\in S\). Comme \( \delta'\) est un est un pgcd de \( aS\), nous avons donc
			\begin{equation}
				a\delta\divides\delta'.
			\end{equation}
			Soit \( u\in A\) tel que \( \delta'=a\delta u\).

			En utilisant la première partie de la preuve, nous avons
			\begin{equation}
				\delta'\in\pgcd(aS)\subset a\pgcd(S).
			\end{equation}
			Donc il existe \( \delta_1\in\pgcd(S)\) tel que \( \delta'=a\delta_1\). En écrivant l'égalité \( \delta' =a\delta u\) avec cette valeur de \( \delta'\), nous trouvons
			\begin{equation}		\label{EQooVHSSooDdVUeW}
				a\delta_1=a\delta u,
			\end{equation}
			et donc \( \delta_1=\delta u\) parce que \( A\) est intègre. Vu que \( \delta_1\in\pgcd(S)\), nous avons aussi \( \delta u\in\pgcd(S)\). En particulier \( \delta u\) divise tous les éléments de \( S\), et donc divise \( \delta\) qui est un pgcd de \( S\) : \( \delta u\divides \delta\). En multipliant par \( a\),
			\begin{equation}
				\delta'=a\delta u\divides a \delta.
			\end{equation}

			\spitem[Résumé]
			%-----------------------------------------------------------
			Nous avons considéré \( \delta\in\pgcd(S)\) et nous sommes en train de prouver que \( a\delta\in\pgcd(aS)\). Nous avons déjà prouvé que si \( \delta'\in \pgcd(aS)\), alors nous avons \( \delta'\divides a\delta\).

			Nous posons \( y\in A\) tel que \( \delta' y=a\delta\).

			\spitem[Et enfin]
			%-----------------------------------------------------------
			Soit \( d\) divisant tous les éléments de \( a\delta\). Donc \( d\divides \delta'\) : il existe \( x\in A\) tel que \( dx=\delta'\). En multipliant par \( y\),
			\begin{equation}
				dxy=\delta' y=a\delta.
			\end{equation}
			Nous avons montré que \( d\) divise \( a\delta\), ce qu'il nous fallait.
		\end{subproof}
	\end{subproof}
\end{proof}

\begin{lemma}[\cite{BIBooSJZQooOytVhm}]		\label{LEMooZSUNooUmYmgt}
	Soient un anneau intègre \( A\) et une partie \( S\subset A\). Si \( \delta\in\pgcd(S)\), alors \( \pgcd(S/\delta)\) ne contient que des inversibles.
\end{lemma}

\begin{proof}
	En utilisant le lemme \ref{LEMooMHZQooIcSNSf}, nous avons
	\begin{equation}
		\pgcd(S)=\pgcd\big( \delta(S/\delta) \big)=\delta\pgcd(S/\delta).
	\end{equation}
	Soit \( d\in\pgcd(S/\delta)\). Notre but est de montrer que \( d\) est inversible. D'abord
	\begin{equation}
		\delta d\in \delta\pgcd(S/\delta)=\pgcd(S).
	\end{equation}
	Donc \( \delta d\) divise tous les éléments de \( S\). Étant donné que \( \delta\) est un pgcd de \( S\), la définition \ref{DefrYwbct}\ref{ITEMooVCKGooWDXZOj} nous dit que \( \delta d\) divise \( \delta\). Il existe donc \( d'\in A\) tel que \( \delta dd'=\delta\), c'est à dire
	\begin{equation}
		\delta(dd'-1)=0.
	\end{equation}
	Étant donné que \( \delta\neq 0\) et que \( A\) est intègre nous avons \( dd'-1=0\) (définition \ref{DEFooTAOPooWDPYmd}\ref{ITEMooLAJCooFwxXrV}), c'est-à-dire \( dd'=1\). Cela montre que \( d'\) est un inverse de \( d\), et donc que \( d\) est inversible.
\end{proof}

\begin{lemma}[\cite{MonCerveau}]		\label{LEMooZKASooKstTuK}
	Soit un anneau intègre \( A\). Si \( \delta\in\pgcd(S)\) et si \( u\in A\) est inversible, alors \( \delta u\in\pgcd(S)\).
\end{lemma}

\begin{proof}
	Soit \( s\in S\). Si \( \delta x=s\), alors \( u\delta (u^{-1} x)=s\). Donc \( u\delta\) divise tous les éléments de \( S\). De plus si \( d\divides S\), alors \( d\divides \delta\). Dans ce cas il existe \( y\) tel que \( dy=\delta\). Nous avons alors aussi
	\begin{equation}
		dxu=\delta u,
	\end{equation}
	de telle sorte que \( d\) divise \( \delta u\).
\end{proof}


%--------------------------------------------------------------------------------------------------------------------------- 
\subsection{Fonction puissance}
%---------------------------------------------------------------------------------------------------------------------------

Voici une première définition de la fonction puissance. Il y en aura d'autres, de plus en plus générales. Voir le thème \ref{THEMEooBSBLooWcaQnR}.
\begin{definition}\label{DEFooGVSFooFVLtNo}
	Si \( A\) est un anneau, si \( a\in A\) et si \( n\in \eN\), nous définissons \( a^n\) par récurrence :
	\begin{enumerate}
		\item
		      \( a^0=1\) (l'unité pour la multiplication dans \( A\)),
		\item       \label{ITEMooOUIPooGjAgQb}
		      \( a^{k+1}=a\cdot a^{k}\).
	\end{enumerate}
	Si vous n'êtes pas \randomGender{sûr}{sure} de vous, ne citez pas le théorème \ref{THOooEJPYooZFVnez}. Il est indispensable pour faire fonctionner cette définition, mais vous pouvez faire comme si vous n'avez rien vu.
\end{definition}

Le lemme suivant dit que le point \ref{ITEMooOUIPooGjAgQb} de la définition \ref{DEFooGVSFooFVLtNo} aurait pu être écrit \( a^k\cdot a\) au lieu de \( a\cdot a^k\).
\begin{lemma}[\cite{MonCerveau}]        \label{LEMooWPARooYLZlzr}
	Si \( A\) est un anneau, si \( a\in A\) et si \( n\in \eN\), alors
	\begin{equation}
		a^n=a\cdot a^{n-1}=a^{n-1}\cdot a.
	\end{equation}
\end{lemma}

\begin{proof}
	Cela se prouve par récurrence. Pour \( n=1\) c'est l'égalité \( a=a^0a\) qui est correcte parce que par définition \( a^0=1\).

	Supposons que le résultat soit bon pour \( n\) et voyons ce que ça donne pour \( n+1\) :
	\begin{subequations}
		\begin{align}
			a^{n+1} & =aa^n        & \text{Définition de } a^{n+1}            \\
			        & =a(a^{n-1}a) & \text{hypothèse de récurrence pour } a^n \\
			        & =(aa^{n-1})a & \text{associativité}                     \\
			        & =a^na        & \text{Définition de } a^n.
		\end{align}
	\end{subequations}
\end{proof}

\begin{proposition}[\cite{MonCerveau}]	\label{PROPooPLLSooUpiLKa}
	Soit un naturel \( b\in\eN\setminus\{ 0 \}\). Soit un naturel \( m>0\). Il existe \( n\in \eN\) tel que \( b^n>m\).
	%TODOooDFWXooMCkcsJ. Prouver ça.
\end{proposition}



%-------------------------------------------------------
\subsection{Anneau euclidien}
%----------------------------------------------------


\begin{definition}[\wikipedia{fr}{Anneau_euclidien}{Wikipédia}] \label{DefAXitWRL}
	Soit \( A\) un anneau intègre\footnote{Défnition \ref{DEFooTAOPooWDPYmd}.}. Un \defe{stathme euclidien}{stathme euclidien} sur \( A\) est une application \( \alpha\colon A\setminus\{ 0 \}\to \eN\) tel que
	\begin{enumerate}
		\item       \label{ITEMooLVJAooLpjgEz}
		      \( \forall a,b\in A\setminus\{ 0 \}\), il existe \( q,r\in A\) tel que
		      \begin{equation}
			      a=qb+r
		      \end{equation}
		      et \( \alpha(r)<\alpha(b)\).
		\item
		      Pour tout \( a,b\in A\setminus\{ 0 \}\), \( \alpha(b)\leq \alpha(ab)\).
	\end{enumerate}
	Un anneau est \defe{euclidien}{euclidien!anneau} si il accepte un stathme euclidien.
\end{definition}
Le stathme est la fonction qui donne le «degré» à utiliser dans la division euclidienne. La contrainte est que le degré du reste soit plus petit que le degré du dividende.

\begin{lemma}       \label{LEMooFUSTooDCcBDb}
	L'ensemble \( \eZ\) avec les opérations usuelles est un anneau intègre\footnote{Anneau intègre, définition \ref{DEFooTAOPooWDPYmd}.}.
\end{lemma}

\begin{example} \label{ExwqlCwvV}
	Le stathme de \( \eN\) pour la division euclidienne usuelle est \( \alpha(n)=n\). Si \( a,b\in \eN\) nous écrivons
	\begin{equation}
		a=qb+r
	\end{equation}
	où \( q\) est l'entier le plus proche \emph{inférieur} à \( a/b\) (on veut que le reste soit positif) et \( r=a-qb\). Nous avons donc
	\begin{equation}
		r-b=a-b(q+1)<a-b\frac{ a }{ b }=0,
	\end{equation}
	ce qui montre que \( r<b\).
\end{example}

Cet exemple ne fonctionne pas avec \( \eZ\) au lieu de \( \eN\) parce que le stathme doit avoir des valeurs dans \( \eN\). Cela ne veut cependant pas dire qu'il n'existe pas de stathme sur \( \eZ\); cela veut seulement dire que \( \alpha(x)=x\) ne fonctionne pas.



%-------------------------------------------------------
\subsection{Anneau principal et idéal premier}
%----------------------------------------------------


\begin{definition}[Idéal principal\cite{BIBooAGIQooBCRVhe}]      \label{DEFooMZRKooBPLAWH}
	Soit un idéal\footnote{Idéal, définition \ref{DefooQULAooREUIU}.} \( I\) dans \( A\).
	\begin{enumerate}
		\item
		      Il est \defe{principal à gauche}{idéal!principal!à gauche} si il existe \( a\in I\) tel que \( I= A a\).
		\item
		      Il est \defe{principal à droite}{idéal!principal!à droite} si il existe \( a\in I\) tel que \( I=a A\).
		\item
		      Nous disons qu'il est \defe{principal}{principal!idéal} bilatère si il existe \( a\in I\) tel que
		      \begin{equation}
			      I=\{ \sum_{i=1}^rx_iay_i\tq x_i,y_i\in A \}.
		      \end{equation}
	\end{enumerate}
\end{definition}

\begin{definition}          \label{DEFooGWOZooXzUlhK}
	Un anneau est \defe{principal}{principal!anneau} si
	\begin{enumerate}
		\item
		      il est commutatif et intègre
		\item
		      tous ses idéaux\footnote{Idéal, définition \ref{DEFooMZRKooBPLAWH}.} sont principaux\footnote{Définition \ref{DEFooMZRKooBPLAWH}.}.
	\end{enumerate}
\end{definition}

Souvent pour prouver qu'un anneau est principal, nous prouvons qu'il est euclidien (définition~\ref{DefAXitWRL}) et nous utilisons la proposition~\ref{Propkllxnv} qui dit qu'un anneau euclidien est principal.

Une manière de prouver qu'un anneau n'est pas principal est de prouver qu'il n'est pas factoriel, théorème~\ref{THOooANCAooBChmwp}.


\begin{proposition}[\cite{ooELVSooZIZCRn}]\label{Propkllxnv}
	Tout anneau euclidien\footnote{Euclidien, définition \ref{DefAXitWRL}.} est principal\footnote{Principal, définition \ref{DEFooMZRKooBPLAWH}}.
\end{proposition}

\begin{proof}
	Soit \( A\) un anneau euclidien et \( \alpha\) un stathme sur \( A\). Nous considérons un idéal \( I\) non nul de \( A\). Nous devons montrer que \( I\) est généré par un élément. En l'occurrence nous allons montrer qu'un élément \( a\in I\setminus\{ 0 \}\) qui minimise \( \alpha(a)\) va générer\footnote{Un tel élément existe\dots} \( I\).

	Soit \( x\in I\). Par construction, il existe \( q,r\in A\) tels que \( x=aq+r\) avec \( r=0\) ou \( \alpha(r)<\alpha(a)\). Étant donné que \( x,a\in I\), \( r\in I\). Si \( r\neq 0\), alors \( r\) contredirait la minimalité de \( \alpha(a)\). Donc \( r=0\) et \( x=aq\), ce qui signifie que \( I\) est principal.
\end{proof}

\begin{proposition}     \label{PROPooPJGLooQSrJTU}
	L'anneau \( \eZ\) est principal\footnote{Définition \ref{DEFooGWOZooXzUlhK}.} et euclidien\footnote{Définition \ref{DefAXitWRL}.}.
\end{proposition}

\begin{proof}
	Nous allons seulement montrer que \( \alpha(x)=| x |\) est un stathme euclidien. Ainsi \( \eZ\) sera euclidien et donc principal par la proposition~\ref{Propkllxnv}.

	D'abord \( \eZ\) est intègre, c'est le lemme \ref{LEMooFUSTooDCcBDb}.

	La condition \( \alpha(b)\leq \alpha(ab)\) est immédiate : \( | a |\leq | ab |\) pour tout \( a,b\in \eZ\).

	Soient maintenant \( a,b\in \eZ\). Nous définissons \( q_0,r_0\in \eN\) tels que
	\begin{equation}
		| a |=q_0| b |+r_0
	\end{equation}
	avec \( r_0<| b |\). Cela existe parce que \( \alpha(x)=x\) est un stathme sur \( \eN\) par l'exemple~\ref{ExwqlCwvV}.

	\begin{subproof}
		\spitem[Si \( a>0\) et \( b>0\)]

		Alors \( a=q_0b+r_0\) et le couple \( (q_0,r_0)\) vérifie les conditions de la définition~\ref{DefAXitWRL}\ref{ITEMooLVJAooLpjgEz}.

		\spitem[Si \( a>0\) et \( b<0\)]

		Alors \( a=-q_0b+r_0\), et le couple \( (-q_0,r_0)\) vérifie les conditions de la définition~\ref{DefAXitWRL}\ref{ITEMooLVJAooLpjgEz}.


		\spitem[Si \( a<0\) et \( b>0\)]
		Alors \( a=-q_0b-r_0\), et le couple \( (-q_0,-r_0)\) vérifie les conditions de la définition~\ref{DefAXitWRL}\ref{ITEMooLVJAooLpjgEz} parce que
		\begin{equation}
			\alpha(-r_0)=r_0<| b |=\alpha(b).
		\end{equation}
		\spitem[Si \( a<0\) et \( b<0\)]
		Alors \( a=q_0b-r_0\), et le couple \( (q_0,-r_0)\) vérifie les conditions de la définition~\ref{DefAXitWRL}\ref{ITEMooLVJAooLpjgEz}.

	\end{subproof}
\end{proof}



%+++++++++++++++++++++++++++++++++++++++++++++++++++++++++++++++++++++++++++++++++++++++++++++++++++++++++++++++++++++++++++
\section{Le groupe et anneau des entiers}
%+++++++++++++++++++++++++++++++++++++++++++++++++++++++++++++++++++++++++++++++++++++++++++++++++++++++++++++++++++++++++++

Certes \( (\eZ,+)\) est un groupe mais en ajoutant la multiplication, \( (\eZ,+,\times)\) devient un anneau\footnote{Définition~\ref{DefHXJUooKoovob}.}.

%---------------------------------------------------------------------------------------------------------------------------
\subsection{Division euclidienne}
%---------------------------------------------------------------------------------------------------------------------------

\begin{theoremDef}[Division euclidienne\cite{ooRBKHooQJqglH}]     \label{ThoDivisEuclide}
	Soient \( a\in\eZ\) et \( b\in\eN^*\). Il existe un unique couple \( (q,r)\in\eZ\times\eN\), avec \( 0\leq r<b\), tel que
	\begin{equation}
		a=bq+r.
	\end{equation}
	L'opération \( (a,b)\mapsto(q,r)\) ainsi définie est la \defe{division euclidienne}{division!euclidienne}. Le nombre \( q\) est le \defe{quotient}{quotient} et \( r\) est le \defe{reste}{reste} de la division de \( a\) par \( b\).
\end{theoremDef}

\begin{proof}
	Remarquons que \( r = a - bq \), et donc, une fois l'existence et l'unicité de \( q\) établie, celle de \( r\) suivra.

	\begin{subproof}
		\spitem[Unicité]
		Nous supposons avoir \( (q,r)\in \eZ\times \eN\) tels que
		\begin{subequations}
			\begin{numcases}{}
				0\leq r<b\\
				a=qb+r.
			\end{numcases}
		\end{subequations}
		Ce système implique que
		\begin{equation}
			0\leq a-qb<b.
		\end{equation}
		En ajoutant \( qb\) dans les trois membres de cette inégalité,
		\begin{equation}
			qb\leq a<(q+1)b.
		\end{equation}
		Cela implique que
		\begin{equation}
			q=\max\{ k\in \eZ\tq kb\leq a \}.
		\end{equation}
		Donc \( q\) est unique et la relation \( a=bq+r\) implique que \( r\) est également unique.

		Soit
		\begin{equation*}
			E = \{ q \in \eZ  | bq \leq a \}.
		\end{equation*}
		La partie \( E\) est non vide (parce qu'elle contient \( -|a| \)) et admet un majorant : l'élément \( |a| \).  Elle admet donc un maximum \( q\) par le lemme \ref{LEMooMYEIooNFwNVI}. Ce maximum vérifie
		\begin{equation}
			bq\leq a<b(q+1).
		\end{equation}
		Cela donne \( 0\leq a-bq<b\) et le résultat, en posant \( r=a-qb\).
	\end{subproof}
\end{proof}


% TODO : À propos de restes, il n'est peut-être pas mal de parler d'algorithme de calcul de la date de pâques.
% L'algorithme de Gauss, Meeus utilise des arrondis.
% http://fr.wikipedia.org/wiki/Calcul_de_la_date_de_Pâques

Le lemme suivant est souvent pris pour la définition d'un nombre premier lorsqu'on parle de \( \eN\) ou \( \eZ\).
\begin{lemma}[\cite{frwiki179832418, MonCerveau}]
	Dans \( \eN\), un nombre est premier si et seulement si il admet exactement deux diviseurs entiers distincts.
\end{lemma}

\begin{proof}
	En deux parties.
	\begin{subproof}
		\spitem[\( \Rightarrow\)]
		Soit un élément premier \( p\in \eN\). Il y a trois possibilités : \( p=0\), \( p=1\) et \( p>1\).

		Le nombre \( p=0\) n'est pas premier parce qu'il est nul. Le nombre \( p=1\) n'est pas premier parce qu'il est inversible. Donc nous savons que si \( p\) est premier, alors \( p>1\).

		Un élément \( p>1\) dans \( \eN\) a toujours au moins deux diviseurs distincts : \( 1\) et \( p\). Soit un diviseur \( k\) de \( p\). Il existe \( l\in \eN\) tel que \( p=kl\). Vu que \( p\) est premier et divise le produit \( kl\), il divise \( k\) ou \( l\). Disons que \( p\) divise \( k\). De cette façon \( p\) divise \( k\) et \( k\) divise \( p\).

		Il existe donc \( n\in \eN\) tel que \( k=np\). En y substituant \( p=kl\), on trouve \( k=np=nkl\). En simplifiant par \( k\), il vient
		\begin{equation}
			1=nl,
		\end{equation}
		ce qui prouve que \( n=l=1\) et donc que \( k=p\) et donc que \( p\) n'a pas d'autres diviseurs que \( 1\) et \( p\).

		\spitem[\( \Leftarrow\)]
		Nous supposons que \( p\in \eN\) ait exactement deux diviseurs entiers distincts. Nous vérifions que \( p\) vérifie les trois conditions de la définition \ref{DEFooZCRQooWXRalw}.

		\begin{enumerate}
			\item
			      \( p\neq 0\) parce que \( 0\) a nettement plus que deux diviseurs distincts.
			\item
			      \( p\neq 1\) parce que \( 1\) a exactement un diviseur. Donc \( p\) n'est pas inversible dans \( \eN\).
			\item
			      Soit \( p\) admettant exactement deux diviseurs distincts. Soit \( p\) divisant le produit \( ab'\) pour certains \( a\) et \( b'\) dans \( \eN\). Nous supposons que \( p\) ne divise pas \( a\), et nous allons prouver que \( p\) divise \( b'\) en supposant d'abord que \( p\) ne divise pas \( b'\).

			      \begin{subproof}
				      \spitem[Un ensemble]
				      Pour cela nous posons
				      \begin{equation}
					      E=\{ x\in \eN\tq p\divides ax, p\notdivides x  \}.
				      \end{equation}
				      Nous posons \( b=\min(E)\). Nous avons pour hypothèse que \( E\) est non vide; en particulier \( 0<b\).
				      \spitem[\( b<p\)]
				      On vérifie que si \( p+k\in E\) alors \( k\in E\). Donc \( b\) ne peut pas être plus grand que \( p\). Vu que \( p\) lui-même n'est pas dans \( E\), nous avons \( b<p\).
				      \spitem[Division euclidienne]
				      Nous effectuons la division euclidienne du théorème \ref{ThoDivisEuclide} :
				      \begin{equation}
					      p=mb+r.
				      \end{equation}
				      En multipliant par \( a\), \( ar=ap-mab\). Vu que \( ab\) est un multiple de \( p\) \( ap-mab\) est un multiple de \( p\). En particulier \( ar\) est divisible en \( p\).
				      \spitem[La contradiction]
				      Nous avons donc \( r\in E\), alors que \( r<b\). Impossible.
			      \end{subproof}
		\end{enumerate}
	\end{subproof}
\end{proof}

\begin{proposition}[\cite{ooTGPAooQTbamu}]     \label{PROPooWMNPooZdvOBt}
	Dans un anneau intègre\footnote{Si pas intègre, voir l'exemple \ref{EXooEIUEooCZCPMC}.} tout élément premier est irréductible\footnote{Toutes les définitions dans le thème \ref{THEMEooVIQIooOcFAQS}.}.
\end{proposition}

\begin{proof}
	Soit \( p\), un élément premier dans un anneau intègre \( A\).
	\begin{subproof}
		\spitem[\( p\) n'est pas inversible]
		Cela fait partie de la définition d'un élément premier.
		\spitem[\( p\) n'est pas un produit d'inversibles]
		Soient \( a,b\in A\) tels que \( p=ab\). Par le point \ref{ITEMooPMTTooCVHPIm} de la définition \ref{DEFooZCRQooWXRalw}, \( p\) divise soit \( a\) soit \( b\). Supposons que \( p\) divise \( a\). Alors il existe \( x\in A\) tel que \( a=px\). En remettant dans \( p=ab\) nous avons :
		\begin{equation}        \label{EQooPYBGooLFHMJZ}
			p=pxb.
		\end{equation}
		Mais l'anneau est intègre et permet donc des simplifications par tout élément non nul. La relation \ref{EQooPYBGooLFHMJZ} donne donc
		\begin{equation}
			1=xb,
		\end{equation}
		ce qui signifie que \( b\) est inversible.

		Un travail similaire montre que \( a\) est inversible si \( p\) divise \( b\).
	\end{subproof}
\end{proof}

\begin{example}
	Si nous avons l'égalité \( 7=ab\) dans \( \eZ\), alors soit \( a\) soit \( b\) vaut \( 1\). Mettons \( a=1\). Dans ce cas, \( b=7\) et n'est donc pas inversible.
\end{example}

Sur un anneau non intègre, la notion d'élément premier n'est pas aussi intéressante que sur un anneau intègre. Par exemple la proposition \ref{PROPooWMNPooZdvOBt} devient fausse.

\begin{example}     \label{EXooEIUEooCZCPMC}
	Soit l'anneau \( \eZ^2\). L'élément \( (1,0)\) est premier mais pas irréductible.
	\begin{subproof}
		\spitem[\( (1,0)\) est premier]
		L'élément \( (1,0)\) est non nul; ça c'est pas cher. Pour qu'il soit inversible, il faudrait \( (1,0)(x,y)=(1,1)\). Entre autres, \( 0\times y=1\), ce qui est impossible. Donc il n'est pas inversible.

		Supposons que \( (1,0)\) divise le produit \( (a,b)(c,d)=(ac,bd)\). Alors il existe \( (x,y)\) tel que \( (1,0)(x,y)=(ac,bd)\). Cela signifie que \( x=ac\) et \( 0\times y=bd\). En particulier, soit \( b=0\) soit \( d=0\). Si \( b=0\), nous avons \( (a,b)=(a,0)\) et effectivement, \( (1,0)\) le divise.
		\spitem[\( (1,0)\) n'est pas irréductible]
		Nous avons \( (1,0)=(1,0)(1,0)\). Donc l'élément \( (1,0)\) est le produit de deux éléments non inversibles.
	\end{subproof}
\end{example}


%---------------------------------------------------------------------------------------------------------------------------
\subsection{Sous-groupes de \texorpdfstring{\( (\eZ,+)\)}{(Z,+)}}
%---------------------------------------------------------------------------------------------------------------------------

\begin{proposition}[liste des sous groupes de \( \eZ\)] \label{PropSsgpZestnZ}
	À propos de sous-groupes de \( \eZ\).
	\begin{enumerate}
		\item
		      Une partie \( H\) du groupe \( (\eZ,+)\) est un sous-groupe si et seulement si il existe \( n\in\eN\) tel que \( H=n\eZ\).
		\item       \label{ITEMooOWNZooUsYRok}
		      Si \( H\) est un sous-groupe de \( (\eZ,+)\), il existe un unique \( n\) tel que \( H=n\eZ\).
	\end{enumerate}
\end{proposition}

\begin{proof}
	Soit \( H\neq\{ 0 \}\) un sous-groupe de \( \eZ\). L'ensemble \( H\cap\eN^*\) contient un élément minimum que nous notons \( n\). Nous avons certainement \( n\eZ\subset H\) parce que \( H\) est un groupe (donc \( n+n\) et \( -n\) sont dans \( H\) dès que \( n\) est dans \( H\)). Nous devons prouver que \( H\subset n\eZ\).

	Si \( x\in H\), par le théorème de division euclidienne~\ref{ThoDivisEuclide}, il existe \( q\in\eZ\) et \( r\in\eN \), uniques, tels que \( x=nq+r\) et \(0 \leq r < n \). Nous savons déjà que \( nq\in H\), donc \( r = x - nq \in H \). Le nombre \( r\) est donc un élément de \( H\) strictement plus petit que \( n\). Mais nous avions décidé que \( n\) serait le plus petit élément de \( H\cap\eN^*\). Par conséquent \( r=0\) et \( x=nq\in n\eZ\).


	En ce qui concerne l'unicité, supposons que \( n\eZ=m\eZ\). Le nombre \( n\) divise \( m\) (parce que \( m\in m\eZ\subset n\eZ\)) et le nombre \( m\) divise \( n\) parce que \( n\in m\eZ\). Par conséquent \( n=m\).
\end{proof}


%--------------------------------------------------------------------------------------------------------------------------- 
\subsection{Théorème de Bézout}
%---------------------------------------------------------------------------------------------------------------------------


Une généralisation de Bézout \ref{ThoBuNjam} à plus de \( 2\) variables et dans un anneau principal : proposition \ref{PROPooXQKMooWJlEFq}.

\begin{theorem}[Théorème de Bézout\footnote{Il y a une super application ici : \url{https://perso.univ-rennes1.fr/matthieu.romagny/agreg/dvt/mauvais_prix.pdf}.}\cite{LSAmvR}, thème~\ref{THEMEooNRZHooYuuHyt}] \label{ThoBuNjam}
	Si \( \{ a_i \}_{i=1,\ldots, N}\) sont des entiers non nuls. Nous avons \( \pgcd(a1,\ldots,a_N)=1\) si et seulement si il existe \( u_1,\ldots,u_N\) tels que
	\begin{equation}
		\sum_ia_iu_i=1.
	\end{equation}
\end{theorem}

\begin{proof}
	En deux parties.
	\begin{subproof}
		\spitem[\( \Rightarrow\)]
		%-----------------------------------------------------------
		Soit \( E=\{ \sum_{i=1}^Na_iu_i\tq u_i\in \eZ \}\). Cet ensemble est non vide. Il contient par exemple \( a_1\). Nous posons \( m=\min\big( E\setminus\{ 0 \} \big)\), et nous considérons des \( v_i\in \eZ\) tels que
		\begin{equation}		\label{EQooQSJLooOGbTzx}
			m=\sum_ja_jv_j.
		\end{equation}
		Nous prouvons à présent que \( m\) divise tous les \( a_i\). Nous nous fixons \( i\in\{ 1,\ldots,N \}\) et nous considérons la division euclidienne\footnote{Théorème \ref{THOooKDJVooRIJRHP}.} \( a_i=mq+r\) avec \( 0\leq r<m\). En remplaçant dedans la valeur de \( m\) de \eqref{EQooQSJLooOGbTzx} nous trouvons \( a_i=\sum_ja_jv_jq+r\), et en isolant \( r\) :
		\begin{equation}
			r=a_i-\sum_{j}a_jv_jq=a_i(1-v_iq)-\sum_{j\neq i}a_jv_jq\in E.
		\end{equation}
		Mais comme \( 0\leq r<m\) et \( r\in E\) et \( m=\min(E\setminus\{ 0 \})\) nous avons \( r=0\), ce qui signifie que \( m\) divise \( a_i\).

		Donc \( m=1\) parce que \( 1\) est le seul diviseur commun des tous les \( a_i\). Donc \( 1\in E\), ce qu'il fallait démontrer.
		\spitem[\( \Leftarrow\)]
		%-----------------------------------------------------------
		Supposons qu'il existe des nombres \( u_i\) tels que \( \sum_ia_iu_i=1\). Si \( \delta\) divise tous les \( a_i\), alors il existe \( v_i\) tels que \( a_i=\delta v_i\) et donc tels que
		\begin{equation}
			\sum_ia_iu_i=\sum_i\delta v_iu_i=\delta\sum_iv_iu_i,
		\end{equation}
		ce qui montre que \( \delta\divides 1\) et donc que \( \delta=1\). Nous avons démontré que \( 1\) est le seul diviseur commun des tous les \( a_i\).
	\end{subproof}
\end{proof}

\begin{corollary}       \label{CorgEMtLj}
	Soient \( p\) et \( q\) deux entiers premiers entre eux. Alors
	\begin{equation}
		p\eZ+q\eZ=\eZ;
	\end{equation}
	en particulier, pour tout \( x \in \eZ \), il existe \( u_x, v_x \) entiers tels que \(u_x p + v_x q = x \).
\end{corollary}

\begin{proof}
	Soit \( x\in \eZ\). Le théorème de Bézout nous donne \( k\) et \( l\) tels que \( kp+lq=1\). Alors, \( (xk)p+(xl)q=x\).
\end{proof}

Notons que l'application \( p\eZ+q\eZ\) vers \( \eZ\) n'est évidemment pas injective: les \( u_x\) et \( v_x\) ne sont pas uniques à \( x\) fixé, car si \( \alpha p+\beta q=x\), alors \( (\alpha+kq)p+(\beta-kp)q=x\) pour tout \( k\in \eZ\).


\begin{corollary}       \label{CORooLINXooBlUKPG}
	Les groupes quotients du groupe \( (\eZ,+)\) sont \( \eZ/n\eZ\).
\end{corollary}

\begin{proof}
	Tous les idéaux de \( \eZ\) sont de la forme \( n\eZ\). En effet en vertu de la proposition~\ref{PropSsgpZestnZ}, les seuls sous-groupes de \( \eZ\) (en tant que groupe additif) sont les \( n\eZ\). Tous les idéaux sont donc de cette forme. De plus les \( n\eZ\) sont effectivement tous des idéaux\footnote{Définition \ref{DefooQULAooREUIU}.} : si \( a\in n\eZ\) et si \( k\in \eZ\) alors \( ak\in n\eZ\).
\end{proof}

\begin{proposition}     \label{PropZpintssiprempUzn}
	Soient \( n\geq 2\) un entier et \( \phi\colon \eZ\to \eZ/n\eZ\) la surjection canonique. Nous noterons \( \overline a=\phi(a)\). Alors l'ensemble des inversibles de \( \eZ/n\eZ\) est donné par
	\begin{equation}
		U(\eZ/n\eZ)=\phi(P_n)=\{ \overline x\tq 0\leq x\leq n\tq\pgcd(x,n)=1 \}.
	\end{equation}
	où \( P_n\) est l'ensemble \( P_n=\{ x\in\{ 0,\ldots,n-1 \}\tq\pgcd(x,n)=1 \}\).

	De plus,
	\begin{equation}
		\Card\big( U(\eZ/n\eZ) \big)=\phi(n).
	\end{equation}
\end{proposition}

\begin{proof}
	Soit \( 0\leq x\leq n\) tel que \( \pgcd(x,n)=1\). Il existe donc\footnote{Théorème de Bézout~\ref{ThoBuNjam}} \( u,v\in\eZ\) tels que \( ux+vn=1\). En passant aux classes,
	\begin{equation}
		\overline u\overline x=\overline 1,
	\end{equation}
	donc \( \overline u\) est l'inverse de \( \overline x\). Cela prouve que \( \phi(P_n)\subset U(\eZ/n\eZ)\).

	Nous prouvons maintenant l'inclusion inverse. Soient \( \overline x\) et \( \overline y\) inverses l'un de l'autre : \( \overline x\overline y=\overline 1\). Il existe donc \( q\in\eZ\) tel que \( xy-qn=1\), ce qui prouve\footnote{À nouveau avec le Théorème de Bézout.} que \( \pgcd(x,n)=1\).
\end{proof}



%---------------------------------------------------------------------------------------------------------------------------
\subsection{PGCD, PPCM}
%---------------------------------------------------------------------------------------------------------------------------

Puisque \( \eZ\) est un anneau intègre, nous avons la définition \ref{DefrYwbct} de pgcd et de ppcm.
\begin{proposition}[PPCM et PGCD]       \label{PROPooAVRGooUfhjwF}
	Soient \( p,q\in\eZ^*\).
	\begin{enumerate}
		\item
		      Le pgcd de \( p\) et \( q\) est le plus grand diviseur commun de \( p\) et \( q\).
		\item
		      Le ppcm de \( p\) et \( q\) est leur plus petit multiple commun.
	\end{enumerate}
\end{proposition}

\begin{proof}
	Démontrons le premier point. Notons \( \delta\) le pgcd de \( p\) et \( q\). Si \( d\) est un diviseur commun de \( p\) et \( q\), alors \( d\) divise \( \delta\). Dans \( \eZ\), \( d\divides \delta\) implique \( d\leq\delta\) (proposition \ref{PROPooYJBMooZrzkNX}).
\end{proof}

\begin{lemma}
	Soient \( p,q\in\eZ^*\). Les entiers \( \ppcm(p,q)\) et \( \pgcd(p,q)\) fournissent les isomorphismes de groupes suivants :
	\begin{subequations}
		\begin{align}
			p\eZ\cap q\eZ & =\ppcm(p,q)\eZ  \\
			p\eZ + q\eZ   & =\pgcd(p,q)\eZ.
		\end{align}
	\end{subequations}
\end{lemma}

\begin{definition}      \label{DEFooXSPFooPumQSy}
	Nous disons que deux éléments d'un anneau principal\footnote{Anneau principal, définition \ref{DEFooGWOZooXzUlhK}.} sont \defe{premiers entre eux}{premier!deux éléments d'un anneau principal} si leurs diviseurs communs sont inversibles.
\end{definition}

Vu que \( \eZ\) est un anneau principal (proposition \ref{PROPooPJGLooQSrJTU}), la définition \ref{DEFooXSPFooPumQSy} d'éléments premiers entre eux s'applique.

\begin{lemma}       \label{LEMooLKLTooXUdsgB}
	Dans \( \eZ\), les nombres \( p\) et \( q\) sont premiers entre eux si et seulement si \( \pgcd(p,q)=1\).
\end{lemma}

\begin{definition}  \label{DefZHRXooNeWIcB}
	Soit un anneau \( A\). Nous disons que les éléments \( a_1,\ldots,a_n\) sont
	Si nous avons un ensemble d'entiers \( a_i\), nous disons qu'ils sont premiers \defe{dans leur ensemble}{nombre!premier!dans leur ensemble} si
	\begin{equation}
		\pgcd(a_1,\ldots,a_n)=1.
	\end{equation}
\end{definition}

Les nombres \( 2\), \( 4\) et \( 7\) ne sont pas premiers deux à deux (à cause de \( 2\) et \( 4\)), mais ils sont premiers dans leur ensemble parce qu'il n'y a pas de diviseurs communs plus grand que \( 1\), au triplet \( (2, 4, 7)\).

La proposition suivante établit que si \( x\) est assez grand, alors il peut même être écrit comme une combinaison de \( p\) et \( q\) à coefficients positifs. Elle sera utilisée pour démontrer que les états apériodiques d'une chaine de Markov peuvent être atteints à tout moment (assez grand), voir la définition~\ref{DefCxvOaT} et ce qui suit.

\begin{proposition}     \label{PropLAbRSE}
	Soient \( a\) et \( b\) deux éléments de \( \eN\) premiers entre eux. Il existe \( N>0\) tel que tout \( x>N\) appartient à \( a\eN+b\eN\).
\end{proposition}

\begin{proof}
	Soient \( a\) et \( b\), premiers entre eux, et \( x\in \eN\). Disons tout de suite, pour éviter les cas triviaux et pénibles, que \( x\), \( a\) et \( b\) sont strictement positifs.

	\begin{subproof}
		\spitem[Une décomposition pour \( x\)]

		On applique le théorème~\ref{ThoDivisEuclide} de division euclidienne à \( x\) et \( a + b \): il existe des entiers \( p_x, r_x \), uniques, tels que
		\begin{subequations}
			\begin{numcases}{}
				x = (p_x-1)(a+b) + r_x\\
				0 \leq r_x < a+b.
			\end{numcases}
		\end{subequations}
		En d'autres termes, \( p_x(a+b)\) est le premier multiple de \( a+b\) supérieur ou égal à \( x\). De plus, \( p_x\) est strictement positif car \( x\) l'est. Il existe alors des entiers \( u\) et \( v\) tels que
		\begin{equation}    \label{EQooXYSZooJqxPui}
			ua + vb = p_x(a+b) - x
		\end{equation}
		par le corolaire~\ref{CorgEMtLj}. Ainsi, \( x\) peut s'écrire
		\begin{equation}
			x = (p_x - u) a + (p_x - v) b.
		\end{equation}

		\spitem[Des maximums]

		Il s'agit maintenant de savoir si nous pouvons être assuré d'avoir \( p_x > u\) et \( p_x > v\) dès que \( x\) est assez grand. Pour cela, grâce au corolaire~\ref{CorgEMtLj}, nous considérons les nombres \( u_i\) et \( v_i\) définis par
		\begin{equation}
			u_ia+v_ib=i
		\end{equation}
		pour \( i=1,\ldots, a+b\). Nous posons \( u^*=\max\{ u_i \}\), \( v^*=\max\{ v_i   \}\), et \( p^*=\max\{ u^*,v^* \}\).  Nous posons alors \( N = p^*(a+b)\), et considérons \( x>N \).

		\spitem[Nouvelle décomposition pour \( x\)]

		Nous voulons écrire
		\begin{equation}        \label{EQooIKNWooBKItYz}
			x= (p_x - u_k) a + (p_x - v_k) b
		\end{equation}
		pour un certain \( k\). Cela demande \( u_ka+v_kb=ua+vb=p_x(a+b)-x\) par l'équation \eqref{EQooXYSZooJqxPui}. Vu que \( p_x(a+b)-x>0\), les nombres \( u_k\) et \( v_k\) existent : il suffit de prendre \( k=p_x(a+b)-x\).

		\spitem[Conclusion]

		Avec tous ces choix, nous avons d'abord \( x>p^*(a+b)\) et donc
		\begin{equation}
			x=(p_x-1)(a+b)+r_x>p^*(a+b),
		\end{equation}
		ce qui donne
		\begin{equation}
			(p_x-1)(a+b)>p^*(a+b)-r_x>(p^*-1)(a+b).
		\end{equation}
		ou encore \( p_x>p^*\). Nous avons finalement
		\begin{equation}
			p_x \geq p^* \geq u^* \geq u_k
		\end{equation}
		et
		\begin{equation}
			p_x \geq p^* \geq v^* \geq v_k.
		\end{equation}
		De ce fait, la décomposition \eqref{EQooIKNWooBKItYz} est celle que nous voulions.
	\end{subproof}
\end{proof}


%\begin{proof}
%Soit \( x\in \eN\) et \( k_1,l_1\in \eN\) les plus petits entiers tels que \( k_1p\geq x/2\) et \( l_1q\geq x/2\). Nous avons alors
%\begin{equation}
%x\leq k_1p+l_1q<x+(p+q).
%\end{equation}
%Nous posons \( \delta=k_1p+l_1q-x\).
%
%Soient des entiers \( a_i,b_i\) tels que \( a_ip+b_iq=i\). Nous notons
%\begin{subequations}
%\begin{align}
%A=\max\{ a_i\tq i=1,\ldots, k+p \}\\
%B=\max\{ b_i\tq i=1,\ldots, k+p \}
%\end{align}
%\end{subequations}
%Notons que \( A\) et \( B\) sont donnés uniquement en termes de \( p\) et \( q\). Ils ne sont en aucun cas dépendants de \( x\).
%
%Nous avons
%\begin{equation}
%x=k_1p+lq-\delta=(k_1-a_{\delta})p+(l_1+b_{\delta})q
%\end{equation}
%avec \( k_1-a_{\delta}\geq k_1-A\) et \( l_1-b_{\delta}\geq l_1-B\). Si \( x\) est suffisamment grand pour avoir \( k_1>A\) et \( l_1>B\), alors la décomposition souhaitée est trouvée.
%
%Une borne pour \( x\) est donnée par
%\begin{equation}    \label{EqjQpURG}
%x>\max\{ 2pA,2qB \}.
%\end{equation}
%\end{proof}

\begin{normaltext}
	Une méthode pour obtenir les entiers naturels \( u\) et \( v\) qui permettent la décomposition \(x = au + bv \) est d'abord de choisir \( u_0\) et \( v_0\) tels que \( au_0 \) et \( bv_0 \) soient les plus proches possibles de \( x/2\), puis de décomposer le nombre (relativement petit) \( x - au_0 - bv_0 \) en \( au_1 + bv_1 \). Deux nombres \( u\) et \( v\) qui fonctionnent sont alors \( u = u_0 + u_1\) et \( v = v_0 + v_1\).
\end{normaltext}

\begin{example}
	Écrivons \( 1000=u\cdot 7+v\cdot 5\) avec \( u,v\in \eN\). D'abord \( 72\cdot 7=504\) et \( 100\cdot 5=500\). Nous avons donc
	\begin{equation}
		1004=72\cdot 7+100\cdot 5.
	\end{equation}
	Ensuite \( 4=25-21=-3\cdot 7+5\cdot 5\). Au final,
	\begin{equation}
		1000=75\cdot 7+95\cdot 5.
	\end{equation}
\end{example}


%---------------------------------------------------------------------------------------------------------------------------
\subsection{Décomposition en facteurs premiers}
%---------------------------------------------------------------------------------------------------------------------------

\begin{lemma}[Lemme de Gauss]    \label{LemPRuUrsD}
	Soient \( a,b,c\in \eZ\) tels que \( a\) divise \( bc\). Si \( a\) est premier avec \( c\), alors \( a\) divise \( b\).
\end{lemma}
\index{lemme!de Gauss!pour des entiers}

\begin{proof}
	Puisque \( a\) est premier avec \( c\), nous avons \( \pgcd(a,c)=1\) et le théorème de Bézout~\ref{ThoBuNjam} nous donne donc \( s,t\in \eZ\) tels que \( sa+tc=1\). En multipliant par \( b\), nous avons \( sab+tbc=b\). Mais les deux termes du membre de gauche sont multiples de \( a\) parce que \( a\) divise \( bc\). Par conséquent \( b\) est somme de deux multiples de \( a\) et donc est multiple de \( a\).
\end{proof}

Il y a une généralisation du lemme de Gauss pour les anneaux principaux dans \ref{LemSdnZNX}.
\begin{lemma}[Lemme d'Euclide\cite{BTDWooZCyXfb}]       \label{LemAXINooOeuMJZ}
	Soient \( a,b\in \eZ\). Si le nombre premier \( p\) divise le produit \( ab\), alors \( p\) divise \( a\) ou \( b\).
\end{lemma}
\index{Euclide!lemme}

\begin{proof}
	Comme \( p\) est premier, si il ne divise pas \( a\) c'est que \( \pgcd(a,p)=1\). Dans ce cas le lemme de Gauss~\ref{LemPRuUrsD} implique que \( p\) divise \( b\).
\end{proof}
\index{lemme!d'Euclide}

\begin{lemma}		\label{LEMooSRFMooHgEMwj}
	Soient \( a,b\in \eN\) tels que \( a\) divise \( b\) et \( b\) divise \( a\). Alors \( a=b\).
	%TODOooNFZZooGETGlh. Prouver ça.
\end{lemma}

\begin{lemma}[\cite{MonCerveau}]	\label{LEMooFVSJooRysGAm}
	Soient \( k,n,p\in \eZ\). Le nombre \( kp\) divise \( kn\) si et seulement si \( p\) divise \( n\).
\end{lemma}

\begin{proof}
	En deux parties.
	\begin{subproof}
		\spitem[\( \Rightarrow\)]
		%-----------------------------------------------------------
		Nous supposons que \( kp\divides kn\). Il existe \( l\in \eZ\) tel que \( kpl=kn\). On peut simplifier par \( k\) : \( pl=n\), ce qui signifie que \( p\divides n\).

		\spitem[\( \Leftarrow\)]
		%-----------------------------------------------------------
		Si \( p\divides n\), alors il existe \( l\) tel que \( lp=n\). Avec ça, \( kn=klp\), et donc \( kp\divides kn\).
	\end{subproof}
\end{proof}

Le théorème fondamental de l'arithmétique permet de décomposer des nombres en facteurs premiers.

\begin{theorem}[Décomposition en facteurs premiers\cite{RATEooJuqgom}]        \label{ThoAJFJooAveRvY}
	Tout entier strictement positif peut être écrit comme un produit de nombres premiers d'une unique façon, à l'ordre près des facteurs.

	En d'autres termes, pour tout entier \( n>1\), il existe une unique suite finie unique \( (p_1, k_1)\),\ldots \( (p_r, k_r)\) telle que :
	\begin{enumerate}
		\item
		      les \( p_i\) sont des nombres premiers tels que, si \( i < j\), alors \( p_i < p_j\) ;
		\item
		      les \( k_i\) sont des entiers naturels non nuls ;
		\item
		      \( n=\prod_{i=1}^rp_i^{k_i}\).
	\end{enumerate}
\end{theorem}
\index{nombre premier!décomposition}

\begin{proof}
	Soit \( n\) un entier positif. Nous prouvons l'existence d'une décomposition en facteurs premiers par récurrence. Le nombre \( n=1\) est le produit d'une famille finie de nombres premiers : la famille vide\footnote{Voir \ref{NORMooDBOFooQCwbOY}.}.

	Supposons que tout entier strictement inférieur à un certain entier \( n>1\) est produit de nombres premiers. Deux possibilités apparaissent pour \( n\) : il est premier ou non. Si \( n\) est premier, et donc produit d'un unique entier premier, à savoir lui-même, le résultat est vrai. Si \( n\) n'est pas premier, il se décompose sous la forme \( kl\) avec \( k\) et \( l\) strictement inférieurs à \( n\). Dans ce cas, l'hypothèse de récurrence implique que les entiers \( k\) et \( l\) peuvent s'écrire comme produits de nombres premiers. Leur produit aussi, ce qui fournit une décomposition de \( n\) en produit de nombres premiers. Par application du principe de récurrence, tous les entiers naturels peuvent s'écrire comme produit de nombres premiers.

	Nous prouvons maintenant l'unicité. Prenons deux produits de nombres premiers qui sont égaux. Prenons n'importe quel nombre premier \( p\) du premier produit. Il divise le premier produit, et, de là, aussi le second. Par le lemme d'Euclide~\ref{LemAXINooOeuMJZ}, \( p\) doit alors diviser au moins un facteur dans le second produit. Mais les facteurs sont tous des nombres premiers eux-mêmes, donc \( p\) doit être égal à un des facteurs du second produit. Nous pouvons donc simplifier par \( p\) les deux produits. En continuant de cette manière, nous voyons que les facteurs premiers des deux produits coïncident précisément.
\end{proof}

\begin{lemma}[\cite{MonCerveau}]        \label{LEMooDTQQooYoJABt}
	Nous notons \( \mP\) l'ensemble des nombres premiers dans \( \eN\). Soient des suites finies \( (a_p)_{p\in \mP}\) et \( (b_p)_{p\in \mP}\). Nous posons
	\begin{equation}
		\begin{aligned}[]
			a & =\prod_{ p\in\mP}p^{a_p} & \text{ et } &  & b=\prod_{ p\in \mP}p^{b_p}.
		\end{aligned}
	\end{equation}
	Alors \( a\divides b\) si et seulement si \( a_p\leq b_p\) pour tout \( p\).
\end{lemma}

\begin{proof}
	Dire que \( a\divides b\) signifie qu'il existe \( s\in \eN\) tel que \( as=b\); le théorème \ref{ThoAJFJooAveRvY} nous permet de décomposer \( s\) en \( s=\prod_{p\in\mP}p^{s_p}\). Puisque le produit dans \( \eN\) est commutatif et associatif,
	\begin{equation}
		b=as=\prod_{p\in\mP}p^{s_p+a_p}.
	\end{equation}
	Par unicité de la décomposition de \( b\) (toujours le théorème \ref{ThoAJFJooAveRvY}), nous en déduisons que \( b_p=s_p+a_p\geq a_p\).

	Dans l'autre sens, l'hypothèse \( a_p\leq b_p\) implique l'existence de \( s_p\geq 0\) tels que \( b_p=a_p+s_p\). En posant \( s=\prod_{p\in\mP}p^{s_p}\), nous avons
	\begin{equation}
		as=\prod_{p\in\mP}p^{s_p+a_p}=\prod_{p\in \mP}p^{b_p}=b.
	\end{equation}
	Donc \( a\divides b\).
\end{proof}

\begin{lemma}       \label{LEMooGLZHooUcRNgu}
	Soient un nombre premier \( q\) ainsi que \( a\in \eZ\). Soit un entier \( n\geq 1\). Le nombre \( q\) divise \( a\) si et seulement si il divise \( a^n\).
\end{lemma}

\begin{proof}
	Nous numérotons les nombres premiers \( p_i\) pour que \( p_1\) soit \( q\). La décomposition en nombre premiers du théorème \ref{ThoAJFJooAveRvY} nous dit que
	\begin{equation}
		a=q^{a_1}\prod_{i\neq 1}p_i^{a_i}
	\end{equation}
	et
	\begin{equation}
		a^n=q^{na_1}\prod_{i\neq 1}p_i^{na_i}
	\end{equation}
	Nous avons équivalence entre les énoncés suivants :
	\begin{itemize}
		\item \( q\) divise \( a\)
		\item \( a_1\neq 0\)
		\item \( na_1\neq 0\) (parce que \( n\neq 0\))
		\item \( q\) divise \( a^n\).
	\end{itemize}
\end{proof}

\begin{corollary}		\label{CORooWBSQooQOEmaC}
	Si \( n\in \eN\) n'est pas une puissance d'un nombre premier, alors il existe \( a,b\in \eN\) tels que \( \pgcd(a,b)=1\) et \( n=ab\).
\end{corollary}

\begin{proof}
	Si \( n\) n'est pas une puissance d'un nombre premier, alors le théorème \ref{ThoAJFJooAveRvY} de décomposition en nombres premiers dit que
	\begin{equation}
		n=\prod_{i=1}^rp_i^{k_i}
	\end{equation}
	avec au moins \( r=2\), et les \( k_i\) non nuls. En prenant \( a=p_1^{k_1}\) et \( b=\prod_{i=2}^rp_i^{k_i}\) nous avons bien \( n=ab\). Montrons que \( a\) et \( b\) n'ont pas de diviseurs communs. Soit, par l'absurde, \( q\) un diviseur premier commun à \( a\) et \( b\).

	Vu que \( q\) divise \( p_1^{k_1}\), le nombre \( q\) divise \( p_1\) par le lemme \ref{LEMooGLZHooUcRNgu}. Étant donné que \( q\) divise \( \prod_{i=2}^rp_i^{k_i}\), il divise au moins un des facteurs (lemme d'Euclide \ref{LemAXINooOeuMJZ}). Disons que \( q\) divise \( p_i^{k_i}\). Dans ce cas \( q\) divise \( p_i\).

	Donc \( q\) divise \( p_i\) et \( p_1\) et donc \( q=1\) parce que \( p_i\) et \( p_1\) sont des nombres premiers distincts.
\end{proof}


%-------------------------------------------------------
\subsection{Binôme de Newton}
%----------------------------------------------------


\begin{definition}[Coefficient binomial]		\label{DEFooHXNFooOBKgqD}
	Soient des entiers \( n\) et \( k\) avec \( k\leq n\). Nous définissons \defe{coefficients binomiaux}{coefficients binomiaux} par
	\begin{equation}		\label{EQooJJYGooTvmSAt}
		{n\choose k} = \frac{ n! }{ k!(n-k)! }.
	\end{equation}
	\index{coefficient binomial}
	Lorsque \( \alpha\) est un multiindice, la factorielle \( \alpha!\) et les coefficients binomiaux sont définis en \ref{DEFooCQPRooFeWeOS}.
\end{definition}

\begin{lemma}		\label{LEMooHWMNooIINsxu}
	Un formule pour les coefficients binomiaux :
	\begin{equation}
		{n\choose k}+{n\choose k-1}={n+1\choose k}
	\end{equation}
\end{lemma}

\begin{proof}
	Simple mise au même dénominateur :
	\begin{equation}
		\frac{ n! }{ k!(n-k)! }+\frac{ n! }{ (k-1)!(n-k+1)! }=\frac{ n!(n-k+1)+n!k }{ k!(n-k+1)! }=\frac{ n!(n+1) }{  k!(n+1-k)!  }.
	\end{equation}
\end{proof}

\begin{lemma}[\cite{MonCerveau}]		\label{LEMooUTDTooXAmvdF}
	Nous considérons ces deux ensembles :
	\begin{subequations}
		\begin{align}
			A & =\{ I=(i_1,\ldots,i_k)\tq 1\leq i_1<\ldots <i_k\leq n \}                     \\
			C & =\big\{ \text{parties de cardinal \( k\) dans \( \{ 1,\ldots,n \}\)} \big\}.
		\end{align}
	\end{subequations}
	Il existe une bijection entre \( A\) et \( C\).

	Nous avons de plus\footnote{Pour le coefficient binomial, définition \ref{DEFooHXNFooOBKgqD}.}
	\begin{equation}        \label{EQooSPVNooHoPnOe}
		\Card(A)=\Card(C)={n\choose k}
	\end{equation}
	%TODOooGOFJooKrymcy. Prouver ça.
\end{lemma}


%-------------------------------------------------------
\subsection{Divisibilité des coefficients binomiaux}
%----------------------------------------------------

\begin{lemma}[Formule du pion\cite{BIBooRRMHooIQHutK}]	\label{LEMooYBOIooEtVfDQ}
	Soient \( n\) et \( k\) deux nombres entiers tels que \( 1\leq k\leq n\). Nous avons
	\begin{equation}		\label{EQooDNBYooZfJEAe}
		k\binom{ n }{ k }=n\binom{ n-1 }{ k-1 }.
	\end{equation}
	%TODOooRHRWooFjWjjV. Prouver ça.
\end{lemma}

\begin{proposition}[\cite{BIBooDQYEooUxkLBh,BIBooVIABooGCECHK}]	\label{PROPooVPOYooNNugWU}
	Un entier \( n\geq 2\) est premier si et seulement si il divise \( \binom{ n }{ k }\) pour tout \( k=1,\ldots,n-1\).
\end{proposition}

\begin{proof}
	En deux parties.
	\begin{subproof}
		\spitem[\( \Rightarrow\)]
		%-----------------------------------------------------------
		Nous supposons que \( n\) est un nombre premier. La formule du pion \eqref{EQooDNBYooZfJEAe} nous dit que
		\begin{equation}
			k\binom{ n }{ k }=n\binom{ n-1 }{ k-1 }.
		\end{equation}
		Donc \( n\) divise le produit \(k\binom{ n }{ k }\); vu que \( n\) est premier, le lemme d'Euclide \ref{LemAXINooOeuMJZ} dit qu'il divise un des deux facteurs. Le nombre \( n\) ne peut pas diviser \( k\) parce que \( k<n\). Donc il divise l'autre. C'est à dire que \( n\) divise \( \binom{ n }{ k }\).

		\spitem[\( \Leftarrow\)]
		%-----------------------------------------------------------
		Nous allons à la contraposée: nous supposons que \( n\) n'est pas premier, et nous allons montrer qu'il y a (au moins) un des \( \binom{ n }{ k }\) que \( n\) ne divise pas.

		Posons \( n=lp\) où \( p\) est premier. Nous montrons que \( n\) ne divise pas \( \binom{ n }{ p }\). Pour cela nous utilisons encore la formule du pion :
		\begin{equation}
			\binom{ n }{ p }=\frac{ n }{ p }\binom{ n-1 }{ p-1 }=l\frac{ (n-1)\ldots (n-p+1) }{ (p-1)! }.
		\end{equation}
		Le lemme \ref{LEMooFVSJooRysGAm} dit que le tout est divisible par \( lp\) si et seulement si
		\begin{equation}		\label{EQooIHXGooIQorMC}
			\frac{ (n-1)\ldots (n-p+1) }{ (p-1)! }
		\end{equation}
		est divisible par \( p\). Vu que \( p\) divise \( n\), il ne divise aucun des nombres \( n-1\),\ldots,\( n-p+1\). Et comme \( p\) est premier, il ne peut pas diviser le produit sans diviser aucun des facteurs. Et vlan, \( p\) ne divise pas \eqref{EQooIHXGooIQorMC}.
	\end{subproof}
\end{proof}

%-------------------------------------------------------
\subsection{Coefficient binmoniaux et polynômes}
%----------------------------------------------------

\begin{normaltext}		\label{NORMooROXXooXOybXN}
	Une convention d'écriture. Si \( A\) est un anneau, et si \( \lambda\in \eN\) alors nous notons
	\begin{equation}
		\lambda a=\sum_{i=1}^{\lambda}a.
	\end{equation}
	Et nous posons également \( 0a=0\). À gauche : le \( 0\) de \( \eN\) et à droite celui de \( A\).

	Si \( \lambda\in \eZ\), alors lorsque \( \lambda<0\) nous posons \( \lambda a=  -(-\lambda a)  \)
\end{normaltext}


\begin{proposition}[Formule binomiale\cite{ooPTQCooIWykWP}]     \label{PropBinomFExOiL}
	Soit un anneau commutatif \( A\). Pour tout \( x,y\in A\) et \( n\in \eN\), nous avons
	\begin{equation}        \label{EqNewtonB}
		(x+y)^n=\sum_{k=0}^n{n\choose k}x^{n-k}y^k
	\end{equation}
	où les coefficients binomiaux sont donnés dans le lemme \ref{LEMooUTDTooXAmvdF}, et où nous utilisons les conventions d'écriture de \ref{NORMooROXXooXOybXN}.
\end{proposition}
\index{formule binomiale}

\begin{proof}
	La preuve se fait par récurrence. La vérification pour \( n=0\) et \( n=1\) se fait aisément pour peu que l'on se rappelle que \( x^0=1\) et que \( 0!=1\), ce qui donne entre autres \( {0\choose 0}=1\).

	Supposons que la formule \eqref{EqNewtonB} soit vraie pour \( n\geq1\), et prouvons la pour \( n+1\). Nous avons
	\begin{subequations}\label{EqBinTrav}
		\begin{align}
			(x+y)^{n+1} & = (x+y)\cdot \sum_{k=0}^n{n\choose k}x^{n-k}y^k                                                  \label{SUBEQooBEUCooHSRtug}  \\
			            & = \sum_{k=0}^n{n\choose k}x^{n-k+1}y^k+\sum_{k=0}^n{n\choose k}x^{n-k}y^{k+1}                     \label{SUBEQooRURMooGVIbbL} \\
			            & = x^{n+1}+ \sum_{k=1}^n{n\choose k}x^{n-k+1}y^k+\sum_{k=0}^{n-1}{n\choose k}x^{n-k}y^{k+1}+y^{n+1}.
		\end{align}
	\end{subequations}
	Notez le passage de \eqref{SUBEQooBEUCooHSRtug} à \eqref{SUBEQooRURMooGVIbbL} qui demande la commutativité de la multiplication. La formule binomiale ne se généralise pas de façon directe à un anneau non commutatif.

	La seconde grande somme peut être transformée en posant \( i=k+1\) :
	\begin{equation}
		\sum_{k=0}^{n-1}{n\choose k}x^{n-k}y^{k+1} = \sum_{i=1}^n{n\choose i-1}x^{n-(i-1)}y^{i-1+1},
	\end{equation}
	dans lequel nous pouvons immédiatement renommer \( i\) par \( k\). En remplaçant dans la dernière expression de \eqref{EqBinTrav}, nous trouvons
	\begin{equation}
		(x+y)^{n+1}=x^{n+1}+y^{n+1}+\sum_{k=1}^n\left[ {n\choose k}+{n\choose k-1} \right]x^{n-k+1}y^k.
	\end{equation}
	La thèse découle maintenant de la formule du lemme \ref{LEMooHWMNooIINsxu}.
\end{proof}

\begin{lemma}[\cite{MonCerveau}]        \label{LEMooLPCXooYIzJsD}
	Si \( n\geq k\) nous avons
	\begin{equation}
		\frac{ n! }{ k!(n-k)! }\leq \frac{ n^{k-1} }{ k! }.
	\end{equation}
\end{lemma}

\begin{proof}
	Nous décomposons le produit définissant \( n!\) en les facteurs entre \( 1\) et \( n-k\) et ceux entre \( n-k+1\) et \( n\) :
	\begin{equation}
		n!=(n-k)!\prod_{i=n-k+1}^ni\leq n^{k-1}(n-k)!.
	\end{equation}
	Donc
	\begin{equation}
		\frac{ n! }{ k!(n-k)! }\leq \frac{ n^{k-1} }{ k! }.
	\end{equation}
\end{proof}

Tant que nous sommes à démontrer des égalités, en voici une.

\begin{lemma}[\cite{BIBooLDPCooFJcgAl}]     \label{LEMooLPOQooICJYdV}
	Pour \( a,b\in \eR\) et \( n\in \eN\) nous avons
	\begin{equation}
		a^n+(-1)^{n-1}b^n=(a+b)\sum_{k=0}^{n-1}(-1)^ka^{n-1-k}b^k.
	\end{equation}
\end{lemma}

\begin{proof}
	C'est un simple calcul:
	\begin{subequations}
		\begin{align}
			(a+b)\sum_{k=0}^{n-1}(-1)^ka^{n-1-k}b^k & =  \sum_{k=0}^{n-1}(-1)^ka^{n-k}b^k+\sum_{k=0}^{n-1}(-1)^ka^{n-k-1}b^{k+1}                                                 \\
			                                        & = a^n+\sum_{k=1}^{n-1}(-1)^ka^{n-k}b^k+\sum_{k=0}^{n-2}(-1)^ka^{n-k-1}b^{k+1} +(-1)^{n-1}b^n   \label{SUBEQooUCIBooKsuEbh} \\
			                                        & = a^n+(-1)^{n-1}b^n                                            \label{SUBEQooLTIHooZPMwVF}
		\end{align}
	\end{subequations}
	Justifications.
	\begin{itemize}
		\item Pour \eqref{SUBEQooUCIBooKsuEbh}. Dans la première somme, nous avons séparé le terme \( k=0\) et dans la seconde nous avons séparé le terme \( k=n-1\)
		\item Pour \eqref{SUBEQooLTIHooZPMwVF}. Dans la seconde somme, décaler les termes pour sommer de \( 1\) à \( n-1\) et remarquer que ce qu'on obtient annule la première somme.
	\end{itemize}
\end{proof}


%+++++++++++++++++++++++++++++++++++++++++++++++++++++++++++++++++++++++++++++++++++++++++++++++++++++++++++++++++++++++++++
\section{Idéal dans un anneau}
%+++++++++++++++++++++++++++++++++++++++++++++++++++++++++++++++++++++++++++++++++++++++++++++++++++++++++++++++++++++++++++

La définition d'un idéal dans un anneau est la définition~\ref{DefooQULAooREUIU}.

\begin{definition}[\cite{ooLKFGooTUrnhx}]  \label{DefTMNooKXHUd}
	Un \defe{corps}{corps} est un anneau\footnote{Définition \ref{DefHXJUooKoovob}.} \( (A, +,\times)\) dans lequel tout élément non nul est inversible pour l'opération \( \times\) (pour l'opération \( +\), tous les éléments sont inversibles parce que \( (A,+)\) est un groupe).
\end{definition}

\begin{normaltext}
	Dans le Frido, nous ne parlons que de corps commutatifs; nous ne le répéterons pas toujours.
\end{normaltext}

\begin{normaltext}
	Pour savoir ce qu'est un «\emph{ring}» ou «\emph{field}» en anglais, voir \ref{SECooPBZVooCVInFT}.
\end{normaltext}

\begin{definition}  \label{DefAJVTPxb}
	Un sous-ensemble \( B\subset A\) d'un anneau est un \defe{sous anneau}{sous-anneau} si
	\begin{enumerate}
		\item
		      \( 1\in B\)
		\item
		      \( B\) est un sous-groupe pour l'addition
		\item
		      \( B\) est stable pour la multiplication.
	\end{enumerate}
\end{definition}

\begin{remark}
	Un idéal n'est pas toujours un anneau parce que l'identité pourrait manquer. Un idéal qui contient l'identité est l'anneau complet.
\end{remark}

\begin{lemma}       \label{LEMooQAYSooCYJXkC}
	La partie \( 2\eZ\) est un idéal\footnote{Idéal, définition \ref{DefooQULAooREUIU}.} de \( \eZ\). On peut aussi le noter \( (2) \).
	%TODOooXMJAooYcLVcZ. Prouver ça.
\end{lemma}

\begin{proposition}[\cite{BIBooQHHCooXEhgqZ} Premier théorème d'isomorphisme]	\label{PROPooPZGFooEbSBXD}
	Soit des anneaux \( A\) et \( B\) ainsi qu'un idéal bilatère\footnote{Définition \ref{DefooQULAooREUIU}.} \( I\). Nous considérons la projection canonique \(\pi \colon A\to A/I  \).

	Soit un morphisme d'anneaux\footnote{Morphisme d'anneaux, définition \ref{DEFooSPHPooCwjzuz}.} \(f \colon A\to B  \) vérifiant \( f(I)=\{ 0 \}\). Alors il existe un unique morphisme \(\tilde f \colon A/I\to B  \) tel que \( \tilde f\circ\pi=f\).

	\begin{equation}
		\xymatrix{%
			A \ar[r]^{f}\ar[d]_{\pi}       &   B                         \\
			A/I    \ar@{-->}[ru]_{\tilde f}
		}
	\end{equation}
\end{proposition}

\begin{proof}
	Soient \( a,b\in A\) tels que \( \pi(a)=\pi(b)\). Alors \( a-b\in I\) et donc \( f(a-b)=0\). Vu que \( f\) est un morphisme nous en déduisons que \( f(a)=f(b)\). La proposition \ref{PROPooCONJooTpJwBe} dit qu'il existe alors une unique application \(\tilde f \colon A/I\to B  \) telle que
	\begin{equation}
		\tilde f\big( \pi(a) \big)=f(a)
	\end{equation}
	pour tout \( a\in A\).

	Nous devons prouver que cette application est un morphisme d'anneaux. Le neutre de \( A/I\) est la classe \( [1]\) et nous avons \( \tilde f([1])=1\). En ce qui concerne la somme, soient \( \alpha,\beta\in A/I\). Si \( a\in \alpha\) et \( b\in \beta\) nous avons
	\begin{subequations}
		\begin{align}
			\tilde f(\alpha+\beta) & =\tilde f([a+b])                   \\
			                       & =f(a+b)                            \\
			                       & =f(a)+f(b)                         \\
			                       & =\tilde f([a])+\tilde f([b])       \\
			                       & =\tilde f(\alpha)+\tilde f(\beta).
		\end{align}
	\end{subequations}
	Pour la multiplication c'est le même genre de vérifications.
\end{proof}

\begin{proposition}[Premier théorème d'isomorphisme pour les anneaux]   \label{PROPooJALPooHFIObB}
	Soient \( A\) et \( B\) des anneaux et un morphisme \( f\colon A\to B\). Nous considérons l'injection canonique \( j\colon f(A)\to B\) et la surjection canonique \( \phi\colon A\to A/\ker f\). Alors il existe un unique isomorphisme
	\begin{equation}
		\tilde f \colon A/\ker f\to f(A)
	\end{equation}
	tel que \( f=j\circ\tilde f\circ\phi\).

	\begin{equation}
		\xymatrix{%
		A \ar[r]^{f}\ar[d]_{\phi}       &   B                         \\
		A/\ker f    \ar|{-->}[r]_{\tilde f}   &   f(A)\subset B \ar[u]^{j}
		}
	\end{equation}
\end{proposition}
\index{théorème!isomorphisme!premier!pour les anneaux}

\begin{proof}
	Il s'agit d'appliquer la proposition \ref{PROPooPZGFooEbSBXD} au cas particulier \( I=\ker(f)\). Vérifions que \( \tilde f\) est alors bien injective. Supposons que \( \alpha,\beta\in A/\ker(f)\) vérifient \( \tilde f(\alpha)=\tilde f(\beta)\). Soient \( a\in \alpha\) et \( b\in \beta\). Nous avons
	\begin{equation}
		f(a)=\tilde f(\alpha)=\tilde f(\beta)=f(b).
	\end{equation}
	Donc \( f(a-b)=0\), ce qui signifie \( a-b\in\ker(f)\) et donc \( \alpha=\beta\).
\end{proof}

\begin{proposition}     \label{PropIJJIdsousphi}
	Soient \( I\) un idéal de \( A\) et la projection canonique
	\begin{equation}
		\phi\colon A\to A/I.
	\end{equation}
	C'est une bijection entre les idéaux de \( A\) contenant \( I\) et les idéaux de \( A/I\).

	Dit de façon imagée :
	\begin{equation}        \label{EqKbrizu}
		\{ \text{idéaux de } A\text{ contenant } I\}\simeq\{ \text{idéaux de } A/I \}.
	\end{equation}
\end{proposition}

\begin{proof}
	Si \( I\subset J\) et si \( J \) est un idéal de \( A\), alors \( \phi(J)\) est un idéal dans \( A/I\). En effet un élément de \( \phi(J)\) est de la forme \( \phi(j)\) et un élément de \( A/I\) est de la forme \( \phi(i)\). Leur produit vaut
	\begin{equation}
		\phi(i)\phi(j)=\phi(ij)\in\phi(J).
	\end{equation}

	Soit maintenant \( K\) un idéal dans \( A/I\) et soit \( J=\phi^{-1}(K)\). Étant donné qu'un idéal doit contenir \( 0\) (parce qu'un idéal est un groupe pour l'addition), \( [0]\in K\) et par conséquent \( I\subset\phi^{-1}(K)\).
\end{proof}

\begin{proposition}[\cite{MonCerveau}]     \label{AnnCorpsIdeal}\label{PROPooUOCVooZGAVVk}
	Si \( A\) est un anneau, nous avons les équivalences
	\begin{enumerate}
		\item       \label{ITEMooLAAVooXhTcMe}
		      \( A\) est un corps\footnote{Définition \ref{DefTMNooKXHUd}.}.
		\item       \label{ITEMooDGZIooRopYGx}
		      \( A\) est non nul et ses seuls\footnote{Je vous laisse vous poser de grandes questions sur le fait que le vide est un idéal ou non.} idéaux à gauche sont \( \{ 0 \}\) et \( A\).
		\item       \label{ITEMooLPWHooDJpTbR}
		      \( A\) est non nul et ses seuls idéaux à droite sont \( \{ 0 \}\) et \( A\).
	\end{enumerate}
\end{proposition}

\begin{proof}
	Nous allons montrer que le point \ref{ITEMooLAAVooXhTcMe} est équivalent aux deux autres.
	\begin{subproof}
		\spitem[\ref{ITEMooLAAVooXhTcMe} implique \ref{ITEMooDGZIooRopYGx}]
		Si \( I\) est un idéal à gauche différent de \( \{ 0 \}\), alors il contient un certain \( a\neq 0\). Puisque \( A\) est un corps, il contient un inverse \( a^{-1}\), et comme \( I\) est un idéal, \( a^{-1} I\subset I\). En particulier \( a^{-1}a\in I\). Donc \( 1\in I\) et \( I=A\).
		\spitem[\ref{ITEMooDGZIooRopYGx} implique \ref{ITEMooLAAVooXhTcMe}]
		Supposons que les seuls idéaux de \( A\) soient \( \{ 0 \}\) et \( A\). Soit \( a\in A\). Si \( a\) est non nul, alors \( aA\) est un idéal de \( a\). Vu qu'il contient \( a\neq 0\), nous avons \( aA=A\) (par hypothèse, un idéal qui n'est pas \( \{ 1 \}\) est \( A\)). En particulier, \( 1\in aA\), c'est-à-dire qu'il existe \( b\in A\) tel que \( ab=1\). L'élément \( a\) est donc inversible.
		\spitem[\ref{ITEMooLAAVooXhTcMe} implique \ref{ITEMooLPWHooDJpTbR}]
		% -------------------------------------------------------------------------------------------- 
		Comme pour \ref{ITEMooLAAVooXhTcMe} implique \ref{ITEMooDGZIooRopYGx}.
		\spitem[\ref{ITEMooLPWHooDJpTbR} implique \ref{ITEMooLAAVooXhTcMe}]
		% -------------------------------------------------------------------------------------------- 
		Comme pour \ref{ITEMooDGZIooRopYGx} implique \ref{ITEMooLAAVooXhTcMe}.
	\end{subproof}
	Notez que je n'ai pas vérifié les deux derniers points. Donc vous devriez le vérifier et m'écrire si il y a un problème.
\end{proof}

\begin{definition}[\cite{ooWEUDooQybvIx}]      \label{DEFIdealMax}
	Soit un anneau \( A\). Un idéal \( I\neq A\) est dit \defe{idéal maximal}{idéal maximal} si il n'existe pas d'idéal \( J\neq A\) contenant strictement \( I\).
\end{definition}

\begin{proposition}[Thème~\ref{THEMEooZYKFooQXhiPD}]     \label{PROPooSHHWooCyZPPw}
	Un idéal \( I\) dans un anneau \( A \) est maximum si et seulement si \( A/I\) est un corps.
\end{proposition}

\begin{proof}
	Soit un idéal maximum \( I\subset A\). Alors les idéaux contenant \( I\) sont \( A\) et \( I\). L'application \( \phi\) de la proposition~\ref{PropIJJIdsousphi} est une bijection, donc l'ensemble des idéaux de \( A/I\) ne contient que deux éléments. Les seuls idéaux de \( A/I\) sont donc \( \{ 0 \}\) et \( A/I\); donc \( A/I\) est un corps par la proposition~\ref{PROPooUOCVooZGAVVk}.

	Dans l'autre sens, c'est la même chose : si \( A/I\) est un corps, il possède exactement deux idéaux, donc \( A\) ne contient que deux idéaux contenant \( I\). Donc \( I\) est un idéal maximum.
\end{proof}

\begin{theorem}[Théorème de Krull\cite{BIBooAVBIooUFSvVv}]      \label{THOooFWYLooOofaPa}
	Pour tout idéal propre \( I\) d'un anneau commutatif \( A\), il existe au moins un idéal maximal de \( A\) contenant \( I\).
	%TODOooYGPVooJwedcz. Prouver ça.
\end{theorem}


\begin{definition}      \label{DEFooAQSZooVhvQWv}
	Nous disons qu'un idéal \( I\) dans \( A\) est \defe{premier}{premier!idéal} si \( I\) est strictement inclus dans \( A\) et si pour tout \( a,b\in A\) tels que \( ab\in I\) nous avons \( a\in I\) ou \( b\in I\).
\end{definition}

\begin{lemma}       \label{LEMooYRPBooYxXXsi}
	L'idéal nul (réduit à \( \{ 0 \}\)) est premier si et seulement si \( A\) est intègre\footnote{Définition \ref{DEFooTAOPooWDPYmd} d'anneau intègre.}.
\end{lemma}

\begin{proof}
	En deux sens.
	\begin{subproof}
		\spitem[Si \( \{ 0 \}\) est premier]
		Alors \( A\neq \{ 0 \}\) parce que \( I=\{ 0 \}\) est propre (définition d'idéal premier).

		De plus, si \( ab=0\), alors \( ab\in I\) qui est un idéal premier. Donc soit \( a\) soit \( b\) est dans \( I\), c'est-à-dire que soit \( a\) soit \( b\) est nul. Donc \( A\) est intègre.

		\spitem[Si \( A\) est intègre]
		Alors \( A\neq \{ 0 \}\) et l'idéal \( I=\{ 0 \}\) est strictement inclus dans \( A\). Si \( ab\in I\), alors \( ab=0\) et comme \( A\) est intègre, soit \( a\) soit \( b\) est nul, c'est-à-dire appartient à \( I\).
	\end{subproof}
\end{proof}

\begin{proposition}[\cite{ooWEUDooQybvIx}]      \label{PROPooRUQKooIfbnQX}
	Soit un anneau commutatif\footnote{Tous les anneaux du Frido sont commutatifs} et un idéal \( I\) dans \( A\).
	\begin{enumerate}
		\item       \label{ITEMooUGBTooOGrnWl}
		      \( I\) est un idéal premier\footnote{Idéal premier, définition \ref{DEFooAQSZooVhvQWv}.} si et seulement si \( A/I\) est un anneau intègre.
		\item   \label{ITEMooGLXSooUjINqR}
		      \( I\) est un idéal maximal\footnote{Idéal maximal, définition \ref{DEFIdealMax}.} si et seulement si \( A/I\) est un corps.
		\item       \label{ITEMooTFFQooOUajFw}
		      Tout idéal maximal propre est premier.
	\end{enumerate}
\end{proposition}

\begin{proof}
	En plein d'étapes.
	\begin{subproof}
		\spitem[\ref{ITEMooUGBTooOGrnWl}, \( \Rightarrow\)]
		Évacuons le cas trivial pour être sûr. Si \( I=\{ 0 \}\) alors \( A\) est intègre par le lemme \ref{LEMooYRPBooYxXXsi}. Donc \( A/I=A/\{ 0 \}=A\) est intègre également.

		Soient \( a,b\in A\) tels que \( [a][b]=[0]\). Donc \( [ab]=[0]\), c'est-à-dire \( ab\in I\). Puisque \( I\) est un idéal premier nous avons \( a\in I\) ou \( b\in I\), c'est-à-dire \( [a]=0\) ou \( [b]=0\); nous en déduisons que \( A/I\) est un anneau intègre.
		\spitem[\ref{ITEMooUGBTooOGrnWl}, \( \Leftarrow\)]

		Soit \( ab\in I\). Alors \( [ab]=0\), ce qui signifie que \( [a][b]=0\) donc que \( [a]=0\) ou que \( [b]=0\) parce que \( A/I\) est intègre. Mais la condition \( [a]=0\) signifie \( a\in I\), et \( [b]=0\) signifie \( b\in I\). Nous avons donc prouvé que soit \( a\) soit \( b\) est dans \( I\), c'est-à-dire que \( I\) est premier.
		\spitem[\ref{ITEMooGLXSooUjINqR}, \( \Rightarrow\)]

		Nous devons montrer que tout élément non nul de \( A/I\) est inversible. Un élément non nul de \( A/I\) est \( [x]\) avec \( x\in A\setminus I\).

		Nous considérons \( J=Ax+I\), qui est un idéal parce que pour tout \( a\in A\), \( aAx+aI\in Ax+I\). Mais comme \( I\) est maximal, \( J=I\) ou \( J=A\).

		Si \( J=I\), nous aurions que pour tout \( a\in A\) et pour tout \( i\in I\), \( ax+i\in I\). En particulier pour \( a=1\) et \( i=0\) nous aurions \( x\in I\), ce qui est contraire à l'hypothèse faite sur \( x\).

		Donc \( J=A\). En particulier, \( 1\in J\), c'est-à-dire qu'il existe \( a\in A\) et \( i\in I\) tels que \( ax+i=1\). En passant aux classes, \( [ax]=1\), c'est-à-dire \( [a][x]=1\) qui signifie que \( [a]\) est un inverse de \( [x]\) dans \( A/I\).

		Nous avons prouvé que \( A/I\) est un corps.

		\spitem[\ref{ITEMooGLXSooUjINqR}, \( \Leftarrow\)]
		Si \( x\in A\setminus I\), il faut prouver que tout idéal contenant \( I\) et \( x\) est \( A\).

		Un idéal contenant \( I\) et \( x\) doit contenir l'idéal \( J=Ax+I\). Comme \( x\notin I\), nous avons \( [x]\neq 0\) dans \( A/I\). Donc \( [x] \) est inversible et il existe \( a\in A\) tel que \( [ax]=[1]\). C'est-à-dire que \( ax-1\in I\). Nous avons alors
		\begin{equation}
			1=ax+\underbrace{(1-ax)}_{\in I}.
		\end{equation}
		C'est-à-dire que \( 1\in Ax+I\) et donc \( Ax+I=A\).
	\end{subproof}
	Enfin nous prouvons que tout idéal maximal propre est premier.

	Si \( I\) est maximal, \( A/I\) est un corps par le point \ref{ITEMooGLXSooUjINqR}, et vu que \( I\) est propre, le corps \( A/I\) n'est pas réduit à \( \{ 0 \}\). Donc le lemme \ref{LEMooIKNMooMfvQnu} dit que \( A/I\) est un anneau intègre. Le point \ref{ITEMooUGBTooOGrnWl} dit alors que \( I\) est un idéal premier.
\end{proof}

\begin{remark}
	Puisqu'un corps peut être réduit à \( \{0\}\), dans \ref{ITEMooGLXSooUjINqR}, l'idéal peut être \( A\). Mais pas dans \ref{ITEMooTFFQooOUajFw}, parce qu'un idéal premier est propre, ça fait partie de la définition \ref{DEFooAQSZooVhvQWv}.
\end{remark}

\begin{proposition}[\cite{ooOYKZooOJBDHS}]     \label{PROPooHABIooBZZQMj}
	Si \( A\) est un anneau commutatif intègre, alors un idéal \( I\) dans \( A\) est premier si et seulement si \( A/I\) est intègre.
\end{proposition}

\begin{proof}
	Supposons que \( I\) soit un idéal premier. Si \( [a],[b] \in A/I\)  sont tels que \( [a][b]=0\), alors \( [ab]=0\), ce qui signifie que \( ab\in I\). Mais alors, puisque \( I\) est premier, soit \( a\) soit \( b\) est dans \( I\). Cela signifie que soit \( [a]\) soit \( [b]\) est nul dans \( A/I\). Cela prouve que \( A/I\) est un anneau intègre.

	Dans l'autre sens, nous supposons que \( A/I\) est intègre. Cela implique immédiatement que \( I\neq A\) parce que \( A/A\) n'est pas un anneau intègre (tout le monde est évidemment diviseur de zéro).

	Soient donc \( a,b\in A\) tels que \( ab\in I\). Alors \( [a][b]=[ab]=0\) dans \( A/I\), mais comme \( A/I\) est intègre, cela implique que soit \( [a]\) soit \( [b]\) est nul. Autrement dit, soit \( a\) soit \( b\) est dans \( I\).
\end{proof}

\begin{proposition}		\label{PROPooSGLNooBYKNyo}
	Dans un idéal principal, les conditions suivantes sont équivalentes :
	\begin{enumerate}
		\item
		      L'élément	\( p\) est premier.
		\item
		      L'élément \( p\) est irréductible.
		\item
		      L'idéal \( pA\) est un idéal maximal.
	\end{enumerate}
	%TODOooPSCUooQnrpBR. Prouver ça.
\end{proposition}


%+++++++++++++++++++++++++++++++++++++++++++++++++++++++++++++++++++++++++++++++++++++++++++++++++++++++++++++++++++++++++++
\section{Anneau intègre}
%+++++++++++++++++++++++++++++++++++++++++++++++++++++++++++++++++++++++++++++++++++++++++++++++++++++++++++++++++++++++++++
\label{SECAnneauxIntegres}





\begin{lemma}     \label{LEMooZSMEooUmSXWZ}
	Un corps\footnote{Définition~\ref{DefTMNooKXHUd}.} est un anneau intègre.
\end{lemma}

\begin{proof}
	En effet, soient un corps \( \eK\) et deux éléments \( x,y\in \eK\) tels que \( xy=0\). Si \( y\) est inversible, alors nous pouvons multiplier par \( y^{-1}\) pour trouver \( x=0\). Cela prouve que \( \eK\) est un anneau intègre.
\end{proof}


\begin{example}
	L'anneau \( \eZ/6\eZ\) n'est pas intègre parce que \( 3\cdot 2=0\) alors que ni \( 3\) ni \( 2\) ne sont nuls.
\end{example}

Nous verrons au théorème~\ref{ThoBUEDrJ} que l'anneau \( A\) est intègre si et seulement si \( A[X]\) est intègre.

\begin{corollary}   \label{CorZnInternprem}
	L'anneau \( \eZ/n\eZ\) est intègre si et seulement si \( n\) est premier.
\end{corollary}

\begin{proof}
	Supposons que \( n\) soit premier. La proposition \ref{PropZpintssiprempUzn} donne les inversibles de \( \eZ/n\eZ\) par
	\begin{equation}
		U(\eZ/n\eZ)=\{ \overline x\tq 0\leq x\leq n\tq\pgcd(x,n)=1 \}.
	\end{equation}
	Mais comme \( n\) est premier, \( \pgcd(x,n)=1\) pour tout \( x\), et donc tous les éléments de \( \eZ/n\eZ\) sont inversibles. Donc \( \eZ/n\eZ\) est intègre.

	Si \( n\) n'est pas premier, alors \( n=pq\) avec \( 1<p\leq q<n\). Alors
	\begin{equation}
		[p]_n[q]_n=[pq]_n=[0]_n.
	\end{equation}
	Donc lorsque \( n\) n'est pas premier,  l'anneau \( \eZ/n\eZ\) possède des diviseurs de zéro et n'est alors pas intègre.
\end{proof}



\begin{proposition}[Thème~\ref{THEMEooZYKFooQXhiPD}, \cite{MonCerveau}] \label{PropomqcGe}
	Soit \( A\) un anneau principal\footnote{Définition \ref{DEFooGWOZooXzUlhK}.} qui n'est pas un corps. Pour un idéal propre \( I\) de \( A\), les conditions suivantes sont équivalentes :
	\begin{enumerate}
		\item       \label{ITEMooNOVFooEHtcwE}
		      \( I\) est un idéal maximal\footnote{Définition \ref{DEFIdealMax}.};
		\item       \label{ITEMooMQWVooNocVEU}
		      \( I\) est un idéal premier non nul\footnote{Définition \ref{DEFooAQSZooVhvQWv}.};
		\item       \label{ITEMooJBXGooEISNuW}
		      il existe \( p\) irréductible\footnote{Définition \ref{DeirredBDhQfA}.} dans \( A\) tel que \( I=(p)\).
	\end{enumerate}
\end{proposition}

\begin{proof}
	En plusieurs implications.
	\begin{subproof}
		\spitem[\ref{ITEMooNOVFooEHtcwE} implique~\ref{ITEMooMQWVooNocVEU}]
		Par hypothèse, \( I\) est un idéal propre, de plus il n'est pas égal à \( \{ 0 \}\), parce que lorsque \( A\) et \( \{ 0 \} \) sont les seuls idéaux, nous avons un corps (proposition~\ref{PROPooUOCVooZGAVVk}). Étant donné que \( I\) est un idéal maximal, le quotient \( A/I\) est un corps par la proposition~\ref{PROPooSHHWooCyZPPw}.

		Soient maintenant, pour entrer dans le vif du sujet, des éléments \( a,b\in A\) tels que \( ab\in I\). Dans le corps \( A/I\) nous avons \( [ab]=0\), et par définition du produit dans le quotient, \( [a][b]=0\). Par intégrité de l'anneau \( A/I\) (un corps est un anneau intègre, lemme \ref{LEMooZSMEooUmSXWZ}) nous avons soit \( [a]=0\), soit \( [b]=0\), soit les deux en même temps. Dans tous les cas, soit \( a\) soit \( b\) est dans \( I\).

		\spitem[\ref{ITEMooMQWVooNocVEU} implique~\ref{ITEMooJBXGooEISNuW}]
		Maintenant \( I\) est un idéal premier non réduit à \( \{ 0 \}\). Puisque \( A\) est un anneau principal, il existe \( x\in A\) tel que \( I=(x)\). Nous devons prouver que \( x\) peut être choisi irréductible; et nous allons faire plus : nous allons prouver que \( x\) ne peut être que irréductible\quext{ça me semble un peu trop facile. Lisez attentivement, et écrivez-moi pour dire si vous êtes d'accord ou pas.}.

		Supposons que \( x\) ne soit pas irréductible. Alors il existe \( a,b\in A\) non inversibles tels que \( x=ab\). Si \( a\in (x)\) alors il existe \( k\in A\) tel que \( a=xk\), et en particulier, \( a=abk\), c'est-à-dire \( 1=bk\) (parce que \( A\) est principal et donc intègre). Cela signifie que \( b\) est inversible alors que nous avions dit qu'il ne l'était pas. Nous en déduisons que \( a\) n'est pas dans \( (x)\). On montre de manière similaire que \( b\) n'est pas dans \( (x)\) non plus.

		Nous nous retrouvons donc avec \( a,b\in A\) tel que \( ab\in I\) sans que ni \( a\) ni \( b\) ne soient dans \( I\). Cela contredit le fait que \( I\) soit un idéal premier. En conclusion, \( x\) est irréductible.

		\spitem[\ref{ITEMooJBXGooEISNuW} implique~\ref{ITEMooNOVFooEHtcwE}]
		Nous avons \( I=(p)\) avec \( p\) irréductible dans \( A\). Supposons que \( J\) est un idéal différent de \( A\) contenant \( I\). Comme \( A\) est principal, il existe \( y\in A\) tel que \( J=(y)\). En particulier \( p\in J\), donc \( p=ay\) pour un certain \( a\in A\). Mais \( p\) est irréductible, donc soit \( a\) est inversible, soit \( y\) est inversible. Si \( y\) est inversible, alors \( J=A\), ce qui est exclu. Si \( a\) est inversible, alors \( (y)=(p)\), et \( I=J\).
	\end{subproof}
\end{proof}

\begin{normaltext}
	Dans la proposition \ref{PropomqcGe}, l'hypothèse d'idéal propre est importante. En effet dans le cas \( I=A\), nous avons évidemment que \( I\) est un idéal maximum. Mais \( A\) n'est d'abord pas un idéal premier parce qu'un idéal premier doit être strictement inclus dans l'anneau. Et ensuite, \( A\) est en général loin d'être garanti d'être égal à \( (p)\) pour un de ses éléments \( p\).
\end{normaltext}

\begin{proposition}     \label{PropoTMMXCx}
	Soit \( A \) un anneau principal, et soit \( p \in A \) un élément irréductible. Alors
	\begin{enumerate}
		\item
		      \( (p)\) est un idéal maximum.
		\item       \label{ITEMooKPJQooWuPZXS}
		      \( A/(p)\) est un corps.
	\end{enumerate}
\end{proposition}

\begin{proof}
	Nous notons \( I=(p)\). Soit un idéal \( J\) contenant \( I\). Comme \( A\) est principal, \( J\) est monogène : \( J=(q)\). Mais comme \( p\) est dans \( I\) qui est dans \( J\), il existe \( a\in A\) tel que \( p=qa\).

	Puisque \( p\) est irréductible, soit \( q\), soit \( a\) est inversible. Si \( q\) est inversible, alors \( J=A\). Si \( a\) est inversible, alors nous avons \( p=qa\), donc \( q=pa^{-1}\), ce qui signifie que \( q\in(p)\) et donc que \( J=I\).

	Cela prouve que \( (p)\) est un idéal maximum.

	Le fait que \( A/(p)\) soit un corps est maintenant la proposition~\ref{PROPooSHHWooCyZPPw}.
\end{proof}

\begin{example}
	L'anneau \( \eZ\) est principal parce qu'il est intègre et que ses seuls idéaux sont les \( n\eZ\) qui sont principaux : \( n\eZ\) est engendré par \( n\).
\end{example}

\begin{example}[Les idéaux de \( \eZ/n\eZ\)]       \label{EXooCJRPooYkWdyr}

	Les idéaux de \( \eZ/n\eZ\) sont principaux, mais l'anneau \( \eZ/n\eZ\) n'est pas principal lorsque \( n\) n'est pas premier. Nous allons voir ça.

	\begin{subproof}
		\spitem[Les idéaux de \( \eZ/n\eZ\) sont principaux]
		Soit un idéal \( S\) dans \( \eZ/n\eZ\). Nous considérons la projection canonique \( \phi\colon \eZ\to \eZ/n\eZ\). La proposition~\ref{PropIJJIdsousphi} dit que  \( S=\phi(J)\) où \( J\) est un idéal de \( \eZ\) contenant \( n\eZ\). Mais le corolaire~\ref{CORooLINXooBlUKPG} nous dit qu'alors \( J=m\eZ\) pour un certain \( m\). Pour que \( m\eZ\) contienne \( n\eZ\), il faut que \( m\) divise \( n\).

		Bref, \( S=\phi(m\eZ)\) avec \( m\divides n\). Nous montrons maintenant que \( S\) est engendré par \( [m]_n\). D'abord, l'élément \( [m]_n\) est bien dans \( \phi(m\eZ)\). Ensuite un élément de \( \phi(m\eZ)\) est de la forme
		\begin{equation}
			[km]_n=k[m]_n\in ([m]_n).
		\end{equation}
		Donc \( S\subset ([m]_n)\). Et l'inclusion dans l'autre sens est tout aussi immédiate : un élément de \( ([m]_n)\) est de la forme
		\begin{equation}
			k[m]_n=[km]_n=\phi(km)\in \phi(m\eZ).
		\end{equation}

		\spitem[Si \( n\) n'est pas premier, \( \eZ/n\eZ\) n'est pas principal]
		Le fait est que lorsque \( n\) n'est pas premier, \( \eZ/n\eZ\) n'est pas intègre (corolaire~\ref{CorZnInternprem}).

		\spitem[Moralité]
		Un anneau comme \( \eZ/6\eZ\) est un anneau dont tous les idéaux sont principaux, mais qui n'est pas principal.
	\end{subproof}
\end{example}

\begin{example}
	Nous verrons dans la proposition~\ref{PROPooVWRPooGQMenV} que l'anneau des fonctions holomorphes sur un compact de \( \eC\) est principal.
\end{example}

%+++++++++++++++++++++++++++++++++++++++++++++++++++++++
\section{Théorèmes chinois et idéaux}
%+++++++++++++++++++++++++++++++++++++++++++++++++++++++


\begin{proposition}[\cite{BIBooSALPooIrQwPd}]	\label{PROPooFGDKooPiDCAE}
	Soit un anneau \( A\). Soient des idéaux \( I_1,\ldots,I_n\) premiers deux à deux\footnote{Définition \ref{DEFooZFYSooJGZndS}.}. Nous posons
	\begin{equation}
		J_i=\bigcap_{\substack{ j=1 \\ j\neq i } }^nI_j.
	\end{equation}
	Alors les idéaux \( J_i\) sont premiers dans leur ensemble.
\end{proposition}

\begin{proof}
	Nous y allons par récurrence. Pour \( n=2\) c'est facile parce que \( J_1=I_2\) et \( J_2=I_1\). Donc \( J_1+J_2=I_2+I_1=A\).

	Supposons que la propriété soit vraie pour \( n\), et voyons pour \( n+1\). Soient donc des idéaux \( I_1\),\ldots, \( I_{n+1}\) premiers deux à deux.  Pour \( i=1,\ldots,n\) nous posons
	\begin{equation}
		K_i=\bigcap_{\substack{ j=1 \\ j\neq i } }^nI_j.
	\end{equation}
	Par hypothèse de récurrence les \( K_i\) sont premiers dans leur ensemble. Il existe \( (k_1,\ldots,k_n)\in K_1\times\ldots\times K_n\) tels que
	\begin{equation}
		k_1+\ldots+k_n=1.
	\end{equation}

	Par ailleurs \( I_i\) est premier avec \( I_{n+1}\), donc il existe \( a_i\in I_i\) et \( b_i\in I_{n+1}\) tels que \( a_i+b_i=1\). Nous avons donc l'égalité
	\begin{equation}
		k_1(a_1+b_1)+\ldots+k_n(a_n+b_n)=1
	\end{equation}
	que nous réarrangeons en
	\begin{equation}		\label{EQooDOFHooQUWtss}
		(k_1a_1+\ldots+k_na_n)+ k_1b_1+\ldots+k_nb_n=1.
	\end{equation}
	Vu que \( k_i\in K_i\) et \( a_i\in I_i\) et vu que les idéaux sont absorbants,
	\begin{equation}
		k_ia_i  \in I_i\cap\bigcap_{\substack{ j=1 \\ j\neq i }  }^nI_j=\bigcap_{j=1}^nI_j=\bigcap_{\substack{ j=1 \\ j\neq n+1 }  }^{n+1}I_j=J_{n+1}.
	\end{equation}
	Vu que cela est vrai pour chaque \( i\), et vu que les idéaux sont stables par somme, nous avons
	\begin{equation}
		k_1a_1+\ldots+k_na_n\in J_{n+1}.
	\end{equation}

	De même pour chaque \( i\) nous avons \( k_i\in I_i\) et \( b_i\in I_{n+1}\), et donc
	\begin{equation}
		k_ib_i\in K_1\cap I_{n+1}=I_{n+1}\cap\bigcap_{\substack{ j=1 \\ j\neq i }  }^nI_j=\bigcap_{\substack{ j=1 \\ j\neq i }  }^{n+1}I_j=J_i.
	\end{equation}

	Nous pouvons revenir à l'équation \eqref{EQooDOFHooQUWtss} :
	\begin{equation}
		\underbrace{(k_1a_1+\ldots+k_na_n)}_{\in J_{n+1}}+ \underbrace{k_1b_1}_{\in J_1}+\ldots+\underbrace{k_nb_n}_{\in J_n}=1.
	\end{equation}
	Nous avons donc démontré que \( 1\in J_1+\ldots+J_{n+1}\) et donc que ces idéaux sont premiers dans leur ensemble.
\end{proof}

\begin{proposition}[Théorème des restes chinois version congruence\cite{BIBooSALPooIrQwPd}]	\label{PROPooKWZLooCHhSjs}
	Soit un anneau \( A\). Soient \( a_1,\ldots,a_n\in A\). Soient des idéaux premiers entre eux \( I_1,\ldots,I_n\). Nous considérons l'ensemble
	\begin{equation}
		S=\{ x\in A\tq x\in[a_i]_{I_i}\,\forall i \}.
	\end{equation}
	Alors
	\begin{enumerate}
		\item		\label{ITEMooTGOEooZfsiKG}
		      \( S\) est non vide.
		\item		\label{ITEMooCECDooQzGovx}
		      Si \( x\in S\) alors \( S=[x]_{I_1\cap\ldots\cap I_n}\).
	\end{enumerate}
\end{proposition}

\begin{proof}
	En deux parties.
	\begin{subproof}
		\spitem[Pour \ref{ITEMooTGOEooZfsiKG}]
		%-----------------------------------------------------------
		Nous commençons par poser
		\begin{equation}
			J_i=\bigcap_{\substack{ j=1 \\ j\neq i }  }^nI_j
		\end{equation}
		qui sont des idéaux premiers dans leur ensemble par \ref{PROPooFGDKooPiDCAE}. Il existe donc \( (j_1,\ldots,j_n)\in J_1\times J_n\) tel que \( j_1+\ldots+j_n=1\). Nous posons alors
		\begin{equation}
			x=a_1j_1+\ldots+a_nj_n,
		\end{equation}
		et nous vérifions que \( x\in S\). Pour cela nous fixons un \( i\) et nous calculons un petit peu :
		\begin{subequations}
			\begin{align}
				x-a_i & =a_1j_1+\ldots+a_nj_n-(a_ij_1+\ldots+a_ij_n)                  \\
				      & =(a_1-a_i)j_1+\ldots+(a_n-a_i)j_n.		\label{SUBEQooKDVXooKfVntq}
			\end{align}
		\end{subequations}
		Notez que dans la dernière somme, le coefficient de \( j_i\) est \( (a_i-a_i)=0\). Pour les autres termes (\( k\neq i\)) nous avons
		\begin{equation}
			(a_k-a_i)j_k\in J_k=\bigcup_{\substack{ l=1 \\ l\neq k }  }^nI_l\subset I_i.
		\end{equation}
		La dernière inclusion est parce que \( i\neq k\), donc il y a un terme \( l=i\) dans l'intersection. Nous avons démontré que chaque terme de la somme \eqref{SUBEQooKDVXooKfVntq} est dans \( I_i\).

		Donc \( x-a_i\in I_i\). Cela étant valable pour tout \( i\), nous avons bien prouvé que \( x\in S\).
		\spitem[Pour \ref{ITEMooCECDooQzGovx}]
		%-----------------------------------------------------------
		Soit \( x\in S\). Nous prouvons les deux inclusions.
		\begin{subproof}
			\spitem[\( S\subset [x]_{I_1\cap\ldots\cap I_n}\)]
			%-----------------------------------------------------------
			Soit \( s\in S\). Vu que \( x\) et \( s\) sont dans \( S\), pour chaque \( i\) nous avons
			\begin{subequations}
				\begin{align}
					s-a_i=e_i \\
					x-a_i=f_i
				\end{align}
			\end{subequations}
			pour certains éléments \( e_i,f_i\in I_i\). En soustrayant, \( s-x=e_i-f_i\in I_i\). Un tel calcul étant valable pour chaque \( I\), nous avons \( s-x\in I_1\cap\ldots \cap I_n\), c'est à dire \( s\in [x]_{I_1\cap\ldots\cap I_n}\).

			\spitem[\( [x]_{I_1\cap\ldots\cap I_n}\subset S\)]
			%-----------------------------------------------------------
			Nous posons \( s=x+e\) avec \( e\in I_1\cap\ldots \cap I_n\), et nous prouvons que \( s\in S\). Nous avons
			\begin{equation}
				s-a_i=\underbrace{x-a_i}_{\in I_i}-\underbrace{e}_{\in I_i}\in I_i.
			\end{equation}
			Donc \( s\in [a_i]_{I_i}\) pour chaque \( i\). Donc \( s\in S\).
		\end{subproof}
	\end{subproof}
\end{proof}


\begin{theorem}[Théorème des restes chinois version anneaux quotients\cite{BIBooSALPooIrQwPd,BIBooHDJZooEVwGMS}]	\label{THOooCBWWooEGjeSV}
	Soient un anneau \( A\) et des idéaux \( I_1,\ldots,I_n\) premiers entre eux deux à deux.
	\begin{enumerate}
		\item		\label{ITEMooPNFZooInhOQo}
		      Il existe une unique application \(f \colon A/I_1\cap\ldots\cap I_n\to A/I_1\times\ldots\times A/_n  \) vérifiant
		      \begin{equation}
			      f\big( [x] \big)=\big( [x]_{I_1},\ldots,[x]_{I_n} \big)
		      \end{equation}
		      pour tout \( x\in A\).
		\item		\label{ITEMooYSDNooSpMVcB}
		      Cette application est un isomorphisme d'anneaux.
	\end{enumerate}
\end{theorem}

\begin{proof}
	En plusieurs points.
	\begin{subproof}
		\spitem[Existence et unicité]
		%-----------------------------------------------------------
		Nous considérons l'application
		\begin{equation}
			\begin{aligned}
				g\colon A & \to A/I_1\times \ldots \times A/I_n             \\
				x         & \mapsto \big( [x]_{I_1},\ldots,[x]_{I_n} \big).
			\end{aligned}
		\end{equation}
		Affin d'utiliser la proposition \ref{PROPooCONJooTpJwBe} nous vérifions que si \( u\in [x]_{I_1\cap\ldots \cap I_n}\) alors \( g(x)=g(y)\). En effet dire \( y\in[x]\) signifie \( x-y\in I_1\cap\ldots\cap I_n\) et donc pour tout \( j\) nous avons \( x-y\in I_j\). Du coup pour tout \( j\) nous avons \( [x]_{I_j}=[y]_{I_j}\). Voila. Donc la proposition \ref{PROPooCONJooTpJwBe} prouve le point \ref{ITEMooPNFZooInhOQo}.

		\spitem[Injective]
		%-----------------------------------------------------------
		Soient \( \alpha,\beta\in A/I_1\cap\ldots\cap I_n\) tels que \( f(\alpha)=f(\beta)\). Nous considérons \( x\in \alpha\) et \( y\in \beta\). Nous avons
		\begin{subequations}
			\begin{align}
				f(\alpha) & =\big( [x]_{I_1},\ldots,[x]_{I_n} \big)  \\
				f(\beta)  & =\big( [y]_{I_1},\ldots,[y]_{I_n} \big).
			\end{align}
		\end{subequations}
		Donc \( [x]_{I_i}=[y]_{I_i}\) pour tout \( i\). Pour chaque \( i\), il existe un \( e_i\in I_i\) tel que \( x-y=e_i\). Forcément tous les \( e_i\) sont égaux et nous avons donc \( x-y\in I_1\cap\ldots\cap I_n\). Cela prouve que \( \alpha=\beta\) et donc que \( f\) est injective.

		\spitem[Surjective]
		%-----------------------------------------------------------
		Soient \( \alpha_i\in A/I_i\). Nous avons besoin de trouver \( \alpha\in A/I_I\cap\ldots\cap I_n\) tel que \( f(\alpha)=(\alpha_1,\ldots,\alpha_n)\). Pour chaque \( \alpha_i\) nous considérons \( a_i\in \alpha_i\) et nous cherchons \( x\in A\) tel que
		\begin{equation}
			x\in[a_i]_{I_i}
		\end{equation}
		pour tout \( i\). La version congruence du théorème chinois (proposition \ref{PROPooKWZLooCHhSjs}) dit qu'un tel \( x\) existe. En prenant \( \alpha=[x]_{I_1\cap\ldots\cap I_n}\) nous avons bien \( f(\alpha)=(\alpha_1,\ldots,\alpha_n)\).

		\spitem[Morphisme]
		%-----------------------------------------------------------
		L'application \( f\) est un morphisme d'anneaux parce que si \( \alpha,\beta\in A/I_1\cap\ldots\cap I_n\) et si \( x\in \alpha\) et \( y\in \beta\), alors
		\begin{equation}
			f(\alpha+\beta)=f([x]+[y])=f([x+y])=\big( [x+y]_1,\ldots, \big)=\big( [x]_1+[y]_1,\ldots, \big)=f(\alpha)+f(\beta).
		\end{equation}
		Le même genre de vérification marche pour le produit.
	\end{subproof}
\end{proof}

Pour la suite nous rappelons que \( [x]_a\) est la classe de \( x\) pour le quotient par l'idéal \( (a)\). Donc
\begin{equation}
	[x]_a=\{ y\in A\tq x-y\in (a) \}.
\end{equation}
Notez au passage que
\begin{equation}		\label{EQooJAAQooJCcmVE}
	[xa]_a=[0]_a
\end{equation}



\begin{theorem}[Théorème des restes chinois\cite{BIBooJVIDooLZroz}]\index{théorème!chinois!anneau principal}	\label{ThofPXwiM}
	Soit un anneau principal \( A\) ainsi que des éléments \( a_1,\ldots,a_r\in A\setminus\{ 0 \}\) deux à deux premiers entre eux. Nous notons \( a\) le produit \( a=a_1\ldots a_r\).

	\begin{enumerate}
		\item		\label{ITEMooWBSWooGWjnNx}
		      Il existe une unique application
		      \begin{equation}
			      f \colon A/(a)\to A/(a_1)\times\ldots\times A/(a_r)
		      \end{equation}
		      telle que
		      \begin{equation}
			      f([x]_a)=\big( [x]_{a_1},\ldots,[x]_{a_r} \big)
		      \end{equation}
		      pour tout \( x\in A\).
		\item		\label{ITEMooANYPooGPgEQc}
		      Cette application \( f\) est un isomorphisme d'anneaux.
		\item		\label{ITEMooTUOKooHcSwTD}
		      Il existe \( u_1,\ldots,u_r\in A\) tels que
		      \begin{equation}
			      \sum_{i=1}^ru_i\frac{ a }{ a_i }=1.
		      \end{equation}
		      Notez ici que \( a_i\) n'est pas spécialement inversible, mais \( a\) étant le produit de tous les \( a_i\), ça va.
		\item		\label{ITEMooGXOEooUdKmzU}
		      Si des \( u_i\in A\) vérifient \( \sum_{i=1}^ru_ia/a_i=1\) alors la réciproque
		      \begin{equation}
			      f^{-1} \colon A/(a_1)\times \ldots\times A/(a_r)\to  	A/(a)
		      \end{equation}
		      vérifie
		      \begin{equation}
			      f^{-1}\big( [x_1]_{a_1},\ldots,[x_r]_{a_r} \big)=\sum_{i=1}^r[x_iu_i\frac{ a }{ a_i }]
		      \end{equation}
		      pour tout \( x_1,\ldots,x_r\in A\).
	\end{enumerate}
\end{theorem}

\begin{proof}
	En plusieurs points.
	\begin{subproof}
		\spitem[Pour \ref{ITEMooWBSWooGWjnNx} et \ref{ITEMooANYPooGPgEQc}]
		%-----------------------------------------------------------
		Vu que les \( a_i\) sont premiers entre eux, nous avons \( a=a_1\ldots a_r\in\ppcm(a_1,\ldots,a_r)\) et donc \( (a)=\bigcap_{i=1}^r(a_i)\) par \ref{PROPooYTMYooEYxuQc}. Pour l'existence, l'unicité et fait que ce soit un isomorphisme d'anneaux, le théorème \ref{THOooCBWWooEGjeSV} fait tout le boulot.

		\spitem[Pour \ref{ITEMooTUOKooHcSwTD}]
		%-----------------------------------------------------------
		Les \( a_i\) sont premiers entre eux. Montrons que les \( a/a_i\) aussi. Supposons pour cela que \( p\) divise tous les \( a/a_i\). En particulier \( p\divides a/a_1=a_2\dots a_r\). Vu que les \( a_2,\ldots,a_r\) sont premiers entre eux, le lemme de Gauss \ref{LemSdnZNX} dit qu'il existe un \( k=2,\ldots,r\) tel que \( p\divides a_k\). Mais \( p\) divise aussi \( a/a_k\). Donc il existe \( l\neq k\) tel que \( p\divides a_l\). L'élément \( p\) est un diviseur commun de \( a_k\) et \( a_l\). Pas possible parce que les \( a_i\) sont premiers deux à deux.

		Nous avons \( 1\in\pgcd(a_1,\ldots,a_r)\) et donc Bézout \ref{PROPooXQKMooWJlEFq} dit qu'il existe des \( u_i\) tels que
		\begin{equation}	\label{EQooMBGBooKoGKwJ}
			\sum_{i=1}^nu_i\frac{ a }{ a_i }=1.
		\end{equation}

		\spitem[Pour \ref{ITEMooGXOEooUdKmzU}]
		%-----------------------------------------------------------
		Lorsque nous caculons
		\begin{equation}
			f\big( \sum_{i=1}^r[x_iu_ia/a_i] \big),
		\end{equation}
		nous avons un vecteur d'éléments de la forme \( [ \sum_{i=1}^rx_iu_i\frac{ a }{ a_i }]_{a_k}\). En vertu de \eqref{EQooJAAQooJCcmVE}, tous les termes avec \( i\neq k\) sont nuls et donc
		\begin{equation}		\label{EQooZBYVooAoUnfM}
			f\big( \sum_{i=1}^r[x_iu_ia/a_i] \big)=\big( [x_1u_1\frac{ a }{ a_1 }]_{a_1},\ldots,[x_ru_r\frac{ a }{ a_r }]_{a_r} \big).
		\end{equation}

		Par ailleurs nous prenons la classe par rapport à \( (a_k)\) des deux côtés de \eqref{EQooMBGBooKoGKwJ}. Comme précédemment tous les termes \( i\neq k\) sont nuls et il reste
		\begin{equation}
			[u_ka/a_k]_{a_k}=[1]_{a_k}.
		\end{equation}
		Nous avons donc
		\begin{equation}
			[x_ku_k\frac{ a }{ a_k }]=[x_k]_{a_k}\underbrace{[u_k\frac{ a }{ a_k }]_{a_k}}_{=[1]_{a_k}}=[x_k]_{a_k}.
		\end{equation}
		En remettant dans \eqref{EQooZBYVooAoUnfM} nous trouvons ce qu'il faut :
		\begin{equation}
			f\big( \sum_{i=1}^r[x_iu_ia/a_i] \big)=\big( [x_1]_{a_1},\ldots,[x_k]_{a_k} \big).
		\end{equation}
	\end{subproof}
\end{proof}




Nous avons vu dans \ref{PROPooPJGLooQSrJTU} que \( \eZ\) est principal; le lemme suivant nous dit que \( \eZ[X]\) n'est lui, pas principal.
\begin{lemma}[\cite{ooRQHSooEBZpKe}]        \label{LEMooDJSUooJWyxCL}
	Si \( A\) est un anneau intègre\footnote{Définition \ref{DEFooTAOPooWDPYmd}.} qui n'est pas un corps, alors \( A[X]\) n'est pas principal.
\end{lemma}

\begin{proof}
	Soit un élément non nul \( a\in A\).
	\begin{subproof}
		\newcommand{\foo}{A[X]}
		\spitem[Un idéal principal contenant \( a\) et \( X\) est {A[X]}]

		Soit \( (P)\) un idéal principal contenant \( a\) et \( X\). Puisque \( a\in(P)\), il existe \( Q\) tel que \( a=QP\). Donc \( P\) divise \( a\) dans \( \eZ[X]\). L'égalité des degrés indique que \( P\) est un polynôme constant, c'est-à-dire en réalité un élément de \( A\). Soit \( P=k\in A\).

		Comme \( P\) divise \( X\), nous avons aussi \( X=kQ\) pour un certain \( Q\in \eZ[X]\). L'égalité des degrés dit qu'il existe \( k'\in A\) tel que \( Q=k'X\) et donc \( Q=k'X=k'kQ\), ce qui implique que \( kk'=1\). L'idéal engendré par \( k\) contient donc en particulier \( kk'=1\) et donc contient \( A[X]\) en entier :
		\begin{equation}
			1=k'k\in k'(P)=(P).
		\end{equation}

		\spitem[Si \( (a,X)=\foo\) alors \( a\) est inversible]

		Si \( (a,X)=A[X]\), en particulier, \( 1\in (a,X)\), ce qui signifie qu'il existe des polynômes \( U,V\in A[X]\) tels que
		\begin{equation}
			1=UX+Va.
		\end{equation}
		Nous évaluons cette égalité en \( 0\) : comme \( (UX)(0)=0\), nous avons \( 1=V(0)a\), ce qui signifie que \( V(0)\) est un inverse de \( A\). Donc \( a\) est inversible.

		\spitem[Si \( a\) n'est pas inversible alors \( (a,X)\) n'est pas principal]

		Si \( (a,X)\) était principal, alors nous aurions, par ce qui est dit plus haut, \( (a,X)=A[X]\). Mais cette dernière égalité impliquerait que \( a\) soit inversible.
	\end{subproof}
	En conclusion, si \( A\) n'est pas un corps, il possède un élément ni nul ni inversible. Dans ce cas, l'idéal \( (a,X)\) n'est pas principal dans \( A[X]\) et nous en déduisons que \( A[X]\) n'est pas un anneau principal.
\end{proof}

Nous verrons dans le lemme~\ref{LEMooIDSKooQfkeKp} que si \( \eK\) est un corps, alors \( \eK[X]\) est principal.


%+++++++++++++++++++++++++++++++++++++++++++++++++++++++++++++++++++++++++++++++++++++++++++++++++++++++++++++++++++++++++++
\section{Sous-groupe normal}
%+++++++++++++++++++++++++++++++++++++++++++++++++++++++++++++++++++++++++++++++++++++++++++++++++++++++++++++++++++++++++++

\begin{proposition}\label{propGroupeNormal}
	Soit \( N\) un sous-groupe de \( G\). Les propriétés suivantes sont équivalentes :
	\begin{enumerate}
		\item       \label{ITEMooDYEUooOuKEqQ}
		      \( N\) est normal\footnote{Définition \ref{DEFooNIIMooFkZgvX}.} dans \( G\).
		\item       \label{ITEMooPYTEooZhvrUa}
		      \( N\) est une union de classes de conjugaison\footnote{Définition \ref{DEFooOLXPooWelsZV}.} de \( G\),
		\item       \label{ITEMooJWTLooBRmriQ}
		      \( gNg^{-1}\subseteq N\) pour tout \( g\in G\),
		\item       \label{ITEMooVRZIooAorhRY}
		      \( gNg^{-1}= N\) pour tout \( g\in G\),
		\item       \label{ITEMooJGUOooYshOZa}
		      \( gN=Ng\) pour tout \( g\in G\),
		\item       \label{ITEMooMRYRooZifCCe}
		      Le normalisateur\footnote{Définition \ref{DEFooZTSMooBislIy}.} de \( N\) est \( G\) : \( \mN_G(N)=G\).
	\end{enumerate}
\end{proposition}

\begin{proof}
	En plusieurs parties.
	\begin{subproof}
		\spitem[\ref{ITEMooDYEUooOuKEqQ} implique \ref{ITEMooJWTLooBRmriQ}]
		C'est la définition de sous-groupe normal.
		\spitem[\ref{ITEMooJWTLooBRmriQ} implique \ref{ITEMooVRZIooAorhRY}]
		Soit \( g\in G\). Nous avons \( gNg^{-1}\subset N\), mais aussi (en appliquant l'hypothèse à \( g^{-1}\)) \( g^{-1}Ng\subset N\). En combinant nous avons
		\begin{equation}
			N\subset g(g^{-1} Ng)g^{-1}\subset g Ng^{-1}.
		\end{equation}
		Nous avons l'inclusion dans les deux sens. Donc l'égalité.
		\spitem[\ref{ITEMooVRZIooAorhRY} implique \ref{ITEMooJGUOooYshOZa}]
		Soit \( g\in G\). Un élément général de \( gN\) est de la forme \( gn\) avec \( n\in N\). Nous devons trouver un \( n'\in N\) tel que \( gn=n'g\). En posant \( n'=gng^{-1}\) nous avons
		\begin{equation}
			n'=gng^{-1}\in gNg^{-1}\subset N.
		\end{equation}
		Il est immédiat de prouver que \( gn=n'g\). Cela prouve que \( gN\subset Ng\).

		Le même raisonnement donne \( Ng\subset gN\).
		\spitem[\ref{ITEMooJGUOooYshOZa} implique \ref{ITEMooJWTLooBRmriQ}]
		Un élément de \( g Ng^{-1}\) est \( a=gng^{-1}\) avec \( n\in N\). Nous devons prouver que \( a\in N\). Puisque \( gn\in gN\), par hypothèse il existe \( n'\) tel que \( gn=n'g\). En remplaçant dans la définition de \( a\),
		\begin{equation}
			a=gng^{-1}=n'gg^{-1}=n'\in N.
		\end{equation}
		\spitem[\ref{ITEMooJGUOooYshOZa} implique \ref{ITEMooPYTEooZhvrUa}]
		Pour chaque \( a\in G\) nous notons \( C_a\) la classe de conjugaison de \( a\) dans \( G\) :
		\begin{equation}        \label{EQooJIEVooCAshfe}
			C_a=\{ gag^{-1}\tq g\in G \}.
		\end{equation}
		Comme \( a\in C_a\) (prendre \( g=e\) dans \eqref{EQooJIEVooCAshfe}.) nous avons forcément
		\begin{equation}
			N\subset\bigcup_{n\in N}C_n.
		\end{equation}
		Prouvons maintenant l'inclusion inverse. Nous avons déjà prouvé que \ref{ITEMooJGUOooYshOZa} implique \ref{ITEMooJWTLooBRmriQ}. Donc si \( n\in N\), alors \( gng^{-1}\in N\). Nous avons alors
		\begin{equation}
			C_n=\{ gng^{-1}\tq g\in G \}\subset N.
		\end{equation}
		Donc il est vrai que \( N=\bigcup_{n\in N}C_n\).
		\spitem[\ref{ITEMooPYTEooZhvrUa} implique \ref{ITEMooDYEUooOuKEqQ}]
		Nous supposons que \( N\subset G\) est un sous-groupe de la forme
		\begin{equation}
			N=\bigcup_{a\in I}C_a
		\end{equation}
		où \( I\) est une partie de \( G\). Nous devons montrer que pour tout \( g\in G\) et pour tout \( n\in N\) nous avons \( gng^{-1}\in N\). Puisque \( n\in N\), il existe \( a\in I\) tel que \( n\in C_a\) et donc il existe \( k\in G\) tel que \( n=kak^{-1}\). Nous avons donc
		\begin{equation}
			gng^{-1}=g(kak^{-1})g^{-1}=(gk)a(gk)^{-1}\in C_a\subset N.
		\end{equation}

		\randomGender{Le lecteur attentif}{La lectrice attentive} aura remarqué l'utilisation de l'axiome du choix. La prudence l'incitera à ne pas le faire remarquer au jury.
		\spitem[\ref{ITEMooMRYRooZifCCe} si et seulement si \ref{ITEMooJGUOooYshOZa}]
		C'est la définition du normalisateur.
	\end{subproof}
\end{proof}

\begin{definition}
	Soit \( g\in G\) et \( n\in \eZ\). Nous définissons \( g^n\) par
	\begin{enumerate}
		\item
		      \( g^0=e\) et \( g^n=gg^{n-1}\) si \( n\) est positif.
		\item
		      si \( n<0\), nous posons \( g^n=(g^{-1})^{-n}\).
	\end{enumerate}
\end{definition}



\begin{lemma}[\cite{PDFpersoWanadoo,BIBooZFPUooIiywbk}]\label{LemHUkMxp}
	Soient un groupe \( G\) et deux sous-groupes normaux\footnote{Sous-groupe normal, définition \ref{DEFooNIIMooFkZgvX}.} \( H\) et \( K\) tels que \( H\cap K=\{ e \}\). Alors :
	\begin{enumerate}
		\item       \label{ITEMooDFVBooSnnlgR}
		      Tout élément de \( H\) commute avec tout élément de \( K\).
		\item       \label{ITEMooVVBGooZSJqjp}
		      \( HK\) est un sous-groupe de \( G\).
		\item       \label{IMTEooPCBZooQoZFOD}
		      L'application
		      \begin{equation}
			      \begin{aligned}
				      \varphi\colon H\times K & \to HK     \\
				      (h,k)                   & \mapsto hk
			      \end{aligned}
		      \end{equation}
		      est un isomorphisme de groupes.
	\end{enumerate}
\end{lemma}

\begin{proof}
	Point par point.
	\begin{subproof}
		\spitem[\ref{ITEMooDFVBooSnnlgR}]
		Soient \( h\in H\) et \( k\in K\). Nous voulons montrer que \( hk=kh\). Pour cela nous considérons l'élément \( a=hkh^{-1}k^{-1}\). Comme \( H \) est normal dans \( G\), nous avons
		\begin{equation}
			kh^{-1}k^{-1}\in H
		\end{equation}
		et donc \( a\in H\). De même \( K\) étant normal dans \( G\), nous avons \( hkh^{-1}\in K\) et donc \( a\in K\). Au final \( a\in H\cap K=\{ e \}\). Nous avons prouvé que
		\begin{equation}
			hkh^{-1}k^{-1}=e,
		\end{equation}
		et donc que \( hk=kh\).
		\spitem[\ref{ITEMooVVBGooZSJqjp}]
		Puisque \( H\) et \( K\) sont des sous-groupes, \( \{ e \}\) est dans les deux, de telle sorte que \( e\in HK\). De plus si \( h_i\in H\) et \( k_i\in K\), la commutativité du point \ref{ITEMooDFVBooSnnlgR} donne
		\begin{equation}
			(h_1k_1)(h_2k_2)=h_1h_2k_1k_2\in HK.
		\end{equation}
		Donc le produit de deux éléments de \( HK\) est dans \( HK\).
		\spitem[\ref{IMTEooPCBZooQoZFOD}]
		En trois sous-parties.
		\begin{subproof}
			\spitem[Morphisme]
			Soient \( h_i\in H\) et \( k_i\in K\). En utilisant la commutativité du point \ref{ITEMooDFVBooSnnlgR} nous avons
			\begin{subequations}
				\begin{align}
					\varphi\big( (h_1,k_1)(h_2,k_2) \big) & =\varphi(h_1h_2,k_1k_2)            \\
					                                      & =(h_1h_2)(k_1k_2)                  \\
					                                      & =(h_1k_1)(h_2k_2)                  \\
					                                      & =\varphi(h_1,k_1)\varphi(h_2,k_2).
				\end{align}
			\end{subequations}
			\spitem[Injectif]
			Si \( \varphi(h_1,k_1)=\varphi(h_2,k_2)\) nous avons successivement
			\begin{subequations}
				\begin{align}
					h_1k_1                 & =h_2k_2       \\
					h_1 k_1 h_2^{-1}       & =k_2          \\
					h_1k_1h_2^{-1}k_1^{-1} & =k_2k_1^{-1}  \\
					h_1h_2^{-1}            & =k_2k_1^{-1}.
				\end{align}
			\end{subequations}
			Le membre de gauche est un élément de \( H\) et le membre de droite un élément de \( K\). Comme \( H\cap K=\{ e \}\) nous avons \( h_1h_2^{-1}=e\) et \( k_2k_1^{-1}=e\), c'est-à-dire \( h_1=h_2\) et \( k_1=k_2\).
			\spitem[Surjectif]
			Un élément général de \( HK\) est \( hk\) avec \( h\in H\) et \( k\in K\), c'est-à-dire \( \varphi(h,k)\).
		\end{subproof}
	\end{subproof}
\end{proof}

\begin{definition}  \label{DefvtSAyb}
	L'\defe{exposant}{exposant!d'un groupe} du groupe \( G\) est le plus petit entier non nul \( n\) tel que \( g^n=e\) pour tout \( g\in G\). S'il n'existe pas un tel \( n\), nous disons que l'exposant du groupe est infini.
\end{definition}

\begin{proposition} \label{PROPooSWHHooOzqWkw}
	À propos d'exposant de groupe et de ppcm.
	\begin{enumerate}
		\item		\label{ITEMooFIPBooFOaPkU}
		      Si l'ensemble des ordres de tous les éléments d'un groupe est majoré, alors l'exposant du groupe est le plus petit commun multiple des ordres des éléments du groupe.
		\item		\label{ITEMooLIREooYEtcdL}
		      Pour un groupe fini, l'exposant est le \( \ppcm\) des ordres des éléments du groupe.
	\end{enumerate}
\end{proposition}

\begin{proof}
	En deux parties.
	\begin{subproof}
		\spitem[Pour \ref{ITEMooFIPBooFOaPkU}]
		%-----------------------------------------------------------
		Soit \( p\) le ppcm des ordres de tous les éléments du groupe. Si \( g\) est d'ordre \( a\), il existe \( k\in \eN\) tel que \( p=ak\). Avec ça nous avons \(  g^{p}=(g^a)^k=e^k=e \). Donc \( p\) a bien la propriété \( g^p=e\) pour tout \( g\in G\).

		Si \( q<p\), nous montrons que \( q\) ne peut pas avoir cette propriété. Il existe un élément \( g\) dont l'ordre \( a\) ne divise pas \( q\). Par la division euclidienne \ref{ThoDivisEuclide}, nous avons des entiers \( u\) et \( v\) tels que \( q=ua+v\) avec \( v<a\). Nous avons alors
		\begin{equation}
			g^{q}=g^{ua}g^v=g^v\neq e
		\end{equation}
		parce que \( v<a\) et que \( a\) est le plus petit \( n\) tel que \( g^n=e\).

		\spitem[Pour \ref{ITEMooLIREooYEtcdL}]
		%-----------------------------------------------------------
		Même chose. Pour un groupe fini, le ppcm existe toujours.
	\end{subproof}
\end{proof}

Le théorème de Burnside~\ref{ThooJLTit} nous donnera un bon paquet d'exemples de groupes d'exposant fini dans \( \GL(n,\eC)\).

\begin{proposition} \label{PropSRMJooIDPBoW}
	Soit un groupe \( G\). Nous considérons un sous-groupe normal \( H\) de \( G\) ainsi qu'un morphisme \( \psi\colon G\to H\). Alors
	\begin{enumerate}
		\item
		      \( \psi(H)\) est normal dans \( \psi(G)\)
		\item
		      Si \( G/H\) est abélien alors \( \psi(G)/\psi(H)\) est abélien.
	\end{enumerate}
\end{proposition}

\begin{proof}
	Soient \( h\in H\) et \( g\in G\). Alors \( \psi(g)\psi(h)\psi(g)^{-1}=\psi(ghg^{-1})\in\psi(H)\). Donc \( \psi(H)\) est normal dans \( \psi(G)\).

	Pour la seconde partie nous notons \( [\ldots]\) les classes par rapport à \( \psi(H)\) et \( \overline{ \vphantom{g}\ldots }\) celles par rapport à \( H\). Nous avons
	\begin{subequations}
		\begin{align}
			[\psi(g_1)][\psi(g_2)] & =\big[ \psi(g_1)\psi(g_2) \big]            \\
			                       & =\big[ \psi(g_1g_2) \big]                  \\
			                       & =\{ \psi(g_1g_2)\psi(h)\tq h\in H \}       \\
			                       & =\{ \psi(g_1g_2h)\tq h\in H \}             \\
			                       & =\psi\Big(  \{ g_1g_2h\tq h\in H \}  \Big) \\
			                       & =\psi\big( \overline{ g_1g_2 } \big)       \\
			                       & =\psi(\overline{ g_2g_1 })                 \\
			                       & =\text{refaire à l'envers}                 \\
			                       & =[\psi(g_2)][\psi(g_1)].
		\end{align}
	\end{subequations}
	Par conséquent \( \psi(G)/\psi(H)\) est abélien.
\end{proof}
