% This is part of Mes notes de mathématique
% Copyright (c) 2011-2023
%   Laurent Claessens
% See the file fdl-1.3.txt for copying conditions.

%+++++++++++++++++++++++++++++++++++++++++++++++++++++++++++++++++++++++++++++++++++++++++++++++++++++++++++++++++++++++++++ 
\section{Groupes}
%+++++++++++++++++++++++++++++++++++++++++++++++++++++++++++++++++++++++++++++++++++++++++++++++++++++++++++++++++++++++++++

%---------------------------------------------------------------------------------------------------------------------------
\subsection{Définition, unicité du neutre}
%---------------------------------------------------------------------------------------------------------------------------

La définition d'un groupe est la définition \ref{DEFooBMUZooLAfbeM}.

\begin{lemmaDef}[Unicités]  \label{LEMooECDMooCkWxXf}
	Dans un groupe, l'inverse et le neutre sont uniques. Plus précisément, si \( G\) est un groupe nous avons :
	\begin{enumerate}
		\item
		      il existe un unique élément \( e\in G\) tel que \( e g=g e=g\) pour tout \( g\in G\),
		\item       \label{ITEMooOIWTooYqmMPP}
		      pour tout \( g\in G\), il existe un unique élément \( h\in  G\) tel que \(g h=h g=e \).
	\end{enumerate}
	Le \( e\) ainsi défini est nommé \defe{neutre}{neutre!dans un groupe} de \( G\). Le \( h\) tel que \( g h=h g=e\) est nommé l'\defe{inverse}{inverse!dans un groupe} de \( g\) et est noté \( g^{-1}\).
\end{lemmaDef}

\begin{proof}
	Chaque point séparément.
	\begin{enumerate}
		\item
		      Supposons que \( e_1\) et \( e_2\) vérifient la propriété. Nous avons pour tout \( g\in G\) : \( e_1g=ge_1=g\). En particulier pour \( g=e_2\) nous écrivons \( e_1e_2=e_2e_1=e_2\). Mais en partant dans l'autre sens : \( e_2g=ge_2=g\) avec \( g=e_1\) nous avons \( e_2e_1=e_1e_2=e_1\). En égalant ces deux valeurs de \( e_2e_1\) nous avons \( e_1=e_2\).

		      Pour la suite de la preuve nous écrivons \( e\) l'unique neutre de \( G\).

		\item
		      Supposons que \( k_1\) et \( k_2\) soient deux inverses de \( g\). On considère alors le produit \( k_1 g k_2 \). Puisque \(k_1 g = e \), on a \( k_1 g k_2 = e k_2 = k_2 \); mais, comme \(g k_2 = e \), on a aussi \( k_1 g k_2 = k_1 e = k_1 \). Le produit est donc à la fois égal à \( k_1 \) et à \( k_2 \), et donc \( k_1 = k_2 \).
	\end{enumerate}
\end{proof}

\begin{lemma}       \label{LEMooWYLRooNOdZnp}
	Soient deux groupes \( G\) et \( H\). Si \( \alpha\colon G\to H\) est un morphisme de groupes\footnote{Définition \ref{DEFooBEHTooMeCOTX}.}, alors
	\begin{enumerate}
		\item
		      \( \alpha(e_G)=e_H\).
		\item
		      \( \alpha(g^{-1})=\alpha(g)^{-1}\).
	\end{enumerate}
\end{lemma}

\begin{lemmaDef}        \label{LEMooSQQDooCmOvqi}
	Soit un groupe \( G\). L'ensemble des automorphismes de \( G\), noté \( \Aut(G)\), est un groupe pour la composition.
\end{lemmaDef}


\begin{lemma}       \label{LEMooBIBFooBHxFYC}
	Si \( G\) est un groupe et si \( h\in G\), alors les applications
	\begin{equation}
		\begin{aligned}
			L_h\colon G & \to G      \\
			g           & \mapsto hg
		\end{aligned}
	\end{equation}
	et
	\begin{equation}
		\begin{aligned}
			R_h\colon G & \to G      \\
			g           & \mapsto gh
		\end{aligned}
	\end{equation}
	sont des bijections.
\end{lemma}

\begin{proof}
	D'abord si \( L_h(g_1)=L_h(g_2)\), alors \( hg_1=hg_2\) et en multipliant à gauche par \( h^{-1}\) nous avons \( g_1=g_2\); donc \( L_h\) est injective. Ensuite \( L_h\) est surjective parce que si \( g\in G\), alors \( g=L_h(h^{-1} g)\).

	Pour l'application \( R_h\), la preuve est une simple adaptation.
\end{proof}

%--------------------------------------------------------------------------------------------------------------------------- 
\subsection{Groupe ordonné}
%---------------------------------------------------------------------------------------------------------------------------

\begin{definition}[\cite{BIBooZIFGooBrNwWU}]        \label{DEFooEUHFooYvhnLQ}
	Soient un groupe \( (G,+)\), ainsi qu'une relation d'ordre \( \leq\) sur \( G\). Nous disons que la relation d'ordre est \defe{compatible}{} avec la structure de groupe si pour tout \( x,y,z\in G\), si \( x\leq y\) alors \( x+z\leq y+z\) et \( z+x\leq z+y\). Dans ce cas, le triple \( (G,+,\leq)\) est un \defe{groupe ordonné}{groupe ordonné}.

	Si \( (G,\leq)\) est totalement ordonné, nous disons que le groupe est totalement ordonné.
\end{definition}

%---------------------------------------------------------------------------------------------------------------------------
\subsection{Classes de conjugaison}
%---------------------------------------------------------------------------------------------------------------------------

\begin{definition}[classe de conjugaison]       \label{DEFooOLXPooWelsZV}
	Soit un groupe \( G\) et un élément \( g\in G\). La \defe{classe de conjugaison}{classe!de conjugaison} de \( g\) est la partie
	\begin{equation}
		C_g=\{ kgk^{-1}\tq k\in G \}.
	\end{equation}
\end{definition}

\begin{lemma}       \label{LEMooQYBJooYwMwGM}
	Un groupe est commutatif si et seulement si ses classes de conjugaison sont des singletons.
\end{lemma}

\begin{proof}
	Supposons que \( G\) soit commutatif. Alors
	\begin{equation}
		C_g=\{ kgk^{-1}\tq k\in G \}=\{ g \}.
	\end{equation}
	Donc les classes de conjugaison sont des singletons.

	Dans l'autre sens, si les classes sont des singletons, on a \( kgk^{-1}=g\) pour tous \( k,g\in G\). Cela signifie immédiatement que \( G\) est commutatif.
\end{proof}

\begin{definition}[centralisateur\cite{Kropholler}]         \label{defGroupeCentre}
	Soient un groupe \( G\), un sous-groupe \( H\) et un élément \( h\in H\). Le \defe{centralisateur}{centralisateur} de \( h\) dans \( G\) est l'ensemble des éléments de \( G\) qui commutent avec \( h\) :
	\begin{equation}
		Z_G(h)=\{z\in G\tq hz=zh\}.
	\end{equation}
	Le centralisateur de \( H\) dans \( G\) est l'ensemble des éléments de \( G\) qui commutent avec tous les éléments de \( H\) :
	\begin{equation}
		Z_G(H)=\bigcap_{h\in H}Z_G(h).
	\end{equation}
	Le \defe{centre}{centre!d'un groupe} d'un groupe \( G\) est l'ensemble des éléments de \( G\) qui commutent avec tous les autres:
	\begin{equation}
		Z_G=Z_G(G)=\{ z\in G\tq gz=zg , \forall g\in G \}.
	\end{equation}
\end{definition}

\begin{definition}[normalisateur\cite{Kropholler}]          \label{DEFooZTSMooBislIy}
	Soient un groupe \( G\) et un sous-groupe \( H\). Le \defe{normalisateur}{normalisateur} de \( H\) dans \( G\) est
	\begin{equation}
		\mN_G(H)=\{ g\in G\tq gH=Hg \}.
	\end{equation}
\end{definition}

\begin{definition}[Sous-groupe normal]                      \label{DEFooNIIMooFkZgvX}
	Un sous-groupe \( N\) de \( G\) est \defe{normal}{normal!sous-groupe} ou \defe{distingué}{sous-groupe distingué}\index{distingué!sous-groupe} si pour tout \( g\in G\) et pour tout \( n\in N\), \( gng^{-1}\in N\). Autrement dit lorsque \( gNg^{-1}\subset N\).

	Lorsque \( N\) est normal dans \( G\) il est parfois noté \( N\normal G\)\nomenclature[R]{\(N \normal G\)}{Le sous-groupe \( N\) est normal dans \( G\)}.
\end{definition}

\begin{definition}      \label{DEFooUXXTooCCLmQe}
	Un sous-groupe \( H\) de \( G\) est un sous-groupe \defe{caractéristique}{sous-groupe!caractéristique}\index{caractéristique!sous-groupe} si \( \alpha(H)\subset H\) pour tout automorphisme\footnote{Automorphisme de groupe, définition \ref{DEFooBEHTooMeCOTX}.} \( \alpha\) de \( G\).
\end{definition}

\begin{lemma}[\cite{BIBooOHKHooAcewiw}]
	Si \( H\) est un sous-groupe caractéristique de \( G\), alors \( \alpha(H)=H\) pour tout automorphisme \( \alpha\) de \( G\).
\end{lemma}

\begin{proof}
	Si \( \alpha\) est un automorphisme de \( G\), alors \( \alpha^{-1}\) est encore un automorphisme de \( G\). En particulier \( \alpha^{-1}(H)\subset H\).

	Soit \( h\in H\). Nous devons prouver que \( h\in \alpha(H)\). Pour cela :
	\begin{equation}
		h=\alpha\big( \alpha^{-1}(h) \big)\in \alpha\big( \alpha^{-1}(H) \big)\subset\alpha(H).
	\end{equation}
\end{proof}

\begin{definition}[Groupe simple]                 \label{DefGroupeSimple}
	Un groupe est dit \defe{simple}{groupe simple}\index{groupe!simple} si il est non trivial et si les seuls sous-groupes normaux qu'il admet sont lui-même et le sous-groupe réduit à l'élément neutre.
\end{definition}

%+++++++++++++++++++++++++++++++++++++++++++++++++++++++++++++++++++++++++++++++++++++++++++++++++++++++++++++++++++++++++++
\section{Sous-groupe normal}
%+++++++++++++++++++++++++++++++++++++++++++++++++++++++++++++++++++++++++++++++++++++++++++++++++++++++++++++++++++++++++++

\begin{proposition}\label{propGroupeNormal}
	Soit \( N\) un sous-groupe de \( G\). Les propriétés suivantes sont équivalentes :
	\begin{enumerate}
		\item       \label{ITEMooDYEUooOuKEqQ}
		      \( N\) est normal\footnote{Définition \ref{DEFooNIIMooFkZgvX}.} dans \( G\).
		\item       \label{ITEMooPYTEooZhvrUa}
		      \( N\) est une union de classes de conjugaison\footnote{Définition \ref{DEFooOLXPooWelsZV}.} de \( G\),
		\item       \label{ITEMooJWTLooBRmriQ}
		      \( gNg^{-1}\subseteq N\) pour tout \( g\in G\),
		\item       \label{ITEMooVRZIooAorhRY}
		      \( gNg^{-1}= N\) pour tout \( g\in G\),
		\item       \label{ITEMooJGUOooYshOZa}
		      \( gN=Ng\) pour tout \( g\in G\),
		\item       \label{ITEMooMRYRooZifCCe}
		      Le normalisateur\footnote{Définition \ref{DEFooZTSMooBislIy}.} de \( N\) est \( G\) : \( \mN_G(N)=G\).
	\end{enumerate}
\end{proposition}

\begin{proof}
	En plusieurs parties.
	\begin{subproof}
		\spitem[\ref{ITEMooDYEUooOuKEqQ} implique \ref{ITEMooJWTLooBRmriQ}]
		C'est la définition de sous-groupe normal.
		\spitem[\ref{ITEMooJWTLooBRmriQ} implique \ref{ITEMooVRZIooAorhRY}]
		Soit \( g\in G\). Nous avons \( gNg^{-1}\subset N\), mais aussi (en appliquant l'hypothèse à \( g^{-1}\)) \( g^{-1}Ng\subset N\). En combinant nous avons
		\begin{equation}
			N\subset g(g^{-1} Ng)g^{-1}\subset g Ng^{-1}.
		\end{equation}
		Nous avons l'inclusion dans les deux sens. Donc l'égalité.
		\spitem[\ref{ITEMooVRZIooAorhRY} implique \ref{ITEMooJGUOooYshOZa}]
		Soit \( g\in G\). Un élément général de \( gN\) est de la forme \( gn\) avec \( n\in N\). Nous devons trouver un \( n'\in N\) tel que \( gn=n'g\). En posant \( n'=gng^{-1}\) nous avons
		\begin{equation}
			n'=gng^{-1}\in gNg^{-1}\subset N.
		\end{equation}
		Il est immédiat de prouver que \( gn=n'g\). Cela prouve que \( gN\subset Ng\).

		Le même raisonnement donne \( Ng\subset gN\).
		\spitem[\ref{ITEMooJGUOooYshOZa} implique \ref{ITEMooJWTLooBRmriQ}]
		Un élément de \( g Ng^{-1}\) est \( a=gng^{-1}\) avec \( n\in N\). Nous devons prouver que \( a\in N\). Puisque \( gn\in gN\), par hypothèse il existe \( n'\) tel que \( gn=n'g\). En remplaçant dans la définition de \( a\),
		\begin{equation}
			a=gng^{-1}=n'gg^{-1}=n'\in N.
		\end{equation}
		\spitem[\ref{ITEMooJGUOooYshOZa} implique \ref{ITEMooPYTEooZhvrUa}]
		Pour chaque \( a\in G\) nous notons \( C_a\) la classe de conjugaison de \( a\) dans \( G\) :
		\begin{equation}        \label{EQooJIEVooCAshfe}
			C_a=\{ gag^{-1}\tq g\in G \}.
		\end{equation}
		Comme \( a\in C_a\) (prendre \( g=e\) dans \eqref{EQooJIEVooCAshfe}.) nous avons forcément
		\begin{equation}
			N\subset\bigcup_{n\in N}C_n.
		\end{equation}
		Prouvons maintenant l'inclusion inverse. Nous avons déjà prouvé que \ref{ITEMooJGUOooYshOZa} implique \ref{ITEMooJWTLooBRmriQ}. Donc si \( n\in N\), alors \( gng^{-1}\in N\). Nous avons alors
		\begin{equation}
			C_n=\{ gng^{-1}\tq g\in G \}\subset N.
		\end{equation}
		Donc il est vrai que \( N=\bigcup_{n\in N}C_n\).
		\spitem[\ref{ITEMooPYTEooZhvrUa} implique \ref{ITEMooDYEUooOuKEqQ}]
		Nous supposons que \( N\subset G\) est un sous-groupe de la forme
		\begin{equation}
			N=\bigcup_{a\in I}C_a
		\end{equation}
		où \( I\) est une partie de \( G\). Nous devons montrer que pour tout \( g\in G\) et pour tout \( n\in N\) nous avons \( gng^{-1}\in N\). Puisque \( n\in N\), il existe \( a\in I\) tel que \( n\in C_a\) et donc il existe \( k\in G\) tel que \( n=kak^{-1}\). Nous avons donc
		\begin{equation}
			gng^{-1}=g(kak^{-1})g^{-1}=(gk)a(gk)^{-1}\in C_a\subset N.
		\end{equation}

		\randomGender{Le lecteur attentif}{La lectrice attentive} aura remarqué l'utilisation de l'axiome du choix. La prudence l'incitera à ne pas le faire remarquer au jury.
		\spitem[\ref{ITEMooMRYRooZifCCe} si et seulement si \ref{ITEMooJGUOooYshOZa}]
		C'est la définition du normalisateur.
	\end{subproof}
\end{proof}

\begin{definition}
	Soit \( g\in G\) et \( n\in \eZ\). Nous définissons \( g^n\) par
	\begin{enumerate}
		\item
		      \( g^0=e\) et \( g^n=gg^{n-1}\) si \( n\) est positif.
		\item
		      si \( n<0\), nous posons \( g^n=(g^{-1})^{-n}\).
	\end{enumerate}
\end{definition}

L'ordre d'un groupe et l'ordre d'un élément d'un groupe sont deux choses différentes.

\begin{definition}[Ordre d'un groupe]    \label{DEFooKWBCooMlmpCP}
	Soit un groupe \( G\).
	\begin{enumerate}
		\item
		      Si \( G\) est un ensemble fini, l'\defe{ordre}{ordre d'un groupe} de \( G\) est son cardinal\footnote{Définition \ref{PROPooJLGKooDCcnWi}.}, et nous le notons \( | G |\).
		\item
		      Si l'ensemble \( G\) est infini, nous disons que \( | G |=\infty\) et qu'il est d'ordre infini.
	\end{enumerate}
	Oui : nous pourrions simplement toujours dire «cardinalité» et écrire \( \Card(G)\). Au lieu de ça, dans le cas particulier des groupes, il y a une tradition de dire «ordre» et d'écrire \( | G |\).
\end{definition}

\begin{definition}[Ordre d'un élément]      \label{DEFooKSTVooOObpgC}
	L'\defe{ordre}{ordre!d'un élément} d'un élément \( g\) de \( G\) est le naturel
	\begin{equation}
		\min\{ n\in\eN\setminus\{ 0 \}\tq g^n=e \},
	\end{equation}
	si il existe; dans le cas contraire, nous disons que l'ordre de \( g\) est infini.
\end{definition}

\begin{normaltext}
	Nous verrons que le corolaire~\ref{CorpZItFX} au théorème de Lagrange dira que l'ordre d'un élément divise l'ordre du groupe.
\end{normaltext}

\begin{lemma}[\cite{PDFpersoWanadoo,BIBooZFPUooIiywbk}]\label{LemHUkMxp}
	Soient un groupe \( G\) et deux sous-groupes normaux\footnote{Sous-groupe normal, définition \ref{DEFooNIIMooFkZgvX}.} \( H\) et \( K\) tels que \( H\cap K=\{ e \}\). Alors :
	\begin{enumerate}
		\item       \label{ITEMooDFVBooSnnlgR}
		      Tout élément de \( H\) commute avec tout élément de \( K\).
		\item       \label{ITEMooVVBGooZSJqjp}
		      \( HK\) est un sous-groupe de \( G\).
		\item       \label{IMTEooPCBZooQoZFOD}
		      L'application
		      \begin{equation}
			      \begin{aligned}
				      \varphi\colon H\times K & \to HK     \\
				      (h,k)                   & \mapsto hk
			      \end{aligned}
		      \end{equation}
		      est un isomorphisme de groupes.
	\end{enumerate}
\end{lemma}

\begin{proof}
	Point par point.
	\begin{subproof}
		\spitem[\ref{ITEMooDFVBooSnnlgR}]
		Soient \( h\in H\) et \( k\in K\). Nous voulons montrer que \( hk=kh\). Pour cela nous considérons l'élément \( a=hkh^{-1}k^{-1}\). Comme \( H \) est normal dans \( G\), nous avons
		\begin{equation}
			kh^{-1}k^{-1}\in H
		\end{equation}
		et donc \( a\in H\). De même \( K\) étant normal dans \( G\), nous avons \( hkh^{-1}\in K\) et donc \( a\in K\). Au final \( a\in H\cap K=\{ e \}\). Nous avons prouvé que
		\begin{equation}
			hkh^{-1}k^{-1}=e,
		\end{equation}
		et donc que \( hk=kh\).
		\spitem[\ref{ITEMooVVBGooZSJqjp}]
		Puisque \( H\) et \( K\) sont des sous-groupes, \( \{ e \}\) est dans les deux, de telle sorte que \( e\in HK\). De plus si \( h_i\in H\) et \( k_i\in K\), la commutativité du point \ref{ITEMooDFVBooSnnlgR} donne
		\begin{equation}
			(h_1k_1)(h_2k_2)=h_1h_2k_1k_2\in HK.
		\end{equation}
		Donc le produit de deux éléments de \( HK\) est dans \( HK\).
		\spitem[\ref{IMTEooPCBZooQoZFOD}]
		En trois sous-parties.
		\begin{subproof}
			\spitem[Morphisme]
			Soient \( h_i\in H\) et \( k_i\in K\). En utilisant la commutativité du point \ref{ITEMooDFVBooSnnlgR} nous avons
			\begin{subequations}
				\begin{align}
					\varphi\big( (h_1,k_1)(h_2,k_2) \big) & =\varphi(h_1h_2,k_1k_2)            \\
					                                      & =(h_1h_2)(k_1k_2)                  \\
					                                      & =(h_1k_1)(h_2k_2)                  \\
					                                      & =\varphi(h_1,k_1)\varphi(h_2,k_2).
				\end{align}
			\end{subequations}
			\spitem[Injectif]
			Si \( \varphi(h_1,k_1)=\varphi(h_2,k_2)\) nous avons successivement
			\begin{subequations}
				\begin{align}
					h_1k_1                 & =h_2k_2       \\
					h_1 k_1 h_2^{-1}       & =k_2          \\
					h_1k_1h_2^{-1}k_1^{-1} & =k_2k_1^{-1}  \\
					h_1h_2^{-1}            & =k_2k_1^{-1}.
				\end{align}
			\end{subequations}
			Le membre de gauche est un élément de \( H\) et le membre de droite un élément de \( K\). Comme \( H\cap K=\{ e \}\) nous avons \( h_1h_2^{-1}=e\) et \( k_2k_1^{-1}=e\), c'est-à-dire \( h_1=h_2\) et \( k_1=k_2\).
			\spitem[Surjectif]
			Un élément général de \( HK\) est \( hk\) avec \( h\in H\) et \( k\in K\), c'est à dire \( \varphi(h,k)\).
		\end{subproof}
	\end{subproof}
\end{proof}

\begin{definition}  \label{DefvtSAyb}
	L'\defe{exposant}{exposant!d'un groupe} du groupe \( G\) est le plus petit entier non nul \( n\) tel que \( g^n=e\) pour tout \( g\in G\). S'il n'existe pas un tel \( n\), nous disons que l'exposant du groupe est infini.
\end{definition}

\begin{proposition} \label{PROPooSWHHooOzqWkw}
	À propos d'exposant de groupe et de ppcm.
	\begin{enumerate}
		\item
		      Si l'ensemble des ordres de tous les éléments d'un groupe est majoré, alors l'exposant du groupe est le plus petit commun multiple des ordres des éléments du groupe.
		\item
		      Pour un groupe fini, l'exposant est le \( \ppcm\) des ordres des éléments du groupe.
	\end{enumerate}
\end{proposition}

Le théorème de Burnside~\ref{ThooJLTit} nous donnera un bon paquet d'exemples de groupes d'exposant fini dans \( \GL(n,\eC)\).

\begin{proposition} \label{PropSRMJooIDPBoW}
	Soit un groupe \( G\). Nous considérons un sous-groupe normal \( H\) de \( G\) ainsi qu'un morphisme \( \psi\colon G\to H\). Alors
	\begin{enumerate}
		\item
		      \( \psi(H)\) est normal dans \( \psi(G)\)
		\item
		      Si \( G/H\) est abélien alors \( \psi(G)/\psi(H)\) est abélien.
	\end{enumerate}
\end{proposition}

\begin{proof}
	Soient \( h\in H\) et \( g\in G\). Alors \( \psi(g)\psi(h)\psi(g)^{-1}=\psi(ghg^{-1})\in\psi(H)\). Donc \( \psi(H)\) est normal dans \( \psi(G)\).

	Pour la seconde partie nous notons \( [\ldots]\) les classes par rapport à \( \psi(H)\) et \( \overline{ \vphantom{g}\ldots }\) celles par rapport à \( H\). Nous avons
	\begin{subequations}
		\begin{align}
			[\psi(g_1)][\psi(g_2)] & =\big[ \psi(g_1)\psi(g_2) \big]            \\
			                       & =\big[ \psi(g_1g_2) \big]                  \\
			                       & =\{ \psi(g_1g_2)\psi(h)\tq h\in H \}       \\
			                       & =\{ \psi(g_1g_2h)\tq h\in H \}             \\
			                       & =\psi\Big(  \{ g_1g_2h\tq h\in H \}  \Big) \\
			                       & =\psi\big( \overline{ g_1g_2 } \big)       \\
			                       & =\psi(\overline{ g_2g_1 })                 \\
			                       & =\text{refaire à l'envers}                 \\
			                       & =[\psi(g_2)][\psi(g_1)].
		\end{align}
	\end{subequations}
	Par conséquent \( \psi(G)/\psi(H)\) est abélien.
\end{proof}

%--------------------------------------------------------------------------------------------------------------------------- 
\subsection{Permutations, groupe symétrique}
%---------------------------------------------------------------------------------------------------------------------------

Nous donnons ici quelques éléments à propos du groupe symétrique. Beaucoup de choses supplémentaires sont reportées à la section \ref{SECooZFYQooFfopMa}. Voir aussi le thème \ref{THEMEooQEEWooXDhvhv}.


\begin{definition}      \label{DEFooJNPIooMuzIXd}
	Soit un ensemble \( E\). Une \defe{permutation}{permutation} de l'ensemble \( E\) est une bijection \( E\to E\). Le \defe{groupe symétrique}{groupe!symétrique} de \( E\) est le groupe des bijections \( E\to E\); il est noté \( S_E\).

	Le \defe{groupe symétrique}{groupe!symétrique} \( S_n\)\nomenclature[R]{\( S_n\)}{le groupe symétrique} est le groupe des permutations de l'ensemble \( \{ 1,\ldots,n \}\). C'est donc l'ensemble des bijections \( \{ 1,\ldots, n \}\to\{ 1,\ldots, n \}\).
\end{definition}

\begin{definition}      \label{DEFooSupportPermutation}
	Le \defe{support}{support!d'une permutation} d'une permutation \( \sigma\) est l'ensemble constitué des éléments modifiés par \( \sigma\):
	\begin{equation*}
		\supp \sigma = \{ i \in \{1,\ldots,n \} \tq \sigma(i) \neq i\}.
	\end{equation*}
\end{definition}

\begin{definition}[\cite{PDFpersoWanadoo}]      \label{DEFooMVFKooMpXMQy}
	Soient une permutation \( \sigma\in E\) ainsi que \( a\in E\). La \( \sigma\)-\defe{orbite}{orbite d'un élément sous une permutation} de \( a\) est l'ensemble
	\begin{equation}
		\Omega_{\sigma}(a)=\{ \sigma^i(a) \}_{i\in \eN}.
	\end{equation}
\end{definition}

\begin{lemma}[\cite{ooJBHPooToyYYI}]        \label{LEMooSGWKooKFIDyT}
	Le groupe symétrique \( S_n\) est un ensemble fini contenant \( n!\) éléments :
	\begin{equation}
		\Card(S_n)=n!.
	\end{equation}
\end{lemma}

\begin{lemma}[\cite{ooJFLYooKMbycW}]        \label{LEMooUPBOooWbwMTx}
	Deux résultats.
	\begin{enumerate}
		\item
		      Tout groupe est isomorphe à un sous-groupe d'un groupe symétrique.
		\item
		      Tout groupe fini d'ordre \( n\) est isomorphe à un sous-groupe de \( S_n\).
	\end{enumerate}
\end{lemma}

\begin{proof}
	Soit, pour \( g\in G\) donné, l'application
	\begin{equation}
		\begin{aligned}
			\tau_g\colon G & \to G       \\
			x              & \mapsto gx.
		\end{aligned}
	\end{equation}
	Commençons par prouver que cela est une bijection.  D'une part, \( \tau_g(x)=y\) pour \( x=g^{-1} y\) (surjection) et, d'autre part, \( \tau_g(x)=\tau_g(y)\) implique \( gx=gy\) et donc \( x=y\) (injection).

	Nous avons donc \( \tau_g\in S_G\). De plus l'application
	\begin{equation}
		\begin{aligned}
			\varphi\colon G & \to S_G        \\
			g               & \mapsto \tau_g
		\end{aligned}
	\end{equation}
	est un morphisme de groupe. Il est injectif parce que si \( \tau_g=\tau_h\) alors \( gx=hx\) pour tout \( x\). En particulier \( g=h\).

	Donc \( \varphi\colon G\to \Image(\varphi)\) est un isomorphisme entre \( G\) et un sous-groupe de \( S_G\).

	Un groupe fini de cardinal \( n\) est isomorphe à un sous-groupe de \( S_G\); or \( S_G\) est isomorphe à un des \( S_n\).
\end{proof}


%--------------------------------------------------------------------------------------------------------------------------- 
\subsection{Décomposition en cycles}
%---------------------------------------------------------------------------------------------------------------------------

\begin{definition}[cycle\cite{PDFpersoWanadoo}]
	Soit \( E\) un ensemble de cardinal\footnote{Définition \ref{PROPooJLGKooDCcnWi}.} \( n\). Soit un entier \( 1\leq k\leq n\). Un élément \( \sigma\in S_E\) est un \( k\)-\defe{cycle}{cycle} si il ne possède qu'une seule orbite\footnote{Définition \ref{DEFooMVFKooMpXMQy}.} non réduite à un élément et qu'elle est de cardinal \( k\).
\end{definition}

\begin{lemma}[\cite{MonCerveau}]        \label{LEMooADNGooDZpdTb}
	Soient un \( k\)-cycle \( \sigma\) et \( a\in \Omega\). Alors
	\begin{equation}
		\Omega_{\sigma}(a)=\{ a,\sigma(a),\ldots, \sigma^{k-1}(a) \}
	\end{equation}
	et \( \sigma^k(a)=a\).

	En particulier, les éléments \( \sigma^q(a)\) avec \( q=0,\ldots, k-1\) sont tous distincts.
\end{lemma}

\begin{proof}
	Soit \( l\) le plus grand entier tel que les \( \sigma^i(a)\) avec \( 0\leq i\leq l\) soient tous distincts, et notons \( A=\{ \sigma^i(a) \}_{i=0,\ldots, l}\). Cet ensemble satisfait
	\begin{itemize}
		\item \( \Card(A)=l+1\)
		\item \( A\subset \Omega_{\sigma}(a)\), et donc \( \Card(A)\leq \Card(\Omega_{\sigma}(a))=k\) par le lemme \ref{LEMooVFPNooVmdUXY}\ref{ITEMooYJSZooXQXkOX}.
	\end{itemize}
	Que vaut \( \sigma^{l+1}(a)\) ? Par maximalité de \( l\), \( \sigma^{l+1}(a)\) est un des \( \sigma^i(a)\) avec \( i\leq l\). Par injectivité de \( \sigma\), nous avons donc forcément \( \sigma^{l+1}(a)=a\).

	Donc pour tout \( i>l\) il existe \( j\leq l\) tel que \( \sigma^i(a)=\sigma^j(a)\) (parce que \( \sigma^i(a)=\sigma^{i-l-1}(a)\)). Nous en déduisons que
	\begin{equation}
		\Omega_{\sigma}(a)=\{ \sigma^i(a) \}_{0\leq i\leq l}=A.
	\end{equation}
	Le cardinal de \( \Omega_{\sigma}(a)\) étant \( k\) par hypothèse nous avons \( k=l+1\), et donc \( l=k-1\).
\end{proof}

\begin{lemma}      \label{LEMooANVHooOQiTwY}
	Si \( \sigma\) est une cycle de longueur \( k\), et si \( b\in \Omega_{\sigma}(a)\), alors
	\begin{equation}
		\Omega_{\sigma}(a)=\Omega_{\sigma}(b)=\{ \sigma^i(b) \}_{i=0,\ldots, k-1}.
	\end{equation}
\end{lemma}

\begin{proof}
	Comme \( b\in \Omega_{\sigma}(a)\), il existe \( l\leq k-1\) tel que \( b=\sigma^l(a)\). Pour tout \( i\) nous avons \( \sigma^i(a)=\sigma^{k-l+1}(b)\), et donc \( \Omega_{\sigma}(a)\subset\Omega_{\sigma}(b)\).

	Mais pour tout \( i\) nous avons aussi \( \sigma^i(b)=\sigma^{l+1}(a)\) et donc \( \Omega_{\sigma}(b)\subset\Omega_{\sigma}(a)\).

	Nous avons donc montré que \( \Omega_{\sigma}(a)=\Omega_{\sigma}(b)\). La seconde égalité est le lemme \ref{LEMooADNGooDZpdTb} appliqué à \( b\).
\end{proof}

\begin{lemma}[\cite{MonCerveau}]       \label{LEMooMIHGooQfALbc}
	Soient un ensemble fini \( E\), une permutation \( \sigma\in S_E\) ainsi que \( a\in E\). Si \( b\in \Omega_{\sigma}(a)\), alors \( \Omega_{\sigma}(b)=\Omega_{\sigma}(a)\).
\end{lemma}


\begin{lemma}[\cite{PDFpersoWanadoo}]
	Tout \( k\)-cycle est d'ordre\footnote{Définition \ref{DEFooKSTVooOObpgC}.} \( k\).
\end{lemma}

\begin{proof}
	Soit le cycle \( \{ a,\sigma(a),\ldots, \sigma^{k-1}(a) \}\). Tous les \( \sigma^i(a)\) avec \( i\leq k-1\) sont distincts et \( \sigma^k(a)=a\). Donc \( \sigma^k\) est l'identité, et l'ordre de \( \sigma\) est plus petit ou égal à \( k\).

	Si \( i\leq k-1\), alors \( \sigma^i(a)\neq a\) parce que les éléments du cycle sont distincts. Donc \( \sigma^i\neq \id\) pour \( i\leq k-1\). Nous en déduisons que l'ordre de \( \sigma\) est \( k\).
\end{proof}


\begin{lemma}[\cite{PDFpersoWanadoo,BIBooONYNooQolWiC}]       \label{LEMooQLSAooBrXDXw}
	Tout élément du groupe symétrique \( S_n\) peut être décomposé en un nombre fini de cycles de supports disjoints.

	Cette décomposition est unique à l'ordre près de l'écriture des cycles.

	Plus précisément, si \( \sigma\) est une permutation, alors il existe un unique ensemble fini \( \{ \omega_i \}_{i\in I}\) de cycles de supports disjoints tels que\footnote{Ici \( I\) est un ensemble fini et vu que les supports sont disjoints, le produit est commutatif.} \( \sigma=\prod_{i\in I}\omega_i\).
	%TODOooLWGBooZFcosV: l'unicité n'est pas prouvée, et il serait bien de réexprimer en termes de l'ensemble fini dont on parle dans le ``plus précisément''.
\end{lemma}

\begin{proof}
	Soit \( \sigma\in S_E\). Si les éléments \( \{ a,\sigma(a), \ldots \sigma^k(a)\}\) sont distincts, alors soit \( \sigma^{k+1}(a)\) est distincts des autres, soit \( \sigma^{k+1}(a)=a\). Il n'est en effet pas possible d'avoir \( \sigma^{k+1}(a)=\sigma^l(a)\) avec \( l<k\) parce que ça contredirait l'injectivité de \( \sigma\).

	Soit donc \( a\in E\). Nous considérons le cycle \( \big( a,\sigma(a),\ldots, \sigma^k(a) \big)\) où \( k\) est maximum tel que tous les éléments sont distincts.

	Soit ce cycle contient tous les éléments de \( E\), soit il existe un élément \( b\) hors de ce cycle. Dans le second cas, nous considérons le cycle commençant par \( b\).

	Et ça continue\ldots
\end{proof}

\begin{lemma}[\cite{PDFpersoWanadoo}]       \label{LEMooWXXLooIzrwJT}
	Deux cycles de support disjoint commutent.
\end{lemma}

\begin{lemma}[\cite{BIBooEPQHooBvhumd}]     \label{LEMooVVPWooMkRjyR}
	Tout cycle de longueur \( r\) est le produit de \( r-1\) transpositions.
\end{lemma}

\begin{proof}
	Il suffit de vérifier que
	\begin{equation}
		(a_1,\ldots, a_r)=(a_1, a_r)(a_1, a_{r-1})\ldots (a_1, a_2).
	\end{equation}
\end{proof}

\begin{lemma}[\cite{MonCerveau}]        \label{LEMooGGLUooUSzuAx}
	Soit une permutation \( \sigma\in S_E\). Soit un cycle \( c\) et une permutation \( s\) de supports disjoints telles que \( \sigma=s\circ c\). Alors
	\begin{enumerate}
		\item       \label{LEMooUHWTooFptoZU}
		      Si \( a\in\supp(\sigma)\), alors pour tout \( q\in \eN\) nous avons \( c^q(a)=\sigma^q(a)\).
		\item       \label{ITEMooHSDLooIAKYZA}
		      Si \( a\in\supp(c)\), alors
		      \begin{equation}
			      \Omega_{\sigma}(a)=\Omega_c(a)=\supp(c).
		      \end{equation}
		\item
		      Si \( a\in\supp(c)\), alors
		      \begin{equation}
			      c(x)=\begin{cases}
				      \sigma(x) & \text{si } x\in\Omega_{\sigma}(a) \\
				      x         & \text{sinon. }
			      \end{cases}
		      \end{equation}
	\end{enumerate}
\end{lemma}

\begin{proof}
	En plusieurs parties.
	\begin{subproof}
		\spitem[Si \( a\in \supp(c)\), alors \( c(a)=\sigma(a)\)]
		% -------------------------------------------------------------------------------------------- 
		Soit \( a\in \supp(c)\). Nous savons que \( c(a)\neq a\), et vu que \( c\) est injective, nous devons aussi avoir \( c\big( c(a) \big)\neq c(a)\). Donc \( a\) et \( c(a)\) sont dans \( \supp(c)\). Étant donné que les supports de \( c\) et de \( s\) sont disjoints, nous déduisons que \( c(a)\) n'est pas dans le support de \( s\), et donc que
		\begin{equation}
			\sigma(a)=(s\circ c)(a)=s\big( c(a) \big)=c(a).
		\end{equation}
		\spitem[\( \sigma^q(a)=c^q(a)\)]
		% -------------------------------------------------------------------------------------------- 
		Juste une récurrence sur le point précédent : si \( b\in\supp(a)\), alors \( \sigma(b)=c(b)\in\supp(a)\).
		\spitem[Si \( a\in \supp(c)\), alors \( \Omega_{\sigma}(a)=\Omega_c(a)\)]
		% -------------------------------------------------------------------------------------------- 
		Utilisant le point précédent, ainsi que la définition \ref{DEFooMVFKooMpXMQy} d'une orbite,
		\begin{equation}
			\Omega_{\sigma}(a)=\{ \sigma^q(a) \}=\{ c^q(a) \}=\Omega_c(a).
		\end{equation}
		\spitem[\( \supp(c)\subset\Omega_c(a)\)]
		% -------------------------------------------------------------------------------------------- 
		Comme toujours, \( a\) est un élément de \( \supp(c)\). Nous considérons \( b\in\supp(c)\) et nous montrons que \( b\in \Omega_c(a)\). Étant donné que \( b\in\supp(c)\), nous avons \( c(b)\neq b\), de telle sorte que \( \Omega_c(b)\) contienne au moins deux éléments distincts.

		Même chose pour \( a\) : l'ensemble \( \Omega_c(a)\) contient au moins \( a\) et \( c(a)\). Vu que \( c\) est un cycle, il n'existe qu'une seule orbite non triviale. Donc \( \Omega_c(a)=\Omega_c(b)\). En particulier \( b\in\Omega_c(b)=\Omega_c(a)\).
		\spitem[\( \Omega_c(a)\subset\supp(c)\)]
		% -------------------------------------------------------------------------------------------- 
		Soit \( b\in \Omega_c(a)\). Le lemme \ref{LEMooANVHooOQiTwY} nous permet de dire que \( \Omega_c(a)=\Omega_c(b)\). Comme \( \Omega_c(b)\) contient au moins deux éléments (parce qu'il est égal à \( \Omega_c(a)\) et que \( a\) est dans le support de \( c\)), nous savons que \( c(b)\neq b\) et donc que \( b\in\supp(c)\).
		\spitem[La formule pour \( c(x)\)]
		% -------------------------------------------------------------------------------------------- 
		Si \( x\in \Omega_{\sigma)}(a)\), alors \( c(x)=\sigma(x)\) par le point \ref{LEMooUHWTooFptoZU}. Si \( x\) n'est pas dans \( \Omega_c(a)=\supp(c)\), alors \( x\) n'est pas dans le support de \( c\) et donc \( c(x)=x\).
	\end{subproof}
\end{proof}

Le lemme suivant permet d'extraire le cycle de \( \sigma\) associé à un élément de \( E\).
\begin{lemma}[\cite{MonCerveau}]        \label{LEMooFFTBooCZsaFu}
	Soient un ensemble finie \( E\) ainsi que \( \sigma\in S_E\), et \( a\in E\) tel que \( \sigma(a)\neq a\). Nous posons
	\begin{equation}
		\begin{aligned}
			c\colon E & \to E                                                  \\
			x         & \mapsto \begin{cases}
				                    \sigma(x) & \text{si } x\in \Omega_{\sigma}(a) \\
				                    x         & \text{sinon }
			                    \end{cases}
		\end{aligned}
	\end{equation}
	Alors
	\begin{enumerate}
		\item
		      Si \( b\in \Omega_{\sigma}(a)\), nous avons \( \Omega_{c}(b)=\Omega_{\sigma}(a)\).
		\item
		      Si \( b\notin \Omega_{\sigma}(a)\), nous avons \( \Omega_{c}(b)=\{ b \}\).
		\item
		      \( c\) est un cycle.
	\end{enumerate}
\end{lemma}

\begin{proof}
	Si \( b\in \Omega_{\sigma}(a)\), alors \( \sigma^q(b)\in \Omega_{\sigma}(a)\) pour tout \( q\in \eN\), et donc
	\begin{equation}
		c^q(b)=\sigma^q(b)\in\Omega_{\sigma}(a)
	\end{equation}
	pour tout \( q\). Donc nous avons
	\begin{equation}
		\Omega_c(b)=\{ c^q(b)\tq q\in \eN \}=\{ \sigma^q(b)\tq q\in \eN \}=\Omega_{\sigma}(b)=\Omega_{\sigma}(a).
	\end{equation}
	La dernière égalité est le lemme \ref{LEMooMIHGooQfALbc}.

	Si \( b\notin\Omega_{\sigma}(a)\), alors \( c(b)=b\) et \( \Omega_c(b)=\{ b \}\).

	Nous avons prouvé que \( c\) a une seule orbite de taille plus grands ou égale à \( 2\). Donc \( c\) est un cycle.
\end{proof}


\begin{theorem}[\cite{BIBooONYNooQolWiC}]
	Soit un ensemble fini \( E\) de cardinal au moins deux. Soit une permutation \( \sigma\in S_E\).
	\begin{enumerate}
		\item Il existe des cycles \( c_1,\ldots, c_m\) à support disjoints tels que \( \sigma=c_1\circ\ldots \circ c_m\).
		\item
		      Cette décomposition est unique à l'ordre près.
	\end{enumerate}
\end{theorem}

\begin{proof}
	Plusieurs points.
	\begin{subproof}
		\spitem[Existence]
		Nous choisissons des éléments \( \{ a_i \}_{i=1,\ldots, p}\) tels que les \( \Omega_{\sigma}(a_i)\) forment une partition de \( E\) en sous-ensembles disjoints. En posant \( l_k=\min\{ r\tq \sigma^r(a_k)=a_k \}\), nous avons
		\begin{equation}
			\Omega_{\sigma}(a_k)=\{ \sigma^q(a_k) \}_{q=1,\ldots, l_k-1}
		\end{equation}
		et tous les \( \sigma^q(a_k)\) sont distincts pour \( q=1,\ldots, l_k-1\).

		Posons
		\begin{equation}
			\begin{aligned}
				c_k\colon E & \to E                                                    \\
				x           & \mapsto \begin{cases}
					                      \sigma(x) & \text{si } x\in \Omega_{\sigma}(a_k) \\
					                      x         & \text{sinon. }
				                      \end{cases}
			\end{aligned}
		\end{equation}
		Le lemme \ref{LEMooGGLUooUSzuAx} dit que \( c_k\) est un cycle. Vu que \( c(a_k)=\sigma(a_k)\), le cycle \( c\) est un \( l_k\)-cycle.

		Nous montrons à présent que \( \sigma=c_1\circ\ldots c_p\). Soit \( x\in E\). Il existe un \( k\in\{1,\ldots, p \}\) tel que
		\begin{equation}
			x\in \Omega_{\sigma}(a_k)=\Omega_{c_k}(a_k)=\Omega_{c_k}(x),
		\end{equation}
		la dernière égalité est parce que \( x\in \Omega_{c_k}(a_k)\). Nous en déduisons que \( c_k(x)\neq x\). D'autre part si \( l\neq k\), alors \( x\) n'est pas dans \( \Omega_{\sigma}(a_l)\), et donc \( c_l(x)=x\). Au final,
		\begin{equation}
			(c_1\circ\ldots c_p)(x)=c_k(x)=\sigma(x).
		\end{equation}
		\spitem[Unicité]
		% -------------------------------------------------------------------------------------------- 
		Nous supposons avoir \( \sigma=c_1\circ\ldots\circ c_p=\gamma_1\circ\gamma_q\) où les \(  c_i\) et les \( \gamma_j\) sont deux ensembles de cycles de supports disjoints. Nous avons
		\begin{equation}
			\supp(\sigma)=\bigcup_{i=1}^p\supp(c_i)=\bigcup_{j=1}^q\supp(\gamma_j).
		\end{equation}
		Montrons que si \( \supp(c_i)\cap\supp(\gamma_j)\neq \emptyset\), alors \( \supp(c_i)=\supp(\gamma_j)\). En effet si \( a\in\supp(c_i)\cap\supp(\gamma_j)\), alors
		\begin{equation}
			\supp(c_i)=\Omega_{c_i}(a)=\Omega_{\sigma}(a)=\Omega_{\gamma_j}(a)=\supp(\gamma_j).
		\end{equation}
		Et comme les \( \supp(\gamma_j)\) sont disjoints, l'ensemble \( \supp(c_i)\) n'a d'intersection qu'avec un et un seul des \( \supp(\gamma_j)\). Cela définit donc une application
		\begin{equation}
			\begin{aligned}
				u\colon \{ 1,\ldots, p \} & \to \{ 1,\ldots, q \}                                              \\
				i                         & \mapsto \text{l'unique \( j\) tel que} \supp(c_i)=\supp(\gamma_j).
			\end{aligned}
		\end{equation}
		Autrement dit, l'application \( u\) permet d'écrire
		\begin{equation}
			\supp(c_i)=\supp\big( \gamma_{u(i)} \big).
		\end{equation}

		L'application \( u\) est injective. En effet si \( u(i)=u(l)\), nous avons
		\begin{subequations}
			\begin{align}
				\supp(c_i) & =\supp\big( \gamma_{u(i)} \big) \\
				\supp(c_l) & =\supp\big( \gamma_{u(l)} \big) \\
				u(i)       & =u(l).
			\end{align}
		\end{subequations}
		Donc \( \supp(c_i)=\supp(c_l)\). Et comme les supports sont disjoints, \( i=l\).

		L'application \( u\) est surjective. En effet, soit \( j\in \{ 1,\ldots, q \}\). Soit \( a\in \supp(\gamma_j)\). Il existe un \( i\) tel que \( a\in\supp(c_i)\). Nous avons alors \( a\in\supp(\gamma_j)\cap\supp(c_i)\), autrement dit \( u(i)=j\).

		Maintenant l'application \( u\colon \{ 1,\ldots, p \}\to \{ 1,\ldots, q \}\) est bijective. Nous en déduisons que \( p=q\). Concluons en montrant que \( c_i=\gamma_{u(i)}\). Soit \( a\in\supp(c_i)=\supp\big( \gamma_{u(i)} \big)\).

		Nous avons
		\begin{equation}
			c_i(x)=\begin{cases}
				\sigma(x) & \text{si } x\in\Omega_{\sigma}(a) \\
				x         & \text{sinon }
			\end{cases}
		\end{equation}
		et
		\begin{equation}
			\gamma_j(x)=\begin{cases}
				\sigma(x) & \text{si } x\in\Omega_{\sigma}(a) \\
				x         & \text{sinon, }
			\end{cases}
		\end{equation}
		et donc \( c_i=\gamma_j\).
	\end{subproof}
\end{proof}


\begin{lemma}[\cite{Combes}]        \label{LemmvZFWP}
	Soit \( \sigma=(i_1,\ldots, i_k)\in S_n\), un cycle de longueur \( k\) et \( \theta\in S_n\). Alors
	\begin{equation}
		\theta\sigma\theta^{-1}=\big( \theta(i_1),\ldots, \theta(i_k) \big).
	\end{equation}
	Tous les cycles de longueur \( k\) sont conjugués entre eux.
\end{lemma}

\begin{proposition}[Classes de conjugaison et structure en cycles\cite{UXMTXxl}] \label{PropEAHWXwe}
	Une classe de conjugaison\footnote{Définition \ref{DEFooOLXPooWelsZV}.} dans \( S_n\) est formée des permutations ayant une décomposition en cycles disjoints de même structure. Autrement dit, deux permutations \( \sigma\) et \( \sigma'\) sont conjuguées si et seulement si le nombre \( k_i\) de cycles de longueur \( i\) dans \( \sigma\) est le même que le nombre \( k'_i\) de cycles de longueur \( i\) dans \( \sigma'\).
\end{proposition}

\begin{proof}
	Soit \( \sigma=c_1\ldots c_m\) la décomposition de \( \sigma\) en cycles \( c_i\) de supports disjoints. Si \( \tau\) est une permutation, alors
	\begin{equation}
		\sigma'=\tau\sigma\tau^{-1}=(\tau c_1\tau^{-1})\ldots (\tau c_m\tau^{-1}),
	\end{equation}
	mais \( \tau c_i\tau^{-1}\) est un cycle de même longueur que \( c_i\), puisque le lemme~\ref{LemmvZFWP} nous dit que si \( \sigma=(a_1,\ldots, a_k)\), alors \( \tau c\tau^{-1}=\big( \tau(a_1),\ldots, \tau(a_k) \big)\). Notons encore que les cycles \( \tau c_i\tau^{-1}\) restent à support disjoints.

	Donc tous les éléments de la classe de conjugaison de \( \sigma\) sont des permutations de même structure que \( \sigma\).

	Réciproquement, si \( \sigma'=c'_1\ldots c'_m\) est une décomposition de \( \sigma'\) en cycles disjoints tels que la longueur des \( c_i\) est la même que la longueur des \( c'_i\), alors il suffit de construire des permutations \( \tau_i\) telles que \( \tau_i c_i\tau_i^{-1}=c_i'\), à travers le lemme~\ref{LemmvZFWP}. Comme les supports des \( c_i\) et des \( c'_i\) sont disjoints, la permutation \( \tau_1\ldots \tau_m\) conjugue \( \sigma\) et \( \sigma'\).
\end{proof}

\begin{example}     \label{EXooQAXRooBsPURs}
	Voyons les classes de conjugaison de \( S_3\). Étant donné que ce groupe agit par définition sur un ensemble à \( 3\) éléments, aucun élément de \( S_3\) ne possède un cycle de plus de \( 3\) éléments. Il y a donc seulement des cycles de longueur deux ou trois (à part les triviaux). Aucun élément de \( S_3\) n'a une décomposition en cycles disjoints contenant deux cycles de deux ou un cycle de deux et un de trois.

	En résumé il y a trois classes de conjugaison dans \( S_3\). La première est celle contenant seulement l'identité. La seconde est celle contenant les cycles de longueur deux et la troisième contient les cycles de longueur \( 3\).

	Ce sont donc
	\begin{subequations}
		\begin{align}
			C_1 & =\{ \id \}               \\
			C_2 & =\{ (1,2),(1,3),(2,3) \} \\
			C_3 & =\{ (1,2,3),(2,1,3) \}.
		\end{align}
	\end{subequations}
\end{example}

\begin{definition}[transposition]      \label{DEFooXNAFooGTbTTJ}
	Une \defe{transposition}{transposition} est une permutation\footnote{Une permutation est une bijection, définition \ref{DEFooJNPIooMuzIXd}.} qui échange deux éléments de \( E\). Plus précisément, une bijection \( \sigma\colon E\to E\) est une transposition si il existe \( a,b\in E\) tels que
	\begin{equation}
		\sigma(x)=\begin{cases}
			a & \text{si } x=b \\
			b & \text{si } x=a \\
			x & \text{sinon. }
		\end{cases}
	\end{equation}
\end{definition}

\begin{example} \label{ExVYZPzub}
	Les classes de conjugaison de \( S_4\). Nous savons que les classes de conjugaison dans \( S_4\) sont caractérisées par la structure des décompositions en cycles (proposition~\ref{PropEAHWXwe}). Le groupe symétrique \( S_4\) possède donc les classes de conjugaison suivantes.
	\begin{enumerate}
		\item
		      Le cycle vide qui représente la classe constituée de l'identité seule.
		\item
		      Les transpositions (de type \( (a,b)\)) qui sont au nombre de \( 6\).
		\item
		      Les \( 3\)-cycles. Pour savoir \href{http://www.toujourspret.com/techniques/expression/chants/C/cantique_des_etoiles.php}{quel est leur nombre} nous commençons par remarquer qu'il y a \( 4\) façons de prendre \( 3\) nombres parmi \( 4\) et ensuite \( 2\) façons de les arranger. Il y a donc \( 8\) éléments dans cette classe de conjugaison.
		\item
		      Les \( 4\)-cycles. Le premier est arbitraire (parce que c'est cyclique). Pour le second il y a \( 3\) possibilités, et deux possibilités pour le troisième; le quatrième est alors automatique. Cette classe de conjugaison contient donc \( 6\) éléments.
		\item       \label{ITEMooGCMYooKZgFHX}
		      Les doubles transpositions, du type \( (a,b)(c,d)\). Dans ce cas, tous les nombres sont permutés, et l'image de \( 1\) détermine la double transposition. Il y a \( 3\) images possibles, et donc \( 3\) éléments dans cette classe.
	\end{enumerate}
	\index{classe de conjugaison!dans \( S_4\) }
\end{example}

\begin{proposition} \label{PropPWIJbu}
	Tout élément de \( S_n\) peut être écrit sous la forme d'un produit fini de transpositions.

	Si \( E \) est un ensemble fini, tout élément de \( S_E\) pour être écrit sous forme d'un produit fini de transposition de \( E\).
	%TODOooAFZSooPMWvJb prouver cette deuxième partie
\end{proposition}

\begin{proof}
	Un élément de \( S_n\) se décompose en un nombre fini de cycles par le lemme \ref{LEMooQLSAooBrXDXw} et chacun des cycles peut être décomposé en un nombre fini de transpositions par le lemme \ref{LEMooVVPWooMkRjyR}.
\end{proof}

Cette décomposition n'est pas à confondre avec celle en cycles de support disjoints. Par exemple \( (1,2,3)=(1,3)(1,2)\).

Le théorème suivant, qui donne la notion de parité d'une permutation, est la clef pour savoir quelles positions du jeu de taquin sont possibles ou impossibles\cite{BIBooGLIWooAggcqh,BIBooCDDXooYWCtNZ}.
\begin{propositionDef}[parité d'une permutation]\label{PROPooKRHEooAxtmRv}
	À propos de décomposition ne permutations.
	\begin{enumerate}
		\item
		      Si une permutation peut être écrite sous forme d'un produit d'un nombre pair de transpositions, alors toute décomposition en transpositions sera en quantité paire.
		\item
		      Si une permutation peut être écrite sous forme d'un produit d'un nombre impair de transpositions, alors toute décomposition en transpositions sera en quantité impaire.
	\end{enumerate}
	Une permutation qui se décompose en une quantité paire de transpositions est une \defe{permutation paire}{permutation paire} (et \defe{impaire}{permutation impaire} sinon).
\end{propositionDef}


\begin{definition}       \label{DEFooNHXSooQzCPzD}
	La \defe{signature}{signature} est l'application
	\begin{equation}
		\begin{aligned}
			\epsilon\colon S_E & \to \{ -1,1 \}                                   \\
			\sigma             & \mapsto \begin{cases}
				                             1  & \text{si \( \sigma\) est paire }    \\
				                             -1 & \text{si \( \sigma\) est impaire. }
			                             \end{cases}
		\end{aligned}
	\end{equation}
\end{definition}

\begin{lemma}  \label{LEMooWGRXooHWyzLC}
	Nous disons qu'un élément \( \sigma \in S_n\) est une \defe{inversion}{inversion!dans le groupe symétrique} pour les nombres \( i<j\) si \( \sigma(i)>\sigma(j)\). Soit \( N_\sigma\) le nombre d'inversions que \( \sigma\in S_n\) possède (c'est le nombre de couples \( (i,j)\) avec \( i<j\) tels que \( \sigma(i)>\sigma(j)\)). Nous avons
	\begin{equation}
		\epsilon(\sigma)=(-1)^{N_\sigma}
	\end{equation}
	où \( \epsilon\) est la signature\footnote{Définition \ref{DEFooNHXSooQzCPzD}.} dans \( S_n\).
\end{lemma}

\begin{lemma}[\cite{PDFpersoWanadoo}]       \label{LemhxnkMf}
	Un \( k\)-cycle est une permutation impaire si \( k\) est pair et paire si \( k\) est impair.
\end{lemma}

\begin{proposition}[\cite{Combes}]  \label{ProphIuJrC}
	Soit \( S_n\) le groupe symétrique.
	\begin{enumerate}
		\item       \label{ITEMooBQKUooFTkvSu}
		      L'application \( \epsilon\colon S_n\to \{ 1,-1 \}\) est l'unique homomorphisme surjectif de \( S_n\) sur \( \{ -1,1 \}\).
		\item
		      Si \( s=t_1\cdots t_k\) est le produit de \( k\) transpositions, alors \( \epsilon(s)=(-1)^k\).
	\end{enumerate}
\end{proposition}

\begin{proof}
	Soit \( \sigma,\theta \in S_n\). Afin de montrer que \( \epsilon(\sigma\theta )=\epsilon(\sigma)\epsilon(\theta )\), nous divisons les couples \( (i,j)\) tels que \( i\leq j\) en \( 4\) groupes suivant que \( \theta(i)\gtrless \theta(j)\) et \( \sigma\big( \theta(i) \big)\gtrless \sigma\big( \theta(j) \big)\). Nous notons \( N_1\), \( N_2\), \( N_3\) et \( N_4\) le nombre de couples dans chacun des quatre groupes :
	\begin{center}
		\begin{tabular}{c|c|c}
			\(  (i,j)\)              & \(\sigma\big( \theta(i) \big)<\sigma\big( \theta(j) \big)\) & \(\sigma\big( \theta(i) \big)>\sigma\big( \theta(j) \big)\) \\
			\hline
			\( \theta(i)<\theta(j)\) & \( N_1\)                                                    & \( N_2\)                                                    \\
			\hline
			\( \theta(i)>\theta(j)\) & \( N_3\)                                                    & \( N_4\)
		\end{tabular}
	\end{center}
	Nous avons immédiatement \( N_\theta=N_3+N_4\) et \( N_{\sigma\theta}=N_2+N_4\). Les éléments qui participent à \( N_\sigma\) sont ceux où \( \theta(i)\) et \(\theta(j)\) sont dans l'ordre inverse de \( \sigma\big( \theta(i) \big)\) et \( \sigma\big( \theta(j) \big)\) (parce que \(  \theta\) est une bijection). Donc \( N_\sigma=N_2+N_3\). Par conséquent nous avons
	\begin{equation}
		\epsilon(\sigma)\epsilon(\theta)=(-1)^{N_2+N_3}(-1)^{N_3+N_4}=(-1)^{N_2+N_4}=(-1)^{N_{\sigma\theta}}=\epsilon(\sigma\theta).
	\end{equation}
	Nous avons prouvé que \( \epsilon\) est un homomorphisme. Pour montrer que \( \epsilon\) est surjectif sur \( \{ -1,1 \}\) nous devons trouver un élément \( \tau\in S_n\) tel que \( \epsilon(\tau)=-1\). Si \( \tau\) est la transposition \( 1\leftrightarrow 2\) alors le couple \( (1,2)\) est le seul à être inversé par \( \tau\) et nous avons \( \epsilon(\tau)=-1\).

	Avant de montrer l'unicité, nous montrons que si \( \sigma=t_1\ldots t_k\) alors \( \epsilon(\sigma)=(-1)^k\). Pour cela il faut montrer que \( \epsilon(\tau)=-1\) dès que \( \tau\) est une transposition. Soit \( \tau_{ij}\), la transposition \( (i,j)\) et \( \theta=(i,i+1,\ldots, j-1)\) alors le lemme~\ref{LemmvZFWP} dit que
	\begin{equation}
		\tau_{ij}=\theta\tau_{j-1,j}\theta^{-1}.
	\end{equation}
	La signature étant un homomorphisme,
	\begin{equation}
		\epsilon(\tau_{ij})=\epsilon(\theta)\epsilon(\tau_{j-1,j})\epsilon(\theta)^{-1}=\epsilon(\tau_{j-1,j})=-1.
	\end{equation}

	Nous passons maintenant à la partie unicité de la proposition. Soit un homomorphisme surjectif \( \varphi\colon S_n\to \{ -1,1 \}\) et \( \tau\), une transposition telle que \( \varphi(\tau)=-1\) (qui existe parce que sinon \( \varphi\) ne serait pas surjectif\footnote{Nous utilisons ici le fait que tous les éléments de \( S_n\) sont des produits de transpositions, proposition~\ref{PropPWIJbu}.}). Si \( \tau'\) est une autre transposition, il existe \( \sigma\in S_n\) tel que \( \tau'=\sigma\tau\sigma^{-1}\) (lemme~\ref{LemmvZFWP}). Dans ce cas, \( \varphi(\tau')=\varphi(\tau)=-1\), et si \( \sigma=(\tau_1\ldots \tau_k) \),
	\begin{equation}
		\varphi(\sigma)=(-1)^k=\epsilon(\sigma).
	\end{equation}
\end{proof}

\begin{corollary}       \label{CORooZLUKooBOhUPG}
	Si \( \sigma\in S_n\), alors
	\begin{equation}
		\epsilon(\sigma)=\epsilon(\sigma^{-1}).
	\end{equation}
\end{corollary}

\begin{proof}
	Comme énoncé par la proposition \ref{ProphIuJrC}, \( \epsilon\) est un homomorphisme, donc
	\begin{equation}
		\epsilon(\sigma)\epsilon(\sigma^{-1})=\epsilon(\sigma\sigma^{-1})=\epsilon(\id)=1.
	\end{equation}
	Puisque \( \epsilon(\sigma)\) et \( \epsilon(\sigma^{-1})\) ne peuvent valoir que \( \pm1\), ils doivent être tous les deux égaux à \( 1\) ou tous les deux à \( -1\) pour que le produit soit \( 1\).
\end{proof}

%-------------------------------------------------------
\subsection{Permutation un peu ordonnées}
%----------------------------------------------------

\begin{lemma}[\cite{MonCerveau}]		\label{LEMooEOTGooPslULz}
	Deux énoncés.
	\begin{enumerate}
		\item		\label{ITEMooQZDRooUYjcgX}
		      Soit \( \pi\in S_k\) satisfaisant \( \pi(1)=1\). Nous définissons \( \tau\in S_{k-1}\) par \( \tau(i)=\pi(i+1)-1 \).

		      Alors \( \epsilon(\tau)=\epsilon(\pi)\).
		\item		\label{ITEMooBUGZooVCVhKz}
		      L'application
		      \begin{equation}
			      \begin{aligned}
				      \varphi\colon \{ \pi\in S_k\tq \pi(1)=1 \} & \to S_{k-1}                             \\
				      \pi                                        & \mapsto \Big[  \tau(i)=\pi(i+1)-1 \Big]
			      \end{aligned}
		      \end{equation}
		      est une bijection.
	\end{enumerate}
\end{lemma}

\begin{proof}
	Nous nommons \( \pi'\) la restriction de \( \pi\) à \( \{ 2,\ldots,k \}\). Cela est encore une bijection, de telle sorte que \( \pi'\) puisse être écrite sous forme d'un produit de \( n\) transpositions de \( \{ 2,\ldots,k \}\) (proposition \ref{PropPWIJbu}) :
	\begin{equation}
		\pi=\sigma_1\circ\ldots\circ\sigma_n.
	\end{equation}
	Les \( \sigma_i\) étant des transpositions de \( \{ 2,\ldots,k \}\), elles sont des transpositions de \( \{ 1,\ldots,k \}\). Tout cela pour dire que \( \pi\) peut être écrite comme produit de transpositions ne faisant pas intervenir \( 1\).

	Nous introduisons la bijection
	\begin{equation}
		\begin{aligned}
			\psi\colon \{ 2,\ldots,k \} & \to \{ 1,\ldots,k-1 \} \\
			i                           & \mapsto i-1.
		\end{aligned}
	\end{equation}
	En termes de \( \psi\), la définition de \( \tau\) s'écrit \( \psi(\pi)=\psi\circ \pi\circ\psi^{-1}\), et nous avons
	\begin{subequations}
		\begin{align}
			\tau & =\psi\circ \pi\circ\psi^{-1}                                                                           \\
			     & =\psi\circ\sigma_1\circ\ldots\circ\sigma_n\circ\psi^{-1}                                               \\
			     & =\psi\circ\sigma_1\circ\psi^{-1}\circ\psi\circ\sigma_2\circ\ldots\circ\psi\circ\sigma_n\circ\psi^{-1}.
		\end{align}
	\end{subequations}
	Bref, en notant \( \sigma'_i=\psi\sigma_i\psi^{-1}\) nous avons \( \tau=\sigma'_1\circ\ldots\circ\sigma_n\).

	Il suffit maintenant de remarquer que \( \sigma'_i\) est une transposition dans \( \{ 1,\ldots,k-1 \}\). On vérifie pour cela que si \( \sigma=(a,b)\) alors \( \sigma'=\big( \psi(a),\psi(b) \big)\). En effet, pour \( i=1,\ldots,k-1\), il y a trois possibilités : soit \( i=\psi(a)\), soit \( i=\psi(b)\) soit \( i\) n'est ni l'un ni l'autre.

	Si \( i=\psi(a)\), alors \( \sigma'(i)=(\psi\sigma\psi^{-1})\big( \psi(a) \big)=\psi\sigma(a)=\psi(b)\). Même vérification pour montrer que \( \sigma'\big( \psi(b) \big)=\psi(a)\). Si \( i\) n'est ni \( \psi(a)\) ni \( \psi(b)\), alors \( \psi^{-1}(i)\) n'est ni \( a\) ni \( b\) et dans ce cas
	\begin{equation}
		(\psi\sigma\psi^{-1})(i)=\psi\sigma\big( \psi^{-1}(i) \big)=\psi\big( \psi^{-1}(i) \big)=i.
	\end{equation}

	Donc la décomposition \( \tau=\sigma'_1\circ\ldots\circ\sigma_n'\) est une décomposition de \( \tau\) en \( n\) transpositions. Autrement dit le nombre de transpositions est le même pour \( \tau\) que pour \( \pi\).

	En particulier les signatures de \( \tau\) et de \( \pi\) sont les mêmes. Cela finit la preuve du point \ref{ITEMooQZDRooUYjcgX}.

	Prouvons le point \ref{ITEMooBUGZooVCVhKz}.
	\begin{subproof}
		\spitem[Injectif]
		%-----------------------------------------------------------
		Supposons que \( \varphi(\pi)=\varphi(\sigma)\). Alors pour tout \( i=1,\ldots,k-1\) nous avons
		\begin{equation}
			\pi(i+1)-1=\sigma(i+1)-1,
		\end{equation}
		c'est à dire \( \pi(j)=\sigma(j)\) pour tout \( j=2,\ldots,k\). Vu que \( \pi(1)=\sigma(1)\) nous avons bien \( \pi=\sigma\).

		\spitem[Surjectif]
		%-----------------------------------------------------------
		Soit \( \tau\in S_{k-1}\). En posant
		\begin{equation}
			\pi(i)=\begin{cases}
				1           & \text{si } i=1 \\
				\tau(i-1)+1 & \text{sinon, }
			\end{cases}
		\end{equation}
		nous avons bien \( \tau=\varphi(\pi)\).
	\end{subproof}
\end{proof}

\begin{definition}[\cite{MonCerveau, BIBooBVCRooAwDAqk}]		\label{DEFooDFBEooFElghU}
	Nous notons
	\begin{equation}
		S_{(k,l)}=\{ \pi\in S_{k+l}\tq \pi(1)<\ldots \pi(k),\,\pi(k+1)<\ldots<\pi(k+l) \},
	\end{equation}
	et
	\begin{equation}
		A_{(k,l)}=\{ \pi\in S_{(k,l)}\tq \pi(1)=1 \}.
	\end{equation}
\end{definition}

\begin{lemma}[\cite{MonCerveau}]		\label{LEMooCKJAooBIAyVs}
	Deux énoncés.
	\begin{enumerate}
		\item		\label{ITEMooCNWMooIbnHnz}
		      Soit \( \pi\in A_{(k,l)}\). En posant \( \tau(i)=\pi(i+1)-1\) nous avons \( \tau\in S_{(k-1,l)}\).
		\item		\label{ITEMooKNKRooWbLYrF}
		      L'application
		      \begin{equation}
			      \begin{aligned}
				      \varphi\colon A_{(k,l)} & \to S_{(k-1,l)}                          \\
				      \pi                     & \mapsto \Big[  \tau(i)=\pi(i+1)-1  \Big]
			      \end{aligned}
		      \end{equation}
		      est une bijection.
		\item
		      En ce qui concerne la signature, \( \epsilon\big( \varphi(\pi) \big)=\epsilon(\pi)\).
	\end{enumerate}
\end{lemma}

\begin{proof}
	Pour prouver \ref{ITEMooCNWMooIbnHnz}, soit d'abord \( 1\leq i<j\leq k-1\). Nous avons
	\begin{equation}
		\tau(i)=\pi(i+1)-1<\pi(i+j)-1=\tau(j)
	\end{equation}
	parce que \( i+1\) et \( j+1\) sont entre \( 1\) et \( k\). Le même raisonnement tient pour \( k+1\leq i<j\leq k+l\).

	Pour \ref{ITEMooBUGZooVCVhKz}. Même preuve que la partie correspondante du lemme \ref{LEMooEOTGooPslULz}.
\end{proof}

%+++++++++++++++++++++++++++++++++++++++++++++++++++++++++++++++++++++++++++++++++++++++++++++++++++++++++++++++++++++++++++ 
\section{Anneaux}
%+++++++++++++++++++++++++++++++++++++++++++++++++++++++++++++++++++++++++++++++++++++++++++++++++++++++++++++++++++++++++++

\begin{definition}      \label{DEFooKWKGooIOwGTA}
	Un \defe{isomorphisme d'anneaux}{isomorphisme!d'anneaux} est un morphisme d'anneaux\footnote{Définition \ref{DEFooSPHPooCwjzuz}.}, bijectif.
\end{definition}

La distributivité de la partie \ref{ITEMooGMNOooSTGiXw} de la définition \ref{DefHXJUooKoovob} ne traite que de l'addition; pas de la soustraction. Voici une lemme qui dit que ça fonctionne quand même.
\begin{lemma}[\cite{BIBooZFPUooIiywbk}]     \label{LEMooVPYUooRzexke}
	Soient un anneau \( A\) ainsi que \( a,b,c\in A\). Alors
	\begin{equation}
		a(b-c)=ab-ac.
	\end{equation}
\end{lemma}

\begin{proof}
	Nous avons le calcul suivant :
	\begin{subequations}
		\begin{align}
			a(b-c)+ac & =a\big( (b-c)+c \big)     \label{SUBEQooKCOWooFeOHUM} \\
			          & =ab.       \label{SUBEQooMLLOooNRmIYM}
		\end{align}
	\end{subequations}
	Justifications :
	\begin{itemize}
		\item Pour \ref{SUBEQooKCOWooFeOHUM}. Distributivité.
		\item Pour \ref{SUBEQooMLLOooNRmIYM}. Parce que \( (b-c)+c=b\).
	\end{itemize}
	Nous avons donc \( a(b-c)+ac=ab\) et donc l'égalité demandée en ajoutant \( -ac\) des deux côtés.
\end{proof}

\begin{lemma}       \label{LEMooVUSMooWisQpD}
	Pour tout élément \( a\) d'un anneau nous avons \( a\cdot 0=0\).
\end{lemma}

\begin{proof}
	L'élément \( 0\) est le neutre de l'addition. Il peut être écrit \( 1-1\), et en utilisant la distributivité sous la forme du lemme \ref{LEMooVPYUooRzexke},
	\begin{equation}
		a\cdot 0=a\cdot (1-1)=a-a=0.
	\end{equation}
	Notons que la dernière égalité s'écrit en détail \( a-a=a+(-a)\) qui donne le neutre de l'addition.
\end{proof}

\begin{proposition}     \label{PROPooNCCGooXjVyVt}
	Dans un anneau\footnote{Définition \ref{DefHXJUooKoovob}.} non nul, le neutre pour l'addition est distinct du neutre pour la multiplication.
\end{proposition}
\begin{proof}
	Supposons par contraposée que dans un anneau \( A\), \( 1 = 0 \). Alors, pour tout \( a \in A \), on a \( a = 1a = 0a = (1 - 1)a = a - a=0 \), d'où l'on déduit \( -a = 0  \) et par suite, \( a = 0. \)
\end{proof}

\begin{lemma}[\cite{MonCerveau}]        \label{LEMooLTERooVKgqjn}
	Un peu d'arithmétique. Soit un anneau \( A\) et un élément \( a\in A\).
	\begin{enumerate}
		\item       \label{ITEMooUGHCooOPgoeR}
		      \( 1\times 1=1\).
		\item       \label{ITEMooJMBSooVgvVwg}
		      \( (-1)\times a=-a\).
		\item       \label{ITEMooXJGMooKNLlHU}
		      \( -(-a)=a\).
		\item       \label{ITEMooYMRKooHVYYKU}
		      \( (-1)\times (-1)=1\).
	\end{enumerate}
\end{lemma}

\begin{proof}
	En plusieurs parties.
	\begin{subproof}
		\spitem[Pour \ref{ITEMooUGHCooOPgoeR}]
		La définition de \( 1\) est que \( 1\times a=a\) pour tout \( a\). En particulier pour \( a=1\) nous avons le résultat.
		\spitem[Pour \ref{ITEMooJMBSooVgvVwg}]
		Nous avons
		\begin{equation}
			(-1)\times a + a= a\times \big( (-1)+1 \big)=a\times 0=0.
		\end{equation}
		Nous avons utilisé le fait que la multiplication était distributive et que le zéro était absorbant (lemme \ref{LEMooVUSMooWisQpD}).

		\spitem[Pour \ref{ITEMooXJGMooKNLlHU}]
		Nous avons \( -a+a=0\) par définition de la notation \( -a\). Donc \( a\) est bien l'inverse de \( -a\) pour l'addition.

		\spitem[Pour \ref{ITEMooYMRKooHVYYKU}]
		En utilisant les points \ref{ITEMooJMBSooVgvVwg} et \ref{ITEMooXJGMooKNLlHU} nous avons
		\begin{equation}
			(-1)\times (-1)=-(-1)=1.
		\end{equation}
	\end{subproof}
\end{proof}

Soit \( X\) un ensemble et un anneau \( (A, +, \times)\). Nous considérons \( \Fun(X,A)\)\nomenclature[A]{\( \Fun(X,Y)\)}{les applications de \( X\) vers \( Y\)} l'ensemble des applications \( X\to A\). Cet ensemble devient un anneau avec les définitions
\begin{subequations}
	\begin{align}
		(f+g)(x)=f(x)+g(x) \\
		(fg)(x)=f(x)g(x).
	\end{align}
\end{subequations}
C'est la \defe{structure canonique}{structure d'anneau canonique} d'anneau sur \( \Fun(X,A)\).

\begin{definition}
	Le \defe{centralisateur}{centralisateur} de \( x\in A\) dans \( A\) est l'ensemble
	\begin{equation}
		\{ y\in A\tq xy=yx \},
	\end{equation}
	le \defe{centre}{centre!d'un anneau} de \( A\) est
	\begin{equation}
		\{ y\in A\tq xy=yx,\forall x\in A \}.
	\end{equation}
\end{definition}

\begin{definition}[Idéal dans un anneau]  \label{DefooQULAooREUIU}
	Un sous-ensemble \( I\subset A\) est un \defe{idéal à gauche}{idéal!dans un anneau} si
	\begin{enumerate}
		\item
		      \( I\) est un sous-groupe pour l'addition,
		\item
		      pour tout \( a\in A\), \( aI\subset I\).
	\end{enumerate}
	De même nous disons que \( I\subset A\) est une \defe{idéal à droite}{idéal à droite} lorsque \( I\) est un sous-groupe pour l'addition et \( Ia\subset I\) pour tout \( a\in A\).

	Lorsqu'un ensemble est idéal à gauche et à droite, nous disons que c'est un \defe{idéal bilatère}{idéal!bilatère}. Lorsque nous parlons d'idéal sans précision, nous parlons d'idéal bilatère.
\end{definition}

\begin{propositionDef}      \label{PROPooGXMRooTcUGbi}
	Soit \( A\), un anneau, \( I\) un idéal bilatère\footnote{Définition~\ref{DefooQULAooREUIU}.} de \( A\). Nous considérons la relation d'équivalence \( x\sim y\) si et seulement si \( x-y\in I\). Sur le quotient\footnote{Définition \ref{DEFooRHPSooHKBZXl}.}
	\begin{equation}
		A/\sim=A/I,
	\end{equation}
	nous mettons les opérations
	\begin{enumerate}
		\item
		      \( [x]+[y]=[x+y]\)
		\item
		      \( [x][y]=[xy]\).
	\end{enumerate}
	Nous avons alors les résultats suivants :
	\begin{enumerate}
		\item       \label{ITEMooEJPEooRKAqmS}
		      Les opérations sont bien définies,
		\item       \label{ITEMooYBEGooTlHgNz}
		      l'ensemble \( A/I\), muni de ces opérations, est un anneau. Le neutre pour l'addition est \( [0]\), l'inverse de \( [a]\) est \( [-a]\) que nous noterons \( -[a]\).
		\item       \label{ITEMooLNRLooMkoWXZ}
		      la surjection canonique \( \pi\colon A\to A/I\) est un morphisme.
	\end{enumerate}
	Cet anneau est appelé \defe{anneau quotient}{anneau!quotient par un idéal}.
\end{propositionDef}

\begin{proof}
	En plusieurs parties.
	\begin{subproof}
		\spitem[Pour \ref{ITEMooEJPEooRKAqmS}]
		Nous savons que, par définition,
		\begin{equation}
			\bar x=\{ x+i\tq i\in I \}.
		\end{equation}
		Calculons le produit de représentants génériques de \( \bar x\) et de \( \bar y\) :
		\begin{equation}
			(x+i_1)(y+i_2)=xy+xi_2+yi_1+i_1i_2.
		\end{equation}
		Puisque \( I\) est un idéal, nous avons \( xi_2+yi_1+i_1i_2\in I\) et donc bien
		\begin{equation}
			(x+i_1)(y+i_2)\in \overline{ xy }.
		\end{equation}
		\spitem[Pour \ref{ITEMooYBEGooTlHgNz}]
		Il s'agit de vérifier les conditions de la définition \ref{DefHXJUooKoovob}.

		D'abord \( A/I\) est un groupe de neutre \( [0]\). En effet, vu que \( (A,+)\) est un groupe commutatif de neutre \( 0\), nous avons
		\begin{enumerate}
			\item Neutre : $[a]+[0]=[a+0]=[a]$.
			\item Associativité :
			      $[a]+([b]+[c])=[a]+[b+c]=[a+b+c]=[a+b]+[c]$.
			\item Inversibilité : l'inverse de \( [a]\) est \( [-a]\) parce que \( [a]+[-a]= [a-a]=[0] \).
		\end{enumerate}
		Nous pouvons noter \( -[a]\) l'élément \( [-a]\). Le groupe \( A/I\) est commutatif:
		\begin{equation}
			[a]+[b]=[a+b]=[b+a]=[b]+[a].
		\end{equation}
		Donc \( (A/I,+)\) est un groupe commutatif de neutre \( [0]\).

		L'associativité de \( A\) donne l'associativité dans \( A/I\) :
		\begin{equation}
			\big( [a][b] \big)[c]=[ab][c]=[abc]=[a][bc]=[a]\big( [b][c] \big).
		\end{equation}
		Et enfin pour la distributivité,
		\begin{equation}
			[a]\big( [b]+[c] \big)=[a][b+c]=[a(b+c)]=[ab+ac]=[ab]+[ac]=[a][b]+[a][c].
		\end{equation}
		Nous avons prouvé que \( A/I\) est un anneau de neutre \( [0]\) et d'unité \( [1]\).
		\spitem[Pour \ref{ITEMooLNRLooMkoWXZ}]
		Nous devons vérifier les trois conditions de la définition \ref{DEFooSPHPooCwjzuz}. Cela est immédiat parce que \( \pi(x)=[x]\).
	\end{subproof}
\end{proof}


\begin{definition}          \label{DefrYwbct}
	Soient \( A\) un anneau commutatif et \( S\subset A\). Nous disons que \( \delta\in A\) est un \defe{PGCD}{pgcd!dans un anneau intègre} de \( S\) si
	\begin{enumerate}
		\item
		      \( \delta\) divise tous les éléments de \( S\).
		\item       \label{ITEMooVCKGooWDXZOj}
		      si \( d\) divise également tous les éléments de \( S\), alors \( d\) divise \( \delta\).
	\end{enumerate}
	Nous disons que \( \mu\in A\) est un \defe{PPCM}{ppcm!dans un anneau intègre} de \( S\) si
	\begin{enumerate}
		\item
		      \( S\divides \mu\),
		\item
		      si \( S\divides m\), alors \( \mu\divides m\).
	\end{enumerate}
	Si \( P\) et \( Q\) sont des polynômes, ce que nous notons \( \pgcd(P,Q)\) est l'unique polynôme unitaire dans \( \pgcd\big( \{ P,Q \} \big)\). Voir \ref{NORMooUJDJooWfijxT}.
\end{definition}

\begin{remark}
	Au sens de la définition \ref{DefrYwbct}, le pgcd n'est pas unique. Dans \( \eZ\) par exemple les nombres \( 4\) et \( -4\) sont tous deux pgcd de \( \{4,16  \}\).

	Dans \( \eZ\) cependant, nous modifions implicitement la définition et nous n'acceptons que les positifs, de telle sorte à ce que l'unique pgcd soit effectivement le plus grand pour l'ordre usuel sur \( \eZ\).

	Pour l'unicité dans \( \eZ\), voir \ref{LEMooBJVJooFyuFeN}.
\end{remark}

%---------------------------------------------------------------------------------------------------------------------------
\subsection{Anneau intègre}
%---------------------------------------------------------------------------------------------------------------------------

\begin{definition}[Diviseurs dans un anneau]\label{DiviseursAnneau}
	Soient \( a, b \in A \). On dit que \( a\) divise \( b\), ou que \( a\) est un \defe{diviseur (à gauche)}{diviseur!dans un anneau} de \( b\) si il existe \( c \in A \) tel que \( ac = b \). On dit que c'est un diviseur de \( b\) à droite si \( ca = b \) pour un certain \( c \in A \).
\end{definition}
Un cas particulier est le cas des diviseurs de zéro. L'absence de tels diviseurs dans un anneau est une propriété intéressante: on dit dans ce cas que l'anneau est intègre. Nous étudions ces anneaux plus en détail en section~\ref{SECAnneauxIntegres}.

Un élément \( a\in A\) est \defe{régulier à droite}{régulier à droite} si \( ba=0\) implique \( b=0\). Il est régulier à gauche si \( ab=0\) implique \( b=0\).

\begin{definition}[Éléments nilpotents, unipotents]  \label{DEFooHRRYooTmbUTH}
	On dit que \( a \in A \) est \defe{nilpotent}{nilpotent} si il existe \( n \in \eN \) tel que \( a^n = 0 \). Il est dit \defe{unipotent}{unipotent} si \( a-1\) est nilpotent, c'est-à-dire si \( (a-1)^n =0\) pour un certain \( n \in \eN \).
\end{definition}

\begin{definition}[Éléments inversibles]        \label{DEFooCIHVooAhpJxy}
	Un élément \( a \in A \) est dit \defe{inversible}{élément!inversible!dans un anneau} si il existe \( b \in A \) tel que \( ab = 1 \).

	L'ensemble \( U(A)\)\nomenclature[A]{\( U(A)\)}{ensemble des inversibles} des éléments inversibles de \( A\) est un groupe pour la multiplication. Nous notons \( A^*=A\setminus\{ 0 \}\).
\end{definition}

Conformément à la définition \ref{DiviseursAnneau} de diviseur, nous posons la définition suivante pour les diviseurs de zéro.
\begin{definition}[diviseur de zéro\cite{ooTNKJooSCSCZQ}]		\label{DEFooCIYLooFkhVOc}
	Un élément \( a\neq 0\) est un \defe{diviseur de zéro à gauche}{diviseur!de zéro} si il existe \( x\neq 0\) tel que \( ax=0\). L'élément \( a\) est un diviseur de zéro \defe{à droite}{diviseur!de zéro à droite} si il existe \( y\neq 0\) tel que \( ya=0\).

	Nous disons que \( a\) est un \defe{diviseur de zéro}{diviseur de zéro} si il est un diviseur de zéro à gauche ou à droite.
\end{definition}

\begin{propositionDef}[Anneau intègre\cite{MonCerveau}]           \label{DEFooTAOPooWDPYmd}
	Soit \( A\) un anneau non réduit à \( \{ 0 \}\). Les assertions suivantes sont équivalentes:
	\begin{enumerate}
		\item       \label{ITEMooMXMKooXMYpkN}
		      \( A\) ne possède pas de diviseurs de zéro\footnote{Définition \ref{DEFooCIYLooFkhVOc}.}.
		\item       \label{ITEMooLAJCooFwxXrV}
		      La règle du produit nul s'applique dans \( A\): pour tous \( a, b \in A \), si \( ab=0\), alors \( a = 0\) ou \( b = 0\).
		      \index{règle du produit nul}
		\item       \label{ITEMooQNTFooSRrVPK}
		      On peut simplifier par un même élément non-nul, deux expressions produit dans \( A\) qui sont égales: pour tous \( a, b, c \in A \) avec \( a \neq 0 \), si \( ab = ac \), alors \( b = c \).
	\end{enumerate}
	Un anneau non réduit à \( \{ 0 \}\) qui vérifie ces propriétés est dit \defe{intègre}{anneau intègre}.
\end{propositionDef}

\begin{proof}
	En trois implications.
	\begin{subproof}
		\spitem[\ref{ITEMooMXMKooXMYpkN} implique \ref{ITEMooLAJCooFwxXrV}]

		Si \( ab=0\) avec \( b\neq 0\) alors \( a\) est un diviseur de zéro. Vu que nous supposons que \( A\) n'a pas de diviseurs de zéros, \( a\) est nul. De même, si \( a\neq 0\) \( b\) devrait être nul.
		\spitem[\ref{ITEMooLAJCooFwxXrV} implique \ref{ITEMooQNTFooSRrVPK}]

		Si \( ab=ac\), alors \( a(b-c)=0\) et l'hypothèse dit que soit \( a=0\), soit \( b-c=0\). Donc si \( a\neq 0\), alors \( b-c=0\).
		\spitem[\ref{ITEMooQNTFooSRrVPK} implique \ref{ITEMooMXMKooXMYpkN}]
		Si \( A=\{ 0 \}\), le point \ref{ITEMooQNTFooSRrVPK} n'est pas applicable.

		Si \( a\neq 0\) et \( ax=0\), alors nous avons aussi \( ax=a\times 0\). Par propriété de simplification, \( x=0\). Donc \( a\) n'est pas un diviseur de zéro à gauche. Nous prouvons de la même façon qu'il n'y a pas de diviseurs de zéro à droite.
	\end{subproof}
\end{proof}

\begin{lemma}		\label{LEMooIKNMooMfvQnu}
	Un corps est un anneau intègre.
\end{lemma}

\begin{proof}
	Nous vérifions la définition \ref{DEFooTAOPooWDPYmd}\ref{ITEMooQNTFooSRrVPK}. Si \( ab=ac\) avec \( a\neq 0\), alors il suffit de multiplier à gauche par \( a^{-1}\) (qui existe parce que nous sommes dans un corps) pour obtenir \( b=c\).
\end{proof}


\begin{lemma}[\cite{MonCerveau}]		\label{LEMooSFHMooQoKsPV}
	Soient un anneau intègre \( A\), et une partie \( S\subset A\). Si un des \( \pgcd\) de \( S\) est inversible\footnote{Définition \ref{DEFooCIHVooAhpJxy}.}, alors ils le sont tous.
\end{lemma}

\begin{proof}
	Pour rappel, les pgcd d'une partie de \(A\) sont définis dans \ref{DefrYwbct}. Soit un \( \pgcd\) inversible de \( S\), ainsi qu'un autre \( \pgcd\) que nous nommons \( \delta'\). Vu que \( \delta'\) divise tous les éléments de \( S\), il est divisé par \( \delta\) : \( \delta\divides\delta'\). Réciproquement, \( \delta'\divides \delta\).

	Soient \( x\) et \( y\) définis par \( \delta=x\delta'\) et \( \delta'=y\delta\). Nous avons
	\begin{equation}
		\delta=x\delta'=xy\delta.
	\end{equation}
	Comme l'anneau \( A\) est intègre, nous pouvons simplifier par \( \delta\) et voir \( xy=1\), ce qui signifie que \( x\) et \( y\) sont inversibles. Donc si \( \delta\) est inversible, alors \( \delta'=y\delta\) est inversible.
\end{proof}

\begin{lemma}[\cite{MonCerveau,BIBooZXSRooOqGHBA}]		\label{LEMooMHZQooIcSNSf}
	Soient un anneau intègre, \( A\), une partie \( S\subset A\) et un élément \( a\in A\). Nous avons\footnote{Définition du pgcd: \ref{DefrYwbct}.}
	\begin{equation}
		\pgcd(aS)=a\pgcd(S).
	\end{equation}
\end{lemma}

\begin{proof}
	Deux inclusions à prouver.
	\begin{subproof}
		\spitem[\( \pgcd(aS)\subset a\pgcd(S)\)]
		%-----------------------------------------------------------
		Soit un pgcd \( \delta\) de \( aS\). Nous devons trouver un \( \delta'\in\pgcd(S)\) tel que \( \delta=a\delta'\). En termes de notations, nous notons \( S=\{ s_i \}_{i\in I}\). Pour chaque \( i\) nous avons \( \delta\divides as_i\) : il existe \( x_i\in A\) tel que
		\begin{equation}
			\delta x_i=as_i.
		\end{equation}
		Vu que \( a\) divise tous les éléments de \( aS\), il divise n'importe quel pgcd de \( aS\), et en particulier \( a\divides \delta\) : il existe \( \delta'\in A\) tel que \( \delta=a\delta'\). Nous montrons que \( \delta'\in\pgcd(S)\).

		Nous savons que \( \delta x_i=as_i\). En remplaçant \( \delta\) par \( a\delta'\), \( a\delta'x_i=as_i\). Vu que nous sommes dans un anneau intègre, nous pouvons simplifier par \( a\) (définition \ref{DEFooTAOPooWDPYmd}\ref{ITEMooQNTFooSRrVPK}) :
		\begin{equation}
			\delta'x_i=s_i.
		\end{equation}
		Donc \( \delta'\) divise tous les éléments de \( S\), et vérifie la première condition pour être un pgcd de \( S\). Pour la seconde condition, nous supposons que \( d\) divise tous les éléments de \( S\). Nous avons \( d\divides S\), donc \( ad\divides aS\). Et comme \( \delta\) est un pgcd de \( aS\), nous déduisons que \( ad\) divise \( \delta\). Il existe \( y\in A\) tel que
		\begin{equation}
			ady=\delta.
		\end{equation}
		Nous remplaçons \( \delta\) par sa valeur \( a\delta'\) : \( ady=a\delta'\). Encore une fois nous simplifions par \( a\) et nous trouvons \( dy=\delta'\), c'est à dire que \( \delta\) divise \( \delta'\).

		\spitem[\(a \pgcd(S)\subset \pgcd(aS)\)]
		%-----------------------------------------------------------

		Soit un pgcd \( \delta\) de \( S\). Nous voulons que \( a\delta\in \pgcd(aS)\). Vu que \( \delta\in\pgcd(S)\) nous avons \( \delta x_i=s_i\) pour tout \( i\), et donc aussi \( a\delta x_i=as_i\), de telle sorte que \( a\delta\) divise tous les éléments de \( aS\).

		Soit maintenant \( d\in A\) divisant tous les éléments de \( aS\). Nous devons prouver que \( d\divides a\delta\).

		\begin{subproof}
			\spitem[Travail préliminaire]
			%-----------------------------------------------------------

			Nous considérons \( \delta'\in \pgcd(aS)\). Vu que \( \delta\divides s\) pour tout \( s\in S\), nous avons aussi \( a\delta\divides as\) pour tout \( s\in S\). Comme \( \delta'\) est un est un pgcd de \( aS\), nous avons donc
			\begin{equation}
				a\delta\divides\delta'.
			\end{equation}
			Soit \( u\in A\) tel que \( \delta'=a\delta u\).

			En utilisant la première partie de la preuve, nous avons
			\begin{equation}
				\delta'\in\pgcd(aS)\subset a\pgcd(S).
			\end{equation}
			Donc il existe \( \delta_1\in\pgcd(S)\) tel que \( \delta'=a\delta_1\). En écrivant l'égalité \( \delta' =a\delta u\) avec cette valeur de \( \delta'\), nous trouvons
			\begin{equation}		\label{EQooVHSSooDdVUeW}
				a\delta_1=a\delta u,
			\end{equation}
			et donc \( \delta_1=\delta u\) parce que \( A\) est intègre. Vu que \( \delta_1\in\pgcd(S)\), nous avons aussi \( \delta u\in\pgcd(S)\). En particulier \( \delta u\) divise tous les éléments de \( S\), et donc divise \( \delta\) qui est un pgcd de \( S\) : \( \delta u\divides \delta\). En multipliant par \( a\),
			\begin{equation}
				\delta'=a\delta u\divides a \delta.
			\end{equation}

			\spitem[Résumé]
			%-----------------------------------------------------------
			Nous avons considéré \( \delta\in\pgcd(S)\) et nous sommes en train de prouver que \( a\delta\in\pgcd(aS)\). Nous avons déjà prouvé que si \( \delta'\in \pgcd(aS)\), alors nous avons \( \delta'\divides a\delta\).

			Nous posons \( y\in A\) tel que \( \delta' y=a\delta\).

			\spitem[Et enfin]
			%-----------------------------------------------------------
			Soit \( d\) divisant tous les éléments de \( a\delta\). Donc \( d\divides \delta'\) : il existe \( x\in A\) tel que \( dx=\delta'\). En multipliant par \( y\),
			\begin{equation}
				dxy=\delta' y=a\delta.
			\end{equation}
			Nous avons montré que \( d\) divise \( a\delta\), ce qu'il nous fallait.
		\end{subproof}
	\end{subproof}
\end{proof}

\begin{probleme}
	Je ne suis pas certain du lemme \ref{LEMooZSUNooUmYmgt}. Essayez de le démontrer, et envoyez-moi la preuve pour que je puisse l'ajouter.
\end{probleme}
\begin{lemma}[\cite{MonCerveau}]		\label{LEMooZSUNooUmYmgt}
	Soient un anneau intègre \( A\) et une partie \( S\subset A\). Si \( \delta\in\pgcd(S)\), alors \( \pgcd(S/\delta)\) ne contient que des inversibles.
\end{lemma}

\begin{lemma}[\cite{MonCerveau}]		\label{LEMooZKASooKstTuK}
	Soit un anneau intègre \( A\). Si \( \delta\in\pgcd(S)\) et si \( u\in A\) est inversible, alors \( \delta u\in\pgcd(S)\).
\end{lemma}

\begin{proof}
	Soit \( s\in S\). Si \( \delta x=s\), alors \( u\delta (u^{-1} x)=s\). Donc \( u\delta\) divise tous les éléments de \( S\). De plus si \( d\divides S\), alors \( d\divides \delta\). Dans ce cas il existe \( y\) tel que \( dy=\delta\). Nous avons alors aussi
	\begin{equation}
		dxu=\delta u,
	\end{equation}
	de telle sorte que \( d\) divise \( \delta u\).
\end{proof}

%--------------------------------------------------------------------------------------------------------------------------- 
\subsection{Fonction puissance}
%---------------------------------------------------------------------------------------------------------------------------

Voici une première définition de la fonction puissance. Il y en aura d'autres, de plus en plus générales. Voir le thème \ref{THEMEooBSBLooWcaQnR}.
\begin{definition}\label{DEFooGVSFooFVLtNo}
	Si \( A\) est un anneau, si \( a\in A\) et si \( n\in \eN\), nous définissons \( a^n\) par récurrence :
	\begin{enumerate}
		\item
		      \( a^0=1\) (l'unité pour la multiplication dans \( A\)),
		\item       \label{ITEMooOUIPooGjAgQb}
		      \( a^{k+1}=a\cdot a^{k}\).
	\end{enumerate}
\end{definition}

Le lemme suivant dit que le point \ref{ITEMooOUIPooGjAgQb} de la définition \ref{DEFooGVSFooFVLtNo} aurait pu être écrit \( a^k\cdot a\) au lieu de \( a\cdot a^k\).
\begin{lemma}[\cite{MonCerveau}]        \label{LEMooWPARooYLZlzr}
	Si \( A\) est un anneau, si \( a\in A\) et si \( n\in \eN\), alors
	\begin{equation}
		a^n=a\cdot a^{n-1}=a^{n-1}\cdot a.
	\end{equation}
\end{lemma}

\begin{proof}
	Cela se prouve par récurrence. Pour \( n=1\) c'est l'égalité \( a=a^0a\) qui est correcte parce que par définition \( a^0=1\).

	Supposons que le résultat soit bon pour \( n\) et voyons ce que ça donne pour \( n+1\) :
	\begin{subequations}
		\begin{align}
			a^{n+1} & =aa^n        & \text{Définition de } a^{n+1}            \\
			        & =a(a^{n-1}a) & \text{hypothèse de récurrence pour } a^n \\
			        & =(aa^{n-1})a & \text{associativité}                     \\
			        & =a^na        & \text{Définition de } a^n.
		\end{align}
	\end{subequations}
\end{proof}
