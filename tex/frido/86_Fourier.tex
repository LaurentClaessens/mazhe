% This is part of Mes notes de mathématique
% Copyright (c) 2011-2015,2017-2020,2022
%   Laurent Claessens
% See the file fdl-1.3.txt for copying conditions.

%+++++++++++++++++++++++++++++++++++++++++++++++++++++++++++++++++++++++++++++++++++++++++++++++++++++++++++++++++++++++++++
\section{Densité des polynômes trigonométriques}
%+++++++++++++++++++++++++++++++++++++++++++++++++++++++++++++++++++++++++++++++++++++++++++++++++++++++++++++++++++++++++++

%---------------------------------------------------------------------------------------------------------------------------
\subsection{Convergence pour les fonctions continues (via Weierstrass)}
%---------------------------------------------------------------------------------------------------------------------------

Le résultat fondamental qui nous permet d'utiliser les polynômes trigonométriques comme base pour les fonctions \emph{continues} périodiques est le suivant. Notons que pour les fonctions non continues, il y a encore du travail.
\begin{lemma}   \label{LemXGYaRlC}
	Si \( f\colon \eR\to \eC\) est une fonction continue \( 2\pi\)-périodique et si \( \epsilon>0\), alors il existe un polynôme trigonométrique \( P\) tel que \( \| f-P \|_{\infty}\leq \epsilon\).
\end{lemma}

\begin{proof}
	Nous allons utiliser le théorème de Stone-Weierstrass~\ref{ThoWmAzSMF}. Soit le compact de Hausdorff\footnote{Définition \ref{DefYFmfjjm}.}
	\begin{equation}
		S^1=\{ z\in \eC\tq | z |=1 \},
	\end{equation}
	et \( C(S^1,\eC)\) l'algèbre des fonctions continues de \( S^1\) vers \( \eC\). Il suffit de vérifier que les polynômes trigonométriques vérifient les hypothèses du théorème de Stone-Weierstrass. Un polynôme trigonométrique est un polynôme en \( z\) et \( \bar z\) défini sur \( S^1\).
	\begin{enumerate}
		\item
		      Le polynôme constant est dans l'algèbre, ok.
		\item
		      Pour la séparation des points, considérons le polynôme trigonométrique \( x\mapsto  e^{ix}\).
		\item
		      Si \( P\) est un polynôme en \( z\) et \( \bar z\), alors \( \bar P\) l'est aussi.
	\end{enumerate}
	Donc si \( \epsilon>0\) et \( \tilde f\in C(S^1,\eC)\) sont donnés, il existe un polynôme trigonométrique \( P\) tel que
	\begin{equation}
		\sum_t| \tilde f( e^{it})-P(t) |<\epsilon.
	\end{equation}
	Soit \( f\colon \eR\to \eC\) une fonction continue \( 2\pi\)-périodique. Nous considérons \( \tilde f\in C(S^1,\eC)\) donnée par \( \tilde f( e^{it})=f(t)\). Alors \( \sup_t| f(t)-P(t) |\leq \epsilon\).
\end{proof}

%---------------------------------------------------------------------------------------------------------------------------
\subsection{Convergence pour les fonctions continues (via Fejér)}
%---------------------------------------------------------------------------------------------------------------------------

Si nous ne voulons pas passer par le gros théorème de Stone-Weierstrass pour prouver la densité des polynômes trigonométriques dans \( \big( C^0_{2\pi},\| . \|_{\infty} \big)\), nous pouvons passer par le gros théorème de Fejér. C'est ce que nous faisons maintenant.

Si vous vous intéressez seulement au théorème sur les séries de Fourier, vous cherchez probablement le théorème \ref{ThozHXraQ}.

\begin{propositionDef}      \label{PROPooUOKAooGzGZWc}
	Pour chaque \( n\in \eN\), la fonction
	\begin{equation}
		\begin{aligned}
			D_n\colon \eR & \to \eC                        \\
			t             & \mapsto \sum_{k=-n}^n  e^{ikt}
		\end{aligned}
	\end{equation}
	est bien définie. Elle s'appelle \defe{noyau de Dirichlet}{noyau!Dirichlet}\index{Dirichlet!noyau}.
\end{propositionDef}

\begin{definition}
	Le \defe{noyau de Fejér}{noyau!Fejér}\index{Fejér!noyau} est la moyenne de Cesàro\footnote{Définition \ref{DEFooLVRLooTeowkn}.} des noyaux de Dirichlet :
	\begin{equation}
		F_n(t)=\frac{1}{ n }\sum_{k=0}^{n-1}D_k(t).
	\end{equation}
\end{definition}

\begin{lemma}   \label{LemHPoIkwu}
	Le noyau de Dirichlet s'exprime sous la forme
	\begin{equation}
		D_n(t)=\sum_{k=-n}^n e^{-ikt}=\frac{ \sin\left( \frac{ 2n+1 }{ 2 }t \right) }{ \sin(t/2) }
	\end{equation}
\end{lemma}

\begin{proof}
	Nous commençons par mettre en facteur le premier terme :
	\begin{equation}
		D_n(t)=\sum_{k=-n}^n e^{ikt}= e^{-int}\sum_{k=0}^{2n} e^{ikt}.
	\end{equation}
	En utilisant la formule de la somme géométrique,
	\begin{subequations}
		\begin{align}
			D_n(t) & = e^{-int}\frac{ 1-( e^{it})^{2n+1} }{ 1- e^{it} }                                                                       \\
			       & = e^{-int}\frac{ 1- e^{(2n+1)it} }{ 1- e^{it} }                                                                          \\
			       & = e^{-int}\frac{ e^{(2n+1)it/2} }{ e^{i\frac{ t }{ 2 }} }\frac{ e^{-(2n+1)it/2}- e^{(2n+1)it/2} }{ e^{-it/2}- e^{it/2} } \\
			       & = \frac{ (-2i)\sin\left( \frac{ 2n+1 }{ 2 }t \right) }{ (-2i)\sin\left( \frac{ t }{2} \right) }.
		\end{align}
	\end{subequations}
\end{proof}

\begin{theorem}[Théorème de Dirichlet]\index{théorème!Dirichlet}\index{Dirichlet!théorème}
	Soit \( f\) une fonction \( 2\pi\)-périodique et \( C^1\) par morceaux. Pour tout \( x\in \eR\) nous posons
	\begin{equation}
		s_n(x)=\sum_{k=-n}^nc_k(f) e^{ikx}.
	\end{equation}
	Alors nous avons
	\begin{equation}
		\lim_{n\to \infty} s_n(x)=\frac{ f(x^+)+f(x^-) }{ 2 }.
	\end{equation}
\end{theorem}

\begin{lemma}   \label{LemtCAjJz}
	Le noyau de Fejér s'exprime sous la forme
	\begin{equation}    \label{EqLOtzCf}
		F_n(t)=\frac{1}{ n }\left( \frac{ \sin\frac{ nt }{2} }{ \sin\frac{ t }{2} } \right)^2.
	\end{equation}
\end{lemma}
Note : ce noyau est positif. C'est important parce qu'on s'en sert dans la preuve du théorème de Fejér.

\begin{proof}
	L'astuce est de noter \( \sin(x)=\Im( e^{ix})\) et de repartir du résultat à propos du noyau de Dirichlet. En utilisant encore la formule de la série géométrique partielle\footnote{Proposition \ref{PROPooWOWQooWbzukS}.},

	\begin{subequations}
		\begin{align}
			F_n(t) & =\frac{1}{ n\sin(t/2) }\Im\sum_{k=0}^{n-1} e^{(2k+1)it/2}                                                                                                                           \\
			       & =\frac{1}{ n\sin(t/2) }\Im e^{\frac{ it }{ 2 }}\sum_{k=0}^{n-1}                                                                                                                     \\
			       & =\frac{1}{ n\sin(t/2) }\Im e^{\frac{ it }{ 2 }}\left( \frac{ 1- e^{nit} }{ 1- e^{it} } \right)                                                                                      \\
			       & =\frac{1}{ n\sin(t/2) }\Im e^{it/2}\frac{ e^{\frac{ nit }{ 2 }}\left( e^{-\frac{ int }{2}}- e^{\frac{ nit }{2}} \right) }{  e^{\frac{ it }{2}}\left(  e^{-it/2}- e^{it/2} \right) } \\
			       & =\frac{1}{ n\sin(t/2) }\underbrace{\Im e^{\frac{ nit }{2}}}_{\sin(nt/2)}\frac{ \sin\left( \frac{ nt }{ 2 } \right) }{ \sin(\frac{ t }{2}) }                                         \\
			       & =\frac{1}{ n }\left( \frac{ \sin\frac{ nt }{2} }{ \sin\frac{ t }{2} } \right)^2.
		\end{align}
	\end{subequations}
\end{proof}

Le théorème de Fejèr donne la convergence au sens de Cesàro\footnote{Convergence au sens de Cesàro, définition \ref{DEFooLVRLooTeowkn}.} de la série de Fourier dans le cas continu et périodique. Pour avoir une convergence plus forte que Cesàro, il faut plus d'hypothèses, comme le montre le contre-exemple de la proposition \ref{PropREkHdol}. Voir la discussion \ref{NORMooGKKWooFmOBeE}.
\begin{theorem}[Fejèr]      \label{ThoJFqczow}
	Soit \( f\colon \eR\to \eC\) une fonction continue et \( 2\pi\)-périodique. Pour tout \( k\in \eZ\) nous notons
	\begin{equation}
		\begin{aligned}
			e_k\colon \eR & \to \eC          \\
			x             & \mapsto e^{ikx}.
		\end{aligned}
	\end{equation}
	Pour chaque \( n\in \eN\) nous posons
	\begin{subequations}
		\begin{align}
			D_n & =\sum_{k=-n}^ne_k                   & S_n(f)                              & =\sum_{k=-n}^nc_k(f)e_k               \\
			F_n & =\frac{ D_0+\cdots + D_{n-1} }{ n } & \tilde F_n=\sigma_n\big( S(f) \big) & =\frac{1}{ n }\sum_{k=0}^{n-1}S_k(f),
		\end{align}
	\end{subequations}
	où
	\begin{equation}
		c_k(f)=\frac{1}{ 2\pi }\int_{-\pi}^{\pi}f(t) e^{-ikt}dt.
	\end{equation}
	Alors
	\begin{enumerate}
		\item
		      \( \frac{1}{ 2\pi }\int_{-\pi}^{\pi}F_n(t)dt=1\).
		\item
		      Pour tout \( \alpha\in \mathopen] 0 , \pi \mathclose[\), \( F_n\) converge uniformément vers \( 0\) sur \( \mathopen[ -\pi , \pi \mathclose]\setminus\mathopen[ -\alpha , \alpha \mathclose]\).
		\item
		      La suite \( \tilde F_n \) converge uniformément sur \( \eR\) vers \( f\).
		\item   \label{ItemUNQSPmyiv}
		      Le système trigonométrique \( \{ e_k \}_{k\in\eZ}\) est total pour l'espace \( \big( C^0_{2\pi}(\eR),\| . \|_{\infty} \big)\) des fonctions continues \( 2\pi\)-périodiques.
	\end{enumerate}
\end{theorem}
\index{théorème!Fejér}

\begin{proof}
	Un calcul usuel montre que
	\begin{equation}
		\int_{-\pi}^{\pi}e_l(t)dt=\begin{cases}
			0    & \text{si } l\neq 0 \\
			2\pi & \text{si } l=0
		\end{cases}
	\end{equation}
	Nous avons alors
	\begin{equation}
		\frac{1}{ 2\pi }\int_{-\pi}^{\pi}F_n(t)dt=\frac{1}{ 2\pi }\frac{1}{ n }\sum_{k=0}^{n-1}\sum_{l=-k}^k\underbrace{\int_{-\pi}^{\pi}e_l(t)dt}_{2\pi\delta_{l,0}}=\frac{1}{ n }\sum_{k=0}^{n-1}1=1.
	\end{equation}
	Cela prouve déjà le premier point.

	Pour le second point, en partant de l'expression \eqref{EqLOtzCf} et en considérant \( x\in\mathopen[ -\pi, \pi ,  \mathclose]\setminus\mathopen[ -\alpha , \alpha \mathclose]\) (ce qui nous évite l'annulation du dénominateur),
	\begin{equation}
		| F_n(x) |\leq\frac{1}{ (n+1)\sin^2(\alpha/2) },
	\end{equation}
	et donc \( F_n\to 0\) uniformément sur l'ensemble considéré.

	Nous passons maintenant à cette histoire de convergence uniforme de la moyenne de Cesàro vers \( f\). Pour tout \( n\in \eN\) nous avons
	\begin{subequations}
		\begin{align}
			S_n(x) & =\frac{1}{ 2\pi }\sum_{k=-n}^n\left( \int_{-\pi}^{\pi}f(t) e^{-ikt}dt \right) e^{ikx} \\
			       & =\frac{1}{ 2\pi }\int_{-\pi}^{\pi}f(t)\sum_{k=-n}^ne_k(x-t)                           \\
			       & =\frac{1}{ 2\pi }\int_{-\pi}^{\pi}f(t)D_k(x-t).
		\end{align}
	\end{subequations}
	Par conséquent, en effectuant le changement de variable \( u=x-t\) et en utilisant la périodicité,
	\begin{subequations}    \label{EqkDsyAc}
		\begin{align}
			\tilde F_n(x) & =\int_{-\pi}^{\pi}f(t)F_n(x-t)dt     \\
			              & =-\int_{x+\pi}^{x-\pi}f(x-u)F_n(u)du \\
			              & =\int_{-\pi}^{\pi}f(x-u) F_n(u)du.
		\end{align}
	\end{subequations}
	Nous prouvons à présent l'uniforme continuité. Soit \( \epsilon>0\); étant donné que \( f\) est continue et \( 2\pi\)-périodique, elle est uniformément continue et nous considérons \( \delta>0\) tel que \( | x-y |<\delta\) implique \( \big| f(x)-f(y) \big|<\epsilon\). Soit \( M\) un majorant de \( | f |\) sur \( \eR\). L'équation \eqref{EqkDsyAc} nous donne
	\begin{subequations}
		\begin{align}
			\big| f(x)-\tilde F_n(x) \big| & =\left\| \frac{1}{ 2\pi }\int_{-\pi}^{\pi}\big( f(x-t)-f(x) \big)F_n(t)dt \right\|    \label{ykuGGh}                         \\
			                               & \leq\frac{1}{ 2\pi }\int_{\delta\leq| t |\leq \pi}| 2MF_n(t) |dt+\frac{1}{ 2\pi }\int_{-\delta}^{\delta}\epsilon| F_n(t) |dt \\
			                               & \leq\frac{ 2M }{ 2\pi }\int_{\delta\leq | t |\leq\pi}F_n(t)dt+\epsilon'     \label{uRAMyq}
		\end{align}
	\end{subequations}
	Pour obtenir \eqref{ykuGGh} nous avons pu rentrer \( f(x)\) dans l'intégrale en utilisant le premier point. Pour obtenir \eqref{uRAMyq} nous avons d'abord utilisé la positivité de \( F_n\) (lemme~\ref{LemtCAjJz}) pour enlever les valeurs absolues, et nous avons ensuite utilisé le fait que son intégrale valait \( 2\pi\).

	Étant donné que \( F_n\to 0\) uniformément sur \( \mathopen[ -\pi,\pi ,  \mathclose]\setminus\mathopen[ -\alpha , \alpha \mathclose]\), il existe un \( N\) tel que
	\begin{equation}
		\int_{\delta\leq| t |\leq \pi}F_n(t)dt\leq \epsilon
	\end{equation}
	dès que \( n>N\). Le résultat en découle.

	Pour le point~\ref{ItemUNQSPmyiv}, il suffit de remarquer que chacun des \( \tilde F_n\) est une combinaison finie d'éléments du système trigonométrique.
\end{proof}


%---------------------------------------------------------------------------------------------------------------------------
\subsection{Densité dans \texorpdfstring{\(  L^p\)}{Lp}}
%---------------------------------------------------------------------------------------------------------------------------

Nous venons de voir (de deux façons différentes) que les polynômes trigonométriques étaient dense dans \( \big( C^0_{2\pi}(\eR),\| . \|_{\infty} \big)\). Nous avons aussi déjà vu par le théorème~\ref{ThoQGPSSJq} que ces polynômes trigonométriques étaient denses dans \( L^p(S^1)\). Nous présentons à présent une autre façon de prouver cette dernière densité.

\begin{theorem}     \label{ThoDPTwimI}
	Les polynômes trigonométriques sont denses dans \( L^p(S^1)\) pour \( 1\leq p <\infty\).
\end{theorem}

\begin{proof}
	Par les théorèmes~\ref{LemXGYaRlC} ou~\ref{ThoJFqczow} (au choix), nous savons que les polynômes trigonométriques sont denses dans \( \big( C^0_{2\pi}(S^1),\| . \|_{\infty} \big)\). Comme \( S^1\) est compact, la densité est également au sens \( L^p\). En effet si \( \| f_n-f \|_{\infty}\leq \epsilon\), alors
	\begin{equation}
		\| f_n-f \|_{\infty}=\int_0^{2\pi}| f_n-f |^p\leq\int_0^{2\pi}\epsilon^p=2\pi\epsilon^p.
	\end{equation}
	Donc les polynômes trigonométriques sont denses dans \( \big( C^0_{2\pi}(S^1),\| . \|_p \big)\). Mais nous savons par le théorème \ref{ThoILGYXhX}\ref{ItemYVFVrOIv} que les fonctions continues sont denses dans \( L^p(S^1)\).

	Par composition de densités, les polynômes trigonométriques sont denses dans \( L^p(S^1)\).
\end{proof}

%---------------------------------------------------------------------------------------------------------------------------
\subsection{Suite équirépartie, critère de Weyl}
%---------------------------------------------------------------------------------------------------------------------------

\begin{definition}
	Soit \( u\) une suite dans \( \mathopen[ 0 , 1 \mathclose]\). Pour \( 0\leq a\leq b\leq 1\) nous posons
	\begin{equation}
		X_n(a,b)=\Card\big\{  k\in\{ 1,\ldots, n \}\tq u_k\in\mathopen[ a , b \mathclose] \big\}.
	\end{equation}
	Nous disons que la suite \( u\) est \defe{équirépartie}{suite!équirépartie} si pour tout \( 0\leq a<b<1\), on a
	\begin{equation}
		\lim_{n\to \infty} \frac{ X_n(a,b) }{ n }=b-a.
	\end{equation}
\end{definition}
Voir aussi la remarque~\ref{RemUXAkcuH} sur les nombres normaux.

\begin{proposition}[Critère de Weyl\cite{ytMOpe,KXjFWKA}]  \label{PropDMvPDc}
	Soit \( (x_n)\) une suite dans \( \mathopen[ 0 , 1 [\). Les conditions suivantes sont équivalentes.
	\begin{enumerate}
		\item   \label{ItemKWcZTHqi}
		      La suite \( (x_n)\) est équirépartie.
		      \item\label{ItemKWcZTHqii}
		      Pour toute fonction continue à valeurs réelles sur \( \mathopen[ 0 , 1 \mathclose]\),
		      \begin{equation}    \label{EqBSqdjpn}
			      \lim_{n\to \infty} \frac{1}{ n }\sum_{k=1}^nf(x_k)=\int_0^1f(x)dx.
		      \end{equation}
		      \item\label{ItemKWcZTHqiii}
		      Pour tout \( p\in\eN\setminus\{ 0 \}\) nous avons
		      \begin{equation}
			      \lim_{n\to \infty} \frac{1}{ n }\sum_{k=1}^n e^{2i\pi px_k}=0.
		      \end{equation}
	\end{enumerate}
\end{proposition}
\index{convergence!suite numérique}
\index{intégrale!calcul}
\index{densité!dans un espace de fonction!critère de Weyl}
\index{suite!équirépartie!critère de Weyl}

\begin{proof}
	On pose
	\begin{equation}
		S_n(f)=\frac{1}{ n }\sum_{k=1}^nf(x_k).
	\end{equation}

	\begin{subproof}
		\item[Une espèce de lemme]

		Supposons connaitre un ensemble de fonctions \( A\) dense dans \( C^0(\mathopen[ 0 , 1 \mathclose])\) pour toutes les fonctions desquelles nous avons la limite \eqref{EqBSqdjpn}. Alors la limite a lieu pour toute fonction de \( C^0(\mathopen[ 0 , 1 \mathclose])\). En effet, soit \( f\in C^0(\mathopen[ 0 , 1 \mathclose])\) et \( g\in A\) tel que \( \| f-g \|_{\infty}<\epsilon\). Alors
		\begin{subequations}
			\begin{align}
				\left\|   \frac{1}{ n }\sum_{k=1}^nf(x_k)-\int_0^1f(t)dt  \right\| & \leq \left\| \frac{1}{ n }\sum_{k=1}^n\big( f(x_k)-g(x_k)\big) \right\|  \\
				                                                                   & \quad+ \left\| \frac{1}{ n }\sum_{k=1}^n  g(x_k)-\int_0^1g(t)dt \right\| \\
				                                                                   & \quad+ \left\| \int_0^1g(t)dt-\int_0^1f(t)dt \right\|.
			\end{align}
		\end{subequations}
		Le premier terme est majoré par \( \epsilon\). Le troisième a la même majoration : \( \int_0^1\big(  f(t)-g(t)\big)dt\leq \| f-g \|_{\infty}=\epsilon\). Par hypothèse sur l'espace \( A\), le second terme se majore par \( \epsilon\) lorsque \( n\) est grand.

		\item[\ref{ItemKWcZTHqi}\( \Rightarrow\)\ref{ItemKWcZTHqii}]
		Nous supposons que la suite est équirépartie et nous commençons par montrer le résultat pour les fonctions en escalier. Soit donc la fonction en escalier \( \eta(x)=c_j\) sur \( a_{j-1}< x<a_j\). Sur le point \( a_j\) lui-même, la fonction \( \eta\) vaut soit \( c_j\) soit \( c_{j+1}\). Nous avons
		\begin{equation}    \label{EqohMuel}
			\frac{1}{ n }\sum_{k=1}^n\eta(x_k)=\frac{1}{ n }\left[  \sum_{j=1}^n c_jX_n(a_j,a_{j+1})-\sum_{j=1}^n c_jX_n(a_j,a_j)+\sum_{j=1}^n \eta(a_j)X_n(a_j,a_j) \right].
		\end{equation}
		À la limite \( n\to\infty\), les deux derniers termes tombent\quext{J'en profite pour mentionner que mon équation \eqref{EqohMuel} n'est pas la même que celle de \cite{ytMOpe} dans laquelle il me semble voir une faute; quoi qu'il en soit, les termes litigieux tombent.} et il reste
		\begin{equation}
			\lim_{n\to \infty} \frac{1}{ n }\sum_{k=1}^n\eta(x_k)=\sum_{j=1}^n c_j(a_{j-1}-a_j).
		\end{equation}
		Or par construction, pour une fonction en escalier,
		\begin{equation}
			\sum_{j=1}^n c_j(a_{j-1}-a_j)=\int_0^1\eta.
		\end{equation}

		Étant donné que les fonctions en escalier sont denses dans les fonctions continues, l'espèce de lemme plus haut conclut.

		\item[\ref{ItemKWcZTHqii}\( \Rightarrow\)\ref{ItemKWcZTHqi}]
		Nous prouvons maintenant le sens inverse. C'est-à-dire que pour toute fonction continue sur \( \mathopen[ 0 , 1 \mathclose]\), nous avons
		\begin{equation}
			\int_0^1f(x)dx=\lim_{n\to \infty} \frac{1}{ n }\sum_{k=1}^nf(x_k).
		\end{equation}
		Nous devons en déduire que \( (x_n)\) est équirépartie. Pour ce faire, soit \( x\in \mathopen[ 0 , 1 [\) et \( \epsilon>0\) tel que \( x+\epsilon<1\). Nous considérons \( \varphi=\mtu_{\mathopen[ x , 1 [}\) et
		\begin{equation}
			\varphi_{\epsilon(t)}=\begin{cases}
				0                        & \text{si } t\in\mathopen[ 0 , x [           \\
				\frac{ t-x }{ \epsilon } & \text{si } t\in \mathopen[ x , x+\epsilon [ \\
				1                        & \text{si } t\geq x+\epsilon.
			\end{cases}
		\end{equation}
		C'est une fonction continue, donc
		\begin{equation}
			\lim_{n\to \infty} S_n\big( \varphi_{\epsilon}(t) \big)=\int_0^1\varphi_{\epsilon}(t)dt=\int_{x}^{x+\epsilon}\frac{ t-x }{ \epsilon }dt+\int_{x+\epsilon}^11dt=1-x-\frac{ \epsilon }{2}.
		\end{equation}
		Mais \( \varphi_{\epsilon}\leq \varphi\), donc \( S_n(\varphi_{\epsilon})\leq S_n(\varphi)\) et donc
		\begin{equation}
			\liminf_{n\to \infty}S_n(\varphi)\geq 1-x.
		\end{equation}
		Notons que nous ne savons pas si la \emph{vraie} limite de gauche existe; c'est pourquoi nous prenons la limite inférieure, qui existe toujours.

		Nous définissons aussi
		\begin{equation}
			\psi_{\epsilon}(t)=\begin{cases}
				0                                 & \text{si } t\in \mathopen[ 0 , x-\epsilon [ \\
				\frac{ t-x+\epsilon }{ \epsilon } & \text{si } t\in\mathopen[ x-\epsilon , x [  \\
				1                                 & \text{si } t>x.
			\end{cases}
		\end{equation}
		C'est encore une fonction continue et nous trouvons\footnote{Je recommande chaudement de dessiner les fonctions \( \varphi_{\epsilon}\) et \( \psi_{\epsilon}\) pour avoir une idée de la situation.}
		\begin{equation}
			\int_0^1\psi_{\epsilon}(t)dt=1-x+\frac{ \epsilon }{2}.
		\end{equation}
		Puisque \( \psi_{\epsilon}\geq\varphi\), nous avons \( S_n(\psi_{\epsilon})\geq S_n(\varphi)\) et donc
		\begin{equation}
			\limsup_{n}S_n(\varphi)\leq 1-x.
		\end{equation}
		Nous avons déjà obtenu que
		\begin{equation}
			1-x\leq\liminf S_n(\varphi)\leq \limsup S_n(\varphi)\leq 1-x,
		\end{equation}
		donc la limite existe et vaut
		\begin{equation}
			\lim_{n\to \infty} S_n(\varphi)=1-x.
		\end{equation}
		Le résultat est maintenant démontré dans le cas très particulier de la fonction caractéristique \( \varphi=\mtu_{\mathopen[ x , 1 [}\).

		Si nous prenons une fonction caractéristique \( \mtu_{\mathopen[ a , b \mathclose]}\), nous avons le même genre de preuve parce que \( \mtu_{\mathopen[ a , b [}\) est une combinaisons linéaire de fonctions du type \( \mtu_{\mathopen[ x , 1 [}\).

		Nous avons donc
		\begin{equation}
			\lim_{n\to \infty} S_n\big( \mtu_{\mathopen[ a , b \mathclose]} \big)=b-a,
		\end{equation}
		alors que le membre de gauche n'est autre que
		\begin{equation}
			S_n\big( \mtu_{\mathopen[ a , b \mathclose]} \big)=\frac{1}{ n }\sum_{k=1}^n\mtu_{\mathopen[ a , b \mathclose]}(x_k)=\frac{1}{ n }N(n,a,b).
		\end{equation}

		\item[\ref{ItemKWcZTHqii}\( \Rightarrow\)\ref{ItemKWcZTHqiii}]
		Vu que\footnote{Lemme \ref{LEMooHOYZooKQTsXW}.} \( e^{2i\pi px_k}=\cos(2\pi px_k)+i\sin(2\pi px_k)\). est une fonction périodique, c'est immédiat.
		\item[\ref{ItemKWcZTHqiii}\( \Rightarrow\)\ref{ItemKWcZTHqii}]
		Par linéarité, le point~\ref{ItemKWcZTHqii} montre que si \( f\) est un polynôme trigonométrique, alors
		\begin{equation}
			\lim_{n\to \infty} \frac{1}{ n }\sum_{k=1}^nf(x_k)=\int_0^1f(t)dt.
		\end{equation}
		\item[Densité des polynômes trigonométriques]
		Il nous reste à prouver que les polynômes trigonométriques sont denses dans les fonctions continues sur \( \mathopen[ 0 , 1 \mathclose]\). Soit une fonction continue sur \( \mathopen[ 0 , 1 \mathclose]\) avec \( f(0)=f(1)\). Alors le théorème de Stone-Weierstrass dans sa version trigonométrique (lemme~\ref{LemXGYaRlC}) nous donne la densité.

		Si \( f(1)\neq f(0)\) c'est pas très grave : on peut trouver une fonction \( g\) vérifiant \( g(0)=g(1) \) et \( \| f-g \|_{\infty}\leq \epsilon\). Ensuite un polynôme trigonométrique approxime très bien \( g\).
	\end{subproof}
\end{proof}

%+++++++++++++++++++++++++++++++++++++++++++++++++++++++++++++++++++++++++++++++++++++++++++++++++++++++++++++++++++++++++++
\section{Fonctions de Dirichlet}
%+++++++++++++++++++++++++++++++++++++++++++++++++++++++++++++++++++++++++++++++++++++++++++++++++++++++++++++++++++++++++++

\begin{definition}
	Une fonction \( f\colon \eR\to \eC\) est une \defe{fonction de Dirichlet}{fonction!de Dirichlet} si
	\begin{enumerate}
		\item
		      elle est \( 2\pi\)-périodique,
		\item
		      elle est continue par morceaux,
		\item
		      pour tout \( x\in \eR\) nous avons
		      \begin{equation}
			      f(x)=\frac{ f(x^+)+f(x^-) }{2}.
		      \end{equation}
	\end{enumerate}
	Nous notons \( \mD\) l'ensemble des fonctions de Dirichlet.
\end{definition}

\begin{lemma}[\cite{NJsYInp}]   \label{LemVIwMsTC}
	L'ensemble \( C^0(S^1)\) est dense dans l'ensemble des fonctions de Dirichlet \( \big( \mD,\| . \|_2 \big)\).
\end{lemma}

\begin{proof}
	Nous commençons par supposer que \( f\in\mD\) n'a qu'un seul point de discontinuité, \( x_0\). Alors nous considérons la fonction
	\begin{equation}
		f_n(x)=\begin{cases}
			f(x) & \text{si } x\in S^1\setminus B(x_0,\frac{1}{ n }) \\
			d(x) & \text{si }x\in B(x_0,\frac{1}{ n })
		\end{cases}
	\end{equation}
	où \( d\) est le droite joignant \( f(x_0-\frac{1}{ n })\) et \( f(x_0+\frac{1}{ n })\). La fonction \( f_n\) est continue et vérifie
	\begin{equation}
		| f_n(x) |\leq \| f \|_{\infty}
	\end{equation}
	pour tout \( x\). En effet si \( x\) est en dehors de \( B(x_0,\frac{1}{ n })\) c'est évident, et si \( x\in B(x_0,\frac{1}{ n })\), alors \( | f_n(x) |\) est majoré soit par \( f(x_0-\frac{1}{ n })\) soit par \( f(x_0+\frac{1}{ n })\) suivant que \( d\) soit croissant ou décroissant. Avec ça nous avons
	\begin{equation}
		\| f_n-f \|_2^2=\int_{x_0-1/n}^{x_0+1/n}| f_n(x)-f(x) |^2dx\leq \int_{x_0-1/n}^{x_0+1/n}4\| f \|_{\infty}=\frac{ 8\| f \|_{\infty} }{ n }.
	\end{equation}
	Et nous voyons que \( \| f_n-f \|_2\to 0\).

	Si \( f\) contient plusieurs points de discontinuité, on fait le même coup autour de chaque point, en prenant \( n\) assez grand pour que si \( x_0\) est un point de discontinuité, \( B(x_0,\frac{1}{ n })\) n'en contienne pas d'autres.
\end{proof}

Notons que la densité de \( C^0(S^1)\) dans \( \big( \mD,\| . \|_{\infty} \big)\) est impossible, parce qu'une limite uniforme de fonctions continues est continue.

\begin{theorem}
	Le système trigonométrique \( \{ e_n \}_{n\in \eZ}\) est total\footnote{Définition \ref{DEFooQVPHooJaSWyF}.} dans \( \big( \mD,\| . \|_2 \big)\).
\end{theorem}

\begin{proof}
	Soit \( f\in\mD\). Si elle est continue, le théorème de Fejèr~\ref{ThoJFqczow} nous donne convergence uniforme sur \( S^1\) d'une suite de polynômes trigonométriques vers \( f\). Cette convergence est également une convergence \( L^2\) parce que \( S^1\) est compact.

	Prenons donc \( f\in \mD\) non continue et \( \epsilon>0\)\footnote{Par exemple \( \epsilon=0.4\), mais ce n'est qu'un exemple hein. Si vous en voulez un autre, prenez \( p\), un nombre premier puis calculez \( \epsilon=1/p\).}. Par le lemme~\ref{LemVIwMsTC}, il existe une fonction \( g\in C^0(S^1)\) telle que
	\begin{equation}
		\| g-f \|_2\leq \epsilon.
	\end{equation}
	Le théorème de Fejèr donne aussi un polynôme trigonométrique \( P\) tel que \( \| P-g \|_2<\epsilon\); nous avons alors
	\begin{equation}
		\| P-f \|_2\leq \| P-g \|_{2}+\| g-f \|_2\leq 2\epsilon.
	\end{equation}
\end{proof}

Notons que cette histoire de fonctions de Dirichlet n'a pas attaqué le vrai fond du problème de la densité des polynômes trigonométriques dans \(  L^2(S^1)\) parce que nous restons avec une hypothèse de continuité, alors que les représentants des éléments de \( L^2(S^1)\) n'ont strictement aucune régularité à priori.

%+++++++++++++++++++++++++++++++++++++++++++++++++++++++++++++++++++++++++++++++++++++++++++++++++++++++++++++++++++++++++++
\section{Coefficients et série de Fourier}
%+++++++++++++++++++++++++++++++++++++++++++++++++++++++++++++++++++++++++++++++++++++++++++++++++++++++++++++++++++++++++++

\begin{definition}      \label{DEFooJUUIooNMdCtN}
	Pour toutes les fonctions \( f\) définie sur \( \mathopen[ 0 , 2\pi \mathclose[\) ou périodique de période \( 2\pi\), pour lesquelles les expressions ont un sens, nous définissons
	\begin{equation}    \label{EqNDBaXRL}
		c_n(f)=\frac{1}{ 2\pi }\int_0^{2\pi}f(t) e^{-int}dt,
	\end{equation}
	et nous nommons \defe{série de Fourier}{série!de Fourier} associée à \( f\) la série
	\begin{equation}
		S(f)(x)=\sum_{k=-\infty}^{\infty}c_k(f) e^{ikx}.
	\end{equation}
	Nous considérons aussi la suite (nous ne précisons pas dans quel espace)
	\begin{equation}
		S_n(f)(x)=\sum_{k=-n}^nc_k(f) e^{ikx}.
	\end{equation}
	Si la fonction \( f\) est de période \( T\), nous définissons
	\begin{equation}        \label{EQooBOFSooFCJXzu}
		c_n(f)=\frac{1}{ T }\int_0^Tf(t) e^{-2 i \pi n t/T}dt.
	\end{equation}
\end{definition}

Le sport de la théorie des séries de Fourier est de donner des conditions sous lesquelles :
\begin{itemize}
	\item les coefficients de Fourier et la série de Fourier ont un sens,
	\item la série de \( f\) est égale à \( f\).
\end{itemize}

\begin{proposition}[\cite{DupFourEsdgKEI}]  \label{PropmrLfGt}
	Soit \( f\) une fonction continue et \( 2\pi\)-périodique telle que sa série de Fourier converge uniformément. Alors la convergence est vers \( f\).
\end{proposition}

\begin{proof}
	Notons d'abord que \( f\) étant continue sur \(\mathopen[ 0 , 2\pi \mathclose]\), elle y est bornée et \( L^2\). Par conséquent Parseval nous enseigne que
	\begin{equation}
		\| S_N(f)-f \|_{L^2}\to 0.
	\end{equation}
	Cela signifie que
	\begin{equation}
		\lim_{N\to \infty} \frac{1}{ 2\pi }\int_{0}^{2\pi}| f(t)-S_N(t) |^2dt=0.
	\end{equation}
	L'hypothèse de convergence uniforme nous dit que la fonction \( | f(t)-S_N(t) |^2\) converge uniformément vers la fonction \( | f(t)-S(t) |^2\) où nous avons écrit \( S\) la limite de \( S_N\). En permutant la limite et l'intégrale,
	\begin{equation}
		\frac{1}{ 2\pi }\int_0^{2\pi}| f(t)-S(t) |^2dt=0,
	\end{equation}
	ce qui signifie que la fonction \( t\mapsto | f(t)-S(t) |^2\) est la fonction nulle. Nous en déduisons que \( f=S\).
\end{proof}

\begin{proposition}     \label{PropSgvPab}
	Soit \( f\) une fonction \( 2\pi\)-périodique. Si \( \sum_{n\in \eZ}| c_n(f) |<\infty\), alors pour tout \( x\in \eR\) nous avons
	\begin{equation}
		f(x)=\sum_{n\in \eZ}c_n(f) e^{inx}.
	\end{equation}
	De plus, la suite \( (S_n(f))\) converge uniformément vers \( f\).
\end{proposition}

\begin{proof}
	Nous posons
	\begin{equation}
		g(x)=\sum_{n\in \eZ}c_n(f) e^{inx}.
	\end{equation}
	Étant donné les hypothèses, la série de droite converge absolument, la fonction \( g\) est continue sur \( \eR\). Nous avons
	\begin{equation}
		\big| g(x)-(S_n(f))(x) \big|\leq \sum_{| k |> n}| c_k(f) |,
	\end{equation}
	mais le terme de droite tend vers zéro lorsque \( n\to \infty\) parce que c'est le reste d'une série convergente. Cela signifie que \( S_n(f)\) converge uniformément vers \( g\).

	Par ailleurs nous savons que dans \( L^2\) nous avons la convergence \( S_n(f)\to f\) (parce que \( f\) est continue sur le compact \( \mathopen[ 0 , 2\pi \mathclose]\) et donc y est bornée et \( L^2\)), ce qui signifie que \( g=f\) presque partout. Ces deux fonctions étant continues, elles sont égales partout.
\end{proof}

\begin{theorem}[\cite{MonCerveau}]     \label{ThozHXraQ}
	Soit \( f\colon \eR\to \eC\), une fonction \( C^1\) et \( 2\pi\)-périodique. Pour \( n\in \eZ\) nous posons
	\begin{equation}
		c_n(f)=\frac{1}{ 2\pi }\int_0^{2\pi}f(t) e^{-int}dt.
	\end{equation}
	Alors
	\begin{enumerate}
		\item       \label{ITEMooIDVEooJdMEmU}
		      Les coefficients de Fourier sont sommables : \( (c_n)\in \ell^1(\eZ)\)
		\item       \label{ITEMooGIEUooKLyXej}
		      Pour tout \( x\in \eR\) nous avons
		      \begin{equation}
			      f(x)=\sum_{n\in \eZ}c_n(f) e^{inx}.
		      \end{equation}
		\item       \label{ITEMooAUCTooTgJEPv}
		      La convergence est uniforme. C'est à dire que si nous posons
		      \begin{equation}
			      S_N(x)=\sum_{k=-N}^Nc_k(f) e^{ikx},
		      \end{equation}
		      alors
		      \begin{equation}
			      \| S_N-f \|_{\infty}\to 0.
		      \end{equation}
	\end{enumerate}
\end{theorem}

\begin{proof}
	Point par point.
	\begin{subproof}
		\item[Pour \ref{ITEMooIDVEooJdMEmU}]
		Soit \( n\in \eZ\). Nous posons \( g(t)=f(t) e^{-int}\). Nous avons
		\begin{equation}
			0=g(2\pi)-g(0)=\int_0^{2\pi}g'(t)dt=\int_0^{2\pi}\big[ f'(t) e^{-int}-inf(t) e^{-int} \big]dt.
		\end{equation}
		Du coup, \( c_n(f')=inc_n(f)\). La fonction \( f'\) étant bornée (parce que continue sur \( \mathopen[ 0 , 2\pi \mathclose]\)), elle est de carré intégrable sur \( \mathopen[ 0 , 2\pi \mathclose]\) et par les inégalités de Parseval (théorème~\ref{ThoyAjoqP}) nous avons
		\begin{equation}
			\sum_{n\in \eZ}| c_n(f') |^2<\infty.
		\end{equation}
		Par conséquent \( (c_n(f'))\in \ell^2(\eZ)\) et a fortiori \( (c_n(f'))_{n\in \eN}\in \ell^2(\eN)\). L'inégalité de Cauchy-Schwarz nous indique alors
		\begin{equation}
			\sum_{n=1}^{\infty}| c_n(f) |=\sum_{n\in \eN}\frac{1}{ n }| c_n(f') |\leq \left( \sum_n\frac{1}{ n^2 } \right)^{1/2}\left( \sum_{n}| c_n(f') |^2 \right)^{1/2}<\infty.
		\end{equation}
		Nous procédons de même pour \( n<0\). Cela prouve que
		\begin{equation}
			\sum_{n\in \eZ}| c_n(f) | = |c_0(f)|+\sum_{n<0}| c_n(f) |+\sum_{n>0} | c_n(f) |  <\infty.
		\end{equation}
		\item[Pour \ref{ITEMooGIEUooKLyXej}]
		\item[Pour \ref{ITEMooAUCTooTgJEPv}]
	\end{subproof}
\end{proof}

\begin{corollary}[Unicité des coefficients de Fourier\cite{MonCerveau}]   \label{CordgtXlC}
	Soient \( f,g\) deux fonctions continues et \( 2\pi\)-périodiques.
	\begin{enumerate}
		\item       \label{ITEMooPLTIooSDykYF}
		      Si \( c_n(f)=c_n(g)\) alors \( f=g\).
		\item       \label{ITEMooQMMSooEpIFbt}
		      Si \( f(x)=\sum_{n\in \eZ}a_n e^{inx}\), alors \( a_n=c_n(f)\).
	\end{enumerate}
\end{corollary}

\begin{proof}
	En deux points.
	\begin{subproof}
		\item[Pour \ref{ITEMooPLTIooSDykYF}]
		Dans le cas de fonctions continues, le théorème de Fejér \ref{ThoJFqczow} nous enseigne que si nous posons
		\begin{equation}
			S_n(f)(x)=\sum_{k=-n}^{n}c_k(f) e^{ikx}
		\end{equation}
		alors nous avons la convergence
		\begin{equation}
			\frac{1}{ N+1 }\sum_{n=0}^NS_n(f)(x)\to f(x).
		\end{equation}
		Donc en supposant que \( c_k(f)=c_k(g)\), nous avons \( S_n(f)(x)=S_n(g)(x)\) et
		\begin{equation}
			f(x)=\lim_{N\to \infty} \frac{1}{ N+1 }\sum_{n=0}^NS_n(f)(x)=\lim_{N\to \infty} \frac{1}{ N+1 }\sum_{n=0}^NS_n(g)(x)=g(x).
		\end{equation}
		\item[Pour \ref{ITEMooQMMSooEpIFbt}]
		Nous considérons la restriction
		\begin{equation}
			\begin{aligned}
				\tilde f\colon \mathopen[ 0 , 2\pi \mathclose] & \to \eC       \\
				x                                              & \mapsto f(x).
			\end{aligned}
		\end{equation}
		C'est une fonction bornée parce qu'elle est la restriction de \( f\) qui est continue sur, disons, le compact \( \mathopen[ -\delta , 2\pi+\delta \mathclose]\). Elle est donc dans l'espace de Hilbert \( L^2\big( \mathopen[ 0 , 2\pi \mathclose[ \big)\).

		En utilisant la base trigonométrique \eqref{EQooKMYOooLZCNap} (qui est une base par le lemme \ref{LEMooBJDQooLVPczR}), nous écrivons l'hypothèse sous la forme
		\begin{equation}
			\tilde f(x)=\sum_{n\in \eZ}\sqrt{ 2\pi }a_ne_n(x).
		\end{equation}
		Autrement dit, \( \tilde f=\sum_{n\in \eZ}\sqrt{ 2\pi }a_ne_n\). La proposition \ref{PROPooWTOZooYZdlml} permet d'identifier les coefficients :
		\begin{equation}
			\sqrt{ 2\pi }a_n=\langle \tilde f, e_n\rangle .
		\end{equation}
		Nous avons donc
		\begin{subequations}
			\begin{align}
				a_n & =\frac{1}{ \sqrt{ 2\pi } }\int_0^{2\pi}\tilde f(t)\overline{ e_n(t) }dt                \\
				    & =\frac{1}{ \sqrt{ 2\pi } }\int_0^{2\pi}\tilde f(t)\frac{1}{ \sqrt{ 2\pi } } e^{-int}dt \\
				    & =\frac{1}{ 2\pi }\int_0^{2\pi}\tilde f(t) e^{-int}dt                                   \\
				    & =\frac{1}{ 2\pi }\int_0^{2\pi}f(t) e^{-int}dt                                          \\
				    & =c_n(f),
			\end{align}
		\end{subequations}
		ce qu'il fallait démontrer.
	\end{subproof}
\end{proof}

\begin{normaltext}
	La proposition \ref{PropREkHdol} dit que les hypothèses de continuité et de périodicité ne sont pas suffisantes pour assurer la convergence de la série de Fourier. En particulier, pour \ref{CordgtXlC}\ref{ITEMooQMMSooEpIFbt}, l'hypothèse de la convergence de la série est une vraie hypothèse.
\end{normaltext}

\begin{example}
	Considérons la fonction
	\begin{equation}
		f(x)=1-\frac{ x^2 }{ \pi^2 }
	\end{equation}
	sur \( \mathopen[ -\pi , \pi \mathclose]\). Nous la développons en série trigonométrique, et étant paire il n'y a pas de sinus. Un calcul montre que
	\begin{equation}
		a_0=\frac{ 4 }{ 3 }
	\end{equation}
	et
	\begin{equation}
		a_n=(-1)^{n+1}\frac{ 4 }{ n^2\pi^2 },
	\end{equation}
	de telle sorte que
	\begin{equation}
		f(x)=\frac{ 2 }{ 3 }-\frac{ 4 }{ \pi^2 }\sum_{n=1}^{\infty}(-1)^n\frac{ \cos(nx) }{ n^2 }.
	\end{equation}
	Nous avons \( f(\pi)=0\), mais avec le développement,
	\begin{equation}
		f(\pi)=\frac{ 2 }{ 3 }-\frac{ 4 }{ \pi^2 }\sum_{n=1}^{\infty}\frac{1}{ n^2 },
	\end{equation}
	donc
	\begin{equation}
		\sum_{n=1}^{\infty}\frac{1}{ n^2 }=\frac{ \pi^2 }{ 6 }.
	\end{equation}
\end{example}

%---------------------------------------------------------------------------------------------------------------------------
\subsection{Le contre-exemple que nous attendions tous}
%---------------------------------------------------------------------------------------------------------------------------

Nous montrons maintenant que la continuité et la périodicité ne sont pas suffisantes pour avoir convergence de la série de Fourier.

\begin{proposition}[\cite{KXjFWKA}] \label{PropREkHdol}
	Soit \( C^0_{2\pi}(\eR)\) l'ensemble des fonctions périodiques continues muni de la norme uniforme. Nous définissons
	\begin{equation}
		S_n(f)(x)=\sum_{k=-n}^nc_k(f) e^{ikx}.
	\end{equation}
	Alors il existe \( f\in C^0_{2\pi}\) tel que la suite \(n\mapsto S_n(f)(0)\) soit divergente. En particulier \( f\) n'est pas la somme de sa série de Fourier.
\end{proposition}

\begin{proof}
	Nous considérons la forme linéaire
	\begin{equation}
		\begin{aligned}
			l_n\colon C^0_{2\pi} & \to \eC                                \\
			f                    & \mapsto S_n(f)(0)=\sum_{k=-n}^nc_k(f).
		\end{aligned}
	\end{equation}
	\begin{subproof}
		\item[La forme est continue]
		Nous montrons d'abord que \( \| l_n \|\) est continue en montrant que \( \| l_n \|<\infty\) et en utilisant la proposition~\ref{PROPooQZYVooYJVlBd}. Pour cela nous calculons un peu :
		\begin{equation}    \label{EqBELHGya}
			l_n(f)=\sum_{k=-n}^n\frac{1}{ 2\pi }\int_{-\pi}^{\pi}f(t) e^{-ikt}dt=\frac{1}{ 2\pi }\int_{-\pi}^{\pi}f(t)\sum_{k=-n}^n e^{-ikt}dt=\frac{1}{ 2\pi }\int_{-\pi}^{\pi}f(t)D_n(t)dt
		\end{equation}
		où \( D_n(t)\) est le noyau de Dirichlet dont nous connaissons une formule par le lemme~\ref{LemHPoIkwu}. Nous avons donc
		\begin{equation}
			| l_n(f) |\leq \frac{1}{ 2\pi }\int_{-\pi}^{\pi}| D_n(t) |\| f \|_{\infty}dt.
		\end{equation}
		En prenant \( \| f \|_{\infty}=1\) nous avons la borne suivante pour la norme de \( l_n\) :
		\begin{equation}        \label{EqBXoIUiD}
			\| l_n \|\leq \frac{1}{ 2n }\int_{-\pi}^{\pi}| D_n(t) |dt<\infty.
		\end{equation}
		Notons que la convergence de l'intégrale vient de la continuité de la fonction
		\begin{equation}
			t\mapsto \frac{ \sin\left( \frac{ 2n+1 }{2}t \right) }{ \sin\left( \frac{ t }{ 2 } \right) }
		\end{equation}
		qui, elle même, se prouve avec une règle de l'Hospital :
		\begin{equation}
			\lim_{t\to 0} \frac{ \sin(at) }{ \sin(t) }=\lim_{t\to 0} \frac{ a\cos(at) }{ \cos(t) }=a.
		\end{equation}
		Donc \( D_n(t)\) a une limite bien définie pour \( t\to 0\) et est alors une fonction continue sur le compact \( \mathopen[ -\pi , \pi \mathclose]\).

		\item[La norme de \( l_n\) (début)]

		Nous avons prouvé que \( \| l_n \|\leq \frac{1}{ 2\pi }\int_{-\pi}^{\pi}| D_n(t) |dt\). Nous allons à présent prouver que ceci est effectivement la norme de \( l_n\). Pour \( \epsilon>0\) nous considérons la fonction
		\begin{equation}
			\begin{aligned}
				f_{\epsilon}\colon \eR & \to \eC                                         \\
				x                      & \mapsto \frac{ D_n(x) }{ | D_n(x) |+\epsilon }.
			\end{aligned}
		\end{equation}
		C'est une fonction continue et \( 2\pi\)-périodique satisfaisant \( \| f_{\epsilon} \|\leq 1\) parce que le dénominateur est toujours plus grand que le numérateur. Nous nous proposons de calculer
		\begin{equation}
			l_n(f_{\epsilon})=\sum_{k=-n}^n\frac{1}{ 2\pi }\int_{-\pi}^{\pi}f_{\epsilon}(t) e^{-ikt}dt.
		\end{equation}
		Puisque \( f_{\epsilon}(t) e^{-ikt}\) vaut en norme \( | f_{\epsilon}(t) |\), qui est une fonction intégrable (ne dépendant pas de \( k\)) sur \( \mathopen[ -\pi , \pi \mathclose]\), le théorème de la convergence dominée~\ref{ThoConvDomLebVdhsTf} nous permet de permuter la somme et l'intégrale :
		\begin{equation}
			l_n(f_{\epsilon})=\frac{1}{ 2\pi }\int_{-\pi}^{\pi}\frac{ D_n(t) }{ | D_n(t) |+\epsilon }\underbrace{\sum_{k=-n}^n e^{-ikt}}_{=D_n(t)}dt=\frac{1}{ 2\pi }\int_{-\pi}^{\pi}\frac{ \big| D_n(t) \big|^2 }{ | D_n(t) |+\epsilon }dt.
		\end{equation}
		Nous avons donc
		\begin{equation}
			\lim_{\epsilon\to 0}l_n(f_{\epsilon})=\frac{1}{ 2\pi }\int_{-\pi}^{\pi}| D_n(t) |dt.
		\end{equation}
		Mais vue l'inégalité \eqref{EqBXoIUiD} nous avons
		\begin{equation}
			\| l_n \|=\frac{1}{ 2\pi }\int_{-\pi}^{\pi}| D_n(t) |dt.
		\end{equation}
		Notre tâche est maintenant de donner une valeur à cette intégrale.

		\item[Norme de \( l_n\) tend vers \( \infty\)]
		D'abord nous écrivons
		\begin{equation}
			\| l_n \|=\frac{1}{ 2\pi }\int_{-\pi}^{\pi}\frac{ \left| \sin\left( \frac{ 2n+1 }{2}t \right) \right|  }{ \big| \sin(t/2) \big| }dt,
		\end{equation}
		ensuite nous nous souvenons que \( | \sin(x) |\leq | x |\) pour tout \( x\), ce qui nous permet de changer le dénominateur :
		\begin{equation}
			\| l_n \|\geq \frac{ 2 }{ \pi }\int_0^{\pi}\frac{ \left| \sin\left( \frac{ 2n+1 }{2}t \right) \right|  }{ | t | }dt
		\end{equation}
		Nous y effectuons le changement de variable \( u=\frac{ 2n+1 }{2}t\) qui donne
		\begin{equation}
			\| l_n \|\geq \frac{ 2 }{ \pi }\int_{0}^{(n+\frac{ 1 }{2})\pi}\frac{ \big| \sin(u) \big| }{ | u | }du.
		\end{equation}
		Nous y reconnaissons l'intégrale \eqref{EqKNOmLEd} du sinus cardinal que nous savons diverger. Cela donne
		\begin{equation}
			\lim_{n\to \infty} \| l_n \|=\infty.
		\end{equation}
		\item[La conclusion]

		L'espace \( \big( C^0_{2\pi},\| . \|_{\infty} \big)\) est complet\footnote{Parce qu'une limite uniforme de fonctions continues est continue, théorème~\ref{ThoUnigCvCont}.}, donc le théorème de Banach-Steinhaus~\ref{ThoPFBMHBN} s'applique. Par rapport aux notations de l'énoncé de Banch-Steinhaus, nous posons
		\begin{subequations}
			\begin{align}
				E & =\big( C^0_{2\pi},\| . \|_{\infty} \big) \\
				F & =\eR                                     \\
				H & =\{ l_n \}_{n\in \eN}.
			\end{align}
		\end{subequations}
		Comme la suite \( (\| l_n \|)\) n'est pas bornée, il existe \( f\in C^0_{2\pi}\) tel que
		\begin{equation}
			\sup_n\| l_n(f) \|=\infty.
		\end{equation}
		Pour cette fonction nous avons
		\begin{equation}
			\sup_{n\geq 0}S_n(f)(0)=\infty,
		\end{equation}
		et donc la série de Fourier de \( f\) ne converge pas en zéro.

	\end{subproof}
\end{proof}

\begin{normaltext}      \label{NORMooGKKWooFmOBeE}
	La proposition \ref{PropREkHdol} ne contredit pas Fejèr \ref{ThoJFqczow}\ref{ItemUNQSPmyiv}. Alors là c'est subtil, donc soyez bien \randomGender{attentif}{attentive}.

	Le système trigonométrique est total dans l'espace des fonctions continues périodiques sur \( \eR\) : \( \big(  C^0_{2\pi}(\eR),\| . \|_{\infty}\big)\). Cela signifie que si \( f\in C^0_{2\pi}(\eR)\), il  existe une suite de polynômes trigonométriques \( P_k\) tels que \( P_k\stackrel{unif}{\longrightarrow}f\).

	Cela ne signifie pas que cette suite soit la suite des sommes partielles de la série de Fourier. Et en effet, le théorème de Fejèr ne donne pas la convergence de la suite des sommes partielles de Fourier, mais la convergence au sens de Cesàro de la somme des \( c_k(f)e_k\). Ce n'est pas la même chose.

	Notez que le coefficient de \( e_1\) dans \( F_2\) est \( c_1(f)/2\) alors que dans \( F_3\), il est \( 2c_1(f)/3\).

	Il y a donc bien une suite de polynômes trigonométriques qui converge vers \( f\), mais ce n'est pas la suite des sommes partielles de la série de Fourier.

	De plus, \( C^0_{2\pi}(\eR)\) n'est pas contenu dans \( L^2(\eR)\), donc nous ne pouvons pas invoquer la théorie de Hilbert pour dire que le système trigonométrique serait quelque chose comme une base. Si \( f\in C^0_{2\pi}(\eR)\), il n'est pas garanti qu'il existe des nombres \( (a_i)_{i\in \eZ}\) tels que \( f(x)=\sum_{n\in \eZ}a_n e^{inx}\).

	Enfin, me diriez-vous, les fonctions continues et périodiques sur \( \eR\) sont les mêmes que les fonctions définies sur \( \mathopen[ 0 , 2\pi \mathclose[\) avec une petite condition de \( \lim_{x\to 2\pi} f(x)=f(0)\). Les fonctions qui vérifient cela sont une partie de \( L^2\big( \mathopen[ 0 , 2\pi \mathclose[ \big)\), qui est un espace de Hilbert, lui. Or le système trigonométrique est une base hilbertienne (lemme \ref{LEMooBJDQooLVPczR}). Alors oui, la série de Fourier de \( f\) converge vers \( f\) lorsque \( f\) est continue sur \( \mathopen[ 0 , 2\pi \mathclose[\). Cela n'est cependant pas un contre-argument pour deux raisons :
	\begin{itemize}
		\item
		      La convergence \( S_n(f)\to f\) qu'on a dans \( L^2\big( \mathopen[ 0 , 2\pi \mathclose[ \big)\) est seulement une convergence pour la norme \( L^2\), et non une convergence uniforme.
		\item
		      Une convergence \( L^2\) sur \( \mathopen[ 0 , 2\pi \mathclose[\) ne se prolonge pas spécialement en une convergence \( L^2\) sur \( \eR\), et encore moins en une convergence uniforme sur \( \eR\).
	\end{itemize}
\end{normaltext}

%---------------------------------------------------------------------------------------------------------------------------
\subsection{Inégalité isopérimétrique}
%---------------------------------------------------------------------------------------------------------------------------

Le théorème suivant dit que parmi les courbes \( C^1\), le cercle a la plus grande surface possible à périmètre donné.
\begin{theorem}[Inégalité isopérimétrique\cite{KXjFWKA}]    \label{ThoIXyctPo}
	Soit \( f\colon S^1\to \eC \) une courbe de Jordan\footnote{Définition \ref{DEFooQZMSooYYkGDv}} de classe \( C^1\). Nous notons \( L\) sa longueur et \( S\) l'aire contenue de la surface délimitée\footnote{C'est la partie connexe bornée de \( \eC\setminus\gamma\) dont l'existence est donnée par le théorème de Jordan~\ref{ThoHSPWBuh}.} par \( f\). Alors
	\begin{enumerate}
		\item
		      Nous avons l'\defe{inégalité isopérimétrique}{inégalité!isopérimétrique} : \( L^2\geq 4\pi S\).
		\item
		      Nous avons l'égalité \( L^2=4\pi S\) si et seulement si la courbe donnée par \( f\) est un cercle.
	\end{enumerate}
\end{theorem}
\index{base!hilbertienne!utilisation}
\index{inégalité!isopérimétrique}
\index{géométrique!avec des nombres complexes}
\index{courbe!étude métrique}
\index{série!de Fourier!utilisation}
\index{Fourier!série!utilisation}

\begin{proof}
	Nous commençons par considérer un chemin dont la longueur est \( 2\pi\) et nous en considérons son paramétrage normal. Nous allons exprimer l'aire \( S\) en utilisant le théorème de Green, et plus particulièrement la formule de surface \eqref{EqAJGrtOk}.

	Si \( f(s)=x(s)+iy(s)\), nous devons intégrer \( y'x-x'y\), qui n'est rien d'autre que la partie imaginaire de \( f'(s)\overline{ f(s) }\). Donc
	\begin{equation}    \label{EqCSWKbPX}
		S=\frac{ 1 }{2}\imag\int_0^{2\pi}f'(s)\overline{ f(s) }ds
	\end{equation}
	Nous considérons les coefficients de Fourier de \( f\) donnés par la formule \eqref{EqNDBaXRL} :
	\begin{equation}
		c_n(f)=\frac{1}{ 2\pi }\int_0^{2\pi}f(s) e^{-ins}ds.
	\end{equation}
	Ceux de \( f'\) (qui est aussi continue sur le compact \( S^1\) et donc tout autant \( L^2\)) sont donnés par
	\begin{equation}
		c_n(f')=inc_n(f).
	\end{equation}

	D'autre part en vertu du théorème~\ref{ThoLongueurIntegrale}, la longueur de \( \gamma\) s'exprime en termes de l'intégrale de la norme de sa dérivée :
	\begin{equation}
		2\pi=L=\int_0^{2\pi}| f'(s) |ds=\int_0^{2\pi}| f'(s) |^2ds
	\end{equation}
	parce que nous avons choisi un paramétrage normal qui vérifie automatiquement \( | f'(s) |=1\) pour tout \( s\). L'identité de Parseval sous sa forme \eqref{EqMIuCSfz} appliquée à \( f'\) nous enseigne que
	\begin{equation}        \label{EqXSpHuZI}
		L=2\pi=\int_0^{2\pi}| f'(s) |^2ds=2\pi\sum_{n=-\infty}^{\infty}| c_n(f') |^2=2\pi\sum_{n\in \eZ}n^2| c_n(f) |^2,
	\end{equation}
	et donc que
	\begin{equation}        \label{EQooAXAWooIgSDmu}
		\sum_{n\in ZZ}n^2| c_n(f) |^2=1.
	\end{equation}
	Par ailleurs le système trigonométrique étant une base hilbertienne, et les fonctions \( f\) et \( f'\) étant dans \( L^2\big( \mathopen[ 0 , 2\pi \mathclose] \big)\) (parce que continues sur un compact), elles sont égales à leurs séries de Fourier (au sens \( L^2\)), c'est-à-dire que nous avons l'égalité \eqref{EqXMMRpSN}. Nous avons alors
	\begin{subequations}
		\begin{align}
			\langle f', f\rangle_{L^2} & =\langle \sum_{n\in \eZ}c_n(f')e_n, \sum_{m\in \eZ}c_m(f)e_m\rangle                         \\
			                           & =\sum_m\sum_nc_n(f')\overline{ c_m(f) }\underbrace{\langle e_n, e_m\rangle }_{\delta_{n,m}} \\
			                           & =\sum_{n\in \eZ}c_n(f')\overline{ c_n(f) }                                                  \\
			                           & =\sum_nin| c_n(f) |^2
		\end{align}
	\end{subequations}
	où nous avons utilisé la continuité du produit scalaire pour sortir les sommes. Avec cela nous pouvons exprimer l'aire \eqref{EqCSWKbPX} en termes de coefficients de Fourier :
	\begin{equation}    \label{EqOZBMiat}
		S=\frac{ 1 }{2}\imag2\pi\langle f', f\rangle =\pi\sum_{n\in \eZ}n| c_n(f) |^2.
	\end{equation}
	En utilisant les expressions \eqref{EqXSpHuZI} et \eqref{EqOZBMiat} pour \( L\) et \( S\), et en écrivant \( L=2\pi \), nous avons
	\begin{subequations}
		\begin{align}
			L^2-4\pi S & =4\pi^2\left( \sum_{n\in \eZ}n^2| c_n(f) |^2 \right)^2-4\pi^2\sum_{n\in \eZ}n| c_n(f) |^2       \label{SUBEQooJTEWooSQpQFC} \\
			           & =4\pi^2\sum_{n\in \eZ}n| c_n(f) |^2-4\pi^2\sum_{n\in \eZ}n| c_n(f) |^2      \label{SUBEQooBYENooTtoxGt}                     \\
			           & =4\pi^2\sum_{n\in \eZ}| c_n(f) |(n^2-n)                                                                                     \\
			           & \geq 0.
		\end{align}
	\end{subequations}
	Justifications.
	\begin{itemize}
		\item Pour \eqref{SUBEQooJTEWooSQpQFC}. Expression \eqref{EqXSpHuZI} pour \( L\) et \eqref{EqOZBMiat} pour \( S\).
		\item Pour \eqref{SUBEQooBYENooTtoxGt}. La somme dans le premier terme valant \( 1\) par \eqref{EQooAXAWooIgSDmu}, nous pouvons supprimer le carré.
	\end{itemize}
	Cela prouve l'inégalité demandée dans le cas où \( L=2\pi\).

	Si \( \gamma\) n'est pas de longueur \( 2\pi\) mais \( L\), alors nous considérons le chemin \( \sigma(t)=\frac{ 2\pi\gamma(t) }{ L }\). Sa longueur est \( 2\pi\) et son aire, au vu de la formule de Green \eqref{EqCSWKbPX}, est de \( 4\pi^2\frac{ S }{ L^2 }\). L'inégalité isopérimétrique appliquée au chemin \( \sigma\) donne alors \( L^2\geq 4\pi S\).

	Le cas d'égalité s'obtient uniquement si \( c_n=0\) pour tout \( n\) différent de \( 0\) ou \( 1\). Dans ce cas nous avons
	\begin{equation}
		f(s)=c_0(f)+c_1(f) e^{is},
	\end{equation}
	qui est un cercle de centre \( c_0(f)\) et de rayon \( | c_1(f) |\).
\end{proof}

%---------------------------------------------------------------------------------------------------------------------------
\subsection{À propos des coefficients}
%---------------------------------------------------------------------------------------------------------------------------

Pour la suite, nous avons besoin d'une notation pour désigner l'ensemble des suites dans \( \eC\) à index dans \( \eZ\), c'est à dire l'ensemble \( \Fun(\eZ,\eC)\). Pour alléger les notations, nous allons l'écrire \( \eC^{\eZ}\), conformément à des notations déjà introduites par exemple en \ref{DEFooLCJEooBvVmkV}.

Nous considérons l'application
\begin{equation}
	\begin{aligned}
		c\colon \big( L^1_{2\pi},\| . \|_1 \big) & \to \big( \eC^{\eZ},\| .\|_{\infty} \big) \\
		f                                        & \mapsto (c_n(f))_{n\in \eZ}
	\end{aligned}
\end{equation}
qui à une fonction \( 2\pi\)-périodique fait correspondre la suite (bornée) de ses coefficients de Fourier. Nous rappelons la définition
\begin{equation}
	c_n(f)=\frac{1}{ 2\pi }\int_0^{2\pi}f(t) e^{-int} dt.
\end{equation}
Nous allons montrer que cette application est linéaire, continue, injective et non surjective. Pour la continuité, par la linéarité il suffit de la montrer en \( 0\). Nous devons donc montrer que si nous avons une suite de fonctions \( f_k\) qui tend vers \( 0\) au sens \( L^1\), alors \( c(f_k)\to 0\) au sens de la norme \( \| . \|_{\infty}\) sur l'ensemble des suites.

Si nous posons \( r_k=\int_0^{2\pi}| f_k(t) |dt\), alors \( r_k=\| f_k \|_1\) et nous avons \( r_k\to 0\). Mais par définition
\begin{equation}
	| c_n(f_k) |\leq r_k,
\end{equation}
et donc \( \| c(f_k) \|_{\infty}\leq r_k\). L'application \( c\) est donc continue. L'injectivité est donnée par le corolaire~\ref{CordgtXlC}.

Si nous supposons que l'application \( c\) est continue, alors le théorème d'isomorphisme de Banach (\ref{ThofQShsw}) nous dit que cela devrait être un homéomorphisme, c'est-à-dire que \( c^{-1}\) serait également continue. Nous allons montrer qu'il n'en est rien.

Nous considérons la suite de suite
\begin{equation}    \label{EqdMtbOB}
	(c_n)_k=\begin{cases}
		0 & \text{si } n<0 \\
		1 & \text{si } k<n \\
		0 & \text{sinon}.
	\end{cases}
\end{equation}
Ici \( (c_n)_k\) est le terme numéro \( k\) de la suite \( (c_n)\). Par exemple \( c_0=(0,0,\ldots )\) et \( c_2=(1,1,0,\ldots)\).

Par injectivité de l'application qui à une fonction fait correspondre la suite de ses coefficients de Fourier, l'unique fonction qui possède ces coefficients est
\begin{equation}
	f_n(t)=\sum_{k\in \eN}c_{n,k} e^{ikt}.
\end{equation}
En ce qui concerne la norme de \( f_n\), nous avons
\begin{equation}
	\| f_n \|_1=\frac{1}{ 2\pi }\int_0^{2\pi}\sum_{k\in \eN}(c_n)_k|  e^{ikt} |dt=\sum_{k\in \eN}(c_n)_k=n.
\end{equation}
Étant donné que \( \| f_n \|_1=n\), la suite \( (\| f_n \|_1)\) n'est pas bornée alors que la suite de suites \eqref{EqdMtbOB} est bornée dans l'ensemble des suites parce que \( \| c_n \|_{\infty}=1\).

\begin{lemma}       \label{LEMooPUJDooKRBTaU}
	Soit une fonction \( f\colon \eR\to \eC\) qui est \( T\)-périodique et de classe \( C^1\). Alors
	\begin{equation}
		c_n(f')=\frac{ 2\pi n }{ T }.
	\end{equation}
\end{lemma}

\begin{proof}
	Nous rappelons la définition \eqref{EQooBOFSooFCJXzu} des coefficients de Fourier :
	\begin{equation}
		c_n(f)=\frac{1}{ T }\int_0^Tf(t) e^{-2 i \pi n t/T}dt.
	\end{equation}
	Le coefficient pour \( f'\) ne pose pas de problème d'existence parce que \( f'\) est continue sur le compact \( \mathopen[ 0 , T \mathclose]\). Il vaut
	\begin{subequations}
		\begin{align}
			c_n(f') & =\frac{1}{ T }\int_0^Tf'(t) e^{-2 i \pi n t/T}dt                                                                                                                   \\
			        & =\frac{1}{ T }\left[ f(t) e^{-2i\pi nt/T} \right]_0^T-\frac{1}{ T }\int_0^Tf(t)\left( \frac{ -2i\pi n }{ T } \right) e^{-2i\pi nt/T}dt \label{SUBEQooXYOVooGmoXbZ} \\
			        & =\frac{ 2i\pi n }{ T }\frac{1}{ T }\int_0^Tf(t) e^{-2i\pi nt/T}dt \label{SUBEQooXSCEooIJXFxT}                                                                      \\
			        & =\frac{ 2i\pi n }{ T }c_n(f).
		\end{align}
	\end{subequations}
	Justifications.
	\begin{itemize}
		\item Pour \eqref{SUBEQooXYOVooGmoXbZ}. C'est une intégration par partie avec \( u'=f'\) et \( v= e^{-2i\pi nt/T}\).
		\item Pour \eqref{SUBEQooXSCEooIJXFxT}. Comme \( f(T)=f(0)\), et que \( t\mapsto e^{-2i\pi nt/T}\) est périodique de période \( T\), le terme au bord est nul : \( f(T) e^{-2i\pi n}-f(0) e^{i0}=0\).
	\end{itemize}
\end{proof}

\begin{lemma}[\cite{BIBooUBUAooHyhrlg}]     \label{LEMooYJQWooDVvSyj}
	Soit une fonction \( f\colon \eR\to \eC\) de classe \( C^2\) et \( T\)-périodique. Alors
	\begin{equation}
		| c_n(f) |\leq \left( \frac{ T }{ 2\pi } \right)^2 \frac{ \| f'' \|_{\infty} }{ n^2 }.
	\end{equation}
\end{lemma}

\begin{proof}
	En utilisant la définition \eqref{EQooBOFSooFCJXzu} des coefficients de Fourier,
	\begin{equation}
		| c_n(f) |\leq \frac{1}{ T }\int_0^T| f(t) |dt\leq \frac{ 1 }{ T }\| f \|_{\infty}\int_0^T1dt=\| f \|_{\infty}.
	\end{equation}
	En appliquant le lemme \ref{LEMooPUJDooKRBTaU} à \( f'\) nous avons
	\begin{equation}
		c_n(f'')=\left( \frac{ 2i\pi n }{ T } \right)^2c_n(f).
	\end{equation}
	Donc
	\begin{equation}
		| c_n(f) |=\left( \frac{ T }{ 2\pi n } \right)^2| c_n(f'') |\leq \left( \frac{ T }{ 2\pi n } \right)^2\| f'' \|_{\infty}.
	\end{equation}
\end{proof}

%+++++++++++++++++++++++++++++++++++++++++++++++++++++++++++++++++++++++++++++++++++++++++++++++++++++++++++++++++++++++++++
\section{Série de Laurent}
%+++++++++++++++++++++++++++++++++++++++++++++++++++++++++++++++++++++++++++++++++++++++++++++++++++++++++++++++++++++++++++

\begin{theorem}[Série de Laurent\cite{BIBooUBUAooHyhrlg}]       \label{THOooMKJOooVghZyG}
	Soient la couronne
	\begin{equation}
		C(r_1,r_2)=\{ z\in \eC\tq r_1<| z |<r_2 \}
	\end{equation}
	et une fonction holomorphe \( f\colon C(r_1,r_2)\to \eC\). Alors :
	\begin{enumerate}
		\item
		      Il existe une suite \( (a_n) \) dans \( \eC\) telle que
		      \begin{equation}
			      f(z)=\sum_{n\in \eZ}a_nz^n.
		      \end{equation}
		\item       \label{ITEMooUOPHooSJRGKs}
		      Cette suite est unique : si
		      \begin{equation}
			      f(z)=\sum_{n\in \eZ}a_nz^n=\sum_{n\in \eZ}b_nz^n,
		      \end{equation}
		      alors \( a_n=b_n\), pour tout \( n\).
		\item       \label{ITEMooDGGZooJkDSxC}
		      Si on pose, pour \( r\in \mathopen] r_1 , r_2 \mathclose[\),
		      \begin{equation}
			      \begin{aligned}
				      f_r\colon \eR & \to \eC                   \\
				      \theta        & \mapsto f(r e^{i\theta}),
			      \end{aligned}
		      \end{equation}
		      les valeurs \( a_n\) sont liés aux coefficients de Fourier de \( f_r\) par
		      \begin{equation}
			      a_n=\frac{ c_n(f_r) }{ r^n }.
		      \end{equation}
		\item       \label{ITEMooOYCPooZZAyKs}
		      Cette série converge uniformément sur tout compact contenu dans \( C(r_1,r_2)\).
		\item
		      Pour tout \( r_1<s<r_2\), les coefficients sont donnés par\footnote{Pour le dire clairement, ces \( a_n\) ne dépendent pas de \( s\), même si \( s\) entre dans le membre de droite.}
		      \begin{equation}
			      a_n=\frac{1}{ 2\pi i }\int_{C_s}\frac{ f(z) }{ z^{n+1} }dz
		      \end{equation}
		      où \( C_s\) est un cercle centré en \( 0\), et de rayon \( s\).
	\end{enumerate}
	La série ainsi définie est la \defe{série de Laurent}{série de Laurent} de la fonction \( f\).
\end{theorem}

\begin{proof}
	Pour \( r\in\mathopen] r_1 , r_2 \mathclose[\) nous posons
	\begin{equation}
		\begin{aligned}
			f_r\colon \eR & \to \eC                   \\
			\theta        & \mapsto f(r e^{i\theta}).
		\end{aligned}
	\end{equation}
	\begin{subproof}
		\item[Coefficients de Fourier]
		La fonction \( f_r\) est de classe \( C^1\) et périodique. Le théorème \ref{ThozHXraQ} sur les séries de Fourier nous indique que
		\begin{equation}      \label{EQooIHQRooZWqJKL}
			f_r(t)=\sum_{n\in \eZ}c_n(r) e^{int}
		\end{equation}
		avec\footnote{Oui, on devrait écrire \( c_n(f_r)\) pour suivre scrupuleusement les notation introduites plus haut. Mais comme toute la suite de la démonstration sera de voir le tout comme fonction de \( r\), je vous laisse juger.}
		\begin{equation}
			c_n(r)=\frac{1}{ 2\pi }\int_0^{2\pi}f_r(t) e^{-int}dt.
		\end{equation}
		La fonction \( c_n\colon \mathopen] r_1 , r_2 \mathclose[\to \eC\) est une fonction définie par une intégrale que nous voudrions dériver en \( r_0\).
			\item[Digression]
			Deux voies s'offrent à nous.
			\begin{itemize}
				\item Le plus immédiatement disponible est le théorème \ref{ThoMWpRKYp}, mais il demande de travailler avec la dérivée (réelle) de \( r\mapsto f(r e^{i\theta})\) et de se poser des questions quant à son lien avec la dérivée (complexe) de \( f\).
				\item Une façon plus indirecte est de considérer une extension
				      \begin{equation}
					      \begin{aligned}
						      h\colon B(r_0,\delta)_{\eC}\times \mathopen[ 0 , 2\pi \mathclose[ & \to \eC                                \\
						      z,\theta                                                          & \mapsto f(z e^{i\theta}) e^{-in\theta}
					      \end{aligned}
				      \end{equation}
				      où \( \delta\) est assez petit pour que le tout reste dans le domaine de \( f\). Alors nous pouvons utiliser le théorème \ref{ThopCLOVN} qui a l'avantage de ne pas devoir majorer la dérivée. Mais cette voie demande de réellement faire le lien entre la dérivée complexe de \( h\) et la dérivée réelle de \( r\mapsto f(z e^{i\theta})\).
			\end{itemize}
			Nous choisissons la première voie parce qu'en réalité, elle évite complètement de parler de dérivée complexe.

			\item[Permuter dérivée et intégrale]
			Nous allons essayer de la dériver en \( r_0\in \mathopen] r_1 , r_2 \mathclose[\) en utilisant le théorème \ref{ThoMWpRKYp}. Pour y voir plus clair, ce qui joue le rôle de \( f\) dans l'énoncé de \ref{ThoMWpRKYp} est l'application
			\begin{equation}
				\begin{aligned}
					h\colon \mathopen] r_1 , r_2 \mathclose[\times \mathopen[ 0 , 2\pi \mathclose[ & \to \eC                             \\
					(r,\theta)                                                                     & \mapsto  f_r(\theta) e^{-in\theta}.
				\end{aligned}
			\end{equation}
			Nous considérons un intervalle \( I=B(r_0,\delta)\) assez petit pour être dans \( \mathopen] r_1 , r_2 \mathclose[\). Passons en revue les conditions.
			\begin{subproof}
				\item[Pour \ref{ITEMooAFVMooAeCEco}]
				Pour tout \( r\in I\), la fonction \( \theta\mapsto f(r e^{i\theta}) e^{-in\theta}\) est dans \( L^1\big( \mathopen[ 0 , 2\pi \mathclose[ \big)\) pare que
				\begin{equation}
					| f(r e^{i\theta}) e^{-in\theta} |=| f(r e^{i\theta}) |=| \tilde f(r,\theta) |.
				\end{equation}
				La fonction \( \tilde f\) étant continue, elle est bornée sur le compact \( \overline{ I }\times \mathopen[ 0 , 2\pi \mathclose]\).
				\item[Pour \ref{ITEMooXIZXooGPYFyT}]
				Pour chaque \( \theta\), la fonction \( r\mapsto f(r e^{i\theta}) e^{-in\theta}\) est dérivable par la proposition \ref{PROPooAGGMooIVQFQB}\ref{ITEMooRTYYooSTgTAQ}.
				\item[Pour \ref{ITEMooDTTIooWkldfB}]
				En utilisant la proposition \ref{PROPooAGGMooIVQFQB}\ref{ITEMooUUXTooZoDMHI}, nous savons que la fonction
				\begin{equation}
					\frac{ \partial h }{ \partial r  }(r,\theta)=\frac{ \partial \tilde f }{ \partial r }(r,\theta) e^{-in\theta}
				\end{equation}
				est continue et donc bornée sur le compact \( \overline{ I }\times \mathopen[ 0 , 2\pi \mathclose]\). Une fonction constante majorant de \( \partial_rh\) est intégrable sur le compact \( \mathopen[ 0 , 2\pi \mathclose]\).
			\end{subproof}
			En permutant nous avons donc
			\begin{equation}
				c'_n(r_0)=\frac{1}{ 2\pi }\int_0^{2\pi}(\partial r\tilde f)(r_0,\theta) e^{-in\theta}d\theta.
			\end{equation}
			\item[Cauchy-Riemann]
			C'est le moment d'utiliser Cauchy-Riemann en coordonnées polaires sous la forme de la proposition \ref{PROPooAGGMooIVQFQB}\ref{ITEMooDHXTooBjxwjY}. Et tant que nous y sommes, nous notons \( g(\theta)=\tilde f(r_0,\theta)\) pour avoir moins de choses à écrire :
			\begin{subequations}
				\begin{align}
					c'_n(r_0) & =\int_0^{2\pi}(\partial_r\tilde f)(r_0,\theta) e^{-in\theta}d\theta                                        \\
					          & =\frac{1}{ 2\pi }\int_0^{2\pi}\frac{1}{ ir_0 }(\partial_{\theta}\tilde f)(r_0,\theta) e^{-in\theta}d\theta \\
					          & =\frac{1}{ 2\pi ir_0 }\int_{0}^{2\pi}g'(\theta) e^{-in\theta}d\theta.
				\end{align}
			\end{subequations}
			La dernière expression a manifestement envie de se soumettre à une intégration par partie.
			\item[Une intégration par partie]
			Nous posons \( u(\theta)= e^{-in\theta}\) et \( v=g\), de telle sorte que
			\begin{subequations}        \label{EQSooSRDJooXLHhgh}
				\begin{align}
					c'_n(r_0) & =\frac{1}{ 2\pi i r_0 }\left( \big[ e^{-in\theta}g(\theta)\big]_0^{2\pi}-\int_0^{2\pi}(-in) e^{-in\theta}g(\theta)d\theta \right) \\
					          & =\frac{n}{ 2\pi r_0 }\int_0^{2\pi}g(\theta) e^{-in\theta}d\theta.           \label{SUBEQooKVZMooUxcRXn}                           \\
					          & =\frac{n}{ r_0 }c_n(r_0).
				\end{align}
			\end{subequations}
			Justification pour \eqref{SUBEQooKVZMooUxcRXn}. Pour rappel, \( g(\theta)=\tilde f(r_0,\theta)=f(r_0 e^{i\theta})\); donc \( g(0)=g(2\pi)\) et le «terme au bord» est nul.
			\item[Équation différentielle]
			L'équation \eqref{EQSooSRDJooXLHhgh} dit que \( c_n\) satisfait à l'équation différentielle
			\begin{equation}
				c'_n(r)=\frac{ n }{ r }c_n(r)
			\end{equation}
			pour tout \( r\in \mathopen] r_1 , r_2 \mathclose[\). La fonction \( c_n\) est une fonction à valeurs complexes dont les parties réelles et imaginaires vérifient toutes deux l'équation du lemme \ref{LEMooCSAFooTYasYM}. Il existe donc \( a_n\in \eC\) tel que
			\begin{equation}
				c_n(r)=a_nr^n.
			\end{equation}
			Cela prouve au passage le point \ref{ITEMooDGGZooJkDSxC} parce que \( r\) n'est jamais nul.
			\item[La valeur de \( a_n\)]
			Nous avons
			\begin{equation}        \label{EQooFNUHooZbbNAT}
				\begin{aligned}
					a_n & =\frac{ c_n(r) }{ r^n }                                                               \\
					    & =\frac{1}{ 2\pi }\int_0^{2\pi}f(r e^{i\theta}) e^{-in\theta}r^{-n}d\theta             \\
					    & =\frac{1}{ 2\pi }  \int_0^{2\pi}\frac{ f(r e^{i\theta}) }{ ( e^{i\theta}r)^n }d\theta \\
					    & =\frac{1}{ 2\pi i }\int_{C_s}\frac{ f(z) }{ z^{n+1} }dz.
				\end{aligned}
			\end{equation}
			Une justification pour l'intégrale curviligne s'impose. La définition est \ref{DEFooBPLJooZwsmxi}. Dans le cas du cercle, nous considérons
			\begin{equation}
				\begin{aligned}
					C_r\colon \mathopen[ 0 , 2\pi \mathclose[ & \to \eC                \\
					\theta                                    & \mapsto r e^{i\theta},
				\end{aligned}
			\end{equation}
			et donc
			\begin{equation}
				\int_{C_r}\frac{ f(z) }{ z^{n+1} }dz=\int_0^{2\pi}\frac{ f(r e^{i\theta}) }{ (r e^{i\theta})^{n+1} }ri e^{i\theta}d\theta=i\int_0^{2\pi}\frac{ f(r e^{i\theta}) }{ (r e^{i\theta})^n }d\theta.
			\end{equation}
			Le fait que le tout soit égal à \( a_n\) prouve que l'intégrale est en réalité indépendante de \( r\)\quext{Voir ma question \ref{PROPBooYWDNooMXVPLJ}.}.
			\item[Conclusion]
			Reprenons la formule \eqref{EQooIHQRooZWqJKL} :
			\begin{equation}
				f(r e^{i\theta})=f_r(\theta)=\sum_nc_n(r) e^{in\theta}=\sum_na_nr^n e^{in\theta}=\sum_na_n(r e^{i\theta})^n.
			\end{equation}
			Autrement dit,
			\begin{equation}
				f(z)=\sum_na_nz^n,
			\end{equation}
			avec les \( a_n\) donnés par la formule \eqref{EQooFNUHooZbbNAT}, comme nous devions le prouver.
			\item[Point \ref{ITEMooUOPHooSJRGKs} (unicité)]
			Supposons que \( f(z)=\sum_{n\in \eZ}a_nz^n\), fixons \( r\in \mathopen] r_1 , r_2 \mathclose[\), et posons
			\begin{equation}
				\begin{aligned}
					f_r\colon \eR & \to \eC                   \\
					\theta        & \mapsto f(r e^{i\theta}).
				\end{aligned}
			\end{equation}
			C'est une fonction continue et périodique. Elle peut s'écrire sous la forme
			\begin{equation}
				f_r(\theta)=f(r e^{i\theta})=\sum_{n\in \eZ}a_n(r^n e^{in\theta})=\sum_{n\in \eZ}(a_nr^n) e^{in\theta}.
			\end{equation}
			Le corolaire \ref{CordgtXlC}\ref{ITEMooQMMSooEpIFbt} à propos de l'unicité des coefficients de Fourier implique que \( a_nr^n=c_n(f)\) et donc que
			\begin{equation}        \label{EQooBNSMooGLIBqU}
				a_n=\frac{ c_n(f) }{ r^n }.
			\end{equation}
			Donc les coefficients \( a_n\) sont déterminés par \( f\). Si nous avions fait le calcul en partant de \( f(z)=\sum_{n\in \eZ}b_nz^n\), nous aurions eu \( b_n=\frac{ c_n(f) }{ r^n }\), et donc bien \( a_n=b_n\).
			\item[Point \ref{ITEMooOYCPooZZAyKs}]
			La fonction
			\begin{equation}
				\begin{aligned}
					f_r\colon \eR & \to \eC                  \\
					\theta        & \mapsto f(r e^{i\theta})
				\end{aligned}
			\end{equation}
			est continue et périodique de période \( 2\pi\). Le lemme \ref{LEMooYJQWooDVvSyj} nous indique que, pour tout \( r\in \mathopen] r_1 , r_2 \mathclose[\), nous avons
		\begin{equation}        \label{EQooMDNNooPYFQrq}
			| c_n(f_r) |\leq \frac{ \| f_r'' \|_{\infty} }{ n^2 }.
		\end{equation}
		\begin{subproof}
			\item[Sur une couronne]
			Soient deux rayons intermédiaires \( r_1<s_1<s_2<r_2\). La couronne fermée \( \overline{ C(s_1,s_2) }\) est compacte. Nous considérons la fonction
			\begin{equation}
				\begin{aligned}
					g\colon \mathopen[ s_1 , s_2 \mathclose]\times \mathopen[ 0 , 2\pi \mathclose] & \to \eC               \\
					(u,\theta)                                                                     & \mapsto f''_u(\theta)
				\end{aligned}
			\end{equation}
			Nous notons \( g_u\) l'application \( \theta\mapsto g(u,\theta)\); c'est une application définie sur \( \mathopen[ s_1 , s_2 \mathclose]\). Par le lemme \ref{LEMooIVAKooUiEENr}, l'application \( u\mapsto \| g_u \|_{\infty}\) est continue et donc majorée : nous pouvons considérer \( M\in \eR\) tel que \( \| g_u \|_{\infty}<M\) pour tout \( u\in \mathopen[ s_1 , s_2 \mathclose]\).

			Puisque, pour \( r\in\mathopen[ s_1 , s_2 \mathclose]\) nous avons \( g_u=f_u''\), en combinant avec \eqref{EQooMDNNooPYFQrq}, nous voyons qu'il existe \( M\) tel que
			\begin{equation}        \label{EQooAIPEooKdfoXr}
				| c_n(f_r) |\leq \frac{ M }{ n^2 }.
			\end{equation}
			En prenant les notations de la définition \ref{DefVBrJUxo} de la convergence normale, nous posons
			\begin{equation}
				\begin{aligned}
					u_n\colon \overline{ C(s_1,s_2) } & \to \eC        \\
					z                                 & \mapsto a_nz^n
				\end{aligned}
			\end{equation}
			En ce qui concerne sa norme\footnote{Faites bien attention que dans cette partie, \( \| . \|_{\infty}\) est la norme uniforme sur \( \overline{ C(s_1,s_2) }\) et non sur \( C(r_1,r2)\) ou sur \( \eC\).}, nous avons
			\begin{subequations}
				\begin{align}
					\| u_n \|_{\infty} & =| a_n |\sup_{z\in\overline{ C(s_1,s_2) }}| z^n |                           \\
					                   & =| a_n |s_2^n                                                               \\
					                   & =\left| \frac{ c_n(f_s) }{ s^n } \right| s_2^n  \label{SUBEQooRKUXooFPxnLG} \\
					                   & \leq \frac{ | c_n(f_s) | }{ s_2^n } s_2^n                                   \\
					                   & \leq \frac{ M }{ n^2 }      \label{SUBEQooZJFJooAgRKHc}
				\end{align}
			\end{subequations}
			Justifications :
			\begin{itemize}
				\item Pour \eqref{SUBEQooRKUXooFPxnLG}. Prendre n'importe quel \( s\in \mathopen[ s_1 , s_2 \mathclose]\) et c'est bon par \eqref{EQooBNSMooGLIBqU}.
				\item Pour \eqref{SUBEQooZJFJooAgRKHc}. Équation \eqref{EQooAIPEooKdfoXr}.
			\end{itemize}
			Comme la somme \( \sum_{n\in \eZ}M/n^2\) converge, nous avons la convergence normale de \( \sum_{n\in \eZ}a_nz^n\) sur \( \overline{C(s_1,s_2)}\).

			\item[Sur un compact quelconque]
			Soit \( K\), un compact dans \( C(r_1,r_2)\). La fonction \( z\mapsto | z |\) est continue sur \( K\); donc elle a un minimum et un maximum. Nous posons \( s_1=\min_{z\in K}| z |\) et \( s_2=\max_{z\in K}| z |\).

			Ah ah ! non. En fait nous ne définissons pas \( s_1\) et \( s_2\) de cette manière parce qu'il y a un risque que \( s_1=s_2\) et qu'alors \( \overline{ C(s_1,s_2) }\) soit vide et ne contienne donc pas \( K\) -- pour rappel, \( C(s_1,s_2)\) est la couronne ouverte.

			Nous choisissons donc \( s_1\) et \( s_2\) de telle sorte que
			\begin{equation}
				r_1 <s_1 <\min_{z\in K}| z |\leq \max_{z\in K} <s_2 <r_2.
			\end{equation}
			Tout ça pour dire que \( K\subset \overline{ C(s_1,s_2) }\). La convergence normale sur \( \overline{ C(s_1,s_2) }\) déjà prouvée implique la convergence normale sur \( K\).
		\end{subproof}
	\end{subproof}
\end{proof}

\begin{probleme}        \label{PROPBooYWDNooMXVPLJ}
	Sur Wikipédia\cite{BIBooUBUAooHyhrlg}, le fait que \( a_n\) ne dépende pas de \( r\) est prouvé en disant que \( a_n(s)=c_n(s)/s^n\) et \( a_n(t)=c_n(t)/t^n\) et que
	\begin{equation}        \label{EQooCPDTooBDxIKm}
		\frac{ c_n(s) }{ c_n(t) }=\frac{ s^n }{ t^n }.
	\end{equation}
	En mettant tout cela bout à bout,
	\begin{equation}
		\frac{ a_n(s) }{ a_n(t) }=1.
	\end{equation}
	Je ne comprends pas très bien pourquoi cette justification est nécessaire. À mon avis, \randomGender{le rédacteur}{la rédactrice} de la démonstration sur Wikipédia parvient à déduire la relation \eqref{EQooCPDTooBDxIKm} directement depuis l'équation différentielle, sans réellement avoir besoin de la résoudre.

	Si vous comprenez n'hésitez pas à m'écrire, parce que j'ai l'impression d'avoir manqué quelque chose.
\end{probleme}

\begin{probleme}
	L'énoncé de la proposition \ref{PROPooBMZGooLoaGLK} n'est peut-être pas précis. Si vous avez un énoncé correct sous le coude, écrivez-moi.
\end{probleme}

\begin{proposition}     \label{PROPooBMZGooLoaGLK}
	Si \( f\) est holomorphe sur \( B(a,r)\), alors sa série de Laurent est de la forme
	\begin{equation}
		f(z)=\sum_{n=0}^{\infty}a_n(z-a)^n.
	\end{equation}
	avec
	\begin{equation}
		a_n=\frac{1}{ 2\pi i }\int_{\gamma}\frac{ f(z) }{ (z-a)^{n+1} }dz.
	\end{equation}
	où
	\begin{equation}
		\begin{aligned}
			\gamma\colon \mathopen[ 0 , 2\pi \mathclose[ & \to \eC            \\
			t                                            & \mapsto a+ re^{it}
		\end{aligned}
	\end{equation}
	est le cercle de centre \( a\) et de rayon \( r\).
\end{proposition}
