% This is part of Le Frido
% Copyright (c) 2010-2016,2018-2020, 2022
%   Laurent Claessens, Carlotta Donadello
% See the file fdl-1.3.txt for copying conditions.


%--------------------------------------------------------------------------------------------------------------------------- 
\subsection{Thérorème de Tietze (espace normal)}
%---------------------------------------------------------------------------------------------------------------------------

\begin{lemma}[\cite{BIBooQKARooMHqitK}]     \label{LEMooCLVAooTaNGJk}
    Soit un espace topologique normal\footnote{Définition \ref{DEFooNNKVooLtzImT}.} \( X\). Soient \( M\in \eR^+\) et \( A\) fermé dans \( X\), et une application continue \( f\colon A\to \mathopen[ -M , M \mathclose]\).

    Il existe une application continue \( g\colon X\to \mathopen[ -M/3 , M/3 \mathclose]\) telle que pour tout \( x\in A\),
            \begin{equation}
                | f(x)-g(x) |<\frac{ 2M }{ 3 },
            \end{equation}
\end{lemma}

\begin{proof}
    Nous divisons \( A\) en trois parties:
    \begin{subequations}
        \begin{align}
            A_-&=f\big( \mathopen[ -M , -M/3 \mathclose] \big),\\
            A_+&=f\big( \mathopen[ M/3 , 3 \mathclose] \big),\\
            A_0&=f\big( \mathopen[ -M/3 , M/3 \mathclose] \big).
        \end{align}
    \end{subequations}
    Les parties \( A_+\) et \( A_-\) sont fermées parce que \( f\) est continue. D'autre part, \( X\) est normal, de telle sorte que le théorème d'Urysohn \ref{THOooKYYEooLFcNpg} s'applique.

    Nous considérons donc une application continue \( g_1\colon X\to \mathopen[ 0 , 1 \mathclose]\) telle que \( g_1^{-1}(A_-)=\{ 0 \}\) et \( g_1^{-1}(A_+)=\{ 1 \}\). Nous posons alors
    \begin{equation}
        g(x)=\left( \frac{ 2M }{ 3 } \right)g_1(x)-\frac{ M }{ 3 }.
    \end{equation}
    Vérification des propriétés de \( g\).
    \begin{subproof}
        \spitem[\( g\) est continue]
        Parce que \( g_1\) est continue.
    \spitem[\( g\) prend ses valeurs dans \( \mathopen\lbrack -M/3 , M/3 \mathclose\rbrack\)]
        % -------------------------------------------------------------------------------------------- 
    Parce que \( g_1\) prend ses valeurs dans \( \mathopen[ 0 , 1 \mathclose]\) et que \( t\mapsto (2M/3)t-M/3\) est croissante.
    \spitem[\( | f(x)-g(x) |\) sur \( A\)]
    % -------------------------------------------------------------------------------------------- 
    Si \( x\in A_-\), alors \( g_1(x)=0\) et donc \( g(x)=-M/3\). Donc
    \begin{equation}        \label{EQooEUOKooFLRJjz}
        | f(x)-g(x) |=| f(x)-M/3 |\leq 2M/3
    \end{equation}
    parce que \( f(x)\in \mathopen[ -M/3 , M/3 \mathclose]\).

    Si \( x\in A_+\), alors \( g_1(x)=1\) et donc \( g(x)=M/3\). Même fin de raisonnement qu'en \eqref{EQooEUOKooFLRJjz}.

    Si \( x\in A_0\), alors \( f(x)\in \mathopen[ -M/3 , M/3 \mathclose]\) et \( g(x)\in\mathopen[ -M/3 , M/3 \mathclose]\) et donc encore \( | f(x)-g(x) |\leq 2M/3\).
    \end{subproof}
\end{proof}

\begin{lemma}[\cite{BIBooQKARooMHqitK}]     \label{LEMooSKSNooEdgFcR}
    Soient un espace topologique normal \( X\) ainsi qu'un fermé \( A\) dans \( X\). Nous considérons une fonction continue \( f\colon A\to \mathopen[ -M , M \mathclose]\).

    Il existe une application \( g\colon X\to \mathopen[ -M , M \mathclose]\) continue prolongeant \( f\).
\end{lemma}

\begin{proof}
    Nous allons commencer par construire une suite d'applications \( g_i\colon X\to \eR\) telles que
    \begin{enumerate}
        \item       \label{ITEMooGAIVooQYBCZj}
            Pour tout \( x\in A\) et pour tout \(n\in \eN\),
            \begin{equation}
                \big| f(x)-\sum_{i=1}^rng_i(x) \big|\leq \left( \frac{ 2 }{ 3 } \right)^NM
            \end{equation}
        \item       \label{ITEMooOLNAooEJPdbV}
            Pour tout \( x\in X\) et pour tout \( i\),
            \begin{equation}
                | g_i(x) |\leq \left( \frac{ 2 }{ 3 } \right)^i\frac{ M }{2}.
            \end{equation}
    \end{enumerate}
    Nous commençons par construire \( g_1\) à partir du lemme \ref{LEMooCLVAooTaNGJk} appliqué à la fonction \( f\). Nous avons donc une application \( g_1\colon X\to \mathopen[ -M/3 , M/3 \mathclose]\) telle que
    \begin{equation}        \label{EQooGXUIooWnATPw}
        | f(x)-g_1(x) |\leq \frac{ 2M }{ 3 }
    \end{equation}
    pour tout \( x\in A\). Vu que \( g_1\) prend ses valeurs dans \( \mathopen[ -M/3 , M/3 \mathclose]\), elle vérifie la condition \ref{ITEMooOLNAooEJPdbV}. De plus \eqref{EQooGXUIooWnATPw} montre que la condition \ref{ITEMooGAIVooQYBCZj} est vérifiée pour \( n=1\).

    Et c'est parti pour la récurrence. Nous supposons avoir des applications \( g_i\) pour \( i=1,\ldots, k\) qui vérifient la condition \ref{ITEMooOLNAooEJPdbV}, et telles que la condition \ref{ITEMooGAIVooQYBCZj} est satisfaite pour \(  n=1,\ldots, k\). Nous allons maintenant construire \( g_{k+1}\).

    Nous posons 
    \begin{equation}
        \begin{aligned}
            h_k\colon A&\to \eR \\
            x&\mapsto f(x)-\sum_{i=1}^kg_i(x). 
        \end{aligned}
    \end{equation}
    Par hypothèse de récurrence, la fonction \( h_k\) ne prend pas n'importe quelles valeurs dans \( \eR\), mais
    \begin{equation}
        h_k\colon A\to \mathopen\big[  -\left( \frac{ 2 }{ 3 } \right)^kM  , \left( \frac{ 2 }{ 3 } \right)^kM \mathclose\big].
    \end{equation}
    Nous construisons \( g_{k+1}\) à partir de ce \( h_k\) et du lemme \ref{LEMooCLVAooTaNGJk}. Nous avons donc
    \begin{equation}
        g_{k+1}\colon X\to \mathopen\Big[  -\frac{1}{ 3 }\left( \frac{ 2 }{ 3 } \right)^kM  , \frac{1}{ 3 }\left( \frac{ 2 }{ 3 } \right)^kM \mathclose\Big] 
    \end{equation}
    vérifiant
    \begin{equation}
        | h_k(x)-g_{k+1}(x) |\leq \left( \frac{ 2 }{ 3 } \right)^{k+1}M.
    \end{equation}
    La condition \ref{ITEMooGAIVooQYBCZj} est donc maintenant vérifiée jusqu'à \( n=k+1\). Nous vérifions la condition \ref{ITEMooOLNAooEJPdbV} pour \( i=k+1\). Simple calcul :
    \begin{equation}
        | g_{k+1}(x) |\leq \frac{1}{ 3 }\left( \frac{ 2 }{ 3 } \right)^kM=\left( \frac{ 2 }{ 3 } \right)^{k+1}\frac{ M }{ 2 }.
    \end{equation}
    Et voila pour la définition des applications \( g_i\).

    Vu que \( 2/3<1\), la série \( \sum_{i=1}^{\infty}\| g_i \|_{\infty}\) converge normalement (définition \ref{DefVBrJUxo}). Notons \( g\) la somme. Le lemme \ref{LEMooJZTBooIopLok} donne alors la convergence uniforme \( g_i\stackrel{unif}{\longrightarrow}g\), et le théorème \ref{ThoSerUnifCont} nous assure que \( g\) est continue sur \( X\).

    En ce qui concerne la norme de \( g\), nous avons, en utilisant la formule \eqref{EqRGkBhrX} avec \( q=2/3\),
    \begin{equation}
        \| g \|_{\infty}\leq \sum_{i=1}^{\infty}\left( \frac{ 2 }{ 3 } \right)^i\frac{ M }{2}\leq M.
    \end{equation}
    Donc \( | g(x) |\leq M\) pour tout \( x\in X\).

    Enfin nous vérifions que \( g\) prolonge \( f\). Soit \( x\in A\). Prenez la limite \( n\to \infty\) dans l'inégalité
    \begin{equation}
        | f(x)-\sum_{i=1}^ng_i(x) |\leq \left( \frac{ 2 }{ 3 } \right)^nM.
    \end{equation}
    Nous trouvons que \( | f(x)-g(x) |=0\).
\end{proof}

\begin{theorem}[Théorème de Tietze]     \label{THOooXKGWooFUYlux}
    Soit une partie fermée \( A\) de l'espace normal \( X\). Si la fonction \( f\colon A\to \eR\) est continue, alors elle se prolonge en une fonction continue \( g\colon X\to \eR\).
\end{theorem}

\begin{proof}
    Nous supposons dans un premier temps que \( f\) prenne ses valeurs dans \( \mathopen] -M , M \mathclose[\). À fortiori, elle prend ses valeurs dans \( \mathopen[ -M , M \mathclose]\) et le lemme \ref{LEMooSKSNooEdgFcR} dit qu'il existe un prolongement continu \( g\colon X\to \mathopen[ -M , M \mathclose]\). Nous allons construire à partir de là un prolongement continu \( h\colon X\to \mathopen] -M , M \mathclose[\) de \( f\).

    Nous posons
    \begin{equation}
        B=g^{-1}\big( \{ -M,M \} \big).
    \end{equation}
    Étant donné que \( f(A)\subset\mathopen] -M , M \mathclose[\) , nous avons \( B\cap A=\emptyset\). De plus \( A\) est fermé par hypothèse et \( B\) est fermé en tant qu'image réciproque du fermé \( \{ -M,M \}\) par l'application continue \( g\). Nous pouvons donc appliquer le théorème d'Urysohn \ref{THOooKYYEooLFcNpg}.

    Nous considérons donc une application \( g_1\colon X\to \mathopen[ 0 , 1 \mathclose]\) telle que \( g_1=0\) sur \( B\) et \( g_1=1\) sur \( A\). Enfin nous posons \( h=gg_1\). Cette application prend ses valeurs dans \( \mathopen] -M , M \mathclose[\), est continue et si \( x\in A\) nous avons
    \begin{equation}
        h(x)=g(x)g_1(x)=f(x)\time 1=f(x).
    \end{equation}
    Ceci règle la question si \( f\) prend ses valeurs dans \( \mathopen] -M , M \mathclose[\).

    Nous considérons à présent le cas général \( f\colon A\to \eR\). Soit un homéomorphisme \( \phi\colon \eR\to \mathopen] -M , M \mathclose[\) (par exemple via l'exemple \ref{EXooGKPNooZtmJen}). Nous considérons
    \begin{equation}
        \begin{aligned}
        \tilde f\colon A&\to \mathopen] -M , M \mathclose[ \\
            x&\mapsto (\phi\circ f)(x). 
        \end{aligned}
    \end{equation}
    Nous lui appliquons le premier cas pour avoir une fonction \( \tilde g\colon X\to \mathopen] -M , M \mathclose[\) qui prolonge \( \tilde f\). Il suffit maintenant de poser
    \begin{equation}
        \begin{aligned}
            g\colon X&\to \eR \\
            x&\mapsto (\phi^{-1}\circ\tilde g)(x). 
        \end{aligned}
    \end{equation}
    Cela est une application continue et si \( x\in A\), nous avons
    \begin{equation}
        g(x)=(\phi^{-1}\circ \tilde g)(x)=(\phi^{-1}\circ \tilde f)(x)=(\phi^{-1}\circ\phi\circ f)(x)=f(x).
    \end{equation}
\end{proof}

\begin{corollary}
    Soit un fermé \( A\) dans un espace normal \( X\). Si \( f\colon A\to \eC\) est continue, alors elle se prolonge en une fonction continue \( g\colon X\to \eC\).
\end{corollary}

\begin{proof}
    Les parties réelles et imaginaires de \( f\) sont continues. Il suffit de leur appliquer le théorème de Tietze \ref{THOooXKGWooFUYlux}.
\end{proof}


%--------------------------------------------------------------------------------------------------------------------------- 
\subsection{Homotopie entre applications}
%---------------------------------------------------------------------------------------------------------------------------


\begin{definition}[homotopie entre applications]       \label{DEFooPJKLooCvgxsu}
    Soient des espaces topologiques \( X\) et \( Y\). Deux applications continues \( f_1,f_2\colon X\to Y\) sont \defe{homotopes}{applications homotopes} si il existe une application continue \( H\colon \mathopen[ 0 , 1 \mathclose]\times X\to Y \) telle que pour tout \( x\in X \) nous avons
    \begin{subequations}
        \begin{align}
            H(0,x)&=f_1(x)\\
            H(1,x)&=f_2(x).
        \end{align}
    \end{subequations}
\end{definition}

\begin{lemma}       \label{LEMooMGFZooGOaGYl}
    La relation «être homotope à\footnote{Définition \ref{DEFooPJKLooCvgxsu}.}»  est une relation d'équivalence sur \( C(X,Y)\).
\end{lemma}

\begin{proof}
    Pour obtenir \( f\sim f\), il suffit de prendre \( H(s,x)=f(x)\).

    Si \( f\sim g\), nous considérons l'homotopie \( H\colon \mathopen[ 0 , 1 \mathclose]\times X\to Y\). Alors l'application
    \begin{equation}
        \begin{aligned}
            M\colon \mathopen[ 0 , 1 \mathclose]\times X&\to Y \\
            (t,x)&\mapsto H(1-t,x) 
        \end{aligned}
    \end{equation}
    est une homotopie pour \( g\sim f\).

    Si \( f\sim g\) et \( g\sim f\), nous avons les applications continues \( H\) et \( M\) telles que
    \begin{subequations}
        \begin{align}
        H(0,x)&=f(x)&&H(1,x)&=g(x)\\
        M(0,x)&=g(x)&&M(1,x)&=h(x).
        \end{align}
    \end{subequations}
    L'application
    \begin{equation}
        \begin{aligned}
            S\colon \mathopen[ 0 , 1 \mathclose]\times X&\to Y \\
            (t,x)&\mapsto \begin{cases}
                H(2t,x)    &   \text{si } t\in\mathopen[ 0 , \frac{ 1 }{2} \mathclose[\\
                M(2t-1,x)    &    \text{si }t\in\mathopen[ \frac{ 1 }{2} , 1 \mathclose].
            \end{cases}
        \end{aligned}
    \end{equation}
    Nous vérifions que \( S\) est continue en vérifiant la valeur en \( t=1/2\). De plus 
    \begin{equation}
        S(0,)=H(0,x)=f(x)
    \end{equation}
    et
    \begin{equation}
        S(1,x)=M(2-1,x)=M(1,x)=g(x).
    \end{equation}
\end{proof}


\begin{lemma}[\cite{MonCerveau}]   \label{LEMooMJKEooCaVhjD}
    Soient des espaces topologiques \( X\) et \( Y\). Si \( Y\) est connexe par arcs, alors toutes les applications constantes \( X\to Y\) sont homotopes.
\end{lemma}

\begin{proof}
    Soient les applications constantes \( u(x)=u_0\) et \( v(x)=v_0\). Étant donné que \( Y\) est connexe par arcs, il existe une application continue \( \gamma\colon \mathopen[ 0 , 1 \mathclose]\to Y\) telle que \( \gamma(0)=u_0\) et \( \gamma(1)=v_0\). Alors l'application
    \begin{equation}
        \begin{aligned}
            H\colon \mathopen[ 0 , 1 \mathclose]\times \eR^n&\to Y \\
            (t,x)&\mapsto \gamma(t) 
        \end{aligned}
    \end{equation}
    est une homotopie entre les applications \( u\) et \( v\).
\end{proof}

\begin{proposition}[\cite{BIBooQKARooMHqitK}]       \label{PROPooNABDooFtKukO}
    Soit un compact \( K\) de \( \eR^n\). Deux applications \( f,g\colon K\to \eC^*\) sont homotopes dans \( \eC^*\) si et seulement si \( f/g\) est homotope à la fonction 
    \begin{equation}
        \begin{aligned}
            u\colon K&\to \eC \\
            x&\mapsto 1. 
        \end{aligned}
    \end{equation}
\end{proposition}

\begin{proof}
    Supposons que \( H\colon \mathopen[ 0 , 1 \mathclose]\times K\to \eC^*\) est une homotopie entre \( f\) et \( g\). Dans ce cas, l'application
    \begin{equation}
        \begin{aligned}
            M\colon \mathopen[ 0 , 1 \mathclose]\times K&\to \eC^* \\
            (t,x)&\mapsto \frac{ H(t,x) }{ g(x) } 
        \end{aligned}
    \end{equation}
    est une homotopie entre \( f/g\) et \( u\). En effet
    \begin{equation}
        M(0,x)=\frac{ H(0,x) }{ g(x) }=\frac{ f(x) }{ g(x) }
    \end{equation}
    et
    \begin{equation}
        M(1,x)=\frac{ H(1,x) }{ g(x) }=1=u(x).
    \end{equation}

    Dans l'autre sens, si \( H\) est une homotopie entre \( f/g\) et \( u\), alors l'application \( M(t,x)=H(t,x)g(x)\) est une homotopie entre \( f\) et \( g\).
\end{proof}


%--------------------------------------------------------------------------------------------------------------------------- 
\subsection{Logarithme sur un chemin}
%---------------------------------------------------------------------------------------------------------------------------

\begin{definition}
    Soient un espace topologique \( X\), et une application \( f\colon X\to \eC^*\). Nous disons que \( g\colon X\to \eC^*\) est un \defe{logarithme}{logarithme d'une application} de \( f\) si pour tout \( x\in X\) nous avons
    \begin{equation}
        f(x)=\exp\big( g(x) \big).
    \end{equation}
\end{definition}

\begin{definition}[Détermination du logarithme]     \label{DEFooOCDGooGyvvWi}
    Soit un chemin\footnote{Définition \ref{DEFooQZMSooYYkGDv}.} \( \gamma\colon \mathopen[ a , b \mathclose]\to \eC^*\). Nous disons qu'une application \( g\colon \gamma\big( \mathopen[ a , b \mathclose] \big)\to \eC^*\) est une \defe{détermination du logarithme}{détermination du logarithme} sur \( \gamma\) si
    \begin{equation}
        \exp\big( (g\circ\gamma)(t) \big)=\gamma(t)
    \end{equation}
    pour tout \( t\in \mathopen[ a , b \mathclose]\).
\end{definition}

\begin{theorem}     \label{THOooUPANooMiECqe}
    Tout chemin dans \( \eC^*\) admet une détermination du logarithme\footnote{Définition \ref{DEFooOCDGooGyvvWi}.}, et si \( l\) est une détermination sur le chemin \( \gamma\), nous avons
    \begin{equation}
        \int_{\gamma}\frac{ dz }{ z }=l\big( \gamma(b) \big)-l\big( \gamma(a) \big).
    \end{equation}
\end{theorem}

\begin{theorem}[Théorème de Borsuk\cite{BIBooQKARooMHqitK}]     \label{THOooTCUMooEByCKg}
    Soient un compact \( K\) de \( \eR^n\) ainsi qu'une application continue \( f\colon K\to \eC^*\). Les propriétés suivantes sont équivalentes :
    \begin{enumerate}
        \item   \label{ITEMooKZYDooKoEEbl}
            \( f\) admet un logarithme continu.
        \item   \label{ITEMooXVNXooVAHklr}
            \( f\) est homotope à l'application constante \( u\colon \eC^*\to \eR\), \( u(z)=1\).
        \item   \label{ITEMooQDHXooObjxLA}
            \( f\) admet une extension continue \( \tilde f\colon \eR^n\to \eC^*\).
    \end{enumerate}
\end{theorem}

\begin{proof}
    En plusieurs points.
    \begin{subproof}
        \spitem[\ref{ITEMooKZYDooKoEEbl} \( \Rightarrow\) \ref{ITEMooQDHXooObjxLA}]
        % -------------------------------------------------------------------------------------------- 
        Soit un logarithme continue \( g\colon K\to \eC\) de \( f\). Vu que \( K\) est fermé, le théorème de Tietze \ref{THOooXKGWooFUYlux} dit que \( g\) possède une extension continue \( \tilde g\colon \eR^n\to \eC\). Nous posons
        \begin{equation}
            \begin{aligned}
                \tilde f\colon \eR^n&\to \eC^* \\
                x&\mapsto \exp\big( \tilde g(x) \big). 
            \end{aligned}
        \end{equation}
        L'application \( \tilde f\) est continue parce que \( \exp\) et \( \tilde g\) le sont. Elle est une extension de \( f\) parce que si \( x\in K\), nous avons
        \begin{equation}
            \tilde f(x)=\exp\big( \tilde g(x) \big)=\exp\big( g(x) \big)=f(x)
        \end{equation}
        parce que \( \tilde g(x)=g(x)\) et \( g\) est un inverse de \( \exp\).
        \spitem[\ref{ITEMooQDHXooObjxLA} \( \Rightarrow\) \ref{ITEMooXVNXooVAHklr}]
        % -------------------------------------------------------------------------------------------- 
        Soit une extension continue \( \tilde f\colon \eR^n\to \eC^*\) de \( f\). Soit \( x_0\in \eR^n\). Nous posons
        \begin{equation}
            \begin{aligned}
                H\colon \mathopen[ 0 , 1 \mathclose]\times K&\to \eC^* \\
                (t,x)&\mapsto \tilde f\big( (1-t)x+tx_0 \big). 
            \end{aligned}
        \end{equation}
        L'application \( H\) est continue, et pour \( x\in K\) elle vérifie \( H(0,x)=\tilde f(x)=f(x)\) ainsi que \( H(1,x)=\tilde f(x_0)\).

        Donc \( f\) est homotope à l'application constante \( \tilde f(x_0)\). Vu que \( \eC^*\) est connexe par arcs, toutes les applications constantes sont homotopes\footnote{Lemme \ref{LEMooMJKEooCaVhjD}.}. Donc \( f\) est homotope à \( \tilde f(x_0)\) qui est homotope à \( u\). L'homotopie étant une relation d'équivalence\footnote{Lemme \ref{LEMooMGFZooGOaGYl}.}, \( f\) est homotope à \( u\).
        \spitem[\ref{ITEMooXVNXooVAHklr} \( \Rightarrow\) \ref{ITEMooKZYDooKoEEbl}]
        % -------------------------------------------------------------------------------------------- 
        Nous considérons l'homotopie \( H\colon \mathopen[ 0 , 1 \mathclose]\times K\to \eC^*\) entre \( f\) et \( u\). En particulier pour tout \( x\in K\) nous avons \( H(0,x)=f(x)\) et \( H(1,x)=1\).

        La partie \( \mathopen[ 0 , 1 \mathclose]\times K\) est compacte\footnote{Théorème \ref{THOIYmxXuu}.} et \( | H |\) y est continue. Donc elle atteint ses bornes. Mais elle prend ses valeurs dans \( \eC^*\); donc
        \begin{equation}
            \inf_{x,t}| H(x,t) |>0.
        \end{equation}
        Et d'ailleurs cet infimum est un minimum. Nous considérons la norme suivante sur \( \eR\times \eR^n\) :
        \begin{equation}
            \| (t,x) \|_{\infty}=\max\big( | t |,| x | \big).
        \end{equation}
        Toutes les normes étant équivalentes\footnote{Équivalence de normes, théorème \ref{ThoNormesEquiv}.} sur \( \eR\times \eR^n\), nous pouvons caractériser la compacité, la continuité et tout ça en termes de cette norme. L'application \( H\) est uniformément continue\footnote{Théorème de Heine \ref{PROPooBWUFooYhMlDp}.}. Soit \( \epsilon>0\), il existe \( \eta>0\) tel que si \( \| (t,x)-(s,y) \|_{\infty}<\eta\) alors \( \| H(t,x)-H(s,y) \|<\epsilon\).

        Soient un entier \( n>\frac{1}{ n }+1\), \( x\in K\)  et \( k\in\{ 0,\ldots, n \}\). Nous avons
        \begin{equation}
            \big\| \big(\frac{ k }{ n },x\big)-\big(  \frac{ k+1 }{ n },x \big) \big\|_{\infty}=\big\|  (-\frac{1}{ n },0)  \big\|=\frac{1}{ n }<\eta.
        \end{equation}
        Nous avons donc également
        \begin{equation}
            \big| H(\frac{ k }{ n },x)-H(\frac{ k+1 }{ n },x) \big|<\epsilon.
        \end{equation}
        
        Nous posons
        \begin{equation}
            \begin{aligned}
                F_k\colon K&\to \eC^* \\
                x&\mapsto H\big( \frac{ k }{ n },x \big). 
            \end{aligned}
        \end{equation}
        Cette application est continue sur \( K\) et pour tout \( x\in K\) nous avons
        \begin{subequations}        \label{SUBEQSooLJIJooYkxyMO}
            \begin{align}
                \big| \frac{ F_k(x) }{ F_{k+1j}(x) }-1 \big|&=\big| \frac{ F_k(x)-F_{k+1}(x) }{ F_{k+1}(x) } \big|\\
                &=\big| \frac{ H(\frac{ k }{ n },x)-H(\frac{ k+1 }{ n },x) }{ H(\frac{ k+1 }{ n },x) } \big|\\
                &<\frac{ \epsilon }{ | H(\frac{ k+1 }{ n },x) | }.
            \end{align}
        \end{subequations}
        Pour rappel, pour tout \( \epsilon>0\), il existe un \( \eta>0\) tel qu'en posant\( n>\frac{1}{ n }+1\) nous avons \eqref{SUBEQSooLJIJooYkxyMO}. Nous choisissons \( \epsilon\) tel que \( 0<\epsilon<\inf_{(t,x)}| H(t,x) |\), de telle sorte à avoir
        \begin{equation}
            \big| \frac{ F_k(x) }{ F_{k+1j}(x) }-1 \big|<\frac{ \epsilon }{ | H(\frac{ k+1 }{ n },x) | }<1.
        \end{equation}
        
        La fonction \( \ln\colon \eC^*\to \eC^*\) est continue sur \( B(1,1)\)\footnote{Le logarithme est celui défini en \ref{DEFooWDYNooYIXVMC} et sa continuité est le théorème \ref{THOooWUXOooYKvLbJ}.}. Donc la fonction
        \begin{equation}
            \begin{aligned}
                h_k\colon K&\to \eC^* \\
            x&\mapsto \ln\left( \frac{ F_k(x) }{ F_{k+1}(x) } \right) 
            \end{aligned}
        \end{equation}
        est continue. Nous posons
        \begin{equation}
            g=\sum_{k=0}^{n-1}\ln\left( \frac{ F_k }{ F_{k+1} } \right),
        \end{equation}
        qui est également continue sur \( K\). Cette fonction \( g\) est un logarithme continu de \( f\) sur \( K\) parce que\footnote{Dans le calcul suivant, nous utilisons entre autres la formule \( \exp(a+b)=\exp(a)\exp(b)\) de la proposition \ref{PropdDjisy}\ref{ITEMooRLHCooJTuYKV}.}
        \begin{subequations}
            \begin{align}
                \exp\big( g(x) \big)&=\exp\left( \sum_{k=0}^{n-1} \ln\Big( \frac{ f_k(x) }{ F_{k+1}(x) } \Big) \right)\\
                &=\prod_{k=0}^{n-1}\exp\left( \ln\big( \frac{ F_k(x) }{ F_{k+1}(x) } \big) \right)\\
                &=\prod_{k=0}^{n-1}\frac{ F_k(x) }{ F_{k+1}(x) }\\
                &=\frac{ F_0(x) }{ F_n(x) }\\
                &=\frac{ H(0,x) }{ H(1,x) }\\
                &=\frac{ f(x) }{ 1 }\\
                &=f(x).
            \end{align}
        \end{subequations}
    \end{subproof}
\end{proof}

\begin{corollary}[\cite{BIBooQKARooMHqitK}]
    Soit un compact convexe \( K\) de \( \eR^n\). Toute application continue \( K\to \eC^*\) admet un logarithme continu.
\end{corollary}

\begin{proof}
    Soit une application continue \( f\colon K\to \eC^*\). Soit \( x_0\in K\). Vu que \( K\) est convexe, \( f\) est homotope à la fonction constante \( f(x_0)\). Par arc-connexité de \( \eC^*\), l'application \( f\) est alors homotope à la fonction constante \( 1\).

    Le théorème de Borsuk \ref{THOooTCUMooEByCKg} nous dit alors que \( f\) admet un logarithme continu.
\end{proof}

\begin{lemma}
    Si \( K\) est compact dans \( \eC\), alors la partie \( \eC\setminus K\) possède exactement une composante connexe non bornée.
\end{lemma}

\begin{proof}
    La partie \( K\) étant compacte, elle est bornée. Il existe donc \( r>0\) tel que \( K\subset B(0,r)\). Soit \( A=\eC\setminus B(0,r)\). Cela est une partie connexe de \( \eC\) et est donc contenue dans une composante connexe de \( \eC\setminus K\). Il y a donc existence d'une composante connexe non bornée de \( \eC\setminus K\).

    Pour l'unicité, les autres composantes connexes de \( \eC\setminus K\) sont contenues dans \( B(0,r)\) et sont donc bornées.
\end{proof}

%---------------------------------------------------------------------------------------------------------------------------
\subsection{Le théorème de Jordan}
%---------------------------------------------------------------------------------------------------------------------------

Les définitions de chemins, de lacets et de courbes de Jordan sont dans \ref{DEFooQZMSooYYkGDv}.

\begin{lemmaDef}[\cite{BIBooQKARooMHqitK}]     \label{LEMooZPRLooJPvrOE}
    Soit un lacet \( \gamma\colon \mathopen[ a , b \mathclose]\to X\). Nous considérons l'application
    \begin{equation}
        \begin{aligned}
            q\colon \mathopen[ a , b \mathclose[&\to S^1 \\
            t&\mapsto  e^{2i\pi t/(b-a)} 
        \end{aligned}
    \end{equation}
    Il existe une unique application continue \( \tilde \gamma\colon S^1\to \Image(\gamma)\) telle que \( \gamma=\tilde \gamma\circ q\).

    Cette application \( \tilde \gamma\) est l'application \defe{quotient}{quotient d'un chemin} associée à \( \gamma\).
\end{lemmaDef}

\begin{proof}
    La proposition \ref{PROPooZEFEooEKMOPT} dit que \( q\) est une bijection continue. Donc l'existence et l'unicité d'une application \( \tilde \gamma=\gamma\circ q^{-1}\). Cette application est continue comme composée d'applications continues.
\end{proof}

\begin{lemma}[\cite{BIBooQKARooMHqitK}]     \label{LEMooCGVOooVPlSRD}
    Soit \( \gamma\colon \mathopen[ a , b \mathclose]\to X\) un chemin d'image \( \Gamma\).
    \begin{enumerate}
        \item       \label{ITEMooWKVAooCQDvpL}
            Si \( \gamma\) est un chemin de Jordan\footnote{Définition \ref{DEFooQZMSooYYkGDv}.}, alors \( \gamma\colon \mathopen[ a , b \mathclose]\to \Gamma\) est un isomorphisme d'espaces topologiques.
        \item       \label{ITEMooVYMXooEtgPJT}
            Si \( \gamma\) est un lacet de Jordan, alors le quotient\footnote{Définition \ref{LEMooZPRLooJPvrOE}.} \( \tilde \gamma\colon S^1\to \Gamma\) est un homéomorphisme\footnote{Un homéomorphisme est un isomorphisme d'espaces vectoriels, c'est à dire continu, inversible et d'inverse continu.}.
    \end{enumerate}
\end{lemma}

\begin{proof}
    En deux parties.
    \begin{subproof}
        \spitem[Pour \ref{ITEMooWKVAooCQDvpL}]
        % -------------------------------------------------------------------------------------------- 
        Un chemin de Jordan est injectif, et donc bijectif sur son image. Il est donc une bijection continue depuis \( \mathopen[ a , b \mathclose]\) qui est compact. Il est donc un homéomorphisme par le lemme \ref{LEMooNEEVooSeHYzx}.
        \spitem[Pour \ref{ITEMooVYMXooEtgPJT}]
        % -------------------------------------------------------------------------------------------- 
        L'application \( \gamma\colon \mathopen[ a , b \mathclose[\to \Gamma \) est une bijection continue parce que \( \gamma\) est un lacet de Jordan. Son quotient \( \tilde \gamma\colon S^1\to \Gamma\) est donc unie bijection continue depuis le compact \( S^1\). Le lemme \ref{LEMooNEEVooSeHYzx} conclu que c'est un homéomorphisme.
    \end{subproof}
\end{proof}

\begin{corollary}
    À propos de courbes simples et de Jordan.
    \begin{enumerate}
        \item
            Toute courbe simple est homéomorphe à \( \mathopen[ 0 , 1 \mathclose]\).
        \item
            Toute courbe de Jordan est homéomorphe à \( S^1\).
    \end{enumerate}
\end{corollary}

\begin{proof}
    Une courbe simple est l'image d'un lacet de Jordan. Donc \( \gamma\colon \mathopen[ a , b \mathclose]\to \Gamma\) est un homéomorphisme par le lemme \ref{LEMooCGVOooVPlSRD}\ref{ITEMooWKVAooCQDvpL}. Le lemme \ref{LEMooAJDLooIPcmIV} fournit un homéomorphisme entre \( \mathopen[ a , b \mathclose]\) et \( \mathopen[ 0 , 1 \mathclose]\). Une composition d'homéomorphismes est un homéomorphisme.

    Une courbe de Jordan est l'image d'un lacet de Jordan \( \gamma\colon \mathopen[ a , b \mathclose]\to \Gamma\). Le lemme \ref{LEMooCGVOooVPlSRD}\ref{ITEMooVYMXooEtgPJT} dit alors que \( \Gamma\) est homéomorphe à \( S^1\).
\end{proof}


\begin{theorem}[Théorème de Jordan\cite{ooTXKNooIgJrPw, HDJTbua}]\label{ThoHSPWBuh}
	Le complémentaire d'une courbe de Jordan \( \Gamma\) dans un plan affine réel est formé de exactement deux composantes connexes distinctes, dont l'une est bornée et l'autre non. Toutes deux ont pour frontière la courbe \( \Gamma\).
\end{theorem}
\index{théorème!de Jordan}
% TODOooBBSQooKwbHLJ
% Si un jour on travaille sur ce théorème, il y a moyen de revoir la réponse de Alphago dans
% https://math.stackexchange.com/questions/1727310/convex-curve-as-boundary-of-a-convex-set
