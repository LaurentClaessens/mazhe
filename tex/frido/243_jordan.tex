% This is part of Le Frido
% Copyright (c) 2010-2016,2018-2020, 2022
%   Laurent Claessens, Carlotta Donadello
% See the file fdl-1.3.txt for copying conditions.



%---------------------------------------------------------------------------------------------------------------------------
\subsection{Le théorème de Jordan}
%---------------------------------------------------------------------------------------------------------------------------

Les définitions de chemins, de lacets et de courbes de Jordan sont dans \ref{DEFooQZMSooYYkGDv}.

\begin{lemmaDef}[\cite{BIBooQKARooMHqitK}]     \label{LEMooZPRLooJPvrOE}
    Soit un lacet \( \gamma\colon \mathopen[ a , b \mathclose]\to X\). Nous considérons l'application
    \begin{equation}
        \begin{aligned}
            q\colon \mathopen[ a , b \mathclose[&\to S^1 \\
            t&\mapsto  e^{2i\pi t/(b-a)} 
        \end{aligned}
    \end{equation}
    Il existe une unique application continue \( \tilde \gamma\colon S^1\to \Image(\gamma)\) telle que \( \gamma=\tilde \gamma\circ q\).

    Cette application \( \tilde \gamma\) est l'application \defe{quotient}{quotient d'un chemin} associée à \( \gamma\).
\end{lemmaDef}

\begin{proof}
    La proposition \ref{PROPooZEFEooEKMOPT} dit que \( q\) est une bijection continue. Donc l'existence et l'unicité d'une application \( \tilde \gamma=\gamma\circ q^{-1}\). Cette application est continue comme composée d'applications continues.
\end{proof}

\begin{lemma}[\cite{BIBooQKARooMHqitK}]     \label{LEMooCGVOooVPlSRD}
    Soit \( \gamma\colon \mathopen[ a , b \mathclose]\to X\) un chemin d'image \( \Gamma\).
    \begin{enumerate}
        \item       \label{ITEMooWKVAooCQDvpL}
            Si \( \gamma\) est un chemin de Jordan\footnote{Définition \ref{DEFooQZMSooYYkGDv}.}, alors \( \gamma\colon \mathopen[ a , b \mathclose]\to \Gamma\) est un isomorphisme d'espaces topologiques.
        \item       \label{ITEMooVYMXooEtgPJT}
            Si \( \gamma\) est un lacet de Jordan, alors le quotient\footnote{Définition \ref{LEMooZPRLooJPvrOE}.} \( \tilde \gamma\colon S^1\to \Gamma\) est un homéomorphisme\footnote{Un homéomorphisme est un isomorphisme d'espaces vectoriels, c'est à dire continu, inversible et d'inverse continu.}.
    \end{enumerate}
\end{lemma}

\begin{proof}
    En deux parties.
    \begin{subproof}
        \spitem[Pour \ref{ITEMooWKVAooCQDvpL}]
        % -------------------------------------------------------------------------------------------- 
        Un chemin de Jordan est injectif, et donc bijectif sur son image. Il est donc une bijection continue depuis \( \mathopen[ a , b \mathclose]\) qui est compact. Il est donc un homéomorphisme par le lemme \ref{LEMooNEEVooSeHYzx}.
        \spitem[Pour \ref{ITEMooVYMXooEtgPJT}]
        % -------------------------------------------------------------------------------------------- 
        L'application \( \gamma\colon \mathopen[ a , b \mathclose[\to \Gamma \) est une bijection continue parce que \( \gamma\) est un lacet de Jordan. Son quotient \( \tilde \gamma\colon S^1\to \Gamma\) est donc unie bijection continue depuis le compact \( S^1\). Le lemme \ref{LEMooNEEVooSeHYzx} conclu que c'est un homéomorphisme.
    \end{subproof}
\end{proof}

\begin{corollary}
    À propos de courbes simples et de Jordan.
    \begin{enumerate}
        \item
            Toute courbe simple est homéomorphe à \( \mathopen[ 0 , 1 \mathclose]\).
        \item
            Toute courbe de Jordan est homéomorphe à \( S^1\).
    \end{enumerate}
\end{corollary}

\begin{proof}
    Une courbe simple est l'image d'un lacet de Jordan. Donc \( \gamma\colon \mathopen[ a , b \mathclose]\to \Gamma\) est un homéomorphisme par le lemme \ref{LEMooCGVOooVPlSRD}\ref{ITEMooWKVAooCQDvpL}. Le lemme \ref{LEMooAJDLooIPcmIV} fournit un homéomorphisme entre \( \mathopen[ a , b \mathclose]\) et \( \mathopen[ 0 , 1 \mathclose]\). Une composition d'homéomorphismes est un homéomorphisme.

    Une courbe de Jordan est l'image d'un lacet de Jordan \( \gamma\colon \mathopen[ a , b \mathclose]\to \Gamma\). Le lemme \ref{LEMooCGVOooVPlSRD}\ref{ITEMooVYMXooEtgPJT} dit alors que \( \Gamma\) est homéomorphe à \( S^1\).
\end{proof}


\begin{theorem}[Théorème de Jordan\cite{ooTXKNooIgJrPw, HDJTbua}]\label{ThoHSPWBuh}
	Le complémentaire d'une courbe de Jordan \( \Gamma\) dans un plan affine réel est formé de exactement deux composantes connexes distinctes, dont l'une est bornée et l'autre non. Toutes deux ont pour frontière la courbe \( \Gamma\).
\end{theorem}
\index{théorème!de Jordan}
% TODOooBBSQooKwbHLJ
% Si un jour on travaille sur ce théorème, il y a moyen de revoir la réponse de Alphago dans
% https://math.stackexchange.com/questions/1727310/convex-curve-as-boundary-of-a-convex-set
