% This is part of Le Frido
% Copyright (c) 2010-2016,2018-2020, 2022-2023
%   Laurent Claessens, Carlotta Donadello
% See the file fdl-1.3.txt for copying conditions.



%---------------------------------------------------------------------------------------------------------------------------
\subsection{Le théorème de Jordan}
%---------------------------------------------------------------------------------------------------------------------------

Les définitions de chemins, de lacets et de courbes de Jordan sont dans \ref{DEFooQZMSooYYkGDv}.

\begin{lemmaDef}[\cite{BIBooQKARooMHqitK}]     \label{LEMooZPRLooJPvrOE}
    Soit un lacet \( \gamma\colon \mathopen[ a , b \mathclose]\to X\). Nous considérons l'application
    \begin{equation}
        \begin{aligned}
            q\colon \mathopen[ a , b \mathclose]&\to S^1 \\
            t&\mapsto  e^{2i\pi t/(b-a)} 
        \end{aligned}
    \end{equation}
    Il existe une unique application continue \( \tilde \gamma\colon S^1\to \Image(\gamma)\) telle que \( \gamma=\tilde \gamma\circ q\).

    Cette application \( \tilde \gamma\) est l'application \defe{quotient}{quotient d'un chemin} associée à \( \gamma\).
\end{lemmaDef}

\begin{proof}
    La proposition \ref{PROPooZEFEooEKMOPT} dit que \( q\) est une bijection continue. Donc l'existence et l'unicité d'une application \( \tilde \gamma=\gamma\circ q^{-1}\). Cette application est continue comme composée d'applications continues.
\end{proof}

\begin{lemma}[\cite{BIBooQKARooMHqitK}]     \label{LEMooCGVOooVPlSRD}
    Soit \( \gamma\colon \mathopen[ a , b \mathclose]\to X\) un chemin d'image \( \Gamma\).
    \begin{enumerate}
        \item       \label{ITEMooWKVAooCQDvpL}
            Si \( \gamma\) est un chemin de Jordan\footnote{Définition \ref{DEFooQZMSooYYkGDv}.}, alors \( \gamma\colon \mathopen[ a , b \mathclose]\to \Gamma\) est un isomorphisme d'espaces topologiques.
        \item       \label{ITEMooVYMXooEtgPJT}
            Si \( \gamma\) est un lacet de Jordan, alors le quotient\footnote{Définition \ref{LEMooZPRLooJPvrOE}.} \( \tilde \gamma\colon S^1\to \Gamma\) est un homéomorphisme\footnote{Un homéomorphisme est un isomorphisme d'espaces vectoriels, c'est à dire continu, inversible et d'inverse continu.}.
    \end{enumerate}
\end{lemma}

\begin{proof}
    En deux parties.
    \begin{subproof}
        \spitem[Pour \ref{ITEMooWKVAooCQDvpL}]
        % -------------------------------------------------------------------------------------------- 
        Un chemin de Jordan est injectif, et donc bijectif sur son image. Il est donc une bijection continue depuis \( \mathopen[ a , b \mathclose]\) qui est compact. Il est donc un homéomorphisme par le lemme \ref{LEMooNEEVooSeHYzx}.
        \spitem[Pour \ref{ITEMooVYMXooEtgPJT}]
        % -------------------------------------------------------------------------------------------- 
        L'application \( \gamma\colon \mathopen[ a , b \mathclose[\to \Gamma \) est une bijection continue parce que \( \gamma\) est un lacet de Jordan. Son quotient \( \tilde \gamma\colon S^1\to \Gamma\) est donc unie bijection continue depuis le compact \( S^1\). Le lemme \ref{LEMooNEEVooSeHYzx} conclu que c'est un homéomorphisme.
    \end{subproof}
\end{proof}

\begin{corollary}
    À propos de courbes simples et de Jordan.
    \begin{enumerate}
        \item
            Toute courbe simple est homéomorphe à \( \mathopen[ 0 , 1 \mathclose]\).
        \item
            Toute courbe de Jordan est homéomorphe à \( S^1\).
    \end{enumerate}
\end{corollary}

\begin{proof}
    Une courbe simple est l'image d'un lacet de Jordan. Donc \( \gamma\colon \mathopen[ a , b \mathclose]\to \Gamma\) est un homéomorphisme par le lemme \ref{LEMooCGVOooVPlSRD}\ref{ITEMooWKVAooCQDvpL}. Le lemme \ref{LEMooAJDLooIPcmIV} fournit un homéomorphisme entre \( \mathopen[ a , b \mathclose]\) et \( \mathopen[ 0 , 1 \mathclose]\). Une composition d'homéomorphismes est un homéomorphisme.

    Une courbe de Jordan est l'image d'un lacet de Jordan \( \gamma\colon \mathopen[ a , b \mathclose]\to \Gamma\). Le lemme \ref{LEMooCGVOooVPlSRD}\ref{ITEMooVYMXooEtgPJT} dit alors que \( \Gamma\) est homéomorphe à \( S^1\).
\end{proof}


\begin{definition}[\cite{BIBooQKARooMHqitK}]        \label{DEFooURFMooXIaRkl}
    Soit un compact \( K\) dans \( \eC\). Nous notons
    \begin{enumerate}
        \item
            \( \mG(K)\) le groupe multiplicatif des fonctions continues \( f\colon K\to \eC^*\).
        \item
            \( \mE(K)\) le sous-groupe des fonctions admettant un logarithme continu.
        \item
            \( G(K)\) le quotient\footnote{Le groupe \( \mG(K)\) est commutatif; donc pas de problèmes pour que \( \mE(K)\) soit normal.} \( G(K)=\mG(K)/\mE(K)\).
    \end{enumerate}
\end{definition}

\begin{lemma}[\cite{BIBooQKARooMHqitK}]     \label{LEMooHEOWooHTtHsJ}
    Si \( K_1\) et \( K_2\) sont des compacts homéomorphes, alors les groupes \( G(K_1)\) et \( G(K_2)\) sont isomorphes.
\end{lemma}

\begin{proof}
    Soit un homéomorphisme \( \theta\colon K_1\to K_2\). Nous posons
    \begin{equation}
        \begin{aligned}
            \psi\colon \mG(K_1)&\to \mG(K_2)  \\
            f&\mapsto f\circ \theta^{-1}. 
        \end{aligned}
    \end{equation}
    \begin{subproof}
        \spitem[Injective]
        % -------------------------------------------------------------------------------------------- 
        Suppose \( \psi(f)=\psi(g)\). Vu que \( \theta\) est une bijection, l'égalité \( f\circ \theta^{-1}=g\circ\theta^{-1}\) implique \( f=g\). Donc \( \psi\) est injective.
        \spitem[Surjective]
        % -------------------------------------------------------------------------------------------- 
        Soit \( g\in\mG(K_2)\). Nous avons \( f\circ\theta\in\mG(K_1)\), et évidemment \( \psi(f\circ\theta)=f\).
        \spitem[Morphisme]
        % -------------------------------------------------------------------------------------------- 
         Nous avons 
         \begin{equation}
             \psi(fg)=fg\circ\theta^{-1}=(f\circ\theta^{-1})(g\circ\theta^{-1})=\psi(f)\psi(g).
         \end{equation}
    \end{subproof}
    Nous avons prouvé que \( \psi\) est un isomorphisme. Nous devons maintenant voir qu'il passe au quotient, c'est à dire que \( \psi\big( \mE(K_1) \big)\subset\mE(K_2)\). Soit \( f\in\mE(K)\), soit un logarithme continu \( g\) de \( f\), c'est à dire \( \exp\big( g(x) \big)=f(x)\). Pour \( y\in K_2\), il existe \( x\in K_1\) tel que \( y=\theta(x)\). Dans ce cas nous avons
    \begin{equation}
        \exp\big( (g\circ\theta^{-1})(y) \big)=\exp\big( g(x) \big)=f(x)=(f\circ\theta^{-1})(y).
    \end{equation}
    Autrement dit,
    \begin{equation}
        \exp\big( \psi(g)(y) \big)=\psi(f).
    \end{equation}
    Donc \( \psi(g)\) est un logarithme continu de \( \psi(f)\).
\end{proof}

Petite notation. Si un compact \( K\) est donné, pour \( p\in\eC\setminus K\), nous notons
\begin{equation}
    \begin{aligned}
        f_p\colon K&\to \eC^* \\
        z&\mapsto z-p. 
    \end{aligned}
\end{equation}

\begin{lemma}[\cite{BIBooQKARooMHqitK}]     \label{LEMooBZUCooHWfolf}
    Soit un lacet de Jordan\footnote{Définition \ref{DEFooQZMSooYYkGDv}.} \( \gamma\colon \mathopen[ a , b \mathclose]\to \eC\). Nous notons \( \Gamma\) sont image. 
    \begin{enumerate}
        \item
            \( \Gamma\) est compact.
        \item
            L'application\footnote{Rappel définitions \ref{DEFooURFMooXIaRkl} pour \( \mG(\Gamma)\) et \( \mE(\Gamma)\).}
            \begin{equation}
                \begin{aligned}
                    I_{\gamma}\colon \mG(\Gamma)&\to \eZ \\
                    f&\mapsto \Ind(f\circ \gamma, 0) 
                \end{aligned}
            \end{equation}
            est un morphisme surjectif de noyau \( \mE(\Gamma)\).
    \end{enumerate}
\end{lemma}

\begin{proof}
    Nous notons \( \mL(J,\eC^*)\) le groupe multiplicatif des lacets \( J\to \eC^*\). Le fait que \( \Gamma\) soit compact est parce qu'il est l'image du compact \( \mathopen[ a , b \mathclose]\) par l'application continue \( \gamma\) (théorème \ref{ThoImCompCotComp}).

    Nous considérons l'application
    \begin{equation}
        \begin{aligned}
            \xi\colon \mG(\Gamma)&\to \aL(J,\eC^*) \\
            f&\mapsto f\circ \gamma. 
        \end{aligned}
    \end{equation}
    \begin{subproof}
        \spitem[\( \xi\) est un morphisme]
        % -------------------------------------------------------------------------------------------- 
         Nous avons
         \begin{equation}
             \xi(fg)=(fg)\circ \gamma=(f\circ \gamma)(g\circ \gamma)=\xi(f)\xi(g).
         \end{equation}
         \spitem[\( \xi\) est injective]
         % -------------------------------------------------------------------------------------------- 
         Nous nous souvenons de l'application quotient \( \tilde \gamma\colon S^1\to \Gamma\), qui est un homéomorphisme par le lemme \ref{LEMooCGVOooVPlSRD}\ref{ITEMooWKVAooCQDvpL}. Nous nous souvenons également de l'application
         \begin{equation}
             \begin{aligned}
                 q\colon \mathopen[ a , b \mathclose]&\to \eC \\
                 t&\mapsto \exp\big( 2i\pi t/(b-a) \big) 
             \end{aligned}
         \end{equation}
         qui vérifie \( \gamma=\tilde \gamma\circ q\). Nous montrons que ceci est l'inverse de \( \xi\) :
         \begin{equation}
             \begin{aligned}
                 \theta\colon \aL(J,\eC^*)&\to \mG(\Gamma) \\
                 \alpha&\mapsto \tilde \lambda\circ\tilde \gamma^{-1}. 
             \end{aligned}
         \end{equation}
         Soit \( \sigma\in\aL(J,\eC^*)\). Nous avons
         \begin{equation}
             \xi\big( \theta(\sigma) \big)=\theta(\sigma)\circ\gamma=\tilde \sigma\circ\tilde \gamma^{-1}\circ\gamma=\tilde \sigma\circ\tilde \gamma^{-1}\circ\tilde \gamma\circ q=\tilde \sigma\circ q=\sigma.
         \end{equation}
         \spitem[Conclusion 1]
         % -------------------------------------------------------------------------------------------- 
         La proposition \ref{PROPooGAOIooFTOuli} donne un morphisme surjectif \( \psi\colon \aL(J,\eC^*)\to \eZ\). L'application \( I_{\gamma}\) dont nous parlons ici est exactement \( I_{\gamma}=\psi\circ\xi\). Donc \( I_{\gamma}\) est un morphisme surjectif.
         \spitem[Noyau]
         % -------------------------------------------------------------------------------------------- 
         Nous devons encore montrer que le noyau de \( I_{\gamma}\) est \( \mE(\Gamma)\). 

            \begin{subproof}
                \spitem[Dans un sens]
                % -------------------------------------------------------------------------------------------- 
         Soit \( k\in\ker(I_{\gamma})\). Alors \( (\psi\circ\xi)(k)=0\), c'est à dire \( \psi(k\circ\gamma)=0\). La proposition \ref{PROPooGAOIooFTOuli}\ref{ITEMooZFXTooDCXTVU} dit qu'alors \( k\circ\gamma\) est un lacet homotope dans \(\eC^*\) à un lacet constant. Donc il existe une application continue \( H\colon \mathopen[ 0 , 1 \mathclose]\times\mathopen[ a , b \mathclose]\to \eC^*\) telle que
         \begin{subequations}
             \begin{numcases}{}
                 H(0,t)=(k\circ\gamma)(t)\\
                 H(1,t)=1
             \end{numcases}
         \end{subequations}
         pour tout \( t\). Vu que l'application \( \gamma\colon \mathopen[ a , b \mathclose[\to \eC^*\) est injective, nous pouvons considéré son inverse; pour la suite, quand nous écrirons \( \gamma^{-1}\), c'est bien de \( \gamma^{-1}\colon \Gamma\to \mathopen[ a , b \mathclose[\) que nous parlerons. C'est une application continue par le lemme \ref{LEMooKSDKooDbKKeB}. Nous pouvons donc considérer
        \begin{equation}
            \begin{aligned}
                L\colon \mathopen[ 0 , 1 \mathclose]\times\Gamma&\to \eC^* \\
                (u,z)&\mapsto H\big( u,\gamma^{-1}(z) \big). 
            \end{aligned}
        \end{equation}
        Cela est une application continue parce que \( H\) et \( \gamma\) le sont. Nous avons
        \begin{equation}
            L(0,z)=H\big( 0,\gamma^{-1}(z) \big)=(k\circ\gamma)\big( \gamma^{-1}(z) \big)=k(z),
        \end{equation}
        et
        \begin{equation}
            L(1,z)=1.
        \end{equation}
        Donc \( k\) est homotope à une application constante sur \( \Gamma\). Le théorème de Borsuk \ref{THOooTCUMooEByCKg} dit alors que \( k\) admet un logarithme continu sur \( \Gamma\), c'est à dire que \( k\in\mE(\Gamma)\).
        \spitem[Dans l'autre sens]
        % -------------------------------------------------------------------------------------------- 
        C'est le même raisonnement que le premier, mais un peu plus simple parce qu'il n'y a pas de débats sur la continuité. Soit \( k\in \mE(\Gamma)\). Le théorème de Boruk indique que \( k\) est homotope dans \( \eC^*\) à une application constante. Il existe une application continue \( H\colon \mathopen[ 0 , 1 \mathclose]\times \Gamma\to \eC^*\) telle que
        \begin{subequations}
            \begin{numcases}{}
                H(0,z)=k(z)\\
                H(1,z)=1
            \end{numcases}
        \end{subequations}
        pour tout \( z\in \Gamma\). Le chemin \( k\circ\gamma\) est homotope à un lacet constant. En effet nous avons l'application continue
        \begin{equation}
            \begin{aligned}
                L\colon \mathopen[ 0 , 1 \mathclose]\times\mathopen[ a , b \mathclose]&\to \eC^* \\
                (u,t)&\mapsto H\big( u,\gamma(t) \big) 
            \end{aligned}
        \end{equation}
        qui vérifie \( L(0,t)=(k\circ\gamma)(t)\) et \( L(1,t)=1\). Nous avons donc
        \begin{equation}
            0=\psi(k\circ\gamma)=I_{\gamma}(k).
        \end{equation}
        Nous avons prouvé que \( k\in\ker(I_{\gamma})\).
            \end{subproof}
    \end{subproof}
\end{proof}

\begin{lemma}[\cite{BIBooQKARooMHqitK}]     \label{LEMooODIPooBZJPAW}
    Soit un lacet de Jordan \( \gamma\) dont l'image est \( \Gamma\). Des fonctions \( f_0,f_1\in\mG(\Gamma)\) sont homotopes dans \( \eC^*\) si et seulement si les lacets \( f_0\circ \gamma\) et \( f_1\circ \gamma\) le sont.
\end{lemma}

\begin{proof}
    Dans les deux sens.
    \begin{subproof}
    \spitem[\( \Rightarrow\)]
    % -------------------------------------------------------------------------------------------- 
    Nous supposons que \( f_0\) et \( f_1\) sont homotopes : il existe une application continue \( H\colon \mathopen[ 0 , 1 \mathclose]\times \Gamma\to \eC^*\) telle que
    \begin{subequations}
        \begin{numcases}{}
            H(0,z)=f_0(z)\\
            H(1,z)=f_1(z)
        \end{numcases}
    \end{subequations}
    pour tout \( z\in \Gamma\). L'homotopie entre \( f_0\circ\gamma\) et \( f_1\circ\gamma\) est donnée par l'application continue
    \begin{equation}
        \begin{aligned}
            L\colon \mathopen[ 0 , 1 \mathclose]\times\mathopen[ a , b \mathclose]&\to \eC^* \\
            (u,t)&\mapsto H\big( u,\gamma(t) \big). 
        \end{aligned}
    \end{equation}
    Elle vérifie \( L(0,t)=(f_0\circ \gamma)(t)\) et \( L(1,t)=(f_1\circ\gamma)(t)\).
    \spitem[\( \Leftarrow\)]
    % -------------------------------------------------------------------------------------------- 
    Nous supposons l'existence d'une application continue \( H\colon \mathopen[ 0 , 1 \mathclose]\times \mathopen[ a , b \mathclose]\to \eC^*\) telle que \( H(0,t)=(f_0\circ\gamma)(t)\) et \( H(1,t)=(f_1\circ \gamma)(t)\).

    L'application \( \gamma\colon \mathopen[ a , b \mathclose[\to \Gamma\) est continue et bijective. Sa réciproque \( \gamma^{-1}\) est donc également continue (lemme \ref{LEMooKSDKooDbKKeB}). Nous pouvons donc considérer l'application continue
            \begin{equation}
                \begin{aligned}
                    L\colon \mathopen[ 0 , 1 \mathclose]\times \Gamma&\to \eC^* \\
                    (u,z)&\mapsto H\big( u,\gamma^{-1}(z) \big).
                \end{aligned}
            \end{equation}
            Elle vérifie
            \begin{equation}
                L(0,z)=H\big( 0,\gamma^{-1}(z) \big)=(f_0\circ \gamma)\big( \gamma^{-1}(z) \big)=f_0(z)
            \end{equation}
            et \( L(1,z)=f_1(z)\). Donc \( L\) est une homotopie dans \( C^*\) entre \( f_0\) et \( f_1\).
    \end{subproof}
\end{proof}

\begin{corollary}[\cite{BIBooQKARooMHqitK}]
    Si \( \Gamma\) est une courbe de Jordan, alors\footnote{Rappel : \( G(\Gamma)\) est défini en \ref{DEFooURFMooXIaRkl}.} \( G(\Gamma)\simeq \eZ \).
\end{corollary}

\begin{proof}
    Nous allons utiliser le premier théorème d'isomorphisme \ref{ThoPremierthoisomo}. Si \( \gamma\) est un lacet de Jordan dont la courbe est \( \Gamma\), le lemme \ref{LEMooBZUCooHWfolf} dit que \( I_{\gamma}\colon \mG(\Gamma)\to \eZ\) est un morphisme surjectif tel que \( \ker(I_{\gamma})=\mE(\Gamma)\). Le théorème d'isomorphisme dit alors que
    \begin{equation}
        \frac{\mG(\Gamma)}{ \mE(\Gamma) }=I_{\gamma}\big( \mG(\Gamma) \big)=\eZ.
    \end{equation}
\end{proof}

Rappel, si \( K\) est compact dans \( \eC\), et si \( p\in \eC\), nous considérons la fonction \( f_p\in\mG(K)\) par \( f_p(z)=z-p\).

\begin{lemma}[\cite{MonCerveau}]        \label{LEMooAKVFooFMHaOZ}
    Soient une courbe de Jordan \( \gamma\) ainsi que \( p\in \eC\). Nous avons
    \begin{equation}
        \Ind(f_p\circ \gamma,0)=\Ind(\gamma,p).
    \end{equation}
\end{lemma}

\begin{proposition}[\cite{BIBooQKARooMHqitK}]
    Soit \( p\in \eC\).  Si \( V\) est un disque ou un rectangle de centre \( p\), alors \( \mG(\partial V)\) est engendré par \( \mE(\partial V)\) et par \( f_p\).
\end{proposition}

\begin{proof}
    Ici \( \partial V\) peut être vu comme un chemin que nous allons noter aussi \( \gamma\). Utilisant le lemme \ref{LEMooAKVFooFMHaOZ}, nous avons
    \begin{equation}        \label{EQooRSXRooTJwDNP}
        I_{\partial V}(f_p)=\Ind(f_p\circ\partial V,0)=\Ind(\partial V,p).
    \end{equation}
    
    \begin{subproof}
        \spitem[Si \( V\) est un disque]
        % -------------------------------------------------------------------------------------------- 
        Alors \( \partial V\) est le chemin
        \begin{equation}
            \begin{aligned}
                \gamma\colon \mathopen[ 0 , 2\pi \mathclose]&\to \eC \\
                t&\mapsto p+R e^{it}. 
            \end{aligned}
        \end{equation}
        Nous avons alors le calcul suivant :
        \begin{subequations}
            \begin{align}
                \Ind(\gamma,p)&=\frac{1}{ 2\pi i }\int_0^{2\pi}\frac{ \gamma'(t) }{ \gamma(t)-p }dt\\
                &=\frac{1}{ 2\pi i }\int_0^{2\pi}\frac{ Ri e^{it} }{ R e^{it} }dt\\
                &=\frac{1}{ 2\pi i }\int_0^{2\pi}idt\\
                &=1.
            \end{align}
        \end{subequations}
        \spitem[Si \( V\) est un rectangle]
        % -------------------------------------------------------------------------------------------- 
        Supposons que \( V\) ait une hauteur de \( h\) et une largeur de \( l\) tout en étant centré en \( p\). Le chemin \( \partial V\) se décompose en quatre partie dont la première est
        \begin{equation}
            \begin{aligned}
                \gamma_1\colon \mathopen[ 0 , l \mathclose]&\to \eC\setminus \{ p \} \\
                t&\mapsto p-\frac{ hi }{ 2 }-\frac{ l }{2}+t 
            \end{aligned}
        \end{equation}
        Je vous laisse voir les autres parties et calculer les intégrales\quext{Je n'ai pas vérifié; si ça pose un problème, écrivez-moi.}
        
    \end{subproof}
    Dans les deux cas nous avons \( \Ind(\partial V, p)=1\) et donc, en continuant \eqref{EQooRSXRooTJwDNP},
    \begin{equation}
        I_{\partial V}(f_p)=1.
    \end{equation}

    Nous savons par le lemme \ref{LEMooBZUCooHWfolf} que
    \begin{equation}
        \begin{aligned}
            I_{\partial V}\colon \mG(\partial V)&\to \eZ \\
            f&\mapsto \Ind(f\circ \gamma,0) 
        \end{aligned}
    \end{equation}
    est un morphisme surjectif de noyau \( \mE(\Gamma)\). Le fait que ce soit un morphisme implique que
    \begin{equation}
        I_{\partial V}(f_p^n)=nI_{\partial V}(f_p)=n.
    \end{equation}
    Mais \( G(\partial V)=\mG(\partial V)/\mE(\partial V)\). Donc nous avons déjà prouvé que \( \{ f_p^k \}_{k\in \eN}\) contient un élément de chaque classe dans \( \mG(\partial V)/\mE(\partial V)\).

    Soit \( f\in \mG(\partial V)\). En notant \( [.]\) la classe par rapport à \( \mE(\partial V)\), il existe \( k\in \eN\) tel que
    \begin{equation}
        f\in\big[ I_{\partial V}(f_p^k) \big].
    \end{equation}
    Autrement dit, en posant \( \alpha=I_{\partial V}(f_p^k)\), il existe \( h\in \mE(\partial V)\) tel que \( f=\alpha h\). Nous avons donc bien prouvé que \( \{ f_p,\mE(\partial V) \}\) engendre \( \mG(\partial V)\).
\end{proof}

\begin{proposition}[\cite{BIBooQKARooMHqitK}]
    Soit un compact \( K\) de \( \eC\) ainsi que \( p,q\) deux points dans la même composante connexe de \( \eC\setminus K\). Alors
    \begin{enumerate}
        \item
            \( f_p/f_q\in\mE(K)\).
        \item
            Si \( p\) est dans la composante non bornée\footnote{Il y en a une seule, c'est le lemme \ref{LEMooJNPTooScfSvA}.} de \( \eC\setminus K\), alors \( f_p\in\mE(K)\).
    \end{enumerate}
\end{proposition}

\begin{proof}
    Soient \( p\) et \( q\) dans la même composante connexe de \( \eC\setminus K\). La proposition \ref{PROPooYFDBooHbBjzF} nous autorise à considérer un chemin \( \gamma\colon \mathopen[ 0 , 1 \mathclose]\to \eC\setminus K\) tel que \( \gamma(0)=p\) et \( \gamma(1)=q\). Nous posons
    \begin{equation}
        \begin{aligned}
            F\colon \mathopen[ 0 , 1 \mathclose]\times K&\to \eC^* \\
            (t,z)&\mapsto p_{\gamma(t)}(z)=z-\gamma(t). 
        \end{aligned}
    \end{equation}
    C'est une application continue vérifiant \( F(0,z)=f_p(z)\) et \( F(1,z)=f_q(z)\). Donc le applications \( f_p\) et \( f_q\) sont homotopes dans \( \eC^*\). La proposition \ref{PROPooNABDooFtKukO} nous indique qu'alors l'application \( f_p/f_q\) est homotope à \( 1\). Le théorème de Borsuk \ref{THOooTCUMooEByCKg} conclut que \( f_p/f_q\in\mE(K)\).
\end{proof}
<++>

%--------------------------------------------------------------------------------------------------------------------------- 
\subsection{Théorème de Jordan}
%---------------------------------------------------------------------------------------------------------------------------

\begin{theorem}[Théorème de Jordan\cite{ooTXKNooIgJrPw, HDJTbua}]\label{ThoHSPWBuh}
	Le complémentaire d'une courbe de Jordan \( \Gamma\) dans un plan affine réel est formé de exactement deux composantes connexes distinctes, dont l'une est bornée et l'autre non. Toutes deux ont pour frontière la courbe \( \Gamma\).
\end{theorem}
\index{théorème!de Jordan}
% TODOooBBSQooKwbHLJ
% Si un jour on travaille sur ce théorème, il y a moyen de revoir la réponse de Alphago dans
% https://math.stackexchange.com/questions/1727310/convex-curve-as-boundary-of-a-convex-set
