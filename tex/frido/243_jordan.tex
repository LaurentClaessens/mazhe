% This is part of Le Frido
% Copyright (c) 2010-2016,2018-2020, 2022
%   Laurent Claessens, Carlotta Donadello
% See the file fdl-1.3.txt for copying conditions.


%--------------------------------------------------------------------------------------------------------------------------- 
\subsection{Thérorème de Tietze (espace normal)}
%---------------------------------------------------------------------------------------------------------------------------

\begin{lemma}[\cite{BIBooQKARooMHqitK}]     \label{LEMooCLVAooTaNGJk}
    Soit un espace topologique normal\footnote{Définition \ref{DEFooNNKVooLtzImT}.} \( X\). Soient \( M\in \eR^+\) et \( A\) fermé dans \( X\), et une application continue \( f\colon A\to \mathopen[ -M , M \mathclose]\).

    Il existe une application continue \( g\colon X\to \mathopen[ -M/3 , M/3 \mathclose]\) telle que pour tout \( x\in A\),
            \begin{equation}
                | f(x)-g(x) |<\frac{ 2M }{ 3 },
            \end{equation}
\end{lemma}

\begin{proof}
    Nous divisons \( A\) en trois parties:
    \begin{subequations}
        \begin{align}
            A_-&=f\big( \mathopen[ -M , -M/3 \mathclose] \big),\\
            A_+&=f\big( \mathopen[ M/3 , 3 \mathclose] \big),\\
            A_0&=f\big( \mathopen[ -M/3 , M/3 \mathclose] \big).
        \end{align}
    \end{subequations}
    Les parties \( A_+\) et \( A_-\) sont fermées parce que \( f\) est continue. D'autre part, \( X\) est normal, de telle sorte que le théorème d'Urysohn \ref{THOooKYYEooLFcNpg} s'applique.

    Nous considérons donc une application continue \( g_1\colon X\to \mathopen[ 0 , 1 \mathclose]\) telle que \( g_1^{-1}(A_-)=\{ 0 \}\) et \( g_1^{-1}(A_+)=\{ 1 \}\). Nous posons alors
    \begin{equation}
        g(x)=\left( \frac{ 2M }{ 3 } \right)g_1(x)-\frac{ M }{ 3 }.
    \end{equation}
    Vérification des propriétés de \( g\).
    \begin{subproof}
        \spitem[\( g\) est continue]
        Parce que \( g_1\) est continue.
    \spitem[\( g\) prend ses valeurs dans \( \mathopen\lbrack -M/3 , M/3 \mathclose\rbrack\)]
        % -------------------------------------------------------------------------------------------- 
    Parce que \( g_1\) prend ses valeurs dans \( \mathopen[ 0 , 1 \mathclose]\) et que \( t\mapsto (2M/3)t-M/3\) est croissante.
    \spitem[\( | f(x)-g(x) |\) sur \( A\)]
    % -------------------------------------------------------------------------------------------- 
    Si \( x\in A_-\), alors \( g_1(x)=0\) et donc \( g(x)=-M/3\). Donc
    \begin{equation}        \label{EQooEUOKooFLRJjz}
        | f(x)-g(x) |=| f(x)-M/3 |\leq 2M/3
    \end{equation}
    parce que \( f(x)\in \mathopen[ -M/3 , M/3 \mathclose]\).

    Si \( x\in A_+\), alors \( g_1(x)=1\) et donc \( g(x)=M/3\). Même fin de raisonnement qu'en \eqref{EQooEUOKooFLRJjz}.

    Si \( x\in A_0\), alors \( f(x)\in \mathopen[ -M/3 , M/3 \mathclose]\) et \( g(x)\in\mathopen[ -M/3 , M/3 \mathclose]\) et donc encore \( | f(x)-g(x) |\leq 2M/3\).
    \end{subproof}
\end{proof}

\begin{lemma}[\cite{BIBooQKARooMHqitK}]     \label{LEMooSKSNooEdgFcR}
    Soient un espace topologique normal \( X\) ainsi qu'un fermé \( A\) dans \( X\). Nous considérons une fonction continue \( f\colon A\to \mathopen[ -M , M \mathclose]\).

    Il existe une application \( g\colon X\to \mathopen[ -M , M \mathclose]\) continue prolongeant \( f\).
\end{lemma}

\begin{proof}
    Nous allons commencer par construire une suite d'applications \( g_i\colon X\to \eR\) telles que
    \begin{enumerate}
        \item       \label{ITEMooGAIVooQYBCZj}
            Pour tout \( x\in A\) et pour tout \(n\in \eN\),
            \begin{equation}
                \big| f(x)-\sum_{i=1}^rng_i(x) \big|\leq \left( \frac{ 2 }{ 3 } \right)^NM
            \end{equation}
        \item       \label{ITEMooOLNAooEJPdbV}
            Pour tout \( x\in X\) et pour tout \( i\),
            \begin{equation}
                | g_i(x) |\leq \left( \frac{ 2 }{ 3 } \right)^i\frac{ M }{2}.
            \end{equation}
    \end{enumerate}
    Nous commençons par construire \( g_1\) à partir du lemme \ref{LEMooCLVAooTaNGJk} appliqué à la fonction \( f\). Nous avons donc une application \( g_1\colon X\to \mathopen[ -M/3 , M/3 \mathclose]\) telle que
    \begin{equation}        \label{EQooGXUIooWnATPw}
        | f(x)-g_1(x) |\leq \frac{ 2M }{ 3 }
    \end{equation}
    pour tout \( x\in A\). Vu que \( g_1\) prend ses valeurs dans \( \mathopen[ -M/3 , M/3 \mathclose]\), elle vérifie la condition \ref{ITEMooOLNAooEJPdbV}. De plus \eqref{EQooGXUIooWnATPw} montre que la condition \ref{ITEMooGAIVooQYBCZj} est vérifiée pour \( n=1\).

    Et c'est parti pour la récurrence. Nous supposons avoir des applications \( g_i\) pour \( i=1,\ldots, k\) qui vérifient la condition \ref{ITEMooOLNAooEJPdbV}, et telles que la condition \ref{ITEMooGAIVooQYBCZj} est satisfaite pour \(  n=1,\ldots, k\). Nous allons maintenant construire \( g_{k+1}\).

    Nous posons 
    \begin{equation}
        \begin{aligned}
            h_k\colon A&\to \eR \\
            x&\mapsto f(x)-\sum_{i=1}^kg_i(x). 
        \end{aligned}
    \end{equation}
    Par hypothèse de récurrence, la fonction \( h_k\) ne prend pas n'importe quelles valeurs dans \( \eR\), mais
    \begin{equation}
        h_k\colon A\to \mathopen\big[  -\left( \frac{ 2 }{ 3 } \right)^kM  , \left( \frac{ 2 }{ 3 } \right)^kM \mathclose\big].
    \end{equation}
    Nous construisons \( g_{k+1}\) à partir de ce \( h_k\) et du lemme \ref{LEMooCLVAooTaNGJk}. Nous avons donc
    \begin{equation}
        g_{k+1}\colon X\to \mathopen\Big[  -\frac{1}{ 3 }\left( \frac{ 2 }{ 3 } \right)^kM  , \frac{1}{ 3 }\left( \frac{ 2 }{ 3 } \right)^kM \mathclose\Big] 
    \end{equation}
    vérifiant
    \begin{equation}
        | h_k(x)-g_{k+1}(x) |\leq \left( \frac{ 2 }{ 3 } \right)^{k+1}M.
    \end{equation}
    La condition \ref{ITEMooGAIVooQYBCZj} est donc maintenant vérifiée jusqu'à \( n=k+1\). Nous vérifions la condition \ref{ITEMooOLNAooEJPdbV} pour \( i=k+1\). Simple calcul :
    \begin{equation}
        | g_{k+1}(x) |\leq \frac{1}{ 3 }\left( \frac{ 2 }{ 3 } \right)^kM=\left( \frac{ 2 }{ 3 } \right)^{k+1}\frac{ M }{ 2 }.
    \end{equation}
    Et voila pour la définition des applications \( g_i\).

    Vu que \( 2/3<1\), la série \( \sum_{i=1}^{\infty}\| g_i \|_{\infty}\) converge normalement (définition \ref{DefVBrJUxo}). Notons \( g\) la somme. Le lemme \ref{LEMooJZTBooIopLok} donne alors la convergence uniforme \( g_i\stackrel{unif}{\longrightarrow}g\), et le théorème \ref{ThoSerUnifCont} nous assure que \( g\) est continue sur \( X\).

    En ce qui concerne la norme de \( g\), nous avons, en utilisant la formule \eqref{EqRGkBhrX} avec \( q=2/3\),
    \begin{equation}
        \| g \|_{\infty}\leq \sum_{i=1}^{\infty}\left( \frac{ 2 }{ 3 } \right)^i\frac{ M }{2}\leq M.
    \end{equation}
    Donc \( | g(x) |\leq M\) pour tout \( x\in X\).

    Enfin nous vérifions que \( g\) prolonge \( f\). Soit \( x\in A\). Prenez la limite \( n\to \infty\) dans l'inégalité
    \begin{equation}
        | f(x)-\sum_{i=1}^ng_i(x) |\leq \left( \frac{ 2 }{ 3 } \right)^nM.
    \end{equation}
    Nous trouvons que \( | f(x)-g(x) |=0\).
\end{proof}

\begin{theorem}[Théorème de Tietze]     \label{THOooXKGWooFUYlux}
    Soit une partie fermée \( A\) de l'espace normal \( X\). Si la fonction \( f\colon A\to \eR\) est continue, alors elle se prolonge en une fonction continue \( g\colon X\to \eR\).
\end{theorem}

\begin{proof}
    Nous supposons dans un premier temps que \( f\) prenne ses valeurs dans \( \mathopen] -M , M \mathclose[\). À fortiori, elle prend ses valeurs dans \( \mathopen[ -M , M \mathclose]\) et le lemme \ref{LEMooSKSNooEdgFcR} dit qu'il existe un prolongement continu \( g\colon X\to \mathopen[ -M , M \mathclose]\). Nous allons construire à partir de là un prolongement continu \( h\colon X\to \mathopen] -M , M \mathclose[\) de \( f\).

    Nous posons
    \begin{equation}
        B=g^{-1}\big( \{ -M,M \} \big).
    \end{equation}
    Étant donné que \( f(A)\subset\mathopen] -M , M \mathclose[\) , nous avons \( B\cap A=\emptyset\). De plus \( A\) est fermé par hypothèse et \( B\) est fermé en tant qu'image réciproque du fermé \( \{ -M,M \}\) par l'application continue \( g\). Nous pouvons donc appliquer le théorème d'Urysohn \ref{THOooKYYEooLFcNpg}.

    Nous considérons donc une application \( g_1\colon X\to \mathopen[ 0 , 1 \mathclose]\) telle que \( g_1=0\) sur \( B\) et \( g_1=1\) sur \( A\). Enfin nous posons \( h=gg_1\). Cette application prend ses valeurs dans \( \mathopen] -M , M \mathclose[\), est continue et si \( x\in A\) nous avons
    \begin{equation}
        h(x)=g(x)g_1(x)=f(x)\time 1=f(x).
    \end{equation}
    Ceci règle la question si \( f\) prend ses valeurs dans \( \mathopen] -M , M \mathclose[\).

    Nous considérons à présent le cas général \( f\colon A\to \eR\). Soit un homéomorphisme \( \phi\colon \eR\to \mathopen] -M , M \mathclose[\) (par exemple via l'exemple \ref{EXooGKPNooZtmJen}). Nous considérons
    \begin{equation}
        \begin{aligned}
        \tilde f\colon A&\to \mathopen] -M , M \mathclose[ \\
            x&\mapsto (\phi\circ f)(x). 
        \end{aligned}
    \end{equation}
    Nous lui appliquons le premier cas pour avoir une fonction \( \tilde g\colon X\to \mathopen] -M , M \mathclose[\) qui prolonge \( \tilde f\). Il suffit maintenant de poser
    \begin{equation}
        \begin{aligned}
            g\colon X&\to \eR \\
            x&\mapsto (\phi^{-1}\circ\tilde g)(x). 
        \end{aligned}
    \end{equation}
    Cela est une application continue et si \( x\in A\), nous avons
    \begin{equation}
        g(x)=(\phi^{-1}\circ \tilde g)(x)=(\phi^{-1}\circ \tilde f)(x)=(\phi^{-1}\circ\phi\circ f)(x)=f(x).
    \end{equation}
\end{proof}

\begin{corollary}
    Soit un fermé \( A\) dans un espace normal \( X\). Si \( f\colon A\to \eC\) est continue, alors elle se prolonge en une fonction continue \( g\colon X\to \eC\).
\end{corollary}

\begin{proof}
    Les parties réelles et imaginaires de \( f\) sont continues. Il suffit de leur appliquer le théorème de Tietze \ref{THOooXKGWooFUYlux}.
\end{proof}

\begin{lemma}[\cite{BIBooQKARooMHqitK}]       \label{LEMooOFCEooIsuchR}
    Soit un rectangle fermé \( R\subset \eR^2\). Si l'application \( H\colon R\to \Omega\) est continue, alors pour toute fonction continue \( f\colon \Omega\to \eC\) nous avons
    \begin{equation}
        \int_{\partial H}f
    \end{equation}
    où \( \partial H\) désigne la frontière de \( H(R)\) dans \( \eC\), et l'intégrale est celle de la définition \ref{PROPooCUBTooZDcdHX}.
\end{lemma}

\begin{proof}
    La frontière \( \partial H\) peut être décomposée en \( 4\) parties de classe \( C^1\) :
    \begin{enumerate}
        \item
            \begin{equation}
                \begin{aligned}
                    \alpha_1\colon \mathopen[ 0 , 1 \mathclose]&\to \Omega \\
                    t&\mapsto H(t,a), 
                \end{aligned}
            \end{equation}
        \item
            \begin{equation}
                \begin{aligned}
                    \alpha_2\colon \mathopen[ a , b \mathclose]&\to \Omega \\
                    t&\mapsto H(1,t)
                \end{aligned}
            \end{equation}
        \item
            \begin{equation}
                \begin{aligned}
                    \alpha_3\colon \mathopen[ 0 , 1 \mathclose]&\to \Omega \\
                    t&\mapsto H(1-t,b) 
                \end{aligned}
            \end{equation}
        \item
            \begin{equation}
                \begin{aligned}
                    \alpha_4\colon \mathopen[ a , b \mathclose]&\to \Omega \\
                    t&\mapsto H\big(0, b+a-t).
                \end{aligned}
            \end{equation}
    \end{enumerate}
    Étudions les diverses parties \( \alpha_i\).
    \begin{subproof}
        \spitem[Pour \( \alpha_2\)]
        % -------------------------------------------------------------------------------------------- 
    Première observation : \( \alpha_2=\gamma_1\). 
    \spitem[Pour \( \alpha_4\)]
    % -------------------------------------------------------------------------------------------- 
     Nous avons \( \alpha_4=\gamma_0(b+a-t)\) et donc
    \begin{equation}        \label{EQooFWIQooJALDpO}
            \int_{\alpha_4}f=\int_a^b(f\circ \alpha_4)(t)\alpha_4'(t)= -\int_a^b(f\circ \gamma_0)(b+a-t)\gamma_0'(b+a-t).
    \end{equation}
    Nous utilisons le changement de variables\footnote{Théorème \ref{THOooUMIWooZUtUSg}\ref{ITEMooAJGDooGHKnvj}.}
    \begin{equation}
        \begin{aligned}
            \phi\colon \mathopen[ a , b \mathclose]&\to \mathopen[ a , b \mathclose] \\
            t&\mapsto b+a-t .
        \end{aligned}
    \end{equation}
    Le jacobien \( | J_{\phi} |\) est égal à \( 1\) et donc nous continuons \eqref{EQooFWIQooJALDpO} de la façon suivante :
    \begin{subequations}
        \begin{align}
            \int_{\alpha_4}&=-\int_a^b(f\circ\gamma_0)\big( \phi(t) \big)\gamma'\big( \phi(t) \big)dt\\
            &=-\int_a^b(f\circ\gamma)(t)\gamma_0'(t)dt\\
            &=-\int_{\gamma_0}f.
        \end{align}
    \end{subequations}
    \spitem[Lien entre \( \alpha_1\) et \( \alpha_3\)]
    % -------------------------------------------------------------------------------------------- 
    Il y a deux possibilités : soit \( \gamma_1\) et \( \gamma_2\) sont des lacets, soit ce sont des chemins normaux.
    \begin{subproof}
        \spitem[Si ils sont des lacets]
        % -------------------------------------------------------------------------------------------- 
        Alors pour tout \( s\), l'application
        \begin{equation}
            \begin{aligned}
                \Gamma_s\colon \mathopen[ a , b \mathclose]&\to C \\
                t&\mapsto H(s,t) 
            \end{aligned}
        \end{equation}
        est un lacet. En particulier \( H(s,a)=H(s,b)\). Donc \( \alpha_3(t)=H(1-t,b)=\alpha_1(1-t)\). Dans ce cas nous avons
        \begin{subequations}
            \begin{align}
                \int_{\alpha_3}f&=\int_a^bf\big( \alpha_3t \big)\alpha_3'(t)\\
                &=\int_a^bf\big( \alpha_1(1-t) \big)(-)\alpha_1'(1-t)\\
                &=-\int_a^b(f\circ\alpha_1)(t)dt        &\text{chm. var. \( \phi(t)=1-t\)}\\
                &=-\int_{\alpha_1}f.
            \end{align}
        \end{subequations}
        Nous avons alors le calcul
        \begin{equation}
            0=\int_{\partial H}f=\int_{\alpha_1}f+\int_{\gamma_1}f+\underbrace{\int_{\alpha_3}f}_{-\int_{\alpha_1}f}-\int_{\gamma_0}f=\int_{\gamma_1}f-\int_{\gamma_0}f.
        \end{equation}
        Le lemme est prouvé.
    \spitem[Si \( \gamma_0\) et \( \gamma_1\) sont des chemins]
    Dans ce cas nous avons une homotopie à extrémités fixées. Donc \( \alpha_1\) et \( \alpha_3\) sont des chemins constants, et les intégrales dessus sont nulles et
        \begin{equation}
            0=\int_{\partial H}f=\underbrace{\int_{\alpha_1}f}_{=0}+\int_{\gamma_1}f+\underbrace{\int_{\alpha_3}f}_{=0}-\int_{\gamma_0}f=\int_{\gamma_1}f-\int_{\gamma_0}f.
        \end{equation}
    \end{subproof}
    Et le lemme est prouvé.
    \end{subproof}
\end{proof}


\begin{theorem}[\cite{BIBooQKARooMHqitK}]     \label{THOooVTFXooBgvVyD}
    Soit un ouvert \( \Omega\) dans \( \eC\). Soit une fonction holomorphe \( f\colon \Omega\to \eC\). 

    Supposons que l'une des deux conditions suivantes soit respectées :
    \begin{enumerate}
        \item
        Les chemine \( \gamma_1,\gamma_2\colon \mathopen[ a , b \mathclose]\to \Omega \) sont homotopes à extrémités fixées. 
    \item
        Les lacets \( \gamma_1,\gamma_2\colon \mathopen[ a , b \mathclose]\to \Omega \) sont homotopes.
    \end{enumerate}
    Alors
    \begin{equation}
        \int_{\gamma_0}f=\int_{\gamma_1}f.
    \end{equation}
\end{theorem}

\begin{proof}
    Considérons l'homotopie \( H\colon \mathopen[ 0 , 1 \mathclose]\times \mathopen[ a , b \mathclose]\to \Omega\). Étant donné que \( \mathopen[ 0 ,1  \mathclose]\times \mathopen[ a , b \mathclose]\) est un rectangle dans \( \eR^2\), le lemme \ref{LEMooOFCEooIsuchR} nous indique que 
    \begin{equation}
        \int_{\partial H}f=0.
    \end{equation}
    %TODOooBQGFooHRquzz je n'ai pas l'impression que cette preuve soit terminée.
\end{proof}

\begin{definition}[homotopie]       \label{DEFooPJKLooCvgxsu}
    Soient des espaces topologiques \( X\) et \( Y\). Deux applications continues \( f_1,f_2\colon X\to Y\) sont \defe{homotopes}{applications homotopes} si il existe une application continue \( H\colon \mathopen[ 0 , 1 \mathclose]\times X\to Y \) telle que pour tout \( x\in X \) nous avons
    \begin{subequations}
        \begin{align}
            H(0,x)&=f_1(x)\\
            H(1,x)&=f_2(x).
        \end{align}
    \end{subequations}
\end{definition}

\begin{lemma}
    La relation «être homotope à\footnote{Définition \ref{DEFooPJKLooCvgxsu}.}»  est une relation d'équivalence sur \( C(X,Y)\).
\end{lemma}

\begin{proof}
    Pour obtenir \( f\sim f\), il suffit de prendre \( H(s,x)=f(x)\).

    Si \( f\sim g\), nous considérons l'homotopie \( H\colon \mathopen[ 0 , 1 \mathclose]\times X\to Y\). Alors l'application
    \begin{equation}
        \begin{aligned}
            M\colon \mathopen[ 0 , 1 \mathclose]\times X&\to Y \\
            (t,x)&\mapsto H(1-t,x) 
        \end{aligned}
    \end{equation}
    est une homotopie pour \( g\sim f\).

    Si \( f\sim g\) et \( g\sim f\), nous avons les applications continues \( H\) et \( M\) telles que
    \begin{subequations}
        \begin{align}
        H(0,x)&=f(x)&&H(1,x)&=g(x)\\
        M(0,x)&=g(x)&&M(1,x)&=h(x).
        \end{align}
    \end{subequations}
    L'application
    \begin{equation}
        \begin{aligned}
            S\colon \mathopen[ 0 , 1 \mathclose]\times X&\to Y \\
            (t,x)&\mapsto \begin{cases}
                H(2t,x)    &   \text{si } t\in\mathopen[ 0 , \frac{ 1 }{2} \mathclose[\\
                M(2t-1,x)    &    \text{si }t\in\mathopen[ \frac{ 1 }{2} , 1 \mathclose].
            \end{cases}
        \end{aligned}
    \end{equation}
    Nous vérifions que \( S\) est continue en vérifiant la valeur en \( t=1/2\). De plus 
    \begin{equation}
        S(0,)=H(0,x)=f(x)
    \end{equation}
    et
    \begin{equation}
        S(1,x)=M(2-1,x)=M(1,x)=g(x).
    \end{equation}
\end{proof}


\begin{proposition}[\cite{BIBooQKARooMHqitK}]
    Soit un compact \( K\) de \( \eR^n\). Deux applications \( f,g\colon K\to \eC^*\) sont homotopes dans \( \eC^*\) si et seulement si \( f/g\) est homotope à la fonction 
    \begin{equation}
        \begin{aligned}
            u\colon K&\to \eC \\
            x&\mapsto 1. 
        \end{aligned}
    \end{equation}
\end{proposition}

\begin{proof}
    Supposons que \( H\colon \mathopen[ 0 , 1 \mathclose]\times K\to \eC^*\) est une homotopie entre \( f\) et \( g\). Dans ce cas, l'application
    \begin{equation}
        \begin{aligned}
            M\colon \mathopen[ 0 , 1 \mathclose]\times K&\to \eC^* \\
            (t,x)&\mapsto \frac{ H(t,x) }{ g(x) } 
        \end{aligned}
    \end{equation}
    est une homotopie entre \( f/g\) et \( u\). En effet
    \begin{equation}
        M(0,x)=\frac{ H(0,x) }{ g(x) }=\frac{ f(x) }{ g(x) }
    \end{equation}
    et
    \begin{equation}
        M(1,x)=\frac{ H(1,x) }{ g(x) }=1=u(x).
    \end{equation}

    Dans l'autre sens, si \( H\) est une homotopie entre \( f/g\) et \( u\), alors l'application \( M(t,x)=H(t,x)g(x)\) est une homotopie entre \( f\) et \( g\).
\end{proof}


%--------------------------------------------------------------------------------------------------------------------------- 
\subsection{Logarithme sur un chemin}
%---------------------------------------------------------------------------------------------------------------------------

\begin{definition}
    Soient un espace topologique \( X\), et une application \( f\colon X\to \eC^*\). Nous disons que \( g\colon X\to \eC^*\) est un \defe{logarithme}{logarithme d'une application} de \( f\) si pour tout \( x\in X\) nous avons
    \begin{equation}
        f(x)=\exp\big( g(x) \big).
    \end{equation}
\end{definition}

\begin{definition}[Détermination du logarithme]     \label{DEFooOCDGooGyvvWi}
    Soit un chemin\footnote{Définition \ref{DEFooQZMSooYYkGDv}.} \( \gamma\colon \mathopen[ a , b \mathclose]\to \eC^*\). Nous disons qu'une application \( g\colon \gamma\big( \mathopen[ a , b \mathclose] \big)\to \eC^*\) est une \defe{détermination du logarithme}{détermination du logarithme} sur \( \gamma\) si
    \begin{equation}
        \exp\big( (g\circ\gamma)(t) \big)=\gamma(t)
    \end{equation}
    pour tout \( t\in \mathopen[ a , b \mathclose]\).
\end{definition}

\begin{theorem}     \label{THOooUPANooMiECqe}
    Tout chemin dans \( \eC^*\) admet une détermination du logarithme\footnote{Définition \ref{DEFooOCDGooGyvvWi}.}, et si \( l\) est une détermination sur le chemin \( \gamma\), nous avons
    \begin{equation}
        \int_{\gamma}\frac{ dz }{ z }=l\big( \gamma(b) \big)-l\big( \gamma(a) \big).
    \end{equation}
\end{theorem}

C'est future : \ref{LEMooCGVOooVPlSRD}.

%---------------------------------------------------------------------------------------------------------------------------
\subsection{Le théorème de Jordan}
%---------------------------------------------------------------------------------------------------------------------------

Les définitions de chemins, de lacets et de courbes de Jordan sont dans \ref{DEFooQZMSooYYkGDv}.

\begin{lemmaDef}[\cite{BIBooQKARooMHqitK}]     \label{LEMooZPRLooJPvrOE}
    Soit un lacet \( \gamma\colon \mathopen[ a , b \mathclose]\to X\). Nous considérons l'application
    \begin{equation}
        \begin{aligned}
            q\colon \mathopen[ a , b \mathclose[&\to S^1 \\
            t&\mapsto  e^{2i\pi t/(b-a)} 
        \end{aligned}
    \end{equation}
    Il existe une unique application continue \( \tilde \gamma\colon S^1\to \Image(\gamma)\) telle que \( \gamma=\tilde \gamma\circ q\).

    Cette application \( \tilde \gamma\) est l'application \defe{quotient}{quotient d'un chemin} associée à \( \gamma\).
\end{lemmaDef}

\begin{proof}
    La proposition \ref{PROPooZEFEooEKMOPT} dit que \( q\) est une bijection continue. Donc l'existence et l'unicité d'une application \( \tilde \gamma=\gamma\circ q^{-1}\). Cette application est continue comme composée d'applications continues.
\end{proof}

\begin{lemma}[\cite{BIBooQKARooMHqitK}]     \label{LEMooCGVOooVPlSRD}
    Soit \( \gamma\colon \mathopen[ a , b \mathclose]\to X\) un chemin d'image \( \Gamma\).
    \begin{enumerate}
        \item       \label{ITEMooWKVAooCQDvpL}
            Si \( \gamma\) est un chemin de Jordan\footnote{Définition \ref{DEFooQZMSooYYkGDv}.}, alors \( \gamma\colon \mathopen[ a , b \mathclose]\to \Gamma\) est un isomorphisme d'espaces topologiques.
        \item       \label{ITEMooVYMXooEtgPJT}
            Si \( \gamma\) est un lacet de Jordan, alors le quotient\footnote{Définition \ref{LEMooZPRLooJPvrOE}.} \( \tilde \gamma\colon S^1\to \Gamma\) est un homéomorphisme\footnote{Un homéomorphisme est un isomorphisme d'espaces vectoriels, c'est à dire continu, inversible et d'inverse continu.}.
    \end{enumerate}
\end{lemma}

\begin{proof}
    En deux parties.
    \begin{subproof}
        \spitem[Pour \ref{ITEMooWKVAooCQDvpL}]
        % -------------------------------------------------------------------------------------------- 
        Un chemin de Jordan est injectif, et donc bijectif sur son image. Il est donc une bijection continue depuis \( \mathopen[ a , b \mathclose]\) qui est compact. Il est donc un homéomorphisme par le lemme \ref{LEMooNEEVooSeHYzx}.
        \spitem[Pour \ref{ITEMooVYMXooEtgPJT}]
        % -------------------------------------------------------------------------------------------- 
        L'application \( \gamma\colon \mathopen[ a , b \mathclose[\to \Gamma \) est une bijection continue parce que \( \gamma\) est un lacet de Jordan. Son quotient \( \tilde \gamma\colon S^1\to \Gamma\) est donc unie bijection continue depuis le compact \( S^1\). Le lemme \ref{LEMooNEEVooSeHYzx} conclu que c'est un homéomorphisme.
    \end{subproof}
\end{proof}

\begin{corollary}
    À propos de courbes simples et de Jordan.
    \begin{enumerate}
        \item
            Toute courbe simple est homéomorphe à \( \mathopen[ 0 , 1 \mathclose]\).
        \item
            Toute courbe de Jordan est homéomorphe à \( S^1\).
    \end{enumerate}
\end{corollary}

\begin{proof}
    Une courbe simple est l'image d'un lacet de Jordan. Donc \( \gamma\colon \mathopen[ a , b \mathclose]\to \Gamma\) est un homéomorphisme par le lemme \ref{LEMooCGVOooVPlSRD}\ref{ITEMooWKVAooCQDvpL}. Le lemme \ref{LEMooAJDLooIPcmIV} fournit un homéomorphisme entre \( \mathopen[ a , b \mathclose]\) et \( \mathopen[ 0 , 1 \mathclose]\). Une composition d'homéomorphismes est un homéomorphisme.

    Une courbe de Jordan est l'image d'un lacet de Jordan \( \gamma\colon \mathopen[ a , b \mathclose]\to \Gamma\). Le lemme \ref{LEMooCGVOooVPlSRD}\ref{ITEMooVYMXooEtgPJT} dit alors que \( \Gamma\) est homéomorphe à \( S^1\).
\end{proof}


\begin{theorem}[Théorème de Jordan\cite{ooTXKNooIgJrPw, HDJTbua}]\label{ThoHSPWBuh}
	Le complémentaire d'une courbe de Jordan \( \Gamma\) dans un plan affine réel est formé de exactement deux composantes connexes distinctes, dont l'une est bornée et l'autre non. Toutes deux ont pour frontière la courbe \( \Gamma\).
\end{theorem}
\index{théorème!de Jordan}
% TODOooBBSQooKwbHLJ
% Si un jour on travaille sur ce théorème, il y a moyen de revoir la réponse de Alphago dans
% https://math.stackexchange.com/questions/1727310/convex-curve-as-boundary-of-a-convex-set
