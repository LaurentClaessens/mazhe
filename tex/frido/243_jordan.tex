% This is part of Le Frido
% Copyright (c) 2010-2016,2018-2020, 2022-2023
%   Laurent Claessens, Carlotta Donadello
% See the file fdl-1.3.txt for copying conditions.



%---------------------------------------------------------------------------------------------------------------------------
\subsection{Le théorème de Jordan}
%---------------------------------------------------------------------------------------------------------------------------

Les définitions de chemins, de lacets et de courbes de Jordan sont dans \ref{DEFooQZMSooYYkGDv}.

\begin{lemmaDef}[\cite{BIBooQKARooMHqitK}]     \label{LEMooZPRLooJPvrOE}
    Soit un lacet \( \gamma\colon \mathopen[ a , b \mathclose]\to X\). Nous considérons l'application
    \begin{equation}
        \begin{aligned}
            q\colon \mathopen[ a , b \mathclose[&\to S^1 \\
            t&\mapsto  e^{2i\pi t/(b-a)} 
        \end{aligned}
    \end{equation}
    Il existe une unique application continue \( \tilde \gamma\colon S^1\to \Image(\gamma)\) telle que \( \gamma=\tilde \gamma\circ q\).

    Cette application \( \tilde \gamma\) est l'application \defe{quotient}{quotient d'un chemin} associée à \( \gamma\).
\end{lemmaDef}

\begin{proof}
    La proposition \ref{PROPooZEFEooEKMOPT} dit que \( q\) est une bijection continue. Donc l'existence et l'unicité d'une application \( \tilde \gamma=\gamma\circ q^{-1}\). Cette application est continue comme composée d'applications continues.
\end{proof}

\begin{lemma}[\cite{BIBooQKARooMHqitK}]     \label{LEMooCGVOooVPlSRD}
    Soit \( \gamma\colon \mathopen[ a , b \mathclose]\to X\) un chemin d'image \( \Gamma\).
    \begin{enumerate}
        \item       \label{ITEMooWKVAooCQDvpL}
            Si \( \gamma\) est un chemin de Jordan\footnote{Définition \ref{DEFooQZMSooYYkGDv}.}, alors \( \gamma\colon \mathopen[ a , b \mathclose]\to \Gamma\) est un isomorphisme d'espaces topologiques.
        \item       \label{ITEMooVYMXooEtgPJT}
            Si \( \gamma\) est un lacet de Jordan, alors le quotient\footnote{Définition \ref{LEMooZPRLooJPvrOE}.} \( \tilde \gamma\colon S^1\to \Gamma\) est un homéomorphisme\footnote{Un homéomorphisme est un isomorphisme d'espaces vectoriels, c'est à dire continu, inversible et d'inverse continu.}.
    \end{enumerate}
\end{lemma}

\begin{proof}
    En deux parties.
    \begin{subproof}
        \spitem[Pour \ref{ITEMooWKVAooCQDvpL}]
        % -------------------------------------------------------------------------------------------- 
        Un chemin de Jordan est injectif, et donc bijectif sur son image. Il est donc une bijection continue depuis \( \mathopen[ a , b \mathclose]\) qui est compact. Il est donc un homéomorphisme par le lemme \ref{LEMooNEEVooSeHYzx}.
        \spitem[Pour \ref{ITEMooVYMXooEtgPJT}]
        % -------------------------------------------------------------------------------------------- 
        L'application \( \gamma\colon \mathopen[ a , b \mathclose[\to \Gamma \) est une bijection continue parce que \( \gamma\) est un lacet de Jordan. Son quotient \( \tilde \gamma\colon S^1\to \Gamma\) est donc unie bijection continue depuis le compact \( S^1\). Le lemme \ref{LEMooNEEVooSeHYzx} conclu que c'est un homéomorphisme.
    \end{subproof}
\end{proof}

\begin{corollary}
    À propos de courbes simples et de Jordan.
    \begin{enumerate}
        \item
            Toute courbe simple est homéomorphe à \( \mathopen[ 0 , 1 \mathclose]\).
        \item
            Toute courbe de Jordan est homéomorphe à \( S^1\).
    \end{enumerate}
\end{corollary}

\begin{proof}
    Une courbe simple est l'image d'un lacet de Jordan. Donc \( \gamma\colon \mathopen[ a , b \mathclose]\to \Gamma\) est un homéomorphisme par le lemme \ref{LEMooCGVOooVPlSRD}\ref{ITEMooWKVAooCQDvpL}. Le lemme \ref{LEMooAJDLooIPcmIV} fournit un homéomorphisme entre \( \mathopen[ a , b \mathclose]\) et \( \mathopen[ 0 , 1 \mathclose]\). Une composition d'homéomorphismes est un homéomorphisme.

    Une courbe de Jordan est l'image d'un lacet de Jordan \( \gamma\colon \mathopen[ a , b \mathclose]\to \Gamma\). Le lemme \ref{LEMooCGVOooVPlSRD}\ref{ITEMooVYMXooEtgPJT} dit alors que \( \Gamma\) est homéomorphe à \( S^1\).
\end{proof}


\begin{definition}[\cite{BIBooQKARooMHqitK}]        \label{DEFooURFMooXIaRkl}
    Soit un compact \( K\) dans \( \eC\). Nous notons
    \begin{enumerate}
        \item
            \( \mG(K)\) le groupe multiplicatif des fonctions continues \( f\colon K\to \eC^*\).
        \item
            \( \mE(K)\) le sous-groupe des fonctions admettant un logarithme continu.
        \item
            \( G(K)\) le quotient\footnote{Le groupe \( \mG(K)\) est commutatif; donc pas de problèmes pour que \( \mE(K)\) soit normal.} \( G(K)=\mG(K)/\mE(K)\).
    \end{enumerate}
\end{definition}

\begin{lemma}[\cite{BIBooQKARooMHqitK}]     \label{LEMooHEOWooHTtHsJ}
    Si \( K_1\) et \( K_2\) sont des compacts homéomorphes, alors les groupes \( G(K_1)\) et \( G(K_2)\) sont isomorphes.
\end{lemma}

\begin{proof}
    Soit un homéomorphisme \( \theta\colon K_1\to K_2\). Nous posons
    \begin{equation}
        \begin{aligned}
            \psi\colon \mG(K_1)&\to \mG(K_2)  \\
            f&\mapsto f\circ \theta^{-1}. 
        \end{aligned}
    \end{equation}
    \begin{subproof}
        \spitem[Injective]
        % -------------------------------------------------------------------------------------------- 
        Suppose \( \psi(f)=\psi(g)\). Vu que \( \theta\) est une bijection, l'égalité \( f\circ \theta^{-1}=g\circ\theta^{-1}\) implique \( f=g\). Donc \( \psi\) est injective.
        \spitem[Surjective]
        % -------------------------------------------------------------------------------------------- 
        Soit \( g\in[mG(K_2)]\). Nous avons \( f\circ\theta\in\mG(K_1)\), et évidemment \( \psi(f\circ\theta)=f\).
        \spitem[Morphisme]
        % -------------------------------------------------------------------------------------------- 
         Nous avons 
         \begin{equation}
             \psi(fg)=fg\circ\theta^{-1}=(f\circ\theta^{-1})(g\circ\theta^{-1})=\psi(f)\psi(g).
         \end{equation}
    \end{subproof}
    Nous avons prouvé que \( \psi\) est un isomorphisme. Nous devons maintenant voir qu'il passe au quotient, c'est à dire que \( \psi\big( \mE(K_1) \big)\subset\mE(K_2)\). Soit \( f\in\mE(K)\), soit un logarithme continu \( g\) de \( f\), c'est à dire \( \exp\big( g(x) \big)=f(x)\). Pour \( y\in K_2\), il existe \( x\in K_1\) tel que \( y=\theta(x)\). Dans ce cas nous avons
    \begin{equation}
        \exp\big( (g\circ\theta^{-1})(y) \big)=\exp\big( g(x) \big)=f(x)=(f\circ\theta^{-1})(y).
    \end{equation}
    Autrement dit,
    \begin{equation}
        \exp\big( \psi(g)(y) \big)=\psi(f).
    \end{equation}
    Donc \( \psi(g)\) est un logarithme continu de \( \psi(f)\).
\end{proof}

Petite notation. Si un compact \( K\) est donné, pour \( p\in\eC\setminus K\), nous notons
\begin{equation}
    \begin{aligned}
        f_p\colon K&\to \eC^* \\
        z&\mapsto z-p. 
    \end{aligned}
\end{equation}

\begin{lemma}[\cite{BIBooQKARooMHqitK}]     \label{LEMooBZUCooHWfolf}
    Soit un lacet de Jordan\footnote{Définition \ref{DEFooQZMSooYYkGDv}.} \( \gamma\colon \mathopen[ a , b \mathclose]\to \eC\). Nous notons \( \Gamma\) sont image. 
    \begin{enumerate}
        \item
            \( \Gamma\) est compact.
        \item
            L'application\footnote{Rappel définitions \ref{DEFooURFMooXIaRkl} pour \( \mG(\Gamma)\) et \( \mE(\Gamma)\).}
            \begin{equation}
                \begin{aligned}
                    I_{\gamma}\colon \mG(\Gamma)&\to \eZ \\
                    f&\mapsto \Ind(f\circ \gamma, 0) 
                \end{aligned}
            \end{equation}
            est un morphisme injectif de noyau \( \mE(\Gamma)\).
    \end{enumerate}
\end{lemma}

\begin{proof}
    Nous notons \( \mL(J,\eC^*)\) le groupe multiplicatif des lacets \( J\to \eC^*\). Le fait que \( \Gamma\) soit compact est parce qu'il est l'image du compact \( \mathopen[ a , b \mathclose]\) par l'application continue \( \gamma\) (théorème \ref{ThoImCompCotComp}).

    Nous considérons l'application
    \begin{equation}
        \begin{aligned}
            \xi\colon \mG(\Gamma)&\to \aL(J,\eC^*) \\
            f&\mapsto f\circ \gamma. 
        \end{aligned}
    \end{equation}
    \begin{subproof}
        \spitem[\( \xi\) est un morphisme]
        % -------------------------------------------------------------------------------------------- 
         Nous avons
         \begin{equation}
             \xi(fg)=(fg)\circ \gamma=(f\circ \gamma)(g\circ \gamma)=\xi(f)\xi(g).
         \end{equation}
         \spitem[\( \xi\) est injective]
         % -------------------------------------------------------------------------------------------- 
         Nous nous souvenons de l'application quotient \( \tilde \gamma\colon S^1\to \Gamma\), qui est un homéomorphisme par le lemme \ref{LEMooCGVOooVPlSRD}\ref{ITEMooWKVAooCQDvpL}. Nous montrons que ceci est l'inverse de \( \xi\) :
         \begin{equation}
             \begin{aligned}
                 \theta\colon \aL(J,\eC^*)&\to \mG(\Gamma) \\
                 \alpha&\mapsto \tilde \lambda\circ\tilde \gamma^{-1}. 
             \end{aligned}
         \end{equation}
         <++>
    \end{subproof}
    <++>
\end{proof}
<++>


\begin{theorem}[Théorème de Jordan\cite{ooTXKNooIgJrPw, HDJTbua}]\label{ThoHSPWBuh}
	Le complémentaire d'une courbe de Jordan \( \Gamma\) dans un plan affine réel est formé de exactement deux composantes connexes distinctes, dont l'une est bornée et l'autre non. Toutes deux ont pour frontière la courbe \( \Gamma\).
\end{theorem}
\index{théorème!de Jordan}
% TODOooBBSQooKwbHLJ
% Si un jour on travaille sur ce théorème, il y a moyen de revoir la réponse de Alphago dans
% https://math.stackexchange.com/questions/1727310/convex-curve-as-boundary-of-a-convex-set
