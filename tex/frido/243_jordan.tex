% This is part of Le Frido
% Copyright (c) 2010-2016,2018-2020, 2022-2023
%   Laurent Claessens, Carlotta Donadello
% See the file fdl-1.3.txt for copying conditions.



%---------------------------------------------------------------------------------------------------------------------------
\subsection{Le théorème de Jordan}
%---------------------------------------------------------------------------------------------------------------------------

Les définitions de chemins, de lacets et de courbes de Jordan sont dans \ref{DEFooQZMSooYYkGDv}.

\begin{lemmaDef}[\cite{BIBooQKARooMHqitK}]     \label{LEMooZPRLooJPvrOE}
	Soit un lacet \( \gamma\colon \mathopen[ a , b \mathclose]\to X\). Nous considérons l'application
	\begin{equation}
		\begin{aligned}
			q\colon \mathopen[ a , b \mathclose] & \to S^1                    \\
			t                                    & \mapsto  e^{2i\pi t/(b-a)}
		\end{aligned}
	\end{equation}
	Il existe une unique application continue \( \tilde \gamma\colon S^1\to \Image(\gamma)\) telle que \( \gamma=\tilde \gamma\circ q\).

	Cette application \( \tilde \gamma\) est l'application \defe{quotient}{quotient d'un chemin} associée à \( \gamma\).
\end{lemmaDef}

\begin{proof}
	La proposition \ref{PROPooXELTooYKjDav}\ref{ITEMooOHRHooRXvxrL} dit que \( q\) est une bijection continue. Donc l'existence et l'unicité d'une application \( \tilde \gamma=\gamma\circ q^{-1}\). Cette application est continue comme composée d'applications continues.
\end{proof}

\begin{lemma}[\cite{BIBooQKARooMHqitK}]     \label{LEMooCGVOooVPlSRD}
	Soit \( \gamma\colon \mathopen[ a , b \mathclose]\to X\) un chemin d'image \( \Gamma\).
	\begin{enumerate}
		\item       \label{ITEMooWKVAooCQDvpL}
		      Si \( \gamma\) est un chemin de Jordan\footnote{Définition \ref{DEFooQZMSooYYkGDv}.}, alors \( \gamma\colon \mathopen[ a , b \mathclose]\to \Gamma\) est un isomorphisme d'espaces topologiques.
		\item       \label{ITEMooVYMXooEtgPJT}
		      Si \( \gamma\) est un lacet de Jordan, alors le quotient\footnote{Définition \ref{LEMooZPRLooJPvrOE}.} \( \tilde \gamma\colon S^1\to \Gamma\) est un homéomorphisme\footnote{Bijection continue d'inverse continu.}.
	\end{enumerate}
\end{lemma}

\begin{proof}
	En deux parties.
	\begin{subproof}
		\spitem[Pour \ref{ITEMooWKVAooCQDvpL}]
		% -------------------------------------------------------------------------------------------- 
		Un chemin de Jordan est injectif, et donc bijectif sur son image. Il est donc une bijection continue depuis \( \mathopen[ a , b \mathclose]\) qui est compact. Il est donc un homéomorphisme par le lemme \ref{LEMooNEEVooSeHYzx}.
		\spitem[Pour \ref{ITEMooVYMXooEtgPJT}]
		% -------------------------------------------------------------------------------------------- 
		L'application \( \gamma\colon \mathopen[ a , b \mathclose[\to \Gamma \) est une bijection continue parce que \( \gamma\) est un lacet de Jordan. Son quotient \( \tilde \gamma\colon S^1\to \Gamma\) est donc une bijection continue depuis le compact \( S^1\). Le lemme \ref{LEMooNEEVooSeHYzx} conclut que c'est un homéomorphisme.
	\end{subproof}
\end{proof}

\begin{corollary}		\label{CORooPGFLooUTVZMi}
	À propos de courbes simples et de Jordan.
	\begin{enumerate}
		\item
		      Toute courbe simple est homéomorphe à \( \mathopen[ 0 , 1 \mathclose]\).
		\item		\label{ITEMooIUAXooOfvNov}
		      Toute courbe de Jordan est homéomorphe à \( S^1\).
	\end{enumerate}
\end{corollary}

\begin{proof}
	Une courbe simple est l'image d'un lacet de Jordan. Donc \( \gamma\colon \mathopen[ a , b \mathclose]\to \Gamma\) est un homéomorphisme par le lemme \ref{LEMooCGVOooVPlSRD}\ref{ITEMooWKVAooCQDvpL}. Le lemme \ref{LEMooAJDLooIPcmIV} fournit un homéomorphisme entre \( \mathopen[ a , b \mathclose]\) et \( \mathopen[ 0 , 1 \mathclose]\). Une composition d'homéomorphismes est un homéomorphisme.

	Une courbe de Jordan est l'image d'un lacet de Jordan \( \gamma\colon \mathopen[ a , b \mathclose]\to \Gamma\). Le lemme \ref{LEMooCGVOooVPlSRD}\ref{ITEMooVYMXooEtgPJT} dit alors que \( \Gamma\) est homéomorphe à \( S^1\).
\end{proof}


\begin{definition}[\cite{BIBooQKARooMHqitK}]        \label{DEFooURFMooXIaRkl}
	Soit un compact \( K\) dans \( \eC\). Nous notons
	\begin{enumerate}
		\item
		      \( \mG(K)\) le groupe multiplicatif des fonctions continues \( f\colon K\to \eC^*\).
		\item
		      \( \mE(K)\) le sous-groupe des fonctions admettant un logarithme continu\footnote{Logarithme continu, définition \ref{DEFooBBGFooCEdsFR}.}.
		\item
		      \( G(K)\) le quotient\footnote{Le groupe \( \mG(K)\) est commutatif; donc pas de problèmes pour que \( \mE(K)\) soit normal.} \( G(K)=\mG(K)/\mE(K)\).
	\end{enumerate}
\end{definition}

\begin{lemma}[\cite{BIBooQKARooMHqitK}]     \label{LEMooHEOWooHTtHsJ}
	Si \( K_1\) et \( K_2\) sont des compacts homéomorphes, alors les groupes \( G(K_1)\) et \( G(K_2)\) sont isomorphes.
\end{lemma}

\begin{proof}
	Soit un homéomorphisme \( \theta\colon K_1\to K_2\). Nous posons
	\begin{equation}
		\begin{aligned}
			\psi\colon \mG(K_1) & \to \mG(K_2)                \\
			f                   & \mapsto f\circ \theta^{-1}.
		\end{aligned}
	\end{equation}
	\begin{subproof}
		\spitem[Injective]
		% -------------------------------------------------------------------------------------------- 
		Suppose \( \psi(f)=\psi(g)\). Vu que \( \theta\) est une bijection, l'égalité \( f\circ \theta^{-1}=g\circ\theta^{-1}\) implique \( f=g\). Donc \( \psi\) est injective.
		\spitem[Surjective]
		% -------------------------------------------------------------------------------------------- 
		Soit \( f\in\mG(K_2)\). Nous avons \( f\circ\theta\in\mG(K_1)\), et évidemment \( \psi(f\circ\theta)=f\).
		\spitem[Morphisme]
		% -------------------------------------------------------------------------------------------- 
		Nous avons
		\begin{equation}
			\psi(fg)=fg\circ\theta^{-1}=(f\circ\theta^{-1})(g\circ\theta^{-1})=\psi(f)\psi(g).
		\end{equation}
	\end{subproof}
	Nous avons prouvé que \( \psi\) est un isomorphisme. Nous devons maintenant voir qu'il passe au quotient, c'est-à-dire que \( \psi\big( \mE(K_1) \big)\subset\mE(K_2)\). Soit \( f\in\mE(K_1)\), soit un logarithme continu \( g\) de \( f\), c'est-à-dire \( \exp\big( g(x) \big)=f(x)\). Pour \( y\in K_2\), il existe \( x\in K_1\) tel que \( y=\theta(x)\). Dans ce cas nous avons
	\begin{equation}
		\exp\big( (g\circ\theta^{-1})(y) \big)=\exp\big( g(x) \big)=f(x)=(f\circ\theta^{-1})(y).
	\end{equation}
	Autrement dit,
	\begin{equation}
		\exp\big( \psi(g)(y) \big)=\psi(f).
	\end{equation}
	Donc \( \psi(g)\) est un logarithme continu de \( \psi(f)\).
\end{proof}

\begin{definition}		\label{DEFooKXHUooTsKuOe}
	Petite notation. Si un compact \( K\) est donné, pour \( p\in\eC\setminus K\), nous notons
	\begin{equation}
		\begin{aligned}
			f_p\colon K & \to \eC^*    \\
			z           & \mapsto z-p.
		\end{aligned}
	\end{equation}
\end{definition}

\begin{lemma}		\label{LEMooSULYooWiEoyf}
	Nous considérons l'application
	\begin{equation}
		\begin{aligned}
			\phi\colon  \eC & \to \eC                  \\
			x               & \mapsto \lambda x+\alpha
		\end{aligned}
	\end{equation}
	avec \( \lambda\neq 0\) (\( \lambda\in \eC\)) et \( \alpha\in \eC\).

	Soit \( q\in \eC\). Nous avons\footnote{L'application \( f_q\) est celle de la définition \ref{DEFooKXHUooTsKuOe}.}
	\begin{equation}
		f_q\big( \phi(x) \big)=\lambda f_{\phi^{-1}(q)}(x).
	\end{equation}
\end{lemma}

\begin{proof}
	En ce qui concerne la multiplication nous avons $f_q(\lambda x)=\lambda x-q$, et donc
	\begin{equation}
		\frac{1}{ \lambda}f_q(\lambda x)=x-\frac{ q }{ \lambda }=f_{q/\lambda}(x).
	\end{equation}
	Nous retenons : \( f_q(\lambda x)=\lambda f_{q/\lambda}(x)\). En ce qui concerne la translation, c'est plus facile :
	\begin{equation}
		f_q(x+\alpha)=x+\alpha-q=f_{q-\alpha}(x).
	\end{equation}
	Lorsque nous faisons les deux en même temps,
	\begin{equation}
		f_q\big( \phi(x) \big)=f_{q-\alpha}(\lambda x)=\lambda f_{(q-\alpha)/\lambda}(x).
	\end{equation}
	Il suffit maintenant de remarquer que \( \phi^{-1}(q)=\frac{ q-\alpha }{ \lambda }\), ce qui est facilement vérifié :
	\begin{equation}
		\phi\left( \frac{ q-\alpha }{ \lambda } \right)=\lambda\left( \frac{ q-\alpha }{ \lambda } \right)+\alpha=q.
	\end{equation}
\end{proof}

\begin{lemma}[\cite{BIBooQKARooMHqitK}]     \label{LEMooBZUCooHWfolf}
	Soit un lacet de Jordan\footnote{Définition \ref{DEFooQZMSooYYkGDv}.} \( \gamma\colon \mathopen[ a , b \mathclose]\to \eC\). Nous notons \( \Gamma\) son image.
	\begin{enumerate}
		\item
		      \( \Gamma\) est compact.
		\item
		      L'application\footnote{Rappel définitions \ref{DEFooURFMooXIaRkl} pour \( \mG(\Gamma)\) et \( \mE(\Gamma)\).}
		      \begin{equation}
			      \begin{aligned}
				      I_{\gamma}\colon \mG(\Gamma) & \to \eZ                        \\
				      f                            & \mapsto \Ind(f\circ \gamma, 0)
			      \end{aligned}
		      \end{equation}
		      est un morphisme surjectif de noyau \( \mE(\Gamma)\).
	\end{enumerate}
\end{lemma}

\begin{proof}
	Nous notons \( \mL(J,\eC^*)\) le groupe multiplicatif des lacets \( J\to \eC^*\). Le fait que \( \Gamma\) soit compact est parce qu'il est l'image du compact \( \mathopen[ a , b \mathclose]\) par l'application continue \( \gamma\) (théorème \ref{ThoImCompCotComp}).

	Nous considérons l'application
	\begin{equation}
		\begin{aligned}
			\xi\colon \mG(\Gamma) & \to \aL(J,\eC^*)       \\
			f                     & \mapsto f\circ \gamma.
		\end{aligned}
	\end{equation}
	\begin{subproof}
		\spitem[\( \xi\) est un morphisme]
		% -------------------------------------------------------------------------------------------- 
		Nous avons
		\begin{equation}
			\xi(fg)=(fg)\circ \gamma=(f\circ \gamma)(g\circ \gamma)=\xi(f)\xi(g).
		\end{equation}
		\spitem[\( \xi\) est injective]
		% -------------------------------------------------------------------------------------------- 
		Nous nous souvenons de l'application quotient \( \tilde \gamma\colon S^1\to \Gamma\), qui est un homéomorphisme par le lemme \ref{LEMooCGVOooVPlSRD}\ref{ITEMooWKVAooCQDvpL}. Nous nous souvenons également de l'application
		\begin{equation}
			\begin{aligned}
				q\colon \mathopen[ a , b \mathclose] & \to \eC                               \\
				t                                    & \mapsto \exp\big( 2i\pi t/(b-a) \big)
			\end{aligned}
		\end{equation}
		qui vérifie \( \gamma=\tilde \gamma\circ q\). Nous montrons que ceci est l'inverse de \( \xi\) :
		\begin{equation}
			\begin{aligned}
				\theta\colon \aL(J,\eC^*) & \to \mG(\Gamma)                                \\
				\lambda                   & \mapsto \tilde \lambda\circ\tilde \gamma^{-1}.
			\end{aligned}
		\end{equation}
		Soit \( \sigma\in\aL(J,\eC^*)\). Nous avons
		\begin{equation}
			\xi\big( \theta(\sigma) \big)=\theta(\sigma)\circ\gamma=\tilde \sigma\circ\tilde \gamma^{-1}\circ\gamma=\tilde \sigma\circ\tilde \gamma^{-1}\circ\tilde \gamma\circ q=\tilde \sigma\circ q=\sigma.
		\end{equation}
		\spitem[Conclusion 1]
		% -------------------------------------------------------------------------------------------- 
		La proposition \ref{PROPooGAOIooFTOuli} donne un morphisme surjectif \( \psi\colon \aL(J,\eC^*)\to \eZ\). L'application \( I_{\gamma}\) dont nous parlons ici est exactement \( I_{\gamma}=\psi\circ\xi\). Donc \( I_{\gamma}\) est un morphisme surjectif.
		\spitem[Noyau]
		% -------------------------------------------------------------------------------------------- 
		Nous devons encore montrer que le noyau de \( I_{\gamma}\) est \( \mE(\Gamma)\).

		\begin{subproof}
			\spitem[Dans un sens]
			% -------------------------------------------------------------------------------------------- 
			Soit \( k\in\ker(I_{\gamma})\). Alors \( (\psi\circ\xi)(k)=0\), c'est-à-dire \( \psi(k\circ\gamma)=0\). La proposition \ref{PROPooGAOIooFTOuli}\ref{ITEMooZFXTooDCXTVU} dit qu'alors \( k\circ\gamma\) est un lacet homotope dans \(\eC^*\) à un lacet constant. Donc il existe une application continue \( H\colon \mathopen[ 0 , 1 \mathclose]\times\mathopen[ a , b \mathclose]\to \eC^*\) telle que
			\begin{subequations}
				\begin{numcases}{}
					H(0,t)=(k\circ\gamma)(t)\\
					H(1,t)=1
				\end{numcases}
			\end{subequations}
			pour tout \( t\). Vu que l'application \( \gamma\colon \mathopen[ a , b \mathclose[\to \eC^*\) est injective, nous pouvons considérer son inverse; pour la suite, quand nous écrirons \( \gamma^{-1}\), c'est bien de \( \gamma^{-1}\colon \Gamma\to \mathopen[ a , b \mathclose[\) que nous parlerons. C'est une application continue par le lemme \ref{LEMooKSDKooDbKKeB}. Nous pouvons donc considérer
			\begin{equation}
				\begin{aligned}
					L\colon \mathopen[ 0 , 1 \mathclose]\times\Gamma & \to \eC^*                              \\
					(u,z)                                            & \mapsto H\big( u,\gamma^{-1}(z) \big).
				\end{aligned}
			\end{equation}
			Cela est une application continue parce que \( H\) et \( \gamma\) le sont. Nous avons
			\begin{equation}
				L(0,z)=H\big( 0,\gamma^{-1}(z) \big)=(k\circ\gamma)\big( \gamma^{-1}(z) \big)=k(z),
			\end{equation}
			et
			\begin{equation}
				L(1,z)=1.
			\end{equation}
			Donc \( k\) est homotope à une application constante sur \( \Gamma\). Le théorème de Borsuk \ref{THOooTCUMooEByCKg} dit alors que \( k\) admet un logarithme continu sur \( \Gamma\), c'est-à-dire que \( k\in\mE(\Gamma)\).
			\spitem[Dans l'autre sens]
			% -------------------------------------------------------------------------------------------- 
			C'est le même raisonnement que le premier, mais un peu plus simple parce qu'il n'y a pas de débats sur la continuité. Soit \( k\in \mE(\Gamma)\). Le théorème de Boruk indique que \( k\) est homotope dans \( \eC^*\) à une application constante. Il existe une application continue \( H\colon \mathopen[ 0 , 1 \mathclose]\times \Gamma\to \eC^*\) telle que
			\begin{subequations}
				\begin{numcases}{}
					H(0,z)=k(z)\\
					H(1,z)=1
				\end{numcases}
			\end{subequations}
			pour tout \( z\in \Gamma\). Le chemin \( k\circ\gamma\) est homotope à un lacet constant. En effet nous avons l'application continue
			\begin{equation}
				\begin{aligned}
					L\colon \mathopen[ 0 , 1 \mathclose]\times\mathopen[ a , b \mathclose] & \to \eC^*                        \\
					(u,t)                                                                  & \mapsto H\big( u,\gamma(t) \big)
				\end{aligned}
			\end{equation}
			qui vérifie \( L(0,t)=(k\circ\gamma)(t)\) et \( L(1,t)=1\). Nous avons donc
			\begin{equation}
				0=\psi(k\circ\gamma)=I_{\gamma}(k).
			\end{equation}
			Nous avons prouvé que \( k\in\ker(I_{\gamma})\).
		\end{subproof}
	\end{subproof}
\end{proof}

\begin{lemma}[\cite{BIBooQKARooMHqitK}]     \label{LEMooODIPooBZJPAW}
	Soit un lacet de Jordan \( \gamma\) dont l'image est \( \Gamma\). Des fonctions \( f_0,f_1\in\mG(\Gamma)\) sont homotopes dans \( \eC^*\) si et seulement si les lacets \( f_0\circ \gamma\) et \( f_1\circ \gamma\) le sont.
\end{lemma}

\begin{proof}
	Dans les deux sens.
	\begin{subproof}
		\spitem[\( \Rightarrow\)]
		% -------------------------------------------------------------------------------------------- 
		Nous supposons que \( f_0\) et \( f_1\) sont homotopes : il existe une application continue \( H\colon \mathopen[ 0 , 1 \mathclose]\times \Gamma\to \eC^*\) telle que
		\begin{subequations}
			\begin{numcases}{}
				H(0,z)=f_0(z)\\
				H(1,z)=f_1(z)
			\end{numcases}
		\end{subequations}
		pour tout \( z\in \Gamma\). L'homotopie entre \( f_0\circ\gamma\) et \( f_1\circ\gamma\) est donnée par l'application continue
		\begin{equation}
			\begin{aligned}
				L\colon \mathopen[ 0 , 1 \mathclose]\times\mathopen[ a , b \mathclose] & \to \eC^*                         \\
				(u,t)                                                                  & \mapsto H\big( u,\gamma(t) \big).
			\end{aligned}
		\end{equation}
		Elle vérifie \( L(0,t)=(f_0\circ \gamma)(t)\) et \( L(1,t)=(f_1\circ\gamma)(t)\).
		\spitem[\( \Leftarrow\)]
		% -------------------------------------------------------------------------------------------- 
		Nous supposons l'existence d'une application continue \( H\colon \mathopen[ 0 , 1 \mathclose]\times \mathopen[ a , b \mathclose]\to \eC^*\) telle que \( H(0,t)=(f_0\circ\gamma)(t)\) et \( H(1,t)=(f_1\circ \gamma)(t)\).

		L'application \( \gamma\colon \mathopen[ a , b \mathclose[\to \Gamma\) est continue et bijective. Sa réciproque \( \gamma^{-1}\) est donc également continue (lemme \ref{LEMooKSDKooDbKKeB}). Nous pouvons donc considérer l'application continue
		\begin{equation}
			\begin{aligned}
				L\colon \mathopen[ 0 , 1 \mathclose]\times \Gamma & \to \eC^*                              \\
				(u,z)                                             & \mapsto H\big( u,\gamma^{-1}(z) \big).
			\end{aligned}
		\end{equation}
		Elle vérifie
		\begin{equation}
			L(0,z)=H\big( 0,\gamma^{-1}(z) \big)=(f_0\circ \gamma)\big( \gamma^{-1}(z) \big)=f_0(z)
		\end{equation}
		et \( L(1,z)=f_1(z)\). Donc \( L\) est une homotopie dans \( C^*\) entre \( f_0\) et \( f_1\).
	\end{subproof}
\end{proof}

\begin{corollary}[\cite{BIBooQKARooMHqitK}]     \label{CORooDOMAooHtKPTe}
	Si \( \Gamma\) est une courbe de Jordan, alors\footnote{Rappel : \( G(\Gamma)\) est défini en \ref{DEFooURFMooXIaRkl}.} \( G(\Gamma)\simeq \eZ \).
\end{corollary}

\begin{proof}
	Nous allons utiliser le premier théorème d'isomorphisme \ref{ThoPremierthoisomo}. Si \( \gamma\) est un lacet de Jordan dont la courbe est \( \Gamma\), le lemme \ref{LEMooBZUCooHWfolf} dit que \( I_{\gamma}\colon \mG(\Gamma)\to \eZ\) est un morphisme surjectif tel que \( \ker(I_{\gamma})=\mE(\Gamma)\). Le théorème d'isomorphisme dit alors que
	\begin{equation}
		\frac{\mG(\Gamma)}{ \mE(\Gamma) }=I_{\gamma}\big( \mG(\Gamma) \big)=\eZ.
	\end{equation}
\end{proof}

Rappel, si \( K\) est compact dans \( \eC\), et si \( p\in \eC\), nous considérons la fonction \( f_p\in\mG(K)\) par \( f_p(z)=z-p\).

\begin{lemma}[\cite{MonCerveau}]        \label{LEMooAKVFooFMHaOZ}
	Soient une courbe de Jordan \( \gamma\) ainsi que \( p\in \eC\). Nous avons
	\begin{equation}
		\Ind(f_p\circ \gamma,0)=\Ind(\gamma,p).
	\end{equation}
\end{lemma}

\begin{proposition}[\cite{BIBooQKARooMHqitK}]       \label{PROPooUPPEooEQOOkh}
	Soit \( p\in \eC\).  Si \( V\) est un disque ou un rectangle de centre \( p\), alors \( \mG(\partial V)\) est engendré par \( \mE(\partial V)\) et par \( f_p\).
\end{proposition}

\begin{proof}
	Ici \( \partial V\) peut être vu comme un chemin que nous allons noter aussi \( \gamma\). Utilisant le lemme \ref{LEMooAKVFooFMHaOZ}, nous avons
	\begin{equation}        \label{EQooRSXRooTJwDNP}
		I_{\partial V}(f_p)=\Ind(f_p\circ\partial V,0)=\Ind(\partial V,p).
	\end{equation}

	\begin{subproof}
		\spitem[Si \( V\) est un disque]
		% -------------------------------------------------------------------------------------------- 
		Alors \( \partial V\) est le chemin
		\begin{equation}
			\begin{aligned}
				\gamma\colon \mathopen[ 0 , 2\pi \mathclose] & \to \eC             \\
				t                                            & \mapsto p+R e^{it}.
			\end{aligned}
		\end{equation}
		Nous avons alors le calcul suivant :
		\begin{subequations}
			\begin{align}
				\Ind(\gamma,p) & =\frac{1}{ 2\pi i }\int_0^{2\pi}\frac{ \gamma'(t) }{ \gamma(t)-p }dt \\
				               & =\frac{1}{ 2\pi i }\int_0^{2\pi}\frac{ Ri e^{it} }{ R e^{it} }dt     \\
				               & =\frac{1}{ 2\pi i }\int_0^{2\pi}idt                                  \\
				               & =1.
			\end{align}
		\end{subequations}
		\spitem[Si \( V\) est un rectangle]
		% -------------------------------------------------------------------------------------------- 
		Supposons que \( V\) ait une hauteur de \( h\) et une largeur de \( l\) tout en étant centré en \( p\). Le chemin \( \partial V\) se décompose en quatre partie dont la première est
		\begin{equation}
			\begin{aligned}
				\gamma_1\colon \mathopen[ 0 , l \mathclose] & \to \eC\setminus \{ p \}                   \\
				t                                           & \mapsto p-\frac{ hi }{ 2 }-\frac{ l }{2}+t
			\end{aligned}
		\end{equation}
		Je vous laisse voir les autres parties et calculer les intégrales\quext{Je n'ai pas vérifié; si ça pose un problème, écrivez-moi.}

	\end{subproof}
	Dans les deux cas nous avons \( \Ind(\partial V, p)=1\) et donc, en continuant \eqref{EQooRSXRooTJwDNP},
	\begin{equation}
		I_{\partial V}(f_p)=1.
	\end{equation}

	Nous savons par le lemme \ref{LEMooBZUCooHWfolf} que
	\begin{equation}
		\begin{aligned}
			I_{\partial V}\colon \mG(\partial V) & \to \eZ                       \\
			f                                    & \mapsto \Ind(f\circ \gamma,0)
		\end{aligned}
	\end{equation}
	est un morphisme surjectif de noyau \( \mE(\Gamma)\). Le fait que ce soit un morphisme implique que
	\begin{equation}
		I_{\partial V}(f_p^n)=nI_{\partial V}(f_p)=n.
	\end{equation}
	Mais \( G(\partial V)=\mG(\partial V)/\mE(\partial V)\). Donc nous avons déjà prouvé que \( \{ f_p^k \}_{k\in \eN}\) contient un élément de chaque classe dans \( \mG(\partial V)/\mE(\partial V)\).

	Soit \( f\in \mG(\partial V)\). En notant \( [.]\) la classe par rapport à \( \mE(\partial V)\), il existe \( k\in \eN\) tel que
	\begin{equation}
		f\in\big[ I_{\partial V}(f_p^k) \big].
	\end{equation}
	Autrement dit, en posant \( \alpha=I_{\partial V}(f_p^k)\), il existe \( h\in \mE(\partial V)\) tel que \( f=\alpha h\). Nous avons donc bien prouvé que \( \{ f_p,\mE(\partial V) \}\) engendre \( \mG(\partial V)\).
\end{proof}

\begin{proposition}[\cite{BIBooQKARooMHqitK}]       \label{PROPooJNZQooLWDKww}
	Soit un compact \( K\) de \( \eC\) ainsi que \( p,q\) deux points dans la même composante connexe de \( \eC\setminus K\). Alors
	\begin{enumerate}
		\item       \label{ITEMooQMKOooQHhUGn}
		      \( f_p/f_q\in\mE(K)\).
		\item       \label{ITEMooUJLPooApgXIF}
		      Si \( p\) est dans la composante non bornée\footnote{Il y en a une seule, c'est le lemme \ref{LEMooJNPTooScfSvA}.} de \( \eC\setminus K\), alors \( f_p\in\mE(K)\).
	\end{enumerate}
\end{proposition}

\begin{proof}
	Soient \( p\) et \( q\) dans la même composante connexe de \( \eC\setminus K\). La proposition \ref{PROPooYFDBooHbBjzF} nous autorise à considérer un chemin \( \gamma\colon \mathopen[ 0 , 1 \mathclose]\to \eC\setminus K\) tel que \( \gamma(0)=p\) et \( \gamma(1)=q\). Nous posons
	\begin{equation}
		\begin{aligned}
			F\colon \mathopen[ 0 , 1 \mathclose]\times K & \to \eC^*                             \\
			(t,z)                                        & \mapsto f_{\gamma(t)}(z)=z-\gamma(t).
		\end{aligned}
	\end{equation}
	C'est une application continue vérifiant \( F(0,z)=f_p(z)\) et \( F(1,z)=f_q(z)\). Donc les applications \( f_p\) et \( f_q\) sont homotopes dans \( \eC^*\). La proposition \ref{PROPooNABDooFtKukO} nous indique qu'alors l'application \( f_p/f_q\) est homotope à \( 1\). Le théorème de Borsuk \ref{THOooTCUMooEByCKg} conclut que \( f_p/f_q\in\mE(K)\).

	Soit \( p\) dans la composante non bornée de \( \eC\setminus K\). Vu que \( K\) est borné, nous pouvons prendre \( r>0\) tel que \( K\subset B(0,r)\). Soit \( p_0\notin \overline{ B(0,r) }\); ce point est également dans la composante non bornée de \( K\). Par le point \ref{ITEMooQMKOooQHhUGn}, nous avons \( f_p/f_{p_0}\in\mE(K)\).

	Nous allons prouver que \( f_{p_0}\in \mE(K)\). Pour cela nous posons
	\begin{equation}
		\begin{aligned}
			F\colon \mathopen[ 0 , 1 \mathclose]\times K & \to \eC^*       \\
			(t,z)                                        & \mapsto tz-p_0.
		\end{aligned}
	\end{equation}
	Vu que \( t\in \mathopen[ 0 , 1 \mathclose]\) et que \( z\in K\subset B(0,r)\), nous avons toujours \( tz\in B(0,r)\). Étant donné que \( p_0\) n'est pas dans \( B(0,r)\), l'application \( F\) prend bien ses valeurs dans \( \eC^*\).

	L'application \( F\) vérifie également \( F(0,z)=-p_0\) et \( F(1,z)=z-p_0=f_{p_0}(z)\). Donc \( f_{p_0}\) est homotope dans \( \eC^*\) à l'application constante \( -p_0\). Nous en déduisons que \( f_{p_0}\in\mE(K)\). Au final,
	\begin{equation}
		f_p=f_{p_0}\frac{ f_p }{ f_{p_0} }\in \mE(K)
	\end{equation}
	parce que \( f_{p_0}\) et \( f_p/f_{p_0}\) sont dans \( \mE(K)\).
\end{proof}

\begin{lemma}[\cite{BIBooQKARooMHqitK}]     \label{LEMooOTMGooZeUoCy}
	Soient des composantes connexes bornées \( \mO_1\), \ldots, \( \mO_N\) de \( \eC\setminus K\) que nous supposons deux à deux disjointes. Soient \( p_i\in\mO_i\). Si \( n_1,\ldots, n_N\in \eZ\) sont des entiers non tous nuls, alors
	\begin{equation}
		\prod_{i=1}^Nf_{p_i}^{n_i}\notin\mE(K).
	\end{equation}
\end{lemma}

\begin{proof}
	En plusieurs points.
	\begin{subproof}
		\spitem[Par l'absurde]
		% -------------------------------------------------------------------------------------------- 
		Nous notons \( f=\prod_{i=1}^Nf_{p_i}^{n_i}\) et nous supposons par l'absurde que \( f\in\mE(K)\). Par le théorème de Borsuk \ref{THOooTCUMooEByCKg}, nous avons une extension continue \( \tilde f\colon \eC \to \eC^*\).
		\spitem[\( \partial\mO_i\subset K\)]
		% -------------------------------------------------------------------------------------------- 
		C'est le lemme \ref{LEMooWGOCooHSoCzb}.
		\spitem[\( p_j\) n'est pas dans \( \bar\mO_i\)]
		% -------------------------------------------------------------------------------------------- 
		Nous savons que \( p_j\in\mO_j\). Mais \( \partial\mO_i\subset K\) et \( K\cap\mO_j=\emptyset\). Donc \( p_j\) n'est pas dans \( \partial\mO_i\). Comme les \( \mO_i\) sont disjoints, nous avons aussi \( p_j\notin \mO_i\). Bref,
		\begin{equation}
			p_j\notin\bar\mO_i.
		\end{equation}
		\spitem[Quelques fonctions]
		% -------------------------------------------------------------------------------------------- 
		Nous posons
		\begin{equation}
			\begin{aligned}
				G_i\colon \bar\mO_i & \to \eC^*                             \\
				z                   & \mapsto \prod_{j\neq i}(z-p_j)^{n_j}.
			\end{aligned}
		\end{equation}
		Le fait que \( G_i\) prenne ses valeurs dans \( \eC^*\) est parce que \( p_j\notin \bar\mO_i\) pour tout \( i\neq j\).

		Nous reprenons notre extension continue \( \tilde f\colon \eC\to \eC^*\) et nous posons
		\begin{equation}
			\begin{aligned}
				F_i\colon \bar\mO_i & \to \eC^*                               \\
				z                   & \mapsto \frac{ \tilde f(z) }{ G_i(z) }.
			\end{aligned}
		\end{equation}
		Vu que \( G_i\) ne s'annule pas sur \( \bar\mO_i\), cette fonction \( F_i\) est bien définie et continue.

		Nous nous fixons \( i\), et nous allons prouver que \( n_i=0\). Nous posons \( D=B(0,R)\) avec \( R\) assez grand pour contenir \( \bar\mO_i\). Posons
		\begin{equation}
			\begin{aligned}
				H_i\colon \overline{ B(0,R) } & \to \eC^*                                        \\
				z                             & \mapsto \begin{cases}
					                                        F_i(z)        & \text{si } z\in\bar\mO_i \\
					                                        (z-p_i)^{n_i} & \text{sinon. }
				                                        \end{cases}
			\end{aligned}
		\end{equation}
		\spitem[\( H_i\) est continue]
		% -------------------------------------------------------------------------------------------- 
		L'application \( F_i\) est continue sur \( \bar\mO_i\). L'application \( z\mapsto (z-p_i)^{n_i}\) est continue sur \( \eC\).

		Soit \( z\in\partial\mO_i\). Nous avons \( z\in\partial\mO_i\subset K\) et donc \( \tilde f(z)=f(z)\). Donc
		\begin{equation}
			F_i(z)=\frac{ f(z) }{ G_i(z) }=f_{p_i}(z)^{n_i}.
		\end{equation}
		Les deux applications \( F_i\) et \( z\mapsto (z-p_i)^{n_i}\) sont égales sur le bord \( \partial\mO_i\). Donc \( H_i\) est continue sur \( \overline{ B(0,R) }\).
		\spitem[Logarithme continu]
		% -------------------------------------------------------------------------------------------- 
		La partie \( \overline{ B(0,R) }\) est convexe et compacte. Le corolaire \ref{CORooXOOZooUJMKxu} dit que \( H_i\) admet un logarithme continu. La proposition \ref{PROPooCFMFooXjlhfV} nous affirme que pour tout lacet \( \gamma\colon \mathopen[ 0 , 1 \mathclose]\to \overline{ B(0,R) }\), nous avons \( \Ind(H_i\circ \gamma,0)=0\). En particulier le chemin \( \partial B(0,R)\) est dans \( \overline{ B(0,R) }\). Avant de calculer \( \Ind(H_i\circ\gamma,0)\), remarquons que \( \gamma(t)\in\overline{ B(0,R) }\) et reste donc en-dehors de \( \bar\mO_i\). Donc
		\begin{equation}
			H_i\big( \gamma(t) \big)=\big( \gamma(t)-p_i \big)^{n_i}
		\end{equation}
		et
		\begin{equation}
			(H_i\circ\gamma)'(t)=n_i\gamma'(t)\big( \gamma(t)-p_i \big)^{n_i-1}.
		\end{equation}
		De plus si vous voulez une forme explicite de \( \gamma\), c'est pas compliqué : c'est un cercle de rayon \( R\). Nous avons un calcul :
		\begin{subequations}
			\begin{align}
				0= \Ind(H_i\circ\gamma,0) & =\frac{1}{ 2\pi i }\int_{H_i\circ\gamma}\frac{ dz }{ z }                                                               \\
				                          & =\frac{1}{ 2\pi i }\int_0^1\frac{ (H_i\circ\gamma)'(t) }{ (H_i\circ\gamma)(t) }                                        \\
				                          & =\frac{1}{ 2\pi i  }\int_0^1\frac{ n_i\gamma'(t)\big( \gamma(t)-p_i \big)^{n_i-1} }{ \big( \gamma(t)-p_i \big)^{n_i} } \\
				                          & =\frac{1}{ 2\pi i }\int_0^1\frac{ n_i\gamma'(t) }{ \gamma(t)-p_i }dt                                                   \\
				                          & =n_i\frac{1}{ 2\pi i }\int_{\gamma}\frac{ dz }{ z-p_i }                                                                \\
				                          & =n_i\Ind(\gamma,p_i).
			\end{align}
		\end{subequations}
		Vu que \( p_i\in B(0,R)\) et que \( \gamma\) est un cercle autour, nous avons \( \Ind(\gamma,p_i)=1\). Donc
		\begin{equation}
			0=\Ind(H_i\circ \gamma,0)=n_i.
		\end{equation}
	\end{subproof}
\end{proof}

\begin{lemma}[\cite{BIBooQKARooMHqitK,MonCerveau}]     \label{LEMooEJRMooNJhMov}
	Nous notons \( (\mO_i)_{i\in I}\) les composantes connexes bornées de \( \eC\setminus K\). Pour chaque \( i\in I\) nous considérons \( p_i\in\mO_i\). Alors
	\begin{equation}
		\mG(K)=\gr\big( \mE(K),\{ f_{p_i} \}_{i\in I} \big).
	\end{equation}
\end{lemma}

\begin{proof}
	Nous commençons par prouver que si \( q\in\eC\setminus K\), alors
	\begin{equation}
		f_q\in\gr\big( \mE(K),\{ f_{p_i} \}_{i\in I} \big).
	\end{equation}
	Il y a deux possibilités : soit \( q\) est dans la composante non bornée, soit il est dans un des \( \mO_i\).
	\begin{subproof}
		\spitem[Si \( q\) est dans la composante non bornée]
		% -------------------------------------------------------------------------------------------- 
		Alors la proposition \ref{PROPooJNZQooLWDKww}\ref{ITEMooUJLPooApgXIF} dit que \( f_q\in\mE(K)\).
		\spitem[Si \( q\in\mO_i\)]
		% -------------------------------------------------------------------------------------------- 
		Alors la proposition \ref{PROPooJNZQooLWDKww}\ref{ITEMooQMKOooQHhUGn} dit que \( f_q/f_{p_i}\in\mE(K)\). Dans ce cas nous avons
		\begin{equation}
			f_q=f_{p_i}\frac{ f_q }{ f_{p_i} }\in\gr\big( f_{p_i},\mE(K) \big).
		\end{equation}
	\end{subproof}
	Soit \( f\in\mG(K)\). Nous devons prouver l'existence de \( g\in\mE(K)\), \( q_1,\ldots, q_l\in\eC\setminus K\) et \( m_1,\ldots, m_l\in \eZ\) tels que
	\begin{equation}
		f=g\prod_{i=1}^lf_{q_i}^{m_i}.
	\end{equation}

	Nous commençons par supposer que \( K\subset Q=\mathopen[ 0 , 1 \mathclose]\times\mathopen[ 0 , 1 \mathclose]\).
	\begin{subproof}
		\spitem[Les petits carrés]
		% -------------------------------------------------------------------------------------------- 
		Soit \( n\in \eN\). Nous subdivisons le carré \( Q\) en carrés fermés de taille \( 1/n\); il y a \( n^2\) tels carrés. Nous notons \( Q_1,\ldots, Q_l\) ceux qui n'ont pas d'intersection avec \( K\), et nous notons \( A\) l'union des autres.

		Grâce au lemme \ref{LEMooKDGRooMmJqnn}, nous savons que pour tout \( z\in A\), nous avons \( d(z,K)\leq \sqrt{ 2 }/n\).
		\spitem[Prolongement]
		% -------------------------------------------------------------------------------------------- 
		La partie \( K\) est fermée, \( f\) est continue dessus. Le théorème de Tietze \ref{ThoFFQooGvcLzJ} donne un prolongement continu \( F\colon \eC\to \eC\). Nous avons
		\begin{equation}
			K\subset S=\{ z\in \eC\tq F(z)\neq 0 \}.
		\end{equation}
		Vu que \( F\) est continue, \( S\) est ouverte.
		\spitem[Grand \( n\)]
		% -------------------------------------------------------------------------------------------- 
		Nous prouvons que \( n\) peut être choisi assez grand pour que \( F\) ne s'annule pas sur \( A\). La fonction \( F\) est continue et ne s'annule pas sur l'ouvert \( S\) qui contient \( K\). Si \( S=\eC\), alors n'importe quel \( n\) convient.

		Sinon nous considérons
		\begin{equation}
			\begin{aligned}
				\varphi\colon K & \to \mathopen[ 0 , \infty \mathclose[ \\
				z               & \mapsto d(z,S^c).
			\end{aligned}
		\end{equation}
		L'application \( \varphi\) est continue sur le compact \( K\). Elle a donc un minimum global que nous nommons \( s\). Pour tout \( z\in K\) nous avons \( d(z,S^c)\geq s\). Donc pour tout \( z\in K\) nous avons \( B(z,s/2)\subset S\). Nous prenons \( n\) assez grand pour avoir
		\begin{equation}        \label{EQooEWQYooWYsImg}
			\frac{ \sqrt{ 2 } }{ n }<\frac{ s }{ 2 }.
		\end{equation}
		Avec ça nous avons \( A\subset S\). En effet, soit \( z\in A\). Nous avons \( d(z,K)<\sqrt{ 2 }/n<s/2\). Donc il existe \( k\in K\) tel que
		\begin{equation}
			z\in B(k,\frac{ s }{2}).
		\end{equation}
		Comme \( k\in K\), nous avons \( d(k,S^c)=\varphi(k)\geq s\) et donc \( B(k,s/2)\subset S\). Nous avons prouvé que \( z\in S\).

		Bref, nous prenons \( n\) assez grand pour vérifier \eqref{EQooEWQYooWYsImg}, et nous avons alors
		\begin{equation}
			A\subset \{ z\in \eC\tq F(z)\neq 0 \}.
		\end{equation}
		\spitem[Une propriété à prouver]		\label{SPooVQHTooUNukEE}
		% -------------------------------------------------------------------------------------------- 
		Nous allons prouver l'énoncé suivant. Soit \( k=0,\ldots, l\), soit une application continue \( h\colon K\to \eC^*\) admettant un prolongement continu \( F\colon A\cup(Q_1\cup\ldots\cup Q_k)\to \eC^*\). Alors il existe \( m\in \eN\) tel que l'application \( h/f_{q_{k+1}}^m\) admette un prolongement continu \( \tilde F\colon A\cup(Q_1\cup\ldots\cup Q_{k+1})\to \eC^*\).

		\spitem[Note pour \( f\)]
		% -------------------------------------------------------------------------------------------- 

		Jusqu'ici, nous avons prouvé que \( f\) vérifie les hypothèses de la récurrence pour \( k=0\).

		De plus, cette propriété à prouver n'est pas à prouver par récurrence sur \( k\). Le \( k\) est fixé, et nous prouvons la propriété de façon directe.  Une récurrence va venir un peu plus loin.
		\spitem[Preuve de la propriété]
		% -------------------------------------------------------------------------------------------- 
		Soit \( h\colon K\to \eC^*\) continue et admettant un prolongement continu \( F\colon A\cup(Q_1\cup\ldots Q_k)\to \eC^*\). Notons \( B_k=A\cup(Q_1\cup\ldots\cup Q_k)\) et posons
		\begin{equation}
			\begin{aligned}
				g\colon \partial Q_{k+1} & \to \eC^*                                      \\
				z                        & \mapsto \begin{cases}
					                                   F(z) & \text{si } z\in Q_{k+1}\cap B_k \\
					                                   s(z) & \text{sinon. }
				                                   \end{cases}
			\end{aligned}
		\end{equation}
		Voyons ce que \( s\) peut être pour que \( g\) soit continue. La partie \( \partial Q_{k+1}\) est le bord d'un carré et est donc constitué de \( 4\) côtés. La valeur de \( g\) sur certains de ces côtés (ceux qui sont aussi dans \( B_k\)) est prescrite par \( F\). Sur les autres côtés, nous pouvons mettre n'importe quoi pourvu que ce soit égal à \( F\) sur les coins et que \( s(z)\neq 0\). Étant donné que les valeurs de \( F\) sur les coins est non nulle et que \( \eC^*\) est connexe par arcs, c'est possible.

		Nous notons \( q_{k+1}\) le centre de \( Q_{k+1}\). La proposition \ref{PROPooUPPEooEQOOkh} s'applique au compact \( \partial Q_{k+1}\).
		\begin{equation}
			g\in\mG(\partial Q_{k+1})=\gr\big( \mE(\partial Q_{k+1}), f_{q_{k+1}}  \big).
		\end{equation}
		Il existe donc \( \xi\in\mE(\partial Q_{k+1})\) et \( m\in \eZ\) tels que
		\begin{equation}
			g(z)=\xi(z)f_{q_{k+1}}^{m}(z).
		\end{equation}

		Prouvons que \( Q_{k+1}\cap B_k\) est une union de côtés de \( Q_{k+1}\). Nous avons
		\begin{equation}        \label{EQooSPSQooVErzCI}
			Q_{k+1}\cap B_k=(Q_{k+1}\cap A)\cup(Q_{k+1}\cap Q_1)\cup\ldots\cup(Q_{k+1}\cap Q_k).
		\end{equation}
		Nous nous souvenons que \( A=\bigcup_{i=1}^sR_i\) où \( R_i\) sont les carrés qui intersectent \( K\). Donc \( R_i\cap Q_j\) est au maximum les côtés de \( Q_j\). Bref, chacun des termes de l'union à droite de \eqref{EQooSPSQooVErzCI} est une union de côtés de \( Q_{k+1}\).

		L'application \( \xi\in\mE(\partial Q_{k+1}) \) admet un logarithme continu et admet donc une extension continue \( \eC\to \eC^*\) (théorème de Borsuk \ref{THOooTCUMooEByCKg}). Nous considérons cette extension seulement sur \( Q_{k+1}\) et nous l'appelons \(G \colon Q_{k+1}\to \eC^*  \). Nous posons
		\begin{equation}
			\begin{aligned}
				\tilde F\colon A\cup(Q_1\cup\ldots \cup Q_{k+1}) & \to \eC^*                                                                           \\
				z                                                & \mapsto\begin{cases}
					                                                          G(z)                         & \text{si } z\in Q_{k+1}                       \\
					                                                          \frac{F(z)}{(z-q_{k+1})^{m}} & \text{si } z\in A\cup(Q_1\cup\ldots\cup Q_k).
				                                                          \end{cases}
			\end{aligned}
		\end{equation}
		Voyons que cette fonction est continue en montrant que les deux morceaux ont les mêmes valeurs sur l'intersection. Soit \( z\in Q_{k+1}\cap(A\cup Q_1\cup\ldots\cup Q_k)\). Alors \( z\in \partial Q_{k+1}\) parce que nous avons dit que \eqref{EQooSPSQooVErzCI} est dans le bord dans \( Q_{k+1}\). Dans ce cas nous avons
		\begin{subequations}
			\begin{align}
				G(z)=\xi(z) & =\frac{ g(z) }{ f_{q_{k+1}}^{m}(z) } \\
				            & =\frac{F(z)}{f_{q_{k+1}}^{m}(z)},    \\
			\end{align}
		\end{subequations}
		et donc bien la continuité de \( \tilde F\).

		Si \( z\in K\), étant donné que \( K\subset A\) nous avons
		\begin{equation}
			\tilde F(z)=\frac{ F(z) }{ (z-q_{q_{k+1}})^{m}}=\frac{h(z)}{   f_{q_{k+1}}^{m}(z)    },
		\end{equation}
		donc \( \tilde F\) prolonge bien \( h/   f_{q_{k+1}}^{m}      \).

		\spitem[Récurrence]
		%-----------------------------------------------------------
		En appliquant la propriété \ref{SPooVQHTooUNukEE} \( l\) fois, nous construisons une suite d'entiers \( (m_i)_{i=1,\ldots,l}\) et une application continue \(F_l \colon A\cup(Q_1\cup\ldots\cup Q_l)=Q\to \eC^*  \) qui prolonge \( f/\prod_{i=1}^lf_{q_i}^{m_i}\).

		Sur \( K\) nous avons alors
		\begin{equation}
			f=F_l\prod_{i=1}^lf_{q_i}^{m_i}.
		\end{equation}
		C'est une application continue sur le convexe \( Q\). Donc \( F_l\in\mE(Q)\) et donc \( F_l\in\mE(K)\).
	\end{subproof}
	Maintenant vous croyez avoir fini mais non. Vous devez encore vérifier que tout est encore valable si \( K\) n'est pas contenu dans \( [0 ,1]\times [0 ,1]\).

	Soit un compact général \( K\subset \eC\). Pour \( \lambda>0\) et \( \alpha\in \eC\), nous posons \( \phi(x)=\lambda x+\alpha\). Il existe forcément un choix de \( \lambda\) et de \( \alpha\) tels que \( \phi(K)\subset \mathopen[ 0,1\mathclose]\times i\mathopen[ 0,1\mathclose]\).

	Soit \( f\in \mG(K)\). Nous appliquons le résultat déjà prouvé à l'application \( f\circ\phi^{-1}\in \mG\big( \phi(K) \big)\). Nous avons \( g\in\mE\big( \phi(K) \big)\), \( q_1,\ldots,q_l\in\eC\setminus\phi(K)\) et \( m_1,\ldots,m_l\in \eZ\) tels que, pour tout \( y\in\phi(K)\),
	\begin{equation}
		(f\circ\phi^{-1})(y)=g(y)\prod_{i=1}^lf_{q_i}^{m_i}(y).
	\end{equation}
	Pour \( x\in K\) nous avons donc
	\begin{equation}
		f(x)=(g\circ\phi)(x)\prod_{i=1}^lf_{q_i}^{m_i}\big( \phi(x) \big).
	\end{equation}
	C'est le moment de sortir le lemme \ref{LEMooSULYooWiEoyf} :
	\begin{equation}
		f(x)=(g\circ\phi)(x)\prod_{i=1}^lf_{q_i}^{m_i}\big( \phi(x) \big)=(g\circ\phi)(x)\prod_{i=1}^l\lambda^{m_i}f_{\phi^{-1}(q_i)}^{m_i}(x).
	\end{equation}
	En posant
	\begin{subequations}
		\begin{numcases}{}
			\tilde g(x)=\left( \prod_{i=1}^l\lambda^{m_i} \right)(g\circ\phi)(x)\\
			\tilde m_i=m_i\\
			\tilde  q_i=\phi^{-1}(q_i),
		\end{numcases}
	\end{subequations}
	nous avons \( \tilde q_i\in \eC\setminus K\), \( \tilde g\in \mE(K)\) et \( f=\tilde g\prod_{i=1}^lf_{\tilde q_i}\).
\end{proof}

\begin{lemma} 	\label{LEMooBHOGooXxBYGA}
	Une partie ouverte de \( \eC\) a au plus un nombre dénombrable de composantes connexes.
\end{lemma}

\begin{proof}
	Soit un ouvert \( A\) de \( \eC\). Nous notons \( \mB\) une base dénombrable d'ouverts de \( \eC\) (proposition \ref{PropNBSooraAFr}). Une composante connexe de \( A\) est ouverte\footnote{Proposition \ref{PROPooCZJGooRlyEOV} et le fait que \( \eC\) est localement convexe.}. Les composantes connexes de \( A\) sont disjointes, et chacune est l'union d'une quantité (au plus) dénombrable d'éléments de \( \mB\). Si il y avait une quantité non dénombrable de composantes connexes, cela ferait une quantité non dénombrable d'éléments différents de \( \mB\). Donc non.
\end{proof}

\begin{theorem}		\label{THOooDGXSooMMHTds}
	Soit un compact \( K\) dans \( \eC\). Alors
	\begin{enumerate}
		\item
		      Le nombre de composantes connexes bornées de \( \eC\setminus K\) est fini ou dénombrable.
		\item		\label{ITEMooWLGDooZGORrP}
		      En notant \( N\) le nombre de composantes connexes bornées de \( \eC\setminus K\), nous avons
		      \begin{equation}
			      G(K)=\mG(K)/\mE(K)\simeq \eZ^N
		      \end{equation}
		      où, si \( N\) est infini, \( \eZ^N\) désigne l'ensemble des suites finies dans \( \eZ\).
	\end{enumerate}
\end{theorem}

\begin{proof}

	Le fait que le nombre de composantes connexes bornées de \( \eC\setminus K\) est au maximum dénombrable est le lemme \ref{LEMooBHOGooXxBYGA}.

	Soient \( \{ \mO_i \}_{i=0,\ldots,N}\) les composantes connexe bornées de \( \eC\setminus K\) (avec \( i\in \eN\) dans le cas où \( N\) il y a une infinité dénombrable de composants connexes). Pour chaque \( i\) nous prenons un élément \( p_i\in\mO_i\). Nous notons aussi \(\pi \colon \mG(K)\to G(K)  \) le morphisme canonique. Nous posons

	\begin{equation}
		\begin{aligned}
			\phi\colon \eZ^N & \to G(K)                                 \\
			n                & \mapsto \prod_{i=1}^N\pi(f_{p_i})^{n_i}.
		\end{aligned}
	\end{equation}
	Notez que le produit a toujours un sens parce que \( n\in \eZ^N\) est toujours une suite finie (de taille \( N\) ou arbitraire) d'entiers. Pas de soucis de convergence.
	\begin{subproof}
		\spitem[\( \phi\) est un morphisme]
		%-----------------------------------------------------------
		Sur \( \eZ^n\), nous mettons l'addition \( (a+b)_i=a_i+b_i\). Si \( N=\infty\) et si \( a\) et \(b\) ont des longueurs différentes, on prend pour \( N\) le maximum des deux longueurs. Nous avons :
		\begin{subequations}
			\begin{align}
				\phi(a)\phi(b) & =\prod_{i=1}^N\pi(f_{p_i})^{a_i}\prod_{j=1}^N\pi(f_{p_j})^{b_j} \\
				               & =\prod_{i=1}^N\pi(f_{p_i})^{a_i+b_i}                            \\
				               & =\phi(a+b).
			\end{align}
		\end{subequations}
		\spitem[\( \phi\) est injective]
		%-----------------------------------------------------------
		Soit \( a\in \eZ^N\). Si \( a\neq 0\), alors le lemme \ref{LEMooOTMGooZeUoCy} nous dit que \( \prod_{i=1}^Nf_{p_i}^{a_i}\notin\mE(K)\). Vu que dans \( G(K)\), le zéro est \( \mE(K)\), nous avons \( \prod_{i=1}^N\pi(f_{p_i})^{a_i}\neq 0\) dès qu'un des \( a_i\) est non nul.
		\spitem[\( \phi\) est surjective]
		%-----------------------------------------------------------
		Soit \( F\in G(K)\). Vu que \( \pi\) est surjective, il existe \( f\in \mG(K)\) tel que \( \pi(f)=F\). Le lemme \ref{LEMooEJRMooNJhMov} nous dit qu'il existe \( g\in\mE(K) \) et \( a\in \eZ^N\) tels que
		\begin{equation}
			f=g\prod_{i=1}^Nf_{p_i}^{a_i}.
		\end{equation}
		Nous avons alors
		\begin{equation}
			F=\pi(f)=\pi(g)\pi\left(  \prod_{i=1}^Nf_{p_i}^{a_i} \right)=\prod_{i=1}^N\pi(f_{p_i})^{a_i}=\phi(a).
		\end{equation}
	\end{subproof}
\end{proof}

\begin{definition}[\cite{BIBooQKARooMHqitK,MonCerveau}]		\label{DEFooWQSAooDQYaip}
	Si \( (G,+)\) est un groupe abélien, la	\defe{pseudo-dimension}{pseudo-dimension} de \( G\) est le cardinal maximum des familles \( \eZ\)-indépendantes de \( G\).

	Dans ce contexte, une famille \( \eZ\)-indépendante est une partie \( \{ a_i \}_{i\in I}\) de \( G\) telle que si \( \sum_iz_ia_i=0\) avec \( z_i\in \eZ\),  alors \( z_i=0\) pour tout \( i\).

	Si il existe une partie libre idempotente à un ensemble \( A\), mais pas de parties libres surpotentes\footnote{Définition \ref{DEFooXGXZooIgcBCg}.} à \( A\), nous disons que la pseudo-dimension de \( G\) est «la cardialité de \( A\)».

	Dans tous les cas, nous notons \( \pdim(G)\) la pseudo-dimension de \( G\).
\end{definition}

\begin{lemma}[\cite{BIBooQKARooMHqitK, MonCerveau}]		\label{LEMooPASKooFGJgBz}
	À propos de pseudo-dimension.
	\begin{enumerate}
		\item
		      La pseudo-dimension de \( \eZ^n\) est \( n\).
		\item
		      La pseudo-dimension de \( \eZ^{\infty}\)\footnote{Pour rappel, \( \eZ^{\infty}\) est l'ensemble des suites finies d'éléments de \( \eZ\).} est la cardinalité de \( \eN\).
	\end{enumerate}
\end{lemma}

\begin{proof}
	Nous commençons par le cas de \( \eZ^n\) et nous ferons \( \eZ^{\infty}\) après.
	\begin{subproof}
		\spitem[\( \geq n\)]
		%-----------------------------------------------------------
		Prenons la base canonique \( (e_i)_{i=1,\ldots,n}\) de \( \eZ^n\). C'est une partie \( \eZ\)-indépendante de \( \eZ^n\). Donc la pseudo-dimension de \( \eZ^n\) est plus grande ou égale à \( n\).

		\spitem[\( \leq n\)]
		%-----------------------------------------------------------
		Dans l'autre sens, supposons que \( \{ a_i \}_{i\in I}\) soit \( \eZ\)-indépendant dans \( \eZ^n\). Pour rappel, la définition \ref{DEFooDLHWooAvfhgc} de partie libre ne parle que de sommes finies.

		Prouvons que \( \{ a_i \}_{i\in I} \) est \( \eQ\)-libre dans \( \eQ^n\). Soient une partie finie \( J\) de \( I\) ainsi que des rationnels \( \{ q_j \}_{j\in J}\) que \( \sum_{j\in J}q_ja_j=0\). Nous pouvons utiliser le lemme \ref{LEMooMUYAooDLgDcf} pour considérer \( k\in \eN\setminus\{ 0 \}\) tel que \( kq_j\in \eZ\) pour tout \( j\) (ici il est crucial que \( J\) soit fini, sinon on ne peut pas prendre un maximum). Dans ce cas nous avons \( \sum_{j\in J}kq_ja_j=0\), ce qui montre que \( \{ a_i \}\) n'est pas \( \eZ\)-indépendante.

		L'intérêt de \( \eQ\) est d'être un corps (alors que \( \eZ\) ne l'est pas). Nous savons que \( \eQ^n\) est un \( \eQ\)-espace vectoriel de dimension \( n\). Il ne contient donc pas de parties \( \eQ\)-libres de cardinal plus grand que \( n\).
	\end{subproof}

	Nous passons maintenant à la preuve pour \( \eZ^{\infty}\)\cite{MonCerveau}. L'existence d'une partie dénombrable et \( \eZ\)-libre dans \( \eZ^{\infty}\) est facile : il faut juste prendre \( e_i\) étant la suite nulle partout sauf en \( i\). La partie \( \{ e_i \}_{i\in \eN}\) est \( \eZ\)-libre dans \( \eZ^{\infty}\).

	Nous devons maintenant montrer que toute partie libre de \( \eZ^{\infty}\) est au maximum dénombrable. Soit une telle partie \( A\). En posant
	\begin{equation}
		A_n=\{ x\in A\tq x_n\neq 0\text{ et }x_k=0\text{ si }k>n \}.
	\end{equation}
	Étant donné que \( A\) ne contient que des suites finies, \( A=\bigcup_{n\in \eN}A_n\).

	D'abord \( A_n\) est \( \eZ\)-libre dans \( \eZ^n\). En effet considérons une partie finie \( \{ a_i \}_{i\in I}\) de \( A_n\) ainsi que des entiers \( \{ z_i \}_{i\in I}\) tels que \( \sum_{i\in I}z_ia_i=0\). Vu que \( \{ a_i \}_{i\in I}\) est également une partie finie de \( A\), nous en déduisons que \( z_i=0\) pour tout \( i\).

	La partie \( A_n\) étant libre dans \( \eZ^n\), elle est finie (de cardinal au maximum \( n\)). La partie \( A\) étant une union dénombrable de parties finies, elle est au maximum dénombrable.
\end{proof}

\begin{lemma}		\label{LEMooEYDNooEUKUpn}
	Deux groupes abéliens isomorphes ont même pseudo-dimension.
\end{lemma}

\begin{proof}
	Soient des groupes abéliens \( G\) et \( H\) ainsi qu'un isomorphisme \(\phi \colon G\to H  \). Soit une partie \( \eZ\)-indépendante \( \{ a_i \}_{i\in I}\) dans \( G\). Nous montrons que \( \{ \phi(a_i) \}_{i\in I}\) est \( \eZ\)-indépendante dans \( H\).

	En effet si \( \sum_{i\in I}z_i\phi(a_i)=0\), alors \( 0=\sum_i\phi(z_ia_i)=\phi\big( \sum_iz_ia_i \big)\), et donc \( \sum_iz_ia_i=0\), ce qui entraine \( z_i=0\).

	Cela prouve que \( \pdim(H)\geq\pdim(G)\). En inversant les rôles de \( G\) et \( H\) (et en utilisant \( \phi^{-1}\)), nous prouvons que \( \pdim(G)\geq \pdim(H)\).

	Si une des pseudo-dimensions est infinie, le raisonnement donné doit être récrit en disant que \( G\) possède une partie \( \eZ\)-libre de telle cardinalité si et seulement si \( H\) en possède une.
\end{proof}

\begin{corollary}[\cite{BIBooQKARooMHqitK}]		\label{CORooQNUIooKLtWVD}
	Soient des compacts \( K_1\) et \( K_2\) homéomorphes dans \( \eC\). Alors les parties \( \eC\setminus K_1\) et \( \eC\setminus K_2\) ont le même nombre de composantes connexes.
\end{corollary}

\begin{proof}
	Vu que \( K_1\) est homéomorphe à \( K_2\), les groupes \( G(K_1)\) et \( G(K_2)\) sont isomorphes par le lemme \ref{LEMooHEOWooHTtHsJ}. Le théorème \ref{THOooDGXSooMMHTds}\ref{ITEMooWLGDooZGORrP} nous enseigne que, si on désigne par \( N_i\) le nombre de composantes connexes de \( \eC\setminus K_i\), alors \( G(K_i)=\eZ^{N_i}\). Le fait que \( G(K_1)\) soit isomorphe à \( G(K_2)\) implique
	\begin{equation}
		\eZ^{N_1}\simeq \eZ^{N_2}.
	\end{equation}
	Le lemme \ref{LEMooEYDNooEUKUpn} indique alors que \( \pdim(\eZ^{N_1})=\pdim(\eZ^{N_2})\). Et enfin le lemme \ref{LEMooPASKooFGJgBz} conclut que \( N_1=N_2\).
\end{proof}

Dans le lemme suivant, \( S^1(a,r)\) est la sphère centrée en \( a\) et de rayon \( r\) : \( S^1(a,r)=\{ z\in \eC\tq | a-z |=r \}\).
\begin{lemma}		\label{LEMooUSJZooFVAlTa}
	La partie \( \eC\setminus S^1(0,1)\) a exactement 2 composantes connexes.
\end{lemma}

\begin{proof}
	Posons \( A=\{ z\in \eC\tq | z |<1 \}\) et \( B=\{ z\tq| z |>1 \}\). Nous avons \( A\cup B=\eC\setminus S^1\). La partie \( A\) est connexe et même convexe. En ce qui concerne la connexité de \( B\), c'est le corolaire \ref{CORooONAVooKPhQuI}.

	De plus \( A\) et \( B\) sont des ouverts disjoints.

	Supposons que \( C\) soit une composante connexe de \( \eC\setminus S^1(0,1)\) contenant un point de \( A\) et un point de \( B\). Soit \( a\in C\cap A\). Par définition \ref{DEFooFHXNooJGUPPn}\ref{ITEMooBZAQooNwuzaS} d'une composante connexe, \( C\) contient tous les connexes contenant \( a\). En particulier \( A\subset C\). De même si \( b\in C\cap B\), alors \( B\subset C\).

	Dans ce cas nous aurions \( C=A\cup B=\eC\setminus S^1(0,1)\). Mais cela est impossible parce que \( A\) et \( B\) forment un recouvrement de \( \eC\setminus S^1\) par deux ouverts disjoints.

	Donc \( A\) et \( B\) sont des composantes connexes de \( \eC\setminus S^1\). Et ce sont les deux seules parce que \( A\cup B=\eC\setminus S^1\).
\end{proof}

%--------------------------------------------------------------------------------------------------------------------------- 
\subsection{Théorème de Jordan}
%---------------------------------------------------------------------------------------------------------------------------

\begin{theorem}[Théorème de Jordan\cite{BIBooQKARooMHqitK,BIBooYTYUooTrexbq}]\label{ThoHSPWBuh}
	Soit une courbe de Jordan \( \Gamma\) fermée dans \( \eC\).
	\begin{enumerate}
		\item		\label{ITEMooOAKXooMirWiD}
		      La partie \( \eC\setminus \Gamma\) a exactement deux composantes connexes.
		\item	\label{ITEMooICBGooXHMALK}
		      Une composante connexe est bornée, l'autre non.
		\item
		      Les deux composantes connexes ont \( \Gamma\) comme frontière.
		      %TODOooBDBSooLHSIZD Faire ce point à propos de la frontière
		      % j'ai posé la question ici:
		      % https://math.stackexchange.com/questions/4661350/jordan-theorem-the-boundary-statement
	\end{enumerate}
\end{theorem}
\index{théorème!de Jordan}
% TODOooBBSQooKwbHLJ
% Revoir la réponse de Alphago dans
% https://math.stackexchange.com/questions/1727310/convex-curve-as-boundary-of-a-convex-set


\begin{proof}
	En plusieurs parties.
	\begin{subproof}
		\spitem[Pour \ref{ITEMooOAKXooMirWiD}]
		%-----------------------------------------------------------
		Nous savons par le corolaire \ref{CORooPGFLooUTVZMi}\ref{ITEMooIUAXooOfvNov} que \( \Gamma\) est homéomorphe à \( S^1\). Le corolaire \ref{CORooQNUIooKLtWVD} nous dit alors que \( \eC\setminus \Gamma\) a le même nombre de composantes connexes que \( \eC\setminus S^1\), c'est-à-dire deux par le lemme \ref{LEMooUSJZooFVAlTa}.
		\spitem[Pour \ref{ITEMooICBGooXHMALK}]
		%-----------------------------------------------------------
		Le lemme \ref{LEMooJNPTooScfSvA} dit que \( \eC\setminus \Gamma\) a exactement une composante connexe non bornée. Vu que nous venons de voir qu'elle a exactement deux composants connexes, c'est que l'autre est bornée.
	\end{subproof}
\end{proof}
