% This is part of Mes notes de mathématique
% Copyright (c) 2008-2022
%   Laurent Claessens, Carlotta Donadello
% See the file fdl-1.3.txt for copying conditions.

%+++++++++++++++++++++++++++++++++++++++++++++++++++++++++++++++++++++++++++++++++++++++++++++++++++++++++++++++++++++++++++
\section{Espaces vectoriels topologiques}
%+++++++++++++++++++++++++++++++++++++++++++++++++++++++++++++++++++++++++++++++++++++++++++++++++++++++++++++++++++++++++++

\begin{definition}\label{DefEVTopologique}
	Un espace vectoriel \( V\) sur le corps valué\footnote{Définition \ref{DEFooBWXXooAkBBRS}.} \( \eK\) muni d'une topologie est un \defe{espace vectoriel topologique}{espace vectoriel!topologique} si
	\begin{enumerate}
		\item
		      la somme de deux vecteurs est une application continue \( V\times V\to V \); et
		\item
		      la multiplication par un scalaire est une application continue \( \eK\times V\to V\).
	\end{enumerate}
	Ici, sur \( V\times V\) et sur \( \eK\times V\) nous avons la topologie produit.

	Dans toute la suite, nous supposons que \( \eK\) est un corps avec une topologie métrique.
\end{definition}
On le redit quand même: le corps\footnote{Définition~\ref{DefTMNooKXHUd}} lui-même doit avoir sa topologie. Dans la grande majorité des cas, ce corps est \( \eR\) ou \( \eC\) muni de la topologie usuelle.

Mine de rien, le fait que les deux opérations usuelles soient continues a de belles conséquences sur la topologie de l'espace\dots

\begin{proposition}[\cite{ooMKWJooLSkGfh}]      \label{PROPooDXLFooFghbWk}
	Soit un espace vectoriel topologique \( V\). Pour \(x \in V \) et \(\lambda \in \eK, \ \lambda \neq 0 \) fixés, les fonctions \( T_x \) et \( M_\lambda \) définies par:
	\begin{align}
		T_x: & V \to V       &  & \text{et} & M_\lambda:          & V \to V \\
		     & y \mapsto x+y &  &           & y \mapsto \lambda y
	\end{align}
	sont des automorphismes\footnote{Définition \ref{DEFooYPGQooMAObTO}.} de l'espace topologique \(V \).
\end{proposition}

\begin{proof}
	Ce sont des bijections continues, dont les inverses sont respectivement \( T_{-x} \) et \( M_{1/\lambda} \).
\end{proof}

\begin{corollary}[Invariance de la topologie~\cite{ooMKWJooLSkGfh}]\label{PropInvarianceTopologie}
	Toute base de voisinage de \( 0 \) se transporte en tout point de l'espace vectoriel topologique.

	Plus précisément, si \( \{ A_i \}_{i\in I}\) est une base de voisinage de \( 0\), alors \( \{ A_i+a \}_{i\in I}\) est une base de voisinage de \( a\).
\end{corollary}

\begin{lemma}[Changement de variables]      \label{LEMooAHIGooJhpPvo}
	Soient un espace vectoriel topologique\footnote{Définition \ref{DefEVTopologique}.} \( X\) ainsi qu'un espace séparé \( Y\) et une application \( f\colon X\to Y\). Nous supposons que \( \lim_{x\to a}f(x)=\ell\). Alors \( \lim_{x\to b} f(x+a-b)\) existe et vaut \( \ell\).
\end{lemma}

\begin{proof}
	Soit un voisinage \( V\) de \( \ell\) dans \( Y\). Il existe un voisinage \( U\) de \( a\) tel que \( f(U)\subset V\). Nous posons \( U'=U-a+b\). C'est un voisinage de \( b\). En posant \( g(x)=f(x+a-b)\) nous avons
	\begin{equation}
		g(U')=f(U-a+b+a-b)=f(U)\subset V.
	\end{equation}
	Donc \( \lim_{x\to b}g(x)=\ell\). C'est cette égalité qui signifie \( \lim_{x\to b}f(x+a-b)\).
\end{proof}

%--------------------------------------------------------------------------------------------------------------------------- 
\subsection{Corps topologique}
%---------------------------------------------------------------------------------------------------------------------------

\begin{definition}[Anneau topologique\cite{BIBooIAZDooByMAjy}]      \label{DEFooWKLOooPdsxQl}
	Un \defe{anneau topologique}{anneau topologique} est un anneau\footnote{Définition \ref{DefHXJUooKoovob}.} muni d'une topologie dans laquelle l'addition et la multiplication sont continues\footnote{Définition \ref{DefOLNtrxB}\ref{ITEMooEHGWooDdITRV}.}.
\end{definition}

\begin{propositionDef}      \label{PROPooAWAKooKRmbGT}
	Si \( (\eK, | . |)\) est un corps valué\footnote{Définition \ref{DEFooBWXXooAkBBRS}}, alors l'application
	\begin{equation}
		\begin{aligned}
			d\colon \eK\times \eK & \to \eR^+       \\
			(x,y)                 & \mapsto | x-y |
		\end{aligned}
	\end{equation}
	est une distance\footnote{Définition \ref{DefMVNVFsX}.}.

	Un corps valué muni de sa topologie métrique\footnote{Définition \ref{ThoORdLYUu}.} est un corps topologique\footnote{Définition \ref{DEFooWKLOooPdsxQl}.}.
\end{propositionDef}

\begin{lemma}       \label{LEMooCHDTooZsgXEK}
	Les corps \( \eR\) et \( \eC\) sont des corps valués. Leur topologie métrique (en tant que corps valués) est leur topologie usuelle.
\end{lemma}

%--------------------------------------------------------------------------------------------------------------------------- 
\subsection{Voisinage symétrique et équilibré}
%---------------------------------------------------------------------------------------------------------------------------

\begin{definition}[Partie symétrique\cite{ooMKWJooLSkGfh}]
	Une partie \( U\) d'un espace vectoriel topologique est \defe{symétrique}{partie symétrique} si \( x\in U\) implique \( -x\in U\).
\end{definition}

\begin{definition}[Partie équilibrée\cite{ooMKWJooLSkGfh}]
	Une partie \( U\) d'un espace vectoriel topologique \( V\) est \defe{équilibrée}{partie équilibrée} si pour tout \( | \alpha |<1\) dans \( \eK\), \( \alpha U\subset U\).
\end{definition}

\begin{lemma}[\cite{ooMKWJooLSkGfh,MonCerveau}]     \label{LEMooYSWXooNqAcOQ}
	Soit un espace vectoriel topologique.
	\begin{enumerate}
		\item       \label{ITEMooSWWQooTreWIE}
		      Soit un ouvert \( A\) autour de \( 0\) dans l'espace vectoriel topologique \( V\). Il existe \( \delta>0\) dans le corps \( \eK\) et un voisinage ouvert \(W\) de \( 0\) tel que \( \lambda W\subset A\) pour tout \( | \lambda |<\delta\).
		\item       \label{ITEMooXZNHooGVplpu}
		      Tout voisinage de \( 0\) contient un ouvert équilibré.
		\item       \label{ITEMooRLVSooGihcLc}
		      Tout voisinage de \( 0\) contient un ouvert équilibré et symétrique.
	\end{enumerate}
\end{lemma}

\begin{proof}
	En plusieurs parties.
	\begin{subproof}
		\item[Pour \ref{ITEMooSWWQooTreWIE}]
		Nous savons que \( 0\cdot 0=0\) et que l'application
		\begin{equation}
			\begin{aligned}
				f\colon \eK\times V & \to V             \\
				\lambda,x           & \mapsto \lambda x
			\end{aligned}
		\end{equation}
		est continue. La partie \( f^{-1}(A)\) contient \( (0,0)\). Il existe donc un voisinage ouvert \( S\) de \( 0\) dans \( \eK\) et un voisinage ouvert \( W\) de \( 0\) dans \( V\) tel que \( S\times W\subset f^{-1}(A)\).

		Puisque \( \eK\) est un corps dont la topologie est métrique\footnote{Le corps \( \eK\) est un corps valué, et donc métrique par la définition \ref{PROPooAWAKooKRmbGT}.}, il existe une boule \( S'=B(0,\delta)\subset S\).

		Donc nous avons \( f(S'\times W)\subset A \) et pour tout \( | \lambda |<\delta\), \( \lambda W\subset A\).
		\item[Pour \ref{ITEMooXZNHooGVplpu}]
		Soit un voisinage ouvert \( \mO\) de \( 0\) dans \( V\). Par le point \ref{ITEMooSWWQooTreWIE}, nous considérons un voisinage \( W\) de \( 0\) et un \( \delta>0\) tel que \( \lambda W\subset\mO\) pour tout \( | \lambda |<\delta\).

		Nous posons
		\begin{equation}
			U=\{\lambda w<\tq | \lambda |<\delta, w\in W\}.
		\end{equation}
		Nous avons \( U\subset A\) par définition de \( W\). De plus \( U\) est équilibré parce que si \( | \mu |<1\), et si \( x\in U\), il existe \( | \lambda |<\delta\) et \( w\in W\) tels que \( x=\lambda w\). Alors \( \mu x=\mu\lambda w\). Nous avons \( | \mu\lambda |<\delta\) et donc \( \mu\lambda w\in U\).
		\item[Pour \ref{ITEMooRLVSooGihcLc}]
		Nous considérons \( U\) équilibré comme dans \ref{ITEMooXZNHooGVplpu}. Ensuite nous posons \( U'=U\cap (-U)\). La partie \( U'\) est symétrique, elle est ouverte (intersection d'ouverts). Et elle est équilibrée parce que si \( x\in U'\) et \( | \lambda |<1\) alors:
		\begin{itemize}
			\item \( x\in U\) et \( U\) est équilibré, donc \( \lambda x\in U\).
			\item \( x\in -U\) et \( U\) est équilibré, donc il existe \( y\in U\) tel que \( x=-y\). Pour ce \( y\) nous avons \( \lambda y\in U\) et donc \( \lambda x=-\lambda y\in -U\). Donc \( \lambda x\in -U\).
			\item Au final, \( \lambda x\in U\cap (-U)=U'\) et \( U'\) est équilibré.
		\end{itemize}
	\end{subproof}
\end{proof}

\begin{lemma}[\cite{BIBooSHPPooMkbgoC}]     \label{LEMooQEFRooHAxOys}
	Soit un espace vectoriel topologique \( V\) ainsi qu'un voisinage ouvert \( \mO\) de \( 0\) dans \( V\). Il existe des voisinages ouverts \( U_1\) et \( U_2\) de \( 0\) dans \( V\) tels que
	\begin{equation}
		U_1+U_2\subset \mO.
	\end{equation}
\end{lemma}

\begin{proof}
	Par définition d'un espace vectoriel topologique, l'application
	\begin{equation}
		\begin{aligned}
			f\colon V\times V & \to V       \\
			x,y               & \mapsto x+y
		\end{aligned}
	\end{equation}
	est continue. Donc la partie \( f^{-1}(\mO)\) est un ouvert de \( V\times V\) (c'est la définition \ref{DefOLNtrxB}\ref{ITEMooEHGWooDdITRV} de la continuité). La définition \ref{DefIINHooAAjTdY} de la topologie produit, appliquée au point \( (0,0)\in V\times V\) implique qu'il existe des voisinages \( U_1\) et \( U_2\) de \( 0\) dans \( V\) tels que
	\begin{equation}
		U_1\times U_2\subset f^{-1}(\mO).
	\end{equation}
	Donc \( f(U_1\times U_2)\subset\mO\) et en particulier \( U_1+U_2\subset \mO\).
\end{proof}

\begin{proposition}[\cite{ooMKWJooLSkGfh,MonCerveau}]\label{PROPSommeTopologique}
	Soit \( V \) un espace vectoriel topologique, et \( \mO \) un voisinage ouvert de \( 0 \). Il existe un voisinage ouvert \( U\) de \( 0 \) tel que
	\begin{enumerate}
		\item
		      \( U\) est symétrique,
		\item
		      \( U\) est équilibré
		\item
		      \( U\) vérifie \( U + U \subset \mO \).
		\item
		      \( U\) vérifie \( U + U + U + U \subset \mO \).
	\end{enumerate}
\end{proposition}

\begin{proof}
	En plusieurs petits pas.
	\begin{subproof}
		\item[Le point de départ]
		Le lemme \ref{LEMooQEFRooHAxOys} donne des voisinages ouverts \( U_1\) et \( U_2\) de \( 0\) dans \( V\) tels que \( U_1+U_2\subset \mO\).
		\item[Symétrique]
		En posant \( U' = U_1 \cap U_2 \cap (-U_1) \cap (-U_2) \), on a un sous-ensemble symétrique de \( U_1\) et \(U_2\) qui vérifie \( U' + U'\subset W \). De plus \( U'\) est encore un voisinage ouvert de \( 0\) dans \( V\).
		\item[équilibré]
		C'est le moment d'utiliser le lemme \ref{LEMooYSWXooNqAcOQ}. La partie \( U'\) contient un voisinage ouvert \( U''\) de \( 0\) qui est symétrique et équilibré. Ce \( U''\) vérifie encore \( U''+U''\subset \mO\).
		\item[En \( 4\) parties]
		Maintenant nous ré-appliquons tout ce que nous venons de faire à \( U''\) pour obtenir un voisinage symétrique et équilibré de \( 0\) tel que \( U+U\subset U'\). Nous avons alors \( U+U+U+U\subset \mO\).

		Notons que ce \( U\) vérifie à fortiori \( U+U\subset \mO\).
	\end{subproof}
\end{proof}


\begin{lemma}[\cite{MonCerveau}]            \label{LEMooEZIYooBBxdJj}
	Soit un espace vectoriel topologique \( V\) sur le corps \( \eK\). Si \( \mO\) est un ouvert autour de \( 0\) dans \( V\) et si \( \lambda\neq 0\in \eK\), il existe un ouvert \( U\) autour de \( 0\) tel que \( \lambda U\subset \mO\).
\end{lemma}

\begin{proof}
	La réponse est \( U=\lambda^{-1}\mO\). En effet par définition d'un espace vectoriel topologique, la fonction donnée par \( f(x)=\lambda x  \) est continue; donc \( U=f^{-1}(\mO)\) est un ouvert. De plus \( \lambda U=\mO\).
\end{proof}


%---------------------------------------------------------------------------------------------------------------------------
\subsection{Limite de suites}
%---------------------------------------------------------------------------------------------------------------------------

Si \( (x_n)\) est une suite dans un espace vectoriel topologique, rien ne garantit qu'elle ait une limite, ni qu'elle soit unique. Donc lorsque nous écrivons
\begin{equation}
	x_n\stackrel{V}{\longrightarrow}x,
\end{equation}
nous sous-entendons seulement que \( x\) est une limite.

De même, dans la proposition \ref{PROPooZRCBooKiJhDg}, nous montrerons que \( x_n+y_n\stackrel{V}{\longrightarrow}x+y\) et \( \lambda x_n\stackrel{ V}{\longrightarrow}\lambda x\). Cela signifie que si \( x\) et \( y\) sont des limites de \( (x_n)\) et \( (y_n)\), alors \( x+y\) est une limite de \( (x_n+y_n)\) et que \( \lambda x\) est une limite de \( (\lambda x_n)\).

Si \( V\) est un espace vectoriel topologique dans lequel il n'y a pas unicité de la limite\footnote{La proposition \ref{PropUniciteLimitePourSuites} dit qu'il y a unicité de la limite dans les espaces topologiques séparés.}, nous ne pouvons pas exactement dire que le processus de limite est une opération linéaire sur l'ensemble des suites convergentes.

\begin{lemma}       \label{LEMooJDJVooHUKdSe}
	Soient un espace vectoriel topologique \( V\) ainsi qu'une suite \( (x_n)\) dans \( V\). Nous avons
	\begin{equation}
		x_n\stackrel{V}{\longrightarrow}x
	\end{equation}
	si et seulement si
	\begin{equation}
		x_n-x\stackrel{V}{\longrightarrow}0.
	\end{equation}
\end{lemma}

\begin{proposition}[\cite{MonCerveau}]        \label{PROPooZRCBooKiJhDg}
	Soit \( V\), un espace vectoriel topologique. Soient deux suites convergentes \( x_n\stackrel{V}{\longrightarrow}x\) et \( y_n\stackrel{V}{\longrightarrow}y\) ainsi que \( \lambda\in \eK\). Alors
	\begin{enumerate}
		\item       \label{ITEMooSHPAooQyEkgT}
		      \begin{equation}
			      x_n+y_n\stackrel{V}{\longrightarrow}x+y.
		      \end{equation}
		\item   \label{ITEMooYHHYooYATzWE}
		      \begin{equation}
			      \lambda x_n\stackrel{V}{\longrightarrow}\lambda x.
		      \end{equation}
	\end{enumerate}
\end{proposition}

\begin{proof}
	En deux parties.
	\begin{subproof}
		\item[\ref{ITEMooSHPAooQyEkgT}]
		Nous allons montrer que \( x_n+y_n-(x+y)\stackrel{V}{\longrightarrow}0\); ce sera suffisant par le lemme \ref{LEMooJDJVooHUKdSe}.

		Soit un ouvert \( \mO\) autour de \( 0\). Soient des ouverts \( U_1\) et \( U_2\) autour de \( 0\) tels que \( U_1+U_2\subset \mO\) (lemme \ref{LEMooQEFRooHAxOys}).

		Vues les convergences de \( (x_n)\) et de \( (y_n)\), il existe un \( N\) tel que \( n\geq N\) implique \( x_n-x\in U_1\) et \( y_n-y\in U_2\). Dans ce cas, \( x_n+y_n-(x+y)\in U_1+U_2\subset \mO\).

		Donc pour \( n\geq N\) nous avons bien \( x_n-y_n-(x+y)\in \mO\), ce qui signifie que \( x_n+y_n\stackrel{V}{\longrightarrow}x+y\).
		\item[\ref{ITEMooYHHYooYATzWE}]
		En plusieurs étapes.
		\begin{subproof}
			\item[\( x_n-x\stackrel{V}{\longrightarrow}0\)]
			C'est le lemme \ref{LEMooJDJVooHUKdSe}.
			\item[\( \lambda x_n-x\stackrel{V}{\longrightarrow}0\)]
			Soit un ouvert \( \mO\) autour de \( 0\). Par le lemme \ref{LEMooEZIYooBBxdJj}, il existe un ouvert \( U\) autour de \( 0\) tel que \( \lambda U\subset \mO\). Comme \( x_n-x\stackrel{V}{\longrightarrow}0\), il existe \( N\) tel que \( n\geq N\) implique \( x_n-x\in U\).

			Pour ces \( N\) et \( n\) nous avons aussi \( \lambda (x_n-x)\in \lambda U\subset \mO\). Nous avons donc démontré que \( \lambda x_n-\lambda x\stackrel{V}{\longrightarrow}0\).
			\item[Conclusion]
			Encore le lemme \ref{LEMooJDJVooHUKdSe} nous permet de déduire que \( \lambda x_n\stackrel{V}{\longrightarrow}\lambda x\).
		\end{subproof}
	\end{subproof}
\end{proof}

%+++++++++++++++++++++++++++++++++++++++++++++++++++++++++++++++++++++++++++++++++++++++++++++++++++++++++++++++++++++++++++
\section{Continuité de fonctions}
%+++++++++++++++++++++++++++++++++++++++++++++++++++++++++++++++++++++++++++++++++++++++++++++++++++++++++++++++++++++++++++

%---------------------------------------------------------------------------------------------------------------------------
\subsection{Continuité}
%---------------------------------------------------------------------------------------------------------------------------

La définition de la continuité d'une fonction est donnée en \ref{DefOLNtrxB}.

\begin{normaltext}
	Lorsque nous écrivons \( f\colon X\to Y\), nous entendons que \( f\) est définie sur tout \( X\), mais pas qu'elle soit surjective sur \( Y\). En particulier, pour que \( f\) soit continue en \( a\), il faut que \( a\) soit dans le domaine de définition de \( f\).

	Dans le cas de fonctions \( \eR\to \eR\), l'espace \( X\) sera la partie de \( \eR\) sur laquelle \( f\) sera définie, et la topologie sera la topologie induite de \( \eR\).
\end{normaltext}

\begin{proposition}[\cite{BIBooDMSUooCZKkdj}]       \label{PROPooOXBCooIzLaPe}
	Soient deux espaces topologiques \( X\) et \( Y\). Une application \( f\colon X\to Y\) est continue\footnote{Définition \ref{DefOLNtrxB}.} si et seulement si pour tout \( x\in X\) et pour tout voisinage\footnote{Définition \ref{DEFVoisinageooGHZCooLRcpXY}} \( V\) de \( f(x)\), la partie \( f^{-1}(V)\) est un voisinage de \( x\) dans~\( X\).
\end{proposition}

\begin{proof}
	En deux parties.
	\begin{subproof}
		\item[\( \Rightarrow\)]
		Soient \( x\in X\) et un voisinage \( V\) de \( f(x)\) dans \( Y\). Il existe alors un ouvert \( \mO\) de \( Y\) tel que \( f(x)\in \mO\subset V\).

		La partie \( f^{-1}(\mO)\) vérifie :
		\begin{itemize}
			\item \( f^{-1}(\mO)\) est un ouvert de \( X\) parce que \( f\) est continue.
			\item \( x\in f^{-1}(\mO)\)
			\item \( f^{-1}(\mO)\subset f^{-1}(V)\).
		\end{itemize}
		Donc \( f^{-1}(V)\) contient un ouvert contenant \( x\). Donc \( f^{-1}(V)\) est un voisinage de \( x\) dans \( X\).
		\item[\( \Leftarrow\)]
		Soit un ouvert \( \mO\) de \( Y\). Nous devons prouver que \( f^{-1}(\mO)\) est un ouvert de \( X\). Pour cela nous prouvons que \( f^{-1}(\mO)\) contient un ouvert autour de chacun de ses éléments et utilisons le théorème \ref{ThoPartieOUvpartouv}.

		Soit donc \( x\in f^{-1}(\mO)\). La partie \( \mO\) est un voisinage de \( f(x)\). Donc \( f^{-1}(\mO)\) est un voisinage de \( x\). Il existe donc un ouvert \( V\) de \( X\) tel que
		\begin{equation}
			x\in V\subset f^{-1}(\mO).
		\end{equation}
		Nous en déduisons que \( f^{-1}(\mO)\) contient bien un ouvert autour de chacun de ses points.
	\end{subproof}
\end{proof}

La proposition~\ref{PropQZRNpMn} donnera des détails sur ce qu'il se passe lorsque l'espace est métrique.

\begin{theorem} \label{ThoESCaraB}
	Une fonction \( f\colon X\to Y\) est une fonction continue si et seulement si elle est continue en chacun des points de \( X\).
\end{theorem}

\begin{proof}
	En deux parties.
	\begin{subproof}
		\item[\( \Rightarrow\)]
		Nous supposons que \( f\) est une fonction continue. Soient \( a\in X\) et \( W\) un voisinage de \( f(a)\). Nous considérons \( \mO\), un voisinage ouvert de \( f(a)\) contenu dans \( W\); l'ensemble \( f^{-1}(\mO)\) est alors un ouvert contenant \( a\), et l'image de \( f^{-1}(\mO)\) par \( f\) est bien entendu contenue dans \( W\).

		\item[\( \Leftarrow\)]
		Soit \( \mO\) un ouvert de \( Y\). Pour prouver que \( f^{-1}(\mO)\) est un ouvert de \( X\), nous allons considérer un élément \( a\in f^{-1}(\mO)\) et montrer qu'il existe un voisinage ouvert de \( a\) contenu dans \( f^{-1}(\mO)\); le théorème~\ref{ThoPartieOUvpartouv} nous assurera alors que \( f^{-1}(\mO)\) est ouvert.

		L'ensemble \( \mO\) est un voisinage ouvert de \( f(a)\) parce que \( a\) a été choisi dans \( f^{-1}(\mO)\). Donc la continuité de \( f\) en \( a\) nous assure qu'il existe un voisinage \( W\) de \( a\) tel que \( f(W)\subset\mO\). En prenant un ouvert contenant \( a\) à l'intérieur de \( W\) nous avons un voisinage ouvert de \( a\) contenu dans \( f^{-1}(\mO)\).
	\end{subproof}
\end{proof}

\begin{remark}
	À cause de l'éventuelle non unicité de la limite, deux fonctions continues et égales sur un sous-ensemble dense ne sont pas spécialement égales. Ce sera vrai sur les espaces métriques et plus généralement pour les espaces séparés. Voir l'exemple~\ref{EXooSHKAooZQEVLB} et la proposition~\ref{PropFObayrf}.
\end{remark}

\begin{lemma}[\cite{MonCerveau}]  \label{LEMooCQQWooVSAWiy}
	Soient une fonction \( f\colon X\to Y\), et un point d'accumulation \( a\in X\)\footnote{Un point d'accumulation de \( X\) n'est pas spécialement dans \( X\), si \( X\) est un sous-espace d'un autre. Par exemple \( 0\) est un point d'accumulation de \( \mathopen] 0 , 1 \mathclose[\) dans \( \eR\). Ici nous supposons que \( a\in X\), sinon il n'y a de toute façon pas de continuité en \( a\).}. La fonction \( f\) est continue en \( a\) si et seulement si \( f(a)\) est une limite de \( f\) en \( a\).
\end{lemma}

\begin{proof}
	En deux parties.
	\begin{subproof}
		\item[\( \Rightarrow\)]
		Nous supposons que \( f\) est continue en \( a\in X\). Soit un ouvert \( V\) de \( Y\) contenant \( f(a)\). Par continuité de \( f\) au point\footnote{Continuité en un point, définition \ref{DefOLNtrxB}\ref{ITEMooXARPooNzsWLr}.} \( a\), il existe un voisinage \( U\) de \( a\) tel que \( f(U)\subset V\). À fortiori, \( f\big( U\setminus{{a}} \big)\subset W\) comme le demande la définition de la limite.
		\item[\( \Leftarrow\)]
		Nous supposons que \( f(a)\) est une limite de \( f(x)\) lorsque \( x\) tend vers \( a\). Si \( W\) est un ouvert de \( Y\) contenant \( f(a)\), il existe un voisinage \( V\) de \( a\) dans \( X\) tel que \( f\big( V\setminus{{a}} \big)\subset W\). Mais puisque \( f(a)\in W\), nous avons \( f(V)\subset W\).
	\end{subproof}
\end{proof}

\subsubsection{Continuité séquentielle}
%///////////////////////

\begin{definition}  \label{DefENioICV}
	Si \( X\) et \( Y \) sont deux espaces topologiques, une fonction \( f\colon X\to \eR\) est \defe{séquentiellement continue}{continuité!séquentielle} en un point \( a\) si pour toute suite convergente \( x_n\to a\) dans \( X\) nous avons \( f(x_n)\to f(a)\) dans \( Y\).
\end{definition}

\begin{normaltext}
	Nous allons maintenant voir deux résultats disant que si une fonction est continue, alors elle peut être permutée avec une limite de suite. Dans le cas des espaces métriques, la proposition \ref{PropXIAQSXr} montrera la réciproque : si pour toute suite \(x_n\to a\), nous avons \( \lim_{n\to \infty} f(x_n)=y\), alors \( f\) a une limite en \( a\) qui vaut \( y\).
\end{normaltext}

\begin{proposition}[Permuter limite et fonction continue\cite{MonCerveau}] \label{fContEstSeqCont}
	Soient deux espaces topologiques \( X\) et \( Y\) ainsi qu'une fonction \( f\colon X\to Y\). Soit \( a\in X\) et \( \ell\in Y\). Si
	\begin{equation}
		\lim_{x\to a} f(x)=\ell,
	\end{equation}
	alors, pour toute suite \( (x_k) \) telle que \( x_k \to a \), on a
	\begin{equation}
		\lim f(x_k)=\ell.
	\end{equation}
\end{proposition}

\begin{proof}
	Nous considérons une suite \( (x_k)\) qui converge vers \( a\) dans \( X\). Soient \( V\) un voisinage de \( \ell \) et \( W\) un voisinage de \( a\) tels que \( f(W)\subset V\) (définition~\ref{DefYNVoWBx} de la continuité en un point). Par la convergence \( a_k\to a\),  il existe \( N\) tel que pour tout \( k\geq N\), \( a_k\in W\), et donc tel que \( f(a_k)\in V\), ce qui donne la continuité séquentielle de \( f\).
\end{proof}


\subsubsection{Application réciproque}
%//////////////////////

\begin{definition}[injection, surjection, bijection]        \label{DEFooBFCQooPyKvRK}
	Soient des ensembles \( A\) et \( B\) ainsi qu'une application \( f\colon A\to B\).
	\begin{enumerate}
		\item
		      La fonction \( f\) est \defe{injective}{injection} si \( f(x_1)=f(x_2)\), implique \( x_1=x_2\).
		\item
		      La fonction \( f\) est \defe{surjective}{surjection} si tous les éléments de \( B\) sont atteints, c'est-à-dire si pour tout \( y\in B\) il existe \( x\in A\) tel que \( f(x)=y\).
		\item
		      La fonction \( f\) est une \defe{bijection}{bijection} entre \( A\) et \( B\) si elle est injective et surjective, c'est-à-dire si pour tout \( y\in B\) il existe un unique \( x\in A\) tel que \( f(x)=y\).
	\end{enumerate}
\end{definition}
La surjection et l'injection sont des propriétés bien différentes qu'il convient de prouver séparément. De plus une même «formule» peut définir une application injective, surjective, bijective ou non selon le domaine sur laquelle nous la considérons.

\begin{definition}      \label{DEFooTRGYooRxORpY}
	Soit \( f\colon A\to B\) une bijection. L'\defe{application réciproque}{application réciproque} de \( f\) est la fonction
	\begin{equation}
		\begin{aligned}
			f^{-1}\colon B & \to A                                             \\
			y              & \mapsto \text{le } x\in A\text{ tel que } f(x)=y.
		\end{aligned}
	\end{equation}
\end{definition}

Plus généralement si \( f\colon X\to Y\) est une application quelconque et si \( S\subset Y\) nous notons
\begin{equation}
	f^{-1}(S)=\{ x\in X\tq f(x)\in S \},
\end{equation}
et dans le cas où \( S\) est réduit à un unique élément \( y\), nous notons \( f^{-1}(y)\) au lieu de \( f^{-1}\big( \{ y \} \big)\). Si de plus \( f^{-1}(S)\) est un singleton \( x\), nous noterons \( f^{-1}(S)=x\) et non \( f^{-1}(S)=\{ x \}\).

Les plus acharnés parmi les lecteurs se rendront compte de la différence ontologique fondamentale entre \( x\) et \( \{ x \}\).

\begin{proposition}	\label{PropoInvCompCont}
	Soit \( f\colon A\subset\eR^n\to B\subset\eR^m\) une bijection continue. Si \( A\) est compact, alors \( f^{-1}\colon B\to A\) est continue.
\end{proposition}
\index{réciproque!continuité}

\begin{proposition}		\label{PropIntContMOnIvCont}
	Soient \( I\) un intervalle dans \( \eR\) et \( f\colon I\to \eR\) une fonction continue strictement monotone. Alors la fonction réciproque \( f^{-1}\colon f(I)\to \eR\) est continue sur l'intervalle \( f(I)\).
\end{proposition}
\index{réciproque!continuité}

%---------------------------------------------------------------------------------------------------------------------------
\subsection{Continuité et topologie induite}
%---------------------------------------------------------------------------------------------------------------------------
\begin{proposition}[\cite{MonCerveau}]     \label{PROPooNPLBooPfmmym}
	Soit une fonction \( f\colon X\to Y\), continue sur l'ouvert \( A\) de \( X\) au sens où elle est continue en chaque point de \( A\). Alors la fonction restriction \( \tilde f\colon A\to Y\) est également continue pour la topologie sur \( A\), induite\footnote{Définition \ref{DefVLrgWDB}.} de \( X\).
\end{proposition}

\begin{proof}
	Soit \( a\in A\), et montrons que \( \tilde f\) est continue en \( a\), c'est-à-dire que \( \tilde f(a)=f(a)\) soit une limite de \( \tilde f\) en \( a\). Soit un voisinage \( V\) de \( \tilde f(a)\) dans \( Y\). Par la continuité de \( f\), nous avons un ouvert \( W\) de \( X\) tel que
	\begin{equation}
		f\big( W\setminus\{ a \} \big)\subset V.
	\end{equation}
	La partie \( W\cap A\) est un voisinage de \( a\) pour la topologie de \( A\), et vérifie
	\begin{equation}
		f\big( W\cap A\setminus\{ a \} \big)\subset V.
	\end{equation}
	donc \( f(a)\) est une limite de \( \tilde f\) pour \( x\to a\). La fonction \( \tilde f\colon A\to Y\) est continue en chaque point de \( A\).
\end{proof}

Au niveau de la notion de continuité, il n'y a pas trop de changements en passant de \( \eR\) à \( \eQ\) muni de la topologie induite.

\begin{example}     \label{EXooHWIIooYYbfGE}
	Que signifie d'être continue pour une fonction \( f\colon \eQ\to \eR\) ? D'après le théorème~\ref{ThoESCaraB}, il s'agit d'être continue en chaque point de \( \eQ\). Il s'agit donc, par la définition~\ref{DefOLNtrxB} que pour tout \( q\in \eQ\), le nombre \( f(q)\) soit une limite de \( f\) pour \( x\to q\).

	L'espace d'arrivée étant \( \eR\), un voisinage de \( f(q)\) est pris comme une boule de taille \( \epsilon\). La continuité de \( f\) exige qu'il y ait un voisinage \( W\) de \( q\) dans \( \eQ\) tel que pour tout \( q'\in W\) (différent de \( q\)), \( | f(q)-f(q') |<\epsilon\).

	Qu'est-ce qu'un ouvert dans \( \eQ\) ? D'après la définition~\ref{DefVLrgWDB} de la topologie induite, ce sont les ensembles \( \eQ\cap\mO\) avec \( \mO\) ouvert dans \( \eR\). Tout cela pour dire que pour tout \( \epsilon>0\), il doit exister \( \delta>0\) tel que pour tout \( q'\in \eQ\) tel que \( 0<| q-q' |<\delta\), nous ayons \( | f(q)-f(q') |<\epsilon\).

	Bref, c'est exactement le mécanisme usuel de la continuité sur \( \eR\), sauf qu'il ne faut considérer que les rationnels.
\end{example}

\begin{lemma}[Application partielle\cite{MonCerveau}]       \label{LEMooHAODooYSPmvH}
	Soient trois espaces topologiques \( X_1\), \( X_2\) et \( Y\). Nous considérons une fonction continue \( f\colon X_1\times X_2\to Y\) ainsi que \( x_1\in X_1\). Alors l'application
	\begin{equation}
		\begin{aligned}
			g\colon X_2 & \to Y              \\
			x_2         & \mapsto f(x_1,x_2)
		\end{aligned}
	\end{equation}
	est continue.
\end{lemma}

\begin{proof}
	Soit un ouvert \( \mO\) de \( Y\); par hypothèse sur \( f\), la partie \( f^{-1}(\mO)\) est ouverte dans \( X_1\times X_2\). Notre but est de prouver que \( g^{-1}(\mO)\) est un ouvert de \( X_2\). Nous avons :
	\begin{equation}
		g^{-1}(\mO)=\{ x_2\in X_2\tq (x_1,x_2)\in f^{-1}(\mO) \}.
	\end{equation}
	Nous considérons \( x_2\in g^{-1}(\mO)\) et nous prouvons qu'il existe dans \( X_2\) un voisinage de \( x_2\) entièrement contenu dans \( g^{-1}(\mO)\).

	Étant donné que \( (x_1,x_2)\) est dans \( f^{-1}(\mO)\) qui est ouvert, la définition~\ref{DefIINHooAAjTdY} de la topologie sur \( X_1\times X_2\) nous donne des ouverts \( A_1\) dans \( X_1\) et \( A_2\) dans \( X_2\) tels que
	\begin{equation}
		(x_1,x_2)\in A_1\times A_2\subset f^{-1}(\mO).
	\end{equation}

	Nous montrons à présent que \( A_2\subset g^{-1}(\mO)\). Soit \( y_2\in A_2\). Par construction \( (x_1,y_2)\in A_1\times A_2\subset f^{-1}(\mO)\), donc
	\begin{equation}
		g(y_2)=f(x_1,y_2)\in \mO.
	\end{equation}
	Cela termine la démonstration.
\end{proof}

%---------------------------------------------------------------------------------------------------------------------------
\subsection{Continuité et connexité}
%---------------------------------------------------------------------------------------------------------------------------

\begin{proposition} \label{PropConnexiteViaFonction}
	Un espace topologique \( X \) est connexe si et seulement si toute application continue \( X\to \eZ\) est constante.
\end{proposition}

\begin{proposition}\label{PropGWMVzqb}
	L'image d'un ensemble connexe par une fonction continue est connexe.
\end{proposition}

\begin{proof}
	Soit \( f\colon X\to Y\) une application continue entre deux espaces topologiques, et \( E\) une partie connexe de \( X\). Nous devons montrer que \( f(E)\) est connexe dans \( Y\).

	Par l'absurde nous considérons \( A\) et \( B\), deux ouverts de \( Y\) disjoints recouvrant \( f(E)\). Étant donné que \( f\) est continue, les ensembles \( f^{-1}(A)\) et \( f^{-1}(B)\) sont ouverts dans \( X\). De plus ces deux ensembles recouvrent \( E\).

	Si \( x\) est un élément de \( f^{-1}(A)\cap f^{-1}(B)\), alors \( f(x)\in A\cap B\), ce qui est impossible parce que nous avons supposé que \( A\) et \( B\) étaient disjoints. Par conséquent \( f^{-1}(A)\) et \( f^{-1}(B)\) sont deux ouverts disjoints recouvrant \( E\). Contradiction avec la connexité de \( E\). Nous concluons que \( f(E)\) est connexe.
\end{proof}
Une application de ce théorème sera le théorème des valeurs intermédiaires~\ref{ThoValInter}.

\begin{example}
	Les espaces topologiques \( \eR\) et \( \eR^2\) ne sont pas homéomorphes.
\end{example}

\begin{proof}
	Supposons par l'absurde que \( f\colon \eR\to \eR^2\) soit un  homéomorphisme. Nous posons \( E=f\big( \eR\setminus\{ 0 \} \big)\) et \( z_0=f(0)\). Puisque \( f\) est bijective nous avons
	\begin{equation}
		E=\eR^2\setminus\{ z_0 \},
	\end{equation}
	qui est connexe.

	Comme \( E\) est connexe et que \( f^{-1}\) est continue, la proposition~\ref{PropGWMVzqb} nous dit que \( f^{-1}(E)\) est connexe. Mais par définition, \( f^{-1}(E)=\eR\setminus\{ 0 \}\) qui n'est pas connexe.
\end{proof}

%---------------------------------------------------------------------------------------------------------------------------
\subsection{Continuité et compacité}
%---------------------------------------------------------------------------------------------------------------------------

\begin{theorem}     \label{ThoImCompCotComp}
	L'image d'un compact\footnote{Définition~\ref{DefJJVsEqs}.} par une fonction continue est un compact.
\end{theorem}
Dans le cadre des espaces vectoriels normés, ce théorème est démontré en la proposition~\ref{PropContinueCompactBorne}.

\begin{proof}
	Soit \( K\subset X\), un ensemble compact, et étudions \( f(K)\); en particulier, nous considérons \( \Omega\), un recouvrement de \( f(K)\) par des ouverts. Nous avons
	\begin{equation}
		f(K)\subseteq\bigcup_{\mO\in\Omega}\mO.
	\end{equation}
	Par construction, nous avons aussi
	\begin{equation}
		K\subseteq\bigcup_{\mO\in\Omega}f^{-1}(\mO),
	\end{equation}
	en effet, si \( x\in K\), alors \( f(x)\) est dans un des ouverts de \( \Omega\), disons \( f(x)\in \mO\), et évidemment, \( x\in f^{-1}(\mO)\).  Les \( f^{-1}(\mO)\) recouvrent le compact \( K\), et donc on peut en choisir un sous-recouvrement fini, c'est-à-dire un choix de \( \{ f^{-1}(\mO_1),\ldots,f^{-1}(\mO_n) \}\) tels que
	\begin{equation}
		K\subseteq \bigcup_{i=1}^nf^{-1}(\mO_i).
	\end{equation}
	Dans ce cas, nous avons
	\begin{equation}
		f(K)\subseteq\bigcup_{i=1}^n\mO_i,
	\end{equation}
	ce qui prouve la compacité de \( f(K)\).
\end{proof}


%---------------------------------------------------------------------------------------------------------------------------
\subsection{Topologie et matrices}
%---------------------------------------------------------------------------------------------------------------------------

\begin{lemmaDef}[Topologie sur les matrices]      \label{DEFooCQHDooYpUAhG}
	Si \( \eK\) est un corps valué\footnote{Définition \ref{DEFooBWXXooAkBBRS}.}, alors l'opération
	\begin{equation}
		\begin{aligned}
			\| . \|_{\eM}\colon \eM(n\times m,\eK) & \to \eR^+                             \\
			M                                      & \mapsto  \max_{kl}\| M_{kl} \|_{\eK}.
		\end{aligned}
	\end{equation}
	est une norme\footnote{Définition \ref{DefNorme}.}.

	Cette norme est appelée \defe{norme maximum}{norme maximum} et nous considérons sur \( \eM(n\times m, \eK)\) la topologie associée à cette norme\footnote{Définition \ref{DEFooPMVFooPSYVNQ}.}
\end{lemmaDef}

\begin{proposition}     \label{PROPooOEETooPhqWuf}
	La multiplication matricielle est une opération continue.
\end{proposition}
