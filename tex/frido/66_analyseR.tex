% This is part of Mes notes de mathématique
% Copyright (c) 2006-2023
%   Laurent Claessens, Carlotta Donadello
% See the file fdl-1.3.txt for copying conditions.

%+++++++++++++++++++++++++++++++++++++++++++++++++++++++++++++++++++++++++++++++++++++++++++++++++++++++++++++++++++++++++++
\section{Limite de fonctions}
%+++++++++++++++++++++++++++++++++++++++++++++++++++++++++++++++++++++++++++++++++++++++++++++++++++++++++++++++++++++++++++

%---------------------------------------------------------------------------------------------------------------------------
\subsection{Définition}
%---------------------------------------------------------------------------------------------------------------------------

La définition générale de la limite est~\ref{DefYNVoWBx}. Dans le cas de fonctions \( \eR\to \eR\), elle peut s'écrire de façon plus efficace. La proposition suivante montre comment fonctionne la limite pour une fonction définie sur tout \( \eR\).

\begin{proposition}[Caractérisation de la limite]       \label{PropAJQQooQQClfp}
	Soit une fonction \( f\colon \eR\to \eR\) définie sur \( \eR\) et \( a\in \eR\). La fonction \( f\) admet la limite \( \ell\) pour \( x\to a\) si et seulement si il existe un réel \( \ell\) tel que pour tout \( \epsilon>0\), il existe un \( \delta>0\) tel que
	\begin{equation}\label{EqDefLimiteFonction}
		0<| x-a |<\delta\Rightarrow| f(x)-\ell |<\epsilon.
	\end{equation}
\end{proposition}

\begin{proof}
	Il s'agit de montrer l'équivalence avec la définition~\ref{DefYNVoWBx}. Nous allons faire un usage intensif de la proposition \ref{PROPooZXTXooEMLgMn}.
	\begin{subproof}
		\spitem[Sens direct]
		Soient \( \epsilon>0\) et \( V=B(\ell,\epsilon)\). Alors il existe un voisinage \( W\) de \( a\) dans \( \eR\) tel que
		\begin{equation}
			f\big( W\setminus\{ a \} \big)\subset V.
		\end{equation}
		Soit \( \delta\) tel que \( B(a,\delta)\subset W\). Nous avons encore
		\begin{equation}
			f\big( B(a,\delta)\setminus\{ a \} \big)\subset V.
		\end{equation}
		Soit maintenant \( x\in \eR\) tel que \( 0<| x-a |<\delta\). Cela signifie \( x\in B(a,\delta)\setminus\{ a \}\). Pour un tel \( x\) nous avons donc \( f(x)\in B(\ell,\epsilon)\), c'est-à-dire \( | f(x)-\ell |<\epsilon\).

		\spitem[Dans l'autre sens]
		Soient un voisinage \( V\) de \( \ell\) et \( \epsilon>0\) tel que \( B(\ell,\epsilon)\subset V\). Nous considérons \( \delta\) tel que \( | f(x)-\ell |<\epsilon\) pour tout \( x\) satisfaisant \( 0<| x-a |<\delta\).

		Avec tout cela nous posons \( W=B(x,\delta)\), et nous avons
		\begin{equation}
			f\big( W\setminus\{ a \} \big)\subset B(\ell,\epsilon)\subset V.
		\end{equation}
	\end{subproof}
\end{proof}

Si aucun nombre \( \ell\) ne vérifie la condition de la définition, alors on dit que la fonction n'admet pas de limite en \( a\). Lorsque \( f\) possède la limite \( \ell\) en \( a\), nous notons
\begin{equation}
	\lim_{x\to a} f(x)=\ell.
\end{equation}

La proposition suivante a déjà été démontrée dans la proposition~\ref{PropFObayrf}. Nous en donnons ici une démonstration adaptée au cas \( \eR\to \eR\).

\begin{proposition}
	Soit une fonction \( f\colon D\to \eR\). Si \( a\) est un point d'accumulation de \( D\) et si il existe une limite de \( f\) en \( a\), alors il en existe une seule.
\end{proposition}

\begin{proof}
	Nous prouvons qu'il ne peut pas exister deux nombres \( \ell\neq\ell'\) vérifiant tous les deux la condition \eqref{EqDefLimiteFonction}.

	Soient \( \ell\) et \( \ell'\) deux limites de \( f\) au point \( a\). Par définition, pour tout \( \epsilon\) nous avons des nombres \( \delta\) et \( \delta'\) tels que
	\begin{equation}	\label{EqsContf2307Right}
		\begin{aligned}[]
			| x-a |<\delta  & \Rightarrow \big| f(x)-\ell \big|<\epsilon  \\
			| x-a |<\delta' & \Rightarrow \big| f(x)-\ell' \big|<\epsilon
		\end{aligned}
	\end{equation}
	Pour fixer les idées, supposons que \( \delta<\delta'\) (le cas \( \delta\geq\delta'\) se traite de la même manière).

	Étant donné que \( a\) est un point d'accumulation du domaine \( D\) de \( f\), il existe un \( x\in D\) tel que \( | x-a |<\delta\). Évidemment, nous avons aussi \( | x-a |<\delta'\). Les conditions \eqref{EqsContf2307Right} signifient alors que ce \( x\) vérifie en même temps
	\begin{equation}
		| f(x)-\ell |<\epsilon,
	\end{equation}
	et
	\begin{equation}
		| f(x)-\ell' |<\epsilon.
	\end{equation}
	Afin de prouver que \( \ell=\ell'\), nous allons maintenant calculer \( | \ell-\ell' |\) et montrer que cette distance est plus petite que tout nombre. Nous avons (voir remarque~\ref{RemTechniqueIneqs})
	\begin{equation}	\label{EqInesq2307ellellepr}
		| \ell-\ell' |=| \ell-f(x)+f(x)-\ell' |\leq | \ell-f(x) |+| f(x)-\ell' |<\epsilon+\epsilon.
	\end{equation}
	En résumé, pour tout \( \epsilon>0\) nous avons
	\begin{equation}
		| \ell-\ell' |<2\epsilon,
	\end{equation}
	et donc \( | \ell-\ell' |=0\), ce qui signifie que \( \ell=\ell'\).
\end{proof}

\begin{remark}		\label{RemTechniqueIneqs}
	Les inégalités \eqref{EqInesq2307ellellepr} utilisent deux techniques très classiques en analyse qu'il convient d'avoir bien compris. La première est de faire
	\begin{equation}
		| A-B |=| A-C+C-B |.
	\end{equation}
	Il s'agit d'ajouter \( -C+C\) dans la norme. Évidemment, cela ne change rien.

	La seconde technique est l'inégalité
	\begin{equation}
		| A+B |\leq| A |+| B |.
	\end{equation}
\end{remark}

\begin{example}
	Considérons la fonction \( f(x)=2x\), et calculons la limite \( \lim_{x\to 3} f(x)\). Vu que \( f(3)=6\), nous nous attendons à avoir \( \ell=6\). C'est ce que nous allons prouver maintenant. Pour chaque \( \epsilon>0\) nous devons trouver un \( \delta>0\) tel que \( | x-3 |<\delta\) implique \( | f(x)-6 |<\epsilon\). En remplaçant \( f(x)\) par sa valeur en fonction de \( x\) et avec quelques manipulations nous trouvons :
	\begin{equation}
		\begin{aligned}[]
			| f(x)-6 | & <\epsilon             \\
			| 2x-6 |   & <\epsilon             \\
			2| x-3 |   & <\epsilon             \\
			| x-3 |    & <\frac{ \epsilon }{2}
		\end{aligned}
	\end{equation}
	Donc dès que \( | x-3 |<\frac{ \epsilon }{2}\), nous avons \( | f(x)-6 |<\epsilon\). Nous posons donc \( \delta=\frac{ \epsilon }{2}\).

	Plus généralement, nous avons \( \lim_{x\to a} f(x)=2a\), et cela se prouve en étudiant \( | f(x)-2a |\) exactement de la même manière.
\end{example}

%---------------------------------------------------------------------------------------------------------------------------
\subsection{Quelques règles de calcul}
%---------------------------------------------------------------------------------------------------------------------------

Les opérations simples passent à la limite, sauf la division pour laquelle il faut faire attention au dénominateur.
\begin{proposition}     \label{PropOpsSimplesLimites}
	Soient deux fonctions \( f,g\colon \eR\to \eR\). Nous supposons que \( \lim_{x\to a} f(x)=\alpha\) et \( \lim_{x\to a} g(x)=\beta\). Alors
	\begin{enumerate}
		\item   \label{ITEMooOJUWooQpqqnQ}
		      La fonction \( f+g\) a une limite \( x\to a\) qui vaut \( \lim_{x\to a} (f+g)(x)=\alpha+\beta\),
		\item
		      La fonction \( fg\) a une limite en \( a\), qui vaut \( \lim_{x\to a} (fg)(x)=\alpha\beta\),
		\item
		      si il existe un voisinage de \( a\) sur lequel \( g\) ne s'annule pas, et si \( \beta\neq 0\), alors la fonction \( f/g\) a une limite en \( a\) qui vaut \( \lim_{x\to a} \frac{ f(x) }{ g(x) }=\frac{ \alpha }{ \beta }\).
	\end{enumerate}
\end{proposition}

\begin{proof}
	En plusieurs points.
	\begin{subproof}
		\spitem[La somme]
		% -------------------------------------------------------------------------------------------- 
		Soit \( \epsilon>0\). Soient \( \delta_1>0\) et \( \delta_2>0\) tels que
		\begin{subequations}
			\begin{align}
				0<| x-a |<\delta_1\Rightarrow | f(x)-\alpha |<\epsilon \\
				0<| x-a |<\delta_2\Rightarrow | g(x)-\beta |<\epsilon.
			\end{align}
		\end{subequations}
		En prenant \( \delta\leq \min\{ \delta_1,\delta_2 \}\), si \( 0<| x-a |\leq \delta\) alors
		\begin{equation}
			| f(x)+g(x)-\alpha-\beta |\leq | f(x)-\alpha |+| g(x)-\beta |\leq 2\epsilon.
		\end{equation}
		\spitem[Le produit]
		% -------------------------------------------------------------------------------------------- 
		Nous prenons \( \epsilon\), \( \delta_1\) et \( \delta_2\) dans les mêmes conditions que pour la somme. Nous majorons comme ceci :
		\begin{subequations}
			\begin{align}
				| f(x)g(x)-\alpha\beta | & \leq | f(x)g(x)-\alpha g(x) |+| \alpha g(x)-\alpha\beta | \\
				                         & =| g(x) | |f(x)-\alpha |+| \alpha | |g(x)-\beta |         \\
				                         & \leq   \epsilon\big(  | g(x) |  +| \alpha |   \big)       \\
				                         & \leq \epsilon\big( | \beta |+\epsilon+| \alpha | \big).
			\end{align}
		\end{subequations}
		La dernière ligne majore \( | g(x) |\leq | \beta |+\epsilon\).
		\spitem[Le quotient]
		% -------------------------------------------------------------------------------------------- 
		Soit \( \epsilon>0\). Nous considérons \( \delta>0\) tel que nous ayons simultanément \( | f(x)-\alpha |<\epsilon\), \( | g(x)-\beta |<\epsilon\) et \( g(x)\neq 0\) pour tout \( x\in B(a,\delta)\). Nous avons alors le calcul
		\begin{subequations}
			\begin{align}
				| \frac{ f(x) }{ g(x) }-\frac{ \alpha }{ \beta } | & \leq | \frac{ f(x) }{ g(x) }-\frac{ \alpha }{ g(x) } |+| \frac{ \alpha }{ g(x) }-\frac{ \alpha }{ \beta } |            \\
				                                                   & \leq | \frac{1}{ g(x) } | |f(x)-\alpha |+| \alpha | |\frac{1}{ g(x) }-\frac{1}{ \beta } |  \label{SUBEQooOHFHooWZrwcA}
			\end{align}
		\end{subequations}
		Nous avons la majoration
		\begin{equation}
			\frac{1}{ | g(x) | }\leq \frac{1}{ | \beta |-\epsilon },
		\end{equation}
		et comme \( \beta\neq 0\), nous pouvons supposer que \( \epsilon\) est assez petit pour que le dénominateur ne s'annule pas. En ce qui concerne le second terme de \eqref{SUBEQooOHFHooWZrwcA},
		\begin{equation}
			\frac{1}{ g(x) }-\frac{1}{ \beta }=\frac{ \beta-g(x) }{ g(x)\beta }.
		\end{equation}
		Ici aussi nous pouvons majorer \( 1/g(x)\). Le numérateur se majore par \( \epsilon\). En remettant tout ça dans \eqref{SUBEQooOHFHooWZrwcA}, nous avons une majoration qui tend vers zéro lorsque \( \epsilon\to 0\).
	\end{subproof}
\end{proof}

Le résultat suivant est pratique pour le calcul des limites.
\begin{proposition}     \label{PropChmVarLim}
	Quand la limite existe, nous avons
	\[
		\lim_{x\to a}f(x)=\lim_{\epsilon\to 0}f(a+\epsilon),
	\]
	ce qui correspond à un «changement de variables» dans la limite.
\end{proposition}

\begin{proof}
	Si \( A=\lim_{x\to a}f(x)\), par définition,
	\begin{equation}        \label{EqCondFaplusespLim}
		\forall\epsilon'>0,\,\exists\delta \tq | x-a |\leq\delta\Rightarrow| f(x)-A |\leq\epsilon'.
	\end{equation}
	La seule subtilité de la démonstration est de remarquer que si \( | x-a |\leq\delta\), alors \( x\) peut être écrit sous la forme \( x=a+\epsilon\) pour un certain \( | \epsilon |\leq\delta\). En remplaçant \( x\) par \( a+\epsilon\) dans la condition~\ref{EqCondFaplusespLim}, nous trouvons
	\begin{equation}
		\forall\epsilon'>0,\,\exists\delta \tq | \epsilon |\leq\delta\Rightarrow| f(a+\epsilon)-A |\leq\epsilon',
	\end{equation}
	ce qui signifie exactement que \( \lim_{\epsilon\to 0}f(a+\epsilon)=A\).
\end{proof}

Il y a une petite différence de point de vue entre \( \lim_{x\to a}f(x)\) et \( \lim_{\epsilon\to 0}f(a+\epsilon)\). Dans le premier cas, on considère \( f(x)\), et on regarde ce qu'il se passe quand \( x\) se rapproche de \( a\), tandis que dans le second, on considère \( f(a)\), et on regarde ce qu'il se passe quand on s'éloigne un tout petit peu de \( a\). Dans un cas, on s'approche très près de \( a\), et dans l'autre on s'en éloigne un tout petit peu. Le contenu de la proposition~\ref{PropChmVarLim} est de dire que ces deux points de vue sont équivalents.

% Il y a des techniques de calcul de limites décrites sur le site
% http://bernard.gault.free.fr/terminale/limites/limite.html

En plus d'être linéaire, la limite possède les deux propriétés suivantes.
\begin{proposition}     \label{PROPooDQFIooMMwxxJ}
	Si \( f\) et \( g\) sont deux fonctions qui admettent une limite en \( a\), alors
	\begin{equation}
		\lim_{x\to a} (fg)(x)=\lim_{x\to a} f(x)\cdot\lim_{x\to a} g(x).
	\end{equation}
	Si de plus \( \lim_{x\to a} g(x)\neq 0\), alors
	\begin{equation}
		\lim_{x\to a} \left(\frac{ f }{ g }\right)(x)=\frac{ \lim_{x\to a} f(x) }{ \lim_{x\to a} g(x) }.
	\end{equation}
\end{proposition}

\begin{theorem}     \label{ThoLimLinMul}
	Si
	\begin{equation} \label{Eqhypmullimlin}
		\lim_{x\to a}f(x)=b,
	\end{equation}
	alors
	\begin{equation} \label{Eqbutmultlim}
		\lim_{x\to a}(\lambda f)(x)=\lambda b
	\end{equation}
	pour n'importe quel \( \lambda\in\eR\).
\end{theorem}

\begin{proof}
	Soit \( \epsilon>0\). Afin de prouver la propriété \eqref{Eqbutmultlim}, il faut trouver un \( \delta\) tel que pour tout \( x\) dans \( [a-\delta,a+\delta]\), on ait \( | (\lambda f)(x)- \lambda b |\leq\epsilon\). Cette dernière inégalité est équivalente à \( |\lambda|| f(x)-b |\leq\epsilon\). Nous devons donc trouver un \( \delta\) tel que
	\begin{equation}
		| f(x)-b |\leq\frac{ \epsilon }{ | \lambda | }.
	\end{equation}
	soit vraie pour tout \( x\) dans \( [a-\delta,a+\delta]\). Mais l'hypothèse \eqref{Eqhypmullimlin} dit précisément qu'il existe un \( \delta\) tel que pour tout \( x\) dans \( [a-\delta,a+\delta]\) on ait cette inégalité.
\end{proof}

\begin{theorem}     \label{ThoLimLin}
	Si
	\begin{subequations}
		\begin{align}
			\lim_{x\to a}f(x) & = b_1  \\
			\lim_{x\to a}g(x) & = b_2,
		\end{align}
	\end{subequations}
	alors
	\begin{equation}
		\lim_{x\to a}(f+g)(x)=b_1+b_2.
	\end{equation}
\end{theorem}

\begin{proof}
	Soit \( \epsilon>0\). Par hypothèse, il existe \( \delta_1\) tel que
	\begin{equation}    \label{Eqfbunepsdeux}
		| f(x)-b_1 |\leq \frac{ \epsilon }{ 2 }
	\end{equation}
	dès que \( | x-a |\leq\delta_1\). Il existe aussi \( \delta_2\) tel que
	\begin{equation}    \label{Eqgbdeuxepsdeux}
		| g(x)-b_2 |\leq \frac{ \epsilon }{ 2 }.
	\end{equation}
	dès que \( | x-a |\leq \delta_2\). Notons l'astuce de prendre \( \epsilon/2\) dans la définition de limite pour \( f\) et \( g\). Maintenant, ce qu'on voudrait c'est un \( \delta\) tel que l'on ait \( | (f+g)(x)-(b_1+b_2) |\leq \epsilon\) dès que \( | x-a |\leq \delta\). Constatons que \( \delta=\min\{ \delta_1,\delta_2 \}\) fonctionne. En effet, en utilisant l'inégalité \( | a+b |\leq | a |+| b |\), nous trouvons :
	\begin{align}
		| (f+g)(x)-(b_1+b_2) |=| (f(x)-b_1)+(g(x)-b_2) |
		\leq | f(x)-b_1 |+| g(x)-b_2 |.     \label{Eqfplusgfbun}
	\end{align}
	Comme on suppose que \( | x-a |\leq\delta\), on a évidemment \( | x-a |\leq\delta_1\), et donc l'équation \eqref{Eqfbunepsdeux} tient. Mais si \( | x-a |\leq\delta\), on a aussi \( | x-a |\leq\delta_2\), et donc l'équation  \eqref{Eqgbdeuxepsdeux} tient également. Chacun des deux termes de \eqref{Eqfplusgfbun} est donc plus petits que \( \epsilon/2\), et donc, le tout, est plus petit que \( \epsilon\), ce qu'il fallait montrer.
\end{proof}

\begin{proposition}     \label{PROPooVLBWooVttvFK}
	La limite est linéaire : pour toutes fonctions \( f\) et \( g\) admettant une limite en \( a\) et pour tout réels \( \lambda\) et \( \mu\).
	\begin{equation}
		\lim_{x\to a} (\lambda f+\mu g)(x)=\lambda\lim_{x\to a} f(x)+\mu\lim_{x\to a} g(x).
	\end{equation}
\end{proposition}

\begin{proof}
	Il s'agit seulement des deux propriétés des théorèmes \ref{ThoLimLinMul} et \ref{ThoLimLin}.
\end{proof}

\begin{lemma}       \label{LEMooYJGLooVBaglB}
	Soient un espace vectoriel normé \( E\) ainsi qu'une fonction \( f\colon \eR\to E\) telle que \( \lim_{t\to 0} f(t)=v\). Alors pour tout \( \lambda\in \eR\) nous avons
	\begin{equation}
		\lim_{t\to 0} f(\lambda t)=v.
	\end{equation}
\end{lemma}

\begin{proof}
	Nous utilisons la caractérisation \eqref{PropAJQQooQQClfp} de la limite. Soit \( \epsilon>0\). Soit \( \delta>0\) tel que \( \| f(t)-v \|<\epsilon\) pour tout \( | t |<\delta\). Nous considérons alors \( \delta'=\delta/| \lambda |\).

	Si \( | t |<\delta'\), alors \( | \lambda t |<\delta\) et nous avons bien \( \| f(\lambda t)-v \|<\epsilon\).
\end{proof}

\begin{proposition}[\cite{TrenchRealAnalisys}]      \label{PROPooOUPNooTrClHw}
	Soient des fonctions \( f,g\colon \eR\to \eR\) telles que \( \lim_{x\to a} f(x)=\ell\) et \( \lim_{x\to a} g(x)=\ell'\neq 0\). Alors
	\begin{equation}
		\lim_{x\to a} \frac{ f(x) }{ g(x) }=\frac{ \ell }{ \ell' }.
	\end{equation}
\end{proposition}

\begin{proof}
	Nous avons :
	\begin{equation}
		\left| \frac{ f(x) }{ g(x) }-\frac{ \ell }{ \ell' } \right| =\frac{ | \ell'f(x)-g(x)\ell | }{ |g(x)\ell'| }.
	\end{equation}
	Soit \( s\), un minorant de \( | g(x) |\) sur un voisinage de \( a\); puisque la limite en \( a\) est \( \ell'\neq 0\), nous pouvons prendre par exemple \( s=|\ell'|/2\) : \( | g(x) |>|\ell'|/2\) sur \( B(a,\delta)\) dès que \( \delta\) est assez petit. Nous considérons \( x\in B(a,\delta)\). Avec cela nous avons :
	\begin{subequations}
		\begin{align}
			\left| \frac{ f(x) }{ g(x) }-\frac{ \ell }{ \ell' } \right| & =\frac{ | \ell'f(x)-g(x)\ell | }{ |g(x)\ell'| }                                          \\
			                                                            & \leq \frac{ 2 }{ | \ell' |} \left( \frac{ | \ell'f(x)-g(x)\ell | }{ | \ell' | } \right)  \\
			                                                            & \leq \frac{ 2 }{ | \ell' |^2 }\big( | \ell'f(x)-\ell\ell' |+| \ell\ell'-g(x)\ell | \big) \\
			                                                            & =\frac{ 2 }{ | \ell' |^2 }\big( | \ell' | |f(x)-\ell |+| \ell | |\ell'-g(x) | \big).
		\end{align}
	\end{subequations}
	Nous avons utilisé la règle du produit de valeurs absolues du lemme \ref{LemooANTJooYxQZDw}\ref{ITEMooEFMLooYVCuHD}.

	Soient \( \epsilon>0\) et \( \delta\) tel que \( | f(x)-\ell |<\epsilon\) et \( | g(x)-\ell' |<\epsilon\) pour tout \( x\in B(a,\delta)\). Avec cela nous avons
	\begin{equation}
		\left| \frac{ f(x) }{ g(x) }-\frac{ \ell }{ \ell' } \right| \leq\frac{ 2 }{ | \ell' |^2\big( | \ell' |+| \ell | \big) }\epsilon.
	\end{equation}
	D'où la limite attendue.
\end{proof}

\begin{lemma}       \label{LemLimMajorableVois}
	Si \( \lim_{x\to a}f(x)=b\) avec \( a\), \( b\in\eR\), alors il existe un \( \delta>0\) et un \( M>0\) tels que
	\[
		(| x-a |\leq\delta)\Rightarrow | f(x) |\leq M.
	\]
\end{lemma}

Ce que signifie ce lemme, c'est que quand la fonction \( f\) admet une limite finie en un point, alors il est possible de majorer la fonction sur un intervalle autour du point.

\begin{proof}
	Cela va être démontré par l'absurde. Supposons qu'il n'existe pas de \( \delta\) ni de \( M\) qui vérifient la condition. Dans ce cas, pour tout \( \delta\) et pour tout \( M\), il existe un \( x\) tel que \( | x-a |\leq\delta\) et \( | f(x) |> M\). Ceci est valable pour tout \( M\), donc, prenons par exemple, \( b+1000\). Ainsi
	\begin{equation}
		\forall\delta>0,\exists x \tq | x-a |\leq\delta\text{ et }| f(x) |>b+1000.
	\end{equation}
	Cela signifie qu'aucun \( \delta\) ne peut convenir dans la définition de \( \lim_{x\to a}f(x)=b\), ce qui contredit les hypothèses.
\end{proof}

Dans le même ordre d'idée, on peut prouver que si la limite de la fonction en un point est positive, alors elle est positive autour de ce point. Plus précisément, nous avons la proposition suivante.

\begin{proposition} \label{PropoLimPosFPos}
	Si \( f\) est une fonction telle que \( \lim_{x\to a}f(x)>0\), alors il existe un voisinage de \( a\) sur lequel \( f\) est positive.
\end{proposition}

\begin{proof}
	Supposons que \( \lim_{x\to a}f(x)=y_0\). Par la définition de la limite, nous avons pour tout \( x\) dans un voisinage autour de \( a\), \( | f(x)-a |<\epsilon\). Ceci est valable pour tout \( \epsilon\), pourvu que le voisinage soit assez petit. Si nous choisissons un voisinage pour lequel \( | f(x)-a |<\frac{ y_0 }{ 2 }\), alors, sur ce voisinage, \( f\) est positive.
\end{proof}

\begin{theorem}     \label{Tholimfgabab}
	Si
	\begin{align}
		\lim_{x\to a}f(x) & =b_1 & \text{et} &  & \lim_{x\to a}g(x)=b_2,
	\end{align}
	alors
	\begin{equation}
		\lim_{x\to a}(fg)(x)=b_1b_2.
	\end{equation}
\end{theorem}

\begin{proof}
	Soit \( \epsilon>0\), et tentons de trouver un \( \delta\) tel que \( | f(x)g(x)-b_1b_2 |\leq \epsilon\) dès que \( | x-a |\leq \delta\). Nous avons
	\begin{equation}    \label{EqfgbunbdeuxMin}
		\begin{split}
			| f(x)g(x)-b_1b_2 |&=|  f(x)g(x)-b_1b_2 +f(x)b_2-f(x)b_2 |\\
			&=\left|   f(x)\big( g(x)-b_2 \big)+b_2\big( f(x)-b_1 \big)    \right|\\
			&\leq \left|  f(x)\big( g(x)-b_2 \big)  \right|+\left| b_2\big( f(x)-b_1 \big) \right| \\
			&= | f(x) | | g(x)-b_2  |+| b_2 | |f(x)-b_1 |.
		\end{split}
	\end{equation}
	Maintenant, nous savons par le lemme~\ref{LemLimMajorableVois} que pour un certain \( \delta_1\), la quantité \( | f(x) |\) peut être majorée par un certain \( M\) dès que \( | x-a |\leq \delta_1\). Prenons donc un tel \( \delta_1\), et supposons que \( | x-a |\leq \delta_1\). Nous savons aussi que pour n'importe quel choix de \( \epsilon_2\) et \( \epsilon_3\), il existe des nombres \( \delta_2\) et \( \delta_3\) tels que \( | f(x)-b_1 |\leq \epsilon_2\) et \( | g(x)-b_1 |\leq \epsilon_3\) dès que \( | x-a |\leq\delta_2\) et \( | x-a |\leq\delta_3\). Dans ces conditions, la dernière expression \eqref{EqfgbunbdeuxMin} se réduit à
	\begin{equation}
		| f(x)g(x)-b_1b_2 |\leq M\epsilon_2+| b_2 |\epsilon_3.
	\end{equation}
	Pour terminer la preuve, il suffit de choisir \( \epsilon_2\) et \( \epsilon_3\) tels que \( M\epsilon_2+| b_2 |\epsilon_3\leq\epsilon\), et puis prendre \( \delta=\min\{ \delta_1,\delta_2,\delta_3 \}\).

	Remettons les choses dans l'ordre. On se donne un \( \epsilon\) au départ. La première chose est de trouver un \( \delta_1\) qui permette de majorer \( |f(x)|\) par \( M\) selon le lemme~\ref{LemLimMajorableVois}, et puis, choisissons \( \epsilon_2\) et \( \epsilon_3\) tels que \( M\epsilon_2+| b_2 |\epsilon_3\leq\epsilon\). Ensuite, nous prenons, en vertu des hypothèses de limites pour \( f\) et \( g\), les nombres \( \delta_2\) et \( \delta_3\) tels que \( | f(x)-b_1 |\leq \epsilon_2\) et \( | g(x)-b_2 |\leq \epsilon_3\) dès que \( | x-a |\leq \delta_2\) et \( | x-a |\leq \delta_3\).

	Si avec tout ça on prend \( \delta=\min\{ \delta_1,\delta_2,\delta_3 \}\), alors la majoration et les deux inégalités sont valables en même temps, et au final
	\[
		| f(x)g(x)-b_1b_2 |\leq M\epsilon_2+|b_2|\epsilon_3 \leq \epsilon,
	\]
	ce qu'il fallait prouver.
\end{proof}

À l'aide de ces petits résultats, nous pouvons déjà calculer pas mal de limites. Nous pouvons déjà par exemple, calculer les limites de tous les polynômes en tous les nombres réels. En effet, nous connaissons la limite de la fonction \( f(x)=x\). La fonction \( x\mapsto x^2\) n'est rien d'autre que le produit de \( f\) par elle-même. Donc
\[
	\lim_{x\to a}x^2=\big( \lim_{x\to a}x\big)\cdot\big( \lim_{x\to a}x \big)=a^2.
\]
De la même façon, nous trouvons facilement que
\begin{equation}
	\lim_{x\to a}x^n=a^n.
\end{equation}

\begin{lemma}       \label{LEMooLSJZooDauTkc}
	Soit \( t\in \eR\). Nous avons
	\begin{equation}
		\lim_{x\to \infty} \frac{1}{ x }\integer(xt)=t
	\end{equation}
	où \( \integer\) est la fonction partie entière définie en \ref{LEMooLEXTooGAQxGB}.
\end{lemma}

\begin{proof}
	En vertu du lemme \ref{LEMooLEXTooGAQxGB}, nous avons \( y=\integer(y)+\rational(y)\) avec \( \rational(y)\in \mathopen[ 0 , 1 \mathclose[\). Nous avons donc
	\begin{subequations}
		\begin{align}
			\frac{1}{ x }\integer(xt) & =\frac{1}{ x }\big( xt-\rational(xt) \big)    \\
			                          & =\frac{ xt }{ x }+\frac{ \rational(xt) }{ x } \\
			                          & =t+\frac{ \rational(xt) }{ x }.
		\end{align}
	\end{subequations}
	Puisque \( \rational(xt)\in \mathopen[ 0 , 1 \mathclose[\), nous avons
	\begin{equation}
		\lim_{x\to \infty} \frac{ \rational(xt) }{ x }=0.
	\end{equation}
	D'où la limite demandée.
\end{proof}


%+++++++++++++++++++++++++++++++++++++++++++++++++++++++++++++++++++++++++++++++++++++++++++++++++++++++++++++++++++++++++++
\section{Limites pointées et épointées}
%+++++++++++++++++++++++++++++++++++++++++++++++++++++++++++++++++++++++++++++++++++++++++++++++++++++++++++++++++++++++++++
\label{SECooNJSGooGSAtdV}

La limite d'une fonction en un point a déjà été définie en \ref{DefYNVoWBx}. Nous introduisons maintenant une notion très ressemblante.

\begin{definition}[limite pointée\cite{ooCNVFooHdbArS}]     \label{DEFooBAPHooUtIaRS}
	Soient \( X\) et \( Y\) deux espaces topologiques, \( A\) une partie de \( X\), \( f\) une application de \( A\) dans \( Y\), \( a\) un point de \( X\) adhérent à \( A\) et \(\ell \) un point de \( Y\). On dit que \( \ell\) est une \defe{limite pointée}{limite pointée} de \( f\) au point \( a\) si pour tout voisinage \( V\) de \( \ell\), il existe un voisinage \( W\) de a tel que pour tout point \( x\) de \( W\cap A\), l'image \( f(x)\) appartient à \( V\).
\end{definition}

Quelques notations et vocabulaire.
\begin{enumerate}
	\item
	      Nous allons limiter notre discussion au cas des fonctions \( \eR\to \eR\).
	\item
	      La limite de la définition \ref{DefYNVoWBx} sera provisoirement nommée \defe{limite épointée}{limite épointée}, pour ne pas causer de confusion.
	\item
	      Pour bien distinguer la limite pointée de la limite épointée, nous allons noter \( {LP}_{x\to a}f(x)\) pour la limite pointée et \( {LE}_{x\to a}f(x)\) pour la limite épointée.
	\item
	      Nous allons utiliser la caractérisation \ref{PROPooVNGEooPwbxXP} de la continuité de \( f\) en un point.
	\item
	      Nous allons utiliser la caractérisation \ref{PropAJQQooQQClfp} de la limite épointée.
\end{enumerate}

\begin{lemma}       \label{LEMooWAZLooDPvemu}
	Si une fonction \( f\colon \eR\to \eR\) vérifie \( {LP}_{x\to a} f(x)=\ell\), alors elle vérifie \( {LE}_{x\to a}f(x)=\ell\).
\end{lemma}

\begin{proof}
	Soit \( \epsilon>0\). L'existence de la limite pointée dit qu'il existe \( \delta>0\) tel que \( | x-a |<\delta\) implique \( | f(x)-\ell |<\epsilon\). À fortiori, si \( 0<| x-a |<\delta\) nous avons aussi \( | f(x)-\ell |<\epsilon\). Donc la limite épointée, par la caractérisation \ref{PropAJQQooQQClfp}.
\end{proof}

La réciproque n'est pas vraie, comme le montre le lemme suivant.
\begin{lemma}[\cite{BIBooDILKooUcmUVD}]     \label{LEMooOSNGooJpiXbK}
	Soit la fonction
	\begin{equation}
		\begin{aligned}
			f\colon \eR & \to \eR                        \\
			x           & \mapsto \begin{cases}
				                      0 & \text{si } x\neq 0 \\
				                      1 & \text{si }x=0.
			                      \end{cases}
		\end{aligned}
	\end{equation}
	Nous avons
	\begin{enumerate}
		\item       \label{ITEMooNRNCooFhbZwB}
		      \( {LE}_{x\to 0}f(x)=0\).
		\item       \label{ITEMooUSWMooMNPMCT}
		      La limite pointée de \( f\) en \( 0\) n'existe pas.
	\end{enumerate}
\end{lemma}

\begin{proof}
	En deux parties.
	\begin{subproof}
		\spitem[Pour \ref{ITEMooNRNCooFhbZwB}]
		Pour tout \( x\in B(0,\delta)\setminus\{ 0 \}\) nous avons \( f(x)=0\). Donc la limite épointée suit.
		\spitem[Pour \ref{ITEMooUSWMooMNPMCT}]
		Soit \( \delta>0\). Le lemme \ref{LemooHLHTooTyCZYL} nous assure qu'il existe \( x\in \mathopen] -\delta , 0 \mathclose[\). Ce \( x\) est dans \( B(0,\delta)\) et vérifie \( f(x)=0\). Nous avons aussi \( x=0\) dans la même boule. Donc \( f\big( B(0,\delta) \big)\) contient au moins les nombres \( 1\) et \( 0\).

		Il n'existe donc pas de \( \ell\) tel que tout \( f\big( B(0,\delta) \big)\) soit dans \( B(\ell, \epsilon)\).
	\end{subproof}
\end{proof}

\begin{lemma}\label{LEMooTPNEooRurTJJ}
	Soit une fonction \( f\colon \eR\setminus\{ a \}\to \eR\). Alors la limite pointée de \( f\) en \( a\) existe si et seulement si la limite épointée existe. Dans ce cas, elles sont égales.
\end{lemma}

\begin{proof}
	En trois parties. Nous notons \( \Omega=\eR\setminus\{ a \}\) le domaine de \( f\).
	\begin{subproof}
		\spitem[Si la limite pointée existe, alors l'épointée existe]
		Soit \( \epsilon>0\). L'existence de la limite pointée dit qu'il existe \( \delta>0\) tel que \( f\big( B(a,\delta)\cap \Omega \big)\subset  B(\ell,\epsilon)\). Comme \( B(a,\delta)\cap\Omega=B(a,\delta)\setminus\{ a \}\), nous avons la condition épointée.

		\spitem[Si la limite épointée existe, alors la pointée existe]
		Du même tonneau.

		\spitem[Égalité]
		Le jeu des boules du premier point prouve l'égalité au passage.
	\end{subproof}
\end{proof}

Que faut-il donc ajouter à la limite épointé pour obtenir une limite pointée ? Réponse dans le lemme suivant.
\begin{lemma}
	Soit une fonction \( f\colon \eR\to \eR\). Nous avons
	\begin{equation}
		{LP}_{x\to a}f(x)=\ell
	\end{equation}
	si et seulement si les deux conditions suivantes sont réunies :
	\begin{enumerate}
		\item
		      \( {LE}_{x\to a}f(x)=\ell\)
		\item
		      Soit \( f\) n'existe pas en \( a\), soit \( f\) est continue en \( a\).
	\end{enumerate}
\end{lemma}

\begin{proof}
	En deux parties.
	\begin{subproof}
		\spitem[\( \Rightarrow\)]
		L'existence d'une limite pointée implique celle de la limite épointée, et l'égalité entre les deux. Donc \( {LE}_{x\to a}f(x)=\ell\). Supposons que \( f\) existe en \( a\). Soit \( \epsilon>0\). L'existence d'une limite pointée signifie qu'il existe \( \delta>0\) tel que \( | x-a |<\delta\) implique \( | f(x)-\ell |<\epsilon\).

		En particulier avec \( x=a\) nous avons \( | f(a)-\ell |<\epsilon\) pour tout \( \epsilon\). Donc \( f(a)=\ell\).
		\spitem[\( \Leftarrow\)]
		Nous supposons que \( {LE}_{x\to a}f(x)=\ell\). Si \( f\) n'existe pas en \( a\), alors les limites pointée et épointée coïncident\footnote{Lemme \ref{LEMooTPNEooRurTJJ}.}. Si par contre \( f\) existe en \( a\) et \( f(a)=\ell\), alors nous travaillons.

		Soit \( \epsilon>0\). Il existe \( \delta>0\) tel que \(  x\in B(a,\delta)\setminus \{ a \}  \) implique \( | f(x)-\ell |<\epsilon\). Mais si \( x=a\) nous avons \( | f(x)-\ell |=0<\epsilon\). Au final nous avons \( | f(x)-\ell |<\epsilon\) pour tout \( x\in B(a,\delta)\). Donc \( {LP}_{x\to a}f(x)=\ell\).
	\end{subproof}
\end{proof}

\begin{lemma}[limite et continuité]     \label{LEMooNEGOooCllIMN}
	Supposons que \( {LE}_{x\to a}f(x)=\ell\) et que \( f\) est continue en \( a\). Alors \( f(a)=\ell\).
\end{lemma}

\begin{proof}
	Soit \( \epsilon>0\). L'hypothèse sur la limite dit qu'il existe \( \delta_1>0\) tel que \( 0<| x-a |<\delta_1\) implique \( | f(x)-\ell |<\epsilon\).

	L'hypothèse de continuité, avec la caractérisation \ref{PROPooVNGEooPwbxXP}, dit qu'il existe \( \delta_2>0\) tel que \( | x-a |<\delta_2\) implique \( | f(x)-f(a) |<\epsilon\).

	Nous considérons un \( \delta>0\) plus petit que \( \delta_1\) et que \( \delta_2\). Soit aussi un \( x\) satisfaisant \( 0<| x-a |<\delta\). Nous avons
	\begin{equation}
		| f(a)-\ell |\leq| f(a)-f(x) |+| f(x)-\ell |\leq 2\epsilon.
	\end{equation}
	Puisque cela est valide pour tout \( x\), nous déduisons que \( f(a)=\ell\).
\end{proof}

%---------------------------------------------------------------------------------------------------------------------------
\subsection{Théorèmes de composition de limites}
%---------------------------------------------------------------------------------------------------------------------------

La proposition suivante formalise le fait que la limite pointée soit stable par composition.
\begin{theorem}     \label{THOooOYXDooKDPkuW}
	Soient des fonctions \( g\colon \eR\to \eR\) et \( f\colon \Omega\to \eR\) avec \( \Omega\subset \eR\) telles que
	\begin{enumerate}
		\item
		      \( {LP}_{x\to a}g(x)=\ell\)
		\item
		      \( {LP}_{y\to \ell}f(y)=b\).
		\item
		      \( g(\eR)\subset \Omega\)
	\end{enumerate}
	Alors
	\begin{equation}
		{LP}_{x\to a} (f\circ g)(x)=b.
	\end{equation}
\end{theorem}

\begin{proof}
	Soit \( \epsilon>0\). L'hypothèse de limite pour \( f\) donne \( \eta>0\) tel que
	\begin{equation}        \label{EQooLWGIooLqKThy}
		| y-\ell |<\eta \Rightarrow | f(y)-b |<\epsilon.
	\end{equation}

	Soit \( \delta>0\) tel que \( | x-a |<\delta\) implique \( | g(x)-\ell |<\eta\).

	Avec tout ça, si \( | x-a |<\delta\) nous avons \( | g(x)-\ell |<\eta\) et en appliquant l'implication \eqref{EQooLWGIooLqKThy} à \( y=g(x)\) nous trouvons \( | f\big( g(x) \big)-b |<\epsilon\).
\end{proof}

Le théorème suivant, qui traite de la composition de limites pointées, montre que la limite épointée ne passe pas gratuitement à la composition.
\begin{theorem}     \label{THOooNPBQooEMOYpd}
	Soient des fonctions \( g\colon \eR\to \eR\) et \( f\colon \Omega\to \eR\) avec \( \Omega\subset \eR\) telles que
	\begin{enumerate}
		\item
		      \( {LE}_{x\to a}g(x)=\ell\)
		\item
		      \( {LE}_{y\to \ell}f(y)=b\).
		\item
		      \( g(\eR)\subset \Omega\)
		\item       \label{ITEMooUNJAooCDOKcO}
		      Soit \( \ell\notin \Omega\), soit \( f\) est continue en \( \ell\).
	\end{enumerate}
	Alors
	\begin{equation}        \label{EQooTHTVooCvrFdN}
		{LE}_{x\to a}(f\circ g)(x)=b.
	\end{equation}
\end{theorem}

\begin{proof}
	Nous notons \( \Omega\) le domaine de \( f\). Ce sera \( \eR\) ou \( \eR\setminus\{ \ell \}\) selon le cas traité dans \ref{ITEMooUNJAooCDOKcO}.

	Soit \( \epsilon>0\). L'hypothèse de limite épointée pour \( f\) nous dit qu'il existe \( \eta>0\) tel que
	\begin{equation}
		y\in\Omega\cap B(\ell,\eta)\setminus\{ \ell \}
	\end{equation}
	implique \( | f(y)-b |<\epsilon\).

	L'hypothèse de limite épointée pour \( g\), appliquée à ce \( \eta\) dit qu'il existe \( \delta>0\) tel que \( 0<| x-a |<\delta\) implique \( | g(x)-\ell |<\eta\).

	\begin{subproof}
		\spitem[Si \( f\) n'existe pas en \( \ell\)]
		Supposons avoir \( 0<| x-a |<\eta\). Alors nous avons \( | g(x)-\ell |<\delta\). Notons qu'il est impossible d'avoir \( g(x)=\ell\) parce que nous avons supposé \( g(\eR)\subset \Omega\) et que \( \ell\) n'est pas dans \( \Omega\).

		Nous avons quand même \( 0<| g(x)-\ell |<\delta\). La condition de limite pour \( f\) appliquée à \( y=g(x)\) donne alors \( | f\big( g(x) \big)-b |<\epsilon\), ce qui signifie la limite épointée \eqref{EQooTHTVooCvrFdN}.
		\spitem[Si \( f\) est continue en \( \ell\)]
		Le lemme \ref{LEMooNEGOooCllIMN} dit qu'alors \( f(\ell)=b\). L'hypothèse de limite épointée sur \( f\) dit que
		\begin{equation}
			0<| y-\ell |<\eta\,\Rightarrow | f(y)-b |<\epsilon.
		\end{equation}
		Mais puisque \( f(\ell)=b\), nous avons en réalité
		\begin{equation}        \label{EQooHAHHooRiAABt}
			| y-\ell |<\eta\,\Rightarrow | f(y)-b |<\epsilon.
		\end{equation}
		Supposons donc que \( 0<|x-a  |<\delta\). Alors \( | g(x)-\ell |<\eta\). En appliquant \eqref{EQooHAHHooRiAABt} à \( y=g(x)\) nous trouvons
		\begin{equation}
			| f(y)-b |<\epsilon.
		\end{equation}
	\end{subproof}
\end{proof}

\begin{normaltext}      \label{NORMooSLAJooLfDreV}
	Les deux théorèmes sont incomplets. En effet, le théorème «pointé» \ref{THOooOYXDooKDPkuW} ne traite pas le cas où seules des limite épointées existent. Il est donc moins général que le théorème «épointé» \ref{THOooNPBQooEMOYpd}. En contrepartie, le théorème «épointé» ne parvient pas à conclure à l'existence d'une limite pointée dans le cas où elle existe. Sa conclusion est donc moins forte.
\end{normaltext}

Nous devons donc nous atteler à écrire un théorème qui traite tous les cas en obtenant la conclusion la plus forte possible dans chaque cas. Nous allons supposer que
\begin{subequations}
	\begin{numcases}{}
		{LE}_{x\to a}g(x)=\ell\\
		{LE}_{y\to \ell}f(y)=b.
	\end{numcases}
\end{subequations}
Ensuite, les différents cas seront divisés selon quatre critères :
\begin{enumerate}
	\item
	      \( g\) existe en \( a\) ou pas.
	\item
	      \( f\) existe en \( \ell\) ou pas.
	\item
	      \( g\) est continue ou pas en \( a\).
	\item
	      \( f\) est continue ou pas en \( \ell\).
\end{enumerate}
Cela fait \( 2^4=16\) combinaisons. Heureusement certaines sont impossibles : si une fonction n'existe pas en un point, elle ne peut pas y être continue.

Nous avons donc \( 9\) cas résumés dans le théorème suivant.
\begin{equation}
	\begin{array}{|c|c|c|c|c|c|c|c|c|c|}
		\hline%
		g(a)     & 1 & 1 & 1 & 1 & 1 & 1 & 0 & 0 & 0 \\
		\hline%
		f(\ell)  & 1 & 1 & 1 & 1 & 0 & 0 & 1 & 1 & 0 \\
		\hline%
		g\in C^0 & 1 & 1 & 0 & 0 & 1 & 0 & 0 & 0 & 0 \\
		\hline%
		f\in C^0 & 1 & 0 & 1 & 0 & 0 & 0 & 1 & 0 & 0 \\
		\hline%
	\end{array}
\end{equation}

\begin{theorem}[\cite{MonCerveau}]      \label{THOooHXGIooBclAHA}
	Soient des fonctions \( f\) et \( g\) telles que
	\begin{subequations}
		\begin{numcases}{}
			{LE}_{x\to a}g(x)=\ell\\
			{LE}_{y\to \ell}f(y)=b.
		\end{numcases}
	\end{subequations}
	\begin{enumerate}
		\item       \label{ITEMooDXBLooVfhSWg}
		      Si \( f\) est continue en \( \ell\) et si \( g\) est continue en \( a\), alors \( f\circ g\) est continue en \( a\).
		\item       \label{ITEMooIXBQooMDknwN}
		      Nous supposons:
		      \begin{multicols}{2}
			      \begin{enumerate}
				      \item \( g\) définie en \( a\)
				      \item \( f\) définie en \( \ell\)
				      \item \( g\) continue en \( a\)
				      \item \( f\) non continue en \( \ell\).
			      \end{enumerate}
		      \end{multicols}
		      Alors nous ne disons rien.
		\item       \label{ITEMooHTIEooMKDrqx}      % 3
		      Nous supposons:
		      \begin{multicols}{2}
			      \begin{enumerate}
				      \item \( g\) définie en \( a\)
				      \item \( f\) définie en \( \ell\)
				      \item \( g\) non continue en \( a\)
				      \item \( f\) continue en \( \ell\).
			      \end{enumerate}
		      \end{multicols}
		      Alors \( {LE}_{x\to a}(f\circ g)(x)=b\).
		\item   \label{ITEMooVQMDooEtHfwC}      % 4
		      Nous supposons:
		      \begin{multicols}{2}
			      \begin{enumerate}
				      \item \( g\) définie en \( a\)
				      \item \( f\) définie en \( \ell\)
				      \item \( g\) non continue en \( a\)
				      \item \( f\) non continue en \( \ell\).
			      \end{enumerate}
		      \end{multicols}
		      Alors nous ne disons rien.
		\item   \label{ITEMooANFQooWVrfTd}      % 5
		      Nous supposons:
		      \begin{multicols}{2}
			      \begin{enumerate}
				      \item \( g\) définie en \( a\)
				      \item \( f\) non définie en \( \ell\)
				      \item \( g\) continue en \( a\)
				      \item \( f\) non continue en \( \ell\).
			      \end{enumerate}
		      \end{multicols}
		      Alors \( {LE}_{x\to a}(f\circ g)(x)=b\).
		\item   \label{ITEMooDJBHooSlqpOO}      % 6
		      Nous supposons:
		      \begin{multicols}{2}
			      \begin{enumerate}
				      \item \( g\) définie en \( a\)
				      \item \( f\) non définie en \( \ell\)
				      \item \( g\) non continue en \( a\)
				      \item \( f\) non continue en \( \ell\).
			      \end{enumerate}
		      \end{multicols}
		      Alors \( {LE}_{x\to a}(f\circ g)(x)=b\).
		\item   \label{ITEMooUFJHooRzLglZ}      % 7
		      Nous supposons:
		      \begin{multicols}{2}
			      \begin{enumerate}
				      \item \( g\) non définie en \( a\)
				      \item \( f\) définie en \( \ell\)
				      \item \( g\) non continue en \( a\)
				      \item \( f\) continue en \( \ell\).
			      \end{enumerate}
		      \end{multicols}
		      Alors \( {LE}_{x\to a}(f\circ g)(x)=b\).
		\item   \label{ITEMooOAAVooSjoYOv}      % 8
		      Nous supposons:
		      \begin{multicols}{2}
			      \begin{enumerate}
				      \item \( g\) non définie en \( a\)
				      \item \( f\) définie en \( \ell\)
				      \item \( g\) non continue en \( a\)
				      \item \( f\) non continue en \( \ell\).
			      \end{enumerate}
		      \end{multicols}
		      Alors nous ne disons rien.
		\item   \label{ITEMooPVZKooBXJARI}      % 9
		      Nous supposons:
		      \begin{multicols}{2}
			      \begin{enumerate}
				      \item \( g\) non définie en \( a\)
				      \item \( f\) non définie en \( \ell\)
				      \item \( g\) non continue en \( a\)
				      \item \( f\) non continue en \( \ell\).
			      \end{enumerate}
		      \end{multicols}
		      Alors \( {LE}_{x\to a}(f\circ g)(x)=b\).
	\end{enumerate}
\end{theorem}

\begin{proof}
	Cas par cas.
	\begin{subproof}
		\spitem[Cas \ref{ITEMooDXBLooVfhSWg}]
		C'est le théorème \ref{PROPooVNKVooJvxarf} de composition de la continuité.
		\spitem[Cas \ref{ITEMooIXBQooMDknwN}]
		l'exemple du lemme \ref{LEMooOSNGooJpiXbK}.
		\spitem[Cas \ref{ITEMooHTIEooMKDrqx}]
		Théorème \ref{THOooNPBQooEMOYpd}.
		\spitem[Cas \ref{ITEMooVQMDooEtHfwC}]
		Le contre-exemple dans ce cas est \( g=\mtu_0\) et \( f=\mtu_0\).
		\spitem[Cas \ref{ITEMooANFQooWVrfTd}]
		Théorème \ref{THOooNPBQooEMOYpd}.
		\spitem[Cas \ref{ITEMooDJBHooSlqpOO}]
		Théorème \ref{THOooNPBQooEMOYpd}.
		\spitem[Cas \ref{ITEMooUFJHooRzLglZ}]
		Théorème \ref{THOooNPBQooEMOYpd}.
		\spitem[Cas \ref{ITEMooOAAVooSjoYOv}]
		Contre-exemple, un peu artificiel, avec \( g(x)=\frac{ \mtu_0(x) }{ x }\). C'est une fonction qui vaut \( 0\) partout sauf en \( 0\) où elle n'existe pas. Ensuite pour \( f\), nous prenons l'indicatrice de \( \{ 0 \}\) :  \( f=\mtu_0\). Pour tout \( x\neq 0\) nous avons
		\begin{equation}
			(f\circ g)(x)=\mtu_0\left( \frac{ \mtu_0(x) }{ x } \right)=\mtu_0(0)=1.
		\end{equation}
		Donc \( {LE}_{x\to 0}(f\circ g)(x)=1\).

		Mais nous avons pourtant
		\begin{subequations}
			\begin{numcases}{}
				{LE}_{x\to 0}g(x)=0\\
				{LE}_{y\to 0}f(x)=0.
			\end{numcases}
		\end{subequations}
		\spitem[Cas \ref{ITEMooPVZKooBXJARI}]
		Théorème \ref{THOooNPBQooEMOYpd}.
	\end{subproof}
\end{proof}

%---------------------------------------------------------------------------------------------------------------------------
\subsection{Discussion pointée Vs épointée}
%---------------------------------------------------------------------------------------------------------------------------

Résumé:
\begin{enumerate}
	\item
	      Dans l'éducation nationale et dans les programmes en France, c'est la limite pointée qui est donnée.
	\item
	      Dans le Frido ce sera la limite épointée. Autrement dit, nous réserverons la notation \( \lim\) et le mot «limite» pour la limite épointée.
	\item
	      De toutes façons, ça ne change pratiquement rien nulle part. Vous pourriez terminer l'agrégation sans vous en rendre compte. Vous pouvez sauter toute la discussion et reprendre une vie normale.
\end{enumerate}

Le débat pour savoir quelle est la «meilleure» notion a déjà fait couler de nombreux octets\cite{BIBooKNWHooBRoxme,BIBooNUKAooVMqppa,BIBooDILKooUcmUVD,BIBooJDPPooVONaQV,BIBooUIAFooHqKjQh}.


%///////////////////////////////////////////////////////////////////////////////////////////////////////////////////////////
\subsubsection{La limite pointée est plus simple au départ}
%///////////////////////////////////////////////////////////////////////////////////////////////////////////////////////////

Il est vrai que la limite pointée est plus simple de premier abord.

%///////////////////////////////////////////////////////////////////////////////////////////////////////////////////////////
\subsubsection{Le théorème de composition}
%///////////////////////////////////////////////////////////////////////////////////////////////////////////////////////////

Le théorème de composition des limites pointées \ref{THOooOYXDooKDPkuW} est plus propre que le théorème de composition épointé \ref{THOooNPBQooEMOYpd}. Intuitivement, on voudrait que la limite d'une fonction composée soit la composée des limites. Et ça c'est vrai pour la limite pointée, pas pour l'épointée.

%///////////////////////////////////////////////////////////////////////////////////////////////////////////////////////////
\subsubsection{Limite d'une fonction discontinue mais qui existe}
%///////////////////////////////////////////////////////////////////////////////////////////////////////////////////////////

Que pensez-vous que la limite en l'infini de la fonction suivante «devrait» être ?
\begin{equation}
	\begin{aligned}
		f\colon \mathopen[ 0 , \infty \mathclose] & \to \eR                               \\
		x                                         & \mapsto \begin{cases}
			                                                    x^2 & \text{si } x\neq \infty \\
			                                                    0   & \text{si }x=\infty
		                                                    \end{cases}
	\end{aligned}
\end{equation}
Là, tant que \( x\) s'approche de \( \infty\) sans l'atteindre, il n'y a vraiment que de la croissance à perte de vue; jusqu'à l'horizon et au-delà.

On pourrait faire la même remarque avec la fonction indicatrice de \( \{ 0 \}\). Qu'est-ce que la limite en zéro «devrait» être ?

%///////////////////////////////////////////////////////////////////////////////////////////////////////////////////////////
\subsubsection{Point d'étape}
%///////////////////////////////////////////////////////////////////////////////////////////////////////////////////////////

Aucune des deux limites ne donne le résultat «attendu» dans les deux cas. Toutes deux font une chose bien, et une chose pas bien.

%///////////////////////////////////////////////////////////////////////////////////////////////////////////////////////////
\subsubsection{Limite d'une fonction discontinue mais qui existe}
%///////////////////////////////////////////////////////////////////////////////////////////////////////////////////////////

Prenons la fonction indicatrice de \( \{ 0 \}\) :
\begin{equation}
	\begin{aligned}
		f\colon\eR & \to \eR                    \\
		x          & \mapsto \begin{cases}
			                     1 & \text{si } x=0 \\
			                     0 & \text{sinon. }
		                     \end{cases}
	\end{aligned}
\end{equation}

En utilisant la limite pointée, on peut exprimer deux choses:
\begin{itemize}
	\item la limite pointée en zéro n'existe pas
	\item la fonction n'est pas continue en zéro.
\end{itemize}

En utilisant la limite épointée, on peut exprimer deux choses:
\begin{itemize}
	\item la limite épointée en zéro existe et vaut zéro.
	\item la fonction n'est pas continue en zéro.
\end{itemize}

La première paire d'informations est compatible avec la fonction \( 1/x\). Autrement dit, la valeur «inattendue» que prend \( f\) en zéro casse tout ce qu'on aurait pu dire sur un voisinage de zéro.

La seconde paire d'informations donne au moins une idée de ce qu'il se passe autour de zéro. On peut en déduire, par exemple, que \( f\) est intégrable sur un voisinage de zéro parce qu'elle y est bornée et que la valeur en un point ne fait rien à l'intégrabilité\quext{Peut-être qu'il faut ajouter que \( f\) est mesurable ? Mais peut-être que l'existence de la limite implique la mesurabilité ? Dites-moi ce que vous en pensez.}.

%///////////////////////////////////////////////////////////////////////////////////////////////////////////////////////////
\subsubsection{L'enseignement du cas précédent}
%///////////////////////////////////////////////////////////////////////////////////////////////////////////////////////////

La limite épointée donne une information sur ce qu'il se passe «autour» du point sans rien dire de ce qu'il se passe «sur» le point. Si nécessaire, la continuité complète l'information en précisant ce qu'il se passe «sur» le point.

Certaines questions n'ont pas besoin de savoir ce qu'il se passe en un seul point.

La limite épointée «refuse» de dire ce qu'il se passe autour du point parce qu'il y a un problème juste sur ledit point. Un seul point se comporte mal et tout le voisinage passe sous le radar.

%///////////////////////////////////////////////////////////////////////////////////////////////////////////////////////////
\subsubsection{La limite épointée est plus riche}
%///////////////////////////////////////////////////////////////////////////////////////////////////////////////////////////

La classe des fonction admettant une limite pointée est plus grande que celle admettant une limite épointée (lemme \ref{LEMooWAZLooDPvemu}). L'utilisation de la limite épointée permet de décrire quelques cas supplémentaires par rapport à ce que l'on peut faire seulement avec la limite pointée.

Pour être plus précis, comme je le disait précédemment, en \ref{NORMooSLAJooLfDreV}, aucune des deux notions n'est satisfaisante seule :
\begin{itemize}
	\item mettez de la limite pointée dans les hypothèses, vous aurez un théorème moins général;
	\item mettez de la limite pointée dans la thèse, vous aurez un résultat plus fort.
\end{itemize}

Le vrai intérêt de la limite épointée est que \emph{en combinaison avec la notion de continuité} permet d'être plus général et plus précis que ce qu'on peut obtenir avec la limite pointée. Dit autrement, le couple (limite épointée, continuité) est plus fort que le couple (limite pointée, continuité).

D'un certain point de vue, oui, la limite pointée est plus simple, mais elle est plus simple parce qu'elle donne moins d'informations.

%///////////////////////////////////////////////////////////////////////////////////////////////////////////////////////////
\subsubsection{Retour sur le théorème de composition}
%///////////////////////////////////////////////////////////////////////////////////////////////////////////////////////////

Le \emph{vrai} théorème de composition est le théorème \ref{THOooHXGIooBclAHA}. Lui, il passe en revue tous les cas possibles et donne le plus de conclusions possibles dans chaque cas.

Ce théorème s'exprime de façon à peu près convenable à l'aide de limite épointée et de continuité. J'attends de voir le même avec une limite pointée et la continuité.

Je suis très ouvert à la discussion si c'est pour avoir quelque chose de plus simple produisant les mêmes résultats. Je ne suis par contre pas très ouvert pour avoir quelque chose de plus simple, mais donnant moins de résultats. C'est toujours facile d'avoir des résultats plus courts, plus simples et plus intuitifs quand on se contente de moins.

%///////////////////////////////////////////////////////////////////////////////////////////////////////////////////////////
\subsubsection{En français}
%///////////////////////////////////////////////////////////////////////////////////////////////////////////////////////////

La limite épointée rend l'idée de «s'approcher sans atteindre». En français l'expression «être à la limite de tel résultat» signifie le plus souvent être très proche du résultat, mais ne pas y être.

%///////////////////////////////////////////////////////////////////////////////////////////////////////////////////////////
\subsubsection{La question est pédagogique}
%///////////////////////////////////////////////////////////////////////////////////////////////////////////////////////////

Tant qu'on ne m'a pas montré comment on exprime le théorème de composition \ref{THOooHXGIooBclAHA} avec des limites pointées, je resterai sur cette idée : la limite pointée est plus simple, mais elle dit moins.

Cela n'est cependant pas spécialement bloquant. Après tout, ça dépend de ce qu'on veut. D'un point de vue pédagogique, la limite pointée introduit autant de \( \epsilon\) et de \( \delta\) qu'on le veut, et permet d'introduire tous les concepts utiles en analyse.

La question est de savoir à quel point on est prêt à se compliquer la vie pour avoir des théorèmes un micro-cheveu plus complets. Le choix du Frido est de recevoir la difficulté avec résignation et de l'endurer avec courage, pour le plaisir d'avoir des théorèmes qui donnent un peu plus d'information\footnote{C'est une de mes citation préférées. Comme nous sommes entre adultes, je vous donne la référence : \cite{BIBooTOVWooSDsNrc}. Si vous n'avez pas 18 ans, on peut vraiment se demander si le Frido est vraiment une lecture de votre âge.}.

%///////////////////////////////////////////////////////////////////////////////////////////////////////////////////////////
\subsubsection{En fait ça ne change presque rien}
%///////////////////////////////////////////////////////////////////////////////////////////////////////////////////////////

Certains craignent qu'utiliser la limite pointée demande d'ajuster beaucoup de résultats un peu partout\cite{BIBooTOVWooSDsNrc}. Le Frido contient à ma connaissance seulement deux théorèmes dont l'énoncé contient une subtilité due au choix épointé. Le fameux théorème de composition \ref{THOooNPBQooEMOYpd}, et le lemme \ref{LEMooYLIHooFBQyzC}.

Le fait est que l'on ne calcule presque jamais de limites en une valeur où la fonction existe. Si on calcule une limite, c'est précisément parce qu'on regarde un point où la fonction n'existe pas.

Exemples:
\begin{itemize}
	\item Quand on calcule une dérivée, on calcule
	      \begin{equation}
		      \lim_{\epsilon\to 0}\frac{ f(a+\epsilon)-f(a) }{ \epsilon }.
	      \end{equation}
	      Cette fonction de \( \epsilon\) n'existe pas lorsque \( \epsilon=0\). Donc les limites pointées et épointées sont identiques.
	\item
	      De même, l'étude du sinus cardinal \( f(x)=\sin(x)/x\) (lemme \ref{LEMooMJFBooAjtNjV}) est une fonction dont ça ne viendrait à l'idée de personne de calculer la limite pour \( x\to 4\). Et ça tombe bien : la seule limite que ça donne envie de calculer est
	      \begin{equation}
		      \lim_{x\to 0} \frac{ \sin(x) }{ x }.
	      \end{equation}
	      Et encore une fois, la fonction dans le limite n'existe pas au point limite.
	\item
	      Beaucoup de conditions d'intégrabilité demandent des limites à l'infini. Là encore, ce sont des limites vers des points où la fonction n'existe pas. Franchement, qui va vouloir définir
	      \begin{equation}
		      \begin{aligned}
			      f\colon \mathopen[ 0 , \infty \mathclose] & \to \eR                               \\
			      x                                         & \mapsto \begin{cases}
				                                                          x^2 & \text{si } x\neq \infty \\
				                                                          0   & \text{si } x=\infty
			                                                          \end{cases}
		      \end{aligned}
	      \end{equation}
	      sans rigoler ?  OK. Pour cette fonction, il y a une différence entre la limite pointée et épointée. Mais franchement, c'est bien la limite épointée qui donne le résultat «intuitif».
\end{itemize}

%///////////////////////////////////////////////////////////////////////////////////////////////////////////////////////////
\subsubsection{Si vous avez quand même envie de discuter}
%///////////////////////////////////////////////////////////////////////////////////////////////////////////////////////////

Essayez de garder votre salive pour des sujets importants.
\begin{itemize}
	\item
	      L'intégrale de Kurzweil-Henstock\cite{BIBooLGJXooZhEXJf} contre Lebesgue. Là au moins, il y a des choses non triviales à dire, et des vrais résultats mathématiques à la clef.
	\item
	      L'utilisation de \( \tau\) au lieu de \( \pi\).
\end{itemize}


%///////////////////////////////////////////////////////////////////////////////////////////////////////////////////////////
\subsubsection{En très résumé}
%///////////////////////////////////////////////////////////////////////////////////////////////////////////////////////////

Si vous ne voulez pas lire toute ma prose, voici ce dont vous devez être conscient :
\begin{enumerate}
	\item
	      La limite épointée est celle utilisée partout sauf en France.
	\item
	      La limite épointée est un peu plus compliquée que la limite pointée, mais elle permet de prouver plus de choses. En témoigne le théorème «complet» de composition \ref{THOooHXGIooBclAHA} que je doute être facile à exprimer à l'aide des limites pointées et de la continuité\footnote{Il y a bien entendu moyen. Voir par exemple \cite{BIBooDAGXooRltbgK}. Sans ironie, je trouve ce théorème fascinant.}.
	\item
	      La limite pointée «cache» l'information sur tout le voisinage de \( a\) si la fonction se comporte mal juste en \( a\).
\end{enumerate}
Une fois que vous êtes conscients de ces quelque points, vous faites comme vous voulez; ça n'a pratiquement aucune importance. La seule position indéfendable est celle de prendre la limite pointée et de ne pas prévenir \randomGender{le lecteur}{la lectrice} que les sources autres que françaises donnent une définition différente.

%+++++++++++++++++++++++++++++++++++++++++++++++++++++++++++++++++++++++++++++++++++++++++++++++++++++++++++++++++++++++++++
\section{Limites en l'infini}
%+++++++++++++++++++++++++++++++++++++++++++++++++++++++++++++++++++++++++++++++++++++++++++++++++++++++++++++++++++++++++++

Non, sur \( \eR\) nous n'allons pas ajouter \( \infty\) avec la topologie d'Alexandrov de la définition \ref{PROPooHNOZooPSzKIN}. Nous n'allons pas considérer \( \hat \eR=\eR\cup\{ \infty \}\).

%TODOooQBSSooMBXSYs: il faut réexprimer pour que ce soit un théorème que (R U {+-infini},<) soit totalement ordonné.
\begin{definition}[Droite réelle achevée\cite{ooDZRQooPpOXhY}]        \label{DEFooRUyiBSUooALDDOa}
	Nous considérons l'ensemble
	\begin{equation}
		\bar \eR=\eR\cup\{ +\infty,-\infty \}
	\end{equation}
	où \( +\infty\) et \( -\infty\) ne sont pas des éléments de \( \eR\).

	Nous mettons sur \( \bar\eR\) la relation d'ordre en prenant celle de \( \eR\) à laquelle nous ajoutons les règles
	\begin{enumerate}
		\item
		      \( -\infty<x\) pour tout \( x\in\eR\cup\{ +\infty \}\)
		\item
		      \( +\infty>x\) pour tout \( x\in \eR\cup\{ -\infty \}\).
	\end{enumerate}

	Nous mettons une topologie sur \( \bar\eR\) en donnant la base\footnote{Base de topologie, définition \ref{DEFooLEHPooIlNmpi}.} suivante :
	\begin{itemize}
		\item \( \mathopen] a , b \mathclose[\),
		\item \( \mathopen] a , +\infty \mathclose]\),
		\item \( \mathopen[ -\infty , b \mathclose[\)
	\end{itemize}
	pour tous réels \( a\) et \( b\).
\end{definition}

\begin{normaltext}
	La notation «\( \infty\)» peut désigner l'unique élément ajouté dans du compactifié d'Alexandrov\footnote{Le compactifié d'Alexandrov \( \hat \eR\), définition \ref{PROPooHNOZooPSzKIN}.} aussi bien que l'élément infini positifs dans la droite réelle achevée. Si vous faites attention au contexte\footnote{Vous devez toujours avoir parfaitement clairement en tête la topologie que vous manipulez.}, ça ne devrait pas poser de problèmes.

	\randomGender{Certains auteurs}{Certaines autrices} réservent «\( \infty\)» pour Alexandrov et écrivent toujours «\( +\infty\)» et «\( -\infty\)» pour la droite réelle achevée.
\end{normaltext}

\begin{lemma}[\cite{MonCerveau}]
	La topologie sur \( \eR\) induite de celle sur \( \bar \eR\) est la topologie usuelle.
\end{lemma}

\begin{proof}
	Nous notons \( \tau_{\eR}\) la topologie de \( \eR\), \( \tau_{\bar \eR}\) celle de \( \bar \eR\) et \( \tau_i\) celle induite de \( \bar \eR\) sur \( \eR\). Nous devons prouver que \( \tau_i=\tau_{\eR}\).

	\begin{subproof}
		\spitem[\( \tau_i\subset\tau_{\eR}\)]
		Un élément de \( \tau_i\) est de la forme \( \mO=\eR\cap A\) où \( A\) est un élément de \( \tau_{\bar \eR}\). Vu que \( A\) est un ouvert de \( \bar \eR\), il est une réunion d'éléments de la base de topologie\footnote{C'est la proposition \ref{DEFooLEHPooIlNmpi} qui dit ça.}; donc \( A=\bigcup_{i\in I}A_i\) où les \( A_i\) sont des trois types listés dans la définition \ref{DEFooRUyiBSUooALDDOa}.
		\begin{enumerate}
			\item
			      Si \( A_i=\mathopen] a , b \mathclose[\) alors \( \eR\cap A=\mathopen] a , b \mathclose[\) est un ouvert de \( \eR\).
			\item Si \( A_i=\mathopen] a , +\infty \mathclose]\), alors \( \eR\cap A_i=\mathopen] a , +\infty \mathclose[\) est un ouvert de \( \eR\).
			\item Si \( A_i=\mathopen[ -\infty , b \mathclose[\), même chose.
		\end{enumerate}
		Donc \( \eR\cap A=\bigcup_{i\in I}(\eR\cap A_i)\) est une union d'ouverts de \( \eR\).
		\spitem[\( \tau_{\eR}\subset\tau_i\)]
		Comme les \( \mathopen] a , b \mathclose[\) forment une base de topologie de \( \eR\), l'ensemble \( \tau_i\) contient une base de topologie de \( \eR\) et donc contient tout \( \tau_{\eR}\).
	\end{subproof}
\end{proof}

\begin{proposition}[\cite{MonCerveau}]
	Soit une suite \( (x_k)\) dans \( \bar \eR=\eR\cup\{ \pm\infty \}\). Nous avons \( x_k\stackrel{\bar \eR}{\longrightarrow}+\infty\) si et seulement si pour tout \( M>0\) il existe un \( N>0\) tel que \( n\geq N\) implique \( x_n>M\).
\end{proposition}

\begin{proof}
	En deux parties.
	\begin{subproof}
		\spitem[\( \Rightarrow\)]
		Pour tout voisinage \( A\) de \( +\infty\), il existe un \( N\) tel que \( n\geq N\) implique \( x_n\in A\). Soit donc le voisinage \( \mathopen] M , +\infty \mathclose]\), et le \( N\) correspondant. Nous avons alors, pour tout \( n\geq N\), \( x_n\in \mathopen] M , +\infty \mathclose]\) et donc \( x_n\geq M\).
		\spitem[\( \Leftarrow\)]
		Soit un ouvert \( A\) contenant \( +\infty\). Nous avons \( A=\bigcup_{i\in I} A_i\) où les \( A_i\) sont des trois types listés dans la définition \ref{DEFooRUyiBSUooALDDOa}. Comme \( +\infty\in A\), pour au moins un des \( i\), nous avons \( A_i=\mathopen] a , +\infty \mathclose[\).

		Prenons \( N\) tel que \( n\geq N\) implique \( x_n>a\). Alors pour \( n\geq N\) nous avons \( x_n\in A\).
	\end{subproof}
\end{proof}


\begin{lemma}[\cite{MonCerveau}]        \label{LEMooUBFAooEmquQQ}
	Soient \( x>1\) dans \( \eR\) et \( n\geq 1\) dans \( \eN\). Alors \( x^n \geq x\).
\end{lemma}

\begin{proof}
	Pour \( n=1\), nous avons \( x^n=x\) donc d'accord. Supposons que \( x^n\geq x\) pour un certain \( n\in \eN\), et prouvons que \( x^{n+1}\geq x\).

	Calcul avec justifications en-dessous :
	\begin{subequations}
		\begin{align}
			x^{n+1} & =xx^n                                      \\
			        & \geq xx        \label{SUBEQooBBJYooIYaErs} \\
			        & \geq x.        \label{SUBEQooCEJWooMUMchE}
		\end{align}
	\end{subequations}
	Justifications:
	\begin{itemize}
		\item Pour \eqref{SUBEQooBBJYooIYaErs} hypothèse de récurrence.
		\item Pour \eqref{SUBEQooCEJWooMUMchE}, lemme \ref{LEMooKAXFooIPyzJC}.
	\end{itemize}
\end{proof}

\begin{lemma}       \label{LEMooFCIXooJuHFqk}
	Nous considérons l'espace topologique de la droite réelle achevée\footnote{Définition \ref{DEFooRUyiBSUooALDDOa}.} \( \bar \eR\). Si \( n\geq 1\) nous avons
	\begin{equation}        \label{EQooRRFEooLYcuRP}
		\lim_{x\to +\infty} x^n = +\infty
	\end{equation}
	et
	\begin{equation}
		\lim_{x\to +\infty} \frac{1}{ x^n }=0.
	\end{equation}
\end{lemma}

\begin{proof}
	Si \( V\) est un voisinage de \( +\infty\), alors nous devons montrer qu'il existe un voisinage \( W\) de \( +\infty\) tel que \( x^n\in V\) pour tout \( x\in W\).

	Un ouvert est une union d'éléments de la base de topologie\footnote{C'est la définition \ref{DEFooLEHPooIlNmpi}.}. Nous voyons que \( V\) contient au moins une partie de la forme \( \mathopen] R , +\infty \mathclose]\). Nous supposons que \( R>1\).

	Si \( x>R>1\), alors nous avons \( x^n\geq x\) (lemme \ref{LEMooUBFAooEmquQQ}) et donc
	\begin{equation}
		x^n\geq x>R,
	\end{equation}
	ce qui signifie \( x\in V\).

	En prenant \( W=\mathopen] R , +\infty \mathclose]\), nous avons bien \( W^n\subset V\). Cela prouve \eqref{EQooRRFEooLYcuRP}.

	En ce qui concerne la seconde limite, la démonstration est du même type. Remarquez seulement que vous n'avez pas formellement le droit d'utiliser la proposition \ref{PROPooOUPNooTrClHw} en invoquant \( \frac{1}{ +\infty }=0\).
\end{proof}

%---------------------------------------------------------------------------------------------------------------------------
\subsection{Limite en des nombres}
%---------------------------------------------------------------------------------------------------------------------------

Nous posons la définition suivante.
\begin{definition}      \label{DefInfNombre}
	Lorsque \( a\in\eR\), on dit que la fonction \( f\) \defe{tend vers l'infini quand \( x\) tend vers \( a\)}{} si
	\[
		\forall M\in\eR,\exists \delta\tq (| x-a |\leq \delta )\Rightarrow f(x)\geq M\text{ quand }x\in\dom f.
	\]
\end{definition}
Cela signifie que l'on demande que dès que \( x\) est assez proche de \( a\) (c'est-à-dire dès que \( | x-a |\leq\delta\)), alors \( f(x)\) est plus grand que \( M\), et que l'on peut trouver un \( \delta\) qui fait ça pour n'importe quel \( M\). Une autre façon de le dire est que pour toute hauteur \( M\), on peut trouver un intervalle de largeur \( \delta\) autour de \( a\)\footnote{C'est-à-dire un intervalle de la forme \( [a-\delta,a+\delta]\).} tel que sur cet intervalle, la fonction \( f\) est toujours plus grande que \( M\).

Montrons sur un dessin pourquoi je disais que la fonction \( x\to 1/x\) n'est pas de ce type.


Le problème est qu'il n'existe par exemple aucun intervalle autour de \( 0\) sur lequel \( f\) serait toujours plus grande que \( 10\). En effet n'importe quel intervalle autour de \( 0\) contient au moins un nombre négatif. Or quand \( x\) est négatif, \( f\) n'est certainement pas plus grande que \( 10\). Nous y reviendrons.

Pour l'instant, montrons que la fonction \( f(x)=1/x^2\) est une fonction qui vérifie la définition~\ref{DefInfNombre}.  Avant de prendre n'importe quel \( M\), prenons par exemple \( 100\). Nous avons besoin d'un intervalle autour de zéro sur lequel \( f\) est toujours plus grande que \( 100\). C'est vite vu que \( f(0.1)=f(-0.1)=100\), donc l'intervalle \( [-\frac{ 1 }{ 10 },\frac{1}{ 10 }]\) convient. Partout dans cet intervalle, \( f\) est plus grande que \( 100\). Partout ? Ben non : en \( x=0\), la fonction n'est même pas définie, donc c'est un peu dur de dire qu'elle est plus grande que \( 100\). C'est pour cela que nous avons ajouté la condition « quand \( x\in\dom f\) » dans la définition de la limite.

Prenons maintenant un \( M\in\eR\) arbitraire, et trouvons un intervalle autour de \( 0\) sur lequel \( f\) est toujours plus grande que \( M\). La réponse est évidemment l'intervalle de largeur \( 1/\sqrt{M}\), c'est-à-dire
\[
	\left[ -\frac{ 1 }{ \sqrt{M} },\frac{ 1 }{ \sqrt{M} } \right].
\]

\subsection{Limites quand tout va bien}
%--------------------------------------

D'abord définissons ce qu'on entend par la limite d'une fonction en un point quand il n'y a aucun infini en jeu.
\begin{definition}      \label{DefLimPointSansInfini}
	On dit que la fonction \( f\) \defe{tend vers \( b\) quand \( x\) tend vers \( a\)}{} si
	\[
		\forall \epsilon>0,\exists\delta\tq (| x-a |\leq\delta)\Rightarrow | f(x)-b |\leq \epsilon\text{ quand }x\in\dom f.
	\]
	Dans ce cas, nous notons
	\begin{equation}
		\lim_{x\to a}f(x)=b.
	\end{equation}
\end{definition}

Commençons par un exemple très simple : prouvons que \( \lim_{x\to 0}x=0\). C'est donc \( a=b=0\) dans la définition. Prenons \( \epsilon>0\), et trouvons un intervalle autour de zéro tel que partout dans l'intervalle, \( x\leq \epsilon\). Bon ben c'est clair que \( \delta=\epsilon\) fonctionne.

Plus compliqué maintenant, mais toujours sans surprises.

\begin{proposition}
	\[
		\lim_{x\to 0}x^2=0.
	\]

\end{proposition}

\begin{proof}
	Soit \( \epsilon>0\). On veut un intervalle de largeur \( \delta\) autour de zéro tel que \( x^2\) soit plus petit que \( \epsilon\) sur cet intervalle. Cette fois-ci, le \( \delta\) qui fonctionne est \( \delta=\sqrt{\epsilon}\). En effet un élément de l'intervalle \( [-\delta,\delta]\) est un \( r\) de valeur absolue plus petite ou égale à \( \delta\) :
	\[
		| r |\leq\delta=\sqrt{\epsilon}.
	\]
	En prenant le carré de cette inégalité on a :
	\[
		r^2\leq\epsilon,
	\]
	ce qu'il fallait prouver.
\end{proof}

Calculer et prouver des valeurs de limites, mêmes très simples, devient vite de l'arrachage de cheveux à essayer de trouver le bon \( \delta\) en fonction de \( \epsilon\) si on n'a pas quelques théorèmes généraux. Heureusement nous en avons déjà quelques uns : \ref{PROPooVLBWooVttvFK}, \ref{PROPooDQFIooMMwxxJ}, \ref{ThoLimLinMul}, \ref{ThoLimLin}, \ref{PROPooOUPNooTrClHw}.

\begin{proposition}[\cite{MonCerveau}]      \label{PROPooWXBAooAEweSF}
	Soit \( f\colon \eR^2\to \eR\) une application continue dont la variable \( y\) varie dans un compact \( I\) de \( \eR\). Alors la fonction
	\begin{equation}
		\begin{aligned}
			d\colon \eR & \to \eR                      \\
			x           & \mapsto \sup_{y\in I} f(x,y)
		\end{aligned}
	\end{equation}
	est continue.
\end{proposition}

\begin{proof}
	Soit \( x_0\) fixé. Prouvons que \( d\) est continue en \( x_0\). Nous notons \( y_0\) la valeur de \( y\) qui réalise le maximum (par le théorème~\ref{ThoMKKooAbHaro} et le fait que les fonctions projection soient continues, lemme~\ref{LEMooHAODooYSPmvH}). Soit aussi \( \epsilon>0\) tellement fixé que même avec un tournevis hydraulique, il ne bougerait pas. Nous considérons \( \delta\) tel que si \( \| (x,y)-(x_0,y_0) \|\leq \delta\) alors \( \| f(x,y)-f(x_0,y_0) \|<\epsilon\).

	Si \( | x-x_0 |<\delta\) alors pour \( y\) assez proche de \( y_0\) nous avons \( \| (x,y)-(x_0,y_0) \|\leq \delta\), et donc \( \| f(x,y)-f(x_0,y_0) \|\leq \epsilon \). Cela montre qu'il existe \( \delta\) tel que \( | x-x_0 |\leq \delta\) implique \( d(x)\geq d(x_0)-\epsilon\).

	Nous devons encore trouver un \( \delta\) tel que si \( | x-x_0 |\leq \delta\) alors \( d(x)\leq d(x_0)+\epsilon\). Supposons que non. Alors pour tout \( \delta\) il existe un \( x\) tel que \( | x-x_0 |\leq \delta\) et \( d(x)> d(x_0)+\epsilon\). Cela nous donne une suite \( x_i\to x_0\).

	Pour chaque \( x_i\) nous notons \( y_i\) la valeur de \( y\) qui réalise le supremum correspondant. La suite \( (y_i)\) étant contenue dans un compact nous supposons prendre une sous-suite de \( (x_i)\) telle que la suite \( (y_i)\) converge. Nous nommons \( a\) la limite (et non \( y_0\) parce que nous ne savons pas si \( y_i\to y_0\)). Pour chaque \( i\) nous avons
	\begin{equation}
		f(x_i,y_i)>\sup_{y\in I}f(x_0,y)+\epsilon.
	\end{equation}
	En prenant la limite et en utilisant la continuité de \( f\),
	\begin{equation}
		f(x_0,a)>\sup_{y\in I} f(x_0,y)+\epsilon,
	\end{equation}
	ce qui est impossible.
\end{proof}

%---------------------------------------------------------------------------------------------------------------------------
\subsection{Limites de fonctions}
%---------------------------------------------------------------------------------------------------------------------------

Tentons de comprendre ce que signifie qu'un nombre \( \ell\) \emph{ne soit pas} la limite de \( f\) lorsque \( x\to a\). Il s'agit d'inverser la condition de la proposition \ref{PropHOCWooSzrMjl}\ref{ITEMooSHKNooStKGKH}. Le nombre \( \ell\) n'est pas une limite de \( f\) pour \( x\to a\) lorsque
\begin{equation}		\label{EqCaractNonLim}
	\exists\varepsilon>0\tq\,\forall\delta>0,\,\exists x\tq 0<\| x-a \|<\delta\text{ et }\| f(x)-\ell \|>\varepsilon,
\end{equation}
c'est-à-dire qu'il existe un certain seuil \( \varepsilon\) tel qu'on a beau s'approcher aussi proche qu'on veut de \( a\) (distance \( \delta\)), on trouvera toujours un \( x\) tel que \( f(x)\) n'est pas \( \varepsilon\)-proche de \( \ell\).

\begin{lemma}[Unicité de la limite]
	Si \( \ell\) et \( \ell'\) sont deux limites de \( f(x)\) lorsque \( x\) tend vers \( a\), alors \( \ell=\ell'\).
\end{lemma}

\begin{proof}
	Soit \( \varepsilon>0\). Nous considérons \( \delta\) tel que \( \| f(x)-\ell \|<\varepsilon\) pour tout \( x\) tel que \( \| x-a \|<\delta\). De la même manière, nous prenons \( \delta'\) tel que \( \| x-a \|<\delta'\) implique \( \| f(x)-\ell' \|<\varepsilon\). Pour les \( x\) tels que \( \| x-a \|\) est plus petit que \( \delta\) et \( \delta'\) en même temps, nous avons
	\begin{equation}
		\| \ell-\ell' \|=\| \ell-f(x)+f(x)-\ell' \|\leq\| \ell-f(x) \|+\| f(x)-\ell' \|<2\varepsilon,
	\end{equation}
	et donc \( \| \ell-\ell' \|=0\) parce que c'est plus petit que \( 2\varepsilon\) pour tout \( \varepsilon\).
\end{proof}

\begin{proposition}[\cite{MonCerveau}]  \label{PROPooKPOXooEHIXJs}
	Soient un espace vectoriel normé \( (V,\| . \|)\) et \( a\in V\). Soient encore un voisinage \( A\) de \( a\) et deux fonctions \( f,g\colon A\setminus \{ a \}\to \eR\) qui admettent une limite en \( a\).

	Si \( f(x)\leq g(x)\) pour tout \( x\in A\setminus \{ a \}\) alors
	\begin{equation}
		\lim_{x\to a} f(x)\leq \lim_{x\to a} g(x).
	\end{equation}
\end{proposition}

\begin{proposition}     \label{PROPooGQHKooWgykjW}
	Si \( f\colon \eR\to \eR\cup\{ \pm\infty \}\) est continue et croissante, alors il existe \( \ell\in \eR\cup\{ \pm\infty \}\) tel que
	\begin{equation}
		\lim_{x\to \infty} f(x)=\ell.
	\end{equation}
\end{proposition}

%---------------------------------------------------------------------------------------------------------------------------
\subsection{Limite à gauche et à droite}
%---------------------------------------------------------------------------------------------------------------------------

Si \( a\) est à l'intérieur du domaine de \( f\), nous savons ce que signifie \( \lim_{x\to a} f(x)\). Nous donnons également une définition des limites à gauche et à droite.

\begin{definition}
	Soient \( D\subset \eR\) et une fonction \( f\colon D\to \eR\). Si \( a\in \Adh(D)\) nous définissons la \defe{limite à droite}{limite à droite} de \( f\) en \( a\) par
	\begin{equation}        \label{EQooQKHLooMoSXVe}
		\lim_{x\to a^+} f(x)=\lim_{x\to a} \tilde f(x)
	\end{equation}
	où \( \tilde f\) est la fonction \( f\) restreinte à \( D\cap\{ x\tq x>a \}\). La limite \eqref{EQooQKHLooMoSXVe} est souvent écrite sous la forme condensée
	\begin{equation}
		\lim_{\substack{x\to a\\x>a}}f(x).
	\end{equation}
	Pour la limite à gauche c'est un peu la même chose :
	\begin{equation}
		\lim_{x\to a^-} f(x)=\lim_{\substack{x\to a\\x<a}}f(x).
	\end{equation}
\end{definition}

\begin{lemma}       \label{LEMooXJMFooCkzoVi}
	Soient \( D\subset \eR\) et une fonction \( f\colon D\to \eR\). Si \( a\in \Adh(D)\) nous avons \( \lim_{x\to a^+} f(x)=\ell\) si et seulement si pour tout \( \epsilon>0\), il existe \( \delta>0\) tel que  \( x\in\mathopen] a , a+\delta \mathclose[\cap D\) implique \( f(x)\in B(\ell,\epsilon)\).
\end{lemma}

\begin{proof}
	Nous avons les équivalences entre les propriétés suivantes, en utilisant la définition \ref{DefYNVoWBx} de la limite :
	\begin{enumerate}
		\item
		      \( \lim_{x\to a^+} f(x)=\ell\)
		\item
		      \( \lim_{x\to a} \tilde f(x)=\ell\)
		\item
		      Pour tout \( \epsilon>0\), il existe \( \delta>0\) tel que si \( x\in B(a,\delta)\cap D\cap\{ x>a \}\) alors \( f(x)\in B(\ell,\epsilon)\)
		\item
		      Pour tout \( \epsilon>0\), il existe \( \delta>0\) tel que si \( x\in \mathopen] a , a+\delta \mathclose[\cap D\) alors \( f(x)\in B(\ell,\epsilon)\)
	\end{enumerate}
\end{proof}

\begin{proposition}[\cite{ooOMWZooZvUFiG}]      \label{PROPooGDDJooDCmydE}
	Soit une fonction \( f\colon D\to \eR\) où \( D\) est une partie de \( \eR\). Si \( a\in \Adh(D)\) alors la limite \( \lim_{x\to a} f(x)\) existe si et seulement si les limites à gauche et à droite existent et sont égales. Dans ce cas nous avons égalité :
	\begin{equation}
		\lim_{x\to a} f(x)=\lim_{x\to a^+} f(x)=\lim_{x\to a^-} f(x).
	\end{equation}
\end{proposition}

\begin{proof}
	En deux parties.
	\begin{subproof}
		\spitem[\( \Rightarrow\)]
		Nous disons que \( \lim_{x\to a} f(x)=\ell\). Si \( V\) est un voisinage de \( \ell\), il existe un voisinage \( U\) de \( a\) tel que \( f\big( U\cap D\setminus \{ a \} \big)\subset V\). En particulier il existe un \( \delta>0\) tel que si \( x\in \mathopen] a , a+\delta \mathclose[\cap D\), alors \( | f(x)-\ell |<\epsilon\). Cela est la limite à droite (lemme \ref{LEMooXJMFooCkzoVi}).
		\spitem[\( \Leftarrow\)]
		Soit \( \epsilon>0\). Par la limite à droite, il existe \( \delta_1>0\) tel que \( f\big( \mathopen] a , a+\delta_1 \mathclose[\cap D \big)\subset B(\ell,\epsilon)\). Par la limite à gauche, il existe \( \delta_2\) tel que \( f\big( \mathopen] a-\delta_2 , a \mathclose[\cap D \big)\subset B(\ell,\epsilon)\)

		En prenant \( \delta=\min\{ \delta_1,\delta_2 \}\) nous avons bien \( f\big( B(a,\delta)\cap D\setminus\{ a \} \big)\subset B(\ell,\epsilon)\) comme le demande la définition de la limite.
	\end{subproof}
\end{proof}

\begin{normaltext}
	Quelques remarques à propos de la proposition \ref{PROPooGDDJooDCmydE}.
	\begin{enumerate}
		\item
		      Cette proposition ne se généralise pas aux dimensions supérieures. Dans \( \eR^2\) par exemple, il ne faudrait pas croire que si les limites suivant toutes les directions existent alors la limite existe.
		\item
		      Cette proposition est souvent utilisée pour calculer des limites dans lesquelles interviennent des valeurs absolues. Par exemple, durant la démonstration de la proposition \ref{PROPooCNDHooKRwils}.
	\end{enumerate}
\end{normaltext}

%+++++++++++++++++++++++++++++++++++++++++++++++++++++++++++++++++++++++++++++++++++++++++++++++++++++++++++++++++++++++++++
\section{Limite en compactifié d'Alexandrov}
%+++++++++++++++++++++++++++++++++++++++++++++++++++++++++++++++++++++++++++++++++++++++++++++++++++++++++++++++++++++++++++

Nous considérons l'espace topologique localement compact \( \eR\), et son compactifié d'Alexandrov défini en \ref{PROPooHNOZooPSzKIN}. Nous avons donc un point supplémentaire noté \( \infty\). Ce point n'est ni du côté des grands nombres positifs, ni du côté des grands nombres négatifs. Il n'est ni \( +\infty\) ni \( -\infty\).

\begin{proposition}
	Dans cet espace topologique \( \hat \eR=\eR\cup\{ \infty \}\),
	\begin{equation}
		\lim_{x\to 0} \frac{1}{ x }=\infty.
	\end{equation}
\end{proposition}

\begin{proof}
	Soit un voisinage \( V\) de \( \infty\) dans \( \hat \eR\). Il s'écrit \( V=K^c\cup\{ \infty \}\) pour un certain compact de \( \eR\). Le théorème \ref{ThoXTEooxFmdI} nous assure que \( K\) est borné. Donc il existe \( R>0\) tel que \( K\subset B(0,R)\). Pour \( x\in B(0,1/R)\) nous avons
	\begin{equation}
		\left| \frac{1}{ x } \right|>R,
	\end{equation}
	et donc \( 1/x\in K^c\). Donc aussi \( \frac{1}{ x }\in V\).
\end{proof}

De la même façon, dans \( \eC\cup\{ \infty \}\) nous avons
\begin{equation}
	\lim_{z\to 0} \frac{1}{ z }=\infty.
\end{equation}

\begin{normaltext}
	Je vous laisse deviner la topologie à considérer sur \( \bar \eR=\eR\cup\{ +\infty,-\infty \}\). Dans cet espace topologique la limite \( \lim_{x\to 0} \frac{1}{ x }\) n'existe pas.
\end{normaltext}

%---------------------------------------------------------------------------------------------------------------------------
\subsection{Prolongement par continuité}
%---------------------------------------------------------------------------------------------------------------------------

%///////////////////////////////////////////////////////////////////////////////////////////////////////////////////////////
\subsubsection{Discussion avec mon ordinateur}
%///////////////////////////////////////////////////////////////////////////////////////////////////////////////////////////

Voici un extrait de ce que peut donner Sage. Nous lui donnons la fonction
\begin{equation}    \label{EqyEHTBZ}
	f(x)=\frac{ x+4 }{ 3x^2+10x-8 }.
\end{equation}
Cette fonction est inventée exprès pour que le dénominateur s'annule en \( -4\). En fait \( 3x^2+10x-8=(x+4)(3x-2)\), et la fraction peut se simplifier en
\begin{equation}
	f(x)=\frac{1}{ 3x-2 }.
\end{equation}
Et avec cela nous écririons \( f(-4)=-\frac{1}{ 14 }\). Voyons comment cela passe dans Sage.

\begin{verbatim}
----------------------------------------------------------------------
| Sage Version 5.2, Release Date: 2012-07-25                         |
| Type "notebook()" for the browser-based notebook interface.        |
| Type "help()" for help.                                            |
----------------------------------------------------------------------
sage: f(x)=(x+4)/(3*x**2+10*x-8)
sage: f(-4)
---------------------------------------------------------------------------
ValueError                                Traceback (most recent call last)
ValueError: power::eval(): division by zero
\end{verbatim}
Il produit donc une erreur de division par zéro. Cela n'est pas étonnant. Pourtant si on lui demande, il est capable de simplifier. En effet :
\begin{verbatim}
sage: f.simplify_full()
x |--> 1/(3*x - 2)
sage: f.simplify_full()(-4)
-1/14
\end{verbatim}

Nous considérons la question suivante : étant donné une fonction \( f\) définie sur \( I\setminus\{ x_0 \}\), est-il possible de définir \( f\) en \( x_0\) de telle façon à ce qu'elle soit continue ?

\begin{example}
	La fonction
	\begin{equation}
		\begin{aligned}
			f\colon \eR\setminus\{ 0 \} & \to \eR               \\
			x                           & \mapsto \frac{1}{ x }
		\end{aligned}
	\end{equation}
	n'est pas définie pour \( x=0\) et il n'y a pas moyen de définir \( f(0)\) de telle sorte que \( f\) soit continue parce que \( \lim_{x\to 0} \frac{1}{ x }\) n'existe pas.
\end{example}

%///////////////////////////////////////////////////////////////////////////////////////////////////////////////////////////
\subsubsection{Limite et prolongement}
%///////////////////////////////////////////////////////////////////////////////////////////////////////////////////////////

Reprenons l'exemple de la fonction \eqref{EqyEHTBZ} que mon ordinateur refusait de calculer en zéro :
\begin{equation}
	f(x)=\frac{ x+4 }{ 3x^2+10x-8 }=\frac{ x+4 }{ (x+4)(3x-2) }.
\end{equation}
Cette fonction a une condition d'existence en \( x=-4\). Et pourtant, tant que \( x\neq -4\), cela a un sens de simplifier les \( (x+4)\) et d'écrire
\[
	f(x)=\frac{ 1 }{ 3x-2 }.
\]
Étant donné que pour toute valeur de \( x\) différente de \( -4\), la fonction \( f\) s'exprime de cette façon, nous avons
\[
	\lim_{x\to -4}f(x)=\lim_{x\to -4}\left(\frac{ 1 }{ 3x-2 }\right).
\]
Oui, mais la fonction\footnote{Cette fonction \( g\) n'est pas \( f\) parce que \( g\) a en plus l'avantage d'être définie en \( -4\).} \( g(x)=1/(3x-2)\) est continue en \( -4\) et donc sa limite vaut sa valeur. Nous en déduisons que
\[
	\lim_{x\to -4}f(x)=-\frac{ 1 }{ 14 }.
\]
Que dire maintenant de la fonction ainsi définie ?
\begin{equation}
	\tilde f(x)=
	\begin{cases}
		f(x)  & \text{si }x\neq -4 \\
		-1/14 & \text{si }x=-4.
	\end{cases}
\end{equation}
Cette fonction est continue en \( -4\) parce qu'elle y est égale à sa limite. Les étapes suivies pour obtenir ce résultat sont :
\begin{itemize}
	\item Repérer un point où la fonction n'existe pas,
	\item calculer la limite de la fonction en ce point, et en particulier vérifier que cette limite existe, ce qui n'est pas toujours le cas,
	\item définir une nouvelle fonction qui vaut partout la même chose que la fonction originale, sauf au point considéré où l'on met la valeur de la limite.
\end{itemize}
C'est ce qu'on appelle \defe{prolonger la fonction par continuité}{prolongement!par continuité} parce que la fonction résultante est continue. La prolongation de \( f\) par continuité est donc en général définie par
\begin{equation}
	\tilde f(x)=
	\begin{cases}
		f(x)              & \text{si }x\neq y \\
		\lim_{y\to x}f(y) & \text{si } x=y
	\end{cases}
\end{equation}
Dans le cas que nous regardions,
\[
	f(x)=\frac{ x+4 }{ 3x^2+10x-8 },
\]
le prolongement par continuité est donné par
\begin{equation}
	\tilde f =\frac{ 1 }{ 3x-2 }.
\end{equation}
Remarquons que cette fonction n'est toujours pas définie en \( x=2/3\).

%---------------------------------------------------------------------------------------------------------------------------
\subsection{Prolongement par continuité}
%---------------------------------------------------------------------------------------------------------------------------

\begin{propositionDef}[Prolongement par continuité]
	Soit \( f\colon I\setminus\{ x_0 \}\to \eR\) telle que \( \lim_{x\to x_{0}} f(x)=\ell\in \eR\). La fonction
	\begin{equation}
		\begin{aligned}
			\tilde f\colon I & \to \eR                      \\
			\tilde f(x)      & =\begin{cases}
				                    f(x) & \text{si } x\neq x_0 \\
				                    \ell & \text{si } x=x_0
			                    \end{cases}
		\end{aligned}
	\end{equation}
	est une fonction continue sur \( I\) et est appelée le \defe{prolongement par continuité}{prolongement!par continuité} de \( f\) en \( x_0\).
\end{propositionDef}
Vous noterez que dans cet énoncé nous demandons \( \ell\in \eR\). Les cas \( \ell=\pm\infty\) sont donc exclus.

\begin{normaltext}
	Le lemme~\ref{LEMooUAFBooAwiXxj} donnera un autre gros morceau de prolongement par continuité. Là, ce ne sera pas juste une valeur qui manquera, mais carrément la majorité des valeurs; mais par contre, ce ne sera pas vraiment de la prolongation par continuité, mais de la prolongation par Cauchy-continuité.
\end{normaltext}

\begin{example}
	La fonction
	\begin{equation}
		\begin{aligned}
			f\colon \eR\setminus\{ -3,2 \} & \to \eR                                  \\
			x                              & \mapsto  \frac{ x^2+2x-3 }{ (x+3)(x-2) }
		\end{aligned}
	\end{equation}
	admet pour limite \( \lim_{x\to -3} f(x)=\frac{ 4 }{ 5 }\). Son prolongement par continuité en \( x=-3\) est donné par
	\begin{equation}
		\tilde f(x)=\frac{ x-1 }{ x-2 }.
	\end{equation}
	Notons que les fonctions \( f\) et \( \tilde f\) ne sont pas identiques : l'une est définie pour \( x=-3\) et l'autre pas. Lorsqu'on fait le calcul
	\begin{equation}
		\frac{ x^2+2x-3 }{ (x+3)(x-2) }=\frac{ (x-1)(x+3) }{ (x+3)(x-2) }=\frac{ x-1 }{ x-2 },
	\end{equation}
	la simplification n'est pas du tout un acte anodin. Le dernier signe «\( =\)» est discutable parce que les deux dernières expressions ne sont pas égales pour tout \( x\); elles ne sont égales «que» pour les \( x\) pour lesquels les deux expressions existent.
\end{example}

Les fonctions trigonométriques donneront quelques exemples intéressants de prolongements par continuité. Voir l'exemple~\ref{ExQWHooGddTLE}. Et une avec la fonction logarithme dans l'exemple~\ref{EXooAGEOooQdQkrS}.

%---------------------------------------------------------------------------------------------------------------------------
\subsection{Théorème de la bijection}
%---------------------------------------------------------------------------------------------------------------------------

À propos de la continuité de l'application réciproque dans divers cas, voir le thème \ref{THEMEooOXNQooERGEGL}.

\begin{lemma}[\cite{MonCerveau}]		\label{LEMooYUOOooByoQdr}
	Si \( I\) est un intervalle de \( \eR\), et si \( x\in \Adh(I)\), alors \( x\) est dans une des trois situations suivantes\footnote{Définitions de suprémum et d'infimum \ref{DefSupeA}.} :
	\begin{enumerate}
		\item
		      \( x\in \Int(I)\)
		\item
		      \( x=\inf(I)\)
		\item
		      \( x=\sup(I)\)
	\end{enumerate}
\end{lemma}

\begin{proof}
	Soit \( a\in \Adh(I)\). Nous supposons que \( a\) n'est pas dans l'intérieur de \( I\), et nous prouvons que nous sommes dans un des deux autres cas. Soit \( \epsilon>0\).
	\begin{subproof}
		\spitem[\( a+\epsilon\) n'est pas un minorant]
		%-----------------------------------------------------------
		Vu que \( a\) est dans l'adhérence de \( I\), pour tout \( n\), il existe \( x_n\) dans \( B(a,1/n)\cap I\). En choisissant \( n\) assez grand, nous avons
		\begin{equation}
			x_n<a+\frac{1}{ n}<a+\epsilon,
		\end{equation}
		de telle sorte que \( a+\epsilon\) ne soit pas un minorant de \( I\).

		\spitem[\( a-\epsilon\) n'est pas un majorant]
		%-----------------------------------------------------------
		De même nous avons \( x_n>a-\frac{1}{ n}>a-\epsilon\).

		\spitem[\( a\geq x\) ou \( a\leq x\) pour tout \( x\in I\)]
		%-----------------------------------------------------------
		Nous devons encore prouver que soit \( a\geq x\) pour tout \( x\in I\), soit que \( a\leq x\) pour tout \( x\in I\). Supposons que ce ne soit pas le cas. Il existe \( x_-,x_+\in I\) tels que
		\begin{equation}
			x_-<a<x_+.
		\end{equation}
		Vu que \( I\) est un intervalle nous avons \( \mathopen[ x_-,x_+\mathclose]\subset I\). Alors en prenant \( r<(x_+-x_-)/2\) nous avons
		\begin{equation}
			B(a,r)\subset \mathopen] x_-,x_+\mathclose[\subset I,
		\end{equation}
		ce qui signifierait que \( a\) est dans l'intérieur de \( I\).
	\end{subproof}
\end{proof}

\begin{lemma}[\cite{BIBooBGGAooUtCJnG}]		\label{LEMooBDGGooSOdOXb}
	Soient un intervalle \( I\) et une application continue \(f \colon I\to \eR  \). Elle est injective si et seulement si elle est strictement monotone.
\end{lemma}

\begin{proof}
	En deux parties.
	\begin{subproof}
		\spitem[Si \( f\) est injective]
		%-----------------------------------------------------------
		Nous prouvons que \( f\) est strictement monotone par l'absurde en supposant qu'elle ne l'est pas. Si \( f\) n'est pas strictement monotone sur \( I\), il existe \( x_1\), \( x_2\), \( x_3\) et \( x_4\) dans \( I\) tels que
		\begin{equation}
			\begin{aligned}[]
				x_1    & <x_1        & x_3    & <x_4         \\
				f(x_1) & \leq f(x_2) & f(x_3) & \geq f(x_4).
			\end{aligned}
		\end{equation}
		Vu que \( I\) est un intervalle, tous les points entre\( x_1\), \( x_2\), \( x_3\) et \( x_4\) sont dans \( I\). Nous pouvons donc définir
		\begin{equation}
			\begin{aligned}
				g\colon \mathopen[ 0,1\mathclose] & \to \eR                                                       \\
				t                                 & \mapsto f\big( tx_1+(1-t)x_2 \big)-f\big( tx_2+(1-t)x_4\big).
			\end{aligned}
		\end{equation}
		Nous avons
		\begin{equation}
			\begin{aligned}[]
				g(0) & =f(x_3)-f(x_4)\geq 0  \\
				g(1) & =f(x_1)-f(x_2)\leq 0.
			\end{aligned}
		\end{equation}
		Le théorème des valeurs intermédiaires \ref{ThoValInter} nous indique qu'il existe \( t_0\in \mathopen[ 0,1\mathclose]\) tel que \( g(t_0)=0\). Nous posons alors \( u=t_0x_1+(1-t_0)x_3\) et \( v=t_0x_2+(1-t_0)x_4\), de telle sorte que
		\begin{equation}
			0=g(t_0)=f(u)-f(v).
		\end{equation}
		Mais \( u\neq v\) parce que \( x_1<x_2\) et \( x_3<x_4\). Nous avons contredit l'injectivité de \( f\).

		\spitem[Si \( f\) est strictement monotone]
		%-----------------------------------------------------------
		Supposons que \( f\) soit strictement croissante. Nous devons prouver que \( f\) est injective. Pour cela supposons que \( f(u)=f(v)\) avec \( u,v\in I\). Nous avons alors
		\begin{subequations}
			\begin{numcases}{}
				f(u)&\leq f(v)\\
				f(u)&\geq f(v).
			\end{numcases}
		\end{subequations}
		Par stricte monotonie nous en déduisons que \( u\leq v\) et que \( u\geq v\), autrement dit, que \( u=v\).
	\end{subproof}
\end{proof}


\begin{proposition}[\cite{BIBooXWMXooPHJnZb, MonCerveau}] \label{PropOARooUuCaYT}
	Une surjection monotone d'une partie \( X\) de \( \eR\) vers un intervalle est continue\footnote{Pour les topologies induites de \( X\) et de \( I\). Dans le cas de \( I\), si il est ouvert, c'est la topologie usuelle.}. Nous sommes dans une des trois situations du lemme \ref{LEMooYUOOooByoQdr}.
\end{proposition}

\begin{proof}
	Soit une application monotone \(g \colon X\to I  \) où \( I\) est un intervalle. Nous supposons dans un premier temps que \( g\) est croissante. Soit un ouvert \( V\) dans \( I\). Nous allons montrer que \( g^{-1}(V)\) est ouvert dans \( X\).

	Pour cela nous considérons \( x\in g^{-1}(V)\) et nous posons \( y=g(x)\). Nous allons prouver qu'il existe un voisinage \( U\) de \( x\) vérifiant \( g(U)\subset V\).

	Le nombre \( y\) est dans un des trois cas du lemme \ref{LEMooYUOOooByoQdr}.
	\begin{subproof}
		\spitem[\( y\) est à l'intérieur de \( I\)]
		%-----------------------------------------------------------
		Il existe \( r>0\) tel que \( B(y,r)\subset V\). Nous posons alors \(  0<\alpha<r  \), et choisissons \( x_-\in g^{-1}(y-\alpha)\) et \( x_+\in g^{-1}(y+\alpha) \). Vu que \( g\) est surjective sur \( I\), ces nombres existent dans \( X\). Par croissance de \( g\), nous avons
		\begin{equation}
			x_-<x<x_+.
		\end{equation}
		Enfin nous considérons la partie \( U=\mathopen] x_-,x_+\mathclose[\cap X\). Ce \( U\) est un ouvert contenant \( x\).

			Avec tout ça nous avons \( g(U)\subset V\).

			\spitem[Si \( y=\sup(I)\)]
			%-----------------------------------------------------------
			Il existe \( r\) tel que \( \mathopen] y-r,y\mathclose]\subset V\). Nous considérons \( x_-\in g^{-1}(y-r)\), et nous posons \( U=\mathopen] x_-,\infty\mathclose[\cap X\). Avec cela nous avons
		\begin{equation}
			g(U)\subset \mathopen] y-\epsilon,y\mathclose]\subset V.
		\end{equation}
		\spitem[Si \( y=\inf(I)\)]
		%-----------------------------------------------------------
		Dans ce cas nous avons un \( r\) tel que \( \mathopen[ y,y+r\mathclose[\subset V\). Nous considérons \( x_+\in g^{-1}(y+r)\) et l'ouvert \( \mathopen] -\infty,x_+\mathclose[\cap X\). Nous avons encore \( g(U)\subset V\).
	\end{subproof}

	Nous devons encore prouver le cas où \( g\) est décroissante. Dans ce cas la fonction \( h(x)=-g(x)\) est croissante et surjective sur l'intervalle \( -I\). Elle est donc continue. Si \( h\) est continue, alors \( g\) est continue.
\end{proof}

\begin{proposition}
	Soient \( f\colon I\to J\) une bijection et \( f^{-1}\colon J\to I\) sa réciproque. Alors pour tout \( x_0\in I\) nous avons
	\begin{equation}    \label{EqHQRooNmLYbF}
		f^{-1}\big( f(x_0) \big)=x_0
	\end{equation}
	et pour tout \( y_0\in J\) nous avons
	\begin{equation}    \label{EqIYTooQPvZDr}
		f\big( f^{-1}(y_0) \big)=y_0.
	\end{equation}
\end{proposition}

\begin{proof}
	Nous prouvons la relation \eqref{EqHQRooNmLYbF} et nous laissons \eqref{EqIYTooQPvZDr} comme exercice au lecteur.

	Soit \( x_0\in I\). Posons \( y_0=f(x_0)\). La définition de l'application réciproque est que pour \( y\in J\), \( f^{-1}(y)\) est l'unique élément \( x\) de \( I\) tel que \( f(x)=y\). Donc \( f^{-1}(y_0)\) est l'unique élément de \( I\) dont l'image est \( y_0\). C'est donc \( x_0\) et nous avons \( f^{-1}(y_0)=x_0\), c'est-à-dire
	\begin{equation}
		f^{-1}\big( f(x_0) \big)=x_0.
	\end{equation}
\end{proof}

\begin{theorem}[Théorème de la bijection] \label{ThoKBRooQKXThd}
	Soit \( I\) un intervalle et \( f\) une fonction continue et strictement monotone de \( I\) dans \( \eR\). Nous avons alors :
	\begin{enumerate}
		\item
		      \( f(I)\) est un intervalle de \( \eR\) ;
		\item       \label{ITEMooMAWXooZXmVwA}
		      La fonction \( f\colon I\to f(I)\) est bijective
		\item
		      La fonction \( f^{-1}\colon f(I)\to I\) est strictement monotone de même sens que \( f\) ;
		\item \label{ItemEJZooKuFoeFiv}
		      La fonction \( f\colon I\to f(I)\) est un homéomorphisme, c'est-à-dire que \( f^{-1}\colon f(I)\to I\) est continue.
	\end{enumerate}
\end{theorem}

\begin{proof}

	Prouvons les choses point par point.

	\begin{enumerate}
		\item
		      Supposons pour fixer les idées que \( f\) est monotone croissante\footnote{Traitez en tant qu'exercice le cas où \(  f\) est décroissante.}.

		      Soient \( a< b\) dans \( f(I)\). Par définition il existe \( x_1,x_2\in I\) tels que \( a=f(x_1)\) et \( b=f(x_2)\). La fonction \( f\) est continue sur l'intervalle \( \mathopen[ x_1 , x_2 \mathclose]\) et vérifie \( f(x_1)<f(x_2)\). Donc le théorème des valeurs intermédiaires~\ref{ThoValInter} nous dit que pour tout \( t\) dans \( \mathopen[ f(x_2) , f(x_2) \mathclose]\), il existe un \( x_0\in\mathopen[ x_1 , x_2 \mathclose]\) tel que \( f(x_0)=t\). Cela montre que toutes les valeurs intermédiaires entre \( a\) et \( b\) sont atteintes par \( f\) et donc que \( f(I)\) est un intervalle.

		\item
		      Nous prouvons maintenant que \( f\) est bijective en prouvant séparément qu'elle est surjective et injective.

		      \begin{subproof}
			      \spitem[\( f\) est surjective]
			      Une fonction est toujours surjective depuis un intervalle \( I\) vers l'ensemble \(\Image f \).

			      \spitem[\( f\) est injective]
			      Soit \( x\neq y\) dans \( I\); pour fixer les idées nous supposons que \( x<y\). La stricte monotonie de \( f\) implique que \( f(x)<f(y)\) ou que \( f(x)>f(y)\). Dans tous les cas \( f(x)\neq f(y)\).

		      \end{subproof}
		      La fonction \( f\) est donc bijective.

		\item
		      Comme d'accoutumée nous supposons que \( f\) est croissante. Soient \( y_1<y_2\) dans \( f(I)\); nous devons prouver que \( f^{-1}(y_1)\leq f^{-1}(y_2)\). Pour cela nous considérons les nombres \( x_1,x_2\in I\) tels que \( f(x_1)=y_1\) et \( f(x_2)=y_2\). Nous allons en prouver la contraposée en supposant que \( f^{-1}(y_1)>f^{-1}(y_2)\). En appliquant \( f\) (qui est croissante) à cette dernière inégalité il vient
		      \begin{equation}
			      f\big( f^{-1}(y_1) \big)\geq f\big( f^{-1}(y_2) \big),
		      \end{equation}
		      ce qui signifie
		      \begin{equation}
			      y_1\geq y_2
		      \end{equation}
		      par l'équation \eqref{EqIYTooQPvZDr}.

		\item
		      La fonction \( f^{-1}\colon f(I)\to I\) est une fonction monotone et surjective, donc continue par la proposition~\ref{PropOARooUuCaYT}.

	\end{enumerate}
\end{proof}

\begin{example}
	La fonction
	\begin{equation}
		\begin{aligned}
			f\colon \mathopen[ 2 , 3 \mathclose] & \to \mathopen[ 4 , 9 \mathclose] \\
			x                                    & \mapsto x^2
		\end{aligned}
	\end{equation}
	est une bijection. Sa réciproque est la fonction
	\begin{equation}
		\begin{aligned}
			f^{-1}\colon \mathopen[ 4 , 9 \mathclose] & \to \mathopen[ 2 , 3 \mathclose] \\
			x                                         & \mapsto \sqrt{x}.
		\end{aligned}
	\end{equation}
\end{example}

%+++++++++++++++++++++++++++++++++++++++++++++++++++++++++++++++++++++++++++++++++++++++++++++++++++++++++++++++++++++++++++
\section{Limite et continuité}
%+++++++++++++++++++++++++++++++++++++++++++++++++++++++++++++++++++++++++++++++++++++++++++++++++++++++++++++++++++++++++++
\label{SecLimiteFontion}

Voir les remarques dans l'index thématique~\ref{THEMEooGVCCooHBrNNd} pour comprendre la place et la portée de ce qui va venir à propos de limite et de continuité.

\begin{theorem}[Limite et continuité]           \label{ThoLimCont}
	La fonction \( f\) est continue au point \( a\) si et seulement si \( \lim_{x\to a}f(x)=f(a)\).
\end{theorem}

\begin{proof}
	Nous commençons par supposer que \( f\) est continue en \( a\), et nous prouvons que \( \lim_{x\to a}f(x)=f(a)\). Soit \( \epsilon>0\); ce qu'il nous faut c'est un \( \delta\) tel que \( | x-a |\leq\delta\) implique \( | f(x)-f(a) |\leq\epsilon\). La caractérisation \ref{PROPooVNGEooPwbxXP} de la continuité donne l'existence d'un \( \delta\) comme il nous faut.

	Dans l'autre sens, c'est-à-dire prouver que \( f\) est continue au point \( a\) sous l'hypothèse que \( \lim_{x\to a}f(x)=f(a)\), la preuve se fait de la même façon.
\end{proof}

Nous en déduisons que si nous voulons gagner quelque chose à parler de limites, il faut prendre des fonctions non continues. En effet, si une fonction est continue en un point, la limite ne donne aucune nouvelle information que la valeur de la fonction elle-même en ce point.

Prenons une fonction qui fait un saut. Pour se fixer les idées, prenons celle-ci :
\begin{equation}    \label{EqnCtOEL}
	f(x)=
	\begin{cases}
		2x  & \text{si }x\in]-\infty,2[ \\
		x/2 & \text{si }x\in[2,+\infty[
	\end{cases}
\end{equation}
Essayons de trouver la limite de cette fonction lorsque \( x\) tend vers \( 2\). Étant donné que \( f\) n'est pas continue en \( 2\), nous savons déjà que \( \lim_{x\to 2}f(x)\neq f(2)\). Donc ce n'est pas \( 1\). Cette limite ne peut pas valoir \( 4\) non plus parce qu'en prenant n'importe quel \( \epsilon\), la valeur de \( f(2+\epsilon)\) est très proche de \( 2\), et donc, ne peut pas s'approcher de \( 4\). En fait, on peut facilement vérifier que \emph{aucun nombre ne vérifie la condition de limite pour \( f\) en \( 2\)}. Nous disons que la limite n'existe pas.

Il ne faudrait pas en déduire trop vite que si une fonction n'est pas continue en \( a\), alors la limite \( x\to a\) n'existe pas. Ce que dit le théorème~\ref{ThoLimCont} est que si une fonction n'est pas continue en \( a\), alors sa limite (si elle existe) ne vaut pas \( f(a)\).

\begin{example}[Un exemple de continuité ; Thème~\ref{THEMEooGVCCooHBrNNd}]     \label{EXooKREUooLeuIlv}
	Soit la fonction
	\begin{equation}        \label{EQooSYSWooSGsUfR}
		f(x)=\begin{cases}
			x & \text{si } x\neq 0 \\
			4 & \text{si } x=0.
		\end{cases}
	\end{equation}
	Cette fonction n'est pas continue en \( x=0\), et pourtant la limite existe : \( \lim_{x\to 0} f(x)=0\). Etudions cela en détail, pour nous assurer de ce qu'il se passe.

	Considérons l'ouvert \( \mathopen] 3 , 5 \mathclose[\). L'image réciproque de cet ouvert par \( f\) est la partie \( \mathopen] 3 , 5 \mathclose[\cup\{ 0 \}\) qui n'est pas ouvert. Donc la fonction \( f\) n'est pas continue comme fonction \( \eR\to \eR\).

		Considérons pour comprendre la restriction \( f\colon \mathopen[ -1 , 1 \mathclose]\to \eR\). L'image inverse de \( \mathopen] 3 , 5 \mathclose[\) par cette fonction est \( \{ 0 \}\) qui n'est pas un ouvert.

		Plus généralement tant qu'on considère des restrictions de \( f\) sur des parties contenant un voisinage de \( 0\), la fonction ne peut pas être continue\footnote{Les plus acharnés se demanderont ce qu'il se passe pour la restriction de \( f\) à la partie \( \{ 0 \}\) munie de la topologie induite de \( \eR\).}.

		Voyons ce qu'il en est de la continuité ponctuelle de \( f\) en \( x=0\). La définition~\ref{DefOLNtrxB} est celle de la continuité en un point; elle dit que \( f\) sera continue en \( 0\) si \( f(0)=4\) est une limite de \( f\). Nous voilà parti vers la définition~\ref{DefYNVoWBx}.

		Soit le voisinage \( V=\mathopen] 3 , 5 \mathclose[\) de \( f(0)\). Quel que soit le voisinage \( W\) de \( 0\) dans \( \eR\), il existe un \( \epsilon>0\) tel que \( W\subset B(0,\epsilon)\). Nous avons alors
	\begin{equation}
		f\big( W\setminus \{ 0 \} \big)\subset f\big( B(0,\epsilon)\setminus\{ 0 \} \big).
	\end{equation}
	Mais le nombre \( \epsilon/2\) fait partie de \( f\big( B(0,\epsilon)\setminus\{ 0 \} \big)\) et n'est pas dans \( V\). Donc \( f(0)\) n'est pas une limite de \( f\) en zéro. Cette fonction n'est donc pas continue en zéro.
\end{example}

\begin{example}[Même exemple, limite]
	Nous avons vu que, pour la fonction \eqref{EQooSYSWooSGsUfR}, le nombre \( 4\) n'est pas une limite de \( f\) en zéro. Nous montrons à présent que \( 0\) est une limite (et même la seule par la proposition~\ref{PropFObayrf} que nous ne rappellerons plus à chaque fois) de \( f\).

	Montrons que \( 0\) est une limite de \( f\) en zéro, c'est-à-dire que \( \lim_{x\to 0} f(x)=0\).

	Nous suivons la définition~\ref{DefYNVoWBx}. Soit un voisinage \( V\) de \( 0\) dans \( \eR\). Il existe \( \delta\) tel que \( B(0,\delta)\subset V\). En posant \( \epsilon=\delta\) et en définissant \( W=B(0,\epsilon)\) nous avons
	\begin{equation}
		f\big( B(0,\epsilon)\setminus\{ 0 \} \big)=B(0,\epsilon)\setminus\{ 0 \}\subset  B(0,\delta)\subset V.
	\end{equation}
	Donc \( 0\) est une limite de \( f\) en zéro.
\end{example}

Nous avons déjà vu par le corolaire~\ref{CorFHbMqGGyi} qu'une suite croissante et bornée était convergente. Il en va de même pour les fonctions.
\begin{proposition}[\cite{MonCerveau}]            \label{PropMTmBYeU}
	Si la fonction réelle \( f\colon I=\mathopen[ a , b [\to \eR\) est croissante et bornée, alors la limite
	\begin{equation}
		\lim_{x\to b} f(x)
	\end{equation}
	existe et est finie.
\end{proposition}

\begin{proof}
	Commençons par prouver que si \( (x_n)\) est une suite dans \( I\) convergeant vers \( b\), alors \( f(x_n)\) est une suite convergente. Dans un second temps, nous allons prouver que si \( (x_n)\) et \( (x'_n)\) sont deux suites qui convergent vers \( b\), alors les suites convergentes \( f(x_n)\) et \( f(x'_n)\) convergent vers la même limite. Alors le critère séquentiel de la limite d'une fonction conclura (proposition~\ref{PROPooJYOOooZWocoq}).

	Nous pouvons extraire de \( x_n\) une sous-suite croissante \( (x_{\alpha(n)})\). Alors la suite \( f\big( x_{\alpha(n)} \big)\) est une suite croissante et majorée, donc convergente par le corolaire~\ref{CorFHbMqGGyi}\footnote{En gros nous sommes en train de dire que toute la théorie des fonctions convexes est un vulgaire corolaire de Bolzano-Weierstrass.}. Nommons \( \ell\) la limite et montrons qu'elle est aussi limite de \( f\) sur la suite originale.

	Pour tout \( \epsilon>0\), il existe \( K\) tel que si \( n>K\) alors \( \big| f\big( x_{\alpha(n)} \big)-\ell \big|<\epsilon\). Soit \( K'\) tel que pour tout \( n>K'\) nous ayons \( x_n>x_{\alpha(K)}\). Cela est possible parce que la suite est bornée par \( b\) et converge vers \( b\) : il suffit de prendre \( K'\) de telle sorte que \( | x_n-b |\leq | x_{\alpha(n)}-b |\). Si \( n>K'\) alors \( x_n>x_{\alpha(K)}\) et
	\begin{equation}
		f(x_n)\geq f(x_{\alpha(n)})\geq \ell-\epsilon;
	\end{equation}
	en résumé si \( n>K\) alors \( | f(x_n)-\ell |<\epsilon\). Cela prouve que \( f(x_n)\to\ell\).

	Soit maintenant une autre suite \( (x'_n)\) qui converge également vers \( b\). Comme nous venons de le voir la suite \( f(x'_n)\) est convergente et nous nommons \( \ell'\) la limite. Si nous considérons \( (x''_n)\) la suite «alternée» (\( x_1,x'_1,x_2,x'_2,\cdots\)) alors nous avons encore une suite qui converge vers \( b\) et donc \( f(x''_n)\to \ell'\).

	Mais étant donné que \( f(x_n)\) et \( f(x'_n)\) sont des sous-suites, elles doivent converger vers la même valeur. Donc \( \ell=\ell'=\ell''\).
\end{proof}


\begin{proposition}[\cite{MonCerveau}]      \label{PROPooLOQVooULDhZz}
	L'ensemble \( \eR\cup\{ \infty \}\) muni de la topologie de la compactification en un point\footnote{Définition \ref{PROPooHNOZooPSzKIN}.} est connexe par arcs.
\end{proposition}

\begin{proof}
	Nous allons montrer que le chemin
	\begin{equation}
		\begin{aligned}
			\gamma\colon \mathopen[ 0 , 1 \mathclose] & \to \mathopen[ 0 , \infty \mathclose]      \\
			x                                         & \mapsto \begin{cases}
				                                                    \frac{1}{ x } & \text{si } x\neq 0 \\
				                                                    \infty        & \text{si } x=0
			                                                    \end{cases}
		\end{aligned}
	\end{equation}
	est continu au sens de la définition \ref{DefOLNtrxB}\ref{ITEMooEHGWooDdITRV} (qui est le seul sens possible au mot «continu»).

	Soit un ouvert \( \mO\) dans \( \eR\cup\{ \infty \}\). Si cet ouvert ne contient pas \( \infty\), alors \( \gamma^{-1}(\mO)\) est ouvert dans \( \eR\) parce que la fonction \( x\mapsto 1/x\) est continue\footnote{Voir par exemple la proposition \ref{PropOpsSimplesLimites}.}.

	Si \( \infty\in\mO\), alors \( \mO=\{ \infty \}\cup \mO'\) où \( \mO'\) est un ouvert de \( \eR\) ayant la propriété que \( \eR\setminus \mO'\) est compact.

	Nous avons \( \gamma^{-1}(\mO)=\{ 0 \}\cup\gamma^{-1}(\mO')\). Et aussi que \( \gamma^{-1}(\mO')\) est un ouvert de \( \eR\) contenu dans \( \mathopen[ 0 , 1 \mathclose]\).

	Puisque le complémentaire de \( \mO'\) est compact, il existe \( a\in \eR\) tel que \( \mathopen] a , \infty \mathclose[\subset \mO'\). Donc \( \mO'=\mathopen] a , \infty \mathclose[\cup\mO''\) où \( \mO''\) est un ouvert.

	Nous avons :
	\begin{subequations}
		\begin{align}
			\gamma^{-1}(\mO) & =\{ \gamma^{-1}(\infty) \}\cup \gamma^{-1}\big( \mathopen] a , \infty \mathclose[ \big)\cup \gamma^{-1}(\mO'') \\
			                 & =\{ 0 \}\cup\mathopen] 0 , \frac{1}{  a } \mathclose[\cup\gamma^{-1}(\mO'')                                    \\
			                 & =\mathopen[ 0 , \frac{1}{ a } \mathclose[\cup\gamma^{-1}(\mO'').
		\end{align}
	\end{subequations}
	Vous noterez le point essentiel où la topologie de la compactification agit : comme \( \{0\}\) n'est pas un ouvert de \( \mathopen[ 0 , 1 \mathclose]\), nous devons nous assurer que la partie \( \gamma^{-1}(\mO')\) vienne se coller à \( \{0\}\) pour compléter en un ouvert.

	L'ensemble \( \gamma^{-1}(\mO'')\) est un ouvert de \( \eR\) contenu dans \( \mathopen[ 0 , 1 \mathclose]\). Nous avons donc
	\begin{equation}
		\gamma^{-1}(\mO)=\Big( \mathopen] -1 , \frac{1}{ a } \mathclose[\cup\gamma^{-1}(\mO'') \Big)\cap \mathopen[ 0 , 1 \mathclose].
	\end{equation}
	Cela est un ouvert de \( \mathopen[ 0 , 1 \mathclose]\) par définition de la topologie induite\footnote{Définition \ref{DefVLrgWDB}.}.
\end{proof}

%---------------------------------------------------------------------------------------------------------------------------
\subsection{Prolongement des rationnels vers les réels}
%---------------------------------------------------------------------------------------------------------------------------

Si \( f\colon \eQ\to \eR\) est une fonction continue pour la topologie induite, est-ce qu'on peut la prolonger en une fonction continue sur \( \eR\) ? La réponse est hélas non.

\begin{example}[\cite{BIBooOFHOooWZGRPw}]       \label{EXooWZNCooQkKdtJ}
	Puisque \( \sqrt{ 2 }\) est irrationnel\footnote{Proposition \ref{PropooRJMSooPrdeJb}. Le fait que \( \sqrt{ 2 }\) existe dans \( \eR\) est la proposition \ref{PROPooUHKFooVKmpte}.}, ceci définit bien une fonction sur \( \eQ\) :
	\begin{equation}
		\begin{aligned}
			f\colon \eQ & \to \eR                              \\
			q           & \mapsto \begin{cases}
				                      0 & \text{si } q<\sqrt{ 2 }  \\
				                      1 & \text{si } q>\sqrt{ 2 }.
			                      \end{cases}
		\end{aligned}
	\end{equation}
	C'est une fonction continue sur \( \eQ\). En effet, soient \( q\in \eQ\) et \( \epsilon>0\). Nous prenons \( \delta>0\) tel que \( \sqrt{ 2 }\) ne soit pas dans \( B(q,\delta)\). Alors si \( p\in B_{\eQ}(q,\delta)\) nous avons \( f(q)=f(p)\) et donc
	\begin{equation}
		| f(p)-f(q) |<\epsilon.
	\end{equation}

	Il n'est cependant pas possible de la prolonger en une fonction continue sur \( \eR\).
\end{example}

Pour qu'une fonction sur \( \eQ\) puisse être prolongée en une fonction continue sur \( \eR\), il faut un peu plus que la continuité. Il faut la Cauchy-continuité, que nous définissons pas plus tard qu'immédiatement.

\begin{definition}[\cite{BIBooOFHOooWZGRPw}]      \label{DEFooXXOGooXblyKP}
	Soient \( X\) et \( Y\) deux espaces métriques. Une application \( f\colon X\to Y\) est dite \defe{Cauchy-continue}{Cauchy-continue} si pour toute suite de Cauchy \( (x_n)\) dans \( X\), la suite \( \big( f(x_n) \big)\) est de Cauchy dans \( Y\).
\end{definition}

\begin{proposition}[\cite{BIBooERJSooURHjMX}]       \label{PROPooPUNFooFMytOY}
	Soient \( X\) et \( Y\) des espaces métriques\footnote{Il a une distance, définition \ref{DefMVNVFsX}.}. Nous supposons que \( Y\) est complet\footnote{Définition \ref{DEFooHBAVooKmqerL}.}. Nous considérons \( A\subset X\) et une application Cauchy-continue \( f\colon A\to Y\).

	Alors:
	\begin{enumerate}
		\item
		      Il existe un unique prolongement continu \( g\colon \bar A\to Y\) de \( f\).
		\item       \label{ITEMooDEZGooUAvwvF}
		      Ce prolongement est lui-même Cauchy-continu.
	\end{enumerate}
\end{proposition}

\begin{proof}
	En plusieurs parties.
	\begin{subproof}
		\spitem[Une suite]
		% -------------------------------------------------------------------------------------------- 

		Soit \( x\in\bar A\). Nous considérons une suite \( (a_n)\) dans \( A\) telle que \( a_n\stackrel{X}{\longrightarrow}x\). Alors \( (a_n)\) est de Cauchy\footnote{Proposition \ref{PROPooZZNWooHghltd}.} et donc \( \big( f(a_n) \big)\) est de Cauchy dans \( Y\). Vu que \( Y \) est complet, \( \big( f(a_n) \big)\) est une suite convergente.
		\spitem[Indépendance]
		% -------------------------------------------------------------------------------------------- 

		Montrons que si \( (b_n)\) est une autre suite dans \( A\) convergeant vers \( x\), alors \( \lim_n f(a_n)=\lim_nf(b_n)\). Pour cela nous considérons la suite\( (a_0,b_0,a_1,b_1,\ldots, )\). Cela est une suite qui converge vers \( x\). Elle est donc de Cauchy et son image par \( f\) est encore de Cauchy. Toutes les sous-suites de la suite image sont dont convergentes vers la même limite. En prenant les sous-suites paires et impaires, nous avons \( \lim_n f(a_n)=\lim_nf(b_n)\).

		\spitem[Définition de \( g\)]
		% -------------------------------------------------------------------------------------------- 
		Nous pouvons donc définir
		\begin{equation}
			\begin{aligned}
				g\colon \bar A & \to Y                 \\
				x              & \mapsto \lim_n f(a_n)
			\end{aligned}
		\end{equation}
		où \( (a_n)\) est une quelconque suite dans \( A\) convergeant vers \( x\).
		\spitem[\( g\) est séquentiellement continue]
		% -------------------------------------------------------------------------------------------- 
		Soit une suite \( (x_n)\) dans \( \bar A\). Nous devons prouver que \( g(x_n)\stackrel{Y}{\longrightarrow}g(x)\).

		Nous pouvons choisir \( a_n\in A\) tel que \( d(a_n,x_n)<1/n\). Si nous considérons une suite de Cauchy dans \( A\) convergeant vers \( x_n\), et que nous prenons \( a_n\) assez loin dans cette suite, alors nous pouvons même choisir \( a_n\) tel que \( d\big( f(a_n),g(x_n) \big)<1/n\). Nous choisissons \( (a_n)\) de telle sorte à vérifier les deux en même temps.

		Nous avons
		\begin{equation}
			d(a_n,x)\leq d(a_n,x_n)+d(x_n,x)\to 0.
		\end{equation}
		Donc \( (a_n)\) est de Cauchy et par définition de \( g\), nous avons
		\begin{equation}    \label{EQooWQYNooKIOAjc}
			g(x)=\lim_{n\to \infty} f(a_n).
		\end{equation}
		À partir de là nous avons la majoration
		\begin{equation}
			d\big( g(x_n),g(x) \big)\leq d\big( g(x_n),f(a_n) \big)+d\big( f(a_n),g(x) \big).
		\end{equation}
		Le premier terme tend vers zéro par choix de la suite \( (a_n)\) et le second tend vers zéro par \eqref{EQooWQYNooKIOAjc}.
		\spitem[\( g\) est continue]
		% -------------------------------------------------------------------------------------------- 
		La proposition \ref{PropXIAQSXr} donne l'équivalence entre la continuité et la continuité séquentielle.
		\spitem[Unicité]
		% -------------------------------------------------------------------------------------------- 
		Soient \( g_1\) et \( g_2\), deux prolongations continues de \( f\) sur \( \bar A\). Si \( a_n\to x\) dans \( A\), alors nous avons, par continuité de \( g_1\) et \( g_2\) : \( g_1(a_n)\to g_1(x)\) et \( g_2(a_n)\to g_2(x)\).

		Mais \( g_1(a_n)=g_2(a_n)=f(a_n)\); donc les limites sont égales.
		\spitem[\( g\) est Cauchy continue]
		% -------------------------------------------------------------------------------------------- 
	\end{subproof}
\end{proof}

En terme de prolongement continu, nous avons ce lemme qui demande à une fonction d'être Cauchy-continue. Vous pouvez comparer avec le principe de prolongement analytique \ref{ThoAVBCewB} qui donne un énoncé similaire pour un prolongement analytique.
\begin{lemma}[\cite{MonCerveau, BIBooUNVDooGfFtGp,BIBooERJSooURHjMX}]           \label{LEMooUAFBooAwiXxj}
	Soit une fonction Cauchy-continue \footnote{Définition \ref{DEFooXXOGooXblyKP}; nous en avons discuté dans l'exemple~\ref{EXooWZNCooQkKdtJ}.} \( f\colon \eQ\to \eR\).
	\begin{enumerate}
		\item
		      La limite \( \lim_{q\to x} f(q)\) existe pour tout \( x\in \eR\).
		\item
		      Il existe un unique prolongement continu \( \tilde f\colon \eR\to \eR\).
		\item
		      Ce prolongement est donné par
		      \begin{equation}
			      \tilde f(x)=\begin{cases}
				      f(x)               & \text{si } x\in \eQ \\
				      \lim_{q\to x} f(q) & \text{sinon }
			      \end{cases}
		      \end{equation}
	\end{enumerate}
\end{lemma}

\begin{proof}
	C'est tout dans la proposition \ref{PROPooPUNFooFMytOY} en sachant que
	\begin{itemize}
		\item \( \eQ\) est dense dans \( \eR\), proposition \ref{PropooUHNZooOUYIkn}.
		\item   \( \eR\) est complet, théorème \ref{THOooNULFooYUqQYo}.
	\end{itemize}
\end{proof}

\begin{proposition}     \label{PROPooXWHYooFiVYfi}
	Soient des fonctions continues \( f,g\colon \eR\to \eR\). Si \( f\) et \( g\) sont égales sur \( \eQ\), alors elles sont égales sur \( \eR\).
\end{proposition}

\begin{proof}
	Nous pouvons utiliser les propriétés fondamentales des réels et de la continuité. Soit \( x\in \eR\); nous voulons montrer que \( f(x)=g(x)\). En prenant par exemple le lemme \ref{LemooRTGFooYVstwS}, il existe une suite \( q_i\) de rationnels telle que \( q_i\stackrel{\eR}{\longrightarrow}x\).

	Par ailleurs, \( f\) et \( g\) sont continues sur \( \eR\) et donc en chaque point de \( \eR\) (théorème \ref{ThoESCaraB}). Par la caractérisation séquentielle \ref{PropFnContParSuite} de la continuité, nous avons
	\begin{equation}
		f(x)=\lim_{i\to \infty} f(q_i)=\lim_{i\to \infty} g(q_i)=g(x).
	\end{equation}
\end{proof}

\begin{proposition}[\cite{MonCerveau}]      \label{PROPooTNIAooNAJDzL}
	Soit une fonction strictement croissante \( f\colon \eQ\to \eR\). Alors la prolongation continue \( \tilde f\colon \eR\to \eR\) est également strictement croissante.
\end{proposition}

\begin{proof}
	Soient \( x,y\in \eR\) avec \( x<y\). Notons \( d=y-x\). Nous considérons des suites de rationnels \( x_k\to x\) et \( y_l\to y\) telles que pour tout \( k\), \( x_k\in B(x,d/3)\) et \( y_k\in B(y,d/3)\). En particulier, \( x_k<y_l\) pour tout \( k\) et \( l\).

	Soient des rationnels \( q\) et \( q'\) tels que pour tout \( k\),
	\begin{equation}
		x_k<q<q'<y_k.
	\end{equation}
	Pour trouver de tels rationnels, il suffit de les chercher dans \( \mathopen] x+\frac{ d }{ 3 } , y-\frac{ d }{ 3 } \mathclose[\). Cet intervalle étant de longueur \( d/3\), il contient des rationnels.

	Vue la croissance de \( f\) sur \( \eQ\), nous avons, pour tout \( k\) :
	\begin{equation}
		f(x_k)<f(q)<f(q')<f(y_k),
	\end{equation}
	et à la limite :
	\begin{equation}
		\tilde f(x)\leq f(q)<f(q')\leq \tilde f(y).
	\end{equation}
	Notez que les inégalités strictes se changent en inégalités larges au passage à la limite. D'où l'utilité de prendre \emph{deux} rationnels entre \( x_k\) et \( y_k\) pour maintenir une inégalité stricte entre \(\tilde f(x)\) et \( \tilde f(y)\).
\end{proof}

%+++++++++++++++++++++++++++++++++++++++++++++++++++++++++++++++++++++++++++++++++++++++++++++++++++++++++++++++++++++++++++
\section{Espace des fonctions continues}
%+++++++++++++++++++++++++++++++++++++++++++++++++++++++++++++++++++++++++++++++++++++++++++++++++++++++++++++++++++++++++++

\begin{definition}
	Soit \( I\), un intervalle de \( \eR\). L'\defe{oscillation}{oscillation!d'une fonction} sur \( I\) est le nombre
	\begin{equation}
		\omega_f(I)=\sup_{x\in I}f(x)-\inf_{x\in I}f(x).
	\end{equation}
\end{definition}
Pour chaque \( x\) fixé, la fonction
\begin{equation}
	x\mapsto \omega_f\big( B(x,\delta) \big)
\end{equation}
est une fonction positive, croissante et a donc une limite (pour \( \delta\to 0\)). Nous notons \( \omega_f(x)\) cette limite qui est l'\defe{oscillation}{oscillation!d'une fonction en un point} de \( f\) en ce point. Une propriété immédiate est que \( f\) est continue en \( x_0\) si et seulement si \( \omega_f(x_0)=0\).

\begin{lemma}       \label{LemuaPbtQ}
	L'ensemble des points de discontinuité d'une fonction \( f\colon \eR\to \eR\) est une réunion dénombrable de fermés.
\end{lemma}

\begin{proof}
	Soit \( D\) l'ensemble des points de discontinuité de \( f\). Nous avons
	\begin{equation}
		D=\bigcup_{n=1}^{\infty}\{ x\tq \omega_f(x)\geq \frac{1}{ n } \}.
	\end{equation}
	Il nous suffit donc de montrer que pour tout \( \epsilon\), l'ensemble
	\begin{equation}
		\{ x\tq \omega_f(x)<\epsilon \}
	\end{equation}
	est ouvert. Soit en effet \( x_0\) dans cet ensemble. Il existe \( \delta\) tel que \( \omega_f\big( B(x_0,\delta) \big)<\epsilon\). Si \( x\in B(x_0,\delta)\), alors si on choisit \( \delta'\) tel que \( B(x,\delta')\subset B(x_0,\delta)\), nous avons \( \omega_f\big( B(x,\delta') \big)<\epsilon\), ce qui justifie que \( \omega_f(x)<\epsilon\) et donc que \( x\) est également dans l'ensemble considéré.
\end{proof}

\begin{theorem}
	L'ensemble des points de discontinuité d'une limite simple de fonctions continues est de première catégorie.
\end{theorem}

\begin{proof}
	Soit \( (f_n)\) une suite de fonctions qui converge simplement vers \( f\). Nous devons écrire l'ensemble des points de discontinuité de \( f\) comme une union dénombrable d'ensembles tels que sur tout intervalle \( I\), aucun de ces ensembles n'est dense. Nous savons déjà par le lemme~\ref{LemuaPbtQ} que l'ensemble des points de discontinuité  de \( f\) est donné par
	\begin{equation}
		D=\bigcup_{n=1}^{\infty}\{ x\tq \omega_f(x)\geq \frac{1}{  n } \}.
	\end{equation}
	Nous essayons donc de prouver que pour tout \( \epsilon\), l'ensemble
	\begin{equation}
		F=\{ x\tq \omega_f(x)\geq \epsilon \}
	\end{equation}
	est nulle part dense. Soit
	\begin{equation}
		E_n=\bigcap_{i,j>n}\{ x\tq | f_i(x)-f_j(x) |<\epsilon \}.
	\end{equation}
	Nous montrons que cet ensemble est fermé en étudiant le complémentaire. Soit \( x\notin E_n\); alors il existe un couple \( (i,j)\) tel que
	\begin{equation}
		| f_i(x)-f_j(x) |>\epsilon.
	\end{equation}
	Par continuité, cette inégalité reste valide dans un voisinage de \( x\). Donc il existe un voisinage de \( x\) contenu dans \( \complement E_n\) et \( E_n\) est donc fermé.

	De plus nous avons \( E_n\subset E_{n+1}\) et \( \bigcup_nE_n=\eR\). Ce dernier point est dû au fait que pour tout \( x\), il existe \( N\) tel que \( i,j>N\) implique \( | f_i(x)-f_j(x) |\leq \epsilon\). Cela est l'expression du fait que la suite \( \big( f_n(x) \big)_{n\in \eN}\) est de Cauchy.

	Soit \( I\), un intervalle fermé de \( \eR\). Nous voulons trouver un intervalle \( J\subset I\) sur lequel \( f\) est continue. Nous écrivons \( I\) sous la forme
	\begin{equation}
		I=\bigcup_{n=1}^{\infty}(E_n\cap I).
	\end{equation}
	Tous les ensembles \( J_n=E_n\cap I\) ne peuvent être nulle part dense en même temps (à cause du théorème de Baire~\ref{ThoQGalIO}). Il existe donc un \( n\) tel que \( J_n\) contienne un ouvert \( J\). Le but est de montrer que \( f\) est continue sur \( J\). Pour ce faire, nous n'allons pas simplement majorer \( | f(x)-f(x_0) |\) par \( \epsilon\) lorsque \( | x-x_0 |\) est petit. Nous allons majorer l'oscillation de \( f\) sur \( B(x_0,\delta)\) lorsque \( \delta\) est petit. Pour cela nous prenons \( x_0\) et \( x\) dans \( J\) et nous écrivons
	\begin{equation}
		| f(x)-f(x_0) |\leq | f(x)-f_n(x) |+| f_n(x)-f_n(x_0) |.
	\end{equation}
	À ce niveau nous rappelons que \( n\) est fixé par le choix de \( J\), dans lequel \( \epsilon\) est déjà inclus. Nous choisissons évidemment \( | x-x_0 |\leq \delta\) de telle sorte que le second terme soit plus petit que \( \epsilon\) en vertu de la continuité de \( f_n\). Pour le premier terme, pour tout \( i,j\geq n\) nous avons
	\begin{equation}
		| f_i(x)-f_j(x) |<\epsilon.
	\end{equation}
	Si nous posons \( j=n\) et \( i\to\infty\), en tenant compte du fait que \( f_i\to f\) simplement,
	\begin{equation}
		| f(x)-f_n(x) |\leq \epsilon.
	\end{equation}
	Nous avons donc obtenu \( | f(x)-f_n(x_0) |\leq 2\epsilon\). Cela signifie que dans un voisinage de rayon \( \delta\) autour de \( x_0\), les valeurs extrêmes prises par \( f(x) \) sont \( f_n(x_0)\pm 4\epsilon\). Nous avons donc prouvé que pour tout \( \epsilon\), il existe \( \delta\) tel que
	\begin{equation}
		\omega_f\big( \mathopen[ x_0-\delta , x_0+\delta \mathclose] \big)\leq 4\epsilon.
	\end{equation}
	De là nous concluons que
	\begin{equation}
		\lim_{\delta\to 0}\omega_f\big( \mathopen[ x_0-\delta , x_0+\delta \mathclose] \big)=0,
	\end{equation}
	ce qui signifie que \( f\) est continue en \( x_0\).
\end{proof}

\begin{example}
	Une fonction discontinue sur \( \eQ\) et continue ailleurs. La fonction
	\begin{equation}
		f(x)=\begin{cases}
			0             & \text{si } x\notin \eQ \\
			\frac{1}{ q } & \text{si } x=p/q
		\end{cases}
	\end{equation}
	où par «\( x=p/q\)» nous entendons que \( p/q\) est la fraction irréductible\footnote{Proposition \ref{THOooWYQVooRBaAAM}.}.

	Cette fonction est discontinue sur \( \eQ\) parce que si \( q\in \eQ\) alors \( f(q)\neq 0\) alors que dans tous voisinage de \( q\) il existe un irrationnel sur qui la fonction vaudra zéro.

	Montrons que \( f\) est continue sur les irrationnels. Si \( x_0\notin \eQ\) alors \( f(x_0)=0\). Mais si on prend un voisinage suffisamment petit de \( x_0\), nous pouvons nous arranger pour que tous les rationnels aient un dénominateur arbitrairement grand. En effet si nous nous fixons un premier rayon \( r_0>0\) alors il existe un nombre fini de fractions de la forme \( 1\), \( \frac{ k }{2}\), \( \frac{ k }{ 3 }\),\ldots, \( \frac{ k }{ N }\) dans \( B(x_0,r_0)\). Il suffit maintenant de choisir \( 0<r\leq r_0\) tel que ces fractions soient toutes hors de \( B(x_0,r)\). Dans cette boule nous avons \( f<\frac{1}{ N }\). Du coup \( f\) est continue en \( x_0\).
\end{example}

\begin{definition}[Point périodique\cite{TMCHooOaTrJL}]
	Soit \( f\colon I\to I\) une application d'un ensemble \( I\) dans lui-même. Si \( x\in I\) vérifie \( f^n(x)=x\) et \( f^k(x)\neq x\) pour \( k=1,\ldots, n-1\) alors on dit que \( x\) est un point \( n\)-périodique.
\end{definition}

\begin{lemma}       \label{LemAONBooGZBuYt}
	Soit \( I\) un segment\footnote{définition~\ref{DefLISOooDHLQrl}. Un segment est un intervalle fermé borné.} de \( \eR\) et une fonction continue \( f\colon I\to I\). Si \( K\) est un segment fermé avec \( K\subset f(I)\) alors il existe un segment fermé \( L\subset I\) tel que \( K=f(L)\).
\end{lemma}

\begin{proof}
	Mentionnons immédiatement que \( f\) est continue sur \( I\) qui est compact\footnote{Par le lemme~\ref{LemOACGWxV}.}. Par conséquent tous les nombres dont nous allons parler sont finis parce que \( f\) est bornée par le théorème~\ref{ThoMKKooAbHaro}.

	Soit \( K=\mathopen[ \alpha , \beta \mathclose]\). Si \( \alpha=\beta\) alors le segment \( L=\{ a \}\) convient. Nous supposons donc que \( \alpha\neq \beta\) et nous considérons \( a,b\in I\) tels que \( \alpha=f(a)\) et \( \beta=f(b)\). Puisque \( a\neq b\) nous supposons \( a<b\) (le cas \( a>b\) se traite de façon similaire).

	Nous posons
	\begin{equation}
		A=\{ x\in\mathopen[ a , b \mathclose]\tq f(x)=\alpha \}.
	\end{equation}
	C'est un ensemble borné par \( a\) et \( b\). De plus il est fermé; ce dernier point n'est pas tout à fait évident parce que \( f\) n'est pas définie sur \( \eR\), mais sur \( I\), qui est fermé. Le corolaire~\ref{CorNNPYooMbaYZg} n'est donc pas immédiatement utilisable. Prouvons donc que \( Z=\{ x\in \eR\tq f(x)=\alpha \}\) est fermé. Si \( x_0\) est hors de \( Z\) alors, soit \( x_0\) est dans \( I\), soit il est hors de \( I\). Dans ce second cas, le complémentaire de \( I\) étant ouvert, on a un voisinage de \( x_0\) hors de \( I\), et par conséquent hors de \( Z\). Si au contraire \( x_0\in I\) alors il y a (encore) deux cas : soit \( x_0\in\Int(I)\), soit \( x_0\) est sur le bord de \( I\). Dans le premier cas, le théorème des valeurs intermédiaires\footnote{Théorème~\ref{ThoValInter}.} fonctionne. Pour le second cas, nous supposons \( x_0=\max(I)\) (le cas \( x_0=\min(I)\) est similaire). Le théorème des valeurs intermédiaires dit que sur \( \mathopen[ x_0-\epsilon , x_0 \mathclose]\), \( f\neq \alpha\) et en même temps, sur \( \mathopen] x_0 , x_0+\epsilon \mathclose]\), nous sommes en dehors du domaine. Au final \( \{ f(x)=\alpha \}\) est fermé et \( A\) est alors fermé en tant que intersection de deux fermés.

	L'ensemble \( A\) étant non vide (\( a\in A\)), il possède donc un maximum que nous nommons \( u\) :
	\begin{equation}
		u=\max(A).
	\end{equation}
	Nous posons aussi
	\begin{equation}
		B=\{ x\in \mathopen[ u , b \mathclose]\tq f(x)=\beta \}
	\end{equation}
	qui est encore fermé, borné et non vide. Nous pouvons donc définir
	\begin{equation}
		v=\min(B).
	\end{equation}
	Nous prouvons maintenant que \( f\big( \mathopen[ u , v \mathclose] \big)=\mathopen[ \alpha , \beta \mathclose]\). D'abord \( f\big( \mathopen[ u , v \mathclose] \big)\) est un intervalle compact\footnote{Corolaire \ref{CorImInterInter} et théorème~\ref{ThoImCompCotComp}.} contenant \( f(u)=\alpha\) et \( f(v)=\beta\). Par conséquent \( \mathopen[ \alpha , \beta \mathclose]\subset f\big( \mathopen[ u , v \mathclose] \big)\). Pour l'inclusion inverse supposons \( t\in \mathopen[ u , v \mathclose]\) tel que \( f(t)>\beta\). Vu que \( f(a)=\alpha\) et \( \alpha<\beta\), par le théorème des valeurs intermédiaires, il existe \( t_0\in \mathopen[ a , t \mathclose]\) tel que \( f(t_0)=\beta\). Cela donne \( t_0<v\) et donc contredit la minimalité de \( v\) dans \( B\). Nous en déduisons que \( f\big( \mathopen[ u , v \mathclose] \big)\) ne contient aucun élément plus grand que \( \beta\). Même jeu pour montrer que l'intervalle ne contient aucun élément plus petit que \( \alpha\).

	En définitive, le segment \( L=\mathopen[ u , v \mathclose]\) satisfait toutes les exigences.
\end{proof}

Lorsque \( I_2\subset f(I_1)\) nous notons \( I_1\to I_2\) ou, si une ambiguïté est à craindre, \( I_1\stackrel{f}{\longrightarrow}I_2\). Cette flèche se lit «recouvre».
\begin{lemma}[\cite{PAXrsMn,TMCHooOaTrJL}]      \label{LemSSPXooMkwzjb}
	Soient les segments \( I_0,\ldots, I_{n-1}\) tels que nous ayons le cycle
	\begin{equation}
		I_0\to I_1\to\ldots\to I_{n-1}\to I_0.
	\end{equation}
	Alors \( f^n\) admet un point fixe \( x_0\in I_0\) tel que \( f^k(x_0)\in I_k\) pour tout \( k=0,\ldots, n-1\).
\end{lemma}

\begin{proof}
	Nous prouvons les cas \( n=1\) et \( n=2\) séparément.
	\begin{subproof}
		\spitem[\( n=1\)]
		Nous avons \( I_0\to I_0\), c'est-à-dire que \( I_0\subset f(I_0)\). Si \( I_0=\mathopen[ a , b \mathclose]\) alors nous posons \( a=f(\alpha)\) et \( b=f(\beta)\) pour certains \( \alpha,\beta\in I_0\). Nous posons ensuite \( g(x)=f(x)-x\).

		Dans un premier temps, \( g(\alpha)=a-\alpha\leq 0\) parce que \( a=\min(I_0)\) et \( \alpha\in I_0\). Pour la même raison, \( g(\beta)=b-\beta\geq 0\). Le théorème des valeurs intermédiaires donne alors \( t_0\in \mathopen[ \alpha , \beta \mathclose]\subset I_0\) tel que \( g(t_0)=0\). Nous avons donc \( f(t_0)=t_0\).

		\spitem[\( n=2\)]
		Nous avons \( I_0\to I_1\to I_0\). Puisque \( I_1\subset f(I_0)\), le lemme~\ref{LemAONBooGZBuYt} donne un segment \( J_1\subset I_0\) tel que \( f(J_1)=I_1\). Mézalors
		\begin{equation}
			J_1\subset I_0\subset f(I_1)=f^2(J_1).
		\end{equation}
		Nous avons donc \( J_1\stackrel{f^2}{\longrightarrow}J_1\) et par le cas \( n=1\) traité plus haut, la fonction \( f^2 \) a un point fixe \( x_0\) dans \( J_1\). De plus
		\begin{equation}
			f(x_0)\in f(J_1)=I_1,
		\end{equation}
		le point \( x_0\) est donc bien celui que nous cherchions.

		\spitem[Cas général]
		Nous avons
		\begin{equation}
			I_0\to I_1\to\ldots\to I_{n-1}\to I_0.
		\end{equation}
		Puisque \( I_1\subset f(I_0)\), il existe \( J_1\subset I_0\) tel que \( f(J_1)=I_1\). Mais
		\begin{equation}
			I_2\subset f(I_1)=f^2(J_1),
		\end{equation}
		donc il existe \( J_2\subset J_1\) tel que \( I_2=f^2(J_2)\). En procédant ainsi aussi longtemps qu'il le faut, nous construisons les ensembles \( J_1,\ldots, J_{n-1}\) tels que
		\begin{equation}
			J_{n-1}\subset J_{n-2}\subset\ldots\subset J_1\subset J_0
		\end{equation}
		tels que \( I_k=f^k(J_k)\) pour tout \( k=1,\ldots, n-1\). La dernière de ces inclusions est \( I_{n-1}=f^{n-1}(J_{n-1})\), mais \( I_{n-1}\to I_0\), c'est-à-dire que
		\begin{equation}
			I_0\subset f(I_{n-1})=f^n(J_{n-1}),
		\end{equation}
		et il existe \( J_n\subset J_{n-1}\) tel que \( I_0\subset f^n(J_n)\). Mais comme \( J_n\subset J_0\) nous avons en particulier \( J_n\subset f^n(J_n)\).

		Cela donne un point fixe \( x_0\in J_n\) pour \( f^n\). Par construction, nous avons \( J_n\subset J_{n-1}\subset\ldots\subset J_1\subset J_0\) et donc \( x_0\in J_k\) pour tout \( k\). En  particulier
		\begin{equation}
			f^k(x_0)\in f^k(J_k)=I_k
		\end{equation}
		pour tout \( k\).
	\end{subproof}
\end{proof}

\begin{theorem}[Théorème de Sarkowski\cite{PAXrsMn,TMCHooOaTrJL}]
	Soit \( I\), un segment de \( \eR\) et une application continue \( f\colon I\to I\). Si \( f\) admet un point \( 3\)-périodique, alors \( f\) admet des points \( n\)-périodiques pour tout \( n\geq 1\).
\end{theorem}

\begin{proof}
	Soit \( a\in I\) un point \( 3\)-périodique pour \( f\) et notons \( b=f(a)\), \( c=f(b)\). Les points \( b\) et \( c\) sont également des points \( 3\)-périodiques. Quitte à renommer, nous pouvons supposer que \( a\) est le plus petit des trois. Il reste deux possibilités : \( a<b<c\) et \( a<c<b\). Nous traitons d'abord le premier cas.

	Supposons \( a<b<c\). Nous posons \( I_0=\mathopen[ a , b \mathclose]\) et \( I_1=\mathopen[ b , c \mathclose]\). Nous avons immédiatement \( I_1\subset f(I_0)\) et comme \( f(b)=c\) et \( f(c)=a\), \( f(I_1)\) recouvre \( \mathopen[ a , c \mathclose]\) et donc recouvre en même temps \( I_1\) et \( I_2\). Nous avons donc \( I_0\to I_1\), \( I_1\to I_0\) et \( I_1\to I_1\).
	\begin{subproof}
		\spitem[Un point \( 1\)-périodique]
		Nous avons \( I_1\to I_1\) qui prouve que \( f\) a un point fixe dans \( I_1\). C'est le cas \( n=1\) du lemme~\ref{LemSSPXooMkwzjb}. Voilà un point \( 1\)-périodique.
		\spitem[Un point \( 2\)-périodique]
		Nous avons \( I_0\to I_1\to I_0\). Par conséquent, le lemme~\ref{LemSSPXooMkwzjb} dit que \( f^2\) a un point fixe \( x_0\in I_0\) tel que \( f(x_0)\in I_1\). Montrons que \( f(x_0)\neq x_0\). Pour avoir \( x_0=f(x_0)\), il faudrait \( x_0\in I_0\cap I_1=\{ b \}\). Mais \( b\) est un point \( 3\)-périodique, donc ne vérifiant certainement pas \( f^2(b)=b\). Nous en déduisons que \( f(x_0)\neq x_0\) et donc que \( x_0\) est \( 2\)-périodique.
		\spitem[Un point \( 3\)-périodique]
		On en a par hypothèse.
		\spitem[Un point \( n\)-périodique pour \( n\geq 4\)]
		Nous avons le cyle
		\begin{equation}
			I_0\to \underbrace{I_1\to I_1\to\ldots\to I_1}_{\text{n-1} fois}\to I_0.
		\end{equation}
		Le lemme donne alors un point fixe \( x\in I_0\) pour \( f^n\) tel que \( f^k(x)\in I_1\) pour \( k=1,\ldots, n-1\). Est-ce possible que \( x=b\) ? Non parce que \( f^2(b)=a\in I_0\) alors que \( f^2(x)\in I_1\). Mais \( I_0\cap I_1=\{ b \}\).

		Par conséquent la relation \( f^k(x)\in I_1\) exclut d'avoir \( f^k(x)=x\), et le point \( x\) est bien \( n\)-périodique.
	\end{subproof}

	Passons au cas \( a<c<b\). Alors nous posons \( I_0=\mathopen[ a , c \mathclose]\) et \( I_1=\mathopen[ c , b \mathclose]\). Encore une fois \( f(I_0)\) contient \( a\) et \( b\), donc \( I_0\to I_0\) et \( I_0\to I_1\). Mais en même temps \( f(I_1)\) contient \( a\) et \( c\), donc \( I_1\to I_0\).

	Nous pouvons donc refaire comme dans le premier cas, en inversant les rôles de \( I_0\) et \( I_1\). En particulier nous pouvons considérer le cycle
	\begin{equation}
		I_1\to I_0\to I_0\to\ldots\to I_0\to I_1.
	\end{equation}
\end{proof}
