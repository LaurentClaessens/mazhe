% This is part of Mes notes de mathématique
% Copyright (c) 2011-2022
%   Laurent Claessens
% See the file fdl-1.3.txt for copying conditions.

%+++++++++++++++++++++++++++++++++++++++++++++++++++++++++++++++++++++++++++++++++++++++++++++++++++++++++++++++++++++++++++
\section{Les réels}
%+++++++++++++++++++++++++++++++++++++++++++++++++++++++++++++++++++++++++++++++++++++++++++++++++++++++++++++++++++++++++++

Une construction des réels via les coupures de Dedekind est donnée dans \cite{PaulinTopGmVegN}.

\begin{normaltext}      \label{NormooHRDZooRGGtCd}
	La construction des réels va nécessiter un petit «\wikipedia{fr}{bootstrap}{bootstrap}» au niveau de la topologie. En effet la notion de suite de Cauchy est une notion topologique (définition~\ref{DefZSnlbPc}) alors que la topologie métrique (celle entre autres de \( \eQ\)) ne sera définie que par le théorème~\ref{ThoORdLYUu}. Nous avons donc dû définir en la définition~\ref{DefKCGBooLRNdJf} \emph{ex nihilo} les notions de
	\begin{itemize}
		\item
		      suite de Cauchy
		\item
		      suite convergente
		\item
		      complétude
	\end{itemize}
	Nous allons ensuite construire \( \eR\) comme ensemble de classes d'équivalence de suites de Cauchy dans \( \eQ\). Ce ne sera que plus tard, après avoir défini la notion d'espace métrique que nous allons voir que sur \( \eR\), ces trois notions coïncident avec celles topologiques\footnote{Proposition~\ref{PropooUEEOooLeIImr}.}. Et par conséquent que \( \eR\) sera un espace métrique complet\footnote{Théorème~\ref{THOooUFVJooYJlieh} pour la complétude en tant que corps et théorème~\ref{PROPooTFVOooFoSHPg} pour la complétude en tant que espace métrique.}.
	% position 11144-30436
	% position 13984-18006

	Dans cette optique, il est intéressant de lire ce que dit Wikipédia à propos des suites de Cauchy dans l'article consacré à la construction des nombres réels\cite{BIBooPIGUooHzurMI}.
\end{normaltext}

%---------------------------------------------------------------------------------------------------------------------------
\subsection{L'ensemble}
%---------------------------------------------------------------------------------------------------------------------------

Soit \( \modE\) l'ensemble des suites de Cauchy\footnote{Définition~\ref{DefKCGBooLRNdJf}\ref{ItemVXOZooTYpcYN}} dans \( \eQ\). Soit aussi l'ensemble \( \modE_0\) constituée des suites qui convergent vers zéro\footnote{Nous rappelons qu'à ce niveau nous n'avons pas encore prouvé que toutes les suites de Cauchy convergent.}.

En posant
\begin{equation}
	x+y=(x_n+y_n)
\end{equation}
et
\begin{equation}
	xy=(x_ny_n),
\end{equation}
l'ensemble \( \modE\) devient un anneau\footnote{Définition~\ref{DefHXJUooKoovob}.} commutatif dont le neutre de l'addition est la suite constante \( x_n=0\) et le neutre pour la multiplication est la suite constante \( x_n=1\).

\begin{proposition}     \label{PROPooNUQVooAAkicK}
	La partie \( \modE_0\) est un idéal\footnote{Définition~\ref{DefooQULAooREUIU}.} de l'anneau \( \modE\).
\end{proposition}

\begin{proof}
	Nous savons par la proposition~\ref{PropFFDJooAapQlP}\ref{ItemRKCIooJguHdji} que les suites convergentes sont de Cauchy; par conséquent \( \modE_0\subset\modE\).

	L'ensemble structuré \( (\modE_0,+)\) est un sous-groupe de \( \modE\) par les propriétés de la proposition~\ref{PropFFDJooAapQlP} (il s'agit du fait que la somme de deux suites convergeant vers zéro est une suite convergente vers zéro).

	En ce qui concerne la propriété fondamentale des idéaux, si \( x\in\modE_0\) et \( y\in\modE\) nous devons prouver que \( xy\in \modE_0\). Puisque \( (\modE_0,\cdot)\) est commutatif, cela suffira pour être un idéal bilatère. Vu que \( y\) est une suite de Cauchy, elle est bornée; et étant donné que \( x\to 0\) nous avons alors \( xy\to 0\) (par la proposition~\ref{PropFFDJooAapQlP}\ref{ItemRKCIooJguHdjiii}).
\end{proof}

\begin{theoremDef}[L'anneau des réels\cite{RWWJooJdjxEK}]       \label{DefooFKYKooOngSCB}
	Sur l'ensemble quotient \( \modE/\modE_0\), les opérations
	\begin{enumerate}
		\item
		      \( \bar u+\bar v=\overline{ u+v }\)
		\item
		      \( \bar u\cdot \bar v=\overline{ uv }\)
	\end{enumerate}
	sont bien définies et donnent à \( \modE/\modE_0\) une structure de corps commutatif appelé \defe{corps des réels}{réel} et noté \( \eR\)\nomenclature[Y]{\( \eR\)}{l'ensemble des réels}
\end{theoremDef}
\index{construction!des réels}

\begin{proof}
	Nous divisons la preuve en plusieurs parties.
	\begin{subproof}
		\item[Les opérations sont bien définies]
		La partie \( \modE_0\) est un idéal par la proposition \ref{PROPooNUQVooAAkicK}. Le quotient est donc bien défini et est un anneau par la proposition \ref{PROPooGXMRooTcUGbi}\ref{ITEMooYBEGooTlHgNz}.
		\item[Caractérisation des classes]
		Soit \( q\in \eQ\) et une suite \( x\) convergente vers \( q\). Cette suite est de Cauchy comme toute suite convergente. Montrons que
		\begin{equation}
			\bar x=\{ \text{suites qui convergent vers } q \}.
		\end{equation}
		Si \( y\in\bar x\) alors \( y=x+h\) avec \( h\in \modE_0\), et comme \( h_n\to 0\), on a \( y_n\to q\). Réciproquement, si \( y_n\to q\) alors pour chaque \( n\) nous avons
		\begin{equation}
			y_n=x_n+(y_n-x_n),
		\end{equation}
		mais \( y_n-x_n\to 0\). Donc la suite \( y-x\in\modE_0\) ce qui signifie que \( y\in\bar x\).
		\item[Neutre et unité]
		Il est vite vérifié que \( \bar 0\), la classe de la suite constante égale à zéro est neutre pour l'addition. De même, \( \bar 1\), est un neutre pour la multiplication.
		\item[Corps]
		Nous devons prouver que tout élément non nul est inversible. C'est-à-dire que si \( x\in\modE\) ne converge pas vers zéro\footnote{\( x\in\modE\) peut soit ne pas converger du tout, soit converger vers autre chose que zéro.} alors il existe \( y\in \modE\) tel que \( xy\in\bar 1\).

		Nous savons par la proposition~\ref{PropFFDJooAapQlP}\ref{ItemRKCIooJguHdjvi} que \( x\) étant une suite de Cauchy dans \( \eQ\), il existe \( n_0\in \eN\) tel que \( \left( \frac{1}{ x_n } \right)_{n\geq n_0}\) est une suite de Cauchy. Nous posons alors
		\begin{equation}
			y_n=\begin{cases}
				0               & \text{si } n\leq n_0 \\
				\frac{1}{ x_n } & \text{si } n>n_0.
			\end{cases}
		\end{equation}
		Nous avons alors
		\begin{equation}
			(xy)_n=\begin{cases}
				0 & \text{si } n\leq n_0 \\
				1 & \text{si } n>n_0
			\end{cases}
		\end{equation}
		et donc \( xy\in\bar 1\).
	\end{subproof}
\end{proof}

\begin{normaltext}[Quelques notations entre \( \eQ\) et \( \eR\)]      \label{NORMooWBYNooBQaPPk}
	Si \( k\mapsto x_k\) est une suite, nous notons \( (x_k)\) avec des parenthèses la suite elle-même. Le \( k\) dans \( (x_k)\) est un indice muet, et dans les cas où il peut y avoir une ambiguïté, nous pouvons noter \( (x_k)_{k\in \eN}\). Cette dernière notation est plus lourde, mais plus exacte.

	Le mieux est d'écrire simplement \( x\) la suite, mais alors il faut être prudent et ne pas noter \( x\) la limite. Nous éviterons donc d'écrire \( x_k\to x\).

	Si \( (x_k)\) est une suite de Cauchy dans \( \eQ\), nous notons \( \bar x\) l'élément de \( \eR\) qui lui correspond. En fait \( \bar x=(x_k)\) : \( \bar x\) est la suite-elle même, mais pour nous souvenir de l'origine nous allons adopter cette notation.

	D'autre part nous définissons
	\begin{equation}
		\begin{aligned}
			\varphi\colon \eQ & \to \eR                           \\
			q                 & \mapsto \overline{ [k\mapsto q]},
		\end{aligned}
	\end{equation}
	c'est-à-dire que \( \varphi(q)\) est la classe de la suite constante \( x_k=q\).
\end{normaltext}

\begin{proposition}     \label{PropooEPFCooMtDOfP}
	Soit l'application
	\begin{equation}
		\begin{aligned}
			\varphi\colon \eQ & \to \eR          \\
			q                 & \mapsto \bar q .
		\end{aligned}
	\end{equation}
	où par \( \bar q\) nous entendons la classe de la suite constante égale à \( q\) (qui est de Cauchy).
	\begin{enumerate}
		\item
		      C'est un homomorphisme de corps injectif.
		\item
		      \( \Image(\varphi)\) est un sous-corps de \( \eR\)
		\item
		      \( \varphi\colon \eQ\to \Image(\varphi)\) est un isomorphisme de corps.
	\end{enumerate}
\end{proposition}

\begin{proof}
	Le fait que ce soit un homomorphisme est simplement
	\begin{itemize}
		\item \( \varphi(q+q')=\overline{ q+q' }=\bar q+\overline{ q' }\)
		\item \( \varphi(qq')=\overline{ qq' }=\overline{ q }\overline{ q' }\).
	\end{itemize}
	En ce qui concerne l'injectivité, si \( q\) est tel que \( \varphi(q)=\bar 0=\modE_0\), c'est que
	\begin{equation}
		\varphi(q)=\{ \text{suites de Cauchy qui convergent vers zéro} \}
	\end{equation}
	Mais nous savons aussi que\footnote{Voir dans la démonstration du théorème~\ref{DefooFKYKooOngSCB}.}
	\begin{equation}
		\varphi(q)=\bar q=\{ \text{suites de Cauchy qui convergent vers } q \}
	\end{equation}
	Nous en déduisons que \( q=0\).
\end{proof}
Lorsque dans la suite nous parlerons d'un élément de \( \eQ\) comme étant un réel, nous aurons en tête l'image de cet élément par \( \varphi\).

\begin{lemma}       \label{LEMooYLQBooFistHs}
	Soient \( q,l\in \eQ\) tels que \( \bar q=\bar l\). Alors \( q=l\) dans \( \eQ\).
\end{lemma}

\begin{proof}
	La suite constante \( x_n=q\) est un représentant de \( \bar q\), et la suite constante \( y_n=l\) est représentant de \( \bar l\). Dire que \( \bar l=\bar q\) signifie qu'il existe une suite \( z\in \modE_0\) tel que
	\begin{equation}
		x=y+z.
	\end{equation}
	Pour tout \( n\) nous avons donc \( x_n=y_n+z_n\), ou encore
	\begin{equation}
		z_n=q-l
	\end{equation}
	pour tout \( n\). Puisque \( z\) est une suite constante, elle ne peut appartenir à \( \modE_0\) que si elle est la suite constante nulle, c'est-à-dire si \( q=l\).
\end{proof}

%---------------------------------------------------------------------------------------------------------------------------
\subsection{Relation d'ordre}
%---------------------------------------------------------------------------------------------------------------------------

\begin{definition}
	Nous définissons les parties \( \modE^+\) et \( \modE^-\) de \( \modE\) par
	\begin{enumerate}
		\item
		      \( x\in  \modE^+\) si et seulement si pour tout \( \epsilon>0\) (\( \epsilon\) est dans \( \eQ\)), il existe \( N_{\epsilon}\) tel que \( n>N_{\epsilon}\) implique \( x_n>-\epsilon\).
		\item
		      \( x\in  \modE^-\) si et seulement si pour tout \( \epsilon>0\), il existe \( N_{\epsilon}\) tel que \( n>N_{\epsilon}\) implique \( x_n<\epsilon\).
	\end{enumerate}
	Nous notons aussi \( \modE^{++}=\modE^+\setminus\modE_0\).
\end{definition}


Dans le lemme suivant, le point \ref{ITEMooRCIZooHUIymE} peut sembler perturbant. Il s'agit de dire que si \( x\) est la classe de la suite constante \( 0\), alors il est le neutre pour l'addition dans \( \eR\).
\begin{lemma}[À propos du zéro]       \label{LEMooJOYQooCDlhHW}
	Nous avons
	\begin{enumerate}
		\item   \label{ITEMooKRBYooZGhhch}
		      \( \modE^+\cap\modE^-=\{ \bar 0 \}\).
		\item       \label{ITEMooRCIZooHUIymE}
		      Si \( x=\bar 0\) alors \( x=0\).
	\end{enumerate}
\end{lemma}

\begin{lemma}       \label{LEMooSWYXooMKMLYI}
	Les parties \( \modE^+\) et \( \modE^-\) partitionnent \( \modE\) de la façon suivante :
	\begin{enumerate}
		\item
		      \( \modE^+\cap\modE^-=\modE_0\)
		\item       \label{ITEMooZRVXooANHspZ}
		      \( \modE^+\cup\modE^-=\modE\)
	\end{enumerate}
\end{lemma}

\begin{proof}
	On prouve d'abord que \( \modE^+\cap\modE^-\subset\modE_0\), l'inclusion inverse est évidente. Soit \( \epsilon>0\) et \( x\in \modE^+\cap\modE^-\). Il existe un \( N\in \eN\) tel que \( x_n>-\epsilon\) et \( x_n<\epsilon\) pour tout \( n\geq N\). Par conséquent, \( | x_n |\leq \epsilon\) pour tout \( n\geq N\) et la suite \( x\) converge vers zéro, c'est-à-dire \( x\in\modE_0\).

	Pour prouver le second point, soit \( x\in \modE\setminus\modE^-\), et prouvons que \( x\in\modE^+\). La condition \( x\notin \modE^-\) donne qu'il existe un \( \alpha>0\) (dans \( \eQ\)) tel que pour tout \( n\), il existe \( p>n\) avec \( x_p>\alpha\). Mais \( x\) est une suite de Cauchy, donc nous avons un \( n_0\) tel que si \( n,p\geq n_0\) alors \( | x_n-x_p |\leq \frac{ \alpha }{2}\). En particulier, si \( n\geq n_0\), et si \( p>n\) est tel que \( x_p>\alpha\), on obtient
	\begin{equation}
		x_n>\frac{ \alpha }{2}>0
	\end{equation}
	Par conséquent \( x\in\modE^+\) parce que \( x\in\modE\) et les \( x_n \) sont tous positifs à partir d'un certain rang.
\end{proof}

\begin{lemma}[\cite{MonCerveau}]        \label{LEMooXNWSooHbNcAV}
	Si \( x\in \modE^{++}\), alors il existe \( N\) tel que \( x_n>0\) pour tout \( n>N\).
\end{lemma}

\begin{proof}
	La suite \( x\) ne tend pas vers zéro. Donc il existe \( \delta>0\) tel que pour tout \( N>0\) il existe \( n>N\) vérifiant \( x_n>\delta\).

	Mais la suite \( x\) est également de Cauchy. Écrivons cette condition pour \( \delta/2\). Il existe \( N_2>0\) tel que \( p,q>N_2\) implique \( | x_p-x_q |<\delta/2\).

	Nous fixons \( n>N_2\) tel que \( x_n>\delta\). Alors pour tout \( p>N_2\) nous avons aussi
	\begin{equation}
		| x_p-x_n |<\frac{ \delta }{2}.
	\end{equation}
	Cela implique que \( x_p>\delta/2>0\) pour tout \( p>N_2\).
\end{proof}

La proposition suivante est une version plus précise du lemme \ref{LEMooXNWSooHbNcAV}.
\begin{proposition}[\cite{BIBooZFPUooIiywbk}]
	Soit \( x\in \mE^{++}\). Il existe \( r\in \eQ^+\setminus\{ 0 \}\) et \( N\in \eN\) tel que \( x_n\geq r\) pour tout \( n\geq N\).
\end{proposition}

\begin{proof}
	Fixons \( \epsilon\in \eQ^+\setminus\{ 0 \}\), et procédons par l'absurde. Du coup nous savons trois choses sur la suite \( (x_n)\).
	\begin{enumerate}
		\item       \label{ITEMooTNZDooHKZdBe}
		      Hypothèse absurde : pour tout \( q\in \eQ^+\setminus\{ 0 \}\) et pour tout \( N\in \eN\), il existe \( p\geq N\) vérifiant \( x_p<q\).
		\item
		      Vu que \( x\in \mE^{++}\subset \mE^+\), il existe \( P_1\in \eN\) tel que \( x_n>-\epsilon\) pour tout \( n\geq P_1\).
		\item
		      La suite \( x\) est de Cauchy. Donc il existe \( P_2\in \eN\) tel que si \( m,n\geq P_2\), nous avons \( | x_m-x_n |<\epsilon/2\).
	\end{enumerate}
	Nous considérons \( P\geq \max\{ P_1, P_2 \}\) et nous prenons \( q=\epsilon/2\), \( N=P\) dans la propriété \ref{ITEMooTNZDooHKZdBe}. Il existe donc \( p>P\) tel que \( x_p<\epsilon/2\).

	Prenons aussi \( n>p\) et écrivons les deux autres propriétés :
	\begin{enumerate}
		\item \( x_n>-\epsilon\) parce que \( n>p>P>P_1\).
		\item \( | x_n-x_p |<\epsilon/2\) parce que \( n,p>P>P_2\).
	\end{enumerate}
	Du coup nous avons
	\begin{equation}
		x_n<x_p+\frac{ \epsilon }{2}<\epsilon,
	\end{equation}
	et donc \( -\epsilon<x_n<\epsilon\).

	Au final, nous avons prouvé que pour tout \( \epsilon\in \eQ^+\setminus\{ 0 \}\), il existe \( P\) tel que \( n>P\) implique \( | x_n |<\epsilon\). Cela signifie que
	\begin{equation}
		x_n\stackrel{\eQ}{\longrightarrow}0,
	\end{equation}
	c'est-à-dire \( x\in \mE_0\), ce qui est contraire à l'hypothèse.
\end{proof}


\begin{lemma}[\cite{RWWJooJdjxEK}]      \label{LEMooRKSXooFsIohe}
	Quelques propriétés du partitionnement.
	\begin{enumerate}
		\item       \label{ITEMooRQVKooCnwWOY}
		      \( x\in\modE^-\) si et seulement si \( (-x)\in\modE^+\)
		\item       \label{ITEMooJUPOooOBubqA}
		      \( x\in\modE^+\) et \( y\in\modE^+\) implique \( x+y\in\modE^+\)
		\item       \label{ITEMooDQLJooPViuVC}
		      \( x\in\modE^+\) et \( y\in\modE^+\) implique \( xy\in\modE^+\)
		\item
		      Si \( x,y\in\modE\) sont tels que \( x-y\in\modE_0\) alors soit \( x,y\in\modE^+\) soit \( x,y\in\modE^-\).
	\end{enumerate}
\end{lemma}

\begin{proof}
	Point par point.
	\begin{enumerate}
		\item
		      Définition de \( \modE^+\) et \( \modE^-\).
		\item
		      Pour \( n\geq N_{\epsilon/2}\) nous avons \( x_n>-\epsilon/2\) et \( y_n>-\epsilon/2\). Donc \( x_n+y_n>-\epsilon\).
		\item
		      Si \( x\) ou \( y\) est dans \( \modE_0\) alors \( xy\in\modE_0\) et c'est bon. Si par contre \( x,y\in\modE^{++}\) alors le lemme \ref{LEMooXNWSooHbNcAV} nous indique que pour \( n\) suffisamment grand, \( x_n>0\) et \( y_n>0\). Et dans ce cas, \( (xy)_n> 0\), c'est-à-dire \( xy\in\modE^+\).
		\item
		      Supposons que \( x-y\in\modE_0\) avec \( x\in\modE^+\) et prouvons qu'alors \( y\in\modE^+\). Soit donc \( \epsilon>0\); il existe \( n_1\) tel que \( x_n>-\frac{ \epsilon }{2}\) dès que \( n\geq n_1\). Mais \( x-y\in\modE_0\), donc il existe \( n_2\) tel que \( | x_n-y_n |<\frac{ \epsilon }{2}\) dès que \( n\geq n_2\). En prenant \( n\) plus grand que \( n_1\) et \( n_2\), nous avons en même temps
		      \begin{subequations}
			      \begin{numcases}{}
				      x_n>-\frac{ \epsilon }{2}\\
				      | x_n-y_n |<\frac{ \epsilon }{2}.
			      \end{numcases}
		      \end{subequations}
		      Cela implique que \( y_n>-\epsilon\) et donc que \( y\in\modE^+\).

		      Nous pouvons de même prouver que si \( x\in\modE^-\) alors \( y\in\modE^-\).
	\end{enumerate}
\end{proof}

\begin{definition}[Positivité dans \( \eR\)]        \label{DefooLMQIooTgzZXd}
	Vocabulaire et notations.
	\begin{enumerate}
		\item
		      Nous notons \( \eR=\modE/\modE_0\).
		\item
		      Nous notons \( \eR^+=\bar\modE^+\).\nomenclature[Y]{\( \eR^+\)}{les réels positifs ou nuls}
		\item
		      Nous notons \( \eR^-=\bar\modE^-\).
		\item
		      Un élément de \( \eR\) est \defe{positif}{positif} si il est la classe d'une suite de Cauchy appartenant à \( \modE^+\).
		\item
		      Un élément de \( \eR\) est \defe{négatif}{négatif} si il est la classe d'une suite de Cauchy appartenant à \( \modE^-\).
		\item
		      Lorsque nous parlons de nombres réels, le symbole «\( 0\)» signifie \( \modE_0\) ou plus précisément la classe d'un élément de \( \modE_0\) modulo \( \modE_0\).
	\end{enumerate}
\end{definition}

\begin{normaltext}\label{REMooOCXLooKQrDoq}
	Avec les conventions de la définition~\ref{DefooLMQIooTgzZXd}, et en anticipant sur nos connaissances à propos des réels,
	\begin{enumerate}
		\item
		      zéro est positif et négatif.
		\item
		      L'intersection entre \( \eR^+\) et \( \eR^-\) est le singleton \( \{ 0 \}\).
		\item
		      L'ensemble des nombres \emph{strictement} positifs est noté \( (\eR^+)^*\) ou \( \eR^+\setminus\{ 0 \}\).
		\item
		      Le mot «positif» signifie «positif ou nul»; le mot «négatif» signifie «négatif ou nul». Ce sont des conventions qui sont également celles de Wikipédia\cite{ooSBSSooTlnuKi}.
	\end{enumerate}

\end{normaltext}

\begin{definition}[Ordre sur \( \eR\)]      \label{DEFooBXHJooOEYPRI}
	Si \( a,b\in \eR\) nous notons \( a\leq b\) si et seulement si \( b-a\in\overline{ \modE^+ }\).
\end{definition}

\begin{proposition} \label{PROPooYMJVooNAsXae}
	Le couple \( (\eR,\leq)\) est un corps totalement ordonné\footnote{Corps totalement ordonné, définition \ref{DefKCGBooLRNdJf}.}
\end{proposition}

\begin{proof}
	Il s'agit de vérifier, dans l'ordre, les définitions \ref{DefooFLYOooRaGYRk}, \ref{DEFooVGYQooUhUZGr} et \ref{DefKCGBooLRNdJf}\ref{ITEMooOOOVooJWwIQr}. Pour la suite nous considérons \( x,y,z\in \eR\) et des suites de Cauchy \( a,b,c\) telles que \( x=\bar a\), \( y=\bar b\) et \( z=\bar c\).
	\begin{subproof}
		\item[Réflexivité]
		Pour savoir si \( x\geq x\), nous devons nous demander si \( x-x\in\overline{ \modE^+ }\). Nous avons \( x-x=\bar a-\bar a=\bar 0=0\).
		\item[antisymétrie]
		Nous supposons que \( x\geq y\) et \( y\geq x\). Du côté des suites de Cauchy, cela signifie que \( a-b\in\modE^+\) et \( b-a\in \modE^+\). Le lemme \ref{LEMooRKSXooFsIohe}\ref{ITEMooRQVKooCnwWOY} nous indique alors que \( a-b=-(b-a)\in \modE^-\). Donc
		\begin{equation}
			a-b\in\modE^+\cap\modE^-=\{ \bar 0 \}.
		\end{equation}
		Donc le réel \( x-y\) est la classe de la suite constante \( 0\). Le lemme \ref{LEMooJOYQooCDlhHW}\ref{ITEMooRCIZooHUIymE} dit alors que \( x-y=0\) ou encore que \( x=y\).
		\item[transitivité]
		Nous supposons que \( x\leq y\) et \( y\leq z\). Alors nous avons
		\begin{subequations}
			\begin{align}
				c-a & =c-b+b-a            \\
				    & =(c-b)+(b-a)        \\
				    & \in \modE^++\modE^+ \\
				    & \subset \modE^+
			\end{align}
		\end{subequations}
		où nous avons utilisé le lemme \ref{LEMooRKSXooFsIohe}\ref{ITEMooJUPOooOBubqA}. Puisque \( c-a\in\modE^+\), nous avons \( z-x=\overline{ c-a }\geq 0\).
		\item[Ordre total]
		Nous devons prouver que pour \( x,y\in \eR\) nous avons toujours \( x\leq y\) ou \( y\leq x\). Supposons que nous n'ayons pas \( x\leq y\), c'est-à-dire \( \overline{ b-a }\notin \modE^+\). Vu le lemme \ref{LEMooSWYXooMKMLYI}\ref{ITEMooZRVXooANHspZ} nous avons \( \overline{ b-a }\in \modE^-\), ce qui donne, par le lemme \ref{LEMooRKSXooFsIohe}\ref{ITEMooRQVKooCnwWOY} que \( \overline{ a-b }\in \modE^+\), c'est-à-dire \( y\leq x\).
		\item[Corps ordonné]
		Enfin nous devons vérifier les deux conditions de la définition \ref{DefKCGBooLRNdJf}\ref{ITEMooOOOVooJWwIQr}.

		Pour la première condition, nous supposons \( x\leq y\), c'est-à-dire \( b-a\in \modE^+\). Nous avons donc
		\begin{equation}
			(b+c)-(a+c)=b+x-a-c=b-a\in \modE^+,
		\end{equation}
		donc \( \overline{ a+c }\leq \overline{ b+c }\), c'est-à-dire \( x+z\leq y+z\).

		Pour la seconde condition, c'est le lemme \ref{LEMooRKSXooFsIohe}\ref{ITEMooDQLJooPViuVC}.
	\end{subproof}
\end{proof}


\begin{definition}
	Puisque \( \eR\) est un corps totalement ordonné (proposition \ref{PROPooYMJVooNAsXae}), si \( x\in \eR\), nous définissons \( | x |\) conformément à \ref{DefKCGBooLRNdJf}\ref{ItemooWUGSooRSRvYC}.
\end{definition}

\begin{lemma}       \label{LEMooTJAXooKEqPCG}
	L'application
	\begin{equation}
		\begin{aligned}
			\varphi\colon \eQ & \to \eR        \\
			q                 & \mapsto \bar q
		\end{aligned}
	\end{equation}
	dont nous avons déjà parlé dans la proposition~\ref{PropooEPFCooMtDOfP} est strictement croissante.
\end{lemma}

\begin{proof}
	Nous supposons \( q< l\) dans \( \eQ\). Nous devons montrer que \( \bar q\leq \bar l\) dans \( \eR\), c'est-à-dire que \( \bar q\leq \bar l\) et \( \bar q\neq \bar l\).

	Considérons la suite constante \( x_n=l-q\in \eQ\). Pour tout \( \epsilon>0\) dans \( \eQ\) nous avons
	\begin{equation}
		x_n=l-q>0>-\epsilon,
	\end{equation}
	et donc \( x_n\in\modE^+\). Donc \( \overline{ l-q }\geq 0\). Cela signifie \( \bar q\leq \bar l\).

	D'autre part le lemme \ref{LEMooYLQBooFistHs} dit que \( \bar q=\bar l\) uniquement si \( q=l\), ce qui est exclu parce que \( q<l\). Donc \( \bar q\neq\bar l\).
\end{proof}

Voici une version dans \( \eR\) du lemme \ref{LEMooSVDDooWsyxNP}.
\begin{lemma}       \label{LEMooKAXFooIPyzJC}
	Soient \( a>0\) et \( b>1\) dans \( \eR\). Nous avons
	\begin{equation}
		ab>a.
	\end{equation}
\end{lemma}

\begin{remark}
	Comme déjà mentionné plus haut, à chaque fois que nous parlerons d'un élément de \( \eQ\) comme étant un élément de \( \eR\), nous considérons la classe de la suite constante.
\end{remark}

\begin{lemma}       \label{LemooYNOVooOwoRwD}
	Si \( x,y,z\in \eR\) avec \( x>0\) sont tels que \( z>y/x\) alors \( zx>y\).
\end{lemma}

\begin{proof}
	Nous savons que
	\begin{equation}
		z-\frac{ y }{ x }\in \modE^+\setminus\{ 0 \}=\modE^{++}.
	\end{equation}
	Puisque \( x\in\modE^{++}\), multiplier par \( x\) fait rester dans \( \modE^{++}\) :
	\begin{equation}
		zx-x\frac{ y }{ x }\in \modE^{++}.
	\end{equation}
	Un représentant de \( x\frac{ y }{ x }\) est la suite \( n\mapsto x_n\frac{ y_n }{ x_n }=y_n\). Donc \( x\frac{ y }{ x }=y\). Cela signifie que \( zx-y\in\modE^{++}\) et donc que \( zx>y\).
\end{proof}

\begin{lemma}       \label{LemooMWOUooVWgaEi}
	Pour tout \( a\in \eR\), il existe \( p\in \eN\) tel que \( p>a\).
\end{lemma}

\begin{proof}
	Nous allons donner deux preuves différentes de ce lemme.
	\begin{subproof}
		\item[Première façon]

		L'élément \( a\) de \( \eR\) admet un représentant \( (a_n)\) qui est une suite de Cauchy dans \( \eQ\). C'est donc une suite bornée, c'est-à-dire qu'il existe \( m,q\in \eN\) tels que \( | a_n |\leq m/q\) pour tout \( n\) (proposition~\ref{PropFFDJooAapQlP}\ref{ItemRKCIooJguHdjii}). Soit \( M\) un naturel strictement plus grand que \( m/q\)\footnote{Lemme~\ref{LEMooEBTIooGMoHsj}.}.

		La suite de Cauchy \( (M-a_n)_{n\in \eN}\) est constituée de rationnels positifs et est donc dans \( \modE^+\). La classe de \( M-a\) est donc un réel positif\footnote{Et nous allons d'ailleurs arrêter de toujours préciser «la classe de» lorsque ce n'est pas nécessaire.}. Par définition de la relation d'ordre, \( M\geq a\).
		\item[Seconde façon]

		La suite \( (a_n)\) est majorée par \( \frac{ m }{ q }\), donc on a dans \( \eQ\) et pour tout \( n\) :
		\begin{equation}
			a_n\leq \frac{ m }{ q }=M\leq qM.
		\end{equation}
		L'application \( \varphi\colon \eQ\to \eR\) est croissante, donc
		\begin{equation}
			\varphi\big( (a_n) \big)\leq \varphi(qM).
		\end{equation}
	\end{subproof}
\end{proof}

En corolaire, nous avons
\begin{lemma}      \label{LEMooMWOUooVWgbFi}
	Pour tout \( x\in \eR\), il existe \( q\in \eZ\) tel que \( q < x\).
\end{lemma}
\begin{proof}
	Utilisation du lemme précédent avec \( a = -x \): on prend \( q = -p \).
\end{proof}

\begin{theorem}[\cite{RWWJooJdjxEK}]        \label{ThoooKJTTooCaxEny}
	Le corps \( \eR\) est archimédien\footnote{Définition~\ref{DEFooLCWLooYrToFv}.}.
\end{theorem}

\begin{proof}
	La proposition \ref{PROPooYMJVooNAsXae} dit que \( \eR\) est totalement ordonné. Soient \( x,y\in \eR\) avec \( x>0\); posons \( a=\frac{ y }{ x }\). Le lemme~\ref{LemooMWOUooVWgaEi} nous donne un \( p \in \eN\) tel que \(p > a\). Nous concluons alors avec le lemme~\ref{LemooYNOVooOwoRwD} :
	\begin{equation}
		px>ax=\frac{ y }{ x }x=y.
	\end{equation}
\end{proof}

Le lemme suivant n'est pas loin de dire que \( \eQ\) est dense dans \( \eR\), à part que nous n'avons pas encore donné de topologie sur \( \eR\).
\begin{lemma}       \label{LemooHLHTooTyCZYL}
    À propos de rationnels entre des réels.
    \begin{enumerate}
        \item       \label{ITEMooGVTMooQsoTCi}
	Si \( x,y\in \eR\) sont tels que \( x<y\), alors il existe \( s\in \eQ\) tel que \( x<s<y\).
\item       \label{ITEMooCVDSooAjimCL}
    Si \( \epsilon>0\) dans \( \eR\), il existe \( n\in \eN\) tel que  \( \frac{1}{ n }<\epsilon\).
    \end{enumerate}
\end{lemma}

\begin{proof}
	Nous avons par hypothèse que \( y-x>0\) et donc le fait que \( \eR\) soit archimédien (théorème~\ref{ThoooKJTTooCaxEny}) nous donne \( q\in \eN\) tel que \( q(y-x)>1\). Soit
	\begin{equation}
		E=\{ n\in \eZ\tq \frac{ n }{ q }\leq x \}.
	\end{equation}
	Cet ensemble n'est pas vide à cause du lemme~\ref{LEMooMWOUooVWgbFi}; de plus, comme \( |x|q \leq n_0\) pour un certain \( n_0 \) (à cause du lemme~\ref{LemooMWOUooVWgaEi}), l'ensemble \( E\) est majoré par \( n_0\). Donc \( E\) possède un plus grand élément\footnote{Lemme~\ref{LEMooMYEIooNFwNVI}.} \( p\) qui vérifie
	\begin{equation}
		\frac{ p }{ q }\leq x<\frac{ p+1 }{ q }.
	\end{equation}
	De plus \( (p+1)/q<y\) parce que
	\begin{equation}
		\frac{ p+1 }{ q }=\frac{ p }{ q }+\frac{1}{ q }\leq x+\frac{1}{ q }<x+y-x=y
	\end{equation}
	où nous avons utilisé l'inégalité stricte \( y-x>\frac{1}{ q }\).

	Nous avons donc
	\begin{equation}
		x<\frac{ p+1 }{ q }<y,
	\end{equation}
    et le nombre \( (p+1)/q\) convient comme \( s\). Le point \ref{ITEMooGVTMooQsoTCi} est prouvé.

    Pour le point \ref{ITEMooCVDSooAjimCL}, par le point \ref{ITEMooGVTMooQsoTCi} nous considérons \( s\in \eQ\) tel que \( 0<s<\epsilon\). Si \( s=p/q\) avec \( p,q\in \eN\) nous avons
    \begin{equation}
        0<\frac{1}{ q }\leq \frac{ p }{ q }<\epsilon.
    \end{equation}
\end{proof}

\begin{remark}      \label{REMooXOIOooHjwMvA}
	Le lemme~\ref{LemooHLHTooTyCZYL} a également pour conséquence que des ensembles comme \( \mathopen[ -1 , 1 \mathclose]\) ne sont pas bien ordonnés (définition~\ref{DEFooVGYQooUhUZGr}). En effet la partie \( \mathopen] 0 , 1 \mathclose[\) ne possède pas de minimum parce que si \( x\in \mathopen] 0 , 1 \mathclose[\) alors \( 0<x\) et il existe \( s\in \eQ\) (a fortiori \( s\in \eR\)) tel que \( 0<s<x\), c'est-à-dire que \( x\) n'est pas un minimum de \( \mathopen] 0 , 1 \mathclose[\).
\end{remark}

Tant que nous y sommes dans les encadrements de réels\dots
\begin{normaltext}
	Soit \(q_0 \in \eQ \) tel que \( 0 \leq q_0 < 1 \). On définit alors \( d_1 \in \{0, 1\} \) comme valant \( 1 \) si \( 2 q_0 \geq 1 \) et \(0 \) sinon. Puis on pose \( q_1 = 2 q_0 - d_1 \).

	Poursuivant de la sorte, on crée une suite \( (d_n)_{n\geq 1} \): c'est le \defe{développement dyadique}{dyadique!développement} de \( q_0 \).
\end{normaltext}

\begin{lemma}[\cite{MonCerveau}]        \label{LEMooRSLIooVrZMxM}
	Soit \( q,\ q' \) deux rationnels tels que \( 0 \leq q < q' < 1 \). Il existe deux entiers naturels \( a \) et \( N \) tels que \( q < \frac a {2^N} < q' \).
\end{lemma}
\begin{proof}
	On crée les développements dyadiques de \( q \) et \( q' \), que l'on note respectivement \( (d_n)_{n\geq 1} \) et \( (d'_n)_{n\geq 1} \). Notons
	\begin{equation}
		E = \{ n \in \eN \tq d_n \neq d'_n \}.
	\end{equation}
	Comme \( q \neq q' \), les développements dyadiques sont différents\quext{À vérifier tout de même\dots}, l'ensemble \(E\) est non-vide, et il admet un plus petit élément \(N \). Or, \( q < q' \), et donc nécessairement \( d_N < d'_N \). On construit alors \( a = \sum_{i=1}^{N} d_i 2^i \).
\end{proof}

\begin{corollary}\label{CorDensiteDyadiques}
	Pour tous réels \(x,\ y\) tels que \( 0 \leq x < y \leq 1 \), il existe un nombre de la forme \( d = a / 2^n \), avec \( n \in \eN \) et \( a \in \eN,\ a \leq 2^n\), tel que \( x < d < y \).
\end{corollary}

\begin{lemma}[\cite{MonCerveau}]        \label{LEMooEGYLooCGrDrl}
	Soient des réels \( a,b,x,y\) tels que
	\begin{equation}
		a\leq x\leq b
	\end{equation}
	et
	\begin{equation}
		a\leq y\leq b,
	\end{equation}
	alors \( | x-y |\leq | b-a |\).
\end{lemma}

\begin{lemma}       \label{LEMooTPLUooXiCZHJ}
	Soient deux réels \( a,b\) tels que
	\begin{enumerate}
		\item
		      \( a\geq 0\),
		\item
		      \( b\geq 0\)
		\item
		      \( a+b=0\).
	\end{enumerate}
	Alors \( a=0\) et \( b=0\).
\end{lemma}

\begin{lemma}       \label{LEMooNLGSooSGdvAo}
	Si \( a\in \eR\), alors \( a^2\geq 0\) et \( a^2=0\) si et seulement si \( a=0\).
\end{lemma}

\begin{lemma}       \label{LEMooVXAXooNhxtSU}
	Soit un réel strictement positif \( a\). Si \( b>1\), alors \( ab>a\).
\end{lemma}


%---------------------------------------------------------------------------------------------------------------------------
\subsection{Complétude}
%---------------------------------------------------------------------------------------------------------------------------

Le théorème \ref{ThoKHTQJXZ} donne une complétion de tout espace métrique en un espace complet. Il serait tentant de l'utiliser ici pour définir \( \eR\) à partir de \( \eQ\). Cette méthode ne fonctionne cependant pas parce que la démonstration de \ref{ThoKHTQJXZ} utilise le fait que \( \eR\) est complet.

\begin{lemma}       \label{LEMooXCVRooOSZYWv}
	Nous avons
	\begin{equation}
		| \varphi(q) |= \varphi(| q |)
	\end{equation}
	pour tout \( q\in \eQ\).
\end{lemma}

\begin{proof}
	Soit \( q\in \eQ\). Si \( q\geq 0\) alors nous avons d'une part \( | q |=q\) dans \( \eQ\), et d'autre part \( \varphi(q)\geq 0\) dans \( \eR\). Donc au final
	\begin{equation}
		| \varphi(q) |=\varphi(q)=\varphi(| q |).
	\end{equation}

	Supposons au contraire que \( q<0\). Alors \( | q |=-q\) dans \( \eQ\), mais aussi \( \varphi(q)\leq 0\). Donc
	\begin{equation}
		| \varphi(q) |=-\varphi(q)=\varphi(-q)=\varphi(| q |).
	\end{equation}
\end{proof}

\begin{lemma}[\cite{RWWJooJdjxEK, MonCerveau}]      \label{LemooRTGFooYVstwS}
	Toute suite de Cauchy dans \( \eQ\) converge dans \( \eR\) vers le réel qu'elle représente.

	Plus précisément, en suivant les notations de \ref{NORMooWBYNooBQaPPk}, si \( (x_k)\) est une suite de Cauchy dans \( \eQ\), alors
	\begin{enumerate}
		\item
		      \( \varphi(x_k)\) est une suite de Cauchy dans \( \eR\).
		\item
		      \( \varphi(x_k)\stackrel{\eR}{\longrightarrow}\bar x\).
	\end{enumerate}
	Ici \( \bar x\) est la classe de la suite \( x\). C'est donc un élément de \( \eR\).
\end{lemma}

\begin{proof}
	Soit \( (x_n)\) une suite de Cauchy de \( \eQ\), c'est-à-dire que \( x_k\in \eQ\) pour tout \( k\) et qu'elle est de Cauchy. Elle représente un réel \( \bar x\in \eR\), et nous voulons prouver que pour la topologie de \( \eR\) nous avons \( \lim_{n\to \infty} \varphi(x_n)=\bar x\).

	\begin{subproof}
		\item[\( \varphi(x_k)\) est de Cauchy]

		Soit \( \epsilon>0\) dans \( \eR\). Nous considérons \( \epsilon'\in \eQ\) tel que \( 0<\epsilon'<\epsilon\). Plus précisément tel que
		\begin{equation}
			0<\varphi(\epsilon')<\epsilon.
		\end{equation}
		Soit \( N>0\) tel que \( | x_p-x_q |<\epsilon'\) pour tout \( p,q\geq N\). Cela existe parce que \( (x_k)\) est dans Cauchy dans \( \eQ\). Nous avons alors
		\begin{subequations}
			\begin{align}
				| \varphi(x_p)-\varphi(x_q) | & =| \varphi(x_p-x_q) |                                              \\
				                              & =\varphi\big( | x_p-w_q | \big)        \label{SUBEQooZPTIooYsiEqh} \\
				                              & \leq \varphi(\epsilon')            \label{SUBEQooDOFUooUmCZpp}     \\
				                              & \leq \epsilon.
			\end{align}
		\end{subequations}
		Justifications :
		\begin{itemize}
			\item Pour \eqref{SUBEQooZPTIooYsiEqh}. C'est le lemme \ref{LEMooXCVRooOSZYWv}.
			\item Pour \eqref{SUBEQooDOFUooUmCZpp}. L'application \( \varphi\) est croissante, lemme \ref{LEMooTJAXooKEqPCG}.
		\end{itemize}
		Donc la suite \(\varphi(x_k)\) est de Cauchy.

		\item[\( \varphi(x_k)\stackrel{\eR}{\longrightarrow}\bar x\)]
		Nous devons prouver que pour tout \( \epsilon\in \eR\), il existe \( N\) tel que \( n>N\) implique \( \varphi(x_n)\in B\big( \bar x,\epsilon \big)\). Nous allons faire ça en deux parties. D'abord \( \epsilon\in \eQ\) et ensuite \( \epsilon\in \eR\).
		\begin{subproof}
			\item[Avec \( \epsilon\in \eQ\)]
			Soit \( \epsilon\in \eQ\). Puisque \( x\) est une suite de Cauchy dans \( \eQ\), il existe \( N\) tel que si \( p,n>N\) nous avons
			\begin{equation}        \label{EQooJURNooOoSzDZ}
				x_p-\epsilon<x_n<x_p+\epsilon.
			\end{equation}
			Ces inégalités sont dans \( \eQ\). Nous fixons \( p\) et nous commençons par écrire plus en détail la première inéquation :
			\begin{equation}
				x_p-\epsilon-x_n<0.
			\end{equation}
			Autrement dit, pour tout \( n\) nous avons
			\begin{equation}
				(\overline{ x_p-\epsilon })_n<x_n.
			\end{equation}
			Pour rappel, la suite \(  \overline{ x_p-\epsilon } \) est la suite constante dans \( \eQ\). La suite
			\begin{equation}
				n\mapsto x_n-(\overline{ x_p-\epsilon })_n
			\end{equation}
			est dans \( \modE^+\). Donc, en vertu de la définition \ref{DEFooBXHJooOEYPRI} nous avons
			\begin{equation}
				\bar x-\overline{ x_p-\epsilon }\geq 0.
			\end{equation}
			Nous pouvons aussi bien écrire
			\begin{equation}
				\bar x\geq \varphi(x_p)-\varphi(\epsilon).
			\end{equation}
			En prenant l'autre inégalité de \eqref{EQooJURNooOoSzDZ} nous trouvons de la même manière que
			\begin{equation}
				\bar x\leq \varphi(x_p)+\varphi(\epsilon).
			\end{equation}
			Ces deux inégalités ensemble montrent que
			\begin{equation}
				\varphi(x_p)\in B\big( \bar x,\varphi(\epsilon) \big).
			\end{equation}
			\item[Avec \( \epsilon\in \eR\)]
			Nous considérons \( \epsilon\in \eR\) et \( \epsilon'\in \eQ\) tel que \( \varphi(\epsilon')<\epsilon\). Par le point précédent, il existe \( N\) tel que \( p>N\) implique
			\begin{equation}
				\varphi(x_p)\in B\big( \bar x,\varphi(\epsilon') \big).
			\end{equation}
			Étant donné que \( \varphi(\epsilon')<\epsilon\) nous avons
			\begin{equation}
				\varphi(x_p)\in B\big( \bar x,\varphi(\epsilon') \big)\subset B\big( \bar x,\epsilon \big).
			\end{equation}
		\end{subproof}
	\end{subproof}
\end{proof}

\begin{proposition}     \label{PROPooZSQYooWRKNGY}
	Soit une suite convergente \( x_k\stackrel{\eQ}{\longrightarrow}q\). Alors
	\begin{equation}
		\varphi(x_k)\stackrel{\eR}{\longrightarrow}\varphi(q)
	\end{equation}
	où \( \varphi\) est la fonction qui à un rationnel fait correspondre la classe de la suite constante correspondante\footnote{Voir les notations en \ref{NORMooWBYNooBQaPPk}.}.
\end{proposition}

\begin{proof}
	Le fait d'avoir une convergence \( x_k\to q\) dans \( \eQ\) implique que la suite \( (x_k)\) est de Cauchy, par la proposition \ref{PropFFDJooAapQlP}\ref{ItemRKCIooJguHdji}.

	Le lemme \ref{LemooRTGFooYVstwS} nous indique que \( \varphi(x_k)\) est une suite dans \( \eR\) qui converge vers \( \bar q\), la classe de la suite \( (x_k)\).

	À prouver : \( \varphi(x)=\bar q\). Autrement dit, nous devons prouver que la classe de la suite constante \( a_k=q\) et la classe de la suite \( x\) sont les mêmes.

	La suite \( (x_k-q)\) est de Cauchy dans \( \eQ\) et converge vers zéro par hypothèse. Donc les suites \(x\) et \( (q)\) sont dans la même classe.
\end{proof}

\begin{proposition}[\cite{MonCerveau}]     \label{PROPooFGBOooHiZqbs}
	Deux choses à propos de suites de rationnels convergeant vers un réel.
	\begin{enumerate}
		\item       \label{ITEMooMAVYooKFtqlx}
		      Soit un réel \( x\). Il existe une suite de rationnels strictement croissante qui converge vers \( x\).
		\item       \label{ITEMooVOVYooFUbccG}
		      Si de plus \( x>0\), alors la suite (toujours strictement croissante) peut être choisie parmi les rationnels strictement positifs.
	\end{enumerate}
\end{proposition}

\begin{proof}
	Le lemme \ref{LemooHLHTooTyCZYL} nous sera d'une grande aide. Soit \( x\in \eR\). Il existe \( q_0\in \eQ\) tel que \( x-1<q_0<x\). Ensuite nous construisons la suite par récurrence : \( q_k\) est choisi tel que \( q_{k-1}<q_k<x\). Cela règle le point \ref{ITEMooMAVYooKFtqlx}.

	Pour \ref{ITEMooVOVYooFUbccG}. Il suffit de faire la même chose, en partant de \( 0<q_0<x\).
\end{proof}

\begin{theorem}[Complétude de \( \eR\), critère de Cauchy\cite{RWWJooJdjxEK}] \label{THOooUFVJooYJlieh}
	Nous avons :
	\begin{enumerate}
		\item
		      Le corps \( \eR\) est un corps complet (définition~\ref{DefKCGBooLRNdJf}\ref{ITEMooKZZYooDaidGU})
		\item
		      Une suite dans \( \eR\) est convergente (définition~\ref{DefKCGBooLRNdJf}\ref{ITEMooDERQooLmJwFR}) si et seulement si elle est de Cauchy (définition~\ref{DefKCGBooLRNdJf}\ref{ItemVXOZooTYpcYN}).
	\end{enumerate}
\end{theorem}
\index{complet!$\eR$!corps}
\index{critère!de Cauchy}
Notez la grande similitude entre ce théorème et le théorème~\ref{THOooNULFooYUqQYo}. Ils ne sont pas équivalents, ne parlent pas exactement du même objet «\( \eR\)», ni des mêmes notions de suites de Cauchy et de complétude.

\begin{proof}
	Soit \( (x_n)\) une suite de Cauchy dans \( \eR\). Pour chaque \( n\), il existe par le lemme~\ref{LemooHLHTooTyCZYL} un \( y_n\in \eQ\) tel que
	\begin{equation}
		x_n-\frac{1}{ n }<y_n<x_n+\frac{1}{ n }.
	\end{equation}
	\begin{subproof}
		\item[\( (y_n)\) est une suite de Cauchy dans \( \eQ\)]
		Nous prouvons que \( (y_n)\) est une suite de Cauchy dans \( \eQ\) (définition~\ref{DefKCGBooLRNdJf}\ref{ItemVXOZooTYpcYN}). Vu que \( (x_n)\) est de Cauchy pour le corps \( \eR\), si \( \epsilon>0\) dans \( \eR\) est donné, il existe \( n_{\epsilon}\) tel que si \( p,q\geq n_{\epsilon}\), alors \( | x_p-x_q |<\epsilon\).

		Nous avons :
		\begin{equation}
			| y_p-y_q |\leq | y_p-x_p |+| x_p-x_q |+| x_q-y_q |<\frac{1}{ p }+\epsilon+\frac{1}{ q }.
		\end{equation}
		En choisissant \( N_{\epsilon}>\max\{ n_{\epsilon},\frac{1}{ \epsilon } \}\) (ce qui est possible par le lemme~\ref{LemooMWOUooVWgaEi}), et en prenant \( p,q>N_{\epsilon}\), nous avons
		\begin{equation}
			| y_p-y_q |\leq 3\epsilon,
		\end{equation}
		ce qui prouve que \( (y_p)\) est une suite de Cauchy dans \( \eQ\), pour la notion de suite de Cauchy dans \( \eQ\).

		\item[Le réel représenté]

		Puisque \( (y_p)\) est de Cauchy dans \( \eQ\), elle représente un réel que nous notons \( \bar y\).

		\item[Convergence de \( (x_n)\)]

		Nous prouvons que \(     x_n\stackrel{\eR}{\longrightarrow}\bar y \).

		Nous savons qu'une suite de Cauchy de rationnels converge dans \( \eR\) vers le réel qu'elle représente, c'est-à-dire : \( y_n\stackrel{\eR}{\longrightarrow}\bar y\) où chaque \( y_n\in \eQ\) est vu comme la suite constante (cela est le lemme~\ref{LemooRTGFooYVstwS}). Autrement dit, pour \( \epsilon>0\), il existe un \( N_{\epsilon}\in \eN\) tel que si \( p>N_{\epsilon}\) alors \( | \bar y-y_p |<\epsilon\). Pour un tel \( p\) nous avons
		\begin{equation}
			| \bar y-x_p |\leq| \bar y-y_p |+| y_p-x_p |\leq \epsilon+\frac{1}{ p }.
		\end{equation}
		Donc dès que \( p\) est plus grand que \( \max\{ N_{\epsilon},\frac{1}{ \epsilon } \}\), nous avons \( | \bar y-x_p |<2\epsilon\), ce qui signifie que la suite \( (x_n) \) converge vers \( \bar y\) dans \( \eR\).

		Ceci achève de prouver que \( \eR\) est un corps complet.
	\end{subproof}

	En ce qui concerne l'équivalence entre les suites convergentes et de Cauchy, nous venons de prouver que toute suite de Cauchy dans \( \eR\) est convergente. La réciproque est la proposition~\ref{PROPooTFVOooFoSHPg}.

\end{proof}

Nous avons terminé avec la construction des réels. Les propriétés topologiques arrivent en la section~\ref{SECooGKHYooMwHQaD}. En particulier le théorème~\ref{THOooNULFooYUqQYo} pour la complétude de \( \eR\) en tant qu'espace métrique.

%--------------------------------------------------------------------------------------------------------------------------- 
\subsection{Intervalles}
%---------------------------------------------------------------------------------------------------------------------------

Nous avons déjà défini la notion d'intervalle pour un espace totalement ordonné en \ref{DefEYAooMYYTz}. Nous posons quelques notations dans \( \eR\).

\begin{definition}  \label{DEFooAQBUooKLChOW}
	Soient \( a\neq b\) dans \( \eR\). Nous définissons les parties suivantes de \( \eR\) :
	\begin{enumerate}
		\item
		      \( \mathopen] a , b \mathclose[=\{ x\in \eR\tq a<x<b \}\)
		\item
		      \( \mathopen[ a , b \mathclose[=\{ x\in \eR\tq a\leq x<b \}\)
		\item
		      \( \mathopen] a , b \mathclose]=\{ x\in \eR\tq a<x\leq b \}\)
		\item
		      \( \mathopen[ a , b \mathclose]=\{ x\in \eR\tq a\leq x\leq b \}\)
		\item
		      \( \mathopen]-\infty , a \mathclose]=\{ x\in \eR\tq x\leq a \}\)
		\item
		      \( \mathopen]-\infty , a \mathclose[=\{ x\in \eR\tq x< a \}\)
		\item
		      \( \mathopen] a , \infty \mathclose[=\{ x\in \eR\tq x>a \}\)
		\item
		      \( \mathopen[ a , \infty \mathclose[=\{ x\in \eR\tq x\geq a \}\).
		\item
		      \( \mathopen] -\infty , \infty \mathclose[=\eR\).
	\end{enumerate}
	La proposition \ref{PROPooHPMWooQJXCAS} nous dira que tous les intervalles de \( \eR\) sont d'une de ces formes.
\end{definition}

%---------------------------------------------------------------------------------------------------------------------------
\subsection{Maximum, supremum et compagnie}
%---------------------------------------------------------------------------------------------------------------------------

Ce n'est un secret pour personne que $\eR$ est un ensemble totalement ordonné\footnote{Proposition \ref{PROPooYMJVooNAsXae}.} : il y a des éléments plus grands que d'autres, et mieux : à chaque fois que je prends deux éléments différents dans $\eR$, il y en a un des deux qui est plus grand que l'autre. Il n'y a pas d'\emph{ex æquo} dans $\eR$.

\begin{definition}
	Soit \( A\), une partie de \( \eR\).
	\begin{enumerate}
		\item
		      Un nombre \( M\) est un \defe{majorant}{majorant} de \( A\) si \( M\) est plus grand que tous les éléments de \( A\) : pour tout \( x\in A\), \( M\geq x\).
		\item
		      Un nombre \( m\) est un \defe{minorant}{minorant} de \( A\) si \( m\) est plus petit que tous les éléments de \( A\) : pour tout \( x\in A\), \( m\leq x\).
	\end{enumerate}
	Nous parlons de majorant ou de minorants \emph{stricts} lorsque les inégalités sont strictes.
\end{definition}

Nous insistons sur le fait que l'inégalité n'est pas stricte. Ainsi, $1$ est un majorant de $[0,1]$. Dès qu'un ensemble a un majorant, il en a plein. Si $s$ majore l'ensemble $A$, alors $s+1$, $s+4$, et \( s+\frac{ 3 }{ 7 }\) majorent également $A$.

\begin{example}
	Une petite galerie d'exemples de majorants.
	\begin{itemize}
		\item L'intervalle fermé $[4,8]$ admet entre autres $8$ et $130$ comme majorants,
		\item l'intervalle ouvert $]4,8[$ admet également $8$ et $130$ comme majorants,
		\item $7$ n'est pas un majorant de $[1,5]\cup]8,32]$,
		\item $10/10$ majore les notes qu'on peut obtenir à un devoir.
		\item l'intervalle $[4,\infty[$ n'a pas de majorant.
	\end{itemize}
\end{example}

\begin{propositionDef}[Least-upper-bound property\cite{BIBooRRUXooKWzcFo}]		\label{DefSupeA}
	Soit $A$ une partie majorée de $\eR$. Il existe un unique élément \( M\in \eR\) tel que
	\begin{enumerate}
		\item
		      $M\geq x$ pour tout $x\in A$,
		\item
		      pour tout $\varepsilon$, le nombre $M-\varepsilon$ n'est pas un majorant de $a$, c'est-à-dire qu'il existe un élément $x\in A$ tel que $x>M-\varepsilon$.
	\end{enumerate}

	Cet élément est nommé \defe{supremum}{supremum} de $A$ et est noté \( \sup(A)\). De la même façon, \defe{l'infimum}{infimum} de $A$, noté $\inf A$, est le plus grand de ses minorants.
\end{propositionDef}

Par convention, si la partie n'est pas bornée vers le haut, nous dirons que son supremum n'existe pas, ou bien qu'il est égal à $+\infty$, suivant les contextes. Pour votre culture générale, sachez toutefois que $\infty\notin\eR$.

\begin{proof}
	Nous faisons la preuve pour l'infimum.

	\begin{subproof}
		\item[Unicité]

		En ce qui concerne l'unicité, soient \( m_1\) et \( m_2\), deux infimums de \( A\). Supposons \( m_1>m_2\). Alors il existe \( \epsilon>0\) tel que \( m_2<m_2+\epsilon<m_1\) (c'est le lemme~\ref{LemooHLHTooTyCZYL}). Cela prouve que \( m_2+\epsilon\) est un minorant de \( A\) et donc que \( m_2\) n'est pas un infimum.

		\item[Existence]

		Soit $A$, une partie de $\eR$. Nous allons trouver son infimum en suivant une méthode de dichotomie. Pour cela nous allons construire trois suites en même temps de la façon suivante. D'abord nous choisissons un point $x_0$ de $A$ et un point $x_1$ qui minore $A$ (qui existe par hypothèse) :
		\begin{equation}
			\begin{aligned}[]
				x_0 & \text{ est un élément de }A,  \\
				x_1 & \text{ est un minorant de }A, \\
				a_0 & =x_0                          \\
				b_0 & =x_1                          \\
				b_1 & =x_1.
			\end{aligned}
		\end{equation}
		Ensuite, nous faisons la récurrence suivante :
		\begin{equation}
			\begin{aligned}[]
				x_{n+1} & =\frac{ a_n+b_n }{2},        \\
				a_{n+1} & =\begin{cases}
					a_{n}   & \text{si }x_{n+1} \text{ minore } A \\
					x_{n+1} & \text{sinon},
				\end{cases} \\
				b_{n+1} & =\begin{cases}
					x_{n+1} & \text{si }x_{n+1} \text{ minore } A \\
					b_n     & \text{sinon}.
				\end{cases}
			\end{aligned}
		\end{equation}
		Nous allons montrer que \( (a_n)\) et \( (b_n)\) sont des suites convergentes de même limite et que cette limite est l'infimum de \( A\).

		Soit $n\in\eN$; il y a deux possibilités. Soit $a_n=a_{n-1}$ et $b_n=x_n$, soit $a_n=x_n$ et $b_n=b_{n-1}$. Supposons que nous soyons dans le premier cas (le second se traite de façon similaire). Alors nous avons
		\begin{equation}
			\begin{aligned}[]
				| a_n-b_n | & =| a_{n-1}-x_n |                                    \\
				            & =\left| a_{n-1}-\frac{ a_{n-1}+b_{n-1} }{2} \right| \\
				            & =\frac{ 1 }{2}| a_{n-1}-b_{n-1} |,
			\end{aligned}
		\end{equation}
		ce qui prouve que $| a_n-b_n |\to 0$. Nous montrons maintenant que la suite \( (a_n)\) est de Cauchy. En effet nous avons
		\begin{equation}
			| a_n-a_{n-1} |=\begin{cases}
				0 \\
				\left| \frac{ a_n -b_n}{ 2} \right|
			\end{cases}\leq \frac{1}{ 2n }.
		\end{equation}
		Il en est de même pour la suite \( (b_n)\). Ce sont deux suites de Cauchy (donc convergentes par la proposition~\ref{PROPooTFVOooFoSHPg}) qui convergent vers la même limite. Soit \( \ell\) cette limite.

		Le nombre $\ell$ minore $A$. En effet si $a\in A$ est plus petit que $\ell$, les éléments $b_n$ tels que $| b_n-\ell |<| a-\ell |$ ne peuvent pas minorer $A$. D'autre part, pour tout $\epsilon$, le nombre $\ell+\epsilon$ ne peut pas minorer $A$. En effet, $\ell$ est la limite de la suite décroissante $(a_n)$, donc il existe $a_n$ entre $\ell$ et $\ell+\epsilon$. Mais $a_n$ ne minore pas $A$, donc $\ell+\epsilon$ ne minore pas non plus $A$.

		Nous avons prouvé que toute partie minorée de $\eR$ possède un infimum.
	\end{subproof}

	La preuve que toute partie majorée possède un supremum se fait de la même façon.
\end{proof}

\begin{lemma}       \label{LEMooSSVKooDPhSkq}
	Soit une partie \( A\) de \( \eR\). Si \( M\) est un majorant de \( A\), alors \( M\geq \sup(A)\).
\end{lemma}

\begin{proof}
	Si \( M<\sup(A)\), alors en posant \( \epsilon=\sup(A)-M\), le nombre \( \sup(A)-\epsilon\) est encore un majorant de \( A\), ce qui est impossible par définition d'un supremum.
\end{proof}

%///////////////////////////////////////////////////////////////////////////////////////////////////////////////////////////
\subsubsection{Intervalles}
%///////////////////////////////////////////////////////////////////////////////////////////////////////////////////////////

\begin{lemma}[\cite{MonCerveau}]        \label{LEMooRMUCooMKiTGr}
	Soit une partie \( A\) de \( \eR\).
	\begin{enumerate}
		\item       \label{ITEMooIQECooFjJFKz}
		      Nous supposons que \( A\) admette un supremum qui n'est pas dans \( A\). Si \( x\) est un élément de \( A\) strictement plus petit que \( \sup(A)\), alors il existe \( y\in A\) tel que \( x<y<\sup(A)\).
		\item
		      Nous supposons que \( A\) admette un infimum qui n'est pas dans \( A\). Si \( x\) est un élément de \( A\) strictement plus grand que \( \inf(A)\), alors il existe \( y\in A\) tel que \(  \inf(A)<y<x  \).
	\end{enumerate}
\end{lemma}

\begin{proof}
	Soit \( \epsilon>0\). Par définition \ref{DefSupeA} d'un supremum, le nombre \( \sup(A)-\epsilon\) n'est pas un majorant de \( A\). Autrement dit, il existe \( y\in A\) tel que \( y>\sup(A)-\epsilon\). Vu que \( \sup(A)\) est plus grand que tous les éléments de \( A\) et qu'il n'est pas lui-même dans \( A\), nous avons aussi \( \sup(A)>y\). En mettant bout à bout :
	\begin{equation}
		x<\sup(A)-\epsilon<y<\sup(A).
	\end{equation}
\end{proof}

\begin{proposition}[\cite{MonCerveau}]     \label{PROPooHPMWooQJXCAS}
	Tous les intervalles\footnote{Définition \ref{DefEYAooMYYTz}.} de \( \eR\) sont d'une des formes listées dans la définition \ref{DEFooAQBUooKLChOW}.
\end{proposition}

\begin{proof}
	Il y a beaucoup de cas, et nous ne les feront pas tous\footnote{Si vous avez un doute, écrivez-moi.}.
	\begin{subproof}
		\item[Si \( I\) est borné vers le haut et vers le bas]
		Il y a 4 possibilités suivant que \( \inf(I)\) et \( \sup(I)\) soient ou non dans \( I\).
		\item[Si \( \inf(I)\in I\) et \( \sup(I)\in I\)]
		Nous prouvons que \( I=\mathopen[ \inf(I) , \sup(II) \mathclose]\).
		\begin{subproof}
			\item[Dans un sens]
			Si \( x\in I\) nous avons \( \inf(I)\leq x\leq \sup(I)\) parce que \( \inf(I)\) est un minorant de toute élément de \( I\) alors que \( \sup(I)\) est un majorant de tout élément de \( I\). Donc \( x\in \mathopen[ \inf(I) , \sup(I) \mathclose]\).
			\item[Dans l'autre sens]
			Soit \( \inf(I)\leq x\leq \sup(I)\). Vu que \( I\) est un intervalle et que \( \sup(I)\) et \( \inf(I)\) sont deux éléments de \( I\), nous avons \( x\in I\).
		\end{subproof}
		\item[\( \inf(I)\in I\) et \( \sup(I)\notin I\)]
		Nous allons démontrer que \( I=\mathopen[ \inf(I) , \sup(I) \mathclose[\).
				\begin{subproof}
					\item[Dans un sens]
					Soit \( x\in I\). Nous savons (par définition de l'infimum et du supremum) que \( \inf(I)\leq x\leq \sup(x)\). Mais nous sommes dans un cas où \( \sup(I)\neq x\) parce que \( \sup(I)\) n'est pas dans \( I\). Donc
					\begin{equation}
						\inf(I)\leq x<\sup(I),
					\end{equation}
					ce qui montre que \( I\subset \mathopen[ \inf(I) , \sup(I) \mathclose[\).
					\item[Dans l'autre sens]
					Soit \( x\in\mathopen[ \inf(I) , \sup(I) \mathclose[\). Nous utilisons le lemme \ref{LEMooRMUCooMKiTGr}\ref{ITEMooIQECooFjJFKz} : il existe \( y\in I\) tel que \( \inf(I)\leq x<y<\sup(I)\). Vu que \( \inf(I)\in I\), que \( y\in I\) et que \( I\) est un intervalle, nous avons aussi \( x\in I\) et donc \( I\subset \mathopen[ \inf(I) , \sup(I) \mathclose[\).
				\end{subproof}
				\item[\( \inf(I)\notin I\) et \( \sup(I)\in I\)]
				C'est le même raisonnement, mais en utilisant l'autre partie du lemme.
				\item[\( \inf(I)\notin I\) et \( \sup(I)\notin I\)]
				Nous prouvons que \( I=\mathopen] \inf(I) , \sup(I) \mathclose[\).
				\begin{subproof}
					\item[Dans un sens]
					Nous avons
					\begin{equation}
						I\subset\{ x\in \eR\tq \inf(I)\leq x\leq \sup(I) \}.
					\end{equation}
					Mais comme \( \inf(I)\) et \( \sup(I)\) ne sont pas dans \( I\), nous pouvons les enlever dans le membre de droite :
					\begin{equation}
						I\subset\{ x\in \eR\tq \inf(I)<x<\sup(I) \}= \mathopen] \inf(I) , \sup(I) \mathclose[.
					\end{equation}
					\item[Dans l'autre sens]
					Si \( x\in I\), nous avons \( \inf(I)<x<\sup(I)\). En utilisant les deux parties du lemme, nous avons \( y_1\) et \( y_2\) dans \( I\) tels que
					\begin{equation}
						\inf(I)<y_1<x<y_2<\sup(I).
					\end{equation}
					Vu que \( I\) est un intervalle, nous en déduisons que \( x\in I\).
				\end{subproof}
				\item[\( \inf(I)=-\infty\), \( \sup(I)\in I\)]
				Nous prouvons que \( I=\mathopen] -\infty , \sup(I) \mathclose]\).
		\begin{subproof}
			\item[Dans un sens]
			L'inclusion \( I\subset\{ x\in \eR\tq x\leq \sup(I) \}\) est automatique : \( \sup(I)\) majore tous les éléments de \( I\).
			\item[Dans l'autre sens]
			Soit \( x\leq \sup(I)\). Vu que \( I\) n'a pas d'infimum, il existe \( y\in I\) tel que \( y<x<\sup(I)\). Comme \( I\) est un intervalle nous déduisons que \( x\in I\).
		\end{subproof}
	\end{subproof}
	Je vous laisse voir les autres cas.
\end{proof}


%///////////////////////////////////////////////////////////////////////////
\subsubsection{Quelques exemples}
%///////////////////////////////////////////////////////////////////////////

En matière de notations, le maximum de l'ensemble $A$ est noté $\max A$, le supremum est noté $\sup A$. Le minimum et l'infimum sont notés $\min A$ et $\inf A$.

\begin{example}
	Exemples de différence entre majorant, supremum et maximum.
	\begin{itemize}
		\item Le nombre $10$ est un supremum, majorant et maximum de l'intervalle fermé $[0,10]$,
		\item Le nombre $10$ est un majorant et un supremum, mais pas un maximum de l'intervalle ouvert $]0,10[$,
		\item Le nombre $136$ est un majorant, mais ni un maximum ni un supremum de l'intervalle $[0,10]$.
	\end{itemize}
\end{example}

En utilisant les notations concises, ces différents cas s'écrivent ainsi :
\begin{equation}
	\begin{aligned}[]
		10 & =\max[0,10]=\sup[0,10] & 10 & =\sup[0,10[
	\end{aligned}
\end{equation}


\begin{example}
	Si on dit qu'un pont s'effondre à partir d'une charge de $10$ tonnes, alors $10$ tonnes est un \emph{supremum} des charges que le pont peut supporter : si on met $9,999999$ tonnes dessus, il tient encore le coup, mais si on ajoute un gramme, alors il s'effondre (on sort de l'ensemble des charges acceptables).
\end{example}

\begin{example}
	Si on dit qu'un pont résiste jusqu'à $10$ tonnes, alors $10$ tonnes est un \emph{maximum} de la charge acceptable. Sur ce pont-ci, on peut ajouter le dernier gramme. Mais à partir de là, le moindre truc qu'on ajoute, il s'effondre.
\end{example}

\begin{lemma}       \label{LEMooWCUXooFqTwDK}
	À propos de bornes d'un intervalle.
	\begin{enumerate}
		\item
		      La borne inférieure d'un intervalle est son infimum,
		\item
		      la borne supérieure est le supremum.
		\item
		      Si de plus l'intervalle est fermé, l'infimum est un minimum et le supremum est un maximum.
	\end{enumerate}
\end{lemma}


\begin{example}
	Quelques exemples dans les intervalles.
	\begin{enumerate}
		\item
		      $A=\mathopen[ 1 , 2 \mathclose]$. Tous les nombres plus petits ou égaux à $1$ sont minorants, $1$ est infimum et minimum. Le nombre $2$ est un majorant, le maximum et le supremum.
		\item
		      $B=\mathopen] 3 , \pi \mathclose[$. Le nombre $\pi$ est le supremum et est un majorant, mais n'est pas le maximum (parce que $\pi\notin B$). L'ensemble $B$ n'a pas de maximum. Bien entendu, $-1000$ est un minorant.
	\end{enumerate}
	Dans les deux cas, le nombre $53$ est un majorant.
\end{example}

Il existe évidemment de nombreux exemples plus vicieux.

\begin{example}
	Prenons $E=\{ \frac{1}{ n }\tq n\in\eN_0 \}$, dont les premiers points sont indiqués sur la figure~\ref{LabelFigSuiteUnSurn}. Cet ensemble est constitué des nombres $1$, $\frac{ 1 }{2}$, $\frac{1}{ 3 }$, \ldots Le plus grand d'entre eux est $1$ parce que tous les nombres de la forme $\frac{1}{ n }$ avec $n\geq 1$ sont plus petits ou égaux à $1$. Le nombre $1$ est donc maximum de $E$.

	L'ensemble $E$ n'a par contre pas de minimum parce que tout élément de $E$ s'écrit $\frac{1}{ n }$ pour un certain $n$ et est plus grand que $\frac{1}{ n+1 }$ qui est également dans $E$.

	Prouvons que zéro est l'infimum de $E$. D'abord, tous les éléments de $E$ sont strictement positifs, donc zéro est certainement un minorant de $E$. Ensuite, nous savons que pour tout $\varepsilon>0$, il existe un $n$ tel que $\frac{1}{ n }$ est plus petit que $\varepsilon$. L'ensemble $E$ possède donc un élément plus petit que $0+\varepsilon$, et zéro est bien l'infimum.
\end{example}

\newcommand{\CaptionFigSuiteUnSurn}{Les premiers points du type $x_n=1/n$.}
\input{auto/pictures_tex/Fig_SuiteUnSurn.pstricks}

L'exemple suivant est une source classique d'erreurs en ce qui concerne l'infimum. Il sera à relire après avoir vu la définition de limite (définition~\ref{PropLimiteSuiteNum}).

\begin{example}
	Les premiers points de l'ensemble $F=\{ \frac{ (-1)^n }{ n }\tq n\in\eN_0 \}$ sont représentés à la figure~\ref{LabelFigSuiteInverseAlterne}. Bien que (comme nous le verrons plus tard) la limite de la suite $x_n=(-1)^n/n$ soit zéro, il n'est pas correct de dire que zéro est l'infimum de l'ensemble $F$. Le dessin, au contraire, montre bien que $-1$ est le minium (aucun point est plus bas que $-1$), tandis que le maximum est $1/2$.

	Nous reviendrons avec cet exemple dans la suite. Pour l'instant, ayez bien en tête que zéro n'est rien de spécial pour l'ensemble $F$ en ce qui concerne les notions de maximum, minimum et compagnie.
\end{example}
\newcommand{\CaptionFigSuiteInverseAlterne}{Les quelques premiers points du type $(-1)^n/n$.}
\input{auto/pictures_tex/Fig_SuiteInverseAlterne.pstricks}

%--------------------------------------------------------------------------------------------------------------------------- 
\subsection{Racines}
%---------------------------------------------------------------------------------------------------------------------------
\label{SUBSECooMBCNooEqjjTY}

Dans cette section, nous définissons \( \sqrt{ x }\) pour \( x\in\eQ^+\). Vous notez que c'est fait de façon assez algébrique\footnote{Discutable parce que des limites sont utilisées.}, ou en tout cas, en restant proche des définitions. Des définitions plus technologiques utilisant la continuité de \( x\mapsto x^n\) et qui prouvent que l'application est bijective sur un domaine choisi avec prudence existent (voir la définition \ref{DEFooJWQLooWkOBxQ}). Il est même expliqué dans \cite{BIBooMPXEooQLKhku} que la méthode décrite ici permet de définir \( \sqrt[n]{ x }\) pour tout \( n\) entier, et pas seulement pour \( n=2\).

\begin{proposition}     \label{PROPooUHKFooVKmpte}
	Soit \( q\in \eQ^+\). Il existe un unique \( r\in \eR\) tel que \( r^2=q\).

	Plus précisément, en termes des notations de \ref{NORMooWBYNooBQaPPk}, pour tout \( q\in \eQ^+\), il existe un unique \( r\in \eR\) tel que \( r^2=\varphi(q)\).
\end{proposition}

\begin{proof}
	En deux parties : d'abord l'existence et ensuite l'unicité.
	\begin{subproof}
		\item[Existence]
		Si \( q=0\), c'est \( r=0\). Nous supposons \( q>0\). La suite \( (x_k)\) de la proposition \ref{PROPooSTQXooHlIGVf} a la propriété d'être de Cauchy dans \( \eQ\). Donc il existe un réel \( r\) qui est la classe de cette suite. Nous posons donc
		\begin{equation}
			r=\bar x.
		\end{equation}

		En ce qui concerne \( r^2\), nous avons, par définition du produit dans \( \eR\),
		\begin{equation}        \label{EQooPHLFooAZhebM}
			r^2=\bar x^2=\overline{ (x_k^2) },
		\end{equation}
		c'est la classe de la suite de Cauchy donnée par les \( x_k^2\). Posons \( y_k=x_k^2\); la relation \eqref{EQooPHLFooAZhebM} s'écrit
		\begin{equation}
			r^2=\bar y.
		\end{equation}

		La proposition \ref{PROPooSTQXooHlIGVf} nous dit également que \( y\) est une suite de Cauchy et que
		\begin{equation}
			y_k\stackrel{\eQ}{\longrightarrow}q
		\end{equation}
		La proposition \ref{PROPooZSQYooWRKNGY} donne alors \( \bar y=\bar q\), et finalement
		\begin{equation}
			r^2=\bar q=\varphi(q).
		\end{equation}
		Ici tout n'est pas encore terminé avec l'existence parce qu'il faut nous assurer que \( r\geq 0\). Ce n'est pas très compliqué : si \( r<0\), alors nous pouvons faire le choix \( -r\) qui convient tout aussi bien : \( (-r)^2=r^2\).
		\item[Unicité]
		Supposons \( r_1,r_2\in \eR\) tels que \( r_1^2=r_2^2\). La proposition \ref{PROPooYMJVooNAsXae} dit que \( \eR\) est totalement ordonné; disons pour fixer les idées que \( r_1\leq r_2\). Cela signifie, par définition de l'ordre sur \( \eR\), que \( r_2-r_1\geq 0\). En posant \( s=r_2-r_1\) nous avons \( r_2=r_1+s\). Passons au carré; la distribution dans le calcul suivant provient du fait que \( \eR\) est un corps :
		\begin{equation}
			r_2^2=(r_1+s)^2=r_1^2+2r_1s+s^2.
		\end{equation}
		Vu que \( r_1^2=q=r_2^2\), nous avons \( 2r_1s+s^2=0\) ou encore
		\begin{equation}
			s(2r_1+s)=0.
		\end{equation}
		Puisque \( \eR\) est un corps, c'est un anneau intègre\footnote{Lemme \ref{LemAnnCorpsnonInterdivzer}.} et la règle du produit nul s'applique : soit \( s=0\), soit \( 2r_2+s=0\). Puisque \( r_2>0\) et que \( s\geq 0\), nous avons \( 2r_2+s>0\) et donc \( s=0\).

		Nous en déduisons que \( r_1=r_2\).
	\end{subproof}
\end{proof}


%--------------------------------------------------------------------------------------------------------------------------- 
\subsection{Corps valué}
%---------------------------------------------------------------------------------------------------------------------------

\begin{definition}[Valeur absolue, corps valué\cite{BIBooVKGHooSPijZp,BIBooNRMUooJmwzpn}]       \label{DEFooBWXXooAkBBRS}
	Soit un corps \( \eK\). Une \defe{valeur absolue}{valeur absolue} sur \(\eK\) est une application \( | . |\colon \eK\to \eR^+\) telle que
	\begin{enumerate}
		\item
		      \( | x |=0\) si et seulement si \( x=0\),
		\item
		      \( | x+y |\leq | x |+| y |\)
		\item
		      \( | xy |\leq | x | | y |\).
	\end{enumerate}
	Un corps muni d'une valeur absolue est un \defe{corps valué}{corps valué}.
\end{definition}
Un corps valué sera un espace topologique métrique dans la définition \ref{PROPooAWAKooKRmbGT}. Dans le cas d'un corps totalement ordonné, nous avons une valeur absolue donnée par \ref{DefKCGBooLRNdJf}\ref{ItemooWUGSooRSRvYC} et les principales propriétés dans le lemme \ref{LemooANTJooYxQZDw}.

%--------------------------------------------------------------------------------------------------------------------------- 
\subsection{Partie entière, partie fractionnaire}
%---------------------------------------------------------------------------------------------------------------------------

\begin{lemmaDef}[\cite{BIBooSXCNooDUlbEG}]      \label{LEMooLEXTooGAQxGB}
	Pour tout réel \( x\), il existe un unique entier \( n\) vérifiant
	\begin{equation}
		n\leq x<n+1.
	\end{equation}

	Dans ce cas nous avons \( x-n\in\mathopen[ 0 , 1 \mathclose[\).

	Le nombre \( n\) ainsi défini est la \defe{partie entière}{partie entière} de \( x\), et il sera noté \( \integer(x)\). Le nombre \( x-\integer(x)\) est la \defe{partie fractionnaire}{partie fractionnaire} de \( x\) que nous notons \( \rational(x)\).

	Il est toutefois à noter que la partie fractionnaire de \( x\) n'est pas garantie d'être une fraction; cette dénomination est donc un peu trompeuse.
\end{lemmaDef}


%+++++++++++++++++++++++++++++++++++++++++++++++++++++++++++++++++++++++++++++++++++++++++++++++++++++++++++++++++++++++++++
\section{Les complexes}
%+++++++++++++++++++++++++++++++++++++++++++++++++++++++++++++++++++++++++++++++++++++++++++++++++++++++++++++++++++++++++++

La notion de module d'un nombre complexe \( | z |\) sera donnée beaucoup plus tard, dans le lemme \ref{LEMooVHDAooJyoakR}. La raison est que le module demande la racine carrée.

\begin{definition}[Nombres complexes\cite{BIBooBSMSooTkhjce}]
	L'ensemble des \defe{nombres complexes}{nombres complexes} \( \eC\) est l'ensemble \( \eR^2\) muni des opérations suivantes :
	\begin{enumerate}
		\item
		      \begin{equation}        \label{EQooIJWOooZBiKEW}
			      \begin{aligned}
				      \times_{\eC}\colon \eC\times \eC & \to \eC                    \\
				      \big( (x,y),(x',y') \big)        & \mapsto (xx'-yy', xy'+yx')
			      \end{aligned}
		      \end{equation}
		\item
		      \begin{equation}
			      \begin{aligned}
				      +_{\eC}\colon \eC\times \eC & \to \eC              \\
				      \big( (x,y),(x',y') \big)   & \mapsto (x+x', y+y')
			      \end{aligned}
		      \end{equation}
		\item
		      \begin{equation}
			      \begin{aligned}
				      \cdot\colon \eR\times \eC & \to \eC                         \\
				      \big( \lambda,(x,y) \big) & \mapsto (\lambda x, \lambda y).
			      \end{aligned}
		      \end{equation}
	\end{enumerate}
\end{definition}

\begin{lemma}
	Le triplet \( (\eC,+_{\eC}, \times_{\eC})\) est un anneau\footnote{Définition \ref{DefHXJUooKoovob}.} commutatif dont le neutre pour l'addition est \( (0,0)\) et le neutre pour la multiplication est \( (1,0)\).
\end{lemma}

\begin{proof}
	Ce ne sont que des calculs. Juste pour vous montrer, voici la première partie pour l'associativité :
	\begin{subequations}
		\begin{align}
			\big( (a,b)(x,y) \big)(s,t) & =(ax-by,ay+bx)(s,t)                 \\
			                            & =(axs-bys-ayt-bxt,axt-byt+ays+bxs).
		\end{align}
	\end{subequations}
	Nous avons utilisé la distributivité sur \( \eR\), provenant du fait que \( \eR\) est un corps par le théorème \ref{DefooFKYKooOngSCB}.
\end{proof}

\begin{lemma}
	L'anneau \( \eC\) est un corps.
\end{lemma}

\begin{proof}
	% TODOooBRFPooClucQE: justifier la stricte inégalité a^2+b^2>0
	Il suffit de trouver un inverse pour chaque élément non nul. Soit un élément non nul \( (a,b)\in \eC\). En combinant les lemmes \ref{LEMooTPLUooXiCZHJ} et \ref{LEMooNLGSooSGdvAo} nous savons que \( a^2+b^2>0\). En particulier, cet élément est inversible dans \( \eR\), et nous pouvons considérer l'élément suivant de \( \eC\) :
	\begin{equation}
		z=\big( \frac{ a }{ a^2+b^2 }, -\frac{ b }{ a^2+b^2 } \big).
	\end{equation}
	Prouver que \( z(a,b)=(1,0)\) est maintenant juste un calcul.
\end{proof}

\begin{lemma}
	L'application
	\begin{equation}
		\begin{aligned}
			\varphi\colon \eR & \to \eC       \\
			x                 & \mapsto (x,0)
		\end{aligned}
	\end{equation}
	est un morphisme d'anneaux\footnote{Définition \ref{DEFooSPHPooCwjzuz}.}.
\end{lemma}

\begin{proof}
	Simples calculs. Par exemple
	\begin{equation}
		\varphi(xx')=(xx',0)=(x,0)(x',0)=\varphi(x)\varphi(x').
	\end{equation}
\end{proof}

\begin{normaltext}
	Admirez \ldots
	\begin{itemize}
		\item Un nombre complexe est un couple de réels.
		\item Un réel est une classe d'équivalence de suites de Cauchy de rationnels.
		\item Une suite de Cauchy de rationnels est une application \( \eN\to \eQ\) vérifiant certaines propriétés.
		\item Un rationnel est une classe d'équivalences d'éléments de \( \eZ\).
		\item Un élément de \( \eZ\) est une classe d'équivalence de couples de naturels.
		\item Un naturel sera \ldots là c'est plus compliqué. Une construction vraiment rigoureuse des naturels risque d'être en dehors du cadre du Frido.
	\end{itemize}
	Bref, les objets que nous manipulons sont d'une effroyable complexité.
\end{normaltext}

\begin{normaltext}
	À partir de maintenant, lorsque nous parlons de \( \eR\), nous parlons en réalité de \( \varphi(\eR)\subset \eC\).
\end{normaltext}

\begin{lemma}
	Nous avons \( (0,1)^2=(-1,0)\). Nous notons \( i=(0,1)\).
\end{lemma}

\begin{proof}
	Calcul direct à partir de la définition \ref{EQooIJWOooZBiKEW}.
\end{proof}

\begin{proposition}[\cite{MonCerveau}]     \label{PROPooKQHLooMFxNLe}
	L'application
	\begin{equation}
		\begin{aligned}
			f\colon \eR^2 & \to \eC                \\
			(a,b)         & \mapsto a\varphi(1)+bi
		\end{aligned}
	\end{equation}
	est un isomorphisme de \( \eR\)-module\footnote{Module, définition \ref{DEFooHXITooBFvzrR}.}.
\end{proposition}


Cette proposition permet d'écrire tout nombre complexe sous la forme \( a+bi\) pour des réels \( a\) et \( b\).

\begin{definition}      \label{DEFooQDDVooRYDsAJ}
	Si \( z=a+bi\) est un nombre complexe (avec \( a,b\in \eR\)), son \defe{complexe conjugué}{complexe conjugué} est le nombre \( a-bi\).
\end{definition}

L'étude de la série géométrique est reportée à (beaucoup) plus tard, à la proposition \ref{PROPooWOWQooWbzukS}. Dans l'immédiat il nous est possible de calculer la somme partielle.
\begin{lemma}[Somme partielle de la série géométrique]      \label{LEMooAFSCooWEVlvp}
	Soit \( q\in \eC\). Nous avons
	\begin{equation}
		\sum_{n=0}^Nq^n=\frac{ 1-q^{N+1} }{ 1-q }.
	\end{equation}
\end{lemma}

\begin{proof}
	Posons \( S_N=1+q+\ldots+q^N\). Nous avons évidemment $S_N-qS_N=1-q^{N+1}$ et donc
	\begin{equation}    \label{EqASYTiCK}
		S_N=\sum_{n=0}^Nq^n=\frac{ 1-q^{N+1} }{ 1-q }.
	\end{equation}
\end{proof}

