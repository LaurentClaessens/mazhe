% This is part of (everything) I know in mathematics
% Copyright (c) 2011-2023
%   Laurent Claessens
% See the file fdl-1.3.txt for copying conditions.

%+++++++++++++++++++++++++++++++++++++++++++++++++++++++++++++++++++++++++++++++++++++++++++++++++++++++++++++++++++++++++++
\section{Systèmes de coordonnées}
%+++++++++++++++++++++++++++++++++++++++++++++++++++++++++++++++++++++++++++++++++++++++++++++++++++++++++++++++++++++++++++
\label{SECooWTPRooZbOSzO}

La trigonométrie nous offre de nouveaux systèmes de coordonnées qui peuvent se révéler pratiques dans certains cas : les coordonnées polaires sur \( \eR^2\), ainsi que les coordonnées cylindriques et sphériques sur \( \eR^3\).

%---------------------------------------------------------------------------------------------------------------------------
\subsection{Coordonnées polaires}
%---------------------------------------------------------------------------------------------------------------------------

%///////////////////////////////////////////////////////////////////////////////////////////////////////////////////////////
\subsubsection{Ce que ça signifie intuitivement}
%///////////////////////////////////////////////////////////////////////////////////////////////////////////////////////////

On a vu qu'un point \( M\) dans \( \eR^2\) peut être représenté par ses abscisses \( x\) et ses ordonnées \( y\). Nous pouvons également déterminer le même point \( M\) en donnant un angle et une distance comme illustré sur la figure~\ref{LabelFigJWINooSfKCeA}.
\newcommand{\CaptionFigJWINooSfKCeA}{Un point en coordonnées polaires est donné par sa distance à l'origine et par l'angle qu'il fait avec l'horizontale.}
\input{auto/pictures_tex/Fig_JWINooSfKCeA.pstricks}


Le même point \( M\) peut être décrit indifféremment avec les coordonnées \( (x,y)\) ou bien avec \( (r,\theta)\).

\begin{remark}
	L'angle \( \theta\) d'un point n'étant à priori défini qu'à un multiple de \( 2\pi\) près, nous convenons de toujours choisir un angle \( 0\leq\theta<2\pi\). Par ailleurs l'angle \( \theta\) n'est pas défini si \( (x,y)=(0,0)\).

	La coordonnée \( r\) est toujours positive.
\end{remark}

Nous avons dans l'idée de définir \( r\) et \( \theta\) par les formules
\begin{subequations}		\label{EqrthetaxyPoal}
	\begin{numcases}{}
		x=r\cos(\theta)\\
		y=r\sin(\theta).
	\end{numcases}
\end{subequations}

%///////////////////////////////////////////////////////////////////////////////////////////////////////////////////////////
\subsubsection{Coordonnées polaires : le théorème}
%///////////////////////////////////////////////////////////////////////////////////////////////////////////////////////////

\begin{theorem}[Coordonnées polaires\cite{MonCerveau}]     \label{THOooBETSooXSQhdX}
	Soit l'application
	\begin{equation}
		\begin{aligned}
			T\colon \mathopen[ 0 , \infty \mathclose[\times \mathopen[ 0 , 2\pi \mathclose[ & \to \eR^2              \\
			(r,\theta)                                                                      & \mapsto \begin{pmatrix}
				                                                                                          r\cos(\theta) \\
				                                                                                          r\sin(\theta)
			                                                                                          \end{pmatrix}.
		\end{aligned}
	\end{equation}
	\begin{enumerate}
		\item       \label{ITEMooNGOKooFCXmwy}
		      L'application \( T\) est surjective.
		\item       \label{ITEMooMCIOooJiBvug}
		      L'application
		      \begin{equation}
			      T\colon \mathopen] 0 , \infty \mathclose[\times \mathopen[ 0 , 2\pi \mathclose[\to \eR^2\setminus\{ (0,0) \}
		      \end{equation}
		      est bijective.
		\item       \label{ITEMooZFRGooQPDUtX}
		      En considérant la demi-droite \( D=\{ (x,0) \}_{x\geq 0}\), l'application
		      \begin{equation}
			      T\colon \mathopen] 0 , \infty \mathclose[\times \mathopen] 0 , 2\pi \mathclose[\to \eR^2\setminus D
		      \end{equation}
		      est un \(  C^{\infty}\)-difféomorphisme\footnote{L'application est de classe \(  C^{\infty}\) et son inverse est également de classe \(  C^{\infty}\). Le plus souvent, vous voulez seulement utiliser ce théorème dans le but de faire un changement de variables dans une intégrale; vous n'avez donc besoin que d'un \( C^1\)-difféomorphisme.}.
	\end{enumerate}
\end{theorem}

\begin{proof}
	Une bonne partie de ce théorème est une conséquence de \ref{PROPooKSGXooOqGyZj}. La vraie nouveauté de ce théorème sera la régularité.  Nous démontrons point par point.
	\begin{enumerate}
		\item
		      Pour \ref{ITEMooNGOKooFCXmwy}. Soit \( a=(x,y)\in \eR^2\). Nous avons \( a/\| a \|\in S^1\). Par la proposition \ref{PROPooKSGXooOqGyZj}, il existe \( \theta\in \mathopen[ 0 , 2\pi \mathclose]\) tel que
		      \begin{equation}
			      \frac{ a }{ \| a \| }=\big( \cos(\theta),\sin(\theta) \big).
		      \end{equation}
		      Alors \( a=  \| a \|\big( \cos(\theta),\sin(\theta) \big)= T(\| a \|,\theta)\). Voilà. L'application \( T\) est surjective.
		\item
		      Pour \ref{ITEMooMCIOooJiBvug}. En ce qui concerne la surjectivité,
		      \begin{equation}
			      T\big( 0,\mathopen[ 0 , 2\pi \mathclose[ \big)=\{ (0,0) \}.
		      \end{equation}
		      Donc le point \ref{ITEMooNGOKooFCXmwy} donne le surjectif lorsque nous enlevons d'un côté les points avec \( r=0\) et de l'autre le point \( (0,0)\).

		      Pour l'injectivité, nous supposons \( T(r_1,\theta_1)=T(r_2,\theta_2)\). Puisque \( \| T(r,\theta) \|=r\), nous avons tout de suite \( r_1=r_2\). Nous restons donc avec l'égalité
		      \begin{equation}
			      \begin{pmatrix}
				      \cos(\theta_1) \\
				      \sin(\theta_1)
			      \end{pmatrix}=\begin{pmatrix}
				      \cos(\theta_2) \\
				      \sin(\theta_2)
			      \end{pmatrix}.
		      \end{equation}
		      La proposition \ref{PROPooKSGXooOqGyZj} dit alors que \( \theta_1=\theta_2\).
		\item
		      Pour \ref{ITEMooZFRGooQPDUtX}. L'application \( T\) est injective en tant que restriction d'une application injective. Pour le surjectif, soit \( a\in \eR^2\setminus D\). Comme \( a\notin D\), nous avons \( \| a \|\neq 0\) et il est légitime de dire, comme plus haut, qu'il existe \( \theta\in \mathopen[ 0 , 2\pi \mathclose[\) tel que
		      \begin{equation}
			      \frac{ a }{ \| a \| }=\begin{pmatrix}
				      \cos(\theta) \\
				      \sin(\theta)
			      \end{pmatrix}.
		      \end{equation}
		      Ce \( \theta\) n'est pas zéro parce que \( \theta=0\) donne le point \( (1,0)\) qui est sur \( D\).

		      En ce qui concerne l'inverse, nous n'allons pas nous lancer dans une étude subtile de la fonction \eqref{EQooSAYFooRFVSPc}; nous avons déjà démontré la continuité dans le lemme \ref{LEMooEQVRooMAffCw}, et monter dans les dérivées nous semble un peu compliqué. Au lieu de cela, nous allons procéder en deux étapes :
		      \begin{itemize}
			      \item Prouver que \( T\) est de classe \( C^p\) pour tout \( p\) en invoquant seulement des théorèmes à proposition de différentielle,
			      \item
			            En déduire que \( T^{-1}\) est également \( C^p\) pour tout \( p\) en invoquant le théorème d'inversion locale \ref{ThoXWpzqCn}.
		      \end{itemize}

		      Les applications \( (r,\theta)\mapsto r\), \( (r,\theta)\mapsto \sin(\theta)\) et \( (r,\theta)\mapsto \cos(\theta)\) sont de classe \(  C^{\infty}\) grâce au lemme \ref{LEMooDDUZooLwXkRp}. Le lemme \ref{LemDiffProsuid} sur la différentiabilité du produit montre alors que les fonctions \( T_1\) et \( T_2\) données par
		      \begin{subequations}
			      \begin{align}
				      T_1(r,\theta)=r\cos(\theta) \\
				      T_2(r,\theta)=r\sin(\theta)
			      \end{align}
		      \end{subequations}
		      sont différentiables\footnote{Si vous voulez seulement avoir un \( C^1\)-difféomorphisme, calculez explicitement la différentielle et montrez que c'est continu. Vous n'avez pas à utiliser la proposition \ref{PROPooWNCGooHbmcVb} ni rien des produits tensoriels.}. Mieux, la proposition \ref{PROPooWNCGooHbmcVb} montre que ces fonctions \( T_1\) et \( T_2\) sont de classe \( C^p\) pour tout \( p\), c'est-à-dire qu'elles sont de classe \(  C^{\infty}\). Cela montre que les coordonnées polaires sont de classe \(  C^{\infty}\), et il faut encore parler de l'inverse.

		      En ce qui concerne la différentielle,
		      \begin{equation}
			      dT_{(r,\theta)}(u,v)=\begin{pmatrix}
				      u\cos(\theta)-rv\sin(\theta) \\
				      u\sin(\theta)+rv\cos(\theta)
			      \end{pmatrix}.
		      \end{equation}
		      Donc la matrice de la différentielle est
		      \begin{equation}
			      dT_{(r,\theta)}=\begin{pmatrix}
				      \cos(\theta) & -r\sin(\theta) \\
				      \sin(\theta) & r\cos(\theta),
			      \end{pmatrix}
		      \end{equation}
		      dont le déterminant est \( r\) (lemme \ref{LEMooAEFPooGSgOkF} utilisé). Donc la différentielle en \( (r,\theta)\) est une application linéaire inversible parce que \( r\neq 0\) aux points que nous considérons. L'application \( dT_{(r,\theta)}\) est bicontinue parce que nous sommes en dimension finie. Tout cela pour dire que le théorème d'inversion local \ref{ThoXWpzqCn} s'applique, et \( T^{-1}\) est \( C^p\) dès que \( T\) est \( C^p\).

		      Puisque \( T\) est de classe \( C^p\) pour tout \( p\), l'inverse \( T^{-1}\) est également \( C^p\) pour tout \( p\), c'est-à-dire que \( T^{-1}\) est de classe \(  C^{\infty}\).
	\end{enumerate}
\end{proof}

\begin{definition}
	Ce que nous appelons \defe{les coordonnées polaires}{coordonnées polaires} est l'application
	\begin{equation}
		\begin{aligned}
			T\colon \mathopen[ 0 , \infty \mathclose[\times \mathopen[ 0 , 2\pi \mathclose[ & \to \eR^2              \\
			(r,\theta)                                                                      & \mapsto \begin{pmatrix}
				                                                                                          r\cos(\theta) \\
				                                                                                          r\sin(\theta)
			                                                                                          \end{pmatrix}.
		\end{aligned}
	\end{equation}
	du théorème \ref{THOooBETSooXSQhdX}\ref{ITEMooZFRGooQPDUtX}. Selon les circonstances, nous considérons l'une ou l'autre des restrictions pour avoir une bijection ou un difféomorphisme.
\end{definition}

\begin{example}     \label{EXooSDHDooJzDioW}
	Soit à calculer
	\begin{equation}
		\lim_{(x,y)\to(0,0)}\frac{ x^2+y^2 }{ x-y }.
	\end{equation}

	Nous introduisons la fonction
	\begin{equation}
		\begin{aligned}
			f\colon \eR^2\setminus\{ x=y \} & \to \eR                          \\
			(x,y)                           & \mapsto \frac{ x^2+y^2 }{ x-y }.
		\end{aligned}
	\end{equation}
	Une idée souvent fructueuse pour traiter ce genre de limite est de passer aux coordonnées polaires. Attention, si on veut faire les choses très explicitement, c'est un peu lourd en notations. Il s'agit de poser
	\begin{equation}
		\begin{aligned}
			g\colon \big( \mathopen] 0 , \infty \mathclose[\times\mathopen[ 0 , 2\pi \mathclose[ \big)\setminus\big\{ \eR\times\{ \frac{ \pi }{ 4 }\}\cup\eR\times \{ \frac{ 5\pi }{ 4 } \} \} & \to \eR                                                         \\
			(r,\theta)                                                                                                                                                                         & \mapsto \frac{ r^2 }{ r\big( \cos(\theta)-\sin(\theta) \big) }.
		\end{aligned}
	\end{equation}
	Bon. À strictement parler, nous aurions pu dire que \( g\) est définie pour \( r=0\), mais vu que nous voulons seulement calculer la limite pour \( r\to 0\), on n'a pas besoin de la valeur en zéro. De plus les coordonnées polaires ne sont pas bijectives en l'origine. Donc bon \ldots on s'en passe.

	Quel est le lien entre \( f\) et \( g \) ? Du point de vue du calcul, le lien est qu'on a remplacé \( x\) par \( r\cos(\theta)\) et \( y\) par \( r\sin(\theta)\). Le vrai lien est l'égalité
	\begin{equation}
		g=f\circ T
	\end{equation}
	où \( T\) est l'application de coordonnées polaires dont les principales propriétés sont données dans le théorème \ref{THOooBETSooXSQhdX}\ref{ITEMooMCIOooJiBvug}.

	Soit un voisinage \( B\big( (0,0), R \big)\) de \( (0,0)\) dans \( \eR^2\). Le but est de montrer que les valeurs \( f(B)\) se regroupent autour d'une valeur \( \ell\) lorsque \( R\to 0\). Soyons plus précis et nommons \( \ell\) le candidat limite. Soit \( \epsilon>0\); nous devons trouver \( R>0\) tel que \( f\Big( B\big( (0,0),R \big) \Big)\subset B(\ell,\epsilon)\).

	Pour \( R>0\), nous avons
	\begin{equation}
		B\big( (0,0),R \big)=T\big( \mathopen[ 0 , R \mathclose[\times \mathopen[ 0 , 2\pi \mathclose[ \big),
	\end{equation}
	donc
	\begin{equation}
		f(B)=g\big( \mathopen[ 0 , R \mathclose[\times \mathopen[ 0 , 2\pi \mathclose[ \big).
	\end{equation}
	Soit \( r<R\). Nous avons
	\begin{equation}
		\lim_{\theta\to \pi/4} g(r,\theta)=\infty.
	\end{equation}
	Donc \( f(B)\) contient des valeurs arbitrairement grandes, quelle que soit la valeur de \( R\). Il n'y a donc pas de limite possible.

	Si vous voulez un argument un peu plus imagé, en voici un\footnote{Qui satisfera tous vos professeurs, pourvu que vous ayez compris que ce qui se cache est une histoire de valeurs de \( f\) prises sur un voisinage de \( (0,0)\).} basé sur une combinaison entre la méthode des coordonnées polaires et la méthode des chemins.

	Certes \emph{pour chaque \( \theta\)} nous avons \( \lim_{r\to 0} g(r,\theta)=0\), mais il ne faut pas en déduire trop vite que la limite \( \lim_{(x,y)\to(0,0)}g(x,y)\) vaut zéro parce que prendre la limite \( r\to 0\) avec \( \theta\) fixé revient à prendre la limite le long de la droite d'angle \( \theta\).

	Il n'est pas possible de majorer \( g(r,\theta)\) par une fonction ne dépendant pas de \( \theta\) parce que cette fonction tend vers l'infini lorsque \( \theta\to\pi/4\). Est-ce que cela veut dire que la limite n'existe pas ? Cela veut en tout cas dire que la méthode des coordonnées polaires ne parvient pas à résoudre l'exercice. Pour conclure, il faudra encore un peu travailler.

	Nous pouvons essayer de calculer le long d'un chemin plus général \( (r(t),\theta(t))\). Choisissons \( r(t)=t\) puis cherchons \( \theta(t)\) de telle sorte à avoir
	\begin{equation}        \label{EqICrDSe}
		\cos\theta(t)-\sin\theta(t)=t^2.
	\end{equation}
	Le mieux serait de résoudre cette équation pour trouver \( \theta(t)\). Mais en réalité il n'est pas nécessaire de résoudre : montrer qu'il existe une solution suffit. Nous pouvons supposer que \( t^2<1\). Pour \( \theta=\pi/4\) nous avons \( \cos(\theta)-\sin(\theta)=0\) et pour \( \theta=0\) nous avons \( \cos(\theta)-\sin(\theta)=1\). Le théorème des valeurs intermédiaires nous enseigne alors qu'il existe une valeur de \( \theta\) qui résout l'équation \eqref{EqICrDSe}.

	% Laisser en deux lignes, parce que la seconde référence est ok vers le futur.
	Pour être rigoureux, nous devons aussi montrer que la fonction \( \theta(t)\) est continue. Pour cela il faudrait utiliser le théorème de la fonction implicite~\ref{ThoRYN_jvZrZ}.
	Nous verrons dans l'exemple~\ref{ExmeASDLAf} comment s'en sortir sans théorème de la fonction implicite, au prix de plus de calculs.
\end{example}

Les coordonnées polaires sont données par le difféomorphisme
\begin{equation}
	\begin{aligned}
		g\colon \mathopen]0,\infty\mathclose[\times\mathopen]0,2\pi\mathclose[ & \to\eR^2\setminus D                             \\
		(r,\theta)                                                             & \mapsto \big( r\cos(\theta),r\sin(\theta) \big)
	\end{aligned}
\end{equation}
où \( D\) est la demi-droite \( y=0\), \( x\geq 0\). Le fait que les coordonnées polaires ne soient pas un difféomorphisme sur tout \( \eR^2\) n'est pas un problème pour l'intégration parce que le manque de difféomorphisme est de mesure nulle dans \( \eR^2\). Le jacobien est donné par
\begin{equation}
	Jg=\det\begin{pmatrix}
		\partial_rx & \partial_{\theta}x \\
		\partial_ry & \partial_{\theta}y
	\end{pmatrix}=\det\begin{pmatrix}
		\cos(\theta) & -r\sin(\theta) \\
		\sin(\theta) & r\cos(\theta)
	\end{pmatrix}=r.
\end{equation}

La fonction qui donne les coordonnées polaires est
\begin{equation}
	\begin{aligned}
		\varphi\colon \eR^+\times\mathopen] 0 , 2\pi \mathclose[ & \to \eR^2             \\
		(r,\theta)                                               & \mapsto\begin{pmatrix}
			                                                                  r\cos(\theta) \\
			                                                                  r\sin(\theta)
		                                                                  \end{pmatrix}.
	\end{aligned}
\end{equation}
Son Jacobien vaut
\begin{equation}
	J_{\varphi}(r,\theta)=\det\begin{pmatrix}
		\frac{ \partial x(r,\theta) }{ \partial r } & \frac{ \partial x(r,\theta) }{ \partial \theta } \\
		\frac{ \partial y(r,\theta) }{ \partial r } & \frac{ \partial y(r,\theta) }{ \partial \theta }
	\end{pmatrix}=
	\begin{vmatrix}
		\cos(\theta) & -r\sin(\theta) \\
		\sin(\theta) & r\cos(\theta)
	\end{vmatrix}=r.
\end{equation}

\begin{proposition}     \label{PROPooFLUAooDsyMXO}
	Soit la fonction
	\begin{equation}
		\begin{aligned}
			T\colon \mathopen] 0 , +\infty \mathclose[\times \eR & \to \eR^2\setminus\{(0,0)\}                      \\
			(r,\theta)                                           & \mapsto \big( r\cos(\theta),r\sin(\theta) \big).
		\end{aligned}
	\end{equation}
	\begin{enumerate}
		\item
		      Elle est surjective.
		\item
		      Pour tout \( a\in \eR\), l'application \( T\) est bijective sur la bande \( \mathopen] 0 , +\infty \mathclose[\times \mathopen[ a-\pi , a+\pi \mathclose[\).
		\item
		      Si \( a=0\), la fonction inverse est donnée par
		      \begin{equation}
			      T^{-1}(x,y)=\big( \sqrt{ x^2+y^2 },\arctan(y/x) \big).
		      \end{equation}
	\end{enumerate}
\end{proposition}

Soit \( P=(x,y)\) un élément dans \( \eR^2\), on dit que \( r=\sqrt{x^2+y^2}\) est le rayon de \( P\) et que \( \theta=\arctan (y/x) \) est son argument principal. L'origine ne peut pas être décrite en coordonnées polaires parce que si son rayon est manifestement zéro, on ne peut pas lui associer une valeur univoque de l'angle \( \theta\).

\begin{example}
	L'équation du cercle de rayon \( a\) et centre \( (0, 0)\) en coordonnées polaires est \( r=a\).
\end{example}

\begin{example}
	Une équation possible pour la demi-droite \( x=y\), \( x>0\),  est \( \theta=\pi/4\).
\end{example}


%///////////////////////////////////////////////////////////////////////////////////////////////////////////////////////////
\subsubsection{Polaires : calcul de limites}
%///////////////////////////////////////////////////////////////////////////////////////////////////////////////////////////

\begin{example}     \label{EXooBEROooPhPsSU}
	Calculer les limites suivantes :
	\begin{enumerate}
		\item
		      \( \lim_{(x,y)\to(0,0)}\frac{ (xy)^2 }{ (x+y)^2+(x-y)^2 }\)
		\item
		      \( \lim_{(x,y)\to(0,0)}\frac{ xy^3 }{ x^2+y^2 }\)
		\item
		      \( \lim_{(x,y)\to(0,0)}\frac{ x\sin(y) }{ \sqrt{x^2+y^2} }\)
	\end{enumerate}

	Tentez de les faire par vous-même avant de regarder la solution qui suit.
	\begin{enumerate}
		\item
		      Si \( (x,y)\in B(0,\delta)\setminus\{ (0,0) \}\), alors le théorème \ref{THOooBETSooXSQhdX} dit qu'il existe \( 0<r<\delta\) et \( \theta\in\mathopen[ 0 , 2\pi \mathclose[\) tels que \( x=r\cos(\theta)\) et \( y=r\sin(\theta)\). Avec ça nous avons
		      \begin{equation}
			      f(x,y)=\frac{ r^{4}\cos^2(\theta)\sin^2(\theta) }{ r^2\big( \cos(\theta)+\sin(\theta) \big)^2+r^2\big( \cos(\theta)-\sin(\theta) \big)^2}.
		      \end{equation}
		      Vu que \( r\neq 0\) nous pouvons simplifier par \( r^2\). Ce qui reste au dénominateur est une fonction de \( \theta\) continue sur le compact \( \mathopen[ 0 , 2\pi \mathclose] \). Je ne peux pas vous dire quel est le minimum, mais il est facile de voir que ce n'est pas zéro. Bref,
		      \begin{equation}
			      f(x,y)=r^2f(\theta)
		      \end{equation}
		      et \( | f(x,y) |<ar^2\) où \( a\) est une constante que vous pouvez vous amuser à calculer. En posant \( \delta<\sqrt{ \epsilon/a }\) nous avons \( | f(x,y) |<\epsilon\) pour tout \( (x,y)\in B(0,\delta)\), ce qui signifie que \( \lim_{(x,y)\to (0,0)} f(x,y)=0\).
		\item
		      Regardons la technique des coordonnées polaires. Nous remplaçons \( x\) par \( r\cos(\theta)\) et \( y\) par \( r\sin(\theta)\) :
		      \begin{equation}
			      f(r,\theta)=\frac{ r^4\cos(\theta)\sin^3(\theta) }{ r^2 }=r^2\cos(\theta)\sin^3(\theta).
		      \end{equation}
		      Cette fonction tend vers zéro quand \( r\to 0\). Nous avons donc
		      \begin{equation}
			      \lim_{(x,y)\to(0,0)}f(x,y)=0.
		      \end{equation}

		      Pour cet exercice nous pouvons aussi utiliser la règle de l'étau en écrivant d'abord
		      \begin{equation}
			      0\leq | f(x,y) |\leq\frac{ | x | |y^3 | }{ | x^2+y^2 | }.
		      \end{equation}
		      Mais on a \( | x |\leq\sqrt{x^2+y^2}\), \( | y |\leq\sqrt{x^2+y^2}\) et \( | x^2+y^2 |=\big( \sqrt{x^2+y^2} \big)^2\), donc
		      \begin{equation}
			      0\leq| f(x,y) |\leq \frac{ \sqrt{x^2+y^2}\big( \sqrt{x^2+y^2} \big)^3 }{ \big( \sqrt{x^2+y^2} \big)^2 }=\big( \sqrt{x^2+y^2} \big)^2\to 0.
		      \end{equation}

		\item
		      En passant aux polaires, nous avons
		      \begin{equation}
			      f(r,\theta)=\frac{ r\cos\theta\sin\big( r\sin\theta \big) }{ r }=\cos(\theta)\sin\big( r\sin\theta \big).
		      \end{equation}
		      La limite de cette dernière fonction lorsque \( r\to 0\) vaut zéro.

		      Une autre façon de procéder consiste à multiplier et diviser par \( y\) de telle façon à faire apparaitre \( \sin(y)/y\) dont nous connaissons la limite :
		      \begin{equation}
			      f(x,y)=\frac{ \sin(y) }{ y }\cdot\frac{ xy }{ \sqrt{x^2+y^2} }.
		      \end{equation}
		      La limite du premier facteur est \( 1\), tandis que le second peut être traité de façon classique en prenant la valeur absolue et en majorant \( | x |\) par \( \sqrt{x^2+y^2}\).

	\end{enumerate}
\end{example}

%///////////////////////////////////////////////////////////////////////////////////////////////////////////////////////////
\subsubsection{Transformation inverse : théorie}
%///////////////////////////////////////////////////////////////////////////////////////////////////////////////////////////

Voyons la question inverse : comment retrouver \( r\) et \( \theta\) si on connait \( x\) et \( y\) ? Tout d'abord,
\begin{equation}
	r=\sqrt{x^2+y^2}
\end{equation}
parce que la coordonnée \( r\) est la distance entre l'origine et \( (x,y)\). Comment trouver l'angle ? Nous supposons \( (x,y)\neq (0,0)\). Si \( x=0\), alors le point est sur l'axe vertical et nous avons
\begin{equation}
	\theta=\begin{cases}
		\pi/2  & \text{si }y>0 \\
		3\pi/2 & \text{si }y<0
	\end{cases}
\end{equation}
Notez que si \( y<0\), conformément à notre convention \( \theta\geq 0\), nous avons noté \( \frac{ 3\pi }{2}\) et non \( -\frac{ \pi }{ 2 }\).

Supposons maintenant le cas général avec \( x\neq 0\). Les équations \eqref{EqrthetaxyPoal} montrent que
\begin{equation}
	\tan(\theta)=\frac{ y }{ x }.
\end{equation}
Nous avons donc
\begin{equation}
	\theta=\tan^{-1}\left( \frac{ y }{ x } \right).
\end{equation}
La fonction inverse de la fonction tangente est celle définie plus haut.

%///////////////////////////////////////////////////////////////////////////////////////////////////////////////////////////
\subsubsection{Transformation inverse : pratique}
%///////////////////////////////////////////////////////////////////////////////////////////////////////////////////////////

Le code suivant utilise \href{http://www.sagemath.org}{Sage}.

\lstinputlisting{tex/frido/calculAngle.py}

Son exécution retourne :
\begin{verbatim}
(sqrt(2), 1/4*pi)
(sqrt(5), pi - arctan(1/2))
(6, 1/6*pi)
\end{verbatim}
Notez que ce sont des valeurs \emph{exactes}. Ce ne sont pas des approximations, Sage travaille de façon symbolique.

%///////////////////////////////////////////////////////////////////////////////////////////////////////////////////////////
\subsubsection{Coordonnées polaires : dérivées partielles}
%///////////////////////////////////////////////////////////////////////////////////////////////////////////////////////////

Le changement de coordonnées pour les coordonnées polaires est la fonction
\begin{equation}
	f\begin{pmatrix}
		r \\
		\theta
	\end{pmatrix}=\begin{pmatrix}
		x \\
		y
	\end{pmatrix}=\begin{pmatrix}
		r\cos\theta \\
		r\sin\theta
	\end{pmatrix}.
\end{equation}
Considérons une fonction \( g\) sur \( \eR^2\), et définissons la fonction \( \tilde g\) par
\begin{equation}
	\tilde g(r,\theta)=g(r\cos\theta,r\sin\theta).
\end{equation}
La formule \eqref{EqDerCompofg} permet de trouver les dérivées partielles de \( g\) par rapport à \( r\) et \( \theta\) en termes de celles par rapport à \( x\) et \( y\) de \( g\).

Pour faire le lien avec les notations du point précédent, nous avons
\begin{equation}
	\begin{aligned}[]
		f_1(r,\theta) & =r\cos(\theta) \\
		f_2(r,\theta) & =r\sin(\theta) \\
		(x_1,x_2)     & \to(r,\theta)  \\
		(y_1,y_2)     & \to(x,y).
	\end{aligned}
\end{equation}
Nous avons donc
\begin{equation}
	\begin{aligned}[]
		\frac{ \partial \tilde g }{ \partial r }(r,\theta) & =\sum_{i=1}^2\frac{ \partial g }{ \partial x_i }\big( f(r,\theta) \big)\frac{ \partial f_i }{ \partial r }(r,\theta)                        \\
		                                                   & =\frac{ \partial g }{ \partial x }(r\cos\theta,r\sin\theta)\frac{ \partial \big( r\cos\theta \big) }{ \partial r }(r,\theta)                \\
		                                                   & \quad+\frac{ \partial g }{ \partial y }(r\cos\theta,r\sin\theta)\frac{ \partial \big( r\sin\theta\big) }{ \partial r }(r,\theta)            \\
		                                                   & =\cos\theta\frac{ \partial g }{ \partial x }(r\cos\theta,r\sin\theta)+\sin\theta\frac{ \partial g }{ \partial y }(r\cos\theta,r\sin\theta).
	\end{aligned}
\end{equation}

Prenons par exemple \( g(x,y)=\frac{1}{ x^2+y^2 }\). Étant donné que
\begin{equation}
	\frac{ \partial g }{ \partial x }=\frac{ -2x }{ (x^2+y^2)^2 },
\end{equation}
nous avons
\begin{equation}
	\frac{ \partial g }{ \partial x }(r\cos\theta,r\sin\theta)=\frac{ -2\cos\theta }{ r^3 }.
\end{equation}
En utilisant la formule,
\begin{equation}
	\frac{ \partial \tilde g }{ \partial r }(r,\theta)=\cos(\theta)\left( \frac{ -2\cos\theta }{ r^3 } \right)+\sin(\theta)\left( \frac{ -2\sin\theta }{ r^3 } \right)=-\frac{ 2 }{ r^3 }.
\end{equation}
Nous pouvons vérifier directement que cela est correct. En effet
\begin{equation}
	\tilde g(r,\theta)=g(r\cos\theta,r\sin\theta)=\frac{1}{ r^2 },
\end{equation}
dont la dérivée par rapport à \( r\) vaut \( -2/r^3\).

En ce qui concerne la dérivée par rapport à \( \theta\), nous avons
\begin{equation}
	\begin{aligned}[]
		\frac{ \partial \tilde g }{ \partial \theta } & =\frac{ \partial g }{ \partial x }(r\cos\theta,r\sin\theta)\frac{ \partial \big( r\cos(\theta) \big) }{ \partial \theta }+\frac{ \partial g }{ \partial y }(r\cos\theta,r\sin\theta)\frac{ \partial \big( r\sin(\theta) \big) }{ \partial \theta } \\
		                                              & =\left( \frac{ -2\cos\theta }{ r^3 } \right)(-r\sin\theta)+\left( \frac{ -2\sin\theta }{ r^3 } \right)(r\cos\theta)                                                                                                                                \\
		                                              & =0.
	\end{aligned}
\end{equation}

En résumé et avec quelques abus de notation :
\begin{equation}
	\begin{aligned}[]
		\frac{ \partial \tilde g }{ \partial r }      & =\cos(\theta)\frac{ \partial g }{ \partial x }+\sin(\theta)\frac{ \partial g }{ \partial y }    \\
		\frac{ \partial \tilde g }{ \partial \theta } & =-r\sin(\theta)\frac{ \partial g }{ \partial x }+r\cos(\theta)\frac{ \partial g }{ \partial y } \\
	\end{aligned}
\end{equation}

%---------------------------------------------------------------------------------------------------------------------------
\subsection{Coordonnées cylindriques}
%---------------------------------------------------------------------------------------------------------------------------

Les \defe{coordonnées cylindriques}{coordonnées!cylindrique} sont un perfectionnement des coordonnées polaires. Il s'agit simplement de donner le point \( (x,y,z)\) en faisant la conversion \( (x,y)\mapsto(r,\theta)\) et en gardant le \( z\). Les formules de passage sont
\begin{subequations}
	\begin{numcases}{}
		x=r\cos(\theta)\\
		y=r\sin(\theta)\\
		z=z.
	\end{numcases}
\end{subequations}

Soit \( T\) la fonction de \( ]0, +\infty[\times \eR^2\) dans \( \eR^3\setminus\{(0,0,0)\}\) définie par
	\begin{equation}
		\begin{array}{lccc}
			T: & ]0, +\infty[\times \eR\times \eR & \to     & \eR^3\setminus\{(0,0,0)\}         \\
			   & (r, \theta, z)                   & \mapsto & (r\cos \theta, r \sin \theta, z),
		\end{array}
	\end{equation}
	Cette fonction est surjective. Elle est bijective sur chaque bande de la forme  \( ]0, +\infty[\times [a-\pi,a+\pi[\times \eR\), \( a\) dans \( \eR\). Il n'y a presque rien de nouveau par rapport aux coordonnées polaires. Les coordonnées  cylindriques sont intéressantes si on décrit un objet invariant par rapport aux rotations autour de l'axe des \( z\).

	\begin{example}
		Il faut savoir ce que décrivent les équations les plus simples en coordonnées cylindriques,
		\begin{itemize}
			\item \( r\leq a\), pour \( a\) constant dans  \( ]0, +\infty[\), est le cylindre de hauteur infinie qui a pour axe l'axe des \( z\) et pour base le disque de rayon \( a\) centré à l'origine,
			\item \( r= a\) est  la surface du cylindre,
			\item \( \theta = b\) est un demi-plan ouvert et sa fermeture contient l'axe des \( z\),
			\item \( z=c\) est un plan parallèle au plan \( x\)-\( y\).
		\end{itemize}
	\end{example}

	\begin{example}
		Un demi-cône qui a  son sommet en l'origine et  pour axe l'axe des \( z\) est décrit par \( z=d r\).  Si \( d\) est positif  il s'agit  de la moitié supérieure du cône, si \( d<0\) de la moitié inférieure.
	\end{example}

	\begin{example}
		De même,  la sphère de rayon \( a\) et centrée à l'origine est l'assemblage des calottes \( z=\sqrt{a^2-r^2}\) et \( z=-\sqrt{a^2-r^2}\).
	\end{example}

	En ce qui concerne les coordonnées cylindriques, le Jacobien est donné par
	\begin{equation}
		J(r,\theta,z)=\begin{vmatrix}
			\frac{ \partial x }{ \partial r } & \frac{ \partial x }{ \partial \theta } & \frac{ \partial x }{ \partial z } \\
			\frac{ \partial y }{ \partial r } & \frac{ \partial y }{ \partial \theta } & \frac{ \partial y }{ \partial z } \\
			\frac{ \partial z }{ \partial r } & \frac{ \partial z }{ \partial \theta } & \frac{ \partial z }{ \partial z }
		\end{vmatrix}=
		\begin{vmatrix}
			\cos\theta & -r\sin\theta & 0 \\
			\sin\theta & r\cos\theta  & 0 \\
			0          & 0            & 1
		\end{vmatrix}=r.
	\end{equation}
	Nous avons donc \( dx\,dy\,dz=r\,dr\,d\theta\,dz\).

	\begin{subequations}
		\begin{numcases}{}
			x=r\cos\theta\\
			y=r\sin\theta\\
			z=z
		\end{numcases}
	\end{subequations}
	avec \( r\in\mathopen] 0 , \infty \mathclose[\), \( \theta\in\mathopen[ 0 , 2\pi [\) et \( z\in\eR\). Le jacobien vaut \( r\).

	%++++++++++++++++++++++++++++++++++++++++++++++++++++++++++++++++++++++++
	\subsection{Coordonnées sphériques}
	%++++++++++++++++++++++++++++++++++++++++++++++++++++++++++++++++++++++++

	Soit \( T\) la fonction de \( ]0, +\infty[\times \eR^2\) dans \( \eR^3\setminus\{(0,0,0)\}\) définie par
	\begin{equation}
		\begin{array}{lccc}
			T: & ]0, +\infty[\times \eR\times \eR & \to     & \eR^3\setminus\{(0,0,0)\}                                             \\
			   & (\rho, \theta, \phi)             & \mapsto & (\rho\cos \theta\sin \phi, \rho \sin \theta\sin \phi, \rho\cos \phi),
		\end{array}
	\end{equation}
	Cette fonction est surjective. Elle est bijective sur chaque bande de la forme  \( ]0, +\infty[\times [a-\pi,a+\pi[\times [b-\pi/2, b+\pi/2[\), \( a\) et \( b\) dans \( \eR\).  Si \( a =0\) et \( b=-\pi/2\) la fonction inverse \( T^{-1}\) est donnée par
\begin{equation}
	\begin{array}{lccc}
		T^{-1}: & \eR^3\setminus\{(0,0,0)\} & \to     & ]0, +\infty[\times [-\pi,\pi[\times [0, \pi[                                                             \\
		        & (x,y,z)                   & \mapsto & \left(\sqrt{x^2+y^2+z^2}, \arctan \frac{y}{x}, \arccos \left(\frac{z}{\sqrt{x^2+y^2+z^2}}\right)\right).
	\end{array}
\end{equation}
Soit \(  P\) un point dans \( \eR^3\). L'angle \( \phi\) est l'angle entre le demi-axe positif des \( z\) et le vecteur \( \overrightarrow{OP}\), \( \rho\) est la norme de \( \overrightarrow{OP}\) et \( \theta\) est l'argument en coordonnées polaires de la projection de \( \overrightarrow{OP}\) sur le plan \( x\)-\( y\).

\begin{remark}
	Dans la littérature, les angles \( \theta\) et \( \phi\) sont parfois inversés (voire, changent de nom, par exemple \( \varphi\) au lieu de \( \phi\)). Il faut donc être très prudent lorsqu'on veut utiliser dans un cours des formules données dans un autre cours.
\end{remark}

\begin{example}
	Il faut connaitre le sens des équations plus simples,
	\begin{itemize}
		\item \( \rho\leq a\), pour \( a\) constant dans  \( ]0, +\infty[\), est la boule fermée de rayon \( a\) centrée à l'origine,
		\item \( \rho= a\) est  la sphère de rayon \( a\) centrée à l'origine,
		\item \( \theta = b\) est un demi-plan ouvert et sa fermeture contient l'axe des \( z\),
		\item \( \phi= c\) est un demi-cône qui a  son sommet à l'origine et  pour axe l'axe des \( z\).  Si \( c\) est positif  il s'agit  de la moitié supérieure du cône, si \( d<0\) de la moitié inférieure.
	\end{itemize}
\end{example}

Les \defe{coordonnées sphériques}{coordonnées!sphériques} sont ce qu'on appelle les «méridiens» et «longitudes» en géographie. Les formules de transformation sont
\begin{subequations}		\label{SubEqsCoordSphe}
	\begin{numcases}{}
		x=\rho\sin(\theta)\cos(\varphi)\\
		y=\rho\sin(\theta)\sin(\varphi)\\
		z=\rho\cos(\theta)
	\end{numcases}
\end{subequations}
avec \( 0\leq\theta\leq\pi\) et \( 0\leq\varphi<2\pi\).

\begin{remark}
	Attention : d'un livre à l'autre les conventions sur les noms des angles changent. N'essayez donc pas d'étudier par cœur des formules concernant les coordonnées sphériques trouvées autre part. Par exemple sur le premier dessin de \href{http://fr.wikipedia.org/wiki/Coordonnées_sphériques}{wikipédia}, l'angle \( \varphi\) est noté \( \theta\) et l'angle \( \theta\) est noté \( \Phi\). Mais vous noterez que sur cette même page, les conventions de noms de ces angles changent plusieurs fois.
\end{remark}

Les coordonnées sphériques sont données par
\begin{equation}		\label{EqChmVarSpherique}
	\left\{
	\begin{array}{lllll}
		x=r\cos\theta\sin\varphi \\
		y=r\sin\theta\sin\varphi \\
		z=r\cos\varphi
	\end{array}
	\right.
\end{equation}
avec \( r\in] 0,\infty[\), \( \theta\in] 0,2\pi[\) et \( \varphi\in] 0,\pi[\).
Le jacobien associé est \( Jg(r,\theta,\varphi)=-r^2\sin\varphi\). N'oubliez pas que lorsqu'on effectue un changement de variables dans une intégrale, la \emph{valeur absolue} du jacobien apparaît.

Cependant notre convention de coordonnées sphériques fait venir \( \sin(\phi)\) avec \( \phi\in\mathopen[ 0 , \pi [\); vu que le signe de \( \sin(\phi)\) y est toujours positif, cette histoire de valeur absolue est sans grandes conséquences. Ce n'est pas le cas de toutes les conventions possibles.

%///////////////////////////////////////////////////////////////////////////////////////////////////////////////////////////
\subsubsection{Coordonnées sphériques : inverse}
%///////////////////////////////////////////////////////////////////////////////////////////////////////////////////////////

Trouvons le changement inverse, c'est-à-dire trouvons \( \rho\), \( \theta\) et \( \varphi\) en termes de \( x\), \( y\) et \( z\). D'abord nous avons
\begin{equation}
	\rho=\sqrt{x^2+y^2+z^2}.
\end{equation}
Ensuite nous savons que
\begin{equation}
	\cos(\theta)=\frac{ z }{ \rho }
\end{equation}
détermine de façon unique\footnote{Le problème \( \rho=0\) ne se pose pas; pourquoi ?} un angle \( \theta\in\mathopen[ 0 , \pi \mathclose]\). Dès que \( \rho\) et \( \theta\) sont connus, nous pouvons poser \( r=\rho\sin\theta\) et alors nous nous trouvons avec les équations
\begin{subequations}
	\begin{numcases}{}
		x=r\cos(\varphi)\\
		y=r\sin(\varphi),
	\end{numcases}
\end{subequations}
qui sont similaires à celles déjà étudiées dans le cas des coordonnées polaires.
