% This is part of Mes notes de mathématique
% Copyright (c) 2011-2020
%   Laurent Claessens, Carlotta Donadello
% See the file fdl-1.3.txt for copying conditions.

%+++++++++++++++++++++++++++++++++++++++++++++++++++++++++++++++++++++++++++++++++++++++++++++++++++++++++++++++++++++++++++
\section{Extension du corps de base}
%+++++++++++++++++++++++++++++++++++++++++++++++++++++++++++++++++++++++++++++++++++++++++++++++++++++++++++++++++++++++++++
\label{SECooAUOWooNdYTZf}

Nous avons discuté dans la section~\ref{SECooLQVJooTGeqiR} de ce qui arrive au corps lorsqu'on l'étend. Dans cette sections nous allons étudier ce qui arrive aux applications linéaires entre deux \( \eK\)-espaces vectoriels lorsque nous étendons le corps \( \eK\) en un corps \( \eL\).

Soit donc un corps \( \eK\) et deux \( \eK\)-espaces vectoriels \( E\) et \( F\), et entrons dans le vif du sujet\footnote{Le sujet étant le corps étendu.}. Soit \( \eK\) un corps (commutatif) et une extension \( \eL\) de \( \eK\). Soient \( E\) et \( F\), des \( \eK\)-espaces vectoriels de dimension finie.

%---------------------------------------------------------------------------------------------------------------------------
\subsection{Extension des applications linéaires}
%---------------------------------------------------------------------------------------------------------------------------


\begin{definition}[\cite{ooAFBYooYvTCCN}]
    L'espace vectoriel obtenu par \defe{extension du corps de base}{extension!corps de base} de \( E\) est l'espace vectoriel
    \begin{equation}
        E_{\eL}=\eL\otimes_{\eK}E.
    \end{equation}
    Ce dernier est le quotient \( \eL\otimes_{\eK}E=(\eL\times E)/\sim\) par la relation d'équivalence
    \begin{equation}
        (\lambda,v)\sim\big( a\lambda,\frac{1}{ a }v \big)
    \end{equation}
    pour tout \( a\in \eK\). Nous noterons \( [\lambda,v]\) ou \( \lambda\otimes v\) ou encore \( \lambda\otimes_{\eK}v\) la classe de \( (\lambda,v)\).
\end{definition}
Un élément de \( E_{\eL}\) est de la forme \( \sum_k[\lambda_k,v_k]\) avec \( \lambda_k\in \eL\) et \( v_k\in E\). Si \( f\colon E\to F\) est une applications linéaire nous définissons
\begin{equation}
    \begin{aligned}
        f_{\eL}\colon E_{\eL}&\to F_{\eL} \\
        [\lambda,v]&\mapsto [\lambda,f(v)].
    \end{aligned}
\end{equation}

\begin{remark}
    Si deux vecteurs de \( E_{\eL}\) sont linéairement indépendants pour \( \eK\), ils ne le sont pas spécialement pour \( \eL\). Par exemple si \( \eC\) est vu comme \( \eR\)-espace vectoriel, alors \( \{ 1,i \}\) est une partie libre. Mais dans \( \eC\) vu comme \( \eC\)-espace vectoriel, la partie \( \{ 1,i \}\) n'est pas libre.
\end{remark}

Nous définissons aussi l'injection canonique
\begin{equation}
    \begin{aligned}
        \iota\colon E&\to E_{\eL} \\
        v&\mapsto [1,v].
    \end{aligned}
\end{equation}

\begin{proposition}[\cite{ooEPEFooQiPESf}]      \label{PropooWECLooHPzIHw}
    Injectivité et surjectivité respectées.
    \begin{enumerate}
        \item
            L'application \( f_{\eL}\) est injective si et seulement si \( f\) est injective.
        \item
            L'application \( f_{\eL}\) est surjective si et seulement si \( f\) est surjective.
    \end{enumerate}
\end{proposition}

\begin{proof}
    Supposons pour commencer que \( f_{\eL}\) est injective.
    Le diagramme
    \begin{equation}
        \xymatrix{%
            E \ar[r]^-{f}\ar[d]_-{\tau}      &   F\ar[d]^{\tau}\\
            E_{\eL} \ar[r]_{f_{\eL}}  &   F_{\eL}
           }
    \end{equation}
    est un diagramme commutatif. En effet
    \begin{equation}
        (\tau\circ f)(v)=[1,f(v)]
    \end{equation}
    tandis que
    \begin{equation}
        (f_{\eL\circ\tau})(v)=f_{\eL}[1,v]=[1,f(v)].
    \end{equation}
    Donc si \( f(v)=0\) avec \( v\neq 0\) nous aurions \( (\tau\circ f)(v)=0\) et donc aussi \( (f_{\eL}\circ \tau)(v)=0\), alors que \( \tau(v)\neq 0\) dans \( E_{\eL}\).

    Réciproquement, nous supposons que \( f\) est injective et nous prouvons que \( f_{\eL}\) est injective. Par le lemme~\ref{LEMooDAACooElDsYb}\ref{ITEMooEZEWooZGoqsZ}, nous savons qu'il existe \( g\colon F\to E\) telle que \( f\circ g=\id|_F\). Nous en déduisons que \( f_{\eL}\circ g_{\eL}=\id|_{F_{\eL}}\) parce que si \( [\lambda,v]\in F_{\eL}\) alors
    \begin{equation}
        (f_{\eL}\circ g_{\eL})[\lambda,v]=f_{\eL}[\lambda,g(v)]=[\lambda,(f\circ g)(v)]=[\lambda,v].
    \end{equation}
    Notons que \( g\) est injective, donc \( g_{\eL}\) est injective et l'égalité \( f_{\eL}\circ g_{\eL}=\id|_{F_{\eL}} \) implique que \( f_{\eL}\) est également injective.
\end{proof}

\begin{proposition}[\cite{MonCerveau,ooYVQCooFBVEXo}] \label{PROPooMHARooUycAts}
    Soit \( \{ e_i \}_{i=1,\ldots, p}\) une base de \( E\). Alors \( \{ 1\otimes e_i \}_i\) est une base de \( E_{\eL}=\eL\otimes_{\eK}E\).
\end{proposition}

\begin{proof}
    L'espace vectoriel \( E\) peut être écrit comme somme directe \( E=\bigoplus_i\eK e_i\). Si \( \lambda\in \eL\) et \( k\in \eK\) nous avons
    \begin{equation}
        \lambda\otimes ke_i=\frac{ \lambda }{ k }\otimes e_i=\frac{ \lambda }{ k }(1\otimes e_i).
    \end{equation}
    Cela pour introduire que l'application
    \begin{equation}
        \begin{aligned}
            \psi\colon \eL\otimes_{\eK}E&\to \bigoplus_i\eL(1\otimes e_i) \\
            \sum_k \lambda_k\otimes v_k&\mapsto \oplus_i \sum_k(\lambda_k v_{ik})(1\otimes e_i)
        \end{aligned}
    \end{equation}
    où \( v_k=\sum_i v_{ik}e_i\) avec \( v_{ik}\in \eK\) est un isomorphisme de \( \eL\)-espaces vectoriels. La surjectivité est facile. En ce qui concerne l'injectivité, si
    \begin{equation}
        \sum_i\sum_k(\lambda_kv_{ik})(1\otimes e_i)=0
    \end{equation}
    alors les choses suivantes sont nulles également :
    \begin{equation}
        \sum_i\sum_k(\lambda_kv_{ik})(1\otimes e_i)=\sum_{ik}(\lambda_k\otimes v_{ik}e_i)=\sum_k(\lambda_k\otimes \sum_iv_{ik}e_i)=\sum_k(\lambda_k\otimes v_k).
    \end{equation}
    Le dernier est l'argument de \( \psi\). Le fait que ce soit nul implique que \( \psi\) est injective.
\end{proof}

\begin{remark}
    Nous n'avons pas dû prouver que chacun des \( \lambda_k\otimes v_k\) était nul. Et encore heureux, parce que cela pouvait très bien être faux, vu qu'il y a plusieurs façons de noter un élément de \( E_{\eL}\) sous la forme de tels termes.
\end{remark}

\begin{corollary}       \label{CORooTQGHooIKhNtr}
    La \( \eL\)-dimension de \( E_{\eL}\) est égale à la \( \eK\)-dimension de \( E\).
\end{corollary}

%---------------------------------------------------------------------------------------------------------------------------
\subsection{Projections}
%---------------------------------------------------------------------------------------------------------------------------

\begin{probleme}
    Nous allons définir \( \pr\colon \aL(E_{\eL},F_{\eL})\to \aL(E,F)\) en faisant appel à des bases et en prouvant que les choses définies ne dépendent pas des bases choisies. Il y a surement une façon plus «intrinsèque» de faire.
\end{probleme}


Nous savons que \( \eL\) est un \( \eK\)-espace vectoriel dans lequel nous pouvons voir \( \eK\) comme un sous-espace (lemme~\ref{LemooOLIIooXzdppM}). Dans cette optique nous choisissons dans \( \eL\) un supplémentaire de \( \eK\), c'est-à-dire un sous-espace vectoriel de \( \eL\) tel que
\begin{equation}
    \eL=\eK\oplus V.
\end{equation}
Nous avons alors naturellement une projection \( \pr\colon \eL\to \eK\).

Soit \( \{ e_i \}\) une base de \( E \) et \(\{ e_a \}\) une  de\( F\). Nous noterons également \( e_i\) et \( e_a\) les éléments \( \tau e_i\) et \( \tau e_a\) correspondants. Grâce à la proposition~\ref{PROPooMHARooUycAts}, ce sont des bases de \( E_{\eL}\) et \( F_{\eL}\). Si la fonction \( f\colon E_{\eL}\to F_{\eL}\) s'écrit dans ce ces bases comme
\begin{equation}
    f(e_i)=\sum_af_{ai}e_a
\end{equation}
alors nous définissons \( \pr(f)\) par
\begin{equation}        \label{EQooSAFRooJnfkLO}
    (\pr f)e_i=\sum_a\pr(f_{ai})e_a.
\end{equation}

\begin{proposition}[\cite{MonCerveau}]      \label{PROPooOEHTooHyjuZQ}
    L'application \( \pr\) définie en \eqref{EQooSAFRooJnfkLO} est indépendante du choix des bases.
\end{proposition}

\begin{proof}
    Notons que dans ce qui suit, les sommes sur \( a\) ou \( b\) et celles sur \( i\) ou \( j\) ne vont pas jusqu'au même indice (dimensions de \( E\) et \( F\)). De plus nous manipulons deux choses qui se notent \( \pr\). La première est la projection \( \pr\colon \eL\to \eK\) qui ne dépend que d'un choix de supplémentaire et que nous supposons fixée ici. D'autre part il y a \( \pr\colon E_{\eL}\to E\) qui dépend à priori des bases choisies.

    Nous choisissons de nouvelles bases qui sont liées aux anciennes bases par
    \begin{subequations}
        \begin{numcases}{}
            e'_b=\sum_aB_{ab}e_a\\
            e'_i=\sum_jA_{ji}e_j.
        \end{numcases}
    \end{subequations}
    Les matrices \( A\) et \( B\) sont dans \( \GL(\eK)\). Nous allons écrire l'opérateur \( \pr'\) qui correspond à ces bases et montrer que pour toute application linéaire \( f\colon E_{\eL}\to F_{\eL} \) nous avons \( \pr(f)=\pr'(f)\). Nous avons :
    \begin{subequations}
        \begin{align}
            f(e'_j)&=\sum_iA_{ji}f(e_i)\\
            &=\sum_a\sum_b\sum_iA_{ji}f_{ai}(B^{-1})_{ba}e'b\\
            &=\sum_b\Big( \sum_{ai}A_{ji}f_{ai}(B^{-1})_{ba} \Big)e'b,
        \end{align}
    \end{subequations}
    ce qui fait que
    \begin{equation}        \label{EQooUQNBooMWHRbD}
        (\pr'f)e'_j=\sum_b\Big( \pr\big( A_{ji}f_{ai}(B^{-1})_{ba} \big) \Big)e'_b.
    \end{equation}
    Nous calculons maintenant \( (\pr'f)e_j\) en substituant \( e_j=\sum_l(A^{-1})_{lj}e'_l\) et en utilisant \eqref{EQooUQNBooMWHRbD} et la linéarité de \( \pr'\) et la \( \eK\)-linéarité de \( \pr\colon \eL\to \eK\) :
    \begin{subequations}
        \begin{align}
            (\pr'f)\Big( \sum_l(A^{-1})_{lj}e'_l \Big)
            &=\sum_l(A^{-1})_{lj}\sum_b\sum_{ai}\pr\big(A_{li}f_{ai}(B^{-1})_{ba}\big)e_b\\
            &=\sum_a\pr(f_{aj})e_a\\
            &=(\pr f)e_j.
        \end{align}
    \end{subequations}
    Donc \( \pr=\pr'\).
\end{proof}

Note au passage comme toujours : il y a un abus systématique de notation entre \( e_i\in E\) et \( \tau(e_i)=1\otimes e_i\in E_{\eL}\).

\begin{remark}[\cite{MonCerveau}]       \label{REMooBEXGooLgpHzg}
    L'opération \( \pr\colon \aL(E_{\eL},F_{\eL})\to \aL(E,F)\) ne dépend pas des bases choisies un peu partout. Mais elle dépend de l'application \( pr\colon \eL\to \eK\) déjà construite. Et celle-là dépend du choix d'un supplémentaire $V$ qui fournit \( \eL=\eK\oplus V\).

    Si \( \pr(\lambda)=0\) pour un de ces choix, cela n'implique nullement que \( \lambda=0\). Penser à \( i\in \eC\) si la projection \( \pr\colon \eC\to \eR\) est l'application \( (x+iy)\mapsto x\) parallèle à l'axe des imaginaires.

    Par contre si \( \pr(\lambda)=0\) pour tout choix de \( V\), alors nous avons bien \( \lambda=0\). Dans la suit nous «fixons» un choix de \( V\) générique, et lorsque nous rencontrerons l'égalité \( \pr(\lambda)=0\) nous en déduirons \( \lambda=0\).
\end{remark}

\begin{proposition} \label{PROPooPWDKooFNFWRI}
    Si \( f\colon E\to F\) et si \( f_{\eL}e_j=\sum_a(f_{\eL})_{aj}e_a\) et si \( f(e_j)=\sum_af_{aj}e_a\) alors
    \begin{enumerate}
        \item
            \( \pr f_{\eL}=f\),
        \item       \label{ITEMooNMPYooXosGhI}
            \( (f_{\eL})_{ja}=f_{ja} \in \eK\).
    \end{enumerate}
\end{proposition}

\begin{proof}
    Nous avons
    \begin{equation}
        f_{\eL}(e_i)=\sum_a f_{ai}(1\otimes e_a)=\sum_a f_{ai}\tau(e_a),
    \end{equation}
    donc
    \begin{equation}
        (\pr f_{\eL})e_i=\sum_a\pr(f_{ai})e_a=\sum_af_{ai}e_a=f(e_i).
    \end{equation}
    Cela prouve que \( \pr f_{\eL}=f\).

    Par ailleurs,
    \begin{equation}        \label{EQooIOTFooNAdkit}
        f_{\eL}(\tau e_i)=f_{\eL}(1\otimes e_i)=1\otimes f(e_i)=\tau\big( f(e_i) \big)=\sum_af_{ai}\tau(e_a)
    \end{equation}
    alors que par définition,
    \begin{equation}        \label{EQooMYSCooPFWATG}
        f_{\eL}(\tau e_i)=\sum_a(f_{\eL})_{ai}\tau(e_a).
    \end{equation}
    Les éléments \( \tau(e_a)\) formant une base\footnote{Encore la proposition~\ref{PROPooMHARooUycAts}.}, la comparaison de \eqref{EQooIOTFooNAdkit} avec \eqref{EQooMYSCooPFWATG} donne \( (f_{\eL})_{ai}=f_{ai}\in \eK\).
\end{proof}

\begin{lemma}       \label{LEMooWZGSooONEnjZ}
    Soient
    \begin{enumerate}
        \item
            Une base \( \{ e_i \}\) de \( E\) et une application linéaire \( f\colon E\to F\);
        \item
            une base \( \{ e_a \}\) de \( F\) et une application linéaire \( g\colon G\to F\);
        \item
            une base \( \{ e_{\alpha} \} \) de \( G\) et une application linéaire \( \tilde h\colon G_{\eL}\to E_{\eL}\).
    \end{enumerate}
    Alors nous avons
    \begin{equation}
        \pr(f_{\eL}\circ \tilde h)=\pr(f_{\eL})\circ\pr(\tilde h).
    \end{equation}
\end{lemma}

\begin{proof}
    Pour écrire \( \pr(f_{\eL}\circ \tilde h)\) à partir de la définition \eqref{EQooSAFRooJnfkLO} nous commençons par écrire
    \begin{equation}
        (f_{\eL}\circ \tilde h)e_{\alpha}=\sum_a(f_{\eL}\circ \tilde h)_{a\alpha}e_a=\sum_{ai}(f_{\eL})_{ai}(\tilde h)_{i\alpha}e_a=\sum_a\Big( \sum_{i}f_{ai}(\tilde h)_{i\alpha} \Big)e_a
    \end{equation}
    où nous avons utilisé le fait que \( (f_{\eL})_{ai}=f_{ai}\). Donc, en utilisant la \( \eK\)-linéarité de \( \pr\),
    \begin{equation}        \label{EQooZGCGooQsCBQH}
        \pr(f_{\eL}\circ \tilde h)e_{\alpha}=\sum_a\sum_i\pr\Big( f_{ai}(\tilde h)_{i\alpha} \Big)e_a=\sum_a\sum_if_{ai}\pr\Big( (\tilde h)_{i\alpha} \Big)e_a.
    \end{equation}
    D'autre part,
    \begin{equation}
        \begin{aligned}[]
            \pr(f_{\eL})\circ \pr(\tilde h)e_{\alpha}&=\pr(f_{\eL})\sum_i\pr\Big( (\tilde h)_{i\alpha} \Big)e_i\\
            &=\sum_i\pr\Big( (\tilde h)_{i\alpha} \Big)\sum_af_{ai}e_a\\
            &=\sum_{ai}\pr\Big( (\tilde h)_{i\alpha} \Big)f_{ai}e_a,
        \end{aligned}
    \end{equation}
    et c'est égal à \eqref{EQooZGCGooQsCBQH}.
\end{proof}

\begin{remark}
    Nous n'avons en général pas \( \pr(xy)=\pr(x)\pr(y)\) pour tout \( x,y\in \eL\). Par exemple si \( \eK=\eR\) et \( \eL=\eC\) avec la projection canonique,
    \begin{equation}
        \pr(i\cdot i)=\pr(-1)=-1
    \end{equation}
    alors que \( \pr(i)=0\).
\end{remark}

\begin{proposition}
    Soient \( f\in\aL(E,F)\) et \( g\in\aL(F,E)\). Alors il existe \( h\colon G\to E\) tel que \( f\circ h=g\) si et seulement si il existe \( \tilde g\colon G_{\eL}\to E_{\eL}\) tel que \( f_{\eL}\circ \tilde g=g_{\eL}\).
\end{proposition}

\begin{proof}
    Dans le sens direct, il suffit de poser \( \tilde h=h_{\eL}\).

    Dans le sens inverse, si nous avons \( \tilde h\colon G_{\eL}\to E_{\eL}\) tel que \( f_{\eL}\circ\tilde h=g_{\eL}\) alors en appliquant \( \pr\) des deux côtés et en utilisant le lemme~\ref{LEMooWZGSooONEnjZ},
    \begin{equation}
        \pr(f_{\eL})\circ\pr(\tilde h)=\pr(g_{\eL})
    \end{equation}
    c'est-à-dire
    \begin{equation}
        f\circ\pr(\tilde h)=g,
    \end{equation}
    c'est-à-dire que l'application \( \pr\tilde h\colon G\to E\) est la réponse à la proposition.
\end{proof}

%---------------------------------------------------------------------------------------------------------------------------
\subsection{Rang, polynôme minimal, polynôme caractéristique}
%---------------------------------------------------------------------------------------------------------------------------

\begin{proposition}[Stabilité du rang par extension des scalaires\cite{ooEPEFooQiPESf}]     \label{PROPooJFQDooZSsxMf}
    Si \( f\colon E\to F\) est linéaire alors nous avons
    \begin{equation}
        \rang(f)=\rang(f_{\eL}).
    \end{equation}
    où à droite nous considérons le rang de l'application \( \eL\)-linéaire \( f_{\eL}\colon E_{\eL}\to F_{\eL}\).
\end{proposition}

\begin{proof}
    Il existe un supplémentaire \( V\) tel que \( E=\ker(f)\oplus V\) avec \( \dim(V)=\rang(f)\). Nous pouvons factoriser \( f\) en
    \begin{equation}
        f=f_2\circ f_1
    \end{equation}
    avec \( f_1\colon E\to V\) est la projection parallèle à \( \ker(f)\) et est surjective (vers \( V\)) parce que \( \dim(V)=\rang(f)=\dim\big( \Image(f) \big)\). De plus \( f_2\colon V\to F\) est injective parce que si \( v\in V\) est tel que \( f_2(v)=0\) alors on aurait
    \begin{equation}
        f(v)=(f_2\circ f_1)(v)=f_2(v)=0.
    \end{equation}
    Cela donne \( v\in\ker(f)\cap V=\{ 0 \}\). Par la proposition~\ref{PropooWECLooHPzIHw}, les applications \( (f_1)_{\eL}\) et \( (f_2)_{\eL}\) sont respectivement surjective et injective.

    L'application \( (f_2)_{\eL}\colon V_{\eL}\to F_{\eL}\) est forcément surjective sur son image, donc
    \begin{equation}
        (f_2)_{\eL}\colon V_{\eL}\to \Image(f_{\eL})
    \end{equation}
    est un isomorphisme de \( \eL\)-espaces vectoriels. Nous avons alors les égalités
    \begin{equation}        \label{EQooWLOIooKlYWTL}
        \dim_{\eL}(V_{\eL})=\dim_{\eL}\big( \Image(f_{\eL}) \big)=\rang(f_{\eL}).
    \end{equation}
    Mais aussi, par les définitions posées plus haut,
    \begin{equation}        \label{EQooEVCGooAGjmoU}
        \dim(V)=\rang(f)=\dim\big( \Image(f) \big).
    \end{equation}
    Mais le corolaire~\ref{CORooTQGHooIKhNtr} nous dit que \( \dim_{\eL}(V_{\eL})=\dim_{\eK}(V)\). Donc il y a égalité des deux lignes \eqref{EQooWLOIooKlYWTL} et \eqref{EQooEVCGooAGjmoU} donne \( \rang(f)=\rang(f_{\eL})\).
\end{proof}

\begin{proposition}     \label{PROPooZAZFooUFdCUv}
    Nous avons
    \begin{enumerate}
        \item
            \( \det(f)=\det(f_{\eL})\)
        \item
            \( \chi_f=\chi_{f_{\eL}}\).
    \end{enumerate}
\end{proposition}

\begin{proof}
    Dès que l'on a des bases nous avons \( (f_{\eL})_{ai}=f_{ai}\) par la proposition~\ref{PROPooPWDKooFNFWRI}\ref{ITEMooNMPYooXosGhI}. Le nombre \( \det(f)\in \eK\) est un polynôme en les \( f_{ai}\). Entendons nous : il existe un polynôme indépendant de \( f\) et de \( \eK\) et de \( \eL\) donnant le déterminant de n'importe quelle matrice. Donc \( \det(f)=\det(f_{\eL})\).

    Même chose pour le polynôme caractéristique (définition~\ref{DefOWQooXbybYD}) : les coefficients de ce polynôme sont des polynômes en les \( f_{ai}\) qui sont indépendants de \( \eL\), de \( \eK\) et de \( f\).

    Notons que \( \chi_{f_{\eL}}\) est un polynôme à coefficients dans \( \eK\).
\end{proof}

La situation est très différente avec le polynôme minimal\footnote{Définition~\ref{DefCVMooFGSAgL}.}. Autant il existe une «recette» pour créer le polynôme caractéristique, il n'en n'existe pas pour le polynôme minimal (ou en tout cas, il ne suffit pas d'appliquer des polynômes en les coefficients de la matrice). La proposition suivante montre que le polynôme minimal est conservé par extension de corps, mais que pour le voir, il faut travailler plus.

\begin{proposition}[\cite{ooEPEFooQiPESf,MonCerveau}]      \label{PROPooXVZMooXcJrsJ}
    Soit \( \eL\) une extension du corps \( \eK\) et une application linéaire \( f\colon E\to F\) entre deux \( \eK\)-espaces vectoriels. Alors \( \mu_f=\mu_{f_{\eL}}\).
\end{proposition}

\begin{proof}
    Nous allons montrer que l'application
    \begin{equation}
        \begin{aligned}
            \tilde g\colon \frac{ \eL[X] }{ (\mu) }&\to \End(E_{\eL}) \\
            \bar P&\mapsto P(f_{\eL})
        \end{aligned}
    \end{equation}
    est bien définie et injective. La proposition~\ref{PROPooVUJPooMzxzjE} nous dira alors que \( \mu\) est le polynôme minimal de \( f_{\eL}\).

    Pour prouver que l'application \( \tilde g\) est bien définie, nous commençons par prouver que  \( P(f_{\eL})=P(f)_{\eL}\) :
    \begin{subequations}
        \begin{align}
            P(f_{\eL})\lambda\otimes v&=\sum_ka_kf_{\eL}^k\lambda\otimes v\\
            &=\lambda\otimes \sum_ka_kf^k(v)\\
            &=\lambda\otimes P(f)v\\
            &=P(f)_{\eL}\lambda\otimes v.
        \end{align}
    \end{subequations}
    Par conséquent \( \mu(f_{\eL})=0\) et l'application est bien définie.

    Sur \( \eL[X]/(\mu)\) nous considérons la base \( \{ 1,\bar X,\ldots, \bar X^{\deg(\mu)-1} \}\), et \( \End(E_{\eL})\) nous considérons une base qui commence\footnote{Théorème de la base incomplète~\ref{ThonmnWKs}\ref{ITEMooFVJXooGzzpOu}.} par \( \{ f_{\eL}^k \}_{k=0,\ldots, \deg(\mu)-1}\). Montrons tout de même que cette partie est libre (sinon le théorème de la base incomplète ne s'applique pas) : si \( \sum_k\lambda_kf_{\eL}^k=0\) alors
    \begin{equation}        \label{EQooSFHVooLxqUEl}
        \sum_k\pr\big( \lambda_k f_{\eL}^k\big)=0.
    \end{equation}
    Pour détailler ce que cela implique, nous calculons ceci :
    \begin{equation}
        (\lambda f_{\eL})(\tau e_i)=\lambda f_{\eL}(\tau e_i)=\sum_a \lambda f_{ia}e_a,
    \end{equation}
    par conséquent \( \pr(\lambda f_{\eL})e_i=\sum_a\pr(\lambda f_{ia})e_a\), et comme \( \pr\) est \( \eK\)-linéaire et que \( f_{ai}\in \eK\),
    \begin{equation}
        \pr(\lambda f_{\eL})e_i=\pr(\lambda)\sum_a f_{ai}e_a=\pr(\lambda)\pr(f_\eL)e_i=\pr(\lambda)f(e_i).
    \end{equation}
    Appliquer la projection \( \pr\) à l'équation \eqref{EQooSFHVooLxqUEl} donne alors \( \sum_k\pr(\lambda)_kf^k=0\). Mais comme les \( f^k\) sont linéairement indépendantes sur \( \eK\) nous avons pour tout \( k\) : \( \pr(\lambda_k)=0\) (égalité dans \( \eK\)). En nous souvenant de la remarque~\ref{REMooBEXGooLgpHzg} nous en déduisons \( \lambda_k=0\) dans \( \eL\).

    Dans les choix de bases faits, l'application \( \tilde g\) a la forme
    \begin{equation}
        \tilde g=\begin{pmatrix}
            \begin{matrix}
                1    &       &       \\
                    &   1    &       \\
                    &       &   1
            \end{matrix}\\
            \begin{matrix}
                *    &   *    &   *    \\
                *    &   *    &   *    \\
                *    &   *    &   *
            \end{matrix}
        \end{pmatrix},
    \end{equation}
    qui est injective.

    Vu que \( \tilde g\) est injective, \( \mu\) est le polynôme minimal de \( f_{\eL}\) et donc \( \mu=\mu_{\eL}\).
\end{proof}

%+++++++++++++++++++++++++++++++++++++++++++++++++++++++++++++++++++++++++++++++++++++++++++++++++++++++++++++++++++++++++++
\section{Frobenius et Jordan}
%+++++++++++++++++++++++++++++++++++++++++++++++++++++++++++++++++++++++++++++++++++++++++++++++++++++++++++++++++++++++++++

%---------------------------------------------------------------------------------------------------------------------------
\subsection{Matrice compagnon}
%---------------------------------------------------------------------------------------------------------------------------

\begin{definition}      \label{DEFooOSVAooGevsda}
    Soit le polynôme \( P=X^n-a_{n-1}X^{n-1}-\ldots-a_1X-a_0\) dans \( \eK[X]\). La \defe{matrice compagnon}{matrice!compagnon} de \( P\) est la matrice\nomenclature[A]{\( C(P)\)}{matrice compagnon} donnée par
    \begin{equation}
        C(P)=\begin{pmatrix}
            0    &   \cdots    &   \cdots    &   0    &   a_0\\
            1    &   0    &       &   \vdots    &   a_1\\
            0    &   \ddots    &   \ddots    &   \vdots    &   \vdots\\
            \vdots    &   \ddots    &   \ddots    &   0    &   a_{n-2}\\
            0    &   \cdots    &   0    &   1    &   a_{n-1}
        \end{pmatrix}
    \end{equation}
    si \( n\geq 2\) et par \( (a_0)\) si \( n=1\).

    Une matrice est dite compagnon si elle a cette forme.
\end{definition}

\begin{proposition}
    Si \( f\) est l'endomorphisme associé à la matrice \( C(P)\) nous avons
    \begin{equation}
        f(e_i)=\begin{cases}
            e_{i+1}    &   \text{si } i<n\\
            (a_0,\ldots, a_{n-1})    &    \text{si } i=n.
        \end{cases}
    \end{equation}
    De plus l'endomorphisme \( f\) vérifie \( P(f)e_1=0\).
\end{proposition}

\begin{lemma}[\cite{RapportArgreg2011}] \label{LemkVNisk}
    Un polynôme sur un corps commutatif est le polynôme caractéristique de sa matrice compagnon. En d'autres termes nous avons \( \chi_{C(P)}=P\).
\end{lemma}

\begin{proof}
    Nous notons \( f\) l'endomorphisme associé à \( C(P)\). La propriété \( P(f)e_1=0\) nous indique que le polynôme minimal ponctuel de \( f\) en \( e_1\) divise \( P\). L'ensemble des puissances de \( f\) appliquées à \( e_1\), \( \big( f^i(e_1) \big)_{i=1,\ldots, n-1}\) est libre, donc le polynôme minimal ponctuel en \( e_1\) est de degré \( n\) au minimum. En reprenant les notations du théorème~\ref{ThoCCHkoU}, nous avons \( I_{e_1}=(P)\) parce que \( P\) est de degré minimum dans \( I_{e_1}\) et \( \chi_f\in I_{e_1}\).

    Donc \( P\) divise \( \chi_f\) et est de degré égal à celui de \( \chi_f\). Étant donné qu'ils sont tous deux unitaires, ils sont égaux.
\end{proof}

\begin{remark}  \label{RemmQjZOA}
    Les matrices compagnons ne sont pas les seules dont le polynôme caractéristique est égal au polynôme minimal. En fait les matrices dont le polynôme caractéristique est égale au polynôme minimal sont denses dans les matrices. En effet une matrice dont le polynôme minimal n'est pas égal au polynôme caractéristique a un polynôme caractéristique avec une racine double. Il est possible, en modifiant arbitrairement peu la matrice de séparer la racine double en deux racines distinctes.
\end{remark}

%---------------------------------------------------------------------------------------------------------------------------
\subsection{Réduction de Frobenius}
%---------------------------------------------------------------------------------------------------------------------------

\begin{lemma}       \label{LEMooKUQDooKFeIYq}
    Soit un endomorphisme \( f\colon E\to E\) sur l'espace vectoriel de dimension finie \( n\). Nous notons \( \mu\) et \( \chi\) les polynômes minimal et caractéristique. Si \( f\) est cyclique, alors \( \mu=\chi\).
\end{lemma}
Le théorème~\ref{THOooGLMSooYewNxW} donnera une version plus complète de ce lemme.

\begin{proof}
    Soit \( v\) un vecteur cyclique de \( f\), c'est-à-dire que \( \{ f^k(v) \}_{k=0,\ldots, n-1}\) est libre. Donc si \( P\) est un polynôme de degré jusqu'à \( n-1\) nous ne pouvons pas avoir \( P(f)=0\) parce que, appliqué à \( v\), ce serait une combinaisons nulle non triviale des \( f^k(v)\). Donc le polynôme minimal est au minimum de degré \( n\). Mais le polynôme caractéristique est annulateur de degré \( n\) (Cayley-Hamilton~\ref{ThoCalYWLbJQ}), donc il est le polynôme minimal.
\end{proof}

\begin{theorem}[Réduction de Frobenius \cite{AutourFrobCompa,Vialivs,MoncetIVS}]        \label{THOooDOWUooOzxzxm}
    Soit \( E\), un \( \eK\)-espace vectoriel, et \( f\in \End(E)\). Alors il existe une suite de sous-espaces \( E_1,\ldots, E_r\) stables par \( f\) tels que
    \begin{enumerate}
        \item   \label{ItemmpwjnSs}
            \( E=\bigoplus_{i=1}^rE_i\);
        \item
            pour chaque \( E_i\), l'endomorphisme restreint \( f_i=f|_{E_i}\) est cyclique;
        \item
            si \( \mu_i\) est le polynôme minimal de \( f_i\) alors \( \mu_{i+1}\) divise \( \mu_i\);
    \end{enumerate}
    Une telle décomposition vérifie automatiquement \( \mu_1=\mu_f\) et \( \mu_1\cdots \mu_r=\chi_f\), et la suite \( (\mu_i)_{i=1,\ldots, r}\) ne dépend que de \( f\) et non du choix de la décomposition du point~\ref{ItemmpwjnSs}.
\end{theorem}
   \index{réduction!Frobénius}
   \index{Frobénius!réduction}

Les polynômes \( \mu_i\) sont les \defe{invariants de similitude}{invariant!de similitude} de l'endomorphisme \( f\).

\begin{proof}
    Nous commençons par montrer que si une telle décomposition existe, alors
    \begin{subequations}    \label{subEqzcGouz}
        \begin{align}
            \chi_f=\prod_{i=1}^r\mu_i  \label{EqTaxsvb}\\
            \mu_f=\mu_1
        \end{align}
    \end{subequations}
    où \( \chi_f\) est le polynôme caractéristique de \( f\) et \( \mu_f\) est le polynôme minimal. D'abord le polynôme caractéristique de \( f\) devra être égal au produit des polynômes caractéristique des \( f|_{E_i}\), mais ces derniers endomorphismes étant cycliques\footnote{Définition~\ref{DEFooFEIFooNSGhQE}.}, leurs polynôme caractéristiques sont égaux à leurs polynômes minimaux (lemme~\ref{LEMooKUQDooKFeIYq}). Cela prouve l'égalité \eqref{EqTaxsvb}. Ensuite tous les \( \mu_i\) doivent diviser le polynôme minimal, donc \( \ppcm(\mu_1,\ldots, \mu_r)\) divise \(\mu_f\). Cependant le polynôme minimal doit contenir une et une seule fois chacun des facteurs irréductibles du polynôme caractéristique, et chacun de ces facteurs sont dans les polynômes \( \mu_i\). Par conséquent \( \ppcm(\mu_1,\ldots, \mu_r)=\mu_f\). Mais par ailleurs \( \mu_1=\ppcm(\mu_1,\ldots, \mu_r)\) parce qu'on a supposé \( \mu_{i+1}\divides \mu_i\), donc \( \mu_1=\mu_f\).

    Soit \( d\), le degré du polynôme minimal de \( f\) et \( y\in E\) tel que \( \mu_f=\mu_{f,y}\) (voir lemme~\ref{LemSYsJJj}). Le plus petit espace stable sous \( f\) contenant \( y\) est
    \begin{equation}
        E_y=\Span\{ y,f(y),\ldots, f^{d-1}(y) \}.
    \end{equation}
    Nous notons \( e_i=f^{i-1}(y)\). Notons que les vecteurs donnés forment bien une base de \( E_y\) parce que si les \( e_i\) n'était pas linéairement indépendants, alors nous aurions des \( a_k\) tels que \( \sum_ka_ke_k=0\) et avec lesquels
    \begin{equation}
        \big( \sum_ka_kX^k \big)(f)y=0,
    \end{equation}
    ce qui contredirait la minimalité de \( \mu_{f,y}\).

    La difficulté du théorème est de trouver un complément de \( E_y\) qui soit également stable sous \( f\). Nous commençons par étendre\quext{Pour autant que j'aie compris, cette extension manque dans \cite{AutourFrobCompa}. Corrigez-moi si je me trompe.} \( \{ e_1,\ldots, e_d \}\) en une base \( \{ e_1,\ldots, e_n \}\) de \( E\). Ensuite nous allons montrer que
    \begin{equation}
        E=E_y\oplus F
    \end{equation}
    avec
    \begin{equation}
        F=\{ x\in E\tq  e^*_d\big( f^k(x) \big)=0\forall k\in \eN \}.
    \end{equation}
    Par construction, \( F\) est invariant sous \( f\). Montrons pour commencer que \( E_y\cap F=\{ 0 \}\). Un élément de \( E_y\) s'écrit
    \begin{equation}
        z=a_1e_1+\cdots +a_ke_k
    \end{equation}
    avec \( k\leq d\). Étant donné que \( f\) décale les vecteurs de base, nous avons \( e^*_d\big( f^{d-k}(z) \big)=a_k\). Du coup \( z\in F\) si et seulement si \( a_1=\ldots=a_d=0\), c'est-à-dire que \( E_y\cap F=\{ 0 \}\).

    Nous montrons maintenant que \( \dim F=n-d\). Pour cela nous considérons l'application
    \begin{equation}
        \begin{aligned}
            T\colon \eK[F]&\to E^* \\
            g&\mapsto e^*_d\circ g.
        \end{aligned}
    \end{equation}
    Cette application est injective. En effet un élément général de \( \eK[f]\) est
    \begin{equation}
        g=a_1\id+a_2f+\cdots +a_pf^{p-1}
    \end{equation}
    avec \( p\leq d\). Si \( T(g)=0\), alors nous avons en particulier
    \begin{equation}
        0=T(g)e_{_d-p+1}=e^*_d(a_1e_{d-p+1}+a_2e_{d-p+2}+\cdots +a_pe_d)=a_p.
    \end{equation}
    Donc \( a_p=0\) et en appliquant maintenant \( T(g)\) à \( e_{d-p}\) nous obtenons \( a_{p-1}=0\). Au final nous trouvons que \( g=0\) et donc que \( T\) est injective.

    Étant donné que \( \dim\eK[f]=d\) et que \( T\) est injective, \( \dim\Image(T)=d\). Nous regardons l'orthogonal de l'image :
    \begin{subequations}
        \begin{align}
            (\Image(T))^{\perp}&=\{ x\in E\tq T(g)x=0\forall g\in\eK[f] \}\\
            &=\{ x\in E\tq e^*_d\big( g(x) \big)=0\forall g\in \eK[f] \}\\
            &=F.
        \end{align}
    \end{subequations}
    Par conséquent \( F^{\perp}=\Image(T)\). Vu que \( \dim\Image(T)=d\), nous avons donc \( \dim F=n-d\) et il est établi que \( E=E_y\oplus F\).

    Nous avons donc trouvé \( F\), stable par \( f\) et tel que \( E=E_y\oplus F\). Nous devons maintenant nous assurer que cette décomposition tombe bien pour les polynômes minimaux. Si \( P_1\) est le polynôme minimal de \( f|_{E_yj}\), alors par le lemme~\ref{LemAGZNNa} nous avons \( P_1=\mu_{f,y}=\mu_f\) parce que \( f|_{E_y}\) est cyclique sur \( E_y\). Mettons \( P_2\), le polynôme minimal de \( f|_F\). Étant attendu que \( F\) est stable par \( f\), le polynôme \( P_2\) divise \( P_1\). En recommençant la construction sur \( F\), nous construisons un nouvel espace \( F'\) stable sous \( F\) et vérifiant \( \mu_{f|_{F'}}=P_2\), etc.

    Nous passons maintenant à la partie unicité du théorème. Soient deux suites \( F_1,\ldots, F_r\) et \( G_1,\ldots, G_s\) de sous-espaces stables par \( f\) et vérifiant
    \begin{enumerate}
        \item
            \( E=\bigoplus_{i=1}^rF_i\),
        \item
            \( f|_{F_i}\) est cyclique,
        \item
            \( \mu_{f|_{F_{i+1}}}\) divise \( \mu_{f|_{F_i}}\),
    \end{enumerate}
    et, \emph{mutatis mutandis}, les mêmes conditions pour la famille \( \{ G_i \}\). Nous posons \( P_i=\mu_{f_{F_i}}\) et \( Q_i=\mu_{f|_{G_i}}\). Nous allons montrer par récurrence que \( P_i=Q_i\) et \( \dim F_i=\dim G_i\). Il ne sera cependant pas garanti que \( F_i=G_i\). D'abord, \( P_1=Q_1\) parce qu'ils sont tous deux égaux à \( \mu_f\) par les relations \eqref{subEqzcGouz}. Nous supposons que \( P_i=Q_i\) pour \( i\leq 1\leq j-1\) et nous tentons de montrer que \( P_j=Q_j\).

    Nous avons
    \begin{equation}    \label{EqMrCtZO}
        P_j(f)=P_j(f)|_{F_1}\oplus\ldots\oplus P_j(f)|_{F_{j-1}}.
    \end{equation}
    En effet étant donné que \( P_{j+k}\) divise \( P_j\), nous avons\footnote{En vertu du lemme~\ref{LemQWvhYb}.} \( P_{j}(f)=A(f)\circ P_{j+k}(f)\), mais \( P_{j+k}(f)F_{j+k}=0\), donc \( P_j(f)F_{j+k}=0\). Les espaces \( G_i\) n'ayant à priori aucun rapport avec les polynômes \( P_i\), nous écrivons
    \begin{equation}    \label{EqJreLiO}
        P_j(f)=P_j(f)|_{G_1}\oplus\ldots\oplus P_j(f)|_{G_{j-1}}\oplus P_j(f)|_{G_j}\oplus\ldots\oplus P_j(f)|_{G_s}.
    \end{equation}
    Pour \( 1\leq i\leq j-1\), nous avons supposé \( P_i=Q_i\). Étant donné que \( f|_{F_i}\) est semblable à \( C_{_i}\) et \( f|_{G_i}\) est semblable à \( C_{Q_i}\), la matrice de \( f|_{E_i}\) est semblable à la matrice de \( f|_{G_i}\). En particulier,
    \begin{equation}
        \dim P_j(f)F_i=\dim P_j(f)G_i.
    \end{equation}
    En prenant les dimensions des images dans les égalités \eqref{EqMrCtZO} et \eqref{EqJreLiO}, nous trouvons que
    \begin{equation}
        P_j(f)|_{G_j}=\ldots=P_j(f)|_{G_s}=0.
    \end{equation}
    Par conséquent \( P_j\in I_{f|G_j}\) et donc \( P_j\) divise \( Q_j\), qui est générateur de \( I_{f|_{G_j}}\). La situation étant symétrique entre \( P\) et \( Q\), nous montrons de même que \( Q_j\) divise \( P_j\) et donc que \( P_j=Q_j\).

    Ceci achève la démonstration du théorème de réduction de Frobenius.

\end{proof}

\begin{remark}      \label{REMooPVLEooYDRXQI}
    Sous forme matricielle, ce théorème dit que toute matrice est semblable à une matrice de la forme bloc-diagonale
    \begin{equation}
        f=\begin{pmatrix}
            C_{\mu_1}    &       &       \\
                &   \ddots    &       \\
                &       &   C_{\mu_r}
        \end{pmatrix}
    \end{equation}
    où les \( C_{\mu_i}\) sont les matrices compagnon (définition~\ref{DEFooOSVAooGevsda}).

    En particulier, et ceci est très important, deux applications sont semblables si et seulement si elles ont même suite d'invariants de similitude.
\end{remark}


\begin{remark}
    Si nous travaillons sur \( \eR\), la réduite de Frobenius restera une matrice réelle, même si les valeurs propres sont complexes. En effet le procédé de Frobenius ne regarde absolument pas les valeurs propres, mais seulement les facteurs irréductibles du polynôme caractéristique. La réduite de Frobenius ne tente pas de résoudre ces polynômes, mais se contente d'en utiliser les matrices compagnon.

    La situation sera différente dans le cas de la forme normale de Jordan.
\end{remark}

%---------------------------------------------------------------------------------------------------------------------------
\subsection{Forme normale de Jordan}
%---------------------------------------------------------------------------------------------------------------------------

Il existe une preuve directe de la réduction de Jordan ne nécessitant pas la réduction de Frobenius\cite{LecLinAlgAllen}. Cette dernière passe par les espaces caractéristiques\footnote{Aussi appelés «espaces propres généralisés».} et est à mon avis plus compliquée que la démonstration de Frobenius elle-même. Nous allons donc nous contenter de donner la réduction de Jordan comme un cas particulier de Frobenius.

\begin{theorem}[Réduction de Jordan]        \label{ThoGGMYooPzMVpe}
    Soit \( E\) un espace vectoriel sur \( \eK\), et \( f\in\End(E)\) un endomorphisme dont le polynôme caractéristique \( \chi_f\) est scindé\footnote{C'est pour cette hypothèse que \( \eK=\eR\) n'est pas le bon cadre.}. Il existe une base de \( E\) dans laquelle la matrice de \( f\) s'écrit sous la forme
    \begin{equation}
        M=\begin{pmatrix}
            J_{n_1}(\lambda_1)    &       &       \\
                &   \ddots    &       \\
                &       &   J_{n_k}(\lambda_k)
        \end{pmatrix}
    \end{equation}
    où les \( \lambda_i\) sont les valeurs propres de \( f\) (avec éventuelle répétitions) et \( J_n(\lambda)\) représente le bloc \( n\times n\)
    \begin{equation}
        J_n(\lambda)=\begin{pmatrix}
            \lambda    &   1    &       &       &   \\
                &   \lambda    &   1    &       &   \\
                &       &   \lambda    &       &   \\
                &       &       &   \ddots    &   1\\
                &       &       &       &   \lambda
        \end{pmatrix}.
    \end{equation}
    En d'autres termes, \( J_n(\lambda)_{ii}=\lambda\) et \( J_n(\lambda)_{i-1,i}=1\).
\end{theorem}
\index{réduction!Jordan}
\index{Jordan!réduction}

\begin{proof}
    Nous commençons par le cas où \( f\) est nilpotente; nous notons \( M\) sa matrice. Dans ce cas la seule valeur propre est zéro et le polynôme caractéristique est \( X^m\) pour un certain \( m\). Nous savons par le lemme~\ref{LemkVNisk} que (la matrice de) \( f\) est semblable à sa matrice compagnon. En l'occurrence pour \( f\) nous avons
    \begin{equation}
        C_{X^m}=\begin{pmatrix}
             0   &       &       &  0     \\
             1   &   \ddots    &       &   \vdots    \\
                &   \ddots    &   \ddots    &    \vdots   \\
                &       &   1    &   0
         \end{pmatrix}.
    \end{equation}
    Ensuite le changement de base (qui est une similitude) \( (e_1,\ldots, e_n)\mapsto(e_n,\ldots, e_1)\) montre que \( C_{X^m}\) est semblable à un bloc de Jordan \( J_m(0)\).

    Supposons à présent que \( f\) ne soit pas nilpotente. Par l'hypothèse de polynôme caractéristique scindé, nous supposons que \( f\) a \( m\) valeurs propres distinctes et que son polynôme caractéristique est
    \begin{equation}
        \chi_f=(X-\lambda_1)^{l_1}\ldots (X-\lambda_m)^{l_m}.
    \end{equation}
    Le lemme des noyaux (théorème~\ref{ThoDecompNoyayzzMWod}) nous enseigne que
    \begin{equation}
        E=\bigoplus_{i=1}^m\underbrace{\ker(f-\mu_i\mtu)^{l_i}}_{F_i}.
    \end{equation}
    La restriction de \( f-\lambda_i\mtu\) à \( F_i\) est par construction un endomorphisme nilpotent, et donc peut s'écrire comme un bloc de Jordan avec des zéros sur la diagonale. En utilisant la décomposition
    \begin{equation}
        f|_{F_i}=(f-\lambda_i\mtu)|_{F_i}+\lambda_i\mtu_{F_i},
    \end{equation}
    nous voyons que \( f|_{F_i}\) s'écrit comme un bloc de Jordan avec \( \lambda_i\) sur la diagonale.
\end{proof}

\begin{remark}
    Nous pouvons calculer la forme normale de Jordan pour une matrice complexe ou réelle, mais dans les deux cas nous devons nous attendre à obtenir une matrice complexe parce que les valeurs propres d'une matrice réelle peuvent être complexes. Cependant nous demandons que le polynôme caractéristique de \( f\) soit scindé sur \( \eK\). En pratique, la décomposition de Jordan n'est garantie que sur les corps algébriquement clos, c'est-à-dire sur \( \eC\).

    La suite des invariants de similitude sur laquelle repose Frobenius, elle, est disponible sur tout corps, y compris \( \eR\).
\end{remark}

%+++++++++++++++++++++++++++++++++++++++++++++++++++++++++++++++++++++++++++++++++++++++++++++++++++++++++++++++++++++++++++
\section{Commutant et endomorphismes cycliques}
%+++++++++++++++++++++++++++++++++++++++++++++++++++++++++++++++++++++++++++++++++++++++++++++++++++++++++++++++++++++++++++

%---------------------------------------------------------------------------------------------------------------------------
\subsection{Endomorphisme cyclique}
%---------------------------------------------------------------------------------------------------------------------------

\begin{lemma}\label{LemSGmdnE}
    Si \( A\) est la matrice de l'endomorphisme \( f\) alors nous avons équivalence des propriétés suivantes :
    \begin{enumerate}
        \item
            La matrice \( A\) est cyclique.
        \item
            L'endomorphisme \( f\) est cyclique.
    \end{enumerate}
\end{lemma}

Si \( f\) est un endomorphisme de l'espace vectoriel \( E\) et si \( x\in E\), nous notons
\begin{equation}
    E_{f,x}=\Span\{ f^k(x)\tq k\in \eN \}.
\end{equation}

\begin{definition}
    Soit \( E\) un espace vectoriel de dimension finie sur un corps \( \eK\) et un endomorphisme \( f\colon E\to E\). Le \defe{commutant}{commutant} de \( f\) est l'ensemble des endomorphismes de \( E\) qui commutent avec \( f\) :
    \begin{equation}
        \comC(f)=\{ g\in\aL(E,E)\tq g\circ f=f\circ g \}.
    \end{equation}
\end{definition}
Il n'est pas très compliqué de vérifier que \( \comC(f)\) est un sous-espace vectoriel de \( \aL(E,E)\).

Notons l'inclusion évidente \( \eK[f]\subset \comC(f)\). L'inclusion inverse va un peu nous occuper durant les prochaines pages.

%---------------------------------------------------------------------------------------------------------------------------
\subsection{Commutant : cas diagonalisable}
%---------------------------------------------------------------------------------------------------------------------------

\begin{proposition}[\cite{ooKPTNooMmncYA}]      \label{PROPooRHHEooIRGmtl}
    Si \( f\) est diagonalisable, alors
    \begin{equation}        \label{EQooOTFLooPUKAos}
        \dim\big( \comC(f) \big)=\sum_{\lambda\in\Spec(f)}\dim(E_{\lambda})^2.
    \end{equation}
    où les \( E_{\lambda} \) sont les espaces propres de \( f\).
\end{proposition}

\begin{proof}
    D'abord su \( g\in\comC(f)\) alors \( E_{\lambda}\) est stable par \( g\). En effet si \( v\in E_{\lambda}\) alors \( f\big( g(v) \big)=g\big( f(v) \big)=g(\lambda v)=\lambda g(v)\), ce qui montre que \( g(v)\) est un vecteur propre de \( f\) pour la valeur propre \( \lambda\), et donc que \( g(v)\in E_{\lambda}\).

    Nous considérons ensuite l'application
    \begin{equation}
        \begin{aligned}
            \psi\colon \comC(f)&\to \End(E_1)\times \ldots\times \End(E_r) \\
            g&\mapsto  g|_{E_1}\times \ldots\times g|_{E_r}
        \end{aligned}
    \end{equation}
    qui est bien définie parce que \( g\) se restreint aux espaces propres de \( f\). Nous allons noter \( \psi(g)_{\lambda}\) la restriction de \( g\) à \( E_{\lambda}\).
    \begin{subproof}
    \item[\( \psi\) est injective]

        Supposons que \( g,h\in\comC(f)\) tels que \( \psi(g)=\psi(h)\). Vu que \( f\) est diagonalisable nous pouvons décomposer \( x\in  E\) en ses composantes sur les espaces propres\footnote{Théorème~\ref{ThoDigLEQEXR}\ref{ITEMooZNJFooEiqDYp}.} :
        \begin{equation}
            x=\sum_{\lambda\in\Spec(f)}x_{\lambda}
        \end{equation}
        avec \( x_{\lambda}\in E_{\lambda}\).  Nous avons alors
        \begin{equation}
            g(x)=\sum_{\lambda}g(x_{\lambda})=\sum_{\lambda}\psi(g)_{\lambda}(x_{\lambda}).
        \end{equation}
        Vu que nous avons \( \psi(g)_{\lambda}=\psi(h)_{\lambda}\), nous avons aussi
        \begin{equation}
            g(x)=\sum_{\lambda}\psi(g)_{\lambda}(x_{\lambda})=\sum_{\lambda}\psi(h)_{\lambda}(x_{\lambda})=\sum_{\lambda}h(x_{\lambda})=h(x).
        \end{equation}
        Cela prouve \( g=h\) et donc que \( \psi\) est injective.
    \item[\( \psi\) est surjective]
        Si nous avons pour chaque \( \lambda\in\Spec(f)\) un endomorphisme \( g_{\lambda}\) de \( E_{\lambda}\) alors en posant
        \begin{equation}
            g(x)=\sum_{\lambda\in\Spec(f)}g_{\lambda}(x_{\lambda})
        \end{equation}
        alors nous avons bien
        \begin{equation}
            \psi(g)=\big( g_{\lambda_1},\ldots, g_{\lambda_r} \big).
        \end{equation}
    \end{subproof}
    Nous pouvons donc conclure en écrivant
    \begin{equation}
        \dim\big( \comC(f) \big)=\sum_{\lambda\in\Spec(f)}\dim\big( \End(E_{\lambda}) \big)= \sum_{\lambda\in\Spec(f)}\dim(E_{\lambda})^2.
    \end{equation}
\end{proof}

\begin{remark}      \label{REMooUGFQooVzCOvV}
    Nous avons alors immédiatement
    \begin{equation}
        \dim\big( \comC(f) \big)\geq\dim(E)
    \end{equation}
    lorsque \( f\) est diagonalisable.
\end{remark}

En suivant la notation \eqref{EqooOAYDooEpZELo}, un endomorphisme est cyclique lorsqu'il existe \( x\in E\) tel que \( E_x=E\).

\begin{proposition}[\cite{ooKPTNooMmncYA}]      \label{PropooQALUooTluDif}
    Si \( f\) est un endomorphisme diagonalisable d'un espace vectoriel \( E\) de dimension \( n\). Nous avons équivalence entre les faits suivants.
    \begin{enumerate}
        \item\label{ITEMooSOYYooZVibjrii}
            Le polynôme minimal est égal au polynôme caractéristique : \( \mu_f=\chi_f\)
        \item\label{ITEMooSOYYooZVibjrvi}
            L'endomorphisme \( f\) est cyclique.
        \item\label{ITEMooSOYYooZVibjrv}
            \( \comC(f)=\eK[f]\).
        \item\label{ITEMooSOYYooZVibjriv}
            \( \dim\big( \comC(f) \big)=n\)
        \item\label{ITEMooSOYYooZVibjriii}
            L'endomorphisme \( f\) possède \( n\) valeurs propres distinctes.
        \item   \label{ITEMooSOYYooZVibjri}
            \( \dim\big( \eK[f] \big)=n\)
    \end{enumerate}
\end{proposition}

\begin{proof}
    Le point important de cette proposition sont les équivalences~\ref{ITEMooSOYYooZVibjrii}-\ref{ITEMooSOYYooZVibjrv}. Les autres sont des intermédiaires. En particulier, dans le cas diagonalisable, nous allons voir que le point~\ref{ITEMooSOYYooZVibjriii} est essentiellement une reformulation de~\ref{ITEMooSOYYooZVibjrii}.
    \begin{subproof}
        \item[\ref{ITEMooSOYYooZVibjriv} implique~\ref{ITEMooSOYYooZVibjriii}]
            Par la formule \eqref{EQooOTFLooPUKAos}, les espaces propres de \( f\) ont dimension \( 1\). Par conséquent \( f\) possède \( n\) valeurs propres distinctes.
        \item[\ref{ITEMooSOYYooZVibjriii} implique~\ref{ITEMooSOYYooZVibjri}]
            Le théorème~\ref{ThoDigLEQEXR} nous dit que le polynôme minimal est scindé à racines simples. Vu que \( f\) possède \( n\) valeurs propres distinctes, \( \mu\) est de degré \( n\).  Par l'isomorphisme \( \eK[f]=\eK[X]/(\mu)\) de la proposition~\ref{PropooCFZDooROVlaA} nous avons \(\dim\big( \eK[f] \big)= \deg(\mu)=n\) par la proposition~\ref{CorsLGiEN}.
        \item[\ref{ITEMooSOYYooZVibjri} implique~\ref{ITEMooSOYYooZVibjrii}]
            Par l'isomorphisme \( \eK[f]=\eK[X]/(\mu)\) de la proposition~\ref{PropooCFZDooROVlaA} et la proposition~\ref{CorsLGiEN} nous avons \(n=\dim\big( \eK[f] \big)= \deg(\mu)\). Vu que \( \chi\) est un polynôme annulateur (Caley-Hamilton~\ref{ThoCalYWLbJQ}), il est divisé par \( \mu\). Maintenant \( \mu\) et \( \chi\) sont des polynômes unitaires de degré \( n\) et \( \mu\) divise \( \chi\). Ils sont donc égaux.
        \item[\ref{ITEMooSOYYooZVibjrii} implique~\ref{ITEMooSOYYooZVibjrvi}]
            Le fait que $f$ soit diagonalisable permet d'utiliser le théorème~\ref{ThoDigLEQEXR} pour dire que \( \mu\) est scindé à racines simples. L'égalisation avec \( \chi \) nous permet de dire que \( f\) possède \( n\) valeurs propres distinctes. Soient \( \{ e_1,\ldots, e_n \}\) une base de diagonalisation, et prouvons que le vecteur \( v=e_1+\cdots +e_n\) est cyclique. Nous avons
            \begin{equation}
                f^k(v)=\sum_{i=1}^n\lambda_i^ke_i.
            \end{equation}
            Pour prouver que cette famille (avec \( k=0,\ldots, n-1\)) est libre\footnote{Ce sera alors une base parce que \( n\) vecteurs libres dans un espace de dimension \( n\) est toujours une base, proposition \ref{PROPooVEVCooHkrldw}.} nous en prenons une combinaison linéaire nulle et nous prouvons que les coefficients sont tous nuls. Soit donc
            \begin{equation}
                    0=\sum_{l=0}^{n-1}a_lf^l(v)=\sum_{l=0}^{n-1}a_l\sum_{i=1}^n\lambda_i^le_i=\sum_{i=1}^n\Big( \sum_{l=0}^{n-1}a_l\lambda_i^l \Big)e_i.
            \end{equation}
            Vu que cela est nul, nous avons pour tout \( i\) :
            \begin{equation}
                \sum_{l=0}^{n-1}a_l\lambda_i^l=0.
            \end{equation}
            En posant la matrice \( A_{ij}=\lambda_i^j\), cela revient à étudier le système \( \sum_j A_{ij}a_j=0\). Ce système n'a des solutions non nulles que si \( \det(A)= 0\); sinon il possède une unique solution et elle est \( a_j=0\) pour tout \( j\). Nous devons donc calculer le déterminant
            \begin{equation}
                \det\begin{pmatrix}
                    1&\lambda_1&\lambda_1^2&\cdots&\lambda_1^{n-1}\\
                    \vdots&\vdots&\vdots&&\vdots\\
                    1&\lambda_n&\lambda_n^2&\cdots&\lambda_n^{n-1}
                \end{pmatrix}.
            \end{equation}
            Il s'agit du déterminant de Vandermonde déjà étudié par la proposition~\ref{PropnuUvtj}. Nous avons \( \det(A)=\prod_{1\leq i<j\leq n}(\lambda_j-\lambda_i)\). Cela est bien non nul du fait que toutes les valeurs propres soient distinctes.
        \item[\ref{ITEMooSOYYooZVibjrvi} implique~\ref{ITEMooSOYYooZVibjrv}]
            Soit \( v\) un vecteur cyclique de \( f\). Un endomorphisme \( g\) donne lieu à un polynôme par le fait suivant : il existe des uniques \( a_k\) (\( k=0,\ldots, n-1\)) tels que
            \begin{equation}
                g(v)=\sum_{k=0}^{n-1}a_kf^k(v).
            \end{equation}
            Cela donne une application linéaire
            \begin{equation}
                \begin{aligned}
                    \psi\colon \comC(f)&\to \eK[f] \\
                    g&\mapsto P\tq P(f)v=g(v).
                \end{aligned}
            \end{equation}
            C'est une application injective parce que si \( \psi(g)=0\) alors \( g(v)=0\) et pour tout \( k\) nous avons \( g\big( f^k(v) \big)=f^k\big( g(v) \big)=0\). L'endomorphisme \( g\) s'annulant sur une base, est nul.
        \item[\ref{ITEMooSOYYooZVibjrv} implique~\ref{ITEMooSOYYooZVibjriv}]
            Si \( n_1,\ldots, n_r\) sont les dimensions des différents espaces propres de \( f\), nous avons les inégalités
            \begin{equation}
                \dim\big( \eK[f] \big)=\deg(\mu)\leq n=n_1+\cdots +n_r\leq n_1^2+\cdots +n_r^2=\dim\big( \comC(f) \big).
            \end{equation}
            Par hypothèse d'égalité entre le premier et le dernier terme de cette suite d'inégalités, toutes les inégalités sont des égalités et en particulier \( \dim\big( \comC(f) \big)=n\).
        \end{subproof}
            Nous avons fini de prouver toutes les équivalences demandées.
\end{proof}

\begin{example}
    Pour mieux comprendre pourquoi le fait d'avoir \( n\) valeurs propres distinctes est équivalent à être cyclique, notons que si deux valeurs propres sont identiques, alors un morceau de la matrice de \( f\) serait par exemple \( \begin{pmatrix}
          2  &   0    \\
        0    &   2
    \end{pmatrix}\), et dans ce cas n'importe quelle combinaison \( ae_i+be_j\) reste proportionnelle à elle-même après application de \( f\). Si nous avons des valeurs propres différentes par contre, nous avons par exemples dans \( \eR^2\) la matrice \( \begin{pmatrix}
        1    &   0    \\
        0    &   2
    \end{pmatrix}\) qui donne \( f(e_1+e_2)=e_1+2e_2\). La partie \( \{ e_1+e_2,e_1+2e_2 \}\) est une base.
\end{example}

%---------------------------------------------------------------------------------------------------------------------------
\subsection{Commutant : cas général}
%---------------------------------------------------------------------------------------------------------------------------

Nous considérons encore un espace vectoriel \( E\) de dimension finie \( n\) et un endomorphisme \( f\colon E\to E\). Nous notons \( \mu\) sont polynôme minimal et \( \mu_x\) le polynôme minimal ponctuel en \( x\).

\begin{lemma}[\cite{ooEFHLooXpAOFz,AutourFrobCompa,ooEPEFooQiPESf}]       \label{LEMooDFFDooJTQkRu}
    Nous avons
    \begin{equation}
        \dim\big( \comC(f) \big)\geq \dim(E)
    \end{equation}
\end{lemma}

\begin{proof}
    Si \( f\) est donnée, l'espace \( \comC(f)\) est l'espace des solutions de \( fg=gf\). Supposons avoir choisi une base de \( E\) et notons \( A\) la matrice de \( f\) et \( X\) celle de \( g\). L'équation est \( AX-XA=0\).
    \begin{subproof}
        \item[Si \( A\) est trigonalisable]
            Nous supposons avoir choisi la base de telle sorte que \( A\) soit triangulaire supérieure, et nous allons nous contenter de chercher les solutions \( X\) qui sont également triangulaires supérieure. Si il y en a déjà plus que \( n\), a fortiori le résultat sera vrai.

            Le produit de deux matrices triangulaires supérieures étant une matrice triangulaire supérieure, l'équation \( AX-XA\) contient, pour les coefficients de \( X\), \( n(n+1)/2\) équations. Mais il se fait que les termes diagonaux ne sont pas de vraies équations parce que
            \begin{equation}
                (AX-XA)_{kk}=\sum_i\big( A_{ki}X_{ik}-X_{ki}A_{ik} \big)=\sum_{k\leq i\leq k}(A_{ki}X_{ik}-X_{ki}A_{ik})=0.
            \end{equation}
            Nous avons donc au maximum
            \begin{equation}
                \frac{ n(x+1) }{2}-n
            \end{equation}
            équations linéairement indépendantes pour un minimum de \( n(n+1)/2\) inconnues. L'espace des solutions est donc de dimension au minimum \( n\).

            Cela a l'air d'être une majoration assez large, mais il existe des cas d'égalité.

        \item[Si \( A\) n'est pas trigonalisable]

            La preuve que nous donnons ici est valable même pour les endomorphismes trigonalisables.

            Nous considérons le résultat de Frobenius~\ref{THOooDOWUooOzxzxm}. Nous avons donc la structure suivante:
            \begin{itemize}
                \item
            une décomposition en somme directe \( E=E_1\oplus\ldots\oplus E_r\),
        \item
            les espaces \( E_i\) sont fixés par \( f\),
        \item
            les endomorphismes \( f_i=f|_{E_i}\) sont cycliques
        \item
            le polynôme minimal de \( f_i\) est \( \mu_i\) et \( \prod_{i=1}^r\mu_i=\chi_f\).
            \end{itemize}
            Les endomorphismes \( f_i^k\) commutent évidemment avec \( f_j\), et la partie \( \{ f_i^k \}_{k=0,\ldots, \deg(\mu_i)-1}\) est libre. Libre en tout cas en tant que partie de \( \End(E_i)\). Mais en prolongeant par \( 0\) sur \( E\), ça reste libre en tant que partie de \( \End(E)\).

            Bien entendu les \( f_j^k\) et les \( f_i^k\) (\( i\neq j\)) sont linéairement indépendants dans \( \End(E)\) parce qu'ils n'agissent pas sur les mêmes vecteurs. Donc les endomorphismes \( f_i^{k_i}\) avec \( k_i=0,\ldots, \deg(\mu_i)-1\) forment une partie libre de \( \End(E)\) composée d'endomorphismes qui commutent avec \( f\). Il y en a en tout
            \begin{equation}
                \sum_{i=1}^r\deg(\mu_i)=\deg(\chi_f)=n.
            \end{equation}
            Par conséquent \( \dim\big( \comC(f) \big)\geq \dim(E)\).
    \end{subproof}
\end{proof}

\begin{theorem}[\cite{ooRJDSooXpVtMD}]      \label{THOooGLMSooYewNxW}
    Soit un endomorphisme \( f\colon E\to E\) sur l'espace vectoriel de dimension finie \( n\). Nous notons \( \mu\) et \( \chi\) les polynômes minimal et caractéristique. Nous avons équivalence entre les faits suivants :
    \begin{enumerate}
        \item   \label{ITEMooLRXIooLWaYqJii}
            \( \mu=\chi\),
        \item   \label{ITEMooLRXIooLWaYqJi}
            \( f\) est cyclique,
        \item   \label{ITEMooLRXIooLWaYqJiii}
            \( \comC(f)=\eK[f]\).
    \end{enumerate}
\end{theorem}

\begin{proof}
    Plusieurs implications. Notons que~\ref{ITEMooLRXIooLWaYqJii} implique~\ref{ITEMooLRXIooLWaYqJii} a déjà été démontré par le lemme~\ref{LEMooKUQDooKFeIYq}.
    \begin{subproof}
        \item[\ref{ITEMooLRXIooLWaYqJii} implique~\ref{ITEMooLRXIooLWaYqJi}]
            Conformément à ce que nous permet le lemme~\ref{LemSYsJJj} nous choisissons\footnote{Dans toute la suite, nous devrions écrire \( \mu_f\) et \( \mu_{f,a}\) mais nous omettons d'indiquer explicitement la dépendance en \( f\).} \( a\in E\) de telle sorte à avoir \( \mu_a=\mu\). De plus pour \( x\in E\) nous considérons l'application
            \begin{equation}
                \begin{aligned}
                    \varphi_x\colon \eK[X]&\to E \\
                    P&\mapsto P(f)x.
                \end{aligned}
            \end{equation}
            Nous avons \( \varphi_a(P)=P(f)a\). Étant donné que\( E_{a}\) est engendré par les \( f^k(a)\) nous avons \( \varphi_a\big( \eK[X] \big)=E_a\). De plus l'application \( \varphi_a\) passe aux classes pour \( (\mu_a)\). Pour rappel, un élément de \( \eK[X]/(\mu_a)\) est de la forme
            \begin{equation}
                \bar P=\{ P+Q\mu_a \}_{Q\in \eK[X]}.
            \end{equation}
            Nous considérons donc l'application
            \begin{equation}
                \varphi_a\colon \frac{ \eK[X] }{ (\mu_a) }\to E_a
            \end{equation}
            et nous prouvons que c'est un isomorphisme d'espace vectoriel.
            \begin{subproof}
                \item[Linéaire]
                    Parce que \( (\lambda P+Q)(f)=(\lambda P)(f)+Q(f)\).
                \item[Injectif]
                    Si \( \varphi_a(\bar P)=0\) alors \( \varphi_a(P)=0\) (dans la deuxième, \( \varphi_a\) est l'application définie sur les polynômes et non sur les classes), ce qui montrer que \( P\) est annulateur de \( a\). Mais par définition~\ref{DEFooUICRooBGYhqQ} du polynôme minimal ponctuel, \( \mu_a\) est générateur de \( \ker(\varphi_a)\); donc il existe \( Q\in \eK[X]\) tel que \( P=Q\mu_a\). En d'autres termes, du point de vue du quotient, \( \bar P=0\).
                \item[Surjectif]
                    Si \( x\in E_a\) alors il existe des coefficients \( x_k\in \eK\) tels que \( x=\sum_{k=0}^{\deg(\mu_a)-1}x_kf^k(a)\), c'est-à-dire \( x=P(f)a=\varphi_a(P)\).
            \end{subproof}
            Mais par hypothèse et par choix de \( a\) nous avons \( \mu_a=\mu=\chi\), donc en fait \( E_a=\eK[X]/(\chi)\). Mais nous savons que \( \deg(\chi)=\dim(E)\) et que \( \dim\big( \eK[X]/P \big)=\deg(P)\) par la proposition~\ref{PropooCFZDooROVlaA}. Au final nous avons \( \dim(E_a)=\deg(\chi)=\dim(E)\). Et par conséquent \( E_a=E\). Cela prouver que \( a\) est un vecteur cyclique pour \( f\).

        \item[\ref{ITEMooLRXIooLWaYqJi} implique~\ref{ITEMooLRXIooLWaYqJiii}]
            Soit \( g\in \comC(f)\); nous devons prouver que \( g\) est un polynôme de \( f\). Par hypothèse nous avons un vecteur cyclique que nous notons \( v\). Nous avons un polynôme \( P\) (dépendant de \( g\)) tel que \( g(v)=P(f)v\). Nous allons voir que \( g=P(f)\). Soient \( y\in E\) et \( Q\) un polynôme tels que \( y=Q(f)v\); en notant que \( g\) commute avec \( P(f)\) nous avons
            \begin{equation}
                g(y)=g\big( Q(f)v \big)=Q(f)\big( g(v) \big)=Q(f)\big( P(f)v \big)=P(f)Q(f)v=P(f)y.
            \end{equation}
            Donc \( g=P(f)\).

        \item[\ref{ITEMooLRXIooLWaYqJiii} implique~\ref{ITEMooLRXIooLWaYqJii}]

            Nous avons les inégalités :
            \begin{equation}
                n\leq \dim\big( \comC(f) \big)=\dim\big( \eK[f] \big)=\deg(\mu)\leq \deg(\chi)=n.
            \end{equation}
            La première est le lemme~\ref{LEMooDFFDooJTQkRu}. Toutes les inégalités sont des égalités. En particulier \( \deg(\mu)=n\), ce qui signifie que \( \mu=\chi\) parce que \( \mu\) est un polynôme diviseur de \( \chi\), de même degré que \( \chi\) et unitaire tout comme \( \chi\).

    \end{subproof}
\end{proof}

\begin{corollary}[\cite{ooEPEFooQiPESf}]        \label{CORooAKQEooSliXPp}
    En suivant les notations sur les extensions de corps de base de la section~\ref{SECooAUOWooNdYTZf}, l'endomorphisme \( f\colon E\to F\) est cyclique si et seulement si l'endomorphisme \( f_{\eL}\colon E_{\eL}\to F_{\eL}\) est cyclique.
\end{corollary}

\begin{proof}
    Nous savons par le théorème~\ref{THOooGLMSooYewNxW} qu'un endomorphisme est cyclique si et seulement si son polynôme minimal est égal à son polynôme caractéristique. Or par les propositions~\ref{PROPooZAZFooUFdCUv} et~\ref{PROPooXVZMooXcJrsJ}, nous savons que ces polynômes sont identiques pour \( f\) et pour \( f_{\eL}\).
\end{proof}

\begin{theorem}[Similitude et extension de corps\cite{ooEPEFooQiPESf}]      \label{THOooHUFBooReKZWG}
    Les applications linéaires \( f,g\colon E\to E\) sont semblables si et seulement si \( f_{\eL}\) et \( g_{\eL}\) le sont.
\end{theorem}

\begin{proof}
    En ce qui concerne le sens direct, si il existe \( m\in\GL(E)\) tel que \( f=mgm^{-1}\) alors il suffit d'appliquer le lemme~\ref{LEMooWZGSooONEnjZ} pour avoir \( f_{\eL}=m_{\eL}g_{\eL}m_{\eL}^{-1}\).

    Nous considérons les invariants de similitude de \( f\) du théorème~\ref{THOooDOWUooOzxzxm}. Il existe une unique suite de polynômes unitaires \( \mu_i\) ($i=1,\ldots, s$) tels que \( \mu_i\divides \mu_{i+1}\) et pour laquelle nous avons une décomposition \( E=E_1\oplus \ldots\oplus E_s\) pour laquelle \( f|_{E_i}\colon E_1\to E_i\) est cyclique et de polynôme minimal \( \mu\).

    Nous avons aussi \( E_{\eL}=(E_1)_{\eL}\oplus\ldots \oplus (E_s)_{\eL}\) et les \( (E_i)_{\eL}\) sont stables sous \( f_{\eL}\) qui y sera également cyclique (corolaire ~\ref{CORooAKQEooSliXPp}). De plus le polynôme minimal de \( f_{\eL}|_{(E_i)_{\eL}}\) est également \( \mu_i\).

    Autrement dit, la suite \( \mu_i\) est également la suite des invariants de similitude de \( f_{\eL}\). La remarque~\ref{REMooPVLEooYDRXQI} nous permet de conclure que \( f\) et \( g\) sont semblables si et seulement si \( f_{\eL}\) et \( g_{\eL}\) le sont.
\end{proof}
