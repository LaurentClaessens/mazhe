% This is part of Mes notes de mathématique
% Copyright (c) 2011-2018, 2020, 2022-2023
%   Laurent Claessens, Carlotta Donadello
% See the file fdl-1.3.txt for copying conditions.

%+++++++++++++++++++++++++++++++++++++++++++++++++++++++++++++++++++++++++++++++++++++++++++++++++++++++++++++++++++++++++++
\section{Applications mesurables}
%+++++++++++++++++++++++++++++++++++++++++++++++++++++++++++++++++++++++++++++++++++++++++++++++++++++++++++++++++++++++++++

%---------------------------------------------------------------------------------------------------------------------------
\subsection{Propriétés}
%---------------------------------------------------------------------------------------------------------------------------

\begin{definition}[Fonction mesurable] \label{DefQKjDSeC}
	Soient \( (E,\tribA)\) et \( (F,\tribF)\) deux espaces mesurés. Une fonction \( f\colon E\to F\) est \defe{mesurable}{mesurable!fonction} si pour tout \( \mO\in \tribF\), l'ensemble \( f^{-1}(\mO)\) est dans \( \tribA\).
\end{definition}

\begin{proposition}     \label{PROPooEFHKooARJBwW}
	Soient \( (S_i,\tribF_i)\) (\( i=1,2,3\)) des espaces mesurables et des fonctions mesurables \( f\colon S_1\to S_2\) et \( g\colon S_2\to S_3\). Alors la fonction \( g\circ f\colon S_1\to S_3\) est mesurable.
\end{proposition}

\begin{proof}
	Soit \( B\in\tribF_3\). Alors
	\begin{equation}
		(g\circ f)^{-1}(B)=f^{-1}\big( g^{-1}(B) \big)\in f^{-1}(\tribF_2)\subset\tribF_1.
	\end{equation}
\end{proof}

%---------------------------------------------------------------------------------------------------------------------------
\subsection{D'une tribu à l'autre}
%---------------------------------------------------------------------------------------------------------------------------

\begin{lemma}[\cite{TribuLi}]       \label{LemooVDXJooZNYelH}
	Soit une application \( f\colon S_1\to S_2\) et une tribu \( \tribF_2\) sur \( S_2\). Alors \( f^{-1}(\tribF_2)\) est une tribu sur \( S_1\)
\end{lemma}

\begin{proof}
	Il faut prouver les trois propriétés de la définition~\ref{DefjRsGSy} d'une tribu.
	\begin{enumerate}
		\item
		      D'abord \( f\) est définie sur tout \( S_1\), donc \( f^{-1}(S_2)=S_1\) alors que \( S_2\in \tribF_2\).
		\item
		      Soit \( A\in f^{-1}(\tribF_2)\), c'est-à-dire \( A=f^{-1}(B)\) pour un certain \( B\in \tribF_2\). En ce qui concerne le complémentaire :
		      \begin{equation}
			      A^c=f^{-1}(B)^c=S_1\setminus f^{-1}(B)=f^{-1}(S_2\setminus B)=f^{-1}(B^c).
		      \end{equation}
		\item
		      Si \( (A_i)_{i\in \eN}\) sont des éléments de \( f^{-1}(\tribF_2)\) avec \( A_i=f^{-1}(B_i)\) alors
		      \begin{equation}
			      \bigcup_iA_i=\bigcup_if^{-1}(B_i)=f^{-1}\big( \bigcup_iB_i \big).
		      \end{equation}
		      Ce qui est dans la dernière parenthèse est dans \( \tribF_2\) parce que cette dernière est une tribu.
	\end{enumerate}
\end{proof}


Le lemme suivant est également nommé «lemme de transfert».
\begin{lemma}[Lemme de transport]       \label{LemOQTBooWGYuDU}
	Soit \( f\colon S_1\to S_2\) une application et une classe \( \tribC\) de parties de \( S_2\). Alors
	\begin{equation}
		\sigma\big( f^{-1}(\tribC) \big)=f^{-1}\big( \sigma(\tribC) \big).
	\end{equation}
\end{lemma}
\index{lemme!de transport}

\begin{proof}
	Puisque \( \sigma(\tribC)\) est une tribu dans \( S_2\) alors le lemme~\ref{LemQYUJwPC} dit que \( f^{-1}\big( \sigma(\tribC) \big)\) est une tribu qui contient en particulier \(  f^{-1}(\tribC) \). Nous en déduisons que \( \sigma\big( f^{-1}(\tribC) \big)\subset f^{-1}\big( \sigma(\tribC) \big)\).

	Réciproquement. Dans \( S_1\) nous avons la tribu \( \sigma\big( f^{-1}(\tribC) \big)\). Nous pouvons alors considérer la tribu
	\begin{equation}
		\tribF_f=\{ B\subset S_2\tq f^{-1}(B)\in\sigma\big( f^{-1}(\tribC) \big) \}.
	\end{equation}
	Montrons que \( \tribC\subset \tribF_f\). Lorsque \( B\in \tribC\) nous avons \( f^{-1}(B)\in f^{-1}(\tribC)\subset\sigma\big( f^{-1}(\tribC) \big)\). Du coup \( B\in \tribF_f\). Nous avons alors, en passant aux tribus engendrées :
	\begin{equation}
		\sigma(\tribC)\subset\sigma(\tribF_f)=\tribF_f.
	\end{equation}
	Si maintenant \( B\in\sigma(\tribC)\), nous avons \( f^{-1}(B)\in \sigma\big( f^{-1}(\tribC) \big)\), ce qui signifie que
	\begin{equation}
		f^{-1}\big( \sigma(\tribC) \big)\subset\sigma\big( f^{-1}(\tribC) \big).
	\end{equation}
\end{proof}

Le théorème suivant est important pour prouver qu'une application est mesurable. En effet, il permet de ne tester si une application n'est mesurable uniquement que sur une partie génératrice de la tribu d'arrivée\footnote{Typiquement les ouverts pour les boréliens.}.
\begin{theorem}     \label{ThoECVAooDUxZrE}
	Soient des espaces mesurables \( ( S_1,\tribF_1 )\) et \( (S_2,\tribF_2)\) ainsi qu'une application \( f\colon S_1\to S_2\). Si il existe un ensemble de parties \( \tribC\) de \( S_2\) tel que
	\begin{itemize}
		\item \( \sigma(\tribC)=\tribF_2\)
		\item \( f^{-1}(B) \in \tribF_1 \) pour tout \( B\in \tribC\)
	\end{itemize}
	alors \( f\) est mesurable.
\end{theorem}

\begin{proof}
	Par hypothèse, \( \sigma(\tribC)=\tribF_2\) et \( f^{-1}(\tribC)\subset \tribF_1\) et nous pouvons utiliser le lemme de transfert~\ref{LemOQTBooWGYuDU} :
	\begin{equation}
		\sigma\big( f^{-1}(\tribC) \big)=f^{-1}\big( \sigma(\tribC) \big)
	\end{equation}
	qui s'écrit ici
	\begin{equation}
		\sigma\big( f^{-1}(\tribC) \big)=f^{-1}(\tribF_2).
	\end{equation}
	Mais comme \( f^{-1}(\tribC)\subset \tribF_1\), nous avons aussi \( \sigma\big( f^{-1}(\tribC) \big)\subset \tribF_1\), ce qui signifie que
	\begin{equation}
		f^{-1}(\tribF_2)\subset \tribF_1.
	\end{equation}
	Cela est exactement le fait que \( f\) soit mesurable.
\end{proof}


%+++++++++++++++++++++++++++++++++++++++++++++++++++++++++++++++++++++++++++++++++++++++++++++++++++++++++++++++++++++++++++
\section{Tribu borélienne}
%+++++++++++++++++++++++++++++++++++++++++++++++++++++++++++++++++++++++++++++++++++++++++++++++++++++++++++++++++++++++++++

%///////////////////////////////////////////////////////////////////////////////////////////////////////////////////////////
\subsubsection{Définition}
%///////////////////////////////////////////////////////////////////////////////////////////////////////////////////////////

\begin{definition}[Tribu borélienne]        \label{DEFooQBQGooTqGdtY}
	La tribu des \defe{boréliens}{boréliens}\index{tribu!borélienne}, notée \( \Borelien(\eR^d)\) est la tribu engendrée par les ouverts de \( \eR^d\). Plus généralement si \( Y\) est un espace topologique, la tribu des boréliens est la tribu engendrée par les ouverts de \( Y\).
\end{definition}
\index{borélienne!tribu}

\begin{proposition} \label{PROPooYEkvbWBz}
	La tribu engendrée par une base dénombrable de la topologie est celle des boréliens.
\end{proposition}

\begin{proof}
	Si une base de topologie est donnée, tout ouvert peut être écrit comme union d'élément de la base, proposition~\ref{DEFooLEHPooIlNmpi}. Dans le cas d'une base dénombrable, cette union sera forcément dénombrable. Une tribu étant stable par union dénombrable, tout ouvert est dans la tribu engendrée par la base de topologie. Les autres boréliens suivent automatiquement.

	Dit avec plus de lettres et moins de phrases, si \( \tribD\) est une base dénombrable de la topologie de \( X\), et si \( \mO\) est un ouvert de \( X\), nous avons \( \mO=\bigcup_{i=1}^{\infty}A_i\) avec \( A_i\in\tribD\). Puisqu'une tribu est stable par union dénombrable\footnote{Définition~\ref{DefjRsGSy}\ref{ItemooPEQNooYiYNtN}}, nous avons \( \mO\in\sigma(\tribD)\). En conséquence, \( \Borelien(X)\subset\sigma(\tribD)\).

	Mais comme \( \tribD\subset\Borelien(X)\) l'inclusion inverse est automatique. D'où l'égalité \( \Borelien(X)=\sigma(\tribD)\).
\end{proof}

%///////////////////////////////////////////////////////////////////////////////////////////////////////////////////////////
\subsubsection{Les boréliens de \texorpdfstring{\(  \eR\)}{R}}
%///////////////////////////////////////////////////////////////////////////////////////////////////////////////////////////

Nous rappelons que la topologie de \( \eR\) est celle des boules donnée par le théorème~\ref{ThoORdLYUu}. Nous rappelons (voir la proposition~\ref{PropNBSooraAFr} et sa preuve) que les boules ouvertes de la forme \( B(q,r)\) avec \( q,r\in \eQ\) forment une base dénombrable de la topologie de \( \eR\).

\begin{lemma}   \label{LemZXnAbtl}
	Soit \( \{ q_i \}\) une énumération des rationnels. La tribu engendrée par les ouverts \( \sigma_i=\mathopen] q_i , \infty \mathclose[\) est la tribu des boréliens.
\end{lemma}

\begin{proof}
	Si \( a<b\) dans \( \eQ\) alors \( \sigma_a\setminus\sigma_b=\mathopen] a , b \mathclose]\). Ensuite
	\begin{equation}
		\bigcup_{n\in \eN^*}\sigma_a\setminus\sigma_{b-\frac{1}{ n }}=\bigcup_{n\in \eN^*}\mathopen] a , b-\frac{1}{ n } \mathclose]=\mathopen] a , b \mathclose[.
	\end{equation}
	Par union dénombrable, tous les intervalles \( \mathopen] a , b \mathclose[\) avec \( a,b\in \eQ\) sont dans la tribu engendrée par les \( \sigma_i\). Ces boules ouvertes forment une base de la topologie de \( \eR\) par la proposition~\ref{PropNBSooraAFr} et la proposition~\ref{PROPooYEkvbWBz} conclut.
\end{proof}

\begin{example}
	Les singletons sont des boréliens de \( \eR\) parce que
	\begin{equation}
		\{ x \}=\Big( \mathopen] -\infty , x \mathclose[\cup\mathopen] x , +\infty \mathclose[ \Big)^c.
	\end{equation}

	Puisqu'une tribu est stable par union dénombrable, l'ensemble \( \eQ\) est un borélien de \( \eR\). Et comme les tribus sont stables par différence ensembliste (\ref{LemBWNlKfA}\ref{ItemXQVLooFGBQNj}), l'ensemble des irrationnels est un borélien de \( \eR\).
\end{example}

%///////////////////////////////////////////////////////////////////////////////////////////////////////////////////////////
\subsubsection{Diverses expressions}
%///////////////////////////////////////////////////////////////////////////////////////////////////////////////////////////

\begin{lemma}   \label{LEMooUPYDooPVjscA}
	Soient un espace topologique \( X\) et un borélien \( B\) de \( X\). Nous considérons sur \( B\) la topologie induite\footnote{Définition \ref{DefVLrgWDB}.} de \( X\) et les boréliens \( \Borelien(B)\) correspondants. Nous avons :
	\begin{equation}
		\Borelien(B)=\{ A\in \Borelien(X)\tq A\subset B \}=\{ B\cap A\tq A\in \Borelien(X) \}.
	\end{equation}
	En particulier,
	\begin{equation}    \label{EQooEUWVooCBUims}
		\Borelien(B)=\Borelien(X)_B.
	\end{equation}
\end{lemma}

\begin{proof}
	L'égalité
	\begin{equation}
		\{ A\in \Borelien(X)\tq A\subset B \}=\{ B\cap A\tq A\in \Borelien(X) \}
	\end{equation}
	est déjà dans la proposition \ref{PROPooUNNSooMUQKfp}.

	Nous démontrons maintenant que
	\begin{equation}
		\Borelien(B)=\{ A\cap B\tq A\in\Borelien(X) \}.
	\end{equation}
	Pour ce faire, nous nous rappelons du lemme de transport \ref{LemOQTBooWGYuDU}. Soit l'injection canonique \( f\colon B\to X\); pour tout \( A\subset X\) nous avons \( f^{-1}(A)=A\cap B\).

	Nous considérons la classe \( \tribT\) des ouverts de \( X\). Par définition de la topologie induite, les ouverts de \( B\) sont les éléments de \( f^{-1}(\tribT)\). Donc
	\begin{equation}
		\sigma\big( f^{-1}(\tribT) \big)=\Borelien(B).
	\end{equation}
	Mais d'autre part,
	\begin{equation}
		f^{-1}\big( \sigma(\tribT) \big)=\{ A\cap B\tq A\in \Borelien(X) \}.
	\end{equation}
	Donc le lemme de transport \ref{LemOQTBooWGYuDU} nous dit que
	\begin{equation}
		\Borelien(B)=\sigma\big( f^{-1}(\tribT) \big)= f^{-1}\big( \sigma(\tribT) \big)=\{ A\cap B\tq A\in \Borelien(X) \}.
	\end{equation}

	Pour finir, l'égalité \eqref{EQooEUWVooCBUims} se démontre :
	\begin{equation}
		\Borelien(X)_B=\{ B\cap A\tq A\in \Borelien(X) \}=\Borelien(B).
	\end{equation}
\end{proof}

%--------------------------------------------------------------------------------------------------------------------------- 
\subsection{Applications continues et boréliennes}
%---------------------------------------------------------------------------------------------------------------------------

\begin{definition}[Fonction borélienne]     \label{DefHHIBooNrpQjs}
	Une application \( f\colon (\Omega,\tribA)\to (\eR^d,\Borelien(\eR^d))\)\footnote{Tribu des boréliens, définition \ref{DEFooQBQGooTqGdtY}.} est \defe{borélienne}{borélienne!fonction}\index{fonction!borélienne} si elle est mesurable, c'est-à-dire si pour tout \( B\in\Borelien(\eR^d)\) nous avons \( f^{-1}(B)\in\tribA\).

	Si rien n'est précisé, une application entre deux espaces topologiques est borélienne lorsqu'elle est mesurable en considérant la tribu borélienne sur \emph{les deux} espaces.
\end{definition}
Si \( \tribA\) est une tribu sur un ensemble \( E\), nous notons \( m(\tribA)\)\nomenclature[P]{\( m(\tribA)\)}{Ensemble des fonctions \( \tribA\)-mesurables} l'ensemble des fonctions qui sont \( \tribA\)-mesurables.

Le plus souvent lorsque nous parlerons de fonctions \( f\colon X\to Y\) où \( Y\) est un espace topologique, nous considérons la tribu borélienne sur \( Y\). Ce sera en particulier le cas dans la théorie de l'intégration.

Le théorème suivant est très important parce qu'en pratique c'est souvent lui, en conjonction avec la proposition~\ref{PropooLNBHooBHAWiD} qui permet de déduire qu'une fonction est borélienne.
\begin{theorem}[\cite{TribuLi}]     \label{ThoJDOKooKaaiJh}
	Soient \( X\) et \( Y\) deux espaces topologiques. Alors toute application continue \( f\colon X\to Y\) est borélienne\footnote{Définition~\ref{DefHHIBooNrpQjs}.}.
\end{theorem}

\begin{proof}
	Pour vérifier que \( f\) est borélienne, nous devons prouver que \( f^{-1}(B)\) est borélien pour tout borélien \( B\) de \( Y\). Heureusement, le théorème~\ref{ThoECVAooDUxZrE} nous permet de limiter la vérification aux \( B\) appartenant à une classe engendrant les boréliens de \( Y\).

	La classe en question est toute trouvée : ce sont les ouverts. Si \( \mO\) est un ouvert de \( Y\) alors \( f^{-1}(\mO)\) est un ouvert de \( X\) et donc un borélien de \( X\).
\end{proof}

Le théorème suivant donne une importante compatibilité entre l'induction de tribu et l'induction de topologie : la tribu induite à partir des boréliens sur un sous-espace topologique est la tribu des boréliens pour la topologie induite.
\begin{theorem}[\cite{TribuLi}]     \label{ThoSVTHooChgvYa}
	Soit \( X\), un espace topologique et \( Y\subset X\) une partie munie de la topologie induite. Alors
	\begin{equation}
		\Borelien(Y)=\Borelien(X)_Y
	\end{equation}
	où \( \Borelien(X)_Y\) est la tribu sur \( Y\) induite de \( \Borelien(X)\) par la définition~\ref{DefDHTTooWNoKDP}.
\end{theorem}

\begin{proof}
	Nous notons \( \tau_X\) et \( \tau_Y\) les topologies de \( X\) et \( Y\).
	\begin{subproof}
		\spitem[\( \Borelien(Y)\subset\Borelien(X)_Y\)]
		Si \( A\in \tau_Y\) alors \( A=Y\cap \Omega\) pour un \( \Omega\in \tau_X\). Mais puisque \(\Omega\) est un ouvert de \( X\), il est un borélien de \( X\), ce qui donne que \( Y\cap\Omega\) est un élément de \( \Borelien(X)_Y\). Cela prouve que \( \tau_Y\subset\Borelien(X)_Y\), c'est-à-dire que \( \Borelien(X)_Y\) est une tribu sur \( Y\) contenant les ouverts de \( Y\). Nous avons donc
		\begin{equation}
			\Borelien(X)\subset\Borelien(X)_Y.
		\end{equation}
		\spitem[Réciproquement]
		L'application \( \id\colon (Y,\tau_Y)\to (X,\tau_X)\) est continue parce que si \( \Omega\) est ouvert de \( X\) alors \( \id^{-1}(\Omega)=\Omega\cap Y\in \tau_Y\). Par conséquent l'identité est une application borélienne (théorème~\ref{ThoJDOKooKaaiJh}), ce qui signifie que \( \id^{-1}\big( \Borelien(X) \big)\subset\Borelien(Y)\), ou encore que si \( B\in\Borelien(X)\), alors \( \id^{-1}(B)=B\cap Y\in\Borelien(Y)\). Cela signifie que
		\begin{equation}
			\Borelien(X)_Y\subset \Borelien(Y).
		\end{equation}
	\end{subproof}
\end{proof}

\begin{corollary}       \label{CorooMJQYooFfwoTd}
	Si \( U\) est un borélien de l'espace topologique \( X\), alors les boréliens de \( U\) sont les boréliens de \( X\) inclus dans \( U\) :
	\begin{equation}
		\Borelien(U)=\{ B\in\Borelien(X)\tq B\subset U \}.
	\end{equation}
\end{corollary}

\begin{proof}
	Si \( B'\in\Borelien(U)\), le théorème~\ref{ThoSVTHooChgvYa} donne un borélien \( B\in\Borelien(X)\) tel que \( B'=B\cap U\). Mais \( U\) étant borélien de \( X\), l'intersection \( B\cap U\) est encore un borélien de \( X\).
\end{proof}
Ce corolaire s'applique en particulier lorsque \( U\) est un ouvert.

La proposition suivante montre comment il est possible de construire un espace mesuré à partir d'une bijection avec un espace mesuré déjà connu. Attention cependant : la mesure construite dans cette proposition n'est pas celle qui est le plus adapté. Voir la proposition \ref{PROPooILOEooBiumKD} et le blabla \ref{REMooOMYYooNFiKOs}.
\begin{proposition}     \label{PROPooXQHTooUxJoyq}
	Soient un espace mesuré \( (\Omega,\tribA,\mu)\), un ensemble \( \Omega'\) et une bijection \( \varphi\colon \Omega\to \Omega'\). Nous posons
	\begin{enumerate}
		\item
		      \( \tribA'=\varphi(\tribA)\),
		\item
		      \( \mu'(B)=\mu\big( \varphi^{-1}(B) \big)\) pour tout \( B\in\tribA'\).
	\end{enumerate}
	Alors \( (\Omega',\tribA',\mu')\) est un espace mesuré.
\end{proposition}

\begin{proof}
	En plusieurs points.
	\begin{subproof}
		\spitem[\( \tribA'\) est une tribu]
		Il faut vérifier les différents points de la définition \ref{DefjRsGSy}. D'abord, puisque \( \Omega\in\tribA\), nous avons \( \Omega'=\varphi(\Omega)\in \tribA'\). Pour le complémentaire, si \( B\in\tribA'\) alors \( B=\varphi(A)\) pour un certain \( A\in \tribA\). Comme \( \tribA\) est une tribu nous avons alors \( \Omega\setminus A\in\tribA\) et donc \( \varphi(\Omega\setminus A)\in \tribA'\). Mais comme \( \varphi\) est bijective,
		\begin{equation}
			\varphi(\Omega\setminus A)=\Omega'\setminus\varphi(A)=\Omega'\setminus B.
		\end{equation}
		Le complémentaire de \( B\) est donc bien dans \( \tribA'\). Pour la troisième condition, soient \( B_i\in\tribA'\). Pour chaque \( i\), il existe \( A_i\in \tribA\) tel que \( B_i=\varphi(A_i)\). Nous avons \( \bigcup_iA_i\in \tribA\), donc
		\begin{equation}
			\bigcup_iB_i=\bigcup_i\varphi(A_i)=\varphi\big( \bigcup_iA_i \big)\in \tribA'.
		\end{equation}
		Nous avons fini de prouver que \( (\Omega',\tribA')\) était un espace mesurable.
		\spitem[\( \mu'\) est une mesure positive]
		D'abord \( \mu'(\emptyset)=\mu\big( \varphi^{-1}(\emptyset) \big)=\mu(\emptyset)=0\). Ensuite si les \( A_i\) sont disjoints dans \( \tribA'\) nous avons
		\begin{equation}
			\mu'\big( \bigcup_{i=0}^{\infty}A_i \big)=\mu\left( \varphi^{-1}\big( \bigcup_{i=0}^{\infty}A_i \big) \right)=\mu\left( \bigcup_i\varphi^{-1}(A_i) \right)=\sum_{i=0}^{\infty}\mu\big( \varphi^{-1}(A_i) \big)=\sum_i\mu'(A_i).
		\end{equation}
	\end{subproof}
\end{proof}

\begin{proposition}     \label{PROPooIKYYooCFaDaI}
	Soit une bijection continue d'inverse continue \( \varphi\colon \Omega\to \Omega'\). Alors
	\begin{equation}
		\varphi\big( \Borelien(\Omega) \big)=\Borelien(\Omega').
	\end{equation}
\end{proposition}

\begin{proof}
	Si \( A\in \Borelien(\Omega')\), alors \( A=\varphi\big( \varphi^{-1}(A) \big)\in\varphi\big( \Borelien(\Omega) \big)\) parce que \( \varphi\) est continue et donc borélienne (proposition \ref{ThoJDOKooKaaiJh}). Le même raisonnement fonctionne dans l'autre sens parce que nous avons supposé que \( \varphi\) est continue et d'inverse continu.
\end{proof}

%---------------------------------------------------------------------------------------------------------------------------
\subsection{Tribu de Baire}
%---------------------------------------------------------------------------------------------------------------------------

\begin{definition}
	Une partie d'un espace topologique est \defe{rare}{rare} si elle est contenue dans un fermé d'intérieur vide.

	Une partie est \defe{maigre}{maigre} si elle est réunion finie ou dénombrable de parties rares.
\end{definition}

\begin{example}
	L'ensemble \( \eQ\) dans \( \eR\) est maigre mais n'est pas rare parce que \( \bar\eQ=\eR\).
\end{example}

\begin{proposition}[\cite{SFYoobgQUp}]      \label{PROPooCHTWooZFiSMf}
	Soit \( X\) un espace topologique. L'ensemble de parties\footnote{Pour rappel, la tribu borélienne est définie en \ref{DEFooQBQGooTqGdtY}.}
	\begin{equation}
		\Baire(X)=\{ B\cup A\text{ avec } B\text{ borélien et } A\text{ maigre} \}
	\end{equation}
	est une tribu. Elle est appelée la \defe{tribu de Baire}{tribu!de Baire}\index{Baire!tribu} de l'espace \( X\).
\end{proposition}

\begin{proof}
	Nous allons montrer que les boréliens et les maigres vérifient les conditions de la proposition~\ref{PropHYLooLgOCy}.
	\begin{enumerate}
		\item
		      Si \( A\) est maigre, il s'écrit comme \( A=\bigcup_{i\in \eN}R_i\) où les \( R_i\) sont rares. Il existe donc des fermés d'intérieur vide \( F_i\) tels que \( R_i\subset F_i\); en particulier \( A\subset\bigcup_i F_i\). En tant que fermés, \( F_i\in\Borelien(X)\); de plus chaque \( F_i\) est rare, donc \( \bigcup_iF_i\) est maigre. L'ensemble \( A\) est donc bien contenu dans un ensemble maigre et borélien.
		\item
		      Soit \( A\) maigre et \( B\subset A\). Nous avons, avec les mêmes notations, \( A=\bigcup_iR_i\) et \( B=\bigcup_i(R_i\cap B)\). Les ensembles \( R_i\cap B\) sont encore rares, donc \( B\) est une union dénombrable d'ensembles rares. L'ensemble \( B\) est donc maigre.
		\item
		      Si les ensembles \( (A_i)\) sont maigres, alors ils sont unions dénombrables de rares : \( A_i=\bigcup_kR_k^{(i)}\). Nous avons alors
		      \begin{equation}
			      \bigcup_iA_i=\bigcup_{(i,k)\in \eN^2}R_k^{(i)},
		      \end{equation}
		      et donc \( \bigcup_iA_i\) est encore une union dénombrable d'ensembles rares.
	\end{enumerate}
\end{proof}

\begin{proposition}[\cite{SFYoobgQUp}]  \label{PropGRHootvAWq}
	Une partie \( B\) de l'espace topologique \( X\) est dans la tribu de Baire de \( X\) si et seulement si il existe un ouvert \( U\) tel que \( B\Delta U\) est maigre.
\end{proposition}

\begin{proof}
	Nous définissons la relation d'équivalence\footnote{Définition~\ref{DefHoJzMp}} suivante sur \( \partP(X)\) : nous disons que \( A\sim B\) si et seulement si \( A\Delta B\) est maigre.
	\begin{subproof}
		\spitem[Réflexive]
		Nous avons \( A\Delta A=\emptyset\), donc \( A\sim A\).
		\spitem[symétrique] Nous avons \( A\Delta B=B\Delta A\), donc \( \sim\) est symétrique.
		\spitem[transitive] Si \( A,B,C\) sont des parties de \( X\) alors nous avons toujours
		\begin{equation}
			A\Delta C\subset (A\cup B\cup C)\setminus(A\cap B\cap C)=(A\Delta B)\cup (B\Delta C).
		\end{equation}
		Donc si \( A\sim B\) et \( B\sim C\) alors \( A\Delta C\) est contenu dans une union de maigres et est donc maigre.
		\spitem[Autres propriétés de \( \sim\)]
		De plus la relation d'équivalence \( \sim\) vérifie \( A\sim B\) si et seulement si \( A^c\sim B^c\), par le lemme~\ref{LemCUVoohKpWB}\ref{ItemVUCooHAztC}.

		Pour compléter les propriétés de \( \sim\) mentionnons encore le fait que si \( F\) est fermé alors \( F\sim\Int(F)\). En effet \( F\cup\Int(F)=F\) et \( F\cap\Int(F)=\Int(F)\), de telle sorte que \( F\Delta\Int(F)=F\setminus\Int(F)\). Cet ensemble est un fermé parce que son complémentaire est \( F^c\cup\Int(F)\) qui est une union d'ouverts. De plus \( F\subset\Int(F)\) est d'intérieur vide, de telle sorte qu'il est rare et donc maigre.
	\end{subproof}

	Pour la suite de la preuve nous posons
	\begin{equation}
		\tribF=\{ A\subset X\tq\text{il existe un ouvert } U\text{ avec } U\sim A \},
	\end{equation}
	et nous devons prouver que \( \tribF=\Baire(X)\).

	\begin{subproof}
		\spitem[\( \tribF\subset\Baire(X)\)]

		Soit \( A\in\tribF\) et un ouvert \( U\) tel que \( U\sim A\). Alors nous posons \( M=U\Delta A\) qui est maigre. En vertu du lemme~\ref{LemCUVoohKpWB}\ref{ItemVUCooHAztCii}, nous avons
		\begin{equation}
			A=M\Delta U=(M\cup U)\setminus(M\cap U),
		\end{equation}
		ce qui prouve que \( A\) est dans la tribu engendrée par les ouverts et les maigres, laquelle tribu est contenue dans \( \Baire(X)\).
		\spitem[\( \Baire(X)\subset\tribF\)]

		Nous allons montrer que \( \tribF\) est une tribu contenant tous les ouverts et tous les maigres. Alors en particulier \( \tribF\) contiendra \( \Baire(X)\). Si \( U\) est ouvert, \( U\sim U\) et donc \( U\in\tribF\). Si \( M \) est maigre, alors \( M\sim\emptyset\) et donc \( M\in\tribF\). Il reste à prouver que \( \tribF\) est une tribu.
		\begin{subproof}
			\spitem[Vide et tout l'ensemble] C'est facile : \( \emptyset\) et \( X\) sont dans \( \tribF\).
			\spitem[Comlémentaire] Commençons par nous souvenir que \( F\sim\Int(F)\) dès que \( F\) est fermé. Si \( A\in\tribF\) alors il existe un ouvert \( U\) tel que \( A\sim U\) et donc aussi \( A^c\sim U^c\). D'autre part \( U^c\) est fermé, donc \( U^c\sim\Int(U^c)\), donc
			\begin{equation}
				A^c\sim U^c\sim\Int(U^c),
			\end{equation}
			ce qui implique que \( A^c\in\tribF\).
			\spitem[Union dénombrable] Soit \( A_n\in\tribF\) et \( M_n=A_n\Delta U_n\) avec \( M_n\) maigre et \( U_n\) ouvert. Nous allons prouver que
			\begin{equation}
				\bigcup_nA_n\sim\bigcup_nU_n.
			\end{equation}
			Pour cela il faut remarquer que
			\begin{equation}
				\Big( \bigcup_nA_n \Big)\Delta\Big( \bigcup_nU_n \Big)\subset\bigcup_n(A_n\Delta U_n)=\bigcup_nM_n.
			\end{equation}
			Le terme le plus à droite est maigre, ce qui signifie que celui le plus à gauche est contenu dans un maigre et donc est maigre lui-même.
		\end{subproof}
	\end{subproof}
\end{proof}

\begin{proposition}
	Si \( B\) est un borélien de \( X\), alors il existe un ouvert \( U\) et un maigre \( M\) tels que
	\begin{enumerate}
		\item
		      \( B\Delta U\) est maigre,
		\item
		      \( M\Delta U=B\),
		\item
		      \( B\Delta M\) est ouvert.
	\end{enumerate}
\end{proposition}

\begin{proof}
	Puisque \( B\) est borélien, il est aussi dans la tribu de Baire et il existe par la proposition~\ref{PropGRHootvAWq} un ouvert \( U\) tel que \( M=B\Delta U\) est maigre. En prenant ce \( U\) et ce \( M\), les trois conditions sont vérifiées parce que
	\begin{equation}
		M\Delta U=(B\Delta U)\Delta U=B
	\end{equation}
	et
	\begin{equation}
		B\Delta M=M\Delta B=(U\Delta B)\Delta B=U.
	\end{equation}
	Tout ceci par le lemme~\ref{LemCUVoohKpWB}\ref{ItemVUCooHAztCii}.
\end{proof}

%+++++++++++++++++++++++++++++++++++++++++++++++++++++++++++++++++++++++++++++++++++++++++++++++++++++++++++++++++++++++++++
\section{Espace mesuré complet}
%+++++++++++++++++++++++++++++++++++++++++++++++++++++++++++++++++++++++++++++++++++++++++++++++++++++++++++++++++++++++++++

%---------------------------------------------------------------------------------------------------------------------------
\subsection{Partie négligeable}
%---------------------------------------------------------------------------------------------------------------------------

\begin{definition}  \label{DefAVDoomkuXi}
	Soit un espace mesuré \( (X,\tribA,\mu)\). Une partie \( N\) de \( X\) est \defe{négligeable}{négligeable!partie d'un espace mesuré} pour \( \mu\) si il existe \( Y\in\tribA\) tel que \( N\subset Y\) et \( \mu(Y)=0\).
\end{definition}

\begin{lemma}   \label{LemVKNooOCOQw}
	L'ensemble des parties négligeables est stable par union dénombrable.
\end{lemma}

\begin{proof}
	Si les ensembles \( N_i\) sont négligeables, alors pour chaque \( i\) nous avons \( Y_i\in\tribA\) tel que \( N_i\subset Y_i\) et \( \mu(Y_i)=0\). Alors bien entendu \( \bigcup_iN_i\subset \bigcup_iY_i\) et en utilisant \eqref{EqWWFooYPCTt},
	\begin{equation}
		\mu\big( \bigcup_iY_i \big)\leq \sum_i\mu(Y_i)=0.
	\end{equation}
\end{proof}

\begin{definition}  \label{DefBWAoomQZcI}
	L'espace mesuré \( (X,\tribF,\mu)\) est \defe{complet}{complet!espace mesuré} si tout ensemble \( \mu\)-négligeable est dans \( \tribF\).
\end{definition}

Notons que la proposition~\ref{PropHYLooLgOCy} s'applique si \( (X,\tribF,\mu)\) est un espace mesuré et \( \tribN\) est l'ensemble des parties \( \mu\)-négligeables. C'est ce qui permet de donner le théorème suivant, que nous redémontrons de façon indépendante de la proposition~\ref{PropHYLooLgOCy}.
\begin{theorem}[Complétion d'espace mesuré\cite{MesureLebesgueLi,DXTooFCLru,BOQoojbFpP}]   \label{thoCRMootPojn}
	Soit un espace mesuré \( (X,\tribF,\mu)\) et \( \tribN\) l'ensemble des parties \( \mu\)-négligeables de \( X\).
	\begin{enumerate}
		\item
		      Les ensembles suivants sont égaux :
		      \begin{subequations}
			      \begin{align}
				      \tribA & =\{ A\subset X \tq \exists B,C\in\tribF \tq B\subset A\subset C,\mu(C\setminus B)=0 \} \\
				      \tribB & =\{ B\cup N \tq  B\in\tribF,N\in\tribN \}          \label{EqFJIoorxZNU}                \\
				      \tribC & =\{ A\subset X\tq \exists B\in\tribF\tq A\Delta B\in \tribN \}.
			      \end{align}
		      \end{subequations}
		      Ici \( A\Delta B\) est la différence symétrique de \( A\) et \( B\), définition~\ref{DefBMLooVjlSG}.
		\item
		      L'ensemble \( \hat\tribF=\tribA=\tribB=\tribC\) est une tribu.
		\item
		      La définition
		      \begin{equation}
			      \begin{aligned}
				      \mu'\colon \tribB & \to \mathopen[ 0 , \infty \mathclose] \\
				      A\cup N           & \mapsto \mu(A)
			      \end{aligned}
		      \end{equation}
		      est cohérente.
		\item
		      L'application \( \mu'\) ainsi définie est une mesure sur \( (X,\tribA)\).
		\item
		      L'espace \( (X,\tribA,\mu')\) est complet.
		\item
		      La mesure \( \mu'\) prolonge \( \mu\).
		\item   \label{thoCRMootPojnvii}
		      La mesure \( \mu'\) est minimale au sens où toute mesure complète prolongeant \( \mu\) prolonge \( \mu'\).
	\end{enumerate}
\end{theorem}

\begin{proof}
	Commençons par prouver que les trois ensembles \( \tribA\), \( \tribB\) et \( \tribC\) sont égaux.
	\begin{subproof}
		\spitem[\( \tribA\subset\tribB\).]
		Soit \( A\in\tribA\). Alors nous avons des ensembles \( B,C\in\tribF \) tels que \( B\subset A\subset C\) avec \( \mu(C\setminus B)=0\). Alors nous avons aussi \( A=B\cup(C\setminus B)\), ce qui prouve que \( A\in\tribB\).
		\spitem[\( \tribB\subset\tribC\).]
		Soit \( A\in\tribB\), c'est-à-dire que \( A=B\cup N\) avec \( B\in\tribF\) et \( N\in\tribN\). Nous avons évidemment \( A\cup B=A\) et donc
		\begin{equation}
			A\Delta B=(A\cup B)\setminus(A\cap B)=A\setminus(A\cap B)=(B\cup N)\setminus(A\cap B)\subset N.
		\end{equation}
		Pour comprendre la dernière inclusion, si \( x\) appartient à \( A=B\cup N\) sans être dans \( N\) alors \( x\in B\) et donc \( x\in A\cap B\). Par conséquent nous avons \( A\Delta B\subset N\) et donc \( A\Delta B\in\tribN\).
		\spitem[\( \tribC\subset\tribA\)]
		Soit donc \( A\in\tribC\); il existe \( B\in\tribF\) tel que \( A\Delta B\in\tribN\) ou encore, il existe \( D\in\tribF\) tel que \( A\Delta B\subset D\) avec \( \mu(D)=0\). Si nous posons \( B'=B\cap D^c\) et \( C'=B\cup D\) alors nous prétendons avoir
		\begin{equation}
			B'\subset A\subset C'.
		\end{equation}
		Et nous le prouvons. En effet si \( x\in B\cap D^c\) alors en remarquant que \( B\) se divise en
		\begin{equation}
			B=(B\cap A)\cup\big(B\cap (A\Delta B)\big),
		\end{equation}
		et en nous souvenant que \( B\cap (A\Delta B)\subset D\), il vient que \( B\cap D^c\subset B\cap A\). Et en particulier \( x\in A\). D'autre part
		\begin{equation}
			A\subset B\cup(A\Delta B)\subset B\cup D.
		\end{equation}
		Nous avons donc bien \( B'\subset A\subset C'\). Par stabilité de la tribu \( \tribF\) sous les intersections et complémentaires, nous avons aussi \( B',C'\in\tribF\). De plus
		\begin{equation}
			C'\setminus B'=(B\cup D)\setminus(B\cap D^c)\subset D,
		\end{equation}
		et donc
		\begin{equation}
			\mu(C'\setminus B')\leq \mu(D)=0.
		\end{equation}
	\end{subproof}

	Nous avons donc prouvé que \( \tribA\subset\tribB\subset\tribC\subset \tribA\), et donc que \( \tribA=\tribB=\tribC\). Nous pouvons maintenant noter \( \tribA\) indifféremment les trois ensembles.

	Nous prouvons à présent que \( \tribA\) est une tribu.

	\begin{subproof}
		\spitem[Tribu : le vide]
		Pas de problème à \( \emptyset\in\tribA\)

		\spitem[Tribu : complémentaire]
		Soit \( A\in\tribA\). Alors il existe \( B,C\in\tribF\) tels que \( B\subset A\subset C\) avec \( \mu(C\setminus B)=0\). En passant au complémentaire,
		\begin{equation}
			C^c\subset A^c\subset B^c.
		\end{equation}
		Mais \( B^c\setminus C^c=C\setminus B\), donc \( \mu(B^c\setminus C^c)=0\).

		\spitem[Tribu : union dénombrable]
		Soit \( (A_n)\) des éléments de \( \tribA\). Pour chaque \( n\) nous avons des ensembles \( B_n,C_n\in\tribF\) tels que\( B_n\subset A_n\subset C_n\) avec \( \mu(C_n\setminus B_n)=0\). En ce qui concerne les unions nous avons
		\begin{equation}
			\bigcup_nB_n\subset \bigcup_nA_n\subset \bigcup_nC_n,
		\end{equation}
		et
		\begin{equation}
			\big( \bigcup_nC_n\big)\setminus\big( \bigcup_nB_n\big)\subset \bigcup_n(C_n\setminus B_n).
		\end{equation}
		Par conséquent, en utilisant \eqref{EqWWFooYPCTt},
		\begin{equation}
			\mu\left( \big( \bigcup_nC_n\big)\setminus\big( \bigcup_nB_n\big)\right)\leq\mu\left(  \bigcup_n(C_n\setminus B_n)\right)\leq\sum_n\mu(C_n\setminus B_n)=0.
		\end{equation}
		Cela prouve que \( \bigcup_nA_n\in\tribA\), et donc que \( \tribA\) est une tribu.

		\spitem[Définition cohérente]
		Soient \( A,A'\in\tribF\) et \( N,N'\in\tribN\) tels que \( A\cup N=A'\cup N'\). Nous considérons \( Y,Y'\in\tribF\) tel que \( N\subset Y\), \( N'\subset Y'\) et \( \mu(Y)=\mu(Y')=0\). En vertu de \eqref{EqWWFooYPCTt} nous avons
		\begin{equation}
			\mu(A)\leq \mu(A\cup Y)\leq \mu(A'\cup Y\cup Y')\leq\mu(A')+\mu(Y)+\mu(Y')=\mu(A').
		\end{equation}
		En écrivant la même chose en échangeant les primes, nous prouvons également \( \mu(A')\leq \mu(A)\). Au final \( \mu(A)=\mu(A')\), c'est-à-dire
		\begin{equation}
			\mu'(A\cup N)=\mu'(A'\cup N').
		\end{equation}
		La définition de \( \mu'\) est donc cohérente.

		\spitem[\( \mu'\) est une mesure]

		Le fait que \( \mu'\) soit positive et que \( \mu'(\emptyset)\) soit nul ne pose pas de problème. Il faut voir l'union dénombrable disjointe. Si les ensembles \( A_i=B_i\cup N_i\) sont disjoints, alors les \( B_i\) et le \( N_i\) sont tous disjoints deux à deux. De plus l'ensemble \( \bigcup_iN_i\) est négligeable parce que nous avons déjà vu que \( \tribN\) était stable par union dénombrable (\ref{EqWWFooYPCTt}). Donc
		\begin{equation}
			\mu'\left( \bigcup_i B_i\cup N_i \right)=\mu'\Big( \big( \bigcup_iB_i \big)\cup\underbrace{\big( \bigcup_iN_i \big)}_{\in\tribN} \Big)=\mu\big( \bigcup_iB_i \big)=\sum_u\mu(B_i)=\sum_i\mu'(B_i\cup N_i).
		\end{equation}

		\spitem[Espace complet]
		Un ensemble \( \mu'\)-négligeable est automatiquement \( \mu\)-négligeable. En effet si \( H\) est \( \mu'\)-négligeable, il existe \( B\in\tribF\) et \( N\in\tribN\) tels que \( H\subset B\cup N\) avec \( \mu(B)=0\). Comme \( N\) est \( \mu\)-négligeable, il existe \( Y\in\tribF\) tel que \( N\subset Y\) et \( \mu(Y)=0\). Donc \( H\subset B\cup N\subset B\cup Y\) avec \( \mu(B\cup Y)=0\).

		Tous les ensembles \( \mu\)-négligeables faisant partie de \( \tribB\), tous les ensembles \( \mu'\)-négligeables font partie de \( \tribA\).

		\spitem[Prolongement]
		La mesure \( \mu'\) prolonge \( \mu\). En effet si \( A\in\tribF\) alors \( A=A\cup\emptyset\in\tribB\) et \( A\) est \( \mu'\)-mesurable. De plus \( \mu'(A)=\mu'(A\cup\emptyset)=\mu(A)\).
		\spitem[Minimalité]

		Soit un espace mesuré complet \( (X,\tribM,\nu)\) prolongeant \( (X,\tribF,\mu)\). Pour \( A\in\tribA\) nous devons prouver que \( A\in\tribM\) et que \( \mu'(A)=\nu(A)\). Il existe \( B\in\tribF\) et \( N\in\tribN\) tels que \( A=B\cup N\). Puisque \( N\) est \( \mu\)-négligeable, il est également \( \nu\)-négligeable et donc \( \nu\)-mesurable parce que \(\nu\) est complète : \( A\in\tribM\). Nous avons le calcul
		\begin{equation}
			\nu(B)\leq\nu(B\cup N)\leq \nu(B)+\nu(N)=\nu(B).
		\end{equation}
		Vu que le premier et dernier termes de ces inégalités sont égaux, toutes les inégalités sont des égalités et nous avons \( \nu(B)=\nu(B\cup N)\). Nous pouvons enfin faire le calcul
		\begin{subequations}
			\begin{align}
				\nu(A) & =\nu(B\cup N)                              \\
				       & =\nu(B)                                    \\
				       & =\mu(B)        \label{SUBEQooVCROooSeRAjw} \\
				       & =\mu'(B\cup N) \label{SUBEQooBDASooTBBBMs} \\
				       & =\mu'(A).
			\end{align}
		\end{subequations}
		Justifications.
		\begin{itemize}
			\item Pour \eqref{SUBEQooVCROooSeRAjw}. La mesure \( \nu\) prolonge \( \mu\).
			\item Pour \eqref{SUBEQooBDASooTBBBMs}. Définition de \( \mu'\).
		\end{itemize}
		L'égalité \( \mu'(A)=\nu(A)\) est prouvée.
	\end{subproof}
\end{proof}

\begin{definition}
	L'espace mesuré complet \( (X,\tribA,\mu')\) défini par le théorème~\ref{thoCRMootPojn} est l'\defe{espace mesuré complété}{espace!mesuré!complété} de \( (X,\tribF,\mu)\).

	Nous noterons le complété de \( (S,\tribF,\mu)\) par \( (S,\hat\tribF,\hat \mu)\)\nomenclature[Y]{\( (S,\hat\tribF,\hat\mu)\)}{complété de l'espace mesuré \( (S,\hat\tribF,\hat\mu)\)}
\end{definition}

\begin{theorem}[Carathéodory\cite{MesureLebesgueLi}]        \label{ThoUUIooaNljH}
	Soit \( S\) un ensemble et \( m^*\) une mesure extérieure sur \( S\). Alors
	\begin{enumerate}
		\item   \label{RPPooHSWWsi}
		      l'ensemble \( \tribM\) des parties \( m^*\)-mesurables est une tribu,
		\item
		      la restriction de \( m^*\) est une mesure sur \( (S,\tribM)\),
		\item
		      l'espace mesuré \( (S,\tribM,m^*)\) est complet\footnote{Définition~\ref{DefBWAoomQZcI}.}.
	\end{enumerate}
\end{theorem}

\begin{proof}
	Une grosse partie de la preuve sera de prouver la stabilité de \( \tribM\) par union dénombrable quelconque; cela sera divisé en plusieurs parties.
	\begin{subproof}
		\spitem[Tribu : le vide]
		L'ensemble vide est \( m^*\)-mesurable.
		\spitem[Tribu : complémentaire]
		Soit \( A\in\tribM\) et \( X\in S\). La condition qui dirait \( A^c\in\tribM\) est :
		\begin{equation}
			m^*(X)=m^*(X\cap A^c)+m^*(X\cap A),
		\end{equation}
		qui est la même que celle qui dit que \( A\) est dans \( \tribM\).
		\spitem[Tribu : union finie]
		Soient \( A,B\in\tribM\) et \( X\subset S\). Alors, comme \( m^*\) est une mesure extérieure,
		\begin{subequations}
			\begin{align}
				m^*(X) & \leq m^*\big(  X\cap(A\cup B) \big)+m^*\big( X\cap (A\cup B)^c \big)         \\
				       & =    m^*\big( (X\cap A)\cup(X\cap B) \big)+m^*\big( X\cap A^c\cap B^c \big).
			\end{align}
		\end{subequations}
		Mais nous pouvons écrire la première union sous forme d'une union disjointe de la façon suivante :
		\begin{equation}
			(X\cap A)\cup(X\cap B)=(X\cap A)\cup(X\cap B\cap A^c),
		\end{equation}
		ce qui donne
		\begin{subequations}
			\begin{align}
				m^*(X) & \leq m^*(X\cap A)+m^*(X\cap B\cap A^c)+m^*(X\cap A^c\cap B^c)   \label{subeqLYNooRdrgCi} \\
				       & =    m^*(X\cap A)+m^*(X\cap A^c)                                                         \\
				       & =    m^*(X)
			\end{align}
		\end{subequations}
		parce que les deux derniers termes de \eqref{subeqLYNooRdrgCi} se somment à \( m^*(X\cap A^c)\) parce que \( B\in \tribM\). La dernière ligne est le fait que \( A\) soit \( m^*\)-mesurable.
		\spitem[Union finie disjointe]
		Soient \( \{ A_1,\ldots, A_n \}\) des éléments deux à deux disjoints de \( \tribM\). Nous allons maintenant prouver par récurrence que
		\begin{equation}    \label{EqBRIooAnPCd}
			m^*\Big( X\cap\big( \bigcup_{k=1}^nA_k \big) \Big)=\sum_{k=1}^nm^*(X\cap A_k).
		\end{equation}
		Si \( n=1\) le résultat est évident. Sinon, le fait que \( A_{n+1}\) soit \( m^*\)-mesurable donne
		\begin{equation}
			m^*\Big( X\cap\big( \bigcup_{k=1}^{n+1}A_k \big) \Big)=m^*\Big( X\cap\big( \bigcup_{k=1}^{n+1}A_k \big)\cap A_{n+1} \Big)+m^*\Big( X\cap\big( \bigcup_{k=1}^{n+1}A_k \big)\cap A_{n+1}^c \Big).
		\end{equation}
		Le fait que les \( A_k\) soient disjoints implique aussi que
		\begin{equation}
			X\cap\big( \bigcup_{k=1}^{n+1}A_k \big)\cap A_{n+1}=X\cap A_{n+1}
		\end{equation}
		et
		\begin{equation}
			X\cap\big( \bigcup_{k=1}^{n+1}A_k \big)\cap A_{n+1}^c=X\cap\big( \bigcup_{k=1}^nA_k \big)
		\end{equation}
		et donc
		\begin{subequations}
			\begin{align}
				m^*\Big( X\cap\big( \bigcup_{k=1}^{n+1}A_k \big) \Big) & =m^*(X\cap A_{n+1})+m^*\Big( X\cap\big( \bigcup_{k=1}^nA_k \big) \Big) \\
				                                                       & \stackrel{rec.}{=}m^*(X\cap A_{n+1})+\sum_{k=1}^nm^*(X\cap A_k)        \\
				                                                       & =\sum_{k=1}^{n+1}m^*(X\cap A_k).
			\end{align}
		\end{subequations}
		La relation \eqref{EqBRIooAnPCd} est prouvée.

		Notons qu'en particularisant à \( X=S\) nous avons
		\begin{equation}
			m^*\big( \bigcup_{k=1}^nA_k \big)=\sum_{k=1}^nm^*(A_k)
		\end{equation}
		dès que les \( A_k\) sont des éléments deux à deux disjoints de \( \tribM\).

		\spitem[Union dénombrable disjointe]
		Soit \( (A_n)_{n\in \eN}\) une suite d'éléments deux à deux disjoints dans \( \tribM\). Nous allons prouver les affirmations suivantes :
		\begin{itemize}
			\item \( \bigcup_nA_n\in\tribM\)
			\item \( m^*\big( \bigcup_nA_n \big)=\sum_nm^*(A_n)\)
		\end{itemize}
		où toutes les sommes et unions sur \( n\) sont entre \( 1\) et \( \infty\).
		\begin{subproof}
			\spitem[Première affirmation]

			Nous posons \( A=\bigcup_kA_k\) et \( B_n=\bigcup_{k=1}^nA_k\). Nous savons que \( B_n\in\tribM\) pour tout \( n\) par le point précédent. Donc si \( X\in S\) nous avons
			\begin{subequations}
				\begin{align}   \label{EqGXLooRxqqg}
					m^*(X) & =m^*(X\cap B_n)+m^*(X\cap B_n^c)                \\
					       & =   \sum_{k=1}^nm^*(X\cap A_k)+m^*(X\cap B_n^x) \\
					       & \geq\sum_{k=1}^nm^*(X\cap A_k)+m^*(X\cap A^c)
				\end{align}
			\end{subequations}
			où nous avons utilisé la relation \eqref{EqBRIooAnPCd} sur les \( B_n\) ainsi que le fait que \( A^c\subset B_n^c\) (parce que \( B_n\subset A\)). L'inégalité \eqref{EqGXLooRxqqg} étant vraie pour tout \( n\), elle est vraie à la limite :
			\begin{subequations}
				\begin{align}
					m^*(X) & \geq \sum_{k=1}^{\infty}m^*(A\cap A_k)+m^*(X\cap A^c)    \\
					       & \geq m^*\Big( \bigcup_k(X\cap A_k) \Big)+m^*(X\cap A^c)
					=    m^*\Big( X\cap \big( \bigcup_kA_k \big) \Big)+m^*(X\cap A^c) \\
					       & \geq m^*(X\cap A)+m^*(X\cap A^c),
				\end{align}
			\end{subequations}
			ce qui signifie que \( A\in\tribM\).

			\spitem[Seconde affirmation]
			En particularisant à \( X=A\) et en tenant compte des faits que \( A\cap A_k=A_k\) et \( A\cap A^c=\emptyset\),
			\begin{equation}
				m^*(A)\geq \sum_{k=1}^{\infty}m^*(A\cap A_k)+m^*(A\cap A^c),
			\end{equation}
			c'est-à-dire que pour tout \( n\) nous avons
			\begin{equation}
				m^*\big( \bigcup_{k\in \eN}A_k \big)\geq \sum_{k=1}^nm^*(A_k).
			\end{equation}
			L'inégalité est encore vraie à la limite, et l'inégalité inverse étant toujours vraie pour une mesure extérieure,
			\begin{equation}
				m^*\big( \bigcup_{k\in \eN}A_k \big)=\sum_{k=1}^{\infty}m^*(A_k).
			\end{equation}
		\end{subproof}

		\spitem[Union dénombrable quelconque]
		Soit maintenant une suite \( (A_n)_{n\in\eN}\) d'éléments de \( \tribM\) que nous ne supposons plus être disjoints. Nous nous ramenons au cas disjoint en posant
		\begin{subequations}
			\begin{numcases}{}
				B_1=A_1\\
				B_n=A_n\cap\big( \bigcup_{k=1}^{n-1}A_k \big)^c,
			\end{numcases}
		\end{subequations}
		c'est-à-dire que nous mettons dans \( B_n\) les éléments de \( A_n\) qui ne sont dans aucun des \( A_k\) précédents. Autrement dit, nous posons \( B_0=\emptyset\) et \( B_n=A_n\setminus B_{n-1}\). L'ensemble \( \tribM\) étant stable par réunion finie, par complément et par intersection finie nous avons \( B_n\in\tribM\). De plus les \( B_n\) sont disjoints, donc
		\begin{equation}
			\bigcup_{k=1}^{\infty}A_k=\bigcup_{k=1}^{\infty}B_k\in\tribM.
		\end{equation}
		La première égalité se justifie de la façon suivante : si \( x\in\bigcup_{k=1}^{\infty}A_k\) alors nous notons \( n_0\) le plus petit \( n\) tel que \( x\in A_n\) et alors \( x\in B_{n_0}\).
		\spitem[Espace complet]
		Nous prouvons à présent que \( (S,\tribM,m^*)\) est un espace mesuré complet. Soit \( N\) une partie \( m^*\)-négligeable de \( S\) et \( Y\in\tribM\) tel que \( m^*(Y)=0\) et \( N\subset Y\). D'abord \( m^*(N)=0\) parce que
		\begin{equation}
			m^*(N)\leq m^*(Y)=0.
		\end{equation}
		Si \( X\subset S\) nous avons
		\begin{subequations}
			\begin{align}
				X\cap N\subset   N & \Rightarrow m^*(X\cap N)=0             \\
				X\cap N^c\subset X & \Rightarrow m^*(X\cap N^c)\leq m^*(X).
			\end{align}
		\end{subequations}
		Donc
		\begin{equation}
			m^*(X\cap N)+m^*(X\cap N^c)\leq m^*(X),
		\end{equation}
		ce qui montre que \( N\) est \( m^*\)-mesurable.
	\end{subproof}
\end{proof}

\begin{normaltext}

	Ce théorème nous pousse à adopter des éléments de notation. Lorsqu'un espace mesuré \( (S,\tribF,\mu)\) est donné, nous noterons
	\begin{equation}
		(S,\tribM,\mu^*)
	\end{equation}
	l'espace mesuré construit de la façon suivante. D'abord \( \mu^*\) est la mesure extérieure associée à \( \mu\) par la proposition~\ref{PropFDUooVxJaJ}. Ensuite \( \tribM\) est la tribu des parties \( \mu^*\)-mesurables, qui est bien une tribu parce que \( \mu^*\) est une mesure extérieure (\ref{ThoUUIooaNljH}). La proposition \eqref{PropOJFoozSKAE} dit alors que \( \tribF\subset\tribM\). De plus~\ref{ThoUUIooaNljH} nous explique que si \( A\in\tribF\) alors \( \mu(A)=\mu^*(A)\). Tout cela pour dire que
	\begin{equation}    \label{EqXDPooKwWAF}
		(S,\tribF,\mu)\subset (S,\tribM,\mu^*).
	\end{equation}
	Et enfin,~\ref{ThoUUIooaNljH} nous dit que l'espace mesuré \( (S,\tribM,\mu^*)\) est complet.
\end{normaltext}

\begin{example} \label{ExOIXoosScTC}
	Montrons un cas dans lequel \( (S,\tribM,\mu^*)\) n'est pas \( \sigma\)-fini. Soit \( S\) un ensemble non dénombrable et \( \tribF\) la tribu des parties de \( S\) qui sont, soit finis ou dénombrables, soit de complémentaire fini ou dénombrable. Nous y mettons la mesure
	\begin{equation}
		\mu(A)=\begin{cases}
			0      & \text{si } A  \text{ est au plus dénombrable} \\
			\infty & \text{sinon}.
		\end{cases}
	\end{equation}
	Cette mesure n'est pas \( \sigma\)-finie parce qu'aucune union de dénombrables est non dénombrable. De plus \( (S,\tribF,\mu)\) est complet parce que toute partie contenue dans un ensemble fini ou dénombrable est fini ou dénombrable (\ref{PropQEPoozLqOQ}).

	\begin{subproof}
		\spitem[\( \tribF\) n'est pas \( \partP(S)\)]
		La tribu \( \tribF\) est différente de \( \partP(S)\). En effet \( S\) étant infini, il existe par~\ref{PropVCSooMzmIX} une bijection \( \varphi\colon \{ 1,2 \}\times S\to S\). Alors l'ensemble \( \varphi\big( \{ 1 \}\times S \big)\) est non dénombrable et son complémentaire
		\begin{equation}
			\varphi\big( \{ 1 \}\times S \big)^c=\varphi\big( \{ 2 \}\times S \big)
		\end{equation}
		n'est pas dénombrable non plus. Cet ensemble n'est donc pas de \( \tribF\).

		\spitem[\( \tribM\) est \( \partP(S)\)]
		En effet, soit \( A\subset S\); il faut prouver que pour tout \( X\subset S\) nous avons
		\begin{equation}
			\mu^*(X)=\mu^*(X\cap A)+\mu^*(X\cap A^c).
		\end{equation}
		Nous prouvons cela en séparant les cas, suivant que \( X\) est dénombrable ou non.

		Si \( X\) est fini ou dénombrable, alors \( X\cap A\) et \( X\cap A^c\) le sont également, et nous avons \( \mu^*(X)=\mu(X)=0\) ainsi que \( \mu^*(X\cap A)=\mu^*(X\cap A^c)=0\).

		Si au contraire \( X\) n'est pas dénombrable,
		\begin{equation}
			\mu^*(X)=\inf_{\substack{A\in\tribF\\X\subset A}}\mu(A)=\infty,
		\end{equation}
		parce que \( X\) n'étant pas dénombrable, l'ensemble \( A\) ne l'est pas non plus et \( \mu(A)=\infty\). Mais comme \( X\) n'est pas dénombrable, soit \( X\cap A\), soit \( X\cap A^c\) (soit les deux) n'est pas dénombrable non plus; par conséquent
		\begin{equation}
			\mu^*(X\cap A)+\mu^*(X\cap A^c)=\infty.
		\end{equation}
	\end{subproof}

	Par conséquent \( (S,\tribF,\mu) \neq (S,\tribM,\mu^*)\). Mais puisque \( (S,\tribF,\mu)\) est complété nous devons avoir \( (S,\tribF,\mu)=(S,\hat\tribF,\hat\mu)\). Tout cela pour dire que nous avons un exemple avec
	\begin{equation}
		(S,\tribM,\mu^*)\neq (S,\hat\tribF,\hat \mu).
	\end{equation}
\end{example}

Nous avons deux façons de créer un espace complet à partir de \( (S,\tribF,\mu)\).
\begin{enumerate}
	\item
	      Partir de la mesure extérieure \( \mu^*\) et construire \( (S,\tribM,\mu^*)\).
	\item
	      Partir des ensembles \( \mu\)-négligeables, construire \( \hat\tribF\) et ensuite \( (S,\hat\tribF,\hat\mu)\).
\end{enumerate}
Ces deux façons ne sont pas équivalentes en général comme le montre l'exemple~\ref{ExOIXoosScTC}. Mais il sera montré par la proposition~\ref{PropIIHooAIbfj} que si \( (S,\tribF,\mu)\) est \( \sigma\)-fini alors les deux sont équivalent.

\begin{lemma}   \label{LemAESoofkMpi}
	Soit \( (S,\tribF,\mu)\) un espace mesuré. Alors pour tout \( X\subset S\) tel que \( \mu^*(X)<\infty\) il existe \( A\in\tribF\) tel que \( X\subset A\) et \( \mu^*(X)=\mu(A)\).
\end{lemma}
C'est-à-dire que \( \mu^*\) a beau être défini sur toutes les parties de \( S\), ce qu'il faut rajouter pour être \( \mu\)-mesurable, c'est pas grand chose.

\begin{proof}
	Par définition de la mesure extérieure associée à \( \mu\) en tant qu'infimum, pour tout \( n\geq 1\), il existe \( A_n\in\tribF\) tel que \( X\subset A_n\) et \( \mu(A_n)\leq \mu^*(X)+\frac{1}{ 2^n }\). Nous posons \( A=\bigcap_{n\geq 1}A_n\) et nous vérifions que ce \( A\) fait l'affaire.

	D'abord \( A\in\tribF\) parce qu'une tribu est stable par union dénombrable. Ensuite pour tout \( n\geq 1\) nous avons
	\begin{equation}
		\mu(A)\leq \mu(A_n)\leq \mu^*(X)+\frac{1}{ 2^n },
	\end{equation}
	et à la limite \( \mu(A)\leq \mu^*(X)\). Mais \( X\subset A\) implique \( \mu^*(X)\leq \mu(A)\) parce que \( \mu^*(X)\) l'infimum d'un ensemble contenant \( \mu(A)\).
\end{proof}

\begin{corollary}\label{LemXOUNooUbtpxm}
	Soit une mesure \( \mu\) et la mesure extérieure \( \mu^*\) associée\footnote{Par la proposition~\ref{PropFDUooVxJaJ}.}. Une partie \( N\) de \( X\) est négligeable si et seulement si \( \mu^*(N)=0\).
\end{corollary}

\begin{proof}
	Si \( \mu^*\) est la mesure extérieure associée à \( \mu\) et si \( N\) est \( \mu\)-négligeable alors \( \mu^*(N)=0\) parce que
	\begin{equation}
		\mu^*(N)\leq \mu^*(Y)=\mu(Y)=0
	\end{equation}
	pour un certain \( Y\) mesurable de mesure nulle contenant \( N\).

	D'autre part si \( \mu^*(N)=0\) alors le lemme~\ref{LemAESoofkMpi} donne une partie mesurable \( A\) telle que \( N\subset A\) et \( \mu(A)=0\), c'est-à-dire que \( N\) est négligeable.
\end{proof}

\begin{lemma}       \label{LemOAEoocBDaO}
	Si l'espace mesuré \( (S,\tribF,\mu)\) est \( \sigma\)-fini alors l'espace mesuré \( (S,\tribM,\mu^*)\) est également \( \sigma\)-fini.
\end{lemma}

\begin{proof}
	Puisque \( (S,\tribF,\mu)\) est \( \sigma\)-fini, nous avons une suite croissante \( A_n\) d'éléments de \( \tribF\) tels que \( \bigcup_nA_n=S\) et telle que \( \mu(A_n)<\infty\) pour tout \( n\). Étant donné que \( \tribF\subset\tribM\), cette suite convient également pour montrer que \( (S,\tribM,\mu^*)\) est \( \sigma\)-fini parce que \( \mu^*(A_n)=\mu(A_n)<\infty\).
\end{proof}

La proposition suivante montre que si \( (S,\tribF,\mu)\) est \( \sigma\)-finie alors nous avons l'égalité.
\begin{proposition} \label{PropIIHooAIbfj}
	Soit \( (S,\tribF,\mu)\) un espace mesuré \( \sigma\)-fini, \( \mu^*\) la mesure extérieure associée et \( \tribM\) la tribu des ensembles \( \mu^*\)-mesurables\footnote{C'est bien une tribu par~\ref{ThoUUIooaNljH}\ref{RPPooHSWWsi}.}. Alors
	\begin{equation}
		(S,\tribM,\mu^*) = (S,\hat\tribF,\hat\mu).
	\end{equation}
\end{proposition}

\begin{proof}
	La proposition~\ref{PropOJFoozSKAE} indique que tous les éléments de \( \tribF\) sont \( \mu^*\)-mesurables, c'est-à-dire que \( \tribF\subset \tribM\). Mais l'espace \( (S,\tribM,\mu^*)\) est complet par le théorème de Carathéodory~\ref{ThoUUIooaNljH}, donc par minimalité du complété (\ref{thoCRMootPojn}\ref{thoCRMootPojnvii}),
	\begin{equation}
		(S,\hat\tribF,\hat\mu)\subset(S,\tribM,\mu^*)
	\end{equation}
	au sens où \( \hat\tribF\subset\tribM\) et si \( A\in\hat\tribF\) alors \( \hat\mu(A)=\mu^*(A)\). Notons que cette inclusion est vraie même si la mesure n'est pas \( \sigma\)-finie.

	Nous passons à l'inclusion inverse. Soit \( A\in\tribM\), c'est-à-dire que pour tout \( Y\subset S\) nous avons
	\begin{equation}    \label{EqTZAooTCdGg}
		\mu^*(Y)=\mu^*(Y\cap A)+\mu^*(Y\cap A^c).
	\end{equation}
	Nous allons montrer que \( A\in\hat\tribF\) en séparant les cas suivant que \( \mu^*(A)=\infty\), ou non.

	\begin{subproof}
		\spitem[Si \( \mu^*(A)<\infty\)]

		Par le lemme~\ref{LemAESoofkMpi}, il existe \( X\in\tribF\) tel que \( A\subset X\) et \( \mu^*(A)=\mu(X)\). Comme \( (S,\tribF,\mu)\subset (S,\tribM,\mu^*)\) nous avons alors
		\begin{equation}    \label{EqKFQooQaont}
			\mu^*(A)=\mu(X)=\mu^*(X).
		\end{equation}
		Nous écrivons la relation \eqref{EqTZAooTCdGg} avec ce \( X\) en guise de \( Y\), et en nous souvenant que \( X\cap A=A\) et \( X\cap A^c=X\setminus A\) :
		\begin{equation}
			\mu^*(X)=\mu^*(A)+\mu^*(X\setminus A).
		\end{equation}
		En tenant compte de \eqref{EqKFQooQaont} et du fait que \( \mu^*(A)<\infty\), nous pouvons simplifier et trouver \( \mu^*(X\setminus A)=0\). Le lemme~\ref{LemAESoofkMpi} nous donne alors \( B\in\tribF\) tel que \( X\setminus A\subset B\) et \( \mu(B)=\mu^*(X\setminus A)=0\), c'est-à-dire que \( X\setminus A\) est \( \mu\)-négligeable. Par conséquent \( X\setminus A\in\hat\tribF\). En écrivant
		\begin{equation}
			A=X\setminus(X\setminus A),
		\end{equation}
		nous avons écrit \( A\) comme différence de deux éléments de \( \hat\tribF\) et nous concluons que \( A\in\hat\tribF\).

		\spitem[Si \( \mu^*(A)=\infty\)]

		Le lemme~\ref{LemOAEoocBDaO} nous indique que \( (S,\tribM,\mu^*)\) est \( \sigma\)-fini et il existe donc une suite \( (S_n)_{n\geq 1}\) dans \( \tribM\) telle que \( \bigcup_nS_n=S\) et \( \mu^*(S_n)<\infty\). L'ensemble \( A\cap S_n\) est un élément de \( \tribM\) vérifiant
		\begin{equation}
			\mu^*(A\cap S_n)\leq \mu^*(S_n)<\infty,
		\end{equation}
		ce qui implique que \( A\cap S_n\in\hat\tribF\) par la première partie. Maintenant \( A=\bigcup_n(A\cap S_n)\in\hat\tribF\) par union dénombrable d'éléments de la tribu \( \hat\tribF\).
	\end{subproof}
\end{proof}

\begin{proposition}[\cite{MonCerveau}] \label{PROPooAMIEooRomnMG}
	Soit un espace mesuré \( (\Omega,\tribF,\mu)\). Nous considérons un mesurable \( M\in \tribF\) ainsi que
	\begin{itemize}
		\item
		      la tribu induite    \( \tribF_M=\{ A\cap M\tq A\in \tribF \}\),
		\item
		      la tribu complétée  \( \hat\tribF\) de \( \tribF\) dans \( \Omega\),
		\item
		      la tribu complétée  \( \widehat{\tribF_M}\) de \( \tribF_M\) dans \( M\) (où nous avons considéré la mesure restreinte\footnote{Ce n'est pas ce qu'il se passe dans le cas de \( S^1\) par rapport à \( \eC\), voir la proposition \ref{PROPooDLBCooUfQZOa}\ref{ITEMooXDBTooYnauyi} bien que \( S^1\) soit un borélien de \( \eC\).} de \( \mu\)).
		\item
		      la tribu induite    \( (\hat\tribF)_M=\{ A\cap M\tq A\in \hat\tribF \}\) de \( \hat\tribF\) sur \( M\).
	\end{itemize}
	Alors
	\begin{equation}
		(\hat\tribF)_M=\widehat{\tribF_M}.
	\end{equation}
\end{proposition}

\begin{proof}
	L'utilisation de la proposition \ref{PROPooUNNSooMUQKfp} nous donne déjà les expressions alternatives
	\begin{equation}
		(\hat\tribF)_M=\{ A\cap M\tq A\in\hat\tribF \}=\{ A\subset M\tq A\in\hat\tribF \}
	\end{equation}
	et
	\begin{equation}
		\tribF_M=\{ A\cap M\tq A\in\tribF \}=\{ A\subset M\tq A\in\tribF \}.
	\end{equation}

	Pour prouver \( (\hat\tribF)_M=\widehat{\tribF_M}\) il faudra faire deux inclusions, et nous avons l'embarras du choix.
	\begin{subproof}
		\spitem[Première : \( \widehat{\tribF_M}\subset\{ M\cap A\tq A\in\hat\tribF \}\)]
		Un élément de \( \widehat{\tribF_M}\) est de la forme \( B\cup N\) où \( B\in \tribF_M\) et où \( N\) est négligeable\footnote{Pour rappel, une partie est négligeable quand elle est inclue à une partie de mesure nulle.} dans \( M\). Vu que \( B\in\tribF_M\), il existe \( A\in\tribF\) tel que \( B=A\cap M\). Vu que \( B\) et \( N\) sont dans \( M\) nous pouvons «factoriser» l'intersection :
		\begin{equation}
			B\cup N=M\cap (A\cup N)
		\end{equation}
		avec \( N\) négligeable dans \( M\) et donc également négligeable dans \( \Omega\). Donc \( A\cup N\in \hat\tribF\).

		\spitem[Deuxième : \( \{ M\cap A\tq A\in\hat\tribF \}\subset \widehat{\tribF_M}\)]

		Soit \( A\in\hat\tribF\). Nous avons une partie négligeable \( N\) de \( \Omega\) et un élément \( B\in \tribF\) tels que \( A=B\cup N\). Nous avons la décomposition
		\begin{equation}        \label{EQooWWQEooDRlnLN}
			M\cap(B\cup N)=(M\cap B)\cup(M\cap N).
		\end{equation}
		Il s'agit maintenant de nous assurer que cette décomposition implique que \( M\cap(B\cup N)\in \widehat{\tribF_M}\).

		Soit \( N_1\in \tribF\) tel que \( \mu(N_1)=0\) et \( N\subset N_1\). Puisque \( M\cap N_1\in \tribF\) (intersections dans une tribu), nous pouvons écrire
		\begin{equation}
			M\cap N\subset M\cap N_1
		\end{equation}
		avec \( \mu(M\cap N_1)=0\). Cela pour dire que \( M\cap N\) est négligeable dans \( M\). La décomposition \eqref{EQooWWQEooDRlnLN} est donc bien une union d'un élément de \( \tribF_M\) avec un négligeable de \( M\), et donc bien un élément de \( \widehat{\tribF_M}\).
	\end{subproof}
\end{proof}

\begin{normaltext}
	La principale application de la proposition \ref{PROPooAMIEooRomnMG} est le cas où \( \tribF=\Borelien(\eR^n)\) et \( M\) est un borélien \( B\) de \( \eR^n\). Dans ce cas, la proposition explique que la tribu de Lebesgue sur \( B\) (complétée depuis les boréliens de la topologie induite) est donnée directement par l'intersection entre \( B\) et la tribu de Lebesgue de \( \eR^n\). Donc sans devoir passer par la topologie induite, les boréliens et la completion :
	\begin{equation}
		\Lebesgue(\eR^n)_M=\widehat{\Borelien(\eR^n)_M}.
	\end{equation}
	Exemple dans la proposition \ref{PROPooHMSCooRIjcJq} qui donne une structure d'espace mesuré dans \( S^1\) à partir de la mesure de Lebesgue sur \( \eC\).
\end{normaltext}

%---------------------------------------------------------------------------------------------------------------------------
\subsection{Prolongement}
%---------------------------------------------------------------------------------------------------------------------------

Le théorème suivant est parfois nommé théorème d'extension de Carathéodory, par exemple sur Wikipédia. Le théorème de Carathéodory en étant un des ingrédients principaux, on comprend.
\begin{theorem}[Prolongement de Hahn\cite{MesureLebesgueLi}]    \label{ThoLCQoojiFfZ}
	Soit \( \tribA\) une algèbre de parties d'un ensemble \( S\) et \( \mu\) une mesure sur \( (S,\tribA)\). Soit \( \tribF=\sigma(\tribA)\) la tribu engendrée par \( \tribA\). Alors
	\begin{enumerate}
		\item
		      La mesure \( \mu\) se prolonge en une mesure \( m\) sur \( \tribF\).
		\item
		      Si \( \mu\) est \( \sigma\)-finie alors le prolongement est unique et \( m\) est \( \sigma\)-finie.
		\item
		      Si \( \mu\) est finie, alors \( m\) l'est aussi.
	\end{enumerate}
\end{theorem}
\index{théorème!prolongement de Hahn}
\index{prolongement!théorème de Hahn}

\begin{proof}
	La proposition~\ref{PropIUOoobjfIB} nous donne une mesure extérieure \( \mu^*\) sur \( S\) dont la restriction à \( \tribA\) est \( \mu\). Si \( \tribM\) est la tribu des parties \( \mu^*\)-mesurables de \( S\) alors le théorème de Carathéodory~\ref{ThoUUIooaNljH} nous dit que \( (S,\tribM,\mu^*)\) est un espace mesuré.
	\begin{subproof}
		\spitem[\( \tribA\subset\tribM\)]
		Cette partie est une adaptation de ce qui a déjà été fait dans la preuve de la proposition~\ref{PropOJFoozSKAE}. Soit \( A\in\tribA\) et \( X\in S\); nous devons prouver la relation de la définition~\ref{DefTRBoorvnUY}. Comme \( \mu^*\) est une mesure extérieure nous avons automatiquement
		\begin{equation}
			\mu^*(X)\leq \mu^*(X\cap A)+\mu^*(X\cap A^c).
		\end{equation}
		Il reste à prouver l'inégalité inverse. Soit une suite \( B_k\) d'éléments de \( \tribA\) telle que \( X\subset\bigcup_kB_k\); nous avons alors
		\begin{equation}
			\mu^*(X\cap A)\leq \mu^*\big( \bigcup_{k=1}^{\infty}B_k\cap A \big)\leq \sum_{k=1}^{\infty}\mu^*(B_k\cap A)=\sum_k\mu(B_k\cap A)
		\end{equation}
		où nous avons utilisé la définition~\ref{DefUMWoolmMaf}\ref{ItemARKooppZfDaiii} ainsi que le lemme~\ref{LemBFKootqXKl}. De la même façon,
		\begin{equation}
			\mu^*(X\cap A^c)\leq \sum_k\mu(B_k\cap A^c).
		\end{equation}
		Mettant les deux bouts ensemble, en remarquant que \( B_k\cap A\in\tribA\) et donc que \( \mu^*(B_k\cap A)=\mu(B_k\cap A)\),
		\begin{equation}
			\mu^*(X\cap A)+\mu^*(X\cap A^c)\leq \sum_k\mu(B_k\cap A)+\mu(B_k\cap A^c)=\sum_k\mu(B_k).
		\end{equation}
		La somme \( \mu^*(X\cap A)+\mu^*(X\cap A^c)\) est donc inférieure à chacun des éléments de l'ensemble sur lequel on prend l'infimum pour définir\footnote{Définition~\ref{EqRNJooQrcoa}.} \( \mu^*(X)\), donc
		\begin{equation}
			\mu^*(X\cap A)+\mu^*(X\cap A^c)\leq \mu^*(X).
		\end{equation}
	\end{subproof}

	A fortiori nous avons \( \sigma(\tribA)\subset\tribM\) et donc \( (S,\sigma(\tribA),\mu^*)\) est un espace mesuré. Cela prouve l'existence d'une mesure prolongeant \( \mu\) à \( \sigma(\tribA)\).

	\begin{subproof}
		\spitem[Unicité]

		Nous supposons à présent que \( \mu\) est \( \sigma\)-finie. Soient \( m_1\) et \( m_2\) deux mesures prolongeant \( \mu\) et définies sur une tribu contenant \( \tribA\). Nous posons
		\begin{equation}
			\tribC=\{ A\in\tribA\tq\mu(A)<\infty \}.
		\end{equation}
		Dans l'optique d'utiliser le théorème d'unicité des mesures~\ref{ThoJDYlsXu}, nous prouvons que \( \sigma(\tribA)=\sigma(\tribC)\). Vu que \( \mu\) est \( \sigma\)-finie, il existe une suite croissante \( (S_n)\) d'éléments de \( \tribA\) telle que \( S=\bigcup_nS_n\) et \( \mu(S_n)<\infty\). Alors si \( A\in\tribA\) nous avons \( A=\bigcup_n(A\cap S_n)\), et donc \( A\in\sigma(\tribC)\). Donc \( \tribA\subset\sigma(\tribC)\). Mais étant donné que \( \tribC\subset\tribA\) nous avons aussi \( \sigma(\tribC)\subset\sigma(\tribA)\). Au final \( \sigma(\tribA)=\sigma(\tribC)\).

		Les mesures \( m_1\) et \( m_2\) sont des mesures sur \( \sigma(\tribC)\) coïncidant sur \( \tribC\) (parce que \( \tribC\subset\tribA\)). De plus la classe \( \tribC\) est stable par intersection finie et contient une suite croissante dont l'union est \( S\) (parce que \( \mu\) est \( \sigma\)-finie).

		Le théorème~\ref{ThoJDYlsXu} nous dit alors que \( m_1\) et \( m_2\) coïncident sur \( \sigma(\tribC)=\sigma(\tribA)\).

		\spitem[Extension finie et \( \sigma\)-finie]

		Enfin si \( \mu\) est \( \sigma\)-finie il existe \( S_n\in\tribA\) avec \( \mu(S_n)<\infty\) et \( \bigcup_nS_n=S\). Ces ensembles vérifient tout autant \( m(S_n)=\mu(S_n)<\infty\) pour tout prolongement \( m\) de \( \mu\).

		Idem si \( \mu\) est finie, tout prolongement est fini.
	\end{subproof}
\end{proof}

\begin{example}[\cite{MesureLebesgueLi}] \label{ExKCEoolsZrL}
	Soit \( \tribA\), l'algèbre de parties de \( \eR\) formée par les réunions finies d'intervalles de la forme \( \mathopen] -\infty , a \mathclose[\), \( \mathopen[ a , b [\) et \( \mathopen[ b , +\infty [\) avec \( -\infty<a\leq b<+\infty\). Notons que les singletons ne font pas partie de \( \tribA\) parce que \( \mathopen[ a ,a[=\emptyset\). Nous posons
	\begin{equation}
		\mu(A)=\begin{cases}
			0      & \text{si } A=\emptyset \\
			\infty & \text{sinon}.
		\end{cases}
	\end{equation}
	Cela donne une mesure (non \( \sigma\)-finie) sur \( (\eR,\tribA)\).

	Nous allons prouver que la tribu engendrée par \( \tribA\) est la tribu des boréliens et que \( \mu\) accepte (au moins) deux prolongements distincts à \( \sigma(\tribA)\).

	D'abord nous avons
	\begin{equation}
		\mathopen] a , b \mathclose[=\big( \mathopen] -\infty , a \mathclose[\cup \mathopen[ b , +\infty [ \big)\cap\mathopen[ a , b [,
	\end{equation}
	donc toutes les boules ouvertes appartiennent à \( \sigma(\tribA)\). Ces dernières comprenant une base dénombrable de la topologie de \( \eR\) (par la proposition~\ref{PropNBSooraAFr}), tous les ouverts de \( \eR\) sont dans \( \sigma(\tribA)\). Par conséquent \( \Borelien(\eR)\subset(\eR^d)\). Mais en même temps tous les éléments de \( \tribA\) sont des boréliens, donc \( \Borelien(\eR)=\sigma(\tribA)\) parce que la fermeture en tant qu'algèbre de parties est plus petite que la fermeture en tant que tribu.

	La mesure de comptage prolonge \( \mu\) parce qu'à part l'ensemble vide, tous les éléments de \( \tribA\) sont infinis. Notons que les singletons sont dans \( \sigma(\tribA)\), donc la mesure de comptage prend d'autres valeurs que \( 0\) et \( +\infty\).

	Par ailleurs la mesure
	\begin{equation}
		\mu'(A)=\begin{cases}
			0       & \text{si } A=\emptyset \\
			+\infty & \text{sinon}
		\end{cases}
	\end{equation}
	est également une mesure prolongeant \( \mu\) à \( \sigma(\tribA)=\Borelien(\eR)\).

	La mesure de comptage et \( \mu'\) sont deux prolongements distincts de \( \mu\).
\end{example}

%TODO : quelle partie de R n'est pas borélienne ?

\begin{example}[\cite{MesureLebesgueLi}]
	Nous montrons maintenant une mesure non \( \sigma\)-finie qui se prolonge en deux mesures distinctes, toutes deux \( \sigma\)-finies.

	Nous considérons la même algèbre \( \tribA\) de parties que celle donnée dans l'exemple~\ref{ExKCEoolsZrL}, mais cette fois vue sur \( \eQ\) uniquement. La mesure de comptage \( m\) sur \( (\eQ,\tribA)\) n'est pas \( \sigma\)-finie.

	Puisque les singletons sont des boréliens, nous avons \( \sigma(\tribA)=\partP(\eQ)\), ce qui fait que \( (\eQ,\sigma(\tribA),m)\) est un prolongement \( \sigma\)-fini de \( m\). L'espace mesuré \( (\eQ,\sigma(\tribA),2m)\) est également \( \sigma\)-fini et est un prolongement distinct de \( (\eQ,\tribA,m)\).
\end{example}

\begin{proposition}     \label{PROPooORDCooJEsjzR}
	Soient des espaces mesurés \( (S_1,\tribF_1,\mu_1)\) et \( (S_2,\tribF_2, \mu_2)\) ainsi qu'une application \( \varphi\colon S_1\to S_2\) avec les hypothèses suivantes :
	\begin{enumerate}
		\item
		      \( \varphi\) est une bijection,
		\item
		      \( \varphi\) est mesurable d'inverse mesurable,
		\item
		      si \( \mu_1(A)=0\) alors \( \mu_2\big( \varphi(A) \big)=0\),
		\item
		      si \( \mu_2(A)=0\) alors \( \mu_1\big( \varphi^{-1}(A) \big)=0\).
	\end{enumerate}
	Alors
	\begin{equation}
		\hat\tribF_2=\varphi(\hat\tribF_1).
	\end{equation}
\end{proposition}

\begin{proof}
	Nous prouvons que \( \hat\tribF_2\subset \varphi(\hat\tribF_1)\). Vu la symétrie des hypothèses, l'inclusion inverse se fera de même.

	Soit \( A\in\hat\tribF_2\). Nous avons \( A=B\cup N\) avec \( B\in \tribF_2\) et \( N\), une partie \( \mu_2\)-négligeable. Nous considérons \( N_1\in\tribF_2\) tel que \( \mu_2(N_1)=0\) et \( N\subset N_1\). Notre but est maintenant de prouver que \( \varphi^{-1}(B\cup N)\in \hat\tribF_1\).

	Comme \( \varphi\) est une bijection, nous avons
	\begin{equation}
		\varphi^{-1}(B\cup N)=\varphi^{-1}(B)\cup \varphi^{-1}(N).
	\end{equation}
	Là-dedans, \( \varphi^{-1}(B)\in \tribF_1\) parce que \( \varphi\) est borélienne. Il nous reste à voir que \( \varphi^{-1}(N)\) est \( \mu_1\)-négligeable. Puisque \( N\subset N_1\), nous avons \( \varphi^{-1}(N)\subset\varphi^{-1}(N_1)\) où \( \varphi^{-1}(N_1)\in\tribF_1\).

	Par construction, \( \mu_2(N_1)=0\) et par hypothèse, \( \mu_1\big( \varphi^{-1}(N_1) \big)=0\).

	Au total,
	\begin{equation}
		\varphi^{-1}(B\cup N)=\underbrace{\varphi^{-1}(B)}_{\in\tribF_1}\cup\underbrace{\varphi^{-1}(N)}_{\mu_1\text{-négligeable}}\in\hat\tribF_1.
	\end{equation}
\end{proof}

%---------------------------------------------------------------------------------------------------------------------------
\subsection{Mesure image}
%---------------------------------------------------------------------------------------------------------------------------

Le produit d'une mesure par une fonction est défini par la propriété~\ref{PropooVXPMooGSkyBo}.

\begin{propositionDef}[Mesure image\cite{TribuLi}]     \label{PropJCJQooAdqrGA}
	Soient \( (S_1,\tribF_1)\) et \( (S_2,\tribF_2)\) des espaces mesurables. Soit \( \varphi\colon S_1\to S_2\) une application mesurable. Si \( m_1\) est une mesure positive sur \( S_1\) alors l'application définie par
	\begin{equation}
		m_2(A_2)=m_1\big( \varphi^{-1}(A_2) \big)
	\end{equation}
	est une mesure positive sur \( (S_2,\tribF_2)\).

	La mesure \( m_2\) ainsi définie est la \defe{mesure image}{mesure!image} de \( m_1\) par l'application \( \varphi\). Elle est notée \( \varphi(m_1)\).
\end{propositionDef}

\begin{proof}
	Il y a deux choses à vérifier pour avoir une mesure positive\footnote{Définition~\ref{DefBTsgznn}}. D'abord pour l'ensemble vide :
	\begin{equation}
		m_2(\emptyset)=m_1\big( \varphi^{-1}(\emptyset) \big)=m_1(\emptyset)=0.
	\end{equation}
	Ensuite pour l'additivité. Soient \( A_n\) dans \( \tribF_2\) des parties deux à deux disjointes et telles que \( \bigcup_nA_n\in\tribF_2\). Alors nous avons
	\begin{subequations}
		\begin{align}
			m_2\big( \bigcup_nA_n \big) & =m_1\Big( \varphi^{-1}(\bigcup_nA_n) \Big) \\
			                            & =m_1\big( \bigcup_n\varphi^{-1}(A_n) \big) \\
			                            & =\sum_nm_1\big( \varphi(A_n) \big)         \\
			                            & =\sum_nm_2(A_n).
		\end{align}
	\end{subequations}
\end{proof}

\begin{lemma}
	Soient deux espaces mesurables \( (S_1,\tribF_1)\) et \( (S_2,\tribF_2)\) ainsi que deux mesures \( \mu\) et \( \nu\) sur \( (S_1,\tribF_1)\). Si \( \varphi\colon S_1\to S_2\) est mesurable et si \( \mu\leq \nu\) alors \( \varphi(\mu)\leq \varphi(\nu)\).
\end{lemma}

\begin{proof}
	Soit \( B\) mesurable dans \( (S_2,\tribF_2)\) (c'est-à-dire \( B\in \tribF_2\)). Alors
	\begin{equation}
		\varphi(\mu)(B)=\mu\big( \varphi^{-1}(B) \big)\leq\nu\big( \varphi^{-1}(B) \big)=\varphi(\nu)(B).
	\end{equation}
\end{proof}

Il est naturel de se demander comment il faut intégrer par rapport à une mesure image. La réponse sera dans le théorème~\ref{THOooVADUooLiRfGK}.

%---------------------------------------------------------------------------------------------------------------------------
\subsection{Régularité d'une mesure}
%---------------------------------------------------------------------------------------------------------------------------

Certaines mesures ont de la compatibilité avec la topologie. Nous allons étudier ça.

\begin{lemma}[\cite{YHRSDGc}]       \label{LEMooCGKXooYWjRwk}
	Soit un espace topologique métrique \( (\Omega,d)\). Nous considérons sa tribu des boréliens\footnote{Définition \ref{DEFooQBQGooTqGdtY}.} \( \Borelien(\Omega)\) ainsi qu'une mesure finie \( \mu\) sur \( \big(\Omega,\Borelien(\Omega)\big)\).

	Soit un borélien \( A\) de \( \Omega\) et \( \epsilon>0\).

	Il existe un fermé \( F\) et un ouvert \( V\) de \( \Omega\) tels que
	\begin{enumerate}
		\item
		      \( F\subset A\subset V\)
		\item
		      \( \mu(V\setminus F)<\epsilon\).
	\end{enumerate}
\end{lemma}

\begin{proof}
	Soit la famille \( \tribD\) des parties \( D\) de \( \Omega\) qui vérifient la propriété suivante : pour tout \( \epsilon>0\), il existe un fermé \( F\) et un ouvert \( V\) de \( \Omega\) tels que \( F\subset D\subset V\) et \( \mu(V\setminus F)<\epsilon\).

	Nous allons prouver que \( \tribD\) est une tribu qui contient tous les ouverts.

	\begin{subproof}
		\spitem[\( \tribD\) contient les ouverts]
		Soit un ouvert \( D\). Nous posons
		\begin{equation}
			F_n=\{ x\in \Omega\tq d(x,D^c)\geq 2^{-n} \}.
		\end{equation}
		\begin{subproof}
			\spitem[\( F_n\) est fermé]

			Le lemme \ref{LEMooJNRTooZyKiFC} montre que le complémentaire \( F_n^c\) est ouvert. Donc \( F_n\) est fermé.
			\spitem[\( D\subset \bigcup_{n\in \eN}F_n\)]
			Si \( x\in D\), alors il existe \( \delta>0\) tel que \( B(x,\delta)\subset D\) (parce que \( D\) est ouvert). Donc \( d(x,V^c)\geq \delta\). Donc \( x\in F_n\) pour \( 2^{-n}<\delta\).
			\spitem[\( \bigcup_{n\in \eN}F_n\subset D\)]
			Si \( x\in F_n\), nous avons \( d(x,D^c)>0\), c'est-à-dire que \( x\) n'est pas dans \( D^c\). Autrement dit, \( x\in D\).
			\spitem[\( \bigcup_{n\in \eN}F_n = D\)]
			Nous avons donc l'égalité
			\begin{equation}
				D=\bigcup_{n\in \eN}F_n.
			\end{equation}
		\end{subproof}
		Vu que \( F_n\subset F_{n+1}\), le lemme \ref{LemAZGByEs}\ref{ItemJWUooRXNPci} nous indique que
		\begin{equation}
			\lim_{n\to \infty} \mu(F_n)=\mu\big( \bigcup_{k\in \eN}F_k \big)=\mu(D).
		\end{equation}
		Étant donné que la mesure est finie, nous pouvons écrire cela sous la forme
		\begin{equation}
			\mu(D)-\mu(F_n)\to 0.
		\end{equation}
		Pour chaque \( n\) nous avons l'encadrement
		\begin{equation}
			F_n\subset D\subset D
		\end{equation}
		où \( F_n\) et \( D\) sont ouverts. Lorsque \( \epsilon\) est donné, il suffit de prendre \( n\) assez grand pour avoir \( \mu(D\setminus F_n)<\epsilon\) pour avoir un encadrement de \( D\) par un fermé et un ouvert (\( D\) lui-même) dont la différence des mesures est plus petite que \( \epsilon\).

		Tout cela pour dire que \( D\in\tribD\).
		\spitem[\( \tribD\) est une tribu]
		Il faut vérifier les trois points de la définition \ref{DefjRsGSy}.
		\begin{subproof}
			\spitem[\( \Omega\in\tribD\)]
			Nous venons de voir que les ouverts sont dans \( \tribD\). Or \( \Omega\) est un ouvert.
			\spitem[\( D\in \tribD\) implique \( D^c\in \tribD\)]
			Soit \( F\) fermé et \( V\) ouvert tels que \( F\subset D\subset V\). Nous avons aussi
			\begin{equation}
				V^c\subset D^c\subset F^c
			\end{equation}
			où \( V^c\) est fermé et \( F^c\) est ouvert. De plus \( F^c\setminus V^c = V\setminus F\) et donc
			\begin{equation}
				\mu(F^c\setminus V^c)=\mu(V\setminus F).
			\end{equation}
			Nous pouvons donc choisir \( F\) et \( V\) pour avoir \( \mu(F^c\setminus V^c)<\epsilon\).
			\spitem[\( \bigcup_{i\in \eN}D_i\in\tribD\)]
			Soient \( D_i\in \tribD\). Pour chaque \( n\) nous posons
			\begin{equation}
				F_n\subset D_n\subset V_n
			\end{equation}
			en choisissant \( V_n\) et \( F_n\) de telle sorte que \( \mu(V_n\setminus F_n)<2^{-n}\epsilon\).

			Nous posons
			\begin{equation}
				Y_N=\bigcup_{n=0}^NF_n,
			\end{equation}
			et
			\begin{equation}
				Y=\bigcup_{n=0}^{\infty}F_n.
			\end{equation}
			Chacun des \( Y_N\) est fermé en tant qu'union finie de fermés (lemme \ref{LemQYUJwPC}\ref{ItemKJYVooMBmMbG}). Mais \( Y\) ne l'est pas spécialement\footnote{Par exemple \( A_n=\mathopen[ 1/n , 2 \mathclose]\) sont des fermés dont l'union est \( \mathopen] 0 , 2 \mathclose]\) qui n'est pas fermé.}. Le lemme \ref{LemAZGByEs} nous dit cependant que \( \mu(Y)=\lim_{N\to \infty} \mu(Y_N)\).

			Nous posons
			\begin{equation}
				D=\bigcup_{n=0}^{\infty}D_n
			\end{equation}
			ainsi que
			\begin{equation}
				V=\bigcup_{n\in \eN} V_n.
			\end{equation}
			La partie \( V\) est ouverte dans \( \Omega\) comme union d'ouverts (c'est dans le définition d'une topologie). Nous avons, pour tout \( N\), l'encadrement
			\begin{equation}        \label{EQooOALEooLAHpVi}
				Y_N=\bigcup_{n=0}^NF_n\subset Y\subset D\subset V.
			\end{equation}
			Nous prouvons à présent que \( \lim_{N\to \infty} \mu(V\setminus Y_N)=0\), de telle sorte que l'encadrement \eqref{EQooOALEooLAHpVi} dise que \( D\in\tribD\).

			D'abord nous avons
			\begin{equation}        \label{EQooYVVBooCNvSnx}
				V\setminus Y\subset \bigcup_n(V_n\setminus F_n)
			\end{equation}
			parce que si \( x\in V\setminus Y\), alors \( x\in V_i\) pour un certain \( i\), mais vu que \( x\) n'est pas dans \( Y\), il n'est dans aucun des \( F_n\) donc en particulier pas dans \( F_i\) et \( x\in V_n\setminus F_i\).

			Un peu de calcul :
			\begin{subequations}
				\begin{align}
					\mu(V)-\mu(Y) & = \mu(V\setminus Y)   \label{SUBEQooCSQYooYXBhYy}                              \\
					              & \leq\mu\big( \bigcup_n(V_n\setminus F_n) \big)     \label{SUBEQooVUCJooHjObZw} \\
					              & \leq \sum_{n=0}^{\infty}\mu(V_n\setminus F_n)      \label{SUBEQooTAGKooTtYtzw} \\
					              & =\sum_{n=0}^{\infty}2^{-n}\epsilon                                             \\
					              & =2\epsilon.        \label{SUBEQooMDAAooXKEajJyi}
				\end{align}
			\end{subequations}
			Justifications:
			\begin{itemize}
				\item Pour \eqref{SUBEQooCSQYooYXBhYy}, c'est le lemme \ref{LemPMprYuC}.
				\item Pour \eqref{SUBEQooVUCJooHjObZw}, c'est \eqref{EQooYVVBooCNvSnx}.
				\item Pour \eqref{SUBEQooTAGKooTtYtzw}, c'est le lemme \ref{LemPMprYuC}\ref{ITEMooABPYooFQEzqE}.
				\item Pour \eqref{SUBEQooMDAAooXKEajJyi}, c'est la série géométrique \eqref{EqPZOWooMdSRvY}.
			\end{itemize}

			Nous choisissons maintenant \( N\) assez grand pour que \( \mu(Y)-\mu(Y_N)<\epsilon\). Nous avons alors l'encadrement
			\begin{equation}
				Y_N\subset Y\subset D\subset V
			\end{equation}
			avec
			\begin{equation}
				\mu(V\setminus Y_N)=\mu(V)-\mu(Y_N)=\underbrace{\mu(V)-\mu(Y)}_{\leq 2\epsilon}+\mu(Y)-\mu(Y_N)\leq 2\epsilon+\epsilon=3\epsilon.
			\end{equation}
		\end{subproof}
	\end{subproof}
	Nous avons donc montré que \( \tribD\) était une tribu contenant les ouverts. Donc \( \tribD\) contient tous les boréliens.
\end{proof}

\begin{lemma}[\cite{YHRSDGc}]       \label{LEMooZDFVooFUgFGZ}
	Soit un espace topologique métrique \( (\Omega,d)\). Nous considérons sa tribu des boréliens \( \Borelien(\Omega)\) ainsi qu'une mesure \( \mu\) sur \( \big(\Omega,\Borelien(\Omega)\big)\).

	Soient un ouvert \( W\subset \Omega\) tel que \( \mu(W)<\infty\) et un borélien \( A\) tel que \( A\subset W\). Soit aussi \( \epsilon>0\).

	Il existe un fermé \( F\) et un ouvert \( V\) tels que
	\begin{enumerate}
		\item \( \mu(V)<\infty\),
		\item
		      \( \mu(V\setminus F)<\epsilon\),
		\item et \( F\subset A\subset V\).
	\end{enumerate}
\end{lemma}

\begin{proof}
	Vu que la mesure de \( W\) est finie, nous considérons la mesure finie
	\begin{equation}
		\begin{aligned}
			\nu\colon \Borelien(\Omega) & \to \mathopen[ 0 , \mu(W) \mathclose] \\
			B                           & \mapsto \mu(B\cap W).
		\end{aligned}
	\end{equation}
	La partie \( A\) étant borélienne; par le lemme \ref{LEMooCGKXooYWjRwk}, nous avons un fermé \( F\) et un ouvert \( V_1\) ouvert tels que
	\begin{equation}
		F\subset A\subset V_1
	\end{equation}
	et \( \nu(V_1\setminus F)<\epsilon\). Nous posons \( V=V_1\cap W\); vu que \( A\subset W\) et \( A\subset V_1\) nous avons aussi \( A\subset V_1\cap  W\) et donc l'encadrement
	\begin{equation}
		F\subset A\subset V\subset W.
	\end{equation}
	En ce qui concerne la mesure :
	\begin{equation}
		\mu(V\setminus F)=\mu(V)-\mu(F)=\mu(V\cap W)-\mu(F\cap W)=\nu(B)-\nu(F)<\epsilon.
	\end{equation}
\end{proof}

\begin{theorem}[\cite{TribuLi}]     \label{ThoPKGEooVrpsGU}
	Soit \( X\) un espace métrique et \( m\) une mesure positive bornée sur \( \big(X,\Borelien(X)\big)\). Alors si \( B\) est un borélien,
	\begin{enumerate}
		\item
		      Régularité extérieure : \( m(B)=\inf\{ m(\Omega)\text{où } \Omega\text{ est un ouvert contenant } B \}\)
		\item
		      Régularité intérieure : \( m(B)=\sup\{ m(F) \text{où } F\text{ est un fermé, } F\subset B \}\).
	\end{enumerate}
\end{theorem}

\begin{proof}
	Soit \( \tribF\) l'ensemble des \( B\in\Borelien(X)\) tels que pour tout \( \epsilon>0\), il existe \( \Omega_{\epsilon}\) ouvert et \( F_{\epsilon}\) fermé tels que \( F_{\epsilon}\subset B\subset \Omega_{\epsilon}\) et \( m(\Omega_{\epsilon}\setminus F_{\epsilon})\leq \epsilon\). Nous allons montrer que \( \tribF\)
	\begin{itemize}
		\item est une tribu
		\item contient les ouverts
		\item est inclus à la tribu borélienne (ça c'est dans la définition de \( \tribF\)).
	\end{itemize}
	De ces trois points nous déduirons que \( \tribF=\Borelien(X)\).
	\begin{subproof}
		\spitem[\( \tribF\) contient les ouverts]
		Soit \( \Omega\) un ouvert de \( X\). Alors \( \Omega^c\) est fermé et \( d(x,\Omega^c)=0\) si et seulement si \( x\in \Omega^c\) par la proposition~\ref{PropGULUooNzqZKj}. Nous pouvons donc écrire
		\begin{equation}
			\Omega^c=\bigcap_{n\geq 1}\{ x\in X\tq d(x,\Omega^c)<\frac{1}{ n } \}.
		\end{equation}
		En passant au complémentaire et en posant \( F_n=\{ x\in X\tq d(x,\Omega^c)\geq \frac{1}{ n } \}\) nous avons
		\begin{equation}
			\Omega=\bigcup_{n\geq 1}F_n.
		\end{equation}
		Chacun des \( F_n\) est fermé parce que \( F_n\) est l'image réciproque du fermé \( \mathopen[ \frac{1}{ n } , \infty \mathclose[\) par l'application \( x\mapsto d(x,\Omega^c)\) qui est continue. De plus les \( F_n\) forment une suite croissante, donc le lemme~\ref{LemAZGByEs} nous assure que \( m(\Omega)=\lim_{n\to \infty}m(F_n)\). Et le lemme~\ref{LemPMprYuC} que \( m(\Omega\setminus F_n)=m(\Omega)-m(F_n)\).

		Soit \( \epsilon>0\). Il existe alors \( n_{\epsilon}\geq 1\) tel que
		\begin{equation}
			m(\Omega\setminus F_n)=m(\Omega)-m(F_n)\leq \epsilon.
		\end{equation}
		Bref si \( \Omega\) est ouvert nous considérons \( \Omega_{\epsilon}=\Omega\) et \( F_{\epsilon}=F_{n_{\epsilon}}\) et nous avons
		\begin{equation}
			F_{\epsilon}\subset \Omega\subset \Omega_{\epsilon}
		\end{equation}
		avec \( m(\Omega_{\epsilon}\setminus F_{\epsilon})\leq \epsilon\).

		L'ensemble \( \tribF\) contient les ouverts.

		\spitem[\( \tribF\) est une tribu]
		Il y a à vérifier les trois conditions de la définition~\ref{DefjRsGSy}.
		\begin{subproof}
			\spitem[Les ensembles faciles]
			Les ensembles \( X\) et \( \emptyset\) sont dans \( \tribF\) parce qu'ils sont ouverts et fermés.
			\spitem[Complémentaire]
			Soit \( B\in \tribF\), soit \( \epsilon>0\) et les ensembles \( F_{\epsilon} \) et \( \Omega_{\epsilon}\) qui vont avec. Alors en passant au complémentaire nous avons
			\begin{equation}
				\Omega_{\epsilon}^c\subset B^c\subset F_{\epsilon}^c
			\end{equation}
			De plus
			\begin{equation}
				F_{\epsilon}^c\setminus \Omega_{\epsilon}^c=F_{\epsilon}^c\cap(\Omega_{\epsilon}^c)^c=F_{\epsilon}^c\cap \Omega_{\epsilon}=\Omega_{\epsilon}\setminus F_{\epsilon}.
			\end{equation}
			Par conséquent
			\begin{equation}
				m(F_{\epsilon}^c\setminus \Omega_{\epsilon}^c)=m(\Omega_{\epsilon}\setminus F_{\epsilon})\leq \epsilon.
			\end{equation}
			Cela montre que \( B^c\in \tribF\).
			\spitem[Union dénombrable]
			Soient \( (B_n)\) une suite d'éléments de \( \tribF\) et \( \epsilon>0\). Pour chaque \( n\), le lemme \ref{LEMooCGKXooYWjRwk} nous permet de choisir un ouvert \( \Omega_n\) et un fermé \( F_n\) tels que \( F_n\subset  B_n\subset \Omega_n\) et
			\begin{equation}
				m(\Omega_n\setminus F_n)\leq \frac{ \epsilon }{ 2^{n+2} }.
			\end{equation}
			Puisque \( \Omega_n\setminus B_n\subset \Omega_n\setminus F_n\) nous avons aussi
			\begin{equation}
				m(\Omega_n\setminus B_n)\leq m(\Omega_n\setminus F_n)\leq \frac{ \epsilon }{ 2^{n+2} }.
			\end{equation}
			Nous posons \( \Omega=\bigcup_{n\geq 1}\Omega_n\) (un ouvert) et \( B=\bigcup_{n\geq 1}B_n\) ainsi que \( A=\bigcup_{n\geq 1}F_n\) (qui n'est pas spécialement fermé).

			Le but est de majorer \( m(\Omega\setminus F)\) où \( F\) est un fermé qui est encore à déterminer. Calculons déjà ceci :
			\begin{subequations}
				\begin{align}
					\Omega\setminus B & =\bigcup_n\Omega_n\cap\big( \bigcup_kB_k \big)^c                                            \\
					                  & =\left( \bigcup_n\Omega_n \right)\cap\left( \bigcap_kB_k^c \right)                          \\
					                  & =\bigcup_n\Big( \Omega_n\cap\big( \bigcap_kB_k^c \big) \Big)    \label{SUBEQooAUBIooQuHHEK} \\
					                  & \subset\bigcup_n\big( \Omega_n\cap B_n^c \big)           \label{SUBEQooDZGJooKGHobO}        \\
					                  & =\bigcup_n(\Omega_n\setminus B_n)       \label{SUBEQooZCILooIOSiSL}.
				\end{align}
			\end{subequations}
			Justifications.
			\begin{itemize}
				\item Pour \eqref{SUBEQooAUBIooQuHHEK}. Pour toute suite d'ensembles on a \( \left( \bigcup_kA_k \right)^x=\bigcap_kA_k^c\).
				\item Pour \eqref{SUBEQooDZGJooKGHobO}. Si \( x\in \Omega_n\cap\left( \bigcap_kB_k^c \right)\) pour un certain \( n\), alors en particulier \( x\in \bigcap_kB_k^c\) et \( x\in B_n^c\). Donc \( x\in \Omega_n\cap B_n^c\).
				\item Pour \eqref{SUBEQooZCILooIOSiSL}, notez que l'union n'est pas spécialement disjointe.
			\end{itemize}
			Par conséquent,
			\begin{equation}
				m(\Omega\setminus B)\leq \sum_{n=1}^{\infty}m(\Omega_n\setminus B_n)\leq \sum_{n=1}^{\infty}\frac{ \epsilon }{ 2^{n+2} }=\frac{ \epsilon }{ 4 }.
			\end{equation}
			De la même façon nous avons
			\begin{equation}
				B\setminus A=\big( \bigcup_{n=1}^{\infty}B_n \big)\cap\big( \bigcup_{k=1}^{\infty}F_n \big)^c\subset \bigcup_{n=1}^{\infty}B_n\setminus F_n.
			\end{equation}
			Nous avons alors le inégalités de mesures
			\begin{subequations}
				\begin{align}
					m(B\setminus A) & \leq \sum_{n=1}^{\infty}m(B_n\setminus F_n)      \\
					                & \leq \sum_{n=1}^{\infty}m(\Omega_n\setminus F_n) \\
					                & \leq \frac{ \epsilon }{ 4 }.
				\end{align}
			\end{subequations}
			C'est vraiment dommage que \( A\) ne soit pas en général un fermé, sinon il répondrait à la question. Nous posons \( F'_1=F_1\) et \( F'_n=\bigcup_{k=1}^nF_k\). En tant qu'unions finies de fermés, les \( F'_n\) sont des fermés (lemme~\ref{LemQYUJwPC}\ref{ItemKJYVooMBmMbG}). De plus la suite \( (F'_n)\)  est croissante et l'union est \( A\). Par le lemme~\ref{LemAZGByEs}\ref{ItemJWUooRXNPci} nous avons
			\begin{equation}
				m(A)=m\big( \bigcup_nF'_n \big)=\lim_{n\to \infty} m(F'_n).
			\end{equation}
			Il existe donc \( n_{\epsilon}\) tel que
			\begin{equation}
				m(A)-m(F'_{n_{\epsilon}})\leq \epsilon
			\end{equation}
			Nous posons \( F=F'_{n_{\epsilon}}\). Comme \( F\subset A\) nous avons aussi \( m(A\setminus F)=m(A)-m(F)\leq \epsilon\). Et en plus \( F\subset A\subset B\subset \Omega\), ce qui donne bien la propriété voulue \( F\subset B\subset \Omega\). Il reste à nous assurer de \( m(\Omega\setminus F)\). Nous avons d'abord
			\begin{equation}
				m(B\setminus F)=m\big( (B\setminus A)\cup (A\setminus F) \big)=m(B\setminus A)+m(A\setminus F)\leq \frac{ 5\epsilon }{ 4 }.
			\end{equation}
			Et enfin :
			\begin{equation}
				m(\Omega\setminus F)=m\big( (\Omega\setminus B)\cup (B\setminus F) \big)=m(\Omega\setminus B)+m(B\setminus F)\leq \frac{ 6\epsilon }{ 4 }.
			\end{equation}
			Et donc à redéfinition près de \( \epsilon\), c'est d'accord.
		\end{subproof}

		Il est donc établi que \( \tribF\) est une tribu. Qui plus est, l'ensemble \( \tribF\) est une tribu incluse aux boréliens et contenant les ouverts. Ergo \( \tribF=\Borelien(X)\).

		\spitem[Régularité extérieure]
		Soit \( B\) un borélien et \( \epsilon>0\). Alors il existe \( F_{\epsilon}\) fermé et \( \Omega_{\epsilon} \) ouvert tels que \( F_{\epsilon}\subset B\subset \Omega_{\epsilon}\) et \( m(\Omega_{\epsilon}\setminus F_{\epsilon})\leq \epsilon\). Vu que \( B\subset \Omega_{\epsilon}\) pour tout \( \epsilon\), nous avons aussi
		\begin{equation}
			m(B)\leq \inf_{\epsilon}m(\Omega_{\epsilon}).
		\end{equation}
		Mais comme \( m(\Omega_{\epsilon})\geq m(B)\) pour tout \( \epsilon\), nous avons en réalité \( m(B)=\inf_{\epsilon}m(\Omega_{\epsilon})\).

		Soit maintenant un ouvert \( \Omega\) tel que \( B\subset \Omega\). Nous devons prouver l'existence d'un \( \epsilon>0\) tel que \( m(\Omega_{\epsilon})\leq m(\Omega)\). Cela permettra de conclure que l'infimum sur tous les ouverts contenant \( B\) est égal à l'infimum sur les ouverts de la forme \( \Omega_{\epsilon}\).

		Nous posons \( m(\Omega)=m(B)+\delta\) et avec \( \epsilon\leq \delta\) nous avons
		\begin{equation}
			m(\Omega_{\epsilon}\setminus B)\leq m(\Omega_{\epsilon}\setminus F_{\epsilon})\leq \epsilon
		\end{equation}
		et donc aussi
		\begin{equation}
			m(\Omega_{\epsilon})\leq m(B)+\epsilon\leq m(B)+\delta=m(\Omega).
		\end{equation}

		\spitem[Régularité intérieure]
		Elle se fait de même.
	\end{subproof}
\end{proof}

\begin{definition}      \label{DefFMTEooMjbWKK}
	Soit \( X\) un espace topologique et \( m\) une mesure positive sur \( \big( X,\Borelien(X) \big)\).
	\begin{enumerate}
		\item       \label{ItemTTPTooStDcpw}
		      \( m\) est une \defe{mesure de Borel}{mesure!de Borel} si elle est finie sur tout compact.
		\item
		      \( m\) est \defe{régulière extérieurement}{mesure!régulière!extérieure} si \( \forall B\in\Borelien(X)\),
		      \begin{equation}
			      m(B)=\inf\{ m(\Omega)\tq \Omega\text{ est ouvert et } B\subset \Omega \}
		      \end{equation}
		\item
		      \( m\) est \defe{régulière intérieurement}{mesure!régulière!intérieure} si \( \forall B\in\Borelien(X)\),
		      \begin{equation}
			      m(B)=\sup\{ m(K)\tq K\text{ est compact et } K\subset B  \}
		      \end{equation}
		\item
		      \( m\) est une mesure \defe{régulière}{mesure!régulière} si elle est régulière dans les deux sens.
		\item
		      \( m\) est une \defe{mesure de Radon}{mesure!de Radon} si elle est de Borel et régulière.
	\end{enumerate}
\end{definition}
\index{régularité!d'une mesure}

\begin{proposition}     \label{PropNCASooBnbFrc}
	Soit \( X\) un espace localement compact et dénombrable à l'infini\footnote{Définitions~\ref{DefEIBYooAWoESf} et~\ref{DefFCGBooLpnSAK}.} Alors toute mesure de Borel sur \( \big( X,\Borelien(X) \big)\) est de Radon.
\end{proposition}

\begin{proof}
	Nous avons une suite exhaustive\footnote{Définition~\ref{LemGDeZlOo}.} de compacts \( X_k\) tels que
	\begin{equation}
		X=\bigcup_{k\geq 1}X_k=\bigcup_{k\geq 1}\Int(X_k).
	\end{equation}
	\begin{subproof}
		\spitem[Régularité intérieure]
		Soit \( B\), un borélien de \( X\); nous avons \( B=\bigcup_{k\geq 1}(B\cap X_k)\) et comme cette union est croissante,
		\begin{equation}
			m(B)=\lim_{k\to \infty} m(B\cap X_k)
		\end{equation}
		par le lemme~\ref{LemAZGByEs}\ref{ItemJWUooRXNPci}. Dans la suite, il va y avoir beaucoup de considérations sur les topologies induites. Nous nommons \( \tau_k\) la topologie de \( X_k\) induite depuis celle de \( X\). Il ne faudra pas confondre les expressions «un compact \emph{de} \( X_k\)»  et «un compact \emph{dans} \( X_k\)». La première parle d'un compact pour la topologie \( \tau_k\). La seconde parle d'un compact pour la topologie de \( X\), inclus dans \( X_k\).


		Si \( a<m(B)\) alors il existe \( k\geq 1\) tel que \( a<m(B\cap X_k)\), c'est-à-dire
		\begin{equation}
			a<m(B\cap X_k)\leq m(B).
		\end{equation}
		Mais \( (X_k,m)\) est un espace mesuré borné parce que \( m\) est de Borel et \( X_k\) est compact. Par conséquent la (restriction de la) mesure \( m\) est régulière sur l'espace mesuré \( \big( X_k,\Borelien(X_k) \big)\) par le théorème~\ref{ThoPKGEooVrpsGU}. De plus l'ensemble \( B\cap X_k\) est un borélien de \( (X_k,\tau_k)\) parce que
		\begin{equation}
			B\cap X_k\in\Borelien(X)_{X_k}=\Borelien(X_k)
		\end{equation}
		où nous avons utilisé la propriété de compatibilité entre topologie induite et tribu des borélien du théorème~\ref{ThoSVTHooChgvYa}. Il existe donc un fermé \( F_{\epsilon}\) de \( (X_k,\tau_k)\) tel que
		\begin{subequations}
			\begin{numcases}{}
				F_{\epsilon}\subset B\cap X_k               \\
				m(B\cap X_k)\leq m(F_{\epsilon})+\epsilon.
			\end{numcases}
		\end{subequations}
		En mettant bout à bout les inégalités nous avons trouvé
		\begin{equation}        \label{EQooXRESooGsGIFO}
			a<m(B\cap X_k)\leq m(F_{\epsilon})+\epsilon.
		\end{equation}
		L'ensemble \( F_{\epsilon}\) est un compact de \( (X,\tau_X)\). En effet \( X_k\) étant fermé de \( (X,\tau_X)\), le lemme~\ref{LemBWSUooCCGvax} nous dit que \( F_{\epsilon}\) est un fermé de \( (X,\tau_X)\). Mais \( X_k\) étant compact, \( F_{\epsilon}\) est un fermé inclus dans un compact, il est donc compact (lemme~\ref{LemnAeACf}).

		Enfin nous prouvons la régularité intérieure de la mesure \( m\), c'est à dire que
		\begin{equation}
			m(B)=\sup\{ m(K)\tq \text{\( K\) est compact dans \( B\)} \}
		\end{equation}
		en vérifiant les deux conditions de la définition \ref{DefSupeA}. D'abord \( m(B)\geq \{ m(K)\tq \ldots \}\) parce que \( m(B)\geq m(K)\) pour tout \( K\subset B\). Ensuite prenons \( \epsilon>0\), et considérons l'inégalité \eqref{EQooXRESooGsGIFO} avec \( a=m(B)-\epsilon\). Alors nous avons
		\begin{equation}
			m(B)-2\epsilon<m(F_{\epsilon}).
		\end{equation}
		Cela prouve que \( m(B)-2\epsilon\) n'est pas un majorant de \( \{ m(K)\tq\ldots \}\).

		\spitem[Régularité extérieure]

		Soit un borélien \( B\) de \( X\). Si \( m(B)=\infty\) alors tous les ouverts contenant \( B\) ont mesure infinie et \( m(B)\) en est évidemment l'infimum. Nous supposons donc que \( m(B)<\infty\).

		Nous notons \( \tau_k\) la topologie induite de \( X\) sur \( \Int(X_k)\). Nous posons \( B_k=B\cap\Int(X_k)\). L'espace \( \big( \Int(X_k),m \big)\) est un espace mesuré borné et \( B_k\in \Borelien\Big( \Int(X_k) \Big)\). Il existe donc un ouvert \( \Omega_k\) de \( \big( \Int(X_k),\tau_k \big)\) tel que \( B_k\subset \Omega_k\) et
		\begin{equation}
			m(\Omega_k\setminus B_k)\leq \frac{ \epsilon }{ 2^k }.
		\end{equation}
		De plus \( \Int(X_k)\) est un ouvert de \( (X,\tau_X)\), donc en réalité \( \Omega_k\) est un ouvert de \( X\). Nous posons
		\begin{equation}
			\Omega=\bigcup_{k=1}^{\infty}\Omega_k
		\end{equation}
		qui est encore un ouvert de \( (X,\tau_X)\).

		Il est temps de voir que \( \Omega\) vérifie \( m(\Omega\setminus B)\leq \epsilon\). Pour cela,
		\begin{subequations}
			\begin{align}
				\Omega\setminus B & =\big( \bigcup_k\Omega_k \big)\cap\big( \bigcup_lB_l \big)^c \\
				                  & =\big( \bigcup_k\Omega_k \big)\cap\big( \bigcap B_l^c \big)  \\
				                  & =\subset\bigcup_k(\Omega_k\cap B_k^c)                        \\
				                  & =\bigcup_k(\Omega_k\setminus B_k),
			\end{align}
		\end{subequations}
		ce qui donne au niveau des mesures :
		\begin{equation}
			m(\Omega\setminus B)\leq\sum_{k=1}^{\infty}m(\Omega_k\setminus B_k)\leq\sum_{k=1}^{\infty}\frac{ \epsilon }{ 2^k }=\epsilon.
		\end{equation}
	\end{subproof}
\end{proof}

\begin{remark}      \label{RemooOAGCooRHpjxd}
	Exprimé sur \( \eR^N\), la proposition~\ref{PropNCASooBnbFrc} s'exprime en disant que toute mesure de Borel sur \( \eR^N\) est régulière. Typiquement, l'espace \( X\) dont il est question est un ouvert de \( \eR^N\).
\end{remark}

%---------------------------------------------------------------------------------------------------------------------------
\subsection{Théorème de récurrence}
%---------------------------------------------------------------------------------------------------------------------------

Soient \( X\) un espace mesurable, \( \mu\) une mesure finie sur \( X\) et \( \phi\colon X\to X\) une application mesurable\footnote{Définition \ref{DefQKjDSeC}.} préservant la mesure, c'est-à-dire que pour tout ensemble mesurable \( A\subset X\),
\begin{equation}
	\mu\big( \phi^{-1}(A) \big)=\mu(A).
\end{equation}
Si \( A\subset X\) est un ensemble mesurable, un point \( x\in A\) est dit \defe{récurrent}{récurrent!point d'un système dynamique} par rapport à \( A\) si et seulement si pour tout \( p\in \eN\), il existe \( k\geq p\) tel que \( \phi^k(x)\in A\).

\begin{theorem}[\wikipedia{fr}{Théorème_de_récurrence_de_Poincaré}{Théorème de récurrence de Poincaré}.]     \label{ThoYnLNEL}
	Si \( A\) est mesurable dans \( X\), alors presque tous les points de \( A\) sont récurrents par rapport à \( A\).
\end{theorem}

\begin{proof}
	Soit \( p\in \eN\) et l'ensemble
	\begin{equation}
		U_p=\bigcup_{k=p}^{\infty}\phi^{-k}(A)
	\end{equation}
	des points qui repasseront encore dans \( A\) après \( p\) itérations  de \( \phi\). C'est un ensemble mesurable en tant que union d'ensembles mesurables (pour rappel, les tribus sont stables par union dénombrable, comme demandé à la définition~\ref{DefjRsGSy}), et nous avons donc
	\begin{equation}
		\mu(U_p)\leq \mu(X)<\infty.
	\end{equation}
	De plus \( U_p=\phi^{-p}(U_0)\), donc \( \mu(U_p)=\mu(U_0)\). Vu que \( U_p\subset U_0\), nous avons
	\begin{equation}
		\mu(U_0\setminus U_p)=0.
	\end{equation}
	Étant donné que \( A\subset U_0\) nous avons a fortiori que
	\begin{equation}
		\{ x\in A\tq x\notin U_p \}\subset U_0\setminus U_p,
	\end{equation}
	et donc
	\begin{equation}
		\mu\{ x\in A\tq x\notin U_p \}=0.
	\end{equation}
	Cela signifie exactement que l'ensemble des points \( x\) de \( A\) tels que aucun des \( \phi^k(x)\) avec \( k\geq p\) n'est dans \( A\) est de mesure nulle.
\end{proof}

%+++++++++++++++++++++++++++++++++++++++++++++++++++++++++++++++++++++++++++++++++++++++++++++++++++++++++++++++++++++++++++
\section{Mesurabilité des fonctions à valeurs réelles}
%+++++++++++++++++++++++++++++++++++++++++++++++++++++++++++++++++++++++++++++++++++++++++++++++++++++++++++++++++++++++++++

Nous allons parler de la mesurabilité de fonctions
\begin{equation}
	f\colon (S,\tribF)\to \big( \bar \eR,\Borelien(\bar \eR) \big)
\end{equation}
où \( \bar \eR=\eR\cup\{ \pm\infty \}\).

\begin{normaltext}      \label{normooGAAJooUPCbzG}
	Nous convenons que \( 0\times\pm\infty=0\) parce que nous voulons qu'une droite (qui est un rectangle dont une mesure est \( 0\) et l'autre \( \infty\)) soit de mesure nulle dans \( \eR^2\).

	Les produits et sommes \( \pm\infty\pm\pm\infty\) et \( \pm\infty\times \pm\infty\) sont ceux que l'on croit. Sauf bien entendu \( +\infty-\infty\) et \( 1/0\) qui ne sont toujours pas définis.
\end{normaltext}

\begin{lemma}       \label{LEMooBLOLooAdNViv}
	L'ensemble \( B\) est un borélien de \( \bar \eR\) si et seulement si il existe un borélien \( B_0\) de \( \eR\) tel que \( B\) soit \( B_0\) ou \( B_0\cup\{ +\infty \}\) ou \( B_0\cup\{ -\infty \}\) ou \( B_0\cup\{ +\infty,-\infty \}\).
\end{lemma}

\begin{proof}
	Comme la topologie usuelle sur \( \eR\) est la topologie induite de celle sur \( \bar \eR\), la tribu induite l'est aussi par le théorème~\ref{ThoJDOKooKaaiJh}. Donc si \( B\) est un borélien de \( \bar \eR\), l'ensemble \( B\cap \eR\) est un borélien de \( \eR\).
\end{proof}

\begin{lemma}[\cite{TribuLi}]       \label{LemooCRVJooQosHPq}
	Si \( \mS_0\) est l'ensemble des intervalles du type
	\begin{equation}
		\begin{aligned}[]
			\mathopen] \alpha , \beta \mathclose[, &  & \mathopen[ -\infty , \beta \mathclose[, &  & \mathopen] \alpha , +\infty \mathclose]
		\end{aligned}
	\end{equation}
	avec \( -\infty<\alpha<\beta<+\infty\) alors \( \sigma(\mS_0)=\Borelien(\bar\eR)\).
\end{lemma}

\begin{proof}
	Les intervalles \( \mathopen] \alpha , \beta \mathclose[\) engendrent la topologie de \( \eR\)\footnote{Parce toutes les boules sont des intervalles de ce type et que les boules forment une base de topologie, proposition~\ref{PropNBSooraAFr}.}, donc \( \Borelien(\eR)\subset\sigma(\mS_0)\). De plus le lemme~\ref{LemBWNlKfA} nous autorise à dire que
	\begin{equation}
		\bigcap_{n\geq 1}\mathopen[ n , +\infty \mathclose]=\{ +\infty \}\in\sigma(\mS_0).
	\end{equation}
	Par conséquent tous les ensembles énumérés dans le lemme~\ref{LEMooBLOLooAdNViv} font partie de \( \sigma(\mS_0)\). Cela implique que \( \Borelien(\bar\eR)\subset\sigma(\mS_0)\).

	Pour l'inclusion inverse, \( \sigma(\mS_0)\) est engendré par des parties qui font partie de \( \Borelien(\bar \eR)\), donc \( \sigma(\mS_0)\subset\Borelien(\bar \eR)\).
\end{proof}

%---------------------------------------------------------------------------------------------------------------------------
\subsection{Fonctions à valeurs réelles sur un espace mesurable}
%---------------------------------------------------------------------------------------------------------------------------

\begin{theorem}     \label{THOooWHFLooKYGsOm}
	Soient un espace mesurable \( (S,\tribF)\) et une fonction \( f\colon S\to \bar \eR\). Les propriétés suivantes sont équivalentes.
	\begin{enumerate}
		\item\label{ITEMooHAMHooYLqUhVi}
		La fonction \( f\) est mesurable.
		\item\label{ITEMooHAMHooYLqUhVii}
		L'ensemble \( \{ f<a \}\) est dans \( \tribF\) pour tout \( a\in \eR\)
		\item\label{ITEMooHAMHooYLqUhViii}
		L'ensemble \( \{ f\leq a \}\) est dans \( \tribF\) pour tout \( a\in \eR\)
	\end{enumerate}
\end{theorem}

\begin{proof}
	Plusieurs implications à prouver.
	\begin{subproof}
		\spitem[\ref{ITEMooHAMHooYLqUhVi}\( \Rightarrow\)\ref{ITEMooHAMHooYLqUhVii}]
		Puisque \( f\) est mesurable et que \( \mathopen[ -\infty , a \mathclose[\in\Borelien(\bar\eR)\), nous avons \( f^{-1}\big( \mathopen[ -\infty , a \mathclose[ \big)\in\tribF\).
		\spitem[\ref{ITEMooHAMHooYLqUhVii}\( \Rightarrow\)\ref{ITEMooHAMHooYLqUhVi}]
		Nous posons \( \tribA=\{ \mathopen[ -\infty , a \mathclose[\tq a\in \eR \}\).

				Nous avons \( \tribA\subset\mS_0\) (le \( \mS_0\) du lemme~\ref{LemooCRVJooQosHPq}). Et de plus,
				\begin{equation}
					\mathopen] \alpha , \beta \mathclose[=\mathopen[ -\infty , \beta \mathclose[\setminus\mathopen[ -\infty , \alpha \mathclose]=\mathopen[ -\infty , \beta \mathclose[\setminus\bigcap_{n\geq 1}\mathopen[ -\infty , \alpha+\frac{1}{ n } \mathclose[.
				\end{equation}
				Donc \( \mathopen] \alpha , \beta \mathclose[\in\sigma(\tribA)\).

				Et aussi :
				\begin{equation}
					\mathopen] \alpha , +\infty \mathclose]=\bar\eR\setminus\bigcap_{n\in \eN}\mathopen[ -\infty , \alpha+\frac{1}{ n } \mathclose[,
				\end{equation}
				ce qui donne \( \mathopen] \alpha , +\infty \mathclose]\in \sigma(\tribA)\).

		Au final, \( \mS_0\subset\sigma(\tribA)\) et donc \( \sigma(\mS_0)\subset\sigma(\tribA)\). Le lemme~\ref{LemooCRVJooQosHPq} nous dit que \( \sigma(\mS_0)=\Borelien(\bar \eR)\). Nous avons donc bien \( \sigma(\mS_0)=\sigma(\tribA)=\Borelien(\bar\eR)\).

		Par ailleurs, nous savons que \( f^{-1}(\tribA)\subset\tribF\) parce que les éléments de \( \tribA\) sont de la forme \( \{ f<a \}\). Cela donne \( \sigma\big( f^{-1}(\tribA) \big)=\tribF\). Mais \( \sigma\big( f^{-1}(\tribA) \big)\) peut aussi s'exprimer par le lemme de transport \ref{LemOQTBooWGYuDU} : \( \sigma\big( f^{-1}(\tribA) \big)=f^{-1}\big( \sigma(\tribA) \big)\). En combinant les deux,
		\begin{equation}
			f^{-1}\big( \sigma(\tribA) \big)=\tribF,
		\end{equation}
		et en remplaçant \( \sigma(\tribA)\) par \( \Borelien(\bar \eR)\) nous avons ce que nous voulions :
		\begin{equation}
			f^{-1}\big( \Borelien(\bar\eR) \big)\in\tribF,
		\end{equation}
		ce qui signifie que \( f\) est mesurable.
		\spitem[\ref{ITEMooHAMHooYLqUhViii}\( \Rightarrow\)\ref{ITEMooHAMHooYLqUhVii}]
		Nous avons
		\begin{equation}
			\{ f<a \}=\bigcup_{n\geq 1}\{ f\leq a-\frac{1}{ n } \}.
		\end{equation}
		donc ceci est une union dénombrable d'éléments de \( \tribF\). Et \( \{ f<a \}\) est dans \( \tribF\).
		\spitem[\ref{ITEMooHAMHooYLqUhVi}\( \Rightarrow\)\ref{ITEMooHAMHooYLqUhViii}]
		Nous avons
		\begin{equation}
			\{ f\leq a \}=\{ f<a \}\cup f^{-1}\big( \mathopen[ -\infty , a \mathclose] \big).
		\end{equation}
		Le premier ensemble est dans \( \tribF\) par~\ref{ITEMooHAMHooYLqUhVii}. Ensuite \( \mathopen[ -\infty , a \mathclose]\) est un fermé de \( \bar \eR\) et donc un borélien de \( \bar \eR\). Son image réciproque est donc un élément de \( \tribF\) parce que \( f\) est mesurable. Au final nous avons bien \( \{ f\leq a \}\in\tribF\).
	\end{subproof}
\end{proof}

\begin{lemma}[\cite{NBoIEXO}]   \label{LemFOlheqw}
	Une fonction \( f\colon X\to \eR\) est mesurable si et seulement si \( f^{-1}(I)\) est mesurable pour tout \( I\) de la forme \( \mathopen] a , \infty \mathclose[\).
\end{lemma}

\begin{proof}
	Nous devons prouver que \( f^{-1}(A)\) est mesurable dans \( X\) pour tout borélien \( A\) de \( \eR\). Nous posons
	\begin{equation}
		S=\{ A\subset \eR\tq f^{-1}(A)\text{ est mesurable dans } X \}
	\end{equation}
	et nous prouvons que c'est une tribu. D'abord \( f^{-1}(\eR)=X\), et \( X\) est mesurable, donc \( \eR\in S\). Ensuite si \( A\in S\) alors \( f^{-1}(A^c)=f^{-1}(A)^c\). En tant que complémentaire d'un mesurable de \( X\), l'ensemble \( f^{-1}(A)^c\) est mesurable dans \( X\). Et enfin si \( A_n\in S \) alors \( f^{-1}(\bigcup_nA_n)=\bigcup_nf^{-1}(A_n)\) qui est encore mesurable dans \( X\) en tant qu'union de mesurables.

	Donc \( S\) est une tribu qui contient tous les ensembles de la forme \( \mathopen] a , \infty \mathclose]\). Le lemme~\ref{LemZXnAbtl} conclut que \( S\) contient tous les boréliens de \( \eR\).
\end{proof}

\begin{lemma}       \label{LEMooMYUFooKqdDNc}
	Soit \( \bar \eR=\eR\cup\{ \pm\infty \}\). Soit \( \lambda\in\bar \eR\).  Les parties \( \{ x\geq\lambda \}\), \( \{ x>\lambda \}\), \( \{ x\leq \lambda \}\) et \( \{ x<\lambda \}\) sont des boréliens de \( \bar\eR\).
\end{lemma}

\begin{lemma}       \label{LEMooAITEooMjHxvh}
	Soit une application mesurable \( f\colon (\Omega,\tribA)\to \big( \eR\cup\{ \pm\infty \},\Borelien \big)\). Soit \( \lambda\in \eR\); nous définissons
	\begin{equation}
		\begin{aligned}
			f_{\lambda}\colon \Omega & \to \eR                                    \\
			\omega                   & \mapsto \min\big( f(\omega),\lambda \big).
		\end{aligned}
	\end{equation}
	Alors \( f_{\lambda}\) est mesurable
\end{lemma}

\begin{proof}
	Soit \( A\) mesurable (i.e. borélien) dans \( \eR\cup\{ \pm\infty \}\). Nous devons montrer que \( f_{\lambda}^{-1}(A)\) est mesurable. Pour cela nous écrivons
	\begin{equation}
		A=\Big( A\cap\{ x>\lambda \}\Big)\cup\Big(  A\cap\{ x\leq \lambda \}\Big).
	\end{equation}
	Par définition \( f_{\lambda}\) ne prend jamais de valeurs plus grandes que \( \lambda\), donc \( f_{\lambda}^{-1}\big( A\cap\{ x>\lambda \} \big)=\emptyset\). D'autre part, \( f_{\lambda}^{-1}\big( A\cap\{ x\leq \lambda \} \big)=f^{-1}\big( A\cap\{ x\leq \lambda \} \big)\).

	Étant donné que \( A\) et \( \{ x\leq \lambda \}\) sont boréliens\footnote{Lemme \ref{LEMooMYUFooKqdDNc}.}, l'intersection \( A\cap\{ x\leq \lambda \}\) est borélienne, et donc
	\begin{equation}
		f_{\lambda}^{-1}(A)=f^{-1}\big( A\cap\{ x\leq\lambda \} \big)\in\tribA.
	\end{equation}
\end{proof}

\begin{lemma}[\cite{NBoIEXO}]   \label{LemIGKvbNR}
	Soit \( f_n\colon X\to \eR\) une suite de fonctions mesurables\footnote{Ici \( X\) est un espace mesuré et \( \eR\) est muni des boréliens.}. Alors \( \sup_n f_n\) est mesurable.
\end{lemma}

\begin{proof}
	Nous avons
	\begin{subequations}
		\begin{align}
			(\sup f_n)^{-1}\big( \mathopen] a , \infty \mathclose] \big) & =\{ x\in X \tq (\sup f_n)(x)>a \}                                \\
			                                                             & =\bigcup_n\{ x\in X\tq f_n(x)>a \}                               \\
			                                                             & =\bigcup_nf_n^{-1}\big( \mathopen] a , \infty \mathclose] \big).
		\end{align}
	\end{subequations}
	Étant donné que \( f_n\) est mesurable et que \( \mathopen] a , \infty \mathclose]\) est mesurable, chacun des \( f_n^{-1}\big( \mathopen] a , \infty \mathclose] \big) \) est mesurable dans \( X\). L'ensemble \( (\sup f_n)^{-1}\big( \mathopen] a , \infty \mathclose] \big)\) est donc une union dénombrable de parties mesurables. Il est donc mesurable.

	Le lemme~\ref{LemFOlheqw} conclut que \( \sup f_n\) est mesurable.
\end{proof}

\begin{proposition}\label{PropFYPEOIJ}
	Si \( f_n\colon X\to \eR\) est une suite de fonctions mesurables et positives, alors la fonction\footnote{Définition \ref{DEFooYEIUooCAgrxI} pour la série de fonctions.} \( \sum_nf_n\) est mesurable.
\end{proposition}

\begin{proof}
	Nous considérons les fonctions \( s_k(x)=\sum_{n=0}^kf_n(x)\) qui valent éventuellement \( \infty\) en certains points. Nous avons
	\begin{equation}
		\sum_nf_n(x)=\sup_ks_k(x),
	\end{equation}
	donc le lemme~\ref{LemIGKvbNR} nous donne la mesurabilité de la somme de \( f_n\).
\end{proof}

\begin{definition}      \label{ooUDHFooJjKscR}
	Soit \( (S,\tribF)\) un espace mesurable.
	Une \defe{partition mesurable dénombrable}{partition!dénombrable mesurable} de \( S\) est une suite  \( (S_n)_{n\geq 1}\) de parties de \( S\) telles que
	\begin{enumerate}
		\item
		      \( S_n\in\tribF\) pour tout \( n\),
		\item
		      \( S_n\cap S_k=\emptyset\) si \( n\neq k\),
		\item
		      \( S=\bigcup_{n\geq 1}S_n\).
	\end{enumerate}
\end{definition}

\begin{lemma}[Lemme de recollement]     \label{LEMooXAPQooPpZUmP}
	Soit \( (S_n)\) une partition mesurable dénombrable de l'espace mesurable \( (S,\tribF)\). Soit \( (S',\tribF')\) un autre espace mesurable et des fonctions mesurables
	\begin{equation}
		f_n\colon (S_n,\tribF_{S_n})\to (S',\tribF')
	\end{equation}
	où \( \tribF_{S_n}\) est la tribu induite\footnote{Définition~\ref{DefDHTTooWNoKDP}.}. Alors la fonction
	\begin{equation}
		\begin{aligned}
			f\colon (S,\tribF) & \to (S',\tribF')                    \\
			x                  & \mapsto f_n(x) \text{ si } x\in S_n
		\end{aligned}
	\end{equation}
	est mesurable.
\end{lemma}

\begin{proof}
	Soit \( A'\in\tribF'\); nous devons prouver que \( f^{-1}(A')\in \tribF\). Nous savons que
	\begin{equation}        \label{EqooGKFFooEwTdtg}
		f^{-1}(A')=\bigcup_{n\geq 1}f_n^{-1}(A'),
	\end{equation}
	qui est une union dénombrable d'éléments \( f_n^{-1}(A')\in\tribF_{S_n}\).

	Puisque \( S_n\in \tribF\) nous avons \( \tribF_{S_n}\subset\tribF\) parce qu'un élément de \( \tribF_{S_n}\) est de la forme \( S_n\cap B\) avec \( B\in\tribF\). Ainsi, pour chaque \( n\) nous avons
	\begin{equation}
		f_n^{-1}(A')\in\tribF_{S_n}\subset \tribF.
	\end{equation}
	Au final l'égalité \eqref{EqooGKFFooEwTdtg} écrit \( f^{-1}(A')\) comme une union d'éléments de \( \tribF\) et est donc un élément de \( \tribF\).
\end{proof}

\begin{proposition}     \label{PROPooODDVooEEmmTX}
	Soit \( (S,\tribF)\) un espace mesurable et des applications mesurables \( f,g\colon S\to \bar \eR\). Alors les fonctions suivantes sont mesurables :
	\begin{enumerate}
		\item
		      \( \lambda f\) pour tout \( \lambda\in \eR\)
		\item
		      \( f+g\) si elle existe.
		\item
		      \( 1/f\) si elle existe.
		\item
		      \( fg\).
	\end{enumerate}
\end{proposition}

\begin{proof}
	Commençons par clarifier «si elle existe». La fonction \( f+g\) n'existe pas au point \( x\in S\) si \( f(x)=+\infty\) et \( g(x)=-\infty\). La fonction \( 1/f\) n'existe pas au point \( x\in S\) si \( f(x)=0\). Voir le point~\ref{normooGAAJooUPCbzG}.
	\begin{subproof}
		\spitem[La partie où \( f+g\) existe est mesurable]
		La partie de \( S\) sur laquelle \( f+g\) existe est
		\begin{equation}
			\{ x\in S\tq \big( f(x),g(x) \big)\neq (+\infty,-\infty),\big( f(x),g(x) \big)\neq (-\infty,+\infty) \}.
		\end{equation}
		Nous avons
		\begin{equation}
			\{  (f,g)=(+\infty,-\infty) \}=\{ f=\infty \}\cap\{ g=-\infty \}
		\end{equation}
		qui est un ensemble mesurable parce que, par exemple,
		\begin{equation}
			\{ +\infty \}=\bigcap_{n\geq 1}\mathopen[ n , +\infty \mathclose].
		\end{equation}
		Le cas \( (-\infty,+\infty)\) est identique, et au final la partie de \( S\) sur laquelle \( f+g\) n'existe pas est mesurable. Par complémentarité la partie sur laquelle \( f+g\) existe est également mesurable\footnote{Parfois on a envie de dire que l'affirmation «\( A\) est mesurable» ne passe pas le test de Popper.}.
		\spitem[Idem pour la partie sur laquelle \( 1/f\) existe]
		Idem.
		\spitem[Mesurabilité de \( \lambda f\)]
		Si \( \lambda=0\), nous avons une fonction constante dont la mesurabilité est évidente\footnote{Prenez quand même le temps d'y penser.}. Nous supposons \( \lambda>0\). Alors
		\begin{equation}
			\{ \lambda f<a \}=\{ f<a/\lambda \}\in \tribF.
		\end{equation}
		Pour \( \lambda<0\) nous avons de la même manière
		\begin{equation}
			\{ \lambda f<a \}=\{ f>a/\lambda \}\in \tribF.
		\end{equation}
		Ce dernier point est suffisant pour que \( \lambda f\) soit mesurable par le théorème~\ref{THOooWHFLooKYGsOm}\ref{ITEMooHAMHooYLqUhViii} et par complémentarité.
		\spitem[Mesurabilité de \( f+g\)]
		Soit \( a\in \eR\); le théorème~\ref{THOooWHFLooKYGsOm} nous demande d'avoir envie de prouver que \(  \{ f+g <a\} \in \tribF \). Nous avons
		\begin{equation}
			f(x)+g(x)<a
		\end{equation}
		si et seulement si
		\begin{equation}
			f(x)<a-g(x)
		\end{equation}
		si et seulement si
		\begin{equation}
			\exists q\in \eQ\tq f(x)<q<a-g(x).
		\end{equation}
		Donc
		\begin{equation}
			\{ f+g<a \}=\bigcup_{q\in \eQ}\Big( \{ f<q \}\cap\{ g<a-q \} \Big),
		\end{equation}
		qui est une union dénombrable d'éléments de \( \tribF\). Donc \( \{ f+g<a \}\in \tribF\) et \( f+g\) est mesurable.

		Note qu'en toute rigueur il faudrait  «\( \cap\text{là où } f+g\text{ est définie}\)» un peu partout, mais cela ne change rien parce que l'intersection de deux parties mesurables est mesurable.
		\spitem[Mesurabilité de \( 1/f\)]
		Soit \( a\in \eR\). Si \( a>0\) alors
		\begin{equation}
			\{ 1/f<a \}=\{ f<0 \}\cup\{ f>\frac{1}{ a } \}\in\tribF.
		\end{equation}
		et si \( a<0\) alors
		\begin{equation}
			\{ 1/f<a \}=\{ f<0 \}\cap\{ f>\frac{1}{ a } \}\in\tribF.
		\end{equation}
		\spitem[Mesurabilité de \( fg\)]
		Nous allons la prouver en plusieurs fois.
		\begin{subproof}
			\spitem[Si \( f\) est mesurable alors \( f^2\) est mesurable]
			Si \( a\leq 0\) alors \( \{ f^2<a \}=\emptyset\). Si \( a>0\) nous avons
			\begin{equation}
				\{ f^2<a \}=\{ -\sqrt{a}<f<\sqrt{a} \}\in\tribF.
			\end{equation}

			\spitem[\( f\mtu_A\) est mesurable]
			Soit \( A\in \tribF\), et prouvons que \( f\mtu_A\) est mesurable. Par définition,
			\begin{equation}
				(f\mtu_A)(x)=\begin{cases}
					f(x) & \text{si } x\in A     \\
					0    & \text{si } x\notin A.
				\end{cases}
			\end{equation}
			Nous posons \begin{equation}
				\begin{aligned}
					f_1\colon A^c & \to \bar \eR \\
					x             & \mapsto 0
				\end{aligned}
			\end{equation}
			et
			\begin{equation}
				\begin{aligned}
					f_2\colon A & \to \bar \eR  \\
					x           & \mapsto f(x).
				\end{aligned}
			\end{equation}
			Alors nous avons
			\begin{equation}
				(\mtu_Af)(x)=\begin{cases}
					f_1(x) & \text{si } x\in A^c \\
					f_2(x) & \text{si } x\in  A.
				\end{cases}
			\end{equation}
			Les ensembles \( A\) et \( A^c\) forment une partition mesurable dénombrable de \( S\). La fonction \( f_1\) est mesurable; pour prouver que \( f_2\) est mesurable, nous l'écrivons \( f_2=f\circ j_A\) où \( j_A\colon A\to S\) est l'injection canonique. L'application
			\begin{equation}
				j_A\colon (A,\tribF_A)\to (S,\tribF)
			\end{equation}
			est mesurable parce que si \( B\in\tribF\) alors \( j_A^{-1}(B)=A\cap B\in\tribF_A\). D'autre part l'application
			\begin{equation}
				f\colon (S,\tribF)\to \big( \bar \eR,\Borelien(\bar \eR) \big)
			\end{equation}
			est mesurable par hypothèse. La composée \( f_2=f\circ j_A\) est alors mesurable par la proposition~\ref{PROPooEFHKooARJBwW}. Le lemme de recollement~\ref{LEMooXAPQooPpZUmP} nous donne alors la mesurabilité de \( f\mtu_A\).

			\spitem[Le produit \( fg\) est mesurable]
			Nous posons
			\begin{equation}
				F=\{ x\in S\tq | f(x) |<+\infty,| g(x) |<\infty \}.
			\end{equation}
			En tant qu'intersection de deux ensembles mesurables, \( F\) est mesurable. Par la partie précédente, les applications \( f_1=f\mtu_F\) et \( g_1=g\mtu_F\) sont mesurables. L'application \( f_1+g_1\colon S\to \eR\) est encore mesurable. Par conséquent l'application
			\begin{equation}
				f_1g_1=\frac{ 1 }{2}\big( (f_1+g_1)^2-f_1^2-g_1^2 \big)
			\end{equation}
			est mesurable.

			Voyons maintenant ce qui se passe en dehors de \( F\). Nous allons utiliser le lemme de recollement sur la fonction
			\begin{equation}
				(fg)(x)=\begin{cases}
					(f_1g_1)(x) & \text{si } x\in F   \\
					-\infty     & \text{si } x\in\mU  \\
					0           & \text{si } x\in \mV \\
					+\infty     & \text{si } x\in\mW
				\end{cases}
			\end{equation}
			où \( F,\mU,\mV,\mW\) forment une partition mesurable dénombrable\footnote{Définition~\ref{ooUDHFooJjKscR}.} de \( S\). Pour le sport nous montrons que \( \mU\) est mesurable :
			\begin{subequations}
				\begin{align}
					\mU & =\big( \{ f=-\infty \}\cap\{ g>0 \} \big)    \\
					    & \cup\big( \{ f=+\infty \}\cap\{ g<0 \} \big) \\
					    & \cup\big( \{ g=-\infty \}\cap\{ f>0 \} \big) \\
					    & \cup\big( \{ g=+\infty \}\cap\{ f<0 \}\big).
				\end{align}
			\end{subequations}
		\end{subproof}
	\end{subproof}
\end{proof}

\begin{proposition}     \label{ooABKWooPbfSOZ}
	Si \( f_n\colon S\to \bar \eR\) est une suite de fonctions mesurables, alors les fonctions \( \inf_n f_n\) et \( \sup_nf_n\) sont mesurables.
\end{proposition}

\begin{proof}
	Nous avons les découpages
	\begin{equation}
		\{ \inf_nf_n<a \}=\bigcup_n\{ f_n<a \}\in\tribF
	\end{equation}
	et
	\begin{equation}        \label{EQooNYKVooDOjOXM}
		\{ \sup_nf_n\leq a \}=\bigcap_n\{ f_n\leq a \}\in\tribF.
	\end{equation}
	Le théorème~\ref{THOooWHFLooKYGsOm} permet de conclure.
\end{proof}
Note : pour \eqref{EQooNYKVooDOjOXM} nous ne pouvions pas utiliser les inégalités strictes parce que \( \{ \sup_nf_n<a \}\) n'est pas spécialement égal à \( \bigcap_n\{ f_n<a \}\).

\begin{normaltext}
	La proposition~\ref{ooABKWooPbfSOZ} nous permet de définir les parties positives et négatives de \( f\) par \( f^+=\sup(f,0)\) et \( f^-=\sup(-f,0)\). Ce sont des applications mesurables. Nous avons les décompositions
	\begin{subequations}
		\begin{align}
			f     & = f^+-f^-  \\
			| f | & = f^++f^-.
		\end{align}
	\end{subequations}
\end{normaltext}

\begin{corollary}       \label{CORooNXYUooEcvDlP}
	Si \( f\colon S\to \bar \eR\) est mesurable alors les applications \( f^+\), \( f^-\) et \( | f |\) sont mesurables en tant qu'applications \( S\to\bar \eR^+\).
\end{corollary}

\begin{proof}
	Nous faisons la preuve pour \( f^+\). Nous savons que \( f^+\colon S\to \bar \eR\) est mesurable par la proposition~\ref{ooABKWooPbfSOZ}. Nous considérons l'injection canonique \( j\colon \bar \eR^+\to \bar \eR\) et
	\begin{equation}
		\begin{aligned}
			f_1^+\colon S & \to \bar \eR^+  \\
			x             & \mapsto f^+(x).
		\end{aligned}
	\end{equation}
	Alors \( f_1^+=j\circ f^+\) est mesurable. Et c'est bien cela que nous voulions.

\end{proof}
Note : \( f^+\) et \( f_1^+\) sont exactement les mêmes fonctions. Elles ne diffèrent que par la tribu que nous considérons sur l'espace d'arrivée. Nous allons à partir de maintenant les noter toutes deux \( f^+\).

\begin{remark}
	L'application \( | f |\) peut être mesurable sans que \( f\) le soit. Soit en effet une partie \( A\notin \tribF\), et posons
	\begin{equation}
		f(x)=\begin{cases}
			1  & \text{si } x\in A    \\
			-1 & \text{si } x\in A^c.
		\end{cases}
	\end{equation}
	Alors \( f^{-1}(\{ 1 \})=A\) n'est pas mesurable alors que \( | f |(x)=1\) pour tout \( x\).
\end{remark}

Il est temps d'aller relire les définitions~\ref{ooMVZAooVVCOnP}.

\begin{proposition}     \label{PropooMFIBooJzaleK}
	Si les fonctions \( f_n\colon S\to \bar \eR\) sont mesurables alors les fonctions \( \limsup f_n\) et \( \liminf f_n\) sont mesurables.
\end{proposition}

\begin{proof}
	Par le lemme~\ref{ooAQTEooYDBovS} nous écrivons \( \limsup_nf_n(x)=\inf_{n\geq 1}\sup_{k\geq n} f_k(x)\). Pour chaque \( k\) nous considérons la fonction \( g_k=\sup_{n\geq k}f_n\). Par la proposition~\ref{ooABKWooPbfSOZ}, les fonctions \( g_k\) sont mesurables. En utilisant encore la même proposition, \( \inf_{n\geq 1}g_k\) est encore mesurable.
\end{proof}

\begin{proposition}[\cite{ooKKLCooZRxJnn}]      \label{PropooDXBGooSFqrai}
	Si \( f_n\colon S\to \bar \eR\) est une suite de fonctions mesurables dont la limite ponctuelle existe, alors la limite est mesurable.
\end{proposition}

\begin{proof}
	Si la limite existe, elle est égale à la limite supérieure par le lemme~\ref{ooIQIKooXWwAmM}. Or la limite supérieure est mesurable par la proposition~\ref{PropooMFIBooJzaleK}.
\end{proof}

%---------------------------------------------------------------------------------------------------------------------------
\subsection{Fonction étagée}
%---------------------------------------------------------------------------------------------------------------------------

\begin{definition}[\cite{ooARRSooBLWdam}]\label{DefBPCxdel}
	Soit \( (S,\tribF)\) un espace mesurable et une fonction \( f\colon S\to \big( \bar\eR,\Borelien(\bar\eR) \big)\). Il serait dommage de confondre les trois concepts suivants.
	\begin{itemize}
		\item
		      Une \defe{fonction simple}{simple!fonction} est une fonction dont l'image est constituée d'un nombre fini de valeurs.
		\item
		      Une \defe{fonction étagée}{étagée!fonction} est une fonction simple qui est elle-même une fonction mesurable.
		\item
		      Une \defe{fonction en escalier}{escalier} est une fonction étagée dont les valeurs sont constantes sur des intervalles : ce sont donc des fonctions constantes par morceaux.
	\end{itemize}
\end{definition}

Dans les trois cas, la fonction \( f\) peut être écrite comme somme de fonctions caractéristiques :
\begin{equation}
	f(x)=\sum_{j=1}^p\alpha_j\mtu_{A_j}(x)
\end{equation}
où \( A_j=f^{-1}(\alpha_j)\). Ce qui change est la nature des \( A_j\).

\begin{itemize}
	\item Si \( f\) est  simple, les \( A_j\) sont quelconques.
	\item Si \( f\) est étagée, les \( A_i\) peuvent être choisis mesurables parce que \( \{\alpha_i \}\) est un borélien, ce qui fait de \( A_i=f^{-1}(\alpha_i)\) un choix mesurable.
	\item Si \( f\) est en escalier, les \( A_i\) sont des intervalles.
\end{itemize}

\begin{definition}
	La \defe{forme canonique}{forme canonique!fonction simple} d'une fonction simple \( f\) est la suivante. Soit \( \{ \alpha_i \}_{i=1,\ldots, l}\) les valeurs distinctes prises par \( f\) et \( A_i=f^{-1}(\alpha_i)\). La forme canonique de \( f\) est alors
	\begin{equation}
		f=\sum_{i=1}^l\alpha_i\mtu_{A_i}.
	\end{equation}
\end{definition}

\begin{lemma}   \label{LEMooNWLTooCDuRQI}
	Si \( f\) est une fonction simple dont la représentation canonique est
	\begin{equation}
		f=\sum_{i=1}^l\alpha_i\mtu_{A_i},
	\end{equation}
	alors
	\begin{enumerate}
		\item
		      les \( A_i\) sont disjoints,
		\item
		      l'union est égale à tout l'ensemble : \( S=\bigcup_iA_i\).
	\end{enumerate}
\end{lemma}

\begin{probleme}
	Le lemme~\ref{LemYFoWqmS} et le théorème~\ref{THOooXHIVooKUddLi} disent la même chose alors que la preuve du théorème~\ref{THOooXHIVooKUddLi} est beaucoup plus compliquée.La démonstration du lemme serait fausse ?

	M'est avis que ce que le théorème donne en plus est la convergence uniforme en cas de fonction bornée. La suite \eqref{EqooXQYIooSSJwtM} ne va pas converger uniformément.
\end{probleme}

\begin{lemma}[Limite croissante de fonctions étagées\cite{MonCerveau}]    \label{LemYFoWqmS}
	Soit \( f\colon (S,\tribF)\to \bar\eR\) une fonction positive mesurable. Il existe une suite \( f_n\colon S\to \eR\) de fonctions étagées positives telles que \( f_n\to f\) ponctuellement et \( f_n \leq f\).
\end{lemma}

\begin{proof}
	Nous considérons \( (q_n)\) une suite parcourant tous les rationnels positifs\footnote{Nous rappelons que \( \eQ\) est dénombrable et dense dans \( \eR\) par la proposition~\ref{PropooUHNZooOUYIkn}.} avec \( q_0=0\) pour être sûr.
	Pour \( n\in \eN\) nous définissons la fonction
	\begin{equation}        \label{EqooXQYIooSSJwtM}
		f_n(x)= \max\{ q_i\tq i\leq n,\, q_i\leq f(x) \}.
	\end{equation}
	L'ensemble sur lequel le maximum est pris n'est pas vide parce que \( q_0=0\). La fonction \( f_n\) est simple parce qu'elle ne prend que \( n\) valeurs différentes. Nous avons aussi, par construction, \(  f_n(x)\leq f(x) \). Et aussi pour tout \( x\in S\), \( f_n(x)\to f(x)\), parce que \( \eQ\) est dense dans \( \eR\).

	En ce qui concerne le fait que \( f_n\) soit mesurable, nous notons \( \{ r_0,\ldots, r_{n} \}\) l'ensemble des \( \{ q_0,\ldots, q_n \}\) classés dans l'ordre croissant. Nous posons en plus \( r_{n+1}=+\infty\). Nous avons alors
	\begin{equation}
		f_n^{-1}(r_k)=\{ x\in S\tq f(x)\geq r_k,f(x)<r_{k+1} \}=\{ f\geq r_k \}\cap\{ f<r_{k+1} \}.
	\end{equation}
	En tant qu'intersection de deux ensembles mesurables, le théorème \ref{THOooWHFLooKYGsOm} dit que \( f_n^{-1}(r_k)\) est mesurable.
\end{proof}

\begin{remark}
	Pour avoir \(  f_n <| f |\) nous pouvons poser
	\begin{equation}
		f_n(x)=\begin{cases}
			\max\{ q_i\tq i\leq n,\, q_i\leq f(x) \} & \text{si } f(x)\geq 0      \\
			\min\{ q_i\tq i\leq n,\, q_i\geq f(x) \} & \text{si } f(x)< 0\text{.}
		\end{cases}
	\end{equation}
\end{remark}

\begin{theorem}[Théorème fondamental d'approximation, thème \ref{THEMEooKLVRooEqecQk}\cite{TribuLi,YHRSDGc,ooRCYWooNAeaTA}]\label{THOooXHIVooKUddLi}
	Soit un espace mesurable \( (S,\tribA)\).
	\begin{enumerate}
		\item
		      Soit une fonction mesurable \( f\colon S\to \mathopen[ 0 , +\infty \mathclose]\). Alors il existe une suite croissante de fonctions \( \varphi_n\colon S\to \mathopen[ 0 , +\infty \mathclose[\) étagées\footnote{Définition \ref{DefBPCxdel}.} positives dont la limite ponctuelle est \( f\).
		\item
		      Si de plus \( f\) est bornée, la convergence est uniforme.
		\item
		      Idem pour \( f\) à valeurs dans \( \bar \eR\) ou \( \eC\).
	\end{enumerate}
\end{theorem}

\begin{proof}
	Nous découpons l'intervalle \( \mathopen[ 0 , n \mathclose]\) en plusieurs morceaux.
	\begin{equation}
		I_{n,k}=\begin{cases}
			\mathopen[ \frac{ k }{ 2^n } , \frac{ k+1 }{ 2^n } \mathclose[ & \text{si } 0\leq k\leq n2^n-1 \\
			\mathopen[ n , \infty \mathclose]                              & \text{si } k=n2^n.
		\end{cases}
	\end{equation}
	Nous posons \( S_{n,k}=f^{-1}(I_{n,k})\). Ce sont des ensembles mesurables parce que \( f\) est mesurable. Et de plus, pour chaque \( n\), la suite \( (S_{n,k})_{k\geq 0}\) est une partition mesurable finie de \( S\). Nous posons
	\begin{equation}
		\varphi_n=\sum_{k=0}^{n2^n}\frac{ k }{ 2^n }\mtu_{S_{n,k}}.
	\end{equation}
	C'est-à-dire que sur chaque \( S_{n,k}\) nous approximons \( f\) par le bas. La fonction \( \varphi_n\) est étagée et positive : \( 0\leq \varphi_n(x)\leq f(x)\) par construction.
	\begin{subproof}
		\spitem[Croissance]
		Nous allons voir que \( \varphi_n\leq \varphi_{n+1}\). Soit \( k\neq n2^n\). Si \( x\in S_{n,k}\) alors \( \varphi_n(x)=\frac{ k }{ 2^n }\) et nous avons aussi la décomposition
		\begin{equation}
			S_{n,k}=S_{n+1,2k}\cup S_{n+1,2k+1}.
		\end{equation}
		Si \( x\in S_{n+1,2k}\) alors \( \varphi_{n+1}(x)=\frac{ 2k }{ 2^{n+1} }=\frac{ k }{ 2^n }=\varphi_n(x)\). Et si \( x\in S_{n+1,2k+1}\) alors
		\begin{equation}
			\varphi_{n+1}(x)=\frac{ 2k+1 }{ 2^{n+1} }=\frac{ k+\frac{ 1 }{2} }{ 2^n }>\varphi_n(x).
		\end{equation}

		Il reste à traiter le cas \( x\in\{ f\geq n \}\). Dans ce cas nous avons \( \varphi_n(x)=n\). Il y a encore deux cas à traiter :
		\begin{equation}
			\{ f\geq n \}=\{ f\in\mathopen[ n , n+1 \mathclose[ \}\cup\{ f\in\mathopen[ n+1 , \infty \mathclose] \}.
		\end{equation}
		Pour plus de simplicité dans les notations, nous notons \( \bar n=n2^n\), c'est-à-dire que \( I_{n,\bar n}\) est le \( I_{n,k}\) avec le \( k\) le plus grand possible. Nous avons
		\begin{equation}
			I_{n,\bar n}=\mathopen[ n , n+1 \mathclose[\cup\mathopen[ n+1 , \infty \mathclose].
		\end{equation}
		Le premier élément se décompose en \( I_{n+1,k}\) avec \( k<\bar n+1\) (nous préciserons plus tard exactement les valeurs de \( k\)) tandis que le second est \( \mathopen[ n+1 , \infty \mathclose]=I_{n+1,\overline{ n+1 }}\).

		Pour \( x\in S_{n+1,\overline{ n+1 }}\) nous avons
		\begin{equation}
			\varphi_{n+1}(x)=\frac{ (n+1)2^{n+1} }{ 2^{n+1} }=n+1>\varphi_n(x).
		\end{equation}
		Si au contraire \( f(x)\in\mathopen[ n , n+1 \mathclose[ \) nous devons précisément voir quels sont les \( k\) qui font en sorte que \( I_{n+1,k}\) recouvre \( \mathopen[ n , n+1 \mathclose[\). Le plus petit \( k\) est donné par \( \frac{ k }{ 2^{n+1} }=n\), c'est-à-dire \( k=n2^{n+1}\) et le plus grand \( k\) est donné par \( \frac{ k }{ 2^{n+1} }<n+1\), c'est-à-dire \( k=2^{n+1}(n+1)-1\). Donc si \( f(x)\in\mathopen[ n , n+1 \mathclose[\) alors \( x\in S_{n+1,k}\) avec
		\begin{equation}
			n2^{n+1}\leq k\leq (n+1)2^{n+1}-1
		\end{equation}
		Dans ce cas
		\begin{equation}
			\varphi_{n+1}(x)=\frac{ k }{ 2^{n+1} }\geq \frac{ n2^{n+1} }{ 2^{n+1} }=n=\varphi_n(x).
		\end{equation}
		\spitem[Convergence ponctuelle]
		Si \( f(x)<\infty\) alors il existe\footnote{Le vrai snob citera ici le lemme~\ref{LemooMWOUooVWgaEi}.} \( n_0\in\eN\) tel que \( f(x)<n_0\). Pour \( n\geq n_0\) nous avons \( f(x)<n\) et donc \( \varphi_n(x)\) se calcule à partir d'un des intervalles de taille \( 1/2^n\) :
		\begin{equation}
			\varphi_n(x)=\frac{ k }{ 2^n }\leq f(x)<\frac{ k+1 }{ 2^n }.
		\end{equation}
		Donc
		\begin{equation}
			| \varphi_n(x)-f(x) |\leq \frac{1}{ 2^n },
		\end{equation}
		ce qui signifie que \( \lim_{n\to \infty} \varphi_n(x)=f(x)\).

		Si \( f(x)=+\infty\) alors \( f(x)>n\) pour tout \( n\). Et alors \( \varphi_n(x)=n\) pour tout \( n\), ce qui donne bien \( \varphi_n(x)\to \infty\).
		\spitem[Convergence uniforme]
		Soit \( f\) bornée : \( 0\leq f(x)<M\) pour tout \( x\in S\). Soit aussi \( \epsilon>0\). Nous prenons \( n_0>M\) tel que \( \frac{1}{ 2^{n_0} }<\epsilon\). Alors pour tout \( n\geq n_0\) nous avons
		\begin{equation}
			0\leq f(x)-\varphi_n(x)\leq \frac{1}{ 2^n }\leq \frac{1}{ 2^{n_0} }\leq \epsilon.
		\end{equation}
		Notez qu'il n'y a pas de valeurs absolues parce que nous savons déjà que la limite est croissante.
	\end{subproof}
\end{proof}

\begin{theorem}[Doob\cite{ProbaDanielLi}]     \label{ThofrestemesurablesXYYX}
	Soit une application mesurable\footnote{Elle est notée \( X\) parce que l'application usuelle de ce théorème est en théorie des variables aléatoires.} \( X\colon \Omega\to \eR^d\). Une application \( Y\colon \Omega\to \eR^{p}\) est \( \tribA_X\)-mesurable si et seulement si il existe une fonction borélienne \( f\colon \eR^d\to \eR^{p}\) telle que \( Y=f(X)\).
\end{theorem}

\begin{proof}
	En séparant \( Y\) par coordonnées, et en séparant, pour chacune, les parties positives et négatives, nous supposons que \( Y\) est à valeurs réelles positives. Le théorème \ref{THOooXHIVooKUddLi} dit qu'il existe une suite croissante d'applications \( \tribA_X\)-mesurables et étagées \( \varphi_n\colon \Omega\to \mathopen[ 0 , \infty \mathclose[\) telles que \( \varphi_n\to Y\) ponctuellement.

	Vu que \( \varphi_n\) est étagée, il existe des \( \tribA_X\)-mesurables \( A_{nk}\subset \Omega\) tels que \( \varphi_n=\sum_{k=0}^{s_n}a_{ak}\mtu_{A_{nk}}\) avec \( a_{nk}>0\).  Par la définition \ref{DefNOJWooLGKhmJ} de la tribu engendrée par une application, il existe un borélien \( B_{nk}\subset \eR\) tel que \( A_{nk}=X^{-1}(B_{nk})\). Avec ça nous avons
	\begin{subequations}
		\begin{align}
			\varphi_n & =\sum_{k=0}^{s_n}a_{nk}\mtu_{A_{nk}}              \\
			          & =\sum_ka_{nk}\mtu_{X^{-1}(B_{nk})}                \\
			          & =\sum_ka_{nk}(\mtu_{B_{nk}}\circ X)               \\
			          & =\left( \sum_ka_{nk}\mtu_{B_{nk}} \right)\circ X.
		\end{align}
	\end{subequations}
	Nous posons
	\begin{equation}
		\begin{aligned}
			f_n\colon \eR & \to \eR^+                             \\
			f_n           & =\sum_{k=0}^{s_n}a_{nk}\mtu_{B_{nk}}.
		\end{aligned}
	\end{equation}

	Pour \( x\in \eR\), il n'y a que deux possibilités. Soit \( x\in X(\Omega)\), soit non. Si \( x\) n'est pas dans \( X(\Omega)\), alors il n'est dans aucun des \( B_{nk}\) et nous avons \( f_n(x)=0\) pour tout \( n\). Si \( x\in X(\Omega)\), alors il existe \( \omega\in \Omega\) tel que \( x=X(\omega)\). Dans ce cas
	\begin{equation}
		f_n(x)=f_n\big( X(\omega) \big)=\varphi_n(\omega),
	\end{equation}
	qui est croissante en \( n\). De plus si \( x= X(\omega)\), nous avons
	\begin{equation}
		f_n(x)=\varphi_n(\omega)\to Y(\omega).
	\end{equation}
	Nous avons donc montré que pour tout \( x\), \( n\mapsto f_n(x)\) est convergente :
	\begin{equation}
		\lim_{n\to \infty} f_n(x)=\begin{cases}
			0         & \text{si } x\notin X(\Omega) \\
			Y(\omega) & \text{si } x=X(\omega).
		\end{cases}
	\end{equation}
	Nous notons \( f=\lim_{n\to \infty} f_n\), et nous avons
	\begin{equation}
		Y(\omega)=\lim_{n\to \infty} \varphi_n(\omega)=\lim_{n\to \infty} \Big( (f_n\circ X)(\omega) \Big)=\lim_{n\to \infty} f_n\big( X(\omega) \big)=f\big( X(\omega) \big)=(f\circ X)(\omega).
	\end{equation}
	Nous avons donc bien trouvé une application \( f\colon \eR\to \eR\) telle que \( Y=f\circ X\).
\end{proof}

%---------------------------------------------------------------------------------------------------------------------------
\subsection{Fonctions réelles à variables réelles}
%---------------------------------------------------------------------------------------------------------------------------

Nous nous focalisons à présent sur le cas des fonctions
\begin{equation}
	f\colon \big( \eR,\Borelien(\eR) \big)\to \big( \bar\eR,\Borelien(\bar \eR) \big).
\end{equation}

\begin{normaltext}[\cite{MonCerveau}]      \label{NORMooNFOMooYnaflN}
	Anticipons un peu pour expliquer pourquoi ce que nous allons faire maintenant est suffisant pour ce que nous avons en tête\footnote{Pour rappel, nous avons en tête de définir une théorie de la mesure afin d'y définir des intégrales. En particulier nous allons étudier l'intégrale de Lebesgue et en ce qui concerne \( \eR^n\), nous aurons la tribu de Lebesgue.}. Toutes les fonctions mesurables
	\begin{equation}
		f\colon \big( \eR,\Borelien(\eR) \big)\to \big( \bar\eR,\Borelien(\bar \eR) \big)
	\end{equation}
	seront a fortiori mesurables au sens de
	\begin{equation}
		f\colon \big( \eR,\Lebesgue(\eR) \big)\to \big( \bar\eR,\Borelien(\bar \eR) \big)
	\end{equation}
	où \( \Lebesgue(\eR)\) est la tribu de Lebesgue sur \( \eR\), c'est-à-dire la tribu complétée de celle des boréliens (définition~\ref{DefooYZSQooSOcyYN}).
\end{normaltext}

\begin{normaltext}
	Nous allons maintenant donner quelques conditions pour que des fonctions soient mesurables au sens de la tribu des boréliens sur l'espace d'arrivée et de départ. Ces résultats seront donc immédiatement applicables à la théorie de l'intégration où nous considérons la tribu de Lebesgue sur l'espace de départ.

	Autrement dit, les résultats présentés ici sont un peu plus forts que ce dont nous avons réellement besoin \ldots ou alors ce sont les hypothèses que nous allons poser en théorie de l'intégration, qui seront un peu plus fortes que nécessaires. C'est une question de point de vue.
\end{normaltext}

\begin{corollary}       \label{CorooJYDVooCrXVun}
	Si \( I\) est un intervalle de \( \eR\), alors toute application monotone \( f\colon I\to \eR\) est borélienne.
\end{corollary}

\begin{proof}
	Puisque \( f\) est monotone, l'ensemble \( \{ f<a \}\) est un intervalle. Or tous les intervalles sont boréliens, donc \( f\) est mesurable par le théorème~\ref{THOooWHFLooKYGsOm}.
\end{proof}

\begin{definition}
	Si \( I\) est un intervalle de \( \eR\), une fonction \( f\colon I\to \eR\) a une propriété (monotone, mesurable, continue, etc.) \defe{par morceaux}{morceau!fonction continue ou monotone}\index{fonction!monotone!par morceaux}\index{fonction!continue!par morceaux} si il existe une suite strictement croissante de points \( (x_i)_{i\in \eZ}\) dans \( I\) telle que \( f\) ait la propriété sur chacun des ouverts \( \mathopen] x_j ,x_{j+1} \mathclose[.\).
\end{definition}
Dans cette définition, les points sont numérotés par \( \eZ\) et non par \( \eN\) parce que nous nous laissons la liberté d'avoir une infinité de points de chacun des deux côtés.

\begin{proposition}     \label{PropooLNBHooBHAWiD}
	Soit \( I\) un intervalle de \( \eR\) et une fonction \( f\colon I\to \eR\). Si \( f\) est continue ou monotone par morceaux sur \( I\) alors elle y est borélienne.
\end{proposition}

\begin{proof}
	L'ensemble \( \{  \mathopen] x_j , x_{j+1} \mathclose[  \}_{j\in \eZ}\cup\{ x_i \}_{i\in \eZ}\) forme une partition mesurable dénombrable de \( I\) (les singletons sont des boréliens). À une belle redéfinition près de la numérotation (deux fois \( \eZ\) va dans \( \eN\)), nous les appelons \( (I_n)_{n\in \eN}\), et nous définissons les fonctions \( f_k\) comme étant les restrictions de \( f\) aux intervalles \( I_k\).

	Toute fonction sur un singleton est mesurable. Toute fonction continue sur un ouvert est mesurable (théorème~\ref{ThoJDOKooKaaiJh}). Toute fonction monotone sur un ouvert est mesurable (corolaire~\ref{CorooJYDVooCrXVun}).

	Le lemme de recollement~\ref{LEMooXAPQooPpZUmP} donne alors la mesurabilité de \( f\).
\end{proof}

\begin{normaltext}
	Toutes les fonctions que nous pouvons écrire explicitement sont mesurables \ldots en tout cas toutes celles que l'on trouve en pratique. En effet nous avons déjà toutes les fonctions continues par morceaux via la proposition~\ref{PropooLNBHooBHAWiD} et ensuite toutes les limites par la proposition~\ref{PropooDXBGooSFqrai}. Cela donne les séries, les dérivées, les primitives, etc.
\end{normaltext}
