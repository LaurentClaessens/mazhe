%---------------------------------------------------------------------------------------------------------------------------
\subsection{Corps de rupture}
%---------------------------------------------------------------------------------------------------------------------------

\begin{definition}      \label{DEFooVALTooDJJmJv}
    Soit \( P\in\eK[X]\) un polynôme irréductible. Une extension \( \eL\) de \( \eK\) est un \defe{corps de rupture}{corps!de rupture}\index{rupture!corps} pour \( P\) s'il existe \( a\in \eL\) tel que \( P(a)=0\) et \( \eL=\eK(a)\).
\end{definition}

\begin{normaltext}      \label{NORMALooTPOIooVZAfUo}
    Nous insistons sur le fait que nous ne définissons le concept de corps de rupture pour un polynôme irréductible à coefficients dans un corps. Les deux points sont importants : irréductible et à coefficient dans un corps.

    Nous discuterons brièvement le pourquoi de cela dans la section~\ref{SUBSECooEDMJooTXBfOu} et surtout dans la question~\ref{ITEMooUBZIooDDcfWg} des questions difficiles d'algèbre.
\end{normaltext}

\begin{definition}[Polynôme scindé]\label{DefPolynomeScinde}
  Soit \( P\in\eK[X]\) un polynôme irréductible, et \( \eL\) un corps, extension du corps \( \eK\). On dit que \( P \) est \defe{scindé}{polynôme!scindé} dans \( \eL \) si \( P \) se décompose en un produit de polynômes de degré 1 dans \( \eL[X]\).
\end{definition}

\begin{example}     \label{ExemGVxJUC}
    Soit \( \eK=\eQ\) et \( P=X^2-2\). On pose \( a=\sqrt{2}\) et \( \eL=\eQ(\sqrt{2})\subset\eR\). De cette façon \( P\) est scindé dans \( \eL \):
    \begin{equation}
        P=(X-\sqrt{2})(X+\sqrt{2}).
    \end{equation}
    Le corps \( \eQ(\sqrt{2})\) est donc un corps de rupture pour \( P\).
\end{example}

\begin{example}
    Dans l'exemple~\ref{ExemGVxJUC}, nous avions un corps de rupture dans lequel le polynôme \( P\) était scindé. Il n'en est pas toujours ainsi. Prenons
    \begin{equation}
        P=X^3-2
    \end{equation}
    et \( a=\sqrt[3]{2}\). Nous avons, certes, \( P(a)=0\) dans \( \eQ(\sqrt[3]{2})\), mais \( P\) n'est pas scindé parce qu'il y a deux racines complexes.
\end{example}

\begin{example}
    Nous considérons le corps \( \eZ/p\eZ\) où \( p\) est un nombre premier. Si \( s\in \eZ/p\eZ\) n'est pas un carré, alors le polynôme \(P= X^2+s\) est irréductible et un corps de rupture de \( P\) sur \( \eZ/p\eZ\) est donné par \( (\eZ/p\eZ)[X]/(X^2+s)\), c'est à dire l'ensemble des polynômes de degré \( 1\) en \( \sqrt{s}\). Le cardinal en est \( p^2\).
\end{example}

Vu que nous allons abondamment parler du quotient \( \eK[X]/(P)\), nous nous permettons un petit lemme.
\begin{lemma}       \label{LEMooWYYFooXYacdF}
    Soit un corps \( \eK\) et \( P\in \eK[X]\) non constant. Alors \( \eK[X]/(P)\) est un corps si et seulement si \( P\) est irréductible.
\end{lemma}

\begin{proof}
    Nous utilisons le trio d'enfer dont il est question dans le thème~\ref{THEMEooZYKFooQXhiPD}. D'abord \( \eK[X]\) est un anneau principal par le lemme~\ref{LEMooIDSKooQfkeKp}. Donc \( \eK[X]/(P)\) sera un corps si et seulement si \( (P)\) est un idéal maximum (proposition~\ref{PROPooSHHWooCyZPPw}), et cela sera le cas si et seulement si \( (P)\) est engendré par un polynôme irréductible (proposition~\ref{PropomqcGe}).

    Il ne nous reste qu'à montrer que \( (P)\) est engendré par un irréductible si et seulement si \( P\) est irréductible. Il y a un sens dans lequel c'est évident.

    Soit un irréductible \( \mu\) tel que \( (P)=(\mu)\). En particulier \( \mu\in (P)\), c'est à dire qu'il existe \( Q\) tel que \( \mu=PQ\). Vu que \( \mu\) est irréductible, soit \( P\) soit \( Q\) est inversible. Si \( P\) est inversible, c'est à dire constant, ce que nous avons exclu par hypothèse. Si par contre \( Q\) est inversible, alors \( P=k\mu\) pour un certain \( k\in \eK\), ce qui montre que \( P\) est irréductible autant que \( \mu\).
\end{proof}

\begin{proposition}[Existence d'un corps de rupture]        \label{PROPooUBIIooGZQyeE}
    Soit un corps \( \eK\) et un polynôme irréductible non constant \( P\). Alors
    \begin{enumerate}
        \item
            Le corps \( \eL=\eK[X]/(P)\) est un corps de rupture pour \( P\).
        \item
            L'élément \( \bar X\) de \( \eL\) est une racine de \( P\).
        \item
            \( \eL=\eK(\bar X)_{\eL}\)
    \end{enumerate}
\end{proposition}

\begin{proof}
    Commençons par nous convaincre que \( \eK[X]/(P)\) est une extension de \( \eK\) (définition~\ref{DEFooFLJJooGJYDOe}). Le fait que ce soit un corps est le lemme~\ref{LEMooWYYFooXYacdF}. Le morphisme \( j\colon \eK\to \eK[X]/(P)\) est simplement \( k\mapsto \bar k\) où à droite, \( \bar k\) voit \( k\) dans \( \eK[X]\) comme étant le polynôme constant. Notez qu'il est automatiquement injectif (lemme~\ref{LEMooWBOPooZnsZgQ}).

    Il faut maintenant voir que \( \eK[X]/(P)=\eK(\alpha)\) pour un certain \( \alpha\in \eK[X]/(P)\). Grâce à notre compréhension des notations acquise dans~\ref{SUBSUBSECooPNBYooWXEHrg}, nous savons que \( X\in\eK[X]\) et qu'il est donc parfaitement légitime de poser \( \alpha=\bar X\) dans \( \eK[X]/(P)\). Il s'agit simplement de l'ensemble \( \bar X=\{ X+QP\tq Q\in \eK[X] \}\) où \( X\) est une notation pour la suite \( (0,1,0,0,\ldots)\).

    Bref, nous notons \( \alpha=\bar X\) et nous démontrons que \( P(\alpha)=0\) et que \( \eK[X]/(P)=\eK(\alpha)\) (isomorphisme de corps).
    \begin{subproof}
        \item[\( P(\bar X)=0\)]

            C'est le moment de nous souvenir comment la notation des \( X\) fonctionne, et en particulier la pirouette autour de \eqref{EQooABULooFCEasf}. D'abord la définition du produit sur \( \eK[X]/(P)\) est \( \bar P\bar Q=\overline{ PQ }\); en particulier si \( P=\sum_ka_kX^k\), alors \( P(\bar X)=\sum_ka_k\bar X^k=\sum_ka_k\overline{ X^k }\), et
            \begin{equation}
                P(\bar X)=\overline{ P(X) }=\bar P=0.
            \end{equation}
        \item[L'égalité]

            Nous montrons à présent que \( \eK(\bar X)_{\eL}=\eL\). C'est à dire que \( \eL\) est bien engendrée par \( \eK\) et un seul élément. D'abord, \( \eL=\eK[X]/(P)\) contient bien évidemment \( \eK\) et \( \bar X\). Ensuite nous devons prouver que tout sous-corps de \( \eL\) contenant \( \eK\) et \( \bar X\) est en réalité \( \eL\) entier.

            Soit \( Q\in \eK[X]\), et montrons que \( \bar Q\) est dans tout sous-corps de \( \eL\) contenant \( \eK\) et \( \bar X\).

            Par le lemme~\ref{LEMooXFMAooMBgIrN} nous avons \( \bar Q=Q(\bar X)\). Et si un corps contient \( \eK\) et \( \bar X\), il doit contenir tous les polynômes en \( \bar X\) à coefficients dans \( \eK\). Donc un tel corps doit contenir \( Q(\bar X)\) et donc \( \bar Q\).

    \end{subproof}
\end{proof}

\begin{example}
    Soit le polynôme \( P=X^2+1\in \eZ[X]\). Dans le quotient \( \eZ[X]/(P)\) nous avons \( \bar X^2+1=0\) et donc \( \bar X^2=-1\). C'est à dire que \( \eZ[X]/(P)\) contient un élément dont le carré est \( -1\). Avouez que c'est bien ce à quoi nous nous attendions.

    Notons que \( -\bar X\) est également une racine de \( P\) dans \( \eZ[X]/(P)\).

    En calculant dans les polynômes à coefficients dans \( \eZ(\bar X)\) nous avons :
    \begin{equation}
        (X+\bar X)(X-\bar X)=X^2-\bar X^2=X^2+1,
    \end{equation}
    c'est à dire que \( P\) est bien factorisé, et que nous avons retrouvé la multiplication \( x^2+1=(x+i)(x-i)\).
\end{example}

\begin{normaltext}
    Il n'y a évidemment pas unicité d'un corps de rupture pour un polynôme donné. Une raison est qu'un polynôme peut accepter plusieurs racines complètement indépendantes. Le corps étendu par l'une ou l'autre racine donne deux corps de rupture différents. Par exemple dans \( \eQ[X]\), le polynôme
    \begin{equation}
        P=X^4-X^2-2
    \end{equation}
    a pour racines (dans \( \eC\)) les nombres \( \sqrt{ 2 }\) et \( i\). Donc on a deux corps de rupture complètement différents : \( \eQ(\sqrt{ 2 })\) et \( \eQ(i)\).
\end{normaltext}

\begin{normaltext}
    La proposition suivante donne une unicité du corps de rupture dans le cas d'un polynôme irréductible. Et nous comprenons pourquoi : un polynôme irréductible n'a fondamentalement qu'une seule racine «indépendante». Par exemple \( X^2-2\) a pour racines \( \pm\sqrt{ 2 }\). Autre exemple, le polynôme \( X^2+6X+13\) a pour racines, dans \( \eC\), les nombres complexes conjugués \( z=-3+2i\) et \( \bar z=-3-2i\).
\end{normaltext}

\begin{proposition}[\cite{ooUHHUooONXDDl}]          \label{PROPooVJACooNDmlfb}
    Soient un corps \( \eK\) et un polynôme irréductible \( P\in \eK[X]\). Alors toute extension \( \eL\) contenant une racine \( \alpha\) de \( P\) admet un unique morphisme de corps
            \begin{equation}
                \psi\colon \eK[X]/(P)\to \eL
            \end{equation}
            tel que \( \psi(\bar X)=\alpha\).

    Dans un tel cas,
    \begin{enumerate}
        \item
            l'image de \( \psi\) est $\eK(\alpha)_{\eL}$ ,
        \item       \label{ITEMooHRFHooWLIdWU}
            si \( \eL=\eK(\alpha)_{\eL}\) alors \( \psi\) est un isomorphisme.
    \end{enumerate}

\end{proposition}

\begin{proof}
    L'idéal annulateur de \( \alpha\) parmi les polynôme de \( \eK[X]\) n'est pas réduit à \( \{ 0 \}\) parce qu'il contient \( P\). Le lemme~\ref{DefCVMooFGSAgL} s'applique donc et nous avons le polynôme minimal \( \mu\) de \( \alpha\) dans \( \eK[X]\). Il divise \( P\) qui est irréductible, donc
    \begin{equation}
        P=\lambda \mu
    \end{equation}
    pour un certain \( \lambda\in \eK\).

    Nous posons
    \begin{equation}
        \begin{aligned}
            \psi\colon \eK[X]/(P)&\to \eL \\
            \bar Q&\mapsto Q(\alpha).
        \end{aligned}
    \end{equation}
    \begin{subproof}
        \item[Bien définie]
            Si \( \bar Q_1=\bar Q_2\) alors il existe un \( R\in \eK[X]\) tel que \( Q_1=Q_2+RP\). Mais alors \( \psi(\bar Q_1)=Q_1(\alpha)=Q_2(\alpha)+R(\alpha)P(\alpha)=Q_2(\alpha)\).
        \item[Injective]

            Si \( \psi(\bar Q_1)=\psi(\bar Q_2)\) alors \( Q_1-Q_2=R\) pour un certain \( R\in \eK[X]\) vérifiant \( R(\alpha)=0\). Nous avons alors un polynôme \( S\) tel que \( R=S\mu=\lambda^{-1}SP\). Donc \( \bar R=0\) et donc \( \bar Q_1=\bar Q_2\).

        \item[Morphisme]

            Laissé comme exercice; la paresse de l'auteur de ces lignes attend vos contributions.

        \item[La condition]

            Le morphisme \( \psi\) respecte de plus la condition
            \begin{equation}
                \psi(\bar X)=X(\alpha)=\alpha.
            \end{equation}

    \end{subproof}

    En ce qui concerne l'unicité, fixer \( \psi(\bar X)\) est suffisant pour fixer un morphisme. En effet si \( \psi(\bar X)=\alpha\), alors
    \begin{equation}
        \psi(\bar Q)=\psi\Big( \sum_ka_k\bar X^k \Big)=\sum_ka_k\psi(\bar X)^k=\sum_ka_k\alpha^k.
    \end{equation}

    Pour le second point de l'énoncé, il faut remarquer que \( \alpha\) est algébrique et non transcendant. Donc en utilisant les propositions~\ref{PROPooPMNSooOkHOxJ} et~\ref{PropURZooVtwNXE}\ref{ItemJCMooDgEHaji} nous trouvons
    \begin{equation}
        \Image(\psi)=\{ Q(\alpha)\tq Q\in \eK[X] \}=\eK[\alpha]=\eK(\alpha).
    \end{equation}

    Et finalement pour le dernier point, un morphisme de corps est toujours injectif. Si il est également surjectif, il sera bijectif.
\end{proof}

%---------------------------------------------------------------------------------------------------------------------------
\subsection{Pile d'extensions}
%---------------------------------------------------------------------------------------------------------------------------

\begin{lemma}[\cite{MonCerveau}]        \label{LEMooTURZooXnjmjT}
    Soient un corps \( \eK\), des extensions \( \eL_1\),\ldots, \( \eL_n\) et des éléments \( \alpha_i\in \eL_i\) tels que
    \begin{equation}
        \eL_k=\eL_{k-1}(\alpha_k)_{\eL_k}.
    \end{equation}
    Alors
    \begin{equation}
        \eL_n=\eK(\alpha_1,\ldots, \alpha_n)_{\eL_n}.
    \end{equation}
\end{lemma}

\begin{proof}
    Nous démontrons par récurrence sur \( n\), en supposant que c'est fait pour \( n=1\). Supposons donc que le lemme soit correct pour \( n\), et étudions le cas \( n+1\). Nous avons, par définition et par hypothèse de récurrence :
    \begin{equation}
        \eL_{n+1}=\eL_n(\alpha_{n+1})_{\eL_{n+1}}=\Big( \eK(\alpha_1,\ldots, \alpha_n)_{\eL_n} \Big)(\alpha_{n+1})_{\eL_{n+1}}.
    \end{equation}
    Notre tâche sera donc de montrer que
    \begin{equation}\label{EQooIHMGooTlPcsd}
        \Big( \eK(\alpha_1,\ldots, \alpha_n)_{\eL_n} \Big)(\alpha_{n+1})=\eK(\alpha_1,\ldots,\alpha_{n+1})
    \end{equation}
    où nous n'écrivons plus les indices \( \eL_{n+1}\) partout.

    Le membre de gauche est un sous-corps de \( \eL_{n+1}\) contenant à la fois \( \eK \) et tous les \(\alpha_i \), si bien que
    \begin{equation}\label{EQooLLRHooHOjLfk}
        \eK(\alpha_1,\ldots,\alpha_{n+1})\subset \big( \eK(\alpha_1,\ldots, \alpha_n)_{\eL_n} \big)(\alpha_{n+1})_{\eL_{n+1}}.
    \end{equation}

    Il faut donc prouver l'inclusion inverse; c'est-à-dire montrer que tout élément \( x \) du corps \( \big( \eK(\alpha_1,\ldots, \alpha_n)_{\eL_n} \big)(\alpha_{n+1})\) est forcément dans tout sous-corps de \( \eL_{n+1}\) contenant \( \eK\) et les \( \alpha_i\). Un tel élément \( x \) est, par la proposition~\ref{PROPooYSFNooFGbbCi}\ref{ITEMooATPTooVXKdlK}, de la forme \( r(\alpha_{n+1})\) avec \( r\in \eK(\alpha_1,\ldots, \alpha_{n})(X)\), c'est à dire
            \begin{equation}
                P(\alpha_{n+1})Q(\alpha_{n+1})^{-1}
            \end{equation}
            avec \( P,Q\in \eK(\alpha_1,\ldots, \alpha_n)[X]\).

        Prouvons d'abord que \( P(\alpha_{n+1})\) est dans tout sous-corps de \( \eL_{n+1}\) contenant \( \eK\) et les \( \alpha_i\), lorsque \( P\in \eK(\alpha_1,\ldots, \alpha_n)[X]\). Nous avons \( P=\sum_ia_iX^i\) pour \( a_i\in \eK(\alpha_1,\ldots, \alpha_n)\), et donc
            \begin{equation}
                P(\alpha_{n+1})=\sum_ia_i\alpha_{n+1}^i.
            \end{equation}
            Tout corps contenant \( \eK\) et les \( \alpha_1\),\ldots, \( \alpha_n\) contient les \( a_i\). Par produit, tout corps contenant \( \eK\), \( \alpha_1\),\ldots,  \( \alpha_{n+1}\) contient les termes \( a_i\alpha_{n+1}^i\), et donc \( P(\alpha_{n+1})\) par somme.

        De la même façon, si un corps contient \( \eK\) et les \( \alpha_i\)  (\( i=1,\ldots, n+1\)), alors il contient \( Q(\alpha_{n+1})\). Comme c'est un corps, il contient aussi son inverse \( Q(\alpha_{n+1})^{-1}\), et il contient aussi le produit
            \begin{equation}
                r(\alpha_{n+1})=P(\alpha_{n+1})Q(\alpha_{n+1})^{-1}.
            \end{equation}

On vient ainsi de montrer que tout élément \( x \in  \big( \eK(\alpha_1,\ldots, \alpha_n)_{\eL_n} \big)(\alpha_{n+1})\) était dans tout sous-corps de \( \eL_{n+1} \) qui contient \( \eK \) et les \( \alpha_i\)  (\( i=1,\ldots, n+1\)); en d'autres termes:
    \begin{equation}\label{EQooUFSMooJozpqL}
       \big( \eK(\alpha_1,\ldots, \alpha_n)_{\eL_n} \big)(\alpha_{n+1})_{\eL_{n+1}} \subset \eK(\alpha_1,\ldots,\alpha_{n+1}).
    \end{equation}
    Les inclusions \eqref{EQooLLRHooHOjLfk} et \ref{EQooUFSMooJozpqL} prouvent l'égalité d'ensembles \eqref{EQooIHMGooTlPcsd} que l'on voulait montrer.
\end{proof}


