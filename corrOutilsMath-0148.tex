% This is part of Outils mathématiques
% Copyright (c) 2012
%   Laurent Claessens
% See the file fdl-1.3.txt for copying conditions.

\begin{corrige}{OutilsMath-0148}

    Pour le rotationnel, nous calculons
    \begin{equation}
        \nabla\times F=\begin{vmatrix}
            e_x    &   e_y    &   e_z    \\
            \partial_x    &   \partial_y    &   \partial_z    \\
            y\sin(z)    &   x\sin(z)    &   1+xy\cos(z)
        \end{vmatrix}=0.
    \end{equation}
    Le rotationnel étant nul, nous devons avoir un potentiel. La fonction \( f\) dont \( F\) est le gradient doit satisfaire les équations
    \begin{subequations}
        \begin{numcases}{}
            \frac{ \partial f }{ \partial x }=y\sin(z)\\
            \frac{ \partial f }{ \partial y }=x\sin(z)\\
            \frac{ \partial f }{ \partial z }=1+xy\cos(z).
        \end{numcases}
    \end{subequations}
    La fonction \( f(x,y,z)=xy\sin(z)+z\) fait l'affaire.

    Maintenant que nous avons le potentiel, l'intégrale est facile :
    \begin{equation}
        \int_{\sigma}F=f\big( \sigma(1) \big)-f\big( \sigma(0) \big)=f(1,\sin(1),\pi)-f(0,0,0)=\pi-0=\pi.
    \end{equation}

\end{corrige}
