\begin{exercice}\label{exoMatlab0026}

\newcommand{\frtr}[1]{\frac{1}{(#1)^3}}
\newcommand{\frqu}[1]{\frac{1}{(#1)^4}}

Définissez une fonction qui, à un entier $n$, associe la matrice $A_n$ suivante, de genre $(n+1)\times (n+1)$ :
\[ A_n = \begin{pmatrix}
0 & 1 & 0 & 0 & \cdots & \cdots & 0 \\
\frtr{n} & 0 & \frqu{2} & 0 & & & \vdots \\
0 & \frtr{n-1} & 0 & \frqu{3} & & & \vdots \\
0 & 0 & \frtr{n-2} & 0 & \ddots & & 0 \\
\vdots & & & \ddots & \ddots & \frqu{n-1} & 0 \\
\vdots & & & & \frtr{2} & 0 & \frqu{n} \\
0 & \cdots & \cdots & 0 & 0 & 1 & 0 
\end{pmatrix} \]

\emph{Indication : Utilisez la commande \texttt{diag(v,k)}. Si vous n'arrivez pas à créer la fonction, construisez simplement la matrice pour $n=7$ (elle est alors de genre $8\times 8$).}

\corrref{Matlab0026}
\end{exercice}
