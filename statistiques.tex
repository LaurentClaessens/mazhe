%+++++++++++++++++++++++++++++++++++++++++++++++++++++++++++++++++++++++++++++++++++++++++++++++++++++++++++++++++++++++++++
\section{Notations et hypothèses}
%+++++++++++++++++++++++++++++++++++++++++++++++++++++++++++++++++++++++++++++++++++++++++++++++++++++++++++++++++++++++++++

Nous notons \( X\) le caractère à étudier, et \( \Omega\) l'ensemble des individus. Le caractère à étudier est vu comme une fonction sur \( \Omega\) :
\begin{equation}
    X\colon \Omega\to \eR,\eN,\{ 0,1 \},\ldots
\end{equation}
Les \defe{statistiques descriptives}{statistiques!descriptives} sont les techniques pour présenter et résumer les données : diagrammes, graphiques, indicateurs numériques : moyenne, écart-type, médiane, \ldots

Nous faisons les hypothèses suivantes :
\begin{enumerate}
    \item
        Chaque observation \( x_i\) est la réalisation de la variable aléatoire \( X\) qui sera de loi inconnue \( \mu\).
    \item
        Le \( n\)-uple \( (x_1,\ldots,x_n)\) est la réalisation de \( (X_1,\ldots,X_n)\) qui est l'échantillon de taille \( n\).
    \item
        Les variables aléatoires \( X_i\) sont indépendantes et identiquement distribuées, de loi commune \( \mu\). La loi \( \mu\) est la \defe{loi parente}{loi!parente} de l'échantillon.
\end{enumerate}

\begin{example}
    Un échantillon de taille \( 1\) consisterait à tirer au sort une personne dans une population et mesurer sa taille.
\end{example}

\begin{example}
    Une échantillon de taille \( n\) consisterait à tirer au sort \( n\) personnes dans une population et de mesurer leurs tailles.
\end{example}

L'\defe{inférence statistique}{inférence statistique} est l'art de dégager des informations sur la population à partir d'informations partielles : intervalles de confiance, estimateurs, test d'hypothèses, \ldots

En théorie des probabilités, nous connaissons la loi de la variable aléatoire \( X\) et nous en déduisons des informations sur le réalisations de \( X\) : valeur la plus probable, moyenne, intervalle dans lequel \( X(\omega)\) a le plus de chance d'appartenir. En statistique, au contraire, la loi est inconnues et nous cherchons des informations sur la loi à partir d'un échantillon de données numériques observées.

%+++++++++++++++++++++++++++++++++++++++++++++++++++++++++++++++++++++++++++++++++++++++++++++++++++++++++++++++++++++++++++
\section{Modèle statistique}
%+++++++++++++++++++++++++++++++++++++++++++++++++++++++++++++++++++++++++++++++++++++++++++++++++++++++++++++++++++++++++++

Une \defe{modèle statistique}{modèle statistique} est un triplet
\begin{equation}
    \statS=\Big[ (\Omega,\tribF,P),(X_{\theta})_{\theta\in\Theta},(\mu_{\theta})_{\theta\in\Theta} \Big]
\end{equation}
où \( (\Omega,\tribF,P)\) est un espace probabilisé, \( (X_{\theta})\) est une famille de variables aléatoires définies sur \( \Omega\) et telles que pour tout \( \theta\in\Theta\), la variable aléatoire \( X_{\theta}\) suive la loi \( \mu_{\theta}\). Les $\mu_{\theta}$ sont des mesures sur les boréliens de \( \eR\) et pour tout \( B\in\Borelien(\eR)\) nous avons
\begin{equation}
    P(X_{\theta}\in B)=\mu_{\theta}(B).
\end{equation}
\begin{remark}
    D'une certaine manière, l'introduction de \( \mu_{\theta}\) dans la définition est redondante parce que ces mesures sont déjà contenues dans la données des variables aléatoires \( X_{\theta}\).
\end{remark}

\begin{example}[Modèle statistique gaussien]
    Si nous savons que les variables aléatoires \( X_i\) suivent une loi gaussienne, alors nous considérons \( \Theta=\eR\times\eR^+\) et \( \theta=(m,\sigma^2j)\). Dans ce cas, \( \mu_{\theta}=\dN(m,\sigma^2)\) et le but de la statistique est de déterminer la valeur de \( \theta\) qui correspond à une population en partant de l'observation d'un échantillon.
\end{example}

\begin{definition}
    Si \( \Theta\subset\eR^k\), nous disons que le modèle statistique est un modèle \defe{paramétrique}{modèle!paramétrique}.
\end{definition}
Le modèle gaussien est un modèle paramétrique : dès que \( m\) et \( \sigma^2\) sont déterminés, la loi du phénomène \( X\) est connue.

\begin{definition}
    Pour chaque \( \theta\in\theta\), un \defe{échantillon}{échantillon} de taille \( n\) associé à un modèle statistique \( \big[ (\Omega,\tribF,P),(X_{\theta}),(\mu_{\theta}) \big]\) est un vecteur \( \big( X_{\theta,1},\ldots,X_{\theta,n} \big)\) de taille \( n\) de variables aléatoires indépendantes et identiquement distribuées de la même loi que la variable aléatoire \( X_{\theta}\). La loi \( \mu_{\theta}\) est la \defe{loi parente}{loi!parente} de l'échantillon.
\end{definition}

\begin{definition}
    Un \defe{modèle d'échantillonnage}{modèle!échantillonnage} sur le modèle statistique \( \statS\) est une famille \( (X_{\theta,1},\ldots, X_{\theta,n})_{\theta\in\Theta}\) d'échantillons de taille \( n\geq 1\).
\end{definition}

Nous noterons souvent \( (X_1,\ldots,X_n)\) à la place de \( (X_{\theta,1},\ldots,X_{\theta,n})\) un échantillon, mais il faut se souvenir que les \( X_i\) suivent toujours la même loi donnée par \( \theta\). La loi du vecteur \( (X_1,\ldots,X_n)\) est \( \mu_{\theta}\otimes\ldots\otimes\mu_{\theta}\) et est définie sur l'espace \( (\Omega^n,\tribF\otimes\ldots\otimes\tribF,P^{\otimes n})\).

\begin{remark}
    Le travail du statisticien est de proposer un modèle statistique \( \statS\) a priori. Si nous étudions la taille d'une population, nous allons choisir un modèle gaussien. Plus le modèle est précis, plus l'espace \( \Theta\) est petit mais plus il y a de risques que le vérité soit hors de l'ensemble considéré.
\end{remark}
<++>


