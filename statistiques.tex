%+++++++++++++++++++++++++++++++++++++++++++++++++++++++++++++++++++++++++++++++++++++++++++++++++++++++++++++++++++++++++++
\section{Notations et hypothèses}
%+++++++++++++++++++++++++++++++++++++++++++++++++++++++++++++++++++++++++++++++++++++++++++++++++++++++++++++++++++++++++++

Nous notons \( X\) le caractère à étudier, et \( \Omega\) l'ensemble des individus. Le caractère à étudier est vu comme une fonction sur \( \Omega\) :
\begin{equation}
    X\colon \Omega\to \eR,\eN,\{ 0,1 \},\ldots
\end{equation}
Les \defe{statistiques descriptives}{statistiques!descriptives} sont les techniques pour présenter et résumer les données : diagrammes, graphiques, indicateurs numériques : moyenne, écart-type, médiane, \ldots

Nous faisons les hypothèses suivantes :
\begin{enumerate}
    \item
        Chaque observation \( x_i\) est la réalisation de la variable aléatoire \( X\) qui sera de loi inconnue \( \mu\).
    \item
        Le \( n\)-uple \( (x_1,\ldots,x_n)\) est la réalisation de \( (X_1,\ldots,X_n)\) qui est l'échantillon de taille \( n\).
    \item
        Les variables aléatoires \( X_i\) sont indépendantes et identiquement distribuées, de loi commune \( \mu\). La loi \( \mu\) est la \defe{loi parente}{loi!parente} de l'échantillon.
\end{enumerate}

\begin{example}
    Un échantillon de taille \( 1\) consisterait à tirer au sort une personne dans une population et mesurer sa taille.
\end{example}

\begin{example}
    Une échantillon de taille \( n\) consisterait à tirer au sort \( n\) personnes dans une population et de mesurer leurs tailles.
\end{example}

L'\defe{inférence statistique}{inférence statistique} est l'art de dégager des informations sur la population à partir d'informations partielles : intervalles de confiance, estimateurs, test d'hypothèses, \ldots

En théorie des probabilités, nous connaissons la loi de la variable aléatoire \( X\) et nous en déduisons des informations sur le réalisations de \( X\) : valeur la plus probable, moyenne, intervalle dans lequel \( X(\omega)\) a le plus de chance d'appartenir. En statistique, au contraire, la loi est inconnues et nous cherchons des informations sur la loi à partir d'un échantillon de données numériques observées.

%+++++++++++++++++++++++++++++++++++++++++++++++++++++++++++++++++++++++++++++++++++++++++++++++++++++++++++++++++++++++++++
\section{Modèle statistique}
%+++++++++++++++++++++++++++++++++++++++++++++++++++++++++++++++++++++++++++++++++++++++++++++++++++++++++++++++++++++++++++

Une \defe{modèle statistique}{modèle statistique} est un triplet
\begin{equation}
    kkk
\end{equation}
<++>
