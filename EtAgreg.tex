% This is part of Mes notes de mathématique
% Copyright (c) 2015
%   Laurent Claessens
% See the file fdl-1.3.txt for copying conditions.

%+++++++++++++++++++++++++++++++++++++++++++++++++++++++++++++++++++++++++++++++++++++++++++++++++++++++++++++++++++++++++++ 
\section*{Ce cours à l'agrégation ?}
%+++++++++++++++++++++++++++++++++++++++++++++++++++++++++++++++++++++++++++++++++++++++++++++++++++++++++++++++++++++++++++

Peut-on utiliser ce cours pour \textbf{les oraux d'\href{http://agreg.org/}{agrégation}} (de mathématiques) ?  Cela est une question qui m'est arrivée quelque fois.  La réponse courte est : « je ne sais pas », et voici quelques éléments de réponses plus longues.

\begin{enumerate}
    \item
        La version 2012 a été disponible dans la bibliothèque à Paris, pour au moins la cession 2013.
    \item
        En 2014 un candidat (qui a réussi, je n'en dirai pas plus) en a amené une version avec lui et l'a utilisé sans être inquiété.
    \item 
        Il m'a été signalé en 2012 (de façon non officielle : par courrier privé) que les candidats ne pourraient pas apporter de copies imprimées « à la maison ».
    \item
        Quelque mois avant les oraux de 2015, à cause d'un changement de règlement de dernière minute, il a été refusé\footnote{De façon on ne peut moins officielle : par mail privé à moi-même et à au moins un autre candidat qui avait demandé.} parce que les ouvrages « non commercialisés ou non suffisamment diffusés » ont besoin d'une autorisation explicite.
    \item
        J'ai demandé cette autorisation en mai 2015. Lorsque j'aurai une réponse, je vous le ferai savoir ici\footnote{Si vous croyez en la règle du « trois mois sans réponses équivaut à approbation »\ldots}.
    \item
        La version 2013 de ce cours est disponible dans la bibliothèque universitaire de l'ENS Cachan :
        \begin{center}
            \url{https://catalogue.ens-cachan.fr/cgi-bin/koha/opac-detail.pl?biblionumber=59258}
        \end{center}
        Peut-être que cela suffit pour dire que ce cours n'est plus dans la catégorie « non suffisamment diffusé »\ldots

        D'ailleurs, si une version est disponible dans la bibliothèque de votre établissement, faites-le moi savoir.
    \item
        J'ai demandé ce que ferait un surveillant si un candidat sortait de son sac une version avec le cachet et la cote d'une bibliothèque. Pas de réponses.
    \item
        Enfin, ce cours est commercialisé. En effet pour la somme de 100€, je vous en imprime et envoie une version dédicacée (envoyez-moi un mail pour me dire la dédicace que vous souhaitez). Frais d'envoi non compris. Vous noterez que 100€, c'est tout à fait dans les prix de ce que vous devriez débourser dans le commerce pour un ouvrage de cette taille\footnote{La licence vous permet de le commercialiser pour votre profit, donc n'hésitez pas à me faire concurrence.}.
\end{enumerate}

Bref, la réponse est : \textbf{« Je ne sais pas »}. Mais en tout cas, il n'y a pas d'arguments en béton\footnote{Et encore moins officiels.} pour dire que vous auriez des ennuis si vous sortiez de votre sac la version de votre bibliothèque.

Quoi qu'il en soit, je vous encourage à poser la question vous même, surtout si vous êtes intervenant dans une préparation agreg et que vous avez des étudiants intéressés.

Histoire d'être cohérent, la demande que j'ai faite au jury d'agrégation concerne ces fichiers-ci (divisé en 3 volumes pour faciliter l'impression) :
\begin{center}
    \url{http://student.ulb.ac.be/~lclaesse/lefrido_septembre2015-volume1.pdf}\\
    \url{http://student.ulb.ac.be/~lclaesse/lefrido_septembre2015-volume2.pdf}\\
    \url{http://student.ulb.ac.be/~lclaesse/lefrido_septembre2015-volume3.pdf}
\end{center}

Tenez-moi au courant. 
