% This is part of Mes notes de mathématique
% Copyright (c) 2015-2016
%   Laurent Claessens
% See the file fdl-1.3.txt for copying conditions.

%+++++++++++++++++++++++++++++++++++++++++++++++++++++++++++++++++++++++++++++++++++++++++++++++++++++++++++++++++++++++++++ 
\section*{Ce cours à l'agrégation ?}
%+++++++++++++++++++++++++++++++++++++++++++++++++++++++++++++++++++++++++++++++++++++++++++++++++++++++++++++++++++++++++++

Peut-on utiliser ce cours pour \textbf{les oraux d'\href{http://agreg.org/}{agrégation}} (de mathématiques) ?  Cela est une question qui m'est arrivée quelques fois.  

Le règlement interdit d'apporter avec soi une version imprimée chez soi, et oblige de n'utiliser que des ressources commercialisées. Cela fait que le Frido \emph{tel que vous l'avez sous les yeux} n'est pas utilisable à l'agrégation. Pour utiliser le Frido, vous devrez payer; j'en suis le premier désolé.

Je suis en train de travailler à une version qui sera commercialisée sur \href{http://www.thebookedition.com/fr/}{thebookedition.com}. Elle sera publiée en deux volumes d'environ \( 800\) pages et \( 30\) euros chacun (plus les frais d'envoi). Notez que c'est à peu près le prix que vous auriez payé pour l'imprimer dans votre magasin de photocopie préféré.

Une version préliminaire de ce qui sera publié est téléchargeable ici :
\begin{itemize}
    \item \url{http://laurent.claessens-donadello.eu/pdf/prémilimaire-thebookedition_vol1.pdf}
    \item \url{http://laurent.claessens-donadello.eu/pdf/prémilimaire-thebookedition_vol2.pdf}
\end{itemize}
Lorsque ce sera dûment commercialisé, la grande question sera : avez-vous le droit de demander à votre université d'imprimer \emph{ces pdf-là} et de les mettre dans la malle ? Le règlement dit que le document doit être commercialisé, pas que le candidat doit passer par la voie commerciale pour se le procurer \ldots

Deux mots sur mon modèle économique. J'ai choisi de ne pas faire de bénéfice sur les ventes : à chaque copie vendue sur internet, je gagne zéro\footnote{En hexadécimal, ça se note \( 0\).} euros. Mon modèle économique est le don. Si vous réussissez l'agrégation et que vous pensez que le Frido y est pour quelque chose, n'hésitez pas à donner si votre situation vous le permet.

\vfill

J'accepte les donations.

% Moralement, vous devriez considérer ce fichier comme une section invariante
% au sens de la licence FDL.
% Autrement dit, je ne serais pas content que ce fichier soit modifié.

\begin{description}
\item[IBAN] FR76 3000 4004 0600 0035 2497 784
\item[BIC] BNPAFRPPBSC
\end{description}

