% This is part of Exercices de mathématique pour SVT
% Copyright (C) 2010
%   Laurent Claessens et Carlotta Donadello
% See the file fdl-1.3.txt for copying conditions.

\begin{exercice}\label{exoTD3-0009}

	Modèle de Hassel.

	Soit la suite $(u_n)_{n\in\eN}$ définie par
	\begin{equation}
		\begin{cases}
			u_{n+1}=\frac{ au_n }{ (1+u_n)^b }	&	\text{$\forall n\in\eN_0$}\\
			u_0=x,
		\end{cases}
	\end{equation}
	où $a$, $b$ et $x\geq 0$ sont des nombres réels.
	\begin{enumerate}
		\item
			On suppose que $a=\frac{ 1 }{2}$ et que $b$ et $x$ sont des nombres positifs quelconques. Montrer que $u_n\geq 0$ pour tout $n\in\eN$ et que la suite $(u_n)$ est décroissante. En déduire que $\lim_{n\to\infty}u_n=0$.
		\item
			On suppose que $a=2$, $b=1$ et $x=2$. Montrer que $\lim_{n\to\infty}u_n=1$.
		\item
			On suppose que $a=2$, $b=1$ et $x=\frac{ 1 }{2}$. Montrer que $\lim_{n\to\infty}u_n=1$.
		\item
			On suppose que $a=2$, $b=\frac{ 1 }{2}$ et $x=2$. Montrer que $\lim_{n\to\infty}u_n=3$.
	\end{enumerate}

\corrref{TD3-0009}
\end{exercice}
