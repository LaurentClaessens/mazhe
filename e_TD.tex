% This is part of Exercices de mathématique pour SVT
% Copyright (C) 2010
%   Laurent Claessens et Carlotta Donadello
% See the file fdl-1.3.txt for copying conditions.

\documentclass[a4paper,12pt]{article}

\usepackage{latexsym}
\usepackage{amsfonts}
\usepackage{amsmath}
\usepackage{amsthm}
\usepackage{amssymb}
\usepackage{bbm}

\usepackage{pstricks,pst-eucl,pstricks-add,calc}	% Les dépendances de phystricks.
\usepackage{graphicx}					% Pour l'inclusion d'image en pfd.
\usepackage{subfigure}


%\usepackage{fancyvrb}

\usepackage{stmaryrd}		% Pour le \obslash
\usepackage{xstring}		% Utilisé pour les références vers wikipédia
\usepackage{cases}
\usepackage{multicol}
\usepackage[all]{xy}


\usepackage{ifthen}
\usepackage[cdot,thinqspace,amssymb]{SIunits} 
 % L'option amssymb sert à éviter un conflit avec la commande \square de amssymb. Note qu'elle n'est plus accessible. Si tu en as besoin, faudra RTFM
%ftp://ftp.belnet.be/packages/ctan/macros/latex/contrib/SIunits/SIunits.pdf

\usepackage[nottoc]{tocbibind}
\usepackage[ps2pdf]{hyperref} 		%Doit être appelé en dernier.
\hypersetup{colorlinks=true,linkcolor=blue}


%%%%%%%%%%%%%%%%%%%%%%%%%%
%
%   Trucs mathématiques
%
%%%%%%%%%%%%%%%%%%%%%%%%

% ENSEMBLES DE NOMBRES
\newcommand{\eR}{\mathbbm{R}}
\newcommand{\eZ}{\mathbbm{Z}}
\newcommand{\eC}{\mathbbm{C}}
\newcommand{\eQ}{\mathbbm{Q}}
\newcommand{\eN}{\mathbbm{N}}

% ENSEMBLES de fonctions
\newcommand{\aL}{\mathcal{L}}		% Les applications linéaires
\newcommand{\aC}{\mathcal{C}}		% Les fonctions C^1, C^2 etc

% AUTRES
\newcommand{\sdS}{\mathcal{S}}		% L'ensemble des subdivisions d'un intervalle.



\newcommand{\mF}{\mathcal{F}}
\newcommand{\mG}{\mathcal{G}}
\newcommand{\mI}{\mathcal{I}}


\newcommand{\mtu}{\mathbbm{1}}  			% La matrice unité


%\newcommand{\efrac}[2]{\frac{ \displaystyle #1 }{\displaystyle #2 }}
%%%%%%%%%%%%%%%%%%%%%%%%%%
%
%   Numérotations en tout genre
%
%%%%%%%%%%%%%%%%%%%%%%%%

\setcounter{tocdepth}{3}		% Profondeur de la table des matièes
\setcounter{secnumdepth}{2}		% Profondeur dans le texte

%%%%%%%%%%%%%%%%%%%%%%%%%%
%
%   Les lignes magiques pour le texte en français.
%
%%%%%%%%%%%%%%%%%%%%%%%%

\usepackage[utf8]{inputenc}
\usepackage[T1]{fontenc}

\usepackage{textcomp}
%\usepackage{mathpazo}
\usepackage{lmodern}
%\usepackage{microtype}		% Pour la page de garde CTU
\usepackage{fancyhdr}
\usepackage{fancybox}
\usepackage[margin=2cm]{geometry} 
\usepackage[english,frenchb]{babel}

\usepackage{SystemeCorr}



%%%%%%%%%%%%%%%%%%%%%%%%%%%%%%%%%%%%%%%%%%%%%%%%%%%%%%%%%%%%%%%%%%%%%%%%%%%%%%%%%%
%
% Frontespizi e pie' di pagina
%
%%%%%%%%%%%%%%%%%%%%%%%%%%%%%%%%%%%%%%%%%%%%%%%%%%%%%%%%%%%%%%%%%%%%%%%%%%%%%%%%%%
\pagestyle{fancy}
\fancyhf{}%
\fancyfoot[C]{}%
\fancyhead[RE]{Analyse appliquée - CTU - \textsc{2010/11}\hfill  \textsc{\footnotesize Université de Franche-Comté}} %
\fancyhead[LO]{Analyse appliquée - CTU  -\textsc{2010/11}\hfill  \textsc{\footnotesize Université de Franche-Comté}}
\renewcommand{\headrulewidth}{0.5pt}
\renewcommand{\footrulewidth}{0pt}
\addtolength{\headheight}{3.5pt}
\fancypagestyle{plain}{\fancyhead{}
\renewcommand{\headrulewidth}{0pt}}
\vfuzz2pt



%%%%%%%%%%%%%%%%%%%%%%%%%%
%
%   Les théorèmes et choses attenantes
%
%%%%%%%%%%%%%%%%%%%%%%%%


\newcounter{numtho}
\newcounter{numprob}

\makeatletter
\@addtoreset{numtho}{chapter}
\@addtoreset{CountExercice}{section}
\makeatother

\newtheoremstyle{MyTheorems}%
		{9pt}{9pt}%
		{\itshape}%
		{}%
		{\bfseries}{.}%
		{\newline}%
		{}%
\newtheoremstyle{MyExamples}%
		{9pt}{9pt}%
		{}%
		{}%
		{\bfseries}{.}%
		{\newline}%
		{}%
\newtheoremstyle{MyRemarks}%
		{9pt}{9pt}%
		{}%
		{}%
		{\bfseries}{.}%
		{\newline}%
		{}%

\theoremstyle{MyExamples}	%\newtheorem{exemple}[numtho]{Exemple}		% Pour unification, ne plus utiliser
	                        \newtheorem{example}[numtho]{Exemple}

\theoremstyle{MyRemarks}	\newtheorem{remark}[numtho]{Remarque}

				\newtheorem{amusement}[numtho]{Amusement}
				\newtheorem{erreur}[numtho]{Error}
				\newtheorem{probleme}[numprob]{\fbox{\bf Problèmes et choses à faire}}

\theoremstyle{MyTheorems}
			\newtheorem{lemma}[numtho]{Lemme}
			\newtheorem{corollary}[numtho]{Corollaire}
			\newtheorem{theorem}[numtho]{Théorème}		

			\newtheorem{definition}[numtho]{Définition}
			\newtheorem{proposition}[numtho]{Proposition}
			%\newtheorem{theoreme}[numtho]{Théoreme}		% Pour unification, ne plus utiliser

% TODO : unifier theorem et theoreme dans le texte

\renewcommand{\thenumtho}{\thechapter.\arabic{numtho}}
% La numérotation des équations change dans les corrigés
\renewcommand{\theequation}{\arabic{equation}}
%\renewcommand{\theCountExercice}{\arabic{section}.\arabic{CountExercice}}		% Ce compteur est défini dans SystemeCorr.sty
\newcommand{\defe}[2]{\textbf{#1}\index{#2}}

%\newcounter{CounterExample}
%\renewcommand{\theCounterExample}{\thechapter.\arabic{CounterExample}}
%\newenvironment{exemple}{\noindent{\bf Exemple \theCounterExample} }{\hfill $\triangle$}
%\renewcommand{\theenumi}{(\roman{enumi})}


%%%%%%%%%%%%%%%%%%%%%%%%%%
%
%   Les macros qui font des choses
%
%%%%%%%%%%%%%%%%%%%%%%%%

\newcommand{\tq}{\text{ tel que }}
\newcommand{\tqs}{\text{ tels que }}
\newcommand{\mA}{\mathcal{A}}
\newcommand{\mO}{\mathcal{O}}
\newcommand{\quext}[1]{ \footnote{\textsf{#1}}  }


%%%%%%%%%%%%%%%%%%%%%%%%%%
%
%   Bibliographie, index et liste des notations
%
%%%%%%%%%%%%%%%%%%%%%%%%

\usepackage{makeidx}
\usepackage[nottoc]{tocbibind}		% Le paquetage qui fait en sorte que la biblio soit inclue correctement dans la table des matières.
\usepackage[refpage]{nomencl}
\renewcommand{\nomname}{Liste des notations}
%
%   Comment introduire des éléments dans l'index des notations.
%
% La liste des tags à mettre pour bien classer mes notations est :
% T		pour la topologie et théorie des ensembles
%
% La syntaxe est facile, par exemple 
% 		$\SL(2,\eR)$\nomenclature[G]{$\SL(2,\eR)$}{Le groupe de matrices deux par deux de déterminant 1.}
\renewcommand{\nomgroup}[1]{%
    \ifthenelse{\equal{#1}{T}}{\item[\textbf{Topologie et théorie des ensembles}]}{}%
    \ifthenelse{\equal{#1}{C}}{\item[\textbf{Courbes paramétrées}]}{}%
}

%%%%%%%%%%%%%%%%%%%%%%%%%%
%
%   DeclareMathOperator
%
%%%%%%%%%%%%%%%%%%%%%%%%

\DeclareMathOperator{\signe}{sgn}
\DeclareMathOperator{\Int}{Int}		% Intérieur d'un ensemble.
\DeclareMathOperator{\Diam}{Diam}	
\DeclareMathOperator{\id}{Id}	
\DeclareMathOperator{\Graph}{Graph}	
\DeclareMathOperator{\Dom}{Domaine}	
\DeclareMathOperator{\pr}{proj}
\DeclareMathOperator{\dom}{dom}
\DeclareMathOperator{\arctg}{arctg}
\DeclareMathOperator{\cotg}{cotg}
\DeclareMathOperator{\cosec}{cosec}
\DeclareMathOperator{\arcsinh}{arcsinh}


%%%%%%%%%%%%%%%%%%%%%%%%%%%%%%%%%%%%%%
%
% les petis yeux 
%
%%%%%%%%%%%%%%%%%%%%%%%%%%%%%%%%%%%%%%%%%%%%%

\newcommand{\coolexo}{$\circledast\circledast$}
\newcommand{\boringexo}{$\circleddash\circleddash$}
\newcommand{\minsyndical}{$\odot\odot$}
\newcommand{\mortelexo}{$\obslash\oslash$}
