% This is part of Exercices et corrigés de CdI-1
% Copyright (c) 2011
%   Laurent Claessens
% See the file fdl-1.3.txt for copying conditions.

\begin{corrige}{OutilsMath-0094}

    Tant la forme de la fonction que celle du domaine d'intégration demandent de passer aux coordonnées sphériques. Cédant à la pression, nous passons aux sphériques. Le domaine est donné par
    \begin{equation}
        \begin{aligned}[]
            \rho&\colon 0\to 1\\
            \theta&\colon 0\to \pi\\
            \varphi&\colon 0\to 2\pi.
        \end{aligned}
    \end{equation} 
    La fonction quant à elle vaut
    \begin{equation}
        f(\rho,\theta,\varphi)= e^{\rho^{3}}.
    \end{equation}
    Nous devons donc effectuer l'intégrale
    \begin{equation}
        I=\int_0^{2\pi}d\varphi\int_0^{\pi}\sin(\theta)d\theta\int_0^1 e^{\rho^3}r^2d\rho=\frac{ 3\pi }{ 3 }\int_0^1 e^{\rho^3}3\rho^2d\rho.
    \end{equation}
    Pour effectuer la dernière intégrale, nous faisons le changement de variable $u=\rho^3$, $du=3\rho^2d\rho$ :
    \begin{equation}
        I=\frac{ 4\pi }{ 3 }\int_0^1 e^{u}du=\frac{ 4\pi }{ 3 }(e-1).
    \end{equation}

\end{corrige}
