% This is part of Agregation : modélisation
% Copyright (c) 2011
%   Laurent Claessens
% See the file fdl-1.3.txt for copying conditions.

Ce chapitre provient en grande partie de \cite{ProbaDanielLi}.

\begin{definition}
    Si \( \Omega\) est un ensemble, un ensemble \( \tribA\) de sous-ensembles de \( \Omega\) est une \defe{tribu}{tribu} si
    \begin{enumerate}
        \item
            \( \Omega\in\tribA\);
        \item
            \( A\cup B\in\tribA\) pour tout \( A,B\in\tribA\), ce qui signifie que toutes les unions finies d'éléments de \( \tribA\) sont dans \( \tribA\);
        \item
            \( \complement A\in A\) pour tout \( A\in\tribA\);
        \item
            si \( A_n\) est une suite dénombrable d'éléments de \( \tribA\), alors \( \sup_{n\geq 1}A_n\in\tribA\).
    \end{enumerate}
    Le couple \( (\Omega,\tribA)\) est alors un \defe{espace mesuré}{espace!mesuré}.
\end{definition}

Une \defe{\wikipedia{en}{Measure_space}{mesure}}{mesure} sur l'espace mesurable \( (\Omega,\tribA)\) est une application \( \mu\colon \tribA\to \eR\cup\{ \infty \}\) telle que
\begin{enumerate}
    \item
        \( \mu(A)\geq 0\) pour tout \( A\in\tribA\);
    \item
        \( \mu(\emptyset)=0\);
    \item
        \( \mu\left( \bigcup_{i=0}^{\infty}A_i\right)=\sum_{i=0}^{\infty}\mu(A_i)\) si les \( A_i\) sont des éléments de \( \tribA\) deux à deux disjoints.
\end{enumerate}

Une \defe{mesure de probabilité}{mesure!probabilité} sur un espace mesuré \( (\Omega,\tribA)\) est une mesure positive telle que \( P(\Omega)=1\). Dans ce cas, le triple \( (\Omega,\tribA,P)\) est un \defe{espace de probabilité}{espace!de probabilité}.

Un point \( \omega\in\Omega\) est une \defe{observation}{observation}, une partie mesurable \( A\in\tribA\) est un \defe{événement}{événement}. L'ensemble \( A\cup B\) représente l'événement \( A\) ou \( B\) tandis que l'ensemble \( A\cap B\) représente l'événement \( A\) et \( B\).


Si les \( A_n\) sont des événements, nous définissons la \defe{limite supérieur}{limite!supérieure} et la \defe{limite inférieure}{limite!inférieure} de la suite \( A_n\) par
\begin{equation}
    \limsup_{n\to\infty}A_n=\bigcap_{n\geq 1}\bigcup_{k\geq n}A_k
\end{equation}
et
\begin{equation}
    \liminf_{n\to\infty}A_n=\bigcup_{n\geq 1}\bigcap_{k\geq n}A_k
\end{equation}
Si \( \omega\in\liminf A_n\), alors \( \omega\) réalise tous les \( A_n\) sauf un nombre fini.

Nous avons
\begin{equation}
    \limsup A_n=\{ \omega\in\Omega\tq \omega\in A_n\text{pour une infinité de \( n\)} \}.
\end{equation}

\begin{theorem}[Borel-Cantelli]\index{théorème!Borel-Cantelli}
    Si
    \begin{equation}
        \sum_{n=1}^{\infty}P(A_n)<\infty
    \end{equation}
    alors \( P(\limsup A_n)=0\).
\end{theorem}

\begin{proof}
    La condition \( \sum_{n\geq 1}P(A_n)<\infty\) signifie que la fonction
    \begin{equation}
        \varphi=\sum_{n\geq 1}\caract_{A_n}
    \end{equation}
    est \( P\)-intégrable. Par conséquent, elle est finie presque partout (au sens de \( P\)), c'est à dire
    \begin{equation}
        P(\varphi=\infty)=0.
    \end{equation}
    Les points \( \omega\) sur lesquels \( \varphi(\omega)=\infty\) sont ceux tels que
    \begin{equation}
        \sum_{n\geq 1}\caract_{A_n}(\omega)=\infty,
    \end{equation}
    c'est à dire les \( \omega\) qui appartiennent à une infinité d'ensembles \( A_n\), ou encore les \( \omega\in\limsup A_n\). Nous avons donc montré que
    \begin{equation}
        \{ \omega\tq \varphi(\omega)=\infty \}=\{ \omega\in\Omega\tq \omega\in A_n\text{pour une infinité de \( n\)} \}=\limsup A_n.
    \end{equation}
    Or l'hypothèse signifie que la probabilité du membre de gauche est nulle.
\end{proof}

\begin{corollary}
    Si \( \sum_{n=1}^{\infty}P(\complement A_n)<\infty\), alors presque surement tous les \( B_n\) sont réalisés à l'exception d'un nombre fini.
\end{corollary}

%+++++++++++++++++++++++++++++++++++++++++++++++++++++++++++++++++++++++++++++++++++++++++++++++++++++++++++++++++++++++++++
\section{Variables aléatoires}
%+++++++++++++++++++++++++++++++++++++++++++++++++++++++++++++++++++++++++++++++++++++++++++++++++++++++++++++++++++++++++++

\begin{definition}
    Une \defe{variable aléatoire}{variable aléatoire} est une application mesurable
    \begin{equation}
        X\colon (\Omega,\tribA)\to (\eR^d,\Borelien(\eR^d)).
    \end{equation}
\end{definition}
Nous convenons que \( \eR^1=\bar\eR\), c'est à dire que dans le cas où la variable aléatoire \( X\) est réelle, nous acceptons les valeurs \( \pm\infty\).

On dit que la variable aléatoire \( X\) a un \defe{moment d'ordre \( p\)}{moment} si \( X\in L^p(\Omega,\tribA,P)\) (\( 1\leq p<\infty\)). Si \( X\) est \defe{intégrable}{variable aléatoire!intégrable} (c'est à dire si \( X\in L^1\)), alors nous définissons
\begin{equation}
    E(X)=\int_{\Omega}XdP\in\eR^d.
\end{equation}
Si \( E(X)=0\) nous disons que la variable aléatoire est \defe{centrée}{variable aléatoire!centrée}. La variable aléatoire \( X-E(X)\) est la variable aléatoire centrée associée à \( X\).

Si \( X\in L^2(\Omega,\tribA,P)\) alors nous définissons la \defe{variance}{variance} de \( X\) par
\begin{equation}
    \Var(X)=E\big( [X-E(X)]^2 \big).
\end{equation}

\begin{proposition}
    La variance de la variable aléatoire \( X\) peut être exprimée par la formule
    \begin{equation}
        \Var(X)=E(X^2)-[E(X)]^2
    \end{equation}
    où \( X^2=X\cdot X\) et \( E(X)^2=\) sont des produits scalaires dans \( \eR^d\).
\end{proposition}

\begin{proof}
    De façon explicité, nous avons
    \begin{equation}
        E\big( [X-E(X)]^2 \big)=\int_{\Omega}\big( X(\omega)-E(X) \big)\cdot\big( X(\omega)-E(X) \big)dP(\omega)
    \end{equation}
    où \( E(X)\in\eR^d\) est une constante. En développant le produit scalaire nous avons
    \begin{subequations}
        \begin{align}
            E\big( [X-E(X)]^2 \big)&=E\big( X^2-2X\cdot E(X)+E(X)^2 \big)\\
            &=E(X^2)-2E(X)^2+E(X)^2\\
            &=E(X^2)-E(X)^2.
        \end{align}
    \end{subequations}
\end{proof}

