% This is part of Agregation : modélisation
% Copyright (c) 2011
%   Laurent Claessens
% See the file fdl-1.3.txt for copying conditions.

Ce chapitre provient en grande partie de \cite{ProbaDanielLi}.

%+++++++++++++++++++++++++++++++++++++++++++++++++++++++++++++++++++++++++++++++++++++++++++++++++++++++++++++++++++++++++++
\section{Rappels}
%+++++++++++++++++++++++++++++++++++++++++++++++++++++++++++++++++++++++++++++++++++++++++++++++++++++++++++++++++++++++++++

Ici nous utilisons la convention de la transformée de Fourier de \wikipedia{fr}{Transformée_de_Fourier}{wikipedia}, c'est à dire
\begin{subequations}
    \begin{align}
        \hat f(\xi)&=\int_{\eR} e^{-i\xi x}f(x)dx\\
        f(x)&=2\pi\int_{\eR} e^{i\xi x}\hat f(\xi)d\xi.
    \end{align}
\end{subequations}


%+++++++++++++++++++++++++++++++++++++++++++++++++++++++++++++++++++++++++++++++++++++++++++++++++++++++++++++++++++++++++++
\section{Théorie de la mesure}
%+++++++++++++++++++++++++++++++++++++++++++++++++++++++++++++++++++++++++++++++++++++++++++++++++++++++++++++++++++++++++++

%---------------------------------------------------------------------------------------------------------------------------
\subsection{Espaces mesurables et mesurés}
%---------------------------------------------------------------------------------------------------------------------------

\begin{definition}
    Si \( \Omega\) est un ensemble, un ensemble \( \tribA\) de sous-ensembles de \( \Omega\) est une \defe{tribu}{tribu} si
    \begin{enumerate}
        \item
            \( \Omega\in\tribA\);
        \item
            \( \complement A\in A\) pour tout \( A\in\tribA\);
        \item
            si \( (A_i)_{i\in I}\) est un ensemble au plus dénombrable d'éléments de \( \tribA\), alors \( \sup_{n\geq 1}A_n=\bigcup_{i\in I}A_i\in\tribA\).
    \end{enumerate}
    Le couple \( (\Omega,\tribA)\) est alors un \defe{espace mesuré}{espace!mesuré}.
\end{definition}

\begin{lemma}
    Une tribu est stable par intersections au plus dénombrables.
\end{lemma}

\begin{proof}
    Soit \( (A_i)_{i\in I}\) une famille au plus dénombrable d'éléments de la tribu \( \tribA\). Nous devons prouver que \( \bigcap_{i\in I}A_i\) est également un élément de \( \tribA\). Pour cela nous passons au complémentaire :
    \begin{equation}
        \complement\left( \bigcap_{i\in I}A_i \right)=\bigcup_{i\in I}\complement A_i.
    \end{equation}
    La définition d'une tribu implique que le membre de droite est un élément de la tribu. Par stabilité d'une tribu par complémentaire, l'ensemble \( \bigcap_{i\in I}A_i\) est également un élément de la tribu.
\end{proof}

La tribu que nous utiliserons toujours dans \( \eR^d\) est la tribu des \defe{boréliens}{boréliens}, notée \( \Borelien(\eR^d)\), qui est la tribu engendrée par les ouverts de \( \eR^d\). Une fonction \( f\colon (\Omega,\tribA)\to (\eR^d,\Borelien(\eR^d))\) est \defe{borélienne}{borélienne} si pour tout \( \mO\in\Borelien\), \( f^{-1}(\mO)\in\tribA\).

\begin{definition}
    Une \defe{\wikipedia{en}{Measure_space}{mesure}}{mesure} sur l'espace mesurable \( (\Omega,\tribA)\) est une application \( \mu\colon \tribA\to \eR\cup\{ \infty \}\) telle que
    \begin{enumerate}
        \item
            \( \mu(A)\geq 0\) pour tout \( A\in\tribA\);
        \item
            \( \mu(\emptyset)=0\);
        \item
            \( \mu\left( \bigcup_{i=0}^{\infty}A_i\right)=\sum_{i=0}^{\infty}\mu(A_i)\) si les \( A_i\) sont des éléments de \( \tribA\) deux à deux disjoints.
    \end{enumerate}
\end{definition}

Une application entre espace mesurés
\begin{equation}
    f\colon (\Omega,\tribA)\to (\Omega',\tribA')
\end{equation}
est \defe{mesurable}{mesurable!application} si pour tout \( B\in\tribA'\), l'ensemble \( f^{-1}(B)\) est dans \( \tribA\).

Si \( \mu\) est une mesure sur \( \eR^d\), une fonction \( f\colon \eR^d\to \eR\) est une \defe{densité}{densité d'une mesure} si pour tout \( A\subset\eR^d\) nous avons
\begin{equation}
    \mu(A)=\int_Af(x)dx
\end{equation}
où \( dx\) est la mesure de Lebesgue.

%---------------------------------------------------------------------------------------------------------------------------
\subsection{Mesure dominée}
%---------------------------------------------------------------------------------------------------------------------------

Soient \( \mu\) et \( \nu\) deux mesures sur le même espace \( \Omega\) et la même tribu \( \tribA\). Nous disons que la mesure \( \mu\) est \defe{dominée}{dominée!mesure}\cite{PersoFeng} par \( \nu\) si pour tout ensemble mesurable \( A\), \( \nu(A)=0\) implique \( \mu(A)=0\).

La mesure \( \mu\) est \defe{portée}{portée!mesure} par l'ensemble \( E\in\tribA\) si pour tout \( A\in\tribA\), 
\begin{equation}
    \mu(A)=\mu(A\cap E).
\end{equation}

Nous écrivons que \( \mu\perp\nu\)\nomenclature{\( \mu\perp\nu\)}{mesures perpendiculaires} si il existe un ensemble \( E\in\tribA\) tel que \( \mu\) soit porté par \( E\) et \( \nu\) soit porté par \( \complement E\).

\begin{theorem}[Radon-Nykodym]\index{Radon-Nykodym}
    Soient \( \mu\) et \( \nu\) deux mesures \( \sigma\)-finies sur un espace métrisable \( (\Omega,\tribA)\).
    \begin{enumerate}
        \item
            Il existe un unique couple de mesures \( \mu_1\) et \( \mu_2\) telles que
            \begin{enumerate}
                \item
                    \( \mu=\mu_1+\mu_2\)
                \item
                    \( \mu_1\) est dominé par \( \nu\)
                \item
                    \( \mu_2\perp\nu\).
            \end{enumerate}
            Dans ce cas, les mesures \( \mu_1\) et \( \mu_2\) sont positives et \( \sigma\)-finies.
        \item
            À égalité \( \nu\)-presque partout près, il existe une unique fonction mesurable positive \( f\) telle que pour tout mesurable \( A\),
            \begin{equation}
                \mu_1(A)=\int_Ad\nu=\int_{\Omega}\mtu_Afd\nu.
            \end{equation}
    \end{enumerate}
\end{theorem}

\begin{corollary}       \label{CorDomDens}
    Soient \( \mu\) et \( \nu\), deux mesures positives \( \sigma\)-finies sur \( (\Omega,\tribA)\). Alors \( \nu\) domine \( \mu\) si et seulement si \( \mu\) possède une densité par rapport à \( \nu\).
\end{corollary}
 
\begin{proof}
    Si \( \mu\) est dominée par \( \nu\), alors la décomposition \( \mu=\mu+0\) satisfait le théorème de Radon-Nykodym. Par conséquent il existe une fonction \( f\) telle que
    \begin{equation}
        \mu(A)=\int_Afd\nu.
    \end{equation}
    Cette fonction est alors une densité pour \( \mu\) par rapport à \( \nu\).

    Pour la réciproque, nous supposons que \( \mu\) a une densité \( f\) par rapport à \( \nu\), et que \( A\) est une ensemble de \( \nu\)-mesure nulle :
    \begin{equation}
        \nu(A)=\int_{\Omega}\mtu_Ad\nu=0.
    \end{equation}
    Cela signifie que la fonction \( \mtu_A\) est \( \nu\)-presque partout nulle. La fonction produit \( \mtu_Af\) est également nulle \( \nu\)-presque partout, et par conséquent
    \begin{equation}
        \mu(A)=\int_{\Omega}\mtu_Afd\nu=0.
    \end{equation}
\end{proof}

%---------------------------------------------------------------------------------------------------------------------------
\subsection{Intégrale par rapport à une mesure}
%---------------------------------------------------------------------------------------------------------------------------

Une fonction \( f\colon (\Omega,\tribA)\to (\Omega',\tribA')\) est \defe{mesurable}{mesurable!fonction} si 
\begin{equation}
    f^{-1}(E)\in\tribA
\end{equation}
pour tout \( E\in\tribA'\).


Une mesure \( \mu\) sur un espace mesurable \( (\Omega,\tribA)\) permet de définir une fonctionnelle linéaire sur l'ensemble des fonctions mesurables \( \Omega\to \eR\). Cette fonctionnelle linéaire est l'intégrale que nous allons définir à présent.

D'abord nous considérons les fonction \defe{simples}{simple!fonction}\index{fonction!simple}, c'est à dire les fonctions de la forme
\begin{equation}
    f=\sum_{i=1}^Na_i\caract_{E_i}
\end{equation}
où \( a_i\in\eR\) tandis que les \( E_i\) sont des ensembles \( \mu\)-mesurables. Si \( Y\in \tribA\) nous définissons
\begin{equation}
    \int_Yfd\mu=\sum_ia_i\mu(Y\cap E_i).
\end{equation}
Pour une fonction \( \mu\)-mesurable générale \( f\colon \Omega\to \mathopen[ 0 , \infty \mathclose]\) nous définissons l'intégrale de \( f\) sur \( Y\) par
\begin{equation}        \label{EqDefintYfdmu}
    \int_Yfd\mu=\sup\Big\{ \int_Yhd\mu\,\text{où \( h\) est une fonction simple et mesurable telle que \( 0\leq h\leq f\)} \Big\}.
\end{equation}
Maintenant nous définissons
\begin{equation}
    \mu(f)=\int_{\Omega}f
\end{equation}
si \( f\) est une fonction mesurable sur \( \Omega\).

\begin{remark}
    Dans \( \eR^d\), quasiment toutes les fonctions et ensembles sont mesurables. En effet la construction d'ensembles non mesurables demande obligatoirement l'utilisation de l'axiome du choix; de tels ensembles doivent être construits «exprès pour». Il y a très peu de chances pour que vous tombiez sur un ensemble non mesurable de \( \eR^d\) sans que vous ne vous en rendiez compte.

    Par exemple la variable aléatoire 
    \begin{equation}
        X(\omega)=\begin{cases}
            \frac{1}{ \omega }    &   \text{si $ \omega\neq 0$}\\
            \infty    &    \text{$\omega=0$}.
        \end{cases}
    \end{equation}
    est mesurable, mais non intégrable.
\end{remark}

%---------------------------------------------------------------------------------------------------------------------------
\subsection{Dérivation sous le signe intégral}
%---------------------------------------------------------------------------------------------------------------------------

Une question classique est la dérivation par rapport à \( x\) d'une fonction du type
\begin{equation}
    F(x)=\int f(x,t)dt.
\end{equation}
Il existe divers théorèmes qui répondent à ces questions. Dans notre cadre nous utiliserons le suivant.
\begin{theorem}     \label{ThoDerSousIntegrale}
    Soit \( A\) un ouvert de \( \eR\) et \( \Omega\), un espace mesuré. Soit une fonction \( f\colon A\times \Omega\to \eR\) qui satisfait
    \begin{enumerate}
        \item
            La fonction \( f\) est mesurable en tant que fonction \( A\times\Omega\to \eR\). Pour chaque \( x\in A\), la fonction \( f(x,\cdot)\) est intégrable sur \( \Omega\).
        \item
            Pour presque tout \( \omega\in\Omega\), la fonction \( f(x,\omega)\) est une fonction absolument continue de \( x\).
        \item
            La fonction \( \frac{ \partial f }{ \partial x }\) est localement intégrable, c'est à dire que pour tout \( \mathopen[ a , b \mathclose]\subset A\),
            \begin{equation}
                \int_a^b\int_{\Omega}\left| \frac{ \partial f }{ \partial x }(x,\omega) \right| d\omega\,dx<\infty.
            \end{equation}
    \end{enumerate}
    Alors la fonction de \( x\)
    \begin{equation}
        F(x)=\int_{\Omega}f(x,\omega)d\omega
    \end{equation}
    est absolument continue et pour presque tout \( x\in A\), la dérivée est donné par
    \begin{equation}
        \frac{ d }{ dx }\int_{\Omega}f(x,\omega)d\omega=\int_{\Omega}\frac{ \partial f }{ \partial x }(x,\omega)d\omega.
    \end{equation}
\end{theorem}

\begin{proposition}[Innégalité de Hölder]
    Soit \( \Omega\) un espace mesuré et \( 1\leq p\), \( q\leq\infty\) satisfaisant \( \frac{1}{ p }+\frac{1}{ q }=1\). Soient \( f\in L^p(\Omega)\), \( g\in L^q(\Omega)\). Alors le produit \( fg\) est dans \( L^1(\Omega)\) et nous avons
    \begin{equation}
        \| fg \|_1\leq \| f \|_p\| g \|_q.
    \end{equation}
\end{proposition}

\begin{remark}      \label{RemNormuptNird}
    Dans le cas d'un espace de probabilité, la fonction constante \( g=1\) appartient à \( L^p(\Omega)\). En prenant \( p=q=2\) nous obtenons
    \begin{equation}
        \| f \|_1\leq\| f \|_2.
    \end{equation}
\end{remark}


