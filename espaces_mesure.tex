% This is part of Agregation : modélisation
% Copyright (c) 2011
%   Laurent Claessens
% See the file fdl-1.3.txt for copying conditions.

Pour ce chapitre, \cite{FubiniBMauray,ProbaDanielLi}.

%+++++++++++++++++++++++++++++++++++++++++++++++++++++++++++++++++++++++++++++++++++++++++++++++++++++++++++++++++++++++++++
\section{Rappels}
%+++++++++++++++++++++++++++++++++++++++++++++++++++++++++++++++++++++++++++++++++++++++++++++++++++++++++++++++++++++++++++

Ici nous utilisons la convention de la transformée de Fourier de \wikipedia{fr}{Transformée_de_Fourier}{wikipedia}, c'est à dire
\begin{subequations}
    \begin{align}
        \hat f(\xi)&=\int_{\eR} e^{-i\xi x}f(x)dx\\
        f(x)&=2\pi\int_{\eR} e^{i\xi x}\hat f(\xi)d\xi.
    \end{align}
\end{subequations}


%+++++++++++++++++++++++++++++++++++++++++++++++++++++++++++++++++++++++++++++++++++++++++++++++++++++++++++++++++++++++++++
\section{Théorie de la mesure}
%+++++++++++++++++++++++++++++++++++++++++++++++++++++++++++++++++++++++++++++++++++++++++++++++++++++++++++++++++++++++++++

%---------------------------------------------------------------------------------------------------------------------------
\subsection{Espaces mesurables et mesurés}
%---------------------------------------------------------------------------------------------------------------------------

\begin{definition}
    Si \( \Omega\) est un ensemble, un ensemble \( \tribA\) de sous-ensembles de \( \Omega\) est une \defe{tribu}{tribu} si
    \begin{enumerate}
        \item
            \( \Omega\in\tribA\);
        \item
            \( \complement A\in A\) pour tout \( A\in\tribA\);
        \item
            si \( (A_i)_{i\in I}\) est un ensemble au plus dénombrable d'éléments de \( \tribA\), alors \( \sup_{n\geq 1}A_n=\bigcup_{i\in I}A_i\in\tribA\).
    \end{enumerate}
    Le couple \( (\Omega,\tribA)\) est alors un \defe{espace mesuré}{espace!mesuré}.
\end{definition}

\begin{lemma}
    Une tribu est stable par intersections au plus dénombrables.
\end{lemma}

\begin{proof}
    Soit \( (A_i)_{i\in I}\) une famille au plus dénombrable d'éléments de la tribu \( \tribA\). Nous devons prouver que \( \bigcap_{i\in I}A_i\) est également un élément de \( \tribA\). Pour cela nous passons au complémentaire :
    \begin{equation}
        \complement\left( \bigcap_{i\in I}A_i \right)=\bigcup_{i\in I}\complement A_i.
    \end{equation}
    La définition d'une tribu implique que le membre de droite est un élément de la tribu. Par stabilité d'une tribu par complémentaire, l'ensemble \( \bigcap_{i\in I}A_i\) est également un élément de la tribu.
\end{proof}

La tribu que nous utiliserons toujours dans \( \eR^d\) est la tribu des \defe{boréliens}{boréliens}, notée \( \Borelien(\eR^d)\), qui est la tribu engendrée par les ouverts de \( \eR^d\). Une fonction \( f\colon (\Omega,\tribA)\to (\eR^d,\Borelien(\eR^d))\) est \defe{borélienne}{borélienne} si pour tout \( \mO\in\Borelien\), \( f^{-1}(\mO)\in\tribA\).

\begin{definition}
    Une \defe{\wikipedia{en}{Measure_space}{mesure}}{mesure} sur l'espace mesurable \( (\Omega,\tribA)\) est une application \( \mu\colon \tribA\to \eR\cup\{ \infty \}\) telle que
    \begin{enumerate}
        \item
            \( \mu(A)\geq 0\) pour tout \( A\in\tribA\);
        \item
            \( \mu(\emptyset)=0\);
        \item
            \( \mu\left( \bigcup_{i=0}^{\infty}A_i\right)=\sum_{i=0}^{\infty}\mu(A_i)\) si les \( A_i\) sont des éléments de \( \tribA\) deux à deux disjoints.
    \end{enumerate}
    Une mesure est \defe{\( \sigma\)-finie}{mesure!$\sigma$-finie} si il existe une suite croissante (pour l'inclusion) d'éléments \( (E_n)_{n\in\eN}\) de la tribu, tous de mesure finie et tels que \( \Omega=\bigcup_{n\in \eN}E_n\). Si la mesure est $\sigma$-finie, nous disons que l'espace \( (\Omega,\tribA,\mu)\) est un espace mesuré $\sigma$-fini.
\end{definition}

\begin{definition}
    Une application entre espace mesurés
    \begin{equation}
        f\colon (\Omega,\tribA)\to (\Omega',\tribA')
    \end{equation}
    est \defe{mesurable}{mesurable!application} si pour tout \( B\in\tribA'\), l'ensemble \( f^{-1}(B)\) est dans \( \tribA\).
\end{definition}

Si \( \mu\) est une mesure sur \( \eR^d\), une fonction \( f\colon \eR^d\to \eR\) est une \defe{densité}{densité d'une mesure} si pour tout \( A\subset\eR^d\) nous avons
\begin{equation}
    \mu(A)=\int_Af(x)dx
\end{equation}
où \( dx\) est la mesure de Lebesgue.

%---------------------------------------------------------------------------------------------------------------------------
\subsection{Mesure produit}
%---------------------------------------------------------------------------------------------------------------------------

Si \( \tribA\) et \( \tribB\) sont deux tribus sur deux ensembles \( \Omega_1\) et \( \Omega_2\), nous définissons la \defe{tribu produit}{tribu!produit} \( \tribA\otimes\tribB\) comme étant la tribu engendrée par 
\begin{equation}
    \{ A\times B\tq A\in\tribA,B\in\tribB \}.
\end{equation}


\begin{theorem}\index{mesure!produit}
    Soient \( \mu_i\) des mesures $\sigma$-finies sur \( (\Omega_i,\tribA_i)\) (\( i=1,2\)). Il existe une et une seule mesure, notée \( \mu_1\otimes \mu_2\), sur \( (\Omega_1\times\Omega_2,\tribA_1\otimes\tribA_2)\) telle que
    \begin{equation}
        (\mu_1\otimes\mu_2)(A_1\times A_2)=\mu_1(A_1)\mu_2(A_2)
    \end{equation}
    pour tout \( A_1\in \tribA_1\) et \( A_2\in\tribA_2\).
\end{theorem}
Une preuve peut être trouvée dans \cite{FubiniBMauray}.

%---------------------------------------------------------------------------------------------------------------------------
\subsection{Mesure dominée}
%---------------------------------------------------------------------------------------------------------------------------

Soient \( \mu\) et \( \nu\) deux mesures sur le même espace \( \Omega\) et la même tribu \( \tribA\). Nous disons que la mesure \( \mu\) est \defe{dominée}{dominée!mesure}\cite{PersoFeng} par \( \nu\) si pour tout ensemble mesurable \( A\), \( \nu(A)=0\) implique \( \mu(A)=0\).

La mesure \( \mu\) est \defe{portée}{portée!mesure} par l'ensemble \( E\in\tribA\) si pour tout \( A\in\tribA\), 
\begin{equation}
    \mu(A)=\mu(A\cap E).
\end{equation}

Nous écrivons que \( \mu\perp\nu\)\nomenclature{\( \mu\perp\nu\)}{mesures perpendiculaires} si il existe un ensemble \( E\in\tribA\) tel que \( \mu\) soit porté par \( E\) et \( \nu\) soit porté par \( \complement E\).

\begin{theorem}[Radon-Nykodym]\index{Radon-Nykodym}
    Soient \( \mu\) et \( \nu\) deux mesures \( \sigma\)-finies sur un espace métrisable \( (\Omega,\tribA)\).
    \begin{enumerate}
        \item
            Il existe un unique couple de mesures \( \mu_1\) et \( \mu_2\) telles que
            \begin{enumerate}
                \item
                    \( \mu=\mu_1+\mu_2\)
                \item
                    \( \mu_1\) est dominé par \( \nu\)
                \item
                    \( \mu_2\perp\nu\).
            \end{enumerate}
            Dans ce cas, les mesures \( \mu_1\) et \( \mu_2\) sont positives et \( \sigma\)-finies.
        \item
            À égalité \( \nu\)-presque partout près, il existe une unique fonction mesurable positive \( f\) telle que pour tout mesurable \( A\),
            \begin{equation}
                \mu_1(A)=\int_Ad\nu=\int_{\Omega}\mtu_Afd\nu.
            \end{equation}
    \end{enumerate}
\end{theorem}

\begin{corollary}       \label{CorDomDens}
    Soient \( \mu\) et \( \nu\), deux mesures positives \( \sigma\)-finies sur \( (\Omega,\tribA)\). Alors \( \nu\) domine \( \mu\) si et seulement si \( \mu\) possède une densité par rapport à \( \nu\).
\end{corollary}
 
\begin{proof}
    Si \( \mu\) est dominée par \( \nu\), alors la décomposition \( \mu=\mu+0\) satisfait le théorème de Radon-Nykodym. Par conséquent il existe une fonction \( f\) telle que
    \begin{equation}
        \mu(A)=\int_Afd\nu.
    \end{equation}
    Cette fonction est alors une densité pour \( \mu\) par rapport à \( \nu\).

    Pour la réciproque, nous supposons que \( \mu\) a une densité \( f\) par rapport à \( \nu\), et que \( A\) est une ensemble de \( \nu\)-mesure nulle :
    \begin{equation}
        \nu(A)=\int_{\Omega}\mtu_Ad\nu=0.
    \end{equation}
    Cela signifie que la fonction \( \mtu_A\) est \( \nu\)-presque partout nulle. La fonction produit \( \mtu_Af\) est également nulle \( \nu\)-presque partout, et par conséquent
    \begin{equation}
        \mu(A)=\int_{\Omega}\mtu_Afd\nu=0.
    \end{equation}
\end{proof}

%---------------------------------------------------------------------------------------------------------------------------
\subsection{Intégrale par rapport à une mesure}
%---------------------------------------------------------------------------------------------------------------------------

Une fonction \( f\colon (\Omega,\tribA)\to (\Omega',\tribA')\) est \defe{mesurable}{mesurable!fonction} si 
\begin{equation}
    f^{-1}(E)\in\tribA
\end{equation}
pour tout \( E\in\tribA'\).


Une mesure \( \mu\) sur un espace mesurable \( (\Omega,\tribA)\) permet de définir une fonctionnelle linéaire sur l'ensemble des fonctions mesurables \( \Omega\to \eR\). Cette fonctionnelle linéaire est l'intégrale que nous allons définir à présent.

D'abord nous considérons les fonction \defe{simples}{simple!fonction}\index{fonction!simple}, c'est à dire les fonctions de la forme
\begin{equation}
    f=\sum_{i=1}^Na_i\caract_{E_i}
\end{equation}
où \( a_i\in\eR\) tandis que les \( E_i\) sont des ensembles \( \mu\)-mesurables. Si \( Y\in \tribA\) nous définissons
\begin{equation}
    \int_Yfd\mu=\sum_ia_i\mu(Y\cap E_i).
\end{equation}
Pour une fonction \( \mu\)-mesurable générale \( f\colon \Omega\to \mathopen[ 0 , \infty \mathclose]\) nous définissons l'intégrale de \( f\) sur \( Y\) par
\begin{equation}        \label{EqDefintYfdmu}
    \int_Yfd\mu=\sup\Big\{ \int_Yhd\mu\,\text{où \( h\) est une fonction simple et mesurable telle que \( 0\leq h\leq f\)} \Big\}.
\end{equation}
Maintenant nous définissons
\begin{equation}
    \mu(f)=\int_{\Omega}f
\end{equation}
si \( f\) est une fonction mesurable sur \( \Omega\).

\begin{remark}
    Dans \( \eR^d\), quasiment toutes les fonctions et ensembles sont mesurables. En effet la construction d'ensembles non mesurables demande obligatoirement l'utilisation de l'axiome du choix; de tels ensembles doivent être construits «exprès pour». Il y a très peu de chances pour que vous tombiez sur un ensemble non mesurable de \( \eR^d\) sans que vous ne vous en rendiez compte.

    Par exemple la variable aléatoire 
    \begin{equation}
        X(\omega)=\begin{cases}
            \frac{1}{ \omega }    &   \text{si $ \omega\neq 0$}\\
            \infty    &    \text{$\omega=0$}.
        \end{cases}
    \end{equation}
    est mesurable, mais non intégrable.
\end{remark}

\begin{lemma}   \label{Lemfobnwt}
    Soit \( f\) une fonction mesurable positive ou nulle telle que
    \begin{equation}
        \int_{\Omega}fd\mu=0.
    \end{equation}
    Alors \( f=0\) \( \mu\)-presque partout.
\end{lemma}

\begin{proof}
    L'ensemble des points \( x\in\Omega\) tels que \( f(x)\neq 0\) peut s'écrire comme une union dénombrable disjointe :
    \begin{equation}
        \{ x\in\Omega\tq f(x)\neq 0 \}=\bigcup_{i=0}^{\infty}E_i
    \end{equation}
    avec
    \begin{subequations}
        \begin{align}
            E_0&=\{ x\in\Omega\tq f(x)>1 \}\\
            E_i&=\{ x\in\Omega\tq \frac{1}{ i+1 }\leq f(x)<\frac{1}{ i } \}.
        \end{align}
    \end{subequations}
    Si un des ensembles \( E_i\) est de mesure non nulle, alors nous pouvons considérer la fonction simple \( h(x)=\frac{1}{ i+1 }\mtu_{E_i}\) dont l'intégrale sur \( \Omega\) est strictement positive. Par conséquent le supremum de la définition \eqref{EqDefintYfdmu} est strictement positif.

    Nous savons donc que \( \mu(E_i)=0\) pour tout \( i\). Étant donné que la mesure d'une union disjointe dénombrable est égale à la somme des mesures, nous avons
    \begin{equation}
        \mu\{ x\in\Omega\tq f(x)\neq 0 \}=0,
    \end{equation}
    ce qui signifie que \( f\) est nulle \( \mu\)-presque partout.
\end{proof}

%---------------------------------------------------------------------------------------------------------------------------
\subsection{Dérivation sous le signe intégral}
%---------------------------------------------------------------------------------------------------------------------------

Une question classique est la dérivation par rapport à \( x\) d'une fonction du type
\begin{equation}
    F(x)=\int f(x,t)dt.
\end{equation}
Il existe divers théorèmes qui répondent à ces questions. Dans notre cadre nous utiliserons le suivant.
\begin{theorem}     \label{ThoDerSousIntegrale}
    Soit \( A\) un ouvert de \( \eR\) et \( \Omega\), un espace mesuré. Soit une fonction \( f\colon A\times \Omega\to \eR\) qui satisfait
    \begin{enumerate}
        \item
            La fonction \( f\) est mesurable en tant que fonction \( A\times\Omega\to \eR\). Pour chaque \( x\in A\), la fonction \( f(x,\cdot)\) est intégrable sur \( \Omega\).
        \item
            Pour presque tout \( \omega\in\Omega\), la fonction \( f(x,\omega)\) est une fonction absolument continue de \( x\).
        \item
            La fonction \( \frac{ \partial f }{ \partial x }\) est localement intégrable, c'est à dire que pour tout \( \mathopen[ a , b \mathclose]\subset A\),
            \begin{equation}
                \int_a^b\int_{\Omega}\left| \frac{ \partial f }{ \partial x }(x,\omega) \right| d\omega\,dx<\infty.
            \end{equation}
    \end{enumerate}
    Alors la fonction de \( x\)
    \begin{equation}
        F(x)=\int_{\Omega}f(x,\omega)d\omega
    \end{equation}
    est absolument continue et pour presque tout \( x\in A\), la dérivée est donné par
    \begin{equation}
        \frac{ d }{ dx }\int_{\Omega}f(x,\omega)d\omega=\int_{\Omega}\frac{ \partial f }{ \partial x }(x,\omega)d\omega.
    \end{equation}
\end{theorem}

\begin{proposition}[Innégalité de Hölder]
    Soit \( \Omega\) un espace mesuré et \( 1\leq p\), \( q\leq\infty\) satisfaisant \( \frac{1}{ p }+\frac{1}{ q }=1\). Soient \( f\in L^p(\Omega)\), \( g\in L^q(\Omega)\). Alors le produit \( fg\) est dans \( L^1(\Omega)\) et nous avons
    \begin{equation}
        \| fg \|_1\leq \| f \|_p\| g \|_q.
    \end{equation}
\end{proposition}

\begin{remark}      \label{RemNormuptNird}
    Dans le cas d'un espace de probabilité, la fonction constante \( g=1\) appartient à \( L^p(\Omega)\). En prenant \( p=q=2\) nous obtenons
    \begin{equation}
        \| f \|_1\leq\| f \|_2.
    \end{equation}
\end{remark}

%+++++++++++++++++++++++++++++++++++++++++++++++++++++++++++++++++++++++++++++++++++++++++++++++++++++++++++++++++++++++++++
\section{Suite et séries de fonctions}
%+++++++++++++++++++++++++++++++++++++++++++++++++++++++++++++++++++++++++++++++++++++++++++++++++++++++++++++++++++++++++++

Source : \cite{TrenchRealAnalisys}.

Nous considérons un espace normé \( (\Omega,\| . \|)\). Nous disons qu'une suite de fonctions \( f_n\) \defe{converge}{convergence!en norme} vers \( f\) pour la norme \( \| . \|\) si \( \forall \epsilon>0\), \( \exists N\) tel que \( n\geq N\) implique \( \| f_n-f \|<\epsilon\).

Dans le cas particulier de la norme 
\begin{equation}
    \| f \|_{\infty}=\sup_{x\in\Omega}| f(x) |,
\end{equation}
nous parlons que \defe{convergence uniforme}{convergence!uniforme}.

\begin{theorem}[Critère de Cauchy]  \label{ThoCauchyZelUF}
    Une suite de fonctions  \( (f_n)_{n\in\eN}\) sur \( \Omega\) converge en norme sur \( \Omega\) si et seulement si \( \forall\epsilon>0\), \( \exists N\) tel que
    \begin{equation}
        \| f_n-f_m \|<\epsilon
    \end{equation}
    pour \( n,m>N\).
\end{theorem}

\begin{corollary}       \label{CorCauchyCkXnvY}
    La série \( \sum f_n\) converge en norme sur \( \Omega\) si et seulement si \( \exists N\) tel que
    \begin{equation}
        \| f_n+\ldots+f_m \|\leq \epsilon
    \end{equation}
    pour tout \( n,m>N\).
\end{corollary}

\begin{proof}
    L'hypothèse montre que la suite des sommes partielles de la série \( \sum f_n\) vérifie le critère de Cauchy du théorème \ref{ThoCauchyZelUF}.
\end{proof}

\begin{definition}
    Nous disons qu'un sous ensemble \( A\) de \( \Omega\) est \defe{complet}{complet} si toute suite de Cauchy d'éléments de \( A\) converge vers un élément de \( A\).
\end{definition}

\begin{theorem}[Weierstrass]
    La série de fonctions \( \sum_{n=1}^{\infty}f_n\) converge en norme si \( \| f_n \|\leq M_n\) avec \( \sum_nM_n<\infty\).
\end{theorem}

\begin{proof}
    Si \( (S_n)_{n\in\eN}\) est la suite des sommes partielles de \( \sum M_n\), alors le critère de Cauchy pour la convergence de suites numériques dit que si \( m\) et \( n\) sont assez grands, \( S_m-S_{n-1}<\epsilon\), c'est à dire
    \begin{equation}
        M_n+\ldots+M_n<\epsilon.
    \end{equation}
    Par conséquent nous avons
    \begin{equation}
        \| f_n+\ldots+f_m \|\leq\| f_n \|+\ldots\| f_m \|\leq \epsilon.
    \end{equation}
    Le corollaire \ref{CorCauchyCkXnvY} montre alors la convergence de la série.
\end{proof}

%+++++++++++++++++++++++++++++++++++++++++++++++++++++++++++++++++++++++++++++++++++++++++++++++++++++++++++++++++++++++++++
\section{Théorème de Fubini}
%+++++++++++++++++++++++++++++++++++++++++++++++++++++++++++++++++++++++++++++++++++++++++++++++++++++++++++++++++++++++++++

\begin{theorem}[Fubini-Tonelli]\index{Funibi-Tonelli}\index{théorème!Fubini-Tonelli}
    Soient \( (\Omega_i,\tribA_i,\mu_i)\) des espaces mesurés $\sigma$-finis. Si
    \begin{equation}
        f\colon \Omega_1\times\Omega_2\to \mathopen[ 0 , \infty \mathclose]
    \end{equation}
    est mesurable pour \( \mu_1\otimes \mu_2\), alors
    \begin{enumerate}
        \item
            Les fonctions
            \begin{subequations}
                \begin{align}
                    x&\mapsto\int_{\Omega_2}f(x,y)d\mu_2(y)\\
                    y&\mapsto\int_{\Omega_2}f(x,y)d\mu_1(x)
                \end{align}
            \end{subequations}
            sont mesurables
        \item
            Nous avons la formule pratique
            \begin{equation}
                \int_{\Omega_1\times\Omega_2}fd(\mu_1\otimes\mu_2)=\int_{\Omega_1}\left( \int_{\Omega_2}f(x,y)d\mu_2(y) \right)d\mu_1(x)
                =\int_{\Omega_2}\left( \int_{\Omega_1}f(x,y)d\mu_1(x) \right)d\mu_2(y).
            \end{equation}
    \end{enumerate}
\end{theorem}

\begin{theorem}
    Soient \( \mu_i\) des mesures \( \sigma\)-finies sur les espaces mesurables \( (\Omega_i,\tribA_i)\) ($i=1,2$). Nous considérons une fonction \( f\colon \Omega_1\otimes\Omega_2\to \eR,\eC\) qui soit mesurable pour la tribu \( \tribA_1\otimes \tribA_2\) et intégrable pour la mesure \( \mu_1\otimes \mu_2\). Alors
    \begin{enumerate}
        \item
            La fonction \( x\mapsto f(x,y)\) est \( \mu_1\)-intégrable pour presque tout \( y\) (par rapport à \( \mu_2\)).
        \item
            La fonction
            \begin{equation}
                y\mapsto\int_{\Omega_1}f(x,y)d\mu_1(x)
            \end{equation}
            est \( \mu_2\)-intégrable. 
        \item
            Nous avons la formule de Fubini
            \begin{equation}
                \int_{\Omega_1\times\Omega_2}fd(\mu_1\otimes\mu_2)=\int_{\Omega_2}\left( \int_{\Omega_1}f(x,y)d\mu_1(x)\right)d\mu_2(y).
            \end{equation}
    \end{enumerate}
\end{theorem}

\begin{example}
    Nous montrons que le théorème ne tient pas si une des deux mesures n'est pas \( \sigma\)-finie. Soit \( I=\mathopen[ 0 , 1 \mathclose]\). Nous considérons l'espace mesuré
    \begin{equation}
        (I,\Borelien(I),\lambda)
    \end{equation}
    où \( \Borelien(I)\) est la tribu des boréliens sur \( I\) et \( \lambda\) est la mesure de Lebesgue (qui est $\sigma$-finie). D'autre part nous considérons l'espace mesuré
    \begin{equation}
        (I,\partP(I),m)
    \end{equation}
    où \( \partP(I)\) est l'ensemble des parties de \( I\) et \( m\) est la mesure de comptage. Cette dernière n'est pas $\sigma$-finie parce que les seuls ensembles de mesure finie pour la mesure de comptage sont des ensembles finis, or une union dénombrable d'ensemble finis ne peut pas recouvrir l'intervalle \( I\).

    Nous allons montrer que dans ce cadre, l'intégrale de la fonction indicatrice de la diagonale sur \( I^2\) ne vérifie pas le théorème de Fubini. Étant donné que \( \Borelien(I)\subset\partP(I)\) nous avons
    \begin{equation}
        \Borelien(I^2)\subset\Borelien(I)\otimes\partP(I).
    \end{equation}
    Soit \( \Delta=\{ (x,x)\tq x\in I \}\). La fonction
    \begin{equation}
        \begin{aligned}
            g\colon I^2&\to \eR \\
            (x,y)&\mapsto x-y 
        \end{aligned}
    \end{equation}
    est continue et \( \Delta=g^{-1}(\{ 0 \})\) est donc fermé dans \( I^2\). L'ensemble \( \Delta\) est donc un borélien de \( I^2\) et par conséquent un élément de la tribu \( \Borelien(I)\otimes\partP(I)\). La fonction indicatrice \( \mtu_{\Delta}\) est alors mesurable pour l'espace mesuré
    \begin{equation}
        (I\times I,\Borelien(I)\otimes\partP(I),\lambda\otimes m).
    \end{equation}
    Pour \( x\) fixé nous avons
    \begin{equation}
        \mtu_{\Delta}(x,y)=\begin{cases}
            1    &   \text{si \( y= x\)}\\
            1    &    \text{si \( y\neq x\)}
        \end{cases}=\mtu_{\{ x \}}(y),
    \end{equation}
    et donc
    \begin{subequations}
        \begin{align}
            A_1&=\int_I\left( \int_I\mtu_{\Delta}(x,y)dm(y) \right)d\lambda(x)\\
            &=\int_I\left( \int_I\mtu_{\{ x \}}(y)dm(y) \right)d\lambda(x)\\
            &=\int_I\Big( m(\{ x \}) \Big)d\lambda(x)\\
            &=\int_I 1d\lambda(x)\\
            &=1.
        \end{align}
    \end{subequations}
    Par contre le support de \( \mtu_{\Delta}\) étant de mesure nulle pour la mesure de Lebesgue, nous avons
    \begin{equation}
        \int_I\mtu_{\Delta}(x,y)d\lambda(x)=0
    \end{equation}
    et par conséquent
    \begin{equation}
        A_2=\int_I\left( \int_I\mtu_{\Delta}(x,y)d\lambda(x) \right)dm(y)=0.
    \end{equation}
    Nous voyons donc que le théorème de Fubini ne s'applique pas.
\end{example}

\begin{example}
    Le théorème de Fubini est utilisé dans le calcul de l'intégrale gaussienne
    \begin{equation}
        G=\int_{\eR} e^{-x^2}dx.
    \end{equation}
    Par symétrie nous pouvons nous contenter de calculer
    \begin{equation}
        G_+=\int_0^{\infty} e^{-x^2}dx.
    \end{equation}
    L'astuce est de passer par l'intermédiaire
    \begin{subequations}
        \begin{align}
            H&=\int_{\eR^+\times\eR^+} e^{-(x^2+y^2)}dxdy       \label{EqIntFausasub}\\
            &=\int_{\eR^+}\left( \int_{\eR^+} e^{-x^2} e^{-y^2}dx \right)dy\\
            &=\left( \int_I e^{-x^2} dx\right)^2\\
            &=G_+^2
        \end{align}
    \end{subequations}
    L'intégrale \eqref{EqIntFausasub} se calcule en passant aux coordonnées polaires et le résultat est \( H=\frac{ \pi }{ 4 }\). Nous avons alors \( G=\frac{ \sqrt{\pi} }{ 2 }\) et
    \begin{equation}
        \int_{\eR} e^{-x^2}=\sqrt{\pi}.
    \end{equation}
\end{example}

\begin{example} \label{ExempInversSumIntFub}
    Le théorème de Fubini-Tonelli nous permet également d'inverser des sommes et des séries. En effet une somme n'est rien d'autre qu'une intégrale pour la mesure de comptage :
    \begin{equation}
        \sum_{n=0}^{\infty}a_n=\int_{\eN}a_ndm(n).
    \end{equation}
    Considérons une suite de fonctions \( f_n\colon \eR^d\to \eR\) \emph{positives}, la quantité
    \begin{equation}    \label{EqAcalculParFubIntSum}
        I=\sum_{n=0}^{\infty}\int_{\eR^n}f_n(x)dx
    \end{equation}
    et les espaces mesurés \( (\eN,\partP(\eN),m)\), \( (\eR^n,\Borelien(\eR^n),\lambda)\) où \( \lambda\) est la mesure de Lebesgue. En écrivant la formule \eqref{EqAcalculParFubIntSum}, nous supposons que pour chaque \( n\), la fonction \( f_n\) est intégrable sur \( \eR^d\) et que le résultat soit sommable. Nous pouvons la récrire sous la forme
    \begin{equation}
        \int_{\eN}\left( \int_{\eR^n}f(n,x)dx \right)dm(n)
    \end{equation}
    avec la notation évidente \( f(n,x)=f_n(x)\). Prouvons que la fonction \( f\colon \eN\times\eR^d\to \eR\) ainsi définie est une fonction mesurable pour l'espace mesuré
    \begin{equation}
        \big( \eN\times\eR^d,\partP(\eN)\otimes\Borelien(\eR^d),m\otimes\lambda \big).
    \end{equation}
    Si \( A\subset\eR\), nous avons
    \begin{equation}
        f^{-1}(A)=\bigcup_{n\in\eN}\{ n \}\times f_n^{-1}(A).
    \end{equation}
    Chacun des ensembles dans l'union appartient à la tribu \( \partP(\eN)\times\Borelien(\eR^d)\) tandis que les tribus sont stables sous les unions dénombrables. La fonction \( f\) est donc mesurable. La fonction \( f\) est donc mesurable. Comme nous avons supposé que \( f\) était positive, le théorème de Fubini-Tonelli s'applique et nous avons
    \begin{equation}
        I=\int_{\eR^d}\left( \int_{\eN}f(n,x)dm(n) \right)dx=\int_{\eR^d}\sum_{n\in \eN}f_n(x)dx.
    \end{equation}
\end{example}

%+++++++++++++++++++++++++++++++++++++++++++++++++++++++++++++++++++++++++++++++++++++++++++++++++++++++++++++++++++++++++++
\section{L'espace \texorpdfstring{$L^2$}{$L^2$}}
%+++++++++++++++++++++++++++++++++++++++++++++++++++++++++++++++++++++++++++++++++++++++++++++++++++++++++++++++++++++++++++

Soit \( (\Omega,\tribA,\mu)\) un espace mesuré. Nous considérons l'opération
\begin{equation}    \label{DefProdScalLubrgTj}
    \langle f, g\rangle =\int_{\Omega}f(\omega)g(\omega)d\mu(\omega)
\end{equation}
et la norme associée
\begin{equation}
    \| f \|_2=\sqrt{\langle f, f\rangle }.
\end{equation}
Nous considérons l'ensemble
\begin{equation}
    \mL^2(\Omega,\mu)=\{ f\colon \Omega\to \eR\tq \| f \|_2<\infty \}
\end{equation}
et la relation d'équivalence \( f\sim g\) si et seulement si \( f(x)=g(x)\) pour \( \mu\)-presque tout \( x\). L'espace que nous considérons est
\begin{equation}
    L^2=\mL^2/\sim.
\end{equation}

\begin{lemma}
    La formule \eqref{DefProdScalLubrgTj} définit un produit scalaire sur \( L^2\).
\end{lemma}

\begin{proof}
    Nous devons d'abord montrer que la formule passe au quotient. Pour cela, nous considérons des fonctions \( \alpha\) et \( \beta\) nulles presque partout et nous regardons le produit de \( f_1=f+\alpha\) par \( g_1=g+\beta\) :
    \begin{equation}
        \langle f_1, g_1\rangle =\int fg+\beta f+\alpha g+ \alpha\beta.
    \end{equation}
    Les fonction \( \beta f\), \( \alpha g\) et \( \alpha\beta\) étant nulles presque partout, leur intégrale est nulle et nous avons bien \( \langle f_1, g_1\rangle =\langle f,g \rangle \). Nous pouvons donc considérer le produit sur l'ensemble des classes.

    Pour vérifier que la formule est un produit scalaire, le seul point non évidement est de prouver que \( \langle f, f\rangle =0\) implique \( f=0\). Cela découle du fait que
    \begin{equation}
        \langle f, f\rangle =\int_{\Omega}| f |^2.
    \end{equation}
    La fonction \( x\mapsto | f(x) |^2\) vérifie les hypothèses du lemme \ref{Lemfobnwt}. Par conséquent \( | f(x) |^2\) est presque partout nulle.
\end{proof}

Nous montrons maintenant que \( L^2\) est un espace vectoriel complet. Soit \( (f_n)\) une suite ce Cauchy dans \( (L^2,\| . \|_2)\). Pour tout \( n\in\eN\), il existe \( N_n\in\eN\) tel que si \( p,q\geq N_n\) alors
\begin{equation}
    \| f_p-f_q \|<2^{-n}.
\end{equation}
Nous avons 
\begin{equation}
    \| \sum_{n=1}^{\infty} f_{N_n+1}-f_{N_n} \|\leq\sum_{n}\| f_{N_n+1}-f_{N_n} \|<\infty
\end{equation}
parce que la série des \( 2^{-n}\) converge.
