% This is part of Analyse Starter CTU
% Copyright (c) 2014
%   Laurent Claessens,Carlotta Donadello
% See the file fdl-1.3.txt for copying conditions.


%+++++++++++++++++++++++++++++++++++++++++++++++++++++++++++++++++++++++++++++++++++++++++++++++++++++++++++++++++++++++++++ 
\section{Équations différentielles linéaires du second ordre}
%+++++++++++++++++++++++++++++++++++++++++++++++++++++++++++++++++++++++++++++++++++++++++++++++++++++++++++++++++++++++++++

\begin{definition}[Équation différentielle linéaire du second ordre]
Une  \defe{équation différentielle linéaire du second ordre}{équation différentielle!linéaire du second ordre} est une équation différentielle de la forme 
\begin{equation}\label{eq_lin_ordre_deux}
  a(x)y'' + b(x) y' + c(x)y = d(x), \quad\text{pour } x\in I, 
\end{equation}
o\`u $a$, $b$, $c$ et $d$ sont des fonction de $\eR$ dans $\eR$ et $a\neq 0$ pour tout $x\in I$ . 

On dit que $a$, $b$, $c$ et $d$ sont les coefficients de l'équation \eqref{eq_lin_ordre_deux}.
\end{definition}
Dans ce cours nous allons étudier exclusivement le cas où $a$, $b$ et $c$ sont des fonctions constantes. 
\begin{definition}[Équation différentielle linéaire du second ordre homogène]
Une  \defe{équation différentielle linéaire du second ordre homogène}{équation différentielle!linéaire du second ordre, homogène} est une équation différentielle de la forme \eqref{eq_lin_ordre_deux}, telle que le coefficient $d$ est nul.
\end{definition}
À toute équation de la forme \eqref{eq_lin_ordre_deux} on peut associer une équation homogène exactement comme on a fait dans la section précedente pour les équations linéaires du premier ordre.
\subsection{Résolution des équations différentielles linéaires du second ordre homogènes à coefficients constants}
\begin{remark}
 L'application qui à la fonction $y$ fait correspondre $a(x)y'' + b(x) y' + c(x)y$ est linéaire, au sens de la remarque \ref{remarque_lineaire}. 

Cela nous dit en particulier, que si $y_1$ et $y_2$ sont deux solutions de l'équation homogène alors toute leur combinaison de la forme $z = \lambda y_1 + \mu y_2$, avec $\lambda$ et $\mu$ dans $\eR$, est encore une solution.
 \end{remark}

\begin{framed}
  Jusqu'ici nous avons toujours travaillé avec des fonctions définies sur $\eR$ et à valeurs dans $\eR$. Dans cette section nous nous authorisons à passer par des fonctions définies sur $\eR$ et à valeurs dans $\eC$, mais cela sera uniquement une étape dans nos calculs. Au final toutes les solutions que nous allons considérer sont des fonctions à valeurs dans $\eR$.  
\end{framed}

La solution générale \textbf{à valeurs dens les complexes} d'une équation de ce type a la forme 
\begin{equation}\label{sol_gen_ordre_deux_hom}
  \mathcal{Y}_h^\eC  = \left\{C_1 e^{r_1x} +C_2 e^{r_2x} \,:\, C_1,\, C_2 \in \eC, \: x\in I \right\},
\end{equation}
où $r_1$ et $r_2$ sont aussi des nombres complexes. Remarquez que la solution générale est une famille à deux paramètres. Il faut aussi observer que en tout cas l'intervalle $I$ dans lequel varie $x$ est un intervalle dans $\eR$, parce que $I$ est une des données du problème.  

À partir de cette information nous pouvons, pour toute équation donnée, chercher la solution générale \textbf{complexe} par substitution. Il suffit de remplacer $y$ dans l'équation par $e^{rx}$ et chercher les valeurs de $r$ qui nous conviennent. 

Si notre équation de départ est 
\begin{equation}\label{eq_lin_ordre_deux_hom}
  ay'' + by' + cy = 0, \quad\text{pour } x\in I, 
\end{equation}
alors la substitution nous donne
\[
e^{rx}\left(ar^2+br+c\right)=0.
\]
Il est connu que la fonction exponentielle ne prend pas la valeur $0$, par consequent ce qui s'annulle est le polin\^ome de degré deux $ar^2+br+c$. Il est donc très facile de trouver les valeurs de $r$ qu'on pourra utiliser comme $r_1$ et $r_2$ dans la solution générale \textbf{complexe}.
\begin{description}
  \item[Si $b^2 - 4ac >0$ :] le polin\^ome admet deux solutions réelles et distinctes, $r_1$ et $r_2$ ;
  \item[Si $b^2 - 4ac <0$ :] le polin\^ome admet deux solutions complexes conjuguées, $r_1 = \alpha + i \beta$ et $r_2 = \alpha - i \beta$ ;
  \item[Si $b^2 - 4ac =0$ :] le polin\^ome admet une solution réelle double $r=r_1 = r_2$.
\end{description}
Il faut maintenant écrire la solution générale \textbf{réelle} de l'équation, qui est celle que nous intéresse vraiment. La façon de l'obtenir est différente dans les trois cas.
\begin{description}
  \item[Si $b^2 - 4ac >0$ :] la solution générale réelle a la m\^eme forme que la solution complexe, \eqref{sol_gen_ordre_deux_hom}, il suffit de prendre les paramètres $C_1$ et $C_2$ dans $\eR$ plut\^ot que dans $\eC$. 
\begin{equation}\label{sol_gen_reelle_ordre_deux_hom}
  \mathcal{Y}_h  = \left\{C_1 e^{r_1x} +C_2 e^{r_2x} \,:\, C_1,\, C_2 \in \eR, \: x\in I\right\},
\end{equation}
  \item[Si $b^2 - 4ac <0$ :] le polin\^ome admet deux solutions complexes conjuguées, $r_1 = \alpha + i \beta$ et $r_2 = \alpha - i \beta$ ; Il faut alors utiliser les formules suivantes 
    \begin{equation}
      \begin{array}{l}
        e^{\alpha + i \beta} =e^{\alpha}(\cos(\beta) + i \sin(\beta))\\
        e^{\alpha - i \beta} =e^{\alpha}(\cos(\beta) - i \sin(\beta)).
      \end{array}
    \end{equation}
    La somme $e^{r_1x} +e^{r_2x}$, où $x$ est dans $I\in\eR$, vaut 
    \[
    e^{(\alpha + i \beta)x} + e^{(\alpha - i \beta)x}=e^{\alpha x}(\cos(\beta x) + i \sin(\beta x )) + e^{\alpha x}(\cos(\beta x) - i \sin(\beta x)) =2 e^{\alpha x}\cos(\beta x)
    \]
    et la différence $e^{r_1x} -e^{r_2x}$ vaut
    \[
    e^{(\alpha + i \beta)x} - e^{(\alpha - i \beta)x}=e^{\alpha x}(\cos(\beta x) + i \sin(\beta x )) - e^{\alpha x}(\cos(\beta x) - i \sin(\beta x)) =2 e^{\alpha x}\sin(\beta x).
    \]
    Par ces deux calculs élémentaires nous avons trouvé deux fonctions à valeurs dans $\eR$ qui n'ont pas de zéros en commun. Elles sont les génératrices de la famille des solutions réelles de l'équation différentielle (la solution générale)
    \begin{equation}\label{sol_gen_reelle_ordre_deux_hom_complconj}
      \mathcal{Y}_h  = \left\{ e^{\alpha x}\left(C_1\cos(\beta x) +C_2\sin(\beta x)\right)  \,:\, C_1,\, C_2 \in \eR, \: x\in I\right\},
    \end{equation}
  \item[Si $b^2 - 4ac =0$ :] le polin\^ome admet une solution réelle double $r=r_1 = r_2$. Dans ce cas la solution générale de l'équation est la famille
    \begin{equation}\label{sol_gen_reelle_ordre_deux_hom_doublerac}
      \mathcal{Y}_h  = \left\{(C_1  +C_2x) e^{r x} \,:\, C_1,\, C_2 \in \eR, \: x\in I\right\}.
    \end{equation} 
    Pour justifier cette formule nous observons d'abord que toute fonction $x\mapsto Ce^{rx}$, pour $C\in\eR$ est une solution de l'équation différentielle (par construction). Ensuite nous utilisons la méthode de variation de la constante. On trouve rapidement que si une fonction de la forme $x\mapsto C(x)e^{rx}$ est une solution alors $C(x)$ est un polyn\^onme de degré au plus 1, c'est à dire $C(x) = C_1 + C_2 x$ avec $C_1$ et $C_2$ dans $\eR$. 
\end{description}

\subsection{Équations différentielles linéaires du second ordre à coéfficients constants, non homogènes}

Nous ne présentons pas une méthode générale pour la résolution de ces équations. Comme dans le cas des équations différentielles linéaires du premier ordre non homogènes, la solution générale de \eqref{eq_lin_ordre_deux} est donnée par la somme d'une solution particulière et de la solution générale de l'équation homogène associée. La recherche d'une solution particulière est facilité par le fait que les coefficients de \eqref{eq_lin_ordre_deux} sont supposés constants, c'est à dire que $a$, $b$ et $c$ sont des fonctions constantes. Il faut essayer de déviner la forme d'une solution particulière à partir de la forme du second membre de l'équation, la fonction $d$. Si $d$ est un polyn\^ome  il faut essayer avec un polyn\^ome du m\^eme degré, si $d$ est un'exponentielle, par exemple $d(x) = e^{5x}$, on pourra essayer avec un multiple de la m\^eme fonction exponentielle, dans l'exemple $f(x) = k e^{5x}$, avec $k$ à determiner. Si $d$ est une combinaison linéaire de sinus et cosinus, comme par exemple $12\cos(x) + 2\sin(x)$, on peut essayer avec $k_1\cos(x) + k_2\sin(x)$. 

\begin{example}
  On considère l'équation différentielle 
  \begin{equation}\label{exemple_non_hom}
    y'' + 12y' + 36 y = -192 e^{2x}, \quad x\in\eR.
  \end{equation}
  Son équation homogène associée est 
\begin{equation}\label{exemple_hom_ass}
    y'' + 12y' + 36 y = 0,
  \end{equation}
dont le polyn\^ome characteristique est $r^2 + 12 r + 36$. Ce polyn\^ome admet une racine double, qui est $-6$, par conséquent la solution générale de \eqref{exemple_hom_ass} est 
\begin{equation*}
      \mathcal{Y}_h  = \left\{(C_1  +C_2x) e^{-6 x} \,:\, C_1,\, C_2 \in \eR, \: x\in \eR\right\}.
    \end{equation*} 
Le membre de droite de \eqref{exemple_non_hom} est une fonction exponentielle, nous allons donc chercher une solution particulière de \eqref{exemple_non_hom} de la forme $f(x) = ke^{2x}$. Par substitution nous trouvons
\[
  ke^{2x}(4 + 12 \times 2 +36) = -192 e^{2x},   
\]
ce qui veut dire que $k$ doit \^etre $-3$. 

La solution générale de l'équation \eqref{exemple_non_hom} est donc 
\begin{equation*}
      \mathcal{Y}  = \left\{(C_1  +C_2x) e^{-6 x} -3e^{2x} \,:\, C_1,\, C_2 \in \eR, \: x\in \eR\right\}.
    \end{equation*} 
\end{example}

\begin{example}
  Nous allons résoudre l'équation 
  \begin{equation}
    y'' + 12y' + 36 y = 12\cos(x) + 2\sin(x), \quad x\in\eR. 
  \end{equation}

Cette équation a comme homogène associée l'équation \eqref{exemple_hom_ass}, comme dans l'exemple précedent. Il nous suffit donc de trouver une solution particulière de \eqref{exemple_non_hom}.

Nous pouvons essayer avec $f(x)= k_1\cos(x) + k_2\sin(x)$. Par substitution on trouve
\begin{equation*}
  \begin{aligned}
    -\left(k_1\cos(x) + k_2\sin(x)\right) & +12 \left(-k_1\sin(x) + k_2\cos(x)\right) + 36\left(k_1\cos(x) + k_2\sin(x)\right)\\
    &= 12\cos(x) + 2\sin(x) 
  \end{aligned}
\end{equation*}

Cette équation doit \^etre satisfaite pour tout valeur de $x$, en particulier pour $x= 0$ et $x = \pi/2$. Cela revient à considère séparemment les coefficients des fonctions sinus et cosinus. Il faut alors que $k_1$ et $k_2$ soient solutions du système 
\begin{equation*}
  \begin{cases}
    -k_1 + 12 k_2 + 36 k_1& = 12, \\
    -k_2 - 12 k_1 + 36 k_2& = 2.
  \end{cases}
\end{equation*}
On trouve $k_1= 396/1369$ et $k_2 = 214/1369$, et la solution générale de notre équation est 
\begin{equation*}
   \mathcal{Y}  = \left\{(C_1  +C_2x) e^{-6 x} +\frac{396}{1369}\cos(x) + \frac{214}{1369}\sin(x) \,:\, C_1,\, C_2 \in \eR, \: x\in \eR\right\}.
\end{equation*}
\end{example}

\begin{example}
   Nous allons résoudre l'équation 
  \begin{equation}
    y'' + 12y' + 36 y = 10x^2+3, \quad x\in\eR. 
  \end{equation}

Cette équation a comme homogène associée l'équation \eqref{exemple_hom_ass}, comme dans l'exemple précedent. Il nous suffit donc de trouver une solution particulière de \eqref{exemple_non_hom}. 

Nous pouvons essayer avec $f(x)= k_1x^2+ k_2x + k_3$. Par substitution on trouve
\begin{equation*}
    \left(2k_1\right)  +12 \left(2k_1x+ k_2\right) + 36\left(k_1x^2+ k_2x + k_3\right)=  10x^2+3. 
\end{equation*}

Pour trouver les bonnes valeurs des coefficients nous devons résoudre le système \begin{equation*}
  \begin{cases}
    36 k_1& = 10, \\
    24k_1 + 36 k_2& = 0,\\
    2k_1 + 12 k_2 + 36 k_3& = 3,
  \end{cases}
\end{equation*}
ce qui donne $k_1= 5/18$, $k_2 = -5/27$ et $k_3 = 7/54$. La solution générale de notre équation est 
\begin{equation*}
   \mathcal{Y}  = \left\{(C_1  +C_2x) e^{-6 x} +\frac{5}{18}x^2 - \frac{5}{27}x + \frac{7}{54} \,:\, C_1,\, C_2 \in \eR, \: x\in \eR\right\}.
\end{equation*}
\end{example}
