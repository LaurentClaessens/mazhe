% This is part of Mes notes de mathématique
% Copyright (c) 2012-2013
%   Laurent Claessens, Carlotta Donadello
% See the file fdl-1.3.txt for copying conditions.

%+++++++++++++++++++++++++++++++++++++++++++++++++++++++++++++++++++++++++++++++++++++++++++++++++++++++++++++++++++++++++++
					\section{Topologie en général}
%+++++++++++++++++++++++++++++++++++++++++++++++++++++++++++++++++++++++++++++++++++++++++++++++++++++++++++++++++++++++++++

\begin{definition}		\label{DefTopologieGene}
Soit $E$, un ensemble et $\mT$, une partie de l'ensemble de ses parties qui vérifie les propriétés suivantes
\begin{enumerate}

\item
les ensembles $\emptyset$ et $E$ sont dans $\mT$,

\item
Si $I$ est n'importe quel ensemble et si pour tout $i\in I$, nous avons un élément $\mO_i\in\mT$, alors $\cup_{i\in I}\mO_i\in\mT$,

\item
Si $J$ est un ensemble fini et si pour tout $j\in J$, nous avons un élément $\mO_j\in\mT$, alors $\cap_{j\in J}\mO_j\in\mT$.

\end{enumerate}
Les deux dernières propriétés s'énoncent en disant que toute réunions d'éléments de $\mT$ est un élément de $\mT$ et que toute intersection \emph{finie} d'éléments de $\mT$ est un élément de $\mT$.

Un tel choix $\mT$ de sous-ensembles de $E$ est une  \defe{\href{http://fr.wikipedia.org/wiki/Espace_topologique}{topologie}}{topologie} sur $E$, et les éléments de $\mT$ sont appelés des \defe{ouverts}{ouvert}. Nous disons que un sous ensemble $A$ de $E$ est \defe{fermé}{fermé} si son complémentaire, $A^c$ est ouvert.
\end{definition}

\begin{definition}
    Un \defe{homéomorphisme}{homéomorphisme} est une application bijective continue entre deux espaces topologiques dont la réciproque est continue. Deux espaces topologiques homéomorphes sont dits \defe{isomorphes}{isomorphisme!d'espaces topologiques}.
\end{definition}

Dès que nous avons une topologie, nous avons une notion de convergence de suite : nous disons qu'une suite $x_n$ d'éléments de $E$ \defe{converge}{convergence!en topologie} vers l'élément $x$ de $E$ si pour tout ouvert $\mO$ contenant $x$, il existe un $K$ tel que $k>K$ implique $x_k\in\mO$. Cette définition est exactement celle donnée pour la convergence de suites dans $\eR^n$, à part que nous avons remplacé le mot \og boule\fg{} par \og ouvert\fg.

Dans un espace topologique, nous avons une caractérisation très importante des ouverts.
\begin{theorem}		\label{ThoPartieOUvpartouv}
Une partie $A$ de $E$ est ouverte si et seulement si pour tout $x\in A$, il existe un ouvert autour de $x$ contenu dans $A$.
\end{theorem}

\begin{proof}
Le sens direct est évident : $A$ lui-même est un ouvert autour de $x\in A$, qui est inclus à $A$.

Pour le sens inverse, pour chaque $x\in A$, nous considérons l'ensemble $\mO_x\subset A$, un ouvert autour de $x$. Nous avons que
\begin{equation}	\label{EqAUniondesOx}
	A=\bigcup_{x\in A}\mO_x.
\end{equation}
En effet $A\subset\cup_{x\in A}\mO_x$ parce que tous les éléments de $A$ sont dans un des $\mO_x$, par construction. D'autre part, $\cup_{x\in A}\mO_x\subset A$ parce que chacun des $\mO_x$ est compris dans $A$.

L'union du membre de droite de \eqref{EqAUniondesOx} est une union d'ouverts et est donc un ouvert. Cela prouve que $A$ est un ouvert.

\end{proof}

Une utilisation typique de ce théorème est faite à l'exercice \ref{exo0083}.

%+++++++++++++++++++++++++++++++++++++++++++++++++++++++++++++++++++++++++++++++++++++++++++++++++++++++++++++++++++++++++++
\section{Topologie induite}
%+++++++++++++++++++++++++++++++++++++++++++++++++++++++++++++++++++++++++++++++++++++++++++++++++++++++++++++++++++++++++++

Soit \( X\) un espace topologique et \( A\subset X\). L'ensemble \( A\) devient un espace topologique en lui-même par la \defe{topologie induite}{topologie!induite} de \( X\). Un ouvert de \( A\) est un ensemble de la forme \( A\cap\mO\) où \( \mO\) est un ouvert de \( X\).

\begin{lemma}       \label{LemkUYkQt}
    Si \( B\subset A\) alors la fermeture de \( B\) pour la topologie de \( A\) (induite de \( X\)) que nous noterons \( \tilde B\) est 
    \begin{equation}
        \tilde B=\bar B\cap A
    \end{equation}
    où \( \bar B\) est la fermeture de \( B\) pour la topologie de \( X\).
\end{lemma}

\begin{proof}
    Si \( a\in \bar B\cap A\), un ouvert de \( A\) autour de \( a\) est un ensemble de la forme \( \mO\cap A\) où \( \mO\) est un ouvert de \( X\). Vu que \( a\in\bar B\), l'ensemble \( \mO\) intersecte \( B\) et donc \( (\mO\cap A)\cap B\neq \emptyset\). Donc \( a\) est bien dans l'adhérence de \( B\) au sens de la topologie de \( A\).

    Pour l'inclusion inverse, soit \( a\in \tilde  B\), et montrons que \( a\in \bar B\cap A\). Par définition \( a\in A\), parce que \( \tilde B\) est une fermeture dans l'espace topologique \( A\). Il faut donc seulement montrer que \( a\in\bar B\). Soit donc \( \mO\) un ouvert de \( X\) contenant \( a\); par hypothèse \( \mO\cap A\) intersecte \( B\) (parce que tout ouvert de \( A\) contenant \( a\) intersecte \( B\)). Donc \( \mO\) intersecte \( B\). Cela signifie que tout ouvert (de \( X\)) contenant \( a\) intersecte \( B\), ou encore que \( a\in \bar B\).
\end{proof}

\begin{example} \label{ExloeyoR}
    Si \( A\) est un ouvert de \( X\), on pourrait croire que la topologie induite n'a rien de spécial. Il est vrai que \( B\) sera ouvert dans \( A\) si et seulement si il est ouvert dans \( X\), mais des choses se passent quand même. Prenons \( X=\eR\) et \( A=\mathopen] 0 , 1 \mathclose[\). Si \( B=\mathopen] \frac{ 1 }{2} , 1 \mathclose[ \), alors la fermeture de \( B\) dans \( A\) sera \( \tilde B=\mathopen[ \frac{ 1 }{2} , 1 [\) et non \( \mathopen[ \frac{ 1 }{2} , 1 \mathclose]\) comme on l'aurait dans \( \eR\).
\end{example}

%+++++++++++++++++++++++++++++++++++++++++++++++++++++++++++++++++++++++++++++++++++++++++++++++++++++++++++++++++++++++++++ 
\section{Compacts}
%+++++++++++++++++++++++++++++++++++++++++++++++++++++++++++++++++++++++++++++++++++++++++++++++++++++++++++++++++++++++++++

\begin{definition}
  Une partie $A$ d'un espace topologique est \defe{compacte}{compact} si il vérifie la propriété de Borel-Lebesgue : pour tout recouvrement de $A$ par des ouverts (c'est-à-dire une collection d'ouverts dont la réunion contient $A$) on peut tirer un recouvrement fini.
\end{definition}
\begin{remark}
    Certaines sources (dont \wikipedia{fr}{Compacité_(mathématiques)}{wikipédia}) disent que pour être compact il faut aussi être \defe{séparable}{séparable} (c'est à dire que deux points distincts ont des voisinages distincts). Pour ces sources, un espace qui ne vérifie que la propriété de Borel-Lebesgue est alors dit \defe{quasi-compact}{quasi-compact}\index{compact!quasi}.
\end{remark}

\begin{theorem}[Tykhonov]\index{théorème!Tykhonov}\label{ThoFWXsQOZ}
    Un produit quelconque d'espaces métriques non vides est compact si et seulement si chacun de ses facteurs est compact.
\end{theorem}
Nous n'allons donner la preuve que dans le cas d'un produit dénombrable, dans le théorème \ref{ThoCDhbZbf}.


\subsection{Connexité}
%----------------------

Dès qu'un ensemble est muni d'une métrique, nous pouvons définir les boules ouvertes, les voisinages et les sous-ensembles ouverts. Dès que l'on a identifié les sous-ensemble ouverts de $E$, nous disons que $E$ devient un \defe{espace topologique}{espace topologique}. Nous allons maintenant un pas plus loin.

\begin{definition}
 Lorsque $E$ est un espace topologique, nous disons qu'un sous-ensemble $A$ est \defe{non connexe}{connexe} quand on peut trouver des ouverts $O_1$ et $O_2$ tels que
\begin{equation}    \label{EqDefnnCon}
  A=(A\cap O_1)\cup (A\cap O_2),
\end{equation}
et tels que $A\cap O_1\neq\emptyset$, et $A\cap O_2\neq\emptyset$.
Si un sous-ensemble n'est pas non-connexe, alors on dit qu'il est connexe.
\end{definition}
Une autre façon d'exprimer la condition \eqref{EqDefnnCon} est de dire que $A$ n'est pas connexe quand il est contenu dans la réunion de deux ouverts disjoints qui intersectent tous les deux $A$.

\begin{proposition}\label{PropGWMVzqb}
    L'image d'un ensemble connexe par une fonction continue est connexe.
\end{proposition}

\begin{proof}
    Soit \( f\colon X\to Y\) une application continue entre deux espaces topologiques, et \( E\) une partie connexe de \( X\). Nous devons montrer que \( f(E)\) est connexe dans \( Y\).

    Par l'absurde nous considérons \( A\) et \( B\), deux ouverts de \( Y\) disjoints recouvrant \( f(E)\). Étant donné que \( f\) est continue, les ensembles \( f^{-1}(A)\) et \( f^{-1}(B)\) sont ouverts dans \( X\). De plus ces deux ensembles recouvrent \( E\).

    Si \( x\) est un élément de \( f^{-1}(A)\cap f^{-1}(B)\), alors \( f(x)\in A\cap B\), ce qui est impossible parce que nous avons supposé que \( A\) et \( B\) étaient disjoints. Par conséquent \( f^{-1}(A)\) et \( f^{-1}(B)\) sont deux ouverts disjoints recouvrant \( E\). Contradiction avec la connexité de \( E\). Nous concluons que \( f(E)\) est connexe.
\end{proof}

%+++++++++++++++++++++++++++++++++++++++++++++++++++++++++++++++++++++++++++++++++++++++++++++++++++++++++++++++++++++++++++ 
\section{Espaces métriques}
%+++++++++++++++++++++++++++++++++++++++++++++++++++++++++++++++++++++++++++++++++++++++++++++++++++++++++++++++++++++++++++

%--------------------------------------------------------------------------------------------------------------------------- 
\subsection{Compacité}
%---------------------------------------------------------------------------------------------------------------------------

<++>

%--------------------------------------------------------------------------------------------------------------------------- 
\subsection{Produit dénombrables d'espaces métriques}
%---------------------------------------------------------------------------------------------------------------------------

\begin{definition}
    Soient \( (E_n,d_n)\) des espaces métriques. Sur l'ensemble produit \( E=\bigprod_{i=1}^{\infty}E_i\) nous définissons la métrique
    \begin{equation}
        d(x,y)=\sum_{i=1}^{\infty}\frac{1}{ 2^i }d'_n(x_i,y_i)
    \end{equation}
    où \( d'_i=\min(d_i,1)\).
\end{definition}
On peut montrer que ce \( d\) est bien une distance et que \( (E,d)\) est alors bien un espace métrique.
%TODO: le faire.

\begin{theorem}\label{ThoCDhbZbf}
    Un produit dénombrable d'espaces métriques non vides est compact si et seulement si chacun de ses facteurs est compact.
\end{theorem}
Note : ce résultat est encore valable pour un produit quelconque, c'est le théorème de Tykhonov \ref{ThoFWXsQOZ}.
