\begin{corrige}{LimiteContinue0003}

	Il faut d'abord bien comprendre ce que signifie la notation
	\begin{equation}
		\lim_{y\to 0} \lim_{x\to 0} f(x,y).
	\end{equation}
	Lorsque nous écrivons $\lim_{x\to 0} f(x,y)$, nous considérons $f(x,y)$ comme fonction de la seule variable $x$; la variable $y$ est alors traitée comme un paramètre. Pour considérer $y$ comme un paramètre, il y a deux cas à traiter : $y=0$ et $y\neq 0$. Dans le cas $y=0$, la fonction $f$ se réduit à la fonction constante $0$. Nous devons donc considérer la fonction
	\begin{equation}
		g(x)=0
	\end{equation}
	dont le domaine est $x\neq 0$. Évidement, la limite de $g$ lorsque $x\to 0$ est zéro. Dans le ces $y\neq 0$ nous considérons la fonction
	\begin{equation}
		g(x)=\frac{ x^2y^2 }{ x^2y^2+(x-y)^2 },
	\end{equation}
	dont le domaine est $\eR$. Nous trouvons que $\lim_{x\to 0} g(x)=0$.

	Tout cela pour dire que $\lim_{x\to 0} f(x,y)=0$. Par conséquent, $\lim_{y\to 0} \lim_{x\to 0} f(x,y)=0$.

	De la même façon, nous avons $\lim_{y\to 0} f(x,y)=0$.

	Étudions maintenant la limite simultanée $\lim_{(x,y)\to(0,0)}f(x,y)$. Une bonne astuce pour regarder plusieurs chemins d'un coup est de considérer le chemin $\gamma(t)=(t,kt)$. Cela teste toute les droites (sauf une, laquelle ?) en un seul calcul. Nous avons
	\begin{equation}
		f(t,kt)=\frac{ k^2t^4 }{ k^2t^4+t^2(1-k)^2 }=\frac{ k^2t^2 }{ k^2t^2+(1-k)^2 }.
	\end{equation}
	La limite de cela lorsque $t\to 0$ vaut $\frac{ 0 }{ (1-k)^2 }=0$ pour tout $k$ sauf pour $k=1$. Le chemin qui correspond à $k=1$ apparaît donc spécial. Regardons le de plus près. Nous avons
	\begin{equation}
		f(t,t)=\frac{ t^4 }{ t^4 }=1,
	\end{equation}
	et donc la limite le long de ce chemin est $1$. Nous avons donc trouvé deux chemins de limites différentes, et nous concluons que $\lim_{(x,y)\to(0,0)}f(x,y)$ n'existe pas.
\end{corrige}

