\begin{corrige}{controlecontinu0003}

    \begin{enumerate}
        \item
            Les points où la continuité n'est pas assurée sont les points de la forme \( (0,a)\) avec \( a\in\eR\). Nous devons donc calculer, pour tout \( a\) la limite
            \begin{equation}
                \lim_{(x,y)\to(0,a)}\frac{ y\sin^2(2x) }{ x }
            \end{equation}
            Nous pouvons calculer séparément les deux limites
            \begin{subequations}
                \begin{align}
                    \lim_{(x,y)\to(0,a)}y&=0\\
                    \lim_{(x,y)\to(0,a)}\frac{ \sin^2(2x) }{ x }&=0.
                \end{align}
            \end{subequations}
            Étant donné que les deux limites existent, la limite du produit est égale au produit des limites (proposition \ref{PropOpsSimplesLimites}). Notre limite vaut donc zéro et la fonction est continue (parce que la limite vers \( (0,a)\)) est égale à la valeur en \( (0,a)\).

            \item
                Le calcul montre que 
                \begin{subequations}        \label{SubeqsDerpartGAzztc}
                    \begin{align}
                        \frac{ \partial f }{ \partial x }&=\frac{ 4y\sin(2x)\cos(2x) }{ x }-y\frac{ \sin^2(2x) }{ x^2 }\\
                        \frac{ \partial f }{ \partial y }&=\frac{ \sin^2(2x) }{ x }.
                    \end{align}
                \end{subequations}
                En posant \( x=\frac{ \pi }{ 6 }\), \( y=4\) nous trouvons
                \begin{subequations}
                    \begin{align}
                        \frac{ \partial f }{ \partial x }(\frac{ \pi }{ 6 },4)&=\frac{ 24\sqrt{3} }{ \pi }-\frac{ 108 }{ \pi^2 }\\
                        \frac{ \partial f }{ \partial y }(\frac{ \pi }{ 6 },4)&=\frac{ 9 }{ 2\pi }.
                    \end{align}
                \end{subequations}
                
            \item
                Le point \( (0,2)\) n'étant pas dans le domaine sur lequel la fonction est «facile», nous devons utiliser la définition de la dérivée directionnelle :
                \begin{subequations}
                    \begin{align}
                        \frac{ \partial f }{ \partial x }(0,2)&=\lim_{t\to 0} \frac{ f(t,2)-f(0,2) }{ t }\\
                        &=\lim_{t\to 0} \frac{ 2\sin^2(2t) }{ t^2 }\\
                        &=2\lim_{t\to 0} \frac{ 2\sin(2t) }{ 2t }\frac{ 2\sin(2t) }{ 2t }\\
                        &=8.
                    \end{align}
                \end{subequations}
                Dans ce calcul, nous avons multiplié le numérateur et le dénominateur par \( 4\). La dérivée selon \( y\) est plus simple parce que \( f(0,2+t)=f(0,2)=0\), ce qui donne
                \begin{equation}        \label{EqAGzztczdttozfrac}
                    \frac{ \partial f }{ \partial y }(0,2)=\lim_{t\to 0} \frac{ f(0,2+t)-f(0,2) }{ t }=0.
                \end{equation}

            \item
                Nous devons vérifier si en prenant la limite \( (x,y)\to(0,2)\) des fonctions \eqref{SubeqsDerpartGAzztc} sont bien les nombres \( \partial_xf(0,2)\) et \( \partial_yf(0,2)\). La première limite à calculer est
                \begin{equation}        \label{EqlimxyAGzztcqymyfrac}
                    \lim_{(x,y)\to(0,2)}\frac{ 4y\sin(2x)\cos(2x) }{ x }-y\frac{ \sin^2(2x) }{ x^2 }.
                \end{equation}
                En décomposant et en sachant que \( \lim_{x\to 0} \sin(2x)/x=2\), nous trouvons que la limite \eqref{EqlimxyAGzztcqymyfrac} vaut zéro, ce qui ne correspond pas à la valeur de \( \partial_xf(0,2)\) donnée par \eqref{EqAGzztczdttozfrac}. La dérivée partielle par rapport à \( x\) n'est donc pas continue en \( (0,2)\). En ce qui concerne la dérivée par rapport à \( y\), nous avons
                \begin{equation}
                    \lim_{(x,y)\to(0,2)}\frac{ \sin^2(2x) }{ x }=0,
                \end{equation}
                cette dérivée directionnelle est donc continue.

            \item
                 Nous ne pouvons pas utiliser la proposition \ref{Diff_totale} et dire qu'il y a une dérivée partielle non continue pour conclure que la fonction n'est pas différentiable. En effet, cette proposition dit que si les dérivées partielles sont continues, alors la fonction est différentiable; elle ne dit rien dans le cas où les dérivées partielles ne sont pas continues. Nous pouvons par contre utiliser le lemme \ref{LemdfaSurLesPartielles} qui nous dit que si la fonction est différentiable, alors la différentie est donnée par l'application
                \begin{equation}
                    T(u)=\frac{ \partial f }{ \partial x }(0,2)u_1+\frac{ \partial f }{ \partial y }(0,2)u_2=8u_1.
                \end{equation}
                Affin de vérifier si cela est bien la différentielle, nous mettons ce \( T\) dans la définition \ref{DefDifferentiellePta} et nous voyons si la limite vaut zéro ou non :
                \begin{equation}
                        \lim_{h\to (0,0)} \frac{ \| f\big( (0,2)+h \big)-f(0,2)-T(h) \| }{ \| h \| }=\lim_{h\to (0,0)} \frac{ \| (2+h_2)\sin^2(2h_1)-8h_1 \| }{ \| h \| }.
                \end{equation}
                Remarque : la norme au numérateur est la norme dans \( \eR\) et est donc une simple valeur absolue. Nous pouvons l'enlever parce que ça ne change qu'un signe, ce qui ne change pas le fait que la fraction tende ou non vers zéro.

                En passant au coordonnées polaire, \( h=(r\cos(\theta),r\sin(\theta))\) nous trouvons à vérifier la limite
                \begin{equation}
                    \lim_{r\to 0} \frac{ (2+r\sin\theta)\sin^2(2r\cos\theta)-8r\cos\theta }{ r }.
                \end{equation}
                Nous pouvons calculer la limite terme par terme:
                \begin{subequations}
                    \begin{align}
                        \lim_{r\to 0} \frac{ 2\sin^2(2r\cos\theta) }{ r }&=0\\
                        \lim_{r\to 0} \frac{  r\sin\theta\sin^2(2r\cos\theta)}{r}&=0\\
                        \lim_{r\to 0} \frac{ -8r\cos\theta }{ r }=-8\cos\theta.
                    \end{align}
                \end{subequations}
                La limite \( r\to 0\) dépend donc de \( \theta\). Par conséquent la limite \( (h_1,h_2)\to(0,0)\) n'existe pas et en particulier la fonction n'est pas différentiable.
            
    \end{enumerate}

\end{corrige}
