% This is part of Mes notes de mathématique
% Copyright (C) 2010-2013,2016
%   Laurent Claessens
% See the file LICENCE.txt for copying conditions.

%+++++++++++++++++++++++++++++++++++++++++++++++++++++++++++++++++++++++++++++++++++++++++++++++++++++++++++++++++++++++++++ 
\section{Estimation de l'ordre de convergence}
%+++++++++++++++++++++++++++++++++++++++++++++++++++++++++++++++++++++++++++++++++++++++++++++++++++++++++++++++++++++++++++

Comment estimer numériquement l'ordre \( p\) de convergence de la méthode ? Soit une suite \( (x_n)\) convergente vers \( \alpha\). Considérons les \( 4\) termes \( x_{n-3}\), \( x_{n-2}\), \( x_{n-1}\), \( x_n\). Alors nous pouvons écrire l'approximation
\begin{equation}
    \frac{ | x_n -x_{n-1}| }{ | x_{n-1}-x_{n-2} | }\simeq \left( \frac{ | x_{n-1}-x_{n-2} | }{ | x_{n-1}-x_{n-3} | } \right)^p.
\end{equation}
Cette approximation ne serait pas trop mauvaise tant que \( n\) est assez grand pour que la convergence soit bien engagée. Passons au logarithme :
\begin{equation}
    \ln \frac{ | x_n -x_{n-1}| }{ | x_{n-1}-x_{n-2} | }\simeq p\ln \left( \frac{ | x_{n-1}-x_{n-2} | }{ | x_{n-1}-x_{n-3} | } \right).
\end{equation}
et donc
\begin{equation}
    p\simeq \frac{ \ln\left( \frac{ | x_n -x_{n-1}| }{ | x_{n-1}-x_{n-2} } \right) }{ \ln \left(\frac{ | x_{n-1}-x_{n-2} | }{ | x_{n-1}-x_{n-3} | } \right)}.
\end{equation}
Avec cette approximation, en réalité nous calculons une suite \( (p_i)\) qui sont les approximation de \( p\) à partir des termes \( i\) à \(i+3 \) de la suite \( (x_n)\). Il s'agit d'une suite d'estimations de \( p\).

\begin{enumerate}
    \item
Dans le cas de la bisection, nous obtenons toujours \( p_i=1\).
\item
    Dans le cas de la méthode de Newton (\ref{SECooIKXNooACLljs}) nous avons \( p=2\). Mais les premières valeurs de \( p_i\) peuvent être aussi bien \( 0\) que \( 7\). Après quelque itérations pourtant les \( p_i\) se regroupent autour de \( 2\).
\end{enumerate}
En tout cas, le plus important est de savoir si \( p>1\) ou non. Rappel : nous voulons la superlinéarité parce que nous voulons utiliser le test d'arrêt de la différence entre deux termes, voir \ref{NTooVXLXooXlAGEq}. 

%+++++++++++++++++++++++++++++++++++++++++++++++++++++++++++++++++++++++++++++++++++++++++++++++++++++++++++++++++++++++++++ 
\section{Autres méthodes}
%+++++++++++++++++++++++++++++++++++++++++++++++++++++++++++++++++++++++++++++++++++++++++++++++++++++++++++++++++++++++++++

%--------------------------------------------------------------------------------------------------------------------------- 
\subsection{Méthode de Schröder}
%---------------------------------------------------------------------------------------------------------------------------

La formule est
\begin{equation}
    x_{n+1}=x_n-\frac{ f(x_n)f'(x_n) }{ f'(x_n)^2-f(x_n)f''(x_n) }
\end{equation}
Cette méthode est d'ordre \( 2\) pour toute racine et toute valeur de multiplicité. Le problème de cette méthode est qu'elle demande \( 3\) évaluations de \( f\). Son efficacité :
\begin{equation}
    E=\sqrt[3]{ 2 }\simeq 1.25
\end{equation}
Cela est donc moins efficace que Newton.

%--------------------------------------------------------------------------------------------------------------------------- 
\subsection{Halley}
%---------------------------------------------------------------------------------------------------------------------------

Il a \( p=3\) lorsque \( \alpha\) est racine simple. Mais encore \( p=1\) pour les racines multiples. Plus efficace que Newton pour les racines simples, mais même problème pour les racines multiples.

\begin{equation}
    x_{n+1}=x_n-\frac{ 2f(x_n)f'(x_n) }{ 2f'(x_n)^2-f(x_n)f''(x_n) }
\end{equation}

%+++++++++++++++++++++++++++++++++++++++++++++++++++++++++++++++++++++++++++++++++++++++++++++++++++++++++++++++++++++++++++ 
\section{Méthode des sécantes variables}
%+++++++++++++++++++++++++++++++++++++++++++++++++++++++++++++++++++++++++++++++++++++++++++++++++++++++++++++++++++++++++++
\label{SECooIUEUooVcHAoc}

Supposons de ne pas avoir \( f\) analytique, mais seulement la possibilité de calculer \( f(x)\) pour tout \( x\). Newton ne fonctionne pas, mais la bisection fonctionne.

Nous pouvons approximer
\begin{equation}
    f'(x_n)=\frac{ f(x_n)-f(x_{n-1}) }{ x_n-x_{n-1} }.
\end{equation}
En substituant dans la formule de Newton, nous obtenons
\begin{equation}
    x_{n+1}=x_n-\frac{ f(x_n)(x_n-x_{n-1}) }{ f(x_n)-f(x_{n-1}) }.
\end{equation}

Il s'agit de prendre la droite qui passe par \( (x_{n-1},f(x_{n-1}))\) et par \( (x_n,f(x_n))\) et de prendre l'intersection de cette droite avec l'axe \( y=0\). Cela donne le \( x_{n+1}\).

Pour cette méthode, il ne faut pas seulement \( x_0\) mais également \( x_1\).

L'ordre de convergence est le nombre d'or
\begin{equation}    \label{EQooQEFCooUsGVjP}
    p=\frac{ 1-\sqrt{ 5 } }{ 2 }\simeq 1.618.
\end{equation}
Cela est donc superlinéaire.

La nombre d'évaluations est \( s=1\) (il y a deux apparitions de \( f\) dans la formule, mais l'une des deux est récupérée dans l'itération suivante). Donc l'efficacité est
\begin{equation}
    E=p.
\end{equation}
Donc bien efficace.

\begin{proposition}
    Si \( \alpha\) est racine simple, il existe un voisinage de \( \alpha\) tel que pour tout choix de \( x_0\), \( x_1\) dans ce voisinage, la méthode converge.
\end{proposition}

Psychologiquement, on est tenté de prendre \( x_0\) et \( x_1\) de part et d'autre de \( \alpha\) (pensant à la bisection), mais en réalité ce n'est pas obligatoire du tout et n'a aucune influence. Il faut seulement les prendre très proches de \( \alpha\).

\begin{remark}
    La méthode de la sécante est souvent écrite sous la forme
    \begin{equation}        \label{EQooYVKLooKTFjwv}
        x_{n+1}=\frac{ x_{n-1}f(x_n)-x_nf(x_{n-1}) }{ f(x_n)-f(x_{n-1}) }.
    \end{equation}
    C'est évidemment algébriquement équivalent. 

    Les formules \eqref{EQooQEFCooUsGVjP} et \eqref{EQooYVKLooKTFjwv} ont toutes deux des erreurs de cancellation. Laquelle est la plus grave ?

    Dans la première, si la fraction est mal calculée, elle ne fait que modifier \( x_n\). C'est à dire qu'on peut espérer qu'à la prochaine itération, ça aille mieux. En tout cas, dans ce cas si la fraction est mal calculée, ça ne détruit pas tout.

    Dans la seconde, c'est la valeur elle-même qui risque d'être mal calculée. Et si la fraction est mal calculée, alors on casse complètement l'éventuel bonne approximation que nous avions déjà.
\end{remark}

%--------------------------------------------------------------------------------------------------------------------------- 
\subsection{Aitken}
%---------------------------------------------------------------------------------------------------------------------------

La méthode du \( \Delta^2\) de Aitken est une méthode d'accélération de la convergence.

Soit \( (x_n)\) une suite qui converge. Nous voudrions une nouvelle suite \( (y_n)\) telle que
\begin{equation}
    \lim_{n\to \infty} \frac{ y_n-\alpha }{ x_n-\alpha }
\end{equation}
C'est la définition d'une convergence accélérée.

La façon de faire est :
\begin{equation}
    y_n=\frac{ x_{n+2}x_n-x_{n+1}^2 }{ x_{n+2}-2x_{n+1}+x_n }=x_n-\frac{ (x_{n+1}-x_n) }{ x_{n+2}-2x_{x+1}+x_n }.
\end{equation}
La première expressions a deux cancellations (la seconde une seule) et de plus la première est $y_n$ elle-même alors que la seconde est une correction.

Donc la seconde expression est numériquement meilleure.

L'opérateur \( \Delta\) appliqué à une suite est :
\begin{equation}
    (\Delta x)_n=x_{n+1}-x_n
\end{equation}
Donc
\begin{equation}
    (\Delta^2x)_n= (\Delta x)_{n+1}-(\Delta x)_n=x_{n+2}-x_{n+1}-x_{n+1}+x_n=x_{n+2}-2x_{n+1}+x_n.
\end{equation}
L'accélération a alors la formule
\begin{equation}
    y_n=\frac{ (\Delta x)_n^2 }{ (\Delta^2x)_n }.
\end{equation}

Le problème est que ça accélère tellement que l'on arrive vite à des erreurs de cancellations, et donc à une précision en pics oscillants.

%+++++++++++++++++++++++++++++++++++++++++++++++++++++++++++++++++++++++++++++++++++++++++++++++++++++++++++++++++++++++++++ 
\section{Équations algébrique}
%+++++++++++++++++++++++++++++++++++++++++++++++++++++++++++++++++++++++++++++++++++++++++++++++++++++++++++++++++++++++++++

C'est une équation du type \( P(x)=0\) où \( P\) est un polynôme. Soit un polynôme de degré \( n\). Nous en savons des choses.

\begin{enumerate}
    \item
        L'équation a exactement \( n\) solutions dans \( \eC\) en comptant les multiplicités.
    \item
        Les racines complexes arrivent par paire complexes conjuguée. Elles sont donc toujours en nombre pair.
\end{enumerate}

Si donc nous avons \( n=3\), nous ne pouvons pas avoir \( 2\) racine réelles. Il y en a donc \( 1\) ou \( 3\) réelles. Pas zéro ni deux.

Quelque méthodes : Müller, matrice compagnon, Laguerre.

%---------------------------------------------------------------------------------------------------------------------------
\subsection{Résoudre un système linéaire}
%---------------------------------------------------------------------------------------------------------------------------

Pour résoudre un système linéaire d'équations, nous échelonnons la matrice du système. Soit à résoudre le système $Ax=b$ où
\begin{equation}
	\begin{aligned}[]
		A&=\begin{pmatrix}
			2   &   4   &   -6  \\
			1   &   5   &   3   \\
			1   &   3   &   2
		\end{pmatrix}, &\text{et}&&b=\begin{pmatrix}
			-4  \\
			10  \\
			5
		\end{pmatrix}.
	\end{aligned}
\end{equation}
En termes de problèmes, on écrit $F\big( x,(A,b) \big)=Ax-b$. La donnée de ce problème est le couple $(A,b)$.

En ce qui concerne l'algorithme, on pose comme premier problème
\begin{equation}
	F_1\big(x_1,(A_1,b_1)\big)=A_1x_1-b_1=0
\end{equation}
avec $A_1=A$ et $b_1=b$.

Ensuite, on commence à échelonner et le second problème est
\begin{equation}
	F_2\big(x_2,(A_2,b_2)\big)=A_2x_2-b_2=0
\end{equation}
avec
\begin{equation}
	\begin{aligned}[]
		A&=\begin{pmatrix}
			2   &   4   &   -6  \\
			0   &   3   &   6   \\
			0   &   1   &   5
		\end{pmatrix}, &\text{et}&&b=\begin{pmatrix}
			-4  \\
			12  \\
			13
		\end{pmatrix}.
	\end{aligned}
\end{equation}
Le troisième problème sera
\begin{equation}
	F_3\big(x_3,(A_3,b_3)\big)=A_3x_3-b_3=0
\end{equation}
avec
\begin{equation}
	\begin{aligned}[]
		A&=\begin{pmatrix}
			2   &   4   &   -6  \\
			0   &   3   &   6   \\
			0   &   0   &   3
		\end{pmatrix}, &\text{et}&&b=\begin{pmatrix}
			-4  \\
			12  \\
			3
		\end{pmatrix}.
	\end{aligned}
\end{equation}
Ce problème est facile à résoudre «à la main». Nous nous arrêtons donc ici avec l'algorithme, et nous trouvons le $x_3$ qui résous le problème $F_3$.

%--------------------------------------------------------------------------------------------------------------------------- 
\subsection{Caractéristiques}
%---------------------------------------------------------------------------------------------------------------------------

L'algorithme de résolution de systèmes linéaires d'équations a les propriétés suivantes, à mettre en contraste avec celles de Newton :
\begin{enumerate}

	\item
		Pour résoudre le problème numéro $n$, il n'a pas fallu résoudre le problème numéro $n-1$.
	\item
		Toutes les solutions $x_n$ des problèmes intermédiaires sont solutions du problème de départ. Nous avons $F_n(x,d_n)=0$ pour tout $n$ (ici, $d_n=(A_n,b_n)$).
	\item
		D'un problème à l'autre, les données changent énormément : la matrice échelonnée peut être très différente de la matrice de départ.

\end{enumerate}

%---------------------------------------------------------------------------------------------------------------------------
\subsection{Définitions}
%---------------------------------------------------------------------------------------------------------------------------

	Nous allons maintenant formaliser en donnant quelques définitions pour nommer les propriétés que nous avons vues. D'abord, un algorithme est une suite de problèmes. Un \defe{algorithme}{algorithme} pour résoudre un problème $F(x,d)=0$ est une suite de problèmes $\{F_n(x_n,d_n)=0\}_{n\in\eN}$.

\begin{definition}
	Un tel algorithme est dit  \defe{fortement consistant}{algorithme!fortement consistant} si pour toutes données admissibles $d_n$, on a
	\begin{equation}
		F_n(x,d_n)=0\quad\forall \;n,
	\end{equation}
	où $x$ est la solution de $F(x,d)=0$.
\end{definition}
L'algorithme des matrices est fortement consistant, mais pas l'algorithme de Newton.

\begin{definition}
	Un algorithme est \defe{consistant}{algorithme!consistant} si $\lim_{n\to\infty}F_n(x,d_n)=0$.
\end{definition}
Dans le cas de l'algorithme de Newton, c'est plutôt une telle consistance qu'on attend.

L'algorithme est dit \defe{stable}{algorithme!stable} si pour tout $n$ le problème correspondant est stable.  Dans ce cas, on note $K^{\mbox{num}}$ le  \defe{conditionnement relatif asymptotique}{conditionnement!relatif asymptotique} défini par
\begin{equation}
	K^{\mbox{num}}=\limsup_nK_n
\end{equation}
où $K_n$ est le conditionnement relatif du problème $F_n(x_n,d_n)=0$.

\begin{definition}      \label{DefAlgoConverge}
	Un algorithme est dit \defe{convergent}{algorithme!convergent} (en $d$) si pour tout $\epsilon>0$, il existe $N=N(\epsilon)$ et $\delta=\delta(N,\epsilon)$ tels que pour $n\geq0$ et $|d-d_n|<\delta$, on ait $|x(d)-x_n(d_n)|<\epsilon$.
\end{definition}

\begin{remark}      \label{RemConvAlgoNewton}
Dans le cas de l'algorithme de Newton, nous avons vu que la donnée $d_n$ du problème $F_n$ était en fait la même que la donnée initiale $d$, donc nous avons $d_n=d$, et par conséquent nous avons toujours $| d-d_n |<\delta$. Dans ce cas, la définition de la convergence revient à demander que la suite numérique des $x_n$ converge vers la solution $x$.
\end{remark}

\begin{remark}
Dans le cas des matrices par contre, les données sont très différentes les unes des autres, nous avons donc en général que $| d-d_n |>\delta$. Mais en revanche nous savons que tous les problèmes intermédiaires $F_n$ acceptent une solution unique\footnote{Nous n'envisageons que le cas où le déterminant est non nul.} $x_n(d_n)=x(d)$. Par conséquent, $| x_n(d_n)-x(d) |$ est toujours plus petit que $\epsilon$. L'algorithme des matrice est donc toujours un algorithme convergent.
\end{remark}

%+++++++++++++++++++++++++++++++++++++++++++++++++++++++++++++++++++++++++++++++++++++++++++++++++++++++++++++++++++++++++++
\section{Équations non linéaire}
%+++++++++++++++++++++++++++++++++++++++++++++++++++++++++++++++++++++++++++++++++++++++++++++++++++++++++++++++++++++++++++

Certains équations non linéaires sont résoluble explicitement, par exemples les polynômes de degré jusqu'à \( 4\) ou des choses comme
\begin{equation}
	\sin^2(x)+3\sin(x)+5=0.
\end{equation}
Mais ces exemples sont très rares.

Nous allons étudier des équations du type \( f(x)=0\), dans \( \eR\).

\begin{enumerate}
	\item
Un problème écrit sous la forme \( x=g(x)\) peut utiliser des théorèmes de points fixes.
\item
	Un problème sous la forme \( f(x)=0\) peut utiliser des méthodes de bisection, Newton ou autres.
\end{enumerate}
Il y a évidemment beaucoup de façons de transformer un problème pour passer d'une forme à l'autre.

\begin{example}
	Soit \( f(x)=x^2-a=0\) avec \( a>0\). Nous pouvons l'écrire
	\begin{equation}
		x^2+x-a=x
	\end{equation}
	qui donne une forme \( g(x)=x\) pour \( g(x)=x^2+x-a\).

	Ou encore \( x=\frac{ a }{ x }\) et donc \( g(x)=a/x\) (si par ailleurs on sait que \( x\neq 0\)). Notons que \( x\neq 0\) n'est pas une hypothèse très forte parce qu'on la vérifie directement sur \( a\).
\end{example}

\begin{example}
	Soit l'équation à résoudre
	\begin{equation}
		f(x)=x^2-2-\ln(x)=0
	\end{equation}
	Les solutions de cette équations peuvent être vues comme les intersections avec l'axe \( X\) du graphe \( y=x^2-2-\ln(x)\). Tracer peut donc aider. Par ailleurs, il faut noter que
	\begin{equation}
		\lim_{x\to \pm\infty} f(x)=\infty,
	\end{equation}
	donc les solutions sont certainement contenues dans un compact de \( \eR\).

	À part tracer nous pouvons écrire
	\begin{equation}
		x^2-2=\ln(x).
	\end{equation}
	Et là, ce sont deux fonctions dont nous pouvons tracer le graphe pour trouver graphiquement les points d'intersection. Une étude de fonction montre vite qu'il y a exactement deux solutions, qu'elles sont strictement positives. Pour trouver des bornes, il faut calculer par exemple pour \( x=2\) les valeurs de \( \ln(x)\) et \( x^2-2\) pour voir si le graphe de \( x^2-2\) est déjà plus haut.
\end{example}

La majorité des méthodes numériques de résolution d'équation du type \( f(x)=0\) ou \( x=g(x)\) seront sous la forme de suites. Avec questions à la clefs :
\begin{enumerate}
	\item
		Quel point de départ choisir ?
	\item
		Convergence ?
	\item
		Est-ce que la limite est bien une solution ?
	\item
		Vu que la limite est unique, comment faire si l'équation a plusieurs solutions ? (souvent c'est le choix du point initial qui va jouer sur ce point)
\end{enumerate}

\begin{normaltext}
	Si la fonction est très plate, il est possible d'avoir
	\begin{equation}
		| f(\tilde \alpha) |\leq \epsilon
	\end{equation}
	sans que \( \tilde \alpha\) ne soit une bonne approximation.

	Lorsqu'on fait tourner une méthode itérative résolvant \( f(x)=0\), il n'est pas suffisant de s'arrêter lorsque
	\begin{equation}
		f(x_n)\leq \epsilon_1.
	\end{equation}
	Il faut aussi s'assurer que, si \( \bar x\) est la solution exacte, \( | x_n-\bar x |\leq \epsilon_2\). Ici \( \epsilon_1\) et \( \epsilon_2\) sont deux «précisions» que nous nous fixons au départ.

	Évidemment, vérifier la condition \( | x_n-\bar x |\leq \epsilon_2\), il faudrait savoir \( \bar x\). Et savoir \( \bar x\) c'est justement le problème. Nous sommes donc amenés à faire des estimation de \( | x_n-\bar x |\).
\end{normaltext}

\begin{normaltext}
    Lorsque nous effectuons une méthode itérative, il faut donc contrôler deux grandeurs :
    \begin{subequations}
        \begin{align}
            | \bar x-x_n |\leq \epsilon_1\\
            | x_{n+1}-x_n |\leq \epsilon_2.
        \end{align}
    \end{subequations}
\end{normaltext}

\begin{proposition}
Soit \( p\) l'ordre de convergence de la suite \( (x_n)\) vers \( \bar x\). Si \( p>1\) et \( | x_{n+1}-x_n |\leq \epsilon_2\) alors \( | \bar x-x_n |\leq \epsilon_2\).
\end{proposition}

%---------------------------------------------------------------------------------------------------------------------------
\subsection{Méthode de bisection}
%---------------------------------------------------------------------------------------------------------------------------

Il y a ce théorème des valeurs intermédiaires.
\begin{theorem}
	Soit \( f\) continue sur \( \mathopen[ a , b \mathclose]\) telle que \( f(a)f(b)<0\). Alors il existe au moins une solution à l'équation \( f(x)=0\) sur l'intervalle \( \mathopen] a , b \mathclose[\).
\end{theorem}

Pour démarrer une bisection, il est toujours bon de prendre l'intervalle \( \mathopen[ a , b \mathclose]\) de façon à ne contenir qu'une seule solution.

Soit donc un premier intervalle \( \mathopen[ a_0 , b_O \mathclose]\) tel que \( f(a_0)f(b_0)<0\) et ne contenant qu'une seule solution. À chaque itération nous considérons la moitié de l'intervalle précédent, mais la moitié contenant la solution.

Le test d'arrêt de la méthode de bisection se base uniquement sur la taille de l'intervalle qui reste. En effet si nous avons
\begin{equation}
	| b_n-a_n |\leq \epsilon
\end{equation}
nous avons certainement
\begin{equation}
	| \bar x-x_n |\leq \frac{ \epsilon }{2}
\end{equation}
où \( x_n\) est le point du milieu de \( \mathopen[ a_n , b_n \mathclose]\).

\begin{normaltext}
    La fonction \( f\) n'intervient dans la méthode que via son signe, pas via ses valeurs exactes.
\end{normaltext}

\begin{normaltext}
    Notons que le théorème des valeurs intermédiaires n'est pas très puissant pour choisir l'intervalle de départ; penser à la fonction
    \begin{equation}
        f(x)=x^2-5
    \end{equation}
    sur l'intervalle \( \mathopen[ -10 , 10 \mathclose]\). Il y a bien deux solutions dans l'intervalle, mais elles sont invisibles du théorème des valeurs intermédiaires. La fonction \( x\mapsto x^2\) a sa solution en \( x=0\), mais elle aussi n'est pas visible.
\end{normaltext}

\begin{normaltext}
    Certes la méthodes de bisection assure la convergence vers une solution, mais elle n'assure pas la convergence monotone. Il peut arriver que \( | \bar x-x_n |<| \bar x-x_{n+1} |\). C'est le cas lorsque la solution est très proche du milieu de l'intervalle choisit. Le \( x_0\) est alors proche de \( \bar x\) alors que \( x_1\) sera à une distance de \( \bar x\) d'environ un quart de l'intervalle de départ.
\end{normaltext}

Supposons déjà avoir trouvé un intervalle \( \mathopen[ a , b \mathclose]\) dans lequel se trouve une unique solution à \( f(x)=0\). Voici un algorithme possible.

\lstinputlisting{codeSnip_1.py}

Plusieurs remarques :
\begin{enumerate}
    \item
Le fait de retourner le nombre d'itérations effectuées permet à l'utilisateur de savoir la précision et si le nombre maximum d'itérations est dépassé. Si ce \info{n} retourné est égal à \info{nmax}, l'utilisateur sait que le \info{x} retourné n'est pas fiable.
\item
    La ligne \info{from \_\_future\_\_ import division} fait en sorte que l'opération \info{/} est bien la division usuelle. Sinon, le défaut en python 2 est que \info{/} soit la division \emph{entière}, c'est à dire que \( 1/2=0\) en python 2.o
    En python 3, le symbole \info{/} désigne bien la division usuelle, mais Sage utilise Python 2.
\item
    Même si l'intervalle \( \mathopen[ a , b \mathclose]\) contient plus d'une solution, la méthode fonctionne et donne une solution. Il est simplement éventuellement très compliqué de savoir laquelle.
\item
    Nous faisons \info{amp=toll+1} parce que nous voulons absolument lancer le cycle au moins une fois. Sinon, le \info{x} à retourner ne serait pas définit au moment de sortir du cycle (si le cycle n'est pas exécuté).
\item
    Calculer le point milieu d'un intervalle \( \mathopen[ a , b \mathclose]\) est par la formule \( (a+b)/2\) sauf que cette opération est numériquement dangereuse parce qu'à cause de l'arithmétique en précision finie, il est possible que cela tombe \emph{exactement} sur \( a\) ou \( b\). D'où le fait de calculer le point milieu par
    \begin{equation}
        x=a+\frac{ amp }{2}.
    \end{equation}
\item
    Dans le cas \info{Problème ZERO} nous déduisons \( f(x)=0\). Attention que c'est pas que \( f(x)=0\) mais simplement que en mettant \( x\) dans \( f\), la \emph{machine} retourne son zéro.

    Il peut cependant avoir une fonction telle que \( f(1)=10^{-50}\) et \( f(2)=0\). L'algorithme de bisection risque de s'arrêter si \( x_n=1\). Parce que la machine risque de calculer \( f(x_n)=0\).

    Quoi qu'il en soit, nous y mettons \info{amp=0} pour être sûr de sortir de la boucle dès la prochaine vérification.
\item
    Il y a moyen de sauver les valeurs de \( f(a)\) et \( f(x)\) pour ne pas les recalculer, et en particulier au moment de faire \info{b=x} nous pouvons poser \(\info{fa=fx}\).
\end{enumerate}

Si \( \tau\) est la précision de la solution voulue, nous pouvons fixer a priori le nombre d'itérations à faire grâce à la formule
\begin{equation}
    n\geq\left\lceil  \log_2\big( \frac{ b-a }{ \tau } \big)  \right\rceil.
\end{equation}
Il y a un ``\( \geq\)'' et non une égalité parce qu'en arithmétique numérique, le nombre obtenu à droite pourrait ne pas être le bon à \( 1\) près.

Ici pour \( \nu\in \eR\) le nombre \( \lceil\nu\rceil\) est le plus petit entier à être plus grand ou égal à \( \nu\).

\begin{normaltext}
    Notons l'importance de la continuité de \( f\). Par exemple que ferait la bisection sur la fonction \( f(x)=1/x\) pour l'intervalle $\mathopen[ -3 , 1 \mathclose]$ ? 

    Il y a changement de signe sans avoir de racine.
\end{normaltext} 

Vu que \( 2^{10}\) est déjà \( 1024\). Donc si on veut de la précision de l'ordre de \( 1/1000\), dix itération suffisent. Si donc nous avons besoin de \( 200\) itérations pour atteindre la précision voulue, c'est l'occasion de trouver un intervalle plus petit. Par exemple en traçant la fonction, en faisant un zoom et en trouvant des valeurs de \( a\) et \( b\) qui sont déjà proches.


\begin{normaltext}
    Dans le monde réel, il arrive souvent d'utiliser une méthode de bisection pour se donner un point de départ pour une autre méthode.
\end{normaltext}

%+++++++++++++++++++++++++++++++++++++++++++++++++++++++++++++++++++++++++++++++++++++++++++++++++++++++++++++++++++++++++++ 
\section{Efficacité}
%+++++++++++++++++++++++++++++++++++++++++++++++++++++++++++++++++++++++++++++++++++++++++++++++++++++++++++++++++++++++++++

\begin{definition}
    L'\defe{efficacité}{efficacité!d'une méthode itérative} est le nombre
    \begin{equation}
        E=\sqrt[s]{ p }
    \end{equation}
    où \( p\) est l'ordre de convergence de la méthode et \( s\) est le nombre de fois qu'il faut calculer une valeur de la fonction à chaque itération (nous ne comptons pas l'initialisation).
\end{definition}
Que le nombre de d'évaluations de \( f\) intervienne est logique parce que chaque évaluation provoque une erreur possible.

\begin{example}[Bisection]
    Pour la méthode de bisection, nous avons \( s=1\) parce que chercher \( x_{n+1}\), il faut seulement calculer \( f(x_n)\).
\end{example}

\begin{example}[Newton]
    Pour l'algorithme de Newton nous avons \( p=2\) et il y a deux évaluations à chaque itérations (une fois \( f\) et une fois \( f'\)), donc \( s=2\) et \( E=\sqrt{ 2 }\).
\end{example}

%+++++++++++++++++++++++++++++++++++++++++++++++++++++++++++++++++++++++++++++++++++++++++++++++++++++++++++++++++++++++++++ 
\section{Exercices}
%+++++++++++++++++++++++++++++++++++++++++++++++++++++++++++++++++++++++++++++++++++++++++++++++++++++++++++++++++++++++++++

\Exo{mazhe-0002}
\Exo{mazhe-0003}
\Exo{mazhe-0004}

