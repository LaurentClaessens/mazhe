% This is part of Analyse Starter CTU
% Copyright (c) 2014
%   Laurent Claessens,Carlotta Donadello
% See the file fdl-1.3.txt for copying conditions.

\begin{corrige}{autoanalyseCTU-10}

\begin{enumerate}
\item On sait que $\arctan(x)\in\left]-\frac{\pi}{2},\frac{\pi}{2}\right[$ et par conséquent $2\arctan(x)\in\left]-\pi, \pi\right[$. Or, la fonction $\tan$ n'est pas définie en $\pm\frac{\pi}{2}$, donc il faut éviter les valeurs de $x$ pour lesquels $2\arctan(x)=\pm\frac{\pi}{2}$, c'est à dire qu'il faut imposer la condition $x\neq \pm 1$. Le domaine de la fonction $f$ est donc $\eR\setminus\{\pm 1\}$.
\item Pour plus de lisibilité on appelle $g(x)$ la fonction  $\frac{2x}{1-x^2}$. 

La façon classique de résoudre ce type d'exercice comporte deux pas : 
\begin{itemize}
\item remarquer que $g(0) = 0$ et $f(0) = 0$ c'est à dire que les graphes des fonctions $f$ et $g$ se croisent en correspondance à $x= 0$ ; 
\item  calculer les dérivées de $f$ et de $g$, montrer qu'elles sont égales et conclure que la fonction différence $f-g$ est constante (et donc il s'agit de la fonction nulle, par le point précédent).
\end{itemize}
Dans cet exercice en particulier, cette méthode n'est pas très performante, parce que l'expression de la dérivée de $f$ est compliquée. 

Nous allons donc utiliser la définition de la fonction tangente, les formules pour calculer sinus et cosinus de $2x$ et la définition de $\arctan$. 
\begin{equation*}
  \begin{aligned}
    f(x)=& \tan(2 \arctan (x)) = \frac{\sin(2 \arctan (x))}{\cos(2 \arctan (x))} \\
    &= \frac{2\sin(\arctan (x))\cos(\arctan (x))}{\cos^2(\arctan (x))-\sin^2(\arctan (x))}\\
    &=\frac{2\frac{\sin(\arctan (x))}{\cos(\arctan (x))}}{1-\frac{\sin^2(\arctan (x))}{\cos^2(\arctan (x))}} = \frac{2\tan(\arctan (x))}{1-\tan^2(\arctan (x))}\\
    &= \frac{2x}{1-x^2} = g(x),
  \end{aligned}
\end{equation*}
pour tout $x$ dans l'ensemble de définition de $f$.
\end{enumerate}


\end{corrige}   
