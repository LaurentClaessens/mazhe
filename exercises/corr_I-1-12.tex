% This is part of the Exercices et corrigés de CdI-2.
% Copyright (C) 2008, 2009
%   Laurent Claessens
% See the file fdl-1.3.txt for copying conditions.


\begin{corrige}{_I-1-12}

En $x=0$, nous avons la série $\sum_n (-1)^n$ qui ne converge pas. En $x<0$, le terme général de la série ne converge pas vers zéro, donc il y a divergence de la série. Travaillons donc avec $x>0$.

Prenons un compact $K$ dans $]0,\infty[$, nommons $\alpha$ son minimum. Pour tout $x\in K$, nous avons
\begin{equation}
	\frac{ 1 }{ n^x }\leq\frac{1}{ n^{\alpha} }.
\end{equation}
Sur le compact, la suite de fonctions $b_n(x)=\frac{1}{ n^x }$ converge uniformément vers zéro, de telle sorte que le critère d'Abel (théorème \ref{ThoSerCritAbel}) conclue à la convergence uniforme. Nous avons donc la convergence uniforme sur tout compact dans $]0,\infty[$. Le théorème \ref{ThoSerUnifCont} dit alors que 
\begin{equation}
	f(x)=\sum_{n=1}^{\infty}\frac{ (-1)^n }{ n^x }
\end{equation}
est une fonction continue sur $]0,\infty[$.

\end{corrige}
