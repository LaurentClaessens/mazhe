% This is part of Exercices et corrigés de CdI-1
% Copyright (c) 2011
%   Laurent Claessens
% See the file fdl-1.3.txt for copying conditions.

\begin{corrige}{OutilsMath-0018}

	Il s'agit de la droite d'équation cartésiennes $x+y=1$. L'équation en coordonnées polaires est obtenue en remplaçant $x$ par $r\cos(\theta)$ et $y$ par $r\sin(\theta)$ :
	\begin{equation}
		r\cos(\theta)+r\sin(\theta)=1,
	\end{equation}
	ou encore, en factorisant, 
	\begin{equation}
		r\big( \cos(\theta)+\sin(\theta) \big)=1.
	\end{equation}
	Cela est l'équation, mais maintenant il faut préciser le domaine de variation des variables. En effet si nous prenons par exemple $\theta=\pi$, nous voyons $-r=1$. Mais le rayon doit toujours être positif. Tous les angles ne sont donc pas acceptables. D'ailleurs sur un dessin nous voyons tout de suite que la droite ne passe pas par tous les angles.

	Les angles acceptables sont ceux tels que $\cos(\theta)+\sin(\theta)> 0$. L'équation
	\begin{equation}
		\cos(\theta)=-\sin(\theta)
	\end{equation}
	a pour solutions les angles $\theta_1=\frac{ 3\pi }{ 4 }$ et $\theta_2=-\frac{ \pi }{ 4 }$. Les angles acceptables sont donc ceux entre $\theta_1$ et $\theta_2$. L'équation complète de la droite est alors
	\begin{equation}
		r\big( \cos(\theta)+\sin(\theta) \big)=1,
	\end{equation}
	avec $\theta\in\mathopen] -\frac{ \pi }{ 4 } , \frac{ 3\pi }{ 4 } \mathclose[$.

\end{corrige}
