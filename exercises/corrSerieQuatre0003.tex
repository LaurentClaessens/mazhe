% This is part of Exercices et corrections de MAT1151
% Copyright (C) 2010
%   Laurent Claessens
% See the file LICENCE.txt for copying conditions.

\begin{corrige}{SerieQuatre0003}

	La première chose à faire pour ce genre d'exercice est d'écrire le nombre en somme de puissances de deux :
	\begin{equation}		\label{EaQTdecun}
		235.5=2^7+2^6+2^5+2^3+2+2^{-1},
	\end{equation}
	ensuite on met la plus grande puissance de $2$ en évidence\footnote{Si on avait précisé un $L$ et que la plus grande puissance dépassait $L$, l'exercice serait impossible.}

	Nous écrivons maintenant la suite des $0$ et $1$ en fonction des coefficients des puissances de $2$ : $111010101$. Les zéros correspondent aux puissances $4$, $2$ et $0$ qui ne sont pas présentes dans la décomposition \eqref{EaQTdecun}. Au final, le nombre s'écrit
	\begin{equation}
		\{ [111010101],2,7,0 \}.
	\end{equation}
	Le $7$ est la puissance mise en évidence, le $2$ est la base et le $0$ indique qu'on a un nombre positif.

\end{corrige}
