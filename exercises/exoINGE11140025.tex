% This is part of Un soupçon de physique, sans être agressif pour autant
% Copyright (C) 2006-2009
%   Laurent Claessens
% See the file fdl-1.3.txt for copying conditions.


\begin{exercice}\label{exoINGE11140025}

	Résoudre les équations et inéquations suivantes :
	\begin{enumerate}

		\item
			$\cos(x)=1$
		\item
			$2\sin\frac{ x }{2}=\frac{1}{ 2 }$
		\item
			$\cos(\frac{ \pi }{2}-x)=\frac{ \sqrt{3} }{2}$
		\item
			$\sin(x)=-\cos(x)$
		\item
			$-3\cos^2(3x)+\frac{ 5 }{2}\cos(3x)-1=0$
		\item
			$\tan(2x)=3\tan(x)$
		\item
			$\sin(x-\frac{ \pi }{ 3 })>\sin(x)$
		\item
			$\sin(2x)=0$
		\item
			$\cos(40x)=4$
		\item
			$\sin(x)=\cos(x)$
		\item
			$\sin^2(x)+\cos^2(x)=\frac{1}{ 2 }$
		\item
			$| \cos(2x) |<\frac{1}{ 2 }$
		\item
			$\arcsin(x)=-\frac{ \pi }{ 4 }$

	\end{enumerate}

\corrref{INGE11140025}
\end{exercice}
