% This is part of Analyse Starter CTU
% Copyright (c) 2014
%   Laurent Claessens,Carlotta Donadello
% See the file fdl-1.3.txt for copying conditions.

\begin{corrige}{TD5-00002}
  \begin{enumerate}
 \item $I_{1}=\displaystyle \int_{1}^{\pi} \frac{4x^3}{x^4+5}\, dx = \left[\ln(x^4+5)\right]_1^\pi = \ln\left(\frac{\pi^4+5}{6}\right)$.
  \item[(2)] On utilise le changement de variable $u =4x+1 $. On a alors $du = 4 dx$, les bornes du domaine d'intégration deviennent $u(0) = 1$ et $u(1) =5$ et notre intégrale s'écrit de la forme suivante
    \[
    I_{2}=\int_{0}^{1}\frac{1}{(4x+1)^4}\, dx =\int_{1}^{5}\frac{1}{4u^4}\, du = \left[-\frac{1}{12u^3}\right]_1^5 = -\frac{1}{12\times 125} + \frac{1}{12} = \frac{31}{375}.
    \]
  \item[(3)] Ici on vise à écrire la fonction à intégrer comme $\frac{1}{u^2 + 1}$, pour avoir $\arctan(u) + C$ comme primitive. Pour ainsi faire, nous devons utiliser le changement de variable $u= \frac{x}{\sqrt{5}}$. On a alors $du = \frac{1}{\sqrt{5}} dx$, les bornes du domaine d'intégration deviennent $u(0) = 0$ et $u(\sqrt{5}) =1$ 
    \[
    I_{3}=\int_{0}^{\sqrt{5}}\frac{1}{x^2+5}\, dx = \int_{0}^{1}\frac{\sqrt{5}}{5(u^2+1)}\, du = \frac{1}{\sqrt{5}} [\arctan(u)]_0^1 = \frac{\pi}{4\sqrt{5}}.
    \]
  \item On utilise le changement de variable $u = \sin(x)$, qui comporte $du = \cos(x)\,dx$,  $u(-\pi/2) = -1$ et  $u(0) = 0$. Nous avons alors que  $I_{4}=\displaystyle \int_{-1}^{0} u^2\, du = \left[\frac{u^3}{3}\right]_{-1}^0= \frac{1}{3}$.
  \item[(5)] Le bon changement de variable nous est donné par l'énoncé. On a $du= e^x dx$, les bornes d'intégration deviennent $u(0) = e^0=1$ et $u(1) = e$ et notre intégrale sera 
    \[
    I_{5}=\int_{0}^{1}\frac{e^x}{e^{2x}+4} \, dx =\int_{1}^{e}\frac{1}{u^2+4} \, du. 
    \] 
    Il faut maintenant travailler comme dans le point précédent de cet exercice. Soit $t =\frac{u}{2}$ alors 
    \[
    I_{5}=\int_{1/2}^{e/2}\frac{2}{4(t^2+1)} \, dt = \frac{1}{2} \left[\arctan\left(\frac{e}{2}\right) - \arctan\left(\frac{1}{2}\right)\right].  
    \] 
 \item[(6)] Le bon changement de variable nous est donné par l'énoncé. On a $ (1 + \tan^2(u))du= dx$, les bornes d'intégration deviennent $\arctan(0) = 0$ et $\arctan(1) = \pi/4$ et notre intégrale sera 
    \[
    I_{6}=\int_{0}^{1}\frac{1}{(1+x^2)^{\frac{3}{2}}}\, dx  = \int_{0}^{\pi/4}\frac{1+\tan^2(u)}{(1+\tan^2(u))^{\frac{3}{2}}}\, du= \int_{0}^{\pi/4}\frac{1}{(1+\tan^2(u))^{\frac{1}{2}}}\, du.
    \]
    On observe alors que 
    \[
    \frac{1}{(1+\tan^2(u))^{\frac{1}{2}}} = \cos(u),
    \]
    et donc 
    \[
    I_{6}= \int_{0}^{\pi/4}\cos(u)\, du = \frac{\sqrt{2}}{2}.
    \]
  \item On utilise le changement de variable $x = a\sin(u)$, qui comporte $dx = a\cos(u)\,du$ et, en utilisant la fonction $u(x) = \arcsin(x/a)$, $u(a) = \pi/2$  et  $u(0) = 0$. On obtient
\[
 I_{7}=\int_{0}^{\pi/2}\sqrt{a^2-a^2\sin^2(u)} a\cos(u)\, du = a^2\int_{0}^{\pi/2} |\cos(u)|\cos(u)\, du.
\]
Cette dernière intégrale est égale à $\displaystyle a^2\int_{0}^{\pi/2} \cos^2(u)\, du$, car toutes les valeurs prises par la fonction cosinus lorsque $x$ varie entre $0$ et $\pi/2$ sont positives. En intégrant par parties on trouve que 
 \[
\int_{0}^{\pi/2} \cos^2(u)\, du =\int_{0}^{\pi/2} \sin^2(u)\, du,
\]
d'où on peut écrire 
 \[
\int_{0}^{\pi/2} \cos^2(u)\, du =\int_{0}^{\pi/2} \frac{\sin^2(u) +\cos^2(u)}{2}\, du = \frac{\pi}{4}.
\]
La velauer de l'intégrale $ I_{7}$ est $a^2\frac{\pi}{4}$.
    \item[(8)]  On intègre par parties 
      \begin{equation*}
        \begin{aligned}
           I_{8}=&\int_{-\pi}^{0} e^x\sin(x) \, dx =  \left[e^x\sin(x)\right]_{-\pi}^{0}-\int_{-\pi}^{0} e^x\cos(x) \, dx \\
           &= 0 -\left[e^x\cos(x)\right]_{-\pi}^{0}-\int_{-\pi}^{0} e^x\sin(x) \, dx = -1-e^{-\pi}-I_{8}
           \end{aligned}
      \end{equation*}
      On conclut que $I_{8} = \frac{-1-e^{-\pi}}{2}$.
    \item[(9)]  On intègre par parties 
      \begin{equation*}
        \begin{aligned}
      I_{9}=&\int_{1}^{2} x^2\ln(2x) \, dx = \left[\frac{x^3}{3}\ln(2x)\right]_{1}^{2} - \int_{1}^{2} \frac{x^2}{3} \, dx \\
      =& \left[\frac{8}{3}\ln(4) - \frac{1}{3}\ln(2)\right] -\left[\frac{8}{9}-\frac{1}{9}\right] = \ln\left(\frac{4^{8/3}}{2^{1/3}}\right)-\frac{7}{9} =\ln\left(32\right)-\frac{7}{9}.  
        \end{aligned}
      \end{equation*}
     
    \item[(10)]  Il faut réduire la fraction rationnelle en élément simples à intégrer. Pour le faire on commence par regarder le dénominateur et remarquer qu'il peut s'écrire comme $x^3 (x^2 + 1)$. Notre objectif sera alors d'écrire $\frac{x^4+1}{x^5+x^3}$ comme une somme entre deux fractions rationnelles de dénominateur respectif $x^3$ et $x^2+1$. Comme le polyn\^ome au numérateur a degré 4 il faut prévoir un polyn\^ome de degré 2 au numérateur de $x^3$ et un  polyn\^ome de degré 1 au numérateur de $x^2+1$. On a alors 
      \[
      \frac{x^4+1}{x^5+x^3} = \frac{Ax^2+Bx + C}{x^3} +\frac{Dx+E}{x^2+1}. 
      \]
      Cela va nous donner un système de 5 équations pour les 5 inconnues $A$, $B$, $C$, $D$, $E$. On trouve que $A = -1$, $B = 0$, $C =1$, $D = 2$, $E = 0$. 
      \[
      I_{10}=\int_{1}^{2}\frac{x^4+1}{x^5+x^3}\, dx  = \int_{1}^{2}\frac{-x^2+1}{x^3} +\frac{2x}{x^2+1}\, dx  =\int_{1}^{2}\frac{-1}{x} +\frac{1}{x^3} +\frac{2x}{x^2+1}\, dx .
      \]
      Par un calcul désormais immédiat on trouve $I_{10}= \ln(5/4) + 3/8$.
    \item $\displaystyle I_{11}=\int_{-3}^0\frac{1}{ x^2-3x+2 }dx$. Il est facile de vérifier que $\frac{1}{ x^2-3x+2 } =- \frac{ 1}{ x-1 }+\frac{ 1}{ x-2 }$. En écrivant la fonction à intégrer comme la somme de deux termes nous avons alors 
\[
I_{11}=-\int_{-3}^0\frac{ 1}{ x-1 }\,dx+\int_{-3}^0\frac{ 1}{ x-2 }\,dx = \left[-\ln(|x-1|)+\ln(|x-2|)\right]_{-3}^0 = \ln\left(\frac{8}{5}\right).
\]
    \item Par parties : $\displaystyle I_{12}=\int_1^2\ln^2(x)dx =\left[x\ln^2(x)\right]_{1}^2 - \int_1^22\ln(x)dx = \left[x\ln^2(x)-2x\ln(x) + 2x\right]_{1}^2 = 2\left(\ln^2(2)-2\ln(2) + 1\right) $.
    \item[(13)]  On va essayer le changement de variable $u= e^x+1 $. On a alors $du = e^x \,dx$ et on pourra écrire $e^x = u-1$. Les bornes d'intégration deviennent $u(0) = 2$ et $u(1) = e + 1$.
      \[
      I_{13}=\int_0^1\frac{ e^x-1 }{ e^x+1 }dx = \int_2^{e+1}\frac{u-2}{u(u-1)} \, du.
      \]
      La fonction de $u$ à intégrer peut s'écrire comme la somme de deux fractions rationnelles plus simples de dénominateur respectif $u$ et $u-1$ : pour le faire nous devons trouver $A$ et $B$ dans $\eR$ tels que 
      \[
      \frac{u-2}{u(u-1)} = \frac{A}{u} +\frac{B}{u-1} 
      \]
      On obtient $A=2$ et $B= -1$, donc notre intégrale devient 
      \[
      I_{13}=\int_2^{e+1} \frac{2}{u} -\frac{1}{u-1}\, du = \ln\left(\frac{(e+1)^2}{4e}\right).
      \]
    \item[(14)] Nous utilisons le changement de variable $u= 1+2e^{-x}$. On a alors $du = -2e^{-x} dx$  et $\frac{1}{1-u}du = dx$.  
      \begin{equation*}
        \begin{aligned}
          I_{14}=&\int_1^2\frac{  e^{-2x} }{ (1+2 e^{-x})^2 }\,dx = \int_{1+2/e}^{1+2/e^2}\left(\frac{u-1 }{ 2 }\right)^2\frac{1}{u^2(1-u)}\,dx = -\int_{1+2/e}^{1+2/e^2}\frac{u-1 }{4u^2}\,dx\\
          &= \frac{1}{4}\left[\ln(x)+\frac{1}{x}\right]_{1+2/e^2}^{1+2/e} = \frac{1}{4}\left[\ln\left(\frac{e+2}{e^2+2}\right)+\frac{e}{e+2}-\frac{e^2}{e^2+2}\right].
        \end{aligned}
      \end{equation*}
    \item[(15)]  Le bon changement de variable nous est donné par l'énoncé. On a $ dt= 2xdx$, les bornes d'intégration deviennent $t(0) = 1$ et $t(1) = 2$ et notre intégrale sera 
      \[
      I_{15}=\int_0^1\frac{ x^3 }{ x^2+1 }dx =\int_1^2\frac{ t-1 }{ 2t }dt = \left[\frac{t}{2} -\frac{1}{2}\ln(t)\right]_1^2  = \frac{1}{2}\left(1-\ln(2)\right).
      \]
  \end{enumerate}
\end{corrige}
