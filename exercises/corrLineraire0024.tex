% This is part of the Exercices et corrigés de mathématique générale.
% Copyright (C) 2009
%   Laurent Claessens
% See the file fdl-1.3.txt for copying conditions.
\begin{corrige}{Lineraire0024}

	\begin{enumerate}
		\item
		La base $E$ étant la base canonique, le vecteur $\begin{pmatrix}
			a	\\ 
			b	
		\end{pmatrix}$ a pour coordonnées $(a,b)$. Pour la base $F$, il faut résoudre le système
		\begin{equation}
			\begin{pmatrix}
				a	\\ 
				b	
			\end{pmatrix}=\lambda_1\begin{pmatrix}
				1	\\ 
				1	
			\end{pmatrix}+\lambda_2\begin{pmatrix}
				1	\\ 
				2	
			\end{pmatrix}.
		\end{equation}
		La résolution donne donc $(2a-b,b-a)$. Pour trouver les coordonnées de ce même vecteur dans la base $G$, il faut résoudre
		\begin{equation}
			\left\{
			\begin{array}{ll}
				\lambda_1+2\lambda_2=a\\
				2\lambda_1+\lambda_2=b.
			\end{array}
			\right.
		\end{equation}
		Cela se résous facilement par substitution, et le résultat est $\lambda_1=\frac{ 2b-a }{ 3 }$ et $\lambda_2=\frac{ 2a-b }{ 3 }$.

	
		\item
			Nous avons d'abord $7e_1-2e_2=\begin{pmatrix}
				7	\\ 
				-2	
			\end{pmatrix}$. Ensuite, nous devons résoudre
			\begin{equation}
				\begin{pmatrix}
					7	\\ 
					-2	
				\end{pmatrix}=\lambda_1\begin{pmatrix}
					1	\\ 
					2	
				\end{pmatrix}+\lambda_2\begin{pmatrix}
					2	\\ 
					1	
				\end{pmatrix}.
			\end{equation}
			Cela donne $(-11/3,16/3)$.

	\end{enumerate}
	
\end{corrige}
