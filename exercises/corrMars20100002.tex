% This is part of the Exercices et corrigés de mathématique générale.
% Copyright (C) 2009-2010
%   Laurent Claessens
% See the file fdl-1.3.txt for copying conditions.


\begin{corrige}{Mars20100002}

	La matrice «augmentée» du système est
	\begin{equation}
		\left(\begin{array}{ccc|c}
			 2	&	-1	&	4	&	1	\\
			  0	&	2	&	-6	&	8\\
			   -4	&	4	&	(k-11)	&	6	 
		   \end{array}\right).
	\end{equation}
	Diviser la seconde ligne par deux est toujours une bonne idée pour y voir plus clair. Ensuite on commence à échelonner en faisant $L_3\to L_3+2L_1$:
	\begin{equation}
		\left(\begin{array}{ccc|c}
			 2	&	-1	&	4	&	1	\\
			  0	&	1	&	-3	&	4\\
			   0	&	2	&	k-3	&	8	 
		   \end{array}\right).
	\end{equation}
	Ensuite on fait $L_3\to L_3-2L_2$ pour mettre un dernier bon zéro où il faut:
	\begin{equation}
		\left(\begin{array}{ccc|c}
			 2	&	-1	&	4	&	1	\\
			  0	&	1	&	-3	&	4\\
			   0	&	0	&	k+3	&	0	 
		   \end{array}\right).
	\end{equation}
	Sur cette matrice, on voit tout de suite que la discussion portera sur $k=-3$ ou $k\neq -3$. D'ailleurs, en calculant le déterminant de la matrice initiale, on aurait vu qu'il était un multiple de $k+3$, donc le déterminant est nul si et seulement si $k=-3$.

	Commençons par $k=-3$.

	Dans ce cas, la dernière ligne devient $0=0$ et on peut la barrer. Il reste deux équations pour trois inconnues. Posons $z=\lambda$, c'est à dire que nous prenons $z$ comme paramètre. Alors la seconde ligne devient $y-3\lambda=4$, c'est à dire $y=4+3\lambda$. La première ligne devient alors $2x-4-3\lambda+4\lambda=1$, c'est à dire $x=(5-\lambda)/2$. Les solutions sont donc
	\begin{equation}
		\begin{pmatrix}
			\frac{ 5-\lambda }{2}	\\ 
			4+3\lambda	\\ 
			\lambda	
		\end{pmatrix}.
	\end{equation}
	Pour chaque $\lambda\in\eR$, il y a une solution. Il y a donc une infinité de solutions qui correspondent au cas de déterminant nul.

	Prenons maintenant $k\neq -3$.

	Dans ce cas, la dernière ligne devient $(k+3)z=0$, c'est à dire $z=0$. Les autres lignes donnent alors $y=4$ et $x=5/2$. L'unique solution correspondant aux cas de déterminant non nul est
	\begin{equation}
		\begin{pmatrix}
			5/2	\\ 
			4	\\ 
			0	
		\end{pmatrix}.
	\end{equation}

\end{corrige}
