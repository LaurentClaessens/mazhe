% This is part of Un soupçon de physique, sans être agressif pour autant
% Copyright (C) 2006-2009
%   Laurent Claessens
% See the file fdl-1.3.txt for copying conditions.


\begin{corrige}{SerieUn0016}

	Vérifions avec $n=2$ :
	\begin{equation}
		1+3=4=2^2,
	\end{equation}
	c'est bon.

	Supposons maintenant avoir prouvé la thèse pour un certain $n$. Nous allons montrer qu'alors la thèse est encore vraie pour $n+1$. En écrivant la formule pour $n+1$, nous avons
	\begin{equation}
		\underbrace{1+3+5+\cdots+(2n-1)}_{\text{$=n^2$ par hyp. de réccurence}}+2(n+1)-1=n^2+2n+2-1=(n+1)^2,
	\end{equation}
	ce qu'il fallait.

\end{corrige}
