% This is part of Exercices et corrections de MAT1151
% Copyright (C) 2010,2015
%   Laurent Claessens
% See the file LICENCE.txt for copying conditions.

\begin{corrige}{SerieDeux0006}

    La formule d'itération de Newton est construite pour que $x_{n+1}$ soit l'abscisse d'intersection entre la tangente à $f$ en $x_n$ et l'axe des $x$. \href{https://github.com/LaurentClaessens/phystricks}{Faisons} une petit dessin à la figure \ref{LabelFigMethodeNewton} avec $a<0$ et $b>0$ pour se fixer les idées.

\newcommand{\CaptionFigMethodeNewton}{La méthode de Newton pour une parabole}
\input{Fig_MethodeNewton.pstricks}

Supposons d'abord que $x_n<r_0$. Alors le point est sur la partie décroissante de $P$, et la dérivée est de plus en plus petite (en valeur absolue) jusqu'au sommet $S=(r_0+r_1)/2$. Cela fait que $x_{n+1}>x_n$. La suite des $x_i$ ainsi obtenue est donc croissante.

D'un autre côté, nous avons $P(x_n)>0$ et $P'(x_n)<0$, mais pour tout $x\in\mathopen[ x_n , r_0 \mathclose]$, nous avons $P'(x)<P(x_n)$, donc $x_{n+1}<r_0$. Nous avons donc $x_{n+1}\in\mathopen[ x_n,r_0 ,  \mathclose]$ de telle sorte que la suite des $x_i$ soit bornée et croissante et donc convergente.

Supposons maintenant que $x_n\in\mathopen] r_0,S ,  \mathclose[$. Dans ce cas, nous avons $x_{n+1}<r_0$ et donc on repasse tout de suite au cas précédent.

Regardez ce qu'il se passe quand $x_n=S$ et puis essayez de refaire tous les raisonnements pour les cas où $x_n$ se trouve à droite de $C$.

\end{corrige}
