% This is part of Exercices de mathématique pour SVT
% Copyright (c) 2010
%   Laurent Claessens et Carlotta Donadello
% See the file fdl-1.3.txt for copying conditions.

\begin{corrige}{TD5-0005}

	\begin{enumerate}
		\item
			Cette intégrale se fait par changement de variables en posant $u=t^2$. En ce qui concerne la dérivée,
			\begin{equation}
				\frac{ du }{ dt }=2t
			\end{equation}
			et par conséquent $dt=\frac{ du }{ 2t }$. En remplaçant dans l'intégrale demandée,
			\begin{equation}
				I=\int t e^{-u}\frac{ du }{ 2t }.
			\end{equation}
			Les deux $t$ restants se simplifient et il reste
			\begin{equation}
				\frac{ 1 }{2}\int e^{-u}du=-\frac{1 }{2} e^{-u}+C=-\frac{ 1 }{2} e^{-t^2}+C
			\end{equation}
			où nous avons fait le changement de variable inverse affin d'exprimer la réponse en termes de la variable originale $t$.

			Notez que pour bien obtenir toutes les primitives, nous devons écrire «$+C$» dans la réponse. Cela signifie que pour tout choix de constante $C$, la fonction
			\begin{equation}
				F(t)=-\frac{ 1 }{2} e^{-t^2}+C
			\end{equation}
			est une primitive de la fonction
			\begin{equation}
				f(t)=t e^{-t^2}.
			\end{equation}
		\item
			Nous pouvons couper la fraction en deux :
			\begin{equation}
				\frac{ 3t^2+2 }{ t^3 }=\frac{ 2t^2 }{ t^3 }+\frac{ 2 }{ t^3 }=\frac{ 3 }{ t }+2t^{-3}.
			\end{equation}
			Pour rappel, $t^{-3}=\frac{1}{ t^3 }$. L'intégrale à calculer devient donc
			\begin{equation}
				\begin{aligned}[]
					\int \frac{ 3t^2+2 }{ t^3 }dt=3\int\frac{1}{ t }dt+2\int t^{-3}dt\\
					&=3\ln(t)+2\frac{ t^{-2} }{ -2 }+C\\
					&=3\ln(t)-\frac{1}{ t^2 }+C.
				\end{aligned}
			\end{equation}
	\end{enumerate}

\end{corrige}
