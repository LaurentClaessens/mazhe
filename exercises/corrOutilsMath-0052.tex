% This is part of Exercices et corrigés de CdI-1
% Copyright (c) 2011
%   Laurent Claessens
% See the file fdl-1.3.txt for copying conditions.

\begin{corrige}{OutilsMath-0052}

    Le champ de gravitation est donné par
    \begin{equation}
        F(x,y,z)=\frac{-1}{ (x^2+y^2+z^2)^{3/2} }\begin{pmatrix}
            x    \\ 
            y    \\ 
            z    
        \end{pmatrix},
    \end{equation}
    et le chemin est
    \begin{equation}
        \sigma(t)=\begin{pmatrix}
            R\cos(t)    \\ 
            R\sin(t)    \\ 
            0    
        \end{pmatrix}
    \end{equation}
    où $R$ est le rayon de l'orbite.

    Nous avons
    \begin{equation}
        \sigma'(t)=\begin{pmatrix}
            -R\sin(t)    \\ 
            R\cos(t)    \\ 
            0    
        \end{pmatrix}
    \end{equation}
    et
    \begin{equation}
        \begin{aligned}[]
            F\big( \sigma(t) \big)&=\frac{ -1 }{ \big( R^2\cos^2(t)+R^2\sin^2(t) \big)^{3/2} }\begin{pmatrix}
                R\cos(t)    \\ 
                R\sin(t)    \\ 
                0    
            \end{pmatrix}\\
            &=
            \frac{ R }{ R^3 }\begin{pmatrix}
                \cos(t)    \\ 
                \sin(t)    \\ 
                0    
            \end{pmatrix}\\
            &=\frac{1}{ R^2 }\begin{pmatrix}
                \cos(t)    \\ 
                \sin(t)    \\ 
                0    
            \end{pmatrix}.
        \end{aligned}
    \end{equation}
    Le produit scalaire est facile à calculer :
    \begin{equation}
        F\big( \sigma(t) \big)\cdot \sigma'(t)=0.
    \end{equation}
    Donc le travail est nul, quel que soit le morceau de trajet parcouru. C'est pour cela que les satellites peuvent tourner indéfiniment sans avoir besoin de propulsion. Si ce n'était pas le cas, la Terre serait déjà tombée sur le Soleil depuis longtemps.

\end{corrige}
