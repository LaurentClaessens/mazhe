% This is part of Exercices et corrigés de CdI-1
% Copyright (c) 2011
%   Laurent Claessens
% See the file fdl-1.3.txt for copying conditions.

\begin{corrige}{OptimSS0004}

Une droite verticale n'est solution que si tous les points donnés sont alignés verticalement. Nous cherchons donc une droite oblique sous la forme $y=ax+b$, où $a$ et $b$ sont les paramètres inconnus que nous voulons optimiser.

Le point $(x_i,y_i)$ se projette verticalement sur la droite sur le point $(x_i,ax_i+b)$, et la distance est donc $y_i-ax_i-b$. La fonction à minimiser est donc
\begin{equation}
	f(a,b)=\sum_i(y_i-ax_i-b)^2.
\end{equation}
Les différentielles sont
\begin{equation}
	\begin{aligned}[]
		\partial_af(a,b)&=2\sum_i(y_i-ax_i-b)(-x_i)\\
		\partial_bf(a,b)&=2\sum_i(y_i-ax_i-b)(-1)
	\end{aligned}
\end{equation}
Les points critiques sont donnés par la solution du système d'équation
\begin{subequations}
\begin{numcases}{}
	(\sum_ix_i^2)a+(\sum_ix_i)b=\sum_iy_ix_i\\
	(\sum_ix_i)a+nb=\sum_iy_i.
\end{numcases}
\end{subequations}

\end{corrige}
