% This is part of Exercices et corrigés de CdI-1
% Copyright (c) 2011
%   Laurent Claessens
% See the file fdl-1.3.txt for copying conditions.

\begin{corrige}{0025}



\begin{enumerate}
\item 
 $\lim_{x\rightarrow +\infty} \dfrac{\ln(x)}{x^a}  $

Si $a>0$ alors, par l'Hospital, nous trouvons $\lim_{x\to\infty}\frac{1}{ ax^a }$, dont la limite est nulle. Si $a\leq0$, elle vaut tout aussi clairement l'infini.

\item
  $\lim_{x\rightarrow +\infty} \dfrac{\ln(x)^a}{x^b}  $

Dans le cas où $a$ ou $b$ est nul, le résultat ne fait aucun doute. Si $a$ et $b$ n'ont pas le même signe, le numérateur et le dénominateur vont dans le même sens, et il n'y a aucun problèmes non plus. Il ne reste donc qu'à étudier le cas où $a$ et $b$ sont strictement positifs (le cas négatif se traite en passant à l'inverse). Pour tout $x$ nous avons alors les inégalités
\begin{equation}		\label{Eq0025EtauGrand}
	\frac{ \ln(x)^{\lfloor a\rfloor} }{ x^b }\leq\frac{ \ln(x)^{a} }{ x^b }\leq\frac{ \ln(x)^{\lceil a\rceil} }{ x^b }.
\end{equation}
où $\lfloor a \rfloor$ représente le plafond du réel $a$, et $\lceil a\rceil$, son plancher. On calcule la limite du terme le plus à gauche en appliquant $\lfloor a\rfloor$ fois la règle de l'Hospital. Ce qu'il faut calculer devient :
\[ 
\lim_{x\rightarrow +\infty} \dfrac{a(a-1)(a-2)\ldots(a-(\lfloor a \rfloor-1))\ln(x)^{a-\lfloor a \rfloor}}{b^{\lfloor a \rfloor}x^b} .
\]
On peut maintenant utiliser l'étau, en sachant que pour $x$ suffisamment grand \[0 \leq \dfrac{\ln(x)^a}{x^b} \leq \dfrac{\ln(x)}{x^b}\] et nous avons vu à l'exercice précédent que cette dernière tend vers $0$. Le terme le plus à droite de \eqref{Eq0025EtauGrand} se traite de la même façon, et sa limite est également nulle. La règle de l'étau conclut que
\begin{equation}
	\lim_{x\rightarrow +\infty} \dfrac{\ln(x)^a}{x^b}  =0
\end{equation}
lorsque $a$ et $b$ sont des réels strictement positifs.

\item
 $\lim_{x\rightarrow +\infty} a^x $

Il y a trois cas possibles (on suppose $a$ positif). Quels autres cas y aurait-il eu si on avait admis $a$ négatif?

\begin{enumerate}
	\item 
	$a< 1$.
	Nous avons $x^x<\epsilon$ dès que $x>\log_a(\epsilon)=\frac{ \ln(\epsilon) }{ \ln(a) }$. La dernière expression est plus grande que zéro dès que $\epsilon<1$ et $a<1$.


	\item
	 $a= 1$.
	$\lim_{x\rightarrow +\infty} a^x  \;= \;1$ 

	Trivial

	\item
	 $a> 1$. La dérivée de $x\mapsto a^x$ est $a^x\ln(a)$. Étant donné que $a>1$, nous avons $\ln(a)>0$ et $a^x>1$ (dès que $x$ est assez grand). La dérivée de $a^x$ majore donc $\ln(a)$, et il existe une constante $c$ telle que  la fonction $x\mapsto a^x+c$ soit plus grande que $x\to \ln(a)x$. Donc $\lim_{x\to\infty}a^x=\infty$ lorsque $a>1$.

\begin{alternative}
	Pour le cas $a>1$. Par l'étau: $a^x \; = \; (1+b)^x \; \geq \; 1+nx$ pour $x\geq n$. 
\end{alternative}

\end{enumerate}
\end{enumerate}


\end{corrige}
