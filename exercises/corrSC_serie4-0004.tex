\begin{corrige}{SC_serie4-0004}

	Par définition, si on connait un point $(x_0,y_0)$ du graphe de la solution au problème de Cauchy, la valeur de la dérivée en ce point est donnée par $f'(x_0)=x_0^2+y_0^2/4$. Nous pouvons donc facilement calculer $f'$ sur les abscisses \verb+X+ où Matlab a fournit la solution parce que nous y connaissons le \verb+y+ correspondant.

	De plus, la formule qui donne $f'(x_0)$ en fonction de $x_0$ et $y_0$ est précisément celle qui définit le problème de Cauchy. Pas besoin de la retaper.

\lstinputlisting{SC_exo_4-4.m}

\end{corrige}
