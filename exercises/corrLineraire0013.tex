% This is part of the Exercices et corrigés de mathématique générale.
% Copyright (C) 2009
%   Laurent Claessens
% See the file fdl-1.3.txt for copying conditions.
\begin{corrige}{Lineraire0013}

	\begin{enumerate}

		\item
			oui.
		\item
			Non, et la différence avec le précédent est que la somme de deux polynômes de degré trois peut n'être que de degré deux : $(x^3+x^2)-(x^3+x)=x^2-x$ par exemple.

		\item
			Oui, c'est un plan. Si $(a,b,c)$ et $(a',b',c')$ vérifient la condition, regardons que $(a+a',b+b',c+c')$ vérifie la condition :
			\begin{equation}
				(a+a')+2(b+b')-(c+c')=0.
			\end{equation}
			
		\item
			Non, par exemple $(1,0,0)$ est dans l'ensemble, mais pas $(2,0,0)$ qui est le double.

		\item 
			Non, c'est un plan afin. Par exemple $(2,1)+(5,4)=(7,5)$, mais $(7-5)\neq 1$, alors que $(2,1)$ et $(5,4)$ vérifient la condition.

		\item
			Oui, c'est l'intersection de deux plans. Si $(a,b,c)$ et $(a',b',c')$ vérifient les deux conditions, c'est facile de voir que $(a+a',b+b',c+c')$ vérifie les deux conditions.

		\item
			Non, si $(n,m)\in\eN^2$, il suffit de le multiplier par $\sqrt{2}$ et ce n'est plus dans $\eN^2$.

		\item
			Non, par exemple $(a,b)-(a,b)=(0,0)$ et évidement, $(0,0)$ n'est pas dans l'ensemble.

	\end{enumerate}
	

\end{corrige}
