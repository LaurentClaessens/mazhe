% This is part of Analyse Starter CTU
% Copyright (c) 2014
%   Laurent Claessens,Carlotta Donadello
% See the file fdl-1.3.txt for copying conditions.

\begin{corrige}{autoanalyseCTU-3}

    \begin{wrapfigure}{r}{5.0cm}
   \vspace{-0.5cm}        % à adapter.
   \centering
   \input{Fig_CFMooGzvfRP.pstricks}
\end{wrapfigure}

Les valeurs usuelles et les graphes sont dans la section de rappels \ref{secHTVooJuBtam}.

Pour résoudre \( \sin(x)=\frac{ 1 }{2}\), la première chose est de se souvenir que \( \sin(\pi/6)=\frac{ 1 }{2}\). Il faut ensuite trouver les autres angles du cercle trigonométrique donnant le même sinus. Un dessin peut vraiment aider.

%\begin{center}
%   \input{Fig_CFMooGzvfRP.pstricks}
%\end{center}

Nous avons donc les solutions \( x=\pi/6\) et \( x=5\pi/6\) dans le cercle trigonométrique (c'est à dire entre \( 0\) et \( 2\pi\)). L'ensemble des solutions dans \( \eR\) s'obtient en ajoutant les solutions obtenues par la périodicité de la fonction sinus :
\begin{equation}
    x\in\left\{ \frac{ \pi }{ 6 }+2k\pi\tq k\in \eZ \right\}\cup\left\{ \frac{ 5\pi }{ 6 }+2k\pi\tq k\in \eZ \right\}.
\end{equation}

En ce qui concerne l'équation \( \cos(x)=-\frac{ 1 }{2}\), nous voyons dans les tables de valeurs usuelles que \( \cos(\pi/3)=1/2\). Donc \( \cos(\frac{ 2\pi }{ 3 })=\cos(\frac{ 4\pi }{ 3 })=-\frac{ 1 }{2}\). Les solutions entre \( 0\) et \( 2\pi\) sont donc \( x=2\pi/3\) et \( x=4\pi/3\). Les solutions dans \( \eR\) sont au final
\begin{equation}
    x\in\left\{ \frac{ 2\pi }{ 3 }+2k\pi\tq k\in \eZ \right\}\cup\left\{ \frac{ 4\pi }{ 3 }+2k\pi\tq k\in \eZ \right\}.
\end{equation}

\end{corrige}   
