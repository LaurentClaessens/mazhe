% This is part of Mes notes de mathématique
% Copyright (c) 2011-2012
%   Laurent Claessens
% See the file fdl-1.3.txt for copying conditions.

%---------------------------------------------------------------------------------------------------------------------------
\subsection{Théorème de l'élément primitif}
%---------------------------------------------------------------------------------------------------------------------------

% TODO : fusionner cette définition avec celle du degré.
\begin{definition}
    Soit \( \eK\) un corps. Une extension \( \eL\) de \( \eK\) est dite \defe{finie}{extension!finie} si \( \eL\) est un espace vectoriel de dimension finie sur \( \eK\).
\end{definition}
Notez que la définition d'extension finie ne suppose ni que \( \eK\) ni que \( \eL\) soient finis en tant qu'ensembles.

\begin{theorem}[de l'élément primitif]\index{théorème!élément primitif}
    Si \( \eK\) est un corps fini, toute extension finie de \( \eK\) est simple.

    Si \( \eK\) est un corps quelconque alors toute extension séparable finie est simple.
\end{theorem}

\begin{proof}
    Nous ne donnons la preuve que dans le cas où \( \eK\) est fini. Dans ce cas nous savons par la proposition \ref{PropnfebjI} que le groupe \( \eK^*\) est cyclique. Si de plus \( \eL\) est une extension finie alors \( \eL\) est fini en tant qu'ensemble. Par conséquent \( \eL^*\) est un groupe cyclique. Si \( \alpha\) est un générateur de \( \eL\) alors \( \eL=\eK(\alpha)\) et l'extension est donc simple.

    Une preuve de l'assertion dans le cas où \( \eK\) est infini peut être trouvée sur wikipédia.
\end{proof}

\begin{definition}\label{DefvBFpsY}
    Soit \( P\) un polynôme irréductible de degré \( n\) sur \( \eF_p[X]\). L'\defe{ordre}{ordre!d'un polynôme} de \( P\) est
    \begin{equation}
        \min\{ k\tq P\divides X^k-1 \}.
    \end{equation}

    Soit \( p\), un nombre premier et \( P\) un polynôme unitaire irréductible de degré $n$ dans \( \eF_p[X]\). Nous disons que \( P\) est \defe{primitif}{primitif!polynôme}\index{polynôme!primitif} si les racines de \( P\) sont d'ordre \( p^n-1\) dans \( \eF_p[X]/P\).
\end{definition}

\begin{remark}  \label{RemwwJbYP}
    Si \( \eA\) est un anneau factoriel, \wikipedia{fr}{Polynôme}{il est souvent dit} qu'un polynôme \( P\in \eA[X]\) est primitif si le pgcd de ses coefficients est \( 1\). Cette notion de primitivité n'est apparemment pas reliée à celle de la définition \ref{DefvBFpsY} qui sera celle que nous retiendrons pour la suite. Notons au passage que dans un corps, tous les polynômes non nuls et unitaires sont primitifs au sens du pgcd des coefficients; cette notions de primitivité n'a donc pas d'intérêt dans notre cadre.

    Lorsque nous utiliserons la notion de polynôme primitif au sens du \( \pgcd\), nous le mentionnerons explicitement. Ce sera le cas par exemple dans le théorème \ref{ThofiIpXg}.
\end{remark}

\begin{proposition}
    L'ordre d'un polynôme \( P\) vérifie les propriétés suivantes :
    \begin{enumerate}
        \item
            L'ordre de \( P\) est l'ordre multiplicatif de ses racines
        \item
            L'ordre de \( P\) divise \( p^n-1\).
    \end{enumerate}
\end{proposition}

\begin{lemma}       \label{LemZrUUOz}
    Soit \( p\) un nombre premier et \( P\) un polynôme irréductible unitaire de degré \( n\). Si \( \alpha,\beta\in \eF_p[X]/P\), alors \( (\alpha+\beta)^p=\alpha^p+\beta^p\).
\end{lemma}

\begin{proof}
    La preuve est exactement la preuve classique :
    \begin{equation}
        (\alpha+\beta)^p=\sum_k{k\choose p} A^kB^{p-k}
    \end{equation}
    où les coefficients binomiaux sont dans \( \eF_p\) et donc nuls pour les \( k\) différents de \( p\) et de \( 0\).
\end{proof}

\begin{lemma}
    Si \( \alpha\in \eF_q\) est une racine d'ordre \( k\) de \( P\) (de degré \( n\)) alors les racines de \( X^k-1\) sont \( \{ \alpha^i\tq i=0,\ldots, k-1 \}\).
\end{lemma}

%+++++++++++++++++++++++++++++++++++++++++++++++++++++++++++++++++++++++++++++++++++++++++++++++++++++++++++++++++++++++++++
\section{Corps finis, construction}
%+++++++++++++++++++++++++++++++++++++++++++++++++++++++++++++++++++++++++++++++++++++++++++++++++++++++++++++++++++++++++++

Nous serions donc intéressé à construire $\eF_{q}$ comme quotient de \( \eF_p[X]\) par un polynôme primitif. Le théorème suivant donne une description abstraite de \( \eF_q\) qui va nous servir de point de départ pour la construction.
\begin{theorem}[Théorème de l'élément primitif]\index{théorème!élément primitif}    \label{ThoqSludu}
    Soit \( p\) un nombre premier, \( n\in \eN\) et \( q=p^n\). Soit \( \eK\) un corps à \( q\) éléments. Alors
    \begin{enumerate}
        \item
            Il existe \( \alpha\in \eK\) tel que \( \eK=\eF_p[\alpha]\).
        \item
            Il existe une polynôme irréductible \( P\in\eF_p[X]\) de degré \( n\) tel que
            \begin{equation}        \label{EqWlMhhm}
                \begin{aligned}
                    \phi\colon \eF_p[X]/(P)&\to \eK \\
                    \bar X&\mapsto \alpha 
                \end{aligned}
            \end{equation}
            soit un isomorphisme de corps.
    \end{enumerate}
    Soit \( \alpha\) et \( P\) choisis pour avoir les propriétés citées plus haut. Alors nous avons les propriétés suivantes.
    \begin{enumerate}
        \item
            \( P\) est primitif.
        \item
            \( P\) est scindé dans \( \eK\).
        \item
            L'ensemble des racines de \( P\) est \( \{ \alpha,\alpha^p,\ldots, \alpha^{p^{n-1}} \}\).
        \item
            Le polynôme \( P\) divise \( X^q-X\) dans \( \eF_p[X]\).
    \end{enumerate}
\end{theorem}

\begin{proof}
    Le corps \( \eK\) étant fini, il est cyclique par la proposition \ref{PropnfebjI}. Si \( \alpha\) un générateur de \( \eK^*\) alors
    \begin{equation}
        \eK=\eF_p[\alpha].
    \end{equation}
    Soit \( \ell\) le plus grand entier tel que l'ensemble
    \begin{equation}
        \{ 1,\alpha,\cdots,\alpha^{\ell-1} \}\subset\eK
    \end{equation}
    soit libre. Pour rappel \( \eK\) est une espace vectoriel sur \( \eF_p\). Il existe des \( a_i\in \eF_p\) tels que
    \begin{equation}
        \alpha^{\ell}+a_{\ell-1}\alpha^{\ell-1}+\ldots+a_0=0.
    \end{equation}
    De façon équivalente, il existe un polynôme unitaire \( P\in\eF_p[X]\) de degré \( \ell\) tel que \( P(\alpha)=0\). Étant donné que \( \alpha\) est générateur de \( \eK\),
    \begin{equation}
        \eK=\Span\{ 1,\alpha,\ldots, \alpha^{\ell-1} \}
    \end{equation}
    parce que \( \eK\) est généré par les puissances de \( \alpha\) alors que les puissances de \( \alpha\) plus hautes que \( \ell-1\) peuvent être générées par \( 1,\alpha,\ldots, \alpha^{\ell-1}\). L'espace \( \eK\) est donc un \( \eF_p\)-espace vectoriel de dimension \( \ell\); par conséquent
    \begin{equation}
        \Card(\eK)=p^n=q
    \end{equation}
    et \( \ell=n\).

    Montrons que \( P\) est irréductible dans \( \eF_p\). Si \( P\) était réductible dans \( \eF_p\), l'élément \( \alpha\in \eK\) serait une racine d'un des facteurs, c'est à dire qu'il serait racine d'un polynôme de degré inférieur à \( n\), ce qui contredirait le fait que 
    \begin{equation}
        \{ \alpha^{\ell-1},\ldots, 1 \}
    \end{equation}
    est libre.

    Montrons que l'application
    \begin{equation}
        \begin{aligned}
            \phi\colon \eF_p[X]/(P)&\to \eK \\
            \bar X&\mapsto \alpha 
        \end{aligned}
    \end{equation}
    est un isomorphisme. Pour l'injectivité, deux éléments \( Q_1,Q_2\in \eF_p[X]/(P)\) s'écrivent
    \begin{subequations}
        \begin{align}
            Q_1&=\sum_{k=0}^{n-1}a_k\bar X^k\\
            Q_2&=\sum_{k=0}^{n-1}b_k\bar X^k.
        \end{align}
    \end{subequations}
    Dans ce cas si \( \phi(Q_1)=\phi(Q_2)\) alors
    \begin{equation}
        \phi(Q_1)=\sum_{k=0}^{n-1}a_k\alpha^k=\phi(Q_2)=\sum_{k=0}^{n-1}b_k\alpha^k.
    \end{equation}
    Mais l'ensemble \( \{ 1,\alpha,\ldots, \alpha^{n-1} \}\) étant libre sur \( \eF_p\), cela implique \( a_k=b_k\). La surjectivité de \( \phi\) provient du fait que \( \alpha\) génère \( \eK\).

    Nous passons maintenant à la seconde partie de la démonstration. Soient \( \alpha\in \eK\) tel que \( \eK=\eF_p[\alpha]\) et \( P\in \eF_p[X]\) un polynôme irréductible de degré \( n\) tel que \( \alpha\mapsto \bar X\) soit un isomorphisme entre \( \eK\) et \( \eF_p[X]/(P)\).

    Le polynôme \( P\) est primitif parce que \( \alpha\) est d'ordre \( p^n\) dans \( \eK\) alors que \( \bar X\mapsto \alpha\) est un isomorphisme. Par conséquent \( \bar X\) est d'ordre \( p^n\) dans \( \eF_p[X]/P\).

    Nous commençons par prouver que l'ensemble
    \begin{equation}        \label{EqAcsQHL}
        \{ \alpha,\alpha^p,\alpha^{p^2},\ldots, \alpha^{p^{n-1}} \}
    \end{equation}
    est l'ensemble des racines distinctes de \( P\). Pour cela nous posons
    \begin{equation}
        P(X)=\sum_{k=0}^na_kX^k
    \end{equation}
    avec \( a_k\in\eF_p\). D'abord \( \alpha\) est une racine de \( P\). En effet
    \begin{equation}        \label{EqbTAmKG}
        P(\bar X)=\sum_ka_k\bar X^k=0
    \end{equation}
    parce que cette somme est calculée dans \( \eF_p[X]/(P)\). En appliquant l'isomorphisme \( \phi\) à l'égalité \eqref{EqbTAmKG} nous trouvons
    \begin{equation}
        0=\phi\big( P(\bar X) \big)=\sum_ka_k\phi(\bar X^k)=\sum_ka_k\alpha^k.
    \end{equation}
    Donc \( \alpha\) est bien une racine de \( P\) dans \( \eF_p[X]\). Nous devons montrer qu'il en est de même pour les autres puissances dans l'ensemble \eqref{EqAcsQHL}. Étant donné que pour tout \( x\) dans \( \eF_p\) nous avons \( x^p=x\), nous avons aussi
    \begin{equation}
        P(X^p)=\sum_ka_k(X^p)^k=\sum_ka_k^p(X^p)^k=\sum_k(a_kX^k)^p
    \end{equation}
    alors que nous savons que \( x\mapsto x^p\) est un automorphisme de \( \eF_p\) par la proposition \ref{PropFrobHAMkTY}. Par conséquent
    \begin{equation}
        P(X^p)=\sum_k(a_kX^k)^p=\left( \sum_k a_kX^k\right)^p=P(X)^p.
    \end{equation}
    Nous avons montré que si \( \beta\) est une racine de \( P\), alors \( \beta^p\) est également une racine de \( P\). Nous savons déjà que \( \alpha\) est une racine de \( P\), et que \( \alpha\) est également générateur de \( \eK\), c'est à dire que \( \alpha\) est d'ordre \( q-1\). Les puissances
    \begin{equation}
        \alpha,\alpha^p,\alpha^{p^2},\ldots, \alpha^{p^{n-1}}
    \end{equation}
    sont donc distinctes (\( \alpha^{p^n}=\alpha^q=1\)) et sont toutes des racines de \( P\). Étant donné que \( P\) est de degré \( n\) il ne peut pas y avoir d'autres racines. Nous concluons que l'ensemble
    \begin{equation}
        \{ \alpha,\alpha^p,\alpha^{p^2},\ldots, \alpha^{p^{n-1}}\}
    \end{equation}
    est l'ensemble des racines distinctes de \( P\) dans \( \eK\). Le polynôme \( P\) est alors scindé dans \( \eK[X]\).

    Le dernier point du théorème est de montrer que \( P\) divise \( X^q-X\). Pour cela nous allons montrer que toutes les racines de \( P\) sont des racines de \( X^q-X\). Soit \( \beta\) une racine de \( P\); il s'écrit \( \beta=\alpha^k\) pour un certain \( k\). Étant donné que \( \alpha^{q-1}=e=\alpha^{p^n-1}\),
    \begin{subequations}
        \begin{align}
            \beta^q&=(\alpha^{p^n})^k\\
            &=\left( \alpha^{p^n-1}\alpha \right)^k\\
            &=\left( \alpha^{q-1}\alpha \right)^k\\
            &=\alpha^k\\
            &=\beta.
        \end{align}
    \end{subequations}
    Cela signifie que \( \beta^q=\beta\) et donc que \( \beta\) est racine de \( X^q-X\).
\end{proof}

\begin{corollary}
    Le corps fini à \( q=p^n\) éléments est de caractéristique \( p\).
\end{corollary}

\begin{proof}
    Nous considérons le corps fini \( \eK\) à \( q\) éléments sous la forme \( \eK=\eF_p[X]/P\) comme indiqué par l'équation \eqref{EqWlMhhm}. Soit \( 1_q\) la classe du polynôme \( 1\) modulo \( P\), nous considérons le morphisme
    \begin{equation}
        \begin{aligned}
            \mu\colon \eZ&\to \eF_q \\
            n&\mapsto n1_q. 
        \end{aligned}
    \end{equation}
    Le noyau de cette application est \( \ker\mu=\eZ_p\) parce que \( p1_q=0\), les coefficients étant à comprendre dans \( \eF_p\).
\end{proof}

\begin{definition}  \label{DefnPNCFO}
    Soit \( P\), un polynôme de degré \( n\) et \( p\), un nombre premier. Un élément \( \alpha\in \eF_p[X]/(P)\) est une \defe{racine primitive}{racine!primitive}\index{primitif!racine} si les puissances de \( \alpha\) parcourent tout le groupe multiplicatif \( (\eF_p[X]/P)^*\).
\end{definition}

\begin{lemma}       \label{Lembcerei}
    Soit \( p\) un nombre premier et \( P\), un polynôme de degré \( n\). Si \( \alpha\in \eF_p[X]/P\) est une racine primitive de \( P\) alors les autres racines de \(P\) sont également primitives.
\end{lemma}

\begin{proof}
    Soit \( \alpha\in \eF_p[X]/P\) une racine primitive de \( P\). L'élément \( \alpha^p\) est également une racine parce que si \( P=\sum_ka_kX^k\),
    \begin{equation}
        P(\alpha^p)=\sum_k(a_k\alpha^k)^p=\big( \sum_ka_k\alpha^k \big)^p=0
    \end{equation}
    où nous avons utilisé le fait que \( a_k^p=a_k\) étant donné que \( a_k\in\eF_p\). Par hypothèse \( \alpha\) est une racine primitive; cela implique que les éléments \( \alpha,\alpha^p,\alpha^{p^2},\ldots,\alpha^{p^n-1}\) sont distincts dans \( \eF_p[X]/P\). Ces éléments constituent donc \emph{toutes} les racines de \( P\).

    Soit \( \beta=\alpha^{p^k}\) une racine de \( P\). Montrons que \( \alpha\) est une puissance de \( \beta\). Étant donné que \( (\eF_p[X]/P)^*\) est un groupe à \( p^n-1\) éléments, le corollaire \ref{CorpZItFX} indique que \( \alpha^{p^n}=\alpha\). En particulier avec \( r=p^{n-k}\) nous avons
    \begin{equation}
        \beta^r=\alpha^{rp^k}=\alpha^{p^n}=\alpha.
    \end{equation}
    Par suite toutes les puissances de \( \alpha\) sont des puissances de \( \beta\), ce qui implique que \( \beta\) est générateur du groupe cyclique \( (\eF_p[X]/P)^*\).
\end{proof}

\begin{lemma}       \label{LemkzWjse}
    Soit \( p\) un nombre premier et \( n\), un entier. Un polynôme de degré \( d\) irréductible dans \( \eF_p[X]\) divise \( X^{p^n}-X\) si et seulement si \( d\) divise \( n\).
\end{lemma}

\begin{theorem}
    Soient \( P\) et \( Q\) deux polynômes irréductibles de degré \( n\) dans \( \eF_p[X]\). Alors les quotients \( \eF_p[X]/P\) et \( \eF_p[X]/Q\) sont isomorphes en tant que corps.
\end{theorem}
En guise de démonstration de ce théorème, nous allons démontrer la proposition suivante.
\begin{proposition}
    Si \( \eK\) et \( \eL\) sont deux corps à \( q=p^n\) éléments, alors ils sont isomorphes.
\end{proposition}

\begin{proof}
    Soit \( a\) un élément primitif de \( \eK\) et \( P\) son polynôme minimal. Nous savons que \( \eK\simeq \eF_p[X]/P\) par le théorème de l'élément primitif \ref{ThoqSludu}. L'élément \( a\) est en particulier une racine de \( X^q-X\). Par ailleurs \( P\) divise \( X^q-X\) par le lemme \ref{LemkzWjse}.

    Nous avons aussi
    \begin{equation}
        X^q-X=\prod_{b\in \eL}(X-b)
    \end{equation}
    par la proposition \ref{propQRcUlq}. Étant donné que \( P\) divise \( X^q-X\), un des éléments de \( \eL\) annule \( P\). Soit \( b\in \eL\) tel que \( P(b)=0\). Soit \( Q\) le polynôme minimal de \( b\). Par définition nous avons que \( Q\) divise \( P\), mais \( P\) étant irréductible et unitaire nous avons immédiatement \( P=Q\). En particulier nous avons
    \begin{equation}
        \eF_p[X]/P\simeq\eF_p[X]/Q\simeq \eK.
    \end{equation}
    Nous montrons maintenant que \( \eF_p[X]/Q\simeq \eL\) par l'application
    \begin{equation}
        \begin{aligned}
            \phi\colon \eF_p[X]/Q&\to \eL \\
            \bar X&\mapsto b 
        \end{aligned}
    \end{equation}
    qui se prolonge en \( R(\bar X)\mapsto R(b)\) pour tout \( R\in \eF_p[X]\). Cette application est bien définie parce que \( Q(b)=0\). Elle est injective parce que \( R(b)=0\) ne peut pas avoir lieu avec \( R\in \eF_p[X]/Q\) parce que \( Q\) est le polynôme minimal de \( b\). La surjectivité vient alors du fait que les deux corps ont le même nombre d'éléments.
\end{proof}

%---------------------------------------------------------------------------------------------------------------------------
\subsection{Construction de $\eF_q$}
%---------------------------------------------------------------------------------------------------------------------------

Soit \( p\) un nombre premier et \( n\in \eN\). Nous souhaitons construire un corps à \( q=p^n\) éléments. Nous savons déjà que ce corps est unique (théorème \ref{ThoOzgSfy}) et que nous le notons \( \eF_q\). Le théorème \ref{ThoqSludu} nous incite à l'écrire sous la forme
\begin{equation}
    \eF_q=\eF_p[X]/(P)
\end{equation}
pour un certain polynôme irréductible \( P\in\eF_p[X]\).

%///////////////////////////////////////////////////////////////////////////////////////////////////////////////////////////
\subsubsection{La version du faignant}
%///////////////////////////////////////////////////////////////////////////////////////////////////////////////////////////

Nous pouvons construire le corps à \( p^n\) éléments en prenant le quotient de \( \eF_p[X]\) par n'importe quel polynôme irréductible de degré \( n\). Le résultat est le suivant.
\begin{proposition} \label{PropHfrNCB}
    Soit \( P\) une polynôme unitaire irréductible dans \( \eF_p[X]\). Nous posons \( \eK=\eF_p[X]/(P)\). Alors
    \begin{enumerate}
        \item
            \( \eK\) est un corps à \( q\) éléments.
        \item
            \( \alpha=\bar X\) est une racine de \( P\) dans \( \eK\).
        \item   \label{ItemiEFRTg}
            \( \eK=\eF_p[\alpha]\).
    \end{enumerate}
\end{proposition}

\begin{proof}
    \begin{enumerate}
        \item
            En vertu du corollaire \ref{CorsLGiEN}, \( \eK\) est un corps. Il est aussi un espace vectoriel de dimension \( n\) sur \( \eF_p\), et contient donc \( p^n=q\) éléments.
        \item
            Nous avons \( P(\bar X)=0\) par construction de \( \eK=\eF_p[X]/(P)\).
        \item
            En tant que quotient de \( \eF_p[X]\), les éléments de \( \eK\) sont des polynômes en \( \bar X\).
    \end{enumerate}
\end{proof}

%///////////////////////////////////////////////////////////////////////////////////////////////////////////////////////////
\subsubsection{La version plus élaborée}
%///////////////////////////////////////////////////////////////////////////////////////////////////////////////////////////

Construire \( \eF_q\) comme quotient de \( \eF_p[X]\) par un polynôme irréductible quelconque ne donne pas d'informations sur les générateurs de \( \eF_q^*\), et en particulier il n'est pas toujours vrai que \( \bar X\) est générateur.

\begin{example}     \label{ExemWUdrcs}
    Cherchons à construire \( \eF_{16}\) comme quotient de \( \eF_2\) par un polynôme de degré \( 4\).
    \begin{verbatim}
----------------------------------------------------------------------
| Sage Version 4.7.1, Release Date: 2011-08-11                       |
| Type notebook() for the GUI, and license() for information.        |
----------------------------------------------------------------------
sage: x=polygen(GF(2))
sage: -x-1
x + 1
sage: Q=x**15-1
sage: Q.factor()
(x + 1) * (x^2 + x + 1) * (x^4 + x + 1) * (x^4 + x^3 + 1) * (x^4 + x^3 + x^2 + x + 1)
    \end{verbatim}
    Les polynômes candidats a avoir des racines génératrices sont donc au nombre de \( 3\):
    \begin{subequations}
        \begin{align}
            P_1&=X^4+X+1\\
            P_2&=X^4+X^3+1\\
            P_3&=X^4+X^3+X^2+X+1.
        \end{align}
    \end{subequations}
    Dans le quotient \( \eF_2[X]/P_3\), l'élément \( \bar X\) n'est pas générateur. En effet nous avons \( X^4=X^3+X^2+X+1\) et par conséquent les puissances successives de \( X\) sont
    \begin{subequations}
        \begin{align}
            X&\\
            X^2&\\
            X^3&\\
            X^4&=X^3+X^2+X+1\\
            1.
        \end{align}
    \end{subequations}
    La classe de \( X\) dans \( \eF_2[X]/P_3\) n'est donc pas génératrice du groupe \( (\eF_2[X]/P_3)^*\).

    Le polynôme \( P_1=X^4+X+1\) par contre est primitif parce que les puissances de \( X\) dans \( \eF_2[X]/P_1\) sont
    \begin{subequations}
        \begin{align}
            X\\
            X^2\\
            X^3\\
            X+1\\
            X^2+X\\
            X^3+X^2\\
            X+1+X^3\\
            X^2+1\\
            X^3+X\\
            X+1+X^2\\
            X^2+X+X^3\\
            X^3+X^2+X+1\\
            1+X^2+X^2\\
            1+X^3\\
            1
        \end{align}
    \end{subequations}
    Cela fait \( 15\) puissances distinctes, ce qui prouve que \( P_1\) est primitif. Nous verrons plus loin comment alléger un peu la vérification de la primitivité de \( P_1\).
\end{example}

\begin{proposition}     \label{PropNsLqWb}
    Soit \( P\) un polynôme irréductible unitaire primitif dans \( \eF_p[X]\). Nous considérons \( \eK=\eF_p[X]/P\) et \( \alpha=\bar X\in \eK\). Alors
    \begin{enumerate}
        \item
            Les racines de \( P\) sont \( \{ \alpha,\alpha^p,\ldots, \alpha^{p^{n-1}} \}\) et \( \alpha^q=\alpha\).
        \item
            \( P\) est le polynôme minimal de \( \alpha\).
        \item
            \( P\) est scindé dans \( \eK\).
        \item
            \( P\) divise \( X^q-X\) dans \( \eK\).
        \item
            La famille \( \{1, \alpha,\alpha^2,\ldots, \alpha^{n-1} \}\) est une base de \( \eK\) en tant qu'espace vectoriel sur \( \eF_p\).
        \item
            En tant qu'ensemble,
            \begin{equation}
                \eF_q=\{0, \alpha,\alpha^2,\alpha^3,\ldots, \alpha^{q-1} \},
            \end{equation}
            et les \( \alpha^k\) sont distincts pour \( k=1,\ldots, q-1\).
    \end{enumerate}
\end{proposition}

\begin{proof}
    La plupart des assertions sont des corollaires ou des paraphrases de résultats contenus dans les propositions précédentes.
    \begin{enumerate}
        \item
            L'assertion à propos des racines de \( P\) est contenue dans le lemme \ref{Lembcerei}. D'autre part le groupe \( (\eF_p[X]/P)^*\) est cyclique d'ordre \( q-1\). Par conséquent le corollaire \ref{CorpZItFX} indique que \( \alpha^{q-1}=1\) et donc \( \alpha^q=\alpha\).
        \item
            Soit \( \tilde P\) un polynôme annulateur de \( \alpha\). Nous voyons que si \( \beta\) est racine de \( \tilde P\) alors \( \beta^p\) est également racine de \( \tilde P\) en utilisant les techniques habituelles. Par conséquent toutes les racines de \( P\) sont racines de \( \tilde P\), ce qui implique que \( \tilde P\) est de degré au moins égal à celui de \( P\).
        \item
            Possédant \( n\) racines distinctes dans \( \eK\), le polynôme \( P\) est scindé.
        \item
            D'après le lemme \ref{propQRcUlq} un polynôme irréductible de degré \( n\) divise le polynôme \( X^{p^n}-X\). Une autre façon de montrer ce point est de remarquer que le polynôme \( P\) est scindé et que toutes ses racines sont également racines de \( X^q-X\).
        \item
            Une combinaison linéaire nulle entre les éléments de \( \{ 1,\alpha,\alpha^2,\ldots, \alpha^{n-1} \}\) serait un polynôme annulateur de degré \( n-1\) de \( \alpha\). Cet ensemble est donc libre. Par ailleurs un ensemble libre de \( n\) éléments dans un espace vectoriel de dimension \( n\) est générateur.
        \item
            Si \( \alpha^l=\alpha^k\) avec \( k<l\) et \( k,l\leq q\) alors nous avons \( \alpha^r=1\) avec \( r=l-k<q\), ce qui contredirait la primitivité de \( P\). Les éléments \( 0,\alpha,\ldots, \alpha^{q-1}\) étant distincts et au nombre de \( q\), ils forment tout l'ensemble \( \eF_q\).

    \end{enumerate}
\end{proof}

%---------------------------------------------------------------------------------------------------------------------------
\subsection{Exemple : étude de \texorpdfstring{$\eF_{16}$}{F16}}
%---------------------------------------------------------------------------------------------------------------------------

Dans cette sous section nous voulons construire \( \eF_{16}\). Nous considérons donc \( p=2\) et \( n=4\). Des polynôme irréductibles de degré \( 4\) dans \( \eF_2[X]\) ne sont pas très difficiles à trouver. Par exemple \( X^4+X^3+X^2+X+1\), plus généralement un polynôme contenant un nombre impair de termes non nuls dont le terme indépendant.

Les polynômes primitifs par contre doivent être trouvés parmi les diviseurs irréductibles de \( X^{15}-1\). Montrons que
\begin{equation}
    P=X^4+X^3+1
\end{equation}
est primitif. Nous posons \( \omega=\bar X\in \eF_2[X]/P\). L'ordre de \( \omega\) dans le groupe \( (\eF_2[X]/P)^* \) doit être un diviseur de \( 15\) et donc seulement \( 1\), \( 3\), \( 5\) ou \( 15\). Le fait que l'ordre ne soit ni \( 1\) ni \( 3\) est trivial parce que le degré de \( P\) est \( 4\). Montrons que l'ordre de \( \omega\) n'est pas \( 5\) non plus :
\begin{equation}
    \omega^5=\omega^4\omega=(\omega^3+1)\omega=\omega^4+\omega=\omega^3+\omega+1\neq 1.
\end{equation}
Dans ce calcul nous avons abondamment utilisé le fait que \( -1=1\).

À partir de maintenant nous posons \( \eK=\eF_2[X]/P\). Les racines de \( P\) sont \( \omega,\omega^2,\omega^4\) et \( \omega^8\). En effet si \( \beta\) est une racine de \( P\), alors \( \beta^2\) est une racine en vertu de 
\begin{equation}
    P(\beta^2)=(\beta^2)^4+(\beta^2)^3+1=(\beta^4)^2+(\beta^3)^2+1^2=(\beta^4+\beta^3+1)^2=0.
\end{equation}
Ici nous avons implicitement utilisé le lemme \ref{LemZrUUOz}. D'autre part \( P\) ne peut pas avoir plus de \( 4\) racines.

\begin{proposition}
    L'ensemble \( \{ \omega,\omega^2,\omega^4,\omega^8 \}\) est une base de \( \eF_{16}\) sur \( \eF_2\).
\end{proposition}

\begin{proof}
    Nous savons que \( \{ 1,\omega,\omega^2,\omega^3 \}\) est une base. En effet cet ensemble est libre (sinon \( \omega\) aurait un polynôme annulateur de degré \( 3\)) et générateur parce que l'espace engendré par \( 4\) vecteurs indépendants sur \( \eF_2\) contient \( 2^4=16\) éléments.

    Nous posons \( e_0=1\), \( e_1=\omega\), \( e_2=\omega^2\), \( e_3=\omega^3\) et \( f_1=\omega\), \( f_2=\omega^2\), \( f_3=\omega^4\), \( f_4=\omega^8\). En utilisant le calcul modulo \( \omega^4+\omega^3+1=0\) et \( 2=0\) nous trouvons
    \begin{subequations}
        \begin{align}
            f_1&=\omega\\
            f_2&=\omega^2\\
            f_3&=\omega^3+1\\
            f_4&=\omega^3+\omega^2+\omega.
        \end{align}
    \end{subequations}
    Ensuite nous montrons que les vecteurs \( e_i\) peuvent être construits comme combinaisons linéaires des vecteurs \( f_j\) :
    \begin{subequations}
        \begin{align}
            f_1+f_2+f_3+f_4&=e_0\\
            f_1&=e_1\\
            f_2&=e_2\\
            f_1+f_2+f_4&=e_3.
        \end{align}
    \end{subequations}
    Les quatre vecteurs \( f_j\) forment donc bien un base parce qu'ils sont générateurs d'un espace de dimension \( 4\).
\end{proof}

\Exo{reserve0003}

%---------------------------------------------------------------------------------------------------------------------------
\subsection{Matrices}
%---------------------------------------------------------------------------------------------------------------------------

\begin{proposition}
    Nous avons
    \begin{equation}
        | \GL(n,\eF_p) |=(p^n-1)(p^n-p)\ldots (p^n-p^{n-1}).
    \end{equation}
\end{proposition}

\begin{proof}
    Par construction il existe une bijection entre \( \GL(n,\eF_p)\) et l'ensemble des bases de \( \eF_p^n\). Nous devons donc seulement compter le nombre de bases. Pour le premier vecteur de base nous avons le choix entre les \( p^n-1\) éléments non nuls de \( \eF_p^n\). Pour le second nous avons le choix entre \( p^n-p\) éléments, et ainsi de suite.
\end{proof}

%+++++++++++++++++++++++++++++++++++++++++++++++++++++++++++++++++++++++++++++++++++++++++++++++++++++++++++++++++++++++++++
\section{Mini introduction aux nombres \texorpdfstring{p}{$p$}-adiques}
%+++++++++++++++++++++++++++++++++++++++++++++++++++++++++++++++++++++++++++++++++++++++++++++++++++++++++++++++++++++++++++


\subsection{La flèche d'Achille}\label{s:un}

C'est un grand classique que je donne ici juste comme introduction pour montrer que des série infinies peuvent donner des nombres finis de manière tout à fait intuitive.

Achille tire une flèche vers un arbre situé à $\unit{10}{\meter}$ de lui. Disons que la flèche avance à une vitesse constante de $\unit{1}{\meter\per\second}$. Il est clair que la flèche mettra $\unit{10}{\second}$ pour toucher l'arbre. En $\unit{5}{\second}$, elle aura parcouru la moitié de son chemin. On le note :
\[
\text{temps}=5s+\ldots
\]
Reste \( \unit{5}{\meter}\) à faire. En $\unit{2.5}{\second}$, elle aura fait la moitié de ce chemin chemin, soit $2.5m=\frac{10}{4}m$. On le note :
\[
\text{temps}=\frac{10}{2}s+\frac{10}{4}s+
\]
Reste $2.5m$ à faire. La moitié de ce trajet, soit $\frac{10}{8}m$, est parcouru en $\frac{10}{8}s$; on le note encore, mais c'est la dernière fois !

\[
\text{temps}=\frac{10}{2}s+\frac{10}{4}s+\frac{10}{8}s+
\]
En continuant ainsi à regarder la flèche qui parcours des demi-trajets puis des demi de demi-trajets et encore des demi de demi de demi-trajets, et en sachant que le temps total est $10s$, on trouve :
\[
10\left( \frac{1}{2}+\frac{1}{4}+\frac{1}{8}+\frac{1}{16}+\ldots  \right)=10.
\]
On doit donc croire que la somme jusqu'à l'infini des inverse des puissances de deux vaut $1$ :
\[
   \sum_{n=1}^{\infty}\frac{1}{2^n}=1.
\]
Cela peut être démontré à la loyale.

\subsection{La tortue et Achille}

Maintenant qu'on est convaincu que des sommes infinies peuvent représenter des nombres tout à fait normaux, passons à un truc plus marrant.

Achille, qui marche peinard à $\unit{10}{\meter\per\hour}$, part avec $1m$ d'avance sur une tortue qui avance à $\unit{1}{\meter\per\hour}$. Le temps que la tortue arrive au point de départ d'Achille, Achille aura parcouru $10m$, et le temps que la tortue mettra pour arriver à ce point, eh bien, Achille ne sera déjà plus là : il sera à $100m$. Si la tortue tient bon pendant un temps infini, et si l'on est confiant en le genre de raisonnements faits à la section \ref{s:un}, elle rattrapera Achille dans 
\[
1m+10m+100m+1000m+\ldots
\]
Autant dire que ça ne risque pas d'arriver. Et pourtant, mettons en équations : 
\begin{subequations}
    \begin{numcases}{}
        x_{\text{Achile}}(t)=1+10t\\
        x_{\text{tortue}}(t)=t.
    \end{numcases}
\end{subequations}
La tortue rejoints Achille au temps \( t\) tel que \( x_{\text{Achille}(t)}=x_{\text{tortue}}(t)\). Un mini calcul donne $t=-1/9$. Physiquement, c'est une situation logique. Peut-on en déduire une égalité mathématique du style de 
\[
1+10+100+1000+\ldots=-\frac{1}{9}\; ???
\]
Là où les choses deviennent jolies, c'est quand on cherche à voir ce que peut bien être la valeur d'un hypothétique $x=1+10+100+1000+\ldots$. En effet, logiquement on devrait avoir
\begin{equation*}
\begin{split}
\frac{x}{10}&=\frac{1}{10}+1+10+100+\ldots\\
            &=\frac{1}{10}+x.
\end{split}
\end{equation*}
Reste à résoudre l'équation du premier degré : $\frac{x}{10}=x+\frac{1}{10}$. Ai-je besoin de donner la solution ?

%---------------------------------------------------------------------------------------------------------------------------
\subsection{Dans les nombres \texorpdfstring{p}{$ p$}-adiques, c'est vrai}
%---------------------------------------------------------------------------------------------------------------------------

Nous nous proposons d'apprendre sur les nombres \( p\)-adiques juste ce qu'il faut pour montrer que l'égalité
\begin{equation}
    \sum_{k=0}^{\infty}10^k=-\frac{1}{ 9 }
\end{equation}
est vraie dans les nombres \( 5\)-adiques. Tout ce qu'il faut est sur \wikipedia{fr}{Nombre_p-adique}{wikipedia}.

Soit \( a\in \eN\) et \( p\), un nombre premier. La \defe{valuation}{valuation!$p$-adique} \( p\)-adique de \( a\) est l'exposant de \( p\) dans la décomposition de \( a\) en nombres premiers. On la note \( v_p(a)\). Pour un rationnel on définit
\begin{equation}
    v_p\left( \frac{ a }{ b } \right)=v_p(a)-v_p(b)
\end{equation}
La \defe{valeur absolue}{valeur absolue!$p$-adique} \( p\)-adique de \( r\in \eQ\) est 
\begin{equation}
    | r |_p=p^{-v_p(r)}.
\end{equation}
Nous posons \( | 0 |_p=0\). De là nous considérons la distance
\begin{equation}
    d_p(x,y)=| x-y |_p.
\end{equation}

\begin{lemma}
    L'espace \( (\eQ,d_p)\) est un espace métrique.
\end{lemma}

Nous considérons maintenant \( p=5\). Étant donné que \( a=5\cdot 2\) nous avons \( v_5(10)=1\) et
\begin{equation}
    v_5\left( \frac{1}{ 9 } \right)=v_5(1)-v_5(9)=0.
\end{equation}
Nous avons
\begin{equation}
    \sum_{k=0}^N10^k+\frac{1}{ 9 }=\frac{ 10^{N-1} }{ 9 }
\end{equation}
mais
\begin{equation}
    v_p\left( \frac{ 10^{N-1} }{ 9 } \right)=v_5(10^{N-1})-v_5(9)=N-1.
\end{equation}
Par conséquent
\begin{equation}
    d_5\big( \sum_{k=0}^N,-\frac{1}{ 9 } \big)=| \frac{ 10^{N-1} }{ 9 } |_p=p^{-(N-1)}.
\end{equation}
En passant à la limite,
\begin{equation}
    \lim_{N\to \infty} d_5\big( \sum_{k=0}^N,-\frac{1}{ 9 } \big)=0,
\end{equation}
ce qui signifie que
\begin{equation}
    \sum_{k=0}^{\infty}10^k=-\frac{1}{ 9 }.
\end{equation}
