% This is part of Outils mathématiques
% Copyright (c) 2012
%   Laurent Claessens
% See the file fdl-1.3.txt for copying conditions.

\begin{corrige}{OutilsMath-0149}

    La première intégrale est
    \begin{equation}
        \int_0^1\left( \int_1^3(1-x)^2y \right)dx=\left( \int_0^1(1-x)^2 \right)\left( \int_1^3ydy \right)=-4\int_1^0u^2du=\frac{ 4 }{ 3 }
    \end{equation}
    où nous avons utilisé le changement de variable \( u=1-x\).

    %TODO : faire un dessin de la seconde intégrale

    En ce qui concerne la seconde intégrale, les bornes sont
    \begin{subequations}
        \begin{numcases}{}
            x\colon 0\to 1\\
            y\colon \sqrt{x}\to 1,
        \end{numcases}
    \end{subequations}
    et l'intégrale à calculer est
    \begin{equation}
        \int_0^1dx\int_{\sqrt{x}}^1dy (1-x)^2y=\frac{1}{ 8 }.
    \end{equation}

\end{corrige}
