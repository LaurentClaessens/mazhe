%+++++++++++++++++++++++++++++++++++++++++++++++++++++++++++++++++++++++++++++++++++++++++++++++++++++++++++++++++++++++++++
\section{Avertissement}
%+++++++++++++++++++++++++++++++++++++++++++++++++++++++++++++++++++++++++++++++++++++++++++++++++++++++++++++++++++++++++++

Ceci sont des notes «prises au vol» de certains de mes cours pour l'agrégation. Aucune garantie. Merci de me signaler toute faute ou remarque.

Une version avec ISBN de ce document est en préparation afin qu'il soit acceptable comme livre aux oraux de l'agrégation \ldots j'y travaille et on verra comment les choses vont évoluer.

%+++++++++++++++++++++++++++++++++++++++++++++++++++++++++++++++++++++++++++++++++++++++++++++++++++++++++++++++++++++++++++
\section{Auteurs}
%+++++++++++++++++++++++++++++++++++++++++++++++++++++++++++++++++++++++++++++++++++++++++++++++++++++++++++++++++++++++++++

Le principal auteur et metteur en \LaTeX\ de ce document est votre serviteur, Laurent Claessens.

D'autres ont participé.
\begin{enumerate}
    \item
        Nicolas Richard et Ivik Swan pour les parties des exercices et rappels de calcul différentiel et intégral (Université libre de Bruxelles, 2003-2004) qui leur reviennent.
\end{enumerate}

%+++++++++++++++++++++++++++++++++++++++++++++++++++++++++++++++++++++++++++++++++++++++++++++++++++++++++++++++++++++++++++
					\section*{Ces notes sont les vôtres !}
%+++++++++++++++++++++++++++++++++++++++++++++++++++++++++++++++++++++++++++++++++++++++++++++++++++++++++++++++++++++++++


Il y a encore certainement des erreurs, des fautes de frappe et des choses pas claires. Je compte sur vous (oui : toi !) pour me signaler toute imperfection (y compris d'orthographe).

Plus vous signalez de fautes, meilleure sera la qualité du texte, et plus les étudiants de l'année prochaine vous seront reconnaissants.


%+++++++++++++++++++++++++++++++++++++++++++++++++++++++++++++++++++++++++++++++++++++++++++++++++++++++++++++++++++++++++++
\section{Instructions pour les examens et interrogations}
%+++++++++++++++++++++++++++++++++++++++++++++++++++++++++++++++++++++++++++++++++++++++++++++++++++++++++++++++++++++++++++

Ceci sont des conseils généraux que nous vous conseillons de suivre dans toutes les matières.
\begin{description}
    \item[numéroter] Numérotez clairement toutes les questions. Si votre réponse prend plus d'une page, écrivez «suite au verso», «suite à l'intercalaire \( n\)» etc. À l'endroit où la réponse continue, écrivez «question \( n\), suite».

    \item[vérifiez] Certaines erreurs sont faciles à détecter. Par exemple
        \begin{enumerate}
            \item
                les aires et volumes sont positifs;
            \item
                une intégrale \emph{définie} qui contient «\( dx\)» ne peut pas contenir de \( x\) dans la réponse;

            \item
                en physique et en chimie, les unités doivent être cohérentes : si la réponse est une énergie, vous devez avoir des joules (\unit{\square\metre\kilo\gram\per\square\second}).

        \end{enumerate}
    \item[votre nom] Écrivez votre nom et votre numéro de carte d'étudiant.

    \item[les faciles d'abord] Lisez d'abord toutes les questions avant de répondre. Commencez par les questions faciles.

    \item[justifier] Justifiez vos réponses. N'hésitez pas à écrire des phrases complètes : sujet, verbe, complément. N'abusez pas des symboles dont vous ignorez la signification :
        \begin{enumerate}
            \item
                «\( \Leftrightarrow\)» signifie «si et seulement si», et non «la suite de mon calcul»;
            \item

                «\( \nexists\)» signifie «il n'existe pas», et non «n'existe pas» ou «n'est pas défini».
        \end{enumerate}

    \item[ne pas passer en force] Si vous savez que votre réponse est fausse, mais vous ne savez pas la corriger, écrivez sur votre feuille «cette réponse est fausse pour telle raison, mais je ne sais pas comment corriger». Ne comptez pas sur une inattention du correcteur. En science, affirmer un fait que vous savez être faux s'appelle de la falsification; c'est déontologiquement inacceptable. De la même façon, si vous copiez sur votre voisin\footnote{Indépendamment que c'est sans doute interdit par le règlement; vérifiez avant.}, vous êtes priés de le citer : on ne s'approprie pas le travail d'autrui.

    \item[approximations numériques] Lorsque vous voulez écrire une approximation numérique, réfléchissez au sens de ce que vous allez écrire. En mathématique, ça n'a presque jamais de sens d'écrire une approximation parce que vous ne savez pas dans quel contexte votre calcul pourra être utilisé. Si vous laissez deux décimales à \( \pi\) pour calculer le volume d'eau dans votre piscine gonflable, ça fera l'affaire; si c'est pour calculer la masse du Higgs ou pour mettre un satellite autour de Mars, vous perdez plusieurs millions d'euros.

        En sciences naturelles (physique, chimie ou autres), vous pouvez donner des approximations numériques de façon circonstanciée. Demandez à votre prof de labo.

    \item[orthographe] Sans être obligatoire, ça ne fait jamais de mal. Surtout si le français est votre langue maternelle.
    \item[santé] Mangez des fruits et des légumes de saisons. Choisissez des producteurs locaux qui n'utilisent pas d'engrais synthétisés à base de pétrole. De toutes façons \href{http://www.energybulletin.net/node/51306}{vous n'avez pas le choix}.

\end{description}
