%+++++++++++++++++++++++++++++++++++++++++++++++++++++++++++++++++++++++++++++++++++++++++++++++++++++++++++++++++++++++++++
\section{Parties libres, génératrices et bases}
%+++++++++++++++++++++++++++++++++++++++++++++++++++++++++++++++++++++++++++++++++++++++++++++++++++++++++++++++++++++++++++

\begin{theorem} \label{ThoBaseIncompjblieG}
    Soit \( E\) un espace vectoriel de dimension finie et \( \{ e_i \}_{i\in I}\) une partie génératrice de \( E\).

    \begin{enumerate}
        \item
            Il existe \( J\subset I\) tel que \( \{ e_i \}_{i\in J}\) est une base. Autrement dit : de toute partie génératrice nous pouvons extraire une base.
        \item
            Soit \( \{ f_1,\ldots, f_l \}\) une partie libre. Alors nous pouvons la compléter en utilisant des éléments \( e_i\). C'est à dire qu'il existe \( J\subset I\) tel que \( \{ f_k \}\cup\{ e_i \}_{i\in J}\) soit une base.
    \end{enumerate}
\end{theorem}

%+++++++++++++++++++++++++++++++++++++++++++++++++++++++++++++++++++++++++++++++++++++++++++++++++++++++++++++++++++++++++++
\section{Endomorphismes}
%+++++++++++++++++++++++++++++++++++++++++++++++++++++++++++++++++++++++++++++++++++++++++++++++++++++++++++++++++++++++++++

%---------------------------------------------------------------------------------------------------------------------------
\subsection{Polynôme caractéristique}
%---------------------------------------------------------------------------------------------------------------------------

Soit \( A\) un anneau commutatif et \( \eK\), un corps commutatif. L'injection canonique \( A\to A[X]\) se prolonge en une injection
\begin{equation}
   \eM(A)\to\eM\big( A[X] \big).
\end{equation}
Si \( u\in\eM_n(A)\), nous définissons le \defe{polynôme caractéristique de \( u\)}{polynôme!caractéristique} :
\begin{equation}
    \chi_u(X)=\det(X\mtu_n-u).
\end{equation} 
Ce faisons nous assimilons la matrice \( u\) et l'endomorphisme \( u\colon E\to E\) qu'elle définit. 

Nous avons un morphisme d'algèbre
\begin{equation}
    \begin{aligned}
        \varphi_u\colon\eK[X]&\to \End(E) \\
        P&\mapsto P(u). 
    \end{aligned}
\end{equation}
Cet endomorphisme ne peut pas être injectif parce que \(\eK[X]\) est de dimension infinie tandis que \( \End(E)\) est de dimension finie. Il possède donc un noyau, c'est à dire qu'il existe \( P\in\eK[X]\) tel que \( P(X)=0\). Le \defe{polynôme minimal}{polynôme!minimal} de \( u\) est le polynôme unitaire de plus petit degré qui annule \( u\). Nous le notons \( \mu_u\)\nomenclature[A]{\( \mu_A\)}{polynôme minimal de \( A\)} :
\begin{equation}
    \mu_u(u)=0.
\end{equation}

\begin{lemma}
    Le polynôme \( \chi_u\) est unitaire et de degré \( n\).
\end{lemma}

\begin{lemma}       \label{LemjcztYH}
    Soit \( u\) un endomorphisme et \( E_{\lambda}(u)\)\nomenclature[A]{\( E_{\lambda}(u)\)}{Espace propre de \( u\)} ses espaces propres. La somme des \( A_{\lambda}\) est directe.
\end{lemma}

\begin{proof}
    Soit \( v_i\in V_{\lambda_i}\) un choix de vecteurs propres de \( u\). Si la somme n'est pas directe, nous pouvons considérer une combinaison linéaire des \( v_i\) qui soit nulle :
    \begin{equation}
        v_1+\ldots+v_p=0.
    \end{equation}
    Appliquons \( (A-\lambda_1\mtu)\) à cette égalité :
    \begin{equation}
        (\lambda_2-\lambda_1)v_1+\ldots+(\lambda_p-\lambda_1)v_p=0.
    \end{equation}
    En appliquant encore successivement les opérateurs \( (A-\lambda_i\mtu)\) nous réduisons le nombre de termes jusqu'à obtenir \( v_p=0\).
\end{proof}


\begin{theorem}     \label{ThoNhbrUL}
    Soit \( E\) un \(\eK\)-espace vectoriel de dimension finie \( n\) et un endomorphisme \( u\in\End(E)\). Alors
    \begin{enumerate}
        \item
            Le polynôme caractéristique divise \( (\mu_u)^n\) dans \(\eK[X]\).
        \item
            Les polynômes caractéristiques et minimaux ont mêmes facteurs irréductibles dans \(\eK[X]\).
        \item
            Les polynômes caractéristiques et minimaux ont mêmes racines dans \(\eK[X]\).
        \item
            Le polynôme caractéristique est scindé si et seulement si le polynôme minimal est scindé.
    \end{enumerate}
\end{theorem}


Si \( \lambda\in\eK\) est une racine de \( \chi_u\), l'ordre de l'annulation est la \defe{multiplicité algébrique}{multiplicité!algébrique d'une valeur propre} de la valeur propre \( \lambda\) de \( u\).

\begin{theorem}
    Soit \( u\in\End(E)\) et \( \lambda\in\eK\). Les conditions suivantes sont équivalentes
    \begin{enumerate}
        \item\label{ItemeXHXhHi}
            \( \lambda\in\Spec(u)\)
        \item\label{ItemeXHXhHii}
            \( \chi_u(\lambda)=0\)
        \item\label{ItemeXHXhHiii}
            \( \mu_u(\lambda)=0\).
    \end{enumerate}
\end{theorem}

\begin{proof}
    \ref{ItemeXHXhHi} \( \Leftrightarrow\) \ref{ItemeXHXhHii}. Dire que \( \lambda\) est dans le spectre de \( u\) signifie que l'opérateur \( u-\lambda\mtu\) n'est pas inversible, ce qui est équivalent à dire que \( \det(u-\lambda\mtu)\) est nul ou encore que \( \lambda\) est une racine du polynôme caractéristique de \( u\). 

    \ref{ItemeXHXhHii} \( \Leftrightarrow\) \ref{ItemeXHXhHiii}. Cela est une application directe du théorème \ref{ThoNhbrUL} qui précise que le polynôme caractéristique a les mêmes racines dans \(\eK\) que le polynôme minimal.
\end{proof}

Sur l'ensemble \( \eM_n(\eK)\) des matrices \( n\times n\) à coefficients dans \(\eK\) nous introduisons la relation d'équivalence \( A\sim B\) si et seulement si il existe une matrice \( P\in\GL(n,\eK)\) telle que \( B=P^{-1}AP\). Deux matrices équivalentes en ce sens sont dites \defe{semblables}{semblables!matrices}.

\begin{lemma}
    Une matrice triangulaire supérieure avec des \( 1\) sur la diagonale n'est diagonalisable que si elle est diagonale (c'est à dire si elle est la matrice unité).
\end{lemma}

\begin{proof}
    Si \( A\) est une matrice triangulaire supérieure de taille \( n\) telle que \( A_{ii}=1\), alors \( \det(A-\lambda\mtu)=(1-\lambda)^n\), ce qui signifie que \( \Spec(A)=\{ 1 \}\). Pour la diagonaliser, il faudrait une matrice \( P\in\GL(n,\eK)\) telle que \( \mtu=P^{-1}AP\), ce qui est uniquement possible si \( A=\mtu\).
\end{proof}

%---------------------------------------------------------------------------------------------------------------------------
\subsection{Polynômes d'endomorphismes}
%---------------------------------------------------------------------------------------------------------------------------

Soit \( u\) un endomorphisme d'un \( \eK\)-espace vectoriel \( E\) et \( P\), un polynôme. Nous disons que \( P\) est un polynôme \defe{annulateur}{polynôme!annulateur} de \( u\) si \( P(u)=0\) en tant que endomorphisme de \( E\).

\begin{lemma}       \label{LemQWvhYb}
    Si \( P\) et \( Q\) sont des polynômes dans \( \eK[X]\) et si \( u\) est un endomorphisme d'un \( \eK\)-espace vectoriel \( E\), nous avons
    \begin{equation}
        (PQ)(u)=P(u)\circ Q(u).
    \end{equation}
\end{lemma}

\begin{proof}
    Si \( P=\sum_i a_iX^i\) et \( Q=\sum_j b_jX^j\), alors le coefficient de \( X^k\) dans \( PQ\) est
    \begin{equation}        \label{EqCoefGPyVcv}
        \sum_la_lb_{k-l}.
    \end{equation}
    Par conséquent \( (PQ)(u)\) contient \( \sum_la_lb_{k-l}u^k\). Par ailleurs \( P(u)\circ Q(u)\) est donné par
    \begin{equation}
        \sum_ia_iu^i\left( \sum_jb_ju^j \right)(x)=\sum_{ij}a_ib_ju^{i+j}(x).
    \end{equation}
    Le coefficient du terme en \( u^k\) est bien le même que celui donné par \eqref{EqCoefGPyVcv}.
\end{proof}

\begin{theorem}[Décomposition des noyaux]       \label{ThoDecompNoyayzzMWod}
    Soit \( u\) un endomorphisme du \( \eK\)-espace vectoriel \( E\). Soit \( P\in\eK[X]\) un polynôme tel que \( P(u)=0\). Nous supposons que \( P\) s'écrive comme le produit \( P=P_1\cdots P_n\) de polynômes deux à deux étrangers. Alors
    \begin{equation}
        E=\ker P_1(u)\oplus\ldots\oplus\ker P_n(u).
    \end{equation}
    De plus les projecteurs associés à cette décomposition sont des polynômes en \( u\).
\end{theorem}

\begin{proof}
    Nous posons 
    \begin{equation}
        Q_i=\prod_{j\neq i}P_i.
    \end{equation}
    Par le lemme \ref{LemuALZHn} ces polynômes sont étrangers entre eux et le théorème de Bezout (théorème \ref{ThoBezoutOuGmLB}) donne l'existence de polynômes \( R_i\) tels que
    \begin{equation}
        R_1Q_1+\ldots+R_nQ_n=1.
    \end{equation}
    Si nous appliquons cette égalité à \( u\) et ensuite à \( x\in E\) nous trouvons
    \begin{equation}        \label{EqqVcpUy}
        \sum_{i=1}^n(R_iQ_i)(u)(x)=x,
    \end{equation}
    et en particulier si nous posons \( E_i=\Image\big(P_iQ_i(u)\big)\) nous avons
    \begin{equation}
        E=\sum_{i=1}^nE_i.
    \end{equation}
    Cette dernière somme n'est éventuellement pas une somme directe. Si \( i\neq j\), alors \( Q_iQ_j\) est multiple de \( P\) et nous avons, en utilisant le lemme \ref{LemQWvhYb}, 
    \begin{equation}
        (R_iQ_i)(u)\circ (R_jQ_j)(u)=\big( R_iQ_iR_jQ_j \big)(u)=S_{ij}(u)\circ P(u)=0
    \end{equation}
    où \( S_{ij}\) est un polynôme. 

    Nous pouvons voir \( E\) comme un \( \eK\)-module et appliquer le théorème \ref{ThoProjModpAlsUR}. Les opérateurs \( R_iQ_i(u)\) ont l'identité comme somme et sont orthogonaux, et nous avons donc la décomposition en somme directe :
    \begin{equation}
        E=\bigoplus_{i=1}^nR_iQ_i(u)E.
    \end{equation}

    Afin de terminer la preuve, nous devons montrer que \( R_iQ_i(u)E=\ker P_i(u)\). D'abord nous avons
    \begin{equation}
        P_iR_iQ_i(u)=(R_iP)(u)=R_i(u)\circ P(u)=0,
    \end{equation}
    par conséquent \( \Image(R_iQ_i(u))\subset \ker P_i(u)\). Pour obtenir l'inclusion inverse, nous reprenons l'équation \eqref{EqqVcpUy} avec \( x\in\ker P_i(u)\). Elle se réduit à
    \begin{equation}
        (R_iQ_i)(u)x=x.
    \end{equation}
    Par conséquent \( x\in\Image\big( R_iQ_i(u) \big)\).
\end{proof}

\begin{proposition}     \label{PropAnnncEcCxj}
    Si \( P\) est un polynôme tel que \( P(u)=0\), alors le polynôme minimal \( \mu_u\) divise \( P\)
\end{proposition}

\begin{proof}
    L'ensemble \( I=\{ Q\in \eK[X]\tq Q(u)=0 \}\) est un idéal par le lemme \ref{LemQWvhYb}. Le polynôme minimal de \( u\) est un élément de degré plus bas dans \( I\) et par conséquent \( I=(\mu_u)\) par le théorème \ref{ThoCCHkoU}. Nous concluons que \( \mu_u\) divise tous les éléments de \( I\).
\end{proof}

%---------------------------------------------------------------------------------------------------------------------------
\subsection{Diagonalisation}
%---------------------------------------------------------------------------------------------------------------------------

Ici encore \( \eK\) est un corps commutatif.

\begin{lemma}       \label{LemgnaEOk}
    Soit \( F\) un sous espace stable par \( u\). Soit une décomposition du polynôme minimal
    \begin{equation}
        \mu_u=P_1^{n_1}\ldots P_r^{n_r}
    \end{equation}
    où les \( P_i\) sont des polynômes irréductibles unitaires distincts. Si nous posons \( E_i=\ker P_i^{n_i}\), alors
    \begin{equation}
        F=(F\cap E_1)\oplus\ldots \oplus(F\cap E_r).
    \end{equation}
\end{lemma}

\begin{theorem}     \label{ThoDigLEQEXR}
    Soit \( E\), un espace vectoriel de dimension \( n\) sur le corps commutatif \( \eK\) et \( u\in\End(E)\). Les propriétés suivantes sont équivalentes.
    \begin{enumerate}
        \item       \label{ItemThoDigLEQEXRi}
            Il existe un polynôme \( P\in\eK[X]\) non constant, scindé sur \(\eK\) dont toutes les racines sont simples tel que \( P(u)=0\).
        \item\label{ItemThoDigLEQEXRii}
            Le polynôme \( \mu_u\) est scindé sur \(\eK\) et toutes ses racines sont simples
        \item\label{ItemThoDigLEQEXRiii}
            Tout sous espace de \( E\) possède un supplémentaire stable par \( u\).
        \item\label{ItemThoDigLEQEXRiv}
            L'endomorphisme \( u\) est diagonalisable.
    \end{enumerate}
    
\end{theorem}

\begin{proof}
    \ref{ItemThoDigLEQEXRi}\( \Rightarrow\)\ref{ItemThoDigLEQEXRii}. Étant donné que \( P(u)=0\), il est dans l'idéal des polynôme annulateurs de \( u\), et le polynôme minimal \( \mu_u\) le divise.

    \ref{ItemThoDigLEQEXRii}\( \Rightarrow\)\ref{ItemThoDigLEQEXRiv}. Nous décomposons le polynôme minimal en
    \begin{equation}
        \mu_u(X)=(X-\lambda_1)\ldots(X-\lambda_r)
    \end{equation}
    où les \( \lambda_i\) sont des éléments distincts de \( \eK\). Étant donné que \( \mu_u(u)=0\), le théorème de décomposition des noyaux (théorème \ref{ThoDecompNoyayzzMWod}) nous enseigne que
    \begin{equation}
        E=\ker(u-\lambda_1)\oplus\ldots\oplus\ker(u-\lambda_r).
    \end{equation}
    Mais \( \ker(u-\lambda_i)\) est l'espace propre \( E_{\lambda_i}(u)\). Donc \( u\) est diagonalisable.

    \ref{ItemThoDigLEQEXRiv}\( \Rightarrow\)\ref{ItemThoDigLEQEXRiii}. Soit \( \{ e_1,\ldots, e_n \}\) une base qui diagonalise \( u\), soit \( F\) un sous espace de \( E\) un \( \{ f_1,\ldots, f_r \}\) une base de \( F\). Par le théorème \ref{ThoBaseIncompjblieG} (qui généralise le théorème de la base incomplète), nous pouvons compléter la base de \( F\) par des éléments de la base \( \{ e_i \}\). Le complément ainsi construit est invariant par \( u\).

    \ref{ItemThoDigLEQEXRiii}\( \Rightarrow\)\ref{ItemThoDigLEQEXRiv}. En dimension un, tout endomorphisme est diagonalisable, nous supposons donc que \( \dim E=n\geq 2\). Nous procédons par récurrence sur le nombre de vecteurs propres connus de \( u\). Supposons avoir déjà trouvé \( p\) vecteurs propres \( e_1,\ldots, e_p\) de \( u\). Considérons \( H\), un hyperplan qui contient les vecteurs \( e_1,\ldots, e_p\). Soit \( F\) un supplémentaire de \( H\) stable par \( u\); par construction \( \dim F=1\) et si \( e_{p+1}\in F\), il doit être vecteur propre de \( u\).

    \ref{ItemThoDigLEQEXRiv}\( \Rightarrow\)\ref{ItemThoDigLEQEXRi}. Nous supposons maintenant que \( u\) est diagonalisable. Soient \( \lambda_1,\ldots, \lambda_r\) les valeurs propres deux à deux distinctes, et considérons le polynôme
    \begin{equation}
        P(x)=(X-\lambda_1)\ldots (X-\lambda_r).
    \end{equation}
    Alors \( P(u)=0\). En effet si \( e_i\) est un vecteur propre pour la valeur propre \( \lambda_i\), 
    \begin{equation}
        P(u)e_i=\prod_{j\neq i}(u-\lambda_j)\circ(u-\lambda_i)e_i=0
    \end{equation}
    par le lemme \ref{LemQWvhYb}. Par conséquent \( P(u)\) s'annule sur une base.
\end{proof}

\begin{corollary}
    Si \( u\) est diagonalisable et si \( F\) est une sous espace stable par \( u\), alors
    \begin{equation}
        F=\sum_{\lambda}E_{\lambda}(u)\cap F
    \end{equation}
    où \( E_{\lambda}(u)\) est l'espace propre de \( u\) pour la valeur propre \( \lambda\).
\end{corollary}

\begin{proof}
    Par le théorème \ref{ThoDigLEQEXR}, le polynôme \( \mu_x\) est scindé et ne possède que des racines simples. Notons le
    \begin{equation}
        \mu_u(X)=(X-\lambda_1)\ldots (X-\lambda_r).
    \end{equation}
    Les espaces \( E_i\) du lemme \ref{LemgnaEOk} sont maintenant les espaces propres.
\end{proof}

\begin{lemma}
    Soit \( E\) un \( \eK\)-espace vectoriel et \( u\in\End(E)\). Si \( \Card\big( \Spec(u) \big)=\dim(E)\) alors \( u\) est diagonalisable.
\end{lemma}

\begin{proof}
    Soient \( \lambda_1,\ldots, \lambda_n\) les valeurs propres distinctes de \( u\). Nous savons que les espaces propres correspondants sont en somme directe (lemme \ref{LemjcztYH}). Par conséquent \( \Span\{ E_{\lambda_i}(u) \}\) est de dimension \( n\) est \( u\) est diagonalisable.
\end{proof}

\begin{proposition}
    Soit \( (u_i)_{i\in I}\) une famille d'endomorphismes qui commutent deux à deux.
    \begin{enumerate}
        \item       \label{ItemGqhAMei}
            Si \( i,j\in I\) alors tout sous espace propre de \( u_i\) est stable par \( u_j\). Autrement dit \( u_j\big(E_{\lambda}(u)\big)\subset E_{\lambda}(u)\).
        \item
            Si les \( u_i\) sont diagonalisables, alors ils le sont simultanément.
    \end{enumerate}
\end{proposition}

\begin{proof}

    Supposons que \( u_i\) et \( u_j\) commutent et soit \( x\) un vecteur propre de \( u_i\) : \( u_ix=\lambda x\). Nous montrons que \( u_jx\in E_{\lambda}(u)\). Nous avons
    \begin{equation}
        u_i\big( u_j(x) \big)=u_j\big( u_i(x) \big)=\lambda u_j(x).
    \end{equation}
    Par conséquent \( u_j(x)\) est vecteur propre de \( u_i\) de valeur propre \( \lambda\).

    Montrons maintenant l'affirmation à propos des endomorphismes simultanément diagonalisables. Si \( \dim E=1\), le résultat est évident. Nous supposons également qu'aucun des \( u_i\) n'est multiple de l'identité. Nous effectuons une récurrence sur la dimension.

    Soit \( u_0\) un des \( u_i\) et considérons ses valeurs propres deux à deux distinctes \( \lambda_1,\ldots, \lambda_r\). Pour chaque \( k\) nous avons
    \begin{equation}
        E_{\lambda_k}(u_0)\neq E,
    \end{equation}
    sinon \( u_0\) serait un multiple de l'identité. Par contre nous avons
    \begin{equation}
        E=\bigoplus_{k}E_{\lambda_k}(u_0).
    \end{equation}
    Par le point \ref{ItemGqhAMei}, nous avons \( u_i\colon E_{\lambda_k}(u_0)\to E_{\lambda_k}(u_0)\), et nous pouvons considérer la famille d'opérateurs
    \begin{equation}
        \left( u_i|_{E_{\lambda_k}(u_0)} \right)_{i\in I}.
    \end{equation}
    Ce sont tous des opérateurs qui commutent et qui agissent sur un espace de dimension plus petite. Par hypothèse de récurrence nous avons une base de \( E_{\lambda_k}(u_0)\) qui diagonalise tous les \( u_i\).
\end{proof}

%---------------------------------------------------------------------------------------------------------------------------
\subsection{Sous espaces caractéristiques}
%---------------------------------------------------------------------------------------------------------------------------

Sources : \cite{MneimneReduct} et \wikipedia{fr}{Décomposition_de_Dunford}{divers articles sur wikipédia}.

Lorsqu'un opérateur n'est pas diagonalisable, les valeurs propres jouent quand même un rôle important.

Soit \( E\) un \( \eK\)-espace vectoriel et \( f\in\End(E)\). Pour \( \lambda\in \eK\) nous définissons
\begin{equation}
    F_{\lambda}(f)=\{ v\in E\tq (f-\lambda\mtu)^nv=0, n\in\eN \}.
\end{equation}
C'est l'ensemble de nilpotence de l'opérateur \( f-\lambda\mtu\).

\begin{lemma}
    L'ensemble \( F_{\lambda}(f)\) est non vide si et seulement si \( \lambda\) est une valeur propre de \( f\). L'espace \( F_{\lambda}(f)\) est invariant sous \( f\).
\end{lemma}

\begin{proof}
    Si \( F_{\lambda}(f)\) est non vide, nous considérons \( v\in F_{\lambda}(f)\) et \( n\) le plus petit entier non nul tel que \( (f-\lambda)^nv=0\). Alors \( (f-\lambda)^{n-1}v\) est un vecteur propre de \( f\) pour la valeur propre \( \lambda\). Inversement si \( v\) est une valeur propre de \( f\) pour la valeur propre \( \lambda\), alors \( v\in F_{\lambda}(f)\).

    En ce qui concerne l'invariance, remarquons que \( f\) commute avec \( f-\lambda\mtu\). Si \( x\in F_{\lambda}(f)\) il existe \( n\) tel que \( (f-\lambda\mtu)^nx=0\). Nous avons aussi
    \begin{equation}
        (f-\lambda\mtu)^nf(x)=f\big( (f-\lambda\mtu)^nx \big)=0,
    \end{equation}
    par conséquent \( f(x)\in F_{\lambda}(f)\).
\end{proof}

\begin{remark}
    Toute matrice sur \( \eC\) n'est pas diagonalisable. Considérons en effet l'endomorphisme \( f\) donné par la matrice
    \begin{equation}
        \begin{pmatrix}
            a&    \alpha    &   \beta    \\
            0    &   a    &   \gamma    \\
            0    &   0    &   b
        \end{pmatrix}
    \end{equation}
    où \( a\neq b\), \( \alpha\neq 0\), \( \beta\) et \( \gamma\) sont des nombres complexes quelconques.
    Son polynôme caractéristique est 
    \begin{equation}
        \chi_f(\lambda)=(a-\lambda)^2(b-\lambda)
    \end{equation}
    de telle façon à ce que les valeurs propres soient \( a\) et \( b\). Nous trouvons les vecteurs propres pour la valeur \( a\) en résolvant
    \begin{equation}
        \begin{pmatrix}
            a    &   \alpha    &   \beta    \\
            0    &   a    &   \gamma    \\
            0    &   0    &   b
        \end{pmatrix}\begin{pmatrix}
            x    \\ 
            y    \\ 
            z    
        \end{pmatrix}=\begin{pmatrix}
            ax    \\ 
            ay    \\ 
            az    
        \end{pmatrix}.
    \end{equation}
    L'espace propre \( E_a(f)\) est réduit à une seule dimension générée par \( (1,0,0)\). De la même façon l'espace propre correspondant à la valeur propre \( b\) est donné par 
    \begin{equation}
        \begin{pmatrix}
            \frac{1}{ b-a }\left( \beta+\frac{ \alpha\gamma }{ b-a } \right)    \\ 
            \frac{ \gamma }{ b-a }    \\ 
            1    
        \end{pmatrix}.
    \end{equation}
    Il n'y a donc pas trois vecteurs propres linéairement indépendants, et l'opérateur \( f\) n'est pas diagonalisable.

    Par contre nous pouvons voir que
    \begin{equation}
        (f-\alpha\mtu)^2\begin{pmatrix}
             0   \\ 
            1    \\ 
            0    
        \end{pmatrix}=
        \begin{pmatrix}
            a    &   \alpha    &   \beta    \\
            0    &   a    &   \gamma    \\
            0    &   0    &   b
        \end{pmatrix}
        \begin{pmatrix}
            \alpha    \\ 
            0    \\ 
            0    
        \end{pmatrix}-\begin{pmatrix}
            a\alpha    \\ 
            0    \\ 
            0    
        \end{pmatrix}=\begin{pmatrix}
            0    \\ 
            0    \\ 
            0    
        \end{pmatrix},
    \end{equation}
    de telle sorte que le vecteur \( (0,1,0)\) soit également dans l'espace caractéristique \( F_a(f)\).

    Dans cet exemple, la multiplicité algébrique de la racine \( a\) du polynôme caractéristique vaut \( 2\) tandis que sa multiplicité géométrique vaut seulement \( 1\).
\end{remark}

Le théorème suivant est aussi appelé le théorème de \defe{décomposition primaire}{décomposition!primaire}.
\begin{theorem}[Théorème spectral]\index{théorème!spectral}     \label{ThoSpectraluRMLok}
    Soit \( E\) espace vectoriel de dimension finie sur le corps algébriquement clos \( \eK\) et \( f\in\End(E)\). Alors
    \begin{equation}
        E=F_{\lambda_1}(f)\oplus\ldots\oplus F_{\lambda_k}(f)
    \end{equation}
    où la somme est sur les valeurs propres distinctes de \( f\).

    Les projecteurs sur les espaces caractéristique forment un système complet et orthogonal.
\end{theorem}

Le théorème suivant généralise le théorème de diagonalisabilité \ref{ThoDigLEQEXR} au cas où le polynôme minimum est seulement scindé.
\begin{probleme}
    \begin{enumerate}
\item
    les \( s\) et \( n\) sont uniques
\item
    \( s\) et \( n\) sont des polynôme en \( u\).
\item 
    Dans le cas où le corps n'est pas algébriquement clos, il paraît qu'il faut remplacer «diagonalisable» par «semi-simple».
    \end{enumerate}
\end{probleme}

\begin{definition}
    Un endomorphisme d'un espace vectoriel est \defe{semi-simple}{semi-simple!endomorphisme} si tout sous espace stable par \( u\) possède un supplémentaire stable.
\end{definition}
Si l'espace vectoriel est sur un corps algébriquement clos, alors les endomorphismes semi-simples sont les endomorphismes diagonaux.

\begin{theorem}[Décomposition de Dunford]\index{décomposition!Dunford}\index{Dunford!décomposition}
    Soit \( E\) un espace vectoriel sur le corps algébriquement clos \( \eK\) et \( u\in\End(E)\) un endomorphisme dont le polynôme minimum est scindé. Alors \( u\) se décompose de façon unique sous la forme
    \begin{equation}
        u=s+n
    \end{equation}
    où \( s\) est diagonale, \( n\) est nilpotent et \( [s,n]=0\).

    De plus ces endomorphismes \( s\) et \( n\) sont des polynômes en \( u\) et sont uniques.
\end{theorem}

\begin{proof}
    Le théorème spectral \ref{ThoSpectraluRMLok} nous indique que
    \begin{equation}
        E=\bigoplus_{\lambda}F_{\lambda}(f).
    \end{equation}
    Nous considérons l'endomorphisme \( s\) de \( E\) qui consiste à dilater d'un facteur \( \lambda\) l'espace caractéristique \( F_{\lambda}(f)\) :
    \begin{equation}
        s=\bigoplus_{\lambda}\lambda\mtu|_{F_{\lambda}(f)}.
    \end{equation}
    Nous allons prouver que \( [s,f]=0\) et \( n=f-s\) est nilpotent. Cela impliquera que \( [s,n]=0\).

    Si \( x\in F_{\lambda}(f)\), alors nous avons \( sf(x)=\lambda f(x)\) parce que \( f(x)\in F_{\lambda}(f)\) tandis que \( fs(x)=f(\lambda x)=\lambda f(x)\). Par conséquent \( f\) commute avec \( s\).

    Pour montrer que \( f-s\) est nilpotent, nous en considérons la restriction
    \begin{equation}
        f-s\colon F_{\lambda}(f)\to F_{\lambda}(f).
    \end{equation}
    Cet opérateur est égal à \( f-\lambda\mtu\) et est par conséquent nilpotent.
    
\end{proof}
<++>


