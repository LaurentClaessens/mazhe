% This is a generic document for the first 5 pages of the 
% comercialized Frido.
% There are automatic substitutions done by 'make_first_5_pages.py'

\documentclass[a4paper,twoside,11pt]{book}

\usepackage[utf8]{inputenc}
\usepackage[T1]{fontenc}

\usepackage{etex}
\usepackage{ifthen}
\usepackage{etoolbox}           % Ceci devrait remplacer ifthen.

\usepackage{latexsym}
\usepackage{amsfonts}
\usepackage{amsmath}
\usepackage{amsthm}
\usepackage{amssymb}
\usepackage{mathrsfs}
\usepackage{mathabx}           % For \divides et \widehat.
\usepackage{bbm}

\usepackage{url}
\usepackage{graphicx}                   % Pour l'inclusion d'image en pfd.

\usepackage{textcomp}
\usepackage{lmodern}
\usepackage[a4paper,margin=2cm,left=2.6cm]{geometry}


\begin{document}

\pagestyle{empty}

\phantom{Foo}
\newpage

\phantom{Foo}
\newpage

\begin{center}
	TITLE YEAR, \\
	volume NUMBER \\
	Laurent Claessens
\end{center}

\newpage

Plusieurs extensions et versions de ce livre.
\begin{enumerate}
	\item
	      La version courante, régulièrement mise à jour et qui deviendra petit à petit le Frido YEAR+1. Téléchargeable sur
	      \begin{center}
		      \url{https://laurent.claessens-donadello.eu/pdf/lefrido.pdf}
	      \end{center}
	\item
	      La version la plus complète, contenant beaucoup de géométrie différentielle
	      \begin{center}
		      \url{https://laurent.claessens-donadello.eu/pdf/giulietta.pdf}
	      \end{center}
	\item
	      Et bien entendu les sources \LaTeX\
	      \begin{center}
		      \url{https://github.com/LaurentClaessens/mazhe}
	      \end{center}
\end{enumerate}

\newpage

\phantom{un foobar bleuté}

\vfill

\begin{center}
	Copyright 2011-YEAR  Laurent Claessens, and many contributors. A complete list could be retrieved from the git log.

	Permission is granted to copy, distribute and/or modify this document under the terms of the GNU Free Documentation License, Version 1.3 or any later version published by the Free Software Foundation; with no Invariant Sections, no Front-Cover Texts, and no Back-Cover Texts. A copy of the license is included in the chapter entitled ``GNU Free Documentation~License''.
\end{center}

\begin{center}
	PEPPERCARROT
\end{center}

\begin{center}
	ISBN : RISBN
\end{center}


\end{document}
