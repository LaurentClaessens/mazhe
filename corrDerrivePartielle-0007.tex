% This is part of the Exercices et corrigés de mathématique générale.
%   Laurent Claessens
% See the file fdl-1.3.txt for copying conditions.

\begin{corrige}{DerrivePartielle-0007}

	Voici la fonction Sage qui fournit les informations :

	\lstinputlisting{exo104.sage}

	La sortie est

	\VerbatimInput[tabsize=3]{exo104.txt}

	Notez la présence de \verb+r1+ comme paramètres dans les solutions. Tous les points avec $y=0$ sont des points critiques. Cependant, Sage\footnote{ou, plus précisément, le programme que j'ai écrit avec Sage.} ne parvient pas à conclure la nature de ces points $(x,0)$.

	Notons que le nombre $f(x,y)$ a toujours le signe de $x$ parce que $y^2$ et l'exponentielle sont positives. Toujours ? En tout cas lorsque $x\neq 0$. Prenons un point $(a,0)$ avec $a>0$. Dans un voisinage de ce point, nous avons $f(x,y)>0$ parce que si $a>0$, alors $x>0$ dans un voisinage de $a$. Le point $(a,0)$ est un minimum local parce que $0=f(a,0)\leq f(x,y)$ pour tout $(x,y)$ dans un voisinage de $(a,0)$.

	De la même façon, les points $(a,0)$ avec $a<0$ sont des maxima locaux parce que dans un voisinage, la fonction est négative. 

	Le point $(0,0)$ n'est ni maximum ni minimum local. C'est un point de selle.

\end{corrige}
