% This is part of Mes notes de mathématique
% Copyright (c) 2011
%   Laurent Claessens
% See the file fdl-1.3.txt for copying conditions.

\begin{exercice}\label{exoreserve0003}

    \begin{enumerate}
        \item
            Résoudre dans \( \eF_{16}\) l'équation \( x^5=a\) en discutant éventuellement en fonction de la valeur de \( a\).
        \item
            Montrer qu'il existe quatre éléments \( \gamma\in\eF_{16}\) tels que pour chacun d'eux l'ensemble \( B_{\gamma}=\{ \gamma,\gamma^2,\gamma^4,\gamma^8 \}\) est une base de \( \eF_{16}\) sur \( \eF_2\) telle que le produit de deux éléments de \( B_{\gamma}\) est soit un élement de \( B_{\gamma}\) soit \( 1\).
    \end{enumerate}

\corrref{reserve0003}
\end{exercice}
