% This is part of (almost) Everything I know in mathematics
% Copyright (c) 2013-2015
%   Laurent Claessens
% See the file fdl-1.3.txt for copying conditions.

Most of this chapter come (often very directly) from \cite{Helgason}. Other sources are \cite{Knapp_reprez,DirkEnvFiniteDimNilLieAlg,SamelsonNotesLieAlg,SternLieAlgebra}.

Do you know what is violet and commutative ? Answer in the footnote\footnote{An abelian grape !}.

\section{Lie groups}
%+++++++++++++++++++++++++++++

A \defe{Lie group}{Lie!group}\index{group!Lie} is a manifold $G$ endowed with a group structure such that the inversion map $\dpt{i}{G}{G}$, $i(x)=x^{-1}$ and the multiplication $\dpt{m}{G\times G}{G}$, $m(x,y)=xy$ are differentiable. The \defe{Lie algebra}{Lie!algebra}\index{algebra!Lie} of the Lie group $G$ is the tangent space of $G$ at the identity: $\yG=T_eG$. 

 It is immediate to see that $g\mapsto g^{-1}$ is a smooth homeomorphism and that, for any fixed $g_0, g_1$, the maps
\[ 
\begin{split}
   g&\mapsto g_0g,\\
g&\mapsto gg_0,\\
g&\mapsto g_0gg_1
\end{split}  
\]
are smooth homeomorphisms. When $A\subset G$, we define $A^{-1}=\{ g^{-1}\tq g\in G \}$.

\subsection{Connected component of Lie groups}
%---------------------------------------------

\begin{proposition}		\label{PropUssGpGenere} 
If $G$ is a connected Lie group and $\mU$, a neighbourhood of the identity $e$, then $G$ is generated by $\mU$ in the sense that $\forall g\in G$, there exists a \emph{finite} number of $g_{i}\in \mU$ such that
\[ 
  g=g_1\ldots g_n.
\]
Notice that the number $n$ is function of $g$ in general.
\end{proposition}

\begin{proof}
Eventually passing to a subset, we can suppose that $\mU$ is open. In this case, $\mU^{-1}$ is open because it is the image of $\mU$ under the homeomorphism $g\mapsto g^{-1}$. Now we consider $V=\mU\cap\mU^{-1}$. The main property of this set is that $V=V^{-1}$. Let
\[ 
  [V]=\{ g_1\ldots g_n\tq g_{i}\in V \};
\]
we will prove that $[V]=G$ by proving that it is closed and open in $G$ (the fact that $G$ is connected then concludes).

We begin by openness of $[V]$. Let $g_0=g_1\cdots g_n\in[V]$. We know that $g_0V$ is open because the multiplication by $g_0$ is an homeomorphism. It is clear that $g_0V\subset [V]$ and that $g_0=g_0e\in g_0V$. Hence $g_0\in g_0V\subseteq[V]$. It proves that $[V]$ is open because $g_0V$ is a neighbourhood of $g_0$ in $[V]$.

We now turn our attention to the closeness of $[V]$. Let $h\in\overline{ [V] }$. The set $hV$ is an open set which contains $h$ and $hV\cap [V]\neq \emptyset$ because an open which contains an element of the closure of a set intersects the set (it is almost the definition of the closure). Let $g_0\in hV\cap[V]$. There exists a $h_{1}\in V$ such that $g_0=hh_1$. For this $h_1$, we have $hh_1=g_0=g_1\cdots g_n$, and therefore
\[ 
  h=g_1\cdots g_n h_1^{-1}\in[V].
\]
This proves that $h\in[V]$ because $h_1^{-1}\in V$ from the fact that $V=V^{-1}$.
\end{proof}
 Remark that this proof emphasises the topological aspect of a Lie group: the differential structure was only used to prove thinks like that $A^{-1}$ is open when $A$ is open.

\begin{proposition}
Let $G$ be a Lie group and $G_0$, the identity component of $G$. We have the following:
\begin{enumerate}
\item $G_0$ is an open invariant subgroup of $G$,
\item $G_0$ is a Lie group,
\item the connected components of $G$ are lateral classes of $G_0$. More specifically, if $x$ belongs to the connected component $G_1$, then $G_1=xG_0=G_0x$.
\end{enumerate}

\end{proposition}

\begin{proof}
We know that when $M_{1}$ is open in the manifold $M$, one can put on $M_{1}$ a differential structure of manifold of same dimension as $M$ with the induced topology. Since $G_0$ is open, it is a smooth manifold. In order for $G_0$ to be a Lie group, we have to prove that it is stable under the inversion and that $gh\in G_0$ whenever $g$, $h\in G_0$.

First, $G_0^{-1}$ is connected because  it is homeomorphic to $G_0$ in $G$. The element $e$ belongs to the intersection of $G_0$ and $G_0^{-1}$, so $G_0\cup G_0^{-1}$ is connected as non-disjoint union of connected sets. Hence $G_0\cup G_0^{-1}=G_0$ and we conclude that $G_0^{-1}\subseteq G_0$. The set $G_0G_0$ is connected because it is the image of $G_0\times G_0$ under the multiplication map, but $e\in G_0G_0$, so $G_0G_0\subseteq G_0$ and  $G_0$ is thus closed for the multiplication. Hence $G_0$ is a Lie group.

For all $x\in G$, we have $e=xex^{-1}\in xG_0x^{-1}$, but $xG_0x^{-1}$ is connected. Hence $xG_0x^{-1}\subseteq G_0$, which proves that $G_0$ is an invariant subset of $G$.

Lateral classes $xG_0$ are connected because the left multiplication is an homeomorphism. They are moreover \emph{maximal} connected subsets because, if $xG_0\subset H$ (proper inclusion) with a connected $H$, then $G_0\subset x^{-1}H$ (still proper inclusion). But the definition of $G_0$ is that this proper inclusion is impossible. Therefore, the sets of the form $xG_0$ are maximally connected sets. It is clear that $\cup_{g\in G}gG_0=G$.

Notice that the last point works with $G_0x$ too.
\end{proof}

%+++++++++++++++++++++++++++++++++++++++++++++++++++++++++++++++++++++++++++++++++++++++++++++++++++++++++++++++++++++++++++
\section{Two words about Lie algebra}
%+++++++++++++++++++++++++++++++++++++++++++++++++++++++++++++++++++++++++++++++++++++++++++++++++++++++++++++++++++++++++++

\subsection{The Lie algebra of \texorpdfstring{$\SU(n)$}{SU2}}
%---------------------------------------

Let consider $G=\SU(n)$; the elements are complexes $n\times n$ matrices $U$ such that $UU^{\dag}=\mtu$ and $\det U=1$. An element of the Lie algebra is given by a path $\dpt{u}{\eR}{G}$ in the group with $u(0)=\mtu$. Since for all $t$, $u(t)u(t)^{\dag}=\mtu$,
\begin{equation}
\begin{split}
  0&=\Dsdd{u(t)u(t)^{\dag}}{t}{0}\\
   &=u(0)\Dsdd{u(t)^{\dag}}{t}{0}+\Dsdd{u(t)}{t}{0}u(0)^{\dag}\\
   &=[d_tu(t)]^{\dag}+[d_tu(t)].
\end{split}
\end{equation}
So a general element of the Lie algebra $\mathfrak{su}(n)$ is an anti-hermitian matrix. 

An element of \( \SU(n)\) has also a determinant equal to \( 1\). What condition does it implies on the elements of the Lie algebra ? If \( g(t)\) is a path in \( \SU(n)\) with \( g(0)=\mtu\) we have
\begin{equation}
    \det\begin{pmatrix}
        g_{11}(t)    &   g_{12}(t)    &   \ldots    \\
        f_{21}(t)    &   g_{22}(t)    &   \ldots    \\
        \vdots    &   \vdots    &   \ddots
    \end{pmatrix}=g_{11}(t)M_{11}(t)+g_{12}(t)M_{12}(t)+\ldots=1
\end{equation}
where \( M_{ij}\) is the minor of \( g\). If we derive the left hand side we get
\begin{equation}
    g'_{11}(0)M_{11}(0)+g_{11}(0)M'_{11}(0)+g'_{12}(0)M_{12}(0)+g_{12}(0)M'_{12}(0)+\ldots
\end{equation}
where the numbers \( g'_{ij}(0)\) are the matrix entries of the tangent matrix, that is the matrix elements of a general element in \( \gsu(n)\). Since \( g(0)=\mtu\) we have \( M_{11}(0)=1\), \( g_{11}(0)=1\), \( M_{12}(0)=0\) and \( g_{12}(0)=0\). Thus we have
\begin{equation}
    (\det g)'(0)=X_{11}+M'_{11}(0)
\end{equation}
where \( X=g'(0)\). By induction we found that the trace of \( X\) appears. Thus the elements of \( \gsu(n)\) have vanishing trace.


\subsection{What is  \texorpdfstring{$g^{-1} dg$}{g-1dg} ?}\label{SubSecgmudg}
%--------------------------------

The expression $g^{-1} dg$ is often written in the physical literature. In our framework, the way to gives a sense to this expression is to consider it pointwise acting on a tangent vector. More precisely, the framework is the data of a manifold $M$, a Lie group $G$ and a map $\dpt{g}{M}{G}$. Pointwise, we have to apply $g(x)^{-1} dg_x$ to a tangent vector $v\in T_xM$.

Note that $\dpt{dg_x}{T_xM}{T_{g(0)}G\neq T_eG}$, so $dg_x\notin \yG$. But the product $g(x)^{-1} dg_x v$ is defined by
\begin{equation}
	g(x)^{-1} dg_x v=\Dsdd{ g(x)^{-1} g(v(t)) }{t}{0}\in\yG.
\end{equation}


%--------------------------------------------------------------------------------------------------------------------------- 
\subsection{Adjoint map}
%---------------------------------------------------------------------------------------------------------------------------

The ideas of this short note comes from \cite{Lie}. A more traumatic definition of the adjoint group can be found in \cite{Helgason}, chapter II, \S 5. Let $G$ be a Lie group, and $\mG$, its Lie algebra. We define the \defe{adjoint map}{adjoint!map} at the point $x\in G$ by 
\begin{equation}
\begin{aligned}
 \AD_x\colon G&\to G \\ 
 \AD_xy&=xyx^{-1} 
\end{aligned}
\end{equation}
Then we define 
\[
\dpt{Ad_x:=(d\AD_x)_e}{\mG}{\mG};
\]
the chain rule applied on $\AD_{xy}=\AD_x\circ\AD_y$ leads to $Ad_{xy}=Ad_x\circ Ad_y$, and thus we can see $Ad$ as a group homomorphism $\dpt{Ad}{G}{GL(\mG)}$, $Ad(x)=Ad_x$.

\begin{definition}
This homomorphism is the \defe{adjoint representation}{adjoint!representation!Lie group on its Lie algebra}\index{representation!adjoint} of the group $G$ in the vector space $\mG$.
\end{definition}


Finally, we define
\[
 \dpt{ad:=d(Ad)_1}{\mG}{L(\mG,\mG)}
\]
 where we identify $T_1GL(\mG)$ with $L(\mG,\mG)$.

\begin{lemma}
If $\dpt{f}{G}{G}$ is an automorphism of $G$ (i.e.: $f(xy)=f(x)f(y)$), then $df_e$ is an automorphism of $\mG$: $df[X,Y]=[df X,dfY]$
\label{lem:auto_1}
\end{lemma}

\begin{proof}
First, remark that $f(\AD_xy)=\AD_{f(x)}f(y)$. Now, $\Ad_x X=(d\AD_x)_eX$, so that one can compute:
\begin{equation}
\begin{split}
   df(\Ad_xX)&=\Dsdd{f(\AD_xX(t))}{t}{0}\\
             &=\Dsdd{   \AD_{f(x)}f(X(t))  }{t}{0}\\
	     &=(d\AD_{f(x)})_{f(e)}df X\\
	     &=\Ad_{f(x)}df X.
\end{split}
\end{equation}
On the other hand, we need to understand how does the $\ad$ work.
\[
  \ad XY=\Dsdd{\Ad_{X(t)}}{t}{0}Y=\Dsdd{\Ad_{X(t)}Y}{t}{0}
\]
because $\dpt{\Ad_{X(t)}}{\mG}{\mG}$ is linear, so that $Y$ can enter the derivation (for this, we identify $\mG$ and $T_X\mG$). Since $\Ad_{X(t)}Y$ is a path in $\mG$ the \emph{true space} is
\[   
(\ad X)Y=\Dsdd{ \Ad_{X(t)}Y }{t}{0}\in T_{[X,Y]}\mG\simeq\mG.
\]
For the same reason of linearity, $df$ can get in the derivative in the expression $df\Dsdd{  \Ad_{X(t)}Y  }{t}{0}$. Thus
\begin{equation}
\begin{split}
(\ad X)Y&=\Dsdd{  df\big(\Ad_{X(t)}Y\big)  }{t}{0}\\
        &=\Dsdd{  \Ad_{ f(X(t)) }df Y  }{t}{0}\\
	&=\Dsdd{ \Ad_{f(X(t))} }{t}{0}df Y\\
	&=\ad(dfX)df Y\\
	&=[dfX,df Y]
\end{split}
\end{equation}
because $f(X(t))$ is a path which gives $df X$.

\end{proof}

One can show that $[X,Y]$ is tangent to the curve
\begin{equation}
  c(t)=e^{-\sqrt{s}X}e^{-\sqrt{s}Y}e^{\sqrt{s}X}e^{\sqrt{s}Y}.
\end{equation}

\begin{lemma}
	In the case of Lie algebra, the bracket is given by the derivative of the adjoint action:
	\begin{equation}
		\Dsdd{ \Ad( e^{tX})Y }{t}{0}=[X,Y]
	\end{equation}
\end{lemma}

\begin{proof}
	Let us make $[\tilde X,\tilde Y]_e$ act on a function $f$. Using the definition \eqref{EqDefLieDerivativeVect} and the property of theorem \ref{ThoLieDerrComm}, we have
	\begin{equation}
		\begin{aligned}[]
			[\tilde X,\tilde Y]_ef&=\Dsdd{ (d\varphi_{-t}^X)\tilde Y }{t}{0}f\\
			&=\Dsdd{ (d\varphi_{-t}^X)_{\varphi_t^X(e)}\big( \tilde Y_{\varphi_t^X(e)} \big) }{t}{0}f\\
			&=\Dsdd{ \tilde Y_{ e^{tX}}\cdot(f\circ\varphi_{-t}^X) }{t}{0}
		\end{aligned}
	\end{equation}
	Now, we use the fact that, by definition, $\varphi_t^X(x)=x e^{tX}$, so that $\varphi_s^Y( e^{tX})= e^{tX} e^{sY}$ and we get
	\begin{equation}
		\begin{aligned}[]
			[\tilde X,\tilde Y]_ef&=\Dsdd{ \Dsdd{ f\big( \varphi_{-t}^X( e^{tX} e^{sY}) \big) }{s}{0} }{t}{0}\\
			&=\Dsdd { \Dsdd{ f( e^{tX} e^{sY} e^{-tX}) }{s}{0} } {t}{0}\\
			&=\Dsdd{ \Dsdd{ f\big(  e^{s\Ad( e^{tX})Y} \big) }{s}{0} }{t}{0}\\
			&=\Dsdd{ \big( \Ad( e^{tX})Y \big)_e\cdot f }{t}{0}
		\end{aligned}
	\end{equation}
	
\end{proof}


\section{Fundamental vector field}\label{sec:fond_vec}
%++++++++++++++++++++++++++++++++++++

If $\yG$ is the Lie algebra of a Lie group $G$ acting on a manifold $M$ (the action of $g$ on $x$ being denoted by $x\cdot g$), the \defe{fundamental vector field}{fundamental!vector field} associated with $A\in\yG$ is given by
\begin{equation}			\label{EqDefChmpFond}
   A^*_x=\Dsdd{ x\cdot e^{-tA} }{t}{0}.
\end{equation}
We always suppose that the action is effective. If the action of $G$ is transitive, the fundamental vectors at point $x\in M$ form a basis of $T_xM$. More precisely, we have the

\begin{lemma}
For any $v\in T_xM$, there exists a $A\in\yG$ such that $v=A^*_x$, in other terms
\[ 
  \Span\{ A^*_{x}\tq A\in\yG \}=T_{x}M.
\]
\label{LemFundSpansTan}
\end{lemma}

\begin{proof}
The vector $v$ is given by a path $v(t)$ in $M$. Since the action is transitive, one can write $v(t)=x\cdot c(t)$ for a certain path $c$ in $G$ which fulfills $c(0)=e$. We have to show that $v$ depends only on $c'(0)\in\yG$. We consider 
\begin{equation}  \label{eq_def_RGM}
\begin{aligned}
 R\colon G\times M&\to M \\ 
R(g,x)&= x\cdot g,
\end{aligned}
\end{equation}
so
\begin{equation}\label{eq:v_Rc}
   v=\Dsdd{ R(c(t),x) }{t}{0}=dR_{(e,x)}\big[  (d_tc(t),x)+(c(0),x)   \big].
\end{equation}

\end{proof}



\begin{lemma}
If $A$, $B\in\yG$ are such that $A^*=B^*$, and if the action is effective, then $A=B$.
\label{lem:As_Bs_A_B}
\end{lemma}

\begin{proof}
 We consider once again the map \eqref{eq_def_RGM} and we look at 
\[
  v=\Dsdd{ R(c(t),x) }{t}{0}
   =(dR)_{(e,x)}\Dsdd{ (c(t),x) }{t}{0},
\]
keeping in mind that $c(t)=e^{-tA}$. In order to treat this expression, we define
\begin{subequations}
\begin{align}
  \dpt{R_1}{G}{M},\quad  R_1(h)&=R(h,x),\\
  \dpt{R_2}{M}{M},\quad  R_2(y)&=R(g,y).
\end{align}
\end{subequations}
So
\[
  v=dR_1(X)+dR_2(0)=dR_1c'(0)
\]
and the assumption $A^*_x=B^*_x$ becomes $dR_1 A=dR_1 B$. This makes, for small enough $t$, $R_1(e^{tA}e^{-tB})=x\cdot e^{tA}e^{-tB}=x$; if the action is effective, it imposes $A=B$.
\end{proof}

\begin{lemma}
If we consider the action of a matrix group, $R_g$ acts on the fundamental field by
\[
  dR_g(A^*_{\xi})=\big( \Ad(g^{-1})A \big)^*_{\xi\cdot g}.
\]
\label{lem:dRgAstar}
\end{lemma}

\begin{proof}
Just notice that $e^{-t\Ad(g^{-1})A}=\AD_{g^{-1}}(e^{-tA})=g^{-1} e^{-tA}g$, thus
\begin{equation}
  \big( \Ad(g^{-1})A \big)^*_{\xi\cdot g}=\Dsdd{ \xi\cdot ge^{-t\Ad(g^{-1})A} }{t}{0}=dR_g(A^*_{\xi}).
\end{equation}
\end{proof}

%+++++++++++++++++++++++++++++++++++++++++++++++++++++++++++++++++++++++++++++++++++++++++++++++++++++++++++++++++++++++++++
\section{Exponential map}
%+++++++++++++++++++++++++++++++++++++++++++++++++++++++++++++++++++++++++++++++++++++++++++++++++++++++++++++++++++++++++++

A \defe{topological group}{topological!group} is a group $G$ equipped with a topological structure such that the maps $(x,y)\in G^2\to xy\in G$ and $x\in G\to x^{-1}\in G$ are continuous.

\begin{remark}\label{rem:ouvert}
From the existence of an unique inverse for any element of $G$, the multiplication and the inversion are also open maps.
\end{remark}

A \defe{Lie group}{Lie!group} is a group $G$ which is in the same times an analytic manifold such that the group operations (multiplication and inverse) are analytic. In particular, we \emph{do not} suppose that they are diffeomorphism.

The concept of normal neighbourhood will be widely used for the study of the relations between a Lie group and its algebra. Let $M$ be a differentiable manifold. If $V$ is a neighbourhood of zero in $T_pM$ on which the exponential $\dpt{\exp_p}{T_pM}{M}$ is a diffeomorphism, then $\exp_pV$ is  \defe{normal neighbourhood}{normal!neighbourhood} of $p$.

\begin{lemma}[\cite{Lie}]		\label{lemsur5d}
	Let $G$, $H$ be two Lie groups with algebras $\mG$ and $\mH$. Let $\dpt{\phi}{G}{H}$ be a homomorphism differentiable at $e$, the unit in $G$. Then for all $X\in\mG$, the following formula holds:
	\[
		\phi(\exp X)=\exp(d\phi_eX).
	\]
\end{lemma}

\begin{corollary}\label{Ad_e}
An useful formula:
\[
   \Ad(e^X)=e^{\ad X}.
\]
\end{corollary}

\begin{corollary}
Another useful corollary of lemma \ref{lemsur5d} is the particular case $\phi=\AD(e^X)$:
\[
   e^Xe^Ye^{-X}=e^{Ad(e^Y)X}.
\]
\label{cor:eXeYe-X}
\end{corollary}

\begin{proposition}
	Let $G$ be a connected Lie group.
	\begin{enumerate}

		\item
			All the left invariant vector fields are complete. That means that the map $X\mapsto  e^{X}$ is defined for every $X\in \mG$.
		\item
			The map $\exp\colon \mG\to G$ is a local diffeomorphism in a neighbourhood of $0$ in $\mG$.
	\end{enumerate}
\end{proposition}

\begin{proof}
	\begin{enumerate}

		\item
			The flow is a one parameter subgroup. Thus if $ e^{tX}$ is defined for $t\in[0,a]$, by composition, $ e^{2a}$ is defined. So $ e^{tX}$ is defined for every value of $t$ in $\eR$.
		\item
			Let us consider the manifold $G\times \mG$ and the vector field $\Xi$ defined by
			\begin{equation}
				\Xi_{(g,X)}=\tilde X_g\oplus 0\in T_g(G)\oplus T_X\mG\simeq T_{(g,X)}(G\times \mG).
			\end{equation}
			The flow of that vector field is given by
			\begin{equation}
				\Phi_t(g,X)=\big( g\exp(tX),X \big).
			\end{equation}
			In particular, $\Xi$ is a complete vector field, and we consider the global diffeomorphism
			\begin{equation}
				\begin{aligned}
					\Phi_1\colon G\times \mG&\to G\times \mG \\
					(g,X)&\mapsto \big( g\exp(X),X \big). 
				\end{aligned}
			\end{equation}
			On the point $(e,X)$ we have $\Phi_1(e,X)=(\exp(X),X)$. Thus the exponential is the projection on the first component of $\Phi_1(e,X)$ and we can write
			\begin{equation}
				\exp(X)=\pr_1\circ\Phi_1(e,X).
			\end{equation}
			It is a smooth function since both the projection and $\Phi_1$ are smooth.

			Now, the differential $(d\exp)_0$ is the identity on $\mG$, so that the theorem of inverse function makes $\exp$ a local diffeomorphism.
	\end{enumerate}	
\end{proof}


\begin{theorem}
For any $p\in M$, there exist a $\delta>0$ and a neighbourhood $W$ of $p$ in $M$ such that for every $q\in W$, we have

\begin{itemize}
\item $\exp_q$ is a diffeomorphism on $B\bdelta(0)\subset T_qM$,
\item $\exp_q B\bdelta(0)$ contains $W$
\end{itemize}
\end{theorem}
This theorem says that everywhere on a differentiable manifold, one can find a neighbourhood which is a normal neighbourhood of each of its points. Such a neighbourhood is said a \emph{totally} normal neighbourhood.

\begin{lemma}
In a Lie group, $e$ is an isolated fixed point for the inversion.
\end{lemma}

\begin{proof}
One can use an exponential map in a neighbourhood of $e$. In this neighbourhood, an element $g$ can be written as $g=e^X$ for a certain $X\in\lG$. The equality $g=g^{-1}$ gives (because the exponential is a diffeomorphism) $X=-X$, so that $X=0$ and $g=e$.
\end{proof}

\begin{lemma}
Let $\lG$ be a Lie algebra and $A$, a linear operator on $\lG$ (see as a common vector space) such that $\forall t\in\eR$, the map $e^{tA}$ is an automorphism of $\lG$. Then $A$ is a derivation of $\lG$.
\label{lem:autom_derr}
\end{lemma}

\begin{proof}
Let us consider $X$, $Y\in\lG$;  the assumption is 
\[
  e^{tA}[X,Y]=[e^{tA}X,e^{tA}Y]. 
\]
Since $e^{tA}$ is a linear map, it has a ``good behavior''\ with the derivations: 
\[
\Dsddc{e^{tA}[X,Y]}{t}{0}=\Dsddc{e^{tA}}{t}{0}[X,Y]=A[X,Y].
\]
Using on the other hand the linearity of $\ad$, we can see
\[
  (\ad(e^{tA}X))(e^{tA}Y)
\]
as a product ``matrix times vector''. Then
\begin{equation}
\begin{split}
  \Dsddc{[e^{tA}X,e^{tA}Y]}{t}{0}&=\Dsddc{(\ad e^{tA}X)Y}{t}{0}+\Dsddc{(\ad X)(e^{tA}Y)}{t}{0}\\
                                 &=(\ad AX)Y+(\ad X)(AY).
\end{split}
\end{equation}
Finally, $A[X,Y]=[AX,Y]+[X,AY]$.

\end{proof}

As notational convention, if $G$ and $H$ are Lie groups, their Lie algebra are denoted by $\lG$ and $\lH$.

\begin{lemma}		\label{LemAlgEtGroupesGenere}
	Let $\lG$ be a Lie algebra ans $\lS$ be a subset of $\lG$. The algebra of the group generated by $ e^{\lS}$ is the algebra generated by $\lS$.
\end{lemma}

\subsection{Invariant vector fields}\index{invariant!vector field}
%-----------------------------------

If $G$ is a Lie group, a vector field $\tilde X\in\Gamma^{\infty}(TG)$ is \defe{left invariant}{left invariant!vector field} if 
\begin{equation}
	(dL_g)\tilde X=\tilde X.
\end{equation}
To each element $X$ in the Lie algebra $\mG$, we have an associated left invariant vector field given on the point $g\in G$ by the path
\begin{equation}
	\tilde X_g(t)=ge^{tX}.
\end{equation}
In the same way, we associate a \defe{right invariant}{right!invariant!vector field} given by
\begin{equation}
	\utilde X_g(t)=e^{tX}g
\end{equation}
The invariance means that $(dL_h)_g\tilde X_g=\tilde X_{hg}$ and $(dR_h)_g\utilde X_g=\utilde X_{gh}$. The invariant vector fields are important because they carry the structure of the tangent space at identity (the Lie algebra). More precisely we have the following result:

\begin{theorem}
	The map $X\to X_e$ is a bijection between the left invariant vector fields on a Lie group and its Lie algebra $T_eG$.
\end{theorem}
Invariant vector fields are also often used in order to transport a structure from the identity of a Lie group to the whole group by $A_g(X_g)=A_e(dL_{g^{-1}}X_g)$ where $A_e$ is some structure and $X_g$, a vector at $g$.


\begin{proposition}
	Let $G$ be a Lie group and $\mG$ the vector space of its left invariant vector fields.
	\begin{enumerate}

		\item
			The map
			\begin{equation}
				\begin{aligned}
					\mG\colon &\to T_eG \\
					\tilde X&\mapsto \tilde X_e 
				\end{aligned}
			\end{equation}
			is a vector space isomorphism.
		\item
			We have $[\mG,\mG]\subset \mG$ and $\mG$ is a Lie algebra. Here, the commutator is the bracket of vector fields.

	\end{enumerate}
\end{proposition}
\begin{proof}
	No proof.
\end{proof}

A vector field \( X\) on a Lie group \( G\) is \defe{left invariant}{left invariant!vector field} if \( dL_g(X)=X\) for every \( g\in G\). Here \( L_g\colon G\to G\) is the left translation defined by \( L_g(h)=gh\). More explicitly, the left invariance is expressed by
\begin{equation}
    \Dsdd{ gX_h(t) }{t}{0}=X_{gh}
\end{equation}
where \( X_h(t)\) is the path defining the tangent vector \( X_h\in T_h G\).

We want to prove that the vector space of left invariant vector fields is isomorphic to the tangent vector space \( T_eG\) to \( G\) at identity. If \( X\in T_eG\), we introduce the left invariant vector field \( X^L=dLX\), more explicitly:
\begin{equation}
    X^L_g=\Dsdd{ gX(t) }{t}{0}.
\end{equation}
Then we consider \( \alpha_X\colon I\to G\) the integral curve of maximal length to \( X^L\) trough \( X_e\). Here, \( I\) is the interval on which \( \alpha_X\) is defined. This is the solution of
\begin{subequations}
    \begin{numcases}{}
        \Dsdd{ \alpha_X(t_0+t) }{t}{0}=X_{\alpha_X(t_0)}\\
        \alpha_X(0)=e.
    \end{numcases}
\end{subequations}

\begin{proposition}     \label{PROPooWEYCooCvyHNr}
    Let \( X\in T_eG\). The integral curve has \( \eR\) as domain and for every \( s,t\in\eR\),
    \begin{equation}
        \alpha_X(s+t)=\alpha_X(s)\alpha_X(t).
    \end{equation}
\end{proposition}

\begin{proof}
    Let \( \alpha\) be any integral curve for \( X^L\) and \( y\in G\). If we put \( \alpha_1(t)=y\alpha(t)\), we have
    \begin{equation}
        \Dsdd{ \alpha_1(t) }{t}{0}=X^L_y,
    \end{equation}
    so that \( \alpha_1\) is an integral curve for \( X^L\) trough the point \( y\). 

    Let now \( I\) be the maximal domain of \( \alpha_X\), and \( t_1\in I\). If we set \( x_1=\alpha_X(t_1)\), the path
    \begin{equation}
         \alpha_1(t)=x_1\alpha_X(t)
    \end{equation}
    is an integral curve of \( X^L\) trough \( x_1\) and has the same maximal definition domain \( I\). On the other hand, the maximal integral curve starting at \( e\) being \( \alpha_X\), the maximal integral curve starting at \( \alpha_X(t_1)\) is
    \begin{equation}
        \alpha_2\colon t\mapsto \alpha_X(t+t_1).
    \end{equation}
    Its domain is \( I-t_1\), but since it starts at \( x_1\), it has to be the same as \( \alpha_1\), then \( I\subset I-t_1\) which proves that \( I=\eR\).

    For each \( s\) and \( t\) in \( \eR\), the maximal integral curve starting at \( \alpha_X(s)\) can be written as
    \begin{equation}
        c(t)=\alpha_X(s)\alpha_X(t)
    \end{equation}
    as well as
    \begin{equation}
        d(t)=\alpha_X(s+t),
    \end{equation}
    so again by unicity, \( \alpha_X(s+t)=\alpha_X(s)\alpha_X(t)\).
\end{proof}


%---------------------------------------------------------------------------------------------------------------------------
\subsection{Integral curve and exponential}
%---------------------------------------------------------------------------------------------------------------------------

\begin{definition}
    If \( \alpha_X\) is the integral curve to \( X^L\), we define the \defe{exponential}{exponential!on Lie group}
    \begin{equation}        \label{EqdefExpoLieTgFGp}
        \begin{aligned}
            \exp\colon T_eG&\to G \\
            X&\mapsto \alpha_X(1). 
        \end{aligned}
    \end{equation}
\end{definition}

This definition works on Lie groups thanks to the group structure that allows to build a natural vector field \( X^L\) from the data of a single vector \( X\). On general manifolds, one has not a notion of exponential. However, if one has a Riemannian manifold, one consider the geodesic.

In the case of groups for which the Killing form defines a scalar product, the notion of exponential associated with the Riemannian structure propagated from the Killing form coincides with the definition \eqref{EqdefExpoLieTgFGp}.

A very important point\cite{ooOLNIooDLmxkR} is that when \( G\) is acting on $M$, one can reconstruct the action of \( G\) only knowing the action of \( \lG\). Let \( X\in \lG\) and \( x\in M\). We consider the path
\begin{equation}
    \begin{aligned}
        \gamma\colon \eR&\to M \\
        t&\mapsto \exp(tX)(x). 
    \end{aligned}
\end{equation}
This map satisfies \( \gamma(0)=x\). We also have, using proposition \ref{PROPooWEYCooCvyHNr},
\begin{equation}
    \gamma(t_0+u)=\big( \exp(uX)\circ\exp(t_0X)\big)(x)=\exp(uX)\gamma(t_0).
\end{equation}
In that, we used the fact that \( G\) acts on \( M\), so that we have transformed the product inside the group \( \exp\big( (u+t_0)X \big)= \exp(uX)\exp(t_0x) \) into a composition of map.  Then
\begin{equation}
    \Dsdd{ \gamma(t_0+u) }{u}{0}=X\big( \gamma(t_0) \big)
\end{equation}
We conclude that \( \gamma\) satisfies the differential equation
\begin{equation}        \label{EQooFGSIooUplbmN}
    \gamma'(t)=-X\big( \gamma(t) \big).
\end{equation}
When \( M=\eR^n\), the Cauchy-Lipschitz theorem \ref{ThokUUlgU} provides unicity of the solution on a maximal domain providing the map \( X\colon \eR^n\to \eR^n\) has nice properties.

\begin{normaltext}      \label{NORMooMGAUooIoLtjW}
    If we have to determine the transformations of \( \eR^n\) that satisfies some properties, the strategy is then the following :
    \begin{itemize}
        \item Suppose the searched group to be a connected Lie group.
        \item Write the condition with the groupe element \( \exp(tX)\) and differentiate with respect to \( t\). This point is what physicist call «consider an infinitesimal transformation and neglect the higher order terms».
        \item This provides an equation for \( X\). Typically a differential equation for the map \( X\colon \eR^n\to \eR^n\). Solve it.
        \item The group action is then retrieved solving the differential equation \eqref{EQooFGSIooUplbmN}.
    \end{itemize}
    Using that technique we will determine the isometries of \( \eR^n\) in proposition \ref{PROPooDVIWooAFDNPy} and determine the conformal group around definition \ref{DEFooVKNBooFBWQQM}.  % position 10906-29466 : provides a more precise reference to the result instead of the definition.
\end{normaltext}

\begin{remark}
    When the Lie algebra is made of linear transformations, the last differential equation to solve is actually exponentiating a matrix.
\end{remark}

%--------------------------------------------------------------------------------------------------------------------------- 
\subsection{Example : determining the smooth isometries of the flat vector space}
%---------------------------------------------------------------------------------------------------------------------------

We know from theorem \ref{ThoDsFErq} that the isometries of \( \eR^n\) are affine functions. We give now an alternative proof of that result.

\begin{proposition}     \label{PROPooDVIWooAFDNPy}
    The smooth\footnote{In fact we only need \( C^2\).} isometries of \( (\eR^n,\| . \|)  \)  are affine maps.
\end{proposition}

\begin{proof}
    The condition for a diffeomorphism \( \phi\colon \eR^n\to \eR^n\) to be an isometry is
    \begin{equation}        \label{EQooRKYWooFIKfYZ}
        \| \phi(x)- \phi(y) \|^2=\| x-y \|^2.
    \end{equation}
    We write \( \phi_t(x)= e^{-tX}x\) and take the derivative of \eqref{EQooRKYWooFIKfYZ} with respect to \( t\) at \( t=0\) taking into account that \( \phi_0(x)=x\):
    \begin{equation}        \label{EQooXEKMooGOktOj}
        \big( X(y)-X(x) \big)\cdot (x-y)=0.
    \end{equation}
    We used the fact that \( \Dsdd{ \phi_t(x) }{t}{0}=-X(x)\).

    We write the condition \eqref{EQooXEKMooGOktOj} with \( tx\) and take the derivative with respect to \( t\) : \( dX_0(x)\cdot y+X(y)\cdot x=0\). The same with \( y\) gives
    \begin{equation}
        dX_0(x)\cdot y+dX_0(y)\cdot x=0.
    \end{equation}
    Taking \( x=e_i\) and \( y=e_j\) this equation reads
    \begin{equation}
        \frac{ \partial X_j }{ \partial x_i }+\frac{ \partial X_i }{ \partial x_j }=0.
    \end{equation}
    With \( i=j\) we get \( \frac{ \partial X_i }{ \partial x_i }=0\). The we compute
    \begin{equation}
        \frac{ \partial  }{ \partial x_j }\frac{ \partial X_i }{ \partial x_j }=-\frac{ \partial  }{ \partial x_j }\left( \frac{ \partial X_j }{ \partial x_i } \right)=-\frac{ \partial  }{ \partial x_i }\frac{ \partial X_j }{ \partial x_j }=0.
    \end{equation}
    We used the fact that \( X_j\) is of class \( C^2\) in order to permute the derivatives (lemma \ref{LemPermDerrxyz}). We proved that 
    \begin{equation}
        \frac{ \partial^2 X_i  }{ \partial x_j }=0
    \end{equation}
    for all \( i,j\). Thus \( X\) is linear.
\end{proof}

\section{Universal enveloping algebra}  \label{subsec:env_alg}
%----------------------------------------

Let $\mA$ be a Lie algebra. One knows that the composition law $(X,Y)\to[X,Y]$ is often non associative. In order to build an associative Lie algebra which ``looks like''\ $\mA$, one considers $T(\mA)$, the tensor algebra of $\mA$ (as vector space) and $\mJ$ the two-sided ideal in $T(\mA)$ generated by elements of the form 
\[
   X\otimes Y-Y\otimes X -[X,Y]
\]
for $X$, $Y\in\mA$. The \defe{universal enveloping algebra}{universal!enveloping algebra} of $\mA$ is the quotient \nomenclature{$U(\mA)$}{Universal enveloping algebra} 
\begin{equation}
     U(\mA)=T(\mA)/\mJ.
\end{equation}
For $X\in\mA$, we denote by $X^*$\nomenclature{$X^*$}{Image of a tensor in the universal enveloping algebra} the image of $X$ by canonical projection $\dpt{\pi}{T(\mA)}{U(\mA)}$ and by $1$ the unit in $U(\mA)$. One has $1\neq 0$ if and only if $\mA\neq\{0\}$.

\begin{proposition}[\cite{Helgason}]
Le $V$ be a vector space on $K$. Then there is a natural bijection between the representations of $\mA$\index{representation! of $U(\mA)$} on $V$ and the ones of $U(\mA)$ on $V$. If $\rho$ is a representation of $\mA$ on $V$, the corresponding $\rho^*$ of $U(\mA)$ is given by
\[
   \rho(X)=\rho^*(X^*)
\]
($X\in\mA$).
\end{proposition}

Let $\{X_1,\ldots,X_n\}$ be a basis of $\mA$. For a $n$-uple of complex numbers $(t_i)$, one defines
\begin{equation}
X^*(t)=\sum_{i=1}^nt_iX^*_i.
\end{equation}
On the other hand, we consider a $n$-uple of positive integers $M=(m_1+\ldots m_n)$, and the notation 
\begin{equation}
\begin{split}
   |M|&=m_1+\ldots+m_n\\
   t^M&=t_1^{m_1}\cdots t_n^{m_n}.
\end{split}
\end{equation}

When $|M|>0$, we denote by $X^*(M)\in U(\mA)$ the coefficient of $t^M$ in the expansion of $(|M|!)^{-1} (X^*(t))^{|M|}$. If $|M|=0$, the definition is $X^*(0)=1$. Once again a proposition without proof.

\begin{proposition}
The smallest vector subspace of $U(\mA)$ which contains all the elements of the form $X^*(M)$ is $U(\mA)$ itself:
\[
   U(\mA)=\Span\{ X^*(M):M\in \eN^n \}.
\]
\end{proposition}

\begin{corollary} \label{cor:/24}
    Let $\mA$ be a Banach algebra of dimension $n$, $\mB$ a Banach subalgebra of dimension $n-r$ and a basis $\{X_1,\ldots,X_n\}$ of $\mA$ such that the $n-r$ last basis vectors are in $\mB$. We denotes by $B$ the vector subspace of $U(\mA)$ spanned by the elements of the form $X^*(M)$ with $m=(0,\ldots,0,m_{r+1},\ldots,m_n)$. Then $B$ is a subalgebra of $U(\mA)$.
\end{corollary}

\begin{definition}
Two Lie groups $G$ and $G'$ are \defe{isomorphic}{isomorphism!of Lie groups} when there exists a differentiable group isomorphism between $G$ and $G'$.

They are \defe{locally isomorphic}{locally!isomorphic!Lie groups} when there exists neighbourhoods $\mU$ and $\mU'$of $e$ and $e'$ and a differentiable diffeomorphism $\dpt{f}{\mU}{\mU'}$ such that

$\forall x,y,xy\in\mU$, $f(xy)=f(x)f(y)$, \\and

$\forall x',y',x'y'\in\mU'$, $f^{-1}(x'y')=f^{-1}(x')f^{-1}(y')$.
\end{definition}

The following universal property of the \emph{universal} enveloping algebra explains the denomination:
\begin{proposition}
Let $\dpt{\sigma}{\mG}{\mU(\mG)}$ the canonical inclusion and $A$ an unital complex associative algebra. A linear map $\dpt{\varphi}{\mG}{A}$ such that
\begin{equation}
\varphi[X,Y]=\varphi(X)\varphi(Y)-\varphi(Y)\varphi(X)
\end{equation}
can be extended in only one way to an algebra homomorphism $\dpt{\varphi_0}{\mU(\mG)}{A}$ such that $\varphi_0\circ\sigma=\varphi$ and $\varphi(1)=1$
\label{prop:extunifmap}
\end{proposition}
For a proof, see \cite{Knapp_reprez}.

\subsection{Adjoint map in \texorpdfstring{$\mU(\mG)$}{U(G)}}   \label{ssadjunif}
%/////////////////////////////////////////////

We know that $\dpt{ \Ad(g) }{ \mG }{ \mG }$ fulfils
\[ 
  \Ad(g)[X,y]=[  \Ad(g)X,\Ad(g)Y  ],
\]
and we can define $\dpt{ \Ad(g) }{ \mG }{ \mU(\mG) }$ by $\Ad(g)X=X$ where in the right hand side, $X$ denotes the class of $X$ for the quotients of the tensor algebra which defines the universal enveloping algebra.

When $[A,B]$ is seen in $\mU(\mG)$, we have $[A,B]=A\otimes B-B\otimes A$. Then $\dpt{ \Ad(g) }{ \mG }{ \mU(\mG) }$ fulfils proposition \ref{prop:extunifmap} and is extended in an unique way to $\dpt{ \Ad(g) }{ \mU(\mG) }{ \mU(\mG) }$ with $\Ad(g)1=1$.
 
\begin{lemma} 
    If $D\in\mU(\mG)$, the following properties are equivalent:
    \begin{itemize}
        \item $D\in\mZ(\mG)$
        \item $D\otimes X=X\otimes D$ for all $X\in \mG$
        \item $e^{\ad X}D=D$ for all $X\in\mG$
        \item $\Ad(g)D=D$ for all $g\in G$.
    \end{itemize}
     \label{lem:equivDAd}
\end{lemma}

%---------------------------------------------------------------------------------------------------------------------------
\subsection{Invariant fields}
%---------------------------------------------------------------------------------------------------------------------------

If $X\in\lG$, we have the associated left invariant vector field on $G$ given by $\tilde X_x=dL_xX$. That field is left invariant as operator on the functions because
\begin{equation}
    \tilde X_x(u)=\tilde X_e(L^*_xu)
\end{equation}
as the following computation shows
\begin{equation}
        \tilde X_e(L^*u)=\Dsdd{ (L_x^*u)\big(  e^{tX} \big) }{t}{0}
        =\Dsdd{ u\big( x e^{tX} \big) }{t}{0}
        =\Dsdd{ u\big( \tilde X_x(t) \big) }{t}{0}
        =\tilde X_x(u)
\end{equation}
because the path defining $\tilde X_x$ is $x e^{tX}$.

We can perform the same construction in order to build left invariant fields based on $\mU(\lG)$. If $X$ and $Y$ are elements of $\lG$, the  differential operator on $ C^{\infty}(G)$ associated to $XY\in\mU(\lG)$ is given by
\begin{equation}
    (XY)(f)=\DDsdd{ f\big( X(s)Y(t) \big) }{t}{0}{s}{0}
\end{equation}
The path defining the field $\widetilde{XY}$ is
\begin{equation}
    \widetilde{XY}_x=xX(s)Y(t).
\end{equation}
Thus we have
\begin{equation}        \label{EqInvarUgField}
    \widetilde{(XY)}_e(L^*u)=\widetilde{(XY)}_xu
\end{equation}

\begin{lemma}       \label{LemAdesthioo}
    If \( X,Y\in\lG\) we have
    \begin{equation}
        [\ad(X),\ad(Y)]=\ad([X,Y]).
    \end{equation}
\end{lemma}

\begin{proof}
    Let \( f\in\lG\) and compute the action of \( [\ad(X),\ad(Y)]\):
    \begin{subequations}
        \begin{align}
            [\ad(X),\ad(Y)]f&=\ad(X)[Yf,fY]-\ad(Y)(Xf-fX)\\
            &=(XY-YX)f+f(YX-XY)\\
            &=\ad([X,Y])f.
        \end{align}
    \end{subequations}
\end{proof}

%---------------------------------------------------------------------------------------------------------------------------
                    \subsection{Representation of Lie groups}
%---------------------------------------------------------------------------------------------------------------------------

\begin{proposition}
    Let $G$ be a Lie group and $\mG$ its Lie algebra. A representation $\varphi\colon G\to \End(V)$ of the group induces a representation $\phi\colon \mU(\mG)\to \End(V)$ of the universal enveloping algebra with the definitions
    \begin{subequations}
        \begin{align}
            \phi(X)     &=d\varphi_e(X),\\
            \phi(XY)    &=\phi(X)\circ\phi(Y)
        \end{align}
    \end{subequations}
    where $e$ is the unit in $G$ and $X$, $Y$ are any elements of $\mG$.
\end{proposition}

\begin{proof}
    We have
    \begin{equation}
        \phi(X)=\Dsdd{ \varphi( e^{tX})v }{t}{0}=d\varphi_e(X)v.
    \end{equation}
    Notice that, by linearity of the action of $\varphi( e^{tX})$ on $v$, one can leave $v$ outside the derivation. Now, neglecting the second order terms in $t$ in the derivative, and using the Leibnitz formula, we have
    \begin{equation}
        \begin{aligned}[]
            \phi([X,Y])v    &=  \Dsdd{ \varphi( e^{tXY} e^{-tXY}) }{t}{0}v\\
                    &=  \Dsdd{ \varphi( e^{tXY})\varphi(\mtu) }{t}{0}v+\Dsdd{ \varphi(\mtu)\varphi( e^{-tXY}) }{t}{0}v\\
                    &=  \phi(XY)v-\phi(YX)v\\
                    &=  \big( \phi(X)\phi(Y)-\phi(Y)\phi(X) \big)v\\
                    &=  [\phi(X),\phi(Y)]v,
        \end{aligned}
    \end{equation}
    which is the claim.
\end{proof}

\section{Lie subgroup}
%------------------------

Now a great theorem without proof:
\begin{theorem} \label{tho:loc_isom}
Two Lie groups are locally isomorphic if and only if their Lie algebras are isomorphic.
\end{theorem}
%TODO : a proof

\begin{definition}
Let $G$ be a Lie group. A submanifold $H$ of $G$ is a \defe{Lie subgroup}{Lie!subgroup} of $G$ when
\begin{enumerate}
\item as group, $H$ is a subgroup of $G$,
\item $H$ is a topological group.
\end{enumerate}
\end{definition}

\begin{remark}\label{rem:sub_Lie}
The definition doesn't include that $H$ has the same topology as $G$ (or the induced one). In some literature, the definition of a Lie subgroup include the fact for $H$ to be a topological subgroup. This choice make some proofs much easier and others more difficult; be careful when you try to compare different texts.
\end{remark}

\begin{lemma}
A Lie subgroup is a Lie group.
\end{lemma}
(without proof)

\begin{theorem}		\label{ThoSubGpSubAlg}		\label{tho:gp_alg}
If $G$ is a Lie group, then
\begin{enumerate}
\item\label{ThoSubGpSubAlgi} if $\lH$ is the Lie algebra of a Lie subgroup $H$ of $G$, then it is a subalgebra of $\lG$,
\item Any subalgebra of $\lG$ is the Lie algebra of one and only one connected Lie subgroup of $G$.
\end{enumerate}

\begin{probleme}
À mon avis, il faut dire ``connexe et simplement connexe'', et non juste ``connexe''.
\end{probleme}

\end{theorem}
\begin{proof}

\subdem{First item}
Let $\dpt{i}{H}{G}$ be the identity map; it is a homomorphism from $H$ to $G$, thus $di_e$ is a homomorphism from $\lH$ to $\lG$. Conclusion: $\lH$ is a subalgebra of $\lG$. 

\subdem{Characterization for $\lH$}
Before to go on with the second point, we derive an important characterization of $\lH$ :
\begin{equation}\label{eq:path_alg}
\lH=\{X\in\lG:\text{the map } t\to\exp tX\text{ is a path in $H$}\}.
\end{equation}
For that, consider $\dpt{\exp_H}{\lH}{H}$ and $\dpt{\exp_G}{\lG}{G}$; from unicity of the exponential, for any $X\in\lH$, $\exp_HX=\exp_GX$, so that one can simply write ``$\exp$''\ instead of ``$\exp_h$''\ or ``$\exp_G$''.

Now, if $X\in\lH$, the map $t\to\exp tX$ is a curve in $H$. But it is not immediately clear that such a curve in $H$ is automatically build from a vector in $\lH$ rather than in $\lG$.  More precisely, consider a $X\in\lG$ such that $t\to\exp tX$ is a path (continuous curve) in H. By lemma \ref{lem:var_cont_diff}, the map $t\to\exp tX$ is differentiable and thus by derivation, $X\in\lH$.
The characterisation \eqref{eq:path_alg} is proved.

Thus $\lH$ is a Lie subalgebra of $\lG$.

\subdem{Second item}
For the second part, we consider $\lH$ any subalgebra of $\lG$ and $H$, the smallest subgroup of $G$ which contains $\exp\lH$. We also consider a basis $\{X_1,\ldots,X_n\}$ of $\lG$ such that $\{X_{r+1},\ldots,X_n\}$ is a basis of $\lH$.

By corollary \ref{cor:/24}, the set of linear combinations of elements of the form $X(M)$ with $M=(0,\ldots,0,m_{r+1},\ldots,m_r)$ form a subalgebra of $U(\lG)$. If $X=x_1X_1+\ldots+x_nX_n$, we define $|X|=(x_1^2+\ldots+x_n^2)^{1/2}$ ($x_i\in\eR$).

Let us consider a $\delta>0$ such that $\exp$ is a diffeomorphism (normal neighbourhood) from $B_{\delta}=\{X\in\lG:|X|<\delta\}$ to a neighbourhood $N_e$ of $e\in G$ and such that $\forall x,y,xy\in N_e$,
\begin{equation}\label{eq:coord_xy}
   (xy)_k=\sum_{M,N}C^{[k]}_{MN}x^My^N
\end{equation}
holds\footnote{The validity of this second condition is assured during the proof of theorem \ref{tho:loc_isom} which is not given here.}. We note $V=\exp(\lH\cap B_{\delta})\subset N_e$. The map
\[
   \exp(x_{r+1}X_{r+1}+\ldots+x_nX_n)\to(x_{r+1},\ldots,x_n)
\]
is a coordinate system on $V$ for which $V$ is a connected manifold. But $\lH\cap B_{\delta}$ is a submanifold of $B_{\delta}$, then $V$ is a submanifold of $N_e$ and consequently of~$G$.

Let $x$, $y\in V$ such that $xy\in N_e$ (this exist: $x=y=e$); the canonical coordinates of $xy$ are given by \eqref{eq:coord_xy}. Since $x_k=y_k=0$ for $1\leq k\leq r$, $(xy)_k=0$ for the same $k$ because for $(xy)_k$ to be non zero, one need $m_1=\ldots=m_r=n_1=\ldots=n_r=0$ -- otherwise, $x^M$ or $y^N$ is zero. Now we looks at $C^{[k]}_{MN}$ for such a $k$ (say $k=1$ to fix ideas) : $[k]=(\delta_{11},\ldots,\delta_{1k})=(1,0,\ldots,0)$ and by definition of the $C$'s,
\[
   X(M)X(N)=\sum_PC_{MN}^PX(P).
\]
But we had seen that the set of the $X(A)$ with $A=(0,\ldots,0,a_{r+1},\ldots,a_n)$ form a subalgebra of $U(\lG)$. Then, only terms with $P=(0,\ldots,0,p_{r+1},\ldots,p_n)$ are present in the sum; in particular, $C_{MN}^{[k]}=0$ for $k=1,\ldots,r$. Thus $VV\cap N_e\subset V$.

The next step is to consider $\mV$, the set of all the subset of $H$ whose contains a neighbourhood of $e$ in $V$. We can check that this fulfils the six axioms of a topological group\index{topological!group} :

\begin{enumerate}
\item The intersection of two elements of $\mV$ is in $\mV$;
\item the intersection of all the elements of $\mV$ is $\{e\}$;
\item any subset of $H$ which contains a set of $\mV$ is in $\mV$;
\item If $\mU\in\mV$, there exists a $\mU_1\in\mV$ such that $\mU_1\mU_1\subset\mU$ because $VV\cap N_e\subset V$;
\item if $\mU\in\mV$, then $\mU^{-1}\in\mV$ because the inverse map is differentiable and transforms a neighbourhood of $e$ into a neighbourhood of $e$;
\item if $\mU\in\mV$ and $h\in H$, then $h\mU h^{-1}\in\mV$.
\end{enumerate}

To see this last item, we denote by $\log$ the inverse map of $\dpt{\exp}{B_{\delta}}{N_e}$. By definition of $V$, it sends $V$ on $\lH\cap B_{\delta}$. If $X\in\lG$, there exists one and only one $X'\in\lG$ such that $he^{tX}h^{-1}=e^{tX'}$ for any $t\in\eR$. Indeed we know that $he^{X}h^{-1}=e^{\Ad_hX}$, then $X'$ must satisfy $e^{tX'}=e^{\Ad_htX}$. If it is true for any $t$, then, by derivation, $X'=\Ad_hX$.

The map $X\to X'$ is an automorphism of $\lG$ which sent $\lH$ on itself. So one can find a $\delta_1$ with $0<\delta_1<\delta$ such that 
\[
   h\exp({B_{\delta_1}\cap\lH})h^{-1}\subset V.
\]
Indeed, $he^{\lH} h^{-1}\subset\lH$, so that taking $\delta_1<\delta$, we get the strict inclusion. We can choose $\delta_1$ even smaller to satisfy $he^{B_{\delta_1}}h^{-1}\subset N_e$. Since the map $X\to\log(he^{X}h^{-1})$ from $B_{\delta_1\cap\lH}$ to $B_{\delta}\cap\lH$ is regular, the image of $B_{\delta_1}\cap\lH$ is a neighbourhood of $0$ in $\lH$. Thus $he^{B_{\delta_1}\cap\lH}h^{-1}$ is a neighbourhood of $e$ in $V$. Finally, $h\mU h^{-1}\in\mV$ and the last axiom of a topological group is checked.

This is important because there exists a topology on $H$ such that $H$ becomes a topological group and $\mV$ is a family of neighbourhood of $e$ in $H$. In particular, $V$ is a neighbourhood of $e$ in $H$.

For any $z\in G$, we define the map $\dpt{\phi_z}{zN_e}{B_{\delta}}$ by
\begin{equation}
  \phi_z(ze^{x_1X_1+\ldots+x_nX_n})=(x_1,\ldots,x_n),
\end{equation}
and we denote by $\varphi_z$ the restriction of $\phi_z$ to $zV$. If $z\in H$, then $\varphi_z$ sends the neighbourhood $zV$ of $z$ in $H$ to the open set $B_{\delta}\cap\lH$ in $\eR^{n-r}$. Indeed, an element of $zV$ is a $ze^Z$ with $Z\in\lH\cap B_{\delta}$ which is sent by $\varphi_z$ to an element of $\lH\cap B_{\delta}$. (we just have to identify $x_1X_1+\ldots+x_nX_n$ with $(x_1,\ldots,x_n)$).

Moreover, if $z_1,z_2\in H$, the map $\varphi_{z_1}\circ\varphi_{z_2}^{-1}$ is the restriction to an open subset of $\lH$ of $\phi_{z_1}\circ\phi_{z_2}$. Then $\varphi_{z_1}\circ\varphi_{z_2}^{-1}$ is differentiable. Conclusion: $(H,\varphi_z: z\in H)$ is a differentiable manifold.

Recall that the definition of $\lH$ was to be a subalgebra of $\lG$; therefore $V=e^{\lH\cap B_{\delta}}$ is a submanifold of $G$. But the left translations are diffeomorphism of $H$ and $H$ is the smallest subgroup of $G$ containing $e^{\lH}$. Thus $H$ is a manifold on which the multiplication is diffeomorphic and consequently, $H$ is a Lie subgroup of $G$.

Rest to prove that the Lie algebra of $H$ is $\lH$ and the unicity part of the theorem.

We know that $\dim H=\dim\lH$ and moreover for $i>r$, the map $t\to\exp tX_i$ is a curve in $H$. Now, the fact that $\lH$ is the set of $X\in\lG$ such that $t\to\exp tX$ is a path in $H$ show that $X_i\in\lH$. Then the Lie algebra of $H$ is $\lH$ and $H$ is a connected group because it is generated by $\exp\lH$ which is a connected neighbourhood of $e$ in $H$.

We turn our attention to the unicity part. Let $H_1$ be a connected Lie subgroup of $G$ such that $T_eH_1=\lH$. Since $\exp_{\lH}X=\exp_{\lH_1}X$, $H=H_1$ as set. But $\exp$ is a differentiable diffeomorphism from a neighbourhood of $0$ in $\lH$ to a neighbourhood of $e$ in $H$ and $H_1$, so as Lie groups, $H$ and $H_1$ are the same.


\end{proof}

We state a corollary without proof :

\begin{corollary} 
If $H_1$ and $H_2$ are two Lie subgroups of the Lie group $G$  such that $H_1=H_2$ as topological groups, then $H_1=H_2$ as Lie groups.
 \label{cor:top_subgroup}
\end{corollary}

\begin{proposition}
Let $G_1$ and $G_2$ be two Lie groups with same Lie algebra such that $\pi_0(G_1)=\pi_0(G_2)$ and $\pi_1(G_1)=\pi_1(G_2)$, then $G_1$ and $G_2$ are isomorphic.
\end{proposition}

\begin{proof}
The assumptions of equality of Lie algebras and of the $\pi_0$ make that the universal covering $\tilde G_1$ and $\tilde G_2$ of $G_1$ and $G_2$ are the same. But we know that $G_i=\tilde G_i/\pi_1(G_i)$. Now equality $\pi_1(G_1)=\pi_1(G_2)$ concludes that $G_1=G_2$.
\end{proof}

\begin{lemma}
Let $\lG$ admit a direct sum decomposition (as vector space) $\lG=\lM\oplus\lN$. Then there exists open and bounded neighbourhoods $\mU_m$ and $\mU_n$ of $0$ in $\lM$ and $\lN$ such that the map 
		\begin{equation}
		\begin{aligned}
			\phi \colon \mU_m\times\mU_n &\to G\
			(A,B)&\mapsto e^Ae^B
		\end{aligned}
	\end{equation}	
is a diffeomorphism between $\mU_m\times\mU_n$ and an open neighbourhood of $e$ in $G$.
 \label{lem:decomp}
\end{lemma}


\begin{proof}
Let $\{X_1,\ldots,X_n\}$ be a basis of $\lG$ such that $X_i\in\lM$ for $1\leq i\leq r$ and $X_j\in\lN$ for $r<j\leq n$. We consider $\{t_1,\ldots,t_n\}$, the canonical coordinates of $\exp(x_1X_1+\ldots+x_rX_r)\exp(x_{r+1}X_{r+1}+\ldots+x_nX_n)$ in this coordinate system. By properties of the exponential, the function $\varphi_j$ defined by $t_j=\varphi_j(x_1,\ldots,x_n)$ is differentiable at $(0,\ldots,0)$. If $x_i=\delta_{ij}s$, then $t_i=\delta_{ij}s$ and the Jacobian of 
\[
   \dsd{(\varphi_1,\ldots,\varphi_n)}{(x_1,\ldots,x_n)}
\]
is $1$ for $x_1=\ldots=x_n=0$. Thus $d\varphi_e$ is a diffeomorphism and so $\varphi$ is a locally diffeomorphic.
\end{proof}

\begin{theorem}  
Let $G$ be a Lie group whose Lie algebra is $\lG$ and $H$, a closed subgroup (not specially a \emph{Lie} subgroup) of $G$. Then there exists one and only one analytic structure on $H$ for which $H$ is a topological Lie subgroup of $G$.
\label{tho:diff_sur_ferme}
\end{theorem}

\begin{remark}
A \textit{topological} Lie subgroup\index{topological!Lie subgroup} is stronger that a common Lie subgroup because it needs to be a topological subgroup: it must carry \emph{exactly} the induced topology. In our definition of a Lie group, this feature doesn't appears. 
\end{remark}

\begin{proof}
   Let $\lH$ be the subspace of $\lG$ defined by 
\begin{equation}\label{eq:lH_de_G}
  \lH=\{X\in\lG\tq \forall t\in\eR,\, e^{tX}\in H\}.
\end{equation}
We begin to show that $\lH$ is a subalgebra of $\lG$; i.e. to show that $t(X+Y)\in\lH$ and $t^2[X,Y]\in\lH$ if $X$, $Y\in\lH$. Remark that $X\in\lH$ and $s\in\eR$ implies $sX\in\lH$. Consider now $X$, $Y\in\lH$ and the classical formula :
\begin{subequations}
\begin{align}
\left(  \exp(\frac{t}{n}X)\exp(\frac{t}{n}Y)  \right )^n
                       =\exp( t(X+Y)+\frac{t^2}{2n}[X,Y]+o(\frac{1}{n^2}) ),\\
\left(  \exp(-\frac{t}{n}X)\exp(-\frac{t}{n}Y)\exp(\frac{t}{n}X)\exp(\frac{t}{n}Y)   \right)^{n^2}
                       =\exp\left( t^2[X,Y]+o(\frac{1}{n})\right).
\end{align}
\end{subequations}
The left hand side of these equations are in $H$ for any $n$; but, since $H$ is closed, it keeps in $H$ when $n\to\infty$. The right hand side, at the limit, is just $\exp(t(X+Y))$ and $\exp(t^2[X,Y])$, which keeps in $H$ for any $t$. Thus $X+Y$ and $[X,Y]$ belong to $\lH$. The space $\lH$ is thus a Lie subalgebra of $\lG$.

Let $H^*$ be the connected Lie subgroup of $G$ whose Lie algebra is $\lH$ (existence and unicity from \ref{tho:gp_alg}). From the proof of theorem \ref{tho:gp_alg}, we know that $H^*$ is the smallest subgroup of $G$ containing $\exp\lH$, then it is made up from products and inverses of elements of the type $e^X$ with $X\in\lH$, and thus is is included in $H$ by definition of $\lH$. So, $H^*\subset H$.
  
We will show that if we put on $H^*$ the induced topology from $G$ and if $H_0$ denotes the identity component of $H$, then $H^*=H_0$ as topological groups. For this, we first have to show the equality as set and then prove that if $N$ is a neighbourhood of $e$ in $H^*$, then it is a neighbourhood of $e$ in $H_0$. In facts, the equality as set can be derives from this second fact. Indeed, since $H_0$ is a connected topological group, it is generated by any neighbourhood of $e$, so if one can show that any neighbourhood $N$ of $e$ in $H^*$ is a neighbourhood of $e$ in $H$, then $H^*$ is a neighbourhood of $e$ in $H_0$ and then $H_0$ should be generated by $H^*$, so that $H_0\subset H^*$ (as set). Moreover, the most general element of $H^*$ is product and inverse of $e^X$ with $X\in\lH$ and $e^X$ is connected to $e$ by the path $e^{tX}$ ($\dpt{t}{1}{0}$). Then $H^*\subset H_0$, and $H^*=H_0$ as set. Immediately, $H^*=H_0$ as topological groups from our assertion about neighbourhoods of $e$. Let us now prove it.
  
We consider a neighbourhood $N$ of $e$ in $H^*$ and suppose that this is not a neighbourhood of $e$ in $H$. Thus there exists a sequence $c_k\in H\setminus N$ such that $c_k\to e$ in the sense of the topology on $G$. Indeed, a neighbourhood of $e$ in the sense of $H$ must contains at least a point which is not in $N$ because if we have an open set of $H$ around $e$ included in $N$, then $N$ is a neighbourhood of $e$ for $H$. So we consider a suitable sequence of such open set around $e$ and one element not in $N$ in each of them. There is the $c_k$'s\quext{Je crois qu'on utilise l'axiome du choix.}.

Using lemma \ref{lem:decomp} with a decomposition $\lG=\lH\oplus\lM$ (i.e. $\lM$ : a complementary for $\lH$ for $\lG$), one can find sequences $A_k\in\mU_m$ and $B_k\in\mU_n$ such that
\[
   c_k=e^{A_k}e^{B_k}.
\]
Here, $\mU_m$ is an open neighbourhood of $0$ in $\lM$ and $\mU_h$, an open neighbourhood of $0$ in $\lH$.

As $e^{B_k}\in N$ and $c_k\in H\setminus N$, $A_k\neq 0$ and $\lim A_k=\lim B_k=0$ (because $(A,B)\to e^Ae^B$ is a diffeomorphism and $e^0e^0=e$ -- and also because all is continuous and thus has a good behaviour with respect to the limit). The set $\mU_m$ is open and bounded --this is a part of the lemma. Then there exist a sequence of positive reals numbers $r_k\in$ such that $r_kA_k\in\mU_m$ and $(r_k+1)A_k\notin\mU_m$. We know that $\mU_m$ is a bounded open subset of the vector space $\lM$, then the whole sequences $r_kA_k$ and $(r_k+1)A_k$ are in a compact domain of $\lM$. Then --by eventually considering subsequences-- there are no problems to consider limits of these sequences in $\lM$ : $r_kA_k\to A\in\lM$ (not necessary in $\mU_m$). Since $A_k\to 0$, the point $A$ is the common limit of $r_kA_k\in\mU_m$ and of $(r_k+1)A_k\notin\mU_m$. Thus $A$ is in the boundary of $\mU_m$; in particular, $A\neq 0$.

On the other hand, consider two integers $p,q$ with $q>0$. One can find sequences $s_k,t_k\in\eN$ and $0\leq t_k<q$ such that $pr_k=qs_k+t_k$. It is clear that 
\begin{equation}
  \lim_{k\to\infty}\frac{t_k}{q}A_k=0,
\end{equation}
thus
\[
   \exp \frac{p}{q}A=\lim \exp\frac{pr_k}{a}A_k=\lim (\exp A_k)^{s_k},
\]
which belongs to $H$. By continuity, $\exp tA\in H$ for any $t\in\eR$ and finally $A\in\lH$; this contradict $A\neq 0$ so that $A\in\lM$ (because by definition, $A\in\lM$ and the sum $\lG=\lH\oplus\lM$ is direct).

By its definition, $H^*$ has an analytic structure of Lie subgroup of $G$; but we had just proved that the induced topology from $G$ is the one of $H_0$ which by definition is a submanifold of $G$. So the set $H_0=H^*$ becomes a submanifold of $G$ whose topology is compatible with the analytic structure: thus it is a Lie subgroup of $G$. From analyticity, this structure is extended to the whole $H$.

\begin{probleme}
Est-ce bien vrai, tout \c ca ? En particulier, je n'utilise pas que $H_0$ est ouvert dans $H$ (ce qui est un tho de topo classique : je ne vois pas pourquoi Helgason fait tout un cin\'ema --que je ne comprends pas-- dessus). En prenant $N=H^*$, on a juste d\'emontr\'e que $H_0$ est un voisinage de $e$ dans $H$, mais ça, on le savait bien avant.
\end{probleme}

The unicity part comes from the corollary \ref{cor:top_subgroup}.
\end{proof}


With the notations and the structure of theorem \ref{tho:diff_sur_ferme}, the subgroup $H$ is discrete if and only if $\lH=\{0\}$. Indeed, recall the definition \eqref{eq:lH_de_G} :
\[
  \lH=\{X\in\lG: \forall t\in\eR, e^{tX}\in H\},
\]
and the fact that there exists a neighbourhood of $e$ in $H$ on which the exponential map is a diffeomorphism.

\begin{remark}
This fact should not be placed after the following lemma. In fact, we use here just the existence of normal neighbourhood (which is a common result) while the following lemma gives much more than normal neighbourhood.
\end{remark}

The lemma (without proof) :

\begin{lemma} 
 Let $G$ be a Lie group and $H$, a Lie subgroup of $G$ ($\lG$ and $\lH$ are the corresponding Lie algebras). If $H$ is a topological subspace of $G$ (cf remark \ref{rem:sub_Lie}), then there exists an open neighbourhood $V$ of $0$ in $\mG$ such that
 \begin{enumerate}
 \item $\exp$ is a diffeomorphism between $V$ and an open neighbourhood of $e$ in~$G$,
 \item $\exp(V\cap\lH)=(\exp V)\cap H$.
 \end{enumerate}
\label{lem:sugroup_normal}
\end{lemma}

Now a theorem with proof.

\begin{theorem}
Let $G$ and $H$ be two Lie groups and $\dpt{\varphi}{G}{H}$ a continuous homomorphism. Then $\varphi$ is analytic.
\end{theorem}

\begin{proof}
The Lie algebra of the product manifold $G\times H$ as $\lG\times\lH$ is given in \ref{lemLeibnitz}. We define
\begin{equation}
  K=\{(g,\varphi(g)):g\in G\}\subset G\times H.
\end{equation}
It is clear that $K$ is closed in $G\times H$ because $G$ is closed and $\varphi$ is continuous.
By theorem \ref{tho:diff_sur_ferme}, there exists an unique differentiable structure on $G\times H$ such that $K$ is a topological Lie subgroup of $G\times H$ (i.e. : Lie subgroup + induced topology). The Lie algebra of $K$ is
\begin{equation}
  \lK=\{(X,Y)\in\lG\times\lH:\forall t\in\eR, (e^{tX},e^{tY})\in K\}.
\end{equation}
Let $N_0$ be an open neighbourhood of $0$ in $\lH$ such that $\exp$ is diffeomorphic between $N_0$ and an open neighbourhood $N_e$ of $e$ in $H$. We define $M_0$ and $M_e$ in the same way, for $G$ instead of $H$. We can suppose $\varphi(M_e)\subset N_e$ : if it is not, we consider a smaller $M_e$ : the openness of $N_e$ and the continuity of $\varphi$ make it coherent.

The lemma \ref{lem:sugroup_normal} allow us to consider $M_0$ and $N_0$ small enough to say that 
\[
   \dpt{\exp}{(M_0\times N_0)\cap\lK}{(M_e\times N_e)\cap K}
\]
is diffeomorphic. Now, we are going to show that for any $X\in\lG$, there exists an unique $Y\in\lH$ such that $(X,Y)\in\lK$. The unicity is easy: consider $(X,Y_1),(X,Y_2)\in\lK$; then $(0,Y_1-Y_2)\in\lK$ (because a Lie algebra is a vector space). Then the definition of $\lK$ makes for any $t\in\eR$, $(e,\exp{t(Y_1-Y_2)})\in K$. Consequently, $\exp t(Y_1-Y_2)=\varphi(e)=e$ and then $Y_1-Y_2=0$.

In order to show the existence, let us consider a $r>0$ such that $X_r=(1/r)X$ keeps in $M_0$. This exists because the sequence $X_r\to 0$ (then it comes $M_0$ from a certain $r$). From the definitions, $\exp$ is diffeomorphic between $M_0$ and $M_e$, then $\exp X_r\in M_e$ and $\varphi(\exp X_r)\in N_e$ because $\varphi(M_e)\subset N_e$.

From this, there exists an unique $Y_r\in N_0$ such that $\exp Y_r=\varphi(\exp X_r)$ and an unique $Z_r\in(M_0\times N_0)\cap\lK$ satisfying  $\exp Z_r=(\exp X_r,\exp Y_r)$. But $\exp$ is bijective from $M_0\times N_0$, so that $Z_r=(X_r,Y_r)$ and we can choose $Y=rY_r$ as a $Y\in\lH$ such that $(X,Y)\in\lK$ (it is not really a choice: the unicity was previously shown). We denotes by $\dpt{\psi}{\lG}{\lH}$ the map which gives the unique $Y\in\lH$ associated with $X\in\lG$ such that $(X,Y)\in\lK$. This is a homomorphism between $\lG$ and $\lH$.

By definition, $(X,\psi(X))\in\lK$, i.e. $(\exp tX,\exp t\psi(X))\in K$ or 
\begin{equation}
  \varphi(\exp tX)=\exp t\psi(X).
\end{equation}
Let us now consider a basis $\{X_1,\ldots,X_n\}$ of $\lG$. Since $\varphi$ is a homomorphism,
\begin{equation}\label{eq:coord_vp_exp}
   \varphi\big((\exp t_1X_1)(\exp t_2X_2)\ldots(\exp t_nX_n)\big)
     =\big(\exp t_1\psi(X_1)\big)\ldots\big( \exp t_n\psi(X_n) \big) 
\end{equation}
Now, we apply lemma \ref{lem:decomp} on the decomposition of $\lG$ into the $n$ subspace spanned by the $n$ vector basis (this is $n$ applications of the lemma), the map
\[
  (\exp t_1X_1)\ldots(\exp t_nX_n)\to (t_1,\ldots,t_n)
\]
is a coordinate system around $e$ in $G$. In this case, the relation \eqref{eq:coord_vp_exp} shows that $\varphi$ is differentiable at $e$. Then it is differentiable anywhere in $G$.
\end{proof}


\begin{proposition}
Let $G$ be a Lie group and $H$, a Lie subgroup of $G$ ($\lG$ and $\lH$ are the corresponding Lie algebras). We suppose that $H$ has at most a countable number of connected components. Then 
\begin{equation}
  \lH=\{ X\in\lG:\forall t\in\eR,e^{tX}\in H \}
\end{equation}
\end{proposition}

\begin{proof}
We will once again use the lemma  \ref{lem:decomp} with $\lN=\lH$ and $\lM$, a complementary vector space of $\lH$ in $\lG$. We define 
\[
   V=\exp\mU_m\exp\mU_h
\]
where $\mU_m$ and $\mU_h$ are the sets given by the lemma. We consider on $V$ the induced topology from $G$. If we define
\[
   \mA=\{A\in\mU_m:e^{A}\in H\},
\]
we have 
\begin{equation}\label{eq:union_A}
   H\cap V=\bigcup_{A\in\mA}e^{A}e^{\mU_h}.
\end{equation}
First, the definition of $V$ makes clear that the elements of the form $\exp A\exp\mU_h$ are in $V$. They are also in $H$ because $\exp A\in H$ (definition of $\mA$) and $\exp\mU_h$ still by definition. In order to see the inverse inclusion, let us consider a $h\in H\cap V$. We know that 
\begin{equation}\label{eq:AB_to_exp}
(A,B)\to\exp A\exp B 
\end{equation}
is a diffeomorphism between $\mU_m\times\mU_h$ and a neighbourhood of $e$ in $G$ which we called $V$. Thus any element of $V$ (\emph{a fortiori} in $V\cap H$) can be written as $\exp A\exp B$ with $A\in\mU_m$ and $B\in\mU_h$. Then $h=e^Ae^B$ for some $A\in\mU_m$, $B\in\mU_h$. Since $H$ is a group and $e^B\in H$, in order the product to belongs to $H$, $e^A$ must lies in $H$ : $A\in\mA$.

\begin{remark}\label{rem:union_disj}
Note that since \eqref{eq:AB_to_exp} is diffeomorphic, the union in right hand side of \eqref{eq:union_A} is disjoint. Each member of this union is a neighbourhood in $H$ because it is a set $h\exp\mU_h$ where $\exp\mU_h$ is a neighbourhood of $e$ in $H$.
\end{remark}

Now we consider the map $\dpt{\pi}{V}{\mU_m}$, 
\[
  \pi(e^{X}e^Y)=X
\]
if $X\in\mU_m$ and $Y\in\mU_h$. This is a continuous map which sends $H\cap V$ into $\mA$. The identity component of $H\cap V$ (in the sense of topology of $V$) is sent to a countable subset of $\mU_m$. Indeed by remark \ref{rem:union_disj}, identity component of $H\cap V$ is only one of the terms in the union \eqref{eq:union_A}, namely $A=0$. But we know that $\pi^{-1}(o)=\exp\mU_h$, thus $\exp\mU_h$ is the identity component of $H\cap V$ for the topology of $V$.

Let us consider a $X\in\lG$ such that $\exp tX\in H$ for any $t\in\eR$, and the map $\dpt{\varphi}{\eR}{G}$, $\varphi(t)=\exp tX$. This is continuous, then there exists a connected neighbourhood $\mU$ of $0$ in $\eR$ such that $\varphi(\mU)\subset V$. Then $\varphi(\mU)\subset H\cap V$ and the connectedness of $\varphi(\mU)$ makes $\varphi(\mU)\subset\exp\mU_h$. But $\exp\mU_h$ is an arbitrary small neighbourhood of $e$ in $H$; the conclusion is that $\varphi$ is a continuous map from $\eR$ into $H$. Indeed, we had chosen $X$ such that $\exp tX\in H$.

Moreover, we know that 
\[
  e^{(t_0+\epsilon)X}=e^{t_0X}e^{\epsilon X},
\]
but $\exp \epsilon X$ can be as close to $e$ as we want (this proves the continuity at $t_0$). Then $\varphi$ is a path in $H$. 

In definitive, we had shown that $\exp tX\in H$ implies that $t\to\exp tX$ is a path. Now equation \eqref{eq:path_alg} gives the thesis.

\end{proof}

\begin{corollary}
Let $G$ be a Lie group and $H_1$, $H_2$, two subgroups both having a finite number of connected components (each for his own topology). If $H_1=H_2$ as sets, then $H_1=H_2$ as Lie groups.
\end{corollary}

\begin{proof}
The proposition shows that $H_1$ and $H_2$ have same Lie algebra. But any Lie subalgebra of $\lG$ is the Lie algebra of exactly one connected subgroup of $G$ (theorem \ref{tho:gp_alg}). Then as Lie groups, ${H_1}_0={H_2}_0$. Since $H_1$ and $H_2$ are topological groups, the equality of they topology on one connected component gives the equality everywhere (because translations are differentiable).
\end{proof}

\begin{definition}
A \defe{differentiable subgroup}{differentiable!subgroup} is a connected Lie subgroup.
\end{definition}

\begin{corollary}
Let $G$ be a Lie group, and $K$, $H$ two differentiable subgroups of $G$. We suppose $K\subset H$. Then $K$ is a differentiable subgroup of the Lie group $H$.
\end{corollary}

\begin{proof}
The Lie algebras of $K$ and $H$ are respectively denoted by $\lK$ and $\lH$. We denote by $K^*$ the differentiable subgroup of $H$ which has $\lK$ as Lie algebra. The differentiable subgroups $K$ and $K^*$ have same Lie algebra, and then coincide as Lie groups.
\end{proof}

\label{pg:ex_topo_Lie}
Consider the group $T=S^1\times S^1$ and the continuous map $\dpt{\gamma}{\eR}{T}$ given by
\[
  \gamma(t)=(e^{it},e^{i\alpha t})
\]
with a certain irrational $\alpha$ in such a manner that $\gamma$ is injective and $\Gamma=\gamma(\eR)$ is dense in $T$.

The subset $\Gamma$ is not closed because his complementary in $T$ is not open: any neighbourhood of element $p\in T$ which don't lie in $\Gamma$ contains some elements of $\Gamma$. We will show that the inclusion map $\dpt{\iota}{\Gamma}{T}$ is continuous. An open subset of $T$ is somethings like
\[
  \mO=(e^{iU},e^{iV})
\]
where $U,V$ are open subsets of $\eR$. It is clear that 
\[
   \iota^{-1}(\mO)=\{ \gamma(t)\tq t\in U+2k\pi,\alpha t\in V+2m\pi \},
\]
but the set of elements $t$ of $\eR$ which satisfies it is clearly open. Then $\Gamma$ has at least the induced topology from $T$ (as shown in proposition \ref{prop:topo_sub_manif}). In fact, the own topology of $\Gamma$ is \emph{more} than the induced: the open subsets of $\Gamma$ whose are just some small segments clearly doesn't appear in the induced topology. Thus the present case is an example (and not a counter-example) of theorem \ref{tho:H_ferme}.

This example show the importance of the condition for a topological subspace to have \emph{exactly} the induced topology. If not, any Lie subgroup were a topological Lie subgroup because a submanifold has at least the induced topology. We will go further with this example after the proof.


\begin{theorem} 
Let $G$ be a Lie group and $H$, a Lie subgroup of $G$.
\begin{enumerate}
\item If $H$ is a topological Lie subgroup of $G$, then it is closed in $G$,
\item If $H$ has at most a countable number of connected components and is closed in $G$, then $H$ is a topological subgroup of $G$.
\end{enumerate}
\label{tho:H_ferme}
\end{theorem}

\begin{proof}
\subdem{First point} It is sufficient to prove that if a sequence $h_n\in H$ converges (in $G$) to $g\in G$, then $g\in H$ (this is almost the definition of a closed subset). We consider $V$, a neighbourhood of $0$ in $\lG$ such that 

\begin{itemize}
\item $\exp$ is diffeomorphic between $V$ and an open neighbourhood  of $e$ in $G$,
\item $\exp(V\cap \lH)=(\exp V)\cap H$.
\end{itemize}
This exists by the lemma \ref{lem:sugroup_normal}; we can suppose that $V$ is bounded. Consider $\mU$, an open neighbourhood of $0$ in $\lG$ contained in $V$ such that $\exp-\mU\exp\mU\subset\exp V$.

Since $h_n\to g$, there exists a $N\in\eN$ such that $n\geq N$ implies $h_n\in g\exp\mU$ (i.e $h_n$ is the product of $g$ by an element rather close to $e$; since the multiplication is differentiable, the notion of ``not so far''\ is good to express the convergence notion). From now we only consider such elements in the sequence. So, $h_N^{-1} h_n\in(\exp V)\cap H$ ($n\geq N$) because
\[
   h_N^{-1} h_n\in\exp-\mU g^{-1} g\exp\mU\subset\exp V.
\]
(note that $H$ is a group, then $h_i^{-1}\in H$) From the second point of the definition of $V$, there exists a $X_n\in V\cap\lH$ such that $h^{-1}_N h_n=\exp X_n$ for any $n\geq N$.

Since $V$ in bounded, there exists a subsequence out of $(X_i)$ (which is also called $X_i$) converging to a certain $Z\in\lG$. But $\lH$ is closed in $\lG$ because it is a vector subspace (we are in a finite dimensional case), then $Z\in\lH$ and thus the sequence $(h_i)$ converges to $h_N\exp Z$; therefore $g\in H$.

\subdem{Second point} The subgroup $H$ is closed in $G$ and has a countable number of connected component. Since $H$ is closed, theorem \ref{tho:diff_sur_ferme} it has an analytics structure for which it is a topological Lie subgroup of $G$. We denotes by $H'$ this Lie group.

The identity map $\dpt{I}{H}{H'}$ is continuous\quext{pourquoi ?} (see error \ref{err:gross}). Thus any connected component of $H$ is contained in a connected component of $H'$, the it has only a countable number of connected components. By corollary \ref{cor:top_subgroup}, $H=H'$ as Lie group.

\end{proof}

Now we take back our example with $G=S^1\times S^1$, $H=\gamma(\eR)$. In this case, the theorem doesn't works. Let us see why as deep as possible. We have $\lG=\eR\oplus\eR=\eR^2$ and $\lH=\eR$, a one-dimensional vector subspace of $\lG$. ($\lH$ is a ``direction ''\ in $\lG$) First, we build the neighbourhood $V$ of $0$ in $\lG$. It is standard to require that $\exp$ is diffeomorphic between $V$ and an open around $(1,1)\in S^1\times S^1$. It also must satisfy $e^{V\cap\lH}=e^V\cap H$. This second requirement is impossible.

Intuitively. We can see $V\subset\lG$ as a little disk tangent to  the torus. The exponential map deposits it on the torus, as well that $e^V$ covers a little area on $G$. Then $e^V\cap H$ is one of these amazing open subset of $\Gamma$ which are dense in a certain domain of $G$.

On the other hand, $V\cap\lH$ is just a little vector in $\lH$; the exponential deposits it on a small line in $G$. This is not the same at all. Then lemma \ref{lem:sugroup_normal} fails in our case. Let us review the proof of this lemma until we find a problem. 

Let $W_0\subset\lG$  be a neighbourhood of $0$ which is in bijection with an open around $e$ in $G$. We consider $N_0$, an open subset of $H$ such that $N_0\subset W_0$ and $N_0$ is in bijection with $N_e$, a neighbourhood of $e$ in $G$. Until here, no problems. But now the proof says that there exists an open $U_e$ in $G$ such that $N_e=U_e\cap H$. This is false in our case. Indeed, $N_e=e^{N_0}$ is just a segment in $G$ while any subset of $G$ of the form $U_e\cap H$ is an ``amazing''\ open.

So we see that deeply, the obstruction for a Lie subgroup to be a topological Lie subgroup resides in the fact that the topology of a submanifold is \emph{more} than the induced topology, so that we can't automatically find the open $U_e$ in $G$.
