\begin{corrige}{LimiteContinue0005}

	\begin{enumerate}
		\item
			Nous essayons les chemins $(t,kt)$. Ce que nous trouvons est
			\begin{equation}		\label{EqLC5ftkt}
				f(t,kt)=\frac{ kt^2 }{ 1+k }\to 0
			\end{equation}
			si $k\neq -1$. Mais si $k=-1$, nous ne sommes pas dans le domaine de $f$. Le calcul \eqref{EqLC5ftkt} ne nous a donc pas apporté grand chose. Il nous a cependant appris que lorsque $k\to -1$, nous avons un zéro qui arrive au dénominateur. Nous essayons donc un chemin «plus dynamique», où au lieu de considérer un $k$ fixé, nous prenons $k(t)$ de telle façon que le dénominateur tende vers zéro plus vite que le numérateur, c'est à dire par exemple $1-k=t^3$. Cela donne le calcul
			\begin{equation}
				f\big( t,(t^3-1)t \big)=\frac{ t^3-1 }{ t }.
			\end{equation}
			Et nous voyons que la fonction $f$ n'admet pas de limites le long de ce chemin parce que $\lim_{t\to 0} (t^3-1)/t$ n'existe pas.
                        \item Nous essayons les chemins $(t, at, bt)$. Nous trouvons
                          \begin{equation}
                            f(t,at,bt) = \frac{ab+b^3}{2+ab^2},
                          \end{equation}
                          si $2+ab^2\neq 0$. On voit que les limites faites sur des chemins différents sont différentes. Par exemple, si $a=b=1$ on a $\lim f (t,t,t)= 2/3$ et si $b=0$ on a $f(t,at, 0)= 0$. La fonction $f$ n'a donc pas une limite en l'origine.
                          \item Nous passons aux coordonnées polaires
                            \begin{equation}
                              \lim_{(x,y)\to (0,0)}\frac{|x|+|y|}{x^2+y^2}=\lim_{r\to 0}\frac{r(|\cos \theta|+|\sin \theta|)}{r^2}\leq \lim_{r\to 0} \frac{2}{r}. 
                            \end{equation}
                            Donc si nous considérions $+\infty$ comme une limite valide alors  la limite de $f$ en l'origine  serait $+\infty$. En fait, dans la définition \ref{def_limite} que nous avons donnée de limite, il est explicitement dit que la limite doit être un élément de $\eR$ et cela n'est pas le cas de $+\infty$. Pas conséquent nous devons conclure que la limite de $f$  n'existe pas (ou mieux : que elle n'existe pas dans $\eR$ ). 
                            \item Ce point peut être résolu de la même manière que le point \ref{ex_limcont_0005_uno}. La fonction $f$ est définie dehors des lignes $x=y$ et $x=-y$.  Nous essayons d'abord la limite sur les droites $(t,kt)$, pour $k\neq \pm 1$, et nous trouvons que si une limite existe elle sera zéro. Ensuite nous trouvons une courbe passant par l'origine sur laquelle le dénominateur décroît plus vite que le numérateur, par exemple $(t, t(1+t^4))$ et voyons que la limite de $f$ n'existe pas.
                              \item Ce point peut être fait encore une fois par la méthode des chemins. Cependant, nous l'utilisons comme une excuse pour présenter les coordonnées ellipsoïdales.  L'équation décrivant une ellipsoïde $E$ centrée en l'origine et dont les axes sont parallèle aux axes coordonnés prend la forme 
                                \begin{equation}\label{Ellissoide}
                                  \frac{x^2}{a^2}+\frac{y^2}{b^2}+\frac{z^2}{c^2}=1,
                                \end{equation}
                                où $a$, $b$ et $c$ sont les longueurs des demi-axes. Vous pouvez vérifier que tout point de la forme  
				\begin{equation}
					\begin{aligned}[]
                                    		x&=a\cos\theta\sin\phi\\
						y&=b\sin\theta\sin\phi\\
						z&=c\cos\phi,
					\end{aligned}
				\end{equation}
                                avec  $\theta$ dans $[0,2\pi[$ et $\phi$ dans $[0,\pi]$, est dans $E$. En fait $E$ coïncide avec l'ensemble des points de cette forme. Multiplions maintenant les  deux côtes de \eqref{Ellissoide} par un scalaire $\rho>0$. Cela revient à  dilater (si $\rho >1$) ou contracter (si $\rho <1$) $E$ tout en gardant ses proportions. La réunion de toutes contractions et dilatations possibles de $E$ est l'espace $\eR^3$.  Avec cette justification intuitive, nous  introduisons les coordonnées suivantes, dont les coordonnées sphériques sont un cas particulier (pour quels $a$, $b$ et $c$ retrouvons-nous les coordonnées sphériques ?)
                                    \begin{equation}
                                      \begin{array}{ccc}
                                        \eR^+\times \eR\times \eR& \to& \eR^3\\ 
                                       (\rho, \theta,\phi )&\mapsto&( x=a\rho\cos\theta\sin\phi, y=b\rho\sin\theta\sin\phi, z=c\rho\cos\phi).
                                      \end{array}
                                      \end{equation}
                                    ce changement de variables est un difféomorphisme de la bande $ \eR^+\times [0,2\pi[\times[0,\pi]$ dans $\eR^3\setminus\{(0,0)\}$.

                                        En ce qui concerne notre limite, si nous passons au coordonnées ellipsoïdales avec $a=1$, $b=1/\sqrt{2}$ et $c=1/\sqrt{3}$ nous trouvons 
                                        \begin{equation}
                                          \lim_{(x,y,z)\to(0,0,0)} \frac{xy+yz}{x^2+2y^2+3z^2}=\lim_{\rho\to 0} \frac{1}{\sqrt{2}}\cos\theta\sin\theta\sin^2\phi+\frac{1}{\sqrt{6}}\sin\theta\sin\phi\cos\phi.
                                        \end{equation}
                                        La quantité à droite ne dépend même pas de $\rho$ et prend des valeurs différentes sur les différents droites par l'origine, la limite n'existe donc  pas. 
	\end{enumerate}
\end{corrige}
