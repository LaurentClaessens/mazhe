Les chaînes de Markov interviennent pour la description des systèmes dont l'évolution future ne dépend que de l'état présent.

\begin{definition}
    Soit \( E\) un ensemble au plus dénombrable et \( (\Omega,\tribF,P)\) un espace probabilisé. Une \defe{chaîne de Markov}{chaîne de Markov} à valeurs dans \( E\) est une famille \( (X_n)_{n\in\eN}\) de variables aléatoires telles que pour tout \( x_0,\ldots,x_{n+1}\in E\),
    \begin{equation}
        P(X_{n+1}=x_{n=1}|X_n=x_n,\ldots,X_0=x_0)=P(X_{n+1}=x_{n+1}|X_n=x_n).
    \end{equation}
\end{definition}
Pour une chaîne de Markov, il n'est pas important de savoir l'historique pour prédire la futur : \( X_{n+1}\) est seulement déterminé par \( X_n\).

\begin{remark}
    Il existe une théorie des chaînes de Markov à temps continu ou avec \( E\) non dénombrable, mais ce n'est pas au programme.
\end{remark}

Nous notons
\begin{equation}
    p_n(x,y)=P(X_{n+1}=y|X_n=x)
\end{equation}
la \defe{probabilité de transition}{transition!probabilité} de la chaîne à l'instant \( n\). Si cette probabilité ne dépend pas de \( n\), nous disons que la chaîne de Markov est \defe{homogène}{homogène!chaîne de Markov}, et nous notons \( p(x,y)\) au lieu de \( p_n(x,y)\). Nous notons \( Q\) la matrice (éventuellement infinie) de transition\index{matrice!de transition}
\begin{equation}
    Q_{xy}=p(x,y).
\end{equation}
Nous avons
\begin{equation}
    \sum_{y\in E}p(x,y)=1
\end{equation}
parce que c'est la somme de toutes les transitions possibles en partant de \( x\). Notons aussi que \( p(x,y)\geq 0\).

\begin{definition}
    Une matrice dont tous les éléments sont positifs ou nuls et donc la somme de toutes les lignes sont \( 1\) est une \defe{matrice stochastique}{matrice!stochastique}.
\end{definition}

\begin{lemma}
    Si \( U\) est une matrice stochastique, alors il existe une chaîne de Markov dont la matrice de transition soit \( U\).
\end{lemma}

\begin{remark}
    La somme \( \sum_{x\in E}p(x,y)\) ne vaut pas spécialement \( 1\). Si les états \( x_1\) et \( x_2\) arrivent tous les deux en \( y\) de façon certaine, alors nous avons \( \sum_xp(x,y)\geq 2\). Il n'y a donc pas de limites aux sommes des colonnes.
\end{remark}

\begin{example}
    Nous considérons une fourmi qui se déplace dans un appartement à trois pièces \( A\), \( B\), \( C\). Supposons qu'à chaque minute, elle a une probabilité \( 1/3\) de rester dans la pièce et une probabilité \( 2/3\) de se déplacer. Le plan de l'appartement est
    \begin{equation}
        \xymatrix{%
        A \ar[r]      &  B\ar[r]&C
           }
    \end{equation}
    De la pièce \( A\) est est donc uniquement possible d'aller vers la pièce \( B\); de la \( B\) il est possible d'aller en \( A\) et en \( C\) et de la \( C\) il est uniquement possible d'aller en \( B\).

    La matrice de transition de cette chaîne de Markov est 
    \begin{equation}
        Q=\begin{pmatrix}
            1/3    &   2/3    &   0    \\
            1/3    &   1/3    &   1/3    \\
            0    &   2/3    &   1/3
        \end{pmatrix}
    \end{equation}
\end{example}

%+++++++++++++++++++++++++++++++++++++++++++++++++++++++++++++++++++++++++++++++++++++++++++++++++++++++++++++++++++++++++++
\section{Marche aléatoire sur \texorpdfstring{$\eZ$}{\( \eZ\)}}
%+++++++++++++++++++++++++++++++++++++++++++++++++++++++++++++++++++++++++++++++++++++++++++++++++++++++++++++++++++++++++++

Soit \( (Y_n)\) une suite de variables aléatoires indépendantes et identiquement distribuées valant \( -1\) avec une probabilité \( p\) et \( 1\) avec une probabilité \( (1-p)\). La loi est
\begin{equation}
    Y_n\sim p\delta_{-1}+(1-p)\delta_{1}.
\end{equation}
Nous considérons la variable aléatoire
\begin{equation}
    X_n=X_0+\sum_{i=1}^nY_i
\end{equation}
où \( X_0\) est une variable aléatoire indépendante des \( Y_i\) à valeurs dans \( \eZ\). Nous vérifions à présent que \( X_n\) est une chaîne de Markov avec comme espace d'états \( E=\eZ\). Nous devons montrer que
\begin{equation}        \label{EqAVoirMarkovMAZ}
    P\big( X_{n+1}=x_{n+1}| X_n=x_n,\ldots,X_0=x_0 \big)=P\big( X_{n+1}=x_{n+1}|X_n=x_n \big).
\end{equation}
Étant donné que \( X_{n+1}=X_n+Y_{n+1}\), nous pouvons remplacer \( X_{n+1}=x_{n+1}\) par \( x_n+Y_{n+1}=x_{n+1}\) dans le membre de gauche de \ref{EqAVoirMarkovMAZ}.

