% This is part of Mes notes de mathématique
% Copyright (c) 2011-2014
%   Laurent Claessens
% See the file fdl-1.3.txt for copying conditions.

%+++++++++++++++++++++++++++++++++++++++++++++++++++++++++++++++++++++++++++++++++++++++++++++++++++++++++++++++++++++++++++
\section{Suites de fonctions}
%+++++++++++++++++++++++++++++++++++++++++++++++++++++++++++++++++++++++++++++++++++++++++++++++++++++++++++++++++++++++++++

%---------------------------------------------------------------------------------------------------------------------------
\subsection{Convergence de suites de fonctions}
%---------------------------------------------------------------------------------------------------------------------------

Nous considérons un espace normé \( (\Omega,\| . \|)\). Nous disons qu'une suite de fonctions \( f_n\) \defe{converge}{convergence!en norme} vers \( f\) pour la norme \( \| . \|\) si \( \forall \epsilon>0\), \( \exists N\) tel que \( n\geq N\) implique \( \| f_n-f \|<\epsilon\).

Dans le cas particulier de la norme 
\begin{equation}
    \| f \|_{\infty}=\sup_{x\in\Omega}| f(x) |,
\end{equation}
nous parlons que \defe{convergence uniforme}{convergence!uniforme!suite de fonctions}.

\begin{theorem}[Critère de Cauchy]  \label{ThoCauchyZelUF}
    Une suite de fonctions  \( (f_n)_{n\in\eN}\) sur \( \Omega\) converge en norme sur \( \Omega\) si et seulement si \( \forall\epsilon>0\), \( \exists N\) tel que
    \begin{equation}
        \| f_n-f_m \|<\epsilon
    \end{equation}
    pour \( n,m>N\).
\end{theorem}

\begin{corollary}       \label{CorCauchyCkXnvY}
    La série \( \sum f_n\) converge en norme sur \( \Omega\) si et seulement si \( \exists N\) tel que
    \begin{equation}
        \| f_n+\ldots+f_m \|\leq \epsilon
    \end{equation}
    pour tout \( n,m>N\).
\end{corollary}

\begin{proof}
    L'hypothèse montre que la suite des sommes partielles de la série \( \sum f_n\) vérifie le critère de Cauchy du théorème \ref{ThoCauchyZelUF}.
\end{proof}

%--------------------------------------------------------------------------------------------------------------------------- 
\subsection{Convergence uniforme}
%---------------------------------------------------------------------------------------------------------------------------

\begin{definition}[\cite{TrenchRealAnalisys}]
    Nous disons qu'une suite de fonctions \( (f_n)\) définies sur un ensemble \( A\) \defe{converge uniformément}{convergence!uniforme} vers une fonction \( f\) si
    \begin{equation}
        \lim_{n\to \infty} \| f_n-f \|_A=0
    \end{equation}
    où \( \| g \|_A=\sup_{x\in A}\| g(x) \|\).
\end{definition}

\begin{proposition}[Critère de Cauchy uniforme\cite{LCbyNWQ}]   \label{PropNTEynwq}
    Soit \( X\) un espace topologique et \( (Y,d)\) un espace topologique complet. La suite de fonction \( f_n\colon X\to Y\) converge uniformément sur \( A\) si et seulement si pour tout \( \epsilon>0\) il existe \( N\in \eN\) tel que si \( k,l>N\) alors
    \begin{equation}
        d\big( f_k(x),f_l(x) \big)\leq \epsilon
    \end{equation}
    pour tout \( x\in X\).
\end{proposition}
\index{Cauchy!critère!uniforme}
\index{critère!Cauchy!uniforme}
Grosso modo, cela dit que si qu'une suite de Cauchy pour la norme uniforme est une suite uniformément convergente. Le fait que la suite converge fait partie du résultat et n'est pas une hypothèse. Ce critère sera utilisé pour montrer que \( \big( C(K),\| . \|_{\infty} \big)\) est complet, proposition \ref{PropSYMEZGU}. 

\begin{proof}
    Si \( f_n\stackrel{unif}{\longrightarrow}f\) alors le critère est satisfait; c'est dans l'autre sens que la preuve est intéressante.

    Soit donc une suite de fonctions satisfaisant au critère et montrons qu'elle converge uniformément. Pour tout \( x\in X\) la suite \( n\mapsto f_n(x)\) est de Cauchy dans l'espace complet \( Y\); nous avons donc convergence ponctuelle \( f_n\to f\). Nous devons prouver que cette convergence est uniforme. Soit \( \epsilon>0\) et \( N\in \eN\) tel que si \( k,l>N\) alors
    \begin{equation}
        d\big( f_k(x),f_l(x) \big)\leq \epsilon
    \end{equation}
    pour tout \( x\in X\). Si nous nous fixons un tel \( k\) et un \( x\in A\) nous considérons l'inégalité
    \begin{equation}
        d\big( f_k(x),f_l(x) \big)\leq \epsilon
    \end{equation}
    qui est vraie pour tout \( l\). En passant à la limite \( l\to\infty\) (limite qui commute avec la fonction distance par définition de la topologie) nous avons
    \begin{equation}
        d\big( f_k(x),f(x) \big)\leq \epsilon.
    \end{equation}
    Cette inégalité étant valable pour tout \( x\in X\), ce la signifie que \( f_n\stackrel{unif}{\longrightarrow}f\).
\end{proof}

\begin{theorem}[Limite uniforme de fonctions continues]			\label{ThoUnigCvCont}
    Soit \( A\), un ensemble mesuré et \( f_n\colon A\to \eR^n\), une suite de fonctions continues convergeant uniformément vers \( f\). Si les fonctions \( f_n\) sont toutes continues en \( x_0\in A\), alors \( f\) est continue en \( x_0\).
\end{theorem}

\begin{proof}
    Soit \( \epsilon>0\). Si \( x\in A\) nous avons, pour tout \( n\), la majoration
    \begin{subequations}
        \begin{align}
            \| f(x)-f(x_0) \|&\leq \| f(x)-f_n(x) \|+\| f_n(x)-f_n(x_0) \|+\| f_n(x_0)-f(x_0) \|\\
            &\leq\| f_n(x)-f_n(x_0) \|+2\| f_n-f \|_{\infty}.
        \end{align}
    \end{subequations}
    Grâce à l'uniforme convergence, nous considérons \(N\in \eN\) tel que \( \| f_n-f \|\leq \epsilon\) pour tout \( n\geq N\). Pour de tels \( n\), nous avons
    \begin{equation}
        \| f(x)-f(x_0) \|\leq 2\epsilon\| f_n-f \|+\| f_n(x)-f_n(x_0) \|.
    \end{equation}
    La continuité de \( f_n\) nous fournit un \( \delta>0\) tel que \( \| f_n(x_0)-f_n(x) \|<\epsilon\) dès que \( \| x-x_0 \|<\delta\). Pour ce \( \delta\), nous avons alors \( \| f(x)-f(x_0) \|<\epsilon\).
\end{proof}

\begin{theorem}[Théorème de Dini\cite{JIFGuct}] \label{ThoUFPLEZh}
    Soit \( D\) un espace métrique compact et une suite de fonctions \( f_n\in C(D,\eR)\) telle que
    \begin{enumerate}
        \item
            \( f_n\to g\) ponctuellement,
        \item
            \( g\in C(D,\eR)\),
        \item
            la suite \( (f_n)\) est croissante, c'est à dire que pour tout \( x\in D\) et pour tout \( n\geq 0\) nous avons \( f_{n+1}(x)\geq f_n(x)\).
    \end{enumerate}
    Alors la convergence est uniforme.
\end{theorem}
\index{convergence!uniforme!théorème de Dini}
\index{compacité!théorème de Dini}
\index{théorème!Dini}

\begin{proof}
    Soit \( x\in D\) et \( \epsilon>0\). Il existe \( N(x)\in \eN\) tel que
    \begin{equation}
        g(x)-\epsilon\leq f_{N(x)}\leq g(x).
    \end{equation}
    De plus \( g\) et \( f_{N(x)}\) sont des fonctions continues, donc il existe \( \eta(x)\) tel que si \( y\in B\big( x,\eta(x) \big)\) alors
    \begin{subequations}
        \begin{align}
            g(y)&\in B\big( g(x),\epsilon \big) \label{subEqXKjgKgv}\\
            f_{N(x)}(y)&\in B\big( f_{N(x)}(x),\epsilon \big)   \label{subEqHTiYZLd}.
        \end{align}
    \end{subequations}
    Si \( n\geq N(x)\) et si \( y\in B(x,\eta(x))\) alors nous avons les majorations
    \begin{equation}
            g(y)\geq f_n(y)
            \geq f_{N(x)}(y)
            \geq f_{N(x)}(x)-\epsilon
            \geq g(x)-2\epsilon
            \geq g(y)-3\epsilon.
    \end{equation}
    Justifications :
    \begin{multicols}{2}
        \begin{enumerate}
            \item
                Les deux première inégalités sont la croissance de la suite.
            \item
                La suivante est \eqref{subEqHTiYZLd}.
            \item
                Ensuite il y a le choix de \( N(x)\).
            \item
                Et enfin il y a \eqref{subEqXKjgKgv}.
        \end{enumerate}
    \end{multicols}
    Nous retenons que si \( x\in D\) et si \( n\geq N(x)\) alors
    \begin{equation}    \label{EqJCMktdj}
        g(y)\geq f_n(y)\geq g(y)-3\epsilon
    \end{equation}
    pour tout \( y\in B(x,\eta(x))\).

    Nous utilisons maintenant la compacité de \( D\). Pour chaque \( x\in D\) nous pouvons considérer la boule ouverte \( B\big( x,\eta(x) \big)\); ces boules recouvrent \( D\). Nous en extrayons un sous-recouvrement fini, c'est à dire un ensemble fini d'éléments \( x_1\),\ldots, \( x_K\) tels que
    \begin{equation}
        D=\bigcup_{k=1}^K B\big(x_k,\eta(x_k)\big).
    \end{equation}
    Si à ce moment vous ne comprenez pas pourquoi c'est une égalité au lieu d'une inclusion, il faut lire l'exemple \ref{ExKYZwYxn}. Considérons 
    \begin{equation}
        n\geq N=\max\{ N(x_1),\ldots, N(x_K) \}.
    \end{equation}
    Pour tout \( y\in D\) il existe \( k\in\{ 1,\ldots, K \}\) tel que \( y\in B\big( x_k,\eta(x_k) \big)\), et vu que \( n\geq N(x_k)\) nous reprenons la majoration \eqref{EqJCMktdj} :
    \begin{equation}
        g(y)\geq f_n(y)\geq g(y)-3\epsilon.
    \end{equation}
    Pour le \( n\) choisi nous avons ces inégalités pour tout \( y\in D\), c'est à dire que nous avons \( \| f_n-g \|\leq 3\epsilon\) et donc la convergence uniforme.
\end{proof}

%--------------------------------------------------------------------------------------------------------------------------- 
\subsection{Permuter avec les dérivées partielles}
%---------------------------------------------------------------------------------------------------------------------------

\begin{theorem}		\label{ThoSerUnifDerr}
	Soit $U\subset\eR^n$ ouvert, $f_k\colon U\to \eR$ et $f_k$ de classe $C^1$. Supposons que $f_k$ converge simplement vers $f$ et que $\partial_if_k$ converge uniformément sur tout compact  vers une fonction $g_i$ pour $i=1,\ldots,n$. Alors $f$ est de classe $C^1$ et $\partial_if=g_i$. De plus, $f_k$ converge vers $f$ uniformément.
\end{theorem}
\index{permutation!dérivée et limite}
%TODO : une preuve.


%+++++++++++++++++++++++++++++++++++++++++++++++++++++++++++++++++++++++++++++++++++++++++++++++++++++++++++++++++++++++++++
\section{Théorie de la mesure}
%+++++++++++++++++++++++++++++++++++++++++++++++++++++++++++++++++++++++++++++++++++++++++++++++++++++++++++++++++++++++++++

%---------------------------------------------------------------------------------------------------------------------------
\subsection{Ensembles mesurables}
%---------------------------------------------------------------------------------------------------------------------------

\begin{definition}[\cite{ProbaDanielLi}]  \label{DefjRsGSy}
    Si \( \Omega\) est un ensemble, un ensemble \( \tribA\) de sous-ensembles de \( \Omega\) est une \defe{tribu}{tribu} si 
    \begin{enumerate}
        \item
            \( \Omega\in\tribA\);
        \item
            \( \complement A\in A\) pour tout \( A\in\tribA\);
        \item
            si \( (A_i)_{i\in I}\) est un ensemble au plus dénombrable d'éléments de \( \tribA\), alors \( \sup_{n\geq 1}A_n=\bigcup_{i\in I}A_i\in\tribA\).
    \end{enumerate}
    Le couple \( (\Omega,\tribA)\) est alors un \defe{espace mesuré}{espace!mesuré}.
\end{definition}

\begin{lemma}   \label{LemBWNlKfA}
    Opérations ensemblistes sur les tribus.
    \begin{enumerate}
        \item
    Une tribu est stable par intersections au plus dénombrables.
\item
    Une tribu est stable par différence ensembliste.
    \end{enumerate}
\end{lemma}

\begin{proof}
    Soit \( (A_i)_{i\in I}\) une famille au plus dénombrable d'éléments de la tribu \( \tribA\). Nous devons prouver que \( \bigcap_{i\in I}A_i\) est également un élément de \( \tribA\). Pour cela nous passons au complémentaire :
    \begin{equation}
        \complement\left( \bigcap_{i\in I}A_i \right)=\bigcup_{i\in I}\complement A_i.
    \end{equation}
    La définition d'une tribu implique que le membre de droite est un élément de la tribu. Par stabilité d'une tribu par complémentaire, l'ensemble \( \bigcap_{i\in I}A_i\) est également un élément de la tribu.

    La seconde assertion est immédiate à partir de la première parce que \( A\setminus B=A\cap \complement B\).
\end{proof}

\begin{definition}
    La tribu des \defe{boréliens}{boréliens}, notée \( \Borelien(\eR^d)\) est la tribu engendrée par les ouverts de \( \eR^d\). Plus généralement si \( Y\) est un espace topologique, la tribu des boréliens est la tribu engendrée par les ouverts de \( Y\).
\end{definition}
Le plus souvent lorsque nous parlerons de fonctions \( f\colon X\to Y\) où \( Y\) est un espace topologique, nous considérons la tribu borélienne sur \( Y\). Ce sera en particulier le cas dans la théorie de l'intégration.

\begin{proposition} \label{LemYEkvbWBz}
    La tribu engendrée par une base dénombrable de la topologie est celle des boréliens.
\end{proposition}

\begin{proof}
    Si une base de topologie est donnée, tout ouvert peut être écrit comme union d'élément de la base, proposition \ref{PropMMKBjgY}. Dans le cas d'une base dénombrable, cette union sera forcément dénombrable. Une tribu étant stable par union dénombrable, tout ouvert est dans la tribu engendrée par la base de topologie. Les autres boréliens suivent automatiquement.
\end{proof}

%--------------------------------------------------------------------------------------------------------------------------- 
\subsection{Les boréliens de \texorpdfstring{$ \eR$}{R}}
%---------------------------------------------------------------------------------------------------------------------------

Nous rappelons que la topologie de \( \eR\) est celle des boules donnée par le théorème \ref{ThoORdLYUu}. Notons que les boules ouvertes de la forme \( B(q,r)\) avec \( q,r\in \eQ\) forment une base dénombrable de la topologique de \( \eR\).

\begin{lemma}   \label{LemZXnAbtl}
    Soit \( \{ q_i \}\) une énumération des rationnels. La tribu engendrée par les ouverts \( \sigma_i=\mathopen] q_i , \infty \mathclose[\) est la tribu des boréliens.
\end{lemma}

\begin{proof}
    Si \( a<b\) dans \( \eQ\) alors \( \sigma_a\setminus\sigma_b=\mathopen] a , b \mathclose]\). Ensuite 
    \begin{equation}
        \bigcup_{n\in \eN^*}\sigma_a\setminus\sigma_{b-\frac{1}{ n }}=\bigcup_{n\in \eN^*}\mathopen] a , b-\frac{1}{ n } \mathclose]=\mathopen] a , b \mathclose[.
    \end{equation}
    Par union dénombrable, tous les intervalles \( \mathopen] a , b \mathclose[\) avec \( a,b\in \eQ\) sont dans la tribu engendrée par les \( \sigma_i\).

        Ces boules ouvertes forment une base de la topologie de \( \eR\) et le lemme \ref{LemYEkvbWBz} conclu.
\end{proof}

\begin{proposition}[\cite{OYRmzAa}]
    Tout ouvert de \( \eR^n\) est une union dénombrable de rectangles presque disjoints\footnote{«presque» au sens où les intersections éventuelles sont de mesure de Lebesgue nulle.}.
\end{proposition}

\begin{proof}
    Soit \( G\) un ouvert de \( \eR^n\). Soit \( \{ Q_i^{1} \}_{i\in \eN}\) un découpage de \( \eR^n\) en cubes de côté \( 1\) et dont les sommets sont en les coordonnées entières. Ce sont des cubes presque disjoints. Nous considérons ensuite pour chaque \( k>1\) le découpage \( \{ Q_i^{(k)} \}_{i\in\eN}\) de \( \eR^n\) en cubes de côtés \( 2^{-k}\) qui consiste à découper en \( 2\) les côtés des cubes du découpage \( Q^{(k-1)}\). Ces cubes forment encore un découpage dénombrable de \( \eR^n\) en des cubes presque disjoints. Ensuite nous considérons \( \mE\) l'union de tous les \( Q_i^{(k)}\) contenus dans \( G\).

    Montrons que \( \mE=G\). D'abord \( \mE\subset G\) parce que \( \mE\) est une union d'ensembles contenus dans \( G\). Ensuite si \( x\in G\), il existe une boule de rayon \( r\) autour de \( x\) contenue dans \( G\); alors un des ensembles \( Q_i^{(k)}\) avec \( 2^{-j}<\frac{ r }{2}\) est contenue dans \( B(x,r)\) et donc dans \( \mE\).

    Bien entendu l'union qui donne \( \mE\) n'est pas satisfaisante par ce que les \( Q_i^{(k+1)}\) sont contenus dans les \( Q_i^{(k)}\); les intersections sont donc loin d'être de mesure nulle.

    Nous faisons ceci : 
    \begin{subequations}
        \begin{align}
            R^{(0)}&=\{ Q_i^{(1)} \text{contenu dans \( G\)} \}\\
            R^{(k+1)}&=\{ Q_i^{(k+1)}\text{contenus dans \( G\) et pas dans \( R^{(k)}\)} \}.
        \end{align}
    \end{subequations}
    En fin de compte l'union de tous les ensembles contenus dans les \( R^{(k)}\) forment encore \( \eR^n\), mais sont d'intersection presque vide.
\end{proof}

%--------------------------------------------------------------------------------------------------------------------------- 
\subsection{Fonction mesurable}
%---------------------------------------------------------------------------------------------------------------------------

\begin{definition}[Fonction mesurable] \label{DefQKjDSeC}
    Soit \( (E,\tribA)\) et \( (F,\tribF)\) deux espaces mesurés. Une fonction \( f\colon E\to F\) est \defe{mesurable}{mesurable!fonction} si pour tout \( \mO\in \tribF\), l'ensemble \( f^{-1}(\mO)\) est dans \( \tribA\).

    Une fonction à valeurs dans \( \eR^d\) est \defe{borélienne}{borélienne!fonction}\index{fonction!borélienne} si elle est mesurable pour la tribu des boréliens sur \( \eR^d\). Plus explicitement, \( f\colon (\Omega,\tribA)\to (\eR^d,\Borelien(\eR^d))\) est borélienne si pour tout \( \mO\in\Borelien\) nous avons \( f^{-1}(\mO)\in\tribA\).
\end{definition}
Si \( \tribA\) est une tribu sur un ensemble \( E\), nous notons \( m(\tribA)\)\nomenclature[P]{\( m(\tribA)\)}{Ensemble des fonctions \( \tribA\)-mesurables} l'ensemble des fonctions qui sont \( \tribA\)-mesurables.

\begin{remark}
    Lorsque nous considérons des fonctions à valeurs réelles \( f\colon X\to \eR\) nous utiliserons toujours la tribu borélienne sur \( \eR\). Pour \( X\), cela peut dépendre des contextes. En théorie de l'intégration, nous mettrons sur \( X\) la tribu des ensembles mesurables au sens de Lebesgue sur \( X\), \emph{tout en gardant celle des boréliens sur l'ensemble d'arrivée}.

    Pour toute la partie sur l'intégration, une fonction \( f\colon \eR^n\to \eR^m\) sera mesurable si pour tout borélien \( A\) de \( \eR^m\) l'ensemble \( f^{-1}(A)\) est Lebesgue-mesurable dans \( \eR^n\).

    Étant donné qu'il est franchement difficile de créer des ensembles non mesurables au sens de Lebesgue, il est franchement difficile de créer des fonction non mesurables à valeurs réelles. L'hypothèse de mesurabilité est donc toujours satisfaite dans les cas pratiques.
\end{remark}

\begin{lemma}[\cite{NBoIEXO}]   \label{LemFOlheqw}
    Une fonction \( f\colon X\to \eR\) est mesurable si et seulement si \( f^{-1}(I)\) est mesurable pour tout \( I\) de la forme \( \mathopen] a , \infty \mathclose[\).
\end{lemma}

\begin{proof}
    Nous devons prouver que \( f^{-1}(A)\) est mesurable dans \( X\) pour tout borélien \( A\) de \( \eR\). Nous posons
    \begin{equation}
        S=\{ A\subset \eR\tq f^{-1}(A)\text{ est mesurable dans \( X\)} \}
    \end{equation}
    et nous prouvons que cela est une tribu. D'abord \( f^{-1}(\eR)=X\), et \( X\) est mesurable, donc \( \eR\in S\). Ensuite si \( A\in S\) alors \( f^{-1}(A^c)=f^{-1}(A)^c\). En tant que complémentaire d'un mesurable de \( X\), l'ensemble \( f^{-1}(A)^c\) est mesurable dans \( X\). Et enfin si \( A_n\in S \) alors \( f^{-1}(\bigcup_nA_n)=\bigcup_nf^{-1}(A_n)\) qui est encore mesurable dans \( X\) en tant qu'union de mesurables.

    Donc \( S\) est une tribu qui contient tous les ensembles de la forme \( \mathopen] a , \infty \mathclose]\). Le lemme \ref{LemZXnAbtl} conclu que \( S\) contient tous les boréliens de \( \eR\).
\end{proof}

\begin{lemma}[\cite{NBoIEXO}]   \label{LemIGKvbNR}
    Soit \( f_n\colon X\to \eR\) une suite de fonctions mesurables\footnote{Ici \( X\) est un espace mesuré et \( \eR\) est muni des boréliens.}. Alors \( \sup_n f_n\) est mesurable.
\end{lemma}

\begin{proof}
    Nous avons
    \begin{subequations}
        \begin{align}
            (\sup f_n)^{-1}\big( \mathopen] a , \infty \mathclose] \big)&=\{ x\in X\tq (\sup f_n)(x)>a \}\\
            &=\bigcup_n\{ x\in X\tq f_n(x)>a \}\\
            &=\bigcup_nf_n^{-1}\big( \mathopen] a , \infty \mathclose] \big).
        \end{align}
    \end{subequations}
    Étant donné que \( f_n\) est mesurable et que \( \mathopen] a , \infty \mathclose]\) est mesurable, chacun des \( f_n^{-1}\big( \mathopen] a , \infty \mathclose] \big) \) est mesurable dans \( X\). Nous sommes en présence d'une union dénombrable de mesurables, donc \( (\sup f_n)^{-1}\big( \mathopen] a , \infty \mathclose] \big)\) est mesurable.

    Le lemme \ref{LemFOlheqw} conclu que \( \sup f_n\) est mesurable.
\end{proof}

\begin{proposition}\label{PropFYPEOIJ}
    Si \( f_n\) est une suite de fonctions mesurables et positives, alors la fonction \( \sum_nf_n\) est mesurable.
\end{proposition}

\begin{proof}
    Nous considérons les fonctions \( s_k(x)=\sum_{n=0}^kf_n(x)\) qui vaut éventuellement \( \infty\) en certains points. Nous avons
    \begin{equation}
        \sum_nf_n(x)=\sup_ks_k(x),
    \end{equation}
    donc le lemme \ref{LemIGKvbNR} nous donne la mesurabilité de la somme de \( f_n\).
\end{proof}

%--------------------------------------------------------------------------------------------------------------------------- 
\subsection{Tribu produit}
%---------------------------------------------------------------------------------------------------------------------------

\begin{definition}      \label{DefTribProfGfYTuR}
    Si \( \tribA_1\) et \( \tribA_2\) sont deux tribus sur deux ensembles \( \Omega_1\) et \( \Omega_2\), nous définissons la \defe{tribu produit}{tribu!produit} \( \tribA_1\otimes\tribA_2\) comme étant la tribu engendrée par 
    \begin{equation}
        \{ X\times Y\tq X\in\tribA_1,Y\in\tribA_2 \}.
    \end{equation}
    Ces ensembles sont appelés \defe{rectangles}{rectangle!produit de tribus} de \( (\Omega_1,\tribA_1)\otimes (\Omega_2,\tribA_2)\).
\end{definition}

\begin{lemma}[Propriété des sections\cite{NBoIEXO}] \label{LemAQmWEmN}
    Soient \( \tribA_1\) et \( \tribA_2\) des tribus sur les ensembles \( \Omega_1\) et \( \Omega_2\). Si \( A\in\tribA_1\otimes\tribA_2\) alors pour tout \( x\in \Omega_1\) et \( y\in\Omega_2\), les ensembles
    \begin{subequations}    \label{subEqCTtPccK}
        \begin{align}
            A_1(y)=\{ x\in\Omega_1\tq (x,y)\in A \}\\
            A_2(x)=\{ y\in\Omega_2\tq (x,y)\in A \}
        \end{align}
    \end{subequations}
    sont mesurables.
\end{lemma}
\index{section!propriété des}

\begin{proof}
    Soit \( y\in\Omega_2\); nous allons prouver le résultat pour \( A_1(y)\). Pour cela nous notons 
    \begin{equation}
        S=\{ A\in \tribA_1\otimes\tribA_2\tq \forall y\in\Omega_2, A_1(y)\in\tribA_1 \},
    \end{equation}
    et nous allons noter que \( S\) est une tribu contenant les rectangles. Par conséquent, \( S\) sera égal à \( \tribA_1\otimes \tribA_2\).

    \begin{subproof}
        \item[Les rectangles]

            Considérons le rectangle \( A=X\times Y\) et si \( y\in \Omega_2\) alors
            \begin{equation}
                A_1(y)=\{ x\in \Omega_1\tq (x,y)\in X\times Y \}.  
            \end{equation}
            Donc soit \( y\in Y\) alors \( A_1(y)=X\in\tribA_1\), soit \( y\notin Y\) et alors \( A_1(y)=\emptyset\in\tribA_1\).

        \item[Tribu : ensemble complet]

            Nous avons \( \Omega_1\times \Omega_2\in S\) parce que c'est un rectangle.

        \item[Tribu : complémentaire] Soit \( A\in S\) et montrons que \( A^c\in S\). Nous avons d'abord
            \begin{equation}
                (A^c)_1(y)=\{ x\in \Omega_1\tq (x,y)\in A^c \}.
            \end{equation}
            D'autre part
            \begin{equation}
                A_1(y)^c=\{ x\in\Omega_1\tq (x,y)\notin A \}=\{ x\in \Omega_1\tq (x,y)\in A^c \}=(A^c)_1(y).
            \end{equation}
            Vu que \( \tribA_1\) est une tribu et que par hypothèse \( A_1(y)\in\tribA_1\), nous avons aussi \( A_1(y)^c\in S\), et donc \( (A^c)_1(y)\in \tribA_1\), ce qui prouve que \( A^c\in S\).

        \item[Tribu : union dénombrable] Soit une suite \( A_n\in S\). Nous avons
            \begin{equation}
                (\bigcup_nA_n)_1(y)=\{ x\in\Omega_1\tq (x,y)\in \bigcup_nA_n \}=\bigcup_n\{ x\in\Omega_1\tq (x,y)\in A_n \}=\bigcup_n(A_n)_1(y),
            \end{equation}
            et ce dernier ensemble est dans \( \tribA_1\) parce que c'est une union dénombrable d'éléments de \( \tribA_1\).
        
    \end{subproof}
    Nous avons donc prouvé que \( S\) est une tribu contenant les rectangles, donc \( S\) contient au moins \( \tribA_1\otimes \tribA_2\).
\end{proof}

\begin{corollary}
    Si \( f\colon \Omega_1\times \Omega_2\to \eR\) est une fonction mesurable\footnote{Définition \ref{DefQKjDSeC}.} sur \( X\times Y\) alors pour chaque \( y\) dans \( \Omega_2\), la fonction
    \begin{equation}
        \begin{aligned}
            f_y\colon X&\to \eR \\
            x&\mapsto f(x,y) 
        \end{aligned}
    \end{equation}
    est mesurable.
\end{corollary}

\begin{proof}
    Soit \( \mO\) un ensemble mesurable de \( \eR\) (i.e. un borélien), et \( y\in \Omega_2\). Nous avons
    \begin{equation}
        f_y^{-1}(\mO)=\{ x\in X\tq f(x,y)\in \mO \}=A_1(y)
    \end{equation}
    où
    \begin{equation}
        A=\{ (x,y)\in \Omega_1\times \Omega_2\tq f(x,y)\in \mO \}=f^{-1}(\mO).
    \end{equation}
    Ce dernier est mesurable parce que \( f\) l'est.
\end{proof}

%--------------------------------------------------------------------------------------------------------------------------- 
\subsection{Mesure}
%---------------------------------------------------------------------------------------------------------------------------

\begin{definition}[Mesure]  \label{DefBTsgznn}
    Une \defe{\wikipedia{en}{Measure_space}{mesure}}{mesure} sur l'espace mesurable \( (\Omega,\tribA)\) est une application \( \mu\colon \tribA\to \eR\cup\{ \infty \}\) telle que
    \begin{enumerate}
        \item
            \( \mu(A)\geq 0\) pour tout \( A\in\tribA\);
        \item
            \( \mu(\emptyset)=0\);
        \item   \label{ItemQFjtOjXiii}
            \( \mu\left( \bigcup_{i=0}^{\infty}A_i\right)=\sum_{i=0}^{\infty}\mu(A_i)\) si les \( A_i\) sont des éléments de \( \tribA\) deux à deux disjoints.
    \end{enumerate}
    Une mesure est \defe{\( \sigma\)-finie}{mesure!$\sigma$-finie} si il existe un recouvrement dénombrable de \( \Omega\) par des ensembles de mesure finie. Si la mesure est $\sigma$-finie, nous disons que l'espace \( (\Omega,\tribA,\mu)\) est un espace mesuré $\sigma$-fini.

    La mesure \( \mu\) sur \( \Omega\) est \defe{finie}{mesure!finie} si \( \mu(\Omega)<\infty\).
\end{definition}

\begin{definition}[Ensemble mesurable]\label{DefHGsQxHB}
    Les éléments de \( \tribA\) sont les ensembles \defe{mesurables}{mesurable!ensemble} pour la mesure \( \mu\).
\end{definition}

Si la mesure des \( \sigma\)-finie, nous pouvons choisir le recouvrement croissant pour l'inclusion. En effet si \( (E_n)_{n\in \eN}\) est le recouvrement, il suffit de considérer \( F_n=\bigcup_{k\leq n}E_k\). Ces ensembles \( F_n\) forment tout autant un recouvrement dénombrable, mais il est évidemment croissant.

\begin{lemma}\label{LemKKNtvee} \label{LemPMprYuC}
    Si \( A\subset B\) sont deux ensembles \( \mu\)-mesurables de mesure finie alors
    \begin{equation}
        \mu(B\setminus A)=\mu(B)-\mu(A)
    \end{equation}
    et en particulier
    \begin{equation}
        \mu(B)\geq \mu(A).
    \end{equation}
\end{lemma}

\begin{proof}
    Vu que les ensembles \( B\setminus A\) et \( A\) sont disjoints par la propriété \ref{ItemQFjtOjXiii} de la définition de mesure nous avons
    \begin{equation}
        \mu\big( (B\setminus A)\cup A \big)=\mu(B\setminus A)+\mu(A)
    \end{equation}
    et donc
    \begin{equation}
        \mu(B)=\mu(B\setminus A)+\mu(A)
    \end{equation}
    comme demandé.
\end{proof}

\begin{lemma}\label{LemAZGByEs}
    Si \( (A_k)\) est une suite croissante d'ensembles \( \mu\)-mesurables, alors
    \begin{equation}
        \lim_{n\to \infty} \mu(A_k)=\mu(\bigcup_kA_k).
    \end{equation}
\end{lemma}

\begin{proof}
    Nous faisons le coup de l'union télescopique, en posant \( A_0=\emptyset\) :
    \begin{equation}
        \bigcup_{k=1}^{\infty}A_k=\bigcup_{k=1}^{\infty}(A_k\setminus A_{k-1}).
    \end{equation}
    Les ensembles \( A_k\setminus A_{k-1}\) sont deux à deux disjoints, donc la propriété \ref{ItemQFjtOjXiii} de la définition d'une mesure donne
    \begin{subequations}
        \begin{align}
            \mu(\bigcup_{k=1}^{\infty}A_k)&=\mu\left( \bigcup_{k=1}^{\infty}(A_k\setminus A_{k-1}) \right)\\
            &=\sum_{k=1}^{\infty}\mu(A_k\setminus A_{k-1})\\
            &=   \sum_{k=1}^{\infty}\big( \mu(A_k)-\mu(A_{k-1}) \big)    \label{subEqMDRRorb}\\
            &=\lim_{k\to \infty} \mu(A_k)-\mu(A_0)\\
            &=\lim_{k\to \infty} \mu(A_k).
        \end{align}
    \end{subequations}
    où pour obtenir \ref{subEqMDRRorb}, nous avons utilisé le lemme \ref{LemPMprYuC}.
\end{proof}


\begin{example}
    La mesure de comptage \( m\) sur \( \eN\) est \( \sigma\)-finie parce que \( E_n=\{ 0,\ldots, n \}\) est de mesure finie et \( \bigcup_{n\in \eN}E_n=\eN\).
\end{example}

\begin{example}
    La mesure de Lebesgue sur \( \eR^n\) est \( \sigma\)-finie parce que les boules de rayon \( n\) forment un ensemble dénombrable d'ensembles de mesures finies dont l'union est évidemment tout \( \eR^n\).

    L'intervalle \( I=\mathopen[ 0 , 1 \mathclose]\) muni de la tribu de toutes ses parties et de la mesure de comptage n'est pas un espace mesuré \( \sigma\)-fini.
\end{example}

\begin{example}
    L'intégration à la Riemann n'est pas dans la théorie des espaces mesurés. En effet l'ensemble 
    \begin{equation}
        \tribA=\{   A\subset\mathopen[ 0 , 1 \mathclose]\tq  \text{\( \mtu_A\) est intégrable au sens de Riemann}   \}
    \end{equation}
    n'est pas une tribu. Par exemple les singletons en font partie tandis que \( \mathopen[ 0 , 1 \mathclose]\cap \eQ\) n'en fait pas partie alors que c'est une union dénombrable de singletons.
\end{example}

\begin{definition}
    Si \( \mu\) est une mesure nous disons qu'une propriété est vraie \( \mu\)-\defe{presque partout}{presque partout} si elle est fausse seulement sur un ensemble de mesure nulle.
\end{definition}

Par exemple la fonction de Dirichlet est presque partout égale à la fonction \( 1\) (pour la mesure de Lebesgue).


\begin{definition}
    Une application entre espace mesurés
    \begin{equation}
        f\colon (\Omega,\tribA)\to (\Omega',\tribA')
    \end{equation}
    est \defe{mesurable}{mesurable!application} si pour tout \( B\in\tribA'\), l'ensemble \( f^{-1}(B)\) est dans \( \tribA\).
\end{definition}

Si \( \mu\) est une mesure sur \( \eR^d\), une fonction \( f\colon \eR^d\to \eR\) est une \defe{densité}{densité d'une mesure} si pour tout \( A\subset\eR^d\) nous avons
\begin{equation}
    \mu(A)=\int_Af(x)dx
\end{equation}
où \( dx\) est la mesure de Lebesgue.

%--------------------------------------------------------------------------------------------------------------------------- 
\subsection{Généralités}
%---------------------------------------------------------------------------------------------------------------------------

\begin{lemma}   \label{LemIDITgAy}
    Une union dénombrable d'ensemble de mesure nulle est de mesure nulle.
\end{lemma}

\begin{proof}
    C'est une conséquence immédiate du point \ref{ItemQFjtOjXiii} de la définition d'une mesure : si les \( A_i\) sont de mesure nulle,
    \begin{equation}
        \mu\left( \bigcup_{i=1}^{\infty}A_i \right)\leq \mu(A_i)=0
    \end{equation}
\end{proof}

\begin{definition}
    Si \( (A_n)\) est une suite croissante d'ensembles alors la \defe{limite}{limite!d'ensembles} est
    \begin{equation}
        \lim_nA_n=\bigcup_{i=0}^{\infty}A_i.
    \end{equation}
    Si la suite est décroissante alors la limite est
    \begin{equation}
        \lim_nA_n=\bigcap_{i=0}^{\infty}A_i.
    \end{equation}
\end{definition}
\ifthenelse{\value{isAgreg}=0}{Pour une suite ni croissante ni décroissante d'ensembles, il y a la notion de limite inductive\footnote{\emph{direct limit} en anglais.} qui sera un peu traitée à la section \ref{SecDirectLimit}.}{}

\begin{proposition}[\cite{RArwFWJ}] \label{PropAFNPSsm}
    Soit \( \mu\) une mesure sur \( \Omega\) et \( (S_n)\) une suite croissante d'ensembles \( \mu\)-mesurables de \( \Omega\). Nous notons
    \begin{equation}
        S=\lim_nS_n.
    \end{equation}
    Alors pour tout ensemble mesurable\footnote{Définition \ref{DefHGsQxHB}} \( A\subset\Omega\) nous avons
    \begin{equation}
        \mu(A\cap S)=\lim_{n\to \infty} \mu(A\cap S_n).
    \end{equation}
\end{proposition}
Note : dans la référence le résultat fonctionne pour tout ensemble \( A\) (et non seulement les mesurables) parce que la définition de la mesurabilité est un peu différente.

\begin{proof}
    L'inégalité \( \lim\mu(A\cap S_n)\leq \mu(A\cap S)\) est simple à prouver. En effet pour tout \( n\) nous avons \( A\cap S_n\subset A\cap S\) et donc par le lemme \ref{LemKKNtvee} nous avons
    \begin{equation}
        \mu(A\cap S_n)\leq\mu(A\cap S).
    \end{equation}
    En passant à la limite (qui respecte les inégalités) nous avons l'inégalité.

    Nous passons à l'inégalité dans l'autre sens. D'abord si \( \mu(A\cap S_n)=\infty\) pour un certain \( n\), alors il cela vaut encore \( \infty\) pour tous les \( n\) suivants et la limite est \( \infty\) sans problèmes. Donc nous supposons que \( \mu(A\cap S_n)<\infty\) pour tout \( n\in \eN\). De plus, quitte à renommer les indices, nous pouvons supposer que \( S_0=\emptyset\).

    Un élément \( x\) est dans \( S\) si et seulement si il existe \( n\geq 0\) tel que \( x\in S_{n+1}\). En prenant le plus petit de ces \( n\) nous avons \( x\neq S_n\) (éventuellement \( n=0\)) et donc
    \begin{equation}
        S=\bigcup_{n=0}^{\infty}\big( S_{n+1}\setminus S_n \big).
    \end{equation}
    Par conséquent
    \begin{equation}
            A\cap S=A\cap\bigcup_{n=0}^{\infty}(S_{n+1}\setminus S_n)
            =\bigcup_{n=0}^{\infty}A\cap(S_{n+1}\setminus S_n)
    \end{equation}
    Étant donné que les ensembles \( A\cap(S_{n+1}\setminus S_n)\) sont disjoints,
    \begin{subequations}
        \begin{align}
            \mu(A\cap S)&=\sum_{n=0}^{\infty}\mu\big( A\cap(S_{n+1}\setminus S_n) \big)\\
            &=\sum_{n=0}^{\infty}\mu\Big( (A\cap S_{n+1})\setminus (A\cap S_n) \Big)\\
            &=\sum_{n=0}^{\infty}\big[ \mu(A\cap S_{n+1})-\mu(A\cap S_n) \big]\\
            &=\lim_{n\to \infty} \mu(A\cap S_{n+1})-\underbrace{\mu(A\cap S_0)}_{=0}\label{subeqLTvmTjO}\\
            &=\lim_{n\to \infty} \mu(A\cap S_n).
        \end{align}
    \end{subequations}
    Dans ce calcul nous avons utilisé plusieurs fois le fait que les \( S_n\) et \( A\) étaient mesurables (et la propriété de tribu qui dit que \( A\cap S_n\) est également mesurable) ainsi que le lemme \ref{LemKKNtvee}. Nous avons aussi utilisé la série télescopique dans \( \eR\) pour obtenir \eqref{subeqLTvmTjO}.
\end{proof}

\begin{definition}[\cite{PVWUyDH}]
    Soit \( E\) un ensemble. Une partie \( \tribD\) de \( E\) est un \defe{\( \lambda\)-système}{$\lambda$-système} si
    \begin{enumerate}
        \item
            pour tout \( A,B\in\tribD\) avec \( A\subset B\) implique \( B\setminus A\in \tribD\),
        \item
            si \( (A_k)_{k\geq 1}\) est une suite croissante d'éléments de \( \tribD\) alors \( \bigcup_kA_k\in\tribD\).
    \end{enumerate}
\end{definition}
Note : une tribu est un \( \lambda\)-système.

\begin{lemma}[\cite{PVWUyDH}]
    Une intersection quelconque de \( \lambda\)-systèmes dans \( E\) est un \( \lambda\)-système dans \( E\).
\end{lemma}

\begin{proof}
    Soient \( \{ \tribD_l \}_{l\in L}\) des \( \lambda\)-systèmes indicés par un ensemble \( L\). Si \( A,B\in\bigcap_{l\in L}\tribD_l\) alors \( B\setminus A\in\tribD_l\) pour tout \( l\in L\) et donc \( A\setminus B\in\bigcap_{l\in L}\tribD_l\). De la même façon si \( (A_k)\) est une suite croissante dans \( \bigcap_{l\in L}\tribD_l\) alors pour tout \( l\in L\) nous avons \( \bigcup_kA_k\in\tribD_l\). Donc \( \bigcup_kA_k\in\bigcap_l\tribD_l\).
\end{proof}
Ce lemme est ce qui permet de définir le \( \lambda\)-système \defe{engendré}{engendré!$\lambda$-système} par une classe \( \tribA\) de parties de \( E\) : c'est l'intersection de tous les \( \lambda\)-systèmes de \( E\) contenant \( \tribA\).

\begin{lemma}[\cite{PVWUyDH}]   \label{LemLUmopaZ}
    Soit \( \tribC\) une classe de parties de \( E\) (contenant \( E\) lui-même) qui soit stable par intersection finie. Alors le \( \lambda\)-système engendré par \( \tribC\) coïncide avec la tribu engendrée par \( \tribC\).
\end{lemma}

\begin{proof}
    Nous notons \( \tribE\) le \( \lambda\)-système engendré par \( \tribC\) et \( \tribF\) la tribu engendrée par \( \tribC\). Étant donné que \( \tribF\) est un \( \lambda\)-système nous avons \( \tribE\subset\tribF\). Pour montrer l'inclusion inverse nous allons prouver que \( \tribE\) est une tribu.

    D'abord pour \( C\in\tribC\) nous posons
    \begin{equation}
        \mG_C=\{ A\subset \tribE\tq A\cap C\in\tribE \}.
    \end{equation}
    et pour \( F\in\tribE\),
    \begin{equation}
        \mH_F=\{ A\in\tribE\tq A\cap F\in\tribE \}.
    \end{equation}
    Nous allons montrer que \( \mG_C\) et \( \mH_F\) sont des \( \lambda\)-systèmes et que \( \mG_C=\mH_F=\tribE\).
    
    Nous commençons par \( \mG_C\). Si \( A,B\in\mG_C\) avec \( A\subset B\) alors
    \begin{equation}
        (B\setminus A)\cap C=\underbrace{(B\cap C)}_{\in\tribE}\setminus\underbrace{(A\cap C)}_{\in\tribE}.
    \end{equation}
    Vu que \( \tribE\) est un \( \lambda\)-système et que \( (A\cap C)\subset(B\cap C)\) nous avons bien \( (B\setminus A)\cap C\in\tribE\) et donc \( B\setminus A\in\mG_C\). Soit maintenant \( (A_k)\) une suite croissante dans \( \mG_C\). Nous avons
    \begin{equation}
        \big( \bigcup_{k=1}^{\infty}A_k \big)\cap C=\bigcup_{k=1}^{\infty}(A_k\cap C)
    \end{equation}
    qui est une union d'une suite croissante d'éléments de \( \tribE\). Donc \( \bigcup_{k=1}^{\infty}(A_k\cap C)\in\tribE\), ce qui signifie que \( \bigcup_{k=1}^{\infty}A_k\in\mG_C\). Cela termine la preuve du fait que \( \mG_C\) soit une \( \lambda\)-système. 

    Étant donné que \( \tribC\) est stable par intersection finie, si \( K\in\tribC\) nous avons \( C\cap K\in\tribC\), ce qui signifie que \( K\in\mG_C\). Nous avons donc \( \tribC\subset\mG_C\). Donc \( \mG_C\) est un \( \lambda\)-système vérifiant \( \tribC\subset\mG_C\subset\tribE\). Mais comme \( \tribE\) est le plus petit \( \lambda\)-système contenant \( \tribC\) nous avons en fait \( \mG_C=\tribE\).

    Nous montrons à présent que \( \mH_F\) est un \( \lambda\)-système. Si \( A,B\in\mH_F\) avec \( A\subset B\) alors \( (B\setminus A)\cap F=(B\cap F)\setminus(A\cap F)\). Vu que \( \tribE\) est une \( \lambda\)-système et que \( A\cap F\) et \( B\cap F\) sont dans \( \tribE\) avec \( A\cap F\subset B\cap F\), nous avons
    \begin{equation}
        (B\cap F)\setminus(A\cap F)\in\mH_F.
    \end{equation}
    Soit maintenant \( (A_k)_{k\geq 1}\) une suite croissante dans \( \mH_F\). Pour tout \( k\) nous avons \( A_k\cap F\in\tribE\), ce qui donne
    \begin{equation}
        \big( \bigcap_{k=1}^{\infty}A_k \big)\cap F=\bigcap_{k=1}^{\infty}(A_k\cap F)\in\tribE.
    \end{equation}
    Donc \( \mH_F\) est un \( \lambda\)-système vérifiant \( \tribC\subset\mH_F\subset\tribE\). Nous en concluons que pour tout \( C\in\tribC\) et pour tout \( F\in\tribE\),
    \begin{equation}
        \mG_C=\mH_F=\tribE.
    \end{equation}
    
    Nous allons maintenant prouver que \( \tribE\) est une tribu\footnote{Définition \ref{DefjRsGSy}.}.
    \begin{enumerate}
        \item
            Si \( F\in\tribE\) alors \( E\cap F=F\in\tribE\), ce qui signifie que \( E\in\mH_F=\tribE\).
        \item
            Si \( A\in \tribE\) alors \( E\setminus A\in\tribE\) parce que \( \tribE\) est un \( \lambda\)-système et \( E\in\tribE\). Donc \( \complement A\in\tribE\).
        \item
            Montrons que \( \tribE\) est stable par union finie en considérant \( A,B\in\tribE\). Vu que \( E\) est également un élément de \( \tribE\) nous avons
            \begin{equation}
                E\setminus(A\cup B)=(E\setminus A)\cap(E\setminus B)\in\tribE.
            \end{equation}
            Cela prouve que \( \complement( A\cup B)\in \tribE\). Par complémentarité nous avons aussi \( A\cup B\in\tribE\).
            
            Soient \( A_k\in\tribE\), et nommons \( B_p=A_1\cup\ldots\cup A_p\). Les ensembles \( B_p\) forment une suite croissante d'éléments de \( \tribE\). L'union est donc dans \( \tribE\) et ce dernier est au final stable par union dénombrable.
    \end{enumerate}
    
    Maintenant que \( \tribE\) est une tribu nous avons \( \tribF\subset\tribE\) parce que \( \tribF\) est la plus petite tribu contenant \( \tribC\). Nous en déduisons que \( \tribE=\tribF\), ce qu'il fallait démontrer.
\end{proof}

\begin{theorem}[\cite{PVWUyDH}] \label{ThoJDYlsXu}
    Soient \( \mu\) et \( \nu\), deux mesures sur \( (E,\tribA)\) et une classe \( \tribE\) de parties de \( E\) telles que
    \begin{enumerate}
        \item
            La tribu engendrée par \( \tribE\) soit \( \tribA\).
        \item
            pour tout \( A\in \tribE\), \( \mu(A)=\nu(A)\)
        \item
            si \( A,B\in\tribE\) alors \( A\cap B\in\tribE\)
        \item
            il existe une suite croissante \( (E_n)\) dans \( \tribE\) telle que \( E=\lim E_n\).
    \end{enumerate}
    Alors les mesures \( \mu\) et \( \nu\) coïncident sur \( \tribA\) en entier.
\end{theorem}

\begin{proof}
    Soit \( (E_n)\) la suite des hypothèses; nous considérons \( \mu_n\) et \( \nu_n\), les restrictions de \( \mu\) et \( \nu\) à \( E_n\), c'est à dire
    \begin{subequations}
        \begin{align}
        \mu_n(A)=\mu(A\cap E_n)\\
        \nu_n(A)=\nu(A\cap E_n).
        \end{align}
    \end{subequations}
    Vu que les \( E_n\) sont dans \( \tribE\subset\tribA\) ils sont mesurables au sens de \( \mu\) et \( \nu\). Par la proposition \ref{PropAFNPSsm}, pour tout \( A\in \tribE\) nous avons alors
    \begin{subequations}
        \begin{align}
            \lim_{n\to \infty} \mu_n(A)=\mu(A)\\
            \lim_{n\to \infty} \nu_n(A)=\nu(A)
        \end{align}
    \end{subequations}
    Nous devons donc seulement montrer que pour tout \( A\in\tribA\) et pour tout \( n\in\eN\), \( \mu_n(A)=\nu_n(A)\). Pour cela nous nous fixons un \( n\) et nous considérons la classe
    \begin{equation}
        \tribD=\{ A\in\tribA\tq\mu_n(A)=\nu_n(A) \}.
    \end{equation}
    Le but sera de prouver que \( \tribD=\tribA\).
    
    
    Par hypothèse \( A\cap E_n\in\tribE\) et donc
    \begin{equation}
        \mu(A\cap E_n)=\nu(A\cap E_n)<\infty,
    \end{equation}
    c'est à dire que \( \mu_n=\nu_n\) sur \( \tribE\). Par ailleurs, \( E\cap E_n=E_n\in\tribE\), donc
    \begin{equation}
        \mu_n(E)=\nu_n(E)<\infty.
    \end{equation}
    Par conséquent \( \mu_n=\nu_n\) sur la classe \( \tribE'=\tribE\cup\{ E \}\) : \( \tribE'\subset\tribD\).

    Montrons que \( \tribD\) est un \( \lambda\)-système. Soient \( A,B\in\tribD\) avec \( A\subset B\). Alors, étant donné que les mesures \( \mu_n\) et \( \nu_n\) sont finies, le lemme \ref{LemPMprYuC} nous donne
    \begin{subequations}
        \begin{align}
            \mu_n(B\setminus A)=\mu_n(B)-\mu_n(A)\\
            \nu_n(B\setminus A)=\nu_n(B)-\nu_n(A).
        \end{align}
    \end{subequations}
    Donc \( \mu_n(B\setminus A)=\nu_n(B\setminus A)\) et \( B\setminus A\in\tribD\).

    Soit par ailleurs une suite croissante \( (A_k)_{k\geq 1}\) d'éléments de \( \tribD\). En posant \( B_p=\bigcup_{k=1}^pA_k\), le lemme \ref{LemAZGByEs} nous donne
    \begin{equation}
        \mu_n(\bigcup_{k=1}^{\infty}A_k)=\lim_{p\to \infty} \mu_n(A_p).
    \end{equation}
    Mais vu que pour chaque \( p\) nous avons \( \mu_n(A_p)=\nu_n(A_p)\), nous avons aussi
    \begin{equation}
        \mu_n(\bigcup_{p=1}^{\infty}A_p)=\nu_n(\bigcup_{p=1}^{\infty}A_p).
    \end{equation}
    Donc \( \tribD\) est bel et bien un \( \lambda\)-système contenant \( \tribE'\). Par le lemme \ref{LemLUmopaZ}, le \( \lambda\)-système engendré par \( \tribE'\) est égal à la tribu engendrée par \( \tribE'\), mais par hypothèse la tribu engendrée par \( \tribE\) est \( \tribA\), donc le \( \lambda\)-système engendré par \( \tribE'\) est \( \tribA\). Vu que \( \tribD\) est une \( \lambda\)-système contenant \( \tribE'\), nous avons alors \( \tribA\subset\tribD\) et donc \( \tribA=\tribD\), ce qu'il fallait.
\end{proof}

%---------------------------------------------------------------------------------------------------------------------------
\subsection{Théorème de récurrence}
%---------------------------------------------------------------------------------------------------------------------------

Soit \( X\) un espace mesurable, \( \mu\) une mesure finie sur \( X\) et \( \phi\colon X\to X\) une application mesurable préservant la mesure, c'est à dire que pour tout ensemble mesurable \( A\subset X\),
\begin{equation}
    \mu\big( \phi^{-1}(A) \big)=\mu(A).
\end{equation}
Si \( A\subset X\) est un ensemble mesurable, un point \( x\in A\) est dit \defe{récurrent}{récurrent!point d'un système dynamique} par rapport à \( A\) si et seulement si pour tout \( p\in \eN\), il existe \( k\geq p\) tel que \( \phi^k(x)\in A\).

\begin{theorem}[\wikipedia{fr}{Théorème_de_récurrence_de_Poincaré}{Théorème de récurrence de Poincaré}.]     \label{ThoYnLNEL}
    Si \( A\) est mesurable dans \( X\), alors presque tous les points de \( A\) sont récurrents par rapport à \( A\).
\end{theorem}

\begin{proof}
    Soit \( p\in \eN\) et l'ensemble
    \begin{equation}
        U_p=\bigcup_{k=p}^{\infty}\phi^{-k}(A)
    \end{equation}
    des points qui repasseront encore dans \( A\) après \( p\) itérations  de \( \phi\). C'est un ensemble mesurable en tant que union d'ensembles mesurables (pour rappel, les tribus sont stables par union dénombrable, comme demandé à la définition \ref{DefjRsGSy}), et nous avons donc
    \begin{equation}
        \mu(U_p)\leq \mu(X)<\infty.
    \end{equation}
    De plus \( U_p=\phi^{-p}(U_0)\), donc \( \mu(U_p)=\mu(U_0)\). Vu que \( U_p\subset U_p\), nous avons
    \begin{equation}
        \mu(U_0\setminus U_p)=0.
    \end{equation}
    Étant donné que \( A\subset U_0\) nous avons a fortiori que
    \begin{equation}
        \{ x\in A\tq x\notin U_p \}\subset U_0\setminus U_p,
    \end{equation}
    et donc
    \begin{equation}
        \mu\{ x\in A\tq x\notin U_p \}=0.
    \end{equation}
    Cela signifie exactement que l'ensemble des points \( x\) de \( A\) tels que aucun des \( \phi^k(x)\) avec \( k\geq p\) n'est dans \( A\) est de mesure nulle.
\end{proof}

%--------------------------------------------------------------------------------------------------------------------------- 
\subsection{Fonction simple}
%---------------------------------------------------------------------------------------------------------------------------

\begin{definition}  \label{DefBPCxdel}
Une fonction mesurable \( f\colon X\to \eR\) est \defe{simple}{simple!fonction}\index{fonction!simple} si son image est finie
\begin{equation}
    f(x)=\sum_{j=1}^p\alpha_j\mtu_{A_j}(x)
\end{equation}
où \( A_j=f^{-1}(\alpha_j)\). Notons que \( f\) étant mesurable, les ensembles \( A_j\) sont forcément mesurables.
\end{definition}

\begin{lemma}[Limite croissante de fonctions étagées mesurables]    \label{LemYFoWqmS}
    Soit \( f\colon (\Omega,\tribA)\to \eR\) une fonction mesurable. Il existe une suite \( f_n\colon \Omega\to \eR\) de fonctions simples telles que \( f_n\to f\) ponctuellement et \( | f_n |<f\).
\end{lemma}

\begin{proof}
    Nous considérons \( (q_n)\) une suite parcourant tous les rationnels\footnote{Nous rappelons que \( \eQ\) est dénombrable et dense dans \( \eR\).}.
    Pour \( n\in \eN\) nous définissons la fonction
    \begin{equation}
        f_n(\omega)=\begin{cases}
            \max\{ q_i\tq i\leq n,\, q_i\leq f(\omega) \}    &   \text{si \( f(\omega)\geq 0\)}\\
            \min\{ q_i\tq i\leq n,\, q_i\geq f(\omega) \}    &    \text{si \( f(\omega)< 0\).}
        \end{cases}
    \end{equation}
    La fonction \( f_n\) est simple parce qu'elle ne prend que \( n\) valeurs différentes. Nous avons aussi par construction que \( | f_n(\omega)|\leq |f(\omega) |\). Nous avons aussi pour tout \( \omega\in \Omega\) que \( f_n(\omega)\to f(\omega)\) parce que \( \eQ\) est dense dans \( \eR\).

    En ce qui concerne la mesurabilité de \( f_n\), les étages de \( f_n\) sont les ensembles de la forme \( \{ \omega\in \Omega\tq f(\omega)\in\mathopen[ a , b [ \}\) où \( a\) et \( b\) sont deux éléments de \( \{ q_1,\ldots, q_n \}\) qui sont consécutifs au sens de l'ordre dans \( \eQ\) (et non spécialement au sens de l'ordre d'apparition dans la suite), avec éventuellement \( b=\infty\) si \( a\) est le plus grand. Les ensembles \( \mathopen[ a , b [\) étant mesurables dans \( \eR\) et la fonction \( f\) étant mesurable par hypothèse, les ensembles \( f^{-1}\Big( \mathopen[ a , b [ \Big)\) sont mesurables dans \( (\Omega,\tribA)\).
\end{proof}

\begin{proposition}\label{PropWBavIf}
    Une fonction positive et mesurable sur \( \Omega\) est limite ponctuelle croissante de fonctions simples positives.
\end{proposition}

\begin{proof}
    Soit \( \{ q_i \}\) une énumération des rationnels positifs. Il suffit de poser
    \begin{equation}
        f_n(x)=\max\{ q_i\tq i\leq n,\text{ et }f(x)\geq q_i \}.
    \end{equation}
    <++>
\end{proof}

%+++++++++++++++++++++++++++++++++++++++++++++++++++++++++++++++++++++++++++++++++++++++++++++++++++++++++++++++++++++++++++ 
\section{Intégrale par rapport à une mesure}
%+++++++++++++++++++++++++++++++++++++++++++++++++++++++++++++++++++++++++++++++++++++++++++++++++++++++++++++++++++++++++++

Une mesure \( \mu\) sur un espace mesurable \( (\Omega,\tribA)\) permet de définir une fonctionnelle linéaire sur l'ensemble des fonctions mesurables \( \Omega\to \eR\). Cette fonctionnelle linéaire est l'intégrale que nous allons définir à présent.

Si \( Y\in \tribA\) et si \( f\) est une fonction simple nous définissons
\begin{equation}
    \int_Yfd\mu=\sum_ia_i\mu(Y\cap E_i).
\end{equation}
Pour une fonction \( \mu\)-mesurable générale \( f\colon \Omega\to \mathopen[ 0 , \infty \mathclose]\) nous définissons l'intégrale de \( f\) sur \( Y\) par
\begin{equation}        \label{EqDefintYfdmu}
    \int_Yfd\mu=\sup\Big\{ \int_Yhd\mu\,\text{où \( h\) est une fonction simple et mesurable telle que \( 0\leq h\leq f\)} \Big\}.
\end{equation}
Maintenant nous définissons
\begin{equation}
    \mu(f)=\int_{\Omega}f
\end{equation}
si \( f\) est une fonction mesurable sur \( \Omega\).

\begin{remark}
    Dans \( \eR^d\), quasiment toutes les fonctions et ensembles sont mesurables. En effet la construction d'ensembles non mesurables demande obligatoirement l'utilisation de l'axiome du choix; de tels ensembles doivent être construits «exprès pour». Il y a très peu de chances pour que vous tombiez sur un ensemble non mesurable de \( \eR^d\) sans que vous ne vous en rendiez compte.

    Par exemple la variable aléatoire 
    \begin{equation}
        X(\omega)=\begin{cases}
            \frac{1}{ \omega }    &   \text{si $ \omega\neq 0$}\\
            \infty    &    \text{$\omega=0$}.
        \end{cases}
    \end{equation}
    est mesurable, mais non intégrable.
\end{remark}

Le lemme suivant nous aide à détecter des fonctions presque partout nulles.
\begin{lemma}   \label{Lemfobnwt}
    Soit \( f\) une fonction mesurable positive ou nulle telle que
    \begin{equation}
        \int_{\Omega}fd\mu=0.
    \end{equation}
    Alors \( f=0\) \( \mu\)-presque partout.
\end{lemma}

\begin{proof}
    L'ensemble des points \( x\in\Omega\) tels que \( f(x)\neq 0\) peut s'écrire comme une union dénombrable disjointe :
    \begin{equation}
        \{ x\in\Omega\tq f(x)\neq 0 \}=\bigcup_{i=0}^{\infty}E_i
    \end{equation}
    avec
    \begin{subequations}
        \begin{align}
            E_0&=\{ x\in\Omega\tq f(x)>1 \}\\
            E_i&=\{ x\in\Omega\tq \frac{1}{ i+1 }\leq f(x)<\frac{1}{ i } \}.
        \end{align}
    \end{subequations}
    Si un des ensembles \( E_i\) est de mesure non nulle, alors nous pouvons considérer la fonction simple \( h(x)=\frac{1}{ i+1 }\mtu_{E_i}\) dont l'intégrale sur \( \Omega\) est strictement positive. Par conséquent le supremum de la définition \eqref{EqDefintYfdmu} est strictement positif.

    Nous savons donc que \( \mu(E_i)=0\) pour tout \( i\). Étant donné que la mesure d'une union disjointe dénombrable est égale à la somme des mesures, nous avons
    \begin{equation}
        \mu\{ x\in\Omega\tq f(x)\neq 0 \}=0,
    \end{equation}
    ce qui signifie que \( f\) est nulle \( \mu\)-presque partout.
\end{proof}

\begin{corollary}   \label{CorjLYiSm}
    Soit \( f\) une fonction mesurable sur l'espace mesuré \( (\Omega,\tribA,\mu)\) telle que
    \begin{equation}
        \int_{\Omega}f\mtu_{f>0}d\mu=0.
    \end{equation}
    Alors \( f\leq 0\) presque partout.
\end{corollary}

\begin{proof}
    Nous avons l'égalité d'ensembles
    \begin{equation}
        \{ f\mtu_{f>0}\neq 0 \}=\{ \mtu_{f>0}\neq 0 \}.
    \end{equation}
    Mais lemme \ref{Lemfobnwt} implique que \( f\mtu_{f>0}\) est nulle presque partout, c'est à dire que la mesure de l'ensemble du membre de gauche est nulle par conséquent
    \begin{equation}
        \mu\{ \mtu_{f>0}\neq 0 \}=0.
    \end{equation}
    Cela signifie que la fonction \( f\) est presque partout négative ou nulle.
\end{proof}

\begin{lemma}   \label{LemPfHgal}
    Soit \( f\) une fonction telle que \( | f(x)|\leq g(x) \) pour tout \( x\in\Omega\). Si \( g\) est intégrable, alors \( f\) est intégrable.
\end{lemma}

\begin{proof}
    Nous décomposons \( f\) en parties positives et négatives :
    \begin{subequations}
        \begin{align}
            A_+&=\{ x\in\Omega\tq f(x)>0 \}\\
            A_-&=\{ x\in\Omega\tq f(x)<0 \}.
        \end{align}
    \end{subequations}
    Nous posons \( f_+(x)=f(x)\mtu_{A_+}\) et \( f_-(x)=f(x)\mtu_{A_-}\). Nous avons \( f=f_+-f_-\) et
    \begin{equation}
        \int_{\Omega}f=\int_{A_+}f+\int_{A_-}f
    \end{equation}
    parce que \( \Omega=A_+\cup A_-\cup\{ x\in\Omega\tq f(x)=0 \}\). Si \( \varphi\) est une fonction simple qui majore \( f_+\) nous avons
    \begin{equation}
        \varphi(x)=\sum_{k}a_k\mtu_{E_k}(x)\leq f(x)\mtu_{A_+}(x)\leq g(x).
    \end{equation}
    Par conséquent le supremum qui définit \( \int f_+\) est inférieur au supremum qui définit \( \int g\). La fonction \( f_+\) est donc intégrable. La même chose est valable pour la fonction \( f_-\).
\end{proof}

%---------------------------------------------------------------------------------------------------------------------------
\subsection{Mesure produit}
%---------------------------------------------------------------------------------------------------------------------------

\begin{theorem}[\cite{NBoIEXO}\footnote{Modèle non contractuel : des notations et la définition de \( \lambda\)-système peuvent varier entre la référence et le présent texte.}]    \label{ThoCCIsLhO}
    Soient \( (\Omega_i,\tribA_i,\mu_i)\) (\( i=1,2\)) deux espaces mesurés \( \sigma\)-finie. Soit \( A\in\tribA_1\otimes \tribA_2\). Alors les fonctions\footnote{Voir la notation du lemme \ref{subEqCTtPccK}.}
    \begin{subequations}
        \begin{align}
            x\mapsto\mu_2\big( A_2(x) \big)\\
            y\mapsto\mu_1\big( A_1(y) \big)
        \end{align}
    \end{subequations}
    sont mesurables et
    \begin{equation}    \label{EqRKXwsQJ}
        \int_{\Omega_1}\mu_2\big( A_2(x) \big)d\mu_1(x)=\int_{\Omega_2}\mu_2\big( A_1(y) \big)d\mu_2(y).
    \end{equation}
\end{theorem}

\begin{proof}
    Nous supposons d'abord que \( \mu_1\) et \( \mu_2\) sont finies et nous notons \( \tribD\) le sous-ensemble de \( \tribA_1\otimes \tribA_2\) sur lequel le théorème est correct. Nous allons commencer par prouver que \( \tribD\) est un \( \lambda\)-système.

    \begin{subproof}
        \item[\( \lambda\)-système : différence ensembliste]
            Soient \( A,B\in\tribD\) avec \( A\subset B\). Nous avons
            \begin{subequations}
                \begin{align}
                    (B\setminus A)_1(y)&=\{ x\in \Omega_1\tq(x,y)\in B\setminus A \}\\
                    &=\{ x\in \Omega_1\tq(x,y)\in B\}\setminus\{ x\in \Omega_1\tq(x,y)\in  A \}\\
                    &=B_1(y)\setminus A_1(y).
                \end{align}
            \end{subequations}
            Vu que \( A_1(y)\subset B_1(y)\) et que les mesure sont finies le lemme \ref{LemPMprYuC} nous donne
            \begin{equation}
                \mu_1\big( (B\setminus A)_1(y) \big)=\mu_1\big( B_1(y) \big)-\mu_1\big( A_1(y) \big),
            \end{equation}
            et similairement pour \( 1\leftrightarrow 2\). Les deux fonctions (de \( y\)) à droite étant mesurables, nous avons la mesurabilité de la fonction \( y\mapsto \mu_1\big( (B\setminus A)_1(y) \big)\).

            Prouvons la formule intégrale en nous rappelant que la formule \eqref{EqRKXwsQJ} est supposée correcte pour \( A\) et \( B\) séparément :
            \begin{subequations}
                \begin{align}
                    \int_{\Omega_2}\mu_1\big( (B\setminus A)_1(y) \big)d\mu_2(y)&=\int_{\Omega_2}\mu_1\big( B_1(y) \big)d\mu_2(y)-\int_{\Omega_2}\mu_1\big( A_1(y) \big)d\mu_2(y)\\
                    &=\int_{\Omega_1}\mu_2\big( B_2(x) \big)d\mu_1(x)-\int_{\Omega_1}\mu_2\big( A_2(x) \big)d\mu_1(x)\\
                    &=\int_{\Omega_1}\mu_2\big( (B\setminus A)_2(x) \big)d\mu_1(x).
                \end{align}
            \end{subequations}
            
    
        \item[\( \lambda\)-système : limite de suite croissante]

            Soit \( (A_n)\) une suite croissante dans \( \tribD\); nous posons \( B_n=A_n\setminus A_{n-1}\) et \( A_0=\emptyset\) de telle sorte à travailler avec une suite d'ensembles disjoints qui satisfait \( \bigcup_nA_n=\bigcup_nB_n\). Vu que la suite est croissante nous avons \( A_{n-1}\subset A_n\) et donc \( B_n\in\tribD\) par le point déjà fait sur la différence ensembliste. Nous avons :
            \begin{subequations}
                \begin{align}
                    \mu_1\big( (\bigcup_nB_n)_1(y) \big)&=\{ x\in \Omega_1\tq (x,y)\in\bigcup_nB_n \}\\
                    &=\bigcup_n\{ x\in\Omega_1\tq (x,y)\in B_n \}\\
                    &=\bigcup_n (B_n)_1(y).
                \end{align}
            \end{subequations}
            Par conséquent, par la propriété \ref{ItemQFjtOjXiii} d'une mesure nous avons
            \begin{equation}
                \mu_1\big( (\bigcup_nB_n)_1(y) \big)=\sum_n\mu_1\big( (B_n)_1(y) \big).
            \end{equation}
            En tant que somme de fonctions positives et mesurables, la fonction
            \begin{equation}
                y\mapsto\sum_n\mu_1\big( (B_n)_1(y) \big)
            \end{equation}
            est mesurable par la proposition \ref{PropFYPEOIJ}. Il faut encore vérifier la formule intégrale. Le gros du boulot est de permuter une somme et une intégrale par le corollaire \ref{CorNKXwhdz} :
            \begin{subequations}
                \begin{align}
                    \int_{\Omega_2}\sum_n\mu_1\big( (B_n)_1(y) \big)d\mu_2(y)&=\sum_n\int_{\Omega_2}\mu_1\big( (B_n)_1(y) \big)d\mu_2(y)\\
                    &=\sum_n\int_{\Omega_1}\mu_2\big( (B_n)_2(x) \big)d\mu_1(x)\\
                    &=\int_{\Omega_1}\sum_n\mu_2\big( (B_n)_2(x) \big)d\mu_1(x)\\
                    &=\int_{\Omega_1}\mu_2\big( (\bigcup_nB_n)_1(y) \big)d\mu_1(x).
                \end{align}
            \end{subequations}
    \end{subproof}
    Maintenant que \( \tribD\) est un $\lambda$-système contenant les rectangles, le lemme \ref{LemLUmopaZ} dit que la tribu engendrée par \( \tribD\) (c'est à dire \( \tribA_1\otimes \tribA_2\)) est le $\lambda$-système \( \tribD\) lui-même.

    La preuve est finie dans le cas de mesures finies. Nous commençons maintenant à prouver dans le cas où les mesures \( \mu_1\) et \( \mu_2\) sont seulement \( \sigma\)-finies. Nous considérons des suites croissantes \( \Omega_{i,n}\to\Omega_i\) d'ensembles mesurables et de mesure finie : \( \mu_i(\Omega_{i,n})<\infty\). D'abord remarquons que
    \begin{equation}\label{EqNFuBzBF}
        \mu_2\Big( (A\cap \Omega_{1,j}\times E_{2,j})_2(x) \Big)=\mu_2\Big( A_2(x)\cap \Omega_{2,j} \Big)\mtu_{\Omega_{1,j}}.
    \end{equation}
    En effet,
    \begin{subequations}
        \begin{align}
            \heartsuit&=(A\cap\Omega_{1,j}\times E_{2,j})_2(x)\\
            &=\{ y\in\Omega_2\tq (x,y)\in A\cap \Omega_{1,j}\times E_{2,j} \}\\
            &=\{ y\in \Omega_2\tq (x,y)\in A\times \Omega_{2,j} \}\cap\{ y\in\Omega_2\tq (x,y)\in \Omega_{1,j}\times \Omega_{2,j} \}.
        \end{align}
    \end{subequations}
    Si \( y\in \Omega_{1,j}\) alors \( \{ y\in \Omega_2\tq (x,y)\in \Omega_{1,j}\times \Omega_{2,j} \}=\Omega_{2,j}\) et dans ce cas
    \begin{equation}
        \heartsuit=\{ y\in \Omega_2\tq (x,y)\in A\times \Omega_{2,j} \}\cap \Omega_{2,j}=A_2(x)\cap E_{2,j}.
    \end{equation}
    Et inversement, si \( x\notin \Omega_{1,j}\) alors \( \heartsuit=\emptyset\). Dans les deux cas nous avons \eqref{EqNFuBzBF}.

    Les ensembles \( A\cap \Omega_{1,j}\times \Omega_{2,j}\) étant de mesure finie, nous pouvons leur appliquer la première partie :
    \begin{equation}
        \int_{\Omega_1}\mu_2\Big( (A\cap\Omega_{1,j}\times \Omega_{2,j})_2(x) \Big)d\mu_1(x)=\int_{\Omega_2}\mu_1\Big( (A\cap\Omega_{1,j}\times \Omega_{2,j})_1(y) \Big)d\mu_2(u),
    \end{equation}
    ou encore
    \begin{equation}
        \int_{\Omega_1}\mu_2\Big( A_2(x)\cap \Omega_{2,j} \Big)\mtu_{\Omega_{1,j}}(x)d\mu_1(x)=\int_{\Omega_2}\mu_1\Big( A_1(y)\cap \Omega_{1,j} \Big)\mtu_{\Omega_{2,j}}(y)d\mu_2(y).
    \end{equation}
    Ce que nous avons dans ces intégrales sont (par rapport à \( j\)) des suites croissantes de fonction positives; nous pouvons donc permuter une limite et une intégrale. En sachant que si \( k\to \infty\), alors
    \begin{subequations}
        \begin{align}
            \mtu_{1,j}(x)\to 1\\
            \mu_2\big( A_2(x)\cap \Omega_2,j \big)\to\mu_2\big( A_2(x) \big),
        \end{align}
    \end{subequations}
    nous trouvons le résultat demandé.
\end{proof}

\begin{theorem}[\cite{FubiniBMauray,MesIntProbb}]   \label{ThoWWAjXzi}
    Soient \( \mu_i\) des mesures $\sigma$-finies sur \( (\Omega_i,\tribA_i)\) (\( i=1,2\)). Il existe une et une seule mesure, notée \( \mu_1\otimes \mu_2\), sur \( (\Omega_1\times\Omega_2,\tribA_1\otimes\tribA_2)\) telle que
    \begin{equation}    \label{EqOIuWLQU}
        (\mu_1\otimes\mu_2)(A_1\times A_2)=\mu_1(A_1)\mu_2(A_2)
    \end{equation}
    pour tout \( A_1\in \tribA_1\) et \( A_2\in\tribA_2\). Cette mesure est donnée par la formule\footnote{Voir les notations du lemme \ref{LemAQmWEmN}.}
    \begin{equation}   \label{EqDFxuGtH}
        (\mu_1\otimes \mu_2)(A)=\int_{\Omega_1}\mu_2\big( A_2(x) \big)d\mu_1(x)=\int_{\Omega_2}\mu_1\big( A_1(y) \big)d\mu_2(y).
    \end{equation}
    Cette mesure est la \defe{mesure produit}{mesure!produit} de \( \mu_1\) par \( \mu_2\).
\end{theorem}
\index{mesure!produit}

\begin{proof}
    L'ensemble des rectangles de \( \Omega_1\times \Omega_2\) engendre la tribu \( \tribA_1\otimes\tribA_2\), est fermé par intersection et contient une suite croissante d'ensembles \( P_n\times R_n\) de mesure finie (\( \mu(P_n\times R_n)<\infty\)) telle que \( P_n\times R_n\to \Omega_1\times \Omega_2\). Cette suite est donné par le fait que \( \mu_1\) et \( \mu_2\) sont \( \sigma\)-finies. En effet si \( (X_n)\) et \( (Y_n)\) sont des recouvrements dénombrables de \( \Omega_1\) et \( \Omega_2\) par des ensembles de mesure finie, en posant \( P_n=\bigcup_{k=1}^nX_n\) et \( R_n=\bigcup_{k=1}^nY_n\) nous avons bien une suite croissante de rectangles qui tendent vers \( \Omega_1\times \Omega_2\). Avec ces rectangles en main, le théorème \ref{ThoJDYlsXu} donne l'unicité.

    Nous passons à l'existence de la mesure. Le théorème \ref{ThoCCIsLhO} dit que ces formules ont un sens et sont correctes. Il suffit donc de prouver que dans le cas des rectangles, ces formules se réduisent à \eqref{EqOIuWLQU}. Soit donc \( A=X_1\times X_2\) avec \( X_i\in\tribA_i\). Alors
    \begin{equation}
        A_1(y)=\{ x\in\Omega_1\tq (x,y)\in X_1\times X_2 \}
    \end{equation}
    et
    \begin{equation}
        \mu_1\big( A_1(y) \big)=\mtu_{X_2}(y)\mu_1(X_1),
    \end{equation}
    donc
    \begin{subequations}
        \begin{align}
            (\mu_1\otimes\mu_2)(A)&=\int_{\Omega_2}\mu_1\big( A_1(y) \big)d\mu_2(y)\\
            &=\int_{\Omega_2}\mu_1(X_1)\mtu_{X_2}(y)d\mu_2(y)\\
            &=\mu_1(X_1)\int_{\Omega_2}\mtu_{X_2}(y)d\mu_2(y)\\
            &=\mu_1(X_1)\mu_2(X_2).
        \end{align}
    \end{subequations}
    Pour cela nous avons utilisé le fait que l'intégrale de la fonction caractéristique d'un ensemble mesurable est la mesure de cet ensemble.
\end{proof}

\begin{definition}[Produit d'espaces mesurés]  \label{DefUMlBCAO}
    Si \( (\Omega_i,\tribA_i,\mu_i)\) sont deux espaces mesurés, l'\defe{espace produit}{produit!espaces mesurés} est l'ensemble \( \Omega_1\times \Omega_2\) muni de la tribu produit \( \tribA_1\otimes \tribA_2\) de la définition \ref{DefTribProfGfYTuR} et de la mesure produit \( \mu_1\otimes \mu_2\) définie par le théorème \ref{ThoWWAjXzi}.
\end{definition}

%---------------------------------------------------------------------------------------------------------------------------
\subsection{Mesure dominée}
%---------------------------------------------------------------------------------------------------------------------------

Soient \( \mu\) et \( \nu\) deux mesures sur le même espace \( \Omega\) et la même tribu \( \tribA\). Nous disons que la mesure \( \mu\) est \defe{dominée}{dominée!mesure}\cite{PersoFeng} par \( \nu\) si pour tout ensemble mesurable \( A\), \( \nu(A)=0\) implique \( \mu(A)=0\).

La mesure \( \mu\) est \defe{portée}{portée!mesure} par l'ensemble \( E\in\tribA\) si pour tout \( A\in\tribA\), 
\begin{equation}
    \mu(A)=\mu(A\cap E).
\end{equation}

Nous écrivons que \( \mu\perp\nu\)\nomenclature[Y]{\( \mu\perp\nu\)}{mesures perpendiculaires} si il existe un ensemble \( E\in\tribA\) tel que \( \mu\) soit porté par \( E\) et \( \nu\) soit porté par \( \complement E\).

\begin{theorem}[Radon-Nikodym\cite{NikoLi}]\index{Radon-Nikodym}
    Soient \( \mu\) et \( m\) deux mesures \( \sigma\)-finies sur un espace métrisable \( (\Omega,\tribA)\).
    \begin{enumerate}
        \item
            Il existe un unique couple de mesures \( \mu_1\) et \( \mu_2\) telles que
            \begin{enumerate}
                \item
                    \( \mu=\mu_1+\mu_2\)
                \item
                    \( \mu_1\) est dominé par \( m\)
                \item
                    \( \mu_2\perp m\).
            \end{enumerate}
            Dans ce cas, les mesures \( \mu_1\) et \( \mu_2\) sont positives et \( \sigma\)-finies.
        \item
            À égalité \(  m\)-presque partout près, il existe une unique fonction mesurable positive \( f\) telle que pour tout mesurable \( A\),
            \begin{equation}
                \mu_1(A)=\int_Ad\mu_1=\int_{\Omega}\mtu_Afd m.
            \end{equation}
        \item
            À égalité \( m\)-presque partout près, il existe une unique fonction positive mesurable \( h\) telle que \( \mu_1=hm\).
    \end{enumerate}
\end{theorem}
%TODO : une preuve

\begin{corollary}   \label{CorZDkhwS}
    Si \( \mu\) es une mesure \( \sigma\)-finie dominée par la mesure \( \sigma\)-finie \( m\), alors \( \mu\) possède une unique fonction de densité.
\end{corollary}

\begin{corollary}       \label{CorDomDens}
    Soient \( \mu\) et \( m\), deux mesures positives \( \sigma\)-finies sur \( (\Omega,\tribA)\). Alors \( m\) domine \( \mu\) si et seulement si \( \mu\) possède une densité par rapport à \( m\).
\end{corollary}
 
\begin{proof}
    Si \( \mu\) est dominée par \( m\), alors la décomposition \( \mu=\mu+0\) satisfait le théorème de Radon-Nikodym. Par conséquent il existe une fonction \( f\) telle que
    \begin{equation}
        \mu(A)=\int_Afdm.
    \end{equation}
    Cette fonction est alors une densité pour \( \mu\) par rapport à \( m\).

    Pour la réciproque, nous supposons que \( \mu\) a une densité \( f\) par rapport à \( m\), et que \( A\) est une ensemble de \( m\)-mesure nulle :
    \begin{equation}
        m(A)=\int_{\Omega}\mtu_Adm=0.
    \end{equation}
    Cela signifie que la fonction \( \mtu_A\) est \( m\)-presque partout nulle. La fonction produit \( \mtu_Af\) est également nulle \( m\)-presque partout, et par conséquent
    \begin{equation}
        \mu(A)=\int_{\Omega}\mtu_Afdm=0.
    \end{equation}
\end{proof}

\begin{probleme}
    Est-ce que la démonstration de cela ne demande pas la convergence monotone d'une façon ou d'une autre ?
\end{probleme}

%--------------------------------------------------------------------------------------------------------------------------- 
\subsection{Primitive et intégrale}
%---------------------------------------------------------------------------------------------------------------------------

\begin{definition}
    Soit \( I\) un intervalle de \( \eR\) et une fonction \( f\colon I\to \eR\). La fonction \( F\colon I\to \eR\) est une \defe{primitive}{primitive!fonction} de \( f\) si \( F\) est dérivable sur \( I\) et si \( F'(x)=f(x)\) pour tout \( x\) dans \( I\).
\end{definition}

\begin{proposition}[Primitive et intégrale] \label{PropEZFRsMj}
    Soit \( f\) une fonction intégrable sur \( \mathopen[ a , b \mathclose]\) et \( x_0\) un point de continuité de \( f\). Nous considérons la fonction
    \begin{equation}
        \begin{aligned}
            F\colon \mathopen[ a , b \mathclose]&\to \eR \\
            x&\mapsto \int_{\mathopen[ a , x \mathclose]}f(t)dt.
        \end{aligned}
    \end{equation}
    Cette fonction est dérivable en \( x_0\) et \( F'(x_0)=f(x_0)\).
\end{proposition}

\begin{proof}
    Soit \( \epsilon>0\). Par continuité de \( f\) en \( x_0\), il existe une fonction \( \alpha\) telle que
    \begin{equation}
        f(x_0+h)=f(x_0)+\alpha(h)
    \end{equation}
    avec \( \lim_{h\to 0} \alpha(h)=0\). De plus il existe un \( \delta>0\) tel que \( \alpha(h)<\epsilon\) pour tout \( h<\delta\). À partir de maintenant nous ne considérons plus que de tels \( h\).

    Nous calculons la dérivée de \( F\) en \( x_0\). Pour cela,
    \begin{subequations}
        \begin{align}
            F(x_0+h)-F(x_0)&=\int_{x_0}^{x_0+h}f(t)dt\\
        &=\int_0^hf(x_0+t)dt\\
        &=\int_0^h\big[ f(x_0)+\alpha(t) \big]dt\\
        &=hf(x_0)+\int_0^{h}\alpha(t)dt.
        \end{align}
    \end{subequations}
    Nous avons donc, pour tout \( h<\delta\),
    \begin{equation}
        hf(x_0)-h\epsilon\leq F(x_0+h)-F(x_0)\leq hf(x_0)+h\epsilon.
    \end{equation}
    En divisant par \( h\) et en prenant la limite \( h\to 0\),
    \begin{equation}
        F'(x_0)\in B\big( f(x_0),\epsilon \big).
    \end{equation}
    Cela étant valable pour tout \( \epsilon>0\) nous en déduisons que
    \begin{equation}
        F'(x_0)=f(x_0).
    \end{equation}
\end{proof}

Ce petit résultat nous donne une façon «pratique» de calculer des intégrales en cherchant des primitives. Nous rappelons qu'en vertu du corollaire \ref{CorZeroCst}, une fonction ne possède qu'une seule primitive à constante près.

Le théorème suivant est à utiliser pour calculer des intégrales des fonctions réelle.
\begin{theorem}[Théorème fondamental du calcul intégral]
    Soit \( f\) une fonction continue sur l'intervalle \( I\subset \eR\) et \( F\) une primitive de \( f\) sur \( I\). Alors
    \begin{equation}
        \int_a^bf(t)dt=F(b)-F(a).
    \end{equation}
\end{theorem}

\begin{proof}
    Nous avons vu par la proposition \ref{PropEZFRsMj} que la fonction
    \begin{equation}
        \begin{aligned}
            \tilde F\colon I&\to \eR \\
            x&\mapsto  \int_a^xf(t)dt
        \end{aligned}
    \end{equation}
    était une primitive de \( f\); c'est même l'unique\footnote{Corollaire \ref{CorZeroCst}.} primitive de \( f\) sur \( I\) à s'annuler pour \( x=a\). Nous avons évidemment
    \begin{equation}
        \int_a^bf(t)dt=\tilde F(b).
    \end{equation}
    Si \( F\) est une primitive quelconque, il suffit de soustraire sa valeur en \( x=a\) : \( \tilde F(x)=F(x)-F(a)\) et donc
    \begin{equation}
        \int_a^bf(t)dt=\tilde F(b)=F(b)-F(a),
    \end{equation}
    comme il fallait le prouver.
\end{proof}

%--------------------------------------------------------------------------------------------------------------------------- 
\subsection{Théorème d'approximation}
%---------------------------------------------------------------------------------------------------------------------------

\begin{theorem}[Théorème d'approximation\cite{YHRSDGc}]     \label{ThoAFXXcVa}
    Soit \( (X,\tribB,\mu)\) un espace mesuré où \( \tribB\) sont les boréliens de \( X\). Soit \( A\in \tribB\) tel que \( A\subset W\) où \( W\) est un ouvert avec \( \mu(W)<\infty\). Soit aussi \( \epsilon>0\).
    \begin{enumerate}
        \item
            Il existe un fermé \( F\) et un ouvert \( V\) tels que \( \mu(V)<\infty\) et
            \begin{equation}
                F\subset A\subset V
            \end{equation}
            et \( \mu(V\setminus F)<\epsilon\).
        \item
            Il existe \( f\in C^0(X,\eR)\) nulle hors de \( W\) vérifiant \( 0\leq f\leq 1\) et
            \begin{equation}
                \int_X| \mtu_A-f |^pd\mu(x)<\epsilon.
            \end{equation}
    \end{enumerate}
\end{theorem}
% TODO : la preuve est dans la référence. Il faut replacer ce théorème après la définition de l'intégrale.

% TODO : les mesures à densité doivent être après les intégrales.

%+++++++++++++++++++++++++++++++++++++++++++++++++++++++++++++++++++++++++++++++++++++++++++++++++++++++++++++++++++++++++++ 
\section{Permuter limite et intégrale}
%+++++++++++++++++++++++++++++++++++++++++++++++++++++++++++++++++++++++++++++++++++++++++++++++++++++++++++++++++++++++++++

%--------------------------------------------------------------------------------------------------------------------------- 
\subsection{Convergence uniforme}
%---------------------------------------------------------------------------------------------------------------------------

\begin{proposition}[Permuter limite et intégrale]       \label{PropbhKnth}
    Soit \( f_n\to f\) uniformément sur un ensemble mesuré \( A\) de mesure finie. Alors si les fonctions \( f_n\) et \( f\) sont intégrables sur \( A\), nous avons
    \begin{equation}
        \lim_{n\to \infty} \int_A f_n=\int_A \lim_{n\to \infty} f_n.
    \end{equation}
\end{proposition}

\begin{proof}
    Notons \( f\) la limite de la suite \( (f_n)\). Pour tout \( n\) nous avons les majorations
    \begin{subequations}
        \begin{align}
            \left| \int_A f_n d\mu-\int_A fd\mu \right| &\leq \int_A| f_n-f |d\mu\\
            &\leq \int_A \| f_n-f \|_{\infty}d\mu\\
            &=\mu(A)\| f_n-f \|_{\infty}
        \end{align}
    \end{subequations}
    où \( \mu(A)\) est la mesure de \( A\). Le résultat découle maintenant du fait que \( \| f_n-f \|_{\infty}\to 0\).
\end{proof}
Il existe un résultat considérablement plus intéressant que cette proposition. En effet, l'intégrabilité de \( f\) n'est pas nécessaire. Cette hypothèse peut être remplacée soit par l'uniforme convergence de la suite (théorème \ref{ThoUnifCvIntRiem}), soit par le fait que les normes des \( f_n\) sont uniformément bornées (théorème de la convergence dominée de Lebesgue \ref{ThoConvDomLebVdhsTf}).

\begin{theorem}[\cite{BJblWiS}]			\label{ThoUnifCvIntRiem}
    La limite uniforme d'une suite de fonctions intégrables sur un borné est intégrable, et on peut permuter la limite et l'intégrale. 
    
    Plus précisément, soit \( A\) un ensemble de \( \mu\)-mesure finie et \( f_n\colon A\to \eR\) des fonctions intégrables sur \( A\). Si la limite \( f_n\to f\) est uniforme, alors \( f\) est intégrable sur \( A\) et nous pouvons inverser la limite et l'intégrale :
    \begin{equation}
        \lim_{n\to \infty} \int_A f_n=\int_A\lim_{n\to \infty} f_n.
    \end{equation}
\end{theorem}

\begin{proof}
    Soit \( \epsilon>0\) et \( n\) tel que \( \| f_n-f \|_{\infty}\leq \epsilon\) (ici la norme uniforme est prise sur \( A\)). Étant donné que \( f_n\) est intégrable sur \( A\), il existe une fonction simple \( \varphi_n\) qui minore \( f_n\) telle que
    \begin{equation}
        \left| \int_{A}\varphi_n-\int_A f_n \right| <\epsilon.
    \end{equation}
    La fonction \( \varphi_n+\epsilon\) est une fonction simple qui majore la fonction \( f\). Si \( \psi\) est une fonction simple qui minore \( f\), alors
    \begin{equation}
        \int_A\psi\leq\int_A\varphi_n+\epsilon\leq\int_A f_n+\epsilon\mu(A).
    \end{equation}
    Par conséquent le supremum qui définit \( \int_A f\) existe, ce qui montre que \( f\) est intégrable. Le fait qu'on puisse inverser la limite et l'intégrale est maintenant une conséquence de la proposition \ref{PropbhKnth}.
\end{proof}

\begin{remark}
    L'hypothèse sur le fait que \( A\) est de mesure finie est importante. Il n'est pas vrai qu'une suite uniformément convergente de fonctions intégrables est intégrables. En effet nous avons par exemple la suite
    \begin{equation}
        f_n(x)=\begin{cases}
            1/x    &   \text{si \( x<n\)}\\
            0    &    \text{sinon}
        \end{cases}
    \end{equation}
    qui converge uniformément vers \( f(x)=1/x\) sur \( A=\mathopen[ 1 , \infty [\). Le limite n'est cependant pas intégrable sur \( A\).
\end{remark}

%---------------------------------------------------------------------------------------------------------------------------
\subsection{Convergence monotone}
%---------------------------------------------------------------------------------------------------------------------------

\begin{theorem}[Théorème de la convergence monotone ou de Beppo-Levi\cite{mathmecaChoi}] \label{ThoConvMonFtBoVh}
    Soit un espace mesuré \( (\Omega,\tribA,\mu)\) et \( (f_n)\) une suite croissante de fonctions mesurables à valeurs dans \( \mathopen[ 0 , \infty \mathclose]\). Alors la limite ponctuelle \( \lim_{n\to \infty} f_n\) existe, est mesurable et
    \begin{equation}    \label{EqFHqCmLV}
        \lim_{n\to \infty} \int_{\Omega}f_nd\mu= \int_{\Omega}\lim_{n\to \infty} f_nd\mu,
    \end{equation}
    cette intégrable valant éventuellement \( \infty\).
\end{theorem}
\index{théorème!convergence!monotone}
\index{théorème!Beppo-Levi}

\begin{proof}
    La limite ponctuelle de la suite est la fonction à valeurs dans \( \mathopen[ 0 , \infty \mathclose]\) donnée par
    \begin{equation}
        f(x)=\lim_{n\to \infty} f_n(x).
    \end{equation}
    Ces limites existent parce que pour chaque \( x\) la suite \( f_n(x)\) est une suite numérique croissante. Nous notons
    \begin{equation}
        I_0=\int_{\Omega}fd\mu.
    \end{equation}
    Nous posons par ailleurs
    \begin{equation}
        I_n=\int_{\Omega}f_n.
    \end{equation}
    Cela est une suite numérique croissante qui a par conséquent une limite que nous notons \( I=\lim_{n\to \infty} I_n\). Notre objectif est de montrer que \( I=I_0\). D'abord par croissance de la suite, pour tous $n$ nous avons \( I_n\leq I_0\), par conséquent \( I\leq I_0\).

    Nous prouvons maintenant l'inégalité dans l'autre sens en nous servant de la définition \eqref{EqDefintYfdmu}. Soit une fonction simple \( h\) telle que \( h\leq f\), et une constante \( 0<C<1\). Nous considérons les ensembles
    \begin{equation}
        E_n=\{ x\in\Omega\tq f_n(x)\geq Ch(x) \}.
    \end{equation}
    Ces ensembles vérifient les propriétés \( E_n\subset E_{n+1}\) et \( \bigcup_{n=1}^{\infty}E_n=\Omega\). Pour chaque \( n\) nous avons les inégalités
    \begin{equation}
        \int_{\Omega}f_n\geq\int_{E_n}f_n\geq C\int_{E_n}h.
    \end{equation}
    Si nous prenons la limite \( n\to\infty\) dans ces inégalités,
    \begin{equation}
        \lim_{n\to \infty} \int_{\Omega}f_n\geq C\lim_{n\to \infty} \int_{E_n}h=C\int_{\Omega}h.
    \end{equation}
    Par conséquent \( \lim_{n\to \infty} \int f_n\geq C\int_{\Omega}h\). Mais étant donné que cette inégalité est valable pour tout \( C\) entre \( 0\) et \( 1\), nous pouvons l'écrire sans le \( C\) :
    \begin{equation}        \label{EqzAKEaU}
        \lim_{n\to \infty} \int_{\Omega}f_n\geq \int_{\Omega}h.
    \end{equation}
    Par définition, l'intégrale de \( f\) est donné par le supremum des intégrales de \( h\) où \( h\) est une fonction simple dominée par \( f\). En prenant le supremum sur \( h\) dans l'équation \eqref{EqzAKEaU} nous avons
    \begin{equation}
        \lim_{n\to \infty} \int_{\Omega}f_n\geq\int_{\Omega}f,
    \end{equation}
    ce qu'il nous fallait.
\end{proof}

\begin{remark}
    La proposition \ref{PropWBavIf} ainsi que le lemme \ref{LemYFoWqmS} montrent qu'une fonction mesurable peut-être écrite comme limite croissante de fonctions simples. Cela permet de démontrer des théorèmes en commençant par prouver sur les fonctions simples et en utilisant Beppo-Levi pour généraliser.
\end{remark}

\begin{remark}
    Une des raisons de demander la positivité des fonctions \( f_n\) est de n'avoir pas d'ambiguïté à parler d'intégrales qui valent \( \infty\). Si par exemple nous prenons \( \Omega=\mathopen[ 0 , 1 \mathclose]\) et que nous considérons
    \begin{equation}
        f_n(x)=\begin{cases}
            0    &   \text{si \( x\leq \frac{1}{ n }\)}\\
            \frac{1}{ x }    &    \text{sinon}.
        \end{cases}
    \end{equation}
    Ce sont des fonctions intégrables, mais la limite étant la fonction \( 1/x\), l'égalité \eqref{EqFHqCmLV} est une égalité entre deux intégrales valant \( \infty\).
\end{remark}

\begin{corollary}[Inversion de somme et intégrales] \label{CorNKXwhdz}
    Si \( (u_n)\) est une suite de fonctions mesurables positives ou nulles, alors
    \begin{equation}
        \sum_{i=0}^{\infty}\int u_i=\int\sum_{i=0}^{\infty}u_i.
    \end{equation}
\end{corollary}

\begin{proof}
    Nous considérons la suite des sommes partielles de \( (u_n)\) : \( f_n(x)=\sum_{i=0}^nu_n(x)\). Le théorème de la convergence monotone (théorème \ref{ThoConvMonFtBoVh}) implique que
    \begin{equation}
        \lim_{n\to \infty} \int f_n=\int\lim_{n\to \infty} f_n.
    \end{equation}
    Nous remplaçons maintenant \( f_n\) par sa valeur en termes des \( u_i\) et dans le membre de gauche nous permutons l'intégrale avec la somme finie :
    \begin{equation}
        \lim_{n\to \infty} \sum_{i=0}^{\infty}\int u_n=\int\sum_{i=0}^{\infty}u_n,
    \end{equation}
    ce qu'il fallait démontrer.
\end{proof}

\begin{lemma}[Lemme de Fatou]\index{lemme!Fatou}\index{Fatou}   \label{LemFatouUOQqyk}
    Soit \( (\Omega,\tribA,\mu)\) un espace mesuré et \( f_n\colon \Omega\to \mathopen[ 0 , \infty \mathclose]  \) une suite de fonctions mesurables. Alors la fonction \( f(x)=\liminf f_n(x)\) est mesurable et
    \begin{equation}
        \int_{\Omega}\liminf f_nd\mu\leq\liminf\int_{\Omega}fd\mu.
    \end{equation}
\end{lemma}
%TODO : pour la mesurabilité, il faudra citer un théorème du genre de celui fait avec le sup.

\begin{proof}
    Nous posons 
    \begin{equation}
        g_n(x)=\inf_{i\geq n}f_i(x).
    \end{equation}
    Cela est une suite croissance de fonctions positives mesurables telles que, par définition, 
    \begin{equation}
        \lim_{n\to \infty}g_n(x)=\liminf f_n(x).
    \end{equation}
    Nous pouvons y appliquer le théorème de la convergence monotone,
    \begin{equation}
        \lim_{n\to \infty} \int g_n(x)=\int\liminf f_n(x).
    \end{equation}
    Par ailleurs, pour chaque \( i\geq n\) nous avons
    \begin{equation}
        \int g_n\leq \int f_i,
    \end{equation}
    en passant à l'infimum nous avons
    \begin{equation}
        \int g_n\leq \inf_{i\geq n}\int f_i,
    \end{equation}
    et en passant à la limite nous avons
    \begin{equation}
        \int\liminf f_n=\lim_{n\to \infty} \int g_n\leq \lim_{n\to \infty} \inf_{i\geq n}\int f_i=\liminf_{i\to\infty}\inf f_i.
    \end{equation}
\end{proof}

L'inégalité donnée dans ce lemme n'est en général pas une égalité, comme le montre l'exemple suivant :
\begin{equation}
    f_i=\begin{cases}
        \mtu_{\mathopen[ 0 , 1 \mathclose]}    &   \text{si \( i\) est pair}\\
        \mtu_{\mathopen[ 1 , 2 \mathclose]}    &    \text{si \( i\) est impair}.
    \end{cases}
\end{equation}
Nous avons évidemment \( g_n(x)=0\) tandis que \( \int_{\mathopen[ 0 , 2 \mathclose]}f_i=1\) pour tout \( i\).

%---------------------------------------------------------------------------------------------------------------------------
\subsection{Convergence dominée de Lebesgue}
%---------------------------------------------------------------------------------------------------------------------------

\begin{theorem}[Convergence dominée de Lebesgue]        \label{ThoConvDomLebVdhsTf}
    Soit \( (f_n)_{n\in\eN}\) une suite de fonctions intégrables sur \( (\Omega,\tribA,\mu)\) à valeurs dans \( \eC\) ou \( \eR\). Nous supposons que  \( f_n\to f\) simplement sur \( \Omega\) presque partout et qu'il existe une fonction intégrable \( g\) telle que
    \begin{equation}
        | f_n(x) |< g(x) 
    \end{equation}
    pour presque\footnote{Si il n'y avait pas le «presque» ici, ce théorème serait à peu près inutilisable en probabilité ou en théorie des espaces \( L^p\), comme dans la démonstration du théorème de Fischer-Riesz \ref{ThoGVmqOro} par exemple.} tout \( x\in\Omega\) et pour tout \( n\in \eN\). Alors
    \begin{enumerate}
        \item
            \( f\) est intégrable,
        \item
           $\lim_{n\to \infty} \int_{\Omega}f_n=\int_\Omega f$,
        \item
            $\lim_{n\to \infty} \int_{\Omega}| f_n-f |=0$.
    \end{enumerate}
\end{theorem}
\index{théorème!convergence!dominée de Lebesgue}
\index{dominée!convergence (Lebesgue)}

\begin{proof}

    La fonction limite \( f\) est intégrable parce que \( | f |\leq g\) et \( g\) est intégrable (lemme \ref{LemPfHgal}). Par hypothèse nous avons
    \begin{equation}
        -g(x)\leq f_n(x)\leq g(x).
    \end{equation}
    En particulier la fonction \( g_n=f_n+g\) est positive et mesurable si bien que le lemme de Fatou (lemme \ref{LemFatouUOQqyk}) implique
    \begin{equation}
        \int_{\Omega}\liminf g_n\leq\liminf\int_{\Omega}g_n.
    \end{equation}
    Évidement nous avons \( \liminf g_n=f+g\), de telle sorte que
    \begin{equation}
        \int f+\int g\leq \liminf\int g_n=\liminf\int f_n+\int g,
    \end{equation}
    et le nombre \( \int g\) étant fini, nous pouvons le retrancher des deux côtés de l'inégalité :
    \begin{equation}
        \int f\leq\liminf\int f_n.
    \end{equation}
    Afin d'obtenir une minoration de \( \int f\) nous refaisons exactement le même raisonnement en utilisant la suite de fonctions \( k_n=-f_n\to k=-f\). Nous obtenons que
    \begin{equation}
        \int k\geq\liminf\int k_n=-\limsup\int f_n,
    \end{equation}
    et par conséquent
    \begin{equation}    \label{IneqsndMYTO}
        \liminf\int f_n\leq\int f\leq\limsup\int f_n.
    \end{equation}
    La limite supérieure étant plus grande ou égale à la limite inférieure, les trois quantités dans les inégalités \eqref{IneqsndMYTO} sont égales.

    Nous prouvons maintenant le troisième point. Soit la suite de fonctions
    \begin{equation}
        h_n(x)=| f_n(x)-f(x) |
    \end{equation}
    qui tend ponctuellement vers zéro. De plus
    \begin{equation}
    h_n(x)\leq | f_n(x) |+| f(x) |\leq 2g(x),
    \end{equation}
    ce qui prouve que les \( h_n\) majorés par une fonction intégrable. Donc
    \begin{equation}
        \lim_{n\to \infty} \int_{\Omega}| f_n-f |= \lim_{n\to \infty} \int_{\Omega}h_n(x)dx=\int_{\Omega}\lim_{n\to \infty} | f_n(x)-f(x) |=0
    \end{equation}
\end{proof}

\begin{remark}
    Lorsque nous travaillons sur des problèmes de probabilités, la fonction \( g\) peut être une constante parce que les constantes sont intégrables sur un espace de probabilité.
\end{remark}

\begin{corollary}       \label{CorCvAbsNormwEZdRc}
    Soit \( (a_i)_{i\in \eN}\) une suite numérique absolument convergente. Alors elle est convergente. Il en est de même pour les séries de fonctions si on considère la convergence ponctuelle.
\end{corollary}

\begin{proof}
    L'hypothèse est la convergence de l'intégrale \( \int_{\eN}| a_i |dm(i)\) où \( dm\) est la mesure de comptage. Étant donné que \( | a_i |\leq | a_i |\), la fonction \( a_i\) (fonction de \( i\)) peut jouer le rôle de \( g\) dans le théorème de la convergence dominée de Lebesgue (théorème \ref{ThoConvDomLebVdhsTf}).
\end{proof}
Nous utiliseront ce résultat pour montrer que la transformée de Fourier d'une fonction \( L^1(\eR^d)\) est continue (proposition \ref{PropJvNfj}).

\begin{proposition}[\cite{YHRSDGc}] \label{PropUXjnwLf}
    \begin{enumerate}
        \item
            Une fonction mesurable et positive est limite (simple) d'une suite croissante de fonctions étagées, mesurables et positives.
        \item
            Si \( f\colon \eR^d\to \bar \eR\) est mesurable, alors elle est limite (simple) de fonctions étagées \( f_n\) telles que \( | f_n |\leq | f |\).
    \end{enumerate}
\end{proposition}
%TODO : la preuve est dans le document cité.

%+++++++++++++++++++++++++++++++++++++++++++++++++++++++++++++++++++++++++++++++++++++++++++++++++++++++++++++++++++++++++++ 
\section{Deux gros théorèmes : Fubini et changement de variable}
%+++++++++++++++++++++++++++++++++++++++++++++++++++++++++++++++++++++++++++++++++++++++++++++++++++++++++++++++++++++++++++

%--------------------------------------------------------------------------------------------------------------------------- 
\subsection{Théorème de Fubini-Tonelli et de Fubini}
%---------------------------------------------------------------------------------------------------------------------------

Il existe plusieurs résultats similaires. 
\begin{itemize}
    \item
        le théorème de Fubini-Tonelli \ref{ThoWTMSthY} demande que la fonction soit mesurable et positive;
    \item
        le théorème de Fubini \ref{ThoFubinioYLtPI} demande que la fonction soit intégrable (mais pas spécialement positive);
    \item
        le corollaire \ref{CorTKZKwP} demande l'intégrabilité de la valeur absolue des intégrales partielles pour déduire que la fonction elle-même est intégrable.
\end{itemize}

%TODO : des démonstrations de ces trois théorèmes seraient les bienvenues.

Nous rappelons que \( \eR^n\) muni de la mesure de Lebesgue est un espace mesuré \( \sigma\)-fini, conformément à la définition \ref{DefBTsgznn}.

\begin{theorem}[Fubini-Tonelli\cite{NBoIEXO}]\label{ThoWTMSthY}
    Soient \( (\Omega_i,\tribA_i,\mu_i)\) deux espaces mesurés \( \sigma\)-finis, et \( (\Omega,\tribA,\mu)\) l'espace produit. Soit une fonction \( f\colon \Omega_1\times \Omega_2\to \eR\) une fonction mesurable et positive (valant éventuellement \( \infty\) à certains endroits)
    Alors :
    \begin{enumerate}
        \item
            Les fonction
            \begin{equation}
                F_1\colon x\mapsto \int_{\Omega_2}f(x,y)d\mu_2(y)
            \end{equation}
            et
            \begin{equation}
                F_2\colon y\mapsto \int_{\Omega_1}f(x,y)d\mu_1(x)
            \end{equation}
            sont mesurables.
        \item
            Toutes les intégrales imaginables existent et sont égales :
            \begin{subequations}    \label{EqJRVtOGx}
                \begin{align}
                    \iint_{\Omega_1\times \Omega_2}f(x,y)d(\mu_1\otimes \mu_2)(x,y)&=\int_{\Omega_1}\left[ \int_{\Omega_2}f(x,y)d\mu_2(y) \right]d\mu_1(x)\\
                &=\int_{\Omega_2}\left[ \int_{\Omega_1}f(x,y)d\mu_1(x) \right]d\mu_2(y).
                \end{align}
            \end{subequations}
    \end{enumerate}
\end{theorem}
\index{théorème!Fubini-Tonelli}

\begin{proof}
    Commençons par prouver le théorème dans le cas d'une fonction caractéristique d'un ensemble mesurable : \( f(x,y)=\mtu_{A}(x,y)\) pour un certain ensemble \( A\subset \Omega_1\times \Omega_2\). Dans ce cas,
    \begin{equation}
        F_1(x)=\int_{\Omega_2}\mtu_A(x,y)d\mu_2(y)=\int_{\omega_2}\mtu_{A_1(y)}(x)d\mu_2(y)=\mu_2\big( A_1(x) \big),
    \end{equation}
    et nous avons déjà vu au théorème \ref{ThoCCIsLhO} que cette fonction \( F_1\) était alors mesurable. En utilisant maintenant les égalités \eqref{EqDFxuGtH} ainsi que le fait que \( \mtu_A(x,y)=\mtu_{A_2(x)}(y)\) nous avons
    \begin{subequations}
        \begin{align}
            \iint_{\Omega_1\times \Omega_2}\mtu_A(x,y)d(\mu_1\otimes \mu_2)(x,y)&=(\mu_1\otimes \mu_2)(A)\\
            &=\int_{\Omega_1}\mu_2\big( A_2(x) \big)d\mu_1(x)\\
            &=\int_{\Omega_1}\left[   \int_{\Omega_2}\mtu_{A_2(x)}(y)d\mu_2(y)  \right]d\mu_1(x)\\
            &=\int_{\Omega_1}\left[ \int_{\Omega_2}\mtu_A(x,y)d\mu_2(y) \right]d\mu_1(x).
        \end{align}
    \end{subequations}
    Le théorème étant valable pour les fonctions caractéristiques, il est valable pour les fonctions simples (définition \ref{DefBPCxdel}) par linéarité de l'intégrale.

    Si \( f\) n'est pas une fonction simple, alors la proposition \ref{PropWBavIf} nous donne une suite croissante de fonctions simples et positives convergeant ponctuellement vers \( f\). La partie du théorème sur les fonctions simples dit que pour chaque \( n\) l'intégrale
    \begin{equation}
        \iint_{\Omega_1\times \Omega_2}f_n(x,y)d(\mu_1\otimes\mu_2)(x,y)
    \end{equation}
    peut être décomposée comme il faut en suivant la formule \eqref{EqJRVtOGx}. Il faut pouvoir permuter la limite et l'intégrale dans chacun de cas. D'abord le théorème de la convergence monotone \ref{ThoConvMonFtBoVh} appliqué à l'espace \( \Omega_1\times \Omega_2\) dit que
    \begin{equation}
        \lim_{n\to \infty} \iint_{\Omega_1\times \Omega_2}f_n(x,y)d(\mu_1\otimes \mu_2)(x,y)= \iint_{\Omega_1\times \Omega_2}f(x,y)d(\mu_1\otimes \mu_2)(x,y).
    \end{equation}
    Ensuite, pour chaque \( x\in\Omega_1\), les fonctions
    \begin{equation}
        \sigma_n(y)=\int_{\Omega_1}f_n(x,y)d\mu_1(x)
    \end{equation}
    forment une suite croissante de fonctions mesurables; nous leur appliquons encore le théorème de la convergence monotone :
    \begin{subequations}
        \begin{align}
            \lim_{n\to \infty} \int_{\Omega_2}\left[ \int_{\Omega_1}f_n(x,y)d\mu_1(x) \right]d\mu_2(y)&=\lim_{n\to \infty} \int_{\Omega_2}\sigma_n(y)d\mu_2(y)\\
            &=\int_{\Omega_2}\left[\lim_{n\to \infty} \int_{\Omega_1}f_n(x,y)d\mu_1(x)\right]d\mu_2(y)\\
            &=\int_{\Omega_2}\left[ \int_{\Omega_1}f(x,y)d\mu_1(x) \right]d\mu_2(y)
        \end{align}
    \end{subequations}
    où nous avons utilisé une seconde fois Beppo-Levi.
\end{proof}

\begin{theorem}[Fubini\cite{MesIntProbb}]\label{ThoFubinioYLtPI}
    Soient \( (\Omega_i,\tribA_i,\mu_i)\) deux espaces mesurés \( \sigma\)-finis, et \( (\Omega,\tribA,\mu)\) l'espace produit. Soit 
    \begin{equation}
        f\in L^1\big( (\Omega,\tribA),\eR \big),
    \end{equation}
    c'est à dire une fonction à valeurs réelles mesurable et intégrable sur \( \Omega\). Alors :
    \begin{enumerate}
        \item
            Pour presque tout \( x\in \Omega_1\), la fonction \( y\mapsto f(x,y)\) est \( L^1(\Omega_2)\).
        \item
            Si nous posons
            \begin{equation}
                \varphi_f(x)=\int_{\Omega_2}f(x,y)d\mu_2(y);
            \end{equation}
            alors \( \varphi_f\in L^1(\Omega_1)\).
        \item   \label{ItemQMWiolgiii}
            Nous avons la formule d'inversion d'intégrale
            \begin{subequations}
                \begin{align}
                \int_{\Omega}fd(\mu_1\otimes \mu_2)&=\int_{\Omega_1}\varphi_fd\mu_1\\
                &=\int_{\Omega_1}\left[ \int_{\Omega_2}f(x,y)d\mu_2(y) \right]d\mu_1(x)\\
                &=\int_{\Omega_2}\left[ \int_{\Omega_1}f(x,y)d\mu_1(x) \right]d\mu_2(y).
                \end{align}
            \end{subequations}
    \end{enumerate}
\end{theorem}
\index{théorème!Fubini!espace mesuré}

Si la fonction \( (x,y)\mapsto f(x)g(y)\) satisfait aux hypothèse du théorème de Fubini alors
\begin{equation}    \label{EqTJEEsJW}
    \int_{\Omega_1\times \Omega_2} f(x)g(y)dx\otimes dy=\left( \int_{\Omega_1}f(x)dx \right)\left( \int_{\Omega_2}g(y)dy \right).
\end{equation}
Le théorème de Fubini est souvent utilisé sous cette forme.

\begin{corollary}\label{CorTKZKwP}
    Soient \( (\Omega_i,\tribA_i,\mu_i)\) deux espaces mesurés \( \sigma\)-finis, et \( (\Omega,\tribA,\mu)\) l'espace produit\footnote{Définition \ref{DefUMlBCAO}.}. Soit une fonction mesurable \( f\colon \Omega\to \eR\). Alors les conditions suivantes sont équivalentes
    \begin{enumerate}
        \item
            \( f\in L^1(\Omega)\),
        \item
            \begin{equation}
                \int_{\Omega_1}\left[ \int_{\Omega_2}| f |d\mu_2 \right]d\mu_1 <\infty,
            \end{equation}
        \item
            \begin{equation}
                \int_{\Omega_2}\left[ \int_{\Omega_1}| f |d\mu_1 \right]d\mu_2 <\infty.
            \end{equation}
    \end{enumerate}
\end{corollary}
En pratique, lorsqu'on ne sait pas a priori si \( f\) est intégrable sur \( \Omega_1\times \Omega_2\), nous testons l'intégrabilité en chaine de \( | f |\), et si c'est bon, alors nous savons que \( f\) est intégrable sur le produit et qu'on peut permuter les intégrales.

\begin{example}
    Nous montrons que le théorème ne tient pas si une des deux mesures n'est pas \( \sigma\)-finie. Soit \( I=\mathopen[ 0 , 1 \mathclose]\). Nous considérons l'espace mesuré
    \begin{equation}
        (I,\Borelien(I),\lambda)
    \end{equation}
    où \( \Borelien(I)\) est la tribu des boréliens sur \( I\) et \( \lambda\) est la mesure de Lebesgue (qui est $\sigma$-finie). D'autre part nous considérons l'espace mesuré
    \begin{equation}
        (I,\partP(I),m)
    \end{equation}
    où \( \partP(I)\) est l'ensemble des parties de \( I\) et \( m\) est la mesure de comptage. Cette dernière n'est pas $\sigma$-finie parce que les seuls ensembles de mesure finie pour la mesure de comptage sont des ensembles finis, or une union dénombrable d'ensemble finis ne peut pas recouvrir l'intervalle \( I\).

    Nous allons montrer que dans ce cadre, l'intégrale de la fonction indicatrice de la diagonale sur \( I^2\) ne vérifie pas le théorème de Fubini. Étant donné que \( \Borelien(I)\subset\partP(I)\) nous avons
    \begin{equation}
        \Borelien(I^2)\subset\Borelien(I)\otimes\partP(I).
    \end{equation}
    Soit \( \Delta=\{ (x,x)\tq x\in I \}\). La fonction
    \begin{equation}
        \begin{aligned}
            g\colon I^2&\to \eR \\
            (x,y)&\mapsto x-y 
        \end{aligned}
    \end{equation}
    est continue et \( \Delta=g^{-1}(\{ 0 \})\) est donc fermé dans \( I^2\). L'ensemble \( \Delta\) est donc un borélien de \( I^2\) et par conséquent un élément de la tribu \( \Borelien(I)\otimes\partP(I)\). La fonction indicatrice \( \mtu_{\Delta}\) est alors mesurable pour l'espace mesuré
    \begin{equation}
        (I\times I,\Borelien(I)\otimes\partP(I),\lambda\otimes m).
    \end{equation}
    Pour \( x\) fixé nous avons
    \begin{equation}
        \mtu_{\Delta}(x,y)=\begin{cases}
            1    &   \text{si \( y= x\)}\\
            1    &    \text{si \( y\neq x\)}
        \end{cases}=\mtu_{\{ x \}}(y),
    \end{equation}
    et donc
    \begin{subequations}
        \begin{align}
            A_1&=\int_I\left( \int_I\mtu_{\Delta}(x,y)dm(y) \right)d\lambda(x)\\
            &=\int_I\left( \int_I\mtu_{\{ x \}}(y)dm(y) \right)d\lambda(x)\\
            &=\int_I\Big( m(\{ x \}) \Big)d\lambda(x)\\
            &=\int_I 1d\lambda(x)\\
            &=1.
        \end{align}
    \end{subequations}
    Par contre le support de \( \mtu_{\Delta}\) étant de mesure nulle pour la mesure de Lebesgue, nous avons
    \begin{equation}
        \int_I\mtu_{\Delta}(x,y)d\lambda(x)=0
    \end{equation}
    et par conséquent
    \begin{equation}
        A_2=\int_I\left( \int_I\mtu_{\Delta}(x,y)d\lambda(x) \right)dm(y)=0.
    \end{equation}
    Nous voyons donc que le théorème de Fubini ne s'applique pas.
\end{example}

\begin{example} \label{ExrgMIni}
    Le théorème de Fubini est utilisé dans le calcul de l'intégrale gaussienne
    \begin{equation}
        G=\int_{\eR} e^{-x^2}dx,
    \end{equation}
    alors que la fonction \( x\mapsto  e^{-x^2}\) n'a pas de primitives parmi les fonctions élémentaires.

    Par symétrie nous pouvons nous contenter de calculer
    \begin{equation}
        G_+=\int_0^{\infty} e^{-x^2}dx.
    \end{equation}
    L'astuce est de passer par l'intermédiaire
    \begin{subequations}
        \begin{align}
            H&=\int_{\eR^+\times\eR^+} e^{-(x^2+y^2)}dxdy       \label{EqIntFausasub}\\
            &=\int_{\eR^+}\left( \int_{\eR^+} e^{-x^2} e^{-y^2}dx \right)dy\\
            &=\left( \int_{\eR^+} e^{-x^2} dx\right)^2\\
            &=G_+^2
        \end{align}
    \end{subequations}
    L'intégrale \eqref{EqIntFausasub} se calcule en passant aux coordonnées polaires et le résultat est \( H=\frac{ \pi }{ 4 }\). Nous avons alors \( G=\frac{ \sqrt{\pi} }{ 2 }\) et
    \begin{equation}
        \int_{\eR} e^{-x^2}=\sqrt{\pi}.
    \end{equation}
\end{example}

\begin{example}
    Une variante, qui n'applique pas Fubini sur un domaine non borné. Nous commençons par écrire
\begin{equation}
	I=\int_{-\infty}^{+\infty} e^{-x^2} dx := \lim_{R \to +\infty} \int_{-R}^{+R} e^{-x^2} dx 
\end{equation}
et puis nous faisons le calcul
\begin{equation}		\label{EqCalculInteeemoisxcar}
	\begin{aligned}[]
		I^2 &= \lim_{R \to +\infty} \left( (\int_{-R}^{+R} e^{-x^2} dx)( \int_{-R}^{+R} e^{-y^2} dy) \right) \\
		&= \lim_{R \to +\infty} \left( \iint_{K_R}e^{-(x^2+y^2)} dx dy \right) \\
		&= \lim_{R \to +\infty} \left( \iint_{C_R}e^{-(x^2+y^2)} dx dy \right) 
	\end{aligned}
\end{equation}
où $K$ est le carré de demi côté $R$ centré à l'origine et de côtés parallèles aux axes et $C_R$ est le cercle de rayon $R$ centré à l'origine.

	La première étape à justifier est simplement l'application de Fubini. Pour le passage de l'intégrale du carré vers le cercle, définissons
	\begin{equation}
		\begin{aligned}[]
			I_K(r)&=\int_{K_r}f,&I_C(r)&=\int_{C_r}f
		\end{aligned}
	\end{equation}
	où $K_r$ est la carré de demi côté $r$ et $C_r$ est le cercle de rayon $r$. Le demi côté du carré inscrit à $C_r$ est $\sqrt{2}$, donc pour tout $r$ nous avons
	\begin{equation}
		I_K(\sqrt{2}r)\leq I_C(r)<I_K(r),
	\end{equation}
	et en prenant la limite, nous avons évidement
	\begin{equation}
		\lim_{r\to \infty}I_K(\sqrt{2}r)=\lim_{r\to\infty}I_K(r),
	\end{equation}
	de telle façon à ce que cette limite soit également égale à $\lim_{r\to\infty}I_C(t)$.


    Il ne reste qu'à calculer la dernière intégrale sur le cercle en passant aux coordonnées polaires :
	\begin{equation}
        \iint_{C_R} e^{-(x^2+y^2)}dxdy=\int_0^{2\pi}d\theta\int_0^Rr e^{-r^2}dr=\pi(1- e^{-R^2}).
	\end{equation}
	La limite donne $\pi$, nous en déduisons que
    \begin{equation}    \label{EqFDvHTg}
		\int_{-\infty}^{\infty} e^{-x^2}dx=\sqrt{\pi}.
	\end{equation}

\end{example}

\begin{example} \label{ExempInversSumIntFub}   \index{mesure!de comptage}
    Le théorème de Fubini-Tonelli nous permet également d'inverser des sommes et des séries. En effet une somme n'est rien d'autre qu'une intégrale pour la mesure de comptage :
    \begin{equation}
        \sum_{n=0}^{\infty}a_n=\int_{\eN}a_ndm(n).
    \end{equation}
    Considérons une suite de fonctions \( f_n\colon \eR^d\to \eR\) \emph{positives}, la quantité
    \begin{equation}    \label{EqAcalculParFubIntSum}
        I=\sum_{n=0}^{\infty}\int_{\eR^n}f_n(x)dx
    \end{equation}
    et les espaces mesurés \( (\eN,\partP(\eN),m)\), \( (\eR^n,\Borelien(\eR^n),\lambda)\) où \( \lambda\) est la mesure de Lebesgue. En écrivant la formule \eqref{EqAcalculParFubIntSum}, nous supposons que pour chaque \( n\), la fonction \( f_n\) est intégrable sur \( \eR^d\) et que le résultat soit sommable. Nous pouvons la récrire sous la forme
    \begin{equation}
        \int_{\eN}\left( \int_{\eR^n}f(n,x)dx \right)dm(n)
    \end{equation}
    avec la notation évidente \( f(n,x)=f_n(x)\). Prouvons que la fonction \( f\colon \eN\times\eR^d\to \eR\) ainsi définie est une fonction mesurable pour l'espace mesuré
    \begin{equation}
        \big( \eN\times\eR^d,\partP(\eN)\otimes\Borelien(\eR^d),m\otimes\lambda \big).
    \end{equation}
    Si \( A\subset\eR\), nous avons
    \begin{equation}
        f^{-1}(A)=\bigcup_{n\in\eN}\{ n \}\times f_n^{-1}(A).
    \end{equation}
    Chacun des ensembles dans l'union appartient à la tribu \( \partP(\eN)\times\Borelien(\eR^d)\) tandis que les tribus sont stables sous les unions dénombrables. La fonction \( f\) est donc mesurable. La fonction \( f\) est donc mesurable. Comme nous avons supposé que \( f\) était positive, le théorème de Fubini-Tonelli s'applique et nous avons
    \begin{equation}
        I=\int_{\eR^d}\left( \int_{\eN}f(n,x)dm(n) \right)dx=\int_{\eR^d}\sum_{n\in \eN}f_n(x)dx.
    \end{equation}
\end{example}

%---------------------------------------------------------------------------------------------------------------------------
\subsection{Changement de variables dans une intégrale}
%---------------------------------------------------------------------------------------------------------------------------

\begin{theorem} \label{ThomFeRCi}
    Soit \( \mO\) un ouvert de \( \eR^n\) et \( \mO'\) un ouvert de \( \eR^m\). Soit \( \varphi\colon \mO\to \mO'\) un difféomorphisme \( C^1\). Si \( f\colon \mO\to \eR\) est une fonction mesurable, positive et intégrable, alors
    \begin{equation}
        \int_{\mO}f(u)du=\int_{\mO'}f\big( \varphi^{-1}(v) \big)| J_{\varphi^{-1}}(v) |dv.
    \end{equation}
\end{theorem}
