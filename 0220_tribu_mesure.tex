% This is part of Mes notes de mathématique
% Copyright (c) 2011-2014
%   Laurent Claessens
% See the file fdl-1.3.txt for copying conditions.

%+++++++++++++++++++++++++++++++++++++++++++++++++++++++++++++++++++++++++++++++++++++++++++++++++++++++++++++++++++++++++++
\section{Théorie de la mesure}
%+++++++++++++++++++++++++++++++++++++++++++++++++++++++++++++++++++++++++++++++++++++++++++++++++++++++++++++++++++++++++++

%---------------------------------------------------------------------------------------------------------------------------
\subsection{Ensembles mesurables}
%---------------------------------------------------------------------------------------------------------------------------

\begin{definition}[\cite{ProbaDanielLi}]  \label{DefjRsGSy}
    Si \( \Omega\) est un ensemble, un ensemble \( \tribA\) de sous-ensembles de \( \Omega\) est une \defe{tribu}{tribu} si 
    \begin{enumerate}
        \item
            \( \Omega\in\tribA\);
        \item
            \( \complement A\in A\) pour tout \( A\in\tribA\);
        \item
            si \( (A_i)_{i\in \eN}\) une suite dénombrable d'éléments de \( \tribA\), alors \( \bigcup_{i\in \eN}A_i\in\tribA\).
    \end{enumerate}
    Le couple \( (\Omega,\tribA)\) est alors un \defe{espace mesurable}{espace!mesurable}.
\end{definition}
La définition de mesure sur un espace mesurable est la définition \ref{DefBTsgznn}. Nous trouvons parfois la notation
\begin{equation}
    \bigcup_{k\in \eN}A_k=\sup_{k\geq 0}A_k.
\end{equation}

\begin{lemma}   \label{LemBWNlKfA}
    Opérations ensemblistes sur les tribus.
    \begin{enumerate}
        \item
    Une tribu est stable par intersections au plus dénombrables.
\item
    Une tribu est stable par différence ensembliste.
    \end{enumerate}
\end{lemma}

\begin{proof}
    Soit \( (A_i)_{i\in I}\) une famille au plus dénombrable d'éléments de la tribu \( \tribA\). Nous devons prouver que \( \bigcap_{i\in I}A_i\) est également un élément de \( \tribA\). Pour cela nous passons au complémentaire :
    \begin{equation}
        \complement\left( \bigcap_{i\in I}A_i \right)=\bigcup_{i\in I}\complement A_i.
    \end{equation}
    La définition d'une tribu implique que le membre de droite est un élément de la tribu. Par stabilité d'une tribu par complémentaire, l'ensemble \( \bigcap_{i\in I}A_i\) est également un élément de la tribu.

    La seconde assertion est immédiate à partir de la première parce que \( A\setminus B=A\cap \complement B\).
\end{proof}

\begin{definition}
    La tribu des \defe{boréliens}{boréliens}, notée \( \Borelien(\eR^d)\) est la tribu engendrée par les ouverts de \( \eR^d\). Plus généralement si \( Y\) est un espace topologique, la tribu des boréliens est la tribu engendrée par les ouverts de \( Y\).
\end{definition}
Le plus souvent lorsque nous parlerons de fonctions \( f\colon X\to Y\) où \( Y\) est un espace topologique, nous considérons la tribu borélienne sur \( Y\). Ce sera en particulier le cas dans la théorie de l'intégration.

\begin{proposition} \label{LemYEkvbWBz}
    La tribu engendrée par une base dénombrable de la topologie est celle des boréliens.
\end{proposition}

\begin{proof}
    Si une base de topologie est donnée, tout ouvert peut être écrit comme union d'élément de la base, proposition \ref{PropMMKBjgY}. Dans le cas d'une base dénombrable, cette union sera forcément dénombrable. Une tribu étant stable par union dénombrable, tout ouvert est dans la tribu engendrée par la base de topologie. Les autres boréliens suivent automatiquement.
\end{proof}

%--------------------------------------------------------------------------------------------------------------------------- 
\subsection{Les boréliens de \texorpdfstring{$ \eR$}{R}}
%---------------------------------------------------------------------------------------------------------------------------

Nous rappelons que la topologie de \( \eR\) est celle des boules donnée par le théorème \ref{ThoORdLYUu}. Notons que les boules ouvertes de la forme \( B(q,r)\) avec \( q,r\in \eQ\) forment une base dénombrable de la topologique de \( \eR\).

\begin{lemma}   \label{LemZXnAbtl}
    Soit \( \{ q_i \}\) une énumération des rationnels. La tribu engendrée par les ouverts \( \sigma_i=\mathopen] q_i , \infty \mathclose[\) est la tribu des boréliens.
\end{lemma}

\begin{proof}
    Si \( a<b\) dans \( \eQ\) alors \( \sigma_a\setminus\sigma_b=\mathopen] a , b \mathclose]\). Ensuite 
    \begin{equation}
        \bigcup_{n\in \eN^*}\sigma_a\setminus\sigma_{b-\frac{1}{ n }}=\bigcup_{n\in \eN^*}\mathopen] a , b-\frac{1}{ n } \mathclose]=\mathopen] a , b \mathclose[.
    \end{equation}
    Par union dénombrable, tous les intervalles \( \mathopen] a , b \mathclose[\) avec \( a,b\in \eQ\) sont dans la tribu engendrée par les \( \sigma_i\).

        Ces boules ouvertes forment une base de la topologie de \( \eR\) et le lemme \ref{LemYEkvbWBz} conclu.
\end{proof}

\begin{proposition}[\cite{OYRmzAa}]
    Tout ouvert de \( \eR^n\) est une union dénombrable de rectangles presque disjoints\footnote{«presque» au sens où les intersections éventuelles sont de mesure de Lebesgue nulle.}.
\end{proposition}

\begin{proof}
    Soit \( G\) un ouvert de \( \eR^n\). Soit \( \{ Q_i^{1} \}_{i\in \eN}\) un découpage de \( \eR^n\) en cubes de côté \( 1\) et dont les sommets sont en les coordonnées entières. Ce sont des cubes presque disjoints. Nous considérons ensuite pour chaque \( k>1\) le découpage \( \{ Q_i^{(k)} \}_{i\in\eN}\) de \( \eR^n\) en cubes de côtés \( 2^{-k}\) qui consiste à découper en \( 2\) les côtés des cubes du découpage \( Q^{(k-1)}\). Ces cubes forment encore un découpage dénombrable de \( \eR^n\) en des cubes presque disjoints. Ensuite nous considérons \( \mE\) l'union de tous les \( Q_i^{(k)}\) contenus dans \( G\).

    Montrons que \( \mE=G\). D'abord \( \mE\subset G\) parce que \( \mE\) est une union d'ensembles contenus dans \( G\). Ensuite si \( x\in G\), il existe une boule de rayon \( r\) autour de \( x\) contenue dans \( G\); alors un des ensembles \( Q_i^{(k)}\) avec \( 2^{-j}<\frac{ r }{2}\) est contenue dans \( B(x,r)\) et donc dans \( \mE\).

    Bien entendu l'union qui donne \( \mE\) n'est pas satisfaisante par ce que les \( Q_i^{(k+1)}\) sont contenus dans les \( Q_i^{(k)}\); les intersections sont donc loin d'être de mesure nulle.

    Nous faisons ceci : 
    \begin{subequations}
        \begin{align}
            R^{(0)}&=\{ Q_i^{(1)} \text{contenu dans \( G\)} \}\\
            R^{(k+1)}&=\{ Q_i^{(k+1)}\text{contenus dans \( G\) et pas dans \( R^{(k)}\)} \}.
        \end{align}
    \end{subequations}
    En fin de compte l'union de tous les ensembles contenus dans les \( R^{(k)}\) forment encore \( \eR^n\), mais sont d'intersection presque vide.
\end{proof}

%--------------------------------------------------------------------------------------------------------------------------- 
\subsection{Fonction mesurable}
%---------------------------------------------------------------------------------------------------------------------------

\begin{definition}[Fonction mesurable] \label{DefQKjDSeC}
    Soit \( (E,\tribA)\) et \( (F,\tribF)\) deux espaces mesurés. Une fonction \( f\colon E\to F\) est \defe{mesurable}{mesurable!fonction} si pour tout \( \mO\in \tribF\), l'ensemble \( f^{-1}(\mO)\) est dans \( \tribA\).

    Une fonction à valeurs dans \( \eR^d\) est \defe{borélienne}{borélienne!fonction}\index{fonction!borélienne} si elle est mesurable pour la tribu des boréliens sur \( \eR^d\). Plus explicitement, \( f\colon (\Omega,\tribA)\to (\eR^d,\Borelien(\eR^d))\) est borélienne si pour tout \( \mO\in\Borelien\) nous avons \( f^{-1}(\mO)\in\tribA\).
\end{definition}
Si \( \tribA\) est une tribu sur un ensemble \( E\), nous notons \( m(\tribA)\)\nomenclature[P]{\( m(\tribA)\)}{Ensemble des fonctions \( \tribA\)-mesurables} l'ensemble des fonctions qui sont \( \tribA\)-mesurables.

\begin{remark}
    Lorsque nous considérons des fonctions à valeurs réelles \( f\colon X\to \eR\) nous utiliserons toujours la tribu borélienne sur \( \eR\). Pour \( X\), cela peut dépendre des contextes. En théorie de l'intégration, nous mettrons sur \( X\) la tribu des ensembles mesurables au sens de Lebesgue sur \( X\), \emph{tout en gardant celle des boréliens sur l'ensemble d'arrivée}.

    Pour toute la partie sur l'intégration, une fonction \( f\colon \eR^n\to \eR^m\) sera mesurable si pour tout borélien \( A\) de \( \eR^m\) l'ensemble \( f^{-1}(A)\) est Lebesgue-mesurable dans \( \eR^n\).

    Étant donné qu'il est franchement difficile de créer des ensembles non mesurables au sens de Lebesgue, il est franchement difficile de créer des fonction non mesurables à valeurs réelles. L'hypothèse de mesurabilité est donc toujours satisfaite dans les cas pratiques.
\end{remark}

\begin{lemma}[\cite{NBoIEXO}]   \label{LemFOlheqw}
    Une fonction \( f\colon X\to \eR\) est mesurable si et seulement si \( f^{-1}(I)\) est mesurable pour tout \( I\) de la forme \( \mathopen] a , \infty \mathclose[\).
\end{lemma}

\begin{proof}
    Nous devons prouver que \( f^{-1}(A)\) est mesurable dans \( X\) pour tout borélien \( A\) de \( \eR\). Nous posons
    \begin{equation}
        S=\{ A\subset \eR\tq f^{-1}(A)\text{ est mesurable dans \( X\)} \}
    \end{equation}
    et nous prouvons que cela est une tribu. D'abord \( f^{-1}(\eR)=X\), et \( X\) est mesurable, donc \( \eR\in S\). Ensuite si \( A\in S\) alors \( f^{-1}(A^c)=f^{-1}(A)^c\). En tant que complémentaire d'un mesurable de \( X\), l'ensemble \( f^{-1}(A)^c\) est mesurable dans \( X\). Et enfin si \( A_n\in S \) alors \( f^{-1}(\bigcup_nA_n)=\bigcup_nf^{-1}(A_n)\) qui est encore mesurable dans \( X\) en tant qu'union de mesurables.

    Donc \( S\) est une tribu qui contient tous les ensembles de la forme \( \mathopen] a , \infty \mathclose]\). Le lemme \ref{LemZXnAbtl} conclu que \( S\) contient tous les boréliens de \( \eR\).
\end{proof}

\begin{lemma}[\cite{NBoIEXO}]   \label{LemIGKvbNR}
    Soit \( f_n\colon X\to \eR\) une suite de fonctions mesurables\footnote{Ici \( X\) est un espace mesuré et \( \eR\) est muni des boréliens.}. Alors \( \sup_n f_n\) est mesurable.
\end{lemma}

\begin{proof}
    Nous avons
    \begin{subequations}
        \begin{align}
            (\sup f_n)^{-1}\big( \mathopen] a , \infty \mathclose] \big)&=\{ x\in X\tq (\sup f_n)(x)>a \}\\
            &=\bigcup_n\{ x\in X\tq f_n(x)>a \}\\
            &=\bigcup_nf_n^{-1}\big( \mathopen] a , \infty \mathclose] \big).
        \end{align}
    \end{subequations}
    Étant donné que \( f_n\) est mesurable et que \( \mathopen] a , \infty \mathclose]\) est mesurable, chacun des \( f_n^{-1}\big( \mathopen] a , \infty \mathclose] \big) \) est mesurable dans \( X\). Nous sommes en présence d'une union dénombrable de mesurables, donc \( (\sup f_n)^{-1}\big( \mathopen] a , \infty \mathclose] \big)\) est mesurable.

    Le lemme \ref{LemFOlheqw} conclu que \( \sup f_n\) est mesurable.
\end{proof}

\begin{proposition}\label{PropFYPEOIJ}
    Si \( f_n\) est une suite de fonctions mesurables et positives, alors la fonction \( \sum_nf_n\) est mesurable.
\end{proposition}

\begin{proof}
    Nous considérons les fonctions \( s_k(x)=\sum_{n=0}^kf_n(x)\) qui vaut éventuellement \( \infty\) en certains points. Nous avons
    \begin{equation}
        \sum_nf_n(x)=\sup_ks_k(x),
    \end{equation}
    donc le lemme \ref{LemIGKvbNR} nous donne la mesurabilité de la somme de \( f_n\).
\end{proof}

%--------------------------------------------------------------------------------------------------------------------------- 
\subsection{Tribu produit}
%---------------------------------------------------------------------------------------------------------------------------

\begin{definition}      \label{DefTribProfGfYTuR}
    Si \( \tribA_1\) et \( \tribA_2\) sont deux tribus sur deux ensembles \( \Omega_1\) et \( \Omega_2\), nous définissons la \defe{tribu produit}{tribu!produit} \( \tribA_1\otimes\tribA_2\) comme étant la tribu engendrée par 
    \begin{equation}
        \{ X\times Y\tq X\in\tribA_1,Y\in\tribA_2 \}.
    \end{equation}
    Ces ensembles sont appelés \defe{rectangles}{rectangle!produit de tribus} de \( (\Omega_1,\tribA_1)\otimes (\Omega_2,\tribA_2)\).
\end{definition}

\begin{lemma}[Propriété des sections\cite{NBoIEXO}] \label{LemAQmWEmN}
    Soient \( \tribA_1\) et \( \tribA_2\) des tribus sur les ensembles \( \Omega_1\) et \( \Omega_2\). Si \( A\in\tribA_1\otimes\tribA_2\) alors pour tout \( x\in \Omega_1\) et \( y\in\Omega_2\), les ensembles
    \begin{subequations}    \label{subEqCTtPccK}
        \begin{align}
            A_1(y)=\{ x\in\Omega_1\tq (x,y)\in A \}\\
            A_2(x)=\{ y\in\Omega_2\tq (x,y)\in A \}
        \end{align}
    \end{subequations}
    sont mesurables.
\end{lemma}
\index{section!propriété des}

\begin{proof}
    Soit \( y\in\Omega_2\); nous allons prouver le résultat pour \( A_1(y)\). Pour cela nous notons 
    \begin{equation}
        S=\{ A\in \tribA_1\otimes\tribA_2\tq \forall y\in\Omega_2, A_1(y)\in\tribA_1 \},
    \end{equation}
    et nous allons noter que \( S\) est une tribu contenant les rectangles. Par conséquent, \( S\) sera égal à \( \tribA_1\otimes \tribA_2\).

    \begin{subproof}
        \item[Les rectangles]

            Considérons le rectangle \( A=X\times Y\) et si \( y\in \Omega_2\) alors
            \begin{equation}
                A_1(y)=\{ x\in \Omega_1\tq (x,y)\in X\times Y \}.  
            \end{equation}
            Donc soit \( y\in Y\) alors \( A_1(y)=X\in\tribA_1\), soit \( y\notin Y\) et alors \( A_1(y)=\emptyset\in\tribA_1\).

        \item[Tribu : ensemble complet]

            Nous avons \( \Omega_1\times \Omega_2\in S\) parce que c'est un rectangle.

        \item[Tribu : complémentaire] Soit \( A\in S\) et montrons que \( A^c\in S\). Nous avons d'abord
            \begin{equation}
                (A^c)_1(y)=\{ x\in \Omega_1\tq (x,y)\in A^c \}.
            \end{equation}
            D'autre part
            \begin{equation}
                A_1(y)^c=\{ x\in\Omega_1\tq (x,y)\notin A \}=\{ x\in \Omega_1\tq (x,y)\in A^c \}=(A^c)_1(y).
            \end{equation}
            Vu que \( \tribA_1\) est une tribu et que par hypothèse \( A_1(y)\in\tribA_1\), nous avons aussi \( A_1(y)^c\in S\), et donc \( (A^c)_1(y)\in \tribA_1\), ce qui prouve que \( A^c\in S\).

        \item[Tribu : union dénombrable] Soit une suite \( A_n\in S\). Nous avons
            \begin{equation}
                (\bigcup_nA_n)_1(y)=\{ x\in\Omega_1\tq (x,y)\in \bigcup_nA_n \}=\bigcup_n\{ x\in\Omega_1\tq (x,y)\in A_n \}=\bigcup_n(A_n)_1(y),
            \end{equation}
            et ce dernier ensemble est dans \( \tribA_1\) parce que c'est une union dénombrable d'éléments de \( \tribA_1\).
        
    \end{subproof}
    Nous avons donc prouvé que \( S\) est une tribu contenant les rectangles, donc \( S\) contient au moins \( \tribA_1\otimes \tribA_2\).
\end{proof}

\begin{corollary}
    Si \( f\colon \Omega_1\times \Omega_2\to \eR\) est une fonction mesurable\footnote{Définition \ref{DefQKjDSeC}.} sur \( X\times Y\) alors pour chaque \( y\) dans \( \Omega_2\), la fonction
    \begin{equation}
        \begin{aligned}
            f_y\colon X&\to \eR \\
            x&\mapsto f(x,y) 
        \end{aligned}
    \end{equation}
    est mesurable.
\end{corollary}

\begin{proof}
    Soit \( \mO\) un ensemble mesurable de \( \eR\) (i.e. un borélien), et \( y\in \Omega_2\). Nous avons
    \begin{equation}
        f_y^{-1}(\mO)=\{ x\in X\tq f(x,y)\in \mO \}=A_1(y)
    \end{equation}
    où
    \begin{equation}
        A=\{ (x,y)\in \Omega_1\times \Omega_2\tq f(x,y)\in \mO \}=f^{-1}(\mO).
    \end{equation}
    Ce dernier est mesurable parce que \( f\) l'est.
\end{proof}

%--------------------------------------------------------------------------------------------------------------------------- 
\subsection{Mesure}
%---------------------------------------------------------------------------------------------------------------------------

\begin{definition}[Mesure]  \label{DefBTsgznn}
    Une \defe{\wikipedia{en}{Measure_space}{mesure}}{mesure} sur l'espace mesurable \( (\Omega,\tribA)\) est une application \( \mu\colon \tribA\to \eR\cup\{ \infty \}\) telle que
    \begin{enumerate}
        \item
            \( \mu(A)\geq 0\) pour tout \( A\in\tribA\);
        \item
            \( \mu(\emptyset)=0\);
        \item   \label{ItemQFjtOjXiii}
            \( \mu\left( \bigcup_{i=0}^{\infty}A_i\right)=\sum_{i=0}^{\infty}\mu(A_i)\) si les \( A_i\) sont des éléments de \( \tribA\) deux à deux disjoints.
    \end{enumerate}
    Une mesure est \defe{\( \sigma\)-finie}{mesure!$\sigma$-finie} si il existe un recouvrement dénombrable de \( \Omega\) par des ensembles de mesure finie. Si la mesure est $\sigma$-finie, nous disons que l'espace \( (\Omega,\tribA,\mu)\) est un espace mesuré $\sigma$-fini.

    La mesure \( \mu\) sur \( \Omega\) est \defe{finie}{mesure!finie} si \( \mu(\Omega)<\infty\).
\end{definition}

\begin{definition}[Ensemble mesurable]\label{DefHGsQxHB}
    Les éléments de \( \tribA\) sont les ensembles \defe{mesurables}{mesurable!ensemble} pour la mesure \( \mu\).
\end{definition}

Si la mesure des \( \sigma\)-finie, nous pouvons choisir le recouvrement croissant pour l'inclusion. En effet si \( (E_n)_{n\in \eN}\) est le recouvrement, il suffit de considérer \( F_n=\bigcup_{k\leq n}E_k\). Ces ensembles \( F_n\) forment tout autant un recouvrement dénombrable, mais il est évidemment croissant.

Le lemme suivant complète la propriété \ref{DefBTsgznn}\ref{ItemQFjtOjXiii} lorsque les ensembles ne sont pas disjoints.
\begin{lemma}\label{LemKKNtvee} \label{LemPMprYuC}
    Si \( A\subset B\) sont deux ensembles \( \mu\)-mesurables de mesure finie alors
    \begin{equation}
        \mu(B\setminus A)=\mu(B)-\mu(A)
    \end{equation}
    et en particulier
    \begin{equation}
        \mu(B)\geq \mu(A).
    \end{equation}

    Si \( (M_n)\) est une suite d'éléments de \( \tribA\) pas spécialement disjoints, alors
    \begin{equation}\label{EqWWFooYPCTt}
        \mu\big( \bigcup_kM_k \big)\leq \sum_{k}\mu(M_k).
    \end{equation}
\end{lemma}

\begin{proof}
    Vu que les ensembles \( B\setminus A\) et \( A\) sont disjoints par la propriété \ref{ItemQFjtOjXiii} de la définition de mesure nous avons
    \begin{equation}
        \mu\big( (B\setminus A)\cup A \big)=\mu(B\setminus A)+\mu(A)
    \end{equation}
    et donc
    \begin{equation}
        \mu(B)=\mu(B\setminus A)+\mu(A)
    \end{equation}
    comme demandé.

    Pour la seconde partie nous considérons la suite disjointe
    \begin{subequations}
        \begin{numcases}{}
            M'_0=\emptyset\\
            M'_k=M_k\setminus M'_{k-1}.
        \end{numcases}
    \end{subequations}
    Nous avons \( \bigcup_kM'_k=\bigcup_kM_k\). Le calcul suivant est alors immédiat :
    \begin{equation}
        \mu\big( \bigcup_kM_k \big)=\mu\big( \bigcup_kM'_k \big)=\sum_{k}\mu(M'_k)=\sum_k\mu(M_k\setminus M'_{k-1})\leq \sum_k\mu(M_k).
    \end{equation}
\end{proof}

\begin{lemma}\label{LemAZGByEs}
    Si \( (A_k)\) est une suite croissante d'ensembles \( \mu\)-mesurables, alors
    \begin{equation}
        \lim_{n\to \infty} \mu(A_k)=\mu(\bigcup_kA_k).
    \end{equation}
\end{lemma}

\begin{proof}
    Nous faisons le coup de l'union télescopique, en posant \( A_0=\emptyset\) :
    \begin{equation}
        \bigcup_{k=1}^{\infty}A_k=\bigcup_{k=1}^{\infty}(A_k\setminus A_{k-1}).
    \end{equation}
    Les ensembles \( A_k\setminus A_{k-1}\) sont deux à deux disjoints, donc la propriété \ref{ItemQFjtOjXiii} de la définition d'une mesure donne
    \begin{subequations}
        \begin{align}
            \mu(\bigcup_{k=1}^{\infty}A_k)&=\mu\left( \bigcup_{k=1}^{\infty}(A_k\setminus A_{k-1}) \right)\\
            &=\sum_{k=1}^{\infty}\mu(A_k\setminus A_{k-1})\\
            &=   \sum_{k=1}^{\infty}\big( \mu(A_k)-\mu(A_{k-1}) \big)    \label{subEqMDRRorb}\\
            &=\lim_{k\to \infty} \mu(A_k)-\mu(A_0)\\
            &=\lim_{k\to \infty} \mu(A_k).
        \end{align}
    \end{subequations}
    où pour obtenir \ref{subEqMDRRorb}, nous avons utilisé le lemme \ref{LemPMprYuC}.
\end{proof}


\begin{example}
    La mesure de comptage \( m\) sur \( \eN\) est \( \sigma\)-finie parce que \( E_n=\{ 0,\ldots, n \}\) est de mesure finie et \( \bigcup_{n\in \eN}E_n=\eN\).
\end{example}

\begin{example}
    La mesure de Lebesgue sur \( \eR^n\) est \( \sigma\)-finie parce que les boules de rayon \( n\) forment un ensemble dénombrable d'ensembles de mesures finies dont l'union est évidemment tout \( \eR^n\).

    L'intervalle \( I=\mathopen[ 0 , 1 \mathclose]\) muni de la tribu de toutes ses parties et de la mesure de comptage n'est pas un espace mesuré \( \sigma\)-fini.
\end{example}

\begin{example}
    L'intégration à la Riemann n'est pas dans la théorie des espaces mesurés. En effet l'ensemble 
    \begin{equation}
        \tribA=\{   A\subset\mathopen[ 0 , 1 \mathclose]\tq  \text{\( \mtu_A\) est intégrable au sens de Riemann}   \}
    \end{equation}
    n'est pas une tribu. Par exemple les singletons en font partie tandis que \( \mathopen[ 0 , 1 \mathclose]\cap \eQ\) n'en fait pas partie alors que c'est une union dénombrable de singletons.
\end{example}

\begin{definition}
    Si \( \mu\) est une mesure nous disons qu'une propriété est vraie \( \mu\)-\defe{presque partout}{presque partout} si elle est fausse seulement sur un ensemble de mesure nulle.
\end{definition}

Par exemple la fonction de Dirichlet est presque partout égale à la fonction \( 1\) (pour la mesure de Lebesgue).


\begin{definition}
    Une application entre espace mesurés
    \begin{equation}
        f\colon (\Omega,\tribA)\to (\Omega',\tribA')
    \end{equation}
    est \defe{mesurable}{mesurable!application} si pour tout \( B\in\tribA'\), l'ensemble \( f^{-1}(B)\) est dans \( \tribA\).
\end{definition}

Si \( \mu\) est une mesure sur \( \eR^d\), une fonction \( f\colon \eR^d\to \eR\) est une \defe{densité}{densité d'une mesure} si pour tout \( A\subset\eR^d\) nous avons
\begin{equation}
    \mu(A)=\int_Af(x)dx
\end{equation}
où \( dx\) est la mesure de Lebesgue.

%--------------------------------------------------------------------------------------------------------------------------- 
\subsection{Généralités}
%---------------------------------------------------------------------------------------------------------------------------

\begin{lemma}   \label{LemIDITgAy}
    Une union dénombrable d'ensemble de mesure nulle est de mesure nulle.
\end{lemma}

\begin{proof}
    C'est une conséquence immédiate du point \ref{ItemQFjtOjXiii} de la définition d'une mesure : si les \( A_i\) sont de mesure nulle,
    \begin{equation}
        \mu\left( \bigcup_{i=1}^{\infty}A_i \right)\leq \mu(A_i)=0
    \end{equation}
\end{proof}

\begin{definition}
    Si \( (A_n)\) est une suite croissante d'ensembles alors la \defe{limite}{limite!d'ensembles} est
    \begin{equation}
        \lim_nA_n=\bigcup_{i=0}^{\infty}A_i.
    \end{equation}
    Si la suite est décroissante alors la limite est
    \begin{equation}
        \lim_nA_n=\bigcap_{i=0}^{\infty}A_i.
    \end{equation}
\end{definition}
\ifthenelse{\value{isAgreg}=0}{Pour une suite ni croissante ni décroissante d'ensembles, il y a la notion de limite inductive\footnote{\emph{direct limit} en anglais.} qui sera un peu traitée à la section \ref{SecDirectLimit}.}{}

\begin{proposition}[\cite{RArwFWJ}] \label{PropAFNPSsm}
    Soit \( \mu\) une mesure sur \( \Omega\) et \( (S_n)\) une suite croissante d'ensembles \( \mu\)-mesurables de \( \Omega\). Nous notons
    \begin{equation}
        S=\lim_nS_n.
    \end{equation}
    Alors pour tout ensemble mesurable\footnote{Définition \ref{DefHGsQxHB}} \( A\subset\Omega\) nous avons
    \begin{equation}
        \mu(A\cap S)=\lim_{n\to \infty} \mu(A\cap S_n).
    \end{equation}
\end{proposition}
Note : dans la référence le résultat fonctionne pour tout ensemble \( A\) (et non seulement les mesurables) parce que la définition de la mesurabilité est un peu différente.

\begin{proof}
    L'inégalité \( \lim\mu(A\cap S_n)\leq \mu(A\cap S)\) est simple à prouver. En effet pour tout \( n\) nous avons \( A\cap S_n\subset A\cap S\) et donc par le lemme \ref{LemKKNtvee} nous avons
    \begin{equation}
        \mu(A\cap S_n)\leq\mu(A\cap S).
    \end{equation}
    En passant à la limite (qui respecte les inégalités) nous avons l'inégalité.

    Nous passons à l'inégalité dans l'autre sens. D'abord si \( \mu(A\cap S_n)=\infty\) pour un certain \( n\), alors il cela vaut encore \( \infty\) pour tous les \( n\) suivants et la limite est \( \infty\) sans problèmes. Donc nous supposons que \( \mu(A\cap S_n)<\infty\) pour tout \( n\in \eN\). De plus, quitte à renommer les indices, nous pouvons supposer que \( S_0=\emptyset\).

    Un élément \( x\) est dans \( S\) si et seulement si il existe \( n\geq 0\) tel que \( x\in S_{n+1}\). En prenant le plus petit de ces \( n\) nous avons \( x\neq S_n\) (éventuellement \( n=0\)) et donc
    \begin{equation}
        S=\bigcup_{n=0}^{\infty}\big( S_{n+1}\setminus S_n \big).
    \end{equation}
    Par conséquent
    \begin{equation}
            A\cap S=A\cap\bigcup_{n=0}^{\infty}(S_{n+1}\setminus S_n)
            =\bigcup_{n=0}^{\infty}A\cap(S_{n+1}\setminus S_n)
    \end{equation}
    Étant donné que les ensembles \( A\cap(S_{n+1}\setminus S_n)\) sont disjoints,
    \begin{subequations}
        \begin{align}
            \mu(A\cap S)&=\sum_{n=0}^{\infty}\mu\big( A\cap(S_{n+1}\setminus S_n) \big)\\
            &=\sum_{n=0}^{\infty}\mu\Big( (A\cap S_{n+1})\setminus (A\cap S_n) \Big)\\
            &=\sum_{n=0}^{\infty}\big[ \mu(A\cap S_{n+1})-\mu(A\cap S_n) \big]\\
            &=\lim_{n\to \infty} \mu(A\cap S_{n+1})-\underbrace{\mu(A\cap S_0)}_{=0}\label{subeqLTvmTjO}\\
            &=\lim_{n\to \infty} \mu(A\cap S_n).
        \end{align}
    \end{subequations}
    Dans ce calcul nous avons utilisé plusieurs fois le fait que les \( S_n\) et \( A\) étaient mesurables (et la propriété de tribu qui dit que \( A\cap S_n\) est également mesurable) ainsi que le lemme \ref{LemKKNtvee}. Nous avons aussi utilisé la série télescopique dans \( \eR\) pour obtenir \eqref{subeqLTvmTjO}.
\end{proof}

\begin{definition}[\cite{PVWUyDH}]
    Soit \( E\) un ensemble. Une partie \( \tribD\) de \( E\) est un \defe{\( \lambda\)-système}{$\lambda$-système} si
    \begin{enumerate}
        \item
            pour tout \( A,B\in\tribD\) avec \( A\subset B\) implique \( B\setminus A\in \tribD\),
        \item
            si \( (A_k)_{k\geq 1}\) est une suite croissante d'éléments de \( \tribD\) alors \( \bigcup_kA_k\in\tribD\).
    \end{enumerate}
\end{definition}
Note : une tribu est un \( \lambda\)-système.

\begin{lemma}[\cite{PVWUyDH}]
    Une intersection quelconque de \( \lambda\)-systèmes dans \( E\) est un \( \lambda\)-système dans \( E\).
\end{lemma}

\begin{proof}
    Soient \( \{ \tribD_l \}_{l\in L}\) des \( \lambda\)-systèmes indicés par un ensemble \( L\). Si \( A,B\in\bigcap_{l\in L}\tribD_l\) alors \( B\setminus A\in\tribD_l\) pour tout \( l\in L\) et donc \( A\setminus B\in\bigcap_{l\in L}\tribD_l\). De la même façon si \( (A_k)\) est une suite croissante dans \( \bigcap_{l\in L}\tribD_l\) alors pour tout \( l\in L\) nous avons \( \bigcup_kA_k\in\tribD_l\). Donc \( \bigcup_kA_k\in\bigcap_l\tribD_l\).
\end{proof}
Ce lemme est ce qui permet de définir le \( \lambda\)-système \defe{engendré}{engendré!$\lambda$-système} par une classe \( \tribA\) de parties de \( E\) : c'est l'intersection de tous les \( \lambda\)-systèmes de \( E\) contenant \( \tribA\).

\begin{lemma}[\cite{PVWUyDH}]   \label{LemLUmopaZ}
    Soit \( \tribC\) une classe de parties de \( E\) (contenant \( E\) lui-même) qui soit stable par intersection finie. Alors le \( \lambda\)-système engendré par \( \tribC\) coïncide avec la tribu engendrée par \( \tribC\).
\end{lemma}

\begin{proof}
    Nous notons \( \tribE\) le \( \lambda\)-système engendré par \( \tribC\) et \( \tribF\) la tribu engendrée par \( \tribC\). Étant donné que \( \tribF\) est un \( \lambda\)-système nous avons \( \tribE\subset\tribF\). Pour montrer l'inclusion inverse nous allons prouver que \( \tribE\) est une tribu.

    D'abord pour \( C\in\tribC\) nous posons
    \begin{equation}
        \mG_C=\{ A\subset \tribE\tq A\cap C\in\tribE \}.
    \end{equation}
    et pour \( F\in\tribE\),
    \begin{equation}
        \mH_F=\{ A\in\tribE\tq A\cap F\in\tribE \}.
    \end{equation}
    Nous allons montrer que \( \mG_C\) et \( \mH_F\) sont des \( \lambda\)-systèmes et que \( \mG_C=\mH_F=\tribE\).
    
    Nous commençons par \( \mG_C\). Si \( A,B\in\mG_C\) avec \( A\subset B\) alors
    \begin{equation}
        (B\setminus A)\cap C=\underbrace{(B\cap C)}_{\in\tribE}\setminus\underbrace{(A\cap C)}_{\in\tribE}.
    \end{equation}
    Vu que \( \tribE\) est un \( \lambda\)-système et que \( (A\cap C)\subset(B\cap C)\) nous avons bien \( (B\setminus A)\cap C\in\tribE\) et donc \( B\setminus A\in\mG_C\). Soit maintenant \( (A_k)\) une suite croissante dans \( \mG_C\). Nous avons
    \begin{equation}
        \big( \bigcup_{k=1}^{\infty}A_k \big)\cap C=\bigcup_{k=1}^{\infty}(A_k\cap C)
    \end{equation}
    qui est une union d'une suite croissante d'éléments de \( \tribE\). Donc \( \bigcup_{k=1}^{\infty}(A_k\cap C)\in\tribE\), ce qui signifie que \( \bigcup_{k=1}^{\infty}A_k\in\mG_C\). Cela termine la preuve du fait que \( \mG_C\) soit une \( \lambda\)-système. 

    Étant donné que \( \tribC\) est stable par intersection finie, si \( K\in\tribC\) nous avons \( C\cap K\in\tribC\), ce qui signifie que \( K\in\mG_C\). Nous avons donc \( \tribC\subset\mG_C\). Donc \( \mG_C\) est un \( \lambda\)-système vérifiant \( \tribC\subset\mG_C\subset\tribE\). Mais comme \( \tribE\) est le plus petit \( \lambda\)-système contenant \( \tribC\) nous avons en fait \( \mG_C=\tribE\).

    Nous montrons à présent que \( \mH_F\) est un \( \lambda\)-système. Si \( A,B\in\mH_F\) avec \( A\subset B\) alors \( (B\setminus A)\cap F=(B\cap F)\setminus(A\cap F)\). Vu que \( \tribE\) est une \( \lambda\)-système et que \( A\cap F\) et \( B\cap F\) sont dans \( \tribE\) avec \( A\cap F\subset B\cap F\), nous avons
    \begin{equation}
        (B\cap F)\setminus(A\cap F)\in\mH_F.
    \end{equation}
    Soit maintenant \( (A_k)_{k\geq 1}\) une suite croissante dans \( \mH_F\). Pour tout \( k\) nous avons \( A_k\cap F\in\tribE\), ce qui donne
    \begin{equation}
        \big( \bigcap_{k=1}^{\infty}A_k \big)\cap F=\bigcap_{k=1}^{\infty}(A_k\cap F)\in\tribE.
    \end{equation}
    Donc \( \mH_F\) est un \( \lambda\)-système vérifiant \( \tribC\subset\mH_F\subset\tribE\). Nous en concluons que pour tout \( C\in\tribC\) et pour tout \( F\in\tribE\),
    \begin{equation}
        \mG_C=\mH_F=\tribE.
    \end{equation}
    
    Nous allons maintenant prouver que \( \tribE\) est une tribu\footnote{Définition \ref{DefjRsGSy}.}.
    \begin{enumerate}
        \item
            Si \( F\in\tribE\) alors \( E\cap F=F\in\tribE\), ce qui signifie que \( E\in\mH_F=\tribE\).
        \item
            Si \( A\in \tribE\) alors \( E\setminus A\in\tribE\) parce que \( \tribE\) est un \( \lambda\)-système et \( E\in\tribE\). Donc \( \complement A\in\tribE\).
        \item
            Montrons que \( \tribE\) est stable par union finie en considérant \( A,B\in\tribE\). Vu que \( E\) est également un élément de \( \tribE\) nous avons
            \begin{equation}
                E\setminus(A\cup B)=(E\setminus A)\cap(E\setminus B)\in\tribE.
            \end{equation}
            Cela prouve que \( \complement( A\cup B)\in \tribE\). Par complémentarité nous avons aussi \( A\cup B\in\tribE\).
            
            Soient \( A_k\in\tribE\), et nommons \( B_p=A_1\cup\ldots\cup A_p\). Les ensembles \( B_p\) forment une suite croissante d'éléments de \( \tribE\). L'union est donc dans \( \tribE\) et ce dernier est au final stable par union dénombrable.
    \end{enumerate}
    
    Maintenant que \( \tribE\) est une tribu nous avons \( \tribF\subset\tribE\) parce que \( \tribF\) est la plus petite tribu contenant \( \tribC\). Nous en déduisons que \( \tribE=\tribF\), ce qu'il fallait démontrer.
\end{proof}

\begin{theorem}[\cite{PVWUyDH}] \label{ThoJDYlsXu}
    Soient \( \mu\) et \( \nu\), deux mesures sur \( (E,\tribA)\) et une classe \( \tribE\) de parties de \( E\) telles que
    \begin{enumerate}
        \item
            La tribu engendrée par \( \tribE\) soit \( \tribA\).
        \item
            pour tout \( A\in \tribE\), \( \mu(A)=\nu(A)\)
        \item
            si \( A,B\in\tribE\) alors \( A\cap B\in\tribE\)
        \item
            il existe une suite croissante \( (E_n)\) dans \( \tribE\) telle que \( E=\lim E_n\).
    \end{enumerate}
    Alors les mesures \( \mu\) et \( \nu\) coïncident sur \( \tribA\) en entier.
\end{theorem}

\begin{proof}
    Soit \( (E_n)\) la suite des hypothèses; nous considérons \( \mu_n\) et \( \nu_n\), les restrictions de \( \mu\) et \( \nu\) à \( E_n\), c'est à dire
    \begin{subequations}
        \begin{align}
        \mu_n(A)=\mu(A\cap E_n)\\
        \nu_n(A)=\nu(A\cap E_n).
        \end{align}
    \end{subequations}
    Vu que les \( E_n\) sont dans \( \tribE\subset\tribA\) ils sont mesurables au sens de \( \mu\) et \( \nu\). Par la proposition \ref{PropAFNPSsm}, pour tout \( A\in \tribE\) nous avons alors
    \begin{subequations}
        \begin{align}
            \lim_{n\to \infty} \mu_n(A)=\mu(A)\\
            \lim_{n\to \infty} \nu_n(A)=\nu(A)
        \end{align}
    \end{subequations}
    Nous devons donc seulement montrer que pour tout \( A\in\tribA\) et pour tout \( n\in\eN\), \( \mu_n(A)=\nu_n(A)\). Pour cela nous nous fixons un \( n\) et nous considérons la classe
    \begin{equation}
        \tribD=\{ A\in\tribA\tq\mu_n(A)=\nu_n(A) \}.
    \end{equation}
    Le but sera de prouver que \( \tribD=\tribA\).
    
    
    Par hypothèse \( A\cap E_n\in\tribE\) et donc
    \begin{equation}
        \mu(A\cap E_n)=\nu(A\cap E_n)<\infty,
    \end{equation}
    c'est à dire que \( \mu_n=\nu_n\) sur \( \tribE\). Par ailleurs, \( E\cap E_n=E_n\in\tribE\), donc
    \begin{equation}
        \mu_n(E)=\nu_n(E)<\infty.
    \end{equation}
    Par conséquent \( \mu_n=\nu_n\) sur la classe \( \tribE'=\tribE\cup\{ E \}\) : \( \tribE'\subset\tribD\).

    Montrons que \( \tribD\) est un \( \lambda\)-système. Soient \( A,B\in\tribD\) avec \( A\subset B\). Alors, étant donné que les mesures \( \mu_n\) et \( \nu_n\) sont finies, le lemme \ref{LemPMprYuC} nous donne
    \begin{subequations}
        \begin{align}
            \mu_n(B\setminus A)=\mu_n(B)-\mu_n(A)\\
            \nu_n(B\setminus A)=\nu_n(B)-\nu_n(A).
        \end{align}
    \end{subequations}
    Donc \( \mu_n(B\setminus A)=\nu_n(B\setminus A)\) et \( B\setminus A\in\tribD\).

    Soit par ailleurs une suite croissante \( (A_k)_{k\geq 1}\) d'éléments de \( \tribD\). En posant \( B_p=\bigcup_{k=1}^pA_k\), le lemme \ref{LemAZGByEs} nous donne
    \begin{equation}
        \mu_n(\bigcup_{k=1}^{\infty}A_k)=\lim_{p\to \infty} \mu_n(A_p).
    \end{equation}
    Mais vu que pour chaque \( p\) nous avons \( \mu_n(A_p)=\nu_n(A_p)\), nous avons aussi
    \begin{equation}
        \mu_n(\bigcup_{p=1}^{\infty}A_p)=\nu_n(\bigcup_{p=1}^{\infty}A_p).
    \end{equation}
    Donc \( \tribD\) est bel et bien un \( \lambda\)-système contenant \( \tribE'\). Par le lemme \ref{LemLUmopaZ}, le \( \lambda\)-système engendré par \( \tribE'\) est égal à la tribu engendrée par \( \tribE'\), mais par hypothèse la tribu engendrée par \( \tribE\) est \( \tribA\), donc le \( \lambda\)-système engendré par \( \tribE'\) est \( \tribA\). Vu que \( \tribD\) est une \( \lambda\)-système contenant \( \tribE'\), nous avons alors \( \tribA\subset\tribD\) et donc \( \tribA=\tribD\), ce qu'il fallait.
\end{proof}

%---------------------------------------------------------------------------------------------------------------------------
\subsection{Théorème de récurrence}
%---------------------------------------------------------------------------------------------------------------------------

Soit \( X\) un espace mesurable, \( \mu\) une mesure finie sur \( X\) et \( \phi\colon X\to X\) une application mesurable préservant la mesure, c'est à dire que pour tout ensemble mesurable \( A\subset X\),
\begin{equation}
    \mu\big( \phi^{-1}(A) \big)=\mu(A).
\end{equation}
Si \( A\subset X\) est un ensemble mesurable, un point \( x\in A\) est dit \defe{récurrent}{récurrent!point d'un système dynamique} par rapport à \( A\) si et seulement si pour tout \( p\in \eN\), il existe \( k\geq p\) tel que \( \phi^k(x)\in A\).

\begin{theorem}[\wikipedia{fr}{Théorème_de_récurrence_de_Poincaré}{Théorème de récurrence de Poincaré}.]     \label{ThoYnLNEL}
    Si \( A\) est mesurable dans \( X\), alors presque tous les points de \( A\) sont récurrents par rapport à \( A\).
\end{theorem}

\begin{proof}
    Soit \( p\in \eN\) et l'ensemble
    \begin{equation}
        U_p=\bigcup_{k=p}^{\infty}\phi^{-k}(A)
    \end{equation}
    des points qui repasseront encore dans \( A\) après \( p\) itérations  de \( \phi\). C'est un ensemble mesurable en tant que union d'ensembles mesurables (pour rappel, les tribus sont stables par union dénombrable, comme demandé à la définition \ref{DefjRsGSy}), et nous avons donc
    \begin{equation}
        \mu(U_p)\leq \mu(X)<\infty.
    \end{equation}
    De plus \( U_p=\phi^{-p}(U_0)\), donc \( \mu(U_p)=\mu(U_0)\). Vu que \( U_p\subset U_p\), nous avons
    \begin{equation}
        \mu(U_0\setminus U_p)=0.
    \end{equation}
    Étant donné que \( A\subset U_0\) nous avons a fortiori que
    \begin{equation}
        \{ x\in A\tq x\notin U_p \}\subset U_0\setminus U_p,
    \end{equation}
    et donc
    \begin{equation}
        \mu\{ x\in A\tq x\notin U_p \}=0.
    \end{equation}
    Cela signifie exactement que l'ensemble des points \( x\) de \( A\) tels que aucun des \( \phi^k(x)\) avec \( k\geq p\) n'est dans \( A\) est de mesure nulle.
\end{proof}

%--------------------------------------------------------------------------------------------------------------------------- 
\subsection{Fonction simple}
%---------------------------------------------------------------------------------------------------------------------------

\begin{definition}  \label{DefBPCxdel}
Une fonction mesurable \( f\colon X\to \eR\) est \defe{simple}{simple!fonction}\index{fonction!simple} si son image est finie
\begin{equation}
    f(x)=\sum_{j=1}^p\alpha_j\mtu_{A_j}(x)
\end{equation}
où \( A_j=f^{-1}(\alpha_j)\). Notons que \( f\) étant mesurable, les ensembles \( A_j\) sont forcément mesurables.
\end{definition}

\begin{lemma}[Limite croissante de fonctions étagées mesurables]    \label{LemYFoWqmS}
    Soit \( f\colon (\Omega,\tribA)\to \eR\) une fonction mesurable. Il existe une suite \( f_n\colon \Omega\to \eR\) de fonctions simples telles que \( f_n\to f\) ponctuellement et \( | f_n |<f\).
\end{lemma}

\begin{proof}
    Nous considérons \( (q_n)\) une suite parcourant tous les rationnels\footnote{Nous rappelons que \( \eQ\) est dénombrable et dense dans \( \eR\).}.
    Pour \( n\in \eN\) nous définissons la fonction
    \begin{equation}
        f_n(\omega)=\begin{cases}
            \max\{ q_i\tq i\leq n,\, q_i\leq f(\omega) \}    &   \text{si \( f(\omega)\geq 0\)}\\
            \min\{ q_i\tq i\leq n,\, q_i\geq f(\omega) \}    &    \text{si \( f(\omega)< 0\).}
        \end{cases}
    \end{equation}
    La fonction \( f_n\) est simple parce qu'elle ne prend que \( n\) valeurs différentes. Nous avons aussi par construction que \( | f_n(\omega)|\leq |f(\omega) |\). Nous avons aussi pour tout \( \omega\in \Omega\) que \( f_n(\omega)\to f(\omega)\) parce que \( \eQ\) est dense dans \( \eR\).

    En ce qui concerne la mesurabilité de \( f_n\), les étages de \( f_n\) sont les ensembles de la forme \( \{ \omega\in \Omega\tq f(\omega)\in\mathopen[ a , b [ \}\) où \( a\) et \( b\) sont deux éléments de \( \{ q_1,\ldots, q_n \}\) qui sont consécutifs au sens de l'ordre dans \( \eQ\) (et non spécialement au sens de l'ordre d'apparition dans la suite), avec éventuellement \( b=\infty\) si \( a\) est le plus grand. Les ensembles \( \mathopen[ a , b [\) étant mesurables dans \( \eR\) et la fonction \( f\) étant mesurable par hypothèse, les ensembles \( f^{-1}\Big( \mathopen[ a , b [ \Big)\) sont mesurables dans \( (\Omega,\tribA)\).
\end{proof}

\begin{proposition}\label{PropWBavIf}
    Une fonction positive et mesurable sur \( \Omega\) est limite ponctuelle croissante de fonctions simples positives.
\end{proposition}

\begin{proof}
    Soit \( \{ q_i \}\) une énumération des rationnels positifs. Il suffit de poser
    \begin{equation}
        f_n(x)=\max\{ q_i\tq i\leq n,\text{ et }f(x)\geq q_i \}.
    \end{equation}
\end{proof}

%+++++++++++++++++++++++++++++++++++++++++++++++++++++++++++++++++++++++++++++++++++++++++++++++++++++++++++++++++++++++++++ 
\section{Mesure extérieure}
%+++++++++++++++++++++++++++++++++++++++++++++++++++++++++++++++++++++++++++++++++++++++++++++++++++++++++++++++++++++++++++

\begin{definition}[\cite{MesureLebesgueLi}]
    Une \defe{mesure extérieure}{mesure!extérieure} sur un ensemble \( S\) est une application \( m^*\colon \partP(S)\to \mathopen[ 0 , \infty \mathclose]\) telle que
    \begin{enumerate}
        \item
            \( m^*(\emptyset)=0\),
        \item
            Si \( A\subset B\) dans \( S\) alors \( m^*(A)\leq m^*(B)\)
        \item
            Si les \( A_n\) sont des parties de \( S\) alors
            \begin{equation}    \label{EqZLMooSxvaL}
                m^*\big( \bigcup_{n\in \eN}A_n \big)\leq \sum_{n\in \eN}m^*(A_n).
            \end{equation}
    \end{enumerate}
\end{definition}
La différence avec une mesure est que nous ne demandons pas que \eqref{EqZLMooSxvaL} soit une égalité lorsque les \( A_n\) sont disjoints.

\begin{proposition}[\cite{MesureLebesgueLi}]
    Soit un espace mesuré \( (S,\tribF,\mu)\) et l'application\nomenclature[Y]{\( \mu^*\)}{La mesure extérieure associée à la mesure \( \mu\)}
    \begin{equation}
        \begin{aligned}
            \mu^*\colon \partP(S)&\to \mathopen[ 0 , \infty \mathclose] \\
            X&\mapsto \inf\{ \mu(A)\tq A\in\tribF,X\subset A \}. 
        \end{aligned}
    \end{equation}
    Alors \( \mu^*\) est une mesure extérieure sur \( S\) et sa restriction à \( \tribF\) est égale à \( \mu\).
\end{proposition}

\begin{proof}
    \begin{subproof}
    \item[Le vide]
        D'abord \( \mu^*(\emptyset)=O\) parce que \( \emptyset\in\tribF\).
    \item[Inégalité d'inclusion]

        Soient \( X\subset Y\) dans \( \partP(S)\). Si \( Y\subset A\) alors \( X\subset A\), donc
        \begin{equation}
            \inf \{ \mu(A)\tq A\in\tribF,X\subset A \}\leq \inf \{ \mu(A)\tq A\in\tribF,Y\subset A \},
        \end{equation}
        ce qui signifie que \( \mu^*(X)\leq \mu^*(Y)\).
    \item[Intégalité par union dénombrable]

        Soit \( (X_n)_{n\in \eN}\) une suite de parties de \( S\). Si il existe \( n_0\) tel que \( \mu^*(X_{n_0})=\infty\) alors nous avons automatiquement \( \sum_n\mu^*(X_n)=\infty\) et l'inégalité demandée est évidente parce que n'importe quoi est plus petit ou égal à \( \infty\). Nous supposons donc que \( \mu^*(X_n)<\infty\) pour tout \( n\). 

        Soit \( \epsilon>\) et par définition pour chaque \( n\), il existe un \( A_n\in \tribF\) tel que \( X_n\subset A_N\) et \( \mu(A_n)\leq \mu^*(X_n)+\frac{ \epsilon }{ 2^n }\). Bien entendu nous avons
        \begin{equation}
            \bigcup_nX_n\subset \bigcup_nA_n\in\tribF.
        \end{equation}
        Nous en déduisons que
        \begin{equation}
            \mu^*\big( \bigcup_nX_n \big)\leq\mu\big( \bigcup_nA_n \big).
        \end{equation}
        Mais \( (S,\tribF,\mu)\) étant un espace mesuré,
        \begin{equation}
            \mu\big( \bigcup_nA_n \big)\leq \sum_n\mu(A_n).
        \end{equation}
        Au final nous avons les inégalités
        \begin{equation}
            \mu^*\big( \bigcup_nX_n \big)\leq \mu\big( \bigcup_nA_n \big)\leq\sum_n\mu(A_n)\leq \sum_n\mu^*(X_n)+\epsilon\sum_n\frac{1}{ 2^n }=\sum_n\mu^*(X_n)+\epsilon.
        \end{equation}
        Cela étant vrai pour tout \( \epsilon\),
        \begin{equation}
            \mu^*\big( \bigcup_nX_n \big)\leq\sum_n\mu^*(X_n),
        \end{equation}
        ce qui prouve que \( \mu^*\) est une mesure extérieure.

    \item[Restriction]

    Supposons que \( X\in\tribF\). Alors si \( X\subset A\) nous avons \( \mu(X)\leq \mu(A)\); mais en même temps, \( \mu(X)\) est dans l'infimum qui définit \( \mu^*(X)\) donc
    \begin{equation}
        \mu^*(X)\leq\mu(X)\leq \inf \{ \mu(A)\tq A\in\tribF,X\subset A \}\leq \mu(X)\leq\mu^*(X).
    \end{equation}
    Donc nous avons égalité de tous les éléments de cette chaîne d'inégalité.
    \end{subproof}
\end{proof}


\begin{definition}  \label{DefTRBoorvnUY}
    Soit \( S\) un ensemble et \( m^*\) une mesure extérieure sur \( S\). Une partie \( A\subset X\) est \defe{$ m^*$-mesurable}{mesurable!au sens de $ m^*$} si pour tout \( X\subset S\),
    \begin{equation}
        m^*(X)=m^*(X\cap A)+m^*(X\cap A^c).
    \end{equation}
\end{definition}


\begin{remark}
    L'inégalité
    \begin{equation}
        m^*(X)\leq m^*(X\cap A)+m^*(X\cap A^c)
    \end{equation}
    étant toujours vraie, pour prouver qu'un ensemble est \( m^*\)-mesurable, il est suffisant de prouver l'inégalité inverse : 
    \begin{equation}
        m^*(X)\geq m^*(X\cap A)+m^*(X\cap A^c)
    \end{equation}
\end{remark}
La définition \ref{DefTRBoorvnUY} est motivée par la proposition suivante.

\begin{proposition} \label{PropOJFoozSKAE}
    Soit un espace mesuré \( (S,\tribF,\mu)\) et \( \mu^*\) la mesure extérieure qui va avec. Alors pour tous les éléments de \( \tribF\) sont \( \mu^*\)-mesurables. 
    
    En d'autres termes, pour tout \( A\in\tribF\) et tout \( X\subset S\) nous avons
    \begin{equation}
        \mu^*(X)=\mu^*(X\cap A)+\mu^*(X\cap A^c).
    \end{equation}
\end{proposition}

\begin{proof}
    Vu que \( X=(X\cap A)\cup(X\ca A^c)\),
    \begin{equation}
        \mu^*(X)\leq \mu^*(X\cap A)+\mu^*(X\cap A^c).
    \end{equation}
    Nous devons montrer l'inégalité inverse.

    Soit \( B\in\tribF\) tel que \( X\subset B\). D'une part nous avons \( X\cap A\subset B\cap A\in\tribF\), donc
    \begin{equation}
        \mu^*(X\cap A)\leq \mu(B\cap A). 
    \end{equation}
    Et d'autre part, \( X\cap A^c\subset B\cap A^c\in\tribF\), donc
    \begin{equation}
        \mu^*(X\cap A^c)\leq \mu(B\cap A^c). 
    \end{equation}
    En remettant ensemble,
    \begin{equation}    \label{EqLSMooTyHLB}
        \mu^*(X\cap A)+\mu^*(X\cap A^c)\leq \mu(B\cap A)+\mu(B\cap A^c)=\mu(B).
    \end{equation}
    La dernière égalité est le fait que \( B\cap A\) et \( B\cap A^c\) sont disjoints et que \( \mu\) est une mesure. L'inégalité \eqref{EqLSMooTyHLB} étant vraie pour tout \( B\in \tribF\) tel que \( X\subset B\), elle est encore vraie pour l'infimum :
    \begin{equation}
        \mu^*(X\cap A)+\mu^*(X\cap A^c)\leq \inf \{ \mu(B)\tq B\in\tribF,X\subset B \}=\mu^*(X).
    \end{equation}
    Nous avons donc prouvé que 
    \begin{equation}
        \mu^*(X\cap A)+\mu^*(X\cap A^c)\leq \mu^*(X).
    \end{equation}
\end{proof}

\begin{remark}
Notons la duplicité du vocabulaire. Les ensembles \( \mu\)-mesurables sont les éléments de \( \tribF\), qui sont a priori les seuls sur lesquels \( \mu\) est calculable\footnote{«calculable» au sens où \( \mu\) y vaut un nombre bien définit; après, que ce soit facile ou pas à calculer dans la pratique, c'est une autre histoire.}, alors que les \( \mu^*\)-mesurables sont les parties de \( S\) qui vérifient une certaine propriété (et \( \mu^*\) est calculable sur toutes les parties de \( S\)).
\end{remark}

%--------------------------------------------------------------------------------------------------------------------------- 
\subsection{Espace mesuré complet}
%---------------------------------------------------------------------------------------------------------------------------

\begin{definition}
    Soit un espace mesuré \( (X,\tribA,\mu)\). Une partie \( N\) de \( X\) est \defe{négligeable}{négligeable!partie d'un espace mesuré} pour \( \mu\) si il existe \( Y\in\tribA\) tel que \( N\subset Y\) et \( \mu(Y)=0\).
\end{definition}
Si \( \mu^*\) est la mesure extérieure associée à \( \mu\) et si \( N\) est \( \mu\)-négligeable alors \( \mu^*(N)=0\) parce que 
\begin{equation}
    \mu^*(N)\leq \mu^*(Y)=\mu(Y)=0
\end{equation}
pour un certain \( Y\) mesurable de mesure nulle contenant \( N\).

\begin{lemma}   \label{LemVKNooOCOQw}
    L'ensemble des parties négligeables est stable par union dénombrable.
\end{lemma}

\begin{proof}
    Si les ensembles \( N_i\) sont négligeables, alors pour chaque \( i\) nous avons \( Y_i\in\tribA\) tel que \( N_i\subset Y_i\) et \( \mu(Y_i)=0\). Alors bien entendu \( \bigcup_iN_i\subset \bigcup_iY_i\) et en utilisant \eqref{EqWWFooYPCTt},
    \begin{equation}
        \mu\big( \bigcup_iY_i \big)\leq \sum_i\mu(Y_i)=0.
    \end{equation}
\end{proof}

\begin{definition}  \label{DefBWAoomQZcI}
    L'espace mesuré \( (X,\tribF,\mu)\) est \defe{complet}{complet!espace mesuré} si tout ensemble \( \mu\)-négligeable est dans \( \tribF\).
\end{definition}

\begin{theorem}[Complétion d'espace mesuré\cite{MesureLebesgueLi,DXTooFCLru,BOQoojbFpP}]   \label{thoCRMootPojn}
    Soit un espace mesuré \( (X,\tribF,\mu)\) et \( \tribN\) l'ensemble des parties \( \mu\)-négligeables de \( X\).
    \begin{enumerate}
        \item
            Les ensembles suivants sont égaux :
            \begin{subequations}
                \begin{align}
                    \tribA&=\{ A\subset X\tq\exists B,C\in\tribF\tq B\subset A\subset C,\mu(C\setminus B)=0 \}\\
                    \tribB&=\{ B\cup N\tq  B\in\tribF,N\in\tribN \}    \label{EqFJIoorxZNU}\\
                    \tribC&=\{ A\subset X\tq \exists B\in\tribF\tq A\Delta B\in \tribN \}.
                \end{align}
            \end{subequations}
            Ici \( A\Delta B\) est la différence symétrique de \( A\) et \( B\), définition \ref{DefBMLooVjlSG}.
        \item
            L'ensemble \( \tribA=\tribB=\tribC\) est une tribu.
        \item
            La définition 
            \begin{equation}
                \begin{aligned}
                    \mu'\colon \tribB&\to \mathopen[ 0 , \infty \mathclose] \\
                    A\cup N&\mapsto \mu(A) 
                \end{aligned}
            \end{equation}
            est cohérente.
        \item
            L'application \( \mu'\) ainsi définie est une mesure sur \( (X,\tribA)\).
        \item
            L'espace \( (X,\tribA,\mu')\) est complet.
        \item
            La mesure \( \mu'\) prolonge \( \mu\).
        \item
            La mesure \( \mu'\) est minimale au sens où toute mesure complète prolongeant \( \mu\) prolonge \( \mu'\).
    \end{enumerate}
\end{theorem}

\begin{proof}
    Commençons par prouver que les trois ensembles \( \tribA\), \( \tribB\) et \( \tribC\) sont égaux.
    \begin{subproof}
    \item[\( \tribA\subset\tribB\).]
        Soit \( A\in\tribA\). Alors nous avons des ensembles \( B,C\in\tribF \) tels que \( B\subset A\subset V\) avec \( \mu(C\setminus B)=0\). Alors nous avons aussi \( A=B\cup(C\setminus B)\), ce qui prouve que \( A\in\tribB\).
    \item[\( \tribB\subset\tribC\).] 
        Soit \( A\in\tribB\), c'est à dire que \( A=B\cup N\) avec \( B\in\tribF\) et \( N\in\tribN\). Nous avons évidemment \( A\cup B=A\) et donc
        \begin{equation}
            A\Delta B=(A\cup B)\setminus(A\cap B)=A\setminus(A\cap B)=(B\cup N)\setminus(A\cap B)\subset N.
        \end{equation}
        Pour comprendre la dernière inclusion, si \( x\) appartient à \( A=B\cup N\) sans être dans \( N\) alors \( x\in B\) et donc \( x\in A\cap B\). Par conséquent nous avons \( A\Delta B\subset N\) et donc \( A\Delta B\in\tribN\).
    \item[\( \tribC\subset\tribA\)]
        Soit donc \( A\in\tribC\); il existe \( B\in\tribF\) tel que \( A\Delta B\in\tribN\) ou encore, il existe \( D\in\tribF\) tel que \( A\Delta B\subset D\) avec \( \mu(D)=0\). Si nous posons \( B'=B\cap D^c\) et \( C'=B\cup D\) alors nous prétendons avoir
        \begin{equation}
            B'\subset A\subset C'.
        \end{equation}
        Et nous le prouvons. En effet si \( x\in B\cap D^c\) alors en remarquant que \( B\) se divise en 
        \begin{equation}
            B=(B\cap A)\cup\big(B\cap (A\Delta B)\big),
        \end{equation}
        et en nous souvenant que \( B\cap (A\Delta B)\subset D\), il vient que \( B\cap D^c\subset B\cap A\). Et en particulier \( x\in A\). D'autre part
        \begin{equation}
            A\subset B\cup(A\Delta B)\subset B\cup D.
        \end{equation}
        Nous avons donc bien \( B'\subset A\subset C'\). Par stabilité de la tribu \( \tribF\) sous les intersections et complémentaires nous avons aussi \( B',C'\in\tribF\). De plus
        \begin{equation}
            C'\setminus B'=(B\cup D)\setminus(B\cap D^c)\subset D,
        \end{equation}
         et donc
         \begin{equation}
             \mu(C'\setminus B')\leq \mu(D)=0.
         \end{equation}
    \end{subproof}

    Nous avons donc prouvé que \( \tribA\subset\tribB\subset\tribC\subset \tribA\), et donc que \( \tribA=\tribB=\tribC\). Nous allons donc maintenant noter \( \tribA\) indifféremment les trois ensembles. Nous prouvons à présent que c'est une tribu.

    \begin{subproof}

        \item[Tribu : le vide]
            
            Pas de problèmes à \( \emptyset\in\tribA\)

        \item[Tribu : complémentaire]
            
            Soit \( A\in\tribA\). Alors il existe \( B,C\in\tribF\) tels que \( B\subset A\subset C\) avec \( \mu(C\setminus B)=0\). En passant au complémentaire,
            \begin{equation}
                C^c\subset A^c\subset B^c.
            \end{equation}
            Mais \( B^c\setminus C^c=C\setminus B\), donc \( \mu(B^c\setminus C^c)=0\).

        \item[Tribu : union dénombrable] 

            Soit \( (A_n)\) des éléments de \( \tribA\). Pour chaque \( n\) nous avons des ensembles \( B_n,C_n\in\tribF\) tels que\( B_n\subset A_n\subset C_n\) avec \( \mu(C_n\setminus B_n)=0\). En ce qui concerne les unions nous avons
            \begin{equation}
                \bigcup_nB_n\subset \bigcup_nA_n\subset \bigcup_nC_n,
            \end{equation}
            et 
            \begin{equation}
                \big( \bigcup_nC_n\big)\setminus\big( \bigcup_nB_n\big)\subset \bigcup_n(C_n\setminus B_n).
            \end{equation}
            Par conséquent, en utilisant \eqref{EqWWFooYPCTt},
            \begin{equation}
                \mu\left( \big( \bigcup_nC_n\big)\setminus\big( \bigcup_nB_n\big)\right)\leq\mu\left(  \bigcup_n(C_n\setminus B_n)\right)\leq\sum_n\mu(C_n\setminus B_n)=0.
            \end{equation}
            Cela prouve que \( \bigcup_nA_n\in\tribA\), et donc que \( \tribA\) est une tribu.

        \item[Définition cohérente]

            Soient \( A,A'\in\tribF\) et \( N,N'\in\tribN\) tels que \( A\cup N=A'\cup N'\). Nous considérons \( Y,Y'\in\tribF\) tel que \( N\subset Y\), \( N'\subset Y'\) et \( \mu(Y)=\mu(Y')=0\). En vertu de \eqref{EqWWFooYPCTt} nous avons
            \begin{equation}
                \mu(A)\leq \mu(A\cup Y)\leq \mu(A'\cup Y\cup Y')\leq\mu(A')+\mu(Y)+\mu(Y')=\mu(A').
            \end{equation}
            En écrivant la même chose en échangeant les primes nous prouvons également \( \mu(A')\leq \mu(A)\). Au final \( \mu(A)=\mu(A')\), c'est à dire
            \begin{equation}
                \mu'(A\cup N)=\mu'(A'\cup N').
            \end{equation}
            La définition de \( \mu'\) est donc cohérente.
        \item[\( \mu'\) est une mesure]

            Le fait que \( \mu'\) soit positive et que \( \mu'(\emptyset)\) soit nul ne pose pas de problèmes. Il faut voir l'union dénombrable disjointe. Si les ensembles \( A_i=B_i\cup N_i\) sont disjoints, alors les \( B_i\) et le \( N_i\) sont tous disjoints deux à deux. De plus l'ensemble \( \bigcup_iN_i\) est négligeable parce que nous avons déjà vu que \( \tribN\) était stable par union dénombrable (\ref{EqWWFooYPCTt}). Donc
            \begin{equation}
                \mu'\left( \bigcup_i B_i\cup N_i \right)=\mu'\Big( \big( \bigcup_iB_i \big)\cup\underbrace{\big( \bigcup_iN_i \big)}_{\in\tribN} \Big)=\mu\big( \bigcup_iB_i \big)=\sum_u\mu(B_i)=\sum_i\mu'(B_i\cup N_i).
            \end{equation}
        \item[Espace complet]
            Un ensemble \( \mu'\)-négligeable est automatiquement \( \mu\)-négligeable. En effet si \( H\) est \( \mu'\)-négligeable, il existe \( B\in\tribF\) et \( N\in\tribN\) tels que \( H\subset B\cup N\) avec \( \mu(B)=0\). Vu que \( N\) est \( \mu\)-négligeable, il existe \( Y\in\tribF\) tel que \( N\subset Y\) et \( \mu(Y)=0\). Donc \( H\subset B\cup N\subset B\cup Y\) avec \( \mu(B\cup Y)=0\).

            Tous les ensembles \( \mu\)-négligeables faisant partie de \( \tribB\), tous les ensembles \( \mu'\)-négligeables font partie de \( \tribA\).
        \item[Prolongement]
            La mesure \( \mu'\) prolonge \( \mu\). En effet si \( A\in\tribF\) alors \( A=A\cup\emptyset\in\tribB\) et \( A\) est \( m'\)-mesurable. De plus \( \mu'(A)=\mu'(A\cup\emptyset)=\mu(A)\).
        \item[Minimalité]

            Soit un espace mesuré complet \( (X,\tribM,\nu)\) prolongeant \( (X,\tribF,\mu)\). Pour \( A\in\tribA\) nous devons prouver que \( A\in\tribM\) et que \( \mu'(A)=\nu(A)\). Il existe \( B\in\tribF\) et \( N\in\tribN\) tels que \( A=B\cup N\). Vu que \( N\) est \( \mu\)-négligeable, il est également \( \nu\)-négligeable et donc \( \nu\)-mesurable parce que \(\nu\) est complète : \( A\in\tribM\). En ce qui concerne l'égalité \( \mu'(A)=\nu(A)\) nous avons
            \begin{equation}
                \nu(B)\leq\nu(B\cup N)\leq \nu(B)+\nu(N)=\nu(B),
            \end{equation}
            donc \( \nu(A)=\nu(B\cup N)=\nu(B)=\mu(B)\). La dernière égalité est le fait que \( \nu\) prolonge \( \mu\). Mais par définition de \( \mu'\) nous avons aussi \( \mu'(A)=\mu'(B\cup N)=\mu(B)\). Au final \( \mu'(A)=\nu(A)=\mu(B)\).

    \end{subproof}

\end{proof}

\begin{definition}
    L'espace mesuré complet \( (X,\tribA,\mu')\) défini par le théorème \ref{thoCRMootPojn} est l'\defe{espace mesuré complétée}{espace!mesuré!complété} de \( (X,\tribF,\mu)\).
\end{definition}

\begin{theorem}[Carathéodory\cite{MesureLebesgueLi}]
    Soit \( S\) un ensemble et \( m^*\) une mesure extérieure sur \( S\). Alors
    \begin{enumerate}
        \item
            l'ensemble \( \tribM\) des parties \( m^*\)-mesurables est une tribu,
        \item
            la restriction de \( m^*\) est une mesure sur \( (S,\tribM)\),
        \item
            l'espace mesuré \( (S,\tribM,m^*)\) est complet\footnote{Définition \ref{DefBWAoomQZcI}.}.
    \end{enumerate}
\end{theorem}

\begin{proof}
    Une grosse partie de la preuve sera de prouver la stabilité de \( \tribM\) par union dénombrable quelconque; cela sera divisé en plusieurs parties.
    \begin{subproof}
    \item[Tribu : le vide]
        L'ensemble vide est \( m^*\)-mesurable.
    \item[Tribu : complémentaire]
        Soit \( A\in\tribM\) et \( X\in S\). La condition qui dirait \( A^c\in\tribM\) est :
        \begin{equation}
            m^*(X)=m^*(X\cap A^c)+m^*(X\cap A),
        \end{equation}
        qui est la même que celle qui dit que \( A\) est dans \( \tribM\).
    \item[Tribu : union finie]
        Soient \( A,B\in\tribM\) et \( X\subset S\). Alors, vu que \( m^*\) est une mesure extérieure,
        \begin{subequations}
            \begin{align}
                m^*(X)&\leq m^*\big( X\cap(A\cup B) \big)+m^*\big( X\cap (A\cup B)^x \big)\\
                &=m^*\big( (X\cap A)\cup(X\cap B) \big)+m^*\big( X\cap A^c\cap B^c \big).
            \end{align}
        \end{subequations}
        Mais nous pouvons écrire la première union sous forme d'une union disjointe de la façon suivante :
        \begin{equation}
            (X\cap A)\cup(X\cap B)=(X\cap A)\cup(X\cap B\cap A^c),
        \end{equation}
        ce qui donne 
        \begin{subequations}
            \begin{align}
                m^*(X)&\leq m^*(X\cap A)+m^*(X\cap B\cap A^c)+m^*(X\cap A^c\cap B^c)        \label{subeqLYNooRdrgCi}\\
                &=m^*(X\cap A)+m^*(X\cap A^c)\\
                &=m^*(X)
            \end{align}
        \end{subequations}
        parce que les deux dernier termes de \eqref{subeqLYNooRdrgCi} se somment à \( m^*(X\cap A^c)\) parce que \( B\in \tribM\). La dernière ligne est le fait que \( A\) soit \( m^*\)-mesurable.
    \item[Union finie disjointe]
        Soient \( \{ A_1,\ldots, A_n \}\) des éléments deux à deux disjoints de \( \tribM\). Nous allons maintenant prouver par récurrence que
        \begin{equation}    \label{EqBRIooAnPCd}
            m^*\Big( X\cap\big( \bigcup_{k=1}^nA_k \big) \Big)=\sum_{k=1}^nm^*(X\cap A_k).
        \end{equation}
        Si \( n=1\) le résultat est évident. Sinon, le fait que \( A_{n+1}\) soit \( m^*\)-mesurable donne
        \begin{equation}
            m^*\Big( X\cap\big( \bigcup_{k=1}^{n+1}A_k \big) \Big)=m^*\Big( X\cap\big( \bigcup_{k=1}^{n+1}A_k \big)\cap A_{n+1} \Big)+m^*\Big( X\cap\big( \bigcup_{k=1}^{n+1}A_k \big)\cap A_{n+1}^c \Big).
        \end{equation}
        Le fait que les \( A_k\) soient disjoints implique aussi que
        \begin{equation}
            X\cap\big( \bigcup_{k=1}^{n+1}A_k \big)\cap A_{n+1}=X\cap A_{n+1}
        \end{equation}
        et
        \begin{equation}
            X\cap\big( \bigcup_{k=1}^{n+1}A_k \big)\cap A_{n+1}^c=X\cap\big( \bigcup_{k=1}^nA_k \big)
        \end{equation}
        et donc
        \begin{subequations}
            \begin{align}
                m^*\Big( X\cap\big( \bigcup_{k=1}^{n+1}A_k \big) \Big)&=m^*(X\cap A_{n+1})+m^*\Big( X\cap\big( \bigcup_{k=1}^nA_k \big) \Big)\\
                &\stackrel{rec.}{=}m^*(X\cap A_{n+1})+\sum_{k=1}^nm^*(X\cap A_k)\\
                &=\sum_{k=1}^{n+1}m^*(X\cap A_k).
            \end{align}
        \end{subequations}
        La relation \eqref{EqBRIooAnPCd} est prouvée.

        Notons qu'en particularisant à \( X=S\) nous avons 
        \begin{equation}
            m^*\big( \bigcup_{k=1}^nA_k \big)=\sum_{k=1}^nm^*(A_k)
        \end{equation}
        dès que les \( A_k\) sont des éléments deux à deux disjoints de \( \tribM\).
        
    \item[Union dénombrable disjointe]

        Soit \( (A_n)_{n\in \eN}\) une suite d'éléments deux à deux disjoints dans \( \tribM\). Nous allons prouver les choses suivantes :
        \begin{itemize}
            \item \( \bigcup_nA_n\in\tribM\)
            \item \( m^*\big( \bigcup_nA_n \big)=\sum_nm^*(A_n)\)
        \end{itemize}
        où toutes les sommes et union sur \( n\) sont entre \( 1\) et \( \infty\).

        Nous posons \( A=\bigcup_kA_k\) et \( B_n=\bigcup_{k=1}^nA_k\). Nous savons que \( B_n\in\tribM\) pour tout \( n\) par le point précédent. Donc si \( X\in S\) nous avons
        \begin{subequations}
            \begin{align}   \label{EqGXLooRxqqg}
                m^*(X)&=m^*(X\cap B_n)+m^*(X\cap B_n^c)\\
                &=\sum_{k=1}^nm^*(X\cap A_k)+m^*(X\cap B_n^x)\\
                &\geq\sum_{k=1}^nm^*(X\cap A_k)+m^*(X\cap A^c)
            \end{align}
        \end{subequations}
        où nous avons utilisé la relation \eqref{EqBRIooAnPCd} sur les \( B_n\) ainsi que le fait que \( A^c\subset B_n^c\) (parce que \( B_n\subset A\)). L'inégalité \eqref{EqGXLooRxqqg} étant vraie pour tout \( n\), elle est vraie à la limite :
        \begin{subequations}
            \begin{align}
                m^*(X)&\geq \sum_{k=1}^{\infty}m^*(A\cap A_k)+m^*(X\cap A^c)\\
                &\geq m^*\Big( \bigcup_k(X\cap A_k) \Big)+m^*(X\cap A^c)\\
                &=m^*\Big( X\cap \big( \bigcup_kA_k \big) \Big)+m^*(X\cap A^c)\\
                &=m^*(X\cap A)+m^*(X\cap A^c),
            \end{align}
        \end{subequations}
        ce qui signifie que \( A\in\tribM\). La première des deux choses que nous voulions montrer est faite. En la particularisant à \( X=A\) et en tenant compte des faits que \( A\cap A_k=A_k\) et \( A\cap A^c=\emptyset\),
        \begin{equation}
            m^*(A)\geq \sum_{k=1}^{\infty}m^*(A\cap A_k)+m^*(A\cap A^c),
        \end{equation}
        c'est à dire que pour tout \( n\) nous avons
        \begin{equation}
            m^*\big( \bigcup_{k\in \eN}A_k \big)\geq \sum_{k=1}^nm^*(A_k).
        \end{equation}
        L'inégalité est encore vraie à la limite, et l'inégalité inverse étant toujours vraie pour une mesure extérieure,
        \begin{equation}
            m^*\big( \bigcup_{k\in \eN}A_k \big)=\sum_{k=1}^{\infty}m^*(A_k).
        \end{equation}

    \item[Union dénombrable quelconque]

        Soit maintenant une suite \( (A_n)_{n\in\eN}\) d'éléments de \( \tribM\) que nous ne supposons plus être disjoints. Nous nous ramenons au cas disjoint en posant
        \begin{subequations}
            \begin{numcases}{}
                B_1=A_1\\
                B_n=A_n\cap\big( \bigcup_{k=1}^{n-1}A_k \big)^c,
            \end{numcases}
        \end{subequations}
        c'est à dire que nous mettons dans \( B_n\) les éléments de \( A_n\) qui ne sont dans aucun des \( A_k\) précédents. Autrement dit, nous posons \( B_0=\emptyset\) et \( B_n=A_n\setminus B_{n-1}\). L'ensemble \( \tribM\) étant stable par réunion finie, par complément et par intersection finie nous avons \( B_n\in\tribM\). De plus les \( B_n\) sont disjoints, donc
        \begin{equation}
            \bigcup_{k=1}^{\infty}A_k=\bigcup_{k=1}^{\infty}B_k\in\tribM.
        \end{equation}
        La première égalité se justifie de la façon suivante : si \( x\in\bigcup_{k=1}^{\infty}A_k\) alors nous notons \( n_0\) le plus petit \( n\) tel que \( x\in A_n\) et alors \( x\in B_{n_0}\).
    \item[Espace complet]
        Nous prouvons à présent que \( (S,\tribM,m^*)\) est un espace mesuré complet. Soit \( N\) une partie \( m^*\)-négligeable de \( S\) et \( Y\in\tribM\) tel que \( m^*(Y)=0\) et \( N\subset Y\). D'abord \( m^*(N)=0\) parce que
        \begin{equation}
            m^*(N)\leq m^*(Y)=0.
        \end{equation}
        Si \( X\subset S\) nous avons
        \begin{subequations}
            \begin{align}
                X\cap N\subset   N&\Rightarrow m^*(X\cap N)=0\\
                X\cap N^c\subset X&\Rightarrow m^*(X\cap N^c)\leq m^*(X).
            \end{align}
        \end{subequations}
        Donc
        \begin{equation}
            m^*(X\cap N)+m^*(X\cap N^c)\leq m^*(X),
        \end{equation}
        ce qui montre que \( N\) est est \( m^*\)-mesurable.
    \end{subproof}
\end{proof}

%+++++++++++++++++++++++++++++++++++++++++++++++++++++++++++++++++++++++++++++++++++++++++++++++++++++++++++++++++++++++++++
\section{Mesure de Lebesgue}
%+++++++++++++++++++++++++++++++++++++++++++++++++++++++++++++++++++++++++++++++++++++++++++++++++++++++++++++++++++++++++++

Nous construisons à présent la mesure de Lebesgue sur \( \eR^n\). Un \defe{pavé}{pavé} dans \( \eR^n\) est un ensemble de la forme 
\begin{equation}
    B=\prod_{i=1}^n\mathopen[ a_i , b_i \mathclose];
\end{equation}
le volume d'un tel pavé est défini par \( \Vol(B)=\prod_i(b_i-a_i)\). Soit maintenant \( A\subset \eR^n\). La \defe{mesure externe}{mesure!externe} de \( A\) est le nombre
\begin{equation}
    m^*(A)=\inf\{ \sum_{B\in\mF}\Vol(B)\text{ où \( \mF\) est un ensemble dénombrable de pavés dont l'union recouvre \( A\).} \}
\end{equation}

\begin{definition}  \label{DefKTzOlyH}
Nous disons que \( A\) est \defe{mesurable}{mesurable!Lebesgue} au sens de Lebesgue si pour tout ensemble \( S\subset \eR^n\) nous avons l'égalité
\begin{equation}
    m^*(S)=m^*(A\cap S)+m^*(S\setminus A).
\end{equation}
Dans ce cas nous disons que la mesure de Lebesgue de \( A\) est \( m(A)=m^*(A)\).
\end{definition}

\begin{proposition}     \label{PropNCMToWI}
    Deux fonctions continue égales presque partout pour la mesure de Lebesgue\footnote{Définition \ref{DefKTzOlyH}.} sont égales.
\end{proposition}

\begin{proof}
    Soient \( f\) et \( g\) deux fonctions continues telles que \( f(x)=g(x)\) pour presque tout \( x\in D\). La fonction \( h=f-g\) est alors presque partout nulle et nous devons prouver qu'elle est nulle sur tout \( D\). La fonction \( h\) est continue; si \( h(a)\neq 0\) pour un certain \( a\in D\) alors \( h\) est non nulle sur un ouvert autour de \( a\) par continuité et donc est non nulle sur un ensemble de mesure non nulle.
\end{proof}


\begin{lemma}\label{LemTHBSEs}
    Si \( f\) est une fonction sur \( \mathopen[ a , \infty [\), alors nous avons la formule
    \begin{equation}
        \lim_{b\to \infty}\int_a^bf(x)dx=\int_a^{\infty}f(x)dx
    \end{equation}
    au sens où si un des deux membres existe, alors l'autre existe et est égal.
\end{lemma}

\begin{proof}
    Supposons que le membre de gauche existe. Cela signifie que la fonction
    \begin{equation}
        \psi(x)=\int_a^xf
    \end{equation}
    est bornée. Soit \( M\), un majorant. Pour toute fonction simple \( \varphi\) dominant \( f\), on a que \( \int\varphi\leq M\), donc l'ensemble sur lequel on prend le supremum pour calculer \( \int_a^{\infty}f\) est majoré par \( M\) et possède donc un supremum. Nous avons donc
    \begin{equation}
        \int_a^{\infty}f\leq\lim_{b\to\infty}\int_a^bf.
    \end{equation}
\end{proof}
