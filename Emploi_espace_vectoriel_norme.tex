
Ce chapitre va traiter d'espaces vectoriels muni d'une notion de «taille» d'un vecteur. Le principal exemple qu'il faut avoir en tête tout au long du chapitre est $\eR^m$ muni de la norme euclidienne; nous entreverrons dans la section \ref{normes_equiv} pourquoi l'intuition sur cet exemple particulier fonctionne tellement bien. Il faut cependant garder à l'esprit que d'autres exemples existent : nous verrons le cas de l'espace des applications linéaires au chapitre \ref{Chap_appl_lin}.

La définition d'une norme est donnée dans la section \ref{Sect_definition}. Les sections suivantes, \ref{Sect_boules} et \ref{Sect_topologie}, présentent la topologie des espaces vectoriels de dimension finie. Cette topologie, comme vous verrez, découle immédiatement de la norme introduite. On parle alors de topologie \emph{induite par la norme}. Dans la section \ref{Sect_suites} nous  donnons la définition de suite convergente de vecteurs et dans la section \ref{Sect_fonctions} la définition de fonction continue. Enfin, la section \ref{sec_prod} parle de normes et de la topologie des espaces vectoriels qui sont le produit de deux (ou plus) espaces vectoriels de dimension finie.

Tout le matériel présenté dans ce chapitre est sujet d'examen à l'exclusion de

\begin{itemize}
\item la partie de la section \ref{Sect_fonctions} qui suit la définition \ref{DefContDansEVN};
\item les démonstrations de la section \ref{sec_prod};
\item la sous-section \ref{SubSecPOlynomesCE};
\end{itemize}
