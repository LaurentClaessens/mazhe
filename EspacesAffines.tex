% This is part of Mes notes de mathématique
% Copyright (c) 2011-2012
%   Laurent Claessens
% See the file fdl-1.3.txt for copying conditions.

Ce chapitre provient principalement de \cite{Combes}.

\begin{definition}
    Un \defe{espace affine}{affine!espace}\index{espace!affine} est un ensemble \( \affE\) sur lequel le groupe\footnote{Voir exemple \ref{ExemMaKdwt}.} \( (E,+)\) agit à droite transitivement et librement.
\end{definition}

Étant donné que \( E\) est un groupe commutatif, l'action peut être vue indifféremment à gauche ou à droite. Si \( M\in\affE\) et si \( x\in E\) nous notons \( M+x\) au lieu de \( x\cdot M\) le résultat de l'action de \( x\) sur \( M\).

\begin{remark}
    Lorsque nous écrivons «\( M+x\)», le symbole plus n'est pas une loi de composition interne de \( \affE\), mais une action.
\end{remark}

Soient \( N,M\in\affE\). Par liberté et transitivité de l'action, il existe un unique \( x\in E\) tel que \( M+x=N\). Ce vecteur \( x\) sera noté \( MN\).

\begin{proposition}
    Si \( A,B,C\in\affE\) nous avons les égalités suivantes dans \( E\) :
    \begin{enumerate}
        \item
            \( AB+BC=AC\) (relations de Chasles)\index{relations!de Chasles}\index{Chasles},
        \item
            \( AA=0\),
        \item
            \( AB=-AB\).
    \end{enumerate}
\end{proposition}

Si \( E\) est un espace vectoriel, le groupe \( (E,+)\) agit sur \( E\) par l'action \( t_y(x)=y+x\). Utilisant cette action nous construisons l'\defe{espace affine canonique}{espace!affine!canonique}\index{canonique!espace affine} de \( E\). En particulier nous notons \( \affE_n(\eK)\) l'espace affine canonique de \( \eK^n\) vu comme espace vectoriel sur \( \eK\).

%---------------------------------------------------------------------------------------------------------------------------
\subsection{Repères affines}
%---------------------------------------------------------------------------------------------------------------------------

Soit \( E\) un \( \eK\)-espace vectoriel de dimension \( n\) et \( \affE\) un espace affine construit sur \( E\).
\begin{definition}
    Un multiplet \( (A,e_1,\ldots, e_n)\) où \( A\) est un point de \( \affE\) et \( \{ e_i \}\) est une base de \( E\) est un \defe{repère cartésien}{repère!cartésien!espace affine} de \( \affE\).
\end{definition}
Un tel repère donne une bijection
\begin{equation}
    \begin{aligned}
        \phi\colon \eK^n&\to \affE \\
        (x_1,\ldots, x_n)&\mapsto A+\sum_ix_ie_i. 
    \end{aligned}
\end{equation}
Ces nombres \( x_i\) sont les \defe{coordonnées}{coordonnées!dans un espace affine} du point \( A+\sum_ix_ie_i\) dans le repère \( (A,e_i)\).

Soient \( (A,e_i)\) et \( (A',e'_i)\) deux repères cartésiens pour l'espace affine \( \affE\). Soit \( (a_{ij})\) la matrice de changement de base entre \( \{ e_i \}\) et \( \{ e'_i \}\) dans \( E\). Nous voudrions trouver les \( x_i\) en termes des \( x'_i\).

Pour cela nous considérons un point \( M\) dans \( \affE\) et nous l'écrivons dans les deux bases. Cela fournit l'égalité
\begin{equation}        \label{EqcYfuMg}
    A+\sum_ix_ie_i=A'+\sum_ix'_ie'_i.
\end{equation}
Nous considérons les coordonnées \( (a_i)\) de \( A'\) dans le repère \( (A,e_i)\), c'est à dire
\begin{equation}    \label{EqZNwPHE}
    A'=A+\sum_ia_ie_i.
\end{equation}
En substituant \( e'_i=\sum_ka_{jk}e_k\) et \eqref{EqZNwPHE} dans \eqref{EqcYfuMg} nous trouvons
\begin{equation}
    \sum_kx_ke_k=\sum_ka_ke_k+\sum_{jk}a_{jk}x'_je_k,
\end{equation}
et par conséquent
\begin{equation}
    x_k=a_k+\sum_ja_{jk}x'_j.
\end{equation}

%+++++++++++++++++++++++++++++++++++++++++++++++++++++++++++++++++++++++++++++++++++++++++++++++++++++++++++++++++++++++++++
\section{Classification affine des conique}
%+++++++++++++++++++++++++++++++++++++++++++++++++++++++++++++++++++++++++++++++++++++++++++++++++++++++++++++++++++++++++++

Soit une conique \( f(x,y)=0\) avec
\begin{equation}
    f(x,y)=ax^2+2bxy+cy^2+2dx+2ey+f  
\end{equation}
dans le repère \( R=(A,e_i)\). La signature de la quadratique
\begin{equation}
    q(x,y)=ax^2+2bx+cy^2
\end{equation}
ne dépend pas de la base choisie et un changement de variables
\begin{subequations}
    \begin{numcases}{}
        \tilde x=\alpha x+\beta y\\
        \tilde y=\gamma x+\delta y
    \end{numcases}
\end{subequations}
peut nous amener dans trois cas :
\begin{equation}
    q(x,y)=\begin{cases}
        \tilde x^2+\tilde y^2    &   \text{genre ellipse}\\
        \tilde x^2-\tilde y^2    &    \text{genre hyperbole}\\
        \tilde x^2               &  \text{genre parabole}.   
    \end{cases}
\end{equation}
Dans le troisième cas, la matrice de \( q\) est de rang \( 1\).

Nous cherchons maintenant à savoir si un point \( I=(x_0,y_0)\) est un centre de symétrie de \( f(x,y)=0\). Pour cela nous choisissons le repère centré en \( I\), c'est à dire que nous posons 
\begin{subequations}
    \begin{numcases}{}
        x=x_0+\tilde x\\
        y=y_0+\tilde y.
    \end{numcases}
\end{subequations}
Un peu de calcul montre qu'alors la conique s'écrit
\begin{equation}
    f(x_0,y_0)+q(\tilde x,\tilde y)+(2ax_0+2by_0+2d)\tilde x+(2bx_0+2cy_0+2e)\tilde y=0.
\end{equation}
Le point \( I\) sera un centre de symétrique si les termes linéaires en \( \tilde x\) et \( \tilde y\) s'annulent, c'est à dire si
\begin{subequations}        \label{SyskhiOvW}
    \begin{numcases}{}
        ax_0+by_0+d=0\\
        bx_0+cy_0+e=0.
    \end{numcases}
\end{subequations}
Nous supposons que \( (d,e)\neq (0,0)\), sinon la conique de départ serait déjà centrée. Le déterminant du système \eqref{SyskhiOvW} est 
\begin{equation}
    \delta=ac-b^2.
\end{equation}
Si ce dernier est différent de zéro, le système possède une unique solution et la conique aura alors un unique centre de symétrie.

Si le déterminant du système est nul, il y a soit pas de centre de symétrie, soit une infinité. Dans le premier cas nous sommes en présence d'une parabole, et dans le second cas de deux droites parallèles.

\begin{example}
    Soit 
    \begin{equation}    \label{EqOgsEcz}
        f(x,y)=x^2+2xy-y^2-6x+2y-1=0
    \end{equation}
    donnée dans le repère affine \( R=(A,\{ e_i \})\). Nous commençons par étudier la signature de \( q(x,y)=x^2+2xy-y^2\) dont la matrice symétrique est
    \begin{equation}
        Q=\begin{pmatrix}
            1    &   1    \\ 
            1    &   -1    
        \end{pmatrix}.
    \end{equation}
    Son polynôme caractéristique est \( \lambda^2-2\) sont les racines sont \( \pm\sqrt{2}\). La signature est donc \( (1,-1)\) et nous sommes en présence d'une conique de genre hyperbole. Nous cherchons le centre en posant \( x=\tilde x+x_0\), \( y=\tilde y+y_0\). Le système à résoudre est
    \begin{subequations}
        \begin{numcases}{}
            x_0+y_0-3=0\\
            x_0-y_0+1=0,
        \end{numcases}
    \end{subequations}
    dont l'unique solution est \( (x_0,y_0)=(1,2)\). Nous considérons le repère centré en \( (x_0,y_0)\), c'est à dire le repère
    \begin{equation}
        R'=(I,\{ e_i \})
    \end{equation}
    avec \( I=A+x_0e_1+y_0e_2\) où \( A\) est l'origine du repère dans lequel l'équation \eqref{EqOgsEcz} était donnée.

    Par construction dans ce repère nous avons la conique
    \begin{equation}
        f(x_0,y_0)+q(\tilde x,\tilde y)=0,
    \end{equation}
    c'est à dire
    \begin{equation}
        \tilde x^2+2\tilde x\tilde y-\tilde y^2=0.
    \end{equation}
    Maintenant la nous avons une quadrique centrée nous voulons la mettre sous une forme plus canonique :
    \begin{equation}
        \left( \frac{1}{ \sqrt{2} }(\tilde x+\tilde y) \right)^2-\tilde y^2-1=0.
    \end{equation}
    Nous posons donc
    \begin{subequations}
        \begin{numcases}{}
            X=\frac{1}{ \sqrt{2} }(\tilde x+\tilde y)\\
            Y=\tilde y,
        \end{numcases}
    \end{subequations}
    et nous trouvons l'hyperbole
    \begin{equation}
        X^2-Y^2-1=0.
    \end{equation}
    Cela revient à faire le changement de base
    \begin{subequations}    \label{EqfiVwym}
        \begin{numcases}{}
            e'_1=\sqrt{2}e_1\\
            e'_2=-e_1+e_2.
        \end{numcases}
    \end{subequations}
    Pour rappel, les vecteurs de bases se transforment avec la matrice inverse des coefficients. Étant donné que
    \begin{equation}
        \begin{pmatrix}
            X    \\ 
            Y    
        \end{pmatrix}=\begin{pmatrix}
            1/\sqrt{2}    &   1/\sqrt{2}    \\ 
            0    &   1    
        \end{pmatrix}\begin{pmatrix}
            \tilde x    \\ 
            \tilde y    
        \end{pmatrix},
    \end{equation}
    nous avons
    \begin{equation}
        \begin{pmatrix}
            e'_1    \\ 
            e'_2    
        \end{pmatrix}=\begin{pmatrix}
            1/\sqrt{2}    &   1/\sqrt{2}    \\ 
            0    &   1    
        \end{pmatrix}^{-1}\begin{pmatrix}
            e_1    \\ 
            e_2    
        \end{pmatrix}.
    \end{equation}
    C'est de là que provient le changement \eqref{EqfiVwym}.
\end{example}

%+++++++++++++++++++++++++++++++++++++++++++++++++++++++++++++++++++++++++++++++++++++++++++++++++++++++++++++++++++++++++++
\section{Applications affines}
%+++++++++++++++++++++++++++++++++++++++++++++++++++++++++++++++++++++++++++++++++++++++++++++++++++++++++++++++++++++++++++

\begin{definition}
    Soient \( \affE\) et \( \affE'\) deux espaces affines sur les espaces vectoriels \( E\) et \( E'\) (sur le même corps \( \eK\)). Une application \( f\colon \affE\to \affE'\) est dite \defe{affine}{affine!application} si il existe une application linéaire \( u\colon E\to E'\) telle que pour tout \( M\in\affE\) on ait
    \begin{equation}    \label{EqMqIoWX}
        f(M+x)=f(M)+u(x).
    \end{equation}
\end{definition}
\begin{remark}
    La condition \eqref{EqMqIoWX} pour tout \( M\in\affE\) est équivalente à demander 
    \begin{equation}
        f\circ t_x=t_{u(x)}\circ f
    \end{equation}
    pour tout \( x\in E\).
\end{remark}

\begin{proposition}
    Soit \( f\) une application affine.
    \begin{enumerate}
        \item
            Une application linéaire vérifiant la condition \eqref{EqMqIoWX} est unique. Nous la noterons \( u_f\).
        \item
            L'application \( u_f\) est injective si et seulement si \( f\) est injective.
        \item
            L'application \( u_f\) est surjective si et seulement si \( f\) est surjective.
    \end{enumerate}
    Si \( \affE\) et \( \affE'\) ont même dimension finie, alors en plus \( f\) est injective si et seulement si \( f\) est surjective.
\end{proposition}

\begin{remark}
    La fonction linéaire \( u_f\) qui vérifie \( f(M+x)=f(M)+u_f(x)\) ne dépend pas du point \( M\). En effet si
    \begin{subequations}
        \begin{align}
            f(M+x)&=f(M)+u(x)\\
            f(N+y)&=f(N)+v(y).
        \end{align}
    \end{subequations}
    En effet si \( N=M+a\) nous avons d'une part que
    \begin{equation}
        f(N+y)=f(M+y+a)=f(M)+u(y+a),
    \end{equation}
    et d'autre part
    \begin{equation}
        f(N+y)=f(M+a)+v(y)=f(M)+u(a)+v(y).
    \end{equation}
    Par conséquent \( u=v\).
\end{remark}

\begin{proposition}
    Si \( f\colon \affE\to \affE'\) et \( g\colon \affE\to \affE''\) sont des applications affines, alors \( g\circ f\colon \affE\to \affE''\) est affine et \( u_{g\circ f}=u_g\circ u_f\).
\end{proposition}

\begin{proof}
    Si \( M\in\affE\) et \( x\in E\) nous avons
    \begin{equation}
        \begin{aligned}[]
            (g\circ f)(M+x)&=g\big( f(M)+u_f(x) \big)\\
            &=f\big( f(M) \big)+u_g\big( u_f(x) \big)\\
            &=(g\circ f)(M)+(u_g\circ u_f)(x).
        \end{aligned}
    \end{equation}
\end{proof}

\begin{theorem}
    Soient \( \affE\) et \( \affE'\) deux espaces affines de dimensions finies \( p\) et \( q\) sur \( \eK\). Soient les repères cartésiens \( R=(O,\{ e_i \})\) et \( R'=(O',\{ e_i' \})\). Une application \( f\colon \affE\to \affE'\) est affine si et seulement si il existe une matrice \( a\in\eM_{p,q}(\eK)\) et \( b\in \eK^q\) tels que
    \begin{equation}    \label{EqCmNHjs}
        f(x)=b+ax.
    \end{equation}
\end{theorem}

\begin{remark}
    L'équation \eqref{EqCmNHjs} est écrite en utilisant un abus de notation entre le vecteur \( x\in \eK^p\) et le point de \( \affE\) qui est représenté par \( x\) dans le repère \( (A,\{ e_i \})\).    
\end{remark}

%+++++++++++++++++++++++++++++++++++++++++++++++++++++++++++++++++++++++++++++++++++++++++++++++++++++++++++++++++++++++++++
\section{Isomorphismes}
%+++++++++++++++++++++++++++++++++++++++++++++++++++++++++++++++++++++++++++++++++++++++++++++++++++++++++++++++++++++++++++

\begin{definition}
    Un \defe{isomorphisme}{isomorphisme!espace affine} entre les espaces affines \( \affE\) \( \affE'\) est une application affine \( f\colon \affE\to \affE'\) inversible dont l'inverse est affine.
\end{definition}

\begin{proposition} \label{PropxtFeDE}
    Une application affine bijective est un isomorphisme. Si \( f\) est un isomorphisme d'espaces affines, alors \( u_{f^{-1}}=(u_f)^{-1}\).
\end{proposition}

\begin{proposition}
    Un espace affine de dimension finie \( n\) sur un corps \( \eK\) est isomorphe à l'espace affine canonique \( \affE_n(\eK)\).
\end{proposition}

\begin{proof}
    Si nous considérons le repère \( R=(A,\{ e_i \})\) de l'espace affine \( \affE\) alors l'application
    \begin{equation}
        \begin{aligned}
            \varphi\colon \eK^n&\to \affE \\
            (x_1,\ldots,x_n)&\mapsto A+\sum_ix_ie_i 
        \end{aligned}
    \end{equation}
    est un isomorphisme.
\end{proof}

%+++++++++++++++++++++++++++++++++++++++++++++++++++++++++++++++++++++++++++++++++++++++++++++++++++++++++++++++++++++++++++
\section{Sous espaces affines}
%+++++++++++++++++++++++++++++++++++++++++++++++++++++++++++++++++++++++++++++++++++++++++++++++++++++++++++++++++++++++++++

\begin{definition}
    Soit \( \affE\) un espace affine sur l'espace vectoriel \( E\). Un \defe{sous espace affine}{affine!sous-espace} de \( \affE\) est une orbite de l'action d'un sous-espace vectoriel de \( E\).
\end{definition}

Si \( \affF\) est un sous ensemble de \( \affE\), il sera un sous espace affine de \( \affE\) si et seulement si l'ensemble
\begin{equation}
    F=\{ AB\tq A,B\in\affF \}
\end{equation}
est un sous espace vectoriel de \( E\). Dans ce cas nous disons que \( F\) est la \defe{direction}{direction!sous espace affine} de \( \affF\). Si \( A\in\affF\), alors l'orbite de \( A\) sous \( F\) est \( \affF\). La \defe{dimension}{dimension!sous espace affine} de \( \affF\) est la dimension de sa direction.

Si \( \affF\) et \( \affG\) sont des sous espaces affines de \( \affE\) de directions \( F\) et \( G\), nous disons que \( \affF\) est \defe{parallèle}{parallèle!sous espaces affines} à \( \affG\) si \( F\subset G\).

\begin{proposition}
    Soit \( \affF\) un sous espace affine de dimension \( k\) dans l'espace affine \( \affE\) de dimension \( n\). Alors il existe une application affine \( f\colon \affE\to \eK^{n-k}\) telle que \( \affF=f^{-1}(0)\).
\end{proposition}

\begin{proof}
    Soit \( F\) la direction de \( \affF\) et \( A\in\affF\). Nous considérons une base \( \{ e_i \}\) adaptée à \( F\) au sens \( \{ e_1,\ldots, e_k \}\) est une base de \( F\). Nous considérons maintenant le repère cartésien \( (A,\{ e_i \})\) avec \( A\in\affF\) et nous construisons l'application affine
    \begin{equation}
        \begin{aligned}
            f\colon \affE&\to \eK^{n-k} \\
            A+\sum_{i=1}^nx_ie_i&\mapsto \begin{pmatrix}
                x_{k+1}    \\ 
                \vdots    \\ 
                x_n    
            \end{pmatrix}.
        \end{aligned}
    \end{equation}
    Par construction nous avons \( f(M)=0\) si et seulement si \( M\in\affF\).
\end{proof}

\begin{proposition}
    Soit \( \affE\) un espace affine de dimension \( n\) sur \( \eK\), soit \( f\colon \affE\to \eK^r\) une fonction affine. Pour tout \( a=(a_1,\ldots, a_r)\in \eK^r\), l'ensemble \( f^{-1}(a)\) est un sous espace affine de dimension \( \dim\ker(u_f)\).
\end{proposition}

\begin{proof}
    Nous considérons le repère \( (A,\{ e_i \})\) de \( \affE\). Étant donné que \( f\) est affine nous avons
    \begin{equation}
        f\big( A+\sum_ix_ie_i \big)=f(A)+u_f\big( \sum_ix_ie_i \big).
    \end{equation}
    Nous avons donc \( f\big( A+\sum_ix_ie_i \big)=a\) lorsque
    \begin{equation}
        u_f(\sum_ix_ie_i)=a-f(A).
    \end{equation}
    Nous avons donc
    \begin{equation}
        f^{-1}(a)=A+(u_f)^{-1}\big( a-f(A) \big),
    \end{equation}
    dont la dimension est le rang de \( (u_f)^{-1}=u_{f^{-1}}\) (proposition \ref{PropxtFeDE}). Le rang de \( (u_f)^{-1}\) est le dimension du noyau de \( u_f\).
\end{proof}
