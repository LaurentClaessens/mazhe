
%%%%%%%%%%%%%%%%%%%%%%%%%%
%
   \section{Sobolev spaces}\label{sec_Sobolev}
%
%%%%%%%%%%%%%%%%%%%%%%%%

A lot of theory about Sobolev spaces can be found in \cite{Maslov,Taylor_PDO}. The \defe{Fourier transform}{Fourier transform} of a function $\varphi$ on $\eR^n$ is defined by formulas
\begin{subequations}
\begin{align}
  (F\varphi)(p)=\hat\varphi(p)&=\int_{\eR^N} e^{-2i\pi p\cdot x}\varphi(x)\,dx\\
	\varphi(x)&=\int_{\eR^{N}} e^{2i\pi}x\cdot p\hat\varphi(p)\,dp
\end{align}
\end{subequations}
Main properties of Fourier transform are
\begin{subequations} \label{subeq_prop_Four}
\begin{align}
(\partial_jF\varphi)(p)&=-2i\pi F(x_j\varphi)\\
	(F\partial_j\varphi)&=2i\pi p_j(F\varphi)(p)
\end{align}
\end{subequations}

The associated formula for the delta Dirac ``function'' is 
\begin{equation}
  \int e^{2i\pi k\cdot x}\,dx=\delta(x).
\end{equation}

\begin{proposition}
If $\hat\swS$ denote the set of functions $\hat\varphi$ with $\varphi\in\swS$, then $\hat\swS=\swS$. 
\end{proposition}

\begin{proof}
No proof
\end{proof}

We consider the \defe{Laplace operator}{Laplace operator}\nomenclature{$\Delta f$}{Laplace operator}, or Laplacian,
\begin{equation}
\Delta=\sum_{j=1}^{N}\partial_j^2
\end{equation}
When $k\in\eN$, we consider the following norm on $\swS(\eR^N)$:
\begin{equation} \label{eq_def_norm_Sob}
  \| \varphi \|_{H^k}^2=\int_{\eR^N}\overline{\varphi}(x)[1-(2\pi)^{-2}\Delta]^k\varphi(x)\,dx
\end{equation}
The \defe{Sobolev space}{Sobolev space} $H^k_2(\eR^N)$\nomenclature{$H^k_2(\eR^N)$}{Sobolev space} is the completed of $\swS$ for this norm.

\begin{proposition}
These Sobolev spaces are Hilbert spaces
\end{proposition}

\begin{proof}
No proof
\end{proof}

In order to define Sobolev spaces $H^k$ with $k<0$, we have to find a definition for $[-\Delta+1]^{-l}$. We define
\[ 
  \swS^{(l)}=\{ [1-(2\pi)^{-2}\Delta]^l\varphi\tq \varphi\in\swS \}
\]

\begin{lemma}
For each $\psi\in\swS^{(l)}$, there exists one and only one $\varphi\in\swS$ such that $(-\Delta+1)^l\varphi=\psi$.
\end{lemma}

This unique function $\varphi$ is naturally denoted by $(-\Delta+1)^{-l}$

\begin{proof}
We give the proof with $l=1$, the other are induction. When $\psi\in\swS^{(l)}$, existence is by definition true and only unicity is non trivial. Let $\varphi_1$ and $\varphi_2$ in $\swS$ such that 
\[ 
  [1-(2\pi)^{-2}\Delta]\varphi_1=[1-(2\pi)^{-2}\Delta]\varphi_2,
\]
the function $f=\varphi_1-\varphi_2$ fulfils $(-\Delta+1)f=0$ and thus
\[ 
  \int_{\eR^N}\overline{ f }(x)[1-(2\pi)^{-2}\Delta]f(x)\,dx=0.
\]
Since $f\to 0$ at infinity rapidly, an integration by part of the term containing $\Delta$ leads, up to some constants, to
\[ 
  \int \overline{ f }f-\int \overline{ f }\Delta f=\int | f |^2+\int | \nabla f |^2=0.
\]
This proves that $f=0$.
\end{proof}

Now, the norm \eqref{eq_def_norm_Sob} can be used to define the Sobolev space $H^{-l}$. Elements of $H^k$ ($k<0$) which are not functions are distributions. When $m\ge0$, formulas  \eqref{subeq_prop_Four} give
\begin{equation}  \label{eq_umdpi_spi}
\left( F[1-(2\pi)^{-2}\Delta]^m\psi \right)(p)=\left( (1+p^2)^m\hat\psi \right)(p).
\end{equation}

\begin{proposition}
When $\psi\in\swS^{(m)}$, equality  \eqref{eq_umdpi_spi} holds even for $m<0$. 
\end{proposition}


Let us point out that $m$ keep integer; the general real case will be treated later.

\begin{proof}
Let $m=-k<0$; from definition of the space $\swS^{(m)}$, the function $\varphi=[1-(2\pi)^{-2}\Delta]^{-k}\psi$ exists; we have
\begin{equation}
\hat\psi=\left( F[1-(2\pi)^{-2}\Delta]^k\varphi \right)(p)
	=(p^2+1)^k\hat\varphi(p),
\end{equation}
therefore $\hat\varphi(p)=(p^2+1)^{-k}\hat\psi(p)$. Replacing $\varphi$ by its definition,
\begin{equation}
\left( F[1-(2\pi)^{-2}\Delta]^{-k}\psi \right)(p)=(p^2+1)^{-k}\hat\psi(p).
\end{equation}


\end{proof}

\begin{proposition}
Let $\varphi\in\swS$. There exists a $\psi\in\swS$ such that 
\[ 
  \varphi=[1-(2\pi)^{-2}\Delta]^l\psi
\]

\end{proposition}

\begin{proof}
Let us prove it with $l=1$; other cases are obtained by iteration. We consider the function $\psi$ defined by the condition
\[ 
  \hat\psi(p)=(p^2+1)^{-1}(F\varphi)(p).
\]
For this function we have $[1-(2\pi)^{-2}\Delta]\psi=\varphi$
\end{proof}

The space $\hat H^s(\eR^N)$ is the completed of $\swS$ for the norm
\begin{equation}
\| \varphi \|^2_{\hat H^s}=\int_{\eR^N}(p^2+1)^2| \varphi(p) |^2\,dx
\end{equation}
where $s$ is any positive real.


\begin{theorem}
For each $\varphi\in\swS$ and $k\in\eN$, we have
\[ 
  \| F\varphi \|_{\hat H^k}=\| \varphi \|_{H^k},
\]
in other words, the Fourier transform in $\swS$ is an isometry $\dpt{F}{ H^k }{ \hat H^k }$.

\end{theorem}


\begin{proof}
Using Parseval and equality \eqref{eq_umdpi_spi},
\begin{equation}
\| \varphi \|_{H^k}^2=\int_{\eR^N}\overline{ \varphi }(x)[1-(2\pi)^{-2}\Delta]^k\varphi(x)\,dx
		=\int \overline{ F\varphi(p) }(p^2+1)^k(F\varphi)(p)\,dp\\
		=\| \varphi \|_{\hat H^k}^2.
\end{equation}
\end{proof}
The space $\swS$ is dense in $H^k$ and $F$ is a bounded operator on $\swS$ (because it is isometric). Therefore it can be extended to an unique isometric homomorphism $\dpt{ F }{ H^k }{ \hat H^k }$ which is evidently called \defe{Fourier transform}{Fourier transform}.


In the same way, $F^{-1}$ is extended to an inverse of $F$ and finally,
\[ 
  \dpt{ F }{ H^k(\eR^N) }{ \hat H^k(\eR^N) }
\]
is  an isometric isomorphism. For each $s\in\eR$, we define
\begin{equation}
  \| \varphi \|_{H^s}:=\| F\varphi \|_{\hat H^s},
\end{equation}
and the completed of $\swS$ for this norm is the \emph{Sobolev space} $H^s(\eR^N)$.
