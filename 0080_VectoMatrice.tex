% This is part of Mes notes de mathématique
% Copyright (c) 2011-2015
%   Laurent Claessens
% See the file fdl-1.3.txt for copying conditions.

%+++++++++++++++++++++++++++++++++++++++++++++++++++++++++++++++++++++++++++++++++++++++++++++++++++++++++++++++++++++++++++
\section{Méthode de Gauss pour résoudre des systèmes d'équations linéaires}
%+++++++++++++++++++++++++++++++++++++++++++++++++++++++++++++++++++++++++++++++++++++++++++++++++++++++++++++++++++++++++++


Pour résoudre un système d'équations linéaires, on procède comme suit:
\begin{enumerate}
\item Écrire le système sous forme matricielle. \[\text{p.ex. } \begin{cases} 2x+3y &= 5 \\ x+2y &= 4 \end{cases} \Leftrightarrow \left(\begin{array}{cc|c} 2 & 3 & 5 \\ 1 & 2 & 4 \end{array}\right) \]
\item Se ramener à une matrice avec un maximum de $0$ dans la partie de gauche en utilisant les transformations admissibles:
\begin{enumerate}
\item Remplacer une ligne par elle-même + un multiple d'une autre;
\[\text{p.ex. } \left(\begin{array}{cc|c} 2 & 3 & 5 \\ 1 & 2 & 4 \end{array}\right)  \stackrel{L_1  - 2. L_2 \mapsto L_1'}{\Longrightarrow} \left(\begin{array}{cc|c} 0 & -1 & -3 \\ 1 & 2 & 4 \end{array}\right) \]
\item Remplacer une ligne par un multiple d'elle-même;
\[\text{p.ex. } \left(\begin{array}{cc|c} 0 & -1 & -3 \\ 1 & 2 & 4 \end{array}\right)  \stackrel{-L_1  \mapsto L_1'}{\Longrightarrow} \left(\begin{array}{cc|c} 0 & 1 & 3 \\ 1 & 2 & 4 \end{array}\right) \]
\item Permuter des lignes.
\[\text{p.ex. } \left(\begin{array}{cc|c} 0 & 1 & 3 \\ 1 & 0 & -2 \end{array}\right)  \stackrel{L_1  \mapsto L_2' \text{ et } L_2  \mapsto L_1'}{\Longrightarrow} \left(\begin{array}{cc|c} 1 & 0 & -2 \\ 0 & 1 & 3  \end{array}\right) \]
\end{enumerate}
\item Retransformer la matrice obtenue en système d'équations.
\[\text{p.ex. }  \left(\begin{array}{cc|c} 1 & 0 & -2 \\ 0 & 1 & 3  \end{array}\right) \Leftrightarrow \begin{cases} x &= -2 \\ y &= 3 \end{cases}  \]
\end{enumerate}

\textbf{Remarques :} 
\begin{itemize}
\item Si on obtient une ligne de zéros, on peut l'enlever:
\[\text{p.ex. }  \left(\begin{array}{ccc|c} 3 & 4 & -2 & 2 \\ 4 & -1 & 3 & 0 \\ 0 & 0 & 0 & 0 \end{array}\right) \Leftrightarrow  \left(\begin{array}{ccc|c} 3 & 4 & -2 & 2 \\ 4 & -1 & 3 & 0 \end{array}\right) \]
\item Si on obtient une ligne de zéros suivie d'un nombre non-nul, le système d'équations n'a pas de solution:
\[\text{p.ex. }  \left(\begin{array}{ccc|c} 3 & 4 & -2 & 2 \\ 4 & -1 & 3 & 0 \\ 0 & 0 & 0 & 7 \end{array}\right) \Leftrightarrow  \begin{cases} \cdots \\ \cdots \\ 0x + 0y + 0z = 7 \end{cases} \Rightarrow \textbf{Impossible} \]
\item Si on moins d'équations que d'inconnues, alors il y a une infinité de solutions qui dépendent d'un ou plusieurs paramètres:
\[\text{p.ex. }  \left(\begin{array}{ccc|c} 1 & 0 & -2 & 2 \\ 0 & 1 & 3 & 0 \end{array}\right) \Leftrightarrow  \begin{cases} x - 2z = 2 \\ y + 3z = 0 \end{cases} \Leftrightarrow  \begin{cases} x = 2 + 2\lambda \\ y = -3\lambda \\ z = \lambda \end{cases} \]
\end{itemize}

%+++++++++++++++++++++++++++++++++++++++++++++++++++++++++++++++++++++++++++++++++++++++++++++++++++++++++++++++++++++++++++
\section{Orthogonalité}
%+++++++++++++++++++++++++++++++++++++++++++++++++++++++++++++++++++++++++++++++++++++++++++++++++++++++++++++++++++++++++++

\begin{proposition}			\label{PropVectsOrthLibres}
	si $v_1,\cdots,v_k$ sont des vecteurs non nuls, orthogonaux deux à deux, alors ces vecteurs forment une famille libre.
\end{proposition}
