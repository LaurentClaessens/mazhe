% This is part of Exercices et corrigés de CdI-1
% Copyright (c) 2011
%   Laurent Claessens
% See the file fdl-1.3.txt for copying conditions.

\begin{corrige}{Variete0004}

	La portion d'espace est donnée par les fonctions
	\begin{equation}
		\begin{aligned}[]
			G_1&=x-y\\
			G_2&=x-z\\
			G_3&=x'-1\\
			G_4&=y'.
		\end{aligned}
	\end{equation}
	Les gradients sont
	\begin{equation}
		\begin{aligned}[]
			\nabla G_1&=(1,-1,0,0,0,0)\\
			\nabla G_2&=(1,0,-1,0,0,0)\\
			\nabla G_3&=(0,0,0,1,0,0)\\
			\nabla G_4&=(0,0,0,0,1,0),
		\end{aligned}
	\end{equation}
	qui sont linéairement indépendants. Les candidats seront donc uniquement les points tels que $\nabla L=0$ où
	\begin{equation}
		\begin{aligned}[]
			L(x,y&,z,x',y',z',\lambda_1,\lambda_2,\lambda_3)\\
					&=(x-x')^2+(y-y')^2+(z-z')^2+\lambda_1(x-y)+\lambda_2(x-y)+\lambda_3(x'-1)+\lambda_4 y'.
		\end{aligned}
	\end{equation}
	La résolution du système d'équations est donnée par l'unique point
	\begin{equation}
		(x,y,z,x',y',z')=(\frac{ 1 }{2},\frac{ 1 }{2},\frac{ 1 }{2},1,0,\frac{ 1 }{2}),
	\end{equation}
	et la fonction vaut $\frac{1}{ 2 }$ en ce point.

	Si au lieu de regarder la fonction sur tout le domaine $S$, nous ne la regardons seulement sur la fermeture de l'intersection avec l'ensemble
	\begin{equation}
		C=\{ (x,y,z,x',y',z')\tq (x-x')^2+(y-y')^2+(z-z')^2\leq 10 \},
	\end{equation}
	alors il y a un minimum et un maximum. La fonction vaut $10$ sur le bord de $\overline{S\cap C}$, tandis qu'elle est plus petite que $10$ à l'intérieur. Le minimum n'est donc pas sur le bord. Il y a un seule candidat à être minimum à l'intérieur, c'est le point trouvé.

	Ce point est donc minimum global de $f$ relativement à l'ensemble $\overline{S\cap C}$, et même relativement à $S\cap C$. Le fait que ce soit un minimum local par rapport à $S$ découle maintenant du lemme \ref{LemmeMinSCimpliqueS}.

\end{corrige}
