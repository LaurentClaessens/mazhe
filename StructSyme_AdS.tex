\section{Symmetric space structure on anti de Sitter}\label{SecSymeStructAdS}
%------------------------------------------

The $l$-dimensional anti de Sitter space $AdS_l$ can be described as set of points $(u,t,x_1,\ldots,x_{l-1})\in \eR^{2,l-1}$  such that $u^2+t^2-x_1^2-\ldots-x_{l-1}^2=1$. The next few pages are devoted to describe the homogeneous space structure on $AdS_l$ induced by the transitive an isometric action of $\SO(2,l-1)$. We suppose that the groups $\SO(2,l-1)$ and $\SO(1,l-1)$ are parametrized in such a way that the second, seen as subgroup of the first one, leaves unchanged the vector $(1,0,\ldots,0)$. In this case, proposition 4.3 of chapter II in \cite{Helgason} provides the homogeneous space isomorphism
\begin{equation}
\begin{aligned}
  \SO(2,l-1)/\SO(1,l-1)&\to AdS_l \\ 
[g]&\mapsto  
 g\cdot
\begin{pmatrix}
1\\0\\\vdots
\end{pmatrix}
\end{aligned}
\end{equation}
where the dot denotes the action ``matrix times vector'' of the representative $g\in [g]$ in the defining representation of $\SO(2,l-1)$. As far as notations are concerned, the classes are taken from the right:  $[g]=\{gh\tq h\in H\}$; in particular the class of the identity $e$ is denoted by $\mfo$; the groups $\SO(2,l-1)$ and $\SO(1,l-1)$ are denoted by $G$ and $H$ respectively and their Lie algebras by $\sG$ and $\sH$. Following proposition \ref{PropGHconn}, we can in fact only consider the identity components of $G$ and $H$.

As far as dimensions are concerned, a candidate $R\subset G$ such that $R\cdot\mfo$ is open must satisfy
\begin{equation}\label{cond_dim}
                  \dim\mR\geq\dim M.
\end{equation}

The case that interest us is $G=\SOdn$ and $H=\SOun$:\nomenclature{$AdS_n$}{Anti de Sitter space}
\[
M=AdS_{n+1}=\dfrac{\SOdn}{\SOun},
\]
 so that we have to consider the action of $\SO(2,n)$ on $AdS_n$.  If $ANK$ is the Iwasawa decomposition of $\SO(2,n)$, we can consider more particularly the action of $R=AN$, and ask us if the orbit $R\cdot\mfo$ is open or not. It is easy to see that the condition \eqref{cond_dim} is satisfied. Indeed,
\[
 \dim\lG=\frac{n(n-1)}{2}+2n+1,\qquad\dim\lK=\frac{n(n-1)}{2}+1,
\]
so that $\dim(\mA\oplus\mN)=2n$, but $\dim AdS_n=n$. The Iwasawa subgroup\index{Iwasawa!group} $AN$ is a candidate for $AN\cdot\mfo$ to be open in $AdS_n$.

\begin{proposition}
The homogeneous space $AdS_l$ is reductive\index{reductive!$AdS_n$}.
\label{PropAdSreduct}
\end{proposition}

The proof relies on the following lemma and the fact that $\SO(2,n)$ is semisimple.
\begin{lemma} 
If $G$ is a semisimple Lie group and $H$ a semisimple subgroup of $G$, the restrictions on $H$ of the Killing form of $G$ is nondegenerate.
 \label{lem:Killing_ss_descent}
\end{lemma}

\begin{probleme}
Il faut une citation pour ce lemme.
\label{ProbCitLemDesc}
\end{probleme}

\begin{proof}[Proof of proposition \ref{PropAdSreduct}]
From the Killing form of $G$ , one defines
\[
   \sQ=\sH^{\perp}=\{X\in\sG:B(X,H)=0\,\forall H\in\sH\}.
\]
Let $H$, $H'\in\sH$ and $Y\in\sQ$. From $\ad$-invariance of the Killing form, we have $B([H,Y],H')=0$. Hence $(\ad(\sH)\sQ)\subset \sQ$ and the claim is proved.

\end{proof}

Matrices of $\SO(2,n)$ are $(2+n)\times(2+n)$ matrices while the $n$-dimensional anti de Sitter space is a quotient of $\SO(2,n-1)$. In order to avoid confusions, we will reserve the letter $n$ to the study of the group $\SO(2,n)$ and the letter $l$ will denote the dimension of the anti de Sitter space which will thus be $AdS_l$.

Note however that this proof of the reducibility don't give any explicit form for $\sQ$. Let us provide a matrix representation now. The matrices of $\soun$ seen as matrices of $\sodn$ fulfil $Y^t\sigma+\sigma Y=0$ for the ``metric'' $\sigma=diag(0,-,+,\ldots,+)$. Hence,
\begin{equation}\label{eq:gene_H}
\sH=\soun\leadsto
  \begin{pmatrix}
     \begin{matrix}
       0&0\\
       0&0
     \end{matrix}
                       &  \begin{pmatrix}
		             \cdots 0\cdots\\
			    \leftarrow v^t\rightarrow
                          \end{pmatrix}\\
    \begin{pmatrix}	  
       \vdots & \uparrow\\
         0    & v \\
       \vdots & \downarrow
    \end{pmatrix} &  B
  \end{pmatrix}
\end{equation}
where  $v\in M_{n\times 1}$ and $B\in M_{n\times n}$ is skew-symmetric. Comparing this with the general form of a matrix of $\sodn$ matrix, one immediately finds that, using
\begin{equation}\label{eq:gene_M}
\sQ\leadsto
 \begin{pmatrix}
     \begin{matrix}
       0&a\\
       -a&0
     \end{matrix}
                       &  \begin{pmatrix}		             
			  \leftarrow w^t\rightarrow \\
			     \cdots 0\cdots\\
                          \end{pmatrix}\\
    \begin{pmatrix}	  
      \uparrow   & \vdots\\
          w      &  0\\
      \downarrow & \vdots 
    \end{pmatrix} & 0
  \end{pmatrix},
 \end{equation}
the decomposition $\sG=\sH\oplus\sQ$ is reductive:
\begin{align}\label{EqDefRedHQ}
  [\sH,\sQ]&\subseteq\sQ,
 &[\sQ,\sQ]&\subseteq\sH,
\end{align}
and $B(\sH,\sQ)=0$. In the sequel, we will use the basis of $\sQ$ defined by 
\begin{align}		\label{EqDefBaseqi}
  q_0&=E_{12}-E_{21}, &q_i&=E_{1i}+E_{i1}.
\end{align}
We define the involutive automorphism $\sigma=\id|_{\sH}\oplus(-\id)|_{\sQ}$.  The vector space $\sQ$ can be identified with the tangent space $T_{[e]}AdS_l$, and that identification can be extended by defining $\sQ_g=dL_g\sQ$. In this case $\dpt{d\pi}{\sQ_g}{T_{[g]}AdS_l}$ is a vector space isomorphism.\label{PgdpibaseQTgM} A homogeneous metric on $T_{[g]}AdS_l$ is defined as in subsection \ref{SubsecKillHomo}.

  Cartan decomposition of $\SO(2,l-1)$ are of crucial importance in chapter~\ref{ChapBHinAdS}, so that we want to use a Cartan involution $\theta$ such that $[\sigma,\theta]=0$ (see \cite{Loos} page 153, theorem 2.1). One can show that $X\mapsto -X^t$ has that property. The corresponding Cartan decomposition is described in appendix \ref{SubSecCartandeuxN}.


As a consequence of relations \eqref{EqDefRedHQ}, 
\begin{equation}  \label{EqdpiAdpi}
d\pi\Ad(h)=\Ad(h) d\pi
\end{equation}
because, if $X\in\sQ$, $d\pi^{-1}(X)=\{ X+Y\tq Y\in\sH \}$, so $\Ad(h)Y\in\sH$ and $\Ad(h)X\in \sQ$.

\subsection{\texorpdfstring{$AdS_l$}{AdSl} as symmetric space}\index{symmetric!space}
%--------------------------------------
\label{pg:AdS_n_syme}

\begin{probleme}
On ne parle pas déjà de la structure symétrique aux pages \pageref{SecSymeStructAdS} et \pageref{pgsymcelabelpeutpartir} ?
\end{problemT}

We know the decomposition $\sodn=\sQ\oplus\sH$. From equation \eqref{EqDefRedHQ} one can find an involutive automorphism $\sigma$ of $\sG$ which leaves $\sH$ invariant. 

There exists a neighbourhood $U$ of $0$ in $\sodn$ on which $\exp$ is diffeomorphic to a neighbourhood $V$ of $e$ in $\SOdn$. We define $\dpt{\sigma_G}{V}{V}$ by $\sigma_G(e^X)=e^{\sigma X}$. Now, this $\sigma_G$ can be extended to the whole $G$. From now we will denote by $\sigma$ this map or its differential (i.e. an abuse of notation between $\sigma$ and $d\sigma_e$).

All this make $(\SOdn,\SOun)$ a symmetric pair. Since $H=\SOun$ is connected and fixed by $\sigma$, $H=H_{\sigma}=(H_{\sigma})_0$. Thus theorem \ref{tho:sigma_theta} gives us a Cartan involution $\theta$ on $\sG$ such that $[\sigma,\theta]=0$ and theorem \ref{tho:sym_homo} gives a symmetric structure to $M=G/H$. Now we understand the computations of page \pageref{pg:calcul_sigma_theta}.
