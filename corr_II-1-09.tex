% This is part of the Exercices et corrigés de CdI-2.
% Copyright (C) 2008, 2009
%   Laurent Claessens
% See the file fdl-1.3.txt for copying conditions.


\begin{corrige}{_II-1-09}

Nous suivons la méthode expliquée en \ref{SubSecRicatti}. La première chose à faire est de voir si nous pouvons deviner des solutions particulières.

\begin{enumerate}

\item 
Ici, nous voyons tout de suite que $y_1(t)=\sin(t)$ est une solution. Nous posons donc
\begin{equation}
	y(t)=\sin(t)+\frac{1}{ u },
\end{equation}
et nous trouvons l'équation différentielle
\begin{equation}		\label{EqII109DiffPouru}
	u'+\sin(t)u=1.
\end{equation}
Note : en suivant les notations du rappel théorique, nous avons
\begin{equation}
	\begin{aligned}[]
		a(t)	&=1,\\
		b(t)	&=-\sin(t),\\
		c(t)	&=\cos(t).
	\end{aligned}
\end{equation}
L'équation \eqref{EqII109DiffPouru} est du type de \eqref{EqDiffExempleVarCst}. Dans ces notations, nous cherchons donc $K$ qui vérifie l'équation $K'(t)=g(t)/u_H(t)$, c'est à dire
\begin{equation}
	K'(t)=-\frac{ 1 }{  e^{\cos(t)} }.
\end{equation}
Cette primitive n'est absolument pas simple à calculer. La solution à notre équation différentielle est donc 
\begin{equation}
	z(t)=- e^{\cos(t)}\int_0^t e^{-\cos(u)}du.
\end{equation}
Pour la petite histoire, l'\href{http://integrals.wolfram.com/index.jsp}{intégrateur en ligne} de Wolfram \href{http://reference.wolfram.com/mathematica/tutorial/IntegralsThatCanAndCannotBeDone.html}{ne trouve pas} de forme pour cette intégrale.

\item Cette fois, trouver des solutions particulières est plus compliqué. Nous les cherchons sous la forme $y=at^2+bt+d$. Nous n'essayons pas de substituer directement cela dans l'équation. Au lieu, nous essayons d'abord avec $b=d=0$, puis avec $a=b=0$ puis $a=d=0$ puis avec $a=0$, \emph{et c\ae tera} jusqu'à trouver deux solutions particulières.

Le résultat est que $y_1(t)=-t^2$ est solution, ainsi que $y_2(t)=1+t$. En suivant les notations de \eqref{EqDiffGFeneRicatti}, nous avons
\begin{equation}
	y'=\frac{1}{ 1-t^3 }y^2-\frac{ t }{ 1-t^3 }y-\frac{ 2t }{ 1-t^3 },
\end{equation}
c'est à dire
\begin{equation}
	\begin{aligned}[]
		a(t)	&=\frac{1}{ 1-t^3 }\\
		b(t)	&=-\frac{ t^2 }{ 1-t^3 }\\
		c(t)	&=-\frac{ 2t }{ 1-t^3 }.
	\end{aligned}
\end{equation}
La solution à notre équation se trouve donc sous forme implicite
\begin{equation}
	\frac{ y+t^2 }{ y-1-t }=K e^{I(t)}
\end{equation}
où
\begin{equation}
	I(t)=\int\frac{1}{ 1-t^3 }(-t^2-1-t).
\end{equation}
Affin de faire cette intégrale, une petite simplification s'impose :
\begin{equation}
	\frac{t^2+1+t)}{ 1-t^3 }=\frac{1}{ t-1 }.
\end{equation}
Dont $I(t)=\ln(t-1)$ et nous avons
\begin{equation}
	\frac{ y+t^2 }{ y-1-t }=K(t-1),
\end{equation}
que nous pouvons résoudre par rapport à $y$ :
\begin{equation}
	y(t)=\frac{ 1-kt^2 }{ k-t }.
\end{equation}
Notons que $y_2=1+t$ correspond à $k=1$, tandis que $y_1=-t^2$ correspond à $k\to\infty$.

\end{enumerate}

\end{corrige}
