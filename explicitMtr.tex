\section{Explicit matrix choices}\label{sec_calc}
%++++++++++++++++++++++++++++++++++++++++++++++++

We conclude this chapter by listing the different matrix choices that have been done during our $AdS$ excursion.

The first choice is to parameterize $\SO(2,n)$ and $\SO(1,n)$ in such a way the latter leaves unchanged the vector $(1,0,0,\ldots)$. Then
\begin{equation}  \label{eq:mtrH}
\sH=\soun\leadsto
  \begin{pmatrix}
     \begin{matrix}
       0&0\\
       0&0
     \end{matrix}
                       &  \begin{pmatrix}
                     \cdots 0\cdots\\
                \leftarrow v^t\rightarrow
                          \end{pmatrix}\\
    \begin{pmatrix}
       \vdots & \uparrow\\
         0    & v \\
       \vdots & \downarrow
    \end{pmatrix} &  B
  \end{pmatrix}.
\end{equation}
where  $v$ is $n\times 1$ and $B$ is skew symmetric $n\times n$.  When we speak about $\mathfrak{so}(n)$, we usually refer to the $B$ part of $\sH$.  A complementary space $\sQ$ such that $[\sH,\sQ]\subset\sQ$ is given by
\begin{equation}
\sQ\leadsto
 \begin{pmatrix}
     \begin{matrix}
       0&a\\
       -a&0
     \end{matrix}
                       &  \begin{pmatrix}
              \leftarrow w^t\rightarrow \\
                 \cdots 0\cdots\\
                          \end{pmatrix}\\
    \begin{pmatrix}
      \uparrow   & \vdots\\
          w      &  0\\
      \downarrow & \vdots
    \end{pmatrix} & 0
  \end{pmatrix}.
\end{equation}
We consider the involutive automorphism $\sigma=\id_{\sH}\oplus(-\id)_{\sQ}$ and the corresponding symmetric space structure on $\sG$. As basis of $\sQ$, we choice $q_0$ as the $2\times 2$ antisymmetric upper-left square and as $q_i$, the one obtained with $w$ full of zero apart a $1$ on the $i$th component.
Next we choice the Cartan involution $\theta(X)=-X^t$ which gives rise to a Cartan decomposition
\[
\sG=\sK\oplus\sP.
\]
The latter choice is made in such a way that $[\sigma,\theta]=0$.  It can be computed, but it is not astonishing that the compact part $\sK$ is made of  ``true'' rotations while $\sP$ contains the boost. So
\[
  \sK=
\begin{pmatrix}
  \sod\\
&\son
\end{pmatrix},
\]
where elements of $\SO(2)$ are represented as
\[ 
  \begin{pmatrix}
\cos\mu&\sin\mu\\
-\sin\mu&\cos\mu
\end{pmatrix}.
\]
A common abuse of notation in the text is to identify the angle $\mu$ with the element of $\SO(2)$ itself.

In order to build an Iwasawa decomposition, one has to choose a maximal abelian subalgebra $\sA$ of $\sP$. Since rotations are in $\sK$, they must be boosts and the fact that there are only two time-like directions restricts $\sA$ to a two dimensional algebra. Up to reparametrization, it is thus generated by $t\partial_x+x\partial_t$ and $u\partial_y+y\partial_t$. Our matrix choices are
\[
   J_1=
\begin{pmatrix}
&0\\
0&0&0&1\\
&0\\
&1
\end{pmatrix}\in\sH,
\textrm{ and }
J_2=q_1=
\begin{pmatrix}
0&0&1&0\\
0\\
1\\
0
\end{pmatrix}\in\sQ.
\]
From here, we have to build root spaces. There still remains a lot of arbitrary choices --among them, the positivity notion on the dual space $\sA^*$. An elements $X$ in $\sG_{(a,b)}$ fulfill $\ad(X)J_1=aJ_1$ and $\ad(X)J_2=bJ_2$. The symbol $E_{ij}$ denote the matrix full of zeros with a $1$ on the component $ij$. Results are
\begin{equation}
\sG_{(0,0)}\leadsto
\begin{pmatrix}
&&x&0\\
&&0&y\\
x&0\\
0&y\\
&&&& D
\end{pmatrix},
\end{equation}
where $D\in M_{(n-2)\times(n-2)}$ is skew-symmetric,
\begin{subequations}
\begin{align}
\sG_{(1,0)}&\leadsto W_i=E_{2i}+E_{4i}+E_{i2}-E_{i4},\\
\sG_{(-1,0)}&\leadsto Y_i=-E_{2i}+E_{4i}-E_{i2}-E_{i4},\\
\sG_{(0,1)}&\leadsto V_i=E_{1i}+E_{3i}+E_{i1}-E_{i3},\\
\sG_{(0,-1)}&\leadsto X_i=-E_{1i}+E_{3i}-E_{i1}-E_{i3}
\end{align}
\end{subequations}
with $\dpt{i}{5}{n+2}$ and
\begin{equation}
\sG_{(1,1)}\leadsto M=
\begin{pmatrix}
   0&1&0&-1\\
   -1&0&1&0\\
   0&1&0&-1\\
   -1&0&1&0
\end{pmatrix},
\quad
\sG_{(1,-1)}\leadsto L=
\begin{pmatrix}
   0&1&0&-1\\
   -1&0&-1&0\\
   0&-1&0&1\\
   -1&0&-1&0
\end{pmatrix},
\end{equation}

\begin{equation}
\sG_{(-1,1)}\leadsto N=
\begin{pmatrix}
   0&1&0&1\\
   -1&0&1&0\\
   0&1&0&1\\
   1&0&-1&0
\end{pmatrix},
\quad
\sG_{(-1,-1)}\leadsto F=
\begin{pmatrix}
   0&1&0&1\\
   -1&0&-1&0\\
   0&-1&0&-1\\
   1&0&1&0
\end{pmatrix}.
\end{equation}
The choice of positivity is
\begin{equation}
   \sN=\{V_i,W_j,M,L\}.
\end{equation}

The following result is important in the computation of the light cones: if $k\in \SO(n)$, then the choice $E=q_0+q_2$ of nilpotent element in $\sQ$ gives
\begin{equation} \label{eq:AdkE}
   \Ad(k)E=
\begin{pmatrix}
0&1&w_1&w_2&\ldots\\
-1\\
w_1\\
w_2\\
\vdots
\end{pmatrix}
\end{equation}
where $w$ is the first column of $k$, whose components are restricted by the condition $\sum_{i=1}^{l-1} \, w_i^2=1$.

