% This is part of Mes notes de mathématique
% Copyright (c) 2008-2015
%   Laurent Claessens
% See the file fdl-1.3.txt for copying conditions.

%+++++++++++++++++++++++++++++++++++++++++++++++++++++++++++++++++++++++++++++++++++++++++++++++++++++++++++++++++++++++++++
\section{Fonctions}		\label{Sect_fonctions}
%+++++++++++++++++++++++++++++++++++++++++++++++++++++++++++++++++++++++++++++++++++++++++++++++++++++++++++++++++++++++++++

Soient $(V,\| . \|_V)$ et $(W,\| . \|_W)$ deux espaces vectoriels normés, et une fonction $f$ de $V$ dans $W$. Il est maintenant facile de définir les notions de limites et de continuité pour de telles fonctions en copiant les définitions données pour les fonctions de $\eR$ dans $\eR$ en changeant simplement les valeurs absolues par les normes sur $V$ et $W$.

En nous inspirant de la définition \ref{DefLimiteFonction}, nous écrivons
\begin{definition}		\label{LimiteDansEVN}
	Soit $f\colon V\to W$ une fonction de domaine \( \Domaine(f)\subset V\) et soit $a$ un point d'accumulation de $\Domaine(f)$. Nous disons que $f$ \defe{admet une limite}{limite!espace vectoriel normé} en $a$ si il existe un élément $\ell\in W$ tel que pour tout $\varepsilon>0$, il existe un $\delta>0$ tel que pour tout $x\in \Domaine(f)$,
    \begin{equation}        \label{EqDefLimzxmasubV}
		0<\| x-a \|_V<\delta\,\Rightarrow\,\| f(x)-\ell \|_W<\varepsilon.
	\end{equation}
	Dans ce cas, nous écrivons $\lim_{x\to a} f(x)=\ell$ et nous disons que $\ell$ est la \defe{limite}{limite} de $f$ lorsque $x$ tend vers $a$.
\end{definition}

\begin{remark}
    Le fait que nous limitions la formule \eqref{EqDefLimzxmasubV} aux \( x\) dans le domaine de \( f\) n'est pas anodin. Considérons la fonction \( f(x)=\sqrt{x^2-4}\), de domaine \( | x |\geq 2\). Nous avons
    \begin{equation}
        \lim_{x\to 2} \sqrt{x^2-4}=0.
    \end{equation}
    Nous ne pouvons pas dire que cette limite n'existe pas en justifiant que la limite à gauche n'existe pas. Les points \( x<2\) sont hors du domaine de \( f\) et ne comptent dons pas dans l'appréciation de l'existence de la limite.

    Vous verrez plus tard que ceci provient de la \wikipedia{fr}{Topologie_induite}{topologie induite} de \( \eR\) sur l'ensemble \( \mathopen[ 2 , \infty [\).
\end{remark}

\begin{definition}\label{DefContDansEVN}
	Une fonction $f\colon D\subset V\to W$ entre deux espaces vectoriels normés $V$ et $W$ est dite \defe{continue}{continue!fonction sur espace vectoriel normé} au point $a\in\bar D$ si $f(x)$ admet une limite pour $x$ tendant vers $a$ et si $\lim_{x\to a} f(x)=f(a)$.
\end{definition}

Une caractérisation très importante des fonctions continues est que l'image inverse d'un ouvert par une fonction continue est ouverte.

\begin{theorem}		\label{ThoContiueImageInvOUvert}
	Soient $V$ et $W$ deux espaces vectoriels normés. Une fonction $f$ de $V$ vers $W$ est continue si et seulement si pour tout ouvert $\mO$ dans $W$, l'ensemble $f^{-1}(\mO)$ est ouvert dans $V$.
\end{theorem}

\begin{proof}
	Supposons d'abord que $f$ est continue. Soit $\mO$ un ouvert de $W$, et prouvons que $f^{-1}(\mO)$ est ouvert. Pour cela, nous allons prouver qu'autour de chaque point $x$ de $f^{-1}(\mO)$, il existe une boule contenue dans $f^{-1}(\mO)$. Nous notons $y=f(x)\in\mO$. Étant donné que $\mO$ est ouvert dans $W$, il existe un rayon $r$ tel que
	\begin{equation}
		B_W\big( f(x),r \big)\subset\mO.
	\end{equation}
	Nous avons ajouté l'indice $W$ pour nous rappeler que c'est une boule dans $W$. Mais la continuité de $f$ implique qu'il existe un rayon $\delta$ tel que $\| x-a \|_V<\delta$ implique $\big\| f(x)-f(a) \big\|_W<r$. Ayant choisit un tel $\delta$, nous savons que si $a\in B_V(x,\delta)$, alors $f(a)\in B_W\big( f(x),r \big)\subset \mO$. Dans ce cas, $a\in f^{-1}(\mO)$. Nous avons donc montré que $B_V(x,\delta)\subset f^{-1}(\mO)$, ce qui prouve que $f^{-1}(\mO)$ est ouvert.

	Supposons maintenant que pour tout ouvert $\mO$ de $W$, l'ensemble $f^{-1}(\mO)$ est ouvert. Nous allons montrer qu'alors $f$ est continue. Soit $x\in V$ et $\varepsilon>0$. Nous devons trouver $\delta$ tel que $0<\| x-a \|_V<\delta$ implique $\| f(a)-f(x) \|_W<\varepsilon$.

	Considérons la boule ouverte $\mO=B_W\big( f(x),\varepsilon \big)$, et son image inverse $f^{-1}(\mO)$ qui est également ouverte par hypothèse. Étant donné que $f(x)\in\mO$, nous avons évidemment $x\in f^{-1}(\mO)$ et donc il existe une boule centrée en $x$ et contenue dans $f^{-1}(\mO)$. Soit $\delta$ le rayon de cette boule :
	\begin{equation}
		B_V\big( x,\delta \big)\subset f^{-1}(\mO).
	\end{equation}
	Par définition de l'image inverse, nous avons aussi $g\big( B_V(x,\delta) \big)\subset\mO$. En récapitulant,
	\begin{equation}
		\| x-a \|_V<\delta\Rightarrow a\in B_V(x,\delta)\Rightarrow f(a)\in\mO=B_W\big( f(x),\varepsilon \big)\Rightarrow\| f(a)-f(x) \|_W<\varepsilon.
	\end{equation}
	Ceci conclut la preuve.
\end{proof}

\begin{remark}
	Cette propriété des fonctions continues est tellement importante qu'elle est souvent prise comme définition de la continuité.
\end{remark}

Un résultat important dans la théorie des fonctions sur les espaces vectoriels normés est qu'une fonction continue sur un compact est bornée et atteint ses bornes. Ce résultat sera (dans d'autres cours) énormément utilisé pour trouver des maxima et minima de fonctions. Le théorème exact est le suivant.

\begin{theorem}		\label{WeierstrassEVN}
	Soit $K\subset V$ une partie compacte (fermée et bornée) d'un espace vectoriel normé $v$. Si $f\colon K\subset V\to \eR$ est une fonction continue, alors $f$ est bornée, et atteint ses bornes. 
	
	C'est à dire qu'il existe $x_0\in K$ tel que $f(x_0)=\inf\{ f(x)\tq x\in K \}$ ainsi que $x_1$ tel que $f(x_1)=\sup\{ f(x)\tq x\in K \}$.
\end{theorem}

Ce résultat sera prouvé dans le théorème \ref{ThoWeirstrassRn} dans le cas particulier de $V=\eR^n$ et dans le théorème \ref{ThoMKKooAbHaro} pour le cas général. Nous n'allons donc pas donner de démonstration de ce théorème ici. Nous allons par contre donner la preuve d'un résultat un peu plus général.

\begin{proposition}		\label{PropContinueCompactBorne}
	Soient $V$ et $W$ deux espaces vectoriels normés. Soit $K$, une partie compacte de $V$, et $f\colon K\to W$, une fonction continue. Alors l'image $f(K)$ est compacte dans $W$.
\end{proposition}
Ce résultat est démontré dans un cadre plus général par le théorème \ref{ThoImCompCotComp}.

\begin{proof}
	Nous allons prouver que $f(K)$ est fermée et bornée.
	\begin{description}
		\item[$f(K)$ est fermé] Nous allons prouver que si $(y_n)$ est une suite convergente contenue dans $f(K)$, alors la limite est également contenue dans $f(K)$. Dans ce cas, nous aurons que l'adhérence de $f(K)$ est contenue dans $f(K)$ et donc que $f(K)$ est fermé. Pour chaque $n\in\eN$, le vecteur $y_n$ appartient à $f(K)$ et donc il existe un $x_n\in K$ tel que $f(x_n)=y_n$. La suite $(x_n)$ ainsi construite est une suite dans le fermé $K$ et possède donc une sous-suite convergente (proposition \ref{ThoBolzanoWeierstrassRn}). Notons $(x'_n)$ cette sous-suite convergente, et $a$ sa limite : $\lim(x'_n)=a\in K$. Le fait que la limite soit dans $K$ provient du fait que $K$ est fermé.

			Nous pouvons considérer la suite $f(x'_n)$ dans $W$. Cela est une sous-suite de la suite $(y_n)$, et nous avons $\lim f(x'_n)=a$ parce que $f$ est continue. Par conséquent nous avons
			\begin{equation}
				f(a)=\lim f(x'_n)=\lim y_n.
			\end{equation}
			Cela prouve que la limite de $(y_n)$ est dans $f(K)$ et par conséquent que $f(K)$ est fermé.

		\item[$f(K)$ est borné]
			Si $f(K)$ n'est pas borné, nous pouvons trouver une suite $(x_n)$ dans $K$ telle que
			\begin{equation}		\label{EqfxnWgeqn}
				\| f(x_n) \|_W>n
			\end{equation}
			Mais par ailleurs, l'ensemble $K$ étant compact (et donc fermé), nous avons une sous-suite $(x'_n)$ qui converge dans $K$. Disons $\lim(x'_n)=a\in K$. 
			
			Par la continuité de $f$ nous avons alors $f(a)=\lim f(x'_n)$, et donc
			\begin{equation}
				| f(a) |=\lim | f(x'_n) |.
			\end{equation}
			La suite $f(x'_n)$ est alors une suite bornée, ce qui n'est pas possible au vu de la condition \eqref{EqfxnWgeqn} imposée à la suite de départ $(x_n)$.
	\end{description}
\end{proof}

\begin{corollary}	\label{CorFnContinueCompactBorne}
	Une fonction $f\colon K\to \eR$ où $K$ est une partie compacte d'un espace vectoriel normé est toujours bornée.
\end{corollary}

\begin{proof}
	En effet, la proposition \ref{PropContinueCompactBorne} montre que $f(K)$ est compact et donc borné.
\end{proof}


%+++++++++++++++++++++++++++++++++++++++++++++++++++++++++++++++++++++++++++++++++++++++++++++++++++++++++++++++++++++++++++
\section{Produit fini d'espaces vectoriels normés}\label{sec_prod}
%+++++++++++++++++++++++++++++++++++++++++++++++++++++++++++++++++++++++++++++++++++++++++++++++++++++++++++++++++++++++++++

Dans cette sections nous parlons de produits finis d'espaces. Cela ne signifie pas que chacun des espaces soient séparément de dimension finie.

%---------------------------------------------------------------------------------------------------------------------------
\subsection{Norme}
%---------------------------------------------------------------------------------------------------------------------------

La définition de la norme sur un produit d'espaces vectoriels normés découle immédiatement de la définition de la distance \ref{DefZTHxrHA} :
\begin{definition}  \label{DefFAJgTCE}
    Soient $V$ et $W$ deux espaces vectoriels normés. On appelle \defe{espace produit}{produit!d'espaces vectoriels normés} de $V$ et $W$ le produit cartésien $V\times W$ 
    \begin{equation}
    V\times W=\{(v,w)\,|\, v\in V,\, w\in W\},
    \end{equation}
    muni de la norme $\|\cdot \|_{V\times W}$
    \begin{equation}	\label{EqNormeVxWmax}
        \|(v,w) \|_{V\times W}=\max\{\|v\|_{V},\|w\|_W\}.
    \end{equation}
\end{definition}
Il est presque immédiat de vérifier que le produit cartésien $V\times W$ est un espace vectoriel pour les opération de somme et multiplication par les scalaires définies composante par composante. C'est à dire,  si $(v_1,w_1)$, $(v_2,w_2)$ sont dans $V\times W$ et $a$, $b$ sont des scalaires, alors  
\begin{equation}
 a (v_1,w_1)+ b(v_2,w_2)=(av_1,aw_1)+ (bv_2,bw_2)=(av_1+bv_2,aw_1+bw_2).
\end{equation}

\begin{lemma}
	L'opération $\|\cdot \|_{V\times W}\colon V\times W\to \eR$ est une norme.
\end{lemma}

\begin{proof}
	On doit vérifier les trois conditions de la définition \ref{DefNorme}.
	\begin{itemize}
		\item Soit $(v,w)$ dans $V\times W$ tel que $\|(v,w)\|_{V\times W}=\max\{\|v\|_{V},\|w\|_W\}=0$. Alors $\|v\|_V=0$ et $\|w\|_W=0$, donc $v=0_V$ et $w=0_W$. Cela implique $(v,w)=(0_v,0_w)=0_{V\times W}$. 
		\item Pour tout $a$ dans $\eR$ et $(v,w)$ dans $V\times W$,  la norme $\|a (v,w)\|_{V\times W}$ est donnée par  $\max\{\|av\|_{V},\|aw\|_W\}$. On peut factoriser $\|av\|_{V}=|a|\|v\|_{V}$ et $\|aw\|_W=|a|\|w\|_W$ et donc $\|a (v,w)\|_{V\times W}=|a|\max\{\|v\|_{V},\|w\|_W\}=|a|\|(v,w)\|_{V\times W}$.
		\item Soient $(v_1,w_1)$ et $(v_2,w_2)$ dans $V\times W$. 
		\begin{equation}
			\begin{aligned}
				\|(v_1,w_1)+(v_2,w_2)\|_{V\times W}&=\max\{\|v_1+v_2\|_{V},\|w_1+w_2\|_W\}\\
				&\leq \max\{\|v_1\|_V+\|v_2\|_{V},\|w_1\|_W+\|w_2\|_W\}\\
				&\leq\max\{\|v_1\|_V,\|w_1\|_W\}+ \max\{\|v_2\|_{V},\|w_2\|_W\}\\
				&=\|(v_1,w_1)\|_{V\times W}+\|(v_2,w_2)\|_{V\times W}.
			\end{aligned}
		\end{equation}
	\end{itemize} 
\end{proof}
On remarque tout de suite que la norme $\|\cdot\|_\infty$ sur $\eR^2$ est la norme de l'espace produit $\eR\times\eR$. En outre cette définition nous permet de trouver plusieurs nouvelles normes dans les espaces $\eR^p$. Par exemple, si nous écrivons $\eR^4$ comme $\eR^2\times \eR^2$ on peut munir $\eR^4$ de la norme produit
\[
\|(x_1,x_2,x_3,x_4)\|_{\infty, 2}=\max\{\|(x_1,x_2)\|_\infty, \|(x_3,x_4)\|_2\}. 
\]    
Les applications de projection de l'espace produit $V\times W$ vers les espaces <<facteurs>>, $V$ $W$ sont notées $\pr_V$ et $\pr_W$ et sont définies par
\begin{equation}
	\begin{aligned}
		\pr_V\colon V\times W&\to V \\
		(v,w)&\mapsto v 
	\end{aligned}
\end{equation}
et
\begin{equation}
	\begin{aligned}
		\pr_W\colon V\times W &\to W \\
		(v,w)&\mapsto w. 
	\end{aligned}
\end{equation}
Les inégalités suivantes sont évidentes
\begin{equation}
	\begin{aligned}[]
		\|\pr_V(v,w)\|_V&\leq \|(v,w)\|_{V\times W} \\
		\|\pr_W(v,w)\|_W&\leq \|(v,w)\|_{V\times W}.
	\end{aligned}
\end{equation}
La topologie de l'espace produit est induite par les topologies des espaces <<facteurs>>. La construction est faite en deux passages : d'abord nous disons que une partie $A\times B$ de $V\times W$ est ouverte si $A$ et $B$ sont des parties ouvertes de $V$ et de $W$ respectivement.  Ensuite nous définissons que une partie quelconque de $V\times W$ est ouverte si elle est une intersection finie ou une réunion de parties ouvertes de $V\times W$ de la forme $A\times B$. 

Ce choix de topologie donne deux propriétés utiles de l'espace produit 
\begin{enumerate}
	\item
		Les projections sont des \defe{applications ouvertes}{application!ouverte}. Cela veut dire que l'image par $\pr_V$ (respectivement $\pr_W$) de toute partie ouverte de $V\times W$ est une partie ouverte de $V$ (respectivement $W$). 
	\item 
		Pour toute partir $A$ de $V$ et $B$ de $W$, nous avons $\Int (A\times B)=\Int A\times \Int B$.\label{PgovlABeqbAbB}
\end{enumerate}
Une propriété moins facile a prouver est que pour toute partie $A$ de $V$ et $B$ de $W$ nous avons  $\overline{A\times B}=\bar{A}\times \bar{B}$. Voir le lemme \ref{LemCvVxWcvVW}.
% position 26329
%et l'exercice \ref{exoGeomAnal-0009}.
  
Ce que nous avons dit jusqu'ici est valable pour tout produit d'un nombre fini d'espaces vectoriels normés. En particulier, pour tout $m>0$  l'espace  $\eR^m$ peut être considéré comme le produit de $m$ copies de $\eR$. 

\begin{example}
	Si $V$ et $W$ sont deux espaces vectoriels, nous pouvons considérer le produit $E=V\times W$. Les projections $\pr_V$ et $\pr_W$\nomenclature{$\pr_V$}{projection de $V\times W$ sur $V$}, définies dans la section \ref{sec_prod}, sont des applications linéaires. 

	En effet, la projection $\pr_V\colon V\times W\to V$ est donnée par $\pr_V(v,w)=v$. Alors,
	\begin{equation}
		\begin{aligned}[]
			\pr_V\big( (v,w)+(v',w') \big)&=\pr_V\big( (v+v'),(w+w') \big)\\
			&=v+v'\\
			&=\pr_V(v,w)+\pr_V(v',w'),
		\end{aligned}
	\end{equation}
	et
	\begin{equation}
		\pr_V\big( \lambda(v,w) \big)=\pr_V\big( (\lambda v,\lambda w) \big)=\lambda v=\lambda\pr_V(v,w).
	\end{equation}
	Nous laissons en exercice le soin d'adapter ces calculs pour montrer que $\pr_W$ est également une projection.
\end{example}

\begin{proposition} \label{PropDXR_KbaLC}
    Si \( \mO\) est un voisinage de \( (a,b)\) dans \( V\times W\) alors \( \mO\) contient un ouvert de la forme \( B(a,r)\times B(b,r)\).
\end{proposition}

\begin{proof}
    Vu que \( \mO\) est un voisinage, il contient un ouvert et donc une boule
    \begin{equation}
        B\big( (a,b),r \big)=\{ (v,w)\in V\times W\tq \max\{ \| v-a \|,\| w-b \| \}< r \}.
    \end{equation}
    Évidemment l'ensemble \( B(a,r)\times B(b,r)\) est dedans.
\end{proof}

%---------------------------------------------------------------------------------------------------------------------------
\subsection{Suites}
%---------------------------------------------------------------------------------------------------------------------------

Nous allons maintenant parler de suites dans $V\times W$. Nous noterons $(v_n,w_n)$ la suite dans $V\times W$ dont l'élément numéro $n$ est le couple $(v_n,w_n)$ avec $v_n\in V$ et $w_n\in W$. La notions de convergence de suite découle de la définition de la norme via la définition usuelle \ref{DefCvSuiteEGVN}. Il se fait que dans le cas des produits d'espaces, la convergence d'une suite est équivalente à la convergence des composantes. Plus précisément, nous avons le lemme suivant.
\begin{lemma}		\label{LemCvVxWcvVW}
	La suite $(v_n,w_n)$ converge vers $(v,w)$ dans $V\times W$ si et seulement les suites $(v_n)$ et $(w_n)$ convergent séparément vers $v$ et $w$ respectivement dans $V$ et $W$. 
\end{lemma}

\begin{proof}
	Pour le sens direct, nous devons étudier le comportement de la norme de $(v_n,w_n)-(v,w)$ lorsque $n$ devient grand. En vertu de la définition de la norme dans $V\times W$ nous avons
	\begin{equation}
		\Big\| (v_n,w_n)-(v,w) \Big\|_{V\times W}=\max\big\{ \| v_n-v \|_V,\| w_n-w \|_W \big\}.
	\end{equation}
	Soit $\varepsilon>0$. Par définition de la convergence de la suite $(v_n,w_n)$, il existe un $N\in\eN$ tel que $n>N$ implique
	\begin{equation}
		\max\big\{ \| v_n-v \|_V,\| w_n-w \|_W \big\}<\varepsilon,
	\end{equation}
	et donc en particulier les deux inéquations
	\begin{subequations}
		\begin{align}
			\| v_n-v \|&<\varepsilon\\
			\| w_n-w \|&<\varepsilon.
		\end{align}
	\end{subequations}
	De la première, il ressort que $(v_n)\to v$, et de la seconde que $(w_n)\to w$.

	Pour le sens inverse, nous avons pour tout $\varepsilon$ un $N_1$ tel que $\| v_n-v \|_V\leq\varepsilon$ pour tout $n>N_1$ et un $N_2$ tel que $\| w_n-w \|_W\leq\varepsilon$ pour tout $n>N_2$. Si nous posons $N=\max\{ N_1,N_2 \}$ nous avons les deux inégalités simultanément, et donc
	\begin{equation}
		\max\big\{ \| v_n-v \|_V,\| w_n-w \|_W \big\}<\varepsilon,
	\end{equation}
	ce qui signifie que la suite $(v_n,w_n)$ converge vers $(v,w)$ dans $V\times W$.
\end{proof}

\begin{remark}		\label{RemTopoProdPasRm}
	Il faut remarquer que la norme \eqref{EqNormeVxWmax} est une norme \emph{par défaut}. C'est la norme qu'on met quand on ne sait pas quoi mettre. Or il y a au moins un cas d'espace produit dans lequel on sait très bien quelle norme prendre : les espaces $\eR^m$. La norme qu'on met sur $\eR^2$ est
	\begin{equation}
		\| (x,y) \|=\sqrt{x^2+y^2},
	\end{equation}
	et non la norme «par défaut» de $\eR^2=\eR\times\eR$ qui serait
	\begin{equation}
		\| (x,y) \|=\max\{ | x |,| y | \}.
	\end{equation}
	Les théorèmes que nous avons donc démontré à propos de $V\times W$ ne sont donc pas immédiatement applicables au cas de $\eR^2$.

	Cette remarque est valables pour tous les espaces $\eR^m$. À moins de mention contraire explicite, nous ne considérons jamais la norme par défaut \eqref{EqNormeVxWmax} sur un espace $\eR^m$.
\end{remark}

Étant donné la remarque \ref{RemTopoProdPasRm}, nous ne savons pas comment calculer par exemple la fermeture du produit d'intervalle $\mathopen] 0,1 ,  \mathclose[\times\mathopen[ 4 , 5 [$. Il se fait que, dans $\eR^m$, les fermetures de produits sont quand même les produits de fermetures.

\begin{proposition}		\label{PropovlAxBbarAbraB}
	Soit $A\subset\eR^m$ et $B\subset\eR^m$. Alors dans $\eR^{m+n}$ nous avons $\overline{ A\times B }=\bar A\times \bar B$.
\end{proposition}

La démonstration risque d'être longue; nous ne la faisons pas ici.

%+++++++++++++++++++++++++++++++++++++++++++++++++++++++++++++++++++++++++++++++++++++++++++++++++++++++++++++++++++++++++++ 
\section{Applications multilinéaires}
%+++++++++++++++++++++++++++++++++++++++++++++++++++++++++++++++++++++++++++++++++++++++++++++++++++++++++++++++++++++++++++

\begin{definition}[Application multilinéaire]       \label{DefFRHooKnPCT}
    Une application $T: \eR^{m_1}\times \ldots \times\eR^{m_k}\to\eR^p $ est dite \defe{\( k\)-linéaire}{application!multilinéaire} si pour tout $X=(x_1, \ldots,x_k)$ dans $ \eR^{m_1}\times \ldots \times\eR^{m_k}$ les applications $x_i\mapsto T(x_1, \ldots, x_i,\ldots,x_k)$ sont linéaires pour tout $i$ dans $\{1,\ldots,k\}$, c'est à dire
	\begin{equation}
		\begin{aligned}[]
			T(\cdot,x_2, \ldots, x_i,\ldots,x_k)&\in \mathcal{L}(\eR^{m_1}, \eR^p),\\
			T(x_1,\cdot, \ldots, x_i,\ldots,x_k)&\in \mathcal{L}(\eR^{m_2}, \eR^p),\\
						& \vdots\\
			T(x_1, \ldots, x_i,\ldots,x_{k-1},\cdot)&\in \mathcal{L}(\eR^{m_k}, \eR^p).\\
		\end{aligned}
	\end{equation}
	En particulier lorsque $k=2$, nous parlons d'applications \defe{bilinéaires}{bilinéaire}. Vous pouvez deviner ce que sont les applications \emph{tri}linéaire ou \emph{quadri}linéaire.
\end{definition}

L'ensemble des applications $k$-linéaires de $ \eR^{m_1}\times \ldots \times\eR^{m_k}$ dans $\eR^p$ est noté $\mathcal{L}(\eR^{m_1}\times \ldots \times\eR^{m_k}, \eR^p)$ ou $\mathcal{L}(\eR^{m_1}, \ldots,\eR^{m_k}; \eR^p)$.

\begin{example}
  Soit $A$ une matrice avec $m$ lignes et $n$ colonnes. L'application bilinéaire de $\eR^m\times \eR^n$ dans $\eR$ associée à $A$ est définie par
\[
T_A(x,y)= x^TAy=\sum_{i,j}a_{i,j}x_i y_j, \qquad \forall x\in \eR^m, \, y \in \eR^n.
\]
\end{example}

Nous énonçons la proposition suivante dans le cas d'espaces vectoriels normés\footnote{Sans hypothèses sur la dimension.} parce que nous allons l'utiliser dans ce cas, mais le cas particulier \( E_i=\eR^{m_i}\) et \( F=\eR^p\) est important.
\begin{proposition} \label{PropUADlSMg}
    Soient des espaces vectoriels normés \( E_i\) et \( F\). Une application \( n\)-linéaire
    \begin{equation}
        T\colon E_1\times\ldots\times E_n\to F
    \end{equation}
    est est continue si et seulement s'il existe un réel $L\geq 0$ tel que
  \begin{equation}\label{limitatezza}
     \|T(x_1, \ldots,x_n)\|_F\leq L \|x_1\|_{F_1}\cdots\|x_n\|_{F_n}, \qquad \forall x_i\in E_i.
  \end{equation}
\end{proposition}

\begin{proof}
    Pour simplifier l'exposition nous nous limitons au cas $n=2$ et nous notons $T(x,y)=x*y$

    Supposons que l'inégalité \eqref{limitatezza} soit satisfaite. 
    \begin{equation}\label{LimImplCont}
      \begin{aligned}
        \|x*y-x_0*y_0\|&=\|(x-x_0)*y-x_0*(y-y_0)\|\\
    &\leq \|(x-x_0)*y\|+\|x_0*(y-y_0)\|\\
    &\leq L\|x-x_0\|\|y\| + L\|x_0\|\|y-y_0\|.
      \end{aligned}
    \end{equation}
    Si $x\to x_0$ et $y\to y_0$  on voit que $T$ est continue en passant à la limite aux deux côtes de l'inégalité \eqref{LimImplCont}.

    Soit $T$ continue en $(0,0)$. Évidemment\footnote{Dans la formule suivante, les trois zéros sont les zéros de trois espaces différents.} $0*0=0$, donc il existe $\delta>0$ tel que si $x\in B_{E_1}(0,\delta)$ et $y\in B_{E_2}(0,\delta)$ alors $\|x*y\|\leq 1$. En particulier si \( (x,y)\in B_{E_1\times E_2}(0,\delta)\) nous sommes dans ce cas. Soient maintenant  $x\in E_1\setminus\{ 0 \}$  et $y\in E_2\setminus\{ 0\}$
    \begin{equation}
        x*y=\left(\frac{\|x\|}{\delta}\frac{\delta x}{\|x\|}\right)*\left(\frac{\|y\|}{\delta}\frac{\delta y}{\|y\|}\right)
    =\frac{\|x\|\|y\|}{\delta^2} \left(\frac{\delta x}{\|x\|}\right)*\left(\frac{\delta y}{\|y\|}\right).
     \end{equation}
    On remarque que $\delta x/\|x\|_m$ est dans la boule de rayon $\delta$ centrée en $0_m$ et que $\delta y/\|y\|_n$ est dans la boule de rayon $\delta$ centrée en $0_n$. On conclut 
    \[
     x*y\leq \frac{\|x\|_m\|y\|_n}{\delta^2}.
    \]
    Il faut prendre $L=1/\delta^2$.
\end{proof}

La norme de \( T\) est alors définie comme la plus petite constante \( L\) qui fait fonctionner la proposition \ref{PropUADlSMg}.
\begin{definition}  \label{DefKPBYeyG}
	La norme sur l'espace $\aL(E_1\times \cdots\times E_n, F)$ des applications $k$-linéaires et continues est 
	\begin{equation}
        \|T\|_{E_1\times \ldots\times E_n}=\sup\{ \|T(u_1, \ldots,u_k)\|_{F}\,\vert\,\|u_i\|_{E_i}\leq 1, i=1,\ldots, k \}.
	\end{equation}
\end{definition}
Nous avons donc automatiquement
\begin{equation}    \label{EqYLnbRbC}
    \| T(u,v) \|\leq \| T \|\| u \|\| v \|.
\end{equation}
Et nous notons que cette norme est uniquement définie pour les applications linéaires continues. Ce n'est pas très grave parce qu'alors nous définissons \( \| T \|=\infty\) si \( T\) n'est pas continue. Cela pour retrouver le principe selon lequel on est continue si et seulement si on est borné.

\begin{proposition}\label{isom_isom}
  On définit les fonctions
  \begin{equation}
    \begin{array}{rccc}
      \omega_g: & \mathcal{L}(\eR^{m}\times\eR^{n}, \eR^p)&\to &\mathcal{L}(\eR^{m}, \mathcal{L}(\eR^{n}, \eR^p)),\\
      \omega_d: & \mathcal{L}(\eR^{m}\times\eR^{n}, \eR^p)&\to &\mathcal{L}(\eR^{n}, \mathcal{L}(\eR^{m}, \eR^p)),
    \end{array}
  \end{equation}
par 
\[
\omega_g(T)(x)=T(x,\cdot), \qquad \forall x\in\eR^m,
\]
et
\[
\omega_d(T)(y)=T(\cdot, y), \qquad \forall y\in\eR^n.
\]
Les fonctions $\omega_g$ et $\omega_d$ sont des isomorphismes qui préservent les normes.    
\end{proposition}

