% This is part of Exercices et corrigés de CdI-1
% Copyright (c) 2011
%   Laurent Claessens
% See the file fdl-1.3.txt for copying conditions.

\begin{corrige}{0092}

Soit $\epsilon>0$. Il faut trouver un $\delta$ tel que $| f(x)-f(y) |<\epsilon$ dès que $x$, $y\in I_1\cup I_2$ et $| x-y |<\delta$. Nous savons qu'il existe un tel $\delta_1$ pour $I_1$ et un $\delta_2$ pour $I_2$.

Dans ce genre d'exercice, il est d'usage de choisi $\delta=\min\{ \delta_1,\delta_2 \}$, afin que $\delta$ fonctionne à la fois pour $I_1$ et pour $I_2$. Cependant, cela n'est pas suffisant ici parce que nous n'avons aucune garantie sur ce qu'il se passerait sir $x\in I_1\setminus I_2$ et $y\in I_2\setminus I_2$ (c'est à dire si aucun des deux n'est dans l'intersection de $I_1$ et $I_2$).

La solution, pour empêcher cette situation est de prendre, en plus, $\delta$ plus petit que la moitié\footnote{En réalité, prendre exactement la taille de l'intersection est ne pose problème qu'au cas où les deux intervalles sont ouverts et que un point est à l'infimum de l'intersection et l'autre au supremum.} de la taille de l'intersection entre $I_1$ et $I_2$. De cette manière, deux éléments $x$ et $y$ tels que $| x-y |\leq\delta$ sont automatiquement tous les deux dans $I_1$ ou tous les deux dans $I_2$.

\end{corrige}
