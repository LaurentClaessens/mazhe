% This is part of Exercices et corrigés de CdI-1
% Copyright (c) 2011
%   Laurent Claessens
% See the file fdl-1.3.txt for copying conditions.

\begin{corrige}{0027}

\begin{enumerate}
\item $\lim_{k\to\infty}\left( \frac{ ak+1 }{ k } \right)^k$. 
Cette limite se calcule en essayant de se ramener à la limite qui définit l'exponentielle. Pour ce faire, nous calculons
\begin{equation}
	\begin{aligned}[]
		\lim_{k\to\infty}\left( \frac{ ak+1 }{ k } \right)^k	&=\lim_{k\to\infty}\left( a\big( 1+\frac{ 1/a }{ k } \big) \right)^k= e^{1/a}\lim_{k\to\infty}a^k.
	\end{aligned}
\end{equation}
Si $| a |<1$, alors $a^k\to 0$ et la limite est zéro. Si $a<-1$, l'alternance de signe empêche de trouver une limite. Dans ce cas, nous ne pouvons que déterminer une limite supérieure et une limite inférieure. Si $a>1$, nous avons $a^k\to\infty$ et donc la limite recherchée est l'infini.

Ici nous avons implicitement utilisé les résultats de l'exercice \ref{exo0010}.

\item $ \lim_{k \to  +\infty} \frac{k}{\sin(\frac{\pi}{5}k)+1} + \ln(k)\cos(\frac{\pi}{5}k) $.
Étant donné que $\sin(\pi k/5)$ n'est jamais égal à $-1$, le premier terme est toujours bien défini, et toujours positif. En réalité, ce premier terme est même toujours plus grand que $\frac{ k }{ 2 }$. Le limite que nous cherchons est donc plus grande que celle de $\frac{ k }{ 2 }+\ln(k)\cos\left( \frac{ \pi k }{ 5 } \right)$. Cette dernière limite vaut l'infini par la règle de l'étau :
\begin{equation}
	\frac{ k }{ 2 }-\ln(k)\leq\frac{ k }{2}+\ln(k)\cos\left( \frac{ k\pi }{ 5 } \right)\leq \frac{ k }{ 2 }+\ln(k).
\end{equation}
L'expression la plus à droite tend clairement vers l'infini, tandis que celle la plus à gauche tend également vers l'infini en vertu du travail fait en dessous de l'équation \eqref{Eq0023ffrac} dans la correction de la question \ref{Item0023h} l'exercice \ref{exo0023}.

\item
$ \lim_{k \to  +\infty} \frac{ \ln(k)\big(\sin(\frac{\pi}{3}k) +1\big)}{k} $.
Si nous majorons la quantité $\sin(x)+1$ par $2$, et nous trouvons
\begin{equation}
	\lim_{k\to\infty}\frac{ \ln(k)\left( \sin(k\pi/3)+1 \right) }{ k }\leq\lim_{k\to\infty}\frac{ 2\ln(k) }{ k }=0.
\end{equation}

\item 
$ \lim_{k \to  +\infty } \sqrt[3k]{k} (1 + \frac{1}{3k})^{3k} $.
D'abord, la suite $\left( 1+\frac{ 1 }{ 3k } \right)^{3k}$ est une sous suite de $\left( 1+\frac{ 1 }{ l } \right)^l$ qui converge vers $e$. Ensuite, nous avons $\sqrt[3k]{k}=\sqrt[3]{\sqrt[k]{k}}$, dont la limite est $1$ en vertu de la troisième suite particulière de la page 47 du cours.

Au final, la limite est $e$.

\end{enumerate}

\end{corrige}
