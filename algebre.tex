%%%%%%%%%%%%%%%%%%%%%%%%%%
%
   \section{Algebraic structures}
%
%%%%%%%%%%%%%%%%%%%%%%%%

\subsection{Algebraic structures}
%--------------------------------

A set with operations $(R,+,\cdot)$ is a \defe{ring}{ring} if 
\begin{itemize}
\item $(R,+)$ is an abelian group (we denote the neutral by $0$),
\item the multiplication $(R,\cdot)$ is unital and associative,
\item the multiplication is distributive with respect to addition.
\end{itemize}
A ring is not always a vector space because there are no multiplication by scalar.

A \defe{field}{field!algebraic structure} is a commutative ring whose nonzero elements form a group under multiplication.

An \defe{algebra}{algebra} is a vector space $\cA$ on $\eR$ or $\eC$ (or any field $\eK$) with an operation $\times\colon \cA\times\cA\to \cA$ such that
\begin{itemize}
\item $\times$ is distributive at left and right with respect to addition (vector structure),
\item for all $z,z'\in\cA$ and $A$, $B\in\cA$,
\[ 
  zA\times z'B=(zz')(A\times B).
\]
\end{itemize}
Most of time, the element $A\times B$ is just denoted by $AB$. In the case of Lie algebra, $X\times Y$ is $[X,Y]$. We will often treat with unital associative algebras on $\eC$; these algebras are in fact rings.


When $\cA$ is an algebra, the \defe{opposite}{opposite algebra} $\cA^0$\nomenclature{$\cA^0$}{Opposite algebra} is given by $\cA^0=\cA$ as sets, but multiplication in $\cA^0$ is
\[ 
  a^0b^0:=(ba)^0
\]
where the left hand side product is the one in $\cA^0$ while the right hand side one is the product in~$\cA$. A \defe{trace}{trace!over an algebra} aver an algebra is a linear form which satisfies the cyclic property $\tau(xy)=\tau(yx)$ for every elements $x$ and $y$ in the algebra.

\subsection{Morphisms and such}
%----------------------------

When $(E,\times)$ and $(F,\cdot)$ are two set with composition law and appropriate differential structure, a map $\varphi\colon E\to F$ is a
\begin{description}
\item[monomorphism] if $\ker\varphi=0$,
\item[epimorphism]  if $\varphi$ is surjective,
\item[homomorphism] if $\varphi(x\times y)=\varphi(x)\cdot\varphi(y)$,
\item[homeomorphism] if $\varphi$ and $\varphi^{-1}$ are continuous ($\varphi^{-1}$ needs to exist !).
\end{description}

If $(E,\times)$ and $(F,\cdot)$ are two sets with a composition law, a map $\dpt{\varphi}{E}{F}$ is a \defe{homomorphism}{homomorphism} if it satisfies
\[
     \varphi(x\times y)=\varphi(x)\cdot\varphi(y).
\]
It is a \defe{monomorphism}{monomorphism} when $\ker\varphi=0$ and an \defe{epimorphism}{epimorphism} if $\varphi(E)=F$.

If $M$ and $N$ are differentiable manifolds of respective dimension $m$ and $n$, a map $\dpt{f}{M}{M}$ is \defe{differentiable}{differentiable!map} if the coordinate associated map $\dpt{f_{\alpha i}}{\mU_{\alpha}}{\mU_i}$ is of class $\Cinf$ for any chart $\mU_{\alpha}$ of $M$ and $\mU_i$ of $N$.

The map $\dpt{f}{M}{N}$ is a \defe{diffeomorphism}{diffeomorphism} if $f$ is bijective and if $g$ and $g^{-1}$ are both differentiable. An \defe{homeomorphism}{homeomorphism} is a continuous map which has a continuous inverse.

\begin{remark}
Take care to not confuse \emph{homomorphism} and \emph{hom\underline{eo}morphism}. The first is only a map which respect an algebraic structure (a composition law) while the second also satisfies topological conditions: it must be continuous and open.
\end{remark}

%---------------------------------------------------------------------------------------------------------------------------
					\subsection{Group ring}
%---------------------------------------------------------------------------------------------------------------------------

Let $G$ be a group and $R$, a ring. The set $R[G]$, called the \defe{group ring}{group ring} of $G$ is 
\begin{equation}
	R[G]=\{ \sum_{g\in G}a_gg \}
\end{equation}
where $a_g\in R$ is non vanishing only for a finite number of $g$. The set $R[G]$ becomes a group by combining the composition law of $G$ and $R$ in the following way:
\begin{equation}
	\Big( \sum_ga_gg \Big)\Big( \sum_hb_hh \Big)=\sum_{g,h}(a_gb_h)gh.
\end{equation}
One can show that this definition is the only one for which the property
\[ 
	(1g)(1h)=(1gh)
\]
holds when $R$ has an unit $1$.  If $R$ and $G$ are commutative, then $R[G]$ is commutative.

\section{Tensor products}
%+++++++++++++++++++++++++

%---------------------------------------------------------------------------------------------------------------------------
					\subsection{Tensor product of vector spaces}
%---------------------------------------------------------------------------------------------------------------------------

When $V$ and $W$ are vector spaces, it is well known that the Cartesian product $V\times W$ is a vector space. A \defe{tensor product}{tensor product!of vector spaces} of $V$ and $W$ is a vector space $V\otimes W$ endowed with a bilinear map $\sigma\colon V\times W\to V\otimes W$ such that for every vector space $X$ and bilinear map $B\colon V\times W\to X$, there exists an unique linear map $\varphi\colon V\otimes W\to X$ such that $\varphi\circ\sigma=B$, i.e. the following diagram commutes:
\[ 
\xymatrix{%
   V\times W \ar[dr]_{B}\ar[rr]^{\sigma}		&	&V\otimes W\ar@{.>}[dl]^{\varphi}\\
 					&  X.
}	
\]
One can exhibit an explicit construction for the tensor product. Let $\{ v_{\alpha} \}$ and $\{ w_{\beta} \}$ be basis of $V$ and $W$ respectively and define $V\otimes W$ as the free vector space generated by the symbols $v_{\alpha}\otimes w_{\beta}$ endowed with the bilinear map $V\times W\to V\otimes W$,
\[ 
  \big( \sum_{\alpha} a_{\alpha} v_{\alpha},\sum_{\beta} b_{\beta}w_{\beta} \big)\mapsto\sum_{\alpha\beta} a_{\alpha}b_{\beta}\, v_{\alpha}\otimes w_{\beta}.
\]

One can form in that way tensor product of any \emph{finite} number of vector space. For the infinite product case, we make a direct limit, namely if $V_k$ are normed vector spaces, we choose unit vectors $v_k\in V_k$ and we define
\begin{equation}
\bigotimes_{k=1}^{\infty} V_k=\text{completion of }\left(\lim_{\rightarrow}\bigotimes_{k=1}^N V_k\right),
\end{equation}
the maps being $w_1\mapsto w_1\otimes v_2$, and so on\ldots

\begin{lemma}
If $V_i$ are finite dimensional vector spaces, there exists an isomorphism $\phi\colon (V_1\otimes V2)^*\to V_1^*\otimes V_2^*$. 
\end{lemma}
\begin{proof}
Indeed an element of $(V_1\otimes V_2)^*$ reads $F=F^{i\alpha}(e_i\otimes e_{\alpha})^*$ where $\{ e_i \}$ is a basis of $V_1$ and $\{ e_{\alpha} \}$ is a basis of $V_2$. Then we define
\[ 
  \phi(F)=F^{i\alpha}e_i^*\otimes e_{\alpha}^*.
\]
\end{proof}

\begin{lemma}
When $V_1$ and $V_2$ are finite dimensional, an inner product on $V_1\otimes V_2$ such that
\[ 
  \langle w_1\otimes w_2, z_1\otimes z_2\rangle =\langle w_1, z_1\rangle \langle w_2, z_2\rangle 
\]
exists and is unique.
\end{lemma}

%---------------------------------------------------------------------------------------------------------------------------
					\subsection{Tensor product of groups}
%---------------------------------------------------------------------------------------------------------------------------

Let $A$ and $B$ be two abelian groups, we define the tensor product of $A$ and $B$ as the abelian group $A\otimes_{\eZ}B$ endowed with a map $A\times B\to A\otimes_{\eZ}B$ which is an homomorphism onto $A$ when $b\in B$ is fixed and an homomorphism onto $B$ when $a\in A$ is fixed, and such that for every morphism $A\times B\to C$ to an abelian group $C$, there exists a morphism $A\otimes_{\eZ}B\to C$ which makes the following diagram commute
\[ 
\xymatrix{%
   A\times B \ar[rr]\ar[dr]_{\displaystyle\forall}		&&	A\otimes_{\eZ}B\ar[ld]^{\displaystyle\exists !}\\
								& C.
}
\]

Let $R$ be a ring; $A$, a right $R$-module and $B$, a left $R$-module. We define
\begin{equation}		\label{EqdefAtensRB}
	A\otimes_RB=\frac{ A\otimes_{\eZ}B }{\text{subgroup generated by elements of the form $ar\otimes_{\eZ}b-a\otimes_{\eZ}rb$}}.
\end{equation}

\section{Modules over algebras}
%++++++++++++++++++++++++++++++

Some more details can be found in \cite{Landi}.


Let $\cA$ be an algebra on $\eC$. A vector space $\modE$ is a \defe{right module}{module!right} on $\cA$ if it carry a right representation of $\cA$. In other terms, if we have a right multiplication 
\begin{equation}
\begin{aligned}
 \modE\times\cA&\to \modE \\ 
  (\eta,a)&\mapsto \eta a 
\end{aligned}
\end{equation}
such that
\begin{subequations}
\begin{align}
  \eta(ab)&=(\eta a)b\\
\eta(a+b)&=\eta a+\eta b\\
(\eta+\xi)a&=\eta a+\xi a
\end{align}
\end{subequations}
for all $\eta,\xi\in\modE$ and $a$, $b\in\cA$. Let $\modE$ and $\modF$ be two right $\cA$-module. A \defe{morphism}{morphism!of right module} of $\modE$ on $\modF$ if a $\cA$-linear map $\rho\colon \modE\to \modF$ which fulfils $\rho(\eta a)=\rho(\eta)a$ for all $a\in\cA$ and $\eta\in\modE$. When $\modE$ is a right $\cA$-module, we have a left $\cA^0$-module $\modE^0$ defined by $a^0\eta a$. 

Let $\cA$ and $\cB$ be two algebras; the vector space $\modE$ is a $\cA$-$\cB$-\defe{bimodule}{bimodule} if $\modE$ is a left $\cA$-module, a right $\cB$-module and if the two actions commute: $(a\xi)b=a(\xi b)$ for every $\xi\in\modE$, $a\in\cA$ and $b\in cB$.

The algebra $\cA^e:=\cA\otimes_{\eC}\cA^0$\nomenclature{$\cA^e$}{Enveloping algebra} is the \defe{enveloping algebra}{enveloping algebra} of $\cA$. A $\cA$-bimodule can be seen as a right $\cA^e$-module with definition $\eta(a\otimes b^0)=b\eta a$.

Let $\Gamma$ be a directed set; a family $(e_t)_{t\in \Gamma}$ is \defe{generating}{generating family of a module} for the right module $\modE$ if any element of $\modE$ can be written under the form $\sum_{t\in \Gamma}e_ta_t$ with $a_t\in\cA$ with only a finite non zero terms in the sum. The family $(e_t)$ is \defe{free}{free!family in a module} if it is made of $\cA$-linearly independent elements. A family is a \defe{basis}{basis!of a module} when it is both free and generating. A module is of \defe{finite type}{module!finite type}\index{finite type module} if it possesses a generating family of finite cardinality. Notice that the decomposition is in general not unique.

Let us now consider the set $\cA^N=\eC^N\otimes_{\eC}\cA$\nomenclature[A]{$\cA^N=\eC^N\otimes_{\eC}\cA$}{A complex module over $\cA$}. An element of this space can be written under the form 
\[ 
  \eta=
\begin{pmatrix}
\eta_1\\\vdots\\\eta_N
\end{pmatrix}
\]
with $\eta_i\in\cA$. A module structure on $\cA^N$ is given by the map $(\eC^N\otimes_{\eC}\cA)\times\cA\to \eC^N\otimes_{\eC}\cA$,
\[ 
  \begin{pmatrix}
\eta_1\\\vdots\\\eta_n
\end{pmatrix}a=
\begin{pmatrix}
\eta_1 a\\\vdots\\\eta_N a
\end{pmatrix}.
\]
A basis of this module can be identified with the canonical basis of $\eC^N$ because
\[ 
  \begin{pmatrix}
\eta_1\\\vdots\\\eta_N
\end{pmatrix}=\eta_1
\begin{pmatrix}
1\\\vdots\\0
\end{pmatrix}+\ldots+
\eta_N\begin{pmatrix}
0\\\vdots\\1
\end{pmatrix}.
\]
Hence the module $\cA^N$ is free and of finite type. The inclusion map (inverse of $p$ in a certain sense) $\lambda\colon \modE\to \eC^N\otimes_{\eC}\cA$\label{PgdeflambdaMod} will also be sometimes used. A general element of $\cA^N$ reads $\sum_{j=1}^{N}e_j\otimes_{\eC}a_j$ where $a_j\in\cA$ and $\{ e_j \}$ is the canonical basis of $\eC^N$. If we pose $f_i=e_i\otimes_{\eC}1$, every element of $\cA^{N}$ is $\sum_{j}j_ja_j$, so that $\cA^N$ is a finitely generated module.


Since $p$ is a module homomorphism,
\[ 
  p\big( \sum_{j}f_ja_j \big)=\sum_j p(f_j)a_j
\]
with $f_j\in\cA^N$ and $a_j\in\cA$. We deduce that $\{ p(f_j) \}$ is a basis of the finitely generated module $p\cA^N$.

If $\modE$ is a $\cA$-bimodule, the \defe{center}{center!of a bimodule} of $\modE$ is the set
\begin{equation}
\mZ(\modE)=\{ \xi\in\modE\tq a\xi=\xi a,\;\forall a\in\cA,\,\xi\in\modE \}.
\end{equation}

\begin{proposition}
If we define the maps
\begin{equation}
\begin{aligned}
 j\colon \modE\otimes_{\cA}\Omega^1\cA&\to \modE\otimes_{\eC}\cA \\ 
   \xi\otimes_{\cA}\delta a&\mapsto \xi\otimes_{\eC}a-\xi a\otimes_{\eC}1 
\end{aligned}
\end{equation}
and
\begin{equation}
\begin{aligned}
 m\colon \modE\otimes_{\eC}\cA&\to \modE \\ 
   \xi\otimes_{\eC}&\mapsto \xi a, 
\end{aligned}
\end{equation}
the sequence
\[ 
\xymatrix{%
   0 \ar[r]		&	\modE\otimes_{\cA}\Omega^1\ar[r]^-j	&  \modE\otimes_{\eC}\cA\ar[r]^-m	&\modE\ar[r]	&0
}
\]
is exact
\end{proposition}

\begin{proof}
An element in the kernel of $m$ has the form $\xi\otimes_{\eC}ba-\xi b\otimes_{\eC}a$ which is the image by $j$ of $\xi\otimes_{\cA}(\delta b)a$. Indeed
\begin{align*}
j\big( \xi\otimes_{\cA}(\delta b)a \big)	&=j\big( \xi\otimes_{\cA}\delta(ba)-\xi\otimes_{\cA}b\delta a \big)\\
						&=\xi\otimes_{\eC}ba-\xi ba\otimes_{\eC}A-\xi b\otimes_{\eC}a+\xi ba\otimes_{\eC}1\\
						&=\xi_{\eC}ba-\xi b\otimes_{\eC} a.
\end{align*}
\end{proof}

\subsection{Endomorphism}
%------------------------

Let us prove that an endomorphism $A\in\End_{\cA}(\modE)$ is identified with a matrix $A\in\eM_{N\times N}(\cA)$ such that $A=Ap=pA=pAp$. The action of $A$ must satisfy
\[ 
  A\big( \sum_ip(f_i)a_i \big)=\sum_iA\big( p(f_i) \big)a_i,
\]
so that $A$ acts on the generating part $\{ p(f_i) \}$ of $\modE$. We pose 
\begin{equation}
  Ap(f_i)=\sum_kp(f_i)A_{ik}
\end{equation}
for some $A_{ik}\in\cA$. So we have
\begin{align*}
  Ap\big( \sum_ie_i\otimes_{\eC}a_i \big)&=\sum_i\sum_kp(f_i)A_{ik}a_i
				=\sum_ip(f_i)\sum_kA_{ik}a_i
				=p\Big( \sum_ie_i\otimes_{\eC}\big( \sum_kA_{ik}a_i \big) \Big).
\end{align*}
Using the notation $\xi=p\big( \sum_ie_i\otimes_{\eC}\xi^i \big)$ we have
\begin{equation}
A\xi=p\Big( \sum_ie_i\otimes_{\eC}(A\xi)_i \Big),
\end{equation}
that we note $pA\xi$ by abuse of notation. For the same reason of notations, the equation $\xi=p\big( \sum_ie_i\otimes_{\eC}\xi^i \big)$ is denoted by $\xi=p\xi$, thus we have
\[ 
  A=Ap=pA,
\]
where $A$ and $p$ have different meaning.


\subsection{Dimension of a module} \label{subsec_DimofModule}
%---------------------------------

There exists some free modules which admit several basis of different cardinality; for them, there are no way to define the notion of dimension. For a free module whose all basis have same cardinality, the latter is the \defe{dimension}{dimension!of a free module} of the module.

If $\modE$ is a module of finite type, there exists an integer $N$ and a surjective map $\rho\colon \cA^N\to \modE$. Indeed, the fact the $\modE$ is of finite type gives a generating part $\{ e_1,\ldots,e_N \}$. Then the definition
\[ 
  \rho
\begin{pmatrix}
a_1\\\vdots\\ a_N
\end{pmatrix}
=
\sum_{i=1}^N e_ia_i
\]
works. In general, it is not possible to extract a free part of this generating part. Let us see an example. Consider $ C^{\infty}(S^2)$, the algebra of smooth functions on the sphere and the Lie algebra $\cvec(S^2)$. The latter is a module of finite type on $ C^{\infty}(S^2)$ and a generating part is given by the three following vector fields :
\[ 
  Y_i(x)=\sum_{j,k=1}^3 \epsilon_{ijk}x_j\partial_k.
\]
These vectors are tangent vectors because scalar product $Y_i(x)\cdot x$ vanishes :
\[ 
  Y_i\cdot x =\sum_l (Y_i)_lx_l=\sum_{l,j}\epsilon_{ijl} x_jx_l=0.
\]
In order to prove that this set is generating, let us suppose that $Y_1$ is proportional to $Y_2$ and $Y_3$. If we impose $bY_3=Y_1$, we find, for each $k$ :
\[ 
  \epsilon_{1jk} x_j\partial_k=b\epsilon_{3jk}\partial_k,
\]
taking $k=1$, we find $x_2=0$. The same with $Y_1=aY_2$ leads to $x_3=0$. So the only candidate point where the set of $Y_i$ can fail to be generating is $x_1=\pm 1$, $x_2=x_3=0$. At this point, $Y_2=\pm\partial_3$ and $Y_3=\pm\partial_2$ whose are linearly independent. Hence the set of $Y_j$ is a generating part. It is not a free part (with respect to $ C^{\infty}(S^2)$) because
\[ 
  \sum_{j=1}^3 x_jY_j=0.
\]
We can however not find two vector fields $X_1$ and $X_2$ which form a global basis of $\cvec{S^2}$ because we would have $X_1-X_2\neq0$ everywhere on $S^2$.

\subsection{Tensor product of modules (first)}
%-------------------------------------

Let $\modE_1,\ldots,\modE_n$, and $\modF$, some modules on $\eC$. We denote by $L^n(\modE_1,\ldots,\modE_n;\modF)$ the module of $n$-multi-linear maps $f\colon \modE_1\times\ldots\times\modE_n\to \modF$. Let $\modM$ the free module generated by $n$-uples $(x_1,\ldots,x_n)$, $x_i\in\modE_i$; we denote by $\modN$ the submodule generated by elements of the form
\begin{subequations}
\begin{align}
(x_1,\ldots,x_i+x'_i,\ldots,x_n)&+(x_1,\ldots,x_i,\ldots,x_n)-(x_1,\ldots,x'_i,\ldots,x_n)\\
(x_1,\ldots,zx_i,\ldots,x_n)&-z(x_1,\ldots,x_n)
\end{align}
\end{subequations}
for all $x_i,x'_i\in\modE_i$ and $z\in\eC$.

We have a canonical injection $\modE_1\times\ldots\times\modE_n\to\modM$ and $\modM\to\modM/\modN$. We denote by $\varphi\colon \modE_1\times\ldots\times\modE_n\to \modM/\modN$ the composition of these two. This $\varphi$ is the \defe{tensor product}{tensor product!of modules} $\modE_1\otimes\ldots\otimes\modE_n$, and we write
\[ 
  \xi_1\otimes_{\eC}\xi_2=\varphi(\xi_1,\xi_2).
\]
The tensor product has an universal property.

\begin{proposition}
If $f\colon \modE_1\times\ldots\times\modE_n\to G$ is a $n$-multi-linear, there exists a map $f_*\colon \modM/\modN\to G$ such that the following diagram commutes :
\begin{equation}
\xymatrix{%
   \modE_1\times\ldots\times\modE_n \ar[r]^-{\varphi}\ar[dr]_{f}		&\modM/\modN\ar@{.>}[d]^{f_*}\\
   									&	G
}
\end{equation}
\end{proposition}

\begin{proof}
It is first clear that there exists a $f'\colon \modM\to G$ such that $i\circ f'=f$. This map takes the value $0$ on $\modN$ because $i^{-1}(\modN)=0$. So $f'$ can be factorised by $\modN$ to give $f_*$.
\end{proof}

\subsection{Tensor product of module (second)}
%---------------------------------------------

Let $\modE_1$ and $\modE_2$ be two $\cA$-bimodule. We consider $\modM$, the free $\cA$-bimodule generated by $\modE_1\times\modE_2$, i.e. the set of (formal) linear combinations of elements of the form
\[ 
  a(\xi_1,\xi_2)\text{ and }(\xi_1,\xi_2)a
\]
with $xi_i\in\mA$ and $a\in\mA$. We define the sub-module $\modN$ generated by elements of the form
\begin{subequations}
\begin{align}
i(\xi_1+\eta_1,\xi_2+\eta_2)-i(\xi_1,\xi_2)-i(\xi_1,\eta_2)-i(\eta_1,\xi_2)-i(\eta_1,\eta_2)\\
ai(\xi_1,\xi_2)\\
ai(\xi_1,\xi_2)-i(a\xi_1,\xi_2)\\
ai(\xi_1,\xi_2)-i(\xi_1,a\xi_2)\\
i(\xi_1,\xi_2)a-i(\xi_1,\xi_2a)\\
i(\xi_1,\xi_2)a-i(\xi_1a,\xi_2).
\end{align}
\end{subequations}
We define $\varphi\colon \modE_1\times\modE_2\to \modM/\modN$ as 
\[ 
  \varphi=\pi\circ i
\]
where $i$ is the inclusion of $\modE_1\times\modE_2$ in $\modM$. The definition $\xi_1\otimes_{\cA}\xi_2=\varphi(\xi_1,\xi_2)$ has an universal property.

\begin{proposition}
Let $G$ be a $\cA$-bimodule and a $\cA$-linear map $f\colon \modE_1\times\modE_2\to G$. Then there exists one and only one $\cA$-linear map $f_*\colon \modM/\modN\to G$ such that the diagram
\begin{equation}  \label{eq_diagprodtensunif}
\xymatrix{%
   \modE_1\times\modE_2 \ar[r]^-{\varphi}\ar[dr]_{f}		&	\modM/\modN\ar@{.>}[d]^{f_*}\\
   	&	G
}
\end{equation}
is commutative.
\end{proposition}

\begin{proof}
We first consider a map $f'\colon \modM\to G$ such that $f'\circ i=f$. From $\cA$-linearity of $f'$, we conclude that
\[ 
 \begin{split}
f'\big( ai(\xi_A,\xi_2)-i(a\xi_1,\xi_2) \big)&=a(f'\circ i)(\xi_1,\xi_2)-(f'\circ i)(a\xi_1,\xi_2)\\
		&=af(\xi_2,\xi_2)-f(a\xi_2,\xi_2)\\
		&=0.
\end{split} 
\]
So $f'(\modN)=0$, and we can consider the quotient map $f_*=f'/\modN$ which gives commutativity of diagram \eqref{eq_diagprodtensunif}. We still have to prove the unicity part. Let $g_*\colon \modM/\modN\to G$ be an other map which has the same commutativity property that $f_*$ :
\[ 
  f=f_*\circ \varphi=g_*\circ\varphi
\]
and $f_*(\modN)=g_*(\modN)=0$. A general element in $\modM/\modN$ has the form (of linear combination of) $\varphi(\xi_1,\xi_2)$, so we following computation concludes the proof :
\[ 
  f_*\varphi(\xi_1,\xi_2)-g_*\varphi(\xi_1,\xi_2)=f(\xi_1,\xi_2)-f(\xi_2,\xi_2)=0.
\]
\end{proof}

\subsection{Tensor product of modules (third)}
%---------------------------------------------

This definition of tensor mainly product comes from \cite{Jacobson_alg}.

We suppose the algebra $\cA$ to be unital and associative, so it is a ring. Note that $\eC$ is a ring too; so we will in the same time define $\otimes_{\eC}$ and $\otimes_{\cA}$. Let $\modE$ be a right $\cA$-module and $\modF$ a left $\cA$-module on the ring $R$, in particular, recall that $\modE$ and $\modF$ are vector spaces on $\eC$. For $\eta\in\modE$, $\xi\in\modF$, and $a\in R$, we have maps
\begin{equation}
\begin{aligned}
 \modE\times R&\to \modE \\ 
(\eta,a)&\mapsto \eta a, 
\end{aligned}
\end{equation} 
and
\begin{equation}
\begin{aligned}
 R\times\modF&\to \modF \\ 
(a,\xi)&\mapsto a\xi. 
\end{aligned}
\end{equation}

A \defe{balanced product}{balanced product} of $\modE$ and $\modF$ is an abelian group $(P,+)$ with a map $f\colon \modE\times\modF\to P$ such that 
\begin{subequations}
\begin{align}
  f(\eta+\eta')&=f(\eta,\xi)+f(\eta',\xi)\\
f(\eta,\xi+\xi')&=f(\eta,\xi)+f(\eta,\xi')\\
f(\eta a,\xi)&=f(\eta,a\xi).
\end{align}
\end{subequations}
This balanced product is denoted by $(P,f)$. The function $f$ must fulfil
\[ 
  f(0,\xi)=0=f(\eta,0)
\]
because $f(0,\xi)=f(0+0,\xi)=f(0,\xi)+f(0\xi)$. The same kind of reason leads to
\[ 
  f(-\eta,\xi)=-f(\eta,\xi)
\]
where the minus sign in the right hand side is in the sense of the inverse in the additive group $(P,+)$.

When $(P,f)$ and $(P,g)$ are two balanced products, a \defe{morphism}{morphism!of balanced product} is a group homomorphism $\varphi\colon P\to Q$ such that
\[ 
  g(\eta,\xi)=\varphi\big( f(\eta,\xi) \big).
\]

\begin{definition}
A \defe{tensor product}{tensor product!of modules} of $\modE$ and $\modF$ is a balanced product $(A,\otimes_R)$ such that for each balanced product $(P,f)$, there exists an unique morphism $(A,\otimes_R)\to(P,f)$ such that
\[ 
  \otimes_R(\eta,\xi)=f(\eta,\xi).
\]

\end{definition}
The element $\otimes_R(\eta,\xi)$ is often denoted by $\eta\otimes_R\xi$.

Unicity of tensor product is given by the following proposition.

\begin{proposition}
Let $(A,\otimes_1)$ and $(B,\otimes_2)$ be two tensor products of $\modE$ and $\modF$. There exists an unique isomorphism $\varphi\colon A\to B$ such that
\[ 
  \varphi\big( \otimes_1(\eta,\xi) \big)=\otimes_2(\eta,\xi).
\]

\end{proposition}

\begin{proof}
Since $(A,\otimes_1)$ is a tensor product and $(B,\otimes_2)$ is a balanced product, there exists an unique morphism $\varphi\colon (A,\otimes_1)\to (B,\otimes_2)$ such that
\[ 
  \varphi(\eta\otimes_1\xi)=\eta\otimes_2\xi.
\]
We have to prove that this morphism is an isomorphism. There also exists an unique morphism $\phi\colon (B,\otimes_2)\to (A,\otimes_1)$ such that
\[ 
  \phi(\eta\otimes_2\xi)=\eta\otimes_1\xi.
\]
Then
\[ 
   (\varphi\circ\phi)(\eta\otimes_2\xi)=\varphi(\eta\otimes_1\xi)=\eta\otimes_2\xi  
\]
and $\varphi$ is the inverse of $\phi$.
\end{proof}

\begin{proposition}
If $(A,\otimes)$ is a tensor product of $\modE$ and $\modF$, then the group $A$ is generated by products $\eta\otimes\xi$.
\end{proposition}

\begin{proof}
Let us suppose the existence of $\omega\in A$ which cannot be written under the form $\sum_i\eta_i\otimes\xi_i$. If $(P,f)$ is a balanced product, a morphism $\varphi\colon A\to P$ is not completely defined by the requirement $\varphi(\eta\otimes\xi)=f(\eta,\xi)$ because $\varphi(\omega)$ can take any value in $A$.

\end{proof} 

\subsection{Explicit building of tensor product}
%------------------------------------------------

We will now explicitly build a tensor product $\modE\otimes_R\modF$ that will be named \emph{the} tensor product. First we consider the abelian free group $M$ of basis $\modE\times\modF$, i.e. the set of formal sums
\[ 
  n_1(\eta_1,\xi_1)+\ldots+n_r(\eta_r,\xi_r)
\]
with $n_i\in\eZ$, $\eta_i\in\modE$ and $\xi_i\in\modF$. We put the usual addition and if $(\eta_i,\xi_i)\neq(\eta_j,\xi_j)$ for all $i\neq j$, this sum is zero only if $n_i=0$ for all $i$. In $M$, we consider the subgroup $N$ generated by
\begin{subequations}
\begin{align}
(\eta+\eta',\xi)&-(\eta,\xi)-(\eta',\xi)\\
(\eta,\xi+\xi')&-(\eta,\xi)-(\eta,\xi')\\
(\eta a,\xi)&-(\eta, a\xi)
\end{align}
\end{subequations}
with $\eta,\eta'\in\modE$, $\xi,\xi'\in\modF$ and $a\in R$. Now we define 
\begin{equation}
\modE\otimes_R\modF=M/N
\end{equation}
and 
\begin{equation}
\eta\otimes_R\xi=[(\eta,\xi)].
\end{equation}

\begin{proposition}
The group $(\modE\otimes_R\modF,\otimes_R)$ is a tensor product.
\end{proposition}

\begin{proof}
We begin by proving that $(\modE\otimes_R\modF,\otimes_R)$ is a balanced product. From definition of the equivalence defining the class $[(\eta,\xi)]$, we have
\[ 
 \begin{split}
  \otimes_R(\eta+\eta',\xi)&=\otimes_R(\eta,\xi)-\otimes_R(\eta',\xi)\\
		&=[(\eta+\eta',\xi)]-[(\eta,\xi)]-[(\eta',\xi)]\\
		&=[0].
\end{split} 
\]
and 
\[ 
  \otimes_R(\eta a,\xi)=[(\eta a,\xi)]=[(\eta,a\xi)]=\otimes_R(\eta, a\xi).
\]

Now let $(P,f)$ be a balanced product. Since $M$ is generated by elements of the form $(\eta,\xi)$, the prescription
\begin{equation}
\begin{aligned}
 \varphi\colon M&\to M \\ 
(\eta,\xi)&\mapsto f(\eta,\xi) 
\end{aligned}
\end{equation} 
completely defines $\varphi$. Now we try to defines something on $M/N$ from this $\varphi$. Let $K$ be the kernel of $\varphi$. Elements of the form
\[ 
 \begin{split}
(\eta+\eta',\xi)&-(\eta,\xi)-(\eta',\xi)\\
(\eta,\xi+\xi')&-(\eta,\xi)-(\eta,\xi')\\
(\eta,a,\xi)&-(\eta, a\xi)
\end{split} 
\]
belongs to $K$. Then $N\subset K$ and one has a well defined morphism 
\begin{equation}
\begin{aligned}
 M/N&\to P \\ 
[(\eta,\xi)]&\mapsto f(\eta,\xi).
\end{aligned}
\end{equation}
Uniqueness comes from the fact that elements of the form $[(\eta,\xi)]$ are generating $M/N$.

\end{proof}





\subsection{Unitary group}		\label{SubsecUnitGroup}
%-------------------------

Let $\modE=p\cA^N$ be an Hermitian finite projective module. The algebra of \defe{endomorphism}{endomorphism!of module} is\nomenclature[A]{$\End_{\cA}(\modE)$}{Algebra of endomorphism of the module $\modE$}
\[ 
  \End_{\cA}(\modE)=\{ T\colon \modE\to \modE\tq T(\eta a)=T(\eta)a \}.
\]
We naturally define the involution $*\colon \End_{\cA}(\modE)\to \End_{\cA}(\modE)$ defined by
\begin{equation}
\langle T^*\eta, \xi\rangle =\langle \eta, T\xi\rangle
\end{equation}
If $A\in\eM_{N\times N}(\cA)$ is an endomorphism of $\modE$, in particular for each $\xi\in\modE$, the element $A\xi$ must belongs to $\modE$, so that $pA\xi=A\xi$. That proves that $pA=A$. Since $\xi=p\xi$, we have moreover $pAp=Ap$. Finally
\[ 
  A=Ap=pA=pAp.
\]
So we have an isomorphism $\End_{\cA}(\modE)\simeq p\eM_{N\times N}(\cA)p$. An endomorphism $u$ is \defe{unitary}{unitary!endomorphism of module} if $u^*u=uu^*=1$. We denote by $U(\modE)$\nomenclature[A]{$U(\modE)$}{Space of unitary endomorphism of the module $\modE$} the space of unitary endomorphism of $\modE$.

As an example, let us take $M$, a differentiable manifold and $ C^{\infty}(M)$, the algebra of smooth functions over $M$. We want to study $U_N\big(  C^{\infty}(M) \big)=U\big(  C^{\infty}(M)^N \big)$. An element of that algebra is a $N\times N$ matrix whose entries are functions on $M$ or, equivalently, a function $u\colon M\to \eM_{N\times N}(\eC)$. The $1$ in the unitary condition $u^*u=uu^*=1$ has to be understood as the function on $M$ with $\mtu$ as constant value. So we have the isomorphism
\[ 
  U_N\big(  C^{\infty}(M) \big)\simeq C^{\infty}\big( M,U(N) \big)
\]
where $U(N)$ is the usual unitary group.


%+++++++++++++++++++++++++++++++++++++++++++++++++++++++++++++++++++++++++++++++++++++++++++++++++++++++++++++++++++++++++++
					\section{Module over unital ring}
%+++++++++++++++++++++++++++++++++++++++++++++++++++++++++++++++++++++++++++++++++++++++++++++++++++++++++++++++++++++++++++
\label{SecModUnitalAnneau}

\begin{probleme}
This section is a mess. Several propositions have to be merged. We have to check that everything said in the algebra case hold in the ring case, and then change algebra to ring everywhere. The module over algebra part mainly comes from \cite{Landi}.
\end{probleme}

\subsection{Projective module of finite type on unital algebra}
%--------------------------------------------------------------

Let $\cA$ be an unital algebra.

\begin{proposition}		\label{PropEquivProjModule}
Let $\modE$ be a right $\cA$-module where $\cA$ is an unital algebra. The three following are equivalent:
\begin{enumerate}
\item If $\rho\colon \modF_1\to \modF_2$ is surjective homomorphism of right $\cA$-module, then every homomorphism $\lambda\colon \modE\to \modF_2$ can be lifted into a homomorphism $\tilde\lambda\colon \modE\to \modF_1$ which satisfies $\rho\circ\tilde\lambda=\lambda$. 
\begin{equation}  \label{eq_diag_proj_mod}
\xymatrix{%
            &\modF_1\ar@{>>}[d]^{\rho}\\
\modE \ar@{.>}[ur]^{\tilde\lambda}\ar[r]_{\lambda} &\modF_2
}
\end{equation}
That property is the \defe{lifting property}{lifting property!projective module}.

\item\label{ItemTroisCarecterisationProjectif} If $\rho\colon\modF_1\to \modE$ is a surjective morphism of module, there exists a module morphism $s\colon \modE\to \modF_1$ such that $\rho\circ s=\id|_{\modE}$. 
\[ 
    \xymatrix{ \modF_1 \ar[r]^{\rho} & \modE \ar@/^/@{.>}[l]^s }
\]

\item \label{prop_def_proj_module_iii}  There exists a free module $\modF$ which decomposes as a direct sum which one component is $\modE$:
\[ 
  \modF=\modE\oplus\modE'.
\]
In this case, $\modE'$ is also free.

\end{enumerate}
\label{prop_def_proj_module}
\end{proposition}

\begin{proof}
For a proof, see \cite{Landi} at page $59$.
\end{proof}

A right module which fulfils these properties is said to be \defe{projective}{projective!module}\index{module!projective}.

\subsection{Projective module of finite type on unital ring}
%------------------------------------------------------------

%
%	En termes de notations pour les modules, M devient \modE et N devient \modF.
%

Let $R$ be an unital ring, and $\modE$, $\modF$, \ldots be left modules over $R$. One says that $\modE$ is a \defe{projective module}{projective!module} if for every surjective module homomorphism $\lambda\colon \modF_2\to \modF_1$, there exists a lifting map $\tilde\lambda\colon \modE\to \modF_2$ such that the following diagram commutes:
\begin{equation}		\label{EqLiftPropProjModules}
\xymatrix{%
 					&  \modF_2 \ar@{->>}[d]^{\displaystyle\lambda}\\
   \modE \ar[r]\ar@{.>}[ru]^{\displaystyle\tilde\lambda}			&	\modF_1
}
\end{equation}
where the double arrow denotes the surjectivity. 

\begin{proposition}		\label{PropFGPRkP}
Every finitely generated projective module over the unital ring $R$ has the form 
\[ 
R^kP
\]
for some $k\in\eN$ and some idempotent matrix $P\in\eM_k(R)$. The choices of $k$ and $P$ are not unique.
\end{proposition}

\begin{proof}
We suppose that $\modE$ is projective and finitely generated by $\{ \xi_1,\ldots,\xi_k\}$. A module map $T\colon R^k\to R^k$ is given by a matrix $r$ defined by
\[ 
	T(0,\ldots,1,\ldots,0)=(r_{1j},\ldots,r_{kj})
\]
where, in the left hand side, the $1$ is on the $j$th position, and $T$ is the right matrix multiplication
\begin{equation}
	T(a_1,\ldots,a_k)=
\begin{pmatrix}
a_1&\ldots&a_k
\end{pmatrix}
\begin{pmatrix}
r_{11}	&	r_{21}	&\ldots	&r_{k1}\\
r_{12}	&	r_{22}	&\ldots	&r_{k2}\\
\vdots	&	\vdots	&	&\vdots\\
r_{k1}	&	r_{k2}	&\ldots	&r_{kk}
\end{pmatrix}.
\end{equation}

Item \ref{ItemTroisCarecterisationProjectif} of proposition \ref{PropEquivProjModule} assures the existence of a direct sum decomposition $R^k=\modE\oplus \modF$ of submodules of $R^k$, then, associated with the projection onto $\modE$, there is an idempotent matrix $P\in\eM_k(R)$ such that 
\begin{equation}		\label{EqMRkPdec}
	\modE\simeq R^kP.
\end{equation}
\end{proof}

The following proposition shows that every finitely generated projective module over $R$ come from that construction.
\begin{proposition}
A module $\modE$ is projective of finite type on $\cA$ if and only if there exists a matrix $p\in \eM_{N\times N}(\cA)$ such that
\begin{enumerate}
\item $p$ is idempotent: $p^2=p$,
\item $\modE=p\cA^N$.
\end{enumerate}

\end{proposition}

\begin{proof}

Since $\modE$ is projective as well as of finite type, we have a surjective map $\rho\colon \cA^N\to \modE$. Let us draw the diagram \eqref{eq_diag_proj_mod} in the particular case with $\modF_1=\cA^N$ and $\modF_2=\modE$: 
\begin{equation}
\xymatrix{%
						   		&	\cA^N\ar[d]^{\rho}\\
   \modE \ar[r]_{\lambda=\id}\ar@{.>}[ru]^{\tilde\lambda}	&	\modE
}
\end{equation}
Hence we have a map $\tilde\lambda\colon \modE\to \cA^N$ such that $\rho\circ\tilde\lambda=\id|_{\modE}$. The map $p=\tilde\lambda\circ\rho\colon \cA^N\to \cA^N$ fulfils $p^2=\id|_{\modE}$ and is therefore idempotent in $\End(\cA^N)$. It allows us to decompose $\cA^N$ as a sum of submodules
\[ 
  \cA^N=p\cA^N+(1-p)\cA^N.
\]
We know that $\rho$ and $\tilde\lambda$ are two isomorphism between $\modE$ and $p\cA^N$. We now prove that $\rho$ is bijective on $p\cA^N$. Surjectivity is clear; we have to prove that $\rho\colon p\cA^N\to \modE$ is injective. Let $\rho(pa)=\rho(pb)$ and apply $\tilde\lambda$ on both sides of this equality. Since $\tilde\lambda\rho=p$, we find $p^2a=p^2b$, en therefore $pa=pb$. Remark that this does not imply $a=b$.

For the inverse sense, we suppose a module $\modE$ on $\cA$ and a map $p\in M_{N\times N}(\cA)$ such that $p^2=p$ and $\modE=p\cA^N$. The module $\modE$ fulfils point \ref{prop_def_proj_module_iii} of proposition \ref{prop_def_proj_module} with
\[ 
  \cA^N=p\cA^N\oplus(1-p)\cA^N.
\]  
The fact that $\modE$ is of finite type is clear because $\cA^N$ is such.

\end{proof}
An element of $\modE$ can be seen as a $p$-invariant column vector with entry in $\cA$:
\[ 
  \xi=
\begin{pmatrix}
\xi_1\\\vdots\\\xi_N
\end{pmatrix}
\]
with $\xi_i\in\cA$ and $p\xi=\xi$. Remark that it does \emph{not} mean that $p\xi_i=\xi_i$ for each $i$: such a condition is in fact senseless.

A beautiful result that provides a concrete realisation of projective finite module is the following.

\begin{theorem}[Serre-Swan]
Let $M$ be a compact finite dimensional manifold. A $C^{\infty}(M)$-module $\modE$ is isomorphic to the space of sections $\Gamma(M,E)$ of a vector bundle $E\to M$ if and only if it is finite and projective.

In other words, to any projective finite module, we can associate a vector bundle whose sections give $\modE$.
\end{theorem}

\begin{proof}
No proof
\end{proof}

\begin{lemma}		\label{LemRklQkR}
If $R^kP\simeq \modE\simeq R^lQ$, then there are matrices $U\in\eM_{k\times l}(R)$ and $V\in\eM_{l\times k}(R)$ such that $UV=P$ and $VU=Q$.
\end{lemma}
\begin{proof}
No proof.
\end{proof}
This lemma says that if we perform two times the constructions which lead to decomposition \eqref{EqMRkPdec} with different choices, then the two results are in a certain sense equivalent.

Let $E\subseteq F$ be a $R$-submodule. The \defe{closure}{closure!of a submodule} of $E$ in $F$, denoted by $\Cl_F(E)$\nomenclature[A]{$\Cl_F(E)$}{Closure of the submodule $E$ in $F$}, or $\bar E$ when there are no confusion, is
\begin{equation}
	\Cl_F(E)=\{ f\in F\tq \varphi(f)=0,\,\forall \text{ module map }\varphi\colon F\to R\text{ such that }\varphi|_E=0 \}.
\end{equation}
We are not interested in the topology on $F$ that this definition provides.

The lemmas \ref{LemEprojEEEfproj} and \ref{LemFGenEEEsingleGen} will help to reduce numerous problems to singly generated modules.
\begin{lemma}		\label{LemEprojEEEfproj}
If $E$ is a finite projective module over $R$, then $E\oplus\ldots\oplus E$ is a finite projective module over $\eM_n(R)$. 

If $F$ is a submodule of $E$, then
\[ 
	\Cl_{E\oplus\ldots\oplus E}(F\oplus\ldots\oplus F)=\Cl_E(F)\oplus\ldots\oplus\Cl_E(F).
\]

\end{lemma}
\begin{proof}
No proof.
\end{proof}

\begin{lemma}		\label{LemFGenEEEsingleGen}
If $E$ is generated by $\{ e_1,\ldots, e_n \}$, then $E\oplus\ldots\oplus E$ is generated over $\eM_n(R)$ by the single element $(e_1,\ldots,e_n)$.
\end{lemma}
\begin{proof}
No proof.
\end{proof}


Now we suppose that we have a trace map $\varphi\colon R\to \eA$ where $\eA$ is a commutative ring (or any abelian group). If $S$ is a square matrix of elements in $R$, we define
\begin{equation}
	\varphi(S)=\sum_i\varphi(S_{ii}),
\end{equation}
and one checks that $\varphi(ST)=\varphi(TS)$, so that $\varphi$ is a trace over $\eM_k(R)$ for every $k$. Now, we define 
\begin{equation}			\label{EqPreDefDimModuleRA}
	\varphi(\modE)=\varphi(P)		
\end{equation}
where $P$ is defined by the condition $\modE=R^kP$. Lemma \ref{LemRklQkR} assures that this condition does not depend on the choice. The \defe{dimension function}{dimension!function} associated with the trace $\varphi$ is
\begin{equation}\label{DefDimFunctModule}
\begin{aligned}
	\dim_{\varphi}\colon \{ \text{isomorphism class of finitely generated projective modules} \} &\to \eA\\
   	\modE&\mapsto \varphi(\modE).
\end{aligned}
\end{equation}
It fulfills the condition
\[ 
	\dim_{\varphi}(\modE\oplus \modF)=\dim_{\varphi}(\modE)+\dim_{\varphi}(\modF).
\]
The challenge of section \ref{SecOverModVNalgDim} is to show that the construction of the dimension function \eqref{DefDimFunctModule} extends to all modules over von~Neumann algebras. 

\begin{lemma}
If $H_1$ and $H_2$ are modules over the ring $R$, they are isomorphic if and only if $H_1\oplus\ldots\oplus H_1\simeq H_2\oplus\ldots\oplus H_2$ as module over $\eM_n(R)$.
\end{lemma}


%+++++++++++++++++++++++++++++++++++++++++++++++++++++++++++++++++++++++++++++++++++++++++++++++++++++++++++++++++++++++++++
\section{Bialgebras and co-properties}
%+++++++++++++++++++++++++++++++++++++++++++++++++++++++++++++++++++++++++++++++++++++++++++++++++++++++++++++++++++++++++++

%---------------------------------------------------------------------------------------------------------------------------
\subsection{Bialgebras}
%---------------------------------------------------------------------------------------------------------------------------
Some generalities about bialgebras are taken from \cite{TimmernannInvitation}.

Let $\cA$ be an unital algebra and $r\in\cA\times\cA$. There exists a (non unique) decomposition $r=\sum_i\alpha_i\otimes\beta_i$ with $\alpha_i$, $\beta_i\in\cA$. There exists three ways to extend that in order to embed $r$ in $\cA\otimes\cA\otimes\cA$. We introduce the \defe{Sweedler notation}{sweedler notation} in order to distinguish them:
\begin{align*}
r_{12}&=\sum_i\alpha_i\otimes\beta_i\otimes 1\\
r_{13}&=\sum_i\alpha_i\otimes 1\otimes\beta_i\\
r_{23}&=\sum_i1\otimes\alpha_i\otimes\beta_i.
\end{align*}
One can define $r_{ij}$ in the same way in $\cA^{\otimes n}$. We denote by $\tau\colon \cA\otimes\cA\to \cA\otimes\cA$,
\[ 
  \tau(a\otimes b)=b\otimes a.
\]

If we denote by $\mu\colon \cA\otimes\cA\to \cA$ the algebra law in $\cA$, the unit $1$ defines (and is defined by) the map
\begin{equation}
\begin{aligned}
 \eta\colon \eK&\to \cA \\ 
   \eta(z)&= z1.
\end{aligned}
\end{equation}
The fact that $1$ is an unit is expressed by the equalities $1a=a1=a$ for all $a\in\cA$, or equivalently by the commutativity of the diagrams
\begin{align}		\label{EqDiagUnitHopf}
	\xymatrix{%
	\eK\otimes\cA \ar[r]^{\eta\otimes\id}\ar[rd]_{\psi}		&	\cA\otimes\cA\ar[d]^{\mu}\\
	   	&	\cA
	   }
&&
	\xymatrix{%
	\cA\otimes\eK \ar[r]^{\id\otimes\eta}\ar[rd]_{\psi}		&	\cA\otimes\cA\ar[d]^{\mu}\\
	   	&	\cA
	   }
\end{align}
where we indifferently denoted by $\psi$ the identification $\psi(z\otimes a)=za$ and $\psi(a)=1\otimes a$.

So an algebra can be defined as a vector space endowed with operations $\mu$ and $\eta$ such that the latter two diagrams commute. One define a \defe{coalgebra}{coalgebra} taking the dual of that construction. So a coalgebra is a vector space $\cA$ endowed with two linear maps
\begin{align*}
\epsilon&\colon \cA\to \eK,\\
\Delta&\colon \cA\to \cA\otimes_{\eK}\cA
\end{align*}
such that the following diagrams commute
\begin{align}	\label{EsDigcococo}
\xymatrix{%
&	 					\cA\otimes\cA\otimes\cA\\
\cA\otimes\cA \ar[ur]^{\Delta\otimes\id}&					&	\cA\otimes\cA\ar[ul]_{\id\otimes\Delta}\\
&						\cA\ar[ul]^{\Delta}\ar[ur]_{\Delta}
}
&&	% Début du second diagrame
\xymatrix{%
\eK\otimes\cA 		&	\cA\otimes\cA\ar[l]_{\epsilon\otimes\id}\\
&	\cA\ar[ul]^{\psi}\ar[u]_{\Delta}
   }
&&	% Début du troisième diagramme
\xymatrix{%
\cA\otimes\eK 		&	\cA\otimes\cA\ar[l]_{\id\otimes\epsilon}\\
&	\cA\ar[ul]^{\psi}\ar[u]_{\Delta}
   }\\
   (\id\otimes\Delta)\Delta=(\Delta\otimes \id)\Delta   &&   (\id\otimes\epsilon)\Delta=\id&&(\epsilon\otimes\id)\Delta=\id.
\end{align}
The first diagram expresses the \defe{coassociativity}{coassociativity} of $\Delta$ while the two last ones are the \defe{counit}{counit} definition for $\epsilon$. When these three diagrams commute, we say that $(\cA,\Delta,\epsilon)$ is a coalgebra.


Notice that, since the tensor product is defined by a quotient, we have $1\otimes za=z\otimes a$ for every $z\in\eC$ and $a\in\cA$. We also have $(\epsilon\otimes\id)(a\otimes b)=\epsilon(a)\otimes b=1\otimes\epsilon(a)b$.

\begin{definition}      \label{DefBialgebra}
    A \defe{bialgebra}{bialgebra} structure on $\cA$ is the data of $(\mu,\eta,\Delta,\epsilon)$ such as before (the six diagrams commute) and which fulfills the following compatibility conditions:
    \begin{align}       \label{EqSixBialgebraformdef}
    \Delta(hg)&=\Delta(h)\Delta(g)&			\epsilon(hg)&=\epsilon(h)\epsilon(g)\\
    \Delta(1)&=1\otimes 1		&		\epsilon(1)&=1
    \end{align}
    where in the last equality, the $1$ of the left hand side denotes the unit in $\cA$ while the $1$ in the right hand side is the unit in $\eK$. Such an operation $\Delta$ is said to be a \defe{coproduct}{coproduct} and such an operation $\epsilon$ is a \defe{counit}{counit}.
\end{definition}

It is interesting to write the formula \( \Delta(ab)=\Delta(a)\Delta(b)\) in a more abstract way. The product 
\begin{equation}
    (a\otimes b)(c\otimes d)=ac\otimes bd
\end{equation}
reads $(\mu\otimes\mu)(\id\otimes\sigma\otimes\id)(a\otimes b\otimes c\otimes d)$. Thus we consider
\begin{equation}        \label{EqDefmdeuxAAotAAmtAA}
    \begin{aligned}
        m_2\colon (A\otimes A)\otimes (A\otimes A)&\to A\otimes A \\
        m_2&=(\mu\otimes\mu)\circ(\id\otimes\sigma\otimes\id)
    \end{aligned}
\end{equation}
and we have the property
\begin{equation}        \label{EqDelMumdex}
    \Delta\circ\mu=m_2(\Delta\otimes\Delta).
\end{equation}


Among other interesting formulas, the identification \( A\otimes \eK=A\) allows us to write
\begin{equation}    \label{EqInterAmongOtheaaepsb}
    a=(\id\otimes\epsilon)\Delta(a)=\sum_i(\id\otimes \epsilon)(a_i\otimes b_i)=\sum_ia_i\epsilon(b_i).
\end{equation}

\begin{lemma}       \label{LemUnicityCounit}
    A coalgebra has at most one counit.
\end{lemma}

\begin{proof}
    Let \( \epsilon_1\) and \( \epsilon_2\) be counits of \( (A,\Delta)\). We have
    \begin{equation}
            \epsilon_1=\epsilon_1\circ(\id\otimes\epsilon_2)\Delta
            =(\epsilon_1\otimes\epsilon_2)\Delta
            =\epsilon_2(\epsilon_1\otimes\id)\Delta
            =\epsilon_2.
    \end{equation}
\end{proof}


%---------------------------------------------------------------------------------------------------------------------------
\subsection{Other co-properties}
%---------------------------------------------------------------------------------------------------------------------------
\label{subSecOtherCoPropoerties}

We usually denote by \( \sigma\) the flip operation:
\begin{equation}
    \begin{aligned}
        \sigma\colon \cA\otimes\cA&\to \cA\otimes\cA \\
        a\otimes b&\mapsto b\otimes a. 
    \end{aligned}
\end{equation}

As a general rule, one defines a ``coproperty'' by writing a property as a commutative diagram and then reversing all the arrows. A map \( \varphi\colon \cA\otimes\cA\to \cA\) is skew symmetric if the following diagram commutes:
\begin{equation}
    \xymatrix{%
    \cA\otimes\cA \ar[r]^-{\varphi}\ar[d]_{\sigma}        &   \cA\\
       \cA\otimes\cA \ar[r]_-{\varphi}   &   \cA\ar[u]_{-1}
       }
\end{equation}
Then one say that a map \( \phi\colon \cA\to \cA\otimes\cA\) is \defe{co-skew symmetric}{co-skew symmetric} if the diagram
\begin{equation}
    \xymatrix{%
    \cA\otimes\cA       &   \cA\ar[l]_-{\phi}\ar[d]^{-1}\\
    \cA\otimes\cA\ar[u]^{\sigma}   &     \cA\ar[l]^-{\phi}
    }
\end{equation}
commutes. In formula, it means that for every \( a\in\cA\), we have \( \phi(a)=\sigma\phi(-a)\) or
\begin{equation}        \label{EqDefCoSkewSym}
    (\sigma+\id)\circ\phi=0.
\end{equation}
A typical example of co-skew symmetric map is \( \phi(a)=1\otimes a-a\otimes 1\).

Let \( \xi\) be the cyclic permutation operator
\begin{equation}
    \begin{aligned}
        \xi\colon \cA\otimes\cA\otimes\cA&\to \cA\otimes\cA\otimes\cA \\
        a\otimes b\otimes c&\mapsto b\otimes c\otimes a.
    \end{aligned}
\end{equation}
A map \( \phi\colon \cA\otimes\cA\to \cA\) satisfies the \defe{Jacobi}{Jacobi} relation
\begin{equation}
    \varphi\big( a,\varphi(b,c) \big)=-\varphi\big( b,\varphi(c,a) \big)-\varphi\big( c,\varphi(a,b) \big)
\end{equation}
if the diagram 
\begin{equation}
    \xymatrix{%
    \cA\otimes\cA\otimes\cA\ar[r]^-{\id\otimes\varphi}\ar[d]_{-\xi-\xi^2}   &   \cA\otimes\cA\ar[r]^-{\varphi}      &   \cA\\
    \cA\otimes\cA\otimes\cA\ar[rr]_-{\id\otimes\varphi} &   &   \cA\otimes\cA\ar[u]_-{\varphi}
       }
\end{equation}
commutes. Thus one says that a map \( \phi\colon \cA\to \cA\otimes\cA\) satisfies the \defe{co-Jacobi}{co-Jacobi} relation if the diagram
\begin{equation}
    \xymatrix{%
    \cA\otimes\cA\otimes\cA     &       \cA\otimes\cA\ar[l]_-{\id\otimes\phi}       &       \cA\ar[l]_-{\phi}\ar[d]^-{\phi}\\
    \cA\otimes\cA\otimes\cA\ar[u]^-{-\xi-\xi^2}     &                               &       \cA\otimes\cA\ar[ll]^-{\id\otimes\phi}
    }
\end{equation}
commutes. In formula,
\begin{equation}        \label{EqDefCoJacobi}
    (\id+\xi+\xi^2)\circ(\id\otimes\phi)\circ\phi=0.
\end{equation}

%+++++++++++++++++++++++++++++++++++++++++++++++++++++++++++++++++++++++++++++++++++++++++++++++++++++++++++++++++++++++++++
\section{Hopf algebras}
%+++++++++++++++++++++++++++++++++++++++++++++++++++++++++++++++++++++++++++++++++++++++++++++++++++++++++++++++++++++++++++

%---------------------------------------------------------------------------------------------------------------------------
\subsection{Convolution product}
%---------------------------------------------------------------------------------------------------------------------------

We define the \defe{convolution product}{convolution!over a bialgebra} between two linear maps $f,g\colon \cA\to \cA$ by
\begin{equation}
f\star g=\mu(f\otimes g)\Delta
\end{equation}

\begin{lemma}
The map $\eta\circ\epsilon$ is a neutral for the convolution product.
\end{lemma}

 \begin{proof}
First, notice that in general, $(\eta\circ\epsilon)(a)=\epsilon(a)1$. Let now consider $a\in\cA$ and $a_1$, $a_2\in\cA$ such that $\Delta(a)=a_1\otimes a_2$. We look at
\begin{equation} 
  \mu(f\otimes g)\Delta(a)=f(a_1)g(a_2)
\end{equation}
which, using linearity of $g$, becomes $\mu(\eta\circ\epsilon\otimes g)\Delta(a)=\epsilon(a_1)1g(a_2)=\epsilon(a_1)g(a_2)=g\big( \epsilon(a_1)a_2 \big)$. Now if we see $\epsilon(a_1)a_2\in \eK\otimes\cA$, using the commutativity of the second diagram in \eqref{EsDigcococo}, we deduce $\Delta\big( \epsilon(a_1)a_2 \big)=a_1\otimes a_2=\Delta(a)$, so that $\epsilon(a_1)a_2=a$ by injectivity of $\Delta$.

In the same way, we prove that $a_1\epsilon(a_2)=a$ and that, consequently, 
\[ 
  \mu(f\otimes\eta\circ\epsilon)\Delta(a)=f\big( a_1\epsilon(a_2) \big)=f(a).
\]
\end{proof}

A bialgebra is a Hopf algebra if the identity map $\id\colon \cA\to \cA$ is invertible for the convolution product. In that case the inverse is denoted by $S$ and is called \defe{antipode}{antipode}. The defining property of $S$ is
\[ 
  S\star \id=\id\star S=\eta\circ\epsilon.
\]
That can equivalently be expressed by the commutativity of the diagram
\begin{equation}
    \xymatrix{%
    A\otimes A\ar[d]_{\id\otimes S}         &   A\ar[l]_-{\Delta}\ar[r]^-{\Delta}\ar[d]_{\eta\circ\epsilon}           &       A\otimes A\ar[d]^{S\otimes \id}\\
    A\otimes A\ar[r]_-{\mu}                  &   A                                                                   &       A\otimes A\ar[l]^-{\mu}.
       }
\end{equation}

\begin{definition}      \label{DefHopfAlgebra}
    A tuple \( (A,\mu,\eta,\Delta,\epsilon,S )\) is a \defe{Hopf algebra}{Hopf algebra} if the maps
    \begin{equation}
        \begin{aligned}[]
            \mu&\colon A\otimes A\to A   &   \Delta&\colon A\to A\otimes A    &   S&\colon A\to A\\
            \eta&\colon \eK\to A         &  \epsilon&\colon A\to A
        \end{aligned}
    \end{equation}
    satisfy
    \begin{enumerate}
        \item
            \( (A,\mu,\eta)\) is an associative algebra:       
            \begin{subequations}
                \begin{align}
                    \mu(\mu\otimes\id)=\mu(\id\otimes\mu)\\
                    \mu(\eta\otimes\id)=\mu(\id\otimes\eta),
                \end{align}
            \end{subequations}
        \item
            \( (A,\Delta,\epsilon)\) is a coassociative coalgebra:
            \begin{subequations}
                \begin{align}
                    (\id\otimes\Delta)\Delta=(\Delta\otimes\id)\Delta\\
                    (\id\otimes\epsilon)\Delta=(\epsilon\otimes\id)\Delta=\id,      \label{EqformCounitDef}
                \end{align}
            \end{subequations}
        \item
            the compatibility relations \eqref{EqSixBialgebraformdef} which turn \( (A,\mu,\eta,\Delta,\epsilon)\) into a bialgebra:
            \begin{subequations}
                \begin{align}
                    \Delta\mu&=\mu(\Delta\otimes\Delta)&\epsilon\mu&=\mu(\epsilon\otimes\epsilon)\\
                    \Delta(1)&=1\otimes 1   &\epsilon(1)&=1,
                \end{align}
            \end{subequations}
        \item
            \( S\) is an antipode:
            \begin{equation}        \label{EqDefPorpAntipode}
                \mu(\id\otimes S)\Delta=\mu(S\otimes\id)\Delta=\eta\epsilon.
            \end{equation}
    \end{enumerate}
    In all these conditions, we identify \( A=A\otimes \eK=\eK\otimes A\).
\end{definition}

\begin{lemma}       \label{LemSuuSetaeta}
    We have 
    \begin{enumerate}
        \item       \label{ItemLemSuuSetaetai}
            \( S(1)=1\) and \( S\circ\eta=\eta\);
        \item       \label{ItemLemSuuSetaetaii}
            \( S\mu=\mu(S\otimes S)\sigma\) where \( \mu\) is the multiplication on \( A\) and \( \sigma\) is the flip operator on \( A\otimes A\).
    \end{enumerate}
    
\end{lemma}

\begin{proof}
    On the one hand we have \( 1=(\eta\circ\epsilon)(1)\) and on the other hand, by the property \eqref{EqDefPorpAntipode} of the antipode,
    \begin{equation}
        (\eta\circ\epsilon)(1)=\mu(S\otimes \id)\Delta(1)=S(1)
    \end{equation}
    because \( \Delta(1)=1\otimes 1\). Thus \( S(1)=1\). Now if \( z\in\eK\), we have \( \eta(z)=z\cdot 1 \) and \( S(z\cdot 1)=zS(1)=z\cdot 1\). This proves \ref{ItemLemSuuSetaetai}.

    The statement \ref{ItemLemSuuSetaetaii} is a reformulation of the identity \( S(ab)=S(b)S(a)\).
\end{proof}

\begin{proposition}
    The antipode satisfies
    \begin{equation}
        \Delta\circ S=\sigma(S\otimes S)\Delta=(S\otimes S)\sigma\Delta.
    \end{equation}
\end{proposition}

\begin{proof}
    We will prove that \( \Delta S\) and \( \sigma(S\otimes S)\Delta\) are both inverses of \( \Delta\) in a well chosen algebra. We consider $F=L(A,A\otimes A)$ with the product 
    \begin{equation}
        f\star g=m_2(f\otimes g)\Delta
    \end{equation}
    where
    \begin{equation}
        m_2=(\mu\otimes\mu)(\id\otimes \sigma\otimes\id).
    \end{equation}
    The product \( m_2\) is the one introduced in equation \eqref{EqDefmdeuxAAotAAmtAA}. Note that \( f\star g\colon A\to A\otimes A\). The unit of that algebra is \( \eta_2\epsilon\) where \( \eta_2=\eta\otimes \eta\). That function \( A\to A\otimes A\) has to be understood as identifying \( \eK=\eK\otimes\eK\) and then
    \begin{equation}
        \eta_2\epsilon(a)=(\eta\otimes\eta)\epsilon(a)(1\otimes 1)=\epsilon(a)1\otimes 1.
    \end{equation}
    In the latter formula the ``\( 1\)'' is the unit in \( A\). Let us show that this is the unit of the algebra \( F\). As far as the notations are concerned, we write \( \Delta(a)=\sum_ia_i\otimes b_i\) and  \( f(b_i)=\sum_k f_k(b_i)\otimes g_k(b_i)\), so
    \begin{equation}
        \begin{aligned}[]
            (\eta_2\epsilon\star f)(a)&=m_2(\eta_2\epsilon\otimes f)\Delta(a)\\
            &=\sum_i(\mu\otimes\mu)(\id\otimes\sigma\otimes\id)(\eta_2\epsilon\otimes f)(a_i\otimes b_i)\\
            &=\sum_i(\mu\otimes\mu)(\id\otimes\sigma\otimes\id)\big( \epsilon(a_i)1\otimes 1\otimes f(b_i) \big)\\
            &=\sum_{ik}\epsilon(a_i)(\mu\otimes\mu)(\id\otimes\sigma\otimes\id)\big( 1\otimes 1\otimes f_k(b_i)\otimes g_k(b_i) \big)\\
            &=\sum_{ik}\epsilon(a_i)(\mu\otimes\mu)\big( 1\otimes f_k(b_i)\otimes 1\otimes g_k(b_i) \big)\\
            &=\sum_{ik}\epsilon(a_i)f_k(b_i)\otimes g_k(b_i)\\
            &=\sum_i\epsilon(a_i)f(b_i)\\
            &=\sum_if\big( \epsilon(a_i)b_i\big)\\
            &=f(a).
        \end{aligned}
    \end{equation}
    Thus \( \eta_2\epsilon\star f=f\).

    Now in order to check that \( \Delta\circ S\) is an inverse of \( \Delta\), we have to check that
    \begin{equation}
        m_2(\Delta\otimes \Delta S)\Delta=\eta_2\epsilon.
    \end{equation}

    Notice that \( \eta_2\) satisfies \( \Delta\eta\epsilon=\eta_2\epsilon\). Indeed \( \Delta(1)=1\otimes 1\) and \( \eta\epsilon(a)=\epsilon(a)1\), so
    \begin{equation}
        \Delta\eta\epsilon(a)=\epsilon(a)1\otimes\epsilon(a)1=(\eta\otimes \eta)(a).
    \end{equation}
    Using the formula \eqref{EqDelMumdex} we have
    \begin{equation}
        \begin{aligned}[]
            m_2(\Delta\otimes\Delta S)\Delta&=m_2(\Delta\otimes\Delta)(\id\otimes S)\Delta\\
            &=\Delta\mu(\id\otimes S)\Delta\\
            &=\Delta\eta\epsilon\\
            &=\eta_2\epsilon.
        \end{aligned}
    \end{equation}
    Thus \( \Delta\circ S\) is an inverse of \( \Delta\) in the algebra \( F\). Let us now check that \( \sigma(S\otimes S)\Delta\) is also an inverse of \( \Delta\). We introduce the short hand notation \( \Delta^3=(\Delta\otimes \Delta)\Delta\). Let us show that this operator satisfies
    \begin{equation}        \label{EqDDDrelDifetc}
        (\Delta\otimes\Delta)\Delta=(\id\otimes\Delta\otimes\id)(\id\otimes\Delta)\Delta.
    \end{equation}
    Using the relation \( (\id\otimes\Delta)\Delta=(\Delta\otimes\id)\Delta\) we have
    \begin{equation}
        \begin{aligned}[]
            (\Delta\otimes\Delta)\Delta&=(\Delta\otimes\id\otimes\id)(\id\otimes\Delta)\Delta\\
            &=(\Delta\otimes\id\otimes\id)(\Delta\otimes\id)\Delta\\
            &=\big( (\Delta\otimes\id)\Delta\otimes\id \big)\Delta\\
            &=\big( (\id\otimes\Delta)\Delta\otimes\id \big)\Delta\\
            &=(\id\otimes\Delta\otimes\id)(\Delta\otimes\id)\Delta\\
            &=(\id\otimes\Delta\otimes\id)(\id\otimes\Delta)\Delta.
        \end{aligned}
    \end{equation}
    
    We have
    \begin{equation}
        \begin{aligned}[]
            \diamondsuit&=m_2(\Delta\otimes\sigma(S\otimes S)\Delta)\Delta\\
            &=m_2\Big[ (\id\otimes\id)\otimes\big( \sigma(S\otimes S) \big) \Big](\Delta\otimes \Delta)\Delta\\
            &=m_2\sigma_{34}(\id\otimes\id\otimes S\otimes S)\Delta^3\\
            &=(\mu\otimes\mu)(\id\otimes\sigma\otimes\id)\sigma_{34}(\id\otimes\id\otimes S\otimes S)\Delta^3
        \end{aligned}
    \end{equation}
    where \( \sigma_{34}\) is the flip of the third and fourth component in \( A\otimes A\otimes A\otimes A\). Now we have
    \begin{equation}
        (\id\otimes\sigma\otimes \id)\sigma_{34}(a\otimes b\otimes c\otimes d)=(\id\otimes\sigma\otimes\id)(a\otimes b\otimes d\otimes c)=a\otimes d\otimes b\otimes c,
    \end{equation}
    thus \( (\id\sigma\otimes \id)\sigma_{34}=\sigma_{234}\) where \( \sigma_{234}\) is the cyclic permutation \( a\otimes b\otimes c\otimes d\mapsto a\otimes d\otimes b\otimes c\). Using that and the relation \eqref{EqDDDrelDifetc} we continue the computation:
    \begin{subequations}
        \begin{align}
            \diamondsuit&=(\mu\otimes\mu)\sigma_{234}(\id\otimes\id\otimes S\otimes S)\Delta^3\\
            &=\Big[ \mu(\id\otimes S)\otimes\mu(\id\otimes S) \Big]\sigma_{234}(\Delta\otimes \Delta)\Delta\\
            &=\Big[ \mu(\id\otimes S)\otimes\mu(\id\otimes S) \Big]\sigma_{234}(\id\otimes\Delta\otimes\id)(\id\otimes\Delta)\Delta\\
            &=\Big[ \mu(\id\otimes S)\otimes\mu(\id\otimes S) \Big](\id\otimes\id\otimes\Delta)(\id\otimes\sigma)   (\id\otimes\Delta)\Delta.
        \end{align}
    \end{subequations}
    On the last line we did \( \sigma_{234}(\id\otimes\Delta\otimes\id)=(\id\otimes\id\otimes\Delta)(\id\otimes\sigma)\); one checks that equality by applying both sides to \( a\otimes b\otimes c\). In the following lines we use \( (\id\otimes\epsilon)\sigma=\sigma(\epsilon\otimes\id)\):
    \begin{subequations}
        \begin{align}
            \diamondsuit&=\Big[ \mu(\id\otimes S)\otimes \underbrace{\mu(\id\otimes S)\Delta}_{\eta\circ\epsilon} \Big](\id\otimes\sigma)(\id\otimes\Delta)\Delta\\
            &=\Big[ \mu(\id\otimes S)\otimes\eta \Big](\id\otimes\id\otimes\epsilon)(\id\otimes\sigma)(\id\otimes\Delta)\Delta\\
            &=\Big[ \mu(\id\otimes S)\otimes\eta \Big](\id\otimes \sigma\underbrace{(\epsilon\otimes\id)\Delta}_{1\otimes\id} )\Delta\\
            &=\Big[ \mu(\id\otimes S)\otimes\eta \Big](\id\otimes\id\otimes 1)\Delta\\
            &=\mu(\id\otimes S)\Delta\otimes 1\\
            &=\eta\epsilon\otimes 1.
        \end{align}
    \end{subequations}
    The last line is the map \( a\mapsto \epsilon(a)1\otimes 1\) and then is the unit for \( L(A,A\otimes A)\).   
\end{proof}

%---------------------------------------------------------------------------------------------------------------------------
\subsection{Opposite algebra}
%---------------------------------------------------------------------------------------------------------------------------

From a Hopf algebra \( (A,\Delta)\) we define the \defe{opposite algebra}{opposite algebra} \( (A,\Delta)^{op}=(A^{op},\Delta)\)\nomenclature[A]{$(A,\Delta)^{op}$}{opposite algebra} and the coopposite coalgebra \( (A,\Delta)^{cop}=(A,\sigma\circ\Delta)\)\nomenclature[A]{$(A,\Delta)^{cop}$}{coopposite algebra}. The opposite and coopposite algebras are not always Hopf algebras.

\begin{lemma}       \label{LemATSopHofounon}
    Let \( (A,\Delta)\) be a Hopf algebra and a linear map \( T\colon A\to A\). The following statements are equivalent:
    \begin{enumerate}
        \item\label{ItemLemATSopHofounoni}
            the bialgebra \( (A,\Delta)^{op}\) is Hopf with \( T\) as antipode;
        \item\label{ItemLemATSopHofounonii}
            \( \mu\circ\sigma\circ(T\otimes\id)\circ\Delta=\mu\circ\sigma\circ(\id\otimes T)\circ\Delta=\eta\circ\epsilon\);
        \item\label{ItemLemATSopHofounoniii}
            \( \sum_i b_iT(a_i)=\sum_i T(b_i)a_i=\eta\circ\epsilon(a)\) where \( \Delta a=\sum_i a_i\otimes b_i\);
        \item\label{ItemLemATSopHofounoniv}
            \( \mu\circ(\id\otimes T)\circ\sigma\circ\Delta=\mu\circ(T\otimes\id)\circ\sigma\circ\Delta=\eta\circ\epsilon\);
        \item\label{ItemLemATSopHofounonv}
            the bialgebra \( (A,\Delta)^{cop}\) is Hopf with \( T\) as antipode.
    \end{enumerate}
\end{lemma}

\begin{proof}
    The equivalence \ref{ItemLemATSopHofounoni} \( \Leftrightarrow\) \ref{ItemLemATSopHofounonii} is the definition of an antipode. For the equivalence \ref{ItemLemATSopHofounoni} \( \Leftrightarrow\) \ref{ItemLemATSopHofounoniii}, following the definition of an antipode on \( (A,\Delta)^{op}\), the two following lines are equal to \( \eta\epsilon(a)\):
    \begin{equation}
        \mu\sigma(T\otimes \id)\Delta a=\sum_i\mu\sigma T(a_i)\otimes b_i=\sum_i b_i T(a_i)
    \end{equation}
    while on the other hand
    \begin{equation}
        \mu\sigma(\id\otimes T)\Delta a=\mu\sigma a_i\otimes T(b_i)=\sum_i T(b_i)a_i.
    \end{equation}
    
    The equivalence \ref{ItemLemATSopHofounonii} \( \Leftrightarrow\) \ref{ItemLemATSopHofounoniv} is the fact that \( \sigma(\alpha\otimes\beta)=(\beta\otimes \alpha)\sigma\) whenever \( \alpha,\beta\colon A\to A\).

    The equivalence \ref{ItemLemATSopHofounoniv} \( \Leftrightarrow\) \ref{ItemLemATSopHofounonv} is the definition of the antipode on \( (A,\Delta)^{cop}\) since in the latter algebra the comultiplication is given by \( \sigma\circ\Delta\) instead of \( \Delta\).

    The proof is finished.
\end{proof}

\begin{proposition}     \label{PropAAopAcopHopfSSemu}
    Let \( (A,\Delta)\) be a Hopf algebra. The following statements are equivalent:
    \begin{enumerate}
        \item   \label{ItemPropAAopAcopHopfSSemui}
            \( S\) is bijective;
        \item\label{ItemPropAAopAcopHopfSSemuii}
            \( (A,\Delta)^{op}\) is a Hopf algebra with antipode \( S^{-1}\);
        \item\label{ItemPropAAopAcopHopfSSemuiii}
            \( (A,\Delta)^{cop}\) is a Hopf algebra with antipode \( S^{-1}\).
    \end{enumerate}
    In this case the antipode of \( (A,\Delta)^{op}\) and \( (A,\Delta)^{cop}\) is given by \( S^{-1}\).
\end{proposition}

\begin{proof}
    First we suppose that \( S\) is invertible and we prove that $S^{-1}$ is an antipode for \( (A,\Delta)^{op}\); in this case it will be an antipode for \( (A,\Delta)^{cop}\) by the points \ref{ItemLemATSopHofounoni} and \ref{ItemLemATSopHofounonv} of lemma \ref{LemATSopHofounon}. We have to check that
    \begin{equation}
        \mu^{op}(\id\otimes S^{-1})\Delta=\eta\circ\epsilon.
    \end{equation}
    First we have
    \begin{equation}
        \mu^{op}(\id\otimes S^{-1})\sum_ia_i\otimes b_i=S^{-1}(b_i)a_i.
    \end{equation}
    Let us take the antipode on both sides and use the fact that \( S\) is an antipode for \( (A,\Delta,\mu)\):
    \begin{equation}
        S\mu^{op}(\id\otimes S^{-1})\Delta (a)=\sum_iS(a_i)b_i=\mu(S\otimes \id)\Delta a=\eta\big( \epsilon(a) \big).
    \end{equation}
    Thus 
    \begin{equation}
        \mu^{op}(\id\otimes S^{-1})\Delta=S^{-1}\circ\eta\circ\epsilon,
    \end{equation}
    but \( S^{-1}\circ\eta=\eta\) by lemma \ref{LemSuuSetaeta}, so we have proved \ref{ItemPropAAopAcopHopfSSemui} \( \Rightarrow\) \ref{ItemPropAAopAcopHopfSSemuii},\ref{ItemPropAAopAcopHopfSSemuiii}.

    Let us now prove that \ref{ItemPropAAopAcopHopfSSemuii} \( \Rightarrow\) \ref{ItemPropAAopAcopHopfSSemui}. For this part of the proof we follow\cite{RolandVertignioux}. Let \( (A,\Delta)^{op}\) be a Hopf algebra with antipode \( T\). We are going to prove that \( ST\) is an inverse of \( S\) for the convolution product, so that \( ST=\id\) and \( T=S^{-1}\) and \( S\) is bijective. What we have to check is
    \begin{equation}
        \mu(ST\otimes S)\Delta=\eta\epsilon.
    \end{equation}
    Using \( \mu(S\otimes S)\sigma=S\mu\) we have
    \begin{equation}
        \begin{aligned}[]
            \mu(ST\otimes S)\Delta&=\mu(S\otimes S)(T\otimes \id)\Delta\\
            &=S\mu\sigma(T\otimes \id)\Delta\\
            &=S\mu(\id\otimes T)\sigma\Delta\\
            &=S\circ\eta\circ\epsilon
        \end{aligned}
    \end{equation}
    because \( \mu(\id\otimes T)\sigma\Delta=\eta\circ\epsilon\) from the fact that \( T\) is an antipode for \( (A,\Delta)^{cop}\). Now we conclude because \( S\eta=\eta\).
\end{proof}

\subsection{Dual of a Hopf algebra}\index{dual!of Hopf algebra}
%---------------------------------------------

Let $(\cA,\mu,\Delta,\eta,\epsilon,S)$ be a Hopf algebra and $\cA^*$ the dual as vector space. We can create a Hopf algebra structure $(m^*,\mu^*,\Delta^*,\epsilon^*,S^*)$ on $\cA^*$ in the following way. First we extend the duality by $\langle f\otimes g,x\otimes y, \rangle =\langle f, x\rangle \langle g, y\rangle $ and then we define the following
\begin{subequations}
	\begin{align}
        \langle m^*(f\otimes g), x\rangle &=\langle f\otimes g, \Delta(x)\rangle    \label{subeqDualHopfmult} \\
		\langle \Delta^*(f), x\otimes y\rangle &=\langle f, xy\rangle \\
		\langle \eta^*(\alpha), x\rangle &=\alpha\epsilon(x)\\
		\epsilon^*(f)&=\langle f, 1\rangle \\
		\langle S^*(f), x\rangle &=\langle f, S(x)\rangle 
	\end{align}
\end{subequations}
for every $f$, $g\in\cA^*$ and $x$, $y\in\cA$. One can show that these definitions fulfill all the condition of a Hopf algebra.

We say that a pair $(\cA,\cB)$ is a \defe{dual pair}{dual!pair of Hopf algebra} of $*$-Hopf algebra is there exists a map $\langle ., .\rangle \colon \cA\otimes\cB\to \eC$ which fulfils the conditions
\begin{enumerate}

	\item
		$\langle x\otimes y, \Delta(f)\rangle =\langle ab, f\rangle $,
	\item
		$\epsilon(f)=\langle 1, f\rangle $,
	\item
		$\langle \Delta(x), f\otimes g\rangle =\langle x, fg\rangle $,
	\item
		$\langle x, S(f)\rangle =\langle S(x), f\rangle $,
	\item
		$\langle x, f^*\rangle =\overline{ \langle S(x)^*, f\rangle  }$

\end{enumerate}
for every $x,y\in\cA$ and $f,g\in\cB$

%---------------------------------------------------------------------------------------------------------------------------
\subsection{Involution, $*$-Hopf algebra}
%---------------------------------------------------------------------------------------------------------------------------
\label{subsecHopfInvolution}

We follow \cite{TimmernannInvitation}. Let \( \eK\) be a commutative ring with unity.

\begin{definition}
    An \defe{involution}{involution}\index{involution!on complex vector space} on a complex vector space \( V\) is a map \( a\mapsto a^*\) such that
    
    \begin{enumerate}
        \item
            \( (a+b)^*=a^*+b^*\) for \( a,b\in V\);
        \item
            \( (\lambda a)^*=\overline{ \lambda }a^*\) for every \( \lambda\in\eK\) and \( a\in A\);
        \item
            \( (a^*)^*=a\).
    \end{enumerate}
    An involution\index{involution!on a $\eK$-algebra} on a \( \eK\)-algebra \( A\) is an involution \( a\mapsto a^*\) on \( A\) (as vector space) which satisfies \( (ab)^*=b^*a^*\) for every \( a,b\in A\). An algebra equipped with an involution is a \( *\)-algebra.
\end{definition}

\begin{probleme}
    In \cite{SoibelmanI}, one does not require \( a^{**}=a\) while it required in \cite{TimmernannInvitation}. In \cite{Kassel} they include \( S\big( S(x)^* \big)^*=x\) in the definition of a Hopf\( *\)-algebra.
\end{probleme}

\begin{definition}
    A \defe{\( *\)-coalgebra}{coalgebra!$*$-coalgebra} \( (A,\Delta,*)\) is a coalgebra equipped with an involution such that
    \begin{equation}
        \Delta(a^*)=\Delta(a)^*
    \end{equation}
    where \( (a\otimes b)^*=a^*\otimes b^*\).
\end{definition}

\begin{definition}
    A Hopf algebra \( (A,\Delta)\) equipped with an involution\index{involution!on Hopf algebra} \( *\) such that \( (A,*)\) is a \( *\)-algebra and \( (A,\Delta,*)\) is a \( *\)-coalgebra is a \defe{\( *\)-Hopf algebra}{Hopf algebra!$*$-Hopf algebra}.
\end{definition}


\begin{lemma}       \label{LemcounitstarHopfalg}
    The counit of a counital \( *\)-coalgebra is \( *\)-linear, namely we have \( \epsilon(a^*)=\overline{ \epsilon(a) }\) where the bar stands for the involution in \( \eK\).
\end{lemma}

\begin{proof}
    Let \( (A,\Delta,*)\) be a \( *\)-coalgebra with counit \( \epsilon\). We consider the map
    \begin{equation}
        \begin{aligned}
            \epsilon^*\colon A&\to \eK \\
            a&\mapsto \overline{ \epsilon(a^*) }. 
        \end{aligned}
    \end{equation}
    We prove that $\epsilon^*$ is a counit for \( A\). We have to check the property \eqref{EqformCounitDef} for \( \epsilon^*\). If \( \Delta a=\sum_ia_i\otimes b_i\),
    \begin{equation}
        \begin{aligned}[]
            (\id\otimes\epsilon^*)\Delta(a)&=\sum_ia_i\otimes\overline{ \epsilon(b_i^*) }\\
            &=\sum_ia_i\overline{ \epsilon(b_i^*) }     &&\text{$A\otimes\eK=A$}\\
            &=\sum_i\Big( \epsilon(b_i^*)a_i^* \Big)^*\\
            &=(a^*)^*       &&\text{by formula \eqref{EqInterAmongOtheaaepsb}}\\
            &=a.
        \end{aligned}
    \end{equation}
    By unicity of the counit (lemma \ref{LemUnicityCounit}) we have \( \epsilon^*=\epsilon\) and then \( \overline{ \epsilon(a^*) }=\epsilon(a)\).
\end{proof}

\begin{theorem}
    The antipode of a $*$-Hopf algebra is a bijection and \( S\circ *\circ S\circ *=\id\).
\end{theorem}

\begin{proof}
    We consider the map
    \begin{equation}
        \begin{aligned}
            S^*\colon A&\to A \\
            a&\mapsto S(a^*)^* 
        \end{aligned}
    \end{equation}
    and we show that \( S^*\) is an antipode for the opposite Hopf algebra \( (A,\Delta)^{op}\). We have to check the definition property \eqref{EqDefPorpAntipode} in \( A^{op}\). The multiplication in \( A^{op}\) being
    \begin{equation}
        \mu^{op}(a\otimes b)=ba
    \end{equation}
    we have
    \begin{equation}
        \begin{aligned}[]
            \mu^{op}(S^*\otimes \id)\Delta(a)&=\sum_ib_iS^*(a_i)\\
            &=\sum_ib_iS(a^*)^*\\
            &=\sum_i\Big( S(a^*)b_i^* \Big)^*.
        \end{aligned}
    \end{equation}
    Since \( S\) is an antipode for \( (A,\Delta)\) the property \eqref{EqDefPorpAntipode} provides
    \begin{equation}
        \sum_iS(a_i^*)b_i^*=\eta\epsilon(a^*),
    \end{equation}
    thus we have
    \begin{equation}
        \mu^{op}(S^*\otimes \id)\Delta(a)=\big( \eta\epsilon(a^*) \big)^*=\eta\epsilon(a).
    \end{equation}
    So \( S^*\) is an antipode for \( (A,\Delta)^{op}\) which becomes a Hopf algebra. Using the proposition \ref{PropAAopAcopHopfSSemu} it proves that \( S\) is bijective and that \( S^*\) is the inverse of \( S\).
\end{proof}

\begin{definition}
    Let \( A\) be a Hopf algebra. The action of \( A\) on itself defined by
    \begin{equation}
        \begin{aligned}
            \ad(a)\colon A&\to A \\
            b&\mapsto \sum_i a_i bS(b_i) 
        \end{aligned}
    \end{equation}
    where \( a\in A\) and \( \Delta(a)=\sum_i a_i\otimes b_i\) is the \defe{adjoint action}{adjoint!action!Hopf algebra}\nomenclature[A]{\( \ad\)}{adjoint action on Hopf algebra}.
\end{definition}

%---------------------------------------------------------------------------------------------------------------------------
\subsection{Examples: Hopf algebra of functions and universal enveloping algebra}
%---------------------------------------------------------------------------------------------------------------------------
\label{SubSecHoptUnivecvgp}

The set of function on a Lie group and the universal enveloping algebra of a Lie algebra are the two classical examples of Hopf algebras.
\begin{definition}		\label{DefHopfsurCG}
	Let $C(G)$ be the set of continuous functions on a topological group $G$. If we denote by $1$ the function $1\colon G\to \eC$, $1(x)=1$ for all $x\in G$, the following produces a structure of Hopf algebra on $C(G)$:
\begin{enumerate}
		\item
			$(f\cdot g)(x)=f(x)g(x)$,
		\item
			$\eta(\lambda)=\lambda 1$, the unit,
		\item\label{ItemHopfCGiii}
			$(\Delta f)(x\otimes y)=f(xy)$,
		\item\label{ItemHopfCGiv}
			$\epsilon(f)=f(e)$,
		\item
			$S(f)(x)=f(x^{-1})$.
	\end{enumerate}
\end{definition}

The item \ref{ItemHopfCGiii} deserves some comments. If $f\in C(G)$, the element $\Delta(f)\in C(G)\otimes C(G)$ is given by $\Delta(f)=f_{(1)}\otimes f_{(2)}$ with the requirement that
\begin{equation}
	f_{(1)}(x)f_{(2)}(y)=f(xy).
\end{equation}
Such choice is possible since by density of the functions of the form $f(x)g(y)$ in $C(G\times G)$, see section \ref{SecTensProdCSA}.

Let us check that $(\id\otimes\epsilon)\Delta=\id$. First $(\id\otimes\epsilon)\big( f_{(1)}\otimes f_{(2)} \big)=f_{(1)}\otimes f_{(2)}(e)$. Now we use the identification between $C(G)\otimes \eC$ and $C(G)$ ($(f\otimes z)(g)=zf(g)$) in order to get
\begin{equation}
	\Big( (\id\otimes\epsilon)\big( f_{(1)}\otimes f_{(2)} \big) \Big)(g)=\big( f_{(1)}\otimes f_{(2)}(e) \big)(g)=f_{(1)}(g)f_{(2)}(e)=f(ge)=f(g).
\end{equation}

\begin{probleme}
	Unicity of $f_{(1)}$ and $f_{(2)}$ seems doubtful to me.
\end{probleme}

The following puts a structure of Hopf algebra on the universal enveloping algebra $\mU(\lG)$ when $G$ is a Lie group:
\begin{enumerate}

	\item
		$X\cdot Y=X\otimes Y$, the ordinary multiplication in $\mU(\lG)$,
	\item
		$\Delta X=X\otimes 1+1\otimes X$,
	\item
		$\eta(\lambda)=\lambda 1$,
	\item	\label{ItemCounitUg}
		$\epsilon(1)=1$ and $\epsilon(X)=0$ otherwise
	\item
		$S(X)=-X$.

\end{enumerate}
It is proven in \cite{Tjin} that the two latter constructions are dual. If $\tilde X$ is the left invariant differential operator associated with $X\in\lG$ (see corollary \ref{CorUisomDiff} and the proposition \ref{PropbidiffUU}), the Leibnitz rules reads
\begin{equation}		\label{EqXfgDeltaUnif}
	\tilde X(fg)=\widetilde{\Delta(X)}(f\otimes g).
\end{equation}


\begin{proposition}     \label{PropHopfSurDual}
    Let \( A\) be a finite dimensional Hopf algebra on a field of characteristic zero. The following structure defines a canonical Hopf algebra structure on the dual \( A^*\)
    \begin{subequations}
        \begin{align}
            (f_1f_2)(a)&=(f_1\otimes f_2)\Delta a\\
            (\Delta f)(a\otimes b)&=f(ab)       \label{DefHopfSurAstar}\\
            S(f)a&=f\big( S(a) \big)\\
            \epsilon(f)&=f(1)
        \end{align}
    \end{subequations}
    where \( (f_1\otimes f_2)(a\otimes b)=f_1(a)f_2(b)\) 
\end{proposition}

\begin{proof}
    We check the property \( \Delta\big( S(f) \big)=(S\otimes S)\Delta'(f)\) where \( \Delta'=\sigma\Delta\) and \( \sigma(a\otimes b)=b\otimes a\). On the one hand we have
    \begin{equation}
        \Delta\big( S(f) \big)(a\otimes b)=S(f)(ab)=f\big( S(ab) \big)=f\big( S(b)S(a) \big).
    \end{equation}
    On the other hand, if
    \begin{equation}
        \Delta(f)=\sum_i f_i\otimes g_i
    \end{equation}
    we have
    \begin{equation}
        \begin{aligned}[]
            \clubsuit=(S\otimes S)(\Delta'f)(a\otimes b)&=(S\otimes S)\sum_i\sigma(f_i\otimes g_i)(a\otimes b)\\
            &=\sum_iS(g_i)\otimes S(f_i)(a\otimes b)\\
            &=\sum_i g_i\big( S(a) \big)f_i\big( S(b) \big).
        \end{aligned}
    \end{equation}
    Using the fact that \( \eK\) is commutative (it is a \wikipedia{en}{Field_(mathematics)}{field}), we continue
    \begin{equation}
        \begin{aligned}[]
            \clubsuit&=\sum_i f_i\big( S(b) \big)g_i\big( S(a) \big)\\
            &=\sum_i(f_i\otimes g_i)\big( S(b)\otimes S(a) \big)\\
            &=(\Delta f)\big( S(b)\otimes S(a) \big)\\
            &=f\big( S(b)S(a) \big).
        \end{aligned}
    \end{equation}
\end{proof}

\begin{definition}      \label{DefInvolutionHopf}
    Let \( A\) be a \( \eK\)-algebra with involution. A \( A\)-module \( V\) is \defe{unitarizable}{unitarizable} if there exists an Hermitian scalar product \( V\otimes V\to\eK\) such that
    \begin{equation}
        \langle av_1, v_2\rangle =\langle v_1, a^*v_2\rangle 
    \end{equation}
    for every \( a\in A\), \( v_1,v_2\in V\).
\end{definition}

\begin{proposition}     \label{PropAstarstaralg}
    Let \( A\) be a \( *\)-Hopf algebra. The dual \( A^*\) becomes a \( *\)-Hopf algebra with the involution \( l\mapsto l^*\) defined by
    \begin{equation}
        l^*(a)=\overline{ l\big( S(a)^* \big) }
    \end{equation}
    with \( l\in A^*\) and \( a\in A\).
\end{proposition}

\begin{proof}
    We check the condition \( \Delta(l^*)=\Delta(l)^*\). We recall the definition \eqref{DefHopfSurAstar}: \( \Delta(l)(a\otimes b)=l(ab)\). On the one hand we have
    \begin{equation}        \label{EqCalculDellstraotb}
        \begin{aligned}[]
            \Delta(l^*)(a\otimes b)&=l^*(ab)\\
            &=\overline{ l\big( S(ab)^* \big) }\\
            &=\overline{ l\Big(   \big( S(b)S(a) \big)^* \Big) }\\
            &=\overline{ l\big( S(a)^*S(b)^* \big) }
        \end{aligned}
    \end{equation}
    On the other hand if we write \( \Delta(l)=\sum_if_i\otimes g_i\) we have
    \begin{equation}
        \begin{aligned}[]
            \Delta(l)^*(a\otimes b)&=\sum_i(f_i^*\otimes g_i^*)(a\otimes b)\\
            &=\sum_i\overline{ f_i\big( S(a)^* \big) }\overline{ g_i\big( S(b)^* \big)}\\
            &=\sum_i\overline{ (f_i\otimes g_i)\big( S(a)^*\otimes S(b)^* \big) }\\
            &=\overline{ \Delta(l)\big( S(a)^*\otimes S(b)^* \big) }\\
            &=\overline{ l\big( S(a)^*S(b)^* \big) },
        \end{aligned}
    \end{equation}
    which is the same as the last line of \eqref{EqCalculDellstraotb}.
\end{proof}

Let \( A\) be a Hopf algebra. A linear form \( \alpha\in A^*\) is \defe{left invariant}{left invariant!linear form on a Hopf algebra} if
\begin{equation}
    (\id\otimes\alpha)\circ\Delta=\eta\circ\alpha.
\end{equation}
The linear form \( \alpha\) is \defe{right invariant}{right invariant!linear form on a Hopf algebra} if
\begin{equation}
    (\alpha\otimes\id)\circ\Delta=\eta\circ\alpha.
\end{equation}

%---------------------------------------------------------------------------------------------------------------------------
\subsection{Modules and representations}
%---------------------------------------------------------------------------------------------------------------------------
\label{SubSecMOdulREepe}

\begin{definition}
    Let \( A\) be a Hopf algebra. A \( A\)-\defe{module}{module!over Hopf algebra} is a module for the \emph{algebra} structure.
\end{definition}

Let \( V_1\) and \( V_2\) be two \( A\)-modules. The \defe{tensor product}{tensor product!of modules over Hopf algebra} is the \( \eK\)-module \( V_1\otimes V_2\) with the action of \( A\) given by
\begin{equation}
    a\cdot(v_1\otimes v_2)=\Delta(a)(v_1\otimes v_2).
\end{equation}

If \( A\) is a Hopf algebra on \( \eK\), we say that a \( A\)-module is \defe{finite dimensional}{finite dimensional!module over Hopf algebra} if it is a free \( \eK\)-module of finite type. The homomorphism \( \pi\colon A\to \End V\) corresponding to the module structure is a \defe{finite dimensional representation}{representation!of Hopf algebra}\index{Hopf algebra!representation}.

The following definitions are from \cite{RolandVertignioux}.

If \( V\) is a left \( A\)-module, the dual \( V^*\) becomes a right \( A\)-module by the definition
\begin{equation}
    (\alpha\cdot a)(v)=\alpha(a\cdot v)
\end{equation}
for every \( a\in A\), \( \alpha\in V^*\) and \( v\in V\). If \( V\) is a right module, we turn it into a left module using the antipode:
\begin{equation}
    a\cdot v=v\cdot S(a).
\end{equation}
In such a way, \( V^*\) is a left \( A\)-module by \( a\cdot\alpha=\alpha\cdot S(a)\), that is
\begin{equation}        \label{EqDefacctrleftUqGLstat}
    (a\cdot \alpha)(v)=\alpha\big( S(a)\cdot v \big).
\end{equation}

Yet another way to make \( V^*\) a left \( A\)-module is to use \( S^{-1}\) instead of \( S\). This will also produce a left module structure since \( S^{-1}(b)S^{-1}(a)=S^{-1}(ab)\). This representation of \( A\) will be denoted by \( \mL\) (see equation \eqref{EqDefregularLEftHopf}):
\begin{equation}        \label{EqDefTroisleftmodAstar}
    \begin{aligned}
        \mL\colon A&\to \End(V^*) \\
        \big( \mL(a)f \big)(v)&=f\big( S^{-1}(a)v \big). 
    \end{aligned}
\end{equation}

\begin{remark}
    This is not the regular left representation. One difference is that the regular left representation is well defined by itself while the action \( \mL(a)\colon V^*\to V^*\) is only defined when a representation of \( A\) is given on \( V\).    
\end{remark}

%---------------------------------------------------------------------------------------------------------------------------
\subsection{Matrix elements of modules}
%---------------------------------------------------------------------------------------------------------------------------

Let \( \eK\) be a commutative ring with unit and \( A\), an unital \( \eK\)-algebra. Let \( V\) be a \( A\)-module. A pair \( (l,v)\in V^*\times V\) defines a linear functional on \( A\) by
\begin{equation}
    a\mapsto l(av).
\end{equation}
This functional is the \defe{matrix element}{matrix!element of a module} of the representation \( \pi\colon A\to \End_{\eK}(V)\) corresponding to the pair \( (l,v)\). It is denoted by \( c^{\pi}_{l,v}\) or \( c^V_{l,v}\)\nomenclature[A]{\( c^{\pi}_{l,v}\)}{matrix element of a module}  \nomenclature[A]{\( c^{V}_{l,v}\)}{matrix element of a module}.

In this section we suppose that the \( A\)-modules are projective as \( \eK\)-modules.

\begin{proposition}     \label{PropHopfDual}
    Let \( A\) be a Hopf algebra on \( \eK\). The matrix elements of the finite dimensional \( A\)-modules form a Hopf subalgebra of the dual \( A^{\star}=\Hom_{\eK}(A,\eK)\).
\end{proposition}

The Hopf algebra defined in proposition \ref{PropHopfDual} is denoted by \( A^*\) and is the \defe{dual Hopf algebra}{dual!Hopf algebra}\index{Hopf algebra!dual} of \( A\).

\begin{definition}
    Let \( A\) be a Hopf algebra. The \defe{regular left representation}{regular!left representation of a Hopf algebra}\index{representation!regular!left on Hopf algebra} of \( A\) on \( A^*\) is given by
    \begin{equation}        \label{EqDefregularLEftHopf}
        \mR(a)f=\langle \id\otimes a, \Delta(f)\rangle .
    \end{equation}
    The \defe{regular right}{representation!regular right} is 
    \begin{equation}
        \mL(a)f=\langle S^{-1}(a)\otimes \id, \Delta(f)\rangle 
    \end{equation}
    where \( \Delta\) is the antipode of \( A^*\).
\end{definition}
More explicitly, for every \( a,x\in U_q\lG\),
\begin{equation}
    \big( \mR(a)f \big)(x)=(\Delta f)(x\otimes a)
\end{equation}
and
\begin{equation}
    \big( \mL(a)f \big)(x)=(\Delta f)\big( S^{-1}(a)\otimes x \big).
\end{equation}

One common idea is to see \( A^*\) as a \( U_q\lG\otimes U_q\lG\)-modules by the representation \( \mL\otimes\mR\). With that structure we have
\begin{equation}        \label{EqmRmLAAsurAstar}
    \big( (a\otimes b)f \big)(x)=\big( \mL(a)\mR(b)f \big)(x).
\end{equation}
Using the definition of the coproduct given in proposition \ref{PropHopfSurDual},
\begin{equation}        \label{EqmRmLabf}
    \begin{aligned}[]
        \big( \mL(a)\mR(b)f \big)(x)&=\Delta(\mR(b)f)\big( S^{-1}(a)\otimes x \big)\\
        &=\big( \mR(b)f \big)\big( S^{-1}(a)x \big)\\
        &=\Delta(f)\big( S^{-1}(a)x\otimes b \big)\\
        &=f\big( S^{-1}(a)xb \big).
    \end{aligned}
\end{equation}

%---------------------------------------------------------------------------------------------------------------------------
\subsection{Quasitriangular Hopf algebra}
%---------------------------------------------------------------------------------------------------------------------------

\begin{definition}
    A \defe{quasitriangular}{quasitriangular Hopf algebra}\index{Hopf algebra!quasitriangular} Hopf algebra (or a \defe{braid}{braid Hopf algebra}) is a Hopf algebra $(\cA,\mu,\eta,\Delta,\epsilon,S)$ together with an invertible element $R\in\cA\otimes\cA$ such that 
\begin{enumerate}
\item $R\Delta(x)R^{-1}=\tau\circ\Delta(x)$,\label{ItemCondUnifRi}
\item
\begin{enumerate}
\item $(\Delta\otimes\id)(R)=R_{13}R_{23}$,
\item $(\id\otimes\Delta)(R)=R_{13}R_{12}$\label{ItemCondUnifRiib}
\end{enumerate}

\end{enumerate}
In that situation, the element $R$ is said to be an \defe{universal $R$-matrix}{universal!$R$-matrix}.
\end{definition}


\begin{theorem}[Yang-Baxter equation]\index{Yang-Baxter equation}\index{equation!Yang-Baxter}
An universal $R$-matrix satisfies
\begin{equation}
R_{12}R_{13}R_{23}=R_{23}R_{13}R_{12}.
\end{equation}

\end{theorem}

\begin{proof}
We compute the quantity $(\id\otimes\tau\circ\Delta)R$ in two different ways. First we compute it using \ref{ItemCondUnifRiib}, and then we first use \ref{ItemCondUnifRi} before \ref{ItemCondUnifRiib}. In the first case we have (a sum is implied over repeated indices)
\begin{align*}
  (\id\otimes\tau\circ\Delta)R&=(\id\tau)\circ(\id\otimes\Delta)R\\
		&=(\id\otimes\tau)(R_{13}R_{12})\\
		&=(\id\otimes\tau)(a_i\otimes 1\otimes b_i)(a_k\otimes b_k\otimes 1)\\
		&=(\id\otimes\tau)(a_ia_k\otimes b_k\otimes b_i)\\
		&=a_ia_k\otimes b_i\otimes b_k\\
		&=(a_i\otimes b_i\otimes 1)(a_k\otimes 1\otimes b_k)\\
		&=R_{12}R_{13}.
\end{align*}
Now if we denote $\Delta(b_i)=b_{i1}\otimes b_{i2}$ and $R^{-1}=\alpha_k\otimes\beta_k$, remark that we have $R_{ij}^{-1}=(R^{-1})_{ij}$, so that
\begin{align*}
(\id\otimes\tau\circ\Delta)(a_i\otimes b_i)&=a_i\otimes\Big( (a_j\otimes b_j)(b_{i1}\otimes b_{i2}(\alpha_k\otimes\beta_k))   \Big)\\
		&=\Big( a_i\otimes\big( (a_i\otimes b_j)(b_{i1}\otimes b_{i2}) \big) \Big)(1\otimes\alpha_k\otimes\beta_k)
\end{align*}
The first factor of the last line can be written as
\[ 
  a_i\otimes a_jb_{i1}\otimes b_jb_{i2}=(1\otimes a_j\otimes b_j)(a_i\otimes b_{i1}\otimes b_{i2})=R_{23}\big( (\id\otimes\Delta)R \big)=R_{23}R_{13}R_{12},
\]
while the second factor is $(R^{-1})_{23}$. Finally we find $(\id\otimes\tau\circ\Delta)(a_i\otimes b_i)=R_{23}R_{13}R_{12}R_{23}^{-1}$. Equating with the first value obtained, we find the claim.
\end{proof}



%+++++++++++++++++++++++++++++++++++++++++++++++++++++++++++++++++++++++++++++++++++++++++++++++++++++++++++++++++++++++++++
\section{Convolution semigroup}
%+++++++++++++++++++++++++++++++++++++++++++++++++++++++++++++++++++++++++++++++++++++++++++++++++++++++++++++++++++++++++++

\begin{definition}
	A \defe{convolution semigroup}{convolution!semigroup} of linear functionals on $B$ is a set of functionals $\varphi_t$ such that
	\begin{enumerate}
		\item
			$\varphi_0=\epsilon$,
		\item
			$\lim_{t\searrow 0}\varphi_t(b)=\epsilon(b) $
		\item
			$\varphi_s*\varphi_t=\varphi_{s+t}$
	\end{enumerate}
	for every $b\in B$.
\end{definition}

\begin{definition}
	If $j_1$ and $j_2$ are linear functionals on a coalgebra $C$ taking values in an algebra $A$, we define the \defe{convolution}{convolution!on coalgebra} by
	\begin{equation}
		j_1*j_2=m_A\circ(j_1\otimes j_2)\circ\Delta.
	\end{equation}
\end{definition}

\begin{lemma}
	Let $C$ be a coalgebra. We have
	\begin{enumerate}
		\item
			If $\psi\colon C\to \eC$ is a linear functional on $C$, then the series
			\begin{equation}
				\exp(\psi)a=\sum_{n=0}^{\infty}\frac{ \psi^{*n} }{ n! }(a)=\epsilon(a)+\psi(a)+\frac{ 1 }{2}(\psi *\psi)(a)+\ldots
			\end{equation}
			converges for every $a\in C$. This defines the map $\exp(\psi)\colon C\to \eC$.
		\item
			Let $(\varphi_t)_{t\geq 0}$ be a convolution semigroup on $C$. Then the limit
			\begin{equation}
				L(a)=\lim_{t\searrow 0} \frac{1}{ t }\big( \varphi_t(a)-\epsilon(a) \big)
			\end{equation}
			exists for every $a\in C$. Moreover we have
			\begin{equation}
				\varphi_t=\exp(tL)
			\end{equation}
			for $t\geq 0$.
	\end{enumerate}
\end{lemma}

%+++++++++++++++++++++++++++++++++++++++++++++++++++++++++++++++++++++++++++++++++++++++++++++++++++++++++++++++++++++++++++
\section{Modules bialgebras}
%+++++++++++++++++++++++++++++++++++++++++++++++++++++++++++++++++++++++++++++++++++++++++++++++++++++++++++++++++++++++++++


%---------------------------------------------------------------------------------------------------------------------------
\subsection{Module (bi)algebras}
%---------------------------------------------------------------------------------------------------------------------------
\label{subSecModulebialgebra}

This subsection comes from \cite{GiaquintoZhangTwist} and we follow the notations of \cite{QuantifKhalerian}. The notions of module algebra and module coalgebra will be used when we will study twists in section \ref{SecTheoryTwist}.

Let $B$ be a bialgebra and $\modE$, a left $B$-module. For each $b\in B$, we have the map
\begin{equation}
	\begin{aligned}
		b_l\colon \modE&\to \modE \\
		b_l(m)&=b\cdot m.
	\end{aligned}
\end{equation}
In the same way, if $\modF$ is a right $B$-module, we have the map
\begin{equation}
	\begin{aligned}
		b_r\colon \modF&\to \modF \\
		b_r(m)&=m\cdot b.
	\end{aligned}
\end{equation}
In that case, $\modE\otimes\modE$ becomes a right $B\otimes B$-module by
\begin{equation}
	\begin{aligned}
		(b\otimes b')_l\colon \modE\otimes\modE&\to \modE\otimes\modE \\
		(b\otimes b')_l(m\otimes m')&=(b\cdot m)\otimes(b'\cdot m').
	\end{aligned}
\end{equation}

\begin{definition}		\label{DefBModuleAlgebra}
	A left $B$-module $\eA$ is a \defe{algebra $B$-module}{algebra!$B$-module} if for every $b\in B$ and $a,a'\in\eA$,
	\begin{enumerate}
	
		\item
			$b\cdot 1_{\eA}=\epsilon(b)\cdot 1_{\eA}$. This equality has to be seen in $\eK\subset B$
		\item
			$b\cdot(aa')=\sum(b_{(1)}\cdot a)(b_{(2)}\cdot a')$ where $\Delta_B(b)=\sum b_{(1)}\otimes b_{(2)}$.
	
	\end{enumerate}
	The second condition is the commutativity of the diagram
	\begin{equation}	\label{EqDiagModAlg}
		\xymatrix{%
		\eA\otimes\eA \ar[r]^{\mu_{\eA}}\ar[d]_{\big( \Delta_B(b) \big)_l}	&	\eA\ar[d]^{b_l}\\
		\eA\otimes\eA \ar[r]_{\mu_{\eA}}	&	\eA
		   }
	\end{equation}
	for every $b\in B$.
\end{definition}

\begin{definition}
	We say that $C$ is a coalgebra and $B$ is a bialgebra, we say that $C$ is a $B$-module coalgebra if it is a $B$-module satisfying
	\begin{enumerate}

		\item\label{ItemDefCoalgModuleUn}
			$\Delta(a\cdot b)=\sum (a_{(1)}\cdot b_{(1)})\otimes (a_{(2)}\otimes b_{(2)})$,
		\item
			$\epsilon(a\cdot b)=\epsilon(a)\epsilon(b)$

	\end{enumerate}
	for any $a\in C$ and $b\in B$.
\end{definition}
From a notational point of view, notice that
\begin{equation}
	\begin{aligned}[]
		\Delta_C(a)&=\sum_ia_{(1)}^i\otimes a_{(2)}^i\\
		\Delta_B(b)&=\sum_jb_{(1)}^j\otimes b_{(2)}^j
	\end{aligned}
\end{equation}
and the sum in the point \ref{ItemDefCoalgModuleUn} is a double sum. The formula is equivalent to commutativity of the diagram
\begin{equation}	\label{EqDiagModCoAlg}
	\xymatrix{%
	C\otimes C 		&	C\ar[l]_{\Delta}\\
	C\otimes C \ar[u]^{\Delta(b)_r}	&	C\ar[l]_{\Delta}\ar[u]_{b_r}
	   }
\end{equation}
for every $b\in B$, which is the dual diagram of \eqref{EqDiagModAlg}.

\subsection{Tensor product of representations}\index{tensor product!of algebra representations}
%---------------------------------------------

Let $(\psi_1,V_1)$ and $(\psi_2,V_2)$ be two representations of the algebra $\cA$. We want to define the tensor product $(\psi_1,V_)\otimes(\psi_2,V_2)$. The two first ideas are
\begin{subequations}
\begin{align}
a(v_1\otimes v_2)&=\big( \psi_1(a)v_1 \big)\otimes\big( \psi_2(a)v_2 \big)\\
a(v_1\otimes v_2)&=\psi_1(a)v_1\otimes v_2+v_1\otimes \psi_2(a)v_2.
\end{align}
\end{subequations}
The bad news is that the first possibility cannot be linear while the second one fails in genera to complete the homomorphism condition. Notice however that, on a Lie algebra, the second possibility works because the product is skew-symmetric. 

The trick in order to construct a tensor product of two representations is to take a map $\Delta\colon \cA\to \cA\otimes\cA$ and to define the tensor representation by
\begin{equation}
\Psi=(\psi_1\otimes\psi_2)\Delta,
\end{equation}
in other words if $a\in\cA$ and $\Delta(a)=a_1\otimes a_2$, we define
\begin{equation}
\Psi(a)(v_1\otimes v_2)=(\psi_1\otimes\psi_2)\Delta(a)(v_1\otimes v_2)=\psi_1(a_1)v_1\otimes\psi_2(a_2)v_2.
\end{equation}

\begin{proposition}
If we impose $\Psi$ to be 
\begin{itemize}
\item linear,
\item homomorphic,
\item associative, i.e. $\big((\psi_1,V_1)\otimes(\psi_2,V_2)\big)\otimes(\psi_3,V_3)=(\psi_1,V_1)\otimes\big( (\psi_2,V_2)\otimes(\psi_3,V_3)\big)$,
\end{itemize}
then $\Delta$ must be a comultiplication. 
\end{proposition}
\begin{proof}
No proof.
\end{proof}

\section{Poisson structure}
%++++++++++++++++++++++++++

A \defe{Poisson algebra}{Poisson structure! on an algebra} is a commutative algebra $(\cA,m,\eta)$ endowed with a map $\{ .,. \}\colon \cA\times\cA\to \cA$ called the \emph{Poisson bracket} such that
\begin{enumerate}
\item $\big( \cA,\{ .,. \} \big)$ is a Lie algebra,
\item $\{ ab,c \}=a\{ b,c \}+\{ a,c \}b$
\end{enumerate}
for all $a$, $b$, $c\in\cA$. The second condition shows that $\{ a,. \}$ is a derivation of $\cA$. Since the bracket is bilinear, it defined and is defined by a map $\gamma\colon \cA\otimes\cA\to \cA$ and the properties of the Poisson bracket are translated in terms of algebra structure in the following way.

First the antisymmetry is encoded by the requirement 
\begin{equation}
\gamma\circ\tau=\gamma.
\end{equation}
The Jacobi identities $\{ a,\{ b,c \} \}+\{ b,\{ c,a \} \}+\{ c,\{ a,b \} \}=0 $ becomes
\[ 
  \gamma\big( a\otimes \gamma(b\otimes c) \big)+\gamma\big( b\otimes\gamma(c\otimes a) \big)+\gamma\big( c\otimes\gamma(a\otimes b) \big)=0,
\]
or after rearrangements, 
\begin{equation}
\gamma(1\otimes\gamma)\big( 1\otimes 1\otimes 1+(1\otimes\tau)(\tau\otimes 1)+(\tau\otimes 1)(1\otimes\tau) \big)=0
\end{equation}
and the last condition reads
\begin{equation}
\gamma(m\otimes 1)=m(1\otimes\gamma)\big( 1\otimes 1\otimes 1+(1\otimes\tau) \big).
\end{equation}

\subsection{Homomorphism}
%-----------------------

Let $\big( \cA,m_{\cA},\{.,.\}_{\cA} \big)$ and $\big( \cA,m_{\cB},\{.,.\}_{\cB} \big)$ two Poisson algebras. An \defe{homomorphism}{homomorphism!of Poisson algebra} is a map $f\colon \cA\to \cB$ such that
\begin{enumerate}
\item $f(ab)=f(a)f(b)$,
\item $f\big( \{a,b\}_{\cA} \big)=\{f(a),f(b)\}_{\cB}$
\end{enumerate}
for every $a$, $b\in\cA$. These conditions can be written in terms of $m_{\cA}$, $m_{\cB}$, $\gamma_{\cA}$ and $\gamma_{\cB}$ :
\begin{itemize}
\item $f\circ \gamma_{\cA}=m_{\cB}\circ(f\otimes f)$,
\item $f\circ \gamma_{\cA}=\gamma_{\cB}\circ(f\otimes f)$.
\end{itemize}

\subsection{Tensor product}
%--------------------------

We can build the tensor product $\cA\otimes\cB$ and define the product $m_{\cA\otimes\cB}$ by
\[
(a_1\otimes b_1)\cdot(a_2\otimes b_2)=a_1a_2\otimes b_1b_2
\]
which can immediately be rewritten under the form
\begin{equation}
m_{\cA\otimes\cB}=(m_{\cA}\otimes m_{\cB})(1\otimes\tau \otimes 1).
\end{equation}
We define the Poisson bracket over the tensor product by
\[ 
  \{ a_1\otimes b_1,a_2\otimes b_2 \}=\{ a_1,a_2 \}\otimes b_1b_2+a_1a_2\otimes\{ b_1,b_2 \}
\]
which is nothing else that
\begin{equation}
\gamma_{\cA\otimes \cB}=(\gamma_{\cA}\otimes m_{\cB}+m_{\cA}\otimes \gamma_{\cB})(1\otimes \tau\otimes 1).
\end{equation}

\begin{definition}
A \defe{co-Poisson bialgebra}{co-Poisson bialgebra}\index{bialgebra!Poisson} is a co-commutative bialgebra $(\cA,m,\Delta,\eta,\epsilon)$ with a map $\delta\colon \cA\to \cA\otimes\cA$ such that
\begin{enumerate}
\item $\tau\circ\delta=-\delta$,
\item $\big( 1\otimes 1\otimes 1+(1\otimes\tau)(\tau\otimes 1)+(\tau\otimes 1)(1\otimes\tau) \big)(1\otimes\gamma)\delta=0$,
\item $(\Delta\otimes 1)\delta=(1\otimes 1\otimes 1+\tau\otimes 1)(1\otimes\delta)\Delta$,
\item $(m\otimes m)\circ\delta_{\cA\otimes\cA}=\delta\circ m$
\end{enumerate}
where $\delta_{\cA\otimes\cA}=(1\otimes\tau\otimes 1)(\delta\otimes\Delta+\Delta\otimes \delta)$ is the co-Poisson structure associated with the tensor product space.
\end{definition}

%+++++++++++++++++++++++++++++++++++++++++++++++++++++++++++++++++++++++++++++++++++++++++++++++++++++++++++++++++++++++++++
\section{Lie bialgebra}
%+++++++++++++++++++++++++++++++++++++++++++++++++++++++++++++++++++++++++++++++++++++++++++++++++++++++++++++++++++++++++++

\begin{definition}
    A vector space \( \lG\) equipped with a map \( \phi\colon \lG\to \lG\otimes\lG\) is a \defe{Lie coalgebra}{Lie!coalgebra}\cite{Farnsteiner} if the map \( \phi\) satisfies the co-skew symmetry and co-Jacobi properties, that is if 
    \begin{enumerate}
        \item
            $(\sigma+\id)\circ\phi=0$;
        \item
            $(\id+\xi+\xi^2)\circ(\id\otimes\phi)\circ\phi=0$
    \end{enumerate}
    where \( \sigma\) is the flip operator in \( \lG\otimes\lG\), \( \sigma(u\otimes v)=v\otimes u\) and \( \xi\) is the cyclic permutation operator on \( \lG\otimes\lG\otimes\lG\), \( \xi(u\otimes v\otimes w)=v\otimes w\otimes u\). See subsection \ref{subSecOtherCoPropoerties} for a justification of these expressions.
\end{definition}

\begin{definition}
    A pair \( (\lG,\phi)\) in which \( \lG\) is a Lie algebra and \( \phi\) is a Lie coalgebra structure on \( \lG\) is a \defe{Lie bialgebra}{Lie!bialgebra}\index{bialgebra!Lie} if the commutator and \( \phi\) satisfy the following compatibility property:
    \begin{equation}
        \phi\big( [X,Y] \big)=X\cdot\phi(Y)-Y\cdot\phi(X)
    \end{equation}
    where $X\cdot(Y\otimes Z)=[X,Y]\otimes Z+Y\otimes[X,Z]$.
\end{definition}

\begin{proposition}     \label{PropStandardBialgStruct} 
    Let \( \lG\) be a complex simple Lie algebra with its Killing form \( B\) and a Cartan subalgebra \( \lH\). Let \( n=\dim\lH\) be the rank of \( \lG\). Let \( X_i^{\pm}\) and \( H_i\) (\( i=1,\ldots,n\)) be the Chevalley generators and \( d_i\) be the numbers such that \( d_iA_{ij}=d_jA_{ji}\) given by the lemma \ref{LemRatdjaijdjaji}. Then the cobracket \( \phi\colon \lG\to \lG\otimes\lG\)
    \begin{equation}        \label{EqDefCobrackStandard}
        \begin{aligned}[]
            \phi(H_i)&=0\\
            \phi(X_i^{\pm})&=d_iX_i^{\pm}\wedge H_i
        \end{aligned}
    \end{equation}
    is a Lie bialgebra structure on \( \lG\).
\end{proposition}
This structure is the \defe{standard bialgebra structure}{standard!bialgebra structure} on \( \lG\). The definition \eqref{EqDefCobrackStandard} is given in the spirit of the remark \ref{RemChevDefmapCommXH}.

\begin{proof}
    There are three conditions to be satisfied.
    \begin{enumerate}
        \item
            If one apply \( (\sigma+\id)\circ\phi\) to \( H_i\), we get zero. If one apply this to \( X^{\pm}_i\), one gets
            \begin{equation}
                \begin{aligned}[]
                    (\sigma+\id)\circ\phi(X^{\pm}_i)&=d_i(\sigma+\id)(X_i^{\pm}\otimes H_i-H_i\otimes X^{\pm}_i)\\
                    &=d_i( H_i\otimes X_i^{\pm}-X^{\pm}\otimes H_i + X_i^{\pm}\otimes H_i-H_i\otimes X^{\pm}_i  )\\
                    &=0.
                \end{aligned}
            \end{equation}
        \item
            The co-Jacobi identity is easy to check on \( H_i\). For \( X^{\pm}_i\) we have
            \begin{equation}
                \begin{aligned}[]
                    (\id+\xi+\xi^2)\circ(\id\otimes\phi)\circ\phi(X^{\pm}_i)&=d_i(\id+\xi+\xi^2)\circ(\id\otimes\phi)(X^{\pm}_i\otimes H_i-H_i\otimes X_i^{\pm})\\
                    &=-d_i^2(\id+\xi+\xi^2)(H_i\otimes X^{\pm}_i\otimes H_i-H_i\otimes H_i\otimes X^{\pm}_i)\\
                    &=-d_i^2(H_i\otimes X^{\pm}_i\otimes H_i-H_i\otimes H_i\otimes X^{\pm}_i)\\
                    &\quad+d_i^2(X^{\pm}_i\otimes H_i\otimes H_i-H_i\otimes X^{\pm}_i\otimes H_i)\\
                    &\quad +d_i^2(H_i\otimes H_i\otimes X^{\pm}_i-X^{\pm}_i\otimes H_i\otimes H_i)\\
                    &=0.
                \end{aligned}
            \end{equation}
        \item
            We have to check the value of \( \phi\big( [H_i,X_{j}^{\pm}] \big)\) and \( \phi\big( [X_i^+,X_j^-] \big)\). For the first one we know that \( [H_i,X_j^{\pm}]=\alpha_j(H_i)X_j^{\pm}\), so that
            \begin{equation}        \label{EqphiHXcobrack}
                \phi\big( [H_i,X_j^{\pm}] \big)=\pm\alpha_j(H_i)\phi(X_j^{\pm})=\pm d_i\alpha_j(H_i) X_j^{\pm}\wedge H_i.
            \end{equation}
            On the other hand
            \begin{equation}
                \begin{aligned}[]
                    H_i\cdot \phi(X_j^{\pm})-X_j^{\pm}\cdot\phi(H_i)&=d_j H_i\cdot(X_j^{\pm}\otimes H_j-H_j\otimes X_j^{\pm})\\
                    &=d_j[H_i,X_j^{\pm}]\otimes H_j-d_j H_j\otimes[H_i,X_j^{\pm}]\\
                    &=d_j[H_i,X_j^{\pm}]\wedge H_j,
                \end{aligned}
            \end{equation}
            which is the same as what we found in \eqref{EqphiHXcobrack}.

            We check the property for \( \phi\big( [X_i^+,X_j^-] \big)=\phi(\delta_{ij}H_i)=0\) in the same way (but there are more terms).
    \end{enumerate}
    
\end{proof}


\section{Poisson-Lie group}
%++++++++++++++++++++++++++

\begin{definition}
    A Lie group is a \defe{Poisson-Lie group}{Poisson-Lie group} if the space of smooth functions is Poisson Hopf algebra.
\end{definition}
Notice that the Hopf structure on $ C^{\infty}(G)$ is defined by the group structure of $G$, so that the fact to be a Lie-Poisson group only imposes conditions on the Poisson structure. The condition to be fulfilled is
\begin{equation}	\label{EqCondcmpDelPoiss}
\{ \Delta(a),\Delta(b) \}_{\cA\otimes\cA}=\Delta\big( \{ a,b \}_{\cA} \big).
\end{equation}

We know on the other hand that every Poisson structure on $ C^{\infty}(G)$ reads
\begin{equation}	\label{Poissgeneetafo}
\{ f,h \}(g)=\sum_{ij}\eta^{ij}(g)X_i(g)(f)X_j(g)(h)
\end{equation}
for a certain $\eta$. That Poisson bracket can be rewritten as
\begin{equation}
\{ f,h \}(g)=\eta(g)(df_g\otimes dh_g)
\end{equation}
where we defined
\begin{equation}
\begin{aligned}
 \eta\colon G&\to \mG\otimes\mG \\ 
   \eta(g)&=\eta^{ij}(g)(X_i\otimes X_j).
\end{aligned}
\end{equation}
Let $C^n(G,\mG)$\nomenclature[F]{$C^n(G,\mG)$}{Space of functions from $G\times\cdots\times G$ into $\mG\otimes\mG$} be the set of maps $\lambda\colon G\times\cdots\times G\to \mG\otimes\mG$. The union of that spaces can be turned into a complex taking the coboundary
\begin{equation}
\begin{aligned}
 \delta_G\colon C^n(G,\mG)&\to C^{n+1}(G,\mG) \\ 
   (\delta_G\lambda)(g_1,\cdots,g_{n+1})&=g\cdot \lambda(g_2,\cdots,g_{n+1})\\
					&\quad+\sum_{i=1}^n (-1)^i\lambda(g_1,\cdots,g_ig_{i+1},\cdots,g_{n+1})\\
					&\quad+(-1)^{n+1}\lambda(g_1,\cdots,g_n).
\end{aligned}
\end{equation}
For that, we defined the action of $G$ on $\mG\otimes \mG$ by
\[ 
  g\cdot(X\otimes Y)=\Ad(g)X\otimes\Ad(g)Y=\big( \Ad(g)\otimes\Ad(g) \big)(X\otimes Y),
\]
and we can prove that $\delta_G^2=0$. The interest of that complex is that the compatibility condition \eqref{EqCondcmpDelPoiss} reads $\delta_G\eta=0$, or more specifically
\[ 
  (\delta_G\eta)(g_1,g_2)=g_1\cdot\eta(g_2)-\eta(g_1,g_2)+\eta(g_1)=0.
\]
The claim is that for the Poisson structure \eqref{Poissgeneetafo} to be compatible with the natural Hopf algebra structure, one needs 
\begin{equation}
\delta_G\eta=0.
\end{equation}
That is only compatibility. Every such $\eta$ does not define a Poisson-Lie group. Let us introduce the function $d\eta_e$, that we denote by $\phi_{\eta}$:
\begin{equation}
\begin{aligned}
 \phi_{\eta}\colon \mG&\to \mG\otimes\mG \\ 
   X&\mapsto \Dsdd{ \eta( e^{tX}) }{t}{0}. 
\end{aligned}
\end{equation}

\begin{theorem}
    In order to define a Poisson-Lie group $G$, the map $\phi_{\eta}$ must define a Lie bialgebra structure on \( \lG\).
\end{theorem}


One way to find such a $\eta$ is to take a $r\in\mG\otimes\mG$ and to define $\eta=\delta_Gr$. By the definitions, we have $\delta_Gr(g)=r-g\cdot r$ and thus
\[ 
  \phi_{\eta}(X)=\Dsdd{ \eta( e^{tX}) }{t}{0}=\Dsdd{ r- e^{tX}r }{t}{0}=-X\cdot r.
\]
A generic $r$ reads $r=r^{ij}(X_i\otimes X_j)$ and then we have
\begin{equation}
\phi_{\eta}(X)=[r,1\otimes X+X\otimes 1].
\end{equation}
