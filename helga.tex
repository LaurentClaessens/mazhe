

\section{Groups of transformations}
%-------------------------------------

\begin{definition}
Let $M$ be an Hausdorff space and $G$, a topological group. It is a \defe{topological group of transformations}{topological!group of transformations}\index{transformation!group (topological)} of $M$ if to any element $g\in G$, one associates an homeomorphism $\dpt{g}{M}{M}$, denoted by dot ($p\to g\cdot p$) such that 
$\forall p\in M$; $\forall g,h\in G$,
\begin{enumerate}
\item $(gh)\cdot p=g\cdot(h\cdot p)$,
\item the map $G\times M\to M$; $(g,p)\to g\cdot p$ is continuous.
\end{enumerate}
\label{DefTopoGpTransfo}
\end{definition}
If $g\cdot=id$ only for $g=e$, then the action is \defe{effective}{effective!action}.\index{action!of a group on a manifold}


\begin{lemma}[Category theorem] 
If a locally compact space $M$ can be written as a countable union
\begin{equation}\label{eq:M_union}
   M=\bigcup_{n=1}^{\infty}M_n
\end{equation}
where each $M_i$ is closed in $M$, then at least one of them contains an open subset of $M$.
\label{lem:categ}
\end{lemma}

\begin{proof}
We suppose that none of the $M_i$ contains an open subset of $M$. Let $U_1$ be an open whose closure is compact, $a_1\in U_1\setminus M_1$ and a neighbourhood $U_2$ of $a_1$ such that $\overline{U}_2\subset U_1$ and $\overline{U_2}\cap M_1=\varnothing$. Let $a_2\in U_2\setminus M_2$ and a neighbourhood $U_3$ of $a_2$ such that $\overline{U_3}\subset U_2$ and $\overline{U_3}\cap M_2=\varnothing$\ldots and so on. The existence of the $a_i$ comes from the fact that $U_j$ is open, so that it is contained in no one of the $M_k$.

The decreasing sequence $\overline{U}_1,\overline{U}_2 ,\ldots$ is made up from non empty compacts sets. Then $\bigcap_{i=1}^{\infty}U_i\neq\emptyset$ and the elements of this intersection are in none of the $M_i$; this contradict \eqref{eq:M_union}.
\end{proof}

\begin{theorem}
Let $G$ be a locally compact group with a countable basis. Suppose that it is a transitive, locally compact and Hausdorff topological group of transformation on $M$. Consider $p\in M$ and $H=\{g\in G\tq g\cdot p=p\}$. Then 
\begin{enumerate}
\item $H$ is closed,
\item the map $[g]\to g\cdot p$ is homeomorphic between  $G/H$ and $M$.
\end{enumerate}
\label{tho:homeo_action}
\end{theorem}

\begin{proof}
By definition of a group action, the map $\dpt{\varphi}{G}{M}$, $\varphi(g)=g\cdot p$ is continuous. Then $H=\varphi^{-1}(p)$ is closed in $G$.

As usual, the topology considered on $G/H$ is a topology which makes the canonical projection $\dpt{\pi}{G}{G/H}$ continuous and open. Now we study the map $\dpt{\psi}{G/H}{M}$, $\psi([g])=g\cdot p$ which is well defined because $H$ fixes $p$ by definition. It is clearly injective, and it is surjective because the action is transitive.

Now remark that $\psi=\varphi\circ\pi^{-1}$. Since $\pi$ is continuous and open, and $\varphi$ is continuous, it just remains to be proved that $\varphi$ is open in order for $\psi$ to be continuous and open. In order to do it, consider $V$, an open subset of $G$, $g\in V$ and a compact neighbourhood $U$ of $e$ in $G$ such that $U=U^{-1}$ and $gU^2\subset V$. If $U$ is small and $u$, $v\in U$ close to $e$, then $guv$ can keep in $V$, so that such a $U$ exists.

We can find a sequence $(g_n)$ in $G$ such that $G=\bigcup_ng_nU$; the transitivity of $G$ on $M$ implies that 
\[
  M=\bigcup_ng_nU\cdot p.
\]
Each term in this union is compact, and therefore closed in $M$. By lemma \ref{lem:categ}, one of the $g_nU\cdot p$ contains an open subset of $M$. Since the action ``$g\cdot$''\ is continuous, $U\cdot p$ also contains an open subset in $M$. The conclusion is that one can find a $u\cdot p$ in the interior of $M$, and $p$ is then an interior point of $u^{-1} U\cdot p\subset U^2\cdot p$. Then $g\cdot p$ in in the interior of $V\cdot p$ and $\varphi$ is therefore open.
\end{proof}


The \defe{isotropy group}{isotropy!group} of $M$ (with respect to the action of $G$) is naturally defined by
\[
   H=\{g\in G\tq g\cdot p=p\}.
\]

The unit sphere is an example. Let's consider the vector $\overrightarrow{1}_z$. It is clear that the action of $\SO(3)$ on this vector covers the whole sphere. But there is a $\SO(2)$ subgroup of rotations which leaves $\overrightarrow{1}_z$ unchanged. So the sphere is given by the quotient 
\[
  S^2=\frac{\SO(3)}{\SO(2)}.
\]


\begin{corollary}
Let $G$ and $X$ be two locally compact groups. We suppose $G$ to have countable basis. Then any homomorphism $\dpt{\psi}{G}{X}$ is open.
\end{corollary}

\begin{proof}
By terminology, when we say that a group has some topological property (as the local compactness here), we suppose that the group is a topological group.

For any $g\in G$, we can build an homeomorphism (see remark \ref{rem:ouvert}) $\dpt{\varphi_g}{X}{X}$, $x\to \psi(g)x$, so that $G$ is  a transitive topological group of transformation on $X$. Let us denote by $f$ the identity of $X$. We know that $\varphi_g$ is continuous, open and satisfies 
\[
   \varphi_g(f)=\psi(g)f=\psi(g).
\]

If we define $H=\{h\in G\tq \psi(h)=f\}$, $\psi(gh)=\psi(g)$ for any $h\in H$, so that $\psi$ descent to a map $\dpt{\psi}{X/H}{X}$. This is precisely the map which theorem \ref{tho:homeo_action} assure us to be an homeomorphism.

\end{proof}

\section{Lie groups of transformations}
%----------------------------------------

\begin{definition}
Let $G$ be a Lie group and $M$, an analytic manifold. We say that $G$ is a \defe{Lie group of transformation}{Lie!group!of transformation}\index{transformation!Lie group} of $M$ when

\begin{enumerate}
\item $G$ is a topological group of transformation of $M$,
\item the map $G\times M\to M$, $(g,p)\to g\cdot p$ is analytic.
\end{enumerate}
\label{DefLieGpTransfo}
\end{definition}

In particular, for any $g\in G$, $p\to g\cdot p$ is a diffeomorphism $M\to M$. When $G$ is a Lie transformation group on $M$, for any $X\in\lG$, we define a \defe{fundamental vector field}{fundamental!vector field} $X^{\dag}\in\cvec(M)$ by\footnote{Remark, as usual, that some literature (in particular in \cite{Helgason}) gives it without the minus sign.} 
\begin{equation}
  X^{\dag}_p=\Dsdd{e^{-tX}\cdot p}{t}{0}
\end{equation}
The existence comes from the fact that $(g,p)\to g\cdot p$ and the exponential are analytic and on a function $f\in\Cinf(M)$, the vector field acts as
\[
   (X^{\dag} f)(p)=\lim_{t\to\infty}\frac{ f(e^{-tX}\cdot p)-f(p) }{t}.
\]

\begin{theorem}
Consider $G$, a Lie transformation group on $M$ and $X$, $Y\in\lG$. Then
\begin{equation}
  [X^{\dag},Y^{\dag}]=\lim_{t\to 0}\us{t}(Y^{\dag}-dg_t Y^{\dag})
\end{equation}
where $g(t)=g_t=e^{tX}$.
\end{theorem}

\begin{proof}
We consider $f\in\Cinf(M)$ and $q\in M$. The function $F$ defined by
\begin{equation}
  F(f,q)=f(e^{-tX}\cdot q)
\end{equation}
is analytic with respect to $t$, so that
\begin{equation}\label{eq:def_F_h}
  F(t,q)-F(0,q)=t\int^1_0\left(\dsd{F}{t}\right)(st,q)ds=h(t,q)t
\end{equation}
for a certain function $h\in\Cinf(\eR\times M)$ which satisfies $h(0,q)=(X^{\dag} f)(q)$.
Naturally, $g_t$ can be seen as a map $\dpt{g_t}{M}{M}$ by the action. Then $dg_t$ is a linear map $\dpt{dg_t}{T_qM}{T_{g_t\cdot q}M}$  (we voluntary omit the index $q$ which was fixed; formally, we speak about $(dg_t)_q$)
\[
   dg_tv=\Dsdd{g_t\cdot v(u)}{u}{0}.
\]
Thus in order to compute $(dg_t Y^{\dag})_p$, we have to consider $Y^{\dag}$ at $r=g_t^{-1}\cdot p$. Consider a path $\dpt{v_r}{[0,1]}{M}$ such that $v'_r(0)=Y^{\dag}_r$ and $v_r(0)=r$. So,
\begin{equation}
\begin{split}
(dg_t\cdot Y^{\dag})_pf&=\Dsdd{ f(g_t\cdot v_r(u)) }{u}{0}\\
                     &=\Dsdd{ (f\circ g_t\circ v_r)(u) }{u}{0}\\
		     &=\Dsdd{ F(t,v_r(u)) }{u}{0}\\
		     &=\Dsdd{ F(0,v_r(u))+h(t,v_r(u))t }{u}{0}
\end{split}
\end{equation}
The two terms are computed separately :
\[
   \Dsdd{ F(0,v_r(u)) }{u}{0}=\Dsdd{ f(v_r(u)) }{u}{0}=Y^{\dag}_r(f),
\]
and
\[
    \Dsdd{ th(t,v_r(u)) }{u}{0}=t(Y^{\dag} h)_{(t,v_r(0))}.
\]
Finally,
\begin{equation}\label{eq:Y_h}
  (dg_t\cdot Y^{\dag})_pf=Y^{\dag}_{g_t^{-1}\cdot p}(p)+t(Y^{\dag} h)_{(t,g_t^{-1}\cdot p)}.
\end{equation}
Now we can compute :
\begin{equation}
\begin{aligned}
  \lim_{t\to 0}\us{t}\left( (Y^{\dag}-dg_tY^{\dag})_pf \right)
                     &=\lim_{t\to 0}\us{t}\left\{ (Y^{\dag} f)p-(Y^{\dag} f)(g_t^{-1}\cdot p) \right\}\\
		     &\quad+\lim_{t\to 0}\us{t}\left\{ (Y^{\dag} f)(g_t^{-1}\cdot p)
		                                      -(Y^{\dag}(f\circ g_t) )(g_t^{-1}\cdot p)
					       \right\}\\
   &=\Dsdd{ (Y^{\dag} f)(g_t^{-1}\cdot p) }{t}{0}-\lim_{t\to 0} 
                \left(    (Y^{\dag} h)(t,g_t^{-1}\cdot p)       \right).
\end{aligned}
\end{equation}
The latter equality comes from  \eqref{eq:Y_h}. The first term is computed as following ($Y^{\dag}(f)$ is a function) :
\begin{equation}
\Dsdd{ Y^{\dag}(f)(g_t^{-1}\cdot p) }{t}{0}=\Dsdd{ (Y^{\dag}(f)\circ g_t^{-1})(p) }{t}{0}
                                        =\left( Y^{\dag}(f) \right)_p(g_t^{-1})'(0)
					=\left( X^{\dag} Y^{\dag}(f)\right)_p
\end{equation}
In the expression $(Y^{\dag} h)(t,g_t^{-1}\cdot p)$, we have to consider the dependence on $t$ as a parameter: the vector $Y^{\dag}$ only acts on the ``second slot''\ of $h$. From definition \eqref{eq:def_F_h} of $h$,
\[
   h(t,g_t^{-1}\cdot p)=\us{t}\left(  F(f,g_t^{-1}\cdot p)-F(0,g_t^{-1}\cdot p)  
                             \right),
\]
but $F(0,q)=f(q)$ and $F(t,g_t^{-1}\cdot p)=f(p)$, then
\[
  f(t,g_t^{-1}\cdot p)=\us{t}\left( f(p)-f(e^{-tX}\cdot p) \right)
\]
Taking the limit for small $t$, it becomes 
\[
  \Dsdd{f(e^{-tX}\cdot p)}{t}{0}=(X^{\dag} f)_p
\]

\end{proof}

\section{Cosets}
%------------------

We consider $G$, a Lie group and $H$, a closed subgroup. Then from theorem \ref{tho:diff_sur_ferme},  there exists an unique analytic structure on $H$ for which $H$ is a topological Lie subgroup of $G$. We naturally consider this structure on $H$. We also consider $\lG$ and $\lH$, the Lie algebras of $G$ and $H$, and $\lM$ be a subspace of $\lG$ such that $\lG=\lM\oplus\lH$.

Now we will study the structure of the coset space $G/H$ on which we put the topology such that $\pi$ is continuous and open; this is the \defe{natural topology}{natural topology}\index{topology!natural on $G/H$}\label{pg:natur_topo}. As notations, we define $p_0=\pi(e)$ and $\dpt{\psi}{\lM}{G}$, the restriction to $\lM$ of the exponential.

\begin{lemma}
The dimension of $G/H$ is $\dim (G/H)=\dim G-\dim H$.
 \label{lem:dim_G_H}
\end{lemma}

\begin{proof}
We decompose the Lie algebra $\lG$ as $\lG=\lH\oplus\lM$, and we will see that there exists a real vector space isomorphism $\dpt{\psi}{T_{[e]}(G/H)}{\lM}$ given by
\begin{equation}
   \psi(X)=\Dsdd{ e^{m(t)} }{t}{0}
\end{equation}
if $X(t)=[g(t)]$ with $g(t)=e^{m(t)}e^{h(t)}$ where $m(t)\in\lM$ and $h(t)\in\lH$ (the existence of such a decomposition in reasonably small neighbourhood of $e$ is given by lemma \ref{lem:decomp}). The fact that $\psi$ is surjective is clear. The injectivity is also easy: $\psi(X)=0$ implies that $\exp m(t)$ is a constant. Thus
\[
X=\Dsdd{ [cst\, e^{h(t)}] }{t}{0}=\Dsdd{[cst]}{t}{0}=0.
\]

\end{proof}

\begin{lemma} 
There exists a neighbourhood $U$ of $0$ in $\lM$ such that
 \begin{enumerate}
 \item $\psi$ is homeomorphic on $U$,
 \item $\pi$ sends homeomorphically $\psi(U)$ on a neighbourhood of $p_0$ in $G/H$.
 \end{enumerate}
\label{lem:vois_U}
\end{lemma}

\begin{proof}
By lemma \ref{lem:decomp}, we consider bounded, open and connected neighbourhoods $\mU_m$ and $\mU_h$ of $0$ in $\lM$ and $\lH$ such that 
\[
  \dpt{\phi}{(A,B)}{e^Ae^B}
\]
is a diffeomorphism from $\mU_m\times\mU_h$ to an open neighbourhood of $e$ in $G$. Since $H$ has the induced topology from $G$, we can find a neighbourhood $V$ of $e$ in $G$ such that $V\cap H=\exp\mU_h$. 

Now we take $U$, a compact neighbourhood of $0$ in $\mU_m$ such that 
\begin{equation}\label{eq:UUV}
  e^{-U}e^{U}\subset V.
\end{equation}
So, $\psi$ is an homeomorphism from $U$ to $\psi(U)$. Indeed for $X\in U$, $\psi(X)=e^X=\phi(X,0)$ and $\phi$ is diffeomorphic. 

On the other hand, $\pi$ is bijective on $\psi(U)$. In order to see that it is injective, let us consider $X$, $Y\in U$ such that $\pi(e^{X})=\pi(e^{Y})$. Then $\exp X$ and $\exp Y$ are in the same class with respect to $H$ : $\exp X\in[\exp Y]$. Then $\exp(-X)\exp Y\in H$, and reversing the role\angl of $X$ and $Y$, $\exp(-Y)\exp X\in H$. Since $X',X''\in U$ and \eqref{eq:UUV}, 
\[
  e^{-Y}e^{X}\in V\cap H.
\]
Then there exists a $Z$ in $\mU_h$ such that $\exp X=\exp Y\exp Z$, but $U$ is a subset of $\mU_m$ (so that $(A,B)\to e^Ae^B$ is diffeomorphic), then $X=Y$ and $Z=0$.

Since $\pi$ is bijective on $\psi(U)$, it is homeomorphic because the topology is build in order for $\pi$ to be open and continuous.

On a third hand, $U\times\mU_h$ is a neighbourhood of $(0,0)$ in $\mU_m\times\mU_h$, so that $e^Ue^{\mU_h}$ is a neighbourhood of $e$ in $G$. Since $\pi$ is open, $\pi(\exp U\exp \mU_h)=\pi(\psi(U))$ is a neighbourhood of $p_0$ in $G/H$.
\end{proof}


Let $N_0$ be the interior of $\pi(\psi(U))$ and $\{X_1,\ldots, X_r\}$ a basis of $\lM$. If $g\in G$, we looks at the map
\[
  \pi(g\cdot e^{x_1X_1+\ldots+x_rX_r})\to(x_1,\ldots,x_r).
\]
This is an homeomorphism from $g\cdot N_0$ to an open subset of $\eR^r$ because $\pi$ is homeomorphic from $U$. With this chart, $G/H$ is an analytic manifold \nomenclature{$G/H$}{as analytic manifold} and moreover if $x\in G$, the map
\begin{equation}\label{eq:tau_x_y}
  \dpt{\tau(x)}{[y]}{[xy]}
\end{equation}
is an analytic diffeomorphism of $G/H$. Let us prove it. If we consider $[x]\in G/H$, we can write $x=gm$ for a certain $m\in\psi(U)$. Hence the chart around $[x]$ will be around $[gm]=[ge^{x_1X_1+\ldots+x_rX_r}]$ (in other word, we can find an open set around $[x]$ on which can be parametrised so). We can forget the $g$ because the action is a diffeomorphism. Then we looks at the chart $\dpt{\varphi}{G/H}{\eR^r}$, $\varphi[e^{x_1X_1+\ldots+x_rX_r}]=(x_1,\ldots,x_r)$. The map \eqref{eq:tau_x_y} makes $(y_1,\ldots,y_r)\to( CBH_1(x_1,\ldots,x_r,y_1,\ldots y_r),\ldots, CBH_r(x_1,\ldots,x_r,y_1,\ldots y_r))$. But $CBH$ is a diffeomorphism.


The following theorem is the theorem 4.2, chapter II in \cite{Helgason}
\begin{theorem}\label{Helgason4.2}\label{tho:struc_anal}
Let $G$ be a Lie group, $H$ a closed subgroup of $G$ and $G/H$ with the natural topology. Then $G/H$ has an unique analytic structure with the property that $G$ is a Lie transformation group of $G/H$.
\end{theorem}

\begin{proof}
We denote by $\UU$ the interior of the $U$ given by the lemma \ref{lem:vois_U}, and $B=\psi(\UU)\subset G$. Since $\dpt{\phi}{(A,B)}{\exp A\exp B}$ is a diffeomorphism, $\psi(\UU)=\phi(U,0)$ is a submanifold of $G$. We consider the following diagram :
\[
\xymatrix{
    G\times B  \ar[d]_{\displaystyle I\times\pi}\ar[r]^{\displaystyle\Phi}    &     
                                                                     G\ar[d]^{\displaystyle\pi}\\
    G\times N_0 &                                                             G/H
  }\]
with, for $g\in G$ and $x\in B$,
\[
I\times\pi\colon (g,x)\mapsto (g,[x])
\]
and
\[
\Phi\colon (g,x)\mapsto gx.
\]

\noindent The classes $[x]$ are taken with respect to $H$. The map $\dpt{\mu}{G\times N_0}{G/H}$, $\mu(g,[x])=[gx]$ can be written under the form
\[
   \mu=\pi\circ\Phi\circ(I\times\pi)^{-1}
\]
which is analytic\footnote{Notice that the inverse of $I\times\pi$ exists because $\pi$ is homeomorphic on the spaces considered here.}. So $G$ is a Lie transformation group on $G/H$.

% Faut encore taper l'unicité
\end{proof}

\begin{proposition}
Let $G$ be a transitive transformation Lie group on a $\Cinf$ manifold $M$. Consider $p_0\in M$ and $H$, the stabilizer of $p_{0}$ :
\[
  H=\{ g\in G\tq g\cdot p_0=p_0 \}.
\]
Let
\begin{equation}
\begin{aligned}
 \alpha\colon G/H&\to M \\ 
[g]&\mapsto g\cdot p_{0}.
\end{aligned}
\end{equation}
We have :
\begin{enumerate}
\item The stabilizer $H$ is closed in $G$.
\item If $\alpha$ is homeomorphic, then it is diffeomorphic (if $G/H$ has the analytic structure of theorem \ref{tho:struc_anal}).
\item If $\alpha$ is homeomorphic and if $M$ is connected, then $G_0$, the identity component of $G$, is transitive on $M$.
\end{enumerate}
\label{propHelgason4.3}
\end{proposition}

This comes from \cite{Helgason}, chapter 2, proposition 4.3. The interest of this theorem is the fact that one only has to check the continuity of $\alpha$ and $\alpha^{-1}$ in order to have a diffeomorphism $M\simeq G/H$.

\begin{proof}
\subdem{The group $H$ is closed in $G$}
We consider the map $\dpt{\varphi}{G}{M}$, $\varphi(g)=g\cdot p_0$. This is continuous; therefore $\varphi^{-1}(p_0)$ is closed. Remark that we are in the situation of theorem \ref{tho:homeo_action} 

\subdem{First item}  
We will use lemma \ref{lem:vois_U}. Se denotes by $\lH$, the Lie algebra of $H$ and we consider a $\lM$ such that $\lG=\lM\oplus\lH$; the lemma \ref{lem:vois_U} assure us that we have a neighbourhood $U$ of $0$ in $\lM$ on which $\psi$ is homeomorphic and such that $\pi$ sends homeomorphically $\psi(U)$ to a neighbourhood of $p_0$ in $G/H$. We define $\UU$, the interior of $U$, $B=\psi(\UU)$ and $N_0$, the interior of $\pi(\psi(U))$.

The set $B$ is a submanifold of $G$, diffeomorphic to $N_0$ by $\pi$ because everything is continuous and then everything respect the interiors.
 
\begin{probleme}
C'est n'importe quoi comme justification. C'est lié au problème \ref{prob:diffeo_2}.
\label{prob:diffeo_1}
\end{probleme}
 
Consider $\dpt{\iota}{B}{G}$, the identity and $\dpt{\beta}{G}{M}$, $\beta(g)=g\cdot p_0$. The restriction $\alpha_{N_0}$ of $\alpha$ to $N_0$ is an homeomorphism (this is a part of the assumptions) from $N_0$ to an open subset of $M$ : $N_0$ is open (this is an interior), then its image by an homeomorphism is open.

Now we can see that $\alpha_{N_0}$ is differentiable. The reason is that it can be written as $\alpha_{N_0}=\dpt{\beta\circ\iota\circ\pi^{-1}}{N_0}{M}$. The construction makes $\pi$ a diffeomorphism and $\beta$ a diffeomorphism when $G$ is a Lie group of transformations (as it is the case here); $\iota$ is clear. Now we have to see that the whole $\alpha$ is also differentiable, and then we will have to prove the same for $\alpha^{-1}$.

By definition, $\alpha([g])=g\cdot p_0$ (the classes $[g]$ is taken with respect to $H$). Consider $[n]\in H$; for any $g\in G$, one can write $[g]=[gn^{-1} n]$. Then
\begin{equation}
  \alpha([g])=\alpha([gn^{-1} n])
             =gn^{-1} n\cdot p_0
	     =gn^{-1}\alpha([n])
	     =gn^{-1}\cdot\alpha_{N_0}([n]),
\end{equation}
but the last dot denotes a differentiable action, and $\alpha_{N_0}$ is differentiable. Thus $\alpha$ is differentiable.

In order for $\alpha$ to be a diffeomorphism, we still have to prove that $\alpha^{-1}$ is differentiable., we begin to show that the Jacobian of $\beta$ at $g=e$ has rank $r_{\beta}=\dim M$. We looks at $\dpt{d\beta_e}{\lG}{T_{p_0}M}$, and consider $X\in\ker (d\beta_e)$. For $f\in\Cinf(M)$, we compute
\begin{equation}
  0=(d\beta_e X)f=X(f\circ\beta)=\Dsdd{ f(e^{tX}\cdot p_0) }{t}{0}.
\end{equation}
Let $s\in\eR$, and we write this equation for $f^*$ instead of $f$, which $f^*$ defined by $f^*(q)=f(e^{sX}\cdot q)$ for each $q\in M$:
\begin{equation}
  0=\Dsdd{f^*(e^{tX}\cdot p_0)}{t}{0}
      =\Dsdd{ f(e^{(s+t)X}\cdot p_0) }{t}{0}
      =\Dsdd{ f(e^{tX}\cdot p_0) }{t}{s}.
\end{equation}
Thus $f(e^{sX}\cdot p_0)$ is a constant with respect to $s$. Since $f$ is arbitrary, $e^{sX}\cdot p_0=p_0$ for any $s$. So $X\in\lH$ because $\exp sX\in H$ for any $s$. Then $\ker d\beta_e\subset \lH$.

On the other hand, $\lH\subset\ker d\beta_e$ is clear, then
\[
   \ker d\beta_e=\lH
\]
and $r_{\beta}=\dim\lG-\dim\lH$.

Since $\alpha$ is an homeomorphism, the dimension of the origin and the target space are the same: $\dim G/H=\dim M$. On the other hand, lemma \ref{lem:dim_G_H} gives $\dim G/H=\dim\lG-\dim\lH$, so that $r_{\beta}=\dim M$.

Now we prove that $\alpha^{-1}$ is differentiable. Remark that $\beta(g)=g\cdot p_0$ and $\alpha([g])=g\cdot p_0=\beta(g)$ is a good definition for $\alpha$ because the class are taken with respect to the stabilizer of $p_0$. Since $r_{\beta}=\dim M$, the map $\beta$ is locally a diffeomorphism from a neighbourhood of $e$ to a neighbourhood of $p_0$.

If $p=g\cdot p_0$, $\alpha^{-1}(p)=[g]$ because $[k]\in\alpha^{-1}(o)$ if $\alpha([k])=p$, i.e. $k\cdot p_0=p$. But $k=gr$ for a certain $r\in G$. It is clear that $p=k\cdot p_0=gr\cdot p_0$. In particular, $g\cdot(r\cdot p_0)$. We know that in general $g\cdot p=g\cdot q$ implies $p=q$; here it gives us $r\in H$, so that $k\in [g]$.

We consider a $n\in G$ such that $n\cdot$ and $n^{-1}\cdot$ are diffeomorphic. We can make the following manipulation :
\begin{equation}
   \alpha^{-1}(p)=[g]
                =[gnn^{-1}]
		=\pi(gn)\alpha^{-1}(n^{-1}\cdot p_0).
\end{equation}
Under this form, $\alpha^{-1}$ is diffeomorphic.


\subdem{Second item}
If $\alpha$ is an homeomorphism, then $\beta$ is open. Let us denote by $G_0$ the identity component of $G$. There exists a subset $\{x\bgamma\tq\gamma\in I\}$ of $G$ such that
\[
    G=\bigcup_{\gamma\in I}G_0x\bgamma.
\]
This comes from the fact that the components are all some left translations of the identity component (this is true for any Lie group). Each orbit $G_0x\bgamma\cdot p_0$ is open in $M$ and two orbits are either disjoint either equals. Since $M$ is connected, all these orbits must coincide; thus each orbit contains the whole $M$. In particular, the orbit $G_0\cdot p_0=M$ : $G_0$ is transitive on $M$.

\end{proof}

\begin{probleme}
Il parra\^it que \c ca donne l'unicit\'e pour \ref{tho:struc_anal}.
\end{probleme}

\section{Cosets and homogeneous spaces}
%-----------------------------------------

Proposition \ref{propHelgason4.3} takes its interest in the setting of homogeneous space. An \defe{homogeneous spaces}{homogeneous!space} is a smooth manifolds $M$ which admits a Lie group of transformations. If $p_{0}\in M$ and $H$ is the stabilizer of $p_{0}$, proposition \ref{propHelgason4.3} says that $H$ is closed and therefore theorem \ref{tho:struc_anal} makes $G$ a Lie group of transformations of $G/H$. Hence the latter becomes a homogeneous space. The map $\alpha$ of proposition \ref{propHelgason4.3} gives an isomorphism of homogeneous spaces. 

This allow us to see a homogeneous space as the quotient of a group by a closed subgroup.

\section{Isotropy group}
%--------------------------

If one has a Lie group $G$ and a closed subgroup $H$, we know from theorem \ref{tho:diff_sur_ferme} that $H$ is a topological Lie subgroup of $G$. We naturally consider this structure and the analytic structure on $G/H$ given by \ref{tho:struc_anal}. For this structure of the coset space $G/H$, the group $H$ is the \defe{isotropy group}{isotropy!group}\index{group!isotropy}. We denote by $\dpt{\tau(x)}{G/H}{G/H}$ ($x\in G$) the diffeomorphism $\tau(x)[y]=[xy]$. The group $H^*$ of the linear transformations $(d\tau(h))_{\pi(e)}$ ($h\in H$) is the \defe{linear isotropy group}{isotropy!Linear group}.


Let $N$ be a Lie subgroup of $G$. Then the subset $N\cap H$ is closed
\footnote{If $H'$ denotes the complementary of $H$ in $G$ (which is open in $G$), the complementary $N\cap H'$ of $N\cap H$ is open in $N$ for the induced topology of $N$ from $G$.}. Then $N\cap H$ is closed in $N$ and we look at $N/(N\cap H)$. We can exhibit a bijection between this and the orbit of $\pi(e)$ in $G/H$ with respect to the action of $N$ :
\begin{equation}
  \{ n\pi(e)\tq n\in N \}\simeq N/(N\cap H)
\end{equation}
by the map $\psi$ given by $\psi(n\pi(e))=\overline{n}$. Here the $\overline{ x }$ denotes the class of $x$ with respect to $N\cap H$. Note that for $n\in N$, $\overline{n}\neq\overline{e}$ because there are \emph{a priori} elements in $N\setminus H$. The map $\psi$ is well defined because $m[e]=[m]=[n]$ if $m=nh$ for a certain $h\in H$. Then $\psi([nh])=\overline{nh}$. But in order for $nh$ to belongs to $N$, one needs $h\in N\cap H$; then $\overline{nh}=\overline{n}$. For the same reason, $\psi$ is injective. The surjectivity is clear.

\begin{proposition}			\label{prop:orbit_N_ss_var}
	In this context,
	\begin{enumerate}
		\item The orbit of $e$ by $N$ in $G/H$ is $N/(N\cap H)$. It is submanifold of $G/H$.
		\item If $N$ is a topological subgroup of $G$ and if $H$ is compact, then the submanifold $N/(N\cap H)$ is a closed topological submanifold of $G/H$.
	\end{enumerate}
\end{proposition}

\begin{proof}
\subdem{First item}
We denote by $\overline{n}$, the class of $n$ with respect to $N\cap H$ and by $[g]$, the class of $g$ with respect to $H$.
The following diagram is commutative:
\begin{equation}\label{eq:dig_4.4}
 \xymatrix{
    N  \ar[d]_{\displaystyle\pi_1}\ar[r]^{\displaystyle i} &  G \ar[d]^{\displaystyle\pi}\\
    N/(N\cap H)\ar[r]^-{\displaystyle I} &       G/H
  }
\end{equation}
where $\dpt{\pi_1}{N}{N/(N\cap H)}$ and $\dpt{\pi}{G}{G/H}$ are canonical projections; $\dpt{i}{N}{G}$ is the inclusion; and $\dpt{I}{N/(N\cap H)}{G/H}$ is defined by $\overline{n}\to [i(n)]$. Indeed, $\pi(i(n))=[i(n)]=I( \overline{n} )=I(\pi_1(n))$ for any $n\in N$.

If $\lN$ is the Lie algebra of $N$ and $\lH$ the one of $H$, $\lH_1=\lH\cap\lN$ is the Lie algebra of $N\cap H$. We consider $\lN_1$ and  $\lG_1$ such that $\lN_1\oplus\lH_1=\lN$ and $\lG_1\oplus(\lH\oplus\lN_1)=\lG$. Let us show why is the sum $\lH\oplus\lN_1$ direct. First remark that $\lH\cap\lN_1$ because $\lH\cap\lN_1\subset\lH\cap\lN=\lH_1$, but $\lH_1\cap\lN_1=\{0\}$. Immediately, the sum $\lG=\lG_1\oplus(\lH\oplus\lN_1)$ is direct.

Now we apply lemma \ref{lem:vois_U} to the decomposition $\lN=\lH_1\oplus\lN_1$; this give us a submanifold $B_N\subset N$ which contains $e$ and on which $\pi_1$ is diffeomorphic to an open neighbourhood of $\pi_1(e)$ in $N/(N\cap H)$. The same with $\lG=\lH\oplus(\lN_1+\lG_1)$ gives $B_G\subset G$, a submanifold around $e$ on which $\pi$ is diffeomorphic to a neighbourhood of $\pi(e)$ in $G/H$. We can see $B_N$ as a submanifold of $B_G$.

We denotes $V_1=\pi_1(B_N)$, $V=\pi(B_G)$ and $I_{V_1}$, the restriction of $I$ to $V_1$. The Jacobian of $I_{V_1}$ at $\pi(e)$ has a rank equal to $\dim\big( N/(N\cap H) \big)$. Indeed we can write $I_{V1}$ as $I_{V_1}=\pi\circ i\circ\pi^{-1}$ and $\pi_1$ is a diffeomorphism, so that $\pi^{-1}$ don't change the dimension. The fact that the Jacobian at identity is non zero on a neighbourhood makes it regular on this neighbourhood and the analyticity make it regular everywhere. The characterization for a submanifold given at page \pageref{pg:caract_subvar} gives the first item.

\subdem{Second item}
We know that $N$ is a submanifold of $G$; the commutative diagram \eqref{eq:dig_4.4} shows that $I$ is an homeomorphism because the topologies on $G/H$ and $N/(N\cap H)$ are made in order to $\pi$ and $\pi_1$ are continuous and open. If $N$ is a topological subgroup of $G$, an open subset of $N$ is written under the form $N\cap\mO$ where $\mO$ is open in $G$. If $\overline{n}\in N/(N\cap H)$, $I(\overline{n})=[i(n)]$.

Now we show that $N/(N\cap H)$ is closed. Consider a sequence $(p_k)$ in $N/(N\cap H)$ which converges to $q\in G/H$. The aim is to show that $q\in N/(N\cap H)$. We take a $g\in G$ such that $\pi(g)=q$; we can suppose that the whole sequence $(p_k)$ is in the neighbourhood $g\cdot V$ of $q$. In order to see that it is a neighbourhood, recall that $\pi$ is a diffeomorphism from $V$ to an open neighbourhood of $e$ in $G/H$, thus $V$ is an open neighbourhood of $\pi(e)$ in $G/H$.

Since $\pi$ is diffeomorphic, there exists a sequence $g_k\in gB_G$ such that $\pi(g_k)=p_k$. It satisfies $\lim g_k=g$. On the other hand, for each $k$, $\exists n_k\in N$ such that $\pi_1(n_k)=p_k$; then $\pi(g_k)=p_k=\pi_1(n_k)$, then there exists $h_k\in H$ such that $g_k=n_kh_k$. But $H$ is compact, then $h_k$ is a converging sequence (by eventually passing to a subsequence). Since $g_k$ and $h_k$ converge, $n_k$ also converges. But $N$ is closed in $G$, then $n^*=\lim n_k\in N$. Finally $\pi_1(n^*)=q$, so that $N/(N\cap H)$ is closed.
\end{proof}
