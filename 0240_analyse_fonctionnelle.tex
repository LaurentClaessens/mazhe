% This is part of Mes notes de mathématique
% Copyright (c) 2011-2013
%   Laurent Claessens
% See the file fdl-1.3.txt for copying conditions.

\begin{theorem}[Théorème d'isomorphisme de Banach]  \label{ThofQShsw}
    Une application linéaire continue et bijective entre deux espaces de Banach est un homéomorphisme.
\end{theorem}
\index{théorème!isomorphisme de Banach}

%+++++++++++++++++++++++++++++++++++++++++++++++++++++++++++++++++++++++++++++++++++++++++++++++++++++++++++++++++++++++++++ 
\section{Théorème d'Ascoli}
%+++++++++++++++++++++++++++++++++++++++++++++++++++++++++++++++++++++++++++++++++++++++++++++++++++++++++++++++++++++++++++

\begin{definition}
    Une partie \( A\) d'un espace topologique \( X\) est \defe{relativement compacte}{compact!relatif}\index{relativement!compact} dans \( X\) si sa fermeture est compacte.
\end{definition}

\begin{proposition}[\cite{JIFGuct}] \label{PropDGsPtpU}
    Soient \( E\) et \( F\) deux espaces vectoriels normés sur \( \eR\) ou \( \eC\) et une application \( f\in\aL(E,F)\). Les propriétés suivantes sont équivalentes.
    \begin{enumerate}
        \item
            L'image d'un borné de \( E\) par \( f\) est relativement compact dans \( F\).
        \item   \label{ItemJIkpUbLii}
            L'image par \( f\) de la boule unité fermée est relativement compacte dans \( F\).
        \item
            Si \( (x_n)\) est une suite bornée dans \( E\), alors nous pouvons en extraire une sous-suite \( (x_{\varphi(n)})\) telle que \( \big( fx_{\varphi(n)} \big)\) converge dans \( F\).
    \end{enumerate}
\end{proposition}

\begin{definition}
    Une application vérifiant les conditions équivalentes de la proposition \ref{PropDGsPtpU} est dite \defe{compacte}{compact!opérateur}.
\end{definition}

\begin{definition}  \label{DefUWmVBcZ}
    Soit \( (f_i)_{i\in I}\) une famille de fonctions \( f_i\colon X\to Y\) entre espaces métriques. Cette famille est \defe{équicontinue}{équicontinuité} si pour tout \( \epsilon>0\) et pour tout \( x\in X\), il existe un \( \delta(x,\epsilon)>0\) tel que 
    \begin{equation}
        \| x-y \|_X\leq \delta\,\Rightarrow\,\| f_i(x)-f_i(y) \|_Y\leq \epsilon
    \end{equation}
    pour tout \( i\in I\).
\end{definition}

\begin{theorem}[Théorème d'Ascoli\cite{LBLADXV}]        \label{ThoKRbtpah}
    Soit \( K\) un espace topologique compact et un espace métrique \( (E,d)\). Nous considérons la topologie uniforme sur \( C(K,E)\). Une partie \( A\) de \( C(K,E)\) est relativement compacte si et seulement si les deux conditions suivantes sont remplies :
    \begin{enumerate}
        \item
            \( A\) est équicontinu,
        \item
            \( \forall x\in K\), l'ensemble \( \{ f(x)\tq f\in A \}\) est relativement compact dans \( E\).
    \end{enumerate}
\end{theorem}
%TODO : une preuve est sur Wikipédia.

%+++++++++++++++++++++++++++++++++++++++++++++++++++++++++++++++++++++++++++++++++++++++++++++++++++++++++++++++++++++++++++
\section{Espaces \texorpdfstring{$L^p$}{Lp}}
%+++++++++++++++++++++++++++++++++++++++++++++++++++++++++++++++++++++++++++++++++++++++++++++++++++++++++++++++++++++++++++

Pour la proposition suivante nous notons \( [f]\) la classe dans \( L^p\) de la fonction \( f\colon \Omega\to \eC\). Donc \( f\) représente une vraie fonction.
\begin{proposition}[\cite{bJOSNQ}]  \label{PropWoywYG}
    Soit \( 1\leq p\leq \infty\) et supposons que la suite \( [f_n]\) dans \( L^p(\Omega,\tribF,\mu)\) converge vers \( [f]\) au sens \( L^p\). Alors il existe une sous-suite \( (h_n)\) qui converge ponctuellement \( \mu\)-presque partout vers \( f\).
\end{proposition}
\index{espace!\( L^p\)}
\index{suite!de fonctions}
\index{limite!inversion}

\begin{proof}
    Si \( p=\infty\) nous sommes en train de parler de la convergence uniforme et il ne faut même pas prendre ni de sous-suite ni de «presque partout».

    Supposons que \( 1\leq p<\infty\). Nous considérons une sous-suite \( [h_n]\) de \( [f_n]\) telle que
    \begin{equation}
        \| [h_j]-[f] \|_p<2^{-j},
    \end{equation}
    puis nous posons \( u_k(x)=| h_k(x)-f(x) |^p\). Notons que ce \( u_k\) est une vraie fonction, pas une classe. Et en plus c'est une fonction positive. Nous avons
    \begin{equation}
        \int_{\Omega}u_kd\mu=\int_{\omega}| h_k(x)-f(x) |^pd\mu(x)=\| h_k-f \|_p^p\leq 2^{-kp}.
    \end{equation}
    Vu que \( u_k\) est une fonction positive la suite des sommes partielles de \( \sum_ku_k\) est croissante et vérifie donc le théorème de la convergence monotone \ref{ThoConvMonFtBoVh} :
    \begin{subequations}
        \begin{align}
            \int_{\Omega}\left( \sum_{k=0}^{\infty}u_k(x) \right)d\mu(x)&=\sum_{k=0}^{\infty}\int_{\Omega}u_k(x)d\mu(x)\\
            &\leq\sum_{k=0}^{\infty}2^{-kp}<\infty.
        \end{align}
    \end{subequations}
    Le fait que l'intégrale de la fonction \( \sum_ku_k\) est finie implique que cette fonction est finie \( \mu\)-presque partout. Donc le terme général tend vers zéro presque partout, c'est à dire
    \begin{equation}
        | h_k(x)-f(x) |^p\to 0.
    \end{equation}
    Cela signifie que \( h_k\to f\) presque partout ponctuellement.
\end{proof}

Est-ce qu'on peut faire mieux que la convergence ponctuelle presque partout d'une sous-suite ? En tout cas on ne peut pas espérer grand chose comme convergence pour la suite elle-même, comme le montre l'exemple suivant.

\begin{example}
    Nous allons montrer une suite de fonctions qui converge vers zéro dans \( L^p[0,1]\) (avec \( p<\infty\)) mais qui ne converge ponctuellement pour \emph{aucun} point. Cet exemple provient de \href{http://www.bibmath.net/dico/index.php?action=affiche&quoi=./b/bosseglissante.html}{bibmath.net}. 

%TODO : revoir tous les \href et mettre ceux qui doivent être en bibliographie. Par exemple celui ci-dessus.

    Nous construisons la suite de fonctions par paquets. Le premier paquet est formé de la fonction constante \( 1\).

    Le second paquet est formé de deux fonctions. La première est \( \mtu_{\mathopen[0 , 1/2 \mathclose]}\) et la seconde \( \mtu_{\mathopen[ 1/2 , 1 \mathclose]}\).

    Plus généralement le paquet numéro \( k\) est constitué des \( k\) fonctions \( \mtu_{\mathopen[ i/k , (i+1)/k \mathclose]}\) avec \( i=0,\ldots, k-1\).

    Vu que les fonctions du paquet numéro \( k\) ont pour norme \( \| f \|_p=\frac{1}{ k }\), nous avons évidemment \( f_n\to 0\) dans \( L^p\). Il est par contre visible que chaque paquet passe en revue tous les points de \( \mathopen[ 0 , 1 \mathclose]\). Donc pour tout \( x\) et pour tout \( N\), il existe (même une infinité) \( n>N\) tel que \( f_n(x)=1\). Il n'y a donc convergence ponctuelle nulle part.
\end{example}

%--------------------------------------------------------------------------------------------------------------------------- 
\subsection{Complétude}
%---------------------------------------------------------------------------------------------------------------------------

\begin{proposition}[Inégalité de Minkowski]     \label{PropInegMinkKUpRHg}
    Si \( 1\leq p<\infty\) et si \( f,g\in L^p(\Omega,\tribA,\mu)\) alors
    \begin{enumerate}
        \item
            \( \| f+g \|_p\leq \| f \|_p+\| g_p \|\)
        \item
            Il y a égalité si et seulement si les vecteurs \( f(x)\) et \( g(x)\) sont presque partout colinéaires : il existe \( \alpha,\beta\) tels que \( \alpha f+\beta g=0\) presque partout.
    \end{enumerate}
\end{proposition}

\begin{theorem}[\cite{SuquetFourierProba,UQSGIUo}]
    Pour \( 1\leq p<\infty\), l'espace \( L^p(\Omega,\tribA,\mu)\) est complet.
\end{theorem}
\index{complétude}

\begin{proof}
    Soit \( (f_n)_{n\in\eN}\) une suite de Cauchy dans \( L^p\). Pour tout \( i\), il existe \( N_i\in\eN\) tel que $\| f_p-f_q \|_p\leq 2^{-i}$ pour tout \( p,q\geq N_i\). Nous considérons la sous suite \( g_i=f_{N_i}\), de telle sorte qu'en particulier
    \begin{equation}    \label{EqJLoDID}
        \|g_i-g_{i-1}\|_p\leq 2^{-i}.
    \end{equation}
    Pour chaque \( j\) nous considérons la somme télescopique
    \begin{equation}
        g_j=g_0+\sum_{i=1}^j(g_i-g_{i-1})
    \end{equation}
    et l'inégalité
    \begin{equation}
        | g_j |\leq | g_0 |+\sum_{i=1}^j| g_i-g_{i-1} |.
    \end{equation}
    Nous allons noter
    \begin{equation}        \label{EqSomPaFPQOWC}
        h_j=| g_0 |+\sum_{i=1}^j| g_i-g_{i-1} |.
    \end{equation}
    La suite de fonctions \( (h_j)\) ainsi définie est une suite croissante de fonctions positive qui converge donc (ponctuellement) vers une fonction \( h\) qui peut éventuellement valoir l'infini en certains points. Par continuité de la fonction \( x\mapsto x^p\) nous avons
    \begin{equation}
        \lim_{j\to \infty} h_j^p=h^p,
    \end{equation}
    puis par le théorème de la convergence monotone (théorème \ref{ThoConvMonFtBoVh}) nous avons
    \begin{equation}
        \lim_{j\to \infty} \int_{\Omega}h_j^pd\mu=\int_{\Omega}h^pd\mu.
    \end{equation}
    Utilisant à présent la continuité de la fonction \( x\mapsto x^{1/p}\) nous trouvons
    \begin{equation}
        \lim_{j\to \infty} \left( \int h_j^p \right)^{1/p}=\left( \int | h |^p \right)^{1/p}.
    \end{equation}
    Nous avons donc déjà montré que
    \begin{equation}
        \lim_{j\to \infty} \| h_j \|_p=\left( \int | h |^p \right)^{1/p}
    \end{equation}
    où, encore une fois, rien ne garantit à ce stade que l'intégrale à droite soit un nombre fini. En utilisant l'inégalité de Minkowski (proposition \ref{PropInegMinkKUpRHg}) et l'inégalité \eqref{EqJLoDID} nous trouvons
    \begin{equation}
        \|h_j\|_p\leq \|g_0\|_p+\sum_{i=1}^j\|g_i-g_{i-1}\|_p\leq \|g_0\|_p+1.
    \end{equation}
    En passant à la limite,
    \begin{equation}
        \left( \int| h |^p \right)^{1/p}=\lim_{j\to \infty}\|h_j\|_p \leq \|g_0\|_p+1<\infty.
    \end{equation}
    Par conséquent \( \int| h |^p\) est finie et
    \begin{equation}    \label{EqgLpdUPOBP}
        h\in L^p(\Omega,\tribA,\mu).
    \end{equation}
    En particulier, l'intégrale \( \int h\) est finie (parce que \( p\geq 1\)) et donc que \( h(x)<\infty\) pour presque tout \( x\in\Omega\).

    Nous savons que \( h(x)\) est la limite des sommes partielles \eqref{EqSomPaFPQOWC}, en particulier la série
    \begin{equation}
        \sum_{j=1}^{\infty}| g_i-g_{i-1} |
    \end{equation}
    converge ponctuellement. En vertu du corollaire \ref{CorCvAbsNormwEZdRc}, la série de terme général \( g_i-g_{i-1}\) converge ponctuellement. La suite \( g_i\) converge donc vers une fonction que nous notons \( g\). Par ailleurs la suite \( g_i\) est dominée par \( h\in L^p\), le théorème de la convergence dominée (théorème \ref{ThoConvDomLebVdhsTf}) implique que
    \begin{equation}
        \lim_{j\to \infty} \|g_j-g\|_p=0.
    \end{equation}
    Nous allons maintenant prouver que \( \lim_{n\to \infty\|f_n-g\|_p} =0\). Soit \( \epsilon>0\). Pour tout \( n\) et \( i\) nous avons
    \begin{equation}
        \|f_n-g\|_p=\|f_n-f_{N_i}+f_{N_i}-g\|_p\leq\|f_n-f_{N_i}\|_p+\|f_{N_i}-g\|_p.
    \end{equation}
    Pour rappel, \( f_{N_i}=g_i\). Si \(i\) et \( n\) sont suffisamment grands nous pouvons obtenir que chacun des deux termes est plus petit que \( \epsilon/2\).

    Il nous reste à prouver que \( g\in L^p(\Omega,\tribA,\mu)\). Nous avons déjà vu (équation \eqref{EqgLpdUPOBP}) que \( h\in L^p\), mais \( | g_i |\leq h^p\), par conséquent  \( g\in L^p\).

    Nous avons donc montré que la suite de Cauchy \( (f_n)\) converge vers une fonction de \( L^p\), ce qui signifie que \( L^p\) est complet.
\end{proof}

\begin{theorem}[Fischer-Riesz\cite{KXjFWKA}] \label{ThoGVmqOro}
    Soit un ouvert \( \Omega\) de \( \eR^n\) et \( p\in\mathopen[ 1 , \infty \mathclose]\). Alors
    \begin{enumerate}
        \item\label{ItemPDnjOJzi}
            Toute suite convergente dans \( L^p(\Omega)\) admet une sous-suite convergente presque partout sur \( \Omega\).
        \item\label{ItemPDnjOJzii}
            La sous-suite donnée en \ref{ItemPDnjOJzi} est dominée par un élément de \( L^p(\Omega)\).
        \item\label{ItemPDnjOJziii}
            L'espace \( L^p(\Omega)\) est de Banach.
    \end{enumerate}
\end{theorem}
\index{espace!de fonctions!$L^p$}
\index{complétude!espaces $ L^p$}

\begin{proof}
    Le cas \( p=\infty\) est à séparer des autres valeurs de \( p\) parce qu'on y parle de norme uniforme, et aucune sous-suite n'est à considérer.
    \begin{subproof}
    \item[Cas \( p=\infty\).]
    Nous commençons par prouver dans le cas \( p=\infty\). Soit \( (f_n)\) une suite de Cauchy dans \( L^{\infty}(\Omega)\), ou plus précisément une suite de représentants d'éléments de \( L^p\). Pour tout \( k\geq 1\), il existe \( N_k\geq 0\) tel que si \( m,n\geq N_k\), on a
    \begin{equation}
        \| f_m-f_n \|_{\infty}\leq \frac{1}{ k }.
    \end{equation}
    En particulier, il existe un ensemble de mesure nulle \( E_k\) sur lequel
    \begin{equation}
        | f_m(x)-f_n(x) |\leq\frac{1}{ k },
    \end{equation}
    et si nous posons \( E=\bigcup_{k\in \eN}E_k\), nous avons encore un ensemble de mesure nulle (lemme \ref{LemIDITgAy}). En  résumé, nous avons un \( N_k\) tel que si \( m,n\geq N_k\), alors 
    \begin{equation}    \label{EqKAWSmtG}
        | f_n(x)-f_m(x) |\leq \frac{1}{ k }
    \end{equation}
    pour tout \( x\) hors de \( E\). Donc pour chaque \( x\in\Omega\setminus E\), la suite \( n\mapsto f_n(x)\) est de Cauchy dans \( \eR\) et converge donc. Cela défini donc une fonction
    \begin{equation}
        \begin{aligned}
            f\colon \Omega\setminus E&\to \eR \\
            x&\mapsto \lim_{n\to \infty} f_n(x). 
        \end{aligned}
    \end{equation}
    Cela prouve le point \ref{ItemPDnjOJzi} : la convergence ponctuelle.

    En passant à la limite \( n\to \infty\) dans l'équation \ref{EqKAWSmtG} et tenant compte que cette majoration tient pour presque tout \( x\) dans \( \Omega\), nous trouvons
    \begin{equation}
        \| f-f_n \|_{\infty}\leq \frac{1}{ k }.
    \end{equation}
    Donc non seulement \( f\) est dans \( L^{\infty}\), mais en plus la suite \( (f_n)\) converge vers \( f\) au sens \( L^{\infty}\), c'est à dire uniformément. Cela prouve le point \ref{ItemPDnjOJziii}. En ce qui concerne le point \ref{ItemPDnjOJzii}, la suite \( f_n\) est entièrement (à partir d'un certain point) dominée par la fonction \( 1+| f |\) qui est dans \( L^p\).


    \item[Cas \( p=\infty\).]

        Toute suite convergente étant de Cauchy, nous considérons une suite de Cauchy \( (f_n)\) dans \( L^p(\Omega)\) et ce sera suffisant pour travailler sur le premier point. Pour montrer qu'une suite de Cauchy converge, il est suffisant de montrer qu'une sous-suite converge. Soit \( \varphi\colon \eN\to \eN\) une fonction strictement croissante telle que pour tout \( n\geq 1\) nous ayons
        \begin{equation}
            \| f_{\varphi(n+1)}-f_{\varphi(n)} \|_p\leq \frac{1}{ 2^{n} }.
        \end{equation}
        Pour créer la fonction \( \varphi\), il est suffisant de prendre le \( N_k\) donné par la condition de Cauchy pour \( \epsilon=1/2^k\) et de considérer la fonction définie par récurrence par \( \varphi(1)=N_1\) et \( \varphi(n+1)>\max\{ N_n,\varphi(n-1) \}\). Ensuite nous considérons la fonction
        \begin{equation}
            g_n(x)=\sum_{k=1}^n| f_{\varphi(k+1)}(x)-f_{\varphi(k)}(x) |.
        \end{equation}
        Notons que pour écrire cela nous avons considéré des représentants \( f_k\) qui sont alors des fonctions à l'ancienne. Étant donné que \( g_n\) est une somme de fonctions dans \( L^p\), c'est une fonction \( L^p\), comme nous pouvons le constater en calculant sa norme :
        \begin{equation}
            \| g_n \|_p\leq \sum_{k=1}^n\| f_{\varphi(k+1)}-f_{\varphi(k)} \|_p\leq\sum_{k=1}^n\frac{1}{ 2^k }\leq\sum_{k=1}^{\infty}\frac{1}{ 2^k }=1.
        \end{equation}
        Étant donné que tous les termes de la somme définissant \( g_n\) sont positifs, la suite \( (g_n)\) est croissante. Mais elle est bornée en norme \( L^p\) et donc sujette à obéir au théorème de Beppo-Levi \ref{ThoConvMonFtBoVh} sur la convergence monotone. Il existe donc une fonction \( g\in L^p(\Omega)\) telle que \( g_n\to g\) presque partout.

        Soit un \( x\in \Omega\) pour lequel \( g_n(x)\to g(x)\); alors pour tout \( n\geq 2\) et \( \forall q\geq 0\),
        \begin{subequations}    \label{EqWTHojCq}
            \begin{align}
                | f_{\varphi(n+q)}(x)-f_{\varphi(n)}(x) |&=\left| f_{\varphi(n+q)}(x)-\sum_{k=1}^{q-1}f_{\varphi(n+k)}(x) -\sum_{k=1}^{q-1}f_{\varphi(n+k)}(x)-f_{\varphi(n)}(x) \right| \\
                &=\left| \sum_{k=1}^qf_{\varphi(n+k)}-\sum_{k=1}^qf_{\varphi(n+k-1)}(x) \right|\\
                &\leq \sum_{k=1}^q\Big| f_{\varphi(n+k)}(x)-f_{\varphi(n+k-1)}(x) \Big|\\
                &=g_{n+q+1}(x)-g_{n+1}(x)\\
                &\leq g(x)-g_{n-1}(x).
            \end{align}
        \end{subequations}
        Nous prenons la limite \( n\to \infty\); la dernière expression tend vers zéro et donc
        \begin{equation}
            | f_{\varphi(n+q)}(x)-f_{\varphi(n)}(x) |\to 0
        \end{equation}
        pour tout \( q\). Donc pour presque tout \( x\in \Omega\), la suite \( n\mapsto f_{\varphi(n)}(x)\) est de Cauchy dans \( \eR\) et donc y converge vers un nombre que nous nommons \( f(x)\). Cela définit une fonction
        \begin{equation}
            \begin{aligned}
                f\colon \Omega\setminus E&\to \eR \\
                x&\mapsto \lim_{n\to \infty} f_{\varphi(n)}(x) 
            \end{aligned}
        \end{equation}
        où \( E\) est de mesure nulle. Montrons que \( f\) est bien dans \( L^p(\Omega)\); pour cela nous complétons la série d'inégalités \eqref{EqWTHojCq} en
        \begin{equation}
            \big| f_{\varphi(n+q)}(x)-f_{\varphi(n)}(x) \big|\leq g(x)-g_{n-1}(x)\leq g(x).
        \end{equation}
        En prenant la limite \( q\to \infty\) nous avons l'inégalité
        \begin{equation}    \label{EqMQbDRac}
            | f(x)-f_{\varphi(n)}(x) |\leq g(x)
        \end{equation}
        pour presque tout \( x\in\Omega\), c'est à dire pour tout \( x\in\Omega\setminus E\). Cette inégalité implique deux choses valables pour presque tout \( x\) dans \( \Omega\) :
        \begin{subequations}
            \begin{align}
                f(x)&\in B\big( g(x),f_{\varphi(n)}(x) \big)\\
                f_{\varphi(n)}(x)&\leq | f(x) |+| g(x) |.
            \end{align}
        \end{subequations}
        
        La première inégalité assure que \( | f |^p\) est intégrable sur \( \Omega\setminus E\) parce que \( | f |\) est majorée par \( | g |+| f_{\varphi(n)} |\). Elle prouve par conséquent le point \ref{ItemPDnjOJzi} parce que \(n\mapsto f_{\varphi(n)}\) est une sous-suite convergente presque partout. La seconde montre le point \ref{ItemPDnjOJzii}. 

        Attention : à ce point nous avons prouvé que \( n\mapsto f_{\varphi(n)}\) est une suite de fonctions qui converge \emph{ponctuellement presque partout} vers une fonction \( f\) qui s'avère être dans \( L^p\). Nous n'avons pas montré que cette suite convergeait au sens de \( L^p\) vers \( f\). Ce que nous devons montrer est que
        \begin{equation}    \label{EqJLfnEvj}
            \| f-f_{\varphi(n)} \|_p\to 0.
        \end{equation}
        L'inégalité \eqref{EqMQbDRac} nous donne aussi, toujours pour presque tout \( x\in \Omega\) :
        \begin{equation}
            \big| f(x)-f_{\varphi(n)}(x) \big|^p\leq g(x)^p
        \end{equation}
        ce qui signifie que la suite\quext{À ce point, \cite{KXjFWKA} se contente de majorer \( | f_{\varphi(n)}(x) |\) par \( | f(x) |+|g(x)\), mais je ne comprends pas comment cette majoration nous permet d'utiliser la convergence dominée de Lebesgue pour montrer \eqref{EqJLfnEvj}.} \(    | f-f_{\varphi(n)} |^p    \) est dominée par la fonction \( | g |^p\) qui est intégrable sur \( \Omega\setminus E\) et tout autant sur \( \Omega\) parce que \( E\) est négligeable; cela prouve au passage le point \ref{ItemPDnjOJzii}, et le théorème de la convergence dominée de Lebesgue (\ref{ThoConvDomLebVdhsTf}) nous dit que
        \begin{equation}
            \lim_{n\to \infty} \int_{\Omega} \big| f(x)-f_{\varphi(n)}(x) \big|^pdx=\int_{\Omega}\lim_{n\to \infty} \big| f(x)-f_{\varphi(n)}(x) \big|dx=0.
        \end{equation}
        Cette dernière suite d'inégalités se lit de la façon suivante :
        \begin{equation}
            \lim_{n\to \infty} \| f-f_{\varphi(n)} \|_p=\big\| \lim_{n\to \infty} | f-f_{\varphi(n)} | \big\|_p=0.
        \end{equation}
        Nous en déduisons que la suite \( n\mapsto f_{\varphi(n)}\) est convergente vers \( f\) au sens de la norme \( L^p(\Omega)\). Or la suite de départ \( (f_n)\) était de Cauchy (pour la norme \( L^p\)); donc l'existence d'une sous-suite convergente implique la convergence de la suite entière vers \( f\), ce qu'il fallait démontrer.
    \end{subproof}
\end{proof}

%--------------------------------------------------------------------------------------------------------------------------- 
\subsection{Un cas particulier pour \texorpdfstring{$ L^2$}{ L2}}
%---------------------------------------------------------------------------------------------------------------------------

Les espaces \( L^2\) étant un peu particuliers, leur étude va dans le chapitre sur les espaces de Hilbert, voir section \ref{SecCKZSrZK}. Nous notons ici une conséquence du théorème \ref{ThoGVmqOro} dans ce cas. La proposition suivante est une petite partie du corollaire \ref{CorQETwUdF}, qui sera d'ailleurs démontré de façon indépendante.

\begin{proposition}
    Si nous avons une suite de réels \( (a_k)\) telle que \( \sum_{k=0}^{\infty}| a_k |^2<\infty\) alors la suite
    \begin{equation}
        f_n(x)=\sum_{k=0}^na_k e^{ikx}
    \end{equation}
    converge dans \( L^2\big( \mathopen] 0 , 2\pi \mathclose[ \big)\).
\end{proposition}

\begin{proof}
    Quitte à séparer les parties réelles et imaginaires, nous pouvons faire abstraction du fait que nous parlons d'une série de fonctions à valeurs dans \( \eC\) au lieu de \( \eR\).

    Un simple calcul est :
    \begin{equation}
        \| f_n-f_m \|^2\leq\int_0^{2\pi}\sum_{k=n}^m| a_k |^2dx\leq 2\pi\sum_{k=n}^m| a_k |^2.
    \end{equation}
    Par hypothèse le membre de droite est \( | s_m-s_n |\) où \( s_k\) dénote la suite des somme partielle de la série des \( | a_k |^2\). Cette dernière est de Cauchy (parce que convergente dans \( \eR\)) et donc la limite \( n\to\infty\) (en gardant \( m>n\)) est zéro. Donc la suite des \( f_n\) est de Cauchy dans \( L^2\) et donc converge dans \( L^2\).
\end{proof}

%--------------------------------------------------------------------------------------------------------------------------- 
\subsection{Inégalité de Hölder}
%---------------------------------------------------------------------------------------------------------------------------

\begin{proposition}[Inégalité de Hölder]       \label{ProptYqspT}
    Soit \( \Omega\) un espace mesuré et \( 1\leq p\), \( q\leq\infty\) satisfaisant \( \frac{1}{ p }+\frac{1}{ q }=1\). Soient \( f\in L^p(\Omega)\), \( g\in L^q(\Omega)\). Alors le produit \( fg\) est dans \( L^1(\Omega)\) et nous avons
    \begin{equation}
        \| fg \|_1\leq \| f \|_p\| g \|_q.
    \end{equation}
\end{proposition}
%TODO : une preuve.

\begin{remark}      \label{RemNormuptNird}
    Dans le cas d'un espace de probabilité, la fonction constante \( g=1\) appartient à \( L^p(\Omega)\). En prenant \( p=q=2\) nous obtenons
    \begin{equation}
        \| f \|_1\leq\| f \|_2.
    \end{equation}
\end{remark}

\begin{lemma}   \label{LemTLHwYzD}
    Lorsque \( I\) est borné nous avons \( L^2(I)\subset L^1(I)\). Si \( I\) n'est pas borné alors \( L^2(I)\subset L^1_{loc}(I)\).
\end{lemma}

\begin{proof}
    En effet si \( I\) est borné, alors la fonction constante \( 1\) est dans \( L^2(I)\) et l'inégalité de Hölder \ref{ProptYqspT} nous dit que le produit \( 1u\) est dans \( L^1(I)\).

    Si \( I\) n'est pas borné, nous refaisons le même raisonnement sur un compact \( K\) de \( I\).
\end{proof}

%+++++++++++++++++++++++++++++++++++++++++++++++++++++++++++++++++++++++++++++++++++++++++++++++++++++++++++++++++++++++++++ 
\section{Théorème de représentation de Riesz}
%+++++++++++++++++++++++++++++++++++++++++++++++++++++++++++++++++++++++++++++++++++++++++++++++++++++++++++++++++++++++++++

Dans le théorème suivant, \( E'\) est le dual topologique, c'est à dire l'espace des formes linéaires et continues.
\begin{definition}
    Un espace \( V\) est \defe{réflexif}{réflexif} si l'injection naturelle \( V\to V'\) est surjective.
\end{definition}

\begin{theorem}[\cite{LRBWftc}] \label{ThoSCiPRpq}
    Soit \( 1<p<\infty\) et une mesure quelconque. Alors
    \begin{enumerate}
        \item
            L'espace \( L^p\) est réflexif.
        \item
            Nous avons l'identification \( (L^p)'=L^q\) pour le nombre \( q\) tel que \( \frac{1}{ p }+\frac{1}{ q }=1\). Cette identification est donné de la façon suivante : l'application
            \begin{equation}
                \begin{aligned}
                    \Phi\colon L^q&\to (L^p)' \\
                    u&\mapsto \Big( \Phi_u\colon f\to \int_{\Omega}fu \Big) 
                \end{aligned}
            \end{equation}
            est une bijection isométrique.
    \end{enumerate}
    Si la mesure est \( \sigma\)-finie, alors
    \begin{enumerate}
        \item
            \( (L^1)'=L^{\infty}\)
        \item
            \( L^1\subset (L^{\infty})' \) avec une inclusion stricte sauf dans les cas triviaux.
    \end{enumerate}
\end{theorem}
% TODO : la preuve est sur Wikipédia.

\begin{proposition} \label{PropUKLZZZh}
    Soit \( f\in L^p(\Omega)\) telle que
    \begin{equation}
        \int_{\Omega}f\varphi=0
    \end{equation}
    pour tout \( \varphi\in C^{\infty}_c(\Omega)\). Alors \( f=0\) presque partout.
\end{proposition}

\begin{proof}
    Nous considérons la forme linéaire \( \Phi_f\in (L^q)'\) donnée par
    \begin{equation}
        \begin{aligned}
            \Phi_f\colon L^p&\to \eC \\
            u&\mapsto \int_{\Omega}fu
        \end{aligned}
    \end{equation}
    Par hypothèse cette forme est nulle sur la partie dense \(  C^{\infty}_c(\Omega)\). Si \( (\varphi_n)\) est une suite dans \(  C^{\infty}_c(\Omega)\) convergente vers \( u\) dans \( L^p\), nous avons pour tout \( n\) que
    \begin{equation}
        0=\Phi_f(\varphi_n)
    \end{equation}
    En passant à la limite, nous voyons que \( \Phi_f\) est la forme nulle. Elle est donc égale à \( \Phi_0\). La partie «unicité» du théorème de représentation de Riesz \ref{ThoSCiPRpq} nous indique alors que \( f=0\) dans \( L^p\) et donc \( f=0\) presque partout.
\end{proof}

%+++++++++++++++++++++++++++++++++++++++++++++++++++++++++++++++++++++++++++++++++++++++++++++++++++++++++++++++++++++++++++ 
\section{Espaces de Sobolev}
%+++++++++++++++++++++++++++++++++++++++++++++++++++++++++++++++++++++++++++++++++++++++++++++++++++++++++++++++++++++++++++

Sauf mention du contraire dans cette section \( I\) est un intervalle borné ouvert \( I=\mathopen] a , b \mathclose[\) de \( \eR\).

\begin{proposition} \label{PropHFWNpRb}
    Une fonction \( h\in C^{\infty}_c(I)\) admet une primitive dans \(  C^{\infty}_c(I)\) si et seulement si \( \int_Ih=0\).
\end{proposition}

\begin{proof}
    Si une primitive \( H\) de \( h\) est à support compact, alors
    \begin{equation}
        \int_Ih=H(b)-H(a)=0-0=0.
    \end{equation}
    Pas de problèmes dans ce sens.

    Supposons maintenant que \( \int_Ih=0\). Le fait que \( h\) admette une primitive dans \(  C^{\infty}(I)\) est évident : toute fonction continue admet une primitive. Soit \( H\) une telle primitive et \( \tilde H=H-H(b)\). Alors \( \tilde H(b)=0\) et 
    \begin{equation}
        \tilde H(a)=H(a)-H(b)=-\int_Ih=0.
    \end{equation}
    Nous rappelons que le support d'une fonction est \emph{la fermeture} de l'ensemble des points de non-annulation.

    Supposons que le support de \( h\) soit inclus dans \( \mathopen[ m , M \mathclose]\subset\mathopen] a , b \mathclose[\). En prenant des nombres \( m'\) et \( M'\) tels que \( a<m'<m\) et \( M<M'<b\) (nous insistons sur le caractère strict de ces inégalités), la fonction \( h\) est nulle sur \( \mathopen[ a , m' \mathclose]\) et sur \( \mathopen[ M' , b \mathclose]\); la fonction \( \tilde H\) doit donc y être constante. Mais nous avons déjà vu que \( \tilde H(a)=\tilde H(b)=0\). Donc l'ensemble des points sur lesquels \( \tilde H\) n'est pas nul est inclus à \( \mathopen] m' , M' \mathclose[\) et donc est strictement (des deux côtés) inclus à \( I\).
\end{proof}


\begin{definition}
    Soit \( f\in L^p(\Omega)\) où \( I\) est l'intervalle ouvert \( \mathopen] a , b \mathclose[\). Sa \defe{dérivée au sens des distributions}{dérivée!au sens de distributions} est une fonction \( g\) telle que
        \begin{equation}
            \int_If\varphi'=-\int_Ig\varphi
        \end{equation}
        pour tout \( \varphi\in C^{\infty}_c(I)\).
\end{definition}

\begin{lemma}
    Lorsqu'une fonction admet une dérivée au sens des distributions, cette dernière est unique (et justifie le singulier dans la définition).
\end{lemma}

\begin{proof}
    Soient \( g,h\in L^2\) tels que 
    \begin{equation}
        \int_Iu\varphi'=-\int_Ig\varphi=-\int_Ih\varphi
    \end{equation}
    pour tout \( \varphi\in C^{\infty}_c(I)\). Nous avons alors
    \begin{equation}
        \int_I(g-h)\varphi=0.
    \end{equation}
    Cela implique que \( g-h=0\) presque partout par la proposition \ref{PropUKLZZZh}\footnote{Ou alors par le lemme \ref{LemDQEKNNf} qui est moins général mais tout aussi bien pour ici.}.
\end{proof}

\begin{definition}
    Soit \( I=\mathopen] a , b \mathclose[\) un ouvert borné de \( \eR\). L'\defe{espace de Sobolev}{espace!de Sobolev} \( H^1(I)\)\nomenclature[Y]{\( H^1(I)\)}{espace de Sobolev} est l'ensemble
    \begin{equation}
        H^1(I)=\Big\{   u\in L^2(I)\tq\exists g\in L^2(I)\tq\forall \varphi\in  C^{\infty}_c(I),\int_Iu\varphi'=-\int_Ig\varphi   \Big\}.
    \end{equation}
\end{definition}
 
L'unique élément \( g\) de \( L^2(I)\) vérifiant \( \int_Iu\varphi'=-\int_Ig\varphi\) est noté \( u'\) est est nommé \defe{dérivée}{dérivée!dans Sobolev $ H^1(I)$}; nous verrons dans les prochaines pages pourquoi.

L'espace \( H^1\) accepte le produit scalaire suivant :
\begin{equation}
    \langle u, v\rangle =\int_Iuv+\int_Iu'v',
\end{equation}
et nous notons \( \| . \|_{H^1}\) la norme correspondante qui n'est autre que
\begin{equation}
    \| u \|_{H^1}=\langle u, u\rangle =\| u \|^2_{L^2}+\| u' \|_{L^2}.
\end{equation}

Nous introduisons l'espace \( L^1_{loc}(I)\)\nomenclature[Y]{\( L^1_{loc}(I)\)}{fonctions intégrables sur les compacts de \( I\)} des fonctions étant \( L^1\) sur tout compact de \( I\). 

\begin{proposition} \label{PropLGoLtcS}
    Si \( f\in L^1_{loc}(I)\) est telle que
    \begin{equation}
        \int_If\varphi'=0
    \end{equation}
    pour tout \( \varphi\in  C^{\infty}_c(I)\), alors il existe une constante \( C\) telle que \( f=C\) presque partout.
\end{proposition}

\begin{proof}
    Soit \( \psi\in C^{\infty}_c(I)\) une fonction d'intégrale \( 1\) sur \( I\). Si \( w\in C^{\infty}_c(I)\) alors nous considérons la fonction
    \begin{equation}
        h=w-\psi\int_Iw,
    \end{equation}
    qui est dans \(  C^{\infty}_c(I)\) et dont l'intégrale sur \( I\) est nulle. Par la proposition \ref{PropHFWNpRb}, la fonction \( h\) admet une primitive dans \(  C^{\infty}_c(I)\); et nous notons \( \varphi\) cette primitive. L'hypothèse appliquée à \( \varphi\) donne
    \begin{equation}
        0=\int_If\varphi'=\int_If\left( w-\psi\int_Iw \right)=\int_Ifw-\underbrace{\left( \int_If(x)\psi(x)dx \right)}_C\left( \int_Iw(y)dy \right)=\int_Iw(f-C).
    \end{equation}
    L'annulation de la dernière intégrale implique par la proposition \ref{PropUKLZZZh} que \( f-C=0\) dans \( L^2\), c'est à dire \( f=C\) presque partout.
\end{proof}

\begin{corollary}   \label{CorEVJYihj}
    Si \( u\in H^1(I)\) et si \( u'=0\) alors il existe une constant \( C\) telle que \( u=C\) presque partout.
\end{corollary}

\begin{proof}
    L'hypothèse \( u'=0\) signifie que pour tout fonction \( \varphi\in C^{\infty}_c(I)\),
    \begin{equation}
        \int_Iu\varphi'=\int_Iu'\varphi=0.
    \end{equation}
    La proposition \ref{PropLGoLtcS} nous dit alors qu'il existe une constante \( C\) telle que \( u=C\) presque partout.
\end{proof}

\begin{lemma}   \label{LemMPkbZxX}
    Tout élément de \( H^1(I)\) admet un unique représentant continu.
\end{lemma}
Nous verrons dans le corollaire \ref{CorCEPJGAu} que ce représentant pourra être prolongé par continuité sur \( \bar I\).

\begin{proof}
    Soit \( y_0\in I\) et \( u\in H^1(I)\). Nous considérons la fonction
    \begin{equation}
        \bar u(x)=\int_{y_0}^xu'(t)dt.
    \end{equation}
    Notons que par définition, \( u'\in L^2\) donc l'intégrale ne pose pas de problèmes. Montrons que \( \bar u\) est continue sur \( \bar I\). Pour cela nous considérons \( x\in\bar I\) et \( h\) tel que \( x+h\in \bar I\). Alors
    \begin{equation}
        \big| \bar u(x+h)-\bar u(x) \big|=\big| \int_x^{x+h}u' \big|\leq \int_x^{x+h}| u' |.
    \end{equation}
    Mais la fonction \( | u' |\) est dans \( L^1_{loc}(I)\) par le lemme \ref{LemTLHwYzD}; elle est en particulier intégrable sur un ouvert contenant \( x\) et par conséquent la dernière intégrale tend vers zéro lorsque \( h\) tend vers \( 0\).

    Nous prouvons à présent que \( \bar u\) est dans \( H^1(I)\) et que sa dérivée est égale à \( u'\); pour cela nous allons montrer que pour tout \( \varphi\in  C^{\infty}_c(I)\),
    \begin{equation}
        \int_I\bar u\varphi'=-\int_Iu'\varphi.
    \end{equation}
    Nous avons
    \begin{equation}
            \int_I\bar u\varphi'=\int_I\left( \int_{y_0}^xu'(t)dt\right)\varphi'(x)dx
            =\int_{a}^{y_0}\left( \int_{y_0}^xu'(t)dt\right)\varphi'(x)dx+\int_{y_0}^b\left( \int_{y_0}^xu'(t)dt\right)\varphi'(x)dx.
    \end{equation}
    Pour faire plus court, nous notons \( f(t,x)=u'(t)\varphi'(x)\). La première intégrale vaut
    \begin{subequations}
        \begin{align}
            \int_a^{y_0}\left( \int_{y_0}^x u'(t)\varphi'(x) \right)&=\int_a^{y_0}\left(\int_{y_0}^af(t,x)\mtu_{t<x}(t,x)dt\right)dx\\
            &=\int_{y_0}^a\int_a^{y_0}f(t,x)\mtu_{t>x}dxdt  \label{subeqBVyBPLp}\\
            &=\int_{y_0}^a\int_a^tf(t,x)dxdt\\
            &=-\int_a^{y_0}\int_a^tu'(t)\varphi'(x)dx\,dt
        \end{align}
    \end{subequations}
    La permutation d'intégrales pour obtenir \eqref{subeqBVyBPLp} est due au théorème de Fubini \ref{ThoFubinioYLtPI}\ref{ItemQMWiolgiii}. Par le même petit jeu, la seconde intégrale vaut
    \begin{equation}
        \int_{y_0}^b\int_t^b u'(t)\varphi'(x)dx\,dt.
    \end{equation}
    En refaisant la somme,
    \begin{subequations}
        \begin{align}
            \int_I\bar u\varphi'
            &=-\int_a^{y_0}u'(t)\left( \int_a^t\varphi'(x)dx \right)dt+\int_{y_0}^bu'(t)\left( \int_t^b\varphi'(x)dx \right)dt\\
            &=-\int_a^{y_0}u'(t)\big( \varphi(t)-\varphi(a) \big)dt+\int_{y_0}^bu'(t)\big( \varphi(b)-\varphi(t) \big)\\
            &=-\int_a^bu'\varphi\\
            &=-\int_Iu'\varphi.
        \end{align}
    \end{subequations}
    Notons que \( \varphi(a)=\varphi(b)=0\) parce que \( \varphi\) est à support compact dans \( \mathopen] a , b \mathclose[\). Nous avons donc prouvé que \( \bar u\) est dans \( H^1(I)\) et que \( \bar u'=u'\). Par le corollaire \ref{CorEVJYihj}, nous avons une constante \( C\) telle que \( \bar u=u+C\) presque partout, c'est à dire \( u=\bar u +C\) dans \( H^1(I)\). 

        En résumé, \( \tilde u\tilde u==\bar u+C\) est un représentant continu de \( u\) dans \( L^2(I)\).

        L'unicité du représentant continu est simplement le fait que deux fonctions continues égales presque partout sont égales (proposition  \ref{PropNCMToWI}).
    
\end{proof}

\begin{proposition}     \label{PropGWOIoDg}
    Si \( u\in H^1(I)\), alors
    \begin{equation}
        u(x)-u(y)=\int_y^xu'
    \end{equation}
    pour tout \( x,y\in I\).
\end{proposition}

\begin{proof}
    Pour fixer les idées, nous supposons \( x<y\). Nous considérons une suite \( \varphi_n\in C^{\infty}_c(I)\) convergeant uniformément sur \( I\) vers \( \mtu_{\mathopen[ x , y \mathclose]}\). Nous exigeons de plus que 
    \begin{itemize}
        \item 
        \( \varphi_n'\) est positive sur \( \mathopen[ a , x+\frac{1}{ n } \mathclose]\)
    \item
        \( \varphi_n'\) est négative sur \( \mathopen[ y-\frac{1}{ n } , b \mathclose]\) 
    \item
        \( \varphi_n=1\) sur \( \mathopen[ x+\frac{1}{ n } , y-\frac{1}{ n } \mathclose]\).
    \item
        \( \varphi_n=0\) sur \( \mathopen[ a , x-1/n \mathclose]\) et sur \( \mathopen[ y+1/n , b \mathclose]\).
    \end{itemize}
    Pour chaque \( n\), nous découpons l'intégrale comme
    \begin{equation}        \label{EqRPwqpve}
        -\int_Iu'\varphi_n=\int_Iu\varphi'_n=\int_a^{a-1/n}u\varphi'_n+\int_{x-1/n}^{x+1/n}u\varphi'_n+\int_{x+1/n}^{y-1/n}u\varphi'_n+\int_{y-1/n}^{y+1/n}u\varphi'_n+\int_{y+1/n}^{b}u\varphi'_n.
    \end{equation}
    Par construction de \( \varphi_n\), de ces \( 5\) morceaux, il n'en reste que deux de non nulles :
    \begin{equation}
        \int_Iu\varphi'=\underbrace{\int_{x-1/n}^{x+1/n}u(t)\varphi'_n(t)dt}_A+\underbrace{\int_{y-1/n}^{y+1/n}u(t)\varphi'_n(t)dt}_B
    \end{equation}

    Soit \( \epsilon>0\) et \( n\) suffisamment grand pour avoir \( u(t)\in B\big( u(x),\epsilon \big)\) pour tout \( t\in B(x,\frac{1}{ n })\) et (en même temps) \( u(t)\in B\big( u(y),\epsilon \big)\) pour tout \( t\in B(y,\frac{1}{ n })\). C'est la continuité de \( u\) qui permet de trouver un tel \( n\). Pour cette valeur de \( n\), en tenant compte des hypothèses sur la positivité de \( \varphi_n'\) nous avons
    \begin{equation}
        \int_{x-1/n}^{x+1/n}\big( u(x)-\epsilon \big)\varphi'_n(t)dt\leq\int_{x-1/n}^{x+1/n}u(t)\varphi'_n(t)dt\leq\int_{x-1/n}^{x+1/n}\big( u(x)+\epsilon \big)\varphi'_n(t)dt,
    \end{equation}
    mais par hypothèse sur \( \varphi_n\) nous trouvons
    \begin{equation}
        \int_{x-1/n}^{x+1/n}\varphi'_n(t)dt=\varphi_n(x+\frac{1}{ n })-\varphi(x+\frac{1}{ n })=1.
    \end{equation}
    donc
    \begin{equation}    \label{EqLYrpEdb}
        u(x)-\epsilon\leq\int_{x-1/n}^{x+1/n}u(t)\varphi'_n(t)dt\leq u(x)+\epsilon.
    \end{equation}
    Pour encadrer la seconde, il faut être plus prudent avec les signes parce que \( \varphi'_n\) y est négative. En posant \( \psi_n=-\varphi_n\) nous avons
    \begin{equation}
        -B=\int_{y-1/n}^{y+1/n}u(t)\psi_n(t)dt,
    \end{equation}
    et donc
    \begin{equation}
        u(y)-\epsilon\leq -B\leq u(y)+\epsilon
    \end{equation}
    ou encore
    \begin{equation}
        -\epsilon-u(y)\leq B\leq \epsilon-u(y).
    \end{equation}
    En additionnant avec \eqref{EqLYrpEdb} nous voyons que pour tout \( \epsilon>0\) il existe un \( N(\epsilon)\) tel que nous ayons
    \begin{equation}    \label{EqEBwWUxm}
        u(x)-u(y)-2\epsilon\leq\int_Iu'\varphi_{n}\leq u(x)-u(y)+2\epsilon
    \end{equation}
    pour tout \( n\geq N\). Nous voulons évidemment prendre la limite \( \epsilon\to 0\), c'est à dire \( n\to \infty\). Étant donné que \( \varphi_n(t)<1\) pour tout \( t\) et pour tout \( n\), la fonction \( t\mapsto u'(t)\varphi_n(t)\) est dominée par \( u'\), qui est dans \( L^1(I)\) par le lemme \ref{LemTLHwYzD}. Le théorème de la convergence dominée nous permet donc d'affirmer que
    \begin{equation}
        \lim_{n\to \infty} \int_Iu'\varphi_n=\int_Iu'\mtu_{[x,y]}=\int_x^yu',
    \end{equation}
    et donc les inégalités \eqref{EqEBwWUxm} donnent le résultat, grâce au signe dans \eqref{EqRPwqpve}.
\end{proof}

\begin{corollary}   \label{CorCEPJGAu}
    Si \( [u]\in H^1(I)\), le représentant continu \( u\in C^0(I)\) peut être prolongé par continuité en \( u\in C^0(\bar I)\).
\end{corollary}

\begin{proof}
    Soit \( (x_n)\) une suite strictement croissante dans \( \mathopen] a , b \mathclose[\) convergeant vers \( b\). Nous voulons montrer que la suite \( \big( u(x_n) \big)\) est de Cauchy dans \( \eR\), ce qui nous permettra de définir
        \begin{equation}
            u(b)=\lim_{n\to \infty} u(x_n).
        \end{equation}
        qui sera évidemment continue. Cette construction ne dépendra pas du choix de la suite \( (x_n)\) parce que deux fonctions continues sur \( \bar I\) et égales sur \( I\) sont égales sur \( \bar I\).

        En notant \( u'\) la dérivée de \( u\) dans \( H^1\), nous avons par construction du représentant continu : \( u(x)=\int_{y_0}^xu'(t)dt\). Et donc
        \begin{equation}
            \big| u(x_n)-u(x_{n+p}) \big|=\left| \int_{y_0}^{x_n}u'-\int_{y_0}^{x_{n+p}}u' \right| =\left| \int_{x_n}^{x_{n+p}}u' \right| .
        \end{equation}
        Vu que la suite \( (x_n)\) est de Cauchy et que \( u'\) est intégrable (même sur \( \bar I\)), la limite \( n\to\infty\) de cela est zéro, quelle que soit la valeur de \( p\). Donc \( \big( u(x_n) \big)\) est ce Cauchy dans \( \eR\) et est donc convergente.
\end{proof}
\index{prolongement!par continuité!dans \( H^1(I)\)}

\begin{proposition}[\cite{KXjFWKA}]     \label{ThoESIyxfU}
    Quelque propriétés de l'espace de Sobolev \( H^1(I)\) où \( I=\mathopen] a , b \mathclose[\) est un ouvert borné de \( \eR\).
    \begin{enumerate}
        \item
            \( H^1(I)\) est un espace de Hilbert.
        \item
            \( H^1(I)\) s'injecte de façon compacte dans \( C^0(\bar I)\).
        \item
            \( H^1(I)\) s'injecte de façon continue dans \( L^2(I)\).
    \end{enumerate}
\end{proposition}
\index{espace!de fonctions!Sobolev \( H^1\)}
\index{espace!de Hilbert!espace de Sobolev \( H^1\)}
\index{espace!\( L^2\)!Sobolev}
\index{dérivation!au sens des distribution!Sobolev}


\begin{proof}
    Nous prouvons point par point.
    \begin{enumerate}
        \item
            Le seul critère à vérifier est la complétude. Pour cela nous considérons une suite de Cauchy \( (u_n)\) dans \( H^1(I)\). Si \( \epsilon>0\), alors il existe \( N>0\) tel que pour tout \( p\geq 0\) nous ayons \( \| u_{n+p}-u_n \|_{H^1}^2\leq \epsilon\), c'est à dire
            \begin{equation}
                \| u_{n+p}-u_n \|^2_{L^2}+\| u'_{n+p}-u'_n \|^2_{L^2}+
            \end{equation}
            En particulier les suites \( (u_n)\) et \( (u'_n)\) sont de Cauchy dans \( L^2\) qui est complet par le théorème de Fischer-Riesz \ref{ThoGVmqOro}. Nous notons donc
            \begin{subequations}
                \begin{align}
                    u_n\stackrel{L^2}{\to}u\\
                    u'_n\stackrel{L^2}{\to}v.
                \end{align}
            \end{subequations}
            Nous allons démontrer les points suivants\quext{C'est le moment de lire l'énoncé du problème \ref{ProbTOElufz} et de m'écrire si vous avez une réponse.}
            \begin{itemize}
                \item \( u\in H^1(I)\) avec \( u'=v\).
                \item \( u_n\stackrel{H^1}{\to}u\).
            \end{itemize}
            Pour cela nous introduisons la dérivée faible de \( u\) dans \( L^2\), c'est à dire la forme linéaire continue \( \partial u\) sur \(  C^{\infty}_c(I)\) :
            \begin{equation}
                \begin{aligned}
                    \partial u\colon  C^{\infty}_c(I)&\to \eR \\
                    \varphi&\mapsto \langle \partial u, \varphi\rangle =-\int_Iu\varphi'. 
                \end{aligned}
            \end{equation}
            Pour tout \( \varphi\in C^{\infty}_c(I)\) nous avons
            \begin{subequations}
                \begin{align}
                \big| \langle \partial u, \varphi\rangle -\langle u_n', \varphi\rangle  \big|&=\left| -\int_Iu\varphi'-\int_Iu'_n\varphi \right| \\
                &=\left| -\int_Iu\varphi'-\int_Iu_n\varphi' \right| \\
            &\leq \int_I| u-u_n | |\varphi' |\\
            &\leq\| u-u_n \|_{L^2}\| \varphi' \|_{L^2}\,\text{Cauchy-Schwartz dans \( L^2\)}\\
            &\to 0.
                \end{align}
            \end{subequations}
            À la première ligne, la première intégrale est la définition de l'action de la forme \( \partial u\) sur \( \varphi\) alors que la seconde est seulement un produit scalaire dans \( L^2\). Tout deux sont notés avec les crochets. En tant que limite dans \( \eR\) nous avons
            \begin{equation}
                \lim_{n\to \infty} \langle u'_n, \varphi\rangle =\langle \partial u, \varphi\rangle .
            \end{equation}
            Dans le calcul suivant, les deux crochets sont des produits scalaires dans \( L^2\) :
            \begin{subequations}
                \begin{align}
                \big| \langle u_n', \varphi\rangle -\langle v, \varphi\rangle  \big|&=\left| -\int_Iu'_n\varphi-\int_Iv\varphi \right| \\
            &\leq \int_I| u'_n-v| |\varphi |\\
            &\leq\| u'_n-v \|_{L^2}\| \varphi \|_{L^2}\\
            &\to 0.
                \end{align}
            \end{subequations}
            Donc en tant que limite dans \( \eR\),
            \begin{equation}
                \lim_{n\to \infty} \langle u'_n, \varphi\rangle =\langle v, \varphi\rangle .
            \end{equation}
            Par unicité de la limite nous en déduisons que pour tout \( \varphi\in C^{\infty}_c(I)\),
            \begin{equation}
                \langle \partial u, \varphi\rangle =\langle v, \varphi\rangle .
            \end{equation}
            Encore une fois nous répétons qu'à gauche le crochet est l'application de la forme \( \partial u\) sur \( \varphi\) tandis qu'à droite c'est le produit scalaire dans \( L^2\). 

            Nous sommes maintenant à même de prouver que \( u\in H^1(I)\) et que sa dérivée (au sens de \( H^1\)) est \( v\). En effet
            \begin{equation}
                \int_Iu\varphi'=-\langle \partial u, \varphi\rangle =-\langle v, \varphi\rangle =-\int_Iv\varphi.
            \end{equation}
            Par conséquent nous avons \( u'=v\) dans \( H^1\) et aussi \( u'=v\) presque partout au sens des fonctions.

            Nous pouvons alors prouver que \( u_n\to u\) dans \( H^1(I)\) :
            \begin{equation}
                \| u_n-u \|^2_{H^1(I)}=\| u_n-u \|^2_{L^2}+\| u'_n-u' \|_{L^2}^2.
            \end{equation}
            Mais nous savons déjà que \( u_n\to u\) dans \( L^2\) (d'ailleurs c'est la définition de \( u\)) et que \( u'=v\) alors que par définition de \( v\), nous avons \( u'_n\to v\) dans \( L^2\). Tout cela donne que \( u_n\to u\) dans \( H^1(I)\) et donc que \( H^1(I)\) est un espace complet.

        \item

            L'application que nous allons prouver être compacte entre \( H^1(I)\) et \( C^0(\bar I)\) est
            \begin{equation}
                \begin{aligned}
                    \psi\colon H^1(I)&\to C^0(\bar I) \\
                    [u]&\mapsto \tilde u 
                \end{aligned}
            \end{equation}
            où \( [u]\) désigne une classe de fonction dans \( H^1(I)\) et \( \tilde u\) est son représentant continu prolongé par continuité à \( \bar I\)\footnote{Encore que par soucis d'économie d'encre nous n'allons pas écrire toujours les tildes et noter \( u\) le représentant continu prolongé à \( \bar I\) par le corollaire \ref{CorCEPJGAu}.}, qui existe par le lemme \ref{LemMPkbZxX} et le corollaire \ref{CorCEPJGAu}. Cette application est une injection par l'unicité du représentant continu. Nous allons prouver que c'est une application compacte en utilisant le critère \ref{ItemJIkpUbLii} de la proposition \ref{PropDGsPtpU}. Pour cela nous allons commencer par utiliser le théorème d'Ascoli sur l'ensemble \( \tilde \mB\) des représentants continus des éléments de \( \mB\), prolongés par continuité sur \( \bar I\); c'est à dire \( \tilde B\subset C^0(\bar I)\).

            Soit \( u\in \tilde \mB\); par la proposition \ref{PropGWOIoDg}, nous avons
            \begin{subequations}
                \begin{align}
                    \big| u(x)-u(y) \big|&=\big| \int_y^xu'(t)dt \big|\\
                    &=\left| \int_I\mtu_{[x,y]}(t)u'(t)dt \right| \\
                    &\leq\| \mtu_{\mathopen[ x , y \mathclose]} \|_{L^2}\| u' \|_{L^2}\\
                    &\leq\sqrt{| x-y |}\| u' \|_{H^1}\\
                    &\leq\sqrt{| x-y |}.
                \end{align}
            \end{subequations}
            où nous insistons sur le fait que la continuité n'impliquant pas la dérivabilité, le \( u'\) ici est la dérivé au sens de \( H^1\), et non la dérivée usuelle. Quoi qu'il en soit, l'ensemble \(\tilde  \mB\) est équicontinu\footnote{Définition \ref{DefUWmVBcZ}}. Nous montrons à présent qu'il est également borné pour la norme uniforme. Soit \( u\in\tilde \mB\); vu la construction du représentant continu au lemme \ref{LemMPkbZxX}, nous avons
            \begin{subequations}
                \begin{align}
                \big| u(x) \big|&=\left| \frac{1}{ b-a }\int_a^bu(x)dy \right| \\
                &=\left| \frac{1}{ b-a }\int_a^b\left( \int_y^xu'(t)dt-u(y) \right)dy \right| \\
                &=\left| \frac{1}{ b-a }\int_a\int_y^xu'(t)dtdy-\frac{1}{ b-a }\int_a^b u(y)dy \right| \\
                &\leq\frac{1}{ b-a }\int_a^b\int_a^b| u'(t) |dt\,dy+\frac{1}{ b-a }\int_a^b| u(y) |dy \label{EqCFwSOxh}.
                \end{align}
            \end{subequations}
            À ce niveau, il faut remarquer que dans la première intégrale, le passage de la valeur absolue à l'intérieur de l'intégrale en même temps que l'élargissement des bornes n'a rien d'innocent. Si \( x<y\), les bornes ne sont pas «dans le bon ordre» et nous ne pouvons pas faire la majoration usuelle en entrant simplement la valeur absolue. Ici nous tenons compte de cela en élargissant les bornes, et en les mettant dans le bon ordre. Le passage exact est le suivant : si \( x,y\in\mathopen] a , b \mathclose[\), nous avons
                \begin{equation}
                \left| \int_y^xf(t)dt \right| \leq\left| \int_y^x| f(t) |dt \right| \leq\left| \int_a^b| f(t) |dt \right| =\int_a^b| f(t) |dt.
                \end{equation}
                Notons en particulier que dans le cas du passage vers l'équation \eqref{EqCFwSOxh}, le nombre \( x\) est fixé alors que \( y\) est une variable d'intégration. Donc l'ordre des deux est certainement de temps en temps le «mauvais».

                Quoi qu'il en soit, la première intégrale se réduit à une multiplication par \( b-a\) et le calcul continue :
                \begin{subequations}
                    \begin{align}
                        \big| u(x) \big|&\leq \int_I| u'(t) |dt+\frac{1}{ b-a }\int_I| u |\\
                        &\leq \sqrt{b-a}\| u' \|_{L^2}+\frac{1}{ \sqrt{b-a} }\| u \|_{L^2}\\
                        &\leq\left( \sqrt{b-a}+\frac{1}{ \sqrt{b-a} } \right)\big( \| u' \|_{L^2}+\| u \|_{L^2} \big)\\
                        &\leq\left( \sqrt{b-a}+\frac{1}{ \sqrt{b-a} } \right) \| u \|_{H^1}\\
                        &= \sqrt{b-a}+\frac{1}{ \sqrt{b-a} }.
                    \end{align}
                \end{subequations}
                Donc \( \tilde \mB\) est borné pour la norme \( L^{\infty}\). Et c'est même borné par un nombre facilement calculable connaissant \( I\). En particulier l'ensemble
                \begin{equation}
                    \{ u(x)\tq u\in H^1 \}
                \end{equation}
                est pour, tout \( x\), contenu dans la boule de rayon \( \sqrt{a-b}+\frac{1}{ \sqrt{a-b} }\) et donc est relativement compact dans \( \eR\). Par conséquent le théorème d'Ascoli \ref{ThoKRbtpah} nous dit que l'ensemble \( \tilde B\) est relativement compact dans \( C^0(I)\).

                Par conséquent nous avons montré que l'image par \( \psi\) de la boule unité fermée \( \mB\) de \( H^1(I)\) est relativement compacte dans \( C^0(\bar I)\), ce qui signifie que \( \psi\) est une application compacte.


            \item

                Les éléments de \( H^1(I)\) sont des éléments de \( L^2(I)\); donc l'identité est une injection. Nous devons seulement étudier la continuité. Si \( (u_n)\) est une suite dans \( H^1\) convergeant dans \( H^1\) vers \( u\), alors
                \begin{equation}
                    \| u_n-u \|_{L^2}\leq\| u_n-u \|_{L^2}+\| u'_n-u' \|_{L^2}=\| u_n-u \|_{H^1}\to 0.
                \end{equation}
                Donc la suite des images (par l'identité) converge dans \( L^2\). L'identité est donc continue.

    \end{enumerate}
    
\end{proof}


\begin{probleme}    \label{ProbTOElufz}
    Au point de la preuve auquel vous devriez être si vous lisez ceci, vous pourriez avoir envie de démontrer \( u'=v\) de la façon suivante :
    \begin{equation}
        \int_I u\varphi'=\lim_{n\to \infty} \int_Iu_n\varphi=-\lim_{n\to \infty} \int_Iu'_n\varphi=-\int_Iv\varphi.
    \end{equation}
    J'avoue ne pas trouver d'exemples pour lesquels ça ne marche pas. Est-ce qu'on peut inverser la limite et l'intégrale dans \( L^2\) ?

    Ceci n'invalide pas la preuve donnée, mais ça suggère un sacré raccourcis.
\end{probleme}

%+++++++++++++++++++++++++++++++++++++++++++++++++++++++++++++++++++++++++++++++++++++++++++++++++++++++++++++++++++++++++++
\section{Série de Fourier}
%+++++++++++++++++++++++++++++++++++++++++++++++++++++++++++++++++++++++++++++++++++++++++++++++++++++++++++++++++++++++++++

Source : \cite{MaureyHilbertFourier}

Nous utilisons ici des résultats de bases hilbertiennes de la sous-section \ref{SubsecDxkjut}. Nous considérons l'espace de Hilbert \( L^2\mathopen[ -T/2 , T/2 \mathclose]\) muni du produit scalaire
\begin{equation}
    \langle f, g\rangle =\frac{1}{ T }\int_{-T/2}^{T/2}f(t)\overline{ g(t) }dt.
\end{equation}
Pour toute fonction pour laquelle ça a un sens (que ce soit des fonctions \( L^2\) ou non), nous posons
\begin{equation}\label{EqhIPoPH}
    c_n(f)=\frac{1}{ T }\int_{-T/2}^{T/2}f(t) e^{-2i\pi nt/T}dt.
\end{equation}
Ces nombres sont les \defe{coefficients de Fourier}{coefficients!de Fourier} de \( f\). Leur importance dans le cadre de \( L^2\) provient du fait que la famille de fonctions
\begin{equation}
    e_k(t)=  e^{2i\pi kt/T}
\end{equation}
est une base hilbertienne de \( L^2\mathopen[ -T/2 , T/2 \mathclose]\) et que
\begin{equation}
    c_n(f)=\langle f, e_n\rangle .
\end{equation}


Pour une fonction donnée \( f\in L^2\), nous définissons\nomenclature[Y]{\( S_nf\)}{somme partielle de série de Fourier} 
\begin{equation}
    S_nf=\sum_{k=-n}^n\langle f, e_k\rangle e_k.
\end{equation}
Nous avons alors, par l'inégalité de Parseval, la convergence
\begin{equation}
    S_nf\to f
\end{equation}
au sens \( L^2\). Si nous voulons une vraie convergence ponctuelle voir uniforme \( (S_nf)(x)\to f(x)\), alors il faut ajouter des hypothèses sur la continuité de \( f\) et le comportement des coefficients \( c_n\).

La \defe{série de Fourier}{série!de Fourier} associée à \( f\) est alors
\begin{equation}
    f(x)\sim\sum_{n=-\infty}^{\infty}c_n(f) e^{2\pi i\frac{ n }{ T }x}.
\end{equation}
Cette expression est pour l'instant purement formelle. Cela ne présume ni de la convergence de la série, ni, au cas où elle serait convergente, que la limite soit \( f\).

\begin{proposition}[\cite{DupFourEsdgKEI}]  \label{PropmrLfGt}
    Soit \( f\) une fonction continue et périodique telle que sa série de Fourier converge uniformément. Alors la convergence est vers \( f\).
\end{proposition}
%TODO : ajouter ce théorème à Wikipédia, et le lier dans l'article sur la formule sommatoire de Poisson.

\begin{proof}
    Notons d'abord que \( f\) étant continue sur \(\mathopen[ 0 , 2\pi \mathclose]\), elle y est bornée et \( L^2\). Par conséquent Parseval nous enseigne que 
    \begin{equation}
        \| S_N(f)-f \|_{L^2}\to 0.
    \end{equation}
    Cela signifie que
    \begin{equation}
        \lim_{N\to \infty} \frac{1}{ 2\pi }\int_{0}^{2\pi}| f(t)-S_N(t) |^2dt=0.
    \end{equation}
    L'hypothèse de convergence uniforme nous dit que la fonction \( | f(t)-S_N(t) |^2\) converge uniformément vers la fonction \( | f(t)-S(t) |^2\) où nous avons écrit \( S\) la limite de \( S_N\). En permutant la limite et l'intégrale,
    \begin{equation}
        \frac{1}{ 2\pi }\int_0^{2\pi}| f(t)-S(t) |^2dt=0,
    \end{equation}
    ce qui signifie que la fonction \( t\mapsto | f(t)-S(t) |^2\) est la fonction nulle. Nous en déduisons que \( f=S\).
\end{proof}

\begin{proposition}     \label{PropSgvPab}
    Soit \( f\) une fonction \( 2\pi\)-périodique. Si \( \sum_{n\in \eZ}| c_n(f) |<\infty\), alors pour tout \( x\in \eR\) nous avons
    \begin{equation}
        f(x)=\sum_{n\in \eZ}c_n(f) e^{inx}.
    \end{equation}
    De plus, la suite \( (S_nf)\) converge uniformément vers \( f\).
\end{proposition}

\begin{proof}
    Nous posons 
    \begin{equation}
        g(x)=\sum_{n\in \eZ}c_n(f) e^{inx}.
    \end{equation}
    Étant donné les hypothèses, la série de droite converge absolument, la fonction \( g\) est continue sur \( \eR\). Nous avons
    \begin{equation}
        \big| g(x)-(S_nf)(x) \big|\leq \sum_{| k |> n}| c_k(f) |,
    \end{equation}
    mais le terme de droite tend vers zéro lorsque \( n\to \infty\) parce que c'est le reste d'une série convergente. Cela signifie que \( S_nf\) converge uniformément vers \( g\).

    Par ailleurs nous savons que dans \( L^2\) nous avons la convergence \( S_nf\to f\) (parce que \( f\) est continue sur le compact \( \mathopen[ 0 , 2\pi \mathclose]\) et donc y est bornée et \( L^2\)), ce qui signifie que \( g=f\) presque partout au sens \( L^2\). Ces deux fonctions étant continues, elles sont égales partout.
\end{proof}

\begin{theorem}     \label{ThozHXraQ}
    Soit \( f\), une fonction \( C^1\) et \( 2\pi\)-périodique. Nous notons \( (c_n)_{n\in \eZ}\) la suite de ses coefficients de Fourier. Alors \( (c_n)\in \ell^1(\eZ)\) et pour tout \( x\in \eR\) nous avons
    \begin{equation}
        f(x)=\sum_{n\in \eZ}c_n(f) e^{inx}.
    \end{equation}
\end{theorem}

\begin{proof}
    Soit \( n\in \eZ\). Nous posons \( g(t)=f(t) e^{-int}\). Nous avons
    \begin{equation}
        0=g(2\pi)-g(0)=\int_0^{2\pi}g'(t)dt=\int_0^{2\pi}\big[ f'(t) e^{-int}-inf(t) e^{-int} \big].
    \end{equation}
    Du coup, \( c_n(f')=inc_n(f)\). La fonction \( f'\) étant bornée (parce que continue sur \( \mathopen[ 0 , 2\pi \mathclose]\)), elle est de carré intégrable sur \( \mathopen[ 0 , 2\pi \mathclose]\) et par les inégalités de Parseval (théorème \ref{ThoyAjoqP}) nous avons
    \begin{equation}
        \sum_{n\in \eZ}| c_n(f') |^2<\infty.
    \end{equation}
    Par conséquent \( (c_n)\in \ell^2(\eZ)\) et a forciori \( (c_n)_{n\in \eN}\in \ell^2(\eN)\). L'inégalité de Cauchy-Schwartz nous indique alors
    \begin{equation}
        \sum_{n\in \eN}| c_n(f) |=\sum_{n\in \eN}\frac{1}{ n }| c_n(f') |\leq \left( \sum_n\frac{1}{ n^2 } \right)^{1/2}\left( \sum_{n}| c_n(f') |^2 \right)^{1/2}<\infty.
    \end{equation}
    Nous procédons de même pour \( n<0\). Cela prouve que 
    \begin{equation}
        \sum_{n\in \eZ}| c_n(f) |<\infty.
    \end{equation}
\end{proof}

%+++++++++++++++++++++++++++++++++++++++++++++++++++++++++++++++++++++++++++++++++++++++++++++++++++++++++++++++++++++++++++
\section{Noyaux pour retrouver la fonction}
%+++++++++++++++++++++++++++++++++++++++++++++++++++++++++++++++++++++++++++++++++++++++++++++++++++++++++++++++++++++++++++

Le \defe{noyau de Dirichlet}{noyau!Dirichlet}\index{Dirichlet!noyau} est la fonction
\begin{equation}
    D_n(t)=\sum_{k=-n}^n e^{int}.
\end{equation}
Le \defe{noyau de Fejér}{noyau!Fejér}\index{Fejér!noyau} est la moyenne de Cesaro des noyaux de Dirichlet :
\begin{equation}
    F_n(t)=\frac{1}{ n }\sum_{k=0}^{n-1}D_k(t).
\end{equation}

\begin{lemma}
    Le noyau de Dirichlet s'exprime sous la forme
    \begin{equation}    
        D_n(t)=\frac{ \sin\left( \frac{ 2n+1 }{ 2 }t \right) }{ \sin(t/2) }
    \end{equation}
\end{lemma}
Note : ce noyau n'est pas positif.

\begin{proof}
    Nous commençons par mettre en évidence le premier terme :
    \begin{equation}
        D_n(t)=\sum_{k=-n}^n e^{int}= e^{-int}\sum_{k=0}^{2n} e^{ikt}.
    \end{equation}
    En utilisant la formule de la somme géométrique,
    \begin{subequations}
        \begin{align}
            D_n(t)&= e^{-int}\frac{ 1-( e^{it})^{2n+1} }{ 1- e^{it} }\\
            &= e^{-int}\frac{ 1- e^{(2n+1)it} }{ 1- e^{it} }\\
            &= e^{-int}\frac{  e^{(2n+1)it/2} }{  e^{i\frac{ t }{ 2 }} }\frac{  e^{-(2n+1)it/2}- e^{(2n+1)it/2} }{  e^{-it/2}- e^{it/2} }\\
            &=\frac{ (-2i)\sin\left( \frac{ 2n+1 }{ 2 }t \right) }{ (-2i)\sin\left( \frac{ t }{2} \right) }.
        \end{align}
    \end{subequations}
\end{proof}

\begin{theorem}[Théorème de Dirichlet]\index{théorème!Dirichlet}\index{Dirichlet!théorème}
    Soit \( f\) une fonction \( 2\pi\)-périodique et \( C^1\) par morceaux. Alors pour tout \( x\in \eR\) nous posons
    \begin{equation}
        s_n(x)=\sum_{k=-n}^nc_k(f) e^{ikx}.
    \end{equation}
    Alors nous avons
    \begin{equation}
        \lim_{n\to \infty} s_n(x)=\frac{ f(x^+)+f(x^-) }{ 2 }.
    \end{equation}
\end{theorem}

%---------------------------------------------------------------------------------------------------------------------------
\subsection{Théorème de Fejér}
%---------------------------------------------------------------------------------------------------------------------------

\begin{lemma}   \label{LemtCAjJz}
    Le noyau de Fejér s'exprime sous la forme
    \begin{equation}    \label{EqLOtzCf}
        F_n(t)=\frac{1}{ n }\left( \frac{ \sin\frac{ nt }{2} }{ \sin\frac{ t }{2} } \right)^2.
    \end{equation}
\end{lemma}
Note : ce noyau est positif. C'est important parce qu'on s'en sert dans la preuve du théorème de Fejér.

\begin{proof}
    L'astuce est de noter \( \sin(x)=\Im( e^{ix})\) et de repartir du résultat à propos du noyau de Dirichlet. En utilisant encore la formule de la série géométrique partielle,
    \begin{subequations}
        \begin{align}
            F_n(t)&=\frac{1}{ n\sin(t/2) }\Im\sum_{k=0}^{n-1} e^{(2k+1)it/2}\\
            &=\frac{1}{ n\sin(t/2) }\Im e^{\frac{ it }{ 2 }}\sum_{k=0}^{n-1}\\
            &=\frac{1}{ n\sin(t/2) }\Im e^{\frac{ it }{ 2 }}\left( \frac{ 1- e^{nit} }{ 1- e^{it} } \right)\\
            &=\frac{1}{ n\sin(t/2) }\Im e^{it/2}\frac{  e^{\frac{ nit }{ 2 }}\left(  e^{-\frac{ int }{2}}- e^{\frac{ nit }{2}} \right) }{  e^{\frac{ it }{2}}\left(  e^{-it/2}- e^{it/2} \right) }\\
            &=\frac{1}{ n\sin(t/2) }\underbrace{\Im e^{\frac{ nit }{2}}}_{\sin(nt/2)}\frac{ \sin\left( \frac{ nt }{ 2 } \right) }{ \sin(\frac{ t }{2}) }\\
            &=\frac{1}{ n }\left( \frac{ \sin\frac{ nt }{2} }{ \sin\frac{ t }{2} } \right)^2.
        \end{align}
    \end{subequations}
\end{proof}


\begin{theorem}[Fejèr]
    Soit une fonction continue et \( 2\pi\)-périodique \( f\colon \eR\to \eC\). Pour tout \( k\in \eZ\) nous notons
    \begin{equation}
        \begin{aligned}
            e_k\colon \eR&\to \eC \\
            x&\mapsto  e^{ikx}. 
        \end{aligned}
    \end{equation}
    Pour chaque \( n\in \eN\) nous posons
    \begin{subequations}
        \begin{align}
            D_n&=\sum_{k=-n}^ne_k& \tilde D_n&=\sum_{k=-n}^nc_k(f)e_k\\
            F_n&=\frac{  D_0+\ldots + D_{n-1} }{ n }&  \tilde F_n&=\frac{ \tilde D_0+\ldots +\tilde D_{n-1} }{ n }
        \end{align}
    \end{subequations}
    Alors
    \begin{enumerate}
        \item
            $\frac{1}{ 2\pi }\int_{-\pi}^{\pi}F_n(t)dt=1$.
        \item
            Pour tout \( \alpha\in \mathopen] 0 , \pi \mathclose[\), \( F_n\) converge uniformément sur \( \mathopen[ -\pi , \pi \mathclose]\setminus\mathopen[ -\alpha , \alpha \mathclose]\).
        \item
            La suite \( \tilde F_n \) converge uniformément sur \( \eR\) vers \( f\).
    \end{enumerate}
\end{theorem}

\begin{proof}
    Un calcul usuel montre que
    \begin{equation}
        \int_{-\pi}^{\pi}e_l(t)dt=\begin{cases}
            0    &   \text{si \( l\neq 0\)}\\
            2\pi    &    \text{si \( l=0\)}
        \end{cases}
    \end{equation}
    Nous avons alors
    \begin{equation}
        \int_{-\pi}^{\pi}F_n(t)dt=\frac{1}{ n }\sum_{k=0}^{n-1}\sum_{l=-k}^k\underbrace{\int_{-\pi}^{\pi}e_l(t)dt}_{2\pi\delta_l}=\frac{1}{ n }\sum_{k=0}^{n-1}1=1.
    \end{equation}
    Cela prouve déjà le premier point.

    Pour le second point, en partant de l'expression \eqref{EqLOtzCf} et en considérant \( x\in\mathopen[ -\pi, \pi ,  \mathclose]\setminus\mathopen[ -\alpha , \alpha \mathclose]\) (ce qui nous évite l'annulation du dénominateur),
    \begin{equation}
        | F_n(x) |\leq\frac{1}{ (n+1)\sin^2(\alpha/2) },
    \end{equation}
    et donc \( F_n\to 0\) uniformément sur l'ensemble considéré.

    Nous passons maintenant à cette histoire de convergence uniforme de la moyenne de Cesaro vers \( f\). Pour tout \( n\in \eN\) nous avons
    \begin{subequations}
        \begin{align}
            \tilde  D_n(x)&=\frac{1}{ 2\pi }\sum_{k=-n}^n\left( \int_{-\pi}^{\pi}f(t) e^{-ikt}dt \right) e^{ikx}\\
            &=\frac{1}{ 2\pi }\int_{-\pi}^{\pi}f(t)\sum_{k=-n}^ne_k(x-t)\\
            &=\frac{1}{ 2\pi }\int_{-\pi}^{\pi}f(t)D_k(x-t).
        \end{align}
    \end{subequations}
    Par conséquent, en effectuant le changement de variable \( u=x-t\) et la périodicité,
    \begin{subequations}    \label{EqkDsyAc}
        \begin{align}
            \tilde F_n(x)&=\int_{-\pi}^{\pi}f(t)F_n(x-t)dt\\
            &=-\int_{x+\pi}^{x-\pi}f(x-u)F_n(u)du\\
            &=\int_{-\pi}^{\pi}f(x-u) F_n(u)du.
        \end{align}
    \end{subequations}
    Nous prouvons à présent l'uniforme continuité. Soit \( \epsilon>0\); étant donné que \( f\) est continue et \( 2\pi\)-périodique, elle est uniformément continue et nous considérons \( \delta>0\) tel que \( | x-y |<\delta\) implique \( \big| f(x)-f(y) \big|<\epsilon\). Soit \( M\) un majorant de \( | f |\) sur \( \eR\). L'équation \eqref{EqkDsyAc} nous donne
    \begin{subequations}
        \begin{align}
            \big| f(x)-\tilde F_n(x) \big|&=\big| \frac{1}{ 2\pi }\int_{-\pi}^{\pi}\big( f(x-t)-f(x) \big)F_n(t)dt \big|    \label{ykuGGh}\\
            &\leq\frac{1}{ 2\pi }\int_{\delta\leq| t |\leq \pi}| 2MF_n(t) |dt+\frac{1}{ 2\pi }\int_{-\delta}^{\delta}\epsilon| F_n(t) |dt\\
            &\leq\frac{ 2M }{ 2\pi }\int_{\delta\leq | t |\leq\pi}F_n(t)dt+\epsilon'    \label{uRAMyq}
        \end{align}
    \end{subequations}
    Pour obtenir \eqref{ykuGGh} nous avons pu rentrer \( f(x)\) dans l'intégrale en utilisant le premier point. Pour obtenir \eqref{uRAMyq} nous avons d'abord utilisé la positivité de \( F_n\) (lemme \ref{LemtCAjJz}) pour enlever les valeurs absolues, et nous avons ensuite utilisé le fait que son intégrale valait \( 2\pi\).

    Étant donné que \( F_n\to 0\) uniformément sur \( \mathopen[ -\pi,\pi ,  \mathclose]\setminus\mathopen[ -\alpha , \alpha \mathclose]\), il existe un \( N\) tel que 
    \begin{equation}
        \int_{\delta\leq| t |\leq \pi}F_n(t)dt\leq \epsilon
    \end{equation}
    dès que \( n>N\). Le résultat découle.
\end{proof}

\begin{corollary}   \label{CordgtXlC}
    Soient \( f,g\) deux fonctions continues et \( 2\pi\)-périodiques. Si \( c_n(f)=c_n(g)\) alors \( f=g\).
\end{corollary}

\begin{proof}
    Dans le cas de fonctions continues, le théorème de Fejér nous enseigne que si nous posons 
    \begin{equation}
        S_n(x)=\sum_{k=-n}^{n}c_k(f) e^{ikx}
    \end{equation}
    alors nous avons la convergence
    \begin{equation}
        \frac{1}{ N+1 }\sum_{n=0}^NS_n(f)(x)\to f(x).
    \end{equation}
    C'est à dire qu'une fonction continue est déterminée par ses coefficients de Fourier.
\end{proof}

\begin{example}
    Considérons la fonction
    \begin{equation}
        f(x)=1-\frac{ x^2 }{ \pi^2 }
    \end{equation}
    sur \( \mathopen[ -\pi , \pi \mathclose]\). Nous la développons en série trigonométrique, et étant paire il n'y a pas de sinus. Un calcul montre que
    \begin{equation}
        a_0=\frac{ 4 }{ 3 }
    \end{equation}
    et
    \begin{equation}
        a_n=(-1)^{n+1}\frac{ 4 }{ n^2\pi^2 },
    \end{equation}
    de telle sorte que
    \begin{equation}
        f(x)=\frac{ 2 }{ 3 }-\frac{ 4 }{ \pi^2 }\sum_{n=1}^{\infty}(-1)^n\frac{ \cos(nx) }{ n^2 }.
    \end{equation}
    Nous avons \( f(\pi)=0\), mais vu le développement,
    \begin{equation}
        f(\pi)=\frac{ 2 }{ 3 }-\frac{ 4 }{ \pi^2 }\sum_{n=1}^{\infty}\frac{1}{ n^2 },
    \end{equation}
    donc
    \begin{equation}
        \sum_{n=1}^{\infty}\frac{1}{ n^2 }=\frac{ \pi^2 }{ 6 }.
    \end{equation}
\end{example}

%---------------------------------------------------------------------------------------------------------------------------
\subsection{Polynôme trigonométrique}
%---------------------------------------------------------------------------------------------------------------------------

Un \defe{polynôme trigonométrique}{polynôme!trigonométrique} est une fonction de la forme
\begin{equation}
    P(t)=\sum_{n=-N}^Nc_n e^{int}.
\end{equation}

\begin{lemma}   \label{LemXGYaRlC}
    Si \( f\colon \eR\to \eC\) est une fonction \( 2\pi\)-périodique et si \( \epsilon>0\), alors il existe un polynôme trigonométrique \( P\) tel que \( \| f-P \|_{\infty}\leq \epsilon\).
\end{lemma}

\begin{proof}
    Nous allons utiliser le théorème de Stone-Weierstrass \ref{ThoWmAzSMF}. Soit le compact Hausdorff
    \begin{equation}
        S^1=\{ z\in \eC\tq | z |=1 \},
    \end{equation}
    et \( C(S^1,\eC)\) l'algèbre des fonctions continues de \( S^1\) vers \( \eC\). Il suffit de vérifier que les polynômes trigonométriques vérifient les hypothèse du théorème de Stone-Weierstrass. Un polynôme trigonométrique est un polynôme en \( z\) et \( \bar z\) défini sur \( S^1\).
    \begin{enumerate}
        \item
            Le polynôme constant est dans l'algèbre, ok.
        \item
            Pour la séparation des points, le polynôme trigonométrique \( x\mapsto  e^{ix}\).
        \item
            Si \( P\) est un polynôme en \( z\) et \( \bar z\), alors \( \bar P\) l'est encore.
    \end{enumerate}
    Donc si \( \epsilon>0\) et \( \tilde f\in C(S^1,\eC)\) sont donnés, il existe un polynôme trigonométrique \( P\) tel que
    \begin{equation}
        \sum_t| \tilde f( e^{it})-P(t) |<\epsilon.
    \end{equation}
    Soit \( f\colon \eR\to \eC\) une fonctions \( 2\pi\)-périodique. Nous considérons \( \tilde f\in C(S^1,\eC)\) donnée par \( \tilde f( e^{it})=f(t)\). Alors \( \sup_t| f(t)-P(t) |\leq \epsilon\).
\end{proof}

%---------------------------------------------------------------------------------------------------------------------------
\subsection{Suite équirépartie, critère de Weyl}
%---------------------------------------------------------------------------------------------------------------------------

\begin{definition}
    Soit \( u\) une suite dans \( \mathopen[ 0 , 1 \mathclose]\). Pour \( 0\leq a\leq b\leq 1\) nous posons
    \begin{equation}
        X_n(a,b)=\Card\big\{  k\in\{ 1,\ldots, n \}\tq u_k\in\mathopen[ a , b \mathclose] \big\}.
    \end{equation}
    Nous disons que la suite \( u\) est \defe{équirépartie}{suite!équirépartie} si pour tout \( 0\leq a<b<1\), on a
    \begin{equation}
        \lim_{n\to \infty} \frac{ X_n(a,b) }{ b }=0.
    \end{equation}
\end{definition}

\begin{proposition}[Critère de Weyl\cite{ytMOpe,KXjFWKA}]  \label{PropDMvPDc}
    Soit \( (x_n)\) une suite dans \( \mathopen[ 0 , 1 [\). On a équivalence entre les deux points suivants.
    \begin{enumerate}
        \item   \label{ItemKWcZTHqi}
            La suite \( (x_n)\) est équirépartie.
        \item\label{ItemKWcZTHqii}
            Pour toute fonction continue à valeurs réelles sur \( \mathopen[ 0 , 1 \mathclose]\),
            \begin{equation}    \label{EqBSqdjpn}
                \lim_{n\to \infty} \frac{1}{ n }\sum_{k=1}^nf(x_k)=\int_0^1f(x)dx.
            \end{equation}
        \item\label{ItemKWcZTHqiii}
            Pour tout \( p\in\eN^*\) nous avons
            \begin{equation}
                \lim_{n\to \infty} \frac{1}{ n }\sum_{k=1}^n e^{2i\pi pu_k}=0.
            \end{equation}
    \end{enumerate}
\end{proposition}
\index{convergence!suite numérique}
\index{intégrale!calcul}
\index{densité!dans un espace de fonction!critère de Weyl}

\begin{proof}
    On pose 
    \begin{equation}
        S_n(f)=\frac{1}{ n }\sum_{k=1}^nf(x_k).
    \end{equation}


    \begin{subproof}
    \item[Une espèce de lemme]

        Supposons connaitre un ensemble de fonctions \( A\) dense dans \( C^0(\mathopen[ 0 , 1 \mathclose])\) pour toutes les fonctions duquel nous avons la limite \eqref{EqBSqdjpn}. Alors la limite a lieu pour toute fonction de \( C^0(\mathopen[ 0 , 1 \mathclose])\). En effet, soit \( f\in C^0(\mathopen[ 0 , 1 \mathclose])\) et \( g\in A\) tel que \( \| f-g \|_{\infty}<\epsilon\). Alors
        \begin{subequations}
            \begin{align}
                \left\|   \frac{1}{ n }\sum_{k=1}^nf(x_k)-\int_0^1f(t)dt  \right\|&\leq \left\| \frac{1}{ n }\sum_{k=1}^n\big( f(x_k)-g(x_k)\big) \right\|\\
                &\quad+ \left\| \frac{1}{ n }\sum_{k=1}^n  g(x_k)-\int_0^1g(t)dt   \right\|\\
                &\quad+ \left\| \int_0^1g(t)dt-\int_0^1f(t)dt \right\|.
            \end{align}
        \end{subequations}
        Le premier terme se majore par \( \epsilon\). Le troisième est la même majoration : \( \int_0^1\big(  f(t)-g(t)\big)dt\leq \| f-g \|_{\infty}=\epsilon\). Par hypothèse sur l'espace \( A\), le second terme se majore par \( \epsilon\) lorsque \( n\) est grand.
        

    \item[\ref{ItemKWcZTHqi}\( \Rightarrow\)\ref{ItemKWcZTHqii}]
    Nous supposons que la suite est équirépartie et nous commençons par montrer le résultat pour les fonctions en escalier. Soit donc la fonction en escalier \( \eta(x)=c_j\) sur \( a_{j-1}< x<a_j\). Sur le point \( a_j\) lui-même, la fonction \( \eta\) vaut soit \( c_j\) soit \( c_{j+1}\). Nous avons
    \begin{equation}    \label{EqohMuel}
        \frac{1}{ n }\sum_{k=1}^n\eta(x_k)=\frac{1}{ n }\left[  \sum_{j=1}^mc_jX_n(a_j,a_{j+1})-\sum_{j=1}^mc_jX_n(a_j,a_j)+\sum_{j=1}^m\eta(a_j)X_n(a_j,a_j) \right].
    \end{equation}
    À la limite \( n\to\infty\), les deux derniers termes tombent\footnote{J'en profite pour mentionner que mon équation \eqref{EqohMuel} n'est pas la même que celle de \cite{ytMOpe} dans laquelle il me semble voir une faute; quoi qu'il en soit, les termes litigieux tombent.} et il reste
    %TODO : savoir si c'est moi ou l'autre qui a raison.
    \begin{equation}
        \lim_{n\to \infty} \frac{1}{ n }\sum_{k=1}^n\eta(x_k)=\sum_{j=1}^mc_j(a_{j-1}-a_j).
    \end{equation}
    Or par construction, pour une fonction en escalier,
    \begin{equation}
        \sum_{j=1}^mc_j(a_{j-1}-a_j)=\int_0^1\eta.
    \end{equation}
    
    Étant donné que les fonctions en escalier sont denses dans les fonctions continues, l'espèce de lemme plus haut conclut.
    
    \item[\ref{ItemKWcZTHqii}\( \Rightarrow\)\ref{ItemKWcZTHqi}]
    Nous prouvons maintenant le sens inverse. C'est à dire que pour toute fonction continue sur \( \mathopen[ 0 , 1 \mathclose]\), nous avons
    \begin{equation}
        \int_0^1f(x)dx=\lim_{n\to \infty} \frac{1}{ n }\sum_{k=1}^nf(x_k).
    \end{equation}
    Nous devons en déduire que \( (x_n)\) est équirépartie. Pour ce faire, soit \( x\in \mathopen[ 0 , 1 [\) et \( \epsilon>0\) tel que \( x+\epsilon<1\). Nous considérons \( \varphi=\mtu_{\mathopen[ x , 1 [}\) et
    \begin{equation}
        \varphi_{\epsilon(t)}=\begin{cases}
            0    &   \text{si \( t\in\mathopen[ 0 , x [\)}\\
            \frac{ t-x }{ \epsilon }    &   \text{si \( t\in \mathopen[ x , x+\epsilon [\)}\\
            1    &    \text{si \( t\geq x+\epsilon\)}.
        \end{cases}
    \end{equation}
    Cela est une fonction continue, donc
    \begin{equation}
        \lim_{n\to \infty} S_n\big( \varphi_{\epsilon}(t) \big)=\int_0^1\varphi_{\epsilon}(t)dt=\int_{x}^{x+\epsilon}\frac{ t-x }{ \epsilon }dt+\int_{x+\epsilon}^11dt=1-x-\frac{ \epsilon }{2}.
    \end{equation}
    Mais \( \varphi_{\epsilon}\leq \varphi\), donc \( S_n(\varphi_{\epsilon})\leq S_n(\varphi)\) et donc
    \begin{equation}
        \liminf_{n\to \infty}S_n(\varphi)\geq 1-x.
    \end{equation}
    Notons que nous ne savons pas si la \emph{vraie} limite de gauche existe; c'est pourquoi nous prenons la limite inférieure, qui existe toujours.

    Nous définissons aussi
    \begin{equation}
        \psi_{\epsilon}(t)=\begin{cases}
            0    &   \text{si \( t\in \mathopen[ 0 , x-\epsilon [\)}\\
            \frac{ t-x+\epsilon }{ \epsilon }    &   \text{si \( t\in\mathopen[ x-\epsilon , x [\)}\\
            1    &    \text{si \( t>x\)}.
        \end{cases}
    \end{equation}
    C'est encore une fonction continue et nous trouvons\footnote{Je recommande chaudement de dessiner les fonctions \( \varphi_{\epsilon}\) et \( \psi_{\epsilon}\) pour avoir une idée de la situation.}
    \begin{equation}
        \int_0^1\psi_{\epsilon}(t)dt=1-x+\frac{ \epsilon }{2}.
    \end{equation}
    Vu que \( \psi_{\epsilon}\geq\varphi\), nous avons \( S_n(\psi_{\epsilon})\geq S_n(\varphi)\) et donc
    \begin{equation}
        \limsup_{n}S_n(\varphi)\leq 1-x.
    \end{equation}
    Nous avons déjà obtenu que
    \begin{equation}
        1-x\leq\liminf S_n(\varphi)\leq \limsup S_n(\varphi)\leq 1-x,
    \end{equation}
    donc la limite existe et vaut
    \begin{equation}
        \lim_{n\to \infty} S_n(\varphi)=1-x.
    \end{equation}
    Cela est pour la fonction caractéristique \( \varphi=\mtu_{\mathopen[ x , 1 [}\). Si nous prenons une fonction caractéristique \( \mtu_{\mathopen[ a , b \mathclose]}\), nous avons la même chose parce que \( \mtu_{\mathopen[ a , b [}\) est une combinaisons linéaire de fonctions du type \( \mtu_{\mathopen[ x , 1 [}\).

    Nous avons donc
    \begin{equation}
        \lim_{n\to \infty} S_n\big( \mtu_{\mathopen[ a , b \mathclose]} \big)=b-a,
    \end{equation}
    alors que le membre de gauche n'est autre que
    \begin{equation}
        S_n\big( \mtu_{\mathopen[ a , b \mathclose]} \big)=\frac{1}{ n }\sum_{k=1}^n\mtu_{\mathopen[ a , b \mathclose]}(x_k)=\frac{1}{ n }N(n,a,b).
    \end{equation}
    \item[\ref{ItemKWcZTHqii}\( \Rightarrow\)\ref{ItemKWcZTHqiii}]
        Vu que \(  e^{2i\pi pu_k}=\cos(2\pi pu_k)+\sin(2\pi iu_k)\) est une fonction périodique, c'est immédiat.
    \item[\ref{ItemKWcZTHqiii}\( \Rightarrow\)\ref{ItemKWcZTHqii}]
        Par linéarité, le point \ref{ItemKWcZTHqii} montre que si \( f\) est un polynôme trigonométrique, alors
        \begin{equation}
            \lim_{n\to \infty} \frac{1}{ n }\sum_{k=1}^nf(u_k)=\int_0^1f(t)dt.
        \end{equation}
        Il nous reste à prouver que les polynômes trigonométriques sont denses dans les fonction continues sur \( \mathopen[ 0 , 1 \mathclose]\). Soit une fonction continue sur \( \mathopen[ 0 , 1 \mathclose]\) avec \( f(0)=f(1)\). Alors le théorème de Stone-Weierstrass dans sa version trigonométrique (lemme \ref{LemXGYaRlC}) nous donne la densité.

        Si \( f(1)\neq f(0)\) c'est pas très grave : on peut trouver une fonction \( g\) vérifiant \( g(0)=g(1) \) et \( \| f-g \|_{\infty}\leq \epsilon\). Ensuite un polynôme trigonométrique approxime très bien \( g\).
        .
    \end{subproof}
\end{proof}

%---------------------------------------------------------------------------------------------------------------------------
\subsection{À propos des coefficients}
%---------------------------------------------------------------------------------------------------------------------------

Nous considérons l'application
\begin{equation}
    \begin{aligned}
        c\colon \big( L^1_{2\pi},\| . \|_1 \big)&\to \big( C_0,\| .\|_{\infty} \big) \\
        f&\mapsto (c_n(f))_{n\in \eZ} 
    \end{aligned}
\end{equation}
qui à une fonction \( 2\pi\)-périodique fait correspondre la suite (bornée) de ses coefficients de Fourier. Nous rappelons la définition
\begin{equation}
    c_n(f)=\int_0^{2\pi}f(t) e^{-int}.
\end{equation}
Nous allons montrer que cette application est linéaire, continue, injective et non surjective. Pour la continuité, par la linéarité il suffit de la montrer en \( 0\). Nous devons donc montrer que si nous avons une suite de fonctions \( f_k\) qui tend vers \( 0\) au sens \( L^1\), alors \( c(f_k)\to 0\) au sens de la norme \( \| . \|_{\infty}\) sur l'ensemble des suites.

Si nous posons \( r_k=\int_0^{2\pi}| f_k(t) |dt\), alors \( r_k=\| f_k \|_1\) et nous avons \( r_k\to 0\). Mais par définition
\begin{equation}
    | c_n(f_k) |\leq r_k,
\end{equation}
et donc \( \| c(f_k) \|_{\infty}\leq r_k\). L'application \( c\) est donc continue. L'injectivité est donnée par le corollaire \ref{CordgtXlC}. 

Si nous supposons que l'application \( c\) est continue, alors le théorème d'isomorphisme de Banach (\ref{ThofQShsw}) nous dit que cela devrait être un homéomorphisme, c'est à dire que \( c^{-1}\) serait également continue. Nous allons montrer qu'il n'en est rien.

Nous considérons la suite de suite
\begin{equation}    \label{EqdMtbOB}
    (c_n)_k=\begin{cases}
        1    &   \text{si \( k<n\)}\\
        0    &    \text{sinon}.
    \end{cases}
\end{equation}
Ici \( (c_n)_k\) est le terme numéro \( k\) de la suite \( n\). Par injectivité de l'application qui à une fonction fait correspondre la suite de ses coefficients de Fourier, la seule fonction qui possède ces coefficients est
\begin{equation}
    f_n(t)=\sum_{k\in \eN}c_{n,k} e^{ikt}.
\end{equation}
Étant donné que \( \| f_n \|_1=n\), la suite \( (\| f_n \|_1)\) n'est pas bornée alors que a suite de suites \eqref{EqdMtbOB} est bornée dans l'ensemble des suites parce que \( \| c_n \|_{\infty}=1\).


%+++++++++++++++++++++++++++++++++++++++++++++++++++++++++++++++++++++++++++++++++++++++++++++++++++++++++++++++++++++++++++
\section{Transformée de Fourier}
%+++++++++++++++++++++++++++++++++++++++++++++++++++++++++++++++++++++++++++++++++++++++++++++++++++++++++++++++++++++++++++

Ici nous utilisons la convention de la transformée de Fourier de \wikipedia{fr}{Transformée_de_Fourier}{wikipedia}, c'est à dire
\begin{subequations}
    \begin{align}
        \hat f(\xi)&=\int_{\eR} e^{-i\xi x}f(x)dx\\
        f(x)&=2\pi\int_{\eR} e^{i\xi x}\hat f(\xi)d\xi.
    \end{align}
\end{subequations}

L'\defe{espace de Schwartz}{Schwartz!espace}\index{espace!de Schwartz} \( \swS(\eR^n,\eC)\)\nomenclature[Y]{\( \swS(\eR^n,\eC)\)}{fonctions Schwartz} est l'ensemble des fonctions dont toutes les dérivées décroissent plus vite que l'inverse de tout polynôme, c'est à dire
\begin{equation}
    \swS(\eR^n,\eC)=\{ f\in C^{\infty}(\eR^n,\eC)\tq \forall \alpha,\beta\in \eN^n,\sup_{x\in \eR^n}\big| (x)^{\alpha}D^{\beta}f(x) \big|<\infty \}
\end{equation}
où nous utilisons les notations \( x^{\alpha}=(x_1)^{\alpha_1}\ldots (x_n)^{\alpha_n}\) et \( D^{\beta}=\frac{ \partial^n  }{ \partial \beta_1\ldots\partial \beta_n }\).

\begin{proposition}[\cite{MesIntProbb}]
    La transformée de Fourier est une bijection de \( \swS(\eR^n,\eC)\).    
\end{proposition}

\begin{theorem}[\cite{MesIntProbb}]      \label{ThoRWEoqY}
    Soit \( \mu\) une mesure sur les boréliens de \( \eR^n\) finie sur les compacts. Alors \( C^{\infty}_c(\eR^n,\eR)\) est dense dans \( L^1(\eR^n,\Borelien(\eR^n),\mu)\).
\end{theorem}

\begin{proposition}     \label{PropfqvLOl}
    La transformée de Fourier est un morphisme vis-à-vis de la convolution\index{produit!convolution!et Fourier} sur \( L^1(\eR^n)\) :
    \begin{equation}
        \widehat{f*g}=\hat f\hat g.
    \end{equation}
\end{proposition}

\begin{proof}
    Nous devons étudier l'intégrale
    \begin{equation}
        \widehat{f*g}(\xi)=\int_{\eR}\left[ \int_{\eR} f(y)g(t-y)\right] e^{-it\xi} dt.
    \end{equation}
    Ici nous avons choisit des représentants \( f\) et \( g\) dans les classes de \( L^1\). Montrons que \( f\) est borélienne. D'abord \( f(x)=f_+(x)-f_-(x)\) où \( f_+\) et \( f_-\) sont des fonctions positives. Afin d'alléger les notations nous supposons un instant que \( f\) est positive et nous posons
    \begin{equation}
        f_n(x)=\sum_{k=1}^{2^n} \frac{ k }{ n }\mtu_{f(x)\in\mathopen[ \frac{ k }{ n } , \frac{ k+1 }{ n } [}.
    \end{equation}
    Le fait que \( f\) soit dans \( L^1\) implique que chacune des fonctions \( f_n\) est borélienne et donc que \( f\) l'est aussi en tant que limite ponctuelle de fonctions boréliennes\footnote{Le fait que \( f\) soit borélienne est une conséquence du théorème \ref{ThoRWEoqY}.}.
    
    Nous allons appliquer le théorème de Fubini \ref{CorTKZKwP} à la fonction
    \begin{equation}
        \phi(x,y)=f(x)g(y) e^{-i\xi(x+y)}
    \end{equation}
    qui est borélienne en tant que produit et composé de fonctions boréliennes. Nous avons
    \begin{subequations}
        \begin{align}
            \int_{\eR}\left( \int_{\eR}| f(x) e^{-i\xi x} | |g(y) e^{-i\xi y} |dy \right)dx&=\int_{\eR}\left( | f(x) |\int_{\eR}| g(y) |dy \right)dx\\
            &=\int_{\eR}| f(x) |\| g \|_1\\
            &=\| f \|_1\| g \|_1<\infty.
        \end{align}
    \end{subequations}
    Le théorème est donc applicable. D'abord nous avons :
    \begin{subequations}
        \begin{align}
            \hat f(\xi)\hat g(\xi)&=\left(\int_{\eR}f(x) e^{-i\xi x}dx\right)\left(\int_{\eR}g(y) e^{-i\xi y}dy\right)\\
            &=\int_{\eR}\left( \int_{\eR}f(x)g(y) e^{-i\xi(x+y)}dy \right)dx\\
            &=\int_{\eR}\left( \int_{\eR}f(x)g(t-x) e^{-i\xi t} \right)dx.
        \end{align}
    \end{subequations}
    Jusqu'ici nous n'avons pas utilisé Fubini. Nous avons seulement introduit le nombre \( \int_{\eR}g(y) e^{-i\xi y}dy\) dans l'intégrale par rapport à \( x\) et effectué le changement de variables \( y\mapsto t=x+y\). Maintenant nous appliquons le théorème de Fubini pour inverser l'ordre des intégrales :
    \begin{subequations}
        \begin{align}
            \hat f(\xi)\hat g(\xi)&=\int_{\eR}\left( \int_{\eR}f(x)g(t-x) e^{-it\xi}dx \right)dy\\
            &=\int_{\eR} e^{-it\xi}\left( \int_{\eR}f(x)g(t-x)dx \right)dt\\
            &=\int_{\eR} e^{-it\xi}(f*g)(t)dt\\
            &=\widehat{f*g}(\xi).
        \end{align}
    \end{subequations}
\end{proof}

\begin{proposition}       \label{PropJvNfj}
    Soit une fonction \( f\in L^1(\eR^d)\). Alors sa transformée de Fourier est continue\index{transformée!Fourier!continuité}.
\end{proposition}

\begin{proof}
    Nous considérons une fonction \( f\) définie sur \( \eR^d\) et à valeurs dans \( \eR\) ou \( \eC\). Sa transformée de Fourier est donnée par
    \begin{equation}
        \hat f(\xi)=\int_{\eR^d} e^{-i\xi x}f(x)dx.
    \end{equation}
    Pour montrer que cette fonction \( \hat f\) est continue en \( \xi_0\) nous considérons une suite \( (\xi_n)\to \xi_0\) et nous voulons montrer que \( \hat f(\xi_n)\to\hat f(\xi_0)\). Pour cela nous considérons les fonctions
\begin{equation}
    g_n(x)= e^{-i\xi_nx}f(x)
\end{equation}
qui convergent simplement vers \( g(x)= e^{-i\xi x}f(x)\). Étant donné que
\begin{equation}
    | g_n(x) |<| f(x) |,
\end{equation}
le théorème de la convergence dominée donne alors
\begin{equation}
    \lim_{n\to \infty} \int g_n(x)=\int\lim_{n\to \infty } g_n(x),
\end{equation}
c'est à dire \( \lim_{n\to \infty} \hat f(\xi_n)=\hat f(\xi)\). La fonction \( \hat f\) est donc continue.
\end{proof}

\begin{lemma}
    Soit \( f\in L^1(\eR)\). Alors \( \| \hat f \|_{\infty}\leq \| f \|_1\).
\end{lemma}

\begin{proof}
    Cela est un simple calcul : étant donné que
    \begin{equation}
        \hat f(\xi)=\int_{\eR}f(x) e^{-ix\xi}dx,
    \end{equation}
    nous avons, pour tout \( \xi\),
    \begin{equation}
        | \hat f(\xi) |\leq\int_{\eR}| f(x) |dx,
    \end{equation}
    ce qui signifie exactement \( \| \hat f \|_{\infty}\leq \| f \|_1\).
\end{proof}

\begin{lemma}[Lemme de Riemann-Lebesgue\cite{MaureyHilbertFourier}]     \label{LesmRLaxXkQV}
    Si \( f\) est une fonction \( L^1(\eR)\) alors \( \lim_{\xi\to\pm\infty} \hat f(\xi)=0\).
\end{lemma}

\begin{proof}
    Nous commençons par prouver le résultat dans le cas d'une fonction \( g\) en escalier, et plus précisément par une fonction caractéristique d'un compact \( K=\mathopen[ a , b \mathclose]\). Au niveau de la transformée de Fourier nous avons
    \begin{equation}
        \hat\mtu_{K}(\xi)=\int_a^b e^{-i\xi x}dx=-\frac{1}{ i\xi }( e^{-ib\xi}- e^{-ia\xi}).
    \end{equation}
    Par conséquent
    \begin{equation}
        | \hat\mtu_K(\xi) |\leq \frac{ 2 }{ | \xi | }.
    \end{equation}
    Plus généralement si \( g=\sum_{i=1}^Nc_i\mtu_{K_i}\), alors
    \begin{equation}
        | \hat g(\xi) |\leq \frac{ 2 }{ | \xi | }\sum_{i=1}^N| c_i |,
    \end{equation}
    et donc nous avons effectivement \( \lim_{\xi\to\pm\infty}| \hat g(\xi) |=0\).

    Nous passons maintenant au cas général \( f\in L^1(\eR)\). Étant donné que les fonctions \( L^1\) en escalier sont denses dans \( L^1\), nous considérons une fonction \( g\in L^1(\eR)\) en escalier telle que \( \| f-g \|_1<\epsilon\). Nous avons donc
    \begin{equation}
        \| \hat f-\hat g \|_{\infty}\leq \| f-g \|_1<\epsilon.
    \end{equation}
    Donc
    \begin{equation}
        \| \hat f(\xi) \|\leq \| \hat f(\xi)-\hat g(\xi) \|_| \hat g(\xi) |.
    \end{equation}
    Le premier terme est plus petit que \( \epsilon\). Il nous reste à voir que 
    \begin{equation}
        \lim_{\xi\to \infty} | \hat g(\xi) |=0,
    \end{equation}
    mais cela est le résultat de la première partie de la preuve.    
\end{proof}

\begin{corollary}
    La transformée de Fourier d'une fonction \( L^1(\eR)\) est bornée.
\end{corollary}

\begin{proof}
    Par le corollaire \ref{PropJvNfj}, la transformée de Fourier d'une fonction \( L^1\) est continue. Le lemme de Riemann-Lebesgue \ref{LesmRLaxXkQV} impliquant qu'elle tend vers zéro en \( \pm\infty\), elle doit être bornée.    
\end{proof}

%---------------------------------------------------------------------------------------------------------------------------
\subsection{Formule sommatoire de Poisson}
%---------------------------------------------------------------------------------------------------------------------------

\begin{proposition}[Formule sommatoire de Poisson]\index{Poisson!formule sommatoire}\index{formule!sommatoire de Poisson}   \label{ProprPbkoQ}
    Soit \( f\colon \eR\to \eC\) une fonction continue et \( L^1(\eR)\). Nous supposons que
    \begin{enumerate}
        \item
    il existe \( M>0\) et \( \alpha>1\) tels que
    \begin{equation}
        | f(x) |\leq\frac{ M }{ (1+| x |)^{\alpha} },
    \end{equation}
        \item
            \( \sum_{n=-\infty}^{\infty}| \hat f(2\pi n) |<\infty\).

    \end{enumerate}
    Alors nous avons
    \begin{equation}
        \sum_{n=-\infty}^{\infty}f(n)=\sum_{n=-\infty}^{\infty}\hat f(2\pi n).
    \end{equation}
\end{proposition}
\index{convergence!rapidité}
\index{série!fonctions}
\index{transformation!Fourier}
\index{Fourier}
\index{série!entière}
\index{série!Fourier}

%AFAIRE : Exprimer ce théorème comme truc sur les distributions et sur les machins tempérées, espace de Schwartz.

\begin{proof}
    \begin{subproof}
        \item[Convergence normale]

    
    Nous commençons par montrer qu'il y a convergence normale sur tout compact séparément des séries sur les \( n\geq 0\) et sur les \( n<0\).
    
    Soit \( K\) un compact de \( \eR\) contenu dans \( \mathopen[ -A , A \mathclose]\) et \( n\in \eZ\) tel que \( | n |\geq 2A\). Pour \( x\in K\) nous avons
    \begin{equation}
        | x+n |\geq | n |-| x |\geq | n |-A\geq \frac{ | n | }{ 2 }.
    \end{equation}
    Du coup nous avons un \( \alpha>1\) tel que
    \begin{equation}
        | f(x+n) |\leq \frac{ M }{ \big( 1+| x+n | \big)^{\alpha} }\leq \frac{ M }{ \left( 1+\frac{ | n | }{2} \right)^{\alpha} }.
    \end{equation}
    Lorsque \( n\) est grand, cela a le comportement de \( M/| n |^{\alpha}\) et donc la série
    \begin{equation}
        \sum_{n=0}^{\infty}f(x+n)
    \end{equation}
    est une série convergent normalement. Les deux séries (usuelles) 
    \begin{subequations}
        \begin{align}
            a_-=\sum_{n\leq 0}f(x+n)\\
            a_-=\sum_{n> 0}f(x+n)
        \end{align}
    \end{subequations}
    convergent normalement.
    
\item[Convergence commutative]
    Au sens de la définition \ref{DefIkoheE} nous avons
    \begin{equation}
        \sum_{n\in \eZ}f(x+n)=a_++a_-.
    \end{equation}
    En effet si nous prenons \( J'_0\subset\eN\) fini tel que \( |\sum_{\eN\setminus J_0}f(x+n)-a_+|\leq \epsilon\) et \( J'_1\in -\eN\) tel que \( |\sum_{n\in -\eN\setminus J'_1}f(x+n)|-a_-<\epsilon\), et si nous posons \( J_0=J'_0\cup J'_1\) alors si \( K\) est un ensemble fini de \( \eZ\) contenant \( J_0\) nous avons
    \begin{equation}
        | \sum_{n\in K}f(n+x)-(a_++a_-) |\leq | \sum_{n\in \e^+}f(n+x)-a_+ |+| \sum_{n\in K^-}f(n+x)-a_- |\leq 2\epsilon
    \end{equation}
    où $K^+$ sont les éléments positifs de \(K\) et \( K^-\) sont les \emph{strictement} négatifs. Maintenant que la famille \( \{ f(n+x) \}_{n\in \eZ}\) est une famille sommable, nous savons qu'elle est commutativement sommable et que la proposition \ref{PropoWHdjw} nous permet de sommer dans l'ordre que l'on veut. Nous pouvons donc écrire sans ambigüité l'expression \( \sum_{n\in \eZ}f(x+n)\) ou \( \sum_{n=-\infty}^{\infty}f(x+n)\).
    
    \item[re-convergence normale]

        Nous posons donc sans complexes la série
        \begin{equation}
            F(x)=\sum_{n\in \eZ}f(x+n)
        \end{equation}
        qui converge tant commutativement que normalement. Notons que nous pouvons maintenant dire que la série sur \( \eZ\) converge normalement; pas seulement les deux séries séparément.

    \item[Continuité, périodicité]
        Étant donné que chacune des fonctions \( f(x+n)\) est continue, la convergence normale nous assure que \( F\) est continue.

        De plus \( F\) est périodique parce que
        \begin{equation}
            F(x+1)=\sum_{n=-\infty}^{\infty}f(x+1+n)=\sum_{p=-\infty}^{\infty}f(x+p)
        \end{equation}
        où nous avons posé \( p=1+n\).
        
    \item[Coefficients de Fourier]

        En vertu de la définition \eqref{EqhIPoPH} et de la périodicité de \( F\),
        \begin{subequations}
            \begin{align}
                c_n(F)&=\int_{-1/2}^{1/2}F(t) e^{-2\pi int}dt\\
                &=\int_0^1F(t) e^{-2\pi int}dt\\
                &=\int_0^1\sum_{n\in \eZ}f(t+n) e^{-2 i\pi nt}dt\\
                &=\sum_{n\in \eZ}\int_n^{n+1}f(u) e^{-2\pi i (u-n)t}du\\
                &=\int_{-\infty}^{\infty}f(u) e^{-2\pi inu}du\\
                &=\hat f(2\pi n).
            \end{align}
        \end{subequations}
        où nous avons effectué le changement de variables \( u=t+n\), et permuté l'intégrale et la somme en vertu du fait que la somme converge normalement.

    \item[Conclusion]

        Étant donné l'hypothèse \( \sum_{n\in \eZ}| \hat f(n) |<\infty\) la proposition \ref{PropSgvPab} nous dit que
        \begin{equation}
            F(x)=\sum_{n\in \eZ}c_n(F) e^{2\pi inx},
        \end{equation}
        c'est à dire que
        \begin{equation}
            \sum_{n-\infty}^{\infty}f(x+n)=\sum_{n=-\infty}^{\infty}\hat f(2\pi n) e^{2\pi i nx}.
        \end{equation}
        En écrivant cette égalité en \( x=0\) nous trouvons le résultat :
        \begin{equation}
            \sum_{n\in \eZ}f(n)=\sum_{n\in \eZ}\hat f(2\pi n).
        \end{equation}
    \end{subproof}
\end{proof}

\begin{example}\label{ExDLjesf}
\index{convergence!rapidité}
    La formule sommatoire de Poisson peut être utilisée pour calculer des sommes dans l'espace de Fourier plutôt que dans l'espace direct. Nous allons montrer dans cet exemple l'égalité
    \begin{equation}
        \sum_{n=-\infty}^{\infty} e^{-\alpha n^2}=\sum_{n=-\infty}^{\infty}\sqrt{\frac{ \pi }{ \alpha }} e^{-\pi^2 n^2/\alpha}.
    \end{equation}
    Si \( \alpha\) est grand, alors la somme de gauche est plus rapide, tandis que si \( \alpha\) est petit, c'est le contraire.

    Nous appliquons la formule sommatoire de Poisson à la fonction
    \begin{equation}
        f(x)= e^{-\alpha x^2}.
    \end{equation}
    Nous avons
    \begin{subequations}        \label{EqCDeLht}
        \begin{align}
            \hat f(x)&=\int_{\eR} e^{-\alpha t^2-ixt}dt\\
            &= e^{-x^2/4\alpha}\int_{\eR}e^{ -(\sqrt{\alpha}t+\frac{ ix }{ 2\sqrt{\alpha} })^2 }\\
            &= e^{-x^2/4\alpha}\frac{1}{ \sqrt{\alpha} }\int_{\eR+\frac{ ix }{ 2\sqrt{\alpha} }} e^{-u^2}du.
        \end{align}
    \end{subequations}
    Pour traiter cette intégrale nous utilisons la proposition \ref{PrpopwQSbJg} en considérant le chemin rectangulaire fermé qui joint les points \( -R\), \( R\), \( R+ai\), \( -R+ai\) et \( f(z)= e^{-z^2}\). Calculons l'intégrale sur les deux côtés verticaux. Nous posons
    \begin{equation}
        \gamma_R(t)=R+tia
    \end{equation}
    avec \( t\colon 0\to 1\). Nous avons
    \begin{subequations}
        \begin{align}
            \int_{\gamma_R}f&=\int_0^1f\big( \gamma_R(t) \big)\| \gamma_R'(t) \|dt\\
            &=a e^{-R^2}\int_0^1 e^{-2tRia+at^2}dt,
        \end{align}
    \end{subequations}
    donc en module nous avons
    \begin{equation}
        | \int_{\gamma_R}f |\leq a e^{-R^2}\int_0^1 e^{at^2}dt\leq M e^{-R^2},
    \end{equation}
    où \( M\) est une constante ne dépendant pas de \( R\). Lorsque \( R\to \infty\), la contribution des chemins verticaux s'annule et nous trouvons que
    \begin{equation}    \label{EqjrNxLr}
        \int_{\eR+ai} e^{-u^2}du=\int_{\eR} e^{-u^2}du,
    \end{equation}
    que nous pouvons utiliser pour continuer le calcul \eqref{EqCDeLht}. Nous avons
    \begin{equation}
        \hat f(x)= \frac{ e^{-x^2/4\alpha}}{\sqrt{\alpha}}\int_{R} e^{-u^2}du\\
            =\sqrt{\frac{ \pi }{ \alpha }} e^{-x^2/4\alpha}
    \end{equation}
    où nous avons utilisé la formule \eqref{EqFDvHTg}. Par conséquent ce qui rentre dans la formule sommatoire de Poisson est
    \begin{equation}
        \hat f(2\pi n)=\sqrt{\frac{ \pi }{ \alpha }} e^{-\pi^2 n^2/\alpha}.
    \end{equation}
\end{example}


%+++++++++++++++++++++++++++++++++++++++++++++++++++++++++++++++++++++++++++++++++++++++++++++++++++++++++++++++++++++++++++
\section{Espaces de Bergman}
%+++++++++++++++++++++++++++++++++++++++++++++++++++++++++++++++++++++++++++++++++++++++++++++++++++++++++++++++++++++++++++

Source : \cite{ytMOpe}.

Soit \( \Omega\) un borné dans \( \eC\) et \( D\) le disque unité ouvert de \( \eC\).

\begin{definition}
    L'\defe{espace de Bergman}{espace!de Bergman}\index{Bergman (espace)} sur \( \Omega\), noté \( A^2(\Omega)\)\nomenclature[Y]{\( A^2(\Omega)\)}{espace de Bergman} est l'espace des fonctions holomorphes sur \( \Omega\) qui sont en même temps dans \( L^2(\Omega)\).
\end{definition}
Nous mettons sur \( A^2(\Omega)\) le produit scalaire usuel hérité de \( L^2\) :
\begin{equation}
    \langle f, g\rangle =\int_{\Omega}f(z)\overline{ g(z) }dz.
\end{equation}

\begin{lemma}   \label{LemIZxKfB}
    Soit \( K\subset \Omega\) un compact et \( f\in A^2(\Omega)\). Alors
    \begin{equation}
        \max_{z\in K}| f(z) |\leq \frac{1}{ \sqrt{\pi} }\frac{1}{ d(K,\partial \Omega) }\| f \|_2.
    \end{equation}
\end{lemma}

\begin{proof}
    Soient \( a\in \Omega\) et \( r>0\) tels que \( B(a,r)\subset\Omega\). Nous considérons aussi \( \rho\leq r\). La formule de Cauchy \eqref{EqPzUABM} nous donne   
    \begin{equation}
        f(a)=\frac{1}{ 2\pi i }\int_{B(a,\rho)}\frac{ f(\xi) }{ \xi-a }f\xi=\frac{1}{ 2\pi }\int_0^{2\pi}f(a+\rho e^{i\theta})d\theta
    \end{equation}
    où nous avons utilisé le chemin \( \gamma(\theta)=a+\rho e^{i\theta}\), \( \gamma'(\theta)=i\rho e^{i\theta}\) et \( \rho=| \xi-a |\). Maintenant une astuce est d'écrire
    \begin{equation}
        \frac{ r^2 }{2}f(a)=\int_0^rf(a)\rho d\rho,
    \end{equation}
    et d'y substituer la valeur de \( f(a)\) que nous venons de calculer :
    \begin{subequations}
        \begin{align}
            \frac{ r^2 }{2}f(a)&=\int_0^r\frac{1}{ 2\pi }\int_0^{2\pi}f(a+\rho e^{i\theta})d\theta\rho d\rho\\
            &=\frac{1}{ 2\pi }\int_{B(a,r)}f(z)dz   &   \text{passage aux polaires}\\
            &=\frac{1}{ 2\pi }\langle 1, f\rangle_B   &   \text{produit scalaire sur \( B(a,r)\)}\\
            &\leq\frac{1}{ 2\pi }\sqrt{\langle 1, 1\rangle_B\langle f, f\rangle_B }\\
        \end{align}
    \end{subequations}
    Nous avons donc
    \begin{equation}
        r^2f(a)\leq \frac{1}{ \pi }\sqrt{\langle 1, 1\rangle_B\langle f, f\rangle_B},
    \end{equation}
    et donc
    \begin{equation}
        \pi r^2 f(a)\leq \sqrt{\pi r^2}\| f \|_2,
    \end{equation}
    parce que \( \langle f, f\rangle_B\leq \| f \|_2^2\). En effet le produit scalaire \( \| . \|_2\) est donné par une intégrale sur \( \Omega\) alors que \( B(a,r)\subset \Omega\) et que la fonction qu'on y intègre est positive (c'est \( | f(z) |^2\)). En simplifiant,
    \begin{equation}
        f(a)\leq \frac{1}{ \sqrt{\pi}r }\| f \|_2.
    \end{equation}
    Mais \( r\) a été choisit pour avoir \( B(a,r)\subset\Omega\), donc \( r\leq d(a,\partial \Omega)\) et
    \begin{equation}
        | f(a) |\leq \frac{1}{ d(a,\partial\Omega)\sqrt{\pi} }\| f \|_2.
    \end{equation}
    
    Maintenant si nous prenons \( a\in K\), nous avons encore la minoration \( d(a,\partial K)\leq d(a,\partial \Omega)\) et donc
    \begin{equation}
        | f(a) |\leq\frac{1}{ d(a,\partial K)\sqrt{\pi} }\| f \|_2.
    \end{equation}

\end{proof}

\begin{theorem}
    Soit \( \Omega\) un ouvert de \( \eC\).
    \begin{enumerate}
        \item
            L'espace \( A^2(\Omega)\) est un espace de Hilbert.
        \item
            Si \( D\) est la boule unité dans \( \eC\), une base hilbertienne de \( A^2(D)\) est donnée par les fonctions
            \begin{equation}
                e_n(z)=\sqrt{\frac{ n+1 }{ \pi }}z^n
            \end{equation}
            pour \( n\geq 0\).
    \end{enumerate}
\end{theorem}

\begin{proof}
    Nous commençons par montrer que \( A^2(\Omega)\) est complet. Pour cela nous considérons une suite de Cauchy \( (f_n)\) dans \( A^2(\Omega)\) et un compact \( K\subset \Omega\). Nous savons par le lemme \ref{LemIZxKfB} que
    \begin{equation}
        \max_{z\in K}\big| f_n(z)-f_m(z) \big|\leq \frac{1}{ \sqrt{\pi}d(K,\partial\Omega) }\| f_n-f_m \|_2.
    \end{equation}
    Donc \( f_n\) converge uniformément sur \( K\). Par le théorème de Weierstrass \ref{ThoArYtQO}, la fonction \( f\) est holomorphe. Il existe donc une fonction holomorphe \( f\) qui est limite uniforme sur tout compact de \( \Omega\) de la suite \( (f_n)\).

    Mais \( L^2(\Omega)\) étant complet, la suite \( (f_n)\) a une limite \( g\in L^2(\Omega)\). Ce que nous voudrions faire est prouver que \( f=g\). Notons que tel quel, ce n'est pas vrai parce que \( f\) est une vraie fonction alors que \( g\) est une classe. Ce que nous enseigne la proposition \ref{PropWoywYG} est qu'il existe une sous-suite (qu'on note \( (g_n)\)) qui converge vers \( g\) presque partout. Dans cette dernière phrase, \( g_n\) et \( g\) sont de vraies fonctions, des représentants des classes dans \( L^2\).

    Nous déduisons que \( f=g\) presque partout (ici \( f\) et \( g\) sont les fonctions) parce que la sous-suite converge uniformément vers \( f\) en même temps que presque partout vers \( g\). Donc \( f=g\) dans \( L^2(\Omega)\) (ici \( f\) et \( g\) sont les classes). Donc \( f\in L^2(\Omega)\) et l'espace \( A^2(\Omega)\) est de Hilbert.

    Il nous faut encore prouver que \( (e_n)_{n\geq 0}\) est une base orthonormale. En ce qui concerne les produits scalaires,
    \begin{subequations}
        \begin{align}
            \langle e_m, e_n\rangle &=\sqrt{\frac{ (m+1)(n+1) }{ \pi }}\int_Dz^n\overline{ z^m }dz\\
            &=\sqrt{\frac{ (m+1)(n+1) }{ \pi^2 }}\int_0^1\rho\,d\rho\int_0^{2\pi}d\theta \rho^{m+n} e^{i\theta(n-m)}\\
            &=\sqrt{\frac{ (m+1)(n+1) }{ \pi^2 }}\frac{1}{ m+n+2 }\underbrace{\int_{0}^{2\pi} e^{i\theta(n-m)}d\theta}_{2\pi \delta_{mn}}\\
            &=\sqrt{\frac{ (n+1)^2 }{ \pi^2 }}\frac{1}{ 2n+2 }2\pi \delta_{nm}\\
            &=\delta_{nm}.
        \end{align}
    \end{subequations}
    Donc les fonctions données sont bien orthonormales. Nous devons montrer qu'elles sont denses dans \( A^2(D)\). Soit \( f\in A^2(D)\) et \( c_n(f)=\langle f, e_n\rangle \); nous allons montrer que
    \begin{equation}
        \| f \|_2^2=\sum_{n=0}^{\infty}| \langle f, e_n\rangle  |^2,
    \end{equation}
    parce que le point \ref{ItemQGwoIx} du théorème \ref{ThoyAjoqP} nous indique que ce sera suffisant pour avoir une base hilbertienne.

    Étant donné que \( f\) est holomorphe sur \( D\), le théorème \ref{ThoUHztQe} nous développe \( f\) en série entière :
    \begin{equation}    \label{EqObkbPK}
        f(z)=\sum_{k=0}^{\infty}a_kz^k.
    \end{equation}
    En permutant la somme avec le produit scalaire,
    \begin{equation}
        c_n(f)=\int_Df(z)\bar e_n(z)=\sqrt{\frac{ n+1 }{ \pi }}\int_Df(z)\bar z^ndz.
    \end{equation}
    Afin de profiter de la convergence uniforme de la série \eqref{EqObkbPK} à l'intérieur de \( D\), nous allons exprimer l'intégrale sur \( D\) comme une intégrale sur \( | z |<r\) en faisant tendre \( r\) vers \( 1\) (par le bas). Pour ce faire nous considérons les fonctions
    \begin{equation}
        g_k(z)=\begin{cases}
            f(z)\bar z^n    &   \text{si \( | z |<1-1/k\)}\\
            0    &    \text{sinon.}
        \end{cases}
    \end{equation}
    Ces fonctions sont intégrables sur \( D\) et dominées par \( f(z)\bar z^n\) qui est intégrable sans dépendre de \( k\). Mais nous avons évidemment \( g_k(z)\to f(z)\bar z^n\). Le théorème de la convergence dominée permet alors de permuter l'intégrale et la limite \( k\to \infty\). Cela nous permet d'écrire
    \begin{equation}
        c_n(f)=\sqrt{\frac{ n+1 }{ \pi }}\lim_{r\to 1^-}\int_{| z |<r}\bar z^nf(z)dz=\sqrt{\frac{ n+1 }{ \pi }}\lim_{r\to 1^-}\int_{| z |<r}\sum_{k=0}^{\infty}a_kz^k\bar z^n.
    \end{equation}
    Par la convergence uniforme de la série entière \emph{à l'intérieur} du disque \( D\) nous pouvons permuter l'intégrale et la somme (proposition \ref{PropfeFQWr}) :
    \begin{equation}
        c_n(f)=\sqrt{\frac{ n+1 }{ \pi }}\lim_{r\to 1^-}\sum_{k=0}^{\infty}a_k\int_{| z |<r}z^k\bar z^ndz.
    \end{equation}
    L'intégrale proprement dite est vite calculée et vaut
    \begin{equation}
        \int_{| z |<1}\bar z^nz^kdz=\frac{ \pi r^{2n+2} }{ n+1 }\delta_{kn}.
    \end{equation}
    Nous pouvons donc continuer le calcul de \( c_n(f)\) en effectuant la somme sur \( k\) qui se réduit à changer \( k\) en \( n\) puis en effectuant la limite :
    \begin{equation}
        c_n(f)=\sqrt{\frac{ n+1 }{ \pi }}\lim_{r\to 1^-}\sum_ka_k\frac{ \pi r^{2n+2} }{ n+1 }\delta_{kn}=\sqrt{\frac{ \pi }{ n+1 }}a_n.
    \end{equation}
    
    Nous effectuons le même genre de calculs pour évaluer \( \| f \|^2_2\) :
    \begin{subequations}
        \begin{align}
            \| f \|_2^2&=\int_D| f(z) |^2dz\\
            &=\lim_{r\to 1^-}\int_{| z |<r}f(z)\sum_{k=0}^{\infty}\bar a_k\bar z_kdz\\
            &=\lim_{r\to 1^-}\sum_{k=0}^{\infty}\bar a_k\int_{| z |<r}f(z)\bar z^kdz&\text{permuter \( \sum\) et \( \int\)}\\
            &=\lim_{r\to 1^-}\sum_{k=0}^{\infty}\bar a_ka_k\frac{ \pi r^{2k+2} }{ k+1 }&\text{intégrale déjà faite}.
        \end{align}
    \end{subequations}
    Mais nous savons déjà que \( c_n(f)=\sqrt{\pi/(n+1)}\), donc ce qui est dans la somme est \( \pi\bar a_ka_k/(n+1)=| c_k(f) |^2\). Nous avons donc
    \begin{equation}
        \| f \|^2_2=\lim_{r\to 1^-}\sum_{k=0}^{\infty}| c_k(f) |^2 r^{2k+2}.
    \end{equation}
    La fonction (de \( r\)) constante \( | c_k(f) |^2\) domine \( | c_k(f)r^{2k+2} |\) tout en ayant une somme (sur \( k\)) qui converge; en effet la proposition \ref{PropHKqVHj} nous indique que \( \sum_j| c_k(f) |^2\leq \| f \|_2^2\). Le théorème de la convergence dominée nous permet d'inverser la limite et la somme pour obtenir le résultat attendu :
    \begin{equation}
        \| f \|_2^2=\sum_{k=0}^{\infty}| c_k(f) |^2.
    \end{equation}
\end{proof}

