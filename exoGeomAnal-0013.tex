\begin{exercice}\label{exoGeomAnal-0013}

	Nous avons donné une démonstration du théorème de Weierstrass \ref{ThoWeirstrassRn} (toute fonction sur un compact de $\eR^m$ est bornée et atteint ses bornes). Pouvons nous prouver le cas plus général où on remplace $\eR^m$ par un espace vectoriel normé en nous contentant de recopier mot à mot en remplaçant $\eR^m$ par $V$ ? Quelles sont les étapes qui demandent un surplus de justification ?

	Cet exercice est difficile. Effectuer lesdits «surplus de justification» l'est encore plus parce qu'il faut vérifier quels sont les lemmes et propositions antérieures qui tiennent en remplaçant $\eR^m$ par $V$ dans les énoncés et les démonstration. Sachez pour votre culture générale que quasiment tout ce que nous disons sur les fonctions continues de $\eR^m$ dans $\eR$ est valable à peu près mot à mot pour des fonctions continues d'un espace vectoriel normé quelconque dans $\eR$ (et même souvent vers un autre espace vectoriel normé quelconque).

\corrref{GeomAnal-0013}
\end{exercice}
