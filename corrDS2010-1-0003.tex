% This is part of Exercices de mathématique pour SVT
% Copyright (C) 2010
%   Laurent Claessens et Carlotta Donadello
% See the file fdl-1.3.txt for copying conditions.

\begin{corrige}{DS2010-1-0003}


	Pour les limites des fonctions rationnelles (c'est à dire, des fractions de polynômes), il y a deux techniques à ne pas confondre. La première est la factorisation et simplification; la seconde est la mise en évidence du plus haut degré au numérateur et au dénominateur.

	La mise en évidence se fait lorsqu'on a des limites en $\pm\infty$. La factorisation se fait pour les limites en des nombres.
	\begin{enumerate}
		\item
			Ici il faut factoriser le dénominateur et simplifier.
			\begin{equation}
				\lim_{x\to 4} \frac{ (x-4)^2 }{ x^2-16 }=\lim_{x\to 4} \frac{ (x-4)^2 }{ (x-4)(x+4) }=\lim_{x\to 4} \frac{ (x-4) }{ (x+4) }=\frac{ 0 }{ 8 }=0.
			\end{equation}
		\item
			Ici, il faut mettre en évidence la plus haute puissance de $x$ au numérateur et au dénominateur.
			\begin{equation}
				\begin{aligned}[]
					\lim_{x\to -\infty} \frac{ x^{36}-2x^{25}-12 }{ 2x^{27}-3x^{12}+1200 }&=\lim_{x\to -\infty} \frac{ x^{36}\left( 1-\frac{ 2x^{25} }{ x^{36} }-\frac{ 12 }{ x^{36} } \right) }{ x^{27}\left( 2-\frac{ 3x^{25} }{ x^{27} }-\frac{ 12 }{ x^{27} } \right) }\\
					&=\frac{ x^9\left( 1-\frac{ 2 }{ x^{11} }-\frac{ 12 }{ x^{36} } \right) }{ 2-\frac{ 3 }{ x^{15} }+\frac{ 1200 }{ x^{27} } }\\
					&=\lim_{x\to -\infty} \frac{ x^{9} }{ 2 }\\
					&=-\infty.
				\end{aligned}
			\end{equation}
		\item
			La limite que l'on connaît est celle $\lim_{x\to 0} \frac{ \sin(2x) }{ 2x }=1$. Ici le problème est que nous avons juste $x$ au dénominateur. L'astuce est de multiplier et diviser par $2$ :
			\begin{equation}
				\lim_{x\to 0} \frac{ \sin(2x) }{ x }=\lim_{x\to 0} \frac{ 2\sin(2x) }{ 2x }=2.
			\end{equation}
		\item
			La fonction $\sin(x)$ est bornée, et nous la divisions par quelque chose qui tend vers l'infini. La limite est donc $0$. Plus formellement on peut dire
			\begin{equation}
				0\leq\left| \frac{ \sin(x) }{ x } \right| \leq \frac{ 1 }{ | x | },
			\end{equation}
			tandis que $\frac{ 1 }{ x }\to 0$ si $x\to\infty$.
		\item
			Ici l'astuce est de multiplier et diviser par le «binôme conjugué» du dénominateur, à savoir
			\begin{equation}
				\sqrt{4-x}+\sqrt{4+x}.
			\end{equation}
			Cela fait apparaître au dénominateur le produit
			\begin{equation}
				(\sqrt{4-x}-\sqrt{4+x})(\sqrt{4-x}+\sqrt{4+x})=(4-x)-(4+x)
			\end{equation}
			par le produit remarquable $(\spadesuit-\clubsuit)(\spadesuit+\clubsuit)=\spadesuit^2-\clubsuit^2$. La limite à calculer est donc
			\begin{equation}
				\lim_{x\to 0} \frac{ \sqrt{x}\left( \sqrt{4-x}+\sqrt{4+x} \right) }{ (4-x)-(4+x) }=\lim_{x\to 0} \frac{ 4\sqrt{x} }{ -2x }=\lim_{x\to 0} \frac{ -2 }{ \sqrt{x} }=-\infty.
			\end{equation}

		\item
			Ici nous utilisons le fait que «le logarithme est le moins fort». Donc le logarithme du numérateur tend moins vite vers l'infini que la puissance de $x$ qui est au dénominateur. Nous avons donc
			\begin{equation}
				\lim_{x\to \infty} \frac{ \ln(x) }{ x^{1/1234} }=0.
			\end{equation}
	\end{enumerate}

\end{corrige}
