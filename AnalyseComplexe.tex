% This is part of Mes notes de mathématique
% Copyright (c) 2012
%   Laurent Claessens
% See the file fdl-1.3.txt for copying conditions.

Source : \cite{Holomorphieus}.

%+++++++++++++++++++++++++++++++++++++++++++++++++++++++++++++++++++++++++++++++++++++++++++++++++++++++++++++++++++++++++++
\section{Dérivabilité au sens complexe}
%+++++++++++++++++++++++++++++++++++++++++++++++++++++++++++++++++++++++++++++++++++++++++++++++++++++++++++++++++++++++++++

Dans cette partie, nous désignons par \( \Omega\) un ouvert de \( \eC\). Une fonction \( f\colon \Omega\to \eC\) est $C$-dérivable si la limite
\begin{equation}
    \lim_{h\to 0} \frac{ f(a+h)-f(a) }{ h }
\end{equation}
existe. Dans ce cas, cette limite est la dérivée de \( f\).

Nous identifions \( \eR^2\) à \( \eC\) par l'application \( \varphi\colon \eR^2\to \eC\) l'application \( \varphi(x,y)=x+iy\).

\begin{lemma}
    Une application \( A\colon \eC\to \eC\) est \( \eC\)-linéaire si et seulement si sa matrice en tant qu'application \( \eR^2\to \eR^2\) est la de forme
    \begin{equation}
        \begin{pmatrix}
            \alpha    &   \beta    \\ 
            -\beta    &   \alpha    
        \end{pmatrix}.
    \end{equation}
\end{lemma}

\begin{proposition}
    Une fonction \( f\colon \eC\to \eC\) est $\eC$-dérivable au point \( z_0=x_0+iy_0\) si et seulement si la fonction \( F=\varphi^{-1}\circ f\circ \varphi\) est différentiable en \( (x_0,y_0)\) et si la matrice de \( dF\) est de la forme
    \begin{equation}
        dF=\begin{pmatrix}
            \alpha    &   \beta    \\ 
            -\beta    &   \alpha    
        \end{pmatrix},
    \end{equation}
    c'est à dire si \( dF\) fournit une application \( \eC\)-linéaire.
\end{proposition}

\begin{proof}
    Nous décomposons \( f\) en parties réelles et imaginaires :
    \begin{equation}
        f(x+iy)=P(x,y)+iQ(x,y)
    \end{equation}
    où \( P\) et \( Q\) sont des fonctions réelles. La jacobienne de \( F\) est la matrice
    \begin{equation}
        \begin{pmatrix}
            \frac{ \partial P }{ \partial x }    &   \frac{ \partial P }{ \partial y }    \\ 
            \frac{ \partial Q }{ \partial x }    &   \frac{ \partial Q }{ \partial y }    
        \end{pmatrix},
    \end{equation}
    et la condition dont nous parlons s'écrit comme le système
    \begin{subequations}
        \begin{numcases}{}
            \frac{ \partial P }{ \partial x }=\frac{ \partial Q }{ \partial y }\\
            \frac{ \partial P }{ \partial y }=-\frac{ \partial Q }{ \partial x}.
        \end{numcases}
    \end{subequations}
    Si \( F\) est différentiable en \( (x_0,y_0)\) alors nous avons
    \begin{equation}        \label{EqwlVfiR}
        F\big( (x_0,y_0)+(h,k) \big)=F(x_0,y_0)+dF_{(x_0,y_0)}\begin{pmatrix}
            h    \\ 
            k    
        \end{pmatrix}+s(| h |+| k |)
    \end{equation}
    où \( s\) est une fonction vérifiant \( \lim_{t\to 0} \frac{ s(t) }{ t }=0\). Soit
    \begin{equation}
        dF_{(x_0,y_0)}=\begin{pmatrix}
            \alpha    &   \beta    \\ 
            -\beta    &   \alpha    
        \end{pmatrix}.
    \end{equation}
    Si nous posons \( \sigma=\alpha-i\beta\) et \( w=h+ik\), l'équation \eqref{EqwlVfiR} s'écrit dans \( \eC\) sous la forme
    \begin{equation}        \label{EqYFmoiM}
        f(z_0+w)=f(z_0)+\sigma w+s(|w|),
    \end{equation}
    ce qui implique que \( f\) est $\eC$-dérivable en \( z_0\).

    Supposons maintenant que \( f\) soit $\eC$-dérivable en \( z_0\). Alors nous avons
    \begin{equation}
        f'(z_0)=\lim_{w\to 0} \frac{ f(z_0+w)-f(z_0) }{ w }=\sigma\in \eC,
    \end{equation}
    ce qui se récrit sous la forme
    \begin{equation}
        \lim_{w\to 0} \frac{ f(z_0+w)-f(z_0)-\sigma w }{ w }=0.
    \end{equation}
    Si nous posons \( z_0=x_0+iy_0\), \( w=h+ik\) et \( \sigma=\alpha-i\beta\) nous avons
    \begin{equation}
        \lim_{(h,k)\to (0,0)} \left| \frac{ F\big( (x_0,y_0)+(h,k) \big)-F(x_0,y_0)-\begin{pmatrix}
            \alpha    &   \beta    \\ 
            -\beta    &   \alpha    
        \end{pmatrix}\begin{pmatrix}
            h    \\ 
            k    
        \end{pmatrix}}{ | w | } \right| =0,
    \end{equation}
    ce qui signifie que \( F\) est différentiable et que sa différentielle est la matrice
    \begin{equation}
       \begin{pmatrix}
           \alpha &   \beta    \\ 
           -\beta &   \alpha    
       \end{pmatrix}.
    \end{equation}
\end{proof}

Notons que la formule \eqref{EqYFmoiM} donne un \defe{développement limité}{développement limité!fonction holomorphe} pour les fonctions holomorphes. Si \( f\) est holomorphe en \( z_0\) alors si \( z\) est dans un voisinage de \( z_0\), il existe une fonction \( s\colon \eR\to \eC\) telle que \( \lim_{t\to 0} s(t)/t=0\) et 
\begin{equation}    \label{EqptwBFG}
    f(z)=f(z_0)+f'(z_0)(z-z_0)+s(| z-z_0 |).
\end{equation}

Nous introduisons les opérateurs
\begin{subequations}
    \begin{align}
        \frac{ \partial  }{ \partial z }=\partial=\frac{ 1 }{2}\left( \frac{ \partial  }{ \partial x }-i\frac{ \partial  }{ \partial y } \right)\\
        \frac{ \partial  }{ \partial \bar z }=\bar\partial=\frac{ 1 }{2}\left( \frac{ \partial  }{ \partial x }+i\frac{ \partial  }{ \partial y } \right)
    \end{align}
\end{subequations}
Si \( f\) est une fonction $\eC$-dérivable représentée par la fonction \( F=P+iQ\), les équations de Cauchy-Schwartz signifient que \( \Delta P=\Delta Q=0\), c'est à dire que la fonction \( f\) a des composantes harmoniques.


%+++++++++++++++++++++++++++++++++++++++++++++++++++++++++++++++++++++++++++++++++++++++++++++++++++++++++++++++++++++++++++
\section{Fonctions holomorphes}
%+++++++++++++++++++++++++++++++++++++++++++++++++++++++++++++++++++++++++++++++++++++++++++++++++++++++++++++++++++++++++++
\label{SecoLNvnO}

\begin{definition}
    Soit \( \Omega\) un ouvert dans \( \eC\). Une fonction \( \Omega\colon \Omega\to \eC\) est \defe{holomorphe}{holomorphe}\index{fonction!holomorphe} si elle est \( C^1\) et \( \eC\)-dérivable sur \( \Omega\). 
\end{definition}

\begin{proposition}
    Une fonction \( f\colon \Omega\to \eC\) est $\eC$-dérivable en \( a\in\Omega\) si et seulement si elle est différentiable en \( a\) et si \( df_a\) est une similitude.
\end{proposition}

\begin{theorem}
    Si \( f\in C^1(\Omega)\) alors nous avons équivalence des faits suivants :
    \begin{enumerate}
        \item
            \( f\) est holomorphe sur \( \Omega\),
        \item
            \( f\) vérifie \( \partial_{\bar z}f=0\).
    \end{enumerate}
\end{theorem}

\begin{lemma}       \label{LemtpEOmi}
    Si \( g\) est une fonction continue dans un ouvert \( \Omega\subset \eC\) et si \( g\) admet une primitive complexe sur \( \Omega\) alors 
    \begin{equation}
        \int_{\gamma}g(z)dz=0
    \end{equation}
    pour tout chemin fermé \( \gamma\) de classe \( C^1\) contenu dans \( \Omega\).
\end{lemma}

\begin{proof}
    Nommons \( G\) une primitive de \( g\). Par définition,
    \begin{subequations}
        \begin{align}
            \int_{\gamma}g&=\int_{\gamma}G'\\
            &=\int_0^1G'\big( \gamma(t) \big)\gamma'(t)dt\\
            &=\int_0^1 (G\circ g\gamma)'(t)dt\\
            &=G(\gamma(1))-G\big( \gamma(0) \big)\\
            &=0
        \end{align}
    \end{subequations}
    parce que le chemin est fermé : \( \gamma(0)=\gamma(1)\).
\end{proof}

\begin{lemma}[Goursat\cite{Holomorphieus}]
    Soit \( \Omega\) un ouvert dans \( \eC\) et \( f\) une fonction continue sur \( \Omega\), holomorphe sur \( \Omega\) moins éventuellement un point (nommé \( z_1\in\Omega\)). Soit \( T\), un triangle\footnote{Nous considérons ici le triangle «plein».} fermé inclus à \( \Omega\). Alors nous avons
    \begin{equation}
        \int_{\partial T}f(z)dz=0.
    \end{equation}
\end{lemma}

\begin{proof}
    Nous notons \( \gamma=\partial T\). Dans la suite nous allons définir une suite de triangles \( T^{(n)}\) et nous noterons \( \gamma_n=\partial T^{(n)}\) avec une orientation que nous allons expliquer. Pour commencer nous posons \( T^{(0)}=T\) et \( \gamma_0=\partial T^{(0)}\).

    Nous considérons le cas \( z_1\notin T\), et nous posons
    \begin{equation}
        c=l(\gamma)^{-2}| \int_{\gamma}f |.
    \end{equation}
    Notre objectif est de montrer que \( c=0\). Soit \( A,B,C\) les trois sommes du triangle; nous divisons le triangle de la façons suivante. D'abord nous considérons les points \( A',B,C'\) respectivement milieux de \( BC\), \( AC\) et \( AB\). En traçant le triangle \( A'B'C'\), nous construisons quatre triangles que nous nommons \( T^{(0)}_i\). Le théorème de Thalès assure que le périmètre de chacun des quatre triangles est la moitié du périmètre du grand triangle \( T\).

    Sur \( T\) nous choisissons l'orientation \( ABC\). De façon à être «compatible», nous choisissons les orientations \( AC'B'\), \( BA'C'\) et \( A'CB'\). La somme de ces trois triangles donne \( T\) plus le triangle \( A'C'B'\). Par conséquent nous choisissons sur le triangle central l'orientation (inverse) \( AB'C'\) de façon à avoir
    \begin{equation}
        \int_{\gamma}f=\sum_{i=1}^4\int_{\partial T^{(0)}_i}f.
    \end{equation}
    Cela implique que pour au moins un des quatre triangles (disons \( T^{(0)}_k\) pour fixer les idées) nous ayons
    \begin{equation}
        \int_{\partial T^{(0)}_k}f\geq \frac{1}{ 4 }\int_{\partial T^{(0)}}f
    \end{equation}
    Nous notons \( T^{(1)}\) ce triangle. Comme noté précédemment nous avons
    \begin{equation}
        l(\partial T^{(1)})=\frac{ 1 }{2}l(\partial T^{(0)}),
    \end{equation}
    et donc
    \begin{equation}
        l(\gamma_1)^{-2}| \int_{\gamma_1} |f=4l(\gamma_0)^{-2}| \int_{\gamma_1}f |\geq 4l(\gamma_0)^{-2}\frac{1}{ 4 }| \int_{\gamma_0}f |=c.
    \end{equation}
    En répétant le procédé nous construisons une suite de triangles \( T^{(n)}\) qui satisfont toujours
    \begin{equation}
        l(\partial T^{(n)})=\frac{1}{ 2^n }l(\partial T^{(0)}).
    \end{equation}
    Ces triangles forment une suite de fermés emboités dont le diamètre tend vers zéro. Leur intersection contient donc exactement un point (lemme \ref{LemdCOMQM}) que nous nommons \( z_0\) (et qui appartient évidemment à \( \Omega\)). Étant donné que \( f\) est holomorphe nous utilisons le développement limité \eqref{EqptwBFG} autour de \( z_0\) :
    \begin{equation}
        f(z)=f(z_0)+f'(z_0)(z-z_0)+s(| z-z_0 |)(z-z_0)
    \end{equation}
    avec \( \lim_{t\to 0} s(t)=0\). Nous posons \( g(z)=f(z_0)+f'(z_0)(z-z_0)\) et nous considérons \( \epsilon>0\). Soit \( \alpha>0\) tel que
    \begin{equation}
        | f(z)-g(z) |<\epsilon| z-z_0 |
    \end{equation}
    pour tout \( | z-z_0 |<\alpha\). Le \( \alpha\) à choisir pour obtenir cet effet est celui qui donne \( s(| z-z_0 |)<\epsilon\). Soit \( N\in \eN\) tel que \( l(\gamma_n)<\alpha\) pour tout \( n>N\). D'autre part, deux points dans un triangle sont toujours à distance moindre que la longueur d'un côté, donc pour tout \( z\in T^{(n)}\) nous avons \( | z-z_0 |<\alpha\) et par conséquent pour tout \( z\) dans \( T^{(n)}\) nous avons
    \begin{equation}
        | f(z)-g(z) |<\epsilon| z-z_0 |.
    \end{equation}
    Notons que la fonction \( g\) est une dérivée : c'est la dérivée de la fonction
    \begin{equation}
        G(z)=zf(z_0)+\frac{ 1 }{2}f'(z_0)(z-z_0)^2.
    \end{equation}
    Par conséquent nous avons
    \begin{equation}
        \int_{\gamma_n}g=0
    \end{equation}
    par le lemme \ref{LemtpEOmi}. Nous avons donc
    \begin{subequations}
        \begin{align}
            | \int_{\gamma_n}f |&=|\int_{\gamma_n}(f-g)|\\
            &\leq l(\gamma_n)\max\{ | f(z)-g(z) |\tq z\in T^{(n)} \}\\
            &\leq \epsilon l(\gamma_n)^2,
        \end{align}
    \end{subequations}
    et par conséquent
    \begin{equation}
        c\leq l(\gamma_n)^{-2}| \int_{\gamma_n}f |\leq \epsilon,
    \end{equation}
    ce qui signifie que \( c=0\) parce que \( \epsilon\) est arbitraire. Nous avons donc prouvé le lemme de Goursat dans le cas où le point de non holomorphie \( z_1\) est en dehors de \( T\).

    Si \( z_1\) est sur un côté, disons sur le côté \( AB\), alors nous considérons un vecteur \( v\in \eC\) tel que \( T_{\epsilon}=T+\epsilon v\) ne contienne \( z_1\) pour aucun \( \epsilon\). Le vecteur \( v=z_1-C\) fait par exemple l'affaire. En vertu du point précédent nous avons
    \begin{equation}
        \int_{\partial T_{\epsilon}}f=0
    \end{equation}
    pour tout \( \epsilon>0\). Étant donné que la fonction \( f\) est continue (y compris en \( z_1\)), l'intégrale sur \( \partial T\) est également nulle.

    Si maintenant le point \( z_1\) est à l'intérieur de \( T\) nous décomposons \( T\) en trois triangles ayant \( z_1\) comme sommet commun. Si nous considérons les orientations \( Az_1C\), \( ABz_1\) et \( BCz_1\), alors nous avons
    \begin{equation}
        \int_Tf=\int_{Az_1C}f+\int_{ABz_1}f+\int_{BCz_1}f,
    \end{equation}
    alors que par le point précédent les trois intégrales du membre de droite sont nulles.
\end{proof}

\begin{proposition}[\cite{Holomorphieus}]   \label{PrpopwQSbJg}
    Soit \( \Omega\) un ouvert étoilé et \( f\) une fonction holomorphe sur \( \Omega\) sauf éventuellement en un point \( z_1\) où \( f\) est seulement continue. Alors si \( \gamma\) est un chemin fermé dans \( \Omega\), nous avons
    \begin{equation}
        \int_{\gamma}f=0.
    \end{equation}
\end{proposition}

\begin{definition}
    Une fonction \( f\colon \Omega\to \eC\) est \( \eC\)-analytique sur \( \Omega\) si pour tout \( z_0\in \Omega\) il existe une suite \( (c_n)\) dans \( \eC\) et \( r>0\) tels que
    \begin{equation}
        f(z)=\sum_{n=0}^{\infty}c_n(z-z_0)^n
    \end{equation}
    pour tout \( z\in B(z_0,r)\).
\end{definition}

\begin{proposition}
    Une application \( f\colon \Omega\to \eC\) est $C$-dérivable sur \( \Omega\) si et seulement si elle est différentiable et
    \begin{subequations}        \label{EqmblExI}
        \begin{numcases}{}
            \frac{ \partial u }{ \partial x }=\frac{ \partial v }{ \partial y }\\
            \frac{ \partial u }{ \partial y }=-\frac{ \partial v }{ \partial x }
        \end{numcases}
    \end{subequations}
    où \( f(x+iy)=u(x,y)+iv(x,y)\).
\end{proposition}
Les équations \eqref{EqmblExI} sont les équations de \defe{Cauchy-Riemann}{Cauchy-Riemann}.

\begin{proof}
    La différentielle de \( f\colon \eR^2\to \eR^2\) est donnée par la matrice
    \begin{equation}        \label{EQwtagsz}
        T=\begin{pmatrix}
            \partial_xu(a)    &   \partial_yu(a)    \\ 
            \partial_xv(a)    &   \partial_yv(a)    
        \end{pmatrix}.
    \end{equation}
    Cette matrice est une similitude si et seulement si les équations de Cauchy-Riemann sont satisfaites. En effet si \( 1=\begin{pmatrix}
        1    \\ 
        0    
    \end{pmatrix}\) et \( i=\begin{pmatrix}
        0    \\ 
        1    
    \end{pmatrix}\), la matrice \( T\) est une similitude (écrivons \( \alpha+i\beta\) son coefficient) si
    \begin{subequations}
        \begin{numcases}{}
            T(1)=\alpha+i\beta\\
            T(i)=-\beta+i\alpha,
        \end{numcases}
    \end{subequations}
    c'est à dire
    \begin{equation}
        T=\begin{pmatrix}
            \alpha    &   -\beta    \\ 
           \beta    &   \alpha    
        \end{pmatrix}.
    \end{equation}
    Identifier cette matrice à \eqref{EQwtagsz} fournit le résultat annoncé.
\end{proof}

\begin{proposition}
    Une fonction \( f\colon \Omega\to \eC\) est $C$-dérivable si et seulement si elle est différentiable et \( df_a\) est une similitude.
\end{proposition}


\begin{proposition}     \label{PropRZCKeO}
    Si \( f(z)=\sum_na_nz^n\) a pour rayon de convergence \( R\), alors \( f\) est $C$-dérivable et nous pouvons dériver terme à terme dans la boule ouverte \( B(0,R)\).
\end{proposition}

\begin{proof}
    Cela est exactement la proposition \ref{ProptzOIuG}.
\end{proof}

\begin{definition}
    Une fonction \( f\colon \Omega\to \eC\) est \( \eC\)-analytique si pour tout \( z_0\in\Omega\), il existe une suite \( c_n\) et \( r>0\) tels que
    \begin{equation}
        f(z)=\sum_n c_n(z-z_0)^n
    \end{equation}
    pour tout \( z\in B(z_0,r)\).
\end{definition}

\begin{proposition}
    Une fonction analytique est holomorphe.
\end{proposition}

%---------------------------------------------------------------------------------------------------------------------------
\subsection{Théorème de Cauchy}
%---------------------------------------------------------------------------------------------------------------------------

Cette sous-section veut prouver le théorème de Cauchy. Comme d'habitude, une référence qui ne peut pas rater est \cite{Holomorphieus}.

\begin{definition}
    Soit \( \gamma\) un chemin fermé\footnote{Par abus de langage, nous désignerons par \( \gamma\) à la fois le chemin et don image.} dans \( \eC\). L'\defe{indice}{indice!d'une courbe dans $\eC$} de la courbe \( \gamma\) est la fonction
    \begin{equation}
        \begin{aligned}
            \Ind_{\gamma}\colon \eC\setminus \gamma&\to \eZ \\
            z&\mapsto \frac{1}{ 2\pi i }\int_{\gamma}\frac{ d\omega }{ \omega-z }. 
        \end{aligned}
    \end{equation}
\end{definition}
Le fait que cette fonction prenne ses valeurs dans \( \eZ\) est l'objet du théorème suivant.

\begin{theorem}     \label{ThoDYQQXZ}
    La fonction \( \Ind_{\gamma}\) est constante sur chaque composante connexe de \( \eC\setminus \gamma\) et est nulle sur la composante non bornée.
\end{theorem}

\begin{example} \label{ExradygL}
    Si \( \gamma\) est un cercle de centre \( z_0\in \eC\) et de rayon \( r\), alors 
    \begin{equation}
        \Ind_{\gamma}(z)=\begin{cases}
            2\pi i    &   \text{si \( z\in B(z_0,r)\)}\\
            0    &    \text{sinon}.
        \end{cases}
    \end{equation}
    La seconde ligne provient directement du théorème \ref{ThoDYQQXZ}. Pour la première, le cercle \( \gamma\) se paramètre par
    \begin{equation}
        \gamma(\theta)=z_0+r e^{i\theta},
    \end{equation}
    et l'intégrale vaut
    \begin{equation}
        \int_{\gamma}\frac{ d\omega }{ \omega-z_0 }=\int_0^{2\pi}\frac{1}{ r e^{i\theta} }ir e^{i\theta}d\theta=2\pi i.
    \end{equation}
    L'indice de ce chemin va évidemment jouer un rôle particulier dans la suite.
\end{example}

\begin{theorem}[formule de Cauchy]\index{formule!de Cauchy}\index{Cauchy!formule}   \label{ThoUHztQe}
    Soit \( \Omega\) ouvert dans \( \eC\), \( z_0\in \Omega\) et \( f\), une fonction holomorphe sur \( \Omega\). Soit \( r>0\) tel que \( B(z_0,r)\subset \Omega\). Alors pour tout \( z\in B(z_0,r)\) nous avons
    \begin{equation}
        f(z)=\frac{1}{ 2\pi i }\int_{\partial B(z_0,r)}\frac{ f(\omega) }{ \omega-z }d\omega.
    \end{equation}
\end{theorem}

\begin{proof}
    Soit \( z\in B(z_0,r)\) et considérons la fonction
    \begin{equation}
        g(\omega)=\begin{cases}
            \frac{ f(\omega)-f(z) }{ \omega-z }    &   \text{si \( \omega\neq z\)}\\
            f'(z)    &    \text{si \( \omega=z\)}.
        \end{cases}
    \end{equation}
    Cette fonction est holomorphe sur \( B(z_0,r)\setminus\{ z \}\). Étant holomorphe sur \( B(z_0,r)\setminus\{ z \}\) et continue en \( z\), elle vérifie la proposition \ref{PrpopwQSbJg} et nous avons
    \begin{equation}
        \int_{\gamma}g=0
    \end{equation}
    où \( \gamma\) est le cercle de centre \( z_0\) et de rayon \( r\). Nous avons donc
    \begin{equation}
        0=<int_{\gamma}\frac{ f(\omega) }{ \omega-z }-\int_{\gamma}\frac{ f(z) }{ \omega-z },
    \end{equation}
    et ayant déjà calculé la seconde intégrale dans l'exemple \ref{ExradygL} nous en déduisons
    \begin{equation}
        \int_{\gamma}\frac{ f(\omega) }{ \omega-z }d\omega=2\pi if(z),
    \end{equation}
    ce qu'il fallait.
\end{proof}


\begin{theorem}
    Soit \( \Omega\) ouvert dans \( \eC\) et \( f\), holomorphe sur \( \Omega\). Soient encore \( z_0\in \Omega\) et \( r_0\) tel que \( B(z_0,r_0)\subset \Omega\). Alors sur \( B(z_0,r_0)\), la fonction \( f\) s'écrit
    \begin{equation}
        f(z)=\sum_{n=0}^{\infty}a_n(z-z_0)^n.
    \end{equation}
    De plus nous avons
    \begin{equation}
        a_n=\frac{ f^{(n)}(z_0) }{ n! }=\frac{1}{ 2\pi i }\int_{\gamma}\frac{ f(\omega) }{ (\omega-z_0)^{n+1} }d\omega
    \end{equation}
    où \( \gamma=\partial B(z_0,r)\) avec \( | z-z_0 |<r<r_0\).

    En particulier, \( f\) est infiniment dérivable.
\end{theorem}

\begin{proof}
    Soit \( r>0\) tel que \( | z-z_0 |<r<r_0\). La formule de Cauchy (théorème \ref{ThoUHztQe}) nous dit que
    \begin{equation}
        f(z)=\frac{1}{ 2\pi i }\int_{\gamma}\frac{ f(\omega)}{ \omega-z }d\omega
    \end{equation}
    où \( \gamma=\partial B(z_0,r)\). Nous pouvons paramétrer ce chemin par \( \omega=z_0+r e^{i\theta}\) et \( \theta\in \mathopen[ 0 , 2\pi \mathclose]\). Nous avons
    \begin{subequations}
        \begin{align}
            f(z)&=\frac{1}{ 2\pi i }\int_0^{2\pi}\frac{ f(z_0+r e^{i\theta}) }{ z_0+r e^{i\theta}-z }ri e^{i\theta}d\theta\\
            &=\frac{1}{ 2\pi }\int_0^{2\pi}\frac{ f(z_0+r e^{i\theta}) }{ 1- e^{-i\theta}(z-z_0)/r }d\theta.
        \end{align}
    \end{subequations}
    Nous pouvons développer l'intégrante en puissance de \( (z-z_0)\) en utilisant la formule \ref{EqVmuaqT}. Ici le rôle de \( x\) est tenu par
    \begin{equation}
        e^{-i\theta}(z-z_0)/r
    \end{equation}
    dont le module est bien plus petit que \( 1\), par hypothèse sur \( r\). Nous avons donc
    \begin{equation}
        f(z)=\frac{1}{ 2\pi }\int_0^{2\pi}\sum_{n=0}^{\infty}f(z_0+r e^{i\theta}) e^{-in\theta}r^{-n}(z-z_0)^nd\theta.
    \end{equation}
    L'art est maintenant de permuter la somme et l'intégrale. Pour cela nous remarquons que ce qui se trouve dans la somme est majoré en module par
    \begin{equation}        \label{EqbykTLD}
        M\left| \frac{ z-z_0 }{ r } \right|^n
    \end{equation}
    où \( M\) est le maximum de \( | f |\) sur \( \gamma\). La borne \eqref{EqbykTLD} ne dépend pas de \( \theta\); par conséquent la convergence de la somme est uniforme en \( \theta\) par le critère de Weierstrass (théorème \ref{ThoCritWeierstrass}). Le théorème \ref{ThoCciOlZ} s'applique\footnote{Étant donné que nous savions déjà que la somme était une fonction intégrable, nous sommes loin d'avoir utilisé toute la puissance du théorème.} et nous pouvons permuter la somme avec l'intégrale.

    Ce que nous trouvons est que
    \begin{equation}
        f(z)=\sum_{n=0}^{\infty}a_n(z-z_0)^n
    \end{equation}
    où
    \begin{equation}
        a_n=\frac{1}{ 2\pi }\int_0^{2\pi}f(z_0+r e^{i\theta}) e^{-in\theta}r^{-n}d\theta=\frac{1}{ 2\pi i }\int_{\gamma}\frac{ f(\omega) }{ (\omega-z_0)^{n+1} }.
    \end{equation}
    Cette formule est valable pour \( | z-z_0 |<r\). Sur cette boule, la fonction est donc une série entière Le théorème de Taylor \ref{ThoTGPtDj} nous permet donc d'affirmer que \( f\) est partout infiniment continument dérivable (parce que en chaque point on a un voisinage sur lequel c'est vrai), et d'identifier les coefficients (qui, eux, ne sont valables que localement) sous la forme
    \begin{equation}
        a_n=\frac{ f^{(n)}(z_0) }{ n! }.
    \end{equation}
\end{proof}

%---------------------------------------------------------------------------------------------------------------------------
\subsection{Principe des zéros isolés}
%---------------------------------------------------------------------------------------------------------------------------

\begin{theorem}[Principe des zéros isolés \cite{Holomorphieus}]     \label{ThoukDPBX}
    Soit \( f\) une fonction holomorphe et \( a\), une zéro non isolé de \( f\). Alors \( f\) est nulle sur un voisinage de \( a\).
\end{theorem}

\begin{proof}
    Nous écrivons \( f\) sous la forme d'une série entière autour de \( a\) :
    \begin{equation}        \label{EqgrvfVl}
        f(z)=\sum_{n=0}^{\infty}c_n(z-a)^n
    \end{equation}
    valable sur une boule \( B(a,r)\). Soit \( c_m\) le premier coefficient non nul (si il n'existe pas c'est que \( f\) est nulle sur tout \( B(a,r) \) et alors le théorème est prouvé). Nous avons alors
    \begin{equation}
        f(z)=c_m(z-a)^m\big( 1+\sum_{k=1}^{\infty}d_k(z-a)^k \big)
    \end{equation}
    avec \( d_k=c_{m-k}\). Le rayon de convergence de la série \( \sum_k d_k(z-a)^k\) est le même que celui de \eqref{EqgrvfVl} parce que la suite \( d_kr^{m+k}\) reste bornée (critère d'Abel, lemme \ref{LemmbWnFI}). Si nous posons
    \begin{equation}
        g(z)=1+\sum_{k=1}^{\infty}d_k(z-a)^k,
    \end{equation}
    alors \( g\) est une fonction continue et \( g(a)=1\). De plus 
    \begin{equation}
        f(z)=c_m(z-a)^mg(z).
    \end{equation}

    Soit une suite \( (z_n)\) de zéros de \( f\) convergent vers \( a\). Étant donné que \( g\) est continue, nous devrions avoir \( \lim_{k\to\infty}g(z_k)=g(a)=1\), mais si \( f(z_k=0)\) avec \( z_k\neq a\), alors \( g(z_k)=0\). Cela est un paradoxe qui nous permet de conclure que si la suite \( z_n\) existe bien, alors \( f\) est identiquement nulle sur un voisinage, c'est à dire que tous les \( c_n\) sont nuls.
\end{proof}

\begin{corollary}
    Soit \( f\) une fonction holomorphe sur un ouvert connexe \( \Omega\). Si \( f\) s'annule sur un un ouvert (non vide) de \( \Omega\), alors \( f\) s'annule sur tout \( \Omega\).
\end{corollary}

\begin{proof}
    soit 
    \begin{equation}
        N=\{ z\in \Omega\tq f=0\text{ sur un ouvert autour de $z$} \}.
    \end{equation}
    Le fait que \( N\) soit ouvert est évident à partir de sa définition. Nous allons montrer que \( N\) est également fermé dans \( \Omega\), et donc conclure que \( N=\Omega\). Soit \( (z_n)\) une suite dans \( N\) convergente vers \( z\in \Omega\). Étant donné que \( f(z_n)=0\) et que \( f\) est continue, nous avons
    \begin{equation}
        f(z)=\lim_{n\to \infty} f(z_n)=0,
    \end{equation}
    ce qui fait de \( z\) un zéro non isolé de \( f\). Par conséquent le principe des zéros isolés (théorème \ref{ThoukDPBX}) nous enseigne que \( f\) s'annule dans un voisinage autour de \( z\), c'est à dire que \( z\in N\). L'ensemble \( N\) est donc fermé.
\end{proof}

%---------------------------------------------------------------------------------------------------------------------------
\subsection{Prolongement}
%---------------------------------------------------------------------------------------------------------------------------

\begin{proposition}
    Soit \( \Omega\), un ouvert de \( \eC\) et \( f\colon \Omega\to \eC\) une fonction holomorphe sur \( \Omega\setminus\{ a \}\) (\( a\in \Omega\)). Nous supposons qu'il existe \( r>0\) tel que \( f\) est bornée sur \( B(a,r)\cap\Omega\). Alors \( f\) se prolonge en une fonction holomorphe sur \( \Omega\).
\end{proposition}

\begin{proof}
    Nous définissons la fonction \( g\colon \Omega\to \eC\) par
    \begin{equation}
        g(z)=\begin{cases}
            (z-a)f(z)    &   \text{si \( z\neq a\)}\\
            0    &    \text{si \( z=a\)}.
        \end{cases}
    \end{equation}
    Sur \( \Omega\setminus\{ a \}\), la fonction \( g\) est holomorphe (produit de fonctions holomorphes), et elle est continue en \( a\). Par conséquent elle est holomorphe sur \( \Omega\). Nous la développons en série entière sur une boule \( B(a,r)\) :
    \begin{equation}
        g(z)=\sum_{n=0}^{\infty}c_n(z-a)^n.
    \end{equation}
    Nous avons \( g(a)=c_0=0\). Nous posons
    \begin{equation}
        \varphi(z)=\sum_{n=0}^{\infty}c_{n+1}(z-a)^n.
    \end{equation}
    Si \( z\neq a\), alors \( \varphi(z)=f(a)\) parce que \( \varphi(z)=g(z)/(z-a)\). Mais \( \varphi\) est continue en \( a\), et donc holomorphe en \( a\).

    La fonction \( \varphi\) est par conséquent un prolongement holomorphe de \( f\) en \( a\).
\end{proof}

%+++++++++++++++++++++++++++++++++++++++++++++++++++++++++++++++++++++++++++++++++++++++++++++++++++++++++++++++++++++++++++
\section{Intégrales de fonctions holomorphes}
%+++++++++++++++++++++++++++++++++++++++++++++++++++++++++++++++++++++++++++++++++++++++++++++++++++++++++++++++++++++++++++

\begin{lemma}       \label{LemNAnweA}
    Soit \( f\) une fonction holomorphe sur \( B(z_0,r_0)\). Pour tout \( z\in B(z_0,r)\) (avec \( r<r_0\)) nous avons
    \begin{equation}
        | f'(z) |\leq \frac{ r }{ \big( r-| z-z_0 | \big)^2 }\max\big\{ f(z_0+r e^{i\theta}) \big\}_{\theta\in \eR}.
    \end{equation}
\end{lemma}

\begin{proof}
    Par translation nous pouvons supposer que \( z_0=0\). Étant donné que \( f\) est holomorphe, elle admet un développement en séries entières
    \begin{equation}
        f(z)=\sum_{n=0}^{\infty}a_nz^n
    \end{equation}
    et nous notons \( M=\max\{ f(z)\tq z\in \overline{ B(0,r) } \}\). Nous avons\cite{Holomorphieus} \( r^n| a_n |\leq M\). Par conséquent
    \begin{subequations}
        \begin{align}
            | f'(z) |&=\left| \sum_{n=1}^{\infty}na_nz^{n-1} \right| \\
            &\leq\frac{1}{ r }\sum r^n| a_n |n\left( \frac{ | z | }{ r } \right)^{n-1}\\
            &<\frac{ M }{ r }\sum n\left( \frac{ | z | }{ r } \right)^{n-1}
        \end{align}
    \end{subequations}
    À ce point nous devons utiliser la série de l'exemple \ref{ExGxzLlP}. Nous avons alors
    \begin{equation}
        | f'(z) |\leq \frac{ M }{ r }\frac{ 1 }{ \left( 1-\frac{ | z | }{ r } \right)^2 }=\frac{ Mr }{ (r-| z |)^2 }.
    \end{equation}
\end{proof}

\begin{theorem}
    Soit un espace mesuré \( (\Omega,\infty)\) et \( A\), un ouvert dans \( \eC\). Nous considérons 
    \begin{equation}
        F(z)=\int_{\Omega}f(z,\omega)d\mu(\omega).
    \end{equation}
    Nous supposons que
    \begin{enumerate}
        \item
            la fonction \( f(.,\omega)\) est holomorphe sur \( A\) pour chaque \( \omega\).
        \item
            La fonction \( f(z,.)\) est mesurable sur \( (\Omega,\mu)\).
        \item
            Pour tout compact \( K\subset A\), il existe une fonction \( g_K\colon \Omega\to \eR\) telle que \( | f(z,\omega) |\leq g_K(\omega)\) et telle que
            \begin{equation}
                \int_{\Omega}g_K(\omega)d\mu(\omega)
            \end{equation}
            existe.
    \end{enumerate}
    Alors la fonction \( F\) est holomorphe et
    \begin{equation}
        F'(z)=\int_{\Omega}\frac{ \partial f }{ \partial z }(z,\omega)d\mu(\omega).
    \end{equation}
\end{theorem}

\begin{proof}
    Soient \( z_0\in A\) et \( r>0\) tels que \( K=\overline{ B(z_0,r) }\subset A\). Pour chaque \( \omega\in \Omega\) nous considérons la fonction
    \begin{equation}
        \begin{aligned}
            f_{\omega}\colon \overline{ B(z_0,r) }&\to \eC \\
            z&\mapsto f(z,\omega). 
        \end{aligned}
    \end{equation}
    Étant donné que \( \overline{ B(z_0,r) }\) est compacte, la fonction \( | f_{\omega} |\) est majorée par un nombre que nous notons \( f_K(\omega)\) qui est indépendant de \( z\) (pour autant que $z\in K$). Nous désignons par \( S(z_0,r)\) la frontière de la boule \( B(z_0,r)\). Étant donné que la majoration est valable sur \( \overline{ B(z_0,r) }\), nous avons en particulier
    \begin{equation}
        | f_{\omega}(z) |\leq f_K(\omega)
    \end{equation}
    pour tout \( z\in S\). En utilisant la lemme \ref{LemNAnweA} nous avons
    \begin{subequations}
        \begin{align}
            | f'_{\omega}(z) |&\leq \frac{ r }{ (r-| z-z_0 |)^2 }\max\{ f(z_0+r e^{i\theta}) \}_{\theta\in \eR}\\
            &\leq \frac{ rf_K(\omega) }{ (r-| z-z_0 |)^2 }.
        \end{align}
    \end{subequations}
    Cette majoration est valable pour tout \( z\in B(z_0,r)\). Si nous supposons de plus que \( z\in B(z_0,r/2)\)  nous avons
    \begin{equation}
        | f'(z) |\leq \frac{ rf_K(\omega) }{ \left( r-\frac{ r }{2} \right)^2 }=\frac{ 4 }{ r }f_K(\omega).
    \end{equation}
    Étant donné que la boule \( B(z_0,r/2)\) est convexe, la fonction \( f_{\omega}\) est Lipschitz et pour tout \( h\in \eC\) tel que \( | h |<r/2\) nous avons
    \begin{equation}
        \left| \frac{ f_{\omega}(z_0+h)-f_{\omega}(z_0) }{ h } \right| \leq \frac{ 4f_K(\omega) }{ r }.
    \end{equation}
    Soit maintenant une suite \( (h_n)\) qui converge vers \( 0\) dans \( \eC\). Nous considérons la suite de fonctions correspondantes
    \begin{equation}
        g_n(\omega)=\frac{ f(z_0+h_n,\omega)-f(z_0,\omega) }{ h_n }.
    \end{equation}
    Cette suite de fonction vérifie la convergence ponctuelle
    \begin{equation}
        g_n(\omega)\to\frac{ \partial f }{ \partial z }(z_0,\omega).
    \end{equation}
    De plus \( g_n\) est une fonction (de \( \omega\)) dominée par \( \frac{ 4f_K }{ r }\) qui est intégrable. Par conséquent le théorème de la convergence dominée nous indique que
    \begin{equation}
        \int_{\Omega}g_n(\omega)d\mu(\omega)\to \int_{\Omega}\frac{ \partial f }{ \partial z }(z_0,\omega)d\mu(\omega),
    \end{equation}
    tandis que
    \begin{equation}
        F'(z)=\lim_{n\to \infty} \frac{ F(z_0+h_n)-F(z_0) }{ h_n }=\lim_{n\to \infty} \int_{\Omega}g_N(\omega)d\mu(\omega).
    \end{equation}
\end{proof}

\begin{definition}
    Une \defe{mesure de Radon}{mesure!de Radon} sur un compact \(  K\) de \( \eC\) est une forme linéaire continue sur \( C(K)\). Si \( \mu\) est une mesure de Radon, on définit la \defe{transformée de Cauchy}{transformée!de Cauchy} de \( \mu\) par 
    \begin{equation}
        \begin{aligned}
            \hat \mu\colon \eC\setminus K&\to \eC \\
            z&\mapsto -\frac{1}{ \pi }\mu\left( \frac{1}{ \xi-z } \right). 
        \end{aligned}
    \end{equation}
\end{definition}

\begin{theorem}     \label{ThoJVNTzn}
    Si \( \mu\) est une mesure de Radon sur \( K\) alors \( \hat \mu\) est infiniment \( \eC\)-dérivable sur \( \Omega=\eC\setminus K\) et nous avons
    \begin{equation}
        \hat\mu^{(n)}(z)=-\frac{ n! }{ \pi }\mu\left( \frac{1}{ (\xi-z)^{n+1} } \right).
    \end{equation}
\end{theorem}

\begin{lemma}
    Si \( f\) est holomorphe sur \( \Omega\) et si \( B\) est une boule fermée dans \( \Omega\) alors pour tout \( z\in \Int(B)\) nous avons
    \begin{equation}
        f^{(n)}(z)=\frac{ n! }{ 2i\pi }\int_{\partial B}\frac{ f(\xi) }{ (\xi-z)^{n+1} }d\xi.
    \end{equation}
\end{lemma}

\begin{proof}
    Appliquer le théorème \ref{ThoJVNTzn} à la mesure de Radon
    \begin{equation}
        \mu(\phi)=\int_{\partial B}\phi(\xi)d\xi.
    \end{equation}
\end{proof}

\begin{lemma}
    Si \( f\) est holomorphe sur \( \Omega\) et si \( B\) est une boule fermée dans \( \Omega\) alors pour tout \( z\) dans l'intérieur de \( B\) nous avons
    \begin{equation}
        f^{(n)}(z)=\frac{ n! }{ 2i\pi }\int_{\partial B}\frac{ f(\xi) }{ (\xi-z)^{n+1} }d\xi.
    \end{equation}
\end{lemma}

\begin{theorem}
    Si \( f\) est une fonction holomorphe sur le disque ouvert \( B(z_0,R)\) alors
    \begin{equation}
        f(z)=\sum_{n=0}^{\infty}\frac{ f^{(n)}(z_0) }{ n! }(z-z_0)^n
    \end{equation}
    et cette série converge uniformément sur tout compact.
\end{theorem}

\begin{proof}
    Sans perte de généralité nous supposons que \( z_0=0\). La formule de Cauchy fournit
    \begin{equation}
        f(z)=\frac{1}{ 2\pi i }\int_{\partial B}\frac{ f(\xi) }{ \xi-z }d\xi=\frac{1}{ 2\pi i }\int_{\partial B}\frac{ f(\xi) }{ 1-(z/\xi) }\frac{ d\xi }{ \xi }.
    \end{equation}
    Nous utilisons la série géométrique
    \begin{equation}
        \frac{1}{ 1-(z/\xi) }=\sum_{n=0}^{\infty}\left( \frac{ z }{ \xi } \right)^n,
    \end{equation}
    nous avons
    \begin{subequations}        \label{EqXSgZGw}
        \begin{align}
            f(z)&=\frac{1}{ 2\pi i }\sum_{n=0}^{\infty}\int_{\partial B}\frac{ z^nf(\xi) }{ \xi^{n+1} }\\
            &=\sum_{n=0}^{\infty}\left( \frac{1}{ 2\pi i }\int_{\partial B}\frac{ f(\xi) }{ \xi^{n+1} } \right)z^n.
        \end{align}
    \end{subequations}
    Nous devons maintenant montrer que ce qui se trouve dans la grande parenthèse vaut \( f^{(n)}(0)/n!\). Nous utilisons le théorème de Radon \ref{ThoJVNTzn} à la mesure
    \begin{equation}
        \mu(\phi)=\int_{\partial B}\phi(\xi)d\xi.
    \end{equation}
    La transformée de Cauchy est
    \begin{equation}        \label{EqTzkmeL}
        \hat \mu(z)=-\frac{1}{ \pi }\mu\left( \frac{1}{ \xi-z } \right)=-\frac{1}{ \pi }\int_{\partial B}\frac{1}{ \xi-z }d\xi,
    \end{equation}
    et le théorème assure que
    \begin{equation}
        \hat\mu^{(n)}(z)=-\frac{ n! }{ \pi }\mu\left( \frac{1}{ (\xi-z)^{n+1} } \right)=-\frac{ n! }{ \pi }\int_{\partial B}\frac{ 1 }{ (\xi-z)^{n+1} }d\xi.
    \end{equation}
    En comparant la formule \eqref{EqTzkmeL} avec la formule de Cauchy nous voyons que \( \hat\mu(z)=-2i f(z)\). Par conséquent
    \begin{equation}
        f^{(n)}(z)=-\frac{1}{ 2i }\hat\mu^{(n)}(z)=\frac{ n! }{ 2\pi i }\int_{\partial B}\frac{1}{ (\xi-z)^{n+1} }d\xi,
    \end{equation}
    et
    \begin{equation}
        f^{(n)}(0)=\frac{ n! }{ 2\pi i }\int_{\partial B}\frac{1}{ \xi^{n+1} }d\xi.
    \end{equation}
\end{proof}
% TODO : justifier la permutation entre la somme et l'intégrale.


%+++++++++++++++++++++++++++++++++++++++++++++++++++++++++++++++++++++++++++++++++++++++++++++++++++++++++++++++++++++++++++
\section{Exponentielle complexe}
%+++++++++++++++++++++++++++++++++++++++++++++++++++++++++++++++++++++++++++++++++++++++++++++++++++++++++++++++++++++++++++
Nous suivons les notes \cite{RomainBoilEnt}.

\begin{definition}  \label{DefJilXoM}
    Soit \( z=x+iy\in \eC\). Nous définissons l'\defe{exponentielle}{exponentielle!complexe} de \( z\) par
    \begin{equation}
        \begin{aligned}
            \exp\colon \eC&\to \eC \\
            z&\mapsto \sum_{n=0}^{\infty}\frac{ z^n }{ n! }. 
        \end{aligned}
    \end{equation}
\end{definition}
Le rayon de convergence de cette somme est infini.

\begin{proposition}     \label{PropdDjisy}
    Quelque propriétés de l'exponentielle.
    \begin{enumerate}
        \item
            Le fonction \( \exp\) est continue.
        \item
            Nous avons la formule \(  e^{z+w}= e^{z}+e^w\) pour tout \( z,w\in \eC\).
        \item
            \( (e^z)^{-1}= e^{-z}\)
        \item
            \( (\exp(z))^n=\exp(nz)\).
    \end{enumerate}
\end{proposition}

\begin{proof}
    L'exponentielle est continue parce qu'elle est la somme d'une série entière de rayon de convergence infini (proposition \ref{PropUEMoNF}).

    Les séries \( \exp(z)\) et \( \exp(w)\) ayant un rayon de convergence infini nous pouvons utiliser le produit de Cauchy (théorème \ref{ThokPTXYC}) :
    \begin{subequations}
        \begin{align}
            e^{z} e^{w}&=\sum_{n=0}^{\infty}\left( \sum_{i+j=n}\frac{ z^iw^j }{ i!j! } \right)\\
            &=\sum_{n=0}^{\infty}\left( \sum_{i=0}^n\frac{ z^iw^{n-i} }{ i!(n-i)! } \right)\\
            &=\sum_{n=0}^{\infty}\frac{1}{ n! }\sum_{i=0}^{n}{n\choose i}z^iw^{n-i}\\
            &=\sum_{n=0}^{\infty}\frac{1}{ n! }(z+w)^{n}\\
            &=\exp(z+w).
        \end{align}
    \end{subequations}
    Nous avons utilisé la formule du binôme (proposition \ref{PropBinomFExOiL}).

    Les autres propriétés énoncées sont des corollaires :
    \begin{equation}
        e^{z} e^{-z}= e^{0}=1.
    \end{equation}
\end{proof}

\begin{proposition}
    Si \( z=x+iy\in \eC\) alors
    \begin{equation}
        e^{x+iy}= e^{x}\big( \cos(y)+i\sin(y) \big).
    \end{equation}
\end{proposition}

\begin{proof}
    Par la proposition \ref{PropdDjisy} nous savons que \(  e^{x+iy}= e^{x} e^{iy}\). Nous devons donc seulement étudier \(  e^{iy}\). Nous avons
    \begin{subequations}
        \begin{align}
            e^{iy}&=\sum_{n=0}^{\infty}\frac{ (iy)^n }{ n! }\\
            &=\sum_{n=0}^{\infty}(-1)^n\frac{ y^{2n} }{ (2n)! }+i\sum_{n=0}^{\infty}(-1)^n\frac{ y^{2n+1} }{ (2n+1)! }\\
            &=\cos(y)+i\sin(y).
        \end{align}
    \end{subequations}
    Nous avons utilisé le fait que \( i^{2n}=(-1)^n\) et \( i^{2n+1}=i(-1)^n\).
\end{proof}

\begin{proposition}
    Soit \( z\in\eC\) fixé. La fonction
    \begin{equation}
        \begin{aligned}
            E\colon \eR&\to \eC \\
            t&\mapsto  e^{tz} 
        \end{aligned}
    \end{equation}
    est  \(  C^{\infty}\), sa dérivée est 
    \begin{equation}
        E'(t)=z e^{tz}.
    \end{equation}
    La fonction \( E\) est développable en série entière (voir définition \ref{DefwmRzKh}) sur \( \eR\) en \( t=0\) et
    \begin{equation}
        e^{tz}=\sum_{n=0}^{\infty}\frac{ z^n }{ n! }t^n.
    \end{equation}
\end{proposition}

\begin{proof}
    Nous fixons \( z\in \eC\). Par définition \ref{DefJilXoM}, la série suivante est \(  e^{tz}\) :
    \begin{equation}
        f(t)=\sum_{n=0}^{\infty}\frac{ z^n }{ n! }t^n.
    \end{equation}
    Cette série a un rayon de convergence infini et la fonction \( f\) est donc \(  C^{\infty}\) sur \( \eR\). Nous pouvons la dériver terme à terme :
    \begin{subequations}
        \begin{align}
            f'(t)&=\sum_{n=1}^{\infty}\frac{ z^n }{ n! }nt^{n-1}\\
            &=z\sum_{n=1}^{\infty}\frac{ z^{n-1} }{ (n-1)! }t^{n-1}\\
            &=z e^{tz}.
        \end{align}
    \end{subequations}
\end{proof}

\begin{theorem}
    La fonction exponentielle vérifie les propriétés suivantes.
    \begin{enumerate}
        \item
            \( \exp\) est holomorphe.
        \item
            \( (e^z)'=e^z\).
        \item
            L'exponentielle est développable en série entière,
            \begin{equation}
                e^z=\sum_{n=0}^{\infty}\frac{ z^n }{ n! }
            \end{equation}
            et la série converge normalement sur tout compact de \( \eC\).
    \end{enumerate}
\end{theorem}

\begin{proof}
    En tant que application \( E\colon \eR^2\to \eC\), la fonction
    \begin{equation}
        E(x,y)=e^x(\cos y+i\sin y)
    \end{equation}
    est \( C^{\infty}\). De plus nous avons
    \begin{subequations}
        \begin{align}
            \frac{ \partial E }{ \partial x }(x,y)= e^{x+iy}=E(x,y)\\
            \frac{ \partial E }{ \partial y }(x,y)=iE(x,y),
        \end{align}
    \end{subequations}
    et par conséquent la fonction \( E\) vérifie les équations de Cauchy-Riemann.


    Si \( r\) est fixé, par le critère d'Abel appliqué à la suite \(r/n!\) nous savons que la série \( \sum z^n/n!\) converge normalement sur le compact \( B(0,r)\).
\end{proof}



