%+++++++++++++++++++++++++++++++++++++++++++++++++++++++++++++++++++++++++++++++++++++++++++++++++++++++++++++++++++++++++++
\section{Fonctions holomorphes}
%+++++++++++++++++++++++++++++++++++++++++++++++++++++++++++++++++++++++++++++++++++++++++++++++++++++++++++++++++++++++++++

\begin{definition}
    Soit \( \Omega\) un ouvert dans \( \eC\). Une fonction \( \Omega\colon \Omega\to \eC\) est \defe{holomorphe}{holomorphe}\index{fonction!holomorphe} si elle est \( C^1\) et \( \eC\)-dérivable sur \( \Omega\). 
\end{definition}

\begin{proposition}
    Une fonction \( f\colon\Omega \to \eC\) est $C$-dérivable en \( a\in\Omega\) si et seulement si elle est différentiable en \( a\) et la différentielle \( df_a\) est une similitude.
\end{proposition}

\begin{theorem}
    Si \( f\in C^1(\Omega)\) alors nous avons équivalence des faits suivants :
    \begin{enumerate}
        \item
            \( f\) est holomorphe sur \( \Omega\),
        \item
            \( f\) vérifie \( \partial_{\bar z}f=0\).
    \end{enumerate}
\end{theorem}

\begin{definition}
    Une fonction \( f\colon \Omega\to \eC\) est \( \eC\)-analytique sur \( \Omega\) si pour tout \( z_0\in \Omega\) il existe une suite \( (c_n)\) dans \( \eC\) et \( r>0\) tels que
    \begin{equation}
        f(z)=\sum_{n=0}^{\infty}c_n(z-z_0)^n
    \end{equation}
    pour tout \( z\in B(z_0,r)\).
\end{definition}

%+++++++++++++++++++++++++++++++++++++++++++++++++++++++++++++++++++++++++++++++++++++++++++++++++++++++++++++++++++++++++++
\section{Séries entières}
%+++++++++++++++++++++++++++++++++++++++++++++++++++++++++++++++++++++++++++++++++++++++++++++++++++++++++++++++++++++++++++

Source : \cite{RomainBoilEnt}.

Nous rappelons qu'une série de nombres \( \sum_{n=0}^{\infty}a_n\) converge \defe{absolument}{convergence!absolue} si la série
\begin{equation}
    \sum_{n=0}^{\infty}| a_n |
\end{equation}
converge. Cette définition s'étend immédiatement aux séries dans n'importe quel espace normé.

La convergence est \defe{normale}{convergence!normale} si la suite de fonctions \( f_N(z)=\sum_{n=0}^N a_nz^n\) converge uniformément (c'est à dire pour la norme supremum).

\begin{definition}
    Une \defe{série entière}{série!entière} est une somme de la forme
    \begin{equation}
        \sum_{n=0}^{\infty}a_nz^n
    \end{equation}
    avec \( a_n,z\in\eC\).    
\end{definition}
Une série entière peut définir une fonction
\begin{equation}
    f(z)=\sum_na_nz^n.
\end{equation}
Le but de cette section est d'étudier des conditions sur la suite \( (a_n)\) qui assurent la continuité de \( f\) ou la possibilité de dériver ou intégrer la série terme à terme.


