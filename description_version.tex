
\thispagestyle{empty}

Ce document existe en plusieurs versions.
\begin{description}
    \item[Décembre 2015]

        La version préliminaire de celle qui aura un ISBN en décembre $2015$ :
        \begin{center}
            \url{http://student.ulb.ac.be/~lclaesse/mes_notes-2015.pdf}
        \end{center}

        C'est cette version que vous devriez utiliser si vous comptez passer l'agrégation en $2016$, sous les mêmes conditions que celles expliquées plus bas.

    \item[Décembre 2014] 

        Publiée en 2014 pour l'agrégation 2015.
        \begin{center}
        \url{http://student.ulb.ac.be/~lclaesse/mes_notes-2014.pdf}
        \end{center}

        C'est cette version que vous devriez utiliser si vous comptez passer l'agrégation en $2015$.
        
        Le règlement de l'agrégation vous permet sans ambigüités\footnote{\url{http://agreg.org/Pratique/bibliotheque.html}} d'en apporter une copie, et vous n'avez pas d'autorisation à demander\footnote{\url{http://agreg.org/Pratique/faq.html}}. Cependant le jury m'a fait savoir par un courrier personnel (rien d'officiel donc; vous me croyez ou non) qu'ils n'accepteraient pas que des candidats amènent des versions de ce document imprimés par leurs soins. M'est avis qu'ils ne refuseront pas que vous en mettiez dans la malle\ldots Faites moi savoir la réponse du jury si vous leur posez la question.


    \item[Décembre 2012]

        Publiée en 2012 pour l'agrégation 2013
        \begin{center}
        \url{http://student.ulb.ac.be/~lclaesse/mes_notes-2012.pdf}
        \end{center}

        Elle est disponible en \( 4\) exemplaires dans la bibliothèque de l'agrégation à Paris en 2013. D'où le fait que je croie qu'envoyer des versions plus récentes ne devrait pas poser de problèmes.

    \item[La version la plus complète]

        Une version plus complète, comprenant à la fois de matières plus avancées et des exercices moins avancés : 
        \begin{center}
        \url{http://student.ulb.ac.be/~lclaesse/mazhe.pdf}
        \end{center}

        C'est la version que vous devriez utiliser si l'agrégation ne vous intéresse pas.

\end{description}


\vfill

\LogoEtLicence
\clearpage
