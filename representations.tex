% This is part of Mes notes de mathématique
% Copyright (c) 2011-2012
%   Laurent Claessens
% See the file fdl-1.3.txt for copying conditions.

% TODO : à certains endroits j'utilise la macro \Gl. Est-ce que ça ne devrait pas être \GL ?


%+++++++++++++++++++++++++++++++++++++++++++++++++++++++++++++++++++++++++++++++++++++++++++++++++++++++++++++++++++++++++++
\section{Représentations et caractères}
%+++++++++++++++++++++++++++++++++++++++++++++++++++++++++++++++++++++++++++++++++++++++++++++++++++++++++++++++++++++++++++

Si \( G\) est un groupe, l'ensemble des homomorphismes \( \Hom(G,\eC^*)\) est un groupe pour la multiplication. Un élément de \( \Hom(G,\eC^*)\) est un \defe{caractère abélien}{caractère!abélien}. Le nom «abélien» vient du fait que le caractère prenne ses valeurs dans \( \eC^*\). Nous notons \( \hat G=\Hom(G,\eC^*)\)\nomenclature[R]{\( \hat G\)}{groupe des caractères de \( G\)}.

\begin{theorem}
    Soit \( G\) un groupe abélien fini. Alors \( G\) est isomorphe à \( \hat G\).
\end{theorem}
L'isomorphisme n'est pas canonique.

\begin{proof}
    Étant donné la structure des groupes abéliens finis donnée par le théorème \ref{ThoRJWVJd}, nous commençons par nous concentrer sur \( G=\eZ/n\eZ\). Nous allons montrer que
    \begin{equation}
        \Hom(\eZ/n\eZ)\simeq \gU_n=\{ \xi\in \eC\tq \xi^n=1 \}.
    \end{equation}
    Pour cela nous avons l'isomorphisme
    \begin{equation}
        \begin{aligned}
            \psi\colon \Hom(\eZ,\eC^*)&\to \eC^* \\
            f&\mapsto f(1). 
        \end{aligned}
    \end{equation}
    Notons que si \( f\in \Hom(\eZ,\eC^*)\), alors \( f(k)=f(1)^k\), donc \( \psi\) est bien un isomorphisme. Cela nous amène à définir
    \begin{equation}
        \begin{aligned}
            \varphi\colon \Hom\Big( (\eZ/n\eZ,+),(\eC^*n\cdot) \Big)&\to \gU_n \\
            g&\mapsto f(1). 
        \end{aligned}
    \end{equation}
    Remarquons que pour tout \( f\in\Hom(\eZ/n\eZ,\eC^*)\) on a bien \( f(1)^n=1\). En effet si \( [k]\in \eZ/n\eZ\), alors \( f\big( [k] \big)=f(1)^k\) et en particulier
    \begin{equation}
        f(1)^n=f([n])=f(0)=1.
    \end{equation}
    Donc \( f(1)\in \gU_n\). Le \( \varphi\) est injective parce que si \( f(1)=g(1)\) alors \( f=g\) du fait que \( f(k)=f(1)^k=g(1)^k=g(k)\).

    Nous en sommes à avoir prouvé que \( \hat{\eZ/n\eZ}\simeq \gU_n\). Il faudrait encore montrer que \( \gU_n\simeq \eZ/n\eZ\). Pour cela nous nous rappelons de la sous-section \ref{SubSechZeTuL} nous ayant raconté que le groupe \( \gU_n\) des racines de l'unité était cyclique et d'ordre \( n\). Il est donc bien isomorphe à \( \eZ/n\eZ\).

    Passons au cas où
    \begin{equation}
        G\simeq \eZ/d_1\eZ\times \eZ/d_2\eZ\times \ldots\times \eZ/n_k\eZ.
    \end{equation}
    Dans ce cas nous montrons que 
    \begin{equation}
        \begin{aligned}
            \alpha\colon \bigtimes_{i=1}^k\Hom(\eZ/d_i\eZ,\eC^*)&\to \Hom(G,\eC^*) \\
            \alpha(\chi_1,\ldots, \chi_k)(g_1,\ldots, g_k)&= \chi_1(g_1)\ldots\chi_k(g_k). 
        \end{aligned}
    \end{equation}
    Ce \( \alpha\) est injectif parce qu'en appliquant l'égalité
    \begin{equation}
        \alpha(\chi_1,\ldots, \chi_k)=\alpha(\chi'_1,\ldots, \chi'_k)
    \end{equation}
    à l'élément \( g=(9,\ldots, 1,\ldots, 0)\) alors nous trouvons \( \chi_i(1)=\chi_i'(1)\) parce que \( \chi_j(0)=1\). Du coup \( \chi_i=\chi'_i\).

    L'application \( \alpha\) est en plus surjective. En effet si \( \chi\in\Hom(G,\eC^*)\), alors nous définissons
    \begin{equation}
        \chi_i(g_i)=\chi(0,\ldots, g_i,\ldots, 0),
    \end{equation}
    et nous avons alors \( \alpha(\chi_1,\ldots, \chi_k)=\chi\).

    Nous devons encore montrer que \( \alpha\) est un homomorphisme. Si \( \chi,\chi'\in\bigtimes_{i=1}^k\Hom(\eF_{d_i},\eC^*)\), alors
    \begin{subequations}
        \begin{align}
            \alpha(\chi\chi')(g_1,\ldots, g_k)&=(\chi_1\chi'_1)(g_1)\ldots (\chi_k\chi_k')(g_k)\\
            &=\chi_1(g_1)\ldots \chi_k(g_k)\chi'_1(g_1)\ldots \chi_k'(g_k)\\
            &=\alpha(\chi)(g_1,\ldots, g_k)\alpha(\chi')(g_1,\ldots, g_k)\\
            &=\big( \alpha(\chi)\alpha(\chi') \big)(g_1,\ldots, g_k).
        \end{align}
    \end{subequations}
    Donc \( \alpha(\chi\chi')=\alpha(\chi)\alpha(\chi')\).
\end{proof}

\begin{theorem}
    Soit \( G\) un groupe abélien fini. Les groupes \( G\) et \( \hat{\hat G}\) sont isomorphes et un isomorphisme canonique est donné par \( \alpha\colon g\mapsto f_g\) donné par
    \begin{equation}
        f_g(\chi)=\chi(g).
    \end{equation}
\end{theorem}

\begin{proof}

    D'abord \( f_g\) est bien un caractère de \( \hat G\) parce que
    \begin{equation}
        f_g(\chi\chi')=(\chi\chi')(g)=\chi(g)\chi'(g)=f_g(\chi)f_g(\chi').
    \end{equation}
    Le fait que \( \alpha\) soit un homomorphisme de groupes est direct :
    \begin{equation}
        f_{gg'}(\chi)=\chi(gg')=\chi(g)\chi(g')=f_g(\chi)f_{g'}(\chi)=(f_gf_{g'})(\chi).
    \end{equation}

    D'autre part nous savon que \( G\) et \( \hat{\hat G}\) ont le même cardinal. Il suffit donc de prouver l'injectivité de \( \alpha\) pour être sûr de la bijectivité. Pour cela nous devons prouver que si \( g\neq e\) alors \( f_g\neq f_e\). Nous savons que pour tout caractère \( \chi\in \hat G\), \( f_e(\chi)=\chi(e)=1\). Donc pour tout \( g\in G\setminus\{ e \}\), nous devons trouver \( \chi\in \hat G\) tel que \( \chi(g)\neq 1\).

    En vertu de ce que nous connaissons sur la structure des groupes abéliens finis (théorème \ref{ThoRJWVJd}), nous commençons \( G=\eZ/n\eZ\) et considérons le caractère donné par \( \chi([1])= e^{2i\pi/n}\). Ce \( \chi\) est un isomorphisme entre \( G\) et \( \gU(n)\); nous n'avons \( \chi([k])=0\) que si \( [k]=[n]=[0]\). Pour rappel dans \( \eZ/n\eZ\), le neutre est \( e=0\) et non \( e=1\).

    Passons au cas général :
    \begin{equation}
        G\simeq \eZ/n_1\eZ\times \ldots\times \eZ/n_k\eZ
    \end{equation}
    Si \( g=(g_1,\ldots, g_k)\) est non nul dans \( G\), alors il existe \( i\) tel que \( g_i\neq 0\) et on prend
    \begin{equation}
        \chi(g_1,\ldots, g_k)=\chi_i(g_i)
    \end{equation}
    où \( \chi_i\) est le caractère \( \chi_i([1])= e^{2\pi i/n_i}\). Ce \( \chi\) est alors un caractère non trivial de \( G\).
    
\end{proof}

%---------------------------------------------------------------------------------------------------------------------------
\subsection{Crochet de dualité et transformée de Fourier}
%---------------------------------------------------------------------------------------------------------------------------

Si \( G\) est un groupe abélien, nous définissons le crochet de dualité entre \( G\) et \( \hat G\) par
\begin{equation}
    \begin{aligned}
        \langle ., .\rangle \colon G\times \hat G&\to \eC^* \\
        \langle g, \chi\rangle &=\chi(g). 
    \end{aligned}
\end{equation}
Notons que l'image de ce crochet n'est pas \( \eC^*\) entier, mais seulement le groupe unitaire \( \gU(n)\) où \( n\) est l'exposant\footnote{Définition \ref{DefvtSAyb}.} de \( G\).


Si \( f,g\) sont des applications de \( G\) dans \( \eC\), alors on leur associe le produit scalaire
\begin{equation}
    \langle f, g\rangle =\frac{1}{ | G | }\sum_{s\in G}f(s)\overline{ g(s) }.
\end{equation}

\begin{lemma}
    Les caractères de \( G\) forment une base orthonormée de \( \eC^G\) pour ce produit scalaire.    
\end{lemma}

\begin{proof}
    Étant donné que les \( \chi(s)\) sont des nombres complexe de module \( 1\), nous avons \( \chi(s)\overline{ \chi(s) }=1\) et par conséquent \( \langle \chi, \chi\rangle =1\).

    Si par contre \( \chi\neq\chi'\), alors il existe \( s_\in G\) tel que \( \chi(s_0)\neq \chi'(s_0)\). Dans ce cas en effectuant un changement de variable \( s\to s_0s\) dans la sommation,
    \begin{subequations}
        \begin{align}
            \langle \chi, \chi'\rangle &=\frac{1}{ | G | }\sum_{s\in G}\chi(s)\overline{ \chi'(s) }\\
            &=\frac{1}{ | G | }\sum_{s\in G}\chi(s_0s)\overline{ \chi'(s_0s) }\\
            &=\frac{1}{ | G | }\chi(s_0)\chi'(s_0)\sum_{s\in G}\chi(s)\overline{ \chi'(s) }.
        \end{align}
    \end{subequations}
    Donc nous avons trouvé
    \begin{equation}
        \langle \chi, \chi'\rangle \big( 1-\chi(s_0)\overline{ \chi'(s_0) } \big)=0.
    \end{equation}
    Mais vu que \( \chi(s_0)\neq \chi(s'_0)\), la parenthèse est non nulle (pour rappel \( \chi(s_0)\) est un complexe de module \( 1\)) et par conséquent \( \langle \chi, \chi'\rangle =0\).

    Nous déduisons immédiatement que les caractères forment une famille libre parce que si \( \sum_i\chi_i=0\) (la somme est sur tous les caractères), alors en prenant le produit scalaire avec \( \chi_k\),
    \begin{equation}
        \sum_ia_i\langle \chi_k, \chi_i\rangle =0,
    \end{equation}
    et donc \( a_k=0\).

    Les caractères forment donc un système libre orthonormé. De plus l'espace engendré à la bonne dimension parce que le cardinal de l'ensemble des caractères est la dimension (complexe) de l'espace des fonction de \( G\) dans \( \eC\) parce que, en utilisant l'isomorphisme entre \( G\) et \( \hat G\),
    \begin{equation}
        \Card\hat G=\Card(G)=\dim_{\eC}\eC^G.
    \end{equation}
    La première 
\end{proof}

Du fait que les caractères forment une base orthonormée, nous pouvons écrire, pour toute application \( f\colon G\to \eC\),
\begin{equation}    \label{EqnnsXWC}
    f=\sum_{\chi\in\hat G}\langle \chi, f\rangle \chi.
\end{equation}
À une fonction \( f\colon G\to \eC\) nous associons la \defe{transformée de Fourier}{transformée!Fourier!groupe abélien fini}\index{Fourier!transformée!groupe abélien fini}
\begin{equation}
    \begin{aligned}
        \hat f\colon \hat G&\to \eC \\
        \chi&\mapsto \langle \chi, f\rangle . 
    \end{aligned}
\end{equation}
Nous avons donc aussi une espèce de formule d'inversion
\begin{equation}
    f=\sum_{\chi\in\hat G}\hat f(\chi)\chi
\end{equation}
qui n'est qu'une réécriture de \ref{EqnnsXWC}.

%---------------------------------------------------------------------------------------------------------------------------
\subsection{Groupes non abéliens}
%---------------------------------------------------------------------------------------------------------------------------

Nous avons vu que le groupe des caractères \( \hat G\) contenait toute l'information sur un groupe abélien. Malheureusement, pour les groupes non abéliens, ça ne va pas suffire, et nous allons introduire la notion de représentations, dont les caractères seront un cas particulier de dimension un.

\begin{proposition}
    Soit \( G\) un groupe (pas spécialement abélien). Nous avons
    \begin{equation}
        \hat G\simeq\Hom\big( G/D(G),\eC^* \big).
    \end{equation}
\end{proposition}

\begin{proof}
    Ce qui fait fonctionner la preuve est le fait que su \( f\colon G\to \eC^*\) est un homomorphisme, alors \( f\) s'annule sur \( D(G)\). L'isomorphisme est
    \begin{equation}
        \begin{aligned}
            \psi\colon \hat G&\to \Hom\big( G/D(G),\eC^* \big) \\
            \psi(f)[g]&=f(g). 
        \end{aligned}
    \end{equation}
    Cette application est bien définie parce que si \( f\) est un homomorphisme,
    \begin{equation}
        f(gklk^{-1}l^{-1})=f(g).
    \end{equation}
    D'autre part \( \psi\) est un homomorphisme de groupe parce que
    \begin{equation}
        \psi(f_1f_2)[g]=(f_1f_2)(g)=f_1(g)f_2(g)=\psi(f_1)[g]\psi(f_2)[g]=\big( \psi(f_1)\psi(f_2) \big)[g].
    \end{equation}
    Pour l'injectivité de \( \psi\), soit \( f_1\) et \( f_2\) telles que \( \psi(f_1)=\psi(f_2)\). Alors pour tout \( g\in G\) nous avons
    \begin{equation}
        \psi(f_1)[g]=\psi(f_2)[g]
    \end{equation}
    et donc \( f_1(g)=f_2(g)\).

    Enfin \( \psi\) est surjective. En effet, soit \( \bar f\in\Hom\big( G/D(G),\eC^* \big)\). Alors nous obtenons \( \psi(f)=\bar f\) en posant
    \begin{equation}
        f(g)=\bar f[g].
    \end{equation}
    Il faut juste vérifier que le \( f\) ainsi défini est dans \( \hat G\), c'est à dire que \( f(g_1g_2)=f(g_1)f(g_2)\).
\end{proof}

Cette proposition nous montre que
\begin{equation}
    \hat G=\widehat{G/D(G)},
\end{equation}
alors que \( G/D(G)\) est abélien; il n'est donc pas tellement possible que \( \hat G\) contienne beaucoup d'informations intéressantes sur \( G\).

%---------------------------------------------------------------------------------------------------------------------------
\subsection{Représentations linéaires des groupes finis}
%---------------------------------------------------------------------------------------------------------------------------

Soit \( V\), un \( \eC\)-espace vectoriel de dimension finie. Une \defe{représentation}{représentation} linéaire de \( G\) dans \( V\) est un homomorphisme \( \rho\colon G\to \End(V)\). Nous notons \( (V,\rho)\) cette représentation. Voire \( \rho\) tout court si l'espace vectoriel n'est pas ambigu.

Si \( \dim V=1\), alors \( \GL(V)=\eC^*\) et les représentation sont les caractères abéliens.

\begin{example} \label{ExKUAyUD}
    Considérons le triangle équilatéral \( A,B,C\), par exemple donné par les points
    \begin{subequations}
        \begin{numcases}{}
            A=1\\
            B=(-\frac{ 1 }{2},\frac{ \sqrt{3} }{2})\\
            C=(-\frac{ 1 }{2},-\frac{ \sqrt{3} }{2})\\
        \end{numcases}
    \end{subequations}
    Dans la base (pas orthonormée) \( \{ A,B \}\) de \( \eR^2\), ces trois points sont donnés par
    \begin{equation}
        \begin{aligned}[]
            A&=\begin{pmatrix}
                1    \\ 
                0    
            \end{pmatrix}&B&=\begin{pmatrix}
                0    \\ 
                1    
            \end{pmatrix}&C&=\begin{pmatrix}
                -1    \\ 
                -1    
            \end{pmatrix}.
        \end{aligned}
    \end{equation}
    Le groupe symétrique\index{groupe!symétrique} \( S_3\) agit sur le triangle par permutation des sommets. Vues dans la base \( \{ A,B \}\), les transpositions correspondent aux matrices
    \begin{subequations}
        \begin{align}
            (A,B)&\to\begin{pmatrix}
                0    &   1    \\ 
                1    &   0    
            \end{pmatrix}\\
            (A,C)\to \begin{pmatrix}
                -1    &   0    \\ 
                -1    &   1    
            \end{pmatrix}\\
            (B,C)\to\begin{pmatrix}
                1    &   -1    \\ 
                0    &   -1    
            \end{pmatrix}.
        \end{align}
    \end{subequations}
    La permutation \( (A,B,C)\) s'écrit comme \( (A,B,C)=(A,C)(A,B)\) et on lui associe la matrice
    \begin{equation}
        (A,B,C)\to\begin{pmatrix}
            0    &   -1    \\ 
            1    &   -1    
        \end{pmatrix}.
    \end{equation}
    C'est bien le produit des matrices de \( (A,C)\) et de \( A,B\).
\end{example}

Si \( (V,\rho)\) et \( (V',\rho')\) sont deux représentations du groupe \( G\), alors nous définissons la \defe{somme directe}{somme directe (de représentations)} par \( \big( V\oplus V',\rho\oplus\rho' \big)\) donné par
\begin{equation}
    (\rho\oplus\rho')(g)=\begin{pmatrix}
        \rho(g)    &   0    \\ 
        0    &   \rho'(g)    
    \end{pmatrix}\in \GL(V\oplus V').
\end{equation}

%---------------------------------------------------------------------------------------------------------------------------
\subsection{Module}
%---------------------------------------------------------------------------------------------------------------------------

Nous considérons la \( \eC\)-algèbre \( G[\eC]\)\nomenclature[G]{\( \eC[G]\)}{combinaisons d'éléments de \( G\) à coefficients dans \( \eC\)} des combinaisons (formelles) d'éléments de \( G\) à coefficients dans \( G\), c'est à dire l'ensemble
\begin{equation}
    \eC[G]=\{ \sum_{s\in G}a_ss \}
\end{equation}
avec le produit hérité de la bilinéarité :
\begin{equation}
    \sum_{s\in G}\sum_{t\in G}a_sb_tst=\sum_s\sum_t a_sb_{s^{-1}t}t,
\end{equation}
et la somme
\begin{equation}
    (\sum_sa_ss)+\sum_tb_tt=\sum_{s\in G}(a_s+b_s)s.
\end{equation}
Le tout est une \( \eC\)-algèbre agissant sur \( V\) par
\begin{equation}
    \left( \sum_sa_ss \right)v=\sum_{s\in G}a_s\rho(s)v\in V
\end{equation}

Les sous-modules indécomposables seront les représentations irréductibles.

\begin{definition}
    La représentation \( (V,\rho)\) du groupe \( G\) est \defe{irréductible}{irréductible!représentation}\index{représentation!irréductible} si les seuls sous-espaces invariants de \( V\) sous \( \rho(G)\) sont $V$ et \( \{ 0 \}\).
\end{definition}

\begin{example}
    La représentation de \( S_3\) sur \( \eR^2\) donnée par les permutations des sommets d'un triangle équilatéral donnée dans l'exemple \ref{ExKUAyUD} est irréductible.
\end{example}

La question qui vient est de savoir si une représentation possédant des sous-espaces invariants peut être écrite comme la somme de représentations irréductibles.

\begin{proposition} \label{PropHeyoAN}  \index{représentation!irréductible}
    Soit \( (V,\rho)\) une représentation linéaire de dimension finie d'un groupe fini\footnote{La démonstration marche aussi pour les groupes compacts, mais il faudrait des intégrales.}. Si \( W_1\) est un sous-espace stable\footnote{c'est à dire si \( \rho\) n'est pas irréductible.}, alors il existe un sous-espace \( W_2\) également stable et tel que \( V=W_1\oplus W_2\).

    Toute représentation linéaire est décomposable en représentations irréductibles.
\end{proposition}

\begin{proof}
    Soit \( P\colon V\to V\) un projecteur sur \( W_1\), c'est à dire que \( P^2=P\) et \( P(V)=W_1\). Pour construire un tel projecteur, on peut par exemple prendre un supplémentaire de \( W_1\) dans \( V\) puis utiliser la décomposition\footnote{Ou encore prendre une base de \( W_1\), l'étendre en une base de \( V\) et définir \( P\) comme l'annulation des coefficients des vecteurs «complétant» la base.}. Nous considérons l'opérateur
    \begin{equation}
        P_G=\frac{1}{ | G | }\sum_{g\in G}\rho(g)\circ P\circ \rho(g)^{-1}.
    \end{equation}
    Prouvons que ce \( P_G\) est encore un projecteur. D'abord pour tout \( g\in G\) nous avons
    \begin{equation}
        \rho(g)P_G\rho(g)^{-1}=\frac{1}{ | G | }\sum_{s\in G}\rho(gs)P\rho(gs)^{-1}=P_G.
    \end{equation}
    La dernière égalité est un changement de variables dans la somme\footnote{Et c'est ça qui demande un peu de technique pour écrire la preuve dans le cas d'un groupe compact : il faut une mesure de Haar.}. Cela signifie que \( P_G\rho=\rho P_G\). Nous avons même \( P_GP=P\) parce que si \( v\in W_1\), alors 
    \begin{subequations}
        \begin{align}
            P_G(v)&=\frac{1}{ | G | }\sum_{s\in G}\rho(s)P\underbrace{\rho(s)^{-1} v}_{\in W_1}\\
            &=\frac{1}{ | G | }\sum_s\rho(s)\rho(s)^{-1} v\\
            &=v.
        \end{align}
    \end{subequations}
    Avec cela nous pouvons conclure que \( P_G^2=P_G\) parce que
    \begin{subequations}
        \begin{align}
            P_G\circ P_G&=\frac{1}{ | G | }\sum_g P_G\rho(g)P\rho(g)^{-1}\\
            &=\frac{1}{ | G | }\sum_g \rho(g)P_GP\rho(g)^{-1}\\
            &=\frac{1}{ | G | }\sum_g \rho(g)P\rho(g)^{-1}\\
            &=P_G.
        \end{align}
    \end{subequations}
    Donc \( P_G\) est un projecteur, est stable sous les conjugaisons par \( \rho(g)\) et commute avec \( \rho(g)\). Nous décomposant \( \id\) de façon évidente en
    \begin{equation}
        \id=P_G+(\id-P_G).
    \end{equation}
    Étant donné que l'opérateur \( P_G\) commute avec tous les \( \rho(g)\), les noyaux de \( P_G\) et \( \id-P_G\) sont des sous-espaces invariants.

Citer la conversation
http://forums.futura-sciences.com/mathematiques-superieur/286432-decomposition-somme-directe-dune-representation.html

\end{proof}
<++>



