% This is part of Exercices et corrections de MAT1151
% Copyright (C) 2010
%   Laurent Claessens
% See the file LICENCE.txt for copying conditions.

\documentclass[a4paper,12pt]{book}
%\documentclass[a4paper,12pt,draft]{book}

\usepackage{latexsym}
\usepackage{amsfonts}
\usepackage{amsmath}
\usepackage{amsthm}
\usepackage{amssymb}
\usepackage{bbm}

\usepackage{soul}	% For writing striked words by \st 

\usepackage{pstricks,pst-eucl,pstricks-add}


\usepackage{fancyvrb}			% For the verbatim in the bibliography.

\usepackage{cases}
\usepackage{multicol}

\usepackage{makeidx}
\usepackage[nottoc]{tocbibind}

%%%%%%%%%%%%%%%%%%%%%%%%%%
%
%   The following lines are the ones for a text in French encoded in UTF8
%
%%%%%%%%%%%%%%%%%%%%%%%%

\usepackage[utf8]{inputenc}
\usepackage[T1]{fontenc}

\usepackage{textcomp}
\usepackage{lmodern}
\usepackage[a4paper]{geometry} 
\usepackage[english,frenchb]{babel}

\usepackage{ifthen}	% ifthen is used by the package SystemeCorr
\usepackage{xstring}
\usepackage{SystemeCorr}	% Personal package for the management of exercises/corrections.
				% It has to be called after the encoding declaration because it writes ``Corrigé à la page''

\usepackage[ps2pdf]{hyperref} 		
\hypersetup{colorlinks=true,linkcolor=blue}


%%%%%%%%%%%%%%%%%%%%%%%%%%
%
%   Numbering
%
%%%%%%%%%%%%%%%%%%%%%%%%

\setcounter{tocdepth}{3}	
\setcounter{secnumdepth}{2}


%%%%%%%%%%%%%%%%%%%%%%%%%%
%
%   Math stuff.
%
%%%%%%%%%%%%%%%%%%%%%%%%

\newcommand{\eR}{\mathbbm{R}}
\newcommand{\eZ}{\mathbbm{Z}}
\newcommand{\eC}{\mathbbm{C}}
\newcommand{\eD}{\mathbbm{D}}
\newcommand{\eQ}{\mathbbm{Q}}
\newcommand{\eN}{\mathbbm{N}}
\newcommand{\efrac}[2]{\frac{ \displaystyle #1 }{\displaystyle #2 }}

\newcommand{\mF}{\mathcal{F}}
\newcommand{\mG}{\mathcal{G}}


\newcommand{\mtu}{\mathbbm{1}}  			% unit matrix


\DeclareMathOperator{\id}{id}
\DeclareMathOperator{\Sym}{Sym}
\DeclareMathOperator{\dom}{dom}
\DeclareMathOperator{\arctg}{Arctg}
\DeclareMathOperator{\cotg}{cotg}
\DeclareMathOperator{\arcsinh}{arcsinh}


%%%%%%%%%%%%%%%%%%%%%%%%%%
%
%   theorems and related stuff.
%
%%%%%%%%%%%%%%%%%%%%%%%%


\newcounter{numtho}
\newcounter{numloiphyz}



\newtheoremstyle{mes_tho}%
		{9pt}{9pt}%
		{\itshape}%
		{}%
		{\bfseries}{.}%
		{\newline}%
		{}%

\newtheoremstyle{mes_exemples}%
		{9pt}{9pt}%
		{}%
		{}%
		{\bfseries}{.}%
		{\newline}%
		{}%


\theoremstyle{mes_exemples}	\newtheorem{exemple}[numtho]{Exemple}
				\newtheorem{remark}[numtho]{Remarque}

				\newtheorem{amusement}[numtho]{Amusement}
				\newtheorem{erreur}[numtho]{Error}
				\newtheorem{probleme}[numtho]{\fbox{\bf Probl\`emes et choses \`a faire}}
				%\newtheorem{exercice}{Exercice}			% Les exercices ne se numérotent pas avec les autres, pour que les références soient plus faciles à suivre dans la partie corrigée.


\theoremstyle{mes_tho}
			\newtheorem{theoreme}[numtho]{Théoreme}
			\newtheorem{lemma}[numtho]{Lemme}
			\newtheorem{proposition}[numtho]{Proposition}
			\newtheorem{corollary}[numtho]{Corollaire}
			\newtheorem{theorem}[numtho]{Théorème}
			\newtheorem{definition}[numtho]{Définition}
			\newtheorem{loiphyz}[numloiphyz]{Loi numéro}




% The numbering of the equations change in the corrections.
\renewcommand{\theequation}{\thechapter.\arabic{equation}}

\renewcommand{\theenumi}{\alph{enumi}}
\renewcommand{\theexercice}{\arabic{section}.\arabic{exercice}}
\makeatletter
\@addtoreset{exercice}{section}
\makeatother

\newcommand{\defe}[2]{\textbf{#1}\index{#2}}
