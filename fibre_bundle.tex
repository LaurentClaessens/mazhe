\section{Vector bundle}
%++++++++++++++++++++++

Let $M$ be a smooth manifold. A \defe{$V$-vector bundle}{vector!bundle}\index{bundle!vector} of rank $r$ on $M$ is a smooth manifold $F$ and a smooth projection $\dpt{p}{F}{M}$ such that

\begin{itemize}
\item for any $x\in M$, the fiber $F_x:=p^{-1}(x)$ is a vector space of dimension $r$ on the same field that $V$ (let's say $\eK=\eR$ or $\eC$).
\item for any $x\in M$, there exists an open neighbourhood $\mU$ of $x$ and a ``chart diffeomorphism``{} $\dpt{\phi}{p^{-1}(\mU)}{\mU\times V}$ such that for any $l\in p^{-1}(y)$,
   \begin{itemize}
      \item $\phi(l)=(y,\phi_y(l))$
      \item $\dpt{\phi_y}{E_y}{V}$ is a vector space isomorphism.
   \end{itemize}
\end{itemize}

The pair $(\mU,\phi)$ is a \emph{local trivialization}; $M$ is the \emph{base space}; $F$, the \emph{total space}, $p$ the \emph{projection} and $r$, the \emph{rank} of the bundle. The denominations of total and base spaces will also be used in the same way for principal bundles.

We will sometimes use charts diffeomorphism $\dpt{\phi}{\mU\times V}{p^{-1}(\mU)}$ instead of $\dpt{\phi}{p^{-1}(\mU)}{\mU\times V}$. Since they are diffeomorphism, this difference don't affect anything.

\subsection{Transition functions}
%--------------------------------

The trivializations will be denoted by Greek indices: $\mU_{\alpha}$, $\phi_{\alpha}$,\ldots The symbol $\mU_{\alpha}{}_{\beta}$ naturally denotes $\mU_{\alpha}\cap\mU_{\beta}$. If we consider two local trivializations $(\mU_{\alpha},\phi_{\alpha})$ and $(\mU_{\beta},\phi_{\beta})$, we have to look at $\dpt{\phi_{\alpha}\circ\phi_{\beta}^{-1}}{\mU_{\alpha}{}_{\beta}\times\eK^r}{\mU_{\alpha}{}_{\beta}\times\eK^r}$. We define the \defe{transition functions}{transition function} $\dpt{g_{\alpha}{}_{\beta}}{\mU_{\alpha}{}_{\beta}}{GL(r,\eK)}$ by
\begin{equation}
\phi_{\alpha}\circ\phi_{\beta}^{-1}(x,v)=(x,g\bab(x)v). 
\end{equation}
These functions take their values in $GL(r,\eK)$ because $\dpt{\phi_y}{E_y}{V}$ is a vector space isomorphism. Since $(\phi_{\alpha}\circ\phi_{\beta})^{-1}=\phi_{\beta}\circ\phi_{\alpha}^{-1}$, it is clear that $g\bab(x)=g_{\alpha\beta}(x)^{-1}$.

If $x\in\mU_{\alpha\beta\gamma}=\mU_{\alpha}\cap\mU_{\beta}\cap\mU\bgamma$, we have $\phi_{\alpha}\circ\phi\bgamma^{-1}(x,v)=(x,g_{\alpha\gamma}(x)v)$, but also $\phi_{\alpha}\circ\phi\bgamma^{-1}=\phi_{\alpha}\circ\phi_{\beta}^{-1}\phi_{\beta}\circ\phi\bgamma^{-1}$, then
\begin{equation}
  (x,g_{\alpha\gamma}(x)v)=(\phi_{\alpha}\circ\phi_{\beta}^{-1})(x,g_{\beta\gamma}(x)v)
               =(x,g\bab(x)_{\beta\gamma}(x)v).
\end{equation}
Thus $g_{\alpha\gamma}(x)=g\bab(x)g_{\beta\gamma}(x)$. So, as linear maps, we have
\begin{equation}\label{eq:g_compat}
  g\bab\circ g_{\alpha\gamma}\circ g_{\gamma\alpha}=\mtu.
\end{equation}

\subsection{Inverse construction}\label{subsec:inv_g}
%-------------------------------------------

Let us consider a manifold $M$, an open covering $\{\mU_{\alpha}:\alpha\in I\}$ and some functions $\dpt{g\bab}{\mU\bab}{GL(r,\eK)}$ which fulfill relations \eqref{eq:g_compat}. We will build a vector bundle $E\stackrel{p}{\longrightarrow}M$ whose transition functions are the $g_{\alpha\beta}$'s. Let $\tilde{E}$ be the disjoint union
\[
  \tilde{E}=\bigsqcup_{\alpha\in I}\mU_{\alpha}\times\eK^r,
\]
i.e. triples of the form $(x,v,\alpha)\in M\times\eK^r\times I$ with the condition that $x\in\mU_{\alpha}$. We define an equivalence relation on $\tilde{E}$ by $(x,v,\alpha)\sim(y,w,\beta)$ if and only if $x=y$ and $w=g\bab(x)v$. Next, we define $E=\tilde{E}/\sim$ and $\dpt{\omega}{\tilde{E}}{E}$, the canonical projection. The projection $\dpt{p}{E}{M}$ is naturally defined by $p([x,v,\alpha])=x$. The chart diffeomorphism is $\dpt{\varphi_{\alpha}}{\mU_{\alpha}\times\eK^r}{p^{-1}(\mU_{\alpha})}$, 
\[
  \varphi_{\alpha}(x,v)=\omega(x,v,\alpha).
\]
Now we have to prove that $E$ endowed with the $\varphi_{\alpha}$'s is a vector bundle.

First we prove that $\varphi_{\alpha}$ is surjective. For this we remark that a general element in $p^{-1}(\mU_{\alpha})$ can be written under the form $\omega(x,v,\alpha)$ with $x\in\mU\bab$. But
\begin{equation}
\begin{split}
  \varphi_{\alpha}(x,g\bab(x)w)&=\omega(x,g\bab(x)w,\alpha)\\
                         &=\omega(x,g_{\alpha\beta}(x)g\bab(x)w,\beta)\\
			 &=\omega(x,w\beta),
\end{split}
\end{equation}
then $\varphi_{\alpha}$ is surjective. Now we suppose $\varphi_{\alpha}(x,v)=\varphi_{\alpha}(y,w)$. Then $\omega(x,v,\alpha)=\omega(y,w,\alpha)$ and $x=y$, $w=g_{\alpha\alpha}v$ which immediately gives $v=w$. Then $\varphi_{\alpha}$ is injective.

Finally, we have 
\begin{equation}
  (\varphi\alpha\circ\varphi_{\beta}^{-1})(\omega(x,v,\alpha))=\varphi_{\alpha}(x,g_{\alpha\beta}(x)v)
                                                   =\omega(x,g_{\alpha\beta}(x)v,\alpha),
\end{equation}
which proves that the maps $g$ are the transition functions of the vector bundle $E$.

\subsection{Equivalence of vector bundle}
%----------------------------------------

Let $E\stackrel{p}{\longrightarrow}M$ and $F\stackrel{p'}{\longrightarrow}M$ be two vector bundles on $M$. They are \defe{equivalent}{equivalence!of vector bundle} if there exists a smooth diffeomorphism $\dpt{f}{E}{F}$ such that 

\begin{itemize}
\item $p'\circ f=p$,
\item $\dpt{f|_{E_x}}{E_x}{F_x}$ is a vector space isomorphism.
\end{itemize}

Let $E$ and $F$ be two equivalent vector bundles, $\{\mU_{\alpha}\tq \alpha\in I\}$, an open covering which trivialize $E$ and $F$ in the same time and $\phi^E_{\alpha}$, $\phi^F_{\alpha}$ the corresponding trivializations. A map $\dpt{f}{E}{F}$ reads ``in the trivialization''\ as $\dpt{\phi^F_{\alpha}\circ f|_{p^{-1}(\mu_{\alpha})}\circ\phi^E{}^{-1}_{\alpha}}{\mU_{\alpha}\times\eK^r}{\mU_{\alpha}\times\eK^r}$ and defines a map $\dpt{\lambda_{\alpha}}{\mU_{\alpha}}{GL(r,\eK)}$ by
\begin{equation}
(\phi^F_{\alpha}\circ f|_{p^{-1}(\mu_{\alpha})}\circ\phi^E{}^{-1}_{\alpha})(x,v)=(x,\lambda_{\alpha}(x)v).
\end{equation}
If we denote by $g^E$ the transition functions for $E$ (and $g^F$ for $F$),
\[
 \phi^F_{\alpha}\circ\phi^F_{\beta}{}^{-1}= (\phi^F_{\alpha}\circ f\circ\phi_{\alpha}^E{}^{-1})\circ
                                        (\phi^E_{\alpha}\circ\phi^E_{\beta}{}^{-1})\circ
					(\phi^E_{\beta}\circ f^{-1}\circ\phi_{\beta}^E{}^{-1}),
\]
so that 
\begin{equation}\label{eq:g_l_g_l}
  g\bab^F(x)=\lambda_{\alpha}(x) g^E\bab(x)\lambda(x)^{-1}.
\end{equation}

Once again we have an inverse construction. We consider a vector bundle $E$ on $M$ with transition functions $g^E$ and some maps $\dpt{\lambda_{\alpha}}{\mU_{\alpha}}{GL(r,\eK)}$; then we define $g^F\bab(x)$ by equation \eqref{eq:g_l_g_l}. 

From subsection \ref{subsec:inv_g}, one can construct a vector bundle $F$ on $M$ whose transition functions are these $g^F$. With the trivializations $\phi^F$ of $F$, one can define $\dpt{f}{E}{F}$ by
\[
(\phi^F_{\alpha}\circ f\circ\phi^E_{\alpha}{}^{-1})(x,v)=(x,\lambda_{\alpha}(x)v).
\]

When a basis space $B$ is given, we denote by $\Vect(B)$ the set of isomorphism classes of vector bundles over $B$. In the complex case, we denote it by $\Vect_{\eC}(B)$.

\begin{proposition}
Any vector bundle over $\eR^n$ is trivial.
\end{proposition}

\begin{proof}
Let $\dpt{p}{F}{M}$ be a vector bundle on $M=\eR^n$ and $\{\mU_{\alpha}\}$ be covering of $\eR^n$ by local trivializations. Now consider a partition of unity\index{partition of unity} related to the covering $\mU_{\alpha}$: a set of functions $\dpt{f_{\alpha}}{M}{\eR}$ such that
\begin{itemize}
\item $f_{\alpha}>0$,
\item $\forall x\in M$, one can find a neighbourhood of $x$ in which only a \emph{finite} number of $f_{\alpha}$ is non zero,
\item $\forall x\in M$, $\sum_{\alpha} f_{\alpha}(x)=1$.
\item $f_{\alpha}=0$ outside of $\mU_{\alpha}$.
\end{itemize}
Using that partition of unity, we build the trivialization function $\dpt{f}{F}{\eR^n\times V}$ by $f(l)=(x,\sum_{\alpha} f_{\alpha}(x)\phi_{\alpha x}(l))$.
\end{proof}

The following two propositions have some importance in K-theory.
\begin{proposition}		\label{PropEoplusEprimetriv}
Let $\pi\colon E\to B$ be a complex vector bundle over a basis compact, Hausdorff, connected basis $B$. Then there exists a vector bundle $E'$ such that $E\oplus E'$ is trivial.
\end{proposition}

\begin{proposition}		\label{PropmapfEEsun}
Let $f\colon A\to B$ be a map between the topological spaces $A$ and $B$, and consider a vector bundle $\pi\colon E\to B$. Then there exists one and only one vector bundle $\pi'\colon E'\to A$ and a map $f'\colon E'\to E$ such that $f'|_{E'_x}\colon E'_x\to E_{f(x)}$ is an isomorphism. The vector bundle $E'$ is unique up to isomorphism.
\end{proposition}
Proofs can be found in \cite{VB_and_K}. Let us denote by $f^*(E)$ the function given by proposition \ref{PropmapfEEsun}. It satisfies the following properties
\begin{equation}		\label{EqPropfstarEVect}
\begin{split}
	(fg)^*(E)		&=g^*\big( f^*(E) \big)\\
	\id^*(E)		&=E\\
	f^*(E_1\oplus E_2)	&=f^*(E_1)\oplus f^*(E_2)\\
	f^*(E_1\otimes E_2)	&=f^*(E_1)\otimes f^*(E_2).
\end{split}
\end{equation}



\subsection{Sections of vector bundle}
%-------------------------------------

A \defe{section}{section!of vector bundle} of the vector bundle $p\colon E\to M$ is a smooth map $\dpt{s}{M}{E}$ such that $p\circ s=\id|_M$. The set of all the sections is denoted by $\Gamma^{\infty}(M)$ or simply $\Gamma(E)$.\nomenclature{$\Gamma(E)$}{Space of sections of the vector bundle $E$}

If $(\mU_{\alpha},\phi_{\alpha})$ is a local trivialization, one can describe the section $s$ by a function $\dpt{s_{\alpha}}{\mU_{\alpha}}{V}$ defined by $\phi_{\alpha}(s(x))=(x,s_{\alpha}(x))$, or equivalently by
\[
s(x)=\phi_{\alpha}^{-1}(x,s_{\alpha}(x)).
\]
As usual when we define such a local quantity, we have to ask ourself how are related $s_{\alpha}$ and $s_{\beta}$ on $\mU_{\alpha}\cap\mU_{\beta}$. The best is $s_{\alpha}=s_{\beta}$, but most of the time it is not. Here, we compute
\[
  \phi_{\beta}\circ\phi_{\alpha}^{-1}\circ\phi_{\alpha}(s(s))=(x,g_{\alpha\beta}(x)s_{\alpha}(x)),
\]
which is obviously also equal to $(x,s_{\beta}(x))$. Then
\begin{equation}\label{eq:tr_sec}
s_{\beta}(x)=g_{\alpha\beta}(x)s_{\alpha}(x)
\end{equation}
without summation.

\section{Vector valued differential forms}	\label{SecVectValFiffFor}
%+++++++++++++++++++++++++++++++++++++++++++

Let $E$ be a vector bundle over $M$. A \defe{$E$-valued $p$-form}{vector-valued differential form}\index{differential!form!vector-valued} is a section
\[ 
  e\in\Gamma\big( E\otimes\Wedge^pT^*M \big).
\]
We denote by $\Omega(M,E)=\Gamma\big( E\otimes\Wedge^pT^*M \big)$\nomenclature[D]{$\Omega(M,E)$}{the set of $E$-valued differential forms} the set of $E$-valued differential forms. An element of $\Omega^1(M,E)=\Gamma\big( E\otimes\Wedge T^*M\big)$ always reads  $\sum_is_i\otimes\omega_i$ for some sections $s_i$ and usual differential forms $\omega_i$.

A form of $\Omega^p(M,E)$ can be seen as a fiber morphism $\underbrace{TM\otimes\cdots\otimes TM}_{\text{$p$ times}}\to E$ by associating 
\[ 
  s\otimes\omega(X_1,\cdots,X_p)=s(x)\omega(X_1,\cdots,X_p)\in E_x
\]
to the element $(s\otimes \omega)\in\Omega^p(M,E)$. There exists a wedge product between vector-valued forms. If $e\in\Omega^p(M,E_1)$ and $f\in\Omega^q(M,E_2)$, then we define $e\wedge f\in\Omega^{p+q}(M,E_1\otimes E_2)$ by
\begin{equation}	\label{EqDefwedgevecor}
(e\wedge f)(v_1,\cdots,v_{p+q})=\frac{1}{ p!q! }\sum_{\pi\in S_{p+q}}(-1)^{\pi} e(v_{\pi(1)},\cdots v_{\pi(p)})\otimes f(v_{\pi(p+1)},\cdots,v_{\pi(p+q)})\in E_1\otimes E_2.
\end{equation}
where $(-1)^{\pi}$ stands for the sign of the permutation $\pi$. For example when $e$, $f\in \Omega^1(M,E)$, we have
\[ 
  (e\wedge f)(X,Y)=e(X)\otimes f(Y)-e(Y)\otimes f(X)\in E\otimes E.
\]

When $M$ is a differentiable manifold, the \defe{fundamental $1$-form}{fundamental!$1$-form} is the element $\theta\in\Omega(M,TM)$ such that
\[ 
  \iota(X)\theta=X
\]
for every $X\in \Gamma(TM)$. 

\section{Lie algebra valued differential forms}	\label{SecLiaAlgformval}
%+++++++++++++++++++++++++++++++++++++++++++++++++

An important particular case of vector valued forms is given by Lie algebra valued forms. That case appears for example in the connection theory over principal bundle\footnote{So in Maxwell and other gauge field theories.}. If $\omega$ and $\eta$ are elements of $\Omega^1(M,\mG)$ for some Lie algebra $\mG$, we define
\[ 
		(\omega\wedge\eta)(X,Y)=\omega(X)\otimes\eta(Y)-\omega(Y)\otimes\eta(X).
\]
Combining with the Lie bracket, we define\nomenclature[D]{$[\omega\wedge\eta]$}{Combination of the wedge and the bracket in the case of Lie algebra-valued forms}
\begin{equation}	\label{EqDefomegawedgebeta}
[\omega\wedge\eta](X,Y):=[\omega(X),\eta(Y)]-[\omega(Y),\eta(X)].
\end{equation}
Using the proposition \ref{prop:XY_YX}, we often implicitly transforms the tensor product into a product \eqref{eq:yGyGb} and put
\begin{equation}	\label{EqAbuswesgeomom}
  (\omega\wedge\omega)(X,Y)=[\omega(X),\omega(Y)].
\end{equation}
Let us point out the fact that that kind of formula only holds for a ``wedge square'', but not for a general product $\omega\wedge\eta$. Remark that for $\omega\in\Omega^1(M,\mG)$ and $\beta\in\Omega^2(M,\mG)$, a simple computation of definition \eqref{EqDefwedgevecor} yields
\begin{equation}	\label{EqomwedgebetaXYZ}
(\omega\wedge\beta)(X,Y,Z)=\omega(X)\otimes\beta(Y,Z)-\omega(Y)\otimes\beta(X,Z)+\omega(Z)\otimes\beta(X,Y),
\end{equation}
so that, using the same trick as for equation \eqref{EqAbuswesgeomom}, we find
\[ 
  (\omega\wedge\beta-\beta\wedge\omega)(X,Y,Z)=[\omega(X),\beta(Y,Z)]-[\omega(Y),\beta(X,Z)]+[\omega(Z),\beta(X,Y)].
\]
But that expression is exactly what we find by exchanging the tensor product by Lie bracket in expression \eqref{EqomwedgebetaXYZ}. So we define
\begin{equation}	\label{EqDefCrochwedgedeux}
[\omega\wedge\beta]=\omega\wedge\beta-\beta\wedge\omega
\end{equation}
when $\omega\in\Omega^1(M,\mG)$ and $\beta\in\Omega^2(M,\mG)$. The reader should remark that this is what one would expect from generalisation of definition \eqref{EqDefomegawedgebeta}. 
\section{Principal bundle}
%+++++++++++++++++++++++++

Let $M$ be a manifold and $G$, a Lie group whose unit is denoted by $e$. A $G$-\defe{principal bundle}{principal!bundle}\index{bundle!principal} on $M$ is a smooth manifold $P$, a smooth map $\dpt{\pi}{P}{M}$ and a right action of $G$ on $P$ denoted by $\xi\cdot g$ with $g\in G$ and $\xi\in P$ such that

\begin{itemize}
\item $\pi(\xi\cdot g)=\pi(\xi)$,
\item $\forall \xi\in\pi^{-1}(x)$, $\pi^{-1}(x)=\{\xi\cdot g\tq g\in G\}\simeq G$,
\item $\forall x\in M$, there exists a neighbourhood $\mU_{\alpha}$ of $x$ in $M$, a diffeomorphism $\dpt{\phi_{\alpha}}{\pi^{-1}(\mU_{\alpha})}{\mU_{\alpha}\times G}$ and a diffeomorphism $\dpt{\phi_{\alpha x}}{P}{G}$ such that

\begin{itemize}
\item $\phi_{\alpha}(\xi)=(x,\phi_{\alpha x}(\xi))$,
\item $\phi_{\alpha x}(\xi\cdot g)=\phi_{\alpha x}(\xi)\cdot g$.
\end{itemize}
\end{itemize}
The group $G$ is often called the \defe{structure group}{structure!group}. We suppose that the action is effective. We will sometimes use the notation $P(G,M)$ to precise that $P$ is a principal bundle over $M$ with structure group $G$.

\begin{lemma}\label{lem:phixh}
The map $\phi_{\alpha}^{-1}$ fulfills
\[
  \phi\alpha^{-1}(x,h)\cdot g=\phi_{\alpha}^{-1}(x,hg).
\]
\end{lemma}

\begin{proof}

From the definition of a principal bundle, any $\xi\in P$ can be written under the form $\xi=\phi_{\alpha}^{-1}(x,\phi_{\alpha x}(\xi))$ with $\phi_x$ satisfying $\phi_x(\xi\cdot h)=\phi_x(\xi)h$ for a certain function $\dpt{\phi_x}{P}{G}$.  We consider in particular $\xi=\phi_{\alpha}^{-1}(x,h)\cdot g$. Then $\xi\cdot g^{-1}=\phi_{\alpha}^{-1}(x,h)$. But $\xi\cdot g^{-1}=\phi_{\alpha}^{-1}(x,\phi_{\alpha x}(\xi)g^{-1})$, then $h=\phi{\alpha_x}(\xi)g^{-1}$ and $\phi_{\alpha x}(\xi)=hg$. So we have
\[
\xi=\phi_{\alpha}^{-1}(x,h)\cdot g=\phi_{\alpha}^{-1}(x,\phi_{\alpha x}(\xi))=\phi_{\alpha}^{-1}(x,hg).
\]

\end{proof}

Let 
\[
   R=\{ (x,y)\in P\times P\tq x=y\cdot g\,\textrm{ for a certain $g\in G$}  \}.
\]

\begin{proposition}
The function $\dpt{u}{R}{G}$ defined by the condition
\[
  p\cdot u(p,q)=q.
\]
is differentiable.
\end{proposition}

\begin{proof}
Let $\mU$ be an open subset of $M$ and $\dpt{\sigma}{\mU}{P}$, a section. We consider a differentiable map $\dpt{\rho}{\pi^{-1}(\mU)}{G}$ such that $\rho(\xi\cdot g)=\rho(\xi)\cdot g$ and $\rho(\sigma(x))=e$. Such a map is given by
\[
   \rho(\xi)=\phi_x(\sigma(x))^{-1}\phi_x(\xi)
\]
where $x=\pi(\xi)$. We naturally define $R_{\mU}=R\cap( \pi^{-1}(\mU)\times\pi^{-1}(\mU) )$ and we pick $(\xi,\eta)\in R_{\mU}$. Let $s\in G$ be the one such that $\xi\cdot s=\eta$, so that $\rho(\xi)\cdot s=\rho(\eta)$. Then the restriction of $u$ to $R_{\mU}$ is given by $u(\xi,\eta)=\rho(\xi)^{-1}\rho(\eta)$ which makes $u|_{\mU}$ differentiable. Since this reasoning can be made on every chart open $\mU$, $u$ is differentiable everywhere on $P$.
\end{proof}

The following is a corollary of Leibnitz rule.
\begin{corollary}  \label{cor_PrincLeib}
If $P$ is a $G$-principal bundle and $v$, $a$ are curves in $P$ and $G$ respectively, we can consider the curve $u(t)=v(t)a(t)$. We have:
\[
         \dsdd{u(t)}{t}{0}=\dsdd{v(t)a(0)}{t}{0}+\dsdd{v(0)a(t)}{t}{0}.
\]
\end{corollary}
 The proof is direct. This result is often written as
\begin{equation}
               \dot{u}_t=\dot{v}_ta_t+v_t\dot{a}_t.
\end{equation}
A main application is 
\begin{equation}\label{eq:rdotht}
  \Dsdd{ r\cdot h(t) }{t}{0}=\Dsdd{r\cdot e^{th'(0)}}{t}{0}.
\end{equation}

\subsection{Transition functions}
%--------------------------------


Let $(\mU_{\alpha},\phi_{\alpha})$ be a local trivialization of $P$. This induces transition functions $\dpt{g\bab}{\mU_{\alpha}\cap\mU_{\beta}}{G}$ defined by
\begin{equation}
	\begin{aligned}
		\phi_{\alpha}\circ\phi_{\beta}^{-1}\colon \mU_{\alpha}\cap\mU_{\beta}\times G&\to \mU_{\alpha}\cap\mU_{\beta}\times G \\
		(x,a)&\mapsto (x,g\bab(x)a).\label{eq:transi_princ} 
	\end{aligned}
\end{equation}
Clearly, $g_{\alpha\alpha}=e$ and $g\bab g_{\alpha\beta}=e$ on $\mU_{\alpha}\cap\mU_{\beta}$. Then the triviality
\[
  \phi_{\alpha}\circ\phi_{\beta}^{-1}\circ\phi_{\beta}\circ\phi\bgamma^{-1}\circ\phi\bgamma\circ\phi_{\alpha}^{-1}=\id
\]
implies the compatibility conditions
\begin{equation}
g\bab g_{\beta\gamma} g_{\gamma\alpha}=e
\end{equation}
on $\mU_{\alpha}\cap\mU_{\beta}\cap\mU\bgamma$.

There is an inverse construction. Let $\{\mU_{\alpha}\tq\alpha\in I\}$ be an open covering of $M$ and $\dpt{g\bab}{\mU_{\alpha}\cap\mU_{\beta}}{G}$ a family of functions such that $g_{\alpha\alpha}=e$, $g\bab g_{\alpha\beta}=e$ on $\mU_{\alpha}\cap\mU_{\beta}$ and $g\bab g_{\beta\gamma}g_{\gamma\alpha}=e$ on $\mU_{\alpha}\cap\mU_{\beta}\cap\mU\bgamma$. Then the following construction gives a $G$-principal bundle whose transition functions are the $g\bab$'s.

\begin{itemize}
\item $\tilde{P}=\bigsqcup_{\alpha\in I}\mU_{\alpha}\times G$  (disjoint union),
\item if $(x,a)\in\mU_{\alpha}\times G$ and $(y,b)\in\mU_{\beta}\times G$, then $(x,a)\sim(y,b)$ if and only if $x=y$ and $b=g\bab(x)a$, 
\item $\dpt{\pi}{\tilde{P}}{M}$ is defined by $\pi[(x,a)]=x$ where $[(x,a)]$ is the class of $(x,a)$ for~$\sim$,
\item the action is defined by $[(x,a)]\cdot g=[(x,ag)]$.
\end{itemize}

\begin{theorem}
Let $G$ be a Lie group; $M$, a differentiable manifold; $\{\mU_{\alpha}\}_{\alpha\in I}$, an open covering of $M$ and some functions $\dpt{\varphi\bab}{\mU_{\alpha}\cap\mU_{\beta}}{G}$ such that $\varphi\bab(x)=\varphi_{\alpha\gamma}(x)\varphi_{\gamma\beta}(x)$. Then there exists a principal bundle $P$ whose transition functions are the $\varphi_{\alpha}$'s for the covering $\{\mU_{\alpha}\}_{\alpha\in I}$.
\end{theorem}

\begin{proof}
We consider the topological space 
\begin{equation}
    E=\bigcup_{\alpha\in I}(G\times\mU_{\alpha}\times I)
\end{equation}
where we put the discrete topology on $I$. Each $G\times\mU_{\alpha}\times\{\alpha\}$ is a manifold. Thus $E$ has a structure of differentiable manifold induced from the one of $G\times M$. We consider on $E$ the equivalence relation given by the following subset of $E\times E$:
\[
   R=\left\{\big(  (g,x,\alpha),(h,y,\beta)\big)\in E\times E\tq y=x\text{ and }  h=\varphi_{\alpha\beta}(x)g \right\}.
\]
We will show that $P=E/R$ has a structure of principal bundle. We begin by defining an action of $G$ on $P$ by
\[
  [ (g,x,\alpha)\cdot h ]=[ (gh,x,\alpha) ].
\]
In order to see that this definition is correct, let us consider $[g',x,\beta]=[g,x,\alpha]$. From the definition of the equivalence class, $g'=\varphi_{\alpha\beta}(x)g$. Then $[(g',x,\beta)]\cdot h=[(\varphi_{\alpha\beta}(g)gh,x,\beta)]$, and the form of $R$ shows that this is well $[(gh,x,\alpha)]$. Since the map $(g,h)\to gh$ is differentiable on $G$, the so defined action is a differentiable action of $G$ on $P$ and $G$ is a transformation group on $P$\quext{Faut voir comment ça correspond à la définition de l'autre texte.}.

If $[(g,x,\alpha)]=[(gh,x,\alpha)]$, then $gh=\varphi_{\alpha\alpha}g=g$ and $h=e$. So the action is effective. 

Now we consider the quotient $P/G$. A typical element is
\[
   \overline{ (s,x,i) }=\{ [s,x,i]\cdot g\tq g\in G \}.
\]
The projection $\dpt{\pi}{P}{M}$, $[(s,x,\alpha)]\to x$ is well defined and we can consider $\dpt{\varphi}{P/G}{M}$, $\varphi\overline{(s,x,\alpha)}=x$. It provides a bijection between $P/G$ and $M$. So we can identify $P/G$ and $M$. Now we are going to show that $P$ endowed with the projection $\dpt{\pi}{P}{X}$ is a principal bundle. 

We consider the map 
		\begin{equation}
		\begin{aligned}
			h_{\alpha} \colon G\times\mU_{\alpha} &\to P\
			(g,x)&\mapsto \omega(g,x,\alpha)
		\end{aligned}
	\end{equation}	
%
where $\dpt{\omega}{E}{P=E/R}$ is the canonical projection. Since 
\[
  (\pi\circ h_{\alpha})(g,x)=(\pi\circ\omega)(g,x,\alpha)=\pi[(g,x,\alpha)]=x,
\]
the map $h_{\alpha}$ actually is $\dpt{h_{\alpha}}{G\times\mU_{\alpha}}{\pi^{-1}(\mU_{\alpha})}$. In order to see that $h_{\alpha}$ is surjective on $\pi^{-1}(\mU_{\alpha})$, let us take a general element of $\pi^{-1}(\mU_{\alpha})$ under the form $\omega(g,x,\beta)$ with $x\in\mU_{\alpha}\cap\mU_{\beta}$. Then $(g,x,\beta)\in[ (\varphi\bab(x)g,x,\alpha) ]$ and therefore $\omega(g,x,\beta)=h_{\alpha}(\varphi\bab(x)g,x)$. For the injectivity, remark that $\omega(g,x,\beta)=\omega(h,y,\alpha)$ implies $x=y$ and $h=\varphi_{\beta\beta}(x)g=g$. In particular, $h_{\alpha}(g,x)=h_{\alpha}(h,y)$ implies $x=y$ and $g=h$.

Now we will prove that the inverse of $h_{\alpha}$ is continuous. For this we consider an open set $\Omega\subset G\times\mU_{\alpha}$ and we have to show that $h_{\alpha}(\Omega)$ is open in $\pi^{-1}(\mU_{\alpha})$. 

We recall the \defe{quotient topology}{topology!quotient}: if $A$ is a topological space with an equivalence relation $\sim$ and the canonical projection $\dpt{\varphi}{A}{A/\sim}$, then $V\subset A/\sim$ is open if and only if $\varphi^{-1}(V)\subset A$ is open. So in our case, we have to check the openness of $V=\omega^{-1}( h_{\alpha}(\Omega) )$ in $E$. We consider the open covering 
\[
  \{ G\times\mU_{\alpha}\times\{\alpha\} \}_{\alpha\in I}
\]
of $E$ and we will show that the intersection of $V$ with any of these open set is open. We have to show that 
$\omega^{-1}\big(  h_{\alpha}(\Omega)\cap (G\times\mU_{\alpha}\times\{\beta\})  \big)$ is open for any $\beta\in I$. For this, we define a map $\dpt{\alpha}{G\times(\mU_{\alpha}\cap\mU_{\beta})\times\{\beta\}}{G\times\mU_{\alpha}}$ by
\begin{equation}
  \alpha_{\beta}(g,x,\beta)=(\varphi\bab(x)g,x)
\end{equation}
which is continuous. The set $(h_{\alpha}\circ\alpha_{\beta})^{-1}(h_{\alpha}(\Omega))=\alpha^{-1}_{\beta}(\Omega)$ is open because $h_{\alpha}\circ\alpha_{\beta}$ is the restriction of $\omega$ to $G\times (\mU_{\alpha}\cap\mU_{\beta})\times\{\beta\}$. Then $h_{\alpha}$ is an homeomorphism from $G\times\mU_{\alpha}$ tp $\pi^{-1}(\mU_{\alpha})$. Since it is build from differentiable functions, it is moreover a diffeomorphism.

So we have a chart system $\{ (h_{\alpha},\mU_{\alpha}) \}_{\alpha\in I}$ where $h_{\alpha}$ fulfils the ``good'' properties with respect to $\pi$. It remains to be proved that the $\varphi\bab$'s are the transition functions and that $\pi^{-1}(\pi(\xi))=\xi\cdot G$ for every $\xi\in P$. We begin by the latter. For $\xi=[(g,x,\alpha)]$, $\pi(\xi)=x$ and we have to study the set
\[
  \pi^{-1}(x)=\{ [(h,x,\beta)]\tq h\in G,\beta\in I \}.
\]
Clearly, $[(h,x,\beta)]\cdot G\subset\pi^{-1}(x)$. The fact that there is nothing else than $[(h,x,\beta)]\cdot G$ in $\pi^{-1}(x)$ is seen by
\[
  [h,x,\beta]=[\varphi\bab(x)g,x,\alpha]\in[(h,x,\alpha)]\cdot G.
\]

In order to check the change of charts, let us consider $g'=h_{\beta,x}^{-1}\circ h_{\alpha,x}(g)$ where
\begin{equation}
  h_{\alpha,x}(g)=h_{\alpha}(g,x)=\omega(g,x,\alpha).
\end{equation}
The fact that $h_{\beta}(g',x)=g_{\alpha}(g,x)$ concludes the proof. To see this fact, remark that $h_{\beta,x}(h^{-1}_{\beta,x}\circ h_{\alpha,x}(g))=h_{\alpha,x}(g)$, so that $h_{\alpha}(g',x)=h_{\alpha}(g,x)$ implies $\omega(g',x,\beta)=\omega(g,x,\alpha)$ which proves that $g'=\varphi\bab(g)$.
\end{proof}

The \defe{trivial bundle}{trivial!principal bundle} is simply $P=M\times G$ and $\pi(x,g)=x$ with the action $(x,a)\cdot g=(x,ag)$.

\subsection{Morphisms and such\texorpdfstring{\ldots}{...}}
%----------------------------------

An \defe{homomorphism}{homomorphism!of principal bundle} between $P(G,M)$ and $P'(G',M')$ is a differentiable map $\dpt{h}{P}{P'}$ such that $\forall \xi\in P,g\in G$, 
\begin{equation}\label{eq:def_princ_homo}
   h(\xi\cdot g)=h(\xi)\cdot h_G(g)
\end{equation}
where $\dpt{h_G}{G}{G'}$ is a Lie group homomorphism. From the definition, $h$ maps a fiber to only one fiber, but it is not specially surjective on any fiber. So $h$ induces a homomorphism $\dpt{h_M}{M}{M'}$ such that $\pi'\circ h=h_M\circ\pi$.

An \defe{isomorphism}{isomorphism!of principal bundle} is a homomorphism $\dpt{g}{P(G,M)}{P'(G',M')}$ such that 

\begin{itemize}
\item $h_P$ is a diffeomorphism $P\to P'$,
\item $h_G$ is a Lie group homomorphism $G\to G'$, and
\item $h_M$ is a diffeomorphism $M\to M'$.
\end{itemize}

A principal bundle is \defe{trivial}{trivial!principal bundle} if one can find an isomorphism $\dpt{h}{G\times M}{P}$ such that $\pi\circ h=\id\circ\pr_2$, i.e. the following diagram commutes:
\begin{equation}\label{diag:princ_triv}
\xymatrix{ G\times M \ar[d]^{\pr_2} \ar[r]^h & P\ar[d]^{\pi} \\M \ar[r] _{\id}&M}
\end{equation}
We say that $P$ is \defe{locally trivial}{locally!trivial!principal bundle} if for every $x\in M$, there exists an open neighbourhood $\mU$ in $M$ such that $\pi^{-1}(\mU)$ endowed with the induced structure of principal bundle is trivial.

\subsection{Frame bundle: first}\label{pg:frame_bundle}
%--------------------------------

In the ideas, the building of a vector bundle is just to put a vector space on each point of the base manifold. A principal bundle is to put something on which a group acts on each point. If you have a vector bundle on a manifold, you can consider, on each point $x\in M$, the set of all the basis of the fiber $E_x$ over $x$. The group $GL(r,\eK)$ naturally acts on this set which becomes a candidate to be a $GL(r,\eK)$-principal bundle.

More formally, we consider a vector bundle $\dptvb{F}{p}{M}$, and for each $x$, the set of the basis of the vector space $F_x=p^{-1}(x)$. We define
\[
  P=\bigcup_{x\in M}(\textrm{basis of $F_x$}).
\]
We naturally consider the projection $\dpt{\pi}{P}{M}$, $\pi(b_x)=x$ if $b_x$ is a basis of $F_x$.

Let $\dpt{\phi^F_{\alpha}}{p^{-1}(\mU_{\alpha})}{\mU_{\alpha}\times\eK^r}$ be a local trivialization of $F$, and $\{\overline{e}_1,\ldots,\overline{e}_r\}$, the canonical basis of $\eK^r$. We naturally define
\[
  \ovS_{\alpha i}(x)=\phi^F_{\alpha}{}^{-1}(x,\overline{e}_i).
\]
The set $\{\ovS_{\alpha 1}(x),\ldots,\ovS_{\alpha r}(x)\}$ is a ``reference''{} basis of $F_x$ with respect to the trivialization $\phi_{\alpha}$. If we choose another basis $\{\ovv_1,\ldots\ovv_r\}$ of $F_x$, we can find a matrix $A\in GL(r,\eK)$ such that $\ovv_k=A^l_k\ovS_{\alpha l}(x)$. This gives a bijection
\begin{equation}
	\begin{aligned}
		\phi_{\alpha}^P\colon \pi^{-1}(\mU_{\alpha})&\to \mU_{\alpha}\times GL(r,\eK) \\
		(\ovv_1,\ldots,\ovv_r)&\mapsto (x,A). 
	\end{aligned}
\end{equation}
One can give to $P$ a $GL(r,\eK)$-principal bundle structure such that the $\phi_{\alpha}^P$ are diffeomorphism.

Let $(\mU_{\alpha},\phi_{\alpha}^F)$ be a local trivialization of  $F$ and $\dpt{g\bab^F}{\mU_{\alpha} \cap\mU_{\beta}}{GL(r,\eK)}$. In this case, $(\mU_{\alpha},\phi_{\alpha}^P)$ is a trivialization of $P$ whose transition function is $g\bab^P=g\bab^F$. Indeed
\[
  \phi_{\alpha}^P\circ\phi_{\beta}^P{}^{-1}(x,A)=\phi_{\alpha}^P(\{\ovv_1,\ldots,\ovv_r\})
\]
where $\ovv_s=(\phi^F_{\beta})^{-1}(x,A^l_s\overline{e}_l)$. In order to see it, recall that $\ovv_s=A^l_s\ovS_{\alpha l}(x)$ and that $\phi_{\alpha}^F{}^{-1}(x,\overline{e}_s)=\ovS_{\alpha s}(x)$. Then
\[
   \ovv_s=(\phi_{\beta}^F)^{-1}(x,A^l_s\overline{e}_l)=A^l_s\ovS_{\alpha s}(x).
\]
On the other hand, from the definition of $\phi_{\beta}^P$, the basis $(\phi_{\beta}^P)^{-1}(x,A)$ is the one obtained by applying $A$ on $S$. With all this,
\begin{equation}
\begin{split}
  \phi_{\alpha}^P\circ(\phi_{\beta}^P)^{-1}(x,A)&=\phi_{\alpha}^P\{  (\phi_{\beta}^F)^{-1}(x,A^l_s\overline{e}_l)\}_{s=1,\ldots r}\\
                 &=\phi_{\alpha}^P\{(\phi_{\alpha}^F)^{-1}\circ(\phi_{\alpha}^E\circ\phi_{\beta}^F{}^{-1})(x,A^l_s\overline{e}_l)  \}_{s=1,\ldots r}\\
		 &=\phi_{\alpha}^P\{  (\phi^E_{\alpha})^{-1}(x,g\bab^F(x)^s_iA^l_s\overline{e}_l   ) \}_{i=1,\ldots r}\\
		 &=(x,g^F\bab(x)A).
\end{split}
\end{equation}
The last product $g^F\bab(x)A$ is a matricial product.

\subsection{Frame bundle: second} \label{subsec_frbundle}
%------------------------------

\subsubsection{Basis}\label{subsubsecframebundle}
%////////////////////////////////////////////////

If $M$ is a $m$-dimensional manifold, a \defe{frame}{basis!of $T_xM$} of $T_xM$ is an isomorphism $\dpt{b}{\eR^m}{T_xM}$. In our purpose, we will always deal with (pseudo)Riemannian manifold. So, the tangents spaces $T_xM$ comes with a metric, and we ask a frame to be isometric. In other words, we ask $b$ to be an isometry from $(\eR^m,\cdot)$ to $(T_xM,g_x)$, where the dot denotes the (pseudo)euclidian product on $\eR^m$. Such a frame is given by a base point $x$ of $M$ and a matrix $S$ in $\SO(g_x)$:
\begin{equation}
                 \label{r1504e1}b(v)=(Sv)^i(\partial_i)_x,
\end{equation}
if the vector $v$ is written as $v=v^i\oui$ in the canonical orthogonal frame $\{\oui\}$ of $\eR^m$ and $\SO(g_x)$ is the set of the $m\times m$ matrix $A$ such that $A^tg_xA=g_x$.

This frame intuitively corresponds to the basis of $T_xM$ (see as a ``true''\ vector space) that we would have written by $\{Se_i\}_x$ if $e_i=\dsd{}{x^i}$. In order to follow this idea, we will effectively denote by $\baz{S}{x}$ the map $\dpt{b}{\eR^m}{T_xM}$ given by \eqref{r1504e1}.

We will often write the frame $b$ as $\baz{b}{x}$, making no differences in notation between the $b$ of $\SO(M)$ and the $b$ of $\SO(g_x)$ which implement it.

\begin{remark}
One has to distinguish a \emph{frame} and a \emph{basis}: a basis is only a free and generator set while a frame can be interpreted as an ordered basis.
\end{remark}


\subsubsection{Construction}
%////////////////////////////

We just saw how to build a frame bundle over a manifold. One can get another expression of the frame bundle when we express a basis of $T_xM$ by means of an isomorphism between $\eR^n$ and $T_xM$. If $M$ is a $n$-dimensional manifold, a \defe{frame}{frame} at $x$ is an ordered basis  
\[
   b=(\overline{ b }_1,\ldots,\overline{ b }_n)
\]
of $T_xM$. It is clear that any frame defines an isomorphism (linear bijective map)
\begin{equation}
\begin{aligned}
	\tilde{b}\colon\eR^n&\to T_xM \\ 
    e_i&\mapsto \overline{ e_i }
\end{aligned}
\end{equation}
where $\{e_i\}$ is the canonical basis of $\eR^n$. It is also clear that any isomorphism gives rise to a frame. Then we see a frame of $M$ at $x$ as an isomorphism $\dpt{\tilde{b}}{\eR^n}{T_xM}$. Let $B(M)_x$ be the of all the frames of $M$ at $x$; we define
\[
   B(M)=\bigcup_{x\in M}B(M)_x.
\]
For all $b\in B(M)_x$, we define $p_B(b)=x$ and the action $B(M)\times GL(n,\eR)\to B(M)$ by $b\cdot g=(\overline{ b }'_1,\ldots,\overline{ b }'_n)$ where
\begin{equation}
  \overline{ b }'_j=\overline{ b }_i\bghd{g}{i}{j}.
\end{equation}
It is easy to see that $\dpt{\widetilde{b\cdot g}=\tilde{b}\circ g}{\eR^n}{T_xM}$. So we can give to
\begin{equation}
\xymatrix{
    GL(n,\eR)  \ar@{~>}[r] & B(M) \ar[d]^{p_B}\\& M
  }
\end{equation}
a structure of principal bundle\footnote{Much more details and proofs are given in \cite{Naber}.}. If $(\mU_{\alpha},\varphi_{\alpha})$ is a local coordinate chart on $M$, we define
		\begin{equation}
		\begin{aligned}
			\tilde{\varphi} \colon p_B^{-1}(\mU_{\alpha}) &\to \varphi_{\alpha}(\mU_{\alpha})\times GL(n,\eR)\
			b&\mapsto (\varphi_{\alpha}(x),A(b))
		\end{aligned}
	\end{equation}	
where $A(b)\in GL(n,\eR)$ is defined by the condition $\overline{ b }_j=\bghd{A}{j}{i}\partial_i|_x$. The matrix $A(b)$ is the one which transforms the canonical basis (in the trivialization $\varphi_{\alpha}$) into $b\in B(M)_x$. That's for the principal bundle structure. 

The manifold structure of $B(M)$ is given by $\dpt{\Phi_{\alpha}}{p_B^{-1}(\mU_{\alpha})}{\mU_{\alpha}\times GL(\eR)}$,
\begin{equation}
\begin{split}
  \Phi(b)&=(\varphi_{\alpha}^{-1}\times \id|_{GL(n,\eR)})\circ\tilde{\varphi}(b)\\
         &=(x,A(b))\\
         &=(p_B(b),A(b)).
\end{split}
\end{equation}
It fulfils $A(b\cdot g)=A(b)\cdot g$. A section $\dpt{s}{\mU_{\alpha}}{B(M)}$ is sometimes called a \defe{moving frame}{moving frame}\index{frame!moving} over $\mU_{\alpha}$.

Frame bundle over $\eR^2$ is given as example in page \pageref{Pg_exempleRdeux}
% 
% \subsection{Space-time}\label{subsec:space_time}
% %----------------------
% 
% 
% We say that the basis $\{\gb_0,\ldots,\gb_3\}$ of $T_xM$ is oriented, time oriented and orthonormal if $g(\gb_i,\gb_j)=\eta_{ij}$ (pointwise) and if $\gb_0$ is time-like and future directed. All this is devoted to build the frame bundle
% 
% 
% \[
% \xymatrix{
%     \Lpf  \ar@{~>}[r] & L(M) \ar[d]^{p_L} \\
%     &M.
%   }
% \]
% 
% From discussion in \autoref{subsec:sym_nature}, we know that, in physics, the relevant group is \emph{not} the Lorentz group $\Lpf$, but $\SLdc$ . So we want to build a $\SLdc$-principal bundle which ``fit''{} as deeply as possible the bundle $L(M)$. It is done with a \defe{spin structure}{spin!structure} which is a principal bundle
% 
% \[
% \xymatrix{
%     \SLdc  \ar@{~>}[r] & S(M) \ar[d]^{p_S} \\
%     &M.
%   }
% \]
% with a map $\dpt{\lambda}{S(M)}{L(M)}$ such that $p_L\circ\lambda=p_S$ and $\lambda(\xi\cdot g)=\lambda(\xi)\cdot\mSpin(g)$. See \autoref{sec:spin_str} and \autoref{sec:incl_Lorentz}.

\subsection{Sections of principal bundle}
%----------------------------------------

A \defe{section}{section!of principal bundle} of a $G$-principal bundle is a smooth map $\dpt{s}{M}{P}$ such that $s(x)\in\pi^{-1}(x)$ for any $x\in M$. A trivialization $\phi_{\alpha}^P$ $P$ on $\mU_{\alpha}$ defines a section of $P$ over $\mU_{\alpha}$ by
\[
   \sigma_{\alpha}(x)=(\phi_{\alpha}^P)^{-1}(x,e)
\]
where $e$ is the neutral of the group. In the inverse sense, we have the following:

\begin{proposition}
If $\dpt{\sigma_{\alpha}}{\mU_{\alpha}}{P}$ is local section of $P$ over $\mU_{\alpha}\subset M$, then the definition $\phi_{\alpha}^P(\xi)=(x,a)$ if $\xi=\sigma_{\alpha}(x)\cdot a$ is a local trivialization.
\end{proposition}

\begin{proof}
The function $\phi_{\alpha}^P$ is well defined because $\xi\in\pi^{-1}(\mU_{\alpha})$ implies the existence of a $x\in\mU_{\alpha}$ such that $\xi\in\pi^{-1}(x)=\{\xi\cdot g\}\simeq G$. For this $x$, there exists a $g\in G$ such that $\xi=\sigma_{\alpha}(x)\cdot g$. 

Now we prove that the couple $(x,a)$ is unique in the sense that $s_{\alpha}(x)\cdot a=\sigma_{\alpha}(y)\cdot b$ implies $(x,a)=(y,b)$. The left hand side belongs to $\pi^{-1}(x)$ while the right one belongs to $\pi^{-1}(y)$. Then $x=y$. The condition $\pi^{-1}(x)\simeq G$ imposes the unicity of the $g$ making $\xi=\eta\cdot g$ for each couple, $\xi,\eta\in\pi^{-1}(x)$.
\end{proof}

If $\sigma$ and $\sigma'$ are two sections of the same principal bundle $P$, then there exists a differentiable map $\dpt{f}{M}{G}$ such that $\sigma'(x)=\sigma(x)\cdot f(x)$. So all the sections can be deduced from only one and multiplication by such a $f$.

\begin{theorem}
If $\dpt{\pi}{P(G,M)}{M}$ is a principal bundle, then the four following propositions are equivalent:
\begin{enumerate}
\item\label{enuymai} $P$ is trivial,
\item\label{enuymaii} $P$ has a global section,
\item\label{enuymaiii} there exists a differentiable map $\dpt{\gamma}{P}{G}$ such that $\gamma(\xi\cdot g)=g^{-1}\gamma(\xi)$ for all $\xi\in P$ and $g\in G$,
\item\label{enuymaiv} there exists a differentiable map $\dpt{\rho}{P}{G}$ such that $\rho(\xi\cdot g)=\rho(\xi)g$.
\end{enumerate}
\label{ThoYPrincBTrivSect}
\end{theorem}

\begin{proof}
\subdem{\ref{enuymai}$\Rightarrow$ \ref{enuymaii}} 
The diagram \eqref{diag:princ_triv} commutes and 
		\begin{equation}
		\begin{aligned}
			\tau \colon M &\to G\times M\
			x&\mapsto (e,x)
		\end{aligned}
	\end{equation}	
is a local section of $G\times M$. From it we build the following global section of $P$:
		\begin{equation}
		\begin{aligned}
			\sigma \colon M &\to P\
			x&\mapsto h(e,x).
		\end{aligned}
	\end{equation}	
 This is injective because $\pi\circ h=\pr_2$ and differentiable because this is a composition of $\xdp{x}{(e,x)}$ and $\xdp{(g,x)}{h(g,x)}$.
\subdem{\ref{enuymaii}$\Rightarrow$ \ref{enuymai}}
The principal bundle $P$ admits a global section $\dpt{\sigma}{M}{P}$. From it, we can build the differentiable map
		\begin{equation}
		\begin{aligned}
			h \colon G\times M &\to P\
			(g,x)&\mapsto \sigma(x)\cdot g
		\end{aligned}
	\end{equation}	
which satisfies $h(gh,x)=h(g,x)\cdot h$ and $\pi\circ(g,x)=x$. First we show that $h$ is a fiber homomorphism and an isomorphism between $P$ and $G\times M$ so that $P$ is trivial. For this remark that
\[
  g(gh,x)=g(g,x)\cdot h=\sigma(x)\cdot gh,
\]
hence equation \eqref{eq:def_princ_homo} reduces to $h( (g,x)\cdot h )=h(g,x)\cdot h_G(h)$ which is true with $h_G=\id$. Moreover $\dpt{h}{G\times M}{P}$ is bijective because $\sigma(\pi(\xi))$ belongs to the fiber of $\xi\in P$, therefore there is one and only one $\gamma(\xi)=u(\xi,\sigma(\pi(\xi)))$ such that $\xi\cdot\gamma(\xi)=(\sigma\circ\pi)\xi$. The inverse map is 
		\begin{equation}
		\begin{aligned}
			\theta \colon P &\to G\times M\
			\xi&\mapsto (\gamma(\xi),\pi(\xi))
		\end{aligned}
	\end{equation}	
which is differentiable because $\gamma$ and $\pi$ are. So far we see that $h$ and $h^{-1}$ are differentiable. Then $h$ is an isomorphism between $P$ and $G\times M$.

\subdem{\ref{enuymaii}$\Rightarrow$ \ref{enuymaiii}}
Let $\sigma$ be the global section and define
		\begin{equation}
		\begin{aligned}
			\gamma \colon P &\to G\
			\xi&\mapsto u(\xi,(\sigma\circ\pi)\xi)
		\end{aligned}
	\end{equation}	
where $\dpt{u}{R}{G}$ is the map defined by the condition $\xi\cdot(\xi,\eta)=\eta$. The map $\gamma$ is differentiable and we have to prove that $\gamma(\xi\cdot g)=g^{-1}\gamma(\xi)$. Since $\xi\cdot \gamma(\xi)=\sigma\circ\pi(\xi)$,
\[
  \gamma(\xi\cdot g)=u(\xi\cdot g,(\sigma\circ\pi)(\xi\cdot g))=u(\xi\cdot g,(\sigma\circ\pi)(\xi)).
\]
But $(\xi\cdot g)(g^{-1}\gamma(\xi))=\xi\cdot\gamma(\xi)=x$. So $\gamma(\xi\cdot g)=u(\xi\cdot g,x)$. Thus $\gamma(\xi\cdot g)=g^{-1}\gamma(\xi)$.

\subdem{\ref{enuymaiii}$\Rightarrow$ \ref{enuymaii}}
The given map $\gamma$ fulfils $\xi\cdot g\gamma(\xi\cdot g)=\xi\cdot(\xi)$, so 
		\begin{equation}
		\begin{aligned}
			\varphi \colon P &\to P\
			\xi&\mapsto \xi\cdot(\xi)
		\end{aligned}
	\end{equation}	
is just function of the class of $\xi$, thus we have a section $\dpt{\sigma'}{P/G}{P}$, but we know that $P/G$ and $M$ are isomorphic.

\subdem{\ref{enuymaiii}$\Rightarrow$ \ref{enuymaiv}}
Let us define $\dpt{\rho}{P}{G}$ by $\rho=J\circ\gamma$ with $J(g)=g^{-1}$, thus $\rho(\xi)=\gamma(\xi)^{-1}$ and
\[
  \rho(\xi\cdot g)=\gamma(\xi\cdot g)^{-1}=(g^{-1}\gamma(\xi))^{-1}=\gamma(\xi)^{-1} g=\rho(\xi)g.
\]
\subdem{\ref{enuymaiv}$\Rightarrow$ \ref{enuymaiii}} The proof is just the same with $\rho=J\circ\rho$.
\end{proof}

\begin{definition}  
A section $\psi\in\Gamma(P,TP)$ is \defe{$G$-equivariant}{equivariant!vector field on principal bundle} when
\[ 
  d\tau_{g}\psi(\xi)=\psi(\xi\cdot g).
\]
\label{DefEqVectPrinc}
\end{definition}
Be careful: this \emph{does not} define equivariant sections of the principal bundle.
\subsection{Equivalence of principal bundle}
%-------------------------------------------

Two principal bundles $\dpt{\pi}{P}{M}$ and $\dpt{\pi'}{P'}{M}$ are \defe{equivalent}{equivalence!of principal bundle} if there exists a diffeomorphism $\dpt{\varphi}{P}{P'}$ such that 

\begin{itemize}
\item  $\pi'\circ\varphi=\pi$
\item $\varphi(\xi\cdot g)=\varphi(\xi)\cdot g$.
\end{itemize}

If $\{\mU_{\alpha}\}_{\alpha\in I}$ is an open covering of $M$ on which we have trivializations $\phi_{\alpha}$ of $P$ and $\psi_{\alpha}$ of $P'$, the diffeomorphism $\varphi$ induces some functions $\dpt{\lambda}{\mU_{\alpha}}{G}$ by setting
\[
   (\phi_{\alpha}\circ\varphi^{-1}\circ\psi_{\alpha}^{-1})(x,a)=(x,\lambda_{\alpha}(x)a).
\]
This definition works because from the definitions of principal bundle and equivalence, one sees that $(\phi_{\alpha}\circ\varphi^{-1}\circ\psi_{\alpha}^{-1})(x,\cdot)=(x,\cdot)$.

\subsubsection{Transition functions}
%///////////////////////////////////

We have some transition functions for $P$ and $P'$ given by equations
\[
 \begin{aligned}
   (\phi_{\alpha}\circ\phi_{\beta}^{-1})(x,g)&=(x,g\bab(x)g)\\
   (\psi_{\alpha}\circ\psi_{\beta}^{-1})(x,g)&=(x,g'\bab(x)g).
 \end{aligned}
\]
Now, we want to know what is $g'\bab$ in function of $g\bab$. First remark that $(\psi_{\alpha}\circ\varphi\circ\phi_{\alpha}^{-1})(x,a)=(x,\lambda_{\alpha}(x)^{-1})a$, and next, compute
\begin{equation}
\begin{split}
  (x,g\bab(x)a)a&=(\psi_{\alpha}\circ\varphi\circ\phi_{\beta}^{-1}\circ\phi_{\beta}\circ\varphi^{-1}\circ\psi_{\beta}^{-1})(x,a)\\
                &=(\psi_{\alpha}\circ\varphi\circ\phi_{\beta}^{-1})(x,\lambda_{\beta}(x)a)\\
		&=(\psi_{\alpha}\circ\varphi\circ\phi_{\alpha}^{-1}\circ\phi_{\alpha}\circ\phi_{\beta}^{-1})(x,\lambda_{\beta}(x)a)\\
		&=(x,\lambda_{\alpha}(x)^{-1} g\bab(x)\lambda_{\beta}(x)a).
\end{split}
\end{equation}
Then 
\begin{equation}
    g\bab(x)=\lambda_{\alpha}(x)^{-1} g\bab(x)\lambda_{\beta}.
\end{equation}
One can show that if two principal bundle have transition functions whose fulfill this condition, they are equivalent. A $G$-principal bundle is \defe{trivial}{trivial!principal bundle} if it is equivalent to the one given by $\dpt{\pi_1}{M\times G}{M}$.

\subsection{Reduction of the structural group}
%---------------------------------------------

We say that a principal bundle $P(G,M)$ is \defe{reducible}{reducible!principal bundle} when there exists a principal bundle $P'(H,M)$ such that 

\begin{itemize}
\item $H$ is a subgroup of $G$,
\item there exists an homeomorphism $\dpt{h}{P'}{P}$ such that $\dpt{h_G}{H}{G}$ is an injective homomorphism.
\end{itemize}

In this case we say that $G$ is reducible to $H$ and that $P'$ is a reduced principal bundle.

\begin{theorem}
If $P$ is a principal bundle over $M$, the structural group $G$ is reducible to the Lie subgroup $H$ if and only if there exists an open covering $\{ \mU_i \}_{i\in I}$ of $M$ and transition functions $\varphi_{ij}$ taking their values in $H$.
\end{theorem}
\begin{proof}
No proof.
\end{proof}
The following comes from \cite{Dieu4}. Let us consider the principal bundle
\begin{equation}
\xymatrix{%
   G \ar@{~>}[r]		&	P\ar[d]^{\pi_P}\\
   				&	M
 }
\end{equation}
and $H$, a closed subgroup of $G$. We denote by $j\colon H\to G$ the inclusion map. The principal bundle
\begin{equation}
\xymatrix{%
   H \ar@{~>}[r]		&	Q\ar[d]^{\pi_Q}\\
   				&	M
 }
\end{equation}
is a \defe{reduction}{reduction of a principal bundle} of $P$ to the group $H$ if there exists a map $u\colon Q\to P$ such that $\pi_P\circ u=\pi_Q$ and $u(\xi\cdot h)=u(\xi)\cdot j(h)$. In this case, $u$ is an embedding\quext{plongement} of $Q$ in $P$ and the image is a closed submanifold of $P$.   

Let $M$ be a $n$-dimensional manifold and $B(M)$ be its frame bundle. This is a $\GL(n,\eR)$-principal bundle. If $G$ is a closed subgroup\footnote{Typically $\SO(p,q)$ or $\SO_0(p,q)$.} of $\GL(n,\eR)$, a \defe{$G$-structure}{$G$-structure} is a reduction of $B(M)$ to $G$.

%---------------------------------------------------------------------------------------------------------------------------
					\subsection{Density}
%---------------------------------------------------------------------------------------------------------------------------

A \defe{density}{density} on a $d$-dimensional manifold $M$ is a section of the principal bundle whose fiber $P_x$ over $x\in M$ is the space of homogeneous non vanishing maps
\begin{equation}
\rho\colon \Wedge^dT_xM\to \eR^*_+
\end{equation}
such that $\rho(\lambda v)=| \lambda |\rho(v)$ for every $\lambda\in\eR$ and $v\in\Wedge^d T_xM$.


\section{Associated bundle}  \index{bundle!associated}
%++++++++++++++++++++++++++

Let $\dpt{\pi}{P}{M}$ be a $G$-principal bundle and $\dpt{\rho}{G}{GL(V)}$, a representation of $G$ on a vector space $V$ (on $\eK=\eR$ or $\eC$) of dimension $r$.

The associated bundle $\dptvb{E=P\times_{\rho} V}{p}{M}$ is defined as following. On $P\times V$, we consider the equivalence relation 
\[
   (\xi,v)\sim(\xi\cdot g,\rho(g^{-1})v)
\]
for $g\in G$, $\xi\in P$ and $v\in V$. Then we define

\begin{itemize}
\item $E=P\times_{\rho} V:=P\times V/\sim$,
\item $p[(\xi,v)]=\pi(\xi)$
\end{itemize}
where $[(\xi,v)]$ is the class of $(\xi,v)$ in $P\times V$.

If $\phi^P_{\alpha}(\xi)=(\pi(\xi),a(\xi))$ is a  trivialization of $P$ on $\mU_{\alpha}$, then
\begin{equation}\label{eq:triv_P_E}
\phi^E[(\xi,v)]=(\pi(\xi),\rho(a)v) 
\end{equation}
is a trivialization of $E$.

In order to see that it is a good definition, let us consider $(\eta,w)\sim(\xi,v)$. It immediately gives the existence of a $g\in G$ such that $\eta=\xi\cdot g$ and $w=\rho(g^{-1})v$. Then $\phi^E[ (\xi\cdot g,\rho(g^{-1})v) ]=( \pi(\xi\cdot g),\rho(b)\rho(g^{-1})v )$.  From the definition of $\phi^E$, the vector $b$ is given by $\phi^P(\xi\cdot g)=( \pi(\xi\cdot g),b )$, and the definition of a principal bundle gives $b=\phi_{\pi(\xi)}(\xi\cdot g)=\phi_{\pi(\xi)}(\xi)\cdot g=ag$. The fact that $\rho$ is a homomorphism makes $\rho(ag)\rho(g^{-1})=\rho(a)v$ and $\phi^E$ is well defined.

Let $G$ be a Lie group, $\rho$ a representation of $G$ on $V$ and $M$, a manifold. We consider $\dptvb{P=M\times G}{\pr_1}{M}$, the trivial $G$-principal bundle on $M$. Then $\dptvb{E=P\times_{\rho}V}{p}{M}$ is \defe{trivial}{trivial!principal bundle}, i.e. we can build a $\dpt{\varphi}{P\times_{\rho} V}{M\times V}$ such that $\pr_1\circ\varphi=p$. It is rather easy: we define
\[
  \varphi\big[ \big((x,g),v\big) \big]=(x,\rho(g)v).
\]
It is easy to see that $(\pr_1\circ\varphi)[ (x,g),v ]=x$ and $p[(x,g),v]=\pr_1(x,g)=x$.

\subsection{Transition functions}
%--------------------------------

\begin{proposition}
Let $(\mU_{\alpha},\phi_{\alpha}^P)$ be a trivialization of $\dptvb{P}{\pi}{M}$ whose transition functions are $\dpt{g\bab}{\mU_{\alpha}\cap\mU_{\beta}}{G}$. Then $(\mU_{\alpha},\phi^E_{\alpha})$ given by \eqref{eq:triv_P_E} is a local trivialization of $\dptvb{E}{p}{M}$ whose transition functions $\dpt{g\bab^E}{\mU_{\alpha}\cap\mU_{\beta}}{GL(\dim V,\eK)}$ are given by
\[
   g\bab^E(x)=\rho(g\bab^P(x)).
\]

\end{proposition}

\begin{proof}
If we write $a:=\phi^E_{_{\beta} x}(\pi^{-1}(x))$, we have $\phi_{\beta}^P(\pi^{-1}(x))=(x,a)$ and $\phi_{\alpha}^E\circ(\phi_{\beta}^E)^{-1}(x,v)=\phi^E_{\alpha}[ (\pi^{-1}(x),\rho(a)^{-1} v) ]$. So,
\begin{equation}
\begin{split}
\phi_{\alpha}^E[ (\pi^{-1}(x),\rho(a)^{-1} v) ]&=\Big(x, 
                            \rho\big(\phi_{\alpha x}(\pi^{-1}(x))\big)\rho\big(\phi_{\beta x}(\pi^{-1}(x))\big)^{-1} v   \Big)\\
			&=\Big(   x,\rho\big(\phi_{\alpha x}( \pi^{-1}(x) )\phi_{\beta x}( \pi^{-1}(x) )        \big)    \Big).
\end{split}
\end{equation}
Then 
\begin{equation}
  g\bab^E=\rho\Big(  \phi_{\alpha x}( \pi^{-1}(x) )\phi_{\beta x}\big( \pi^{-1}(x) \big)  \Big)\\
         =\rho(g^P\bab(x)).
\end{equation}

\end{proof}

\subsection{Sections on associated bundle}  \label{sec_fnequiv}
%-----------------------------------------

\subsubsection{Equivariant functions}

We consider a bundle $\dptvb{E=P\times_{\rho} V}{p}{M}$ associated with the principal bundle $\dptvb{P}{\pi}{M}$ and a section $\dpt{\psi}{M}{E}$. 
\[
 \xymatrix{ P \ar[rd]_{\displaystyle\pi^P} &&
 E=P\times_{\rho} V \ar[ld]^{\displaystyle\pi^E} \\ & M }
\]
A \defe{section}{section!of an associated bundle} of $E$ is a map $\dpt{\psi}{M}{E}$ such that $\pi^E\circ\psi=id_M$. We define the function $\dpt{\hpsi}{P}{V}$ by
\begin{equation}\label{eq:equiv_psi}
   \psi(\pi(\xi))=[\xi,\hpsi(\xi)].
\end{equation}
Let us see the condition under which this equation well defines $\hpsi$. First, remark that a $\psi$ defined by this equation is a section because $p[\xi,v]=\pi(\xi)$, so that $(p\circ\psi)(\pi(\xi))=\pi(\xi)$. Now, consider a $\eta$ such that $\pi(\eta)=\pi(\xi)$. Then there exists a $g\in G$ for which $\eta\cdot g=\xi$. For any $g$ and for this one in particular,
\[
  \psi(\pi(\eta))=[\eta,\hpsi(\eta)]=[\eta\cdot g,\rho(g^{-1})\hpsi(\eta)].
\]
Then equation \eqref{eq:equiv_psi} defines $\hpsi$ from $\psi$ if and only if 
\begin{equation}\label{eq:equiv_psi_b}
  \hpsi(\xi\cdot g)=\rho(g^{-1})\hpsi(\xi).
\end{equation}
This condition is called the \defe{equivariance}{equivariant} of $\hpsi$. Reciprocally, any equivariant function $\hpsi$ defines a section of $E=P\times_{\rho} V$.

If $\eta=\xi\cdot g=\chi\cdot k$, one define a sum
\begin{equation}\label{eq:def:som_E}
  [\xi,v]+[\chi,w]=[\eta,\rho(g)v+\rho(k)w].
\end{equation}
If $\dpt{\psi,\eta}{M}{E}$ are two sections defined by the equivariant functions $\dpt{\hpsi,\hat\eta}{P}{V}$, then the section $\psi+\eta$ is defined by the equivariant function $\hat\psi+\hat \eta$.

\subsubsection{For the endomorphism of sections of \texorpdfstring{$E$}{E}}\label{equivendo}
%////////////////////////////////////////////////////

Let us now make a step backward, and take $A$ in $\End{\Gamma(E)}$. We will now see that $A$ defines (and is defined by) an equivariant function $\dpt{\hat A}{P}{\End{V}}$. Let $\dpt{\psi}{M}{E}$ be in $\Gamma(E)$. If $\psi(x)=[\xi,v]$, we define the new section $A\psi$ by
\[
         (A\psi)(x)=[\xi,\hat A(\xi)v]=[\xi,\hat A(\xi)\hat\psi(\xi)].
\]
In order for $A\psi$ to be well defined, the function $\hat A$ must satisfy
\begin{equation}
     \hat A(\xi\cdot g)=\rho(g^{-1})\hat A(\xi)\rho(g)                 \label{equivA}
\end{equation}
for all $g$ in $G$.

\subsubsection{Local expressions}
%////////////////////////////////

We consider a local trivialization $\dpt{\phi^P_{\alpha}}{\pi^{-1}(\mU_{\alpha})}{\mU_{\alpha}\times G}$ of $P$ on $\mU_{\alpha}$ and the corresponding section $\dpt{\sigma_{\alpha}}{\mU_{\alpha}}{P}$ given by 
\[
\sigma_{\alpha}(x)=(\phi^P_{\alpha})^{-1}(x,e).
\]
We saw at page \pageref{eq:triv_P_E} that a trivialization of $P$ gives a trivialization of the associated bundle $E=P\times_{\rho} V$; the definition is
\begin{equation}
  \phi^E_{\alpha}[(\xi,v)]=( \pi(\xi),\rho(a)v )
\end{equation}
if $\phi_{\alpha}^P(\xi)=(\pi(\xi),a)$. With $\xi=\sigma_{\alpha}(x)$, we find
\begin{equation}
   \phi^E_{\alpha}[(\sigma_{\alpha}(x),v)]=(  \pi(\sigma_{\alpha}(x)),\rho(a)v  )
                                 =(x,v).
\end{equation}

The section $\psi$ can also be seen with respect to the ``reference''{} sections $\sigma_{\alpha}$ by means of the definition
\begin{equation}\label{eq:def:psisa}
  \psi(x)=[\sigma_{\alpha}(x),\psisa(x)]
\end{equation}
for a function $\dpt{\psisa}{M}{V}$.

\begin{lemma}
Let $\dpt{\psi}{M}{E}$ be a section and $\dpt{\hpsi}{P}{V}$, the corresponding equivariant function. Then
\[
   \psisa(x)=\hpsi(\sigma_{\alpha}(x)).
\]
\end{lemma}

\begin{proof}
By definition, $\psi(x)=\psi(\pi(\xi))=[\xi,\hpsi(\xi)]$.  Thus if we consider in particular $\xi=\sigma_{\alpha}(x)$,
\begin{equation}
  \phi^E_{\alpha}(\psi(x))=\phi^E_{\alpha}[\xi,\hpsi(\xi)]
                        =\phi^E_{\alpha}[s_{\alpha}(x),\hpsi(\sigma_{\alpha}(x))]
                        =(x,\hpsi(\sigma_{\alpha}(x))).
\end{equation}

\end{proof}

Let us anticipate. A \defe{spinor}{spinor} is a section of an associated bundle $E=P\times_{\rho} V$ where $P$ is a Lorentz-principal bundle, $V=\eC^2$ and $\rho$ is the spinor representation of Lorentz on $\eC^2$. So a spinor $\dpt{\psi}{M}{E}$ is \emph{locally} described by a function $\dpt{\psi\bsa}{M}{\eC^2}$. The latter is the one that we are used to handle in physics. In this picture, the transformation law of $\psi$ under a Lorentz transformation comes naturally.

Let $\{e_i\}$ be a basis of V; we consider some ``reference'' sections $\gamai$ of the associated bundle $E=P\times_{\rho} V$ defined by
\begin{equation}\label{eq:def:gamai}
\gamai(x)=[\phi_{\alpha}^{-1}(x,e),e_i].
\end{equation}
A general section $\dpt{\psi}{M}{E}$ is defined by an equivariant function $\dpt{\hpsi}{P}{V}$ which can be written as $\hpsi(\xi)=a^i(\xi)e_i$. If $\eta=\phi_{\alpha}^{-1}(x,e)$ and $\xi=\eta\cdot g(\xi)$,
\begin{equation}
  \psi(x)=[\xi,a^ie_i]
         =a^i[\eta,\rho(g)e_i]
	 =a^i(\xi)\bghd{\rho(g(\xi))}{i}{j}[\eta,e_j]
	 =c^j(\xi)\gamaj(x).
\end{equation}
Since the left hand side of this equation just depends on $x$, the functions $c^j$ must actually not depend on the choice of $\xi\in\pi^{-1}(x)$. So we have $\dpt{c^j}{M}{\eR}$. Indeed, if we choose $\chi\in\pi^{-1}(x)$, 
\[
  \psi(x)=c^j(\xi)\gamaj(x)
         \stackrel{!}{=}[\xi,a^{i}(\chi)e_i]
	 =\ldots
	 =c^j(\chi)\gamaj(x),
\]
so that $c^j(\xi)=c^j(\chi)$. So any section $\dpt{\psi}{M}{E}$ can be decomposed (over the open set $\mU_{\alpha}$) as 
\begin{equation}
  \psi(x)=s_{\alpha}^i(x)\gamai(x).
\end{equation}


\subsection{Associated and vector bundle}
%----------------------------------------

\subsubsection{General construction}
%///////////////////////////////////

We are going to see that a vector bundle is an associated bundle. For this, we consider a vector bundle $\dpt{p}{F}{M}$ with a fiber $F_x=V$ of dimension $m$. Let $G=GL(V)$, $P$ be the trivial principal bundle $P=M\times G$ and $\rho$ be the definition representation of $G$ on $V$. We set $E=P\times_{\rho} V$. Our aim is to put a vector bundle structure on $E$ which is equivalent to the one of $F$. The bijection $\dpt{b}{F}{E}$ will clearly be 
\begin{equation}
   b(\phi^{-1}(x,v))=[(x,e),v].
\end{equation}
We define the projection $\dpt{q}{E}{M}$ by
\[
   q[(x,g),w]=x
\]
and we have to show that  $q^{-1}(x)=\{  [(x,g),w]\tq g\in G \textrm{ and } w\in V  \}$ is a vector space isomorphic to $V$. The following definitions define a vector space structure:
\begin{itemize}
\item multiplication by a scalar: $\lambda[(x,g),v]=[(x,g),\lambda v]$,
\item addition: $[(x,g),v]+[(x,h),w]=[(x,e),\rho(g)v+\rho(h)w]$.
\end{itemize}
As local trivialization map, we consider 
\begin{equation}
\begin{aligned}
 \chi\colon q^{-1}(\mU)&\to \mU\times V \\ 
    [(x,g),v]  &\mapsto(x,\rho(g)v).
\end{aligned}
\end{equation}
With this structure, the bijection $b$ is an equivalence because $b|_{F_x}$ is a vector space isomorphism and $q\circ b=p$.

\subsection{Equivariant functions for a vector field}	\label{equivvec}
%----------------------------------------------------

In order to define in the same way an equivariant function for a vector field $X\in\cvec(M)$, we need to see $TM$ as an associated bundle.

\begin{proposition}
If $M$ is a $n$ dimensional manifold, we have the following isomorphism:
\[
     \SO(M)\times_{\rhoM}\eR^m\simeq TM
\]
where $\dpt{\rhoM}{\SO(m)\times\eR^m}{\eR^m}$ is defined by $\rhoM(A)v=Av$.
\end{proposition}
\begin{proof}
Recall that an element $b\in \SO(M)_x$ is a map $\dpt{b}{\eR^m}{T_xM}$. The isomorphism is no difficult. It is $\dpt{\psi}{\SO(M)\times_{\rhoM}\eR^m}{TM}$ defined by
\[\psi[b,v]=b(v).\]
It prove no difficult to see that $\psi$ is well defined, injective and surjective.
\end{proof}

Now, let us consider $X\in\cvec(M)$. We can see it as an element of $\Gamma(\SO(M)\times_{\rhoM}\eR^m)$, and define an equivariant function $\dpt{\hX}{\SO(M)}{\eR^m}$.

Let us make it more explicit. A vector field $Y\in\cvec(M)$ is, for each $x$ in $M$, the data of a tangent vector $Y_x\in T_xM$. Hence the formula $b(v)=Y_x$ defines an element $[b,v]$ in $\SO(M)\times_{\rhoM}\eR^m$, and $Y$ defines a section $\tilde{Y}(x)=[b(x),v(x)]$ of $\SO(M)\times_{\rhoM}\eR^m$. The associated equivariant function is given by $\hY(b)=v$ if $b(v)=Y_x$. In other words, the equivariant function $\dpt{\hY}{\SO(M)}{\eR^m}$ associated with the vector field $Y\in\cvec(M)$ is given by
\begin{equation}\label{r1404e1}
  \hY(b)=b^{-1}(Y_x),
\end{equation}
 where $x=\pi(b)$.


\subsection{Gauge transformations}
%---------------------------------

A \defe{gauge transformation}{gauge!transformation!of principal bundle} of the $G$-principal bundle $\dpt{\pi}{P}{M}$ is a diffeomorphism $\dpt{\varphi}{P}{P}$ such that

\begin{itemize}
\item $\pi\circ\varphi=\pi$,
\item  $\varphi(\xi\cdot g)=\varphi(\xi)\cdot g$.
\end{itemize}

When we consider some local sections on $\dpt{\sigma_{\alpha}}{\mU_{\alpha}}{P}$, we can describe a gauge transformation with a function $\dpt{\tilde{\varphi}_{\alpha}}{M}{G}$ by requiring
\[
   \varphi(\sigma_{\alpha}(x))=\sigma_{\alpha}(x)\cdot\tilde{\varphi}_{\alpha}(x).
\]
This formula defines $\varphi$ from $\tilde{\varphi}$ as well as $\tilde{\varphi}$ from $\varphi$.

The group of gauge transformations\index{gauge!transformation!of section of associated bundle} has a natural action on the space of sections given by
\begin{subequations}
   \begin{align}
   (\varphi\cdot\psi)(x)&=[\varphi(\xi),v].
\intertext{if $\psi(x)=[\xi,v]=[\xi,\hat{\psi}(\xi)]$. This law can also be seen on the equivariant function $\hpsi$ which defines $\psi$. The rule is}
   \widehat{\varphi\cdot\psi}(\xi)&=\hpsi(\varphi^{-1}(\xi)).
   \end{align}
\end{subequations}
Indeed, in the same way as before we find $(\varphi\cdot\psi)(x)=[\xi,\widehat{\varphi\cdot\psi}(x)]\stackrel{!}{=}[\varphi(\xi),v]=[\varphi(\xi),\hpsi(\xi)]$. Taking $\xi\to\varphi^{-1}(\xi)$ as representative, $(\varphi\cdot\psi)(x)=[\xi,\hpsi\circ\varphi^{-1}(\xi)]$.


\section{Adjoint bundle}
%+++++++++++++++++++++++

Let $\pi\colon P\to M$ be a $G$-principal bundle. The \defe{adjoint bundle}{adjoint!bundle} is the associated bundle $\Ad(P)=P\times_{\Ad}\mG$. An element of that bundle is an equivalent class given by\nomenclature[D]{$\Ad(P)$}{Adjoint bundle of the principal bundle $P$}
\[ 
  [\xi,X]=[\xi\cdot g,\Ad(g^{-1})X]
\]
for every $g\in G$. Here $\xi\in P$ and $X\in\mG$.

\section{Connection on vector bundle: local description}\label{sec:conn_vect}
%+++++++++++++++++++++++++++++++++++++++++++++++++++++++

A \defe{connection}{connection!on vector bundle} on the vector bundle $p\colon E\to M$ is a bilinear map
		\begin{equation}
		\begin{aligned}
			\nabla \colon \cvec(M)\times\Gamma(E) &\to \Gamma(E),\
			(X,s)&\mapsto \nabla_Xs
		\end{aligned}
	\end{equation}	
 such that 
 \begin{itemize}
 \item $\nabla_{fX}s=f\nabla_Xs$,
 \item $\nabla_X(fs)=(X\cdot f)s+f\nabla_Xs$
 \end{itemize}
for all $X\in\cvec(M)$, $f\in\Cinf(M)$ and $s\in\Gamma(E)$. The operation $\nabla$ is often called a \defe{covariant derivative}{covariant!derivative!on vector bundle}.

An easy example is given on the trivial bundle $E=\pr_1\colon M\times\eC\to M$. For this bundle, $\Gamma(E)=\Cinf(M,\eC)$ and the common derivation is a covariant derivation: $\nabla_Xs=(ds)X$.

\begin{proposition}
The value of $(\nabla_Xs)(x)$ depends only on $X_x$ and $s$ on a neighbourhood of $x\in M$.
\end{proposition}

\begin{proof}
Let $X$, $Y\in\cvec(M)$ such that $Y_z=f(z)X_z$  with $f(x)=1$ and $f(z)\neq 1$ everywhere else. Then
\[
  (\nabla_Ys)(x)-(\nabla_Xs)(x)=(f(x)-1)(\nabla_Xs)(x)=0.
\]
Since it is true for any function, the linearity makes that it cannot depend on $X_z$ with $z\neq x$. If we consider now two sections $s$ and $s'$ which are equals on a neighbourhood of $x$, we can write $s'=fs$ for a certain function $f$ which is $1$ on the neighbourhood. Then
\[
  (\nabla_Xs')(x)-(\nabla_Xs)(x)=(f(x)-1)(\nabla_Xs)(x)+(Xf)s(x)
\]
which zero because on a neighbourhood of $x$, $f$ is the constant $1$.
\end{proof}

This proposition shows that it makes sense to consider only local descriptions of connections.  Let $\{e_1,\ldots,e_r\}$ be a basis of $V$ and consider the local sections $\dpt{\ovS_{\alpha i}}{\mU_{\alpha}}{E}$,
\[
  \ovS_{\alpha i}(x)=\phi_{\alpha}^{-1}(x,e_i).
\]
A local section $\dpt{s_{\alpha}}{\mU_{\alpha}}{V}$ can be decomposed as $s_{\alpha}(x)=s_{\alpha}^i(x)e_i$ with respect to this basis (up to an isomorphism between the different $V$ at each point). Then on $\mU_{\alpha}$,
\begin{equation}
  s_{\alpha}^i\ovS_{\alpha i}(x)=s_{\alpha}^i(x)\phi_{\alpha}^{-1}(x,e_i)
                              =\phi_{\alpha}^{-1}(x,s_{\alpha}^ie_i)
			      =\phi_{\alpha}^{-1}(x,s_{\alpha}(x))
			      =s(x).
\end{equation}
The first equality is the definition of the product $\eR\times F\to F$.

So any $s\in\Gamma(E)$ can be (locally !) written under the form\footnote{be careful on the fact that the ``coefficient'' $s_{\alpha}^i$ depends on $x$ : the right way to express this equation is $s(x)=s^i_{\alpha}(x)\ovS_{\alpha i}(x)$.} $s=s_{\alpha}^i\ovS_{\alpha i}$; in particular $\nabla_X(\ovS_{\alpha i})$ can. We define the coefficients $\theta$ by\nomenclature[D]{$(\theta_{\alpha})_i^j$}{Matrix associated with a connection}
\begin{equation}
 \nabla_X(\ovS_{\alpha i})=(\theta_{\alpha})^j_i(X)\ovS_{\alpha j}.
\end{equation}
where, for each $i$ and $j$, $(\theta_{\alpha})^j_i$ is a $1$-form on $\mU_{\alpha}$. We can consider $\theta_{\alpha}$ as a matrix-valued $1$-form on $\mU_{\alpha}$.

\begin{proposition}	\label{PropFormnabXthe}
The formula
\begin{equation}\label{eq:nab_theta}
   (\nabla_Xs)_{\alpha}=Xs_{\alpha}+\theta_{\alpha}(X)s_{\alpha}
\end{equation}
gives a local description of the connection. 
\label{prop:namba_theta_u}
\end{proposition}

\begin{proof}
For any $s\in\Gamma(E)$, we have
\[ 
\begin{split}
\nabla_Xs&=\nabla_X\big(  \sum_j s^j_{\alpha}\ovS_{\alpha j}  \big)\\
         &=\sum_j\Big(  (Xs_{\alpha}^j)\ovS_{\alpha j} + s^j_{\alpha}\nabla_X\ovS_{\alpha j}     \Big)\\
	 &=\sum_i\Big[   (Xs_{\alpha}^i)+s_{\alpha}^j(\theta_{\alpha})_j^i(X)  \Big]\ovS_{\alpha i}.
\end{split}
\]
\end{proof}

\subsection{Connection and transition functions}
%//////////////////////////////////////////////////

A connection determines some local matrix-valued $1$-forms $\theta_{\alpha}$ on the trivialization $\mU_{\alpha}$. Two natural questions raise. The first is the converse: does a matrix-valued $1$-form defines a connection~? The second is to know  what is $\theta_{\alpha}$ in function of $\theta_{\beta}$ on $\mU_{\alpha}\cap\mU_{\beta}$ ? The answer to the latter is  given by the following proposition :

\begin{proposition}
The $1$-form $\theta_{\alpha}$ relative to the trivialization $(\mU_{\alpha},\phi_{\alpha})$ is related to the $1$-form $\theta_{\beta}$ relative to the trivialization $(\mU_{\beta},\phi_{\beta})$ by
\begin{equation}\label{eq:theta_g}
  \theta_{\beta}=g\bab^{-1} dg\bab+g\bab^{-1}\theta_{\alpha} g\bab.
\end{equation}
\end{proposition}

This equation looks like something you know ? If you think to equation \eqref{trans_A} or \eqref{tr_de_A} or any physical equation of gauge transformation for the bosons, then you are almost right.
 
\begin{proof}
We can use equation \eqref{eq:tr_sec} pointwise on $(\nabla_X s)_{\alpha}$ :
\begin{equation}
\begin{split}
(\nabla_X s)_{\alpha}&=g\bab(\nabla_Xs)_{\beta}\\
                  &=g\bab\big(   Xs_{\beta}+\theta_{\beta}(X)s_{\beta}   \big) \\
                  &=g\bab\big(   X(g_{\alpha\beta} s_{\alpha})+\theta_{\beta}(X)g_{\alpha\beta} s_{\alpha}   \big).
\end{split}
\end{equation} 
We have to compare it with equation \eqref{eq:nab_theta}. Note that $g\bab$ and $\theta_{\alpha}(X)$ are matrices, then one cannot do
\[
   g\bab\theta_{\beta}(X)g_{\alpha\beta} =g\bab g_{\alpha\beta}\theta_{\beta}(X)=\theta_{\beta}(X)
\]
by using $g\bab g_{\alpha\beta}=\mtu$.  Taking carefully subscripts into account, one sees that the correct form is $(g\bab)^i_j\theta_{\beta}(X)^j_k(g_{\alpha\beta})^k_l$. Applying Leibnitz formula ($X(fg)=f(Xg)+(Xf)g$), and making the simplification $g\bab g_{\alpha\beta}=\mtu$ in the first term, we find
\[
  \theta_{\alpha}(X)s_{\alpha}=g\bab(Xg_{\alpha\beta})s_{\alpha}+g_{\alpha\beta}^{-1}\theta_{\beta}(X)g_{\alpha\beta} s_{\alpha}.
\]
The claim follows from the fact that $Xg_{\alpha\beta}=dg_{\alpha\beta}(X)$.
\end{proof}

Notice that formula \eqref{eq:theta_g} shows in particular that $\theta_{\alpha}$ takes its values in the Lie algebra $\gl(V)$, see for example subsection \ref{SubSecgmudg}.

The inverse is given in the
\begin{proposition}	\label{Propformconnve}
If we choose a family of $\gl(V)$-valued $1$-forms $\theta_{\alpha}$ on $\mU_{\alpha}$ satisfying \eqref{eq:theta_g},then the formula
\[
  (\nabla_Xs)_{\alpha}=Xs_{\alpha}+\theta_{\alpha}(X)s_{\alpha}
\]
defines a connection on $E$.\label{prop:thet_conn_F}
\end{proposition}

\begin{proof}
Note that $\theta$ is $\Cinf(M)$-linear, thus
\begin{equation}
  (\nabla_{fX}s)_{\alpha}=(fX)s_{\alpha}+\theta_{\alpha}(fX)s_{\alpha}
                        =f[ Xs_{\alpha}+\theta_{\alpha}(X)s_{\alpha} ]
			=f(\nabla_Xs)_{\alpha}.
\end{equation}
In expressions such that $\theta_{\alpha}(X)(fs_{\alpha})$, the product is a matrix times vector product between $\theta_{\alpha}(X)$ and $s_{\alpha}$; the position of the $f$ is not important. So we can check the second condition :
\begin{equation}
\begin{split}
(\nabla_X(fs))_{\alpha}&=X(fs_{\alpha})+\theta_{\alpha}(X)(fs_{\alpha}) \\
                     &=X(f)s_{\alpha}+f(Xs_{\alpha})+f\theta_{\alpha}(X)s_{\alpha}\\
		     &=df(X)s_{\alpha}+f(\nabla_Xs)_{\alpha}.
\end{split}
\end{equation}
This concludes the proof.
\end{proof}


\subsection{Torsion and curvature}
%----------------------------------

The map $\dpt{T^{\nabla}}{\cvec(X)\times\cvec(X)}{\cvec(X)}$ defined by
\begin{equation}
     T^{\nabla}(X,Y)=\nabla_XY-\nabla_YX-[X,Y]\label{deftorsion}
\end{equation}
is the \defe{torsion}{torsion!of a connection} of the connection $\nabla$. When $T^{\nabla}(X,Y)=0$ for every $X$ and $Y$ in $\cvec(X)$, we say that $\nabla$ is a \defe{torsion free}{torsion!free, connection} connection. Let $X$, $Y$ be in $\cvec(M)$, and consider the map $\dpt{R(X,Y)}{\Gamma(E)}{\Gamma(E)}$ defined by
		\begin{equation}
		\begin{aligned}
			R(X,Y) \colon \Gamma(E) &\to \Gamma(E)\
			s&\mapsto \nabla_X\nabla_Ys-\nabla_Y\nabla_Xs-\nabla_{[X,Y]}s.
		\end{aligned}
	\end{equation}	

For each $x\in M$, $R$ can be seen as a bilinear map $\dpt{R}{T_xM\times T_xM}{\End(E_x)}$. It is called the \defe{curvature}{curvature} of the connection $\nabla$. For every $f\in C^{\infty}(M)$, it satisfies
\[
 R(fX,Y)s=fR(X,Y)s=R(X,Y)fs.
\]
    

In a trivialization $(\mU_{\alpha},\phi_{\alpha})$, we have $(\nabla_Xs)_{\alpha}=Xs_{\alpha}+\theta_{\alpha}(X)s_{\alpha}$. In the expression of $(R(X,Y)s)_{\alpha}$, the terms coming from the $Xs_{\alpha}$ part of covariant derivative make
\[
  XYs_{\alpha}-YXs_{\alpha}-[X,Y]s_{\alpha}=0.
\]
The other terms are no more than matricial product, hence the formula
\begin{equation}
  (R(X,Y)s)_{\alpha}=\Omega_{\alpha}(X,Y)s_{\alpha}
\end{equation}
 defines a $2$-form $\Omega_{\alpha}$ which takes values in $GL(r,\eK)$. We can find an expression for $\Omega$ in terms of $\theta$ :
\[
  \Omega_{\alpha}(X,Y)=X\theta_{\alpha}(Y)-Y\theta_{\alpha}(X)-\theta_{\alpha}([X,Y])+\theta_{\alpha}(X)\theta_{\alpha}(Y)-\theta_{\alpha}(Y)\theta_{\alpha}(X);
\]
it is written as
\begin{equation}\label{eq:Omega_ttheta}
\Omega_{\alpha}=d\theta_{\alpha}+\theta_{\alpha}\wedge\theta_{\alpha}=d\theta_{\alpha}+\frac{1}{2}[\theta_{\alpha},\theta_{\alpha}]
\end{equation}
which is a notational shortcut for
\begin{equation}		\label{EaCurvdVVsq}
  \Omega_{\alpha}(X,Y)=d\theta_{\alpha}(X,Y)+[\theta_{\alpha}(X),\theta_{\alpha}(Y)].
\end{equation}
These equations are called \defe{structure equations}{structure!equations}. Pointwise, the second term is a matrix commutator; be careful on the fact that, when we will speak about principal bundle, the forms $\theta$'s will take their values in a Lie algebra. On $\mU_{\alpha}\cap\mU_{\beta}$, we have
\[
  \Omega_{\beta}(X,Y)=g\bab^{-1}\Omega_{\alpha}(X,Y)g\bab.
\]
The curvature and the connection fulfill the \defe{Bianchi identities}{Bianchi identities} :

\begin{lemma}
  \[
     d\Omega_{\alpha}+[\theta_{\alpha}\wedge\Omega_{\alpha}]=0.
  \]
\end{lemma}

\begin{proof}
For each matricial entry, $\theta_{\alpha}$ is a $1$-form on $\mU_{\alpha}$, then $\theta_{\alpha}(X)$ is a function which to $x\in M$ assign $\theta_{\alpha}(x)(X_x)\in\eR$. So we can apply $d$ and Leibnitz on the product $\theta_{\alpha}(X)\theta_{\alpha}(Y)$.
\[
 d\big(  \theta_{\alpha}(X)\theta_{\alpha}(Y)  \big)=\theta_{\alpha}(X)d\theta_{\alpha}(Y)+d\theta_{\alpha}(X)\theta_{\alpha}(Y).
\]
Differentiating equation \eqref{eq:Omega_ttheta}, $d\Omega_{\alpha}=d\theta_{\alpha}\wedge\theta_{\alpha}-\theta_{\alpha}\wedge d\theta_{\alpha}$. 
\end{proof}


%////////////////////////////////////////////////////////////////////////////////////////////////////////////////////////////
\subsection{Divergence, gradient and Laplacian}

We define the \defe{gradient}{gradient} of a function $f\in C^{\infty}(M)$, denoted by $\nabla f$\nomenclature[F]{$\nabla f$}{Gradient of the function $f$} as the vector field such that
\begin{equation}
	g(\nabla f,X)=X(f).
\end{equation}
The \defe{divergence}{divergence} of a vector field $X\in \Gamma(TM)$, is the function $\nabla\cdot X\in C^{\infty}(M)$\nomenclature[F]{$\nabla\cdot X$}{divergence of the vector field $X$} defined by
\begin{equation}
  (\nabla\cdot X)(x)=\tr\big( v\mapsto\nabla_vX \big)
\end{equation}
where the trace is the one of $v\mapsto\nabla_vX$ seen as an operator on $T_xM$. The \defe{Laplacian}{laplace operator} of the function $f$ is the function $\Delta f$\nomenclature[F]{$\Delta f$}{Laplace operator} given by
\begin{equation}
\Delta f=\nabla\cdot(\nabla f).
\end{equation}

\section{Connexion on vector bundle: algebraic view}
%+++++++++++++++++++++++++++++++++++++++++++++++++++

A \defe{connection}{connection!on vector bundle} on the vector bundle $\pi\colon E\to M$ is a linear map
\[ 
  \nabla\colon \Gamma^{\infty}(E)\to \Gamma^{\infty}(E)\otimes\Omega^1(M)
\]
which satisfies the Leibnitz rule
\begin{equation}
\nabla(\sigma f)=(\nabla\sigma)f+\sigma\otimes df
\end{equation}
for any section $\sigma\colon M\to E$ and function $f\colon M\to \eC$. If $\{ \sigma_i \}$ is a local basis of $E$, one can write $\sigma=\sigma_if^i$ and one defines the \defe{Christoffel symbols}{Christoffel symbol} $\Gamma_{i\mu}^{j}$ in this basis by
\begin{equation}
\nabla \sigma=\nabla (\sigma_if^i)
		=(\nabla \sigma_i)f^i+\sigma_i\otimes f(f^i)
		=f^i\Gamma_{i\mu}^{j}\sigma_j\otimes dx^{\mu}+\sigma_i\otimes d(f^i).
\end{equation}
The notations $d\sigma=\sigma_i\otimes d(f^i)$ and $\Gamma\sigma=f^i\Gamma_{i\mu}^{j}\sigma_j\otimes dx^{\mu}$ lead us to the compact usual form
\[ 
  \nabla\sigma=(d+\Gamma)\sigma.
\]

When $E=TM$ over a (pseudo)Riemannian manifold $M$, we know the Levi-Civita\index{connection!Levi-Civita} connection which is compatible with the metric:
\begin{equation}\label{eq_230605r1}
  g(\nabla X,Y)+g(X,\nabla Y)=d\big( g(X,Y) \big).
\end{equation}
One can see $g$ as acting on $\big( \cvec(M)\otimes\Omega^1(M) \big)\times\cvec(M)$ with
\[ 
 g\big( r^i_{\nu}\partial_i\otimes dx^{\nu},t^j\partial_j \big):=r^i_{\nu}j^jg(\partial_i,\partial_j)dx^{\nu}, 
\]
which at each point is a form. From condition \eqref{eq_230605r1}, we see $\nabla$ as a Levi-Civita connection on the bundle $E=T^*M$ which values in 
\[ 
 \Gamma^{\infty}(T^*M)\otimes\Omega^1(M)\simeq\Omega^1(M)\otimes\Omega^1(M).
\]
This is defined as follows. A $1$-form $\omega$ can always be written under the for $\omega=X^{\flat}:=g(X,.)$ for a certain $X\in\cvec(M)$. Then  \eqref{eq_230605r1} gives
\[ 
  (\nabla X)^{\flat}Y+\omega(\nabla Y)=d(\omega Y),
\]
and we put $\nabla\omega=(\nabla X)^{\flat}$, i.e
\begin{equation}
  (\nabla\omega)Y=d(\omega Y)-\omega(\nabla Y)
\end{equation}
for all $Y\in\cvec(M)$. When $\omega=dx^i$ and $Y=\partial_j$, we find
\begin{equation}
(\nabla dx^i)\partial_j=d(dx^i\partial_j)-dx^i(\nabla\partial_j)
		=d(\delta^i_j)-\Gamma_{jk}^{l}\partial_l\otimes dx^k
		=-\Gamma_{jk}^{l}\delta_l^i\otimes dx^k
		=-\Gamma_{jk}^{i}\,dx^k.
\end{equation}
So we get the local formula
\begin{equation}
\nabla dx^i=-\Gamma_{jk}^{i}\,dx^j\otimes dx^k.
\end{equation}
If the form writes locally $\omega=dx^if_i$,
\begin{equation}
  \nabla\omega=\nabla(dx^i)f_i+dx^i\otimes df_i
		=-f_i\Gamma_{jk}^{i}\,dx^j\otimes dx^k+d\omega
		=(d-\tilde\Gamma)\omega
\end{equation}
where we taken the notations $d\omega=dx^i\otimes df_i$ and $\tilde\Gamma\omega=f_i\Gamma_{jk}^{i}dx^j\otimes dx^k$.

%---------------------------------------------------------------------------------------------------------------------------
\subsection{Exterior derivative}

If $E$ is a $m$-dimensional vector bundle over $M$ and $s\colon M\to E$ is a section, we say that a \defe{exterior derivative}{exterior!derivative} is a map $D\colon \Gamma(E)\to \Gamma(E\otimes \Omega^1M)$ such that for every $f\in C^{\infty}(M)$ we have
\[ 
  D(fs)=s\otimes df+f(Ds).
\]
An exterior derivative can be extended to $D\colon \Gamma(E\otimes\Omega^kM)\to \Gamma(E\otimes\Omega^{k+1}M)$ imposing the condition
\begin{equation}		\label{EqExtExtDerrk}
D(\omega\wedge\alpha)=(D\omega)\wedge\alpha+(-1)^k\omega\wedge d\alpha
\end{equation}
for every $\omega\in\Gamma(E\otimes\Omega^kM)$ and $\alpha\in\Gamma(E\otimes\Omega^lM)$ . The result is an element of $\Gamma(E\otimes\Omega^{k+l+1}M)$.

Coordinatewise expressions are obtained when one choose a specific section $(e_i)$ of the frame bundle of $E$. In that case for each $i$, the derivative $e_i$ is an element of $\Gamma(E\otimes\Omega^1M)$ and we define $\omega_i^j\in\Omega^1(M)$\nomenclature{$\omega_i^j$}{Connection form} by
\begin{equation}
De_i=\sum_{j=1}^ke_j\otimes \omega_i^j.
\end{equation}
For each $i$ and $j$, we have an element $\omega_i^j\in\Omega^1(M)$, so that we say that $\omega\in\Omega^1(M,\gl(m))$. Now a section can be expressed as $s=s^ie_i$ where $s^i$ are functions, so we have
\begin{align}
  D(s)=D(s^ie_i)	&=e_i\otimes ds^i+s^iD(e_i)=e_i\otimes ds^i+s^ie_j\otimes \omega_i^j=e_i\otimes ds^i+e_i\otimes s^j\omega_j^i.
\end{align}
Expressed in component, we find $D(s)^i=ds^i+s^j\omega_j^i$, so that we often write
\begin{equation}
D=d+\omega.
\end{equation}
When a section $e$ is given, we write $s=s^i(e)e_i$, indicating the dependence of the functions $s^i$ in the choice of the frame $e$ :
\[ 
  D(s)=e_i\otimes ds^i(e)+e_i\otimes s^j(e)\omega(e)_j^i.
\]
When we apply both sides to a vector $X\in\Gamma(TM)$, we find
\begin{equation}
D_X(s)=e_i\otimes\Big( X(s^i)+s^j\omega^i_j(X) \Big).
\end{equation}

By convention we say that, when $f\in C^{\infty}(M)$, is a function, $D_X$ reduces to the action of the vector field $X$:
\begin{equation}
  D_X(f)=X(f).
\end{equation}


%----------------------------------------------------------------------------------------------------------------------------
\subsubsection{Covariant exterior derivative}

An important exterior derivative is the covariant exterior derivative. If the vector bundle $E$ is endowed by a covariant derivative $\nabla$, we define the corresponding \defe{covariant exterior derivative}{covariant!derivative!exterior } by the following :
{
\renewcommand{\theenumi}{\arabic{enumi}.}
\begin{enumerate}
\item for a section $s\colon M\to E$ (i.e. a $0$-form) we define
\begin{equation}
   (d_{\nabla}s)(X)=\nabla_Xs,
\end{equation}
\item and on the $1$-form $\sum_i(s_i\otimes\omega_i \big)\in\Gamma(E\otimes T^*M)$,
\begin{equation}
d_{\nabla}\big( \sum_is_i\otimes\omega_i \big)=\sum_i(d_{\nabla}s_i)\wedge\omega_i+\sum_is_i\otimes d\omega_i.
\end{equation}
\end{enumerate}
}		% Fin de la mise en nombre arabes pour la liste énumérée
The latter relation is the condition \eqref{EqExtExtDerrk} with $k=0$.

%\\\\\\\\\\\\\\\\\\\\\\\\\\\\\\\\\\\\\\\\\\\\\\\\\\\\\\\\\\\\\\\\\\\\\\\\\\\\\\\\\\\\\\\\\\\\\\\\\\\\\\\\\\\\\\\\\\\\\\\\\\\\
\subsubsection{Soldering form and torsion}


Let us particularize to the case where $E$ has the same dimension as the manifold. In that case, we can introduce a \defe{soldering form}{soldering form}, that is an element $\theta\in \Omega^1(M,E)$ such that for every $x\in M$ the map $\theta_x\colon T_xM\to E_x$ is a vector space isomorphism.
When a soldering form $\theta$ is given, the \defe{torsion}{torsion!of exterior derivative} is the exterior derivative $D$ is
\begin{equation}
	T=D\theta.
\end{equation}
Using a local frame $e$, we have forms $\theta^i(e)\in\Omega^1(M)$ such that
\[ 
  \theta(X)=\theta^i(X)e_i.
\]
We see $\theta$ as an element of $\Gamma(E\otimes \Omega^1(M))$ by identifying $\theta=e_i\otimes\theta^i$. Thus we have
\[ 
D\theta=D(e_i\otimes\theta^i)	=De_i\wedge\theta^i(e)+e_i\wedge d\theta^i(e)
				=(e_j\otimes^j_i)\wedge\theta^i(e)+e_i\wedge d\theta^i(e),
\]
or in coordinates :
\begin{equation}
  (D\theta)^i=\omega_j^i\wedge \theta^j(e)+d\theta^i(e).
\end{equation}
Notice that it provides the formula
\begin{equation}
T=d_{\omega}\theta
\end{equation}
for the torsion as exterior covariant derivative of the connection form.

%\\\\\\\\\\\\\\\\\\\\\\\\\\\\\\\\\\\\\\\\\\\\\\\\\\\\\\\\\\\\\\\\\\\\\\\\\\\\\\\\\\\\\\\\\\\\\\\\\\\\\\\\\\\\\\\\\\\\\\\\\\\\
\subsubsection{Example : Levi-Civita}

We consider the vector bundle $E=TM$ and the local basis $e_i=\partial_i$. An exterior derivative in this case is a map $D\colon \Gamma(TM)\to \Gamma\Big( TM\otimes\Omega^1M \Big)$. In that particular case, we denote by $\nabla_XY$ the vector field $D(Y)X$, and it is computed by first writing $D(X)_x=\sum_iZ_x^i\otimes\omega_x^i$ with $Z^i\in\Gamma(TM)$ and $\omega^i\in\Omega^1(M)$. The we have
\begin{equation}
D(X)_xY_x=\omega+x^i(Y_x)Z_x^i.
\end{equation}
A good choice of soldering form is $\theta_x=\id$ for every $x\in M$, or $\theta(X)=X$. In coordinates, that soldering form is given by $\theta^i(\partial_j)=\delta^i_j$. The \defe{Christoffel symbols}{Christoffel symbol} are defined by
\begin{equation}
\nabla_{\partial_i}\partial_j=\Gamma_{ij}^k\partial_k,
\end{equation}
and the covariant derivative reads
\begin{equation}		\label{EqCovDerGamChr}
\nabla_XY	= \nabla_{X^i\partial_i}(Y^j\partial_j)
		= X^i\Big( (\partial_iY^j)\partial_j+Y^j\nabla_{\partial_i}\partial_j \Big)
		= \Big( X(Y^k)+X^iY^j\Gamma_{ij}^k \Big) \partial_k.
\end{equation}

We can determine the Christoffel symbols in function of the connection form using the fact that on the one hand, $\nabla_{\partial_i}\partial_j=\Gamma_{ij}^k\partial_k$, and on the other hand,
\[ 
  \nabla_{\partial_i}\partial_j=D(\partial_j)(\partial_i)=\partial_k\otimes\omega_j^k(\omega_i), 
\]
so that
\begin{equation}
	\Gamma_{ij}^k=\omega_j^k(\partial_i)
\end{equation}
Now we can get the same result as equation \eqref{EqCovDerGamChr} using the exterior derivative formalism. First we have $DY=\partial_i\otimes dY^i+\partial_i\otimes X^j\omega_j^i$, so that
\[ 
  (DY)X=\partial_i\otimes dY^i(X)=\partial_i\otimes X^j\omega_j^i(X^k\partial_k),
\]
in which we use the relation $\omega_j^i(X^k\partial_k)=X^k\omega_j^i(\partial_k)=X^k\Gamma_{jk}^i$ to get
\[ 
  (DY)X=\big( X(Y^i)+X^jX^k\Gamma^i_{jk} \big)\partial_i.
\]
Notice that the anti-symmetric part of $\Gamma$ with respect to its two lower indices does not influence the covariant derivative. Let us compute the torsion in terms of $\Gamma$. For that remark that $d\theta^i=0$ because 
\[ 
  (d\theta^i)(X,Y)=X\theta^i(Y)-Y\theta^i(X)-\theta^i\big( [X,Y] \big)=X(Y^i)-Y(X^i)-[X,Y]^i=0.
\]
Thus we have
\begin{align*}
(D\theta)(\partial_k\otimes\partial_l)	&=\big( (D\partial_i)\partial_k \big)\theta^i(\partial_l)-\big( (D\partial_i)\partial_l \big)\theta^i(\partial_k)\\
					&=\delta_l^i\Gamma_{ik}^j\partial_j-\delta_k^i\Gamma_{il}^j\partial_j\\
					&=(\Gamma_{lk}^j-\Gamma^j_{kl})\partial_j.
\end{align*}
The connection $\nabla$ is moreover compatible with the metric because
\[ 
  \nabla_Z\big( g(X,Y) \big)=Z\big( \eta(eX,eY) \big)=\eta\big( \underbrace{D_Z(eX)}_{=e(\nabla_ZX)},eY \big)+\eta\big( eX,D_Z(eY) \big)=g(\nabla_ZX,Y)+g(X,\nabla_ZY).
\]




\section{Connection on principal bundle}  %\label{subsec_defconnprinc}
%------------------------------------------

\subsection{First definition:  \texorpdfstring{$1$}{1}-form}
%------------------------------

\label{pg_connpriic}
We consider a $G$-principal bundle 
\[
\xymatrix{%
   G \ar@{~>}[r]		&	P\ar[d]^{\pi}\\
   				&	  M
}
\]
and $\yG$, the Lie algebra of $G$. 

\begin{definition}
A \defe{connection}{connection!on principal bundle} on $P$  is a $1$-form $\omega\in\Omega(P,\yG)$ which fulfills

\begin{itemize}\label{pg:def:conne}
\item $\omega_{\xi}(A^*_{\xi})=A$,
\item $(R_g^*\omega)_{\xi}(\Sigma)=\Ad(g^{-1})(\omega_{\xi}(\Sigma))$,
\end{itemize}
for all $A\in\yG$, $g\in G$, $\xi\in P$ and $\Sigma\in T_{\xi} P$
\label{defconnform}
\end{definition}
Here, $R_g$ is the right action: $R_g\xi=\xi\cdot g$ and $A^*$ stands for the \defe{fundamental field}{fundamental!vector field} associated with $A$ for the action of $G$ on $P$:
\begin{equation} \label{defastar}
   A^*_{\xi}=\Dsdd{ \xi\cdot e^{-tA} }{t}{0},
\end{equation}
For each $\xi\in P$, we have $\dpt{\omega_{\xi}}{T_{\xi} P}{\yG}$. See section \ref{sec:fond_vec}.

If $\alpha$ is a connection $1$-form on $P$, we say that $\Sigma$ is an \defe{horizontal}{horizontal} vector field if $\alpha_{\xi}(\Sigma)=0$ for all $\xi\in P$. If $X_x\in T_xM$ and $\xi\in\pi^{-1}(x)$, there exists an unique\footnote{See \cite{kobayashi}, chapter II, proposition 1.2.} $\Sigma$ in $T_{\xi} P$ which is horizontal and such that $\pi_*(\Sigma)=X_x$. This $\Sigma$ is called the \defe{horizontal lift}{horizontal!lift} of $X_x$. We can also pointwise construct the horizontal lift of a vector field. The one of $X$ is often denoted by $\overline{X}$; it is an element of $\cvec(P)$.

\subsection[Horizontal space]{Second definition: horizontal space}
%-------------------------------------------------------------------

For each $\xi\in P$, we define the \defe{vertical space}{vertical space} $V_{\xi} P$ as the subspace of $T_{\xi} P$ whose vectors are tangent to the fibers: each $v\in V_{\xi} P$ fulfills $d\pi v=0$. Any such vector is given by a path contained in the fiber of $\xi$. So, $v\in V_{\xi} P$ if and only if there exists a path $g(t)\in G$ such that $v=\Dsdd{\xi\cdot g(t)}{t}{0}$.

A \defe{connection}{connection!on principal bundle} $\Gamma$ is a choice, for each $\xi\in P$, of an \defe{horizontal space}{horizontal!space} $H_{\xi} P$ such that

\begin{itemize}
\item $T_{\xi} P=V_{\xi} P\oplus H_{\xi} P$,
\item $H_{\xi\cdot g}=(dR_g)_{\xi} H_{\xi}$,
\item $H_{\xi} P$ depends on $\xi$ under a differentiable way.
\end{itemize}
The second condition means that the distribution $\xi\to H_{\xi}$ is invariant under $G$. Thanks to the first one, for each $X\in T_{\xi} P$, there exists only one choice of $Y\in H_{\xi} P$ and $Z\in V_{\xi} P$ such that $X=Y+Z$. These are denoted by $vX$ and $hX$ and are naturally named \emph{horizontal} and \emph{vertical components} of $X$. The third condition means that if $X$ is a differentiable vector field on $P$, then $vX$ and $hX$ are also differentiable vector fields. We will often write $V_{\xi}$ and $H_{\xi}$ instead of $V_{\xi} P$ and $V_{\xi} P$.

The word \emph{connection} probably comes from the fact that the horizontal space gives a way to jump from a fiber to the next one. 
When we consider a connection $\Gamma$, we can define a $\yG$-valued connection $1$-form by 
\[
   \omega(X)^*_{\xi}=vX_{\xi}.
\]
The existence is explained in section \ref{sec:fond_vec}. It is clear that $\omega(X)=0$ if and only if $X$ is horizontal. The theorem which connects the two definitions is the following.

\begin{theorem}
If $\Gamma$ is a connection on a $G$-principal bundle, and $\omega$ is its $1$-form, then

\begin{enumerate}
\item\label{enuyai} for any $A\in\yG$, we have $\omega(A^*)=A$,
\item\label{enuyaii} $(R_g)^*\omega=\Ad(g^{-1})\omega$, i.e. for any $X\in T_{\xi} P$, $g\in G$ and $\xi\in M$,
\[
    \omega((dR_g)_{\xi} X)=\Ad(g^{-1})\omega_{\xi}(X)
\]
\end{enumerate}
Conversely, if one has a $\yG$-valued $1$-form on $P$ which fulfills these two requirement, then one has one and only one connection on $P$ whose associated $1$-form is $\omega$.

\end{theorem}

\begin{proof}
\ref{enuyai} The definition of $\omega$ is $\omega(X)^*_{\xi}=vX$. Then $\omega(A^*)^*_{\xi}=vA^*_{\xi}=A^*_{\xi}$ because $A^*$ is vertical. From lemma \ref{lem:As_Bs_A_B}, $\omega(A^*)=A$.

\ref{enuyaii} Let $X\in\cvec(P)$. If $X$ is horizontal, the definition of a connection makes $dR_d X$ also horizontal, then the claim becomes $0=0$ which is true. If $X$ is vertical, there exists a $A\in\yG$ such that $X=A^*$ and a lemma shows that $dR_gX$ is then the fundamental field of $\Ad(g^{-1})A$. Using the properties of a connection,
\begin{equation}
  (R^*_g\omega)_{\xi}(X)=\omega_{\xi\cdot g}(dR_g X)=\Ad(g^{-1})A=\Ad(g^{-1})\omega_{\xi}(X).
\end{equation}

Now we turn our attention to the inverse sense: we consider a $1$-form which fulfills the two conditions and we define
\begin{equation}
   H_{\xi}=\{X\in T_{\xi} P\tq \omega(X)=0\}.
\end{equation}
We are going to show that this prescription is a connection. First consider a $X\in V_{\xi}$, then $X=A^*$ and $\omega(X)=A$. So $H_{\xi}\cap V_{\xi}=0$. Now we consider $X\in T_{\xi} P$ and we decompose it as
\[
   X=A^*+(X-A^*)
\]
where $A^*$ is the vertical component of $X$. If $\omega(dR_g X)=0$ for all $g\in G$, then $\omega(X)=0$, then a vector $X\in H_{\xi}$ fulfills at most $\dim G$ independent constraints $\omega(dR_g X)=0$ and $\dim H_{\xi}$ is at least $\dim P-\dim G$. On the other hand, $\dim V_{\xi}=\dim G$; then
\[
  \dim V_{\xi}+\dim H_{\xi}\geq\dim G+\dim P-\dim G.
\]
Then the equality must holds and $V_{\xi}\oplus H_{\xi}=T_{\xi} P$.

We have now to prove that $\omega$ is the connection form of $H_{\xi}$, i.e. that $\omega(X)$ is the unique $A\in\yG$ such that $A^*_{\xi}$ is the vertical component of $X$. Indeed if $X\in T_{\xi} P$, it can be decomposed as into $A^*\in V_{\xi}$ and $Y\in H_{\xi}$ and
\[
   \omega(X)=\omega(A^*+Y)=\omega(A^*)=A.   
\]

It remains to be proved that the horizontal space $H_{\xi}$ of any connection $\Gamma$ is related to the corresponding $1$-form $\omega$ by $H_{\xi}=\{X\in T_{\xi} P\tq\omega_{\xi}(X)=0\}$. From the connection $\Gamma$, the $1$-form is defined by the requirement that $\omega(X)^*_{\xi}=vX_{\xi}$. For $X\in H_{\xi}$, it is clear that $vX=0$, so that $\omega(X)^*=0$. This implies $\omega(X)=0$ because we suppose that the action of $G$ is effective.

\end{proof}

The projection $\dpt{\pi}{P}{M}$ induces a linear map $\dpt{d\pi}{T_{\xi} P}{T_xM}$. We will see that, when a connection is given, it is an isomorphism between $H_{\xi}$ and $T_xM$ (if $x=\pi(\xi)$). The \defe{horizontal lift}{horizontal!lift}\index{lift!horizontal} of $X\in\cvec(M)$ is the unique horizontal vector field (i.e. it is pointwise horizontal) such that $d\pi(\ovX_{\xi})=X_{\pi(\xi)}$. The proposition which allows this definition is the following.

\begin{proposition}
For a given connection on the $G$-principal bundle $P$ and a vector field $X$ on $M$, there exists an unique horizontal lift of $X$. Moreover, for any $g\in G$, the horizontal lift is invariant under $dR_g$.

The inverse implication is also true: any horizontal field on $P$ which is invariant under $dR_g$ for all $g$ is the horizontal lift of a vector field on $M$. 
\end{proposition}

This proposition comes from \cite{kobayashi}, chapter II, proposition 1.2.

\begin{proof}
We consider the restriction $\dpt{d\pi}{H_{\xi}}{T_{\pi(\xi)}M}$. It is injective because $d\pi(X-Y)$ vanishes only when $X-Y$ is vertical or zero. Then it is zero. It is cleat that $\dpt{d\pi}{T_{\xi} P}{T_{\pi(\xi)}M}$ is surjective. But $d\pi X=0$ if $X$ is vertical, then $d\pi$ is surjective from only $H_{\xi}$.

So we have existence and unicity of an horizontal lift. Now we turn our attention to the invariance. The vector $dR_g\ovX_{\xi}$ is a vector at $\xi\cdot g$. From the definition of a connection, $dR_g H_{\xi}=H_{\xi\cdot g}$, then $dR_g\ovX_{\xi}$ is the unique horizontal vector at $\xi\cdot g$ which is sent to $X_x$ by $d\pi$. Thus it is $\ovX_{\xi\cdot g}$.

For the inverse sense, we consider $\ovX$, an horizontal invariant vector field on $P$. If $x\in M$, we choose $\xi\in\pi^{-1}(x)$ and we define $X_x=d\pi(\ovX_{\xi})$. This construction is independent of the choice of $\xi$ because for $\xi'=\xi\cdot g$, we have
\[
   d\pi(\ovX_{\xi'})=\pi(dR_g\ovX_{\xi})=\pi(\ovX_{\xi}).
\]
\end{proof}

An other way to see the invariance is the following formula:
\[
   \ovX_{\xi\cdot g}=(dR_g)_{\xi} \ovX_{\xi}.
\]
By definition, $\ovX_{\xi\cdot g}$ is the unique vector of $T_{\xi\cdot g}P$ which fulfils $d\pi\ovX_{\xi\cdot g}=X_x$ if $\xi\pi^{-1}(x)$, so the following computation proves the formula:
\begin{equation}
  (d\pi)_{\xi\cdot g}((dR_g)_{\xi}\ovX_{\xi})=d(\pi\circ R_g)_{\xi}\ovX_{\xi}
                                         =d\pi_{\xi}\ovX_{\xi}
					 =X_x.
\end{equation}

\subsection{Curvature}
%////////////////////////

The curvature of a vector or associated bundle satisfies $\Omega_{\alpha}=d\theta_{\alpha}+\theta_{\alpha}\wedge\theta_{\alpha}$. So we naturally define the \defe{curvature}{curvature!on principal bundle} of the connection $\omega$ on a principal bundle as the $\yG$-valued $2$-form
\begin{equation} 
  \Omega=d\omega+\omega\wedge\omega.
\end{equation}
When we consider a local section\label{PgLocSecCurv} $\dpt{\sigma_{\alpha}}{\mU_{\alpha}}{P}$ on $\mU_{\alpha}\subset M$, we can express the curvature with a $2$-form on $M$ instead of $P$ by the formula \label{pg:curv_princ}
\[ 
 F\bsa=\sigma_{\alpha}^*\Omega,
\]
or, more explicitly, by $F\bsa_x(X,Y)=\Omega_{\sigma_{\alpha}(x)}(d\sigma_{\alpha} X,d\sigma_{\alpha} Y))$. Note that if $\yG$ is abelian, $\Omega=d\omega$ and $d\Omega=0$.

\section{Exterior covariant derivative and Bianchi identity} 
%--------------------------------------------------------------

Let $\omega\in\Omega^1(P,\mG)$ be a connection $1$-form on the $G$-principal bundle $P$. Using the operation $[.\wedge .]$ defined in section \ref{SecLiaAlgformval}, we define the \defe{exterior covariant derivative}{exterior!covariant derivative} by\nomenclature{$d_{\omega}$}{Exterior covariant derivative associated with the connection form $\omega$}  %\index{exterior covariant derivative}
\begin{align}
d_{\omega}\alpha&=d\alpha+\frac{ 1 }{2}[\omega\wedge\alpha]&\text{when $\alpha\in\Omega^1(P,\mG)$},\\
d_{\omega}\beta&=d\beta+[\omega\wedge\beta]&\text{when $\beta\in\Omega^2(P,\mG)$},
\end{align}

The \defe{curvature}{curvature!form} is the $2$-form defined by
\begin{equation}
\Omega=d_{\omega}\omega=d\omega+\omega\wedge\omega
\end{equation}
where $d_{\omega}$ is the exterior covariant derivative associated with the connection form $\omega$, and the wedge has to be understood as in equation \eqref{EqAbuswesgeomom}.

\begin{proposition}
The curvature form satisfies the identity
\begin{equation}
d_{\omega}\Omega=0
\end{equation}
which is the Bianchi identity\index{Bianchi identities}
\end{proposition}

\begin{proof}
taking the differential of $\Omega=d\omega+\omega\wedge\omega$, we find
\[ 
  d\Omega=d^2\omega+d\omega\wedge\omega-\omega\wedge d\omega
\]
in which $d^2\omega=0$ and we replace $d\omega$ by $\Omega-\omega\wedge\omega$, so that
\[ 
  d\Omega=\Omega\wedge\omega-\omega\wedge\Omega,
\]
which becomes the Bianchi identity using the definition of $d_{\omega}$ and the notation \eqref{EqDefCrochwedgedeux}.
\end{proof}
Remark that the Bianchi identity reads $d_{\omega}^2\omega=0$, but that in general $d_{\omega}$ does not square to zero. 

\section{Covariant derivative on associated bundle}
%-----------------------------------------------------

Now we consider a general $G$-principal bundle $\dpt{\pi}{P}{M}$ and an associated bundle $E=P\times_{\rho} V$. We define a product $\eR\times E\to E$ by
\begin{equation}\label{eq:def:REE}
  \lambda[\xi,v]=[\xi,\lambda v].
\end{equation}
It is clear that the equivariant function $\widehat{\lambda \psi}$ defines the section $\lambda\psi$. A \defe{covariant derivative}{covariant!derivative!on associated bundle} is a map 
		\begin{equation}
		\begin{aligned}
			\nabla \colon \cvec(M)\times \Gamma(M,E) &\to \Gamma(M,E)\
			(X,\psi)&\mapsto \nabla_X\psi
		\end{aligned}
	\end{equation}	
such that 
\begin{subequations}
\begin{align}
\nabla_{fX}\psi&=f\nabla_X\psi,   \\
\nabla_X(f\psi)&=(X\cdot f)\psi+f\nabla_X\psi                                      \label{eq:def:der_covii}
\end{align}
\end{subequations}
where products have to be understood by formula \eqref{eq:def:REE}.

\begin{theorem}  
A connection on a principal bundle gives rise to a covariant derivative on any associated bundle by the formula
\begin{equation}
  \widehat{\nabla^E_X\psi}(\xi)=\ovX_{\xi}(\hpsi)
\end{equation}
where $\dpt{\hpsi}{P}{V}$ is the function associated with the section $\dpt{\psi}{M}{E}$. 
\label{tho_dercovassoequiv}
\end{theorem}
We have to prove that it is a good definition: the function $\widehat{\nabla^E_X\psi}$ must define a section $\dpt{\nabla_X^R\psi}{M}{E}$ and the association $\psi\to\nabla^E_X\psi$ must be a covariant derivative.

With the discussion of page \pageref{pg:vecto_vecto} about the application of a tangent vector on a map between manifolds, we have $(d\varphi X)f=X(f\circ\varphi)$. By using this equality in the case of $\ovX$ with $\hpsi$ and $R_g$, we find $(dR_g\ovX)(\hpsi)=\ovX(\hpsi\circ R_g)$ and thus
\[
   \ovX_{\xi\cdot g}(\hpsi)=\ovX_{\xi}(dR_g\hpsi).
\]
We prove the theorem step by step.

\begin{proposition}
The function $\widehat{\nabla_X^E\psi}$ defines a section of $P$.
\end{proposition}

\begin{proof}
We have to see that $\widehat{\nabla_X^E\psi}$ is an equivariant function. The equivariance of $\hpsi$ gives $\hpsi\circ R_g=\rho(g^{-1})\hpsi$, thus
\begin{equation}
\widehat{ \nabla_X^E\psi }(\xi\cdot g)=\ovX_{\xi\cdot g}(\hpsi)\\
                                      =\big( (dR_g)_{\xi}\ovX_{\xi}\big)(\hpsi)\\
				      =\ovX_{\xi}(\hpsi\circ R_g)\\
				      =\ovX_{\xi}( \rho(g^{-1})\hpsi )\\
				      =\rho(g^{-1})\ovX_{\xi}(\hpsi).
\end{equation}
The last equality comes from the fact that the product $\rho(g^{-1})\hpsi$ is a linear product ``matrix times vector''{} and that $\ovX_{\xi}$ is linear.
\end{proof}

\begin{theorem}  
The definition
\[
   \widehat{\nabla^E_X\psi}(\xi)=\ovX_{\xi}(\hpsi)
\]
defines a covariant derivative.
\label{tho_nablaE}
\end{theorem}

\begin{proof}
We have to check the two conditions given on page \pageref{sec:conn_vect}.

\subdem{First condition}
By definition, $\widehat{\nabla_{fX}^E\psi}(\xi)=\overline{fX_{\xi}}(\hpsi)$. Now we prove that 
\begin{equation}\label{eq:fXhpsi}
  \overline{fX_{\xi}}(\hpsi)=(f\circ\pi)(\xi)\ovX_{\xi}(\hpsi).
\end{equation}
 This formula is coherent because $\ovX_{\xi}(\hpsi)\in V$ and $(f\circ\pi)(\xi)\in\eR$. By definition of the horizontal lift, $\overline{fX}_{\xi}$ is the unique vector such that

 \begin{itemize}
 \item $d\pi_{\xi}(\overline{fX}_{\xi})=(fX)_x=f(x)d\pi\ovX_{\xi}=(f\circ\pi)(\xi)d\pi\ovX_{\xi}$,
 \item $\omega_{\xi}(\overline{fX}_{\xi})=0$.
 \end{itemize}
We check that $(f\circ\pi)(\xi)\ovX_{\xi}$ also fulfills these two conditions because $d\pi$ and $\omega$ are $\Cinf(P)$-linear. Equation \eqref{eq:fXhpsi} immediately gives 
\begin{equation}
\widehat{\nabla_{fX}^E\psi}(\xi)=(f\circ\pi)(\xi)\widehat{\nabla_X^E\psi}(\xi).
\end{equation}
Now we show that $\widehat{ f\nabla_X^E\psi }$ is the same. The section $\dpt{f\nabla_X^E\psi}{M}{E}$ is given by  $(f\nabla_X^E\psi)(x)=f(x)(\nabla_X^E\psi)(x)$, and by definition of the associated equivariant function,
\[
  f(x)(\nabla_X^E\psi)(x)=[ \xi,f(x)\widehat{\nabla_X^E\psi}(\xi) ].
\]
Then
\begin{equation}
  \widehat{f\nabla_X^E\psi}(\xi)=f(x)\widehat{\nabla_X^E\psi}(\xi)=(f\circ\pi)(\xi)\widehat{\nabla_X^E\psi}(\xi).
\end{equation}
All this shows that
$  \nabla_{fX}^E\psi=f\nabla_X^E\psi$.
\subdem{Second condition}
This is a computation using the Leibnitz rule:
\begin{equation}
\begin{split}
  \widehat{\nabla_X^E(f\psi)}(\xi)&=\ovX_{\xi}( \widehat{f\psi} )
                                  	\stackrel{(a)}{=}\ovX_{\xi}((\pi\circ f)\hpsi)\\
				  &\stackrel{(b)}{=}\ovX_{\xi}(\pi^*f)\hpsi(\xi)+(\pi^*f)(\xi)\ovX_{\xi}\hpsi
				  =d(f\circ\pi)_{\xi}\ovX_{\xi}\hpsi(\xi)+f\widehat{\nabla_X^E\psi}(x)\\
				  &=df_{\pi(\xi)}d\pi_{\xi}\ovX_{\xi}\hpsi(\xi)+f\widehat{\nabla_X^E\psi}(x)
				  =X_x(f)\hpsi(\xi)+f\widehat{\nabla_X^E\psi}(x)\\
				  &=\widehat{(Xf)\psi}(\xi)+\widehat{f\nabla_X^E\psi}(\xi)
\end{split}
\end{equation}
where (a) is because $\widehat{f\psi}=\pi^*f\hat{\psi}$, and (b) is an application of the Leibnitz rule.
\end{proof}




\begin{theorem}
Using the local coordinates related to the sections $\dpt{\sigma_{\alpha}}{\mU_{\alpha}}{P}$, the covariant derivatives reads:
\begin{equation}\label{eq:nabla_coord}
(\nabla_X\psi)\bsa(x)=X_x\psi\bsa-\rho_*(\sigma_{\alpha}^*\omega_x(X))\psi\bsa(x)
\end{equation}
where $\dpt{\rho_*}{\yG}{\End(V)}$ is defined by
\begin{equation}  \label{eq:def_rho_s}
  \rho_*(A)=\Dsdd{\rho(e^{tA})}{t}{0}
\end{equation}

\end{theorem}

\begin{proof}
The problem reduces to the search of $\ovX$ because
\[
   (\nabla_X\psi)\bsa(x)=\widehat{\nabla_X\psi}(\sigma_{\alpha}(x))=\ovX_{\sigma_{\alpha}(x)}(\hpsi).
\]
We claim that $\ovX_{\sigma_{\alpha}(x)}=d\sigma_{\alpha} X_x-\omega(d\sigma_{\alpha} X_x)^*$. We have to check that $d\pi\ovX=X$ and $\omega(\ovX)=0$. The latter comes easily from the fact that $\omega(A^*)=A$. For the first one, remark that $s_{\alpha}$ is a section, then $d(\pi\circ s_{\alpha})=\id$, and $d\pi(ds_{\alpha} X_x)=X_x$, while
\begin{equation}
  d\pi(A^*\bxi)=d\pi\Dsdd{\xi\cdot e^{-tA}}{t}{0}
               =\Dsdd{\pi(\xi)}{t}{0}
               =0.
\end{equation}
Since the horizontal lift is unique, we deduce
\begin{equation}
  (\nabla_X\psi)\bsa(x)=\big(  d\sigma_{\alpha} X_x-\omega(d\sigma_{\alpha} X_x)^*  \big)\hpsi.
\end{equation}
From the definition of a fundamental vector field,
\begin{equation}
\begin{aligned}
    \omega(d\sigma_{\alpha} X_x)^*_{\sigma_{\alpha}(x)}\hpsi
           &=\Dsdd{\hpsi\big(\sigma_{\alpha}(x)\cdot e^{-t\omega(d\sigma_{\alpha} X_x)}  \big) }{t}{0}\\
           &=\Dsdd{\rho(e^{t\omega(d\sigma_{\alpha} X_x)})\hpsi(\sigma_{\alpha}(x))}{t}{0}&&\text{from \eqref{eq:equiv_psi_b}}\\
           &=(d\rho)_e(\omega\circ d\sigma_{\alpha})X_x(\hpsi\circ\sigma_{\alpha})(x)\\
           &=\rho_*\big( (\sigma_{\alpha}^*\omega)(X_x) \big)\psi\bsa(x)    &&\text{by \eqref{eq:def_rho_s}}
\end{aligned}
\end{equation}

\end{proof}
 
We can express the covariant derivative by means of some maps $\dpt{\theta_{\alpha}}{\cvec(M)\times M}{\End(V)}$ given by
 \begin{equation}
\nabla_X\gamai=\bghd{\theta_{\alpha}(X)}{i}{j}\gamaj.
 \end{equation}
where the $\gamai$'s were given in equation \eqref{eq:def:gamai}. By the definition \eqref{eq:def:der_covii},
\[
\begin{split}
  (\nabla_X\psi)(x)&=(X\cdot s^i_{\alpha})_x\gamai(x)+s^i_{\alpha}(x)(\nabla_X\gamai)(x)\\
                   &=(X\cdot s^i_{\alpha})_x\gamai(x)+s^i_{\alpha}(x)\bghd{\theta_{\alpha}(X)}{i}{j}\gamaj(x).
\end{split}
\]
On the othre hand with the notations of equation \eqref{eq:def:psisa}, $\gamsai=e_i$ and $X_x\gamsai=0$. Then equation \eqref{eq:nabla_coord} gives $\theta_{\alpha}(X)=\rho_*(\sigma_{\alpha}^*\omega_x(X))$, or
\begin{equation}
\theta_{\alpha}=\rho_*(\sigma^*_{\alpha}\omega_x).
\end{equation}

\subsection{Curvature on associated bundle}

From the definition \eqref{eq:def:som_E}, it makes sense to define the curvature $2$-form by
\[
  R(X,Y)\psi=\nabla_X\nabla_Y\psi-\nabla_Y\nabla_X\psi-\nabla_{[X,Y]}\psi.
\]
It is also clear that $\psisa$ defines a section of the trivial vector bundle $F=M\times V$ by $x\to (x,\psisa(x))$, so one can define  $\dpt{\Omega_{\alpha}(X,Y)}{\Gamma(M,E)}{\Gamma(M,E)}$ by
\[
  \big(  R(X,Y)\psi  \big)\bsa=\Omega_{\alpha}(X,Y)\psisa
\]
and take back all the work around Bianchi because of the relation \eqref{eq:nabla_coord} which can be written as $(\nabla_X\psi)\bsa(x)=X_x\psisa+\theta_{\alpha}(X)\psisa(x)$ and which is the same as in proposition \ref{prop:thet_conn_F}.

\subsection{Connection on frame bundle}\index{frame!bundle}
%//////////////////////////////////////

\subsubsection{General framework}
%////////////////////////////

The frame bundle was defined at page \pageref{pg:frame_bundle}. Let $\dptvb{F}{p}{M}$ be a $\eK$-vector bundle with some local trivialization $(\mU_{\alpha},\phi^E_{\alpha})$ and the corresponding transition functions $\dpt{g\bab}{\mU_{\alpha}\cap\mU_{\beta}}{GL(r,\eK)}$. We consider $\dpt{\pi}{P}{M}$, the frame bundle of $F$; it is a $GL(r,\eK)$-principal bundle. Let $\nabla$ be a covariant derivative on $F$ and $\theta_{\alpha}$, the associated matrices $1$-form. The frame bundle is
\[
  P=\bigcup_{x\in M}(\text{frame of $F_x$}).
\]
A connection is a $\yG$-valued $1$-form; in our case it is a map
\[
  \dpt{\omega^{\alpha}\bxi}{T\bxi\big(\pi^{-1}(\mU_{\alpha})\big)}{\gl(r,\eK)}.
\]
We define our connection by, for $g\in GL(r,\eK)$, $x\in \mU_{\alpha}$, $X_x\in T_xM$ and $A\in \gl(r,\eK)$,
\begin{equation}\label{eq:def_omega_frame}
  \omega_{S_{\alpha}(x)\cdot g}^{\alpha}
         \big(   {R_g}_*s_{\alpha}(x)_* X_x + A^*_{S_{\alpha}(x)\cdot g}  \big):=A+\Ad(g^{-1})\theta_{\alpha}(X_x).
\end{equation}
where $\dpt{S_{\alpha}}{\mU_{\alpha}}{P}$ is the section defined by the trivialization $\phi^P_{\alpha}$:
\[
   S_{\alpha}(x)=\{ \ovv_{\alpha}={\phi^E_{\alpha}}^{-1}(x,e_i) \}_{i=1,\ldots,r}.
\]
Since $\theta_{\alpha}(X_x)\in\End(\eK^r)\subset\gl(r,\eK)$, the second term of \eqref{eq:def_omega_frame} makes sense. This formula is a good definition of $\omega$ because of the following lemma:

\begin{lemma}
If $\xi=S_{\alpha}(x)\cdot g$ and $\Sigma\in T\bxi P$, there exists a choice of $A\in\yG$, and $X_x\in T_xM$ such that
\begin{equation}\label{eq:geneSigma}
  \Sigma={R_g}_*{s_{\alpha}(X)}_* X_x+A^*_{S_{\alpha}(x)\cdot g}.
\end{equation}
\end{lemma}

\begin{proof}
If $\xi\in P$ is a basis of $E$ at $y$, there exists only one choice of $x\in M$ and $g\in G$ such that $\xi=S_{\alpha}(x)\cdot g$.

Let us consider a general path $\dpt{c}{\eR}{P}$ under the form $c(t)=s_{\alpha}(x(t))\cdot g(t)$ where $x$ and $g$ are path in $M$ and $GL(r,\eK)$. The frame $c(t)$ is the one of $F_{x(t)}$ obtained by the transformation $g(t)$ from $s_{\alpha}(x(t))$. It is a set of $r$ vectors, and each of them can be written as a combination of the vectors of $s_{\alpha}(x(t))$, so we write
\begin{equation}
  c^i(t)=s_{\alpha}^j(x(t))g_j^i(t)
\end{equation}
where $s_{\alpha}^j(x(t))\in F_{x(t)}$ and $g_j^i(t)\in\eK$. We compute $\Sigma=c'(0)$ by using the Leibnitz rule and we denote $x'(0)=X_x$, $x(0)=x$ and $g^i_j(0)=g^i_j$ (the matrix of $g$):
\begin{equation}
\begin{split}
  \Sigma^i&=\Dsdd{  s^j_{\alpha}(x(t))  }{t}{0}g^i_j+s^j_{\alpha}(x)\Dsdd{g^i_j(t)}{t}{0}\\
          &=(ds_{\alpha}^j)_xX_xg^i_j+{g^i_j}'(0)s^j_{\alpha}(x).
\end{split}
\end{equation}
Going to more compact matrix form, it gives
\[
  \Sigma=(ds_{\alpha})_xX_x\cdot g+s_{\alpha}(x)g'(0).
\]
The second term, $s_{\alpha}^j(x)g'^i_j(0)$, is a general vector tangent to a fiber. So it can be written as a fundamental field $A^*\bxi$.

\end{proof}

\begin{lemma}
On $\mU_{\alpha}\cap\mU_{\beta}$, the form fulfills $\omega^{\alpha}=\omega\hbeta$.
\end{lemma}

\begin{proof}
Let $\dpt{\gamma}{\eR}{M}$ be a path whose derivative is $X_x$. Then
\begin{equation}
\begin{split}
   (R_g)_*s_{\alpha}(x)_*X_x&=\Dsdd{s_{\alpha}(\gamma_t)\cdot g}{t}{0}
                          =\Dsdd{  s_{\beta}(\gamma_t)g_{\alpha\beta}(\gamma_t)\cdot g  }{t}{0}\\
                          &=\Dsdd{ s_{\alpha}(\gamma_t)g_{\alpha\beta}(x)\cdot g }{t}{0}
                          +\Dsdd{ s_{\beta}(x)\cdot g_{\alpha\beta}(\gamma_t)\cdot g }{t}{0}.
\end{split}
\end{equation}
What is in the derivative of the first term is $R_{g_{\alpha\beta}(x)g}(s_{\beta}(\gamma_t))$. Taking the derivative, we find the expected ${R_{g_{\alpha\beta}(x)g}}_*{s_{\beta}}_*X_x$.

For the second term, we note $r:=s_{\beta}(x)\cdot g_{\alpha\beta}(g)g$, and we have to compute the following, using equation \eqref{eq:rdotht},
\begin{equation}
\begin{aligned}
 \Dsdd{  r\cdot\Ad_{g^{-1}}( g_{\alpha\beta}^{-1}(x)&g_{\alpha\beta}(\gamma_t) ) }{t}{0}\\
                  &=\Dsdd{ r\cdot\exp t\big(    (d\AD_{g^{-1}})_e( g_{\alpha\beta}^{-1}(x)(dg_{\alpha\beta})_xX_x )   \big)}{t}{0}\\
                  &=\Dsdd{r\cdot\exp t\big( \Ad_{g^{-1}}g_{\alpha\beta}^{-1}(x)dg_{\alpha\beta}}{t}{0}\\
                  &=\left(  \Ad_{g^{-1}}g^{-1}_{\alpha\beta}(x)dg_{\alpha\beta} X_x    \right)^*_r.
\end{aligned}
\end{equation}
Using this, we can perform the computation:
\begin{equation}
\begin{aligned}
\omega\hbeta_{S_{\alpha}(x)\cdot g}\big(  {R_g}_*{s_{\alpha}(x)}_*X_x+A^*_{S_{\alpha}(x)\cdot g}  \big)
                          &=\omega\hbeta_{S_{\beta}(x)\cdot g_{\alpha\beta}(x)g}\Big(  {R_{g_{\alpha\beta}}(x)g}_*{s_{\beta}(x)}_*X_x \\
                              &\qquad + (  \Ad_{g^{-1}}g^{-1}_{\alpha\beta}(x)dg_{\alpha\beta} X_x  )^*_r +A^* \Big)\\
                          &=\Ad_{(g_{\alpha\beta}(x)g)^{-1}}\theta_{\beta}(X_x)\\
			&\quad+\Ad_{g^{-1}}g_{\alpha\beta}^{-1}(x)dg_{\alpha\beta}(X_x)+A\\
                          &=\Ad_{g^{-1}}\big(  (g_{\alpha\beta}^{-1}\theta_{\beta} g_{\alpha\beta}+g_{\alpha\beta}^{-1} dg_{\alpha\beta})(X_x)  \big)+A\\
                          &=\omega^{\alpha}_{S_{\alpha}(x)g}\big( {R_g}_*{s_{\alpha}(x)}_*X_x+A^A_{S_{\alpha}(x)\cdot g} \big).
\end{aligned}
\end{equation}

\end{proof}

\begin{proposition}
The $\omega$ defined by formula \eqref{eq:def_omega_frame} is a connection $1$-form.
\label{prop_omconfrom}
\end{proposition}

\begin{proof}
The first condition, $\omega(A^*\bxi)=A$, is immediate from the definition. The lemma \ref{lem:dRgAstar} gives the second condition in the case $\Sigma=A^*\bxi$. It remains to be checked that $\omega(dR_g\Sigma)=\Ad(g^{-1})\omega(\Sigma)$ in the case $\Sigma=dR_hds_{\alpha} X_x$. This is obtained using the fact that $\Ad$ is a homomorphism.
\end{proof}

\subsubsection{Levi-Civita connection}\label{subsubsec_levi}
%---------------------------------------------

Let $(M,g)$ be a Riemannian manifold. We look at a connection $1$-form $\alpha\in\Omega^1(\SO(M),so(\eR^m))$ on $\SO(M)$, and we define a covariant derivative $\dpt{\nabla^{\alpha}}{\cvec(M)\times T(M)}{T(M)}$, where $T(M)$ is the tensor bundle on $M$ by (cf. theorem \eqref{tho_nablaE})
\begin{eqnarray}\label{r2804e1}
 \widehat{\nabla^{\alpha}_X s}=\overline{X}\hat{s},
\end{eqnarray}
for any $s\in T(M)$.  Our purpose now is to prove that an automatic property of this connection is $\nabla^{\alpha} g=0$. The unique such connection which is torsion-free is the \defe{Levi-Civita}{Levi-Civita connection} one.

The metric $g$ is a section of the tensor bundle $T^*M\otimes T^*M$. So we have, in order to find $\hg$ and to use equation \eqref{r2804e1}, to see $T^*M\otimes T^*M$ as an associated bundle. As done in \ref{equivvec}, we see that
\[
 T^*M\otimes T^*M\simeq \SO(M)\times_{\rho}(V^*\otimes V^*),
\]
with the following definitions:
\begin{itemize}
\item The isomorphism is given by $\psi[b,\alpha\otimes\beta](X\otimes Y)=\alpha(b^{-1} X)\beta(b^{-1} Y)$,
\item $\rho(A)\alpha=\alpha\circ A$,
\item $b\cdot A=b\circ A$.
\end{itemize}
Here, $V=\eR^m$; $\dpt{b}{V}{T_xM}$; $\alpha,\beta\in V^*$; $X$, $Y\in T_xM$ and $A\in \SO(m)$ is seen as $\dpt{A}{V}{V}$.

The following shows that $\psi$ is well defined:
\begin{equation}
\begin{split}
 \psi[b\cdot A,\rho(A^{-1})\alpha\otimes\beta](X\otimes Y)&=(\alpha\circ A)
                            (A^{-1}\circ b^{-1} X)(\beta\circ A)(A^{-1}\circ b^{-1} Y)\\
                                           &=\psi[b,\alpha\otimes\beta](X\otimes Y)
\end{split}
\end{equation}

\begin{proposition}
The function $\hg$ is given by
\[
 \hg(b)(v\otimes w)=g_x(b(v)\otimes b(w))=v\cdot w.
\]
\end{proposition}

\begin{proof}
The second equality is just the fact that $\dpt{b}{(\eR^m,\cdot)}{(T_xM,g_x)}$ is isometric. On the other hand, if $\hg(b)=\alpha\otimes\beta$, we have:
\begin{equation}
\begin{split}
 g_x(X\otimes Y)&=\psi[b,\alpha\otimes\beta](X\otimes Y)
                =\alpha(b^{-1} X)\beta(b^{-1} Y)\\
                &=\alpha\otimes\beta(b^{-1} X\otimes b^{-1} Y)
                =\hg(b)(b^{-1} X\otimes b^{-1} Y).
\end{split}
\end{equation}

Since $b$ is bijective, $X$ and $Y$ can be written as $bv$ and $bw$ respectively for some $v$, $w\in V$, so that
\[g_x(bv\otimes bw)=\hg(b)v\otimes w.\]
\end{proof}

It is now easy to see that $\oX\hg=0$. As $\hg$ takes its values in $V^*\otimes V^*$, $\oX\hg$ belongs to this space and can be applied on $v\otimes w\in V\otimes V$. Let $\oX(t)$ be a path in $\SO(M)$ which defines $\oX$; if $\oX\in T_b\SO(M)$, $\oX(0)=b$. We have
\begin{equation}
 \oX\hg(v\otimes w)	=\dsdd{\hg(\oX(t))v\otimes w}{t}{0}
			=\dsdd{v\cdot w}{t}{0},
\end{equation}
which is obviously zero.

\subsection{Holonomy}
%--------------------

Let the principal bundle 
\begin{equation}
\xymatrix{%
   G \ar@{~>}[r]		&	P\ar[d]^{\pi}\\
   				&	M
 }
\end{equation}
  and $\omega$ a connection on $G$. Let $\gamma\colon [0,1]\to M$, a closed curve piecewise smooth; $\gamma(0)=\gamma(1)=x$. For each $p\in\pi^{-1}(x)$, there exists one and only one horizontal lift $\tilde\gamma\colon [0,1]\to P$ such that $\tilde\gamma(0)=p$. There exists of course an element $g\in G$ such that $\tilde\gamma(1)=p\cdot g$.

We define the following equivariance relation on $P$: we say that $p\sim q$ if and only if $p$ and $q$ can be joined by a piecewise smooth path. The \defe{holonomy group}{holonomy group} at the point $p$ is 
\[ 
  \Hol_p(\omega)=\{ g\in G\tq p\sim p\cdot g \}.
\]
\subsection{Connection and gauge transformation}
%-----------------------------------------------


\begin{proposition}
If $\omega$ is a connection on a $G$-principal bundle and $\varphi$, a gauge transformation, the form $\beta=\varphi^*\omega$ is a connection $1$-form too.
\label{prop:vp_conn}
\end{proposition}

\begin{proof}
It is rather easy to see that $\varphi_*A^*\bxi=A^*_{\varphi(x)}$:
\[
  \varphi_*A^*\bxi=\Dsdd{ \varphi(\xi e^{-tA})  }{t}{0}=\Dsdd{ \varphi(\xi)e^{-tA}  }{t}{0}=A^*_{\varphi(x)}.
\]
The same kind of reasoning leads to $\varphi_*{R_g}_*={R_g}_*\varphi_*$. From here, it is easy to see that
\[
  (\varphi^*\omega)\bxi(A^*\bxi)=\omega_{\varphi(\xi)}(\varphi_*A^*\bxi)=A,
\]
and
\[
\big( R^*_g(\varphi^*\omega)\bxi \big)(\Sigma)
                =(R_g^*\omega)_{\varphi(\xi)}(\varphi_*\Omega)=\Ad(g^{-1})\big( (\varphi^*\omega)\bxi(\Sigma) \big).
\]
\end{proof}
So, the ``gauge transformed'' of a connection is still a connection. It is hopeful in order to define gauge invariants objects (Lagrangian) from connections (electromagnetic fields).

\subsubsection{Local description}

Let $\dpt{\pi}{P}{M}$ be a $G$-principal bundle given with some trivializations $\dpt{\phi^P_{\alpha}}{\pi^{-1}(\mU_{\alpha})}{\mU_{\alpha}\times G}$ over $\mU_{\alpha}\subset M$ and $\dpt{s_{\alpha}}{\mU_{\alpha}}{\pi^{-1}(\mU_{\alpha})}$, a section. In front of that, we consider an associated bundle $\dpt{p}{E=P\times_{\rho} V}{M}$ with a trivialization $\dpt{\phi^E_{\alpha}}{E}{\mU_{\alpha}\times V}$. One can choose a section $s_{\alpha}$ compatible with the trivialization in the sense that $\phi^P_{\alpha}(s_{\alpha}(x)\cdot g)=(x,g)$; the same can be done with $E$ by choosing $\phi^E_{\alpha}([s_{\alpha}(x),v])=(x,v)$.

A section\index{section!local description} $\dpt{\psi}{M}{E}$ is described by a function $\dpt{\psi_{\alpha}}{\mU_{\alpha}}{V}$ defined by $\phi^E_{\alpha}(\psi(x))=(x,\psi_{\alpha}(x))$.  In the inverse sense, $\psi$ is defined (on $\mU_{\alpha}$) from $\psi_{\alpha}$ by
$\psi(x)=[s_{\alpha}(x),\psi_{\alpha}(x)]$.
In the same way, a gauge transformation\index{gauge!transformation!local description} $\dpt{\varphi}{P}{P}$ is described by functions $\dpt{\tilde{\varphi}_{\alpha}}{\mU_{\alpha}}{G}$,
\begin{equation}
  \varphi(s_{\alpha}(x))=s_{\alpha}(x)\cdot\tilde{\varphi}_{\alpha}(x).
\end{equation}
The function $\tilde{\varphi}_{\alpha}$ also fulfil
\begin{equation}
  (\phi_{\alpha}^P\circ\varphi\circ{\phi_{\alpha}^P}^{-1})(x,g)=(x,\tilde{\varphi}(x)\cdot g) 
\end{equation}
because
\begin{equation}
\begin{split}
  (\phi_{\alpha}^P\circ\varphi\circ{\phi_{\alpha}^P}^{-1})(x,g)&=(\phi_{\alpha}^P\circ\varphi)(s_{\alpha}(x)\cdot g)\\
                                                      &=\phi_{\alpha}^P( \varphi(s_{\alpha}(x))\cdot g )\\
                                                      &=\phi_{\alpha}^P( s_{\alpha}(c)\cdot\tilde{\varphi}_{\alpha}(x)g)\\
                                                      &=(x,\tilde{\varphi}_{\alpha}(x)g).
\end{split}
\end{equation}

We know that a connection on $P$ is given by its $1$-form $\omega$. Moreover we have the following:
\begin{proposition}
A connection on $P$ is completely determined on $\pi^{-1}(\mU_{\alpha})$ from the data of the $\yG$-valued $1$-form $\sigma_{\alpha}^*\omega$ on $\mU_{\alpha}$.
\end{proposition}

\begin{proof}
We consider a $1$-form $\omega$ which fulfils the two conditions of page \pageref{pg:def:conne}. Our purpose is to find back $\omega\bxi(\Sigma)$, $\forall\xi\in P,\Sigma\in T\bxi P$ from the data of $\sigma_{\alpha}^*\omega$ alone. For any $\xi$, there exists a $g$ such that $\xi=\sigma_{\alpha}(x)\cdot g$. We have
\begin{equation}
  \Ad_{g^{-1}}(\omega_{\sigma_{\alpha}(x)\Sigma})=(R^*_g\omega)_{\sigma_{\alpha}(x)}(\Sigma)
             =\omega_{\sigma_{\alpha}(x)\cdot g}\big( (dR_g)_{\sigma_{\alpha}(x)}\Sigma \big).
\end{equation}
If we know $s_{\alpha}^*\omega$, then we know $\omega\big(  (ds_{\alpha})_xv  \big)$ for any $v\in T_xM$. So 
\[
   \omega_{\sigma_{\alpha}(x)\cdot g}\big( (dR_g)_{\sigma_{\alpha}(x)}\Sigma\big)
\]
is given from $\sigma_{\alpha}^*\omega$ for every $\Sigma$ of the form $\Sigma=(d\sigma_{\alpha})_xv$. From the form \eqref{eq:geneSigma} of a vector in $T\bxi P$, it just remains to express $\omega_{\sigma_{\alpha}(x)\cdot g}(A^*_{\sigma_{\alpha}(x)\cdot g})$ in terms of $s_{\alpha}^*$. The definition of a connection makes that it is simply $A$.

\end{proof}

\subsubsection{Covariant derivative}

If we have a connection on $P$, we can define a covariant derivative on the associated bundle $E$ by 
\[
  (\nabla_X\psi)\bsa(x)=X_x(\psi_{\alpha})+\rho_*( s_{\alpha}^*\omega_x(X) )\psi\bsa(x),
\]
the matricial $1$-form being given by $\theta_{\alpha}=\rho_*\sigma^*_{\alpha}\omega$. The gauge transformation $\varphi$ acts on the connection $\omega$ by defining $\omega^{\varphi}:=\varphi^*\omega$.

\begin{proposition}
If $\beta=\varphi^*\omega$, then
\[
   s^*_{\alpha}(\beta)=\Ad_{\tilde{\varphi}_{\alpha}(x)^{-1}}s^*_{\alpha}(\omega)+\tilde{\varphi}_{\alpha}(x)^{-1} d\tilde{\varphi}_{\alpha}.
\]
\end{proposition}

\begin{proof}
Let $\dpt{\gamma}{\eR}{M}$ be a path such that $\gamma(0)=x$ and $\gamma'(0)=X_x$. We have to compute the following:
\begin{equation}\label{eq:ppu}
  (s_{\alpha}^*\beta)(X_x)=(s_{\alpha}^*\varphi^*\omega)(X_x)=\omega_{(\varphi\circ s_{\alpha})(x)}\big(  (\varphi\circ s_{\alpha})_*X_x  \big).
\end{equation}
What lies in the derivative is:
\begin{equation}
\begin{split}
  (\varphi\circ s_{\alpha})_*(X_x)&=\Dsdd{ (\varphi\circ s_{\alpha}\circ\gamma)(t) }{t}{0}\\
                            &=\Dsdd{  s_{\alpha}(\gamma(t))\cdot\tilde{\varphi}_{\alpha}(\gamma(t))  }{t}{0}\\
                            &=\Dsdd{ s_{\alpha}(\gamma(t))\cdot\tilde{\varphi}_{\alpha}(\gamma(0)) }{t}{0}
                             +\Dsdd{ s_{\alpha}(\gamma(0))\cdot\tilde{\varphi}_{\alpha}(\gamma(t)) }{t}{0}\\
                            &={R_{\tilde{\varphi}_{\alpha}(x)}}_*{s_{\alpha}}_*X_x
                             +\Dsdd{  s_{\alpha}(x)\cdot\tilde{\varphi}_{\alpha}(x)e^{ t\tilde{\varphi}_{\alpha}(x)^{-1}(d\tilde{\varphi}_{\alpha})_x\gamma'(0)}}{t}{0}.
\end{split}
\end{equation}
A justification of the remplacement $\tilde{\varphi}_{\alpha}(\gamma(t))\to \tilde{\varphi}_{\alpha}(x)e^{t\tilde{\varphi}_{\alpha}(x)^{-1}(d\tilde{\varphi}_{\alpha})_x\gamma'(0)}$ is given in the corresponding proof at page \pageref{pg:justif_s}.
If we put this expression into equation \eqref{eq:ppu}, the first term becomes
\[
\begin{split}
   \omega_{(\varphi\circ s_{\alpha})(x)}\big(  {{R_{\tilde{\varphi}_{\alpha}(x)}}_*{s_{\alpha}}_*X_x}   \big)
           &=(R^*_{\tilde{\varphi}_{\alpha}(x)}\omega)_{s_{\alpha}(x)}({s_{\alpha}}_*X_x)\\
           &=\Ad_{\tilde{\varphi}_{\alpha}(x)^{-1}} \big(\omega_{s_{\alpha}(x)} ( {s_{\alpha}}_*X_x ) \big)\\
           &=\Ad_{\tilde{\varphi}_{\alpha}(x)^{-1}}  (s_{\alpha}^*\omega)(X_x).
\end{split}
\]
The second term is the case of a connection applied to a fundamental vector field.

\end{proof}

\section{Product of principal bundle}\label{sec:produit_bundle}
%++++++++++++++++++++++++++++++++++++

In this section, we build a $G_1\times G_2$-principal bundle from the data of a $G_1$ and a $G_2$-principal bundle. The physical motivation is clear: as far as electromagnetism is concerned, particles are sections of $U(1)$-principal bundle while the relativistic invariance must be expressed by means of a $\SLdc$-associated bundle. So the physical fields must be sections of something as the product of the two bundles. See subsection \ref{subsec:incl_Lorentz}.

\subsection{Putting together principal bundle}
%------------------------------------

Let us consider two principal bundle over the same base space
\[
\xymatrix{
    G_1  \ar@{~>}[r] & P_1 \ar[r]^{p_1}& M,}
\]
and
\[
\xymatrix{
    G_2  \ar@{~>}[r] & P_2 \ar[r]^{p_2}& M.
  }
\]
First we define the set
\begin{equation}
  P_1\circ P_2=\{   (\xi_1,\xi_2)\in P_1\times P_2\tq p_1(\xi_1)=p_2(\xi_2)    \}
\end{equation}
which will be the total space of our new bundle. The projection $\dpt{p}{P_1\circ P_2}{M}$ is naturally defined by
\[
  p(\xi_1,\xi_2)=p_1(\xi_1)=p_2(\xi_2),
\]
while the right action of $G_1\times G_2$ on $P_1\circ P_2$ is given by
\[
  (\xi_1,\xi_2)\cdot(g_1,g_2)=(\xi_1\cdot g_1,\xi_2\cdot g_2)
\]
With all these definitions,
\[
\xymatrix{
    G_1\times G_2  \ar@{~>}[r] & P_1\circ P_2 \ar[d]^{p}\\& M\\
  }
\]
is a $G_1\times G_2$-principal bundle over $M$. We define the natural projections 
		\begin{equation}
		\begin{aligned}
			\pi_i \colon P_1\times P_2 &\to P_i\
			(\xi_1, \xi_2)&\mapsto \xi_i,
		\end{aligned}
	\end{equation}	
%
and if $e_i$ denotes the identity element of $G_i$, we can identify $G_1$ to $G_1\times \{e_2\}$ and $G_2$ to $G_2\times \{e_1\}$; in the same way, $\yG_1=\yG_1\times\{0\}\subset\yG_1\times\yG_2$. So we get the following principal bundles:
\[
\xymatrix{
    G_2  \ar@{~>}[r] & P_1\circ P_2 \ar[r]^{\pi_1}& P_1\\
    G_1  \ar@{~>}[r] & P_1\circ P_2 \ar[r]^{\pi_2}& P_2.
  }
\]
It is clear that the following diagram commutes:
\[
\xymatrix{
    P_1  \ar[rd]_{P_1} & P_1\circ P_2\ar[r]^{\pi_2} \ar[l]_{\pi_1}\ar[d]_p& P_2 \ar[ld]^{p_2}\\
    &M
  }
\]

\subsection{Connections}
%-----------------------

Let $\omega_i$ be a connection on the bundle $\dpt{p_i}{P_i}{M}$. Using the identifications, $\pi_1^*\omega_1$ is a connection on $\dpt{\pi_2}{P_1\circ P_2}{P_2}$ (the same is true for $1\leftrightarrow 2$), and $\pi_1^*\omega_1\oplus\pi_2^*\omega_2$ is a connection on $\dpt{p}{P_1\circ P_2}{M}$. Let us prove the first claim.

Let $A\in\yG_1$. We first have to prove that $\pi_1^*\omega_1(A^*)=A$. For this, remark that $A=(A,0)\in\yG_1\oplus\yG_2$ and
\begin{equation}
   A^*\bxi=\Dsdd{\xi\cdot e^{-t(A,0)}}{t}{0}
          =\Dsdd{(\xi_1,\xi_2)\cdot(e^{-tA},e_2)}{t}{0}
          =\Dsdd{(\xi_1\cdot e^{-tA},\xi_2)}{t}{0},
\end{equation}
so $d\pi_1A^*=\Dsdd{\pi_1(\ldots)}{t}{0}=\omega_1(A^*)=A$. Let now $\Sigma\in T_{(\xi_1,\xi_2)}(P_1\circ P_2)$ be given by the path $(\xi_1(t),\xi_2(t))$. In this case we have
\begin{equation}
\begin{split}
   \big(  R^*_{(g,e_2)\pi_1^*\omega_1}  \big)_{(\xi_1,\xi_2)}\Sigma&=
      (\pi_1^*\omega_1)(dR_{(g,e_2)}\Sigma)\\
     &=\omega_1( \Dsdd{\xi_1(t)\cdot g}{t}{0}  )\\
     &=\omega_1( dR_g\Dsdd{\xi_1(t)}{t}{0} )\\
     &=\Ad(g^{-1})\pi_1^*\omega_1( \Dsdd{( \xi_1(t),\xi_2(t) )}{t}{0} )\\
     &=\Ad(g^{-1})\pi_1^*\omega_1\Sigma.
\end{split}
\end{equation}

\subsection{Representations}
%----------------------------

Let $V$ be a vector space and $\dpt{\rho_i}{G_i}{GL(V)}$ be some representations such that 
\begin{equation}\label{eq:cond_reprez}
   [\rho_1(g_1),\rho_2(g_2)]=0
\end{equation}
for all $g_1\in G_1$ and $g_2\in G_2$ (in the sense of commutators of matrices). In this case, one can define the representation $\dpt{\rho_1\times\rho_2}{G_1\times G_2}{GL(V)}$ by
\begin{equation}
   (\rho_1\times\rho_2)(g_1,g_2)=\rho_1(g_1)\circ\rho_2(g_2)=\rho_2(g_2)\circ\rho_1(g_1).
\end{equation}
The relation \eqref{eq:cond_reprez} is needed in order for $\rho_1\times\rho_2$ to be a representation, as one can check by writing down explicitly the requirement
\[
  (\rho_1\times \rho_2)\big(  (g_1,g_2)(g'_1,g'_2)  \big)=(\rho_1\times \rho_2)(g_1g'_1,g_2g'_2) 
\]

%+++++++++++++++++++++++++++++++++++++++++++++++++++++++++++++++++++++++++++++++++++++++++++++++++++++++++++++++++++++++++++
\section{Hodge decomposition theorem and harmonic forms}
%+++++++++++++++++++++++++++++++++++++++++++++++++++++++++++++++++++++++++++++++++++++++++++++++++++++++++++++++++++++++++++

Among other sources for Hodge decomposition and harmonic forms, we have \cite{JohnsonHodge,CohoHarBound,UndergradDeRham}. Some parts of the wikipedia article \wikipedia{en}{Hodge_dual}{Hodge\_dual} are interesting.

Let $E$ be an oriented Euclidian space of dimension $m=2n$. We define the operation $*$ by
\begin{equation}		\label{EqGradWedge}
	\begin{aligned}
		*\colon \Wedge E&\to \Wedge E \\
		e_{i_1}\wedge e_{i_2}\wedge\cdots\wedge e_{i_k}&\mapsto e_{i_{k+1}}\wedge\cdots\wedge e_{i_m} 
	\end{aligned}
\end{equation}
when $\{ e_i \}$ is an oriented basis of $E$ and $\{ i_k \}$ is an even permutation of $\{ 1,2,\cdots,m \}$. If it is impossible to build an even permutation, then we add a minus sign. We have $**\omega=(-1)^{p(m-p)}\omega$ belongs to $\omega\in\Wedge^pE$.
\begin{example}
	If we consider the space $\eR^4$ with the coordinates $(x,y,z,t)$,
	\begin{equation}
		*(dx\wedge dz\wedge dt)=dy
	\end{equation}
	because $(x,z,t,y)$ is an even permutation of $(x,y,z,t)$. Now, $*dy=dz\wedge dx\wedge dt=-(dx\wedge dz\wedge dt)$ because $(y,z,x,t)$ is an even permutation of $(x,y,z,t)$.

	More generally, if we have a differential $p$-form $\omega$ on a $m$ dimensional space, we have
	\begin{equation}
		*(e_{\sigma(1)}\wedge e_{\sigma(2)}\wedge\ldots\wedge e_{\sigma(p)})=e_{\sigma(p+1)}\wedge \ldots\wedge e_{\sigma(m)}
	\end{equation}
	In order to compute $*(e_{\sigma(p+1)}\wedge \ldots\wedge e_{\sigma(m)})$, we need a permutation of $\big( \sigma(1),\ldots, \sigma(m)\big)$ which \emph{begins} by $\sigma(p+1)\ldots\sigma(m)$. This reduces to permute the $m-p$ elements $\sigma(p+1),\ldots,\sigma(m)$ with the $p$ first elements. Thus we have
	\begin{equation}
		**\omega=(-1)^{p(m-p)}\omega.
	\end{equation}
\end{example}

Let $V$ be a compact, oriented manifold. Each of the spaces of sections $ C^{\infty}\big( V, \Wedge^k_{\eC}(T^*V)\big)$ is endowed with a $2$-form
\begin{equation}		\label{EqProdWedgeHOfge}
	\langle \omega_1, \omega_2\rangle =\int_V\omega_1\wedge *\omega_2.
\end{equation}

\begin{lemma}
	The \defe{codifferential}{codifferential} $\delta$ defined by
	\begin{equation}
		\begin{aligned}[]
			\delta	\colon \Wedge^k_{\eC}(T^*V)&\to \Wedge_{\eC}^{k-1}(T^*V)\\
			\beta				&\mapsto (-1)^{mk+m+1}*d*\beta
		\end{aligned}
	\end{equation}
	is a formal adjoint of $d$ for the product \eqref{EqProdWedgeHOfge}.
\end{lemma}

\begin{proof}
	If $\beta\in\Wedge_{\eC}^k(T^*V)$ and $\alpha\in\Wedge_{\eC}^{k-1}(T^*V)$, we have
	\begin{equation}
		\begin{aligned}[]
			*\delta\beta&=(-1)^{mk+m+1}* *\big( d*\beta \big)\\
			&=(-1)^{mk+m+1}(-1)^{(m-k+1)(m-m+k-1)}d*\beta\\
			&=(-1)^kd*\beta.
		\end{aligned}
	\end{equation}
	Using that formula we find
	\begin{equation}
		\begin{aligned}[]
			\langle d\alpha, \beta\rangle -\langle \alpha, \delta\beta\rangle &=\int_V d\alpha\wedge *\beta-\alpha\wedge *\delta\beta\\
			&=\int_Vd\alpha\wedge *\beta-(-1)^k\alpha\wedge d*\beta\\
			&=\int_Vd\alpha\wedge *\beta+(-1)^{k+1}\alpha\wedge d*\beta\\
			&=\int_Vd(\alpha\wedge *\beta)\\
			&=\int_{\partial V}\alpha\wedge *\beta\\
			&=0.
		\end{aligned}
	\end{equation}
	This proves that $\langle d\alpha, \beta\rangle =\langle \alpha, \delta\beta\rangle$.
	\begin{probleme}
		I do not understand why the integral in the boundary is zero. 
	\end{probleme}	
\end{proof}

Now we define the \defe{Laplace-Beltrami operator}{Laplace-Beltrami operator}\index{operator!Laplace-Beltrami} by\nomenclature[D]{$\Delta$}{Laplace-Beltrami operator}
\begin{equation}
	\Delta=\delta d+d\delta
\end{equation}
and the space of \defe{harmonic forms}{harmonic form}
\begin{equation}
	H^k=\{ \omega\in\Omega^k\tq\Delta\omega=0 \}.
\end{equation}

\begin{lemma}
	If $M$ is a closed manifold, a $k$-form is harmonic if and only if $d\omega=\delta\omega=0$.
\end{lemma}

\begin{proof}
	No proof.
\end{proof}

\begin{theorem}[Hodge decomposition theorem]\index{theorem!Hodge decomposition}
	For every integer $0\leq k\leq m$, the space $H^p$ is finite dimensional and $\Omega^k(M)$ has the orthogonal decomposition
	\begin{equation}
		\Omega^k(M)=H^k\oplus\Delta\big( \Omega^k(M) \big),
	\end{equation}
	i.e. the space splits into the kernel of $\Delta$ and its image.
\end{theorem}

\begin{theorem}
	Let $M$ be a compact orientable manifold of dimension $m$. Any exterior differential $k$-form can be written as a unique sum of an exact form, a coexact form and an harmonic form :
	\begin{equation}
		\omega=d\alpha+\delta\beta+\gamma.
	\end{equation}
	with $\omega\in\Omega^k(M)$, $\alpha\in\Omega^{k-1}(M)$, $\beta\in\Omega^{k+1}(M)$ and $\gamma\in\Omega^k(M)$ harmonic.
\end{theorem}

The operator $\Delta$ commutes with the differential $d$ and we have $d\Delta=\Delta d= d\delta d$ since
\begin{equation}
	d\Delta \omega=dd\delta\omega+d\delta d\omega=d\delta d\omega,
\end{equation}
because $d^2=0$, while
\begin{equation}
	\Delta d\omega=d\delta d\omega+\Delta d d\omega=d\delta d\omega.
\end{equation}

\begin{lemma}
	On a close manifold, $\Delta\omega=0$ if and only if $\delta\omega=d\omega=0$.
\end{lemma}

In the case of a closed manifold, a form is harmonic if and only if is belongs to the kernel of $d+\delta$. Moreover, a form in the image of $d+\delta$ is orthogonal to the harmonic forms :
\begin{equation}
	\langle d\alpha^{k-1}+\delta\beta^{k+1}, \gamma\rangle =0
\end{equation}
whenever $\gamma$ is harmonic on a closed manifold.
