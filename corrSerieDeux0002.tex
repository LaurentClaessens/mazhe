% This is part of Exercices et corrections de MAT1151
% Copyright (C) 2010
%   Laurent Claessens
% See the file LICENCE.txt for copying conditions.

\begin{corrige}{SerieDeux0002}

	\begin{enumerate}

		\item
			Si $\alpha(x)=\lambda x$, nous devons faire le calcul
			\begin{equation}
				\sup_{| x |=1}\{ | \lambda x | \}=| \lambda |.
			\end{equation}
		\item
			Si $A$ est une rotation, par définition la norme de $Ax$ est la même que celle de $x$, donc
			\begin{equation}
				\sup_{| x |=1}\{ \| Ax \| \}=1.
			\end{equation}
			Il est à noter que la norme $\| . \|$ dans le supremum est la norme sur $\eR^2$.
		\item
			Si nous prenons une base orthonormée dans laquelle un des vecteurs est $(1,1,1)$, nous retombons sur le cas précédent. Souvenez vous qu'on peut passer d'une base orthonormée à une autre avec des isométries. Le changement de base n'affecte donc pas la norme dont on prend le supremum.

	\end{enumerate}

\end{corrige}
