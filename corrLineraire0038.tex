% This is part of the Exercices et corrigés de mathématique générale.
% Copyright (C) 2009
%   Laurent Claessens
% See the file fdl-1.3.txt for copying conditions.
\begin{corrige}{Lineraire0038}

	\begin{enumerate}

		\item
			


	La symétrie par rapport au plan $x=-z$ doit laisser invariante les vecteurs qui sont dans le plan, c'est à dire les vecteurs tels que $x=-z$ :
	\begin{equation}
		\begin{aligned}[]
			\begin{pmatrix}
				0	\\ 
				1	\\ 
				0	
			\end{pmatrix},&&\text{et}&&\begin{pmatrix}
				1	\\ 
				0	\\ 
				-1	
			\end{pmatrix}.
		\end{aligned}
	\end{equation}
	En même temps, la symétrie doit changer le signe du vecteur $\begin{pmatrix}
		1	\\ 
		0	\\ 
		1	
	\end{pmatrix}$ qui est perpendiculaire au plan donné. Afin d'écrire la matrice de l'application, il faut trouver l'image des vecteurs de bases $e_1$, $e_2$ et $e_3$. Pour cela, il faut décomposer ces vecteurs de la base canonique dans la base donnée par
	\begin{equation}
		\begin{aligned}[]
			\begin{pmatrix}
				0	\\ 
				1	\\ 
				0	
			\end{pmatrix},&&\begin{pmatrix}
				1	\\ 
				0	\\ 
				-1	
			\end{pmatrix},&&\text{et}&&\begin{pmatrix}
				1	\\ 
				0	\\ 
				1	
			\end{pmatrix}.
		\end{aligned}
	\end{equation}
	Nous trouvons
	\begin{equation}
		\begin{pmatrix}
			1	\\ 
			0	\\ 
			0	
		\end{pmatrix}=\frac{1}{ 2 }\begin{pmatrix}
			1	\\ 
			0	\\ 
			-1	
		\end{pmatrix}+\frac{ 1 }{2}\begin{pmatrix}
			1	\\ 
			0	\\ 
			1	
		\end{pmatrix},
	\end{equation}
	donc, en nous rappelant quel vecteur reste inchangé par $A$ et quel doit changer de signe,
	\begin{equation}
		Ae_1=A\begin{pmatrix}
			1	\\ 
			0	\\ 
			0	
		\end{pmatrix}=
		\frac{1}{ 2 }A\begin{pmatrix}
			1	\\ 
			0	\\ 
			-1	
		\end{pmatrix}+\frac{ 1 }{2}A\begin{pmatrix}
			1	\\ 
			0	\\ 
			1	
		\end{pmatrix}=
		\frac{1}{ 2 }\begin{pmatrix}
			1	\\ 
			0	\\ 
			-1	
		\end{pmatrix}-\frac{ 1 }{2}\begin{pmatrix}
			1	\\ 
			0	\\ 
			1	
		\end{pmatrix}=
		\begin{pmatrix}
			0	\\ 
			0	\\ 
			-1	
		\end{pmatrix}.
	\end{equation}
	Vu que nous avons dit que $e_2$ ne changeait pas, nous avons tout de suite
	\begin{equation}
		Ae_2=\begin{pmatrix}
			0	\\ 
			1	\\ 
			0	
		\end{pmatrix}.
	\end{equation}
	Et enfin,
	\begin{equation}
		Ae_3=A\begin{pmatrix}
			0	\\ 
			0	\\ 
			1	
		\end{pmatrix}=-\frac{ 1 }{2}A\begin{pmatrix}
			1	\\ 
			0	\\ 
			-1	
		\end{pmatrix}+\frac{ 1 }{2}A\begin{pmatrix}
			1	\\ 
			0	\\ 
			1	
		\end{pmatrix}=
		-\frac{ 1 }{2}\begin{pmatrix}
			1	\\ 
			0	\\ 
			-1	
		\end{pmatrix}+\frac{ 1 }{2}\begin{pmatrix}
			-1	\\ 
			0	\\ 
			-1	
		\end{pmatrix}=\begin{pmatrix}
			-1	\\ 
			0	\\ 
			0	
		\end{pmatrix}.
	\end{equation}
	Nous pouvons maintenant écrire la matrice de la symétrie en écrivant simplement les vecteurs $Ae_1$, $Ae_2$ et $Ae_3$ en colonne :
	\begin{equation}
		A=\begin{pmatrix}
			0	&	0	&	-1	\\
			0	&	1	&	0	\\
			-1	&	0	&	0
		\end{pmatrix}.
	\end{equation}
	

	\item
		La symétrie par rapport au plan $x=-y$ laisse inchangés les vecteurs
		\begin{equation}
			\begin{aligned}[]
			f_1&=\begin{pmatrix}
				0	\\ 
				0	\\ 
				1	
			\end{pmatrix},&\text{et}&f_2&=\begin{pmatrix}
				1	\\ 
				-1	\\ 
				0	
			\end{pmatrix},
			\end{aligned}
		\end{equation}
		tandis qu'elle change le signe de
		\begin{equation}
			f_3=\begin{pmatrix}
				1	\\ 
				1	\\ 
				0	
			\end{pmatrix}.
		\end{equation}
		
		Pour écrire la matrice, il faut d'abord calculer $Ae_1$, $Ae_2$ et $Ae_3$. Pour ce faire, nous décomposons $e_1$, $e_2$ et $e_3$ dans la base $\{ f_1,f_2,f_3 \}$ :
		\begin{equation}
			\begin{aligned}[]
				e_1&=\frac{ 1 }{2}(f_3+f_3)\\
				e_2&=\frac{ 1 }{2}(f_3-f_2)\\
				e_3&=f_1,
			\end{aligned}
		\end{equation}
		donc
		\begin{equation}
			\begin{aligned}[]
				Ae_1&=\frac{1 }{2}(f_2-f_3)=\begin{pmatrix}
					0	\\ 
					-1	\\ 
					0	
				\end{pmatrix}\\
				Ae_2&=\frac{1 }{2}(-f_3-f_2)=\begin{pmatrix}
					-1	\\ 
					\	\\ 
					0	
				\end{pmatrix}\\
				Ae_3&=f_1=\begin{pmatrix}
					0	\\ 
					0	\\ 
					1	
				\end{pmatrix}.
			\end{aligned}
		\end{equation}
		La matrice de la symétrie se note alors
		\begin{equation}
			\begin{pmatrix}
				0	&	-1	&	0	\\
				-1	&	0	&	0	\\
				0	&	0	&	1
			\end{pmatrix}.
		\end{equation}
		

	\item
		La matrice produit est
		\begin{equation}
			AB=\begin{pmatrix}
				0	&	0	&	-1	\\
				-1	&	0	&	0	\\
				0	&	1	&	0
			\end{pmatrix}.
		\end{equation}
		Tout vecteur perpendiculaire à un des deux plans doit être retourné par cette matrice, et tout vecteur dans les deux plans en même temps doit être laissé inchangé. Il s'agit donc de la symétrie par rapport à la droite intersection des deux plans. Calculer les vecteurs propres et valeurs propres pour s'en convaincre.
	
	\end{enumerate}

\end{corrige}
