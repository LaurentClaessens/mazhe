% This is part of Mes notes de mathématique
% Copyright (c) 2011-2012
%   Laurent Claessens
% See the file fdl-1.3.txt for copying conditions.

L'espace projectif de \( E\) est l'ensemble des droites vectorielles de \( E\).
\begin{definition}
    Soit \( \eK\) un corps et \( E\) un espace vectoriel de dimension finie sur \( \eK\). Nous définissons sur \( E\setminus\{ 0 \}\) la relation d'équivalence \( u\sim v\) si et seulement si \( u=\lambda v\) pour un certain \( \lambda\in\eK\). Cette relation est la relation de \defe{colinéarité}{colinéarité}. L'ensemble des classes d'équivalence de \( \sim\) est l'\defe{espace projectif}{espace!projectif}\index{projectif!espace} de \( E\) et sera noté \( P(E)\)\nomenclature[A]{\( P(E)\)}{l'espace projectif de $E$}.
\end{definition}

Si \( \dim E=2\), l'ensemble \( P(E)\) est la \defe{droite projective}{droite!projective}\index{projectif!droite}, et si \( \dim E=3\) nous parlons du \defe{plan projectif}{plan!projectif}\index{projectif!plan}.

Étant donné que tous les \( \eK\)-espaces vectoriels de dimensions \( n+1\) sont isomorphes à\( \eK^{n+1}\), nous noterons \( P_n(\eK)\) ou \( P_n\) l'espace projectif \( P(\eK^{n+1})\).

\begin{example}
    Si \( n=1\) et \( \eK=\eR\), l'espace projectif est l'ensemble des droites vectorielles dans le plan usuel. Il y en a une pour chaque point du type \( (x,1)\) avec \( x\in\eR\) et ensuite une horizontale, passant par le point \( (1,0)\). Nous avons donc
    \begin{equation}
        P_1(\eR)=\{ (1,0) \}\cup\{ (x,1)\tq x\in \eR \}.
    \end{equation}
    Le point \( (1,0)\) est dit «point à l'infini».
\end{example}
