% This is part of Mes notes de mathématique
% Copyright (c) 2012-2013
%   Laurent Claessens
% See the file fdl-1.3.txt for copying conditions.


\begin{theorem}[Théorème d'isomorphisme de Banach]  \label{ThofQShsw}
    Une application linéaire continue et bijective entre deux espaces de Banach est un homéomorphisme.
\end{theorem}

%+++++++++++++++++++++++++++++++++++++++++++++++++++++++++++++++++++++++++++++++++++++++++++++++++++++++++++++++++++++++++++
\section{Espaces \texorpdfstring{$L^p$}{Lp}}
%+++++++++++++++++++++++++++++++++++++++++++++++++++++++++++++++++++++++++++++++++++++++++++++++++++++++++++++++++++++++++++

Pour la proposition suivante nous notons \( [f]\) la classe dans \( L^p\) de la fonction \( f\colon \Omega\to \eC\). Donc \( f\) représente une vraie fonction.
\begin{proposition}[\cite{bJOSNQ}]  \label{PropWoywYG}
    Soit \( 1\leq p\leq \infty\) et supposons que la suite \( [f_n]\) dans \( L^p(\Omega,\tribF,\mu)\) converge vers \( [f]\) au sens \( L^p\). Alors il existe une sous-suite \( (h_n)\) qui converge ponctuellement \( \mu\)-presque partout vers \( f\).
\end{proposition}
\index{espace!\( L^p\)}
\index{suite!de fonctions}
\index{limite!inversion}

\begin{proof}
    Si \( p=\infty\) nous sommes en train de parler de la convergence uniforme et il ne faut même pas prendre ni de sous-suite ni de «presque partout».

    Supposons que \( 1\leq p<\infty\). Nous considérons une sous-suite \( [h_n]\) de \( [f_n]\) telle que
    \begin{equation}
        \| [h_j]-[f] \|_p<2^{-j},
    \end{equation}
    puis nous posons \( u_k(x)=| h_k(x)-f(x) |^p\). Notons que ce \( u_k\) est une vraie fonction, pas une classe. Et en plus c'est une fonction positive. Nous avons
    \begin{equation}
        \int_{\Omega}u_kd\mu=\int_{\omega}| h_k(x)-f(x) |^pd\mu(x)=\| h_k-f \|_p^p\leq 2^{-kp}.
    \end{equation}
    Vu que \( u_k\) est une fonction positive la suite des sommes partielles de \( \sum_ku_k\) est croissante et vérifie donc le théorème de la convergence monotone \ref{ThoConvMonFtBoVh} :
    \begin{subequations}
        \begin{align}
            \int_{\Omega}\left( \sum_{k=0}^{\infty}u_k(x) \right)d\mu(x)&=\sum_{k=0}^{\infty}\int_{\Omega}u_k(x)d\mu(x)\\
            &\leq\sum_{k=0}^{\infty}2^{-kp}<\infty.
        \end{align}
    \end{subequations}
    Le fait que l'intégrale de la fonction \( \sum_ku_k\) est finie implique que cette fonction est finie \( \mu\)-presque partout. Donc le terme général tend vers zéro presque partout, c'est à dire
    \begin{equation}
        | h_k(x)-f(x) |^p\to 0.
    \end{equation}
    Cela signifie que \( h_k\to f\) presque partout ponctuellement.
\end{proof}

Est-ce qu'on peut faire mieux que la convergence ponctuelle presque partout d'une sous-suite ? En tout cas on ne peut pas espérer grand chose comme convergence pour la suite elle-même, comme le montre l'exemple suivant.

\begin{example}
    Nous allons montrer une suite de fonctions qui converge vers zéro dans \( L^p[0,1]\) (avec \( p<\infty\)) mais qui ne converge ponctuellement pour \emph{aucun} point. Cet exemple provient de \href{http://www.bibmath.net/dico/index.php?action=affiche&quoi=./b/bosseglissante.html}{bibmath.net}. 

%TODO : revoir tous les \href et mettre ceux qui doivent être en bibliographie. Par exemple celui ci-dessus.

    Nous construisons la suite de fonctions par paquets. Le premier paquet est formé de la fonction constante \( 1\).

    Le second paquet est formé de deux fonctions. La première est \( \mtu_{\mathopen[0 , 1/2 \mathclose]}\) et la seconde \( \mtu_{\mathopen[ 1/2 , 1 \mathclose]}\).

    Plus généralement le paquet numéro \( k\) est constitué des \( k\) fonctions \( \mtu_{\mathopen[ i/k , (i+1)/k \mathclose]}\) avec \( i=0,\ldots, k-1\).

    Vu que les fonctions du paquet numéro \( k\) ont pour norme \( \| f \|_p=\frac{1}{ k }\), nous avons évidemment \( f_n\to 0\) dans \( L^p\). Il est par contre visible que chaque paquet passe en revue tous les points de \( \mathopen[ 0 , 1 \mathclose]\). Donc pour tout \( x\) et pour tout \( N\), il existe (même une infinité) \( n>N\) tel que \( f_n(x)=1\). Il n'y a donc convergence ponctuelle nulle part.
\end{example}

%+++++++++++++++++++++++++++++++++++++++++++++++++++++++++++++++++++++++++++++++++++++++++++++++++++++++++++++++++++++++++++
\section{Complétude}
%+++++++++++++++++++++++++++++++++++++++++++++++++++++++++++++++++++++++++++++++++++++++++++++++++++++++++++++++++++++++++++

\begin{proposition}[Inégalité de Minkowski]     \label{PropInegMinkKUpRHg}
    Si \( 1\leq p<\infty\) et si \( f,g\in L^p(\Omega,\tribA,\mu)\) alors
    \begin{enumerate}
        \item
            \( \| f+g \|_p\leq \| f \|_p+\| g_p \|\)
        \item
            Il y a égalité si et seulement si les vecteurs \( f(x)\) et \( g(x)\) sont presque partout colinéaires : il existe \( \alpha,\beta\) tels que \( \alpha f+\beta g=0\) presque partout.
    \end{enumerate}
\end{proposition}

\begin{theorem}[\cite{SuquetFourierProba,UQSGIUo}]
    Pour \( 1\leq p<\infty\), l'espace \( L^p(\Omega,\tribA,\mu)\) est complet.
\end{theorem}
\index{complétude}

\begin{proof}
    Soit \( (f_n)_{n\in\eN}\) une suite de Cauchy dans \( L^p\). Pour tout \( i\), il existe \( N_i\in\eN\) tel que $\| f_p-f_q \|_p\leq 2^{-i}$ pour tout \( p,q\geq N_i\). Nous considérons la sous suite \( g_i=f_{N_i}\), de telle sorte qu'en particulier
    \begin{equation}    \label{EqJLoDID}
        \|g_i-g_{i-1}\|_p\leq 2^{-i}.
    \end{equation}
    Pour chaque \( j\) nous considérons la somme télescopique
    \begin{equation}
        g_j=g_0+\sum_{i=1}^j(g_i-g_{i-1})
    \end{equation}
    et l'inégalité
    \begin{equation}
        | g_j |\leq | g_0 |+\sum_{i=1}^j| g_i-g_{i-1} |.
    \end{equation}
    Nous allons noter
    \begin{equation}        \label{EqSomPaFPQOWC}
        h_j=| g_0 |+\sum_{i=1}^j| g_i-g_{i-1} |.
    \end{equation}
    La suite de fonctions \( (h_j)\) ainsi définie est une suite croissante de fonctions positive qui converge donc (ponctuellement) vers une fonction \( h\) qui peut éventuellement valoir l'infini en certains points. Par continuité de la fonction \( x\mapsto x^p\) nous avons
    \begin{equation}
        \lim_{j\to \infty} h_j^p=h^p,
    \end{equation}
    puis par le théorème de la convergence monotone (théorème \ref{ThoConvMonFtBoVh}) nous avons
    \begin{equation}
        \lim_{j\to \infty} \int_{\Omega}h_j^pd\mu=\int_{\Omega}h^pd\mu.
    \end{equation}
    Utilisant à présent la continuité de la fonction \( x\mapsto x^{1/p}\) nous trouvons
    \begin{equation}
        \lim_{j\to \infty} \left( \int h_j^p \right)^{1/p}=\left( \int | h |^p \right)^{1/p}.
    \end{equation}
    Nous avons donc déjà montré que
    \begin{equation}
        \lim_{j\to \infty} \| h_j \|_p=\left( \int | h |^p \right)^{1/p}
    \end{equation}
    où, encore une fois, rien ne garantit à ce stade que l'intégrale à droite soit un nombre fini. En utilisant l'inégalité de Minkowski (proposition \ref{PropInegMinkKUpRHg}) et l'inégalité \eqref{EqJLoDID} nous trouvons
    \begin{equation}
        \|h_j\|_p\leq \|g_0\|_p+\sum_{i=1}^j\|g_i-g_{i-1}\|_p\leq \|g_0\|_p+1.
    \end{equation}
    En passant à la limite,
    \begin{equation}
        \left( \int| h |^p \right)^{1/p}=\lim_{j\to \infty}\|h_j\|_p \leq \|g_0\|_p+1<\infty.
    \end{equation}
    Par conséquent \( \int| h |^p\) est finie et
    \begin{equation}    \label{EqgLpdUPOBP}
        h\in L^p(\Omega,\tribA,\mu).
    \end{equation}
    En particulier, l'intégrale \( \int h\) est finie (parce que \( p\geq 1\)) et donc que \( h(x)<\infty\) pour presque tout \( x\in\Omega\).

    Nous savons que \( h(x)\) est la limite des sommes partielles \eqref{EqSomPaFPQOWC}, en particulier la série
    \begin{equation}
        \sum_{j=1}^{\infty}| g_i-g_{i-1} |
    \end{equation}
    converge ponctuellement. En vertu du corollaire \ref{CorCvAbsNormwEZdRc}, la série de terme général \( g_i-g_{i-1}\) converge ponctuellement. La suite \( g_i\) converge donc vers une fonction que nous notons \( g\). Par ailleurs la suite \( g_i\) est dominée par \( h\in L^p\), le théorème de la convergence dominée (théorème \ref{ThoConvDomLebVdhsTf}) implique que
    \begin{equation}
        \lim_{j\to \infty} \|g_j-g\|_p=0.
    \end{equation}
    Nous allons maintenant prouver que \( \lim_{n\to \infty\|f_n-g\|_p} =0\). Soit \( \epsilon>0\). Pour tout \( n\) et \( i\) nous avons
    \begin{equation}
        \|f_n-g\|_p=\|f_n-f_{N_i}+f_{N_i}-g\|_p\leq\|f_n-f_{N_i}\|_p+\|f_{N_i}-g\|_p.
    \end{equation}
    Pour rappel, \( f_{N_i}=g_i\). Si \(i\) et \( n\) sont suffisamment grands nous pouvons obtenir que chacun des deux termes est plus petit que \( \epsilon/2\).

    Il nous reste à prouver que \( g\in L^p(\Omega,\tribA,\mu)\). Nous avons déjà vu (équation \eqref{EqgLpdUPOBP}) que \( h\in L^p\), mais \( | g_i |\leq h^p\), par conséquent  \( g\in L^p\).

    Nous avons donc montré que la suite de Cauchy \( (f_n)\) converge vers une fonction de \( L^p\), ce qui signifie que \( L^p\) est complet.
\end{proof}

\begin{theorem}[Fischer-Riesz\cite{KXjFWKA}] \label{ThoGVmqOro}
    Soit un ouvert \( \Omega\) de \( \eR^n\) et \( p\in\mathopen[ 1 , \infty \mathclose]\). Alors
    \begin{enumerate}
        \item\label{ItemPDnjOJzi}
            Toute suite convergente dans \( L^p(\Omega)\) admet une sous-suite convergente presque partout sur \( \Omega\).
        \item\label{ItemPDnjOJzii}
            La sous-suite donnée en \ref{ItemPDnjOJzi} est dominée par un élément de \( L^p(\Omega)\).
        \item\label{ItemPDnjOJziii}
            L'espace \( L^p(\Omega)\) est de Banach.
    \end{enumerate}
\end{theorem}
\index{espace!de fonctions!$L^p$}
\index{complétude!espaces $ L^p$}

\begin{proof}
    Le cas \( p=\infty\) est à séparer des autres valeurs de \( p\) parce qu'on y parle de norme uniforme, et aucune sous-suite n'est à considérer.
    \begin{subproof}
    \item[Cas \( p=\infty\).]
    Nous commençons par prouver dans le cas \( p=\infty\). Soit \( (f_n)\) une suite de Cauchy dans \( L^{\infty}(\Omega)\), ou plus précisément une suite de représentants d'éléments de \( L^p\). Pour tout \( k\geq 1\), il existe \( N_k\geq 0\) tel que si \( m,n\geq N_k\), on a
    \begin{equation}
        \| f_m-f_n \|_{\infty}\leq \frac{1}{ k }.
    \end{equation}
    En particulier, il existe un ensemble de mesure nulle \( E_k\) sur lequel
    \begin{equation}
        | f_m(x)-f_n(x) |\leq\frac{1}{ k },
    \end{equation}
    et si nous posons \( E=\bigcup_{k\in \eN}E_k\), nous avons encore un ensemble de mesure nulle (lemme \ref{LemIDITgAy}). En  résumé, nous avons un \( N_k\) tel que si \( m,n\geq N_k\), alors 
    \begin{equation}    \label{EqKAWSmtG}
        | f_n(x)-f_m(x) |\leq \frac{1}{ k }
    \end{equation}
    pour tout \( x\) hors de \( E\). Donc pour chaque \( x\in\Omega\setminus E\), la suite \( n\mapsto f_n(x)\) est de Cauchy dans \( \eR\) et converge donc. Cela défini donc une fonction
    \begin{equation}
        \begin{aligned}
            f\colon \Omega\setminus E&\to \eR \\
            x&\mapsto \lim_{n\to \infty} f_n(x). 
        \end{aligned}
    \end{equation}
    Cela prouve le point \ref{ItemPDnjOJzi} : la convergence ponctuelle.

    En passant à la limite \( n\to \infty\) dans l'équation \ref{EqKAWSmtG} et tenant compte que cette majoration tient pour presque tout \( x\) dans \( \Omega\), nous trouvons
    \begin{equation}
        \| f-f_n \|_{\infty}\leq \frac{1}{ k }.
    \end{equation}
    Donc non seulement \( f\) est dans \( L^{\infty}\), mais en plus la suite \( (f_n)\) converge vers \( f\) au sens \( L^{\infty}\), c'est à dire uniformément. Cela prouve le point \ref{ItemPDnjOJziii}. En ce qui concerne le point \ref{ItemPDnjOJzii}, la suite \( f_n\) est entièrement (à partir d'un certain point) dominée par la fonction \( 1+| f |\) qui est dans \( L^p\).


    \item[Cas \( p=\infty\).]

        Toute suite convergente étant de Cauchy, nous considérons une suite de Cauchy \( (f_n)\) dans \( L^p(\Omega)\) et ce sera suffisant pour travailler sur le premier point. Pour montrer qu'une suite de Cauchy converge, il est suffisant de montrer qu'une sous-suite converge. Soit \( \varphi\colon \eN\to \eN\) une fonction strictement croissante telle que pour tout \( n\geq 1\) nous ayons
        \begin{equation}
            \| f_{\varphi(n+1)}-f_{\varphi(n)} \|_p\leq \frac{1}{ 2^{n} }.
        \end{equation}
        Pour créer la fonction \( \varphi\), il est suffisant de prendre le \( N_k\) donné par la condition de Cauchy pour \( \epsilon=1/2^k\) et de considérer la fonction définie par récurrence par \( \varphi(1)=N_1\) et \( \varphi(n+1)>\max\{ N_n,\varphi(n-1) \}\). Ensuite nous considérons la fonction
        \begin{equation}
            g_n(x)=\sum_{k=1}^n| f_{\varphi(k+1)}(x)-f_{\varphi(k)}(x) |.
        \end{equation}
        Notons que pour écrire cela nous avons considéré des représentants \( f_k\) qui sont alors des fonctions à l'ancienne. Étant donné que \( g_n\) est une somme de fonctions dans \( L^p\), c'est une fonction \( L^p\), comme nous pouvons le constater en calculant sa norme :
        \begin{equation}
            \| g_n \|_p\leq \sum_{k=1}^n\| f_{\varphi(k+1)}-f_{\varphi(k)} \|_p\leq\sum_{k=1}^n\frac{1}{ 2^k }\leq\sum_{k=1}^{\infty}\frac{1}{ 2^k }=1.
        \end{equation}
        Étant donné que tous les termes de la somme définissant \( g_n\) sont positifs, la suite \( (g_n)\) est croissante. Mais elle est bornée en norme \( L^p\) et donc sujette à obéir au théorème de Beppo-Levi \ref{ThoConvMonFtBoVh} sur la convergence monotone. Il existe donc une fonction \( g\in L^p(\Omega)\) telle que \( g_n\to g\) presque partout.

        Soit un \( x\in \Omega\) pour lequel \( g_n(x)\to g(x)\); alors pour tout \( n\geq 2\) et \( \forall q\geq 0\),
        \begin{subequations}    \label{EqWTHojCq}
            \begin{align}
                | f_{\varphi(n+q)}(x)-f_{\varphi(n)}(x) |&=\left| f_{\varphi(n+q)}(x)-\sum_{k=1}^{q-1}f_{\varphi(n+k)}(x) -\sum_{k=1}^{q-1}f_{\varphi(n+k)}(x)-f_{\varphi(n)}(x) \right| \\
                &=\left| \sum_{k=1}^qf_{\varphi(n+k)}-\sum_{k=1}^qf_{\varphi(n+k-1)}(x) \right|\\
                &\leq \sum_{k=1}^q\Big| f_{\varphi(n+k)}(x)-f_{\varphi(n+k-1)}(x) \Big|\\
                &=g_{n+q+1}(x)-g_{n+1}(x)\\
                &\leq g(x)-g_{n-1}(x).
            \end{align}
        \end{subequations}
        Nous prenons la limite \( n\to \infty\); la dernière expression tend vers zéro et donc
        \begin{equation}
            | f_{\varphi(n+q)}(x)-f_{\varphi(n)}(x) |\to 0
        \end{equation}
        pour tout \( q\). Donc pour presque tout \( x\in \Omega\), la suite \( n\mapsto f_{\varphi(n)}(x)\) est de Cauchy dans \( \eR\) et donc y converge vers un nombre que nous nommons \( f(x)\). Cela définit une fonction
        \begin{equation}
            \begin{aligned}
                f\colon \Omega\setminus E&\to \eR \\
                x&\mapsto \lim_{n\to \infty} f_{\varphi(n)}(x) 
            \end{aligned}
        \end{equation}
        où \( E\) est de mesure nulle. Montrons que \( f\) est bien dans \( L^p(\Omega)\); pour cela nous complétons la série d'inégalités \eqref{EqWTHojCq} en
        \begin{equation}
            \big| f_{\varphi(n+q)}(x)-f_{\varphi(n)}(x) \big|\leq g(x)-g_{n-1}(x)\leq g(x).
        \end{equation}
        En prenant la limite \( q\to \infty\) nous avons l'inégalité
        \begin{equation}    \label{EqMQbDRac}
            | f(x)-f_{\varphi(n)}(x) |\leq g(x)
        \end{equation}
        pour presque tout \( x\in\Omega\), c'est à dire pour tout \( x\in\Omega\setminus E\). Cette inégalité implique deux choses valables pour presque tout \( x\) dans \( \Omega\) :
        \begin{subequations}
            \begin{align}
                f(x)&\in B\big( g(x),f_{\varphi(n)}(x) \big)\\
                f_{\varphi(n)}(x)&\leq | f(x) |+| g(x) |.
            \end{align}
        \end{subequations}
        
        La première inégalité assure que \( | f |^p\) est intégrable sur \( \Omega\setminus E\) parce que \( | f |\) est majorée par \( | g |+| f_{\varphi(n)} |\). Elle prouve par conséquent le point \ref{ItemPDnjOJzi} parce que \(n\mapsto f_{\varphi(n)}\) est une sous-suite convergente presque partout. La seconde montre le point \ref{ItemPDnjOJzii}. 

        Attention : à ce point nous avons prouvé que \( n\mapsto f_{\varphi(n)}\) est une suite de fonctions qui converge \emph{ponctuellement presque partout} vers une fonction \( f\) qui s'avère être dans \( L^p\). Nous n'avons pas montré que cette suite convergeait au sens de \( L^p\) vers \( f\). Ce que nous devons montrer est que
        \begin{equation}    \label{EqJLfnEvj}
            \| f-f_{\varphi(n)} \|_p\to 0.
        \end{equation}
        L'inégalité \eqref{EqMQbDRac} nous donne aussi, toujours pour presque tout \( x\in \Omega\) :
        \begin{equation}
            \big| f(x)-f_{\varphi(n)}(x) \big|^p\leq g(x)^p
        \end{equation}
        ce qui signifie que la suite\quext{À ce point, \cite{KXjFWKA} se contente de majorer \( | f_{\varphi(n)}(x) |\) par \( | f(x) |+|g(x)\), mais je ne comprends pas comment cette majoration nous permet d'utiliser la convergence dominée de Lebesgue pour montrer \eqref{EqJLfnEvj}.} \(    | f-f_{\varphi(n)} |^p    \) est dominée par la fonction \( | g |^p\) qui est intégrable sur \( \Omega\setminus E\) et tout autant sur \( \Omega\) parce que \( E\) est négligeable; cela prouve au passage le point \ref{ItemPDnjOJzii}, et le théorème de la convergence dominée de Lebesgue (\ref{ThoConvDomLebVdhsTf}) nous dit que
        \begin{equation}
            \lim_{n\to \infty} \int_{\Omega} \big| f(x)-f_{\varphi(n)}(x) \big|^pdx=\int_{\Omega}\lim_{n\to \infty} \big| f(x)-f_{\varphi(n)}(x) \big|dx=0.
        \end{equation}
        Cette dernière suite d'inégalités se lit de la façon suivante :
        \begin{equation}
            \lim_{n\to \infty} \| f-f_{\varphi(n)} \|_p=\big\| \lim_{n\to \infty} | f-f_{\varphi(n)} | \big\|_p=0.
        \end{equation}
        Nous en déduisons que la suite \( n\mapsto f_{\varphi(n)}\) est convergente vers \( f\) au sens de la norme \( L^p(\Omega)\). Or la suite de départ \( (f_n)\) était de Cauchy (pour la norme \( L^p\)); donc l'existence d'une sous-suite convergente implique la convergence de la suite entière vers \( f\), ce qu'il fallait démontrer.
    \end{subproof}
\end{proof}

%--------------------------------------------------------------------------------------------------------------------------- 
\subsection{Un cas particulier pour \texorpdfstring{$ L^2$}{ L2}}
%---------------------------------------------------------------------------------------------------------------------------

Les espaces \( L^2\) étant un peu particuliers, leur étude va dans le chapitre sur les espaces de Hilbert, voir section \ref{SecCKZSrZK}. Nous notons ici une conséquence du théorème \ref{ThoGVmqOro} dans ce cas. La proposition suivante est une petite partie du corollaire \ref{CorQETwUdF}, qui sera d'ailleurs démontré de façon indépendante.

\begin{proposition}
    Si nous avons une suite de réels \( (a_k)\) telle que \( \sum_{k=0}^{\infty}| a_k |^2<\infty\) alors la suite
    \begin{equation}
        f_n(x)=\sum_{k=0}^na_k e^{ikx}
    \end{equation}
    converge dans \( L^2\big( \mathopen] 0 , 2\pi \mathclose[ \big)\).
\end{proposition}

\begin{proof}
    Quitte à séparer les parties réelles et imaginaires, nous pouvons faire abstraction du fait que nous parlons d'une série de fonctions à valeurs dans \( \eC\) au lieu de \( \eR\).

    Un simple calcul est :
    \begin{equation}
        \| f_n-f_m \|^2\leq\int_0^{2\pi}\sum_{k=n}^m| a_k |^2dx\leq 2\pi\sum_{k=n}^m| a_k |^2.
    \end{equation}
    Par hypothèse le membre de droite est \( | s_m-s_n |\) où \( s_k\) dénote la suite des somme partielle de la série des \( | a_k |^2\). Cette dernière est de Cauchy (parce que convergente dans \( \eR\)) et donc la limite \( n\to\infty\) (en gardant \( m>n\)) est zéro. Donc la suite des \( f_n\) est de Cauchy dans \( L^2\) et donc converge dans \( L^2\).
\end{proof}

%+++++++++++++++++++++++++++++++++++++++++++++++++++++++++++++++++++++++++++++++++++++++++++++++++++++++++++++++++++++++++++ 
\section{Inégalité de Hölder}
%+++++++++++++++++++++++++++++++++++++++++++++++++++++++++++++++++++++++++++++++++++++++++++++++++++++++++++++++++++++++++++

\begin{proposition}[Inégalité de Hölder]       \label{ProptYqspT}
    Soit \( \Omega\) un espace mesuré et \( 1\leq p\), \( q\leq\infty\) satisfaisant \( \frac{1}{ p }+\frac{1}{ q }=1\). Soient \( f\in L^p(\Omega)\), \( g\in L^q(\Omega)\). Alors le produit \( fg\) est dans \( L^1(\Omega)\) et nous avons
    \begin{equation}
        \| fg \|_1\leq \| f \|_p\| g \|_q.
    \end{equation}
\end{proposition}
%TODO : une preuve.

\begin{remark}      \label{RemNormuptNird}
    Dans le cas d'un espace de probabilité, la fonction constante \( g=1\) appartient à \( L^p(\Omega)\). En prenant \( p=q=2\) nous obtenons
    \begin{equation}
        \| f \|_1\leq\| f \|_2.
    \end{equation}
\end{remark}

