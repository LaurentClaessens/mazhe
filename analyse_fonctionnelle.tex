% This is part of Mes notes de mathématique
% Copyright (c) 2012-2013
%   Laurent Claessens
% See the file fdl-1.3.txt for copying conditions.


\begin{theorem}[Théorème d'isomorphisme de Banach]  \label{ThofQShsw}
    Une application linéaire continue et bijective entre deux espaces de Banach est un homéomorphisme.
\end{theorem}

%+++++++++++++++++++++++++++++++++++++++++++++++++++++++++++++++++++++++++++++++++++++++++++++++++++++++++++++++++++++++++++
\section{Espaces \texorpdfstring{$L^p$}{Lp}}
%+++++++++++++++++++++++++++++++++++++++++++++++++++++++++++++++++++++++++++++++++++++++++++++++++++++++++++++++++++++++++++
Les espaces \( L^2\) étant un peu particuliers, leur étude va dans le chapitre sur les espaces de Hilbert, voir section \ref{SecCKZSrZK}.

Pour la proposition suivante nous notons \( [f]\) la classe dans \( L^p\) de la fonction \( f\colon \Omega\to \eC\). Donc \( f\) représente une vraie fonction.
\begin{proposition}[\cite{bJOSNQ}]  \label{PropWoywYG}
    Soit \( 1\leq p\leq \infty\) et supposons que la suite \( [f_n]\) dans \( L^p(\Omega,\tribF,\mu)\) converge vers \( [f]\) au sens \( L^p\). Alors il existe une sous-suite \( (h_n)\) qui converge ponctuellement \( \mu\)-presque partout vers \( f\).
\end{proposition}
\index{espace!\( L^p\)}
\index{suite!de fonctions}
\index{limite!inversion}

\begin{proof}
    Si \( p=\infty\) nous sommes en train de parler de la convergence uniforme et il ne faut même pas prendre ni de sous-suite ni de «presque partout».

    Supposons que \( 1\leq p<\infty\). Nous considérons une sous-suite \( [h_n]\) de \( [f_n]\) telle que
    \begin{equation}
        \| [h_j]-[f] \|_p<2^{-j},
    \end{equation}
    puis nous posons \( u_k(x)=| h_k(x)-f(x) |^p\). Notons que ce \( u_k\) est une vraie fonction, pas une classe. Et en plus c'est une fonction positive. Nous avons
    \begin{equation}
        \int_{\Omega}u_kd\mu=\int_{\omega}| h_k(x)-f(x) |^pd\mu(x)=\| h_k-f \|_p^p\leq 2^{-kp}.
    \end{equation}
    Vu que \( u_k\) est une fonction positive la suite des sommes partielles de \( \sum_ku_k\) est croissante et vérifie donc le théorème de la convergence monotone \ref{ThoConvMonFtBoVh} :
    \begin{subequations}
        \begin{align}
            \int_{\Omega}\left( \sum_{k=0}^{\infty}u_k(x) \right)d\mu(x)&=\sum_{k=0}^{\infty}\int_{\Omega}u_k(x)d\mu(x)\\
            &\leq\sum_{k=0}^{\infty}2^{-kp}<\infty.
        \end{align}
    \end{subequations}
    Le fait que l'intégrale de la fonction \( \sum_ku_k\) est finie implique que cette fonction est finie \( \mu\)-presque partout. Donc le terme général tend vers zéro presque partout, c'est à dire
    \begin{equation}
        | h_k(x)-f(x) |^p\to 0.
    \end{equation}
    Cela signifie que \( h_k\to f\) presque partout ponctuellement.
\end{proof}

Est-ce qu'on peut faire mieux que la convergence ponctuelle presque partout d'une sous-suite ? En tout cas on ne peut pas espérer grand chose comme convergence pour la suite elle-même, comme le montre l'exemple suivant.

\begin{example}
    Nous allons montrer une suite de fonctions qui converge vers zéro dans \( L^p[0,1]\) (avec \( p<\infty\)) mais qui ne converge ponctuellement pour \emph{aucun} point. Cet exemple provient de \href{http://www.bibmath.net/dico/index.php?action=affiche&quoi=./b/bosseglissante.html}{bibmath.net}. 

%TODO : revoir tous les \href et mettre ceux qui doivent être en bibliographie. Par exemple celui ci-dessus.

    Nous construisons la suite de fonctions par paquets. Le premier paquet est formé de la fonction constante \( 1\).

    Le second paquet est formé de deux fonctions. La première est \( \mtu_{\mathopen[0 , 1/2 \mathclose]}\) et la seconde \( \mtu_{\mathopen[ 1/2 , 1 \mathclose]}\).

    Plus généralement le paquet numéro \( k\) est constitué des \( k\) fonctions \( \mtu_{\mathopen[ i/k , (i+1)/k \mathclose]}\) avec \( i=0,\ldots, k-1\).

    Vu que les fonctions du paquet numéro \( k\) ont pour norme \( \| f \|_p=\frac{1}{ k }\), nous avons évidemment \( f_n\to 0\) dans \( L^p\). Il est par contre visible que chaque paquet passe en revue tous les points de \( \mathopen[ 0 , 1 \mathclose]\). Donc pour tout \( x\) et pour tout \( N\), il existe (même une infinité) \( n>N\) tel que \( f_n(x)=1\). Il n'y a donc convergence ponctuelle nulle part.
\end{example}




%+++++++++++++++++++++++++++++++++++++++++++++++++++++++++++++++++++++++++++++++++++++++++++++++++++++++++++++++++++++++++++
\section{Complétude}
%+++++++++++++++++++++++++++++++++++++++++++++++++++++++++++++++++++++++++++++++++++++++++++++++++++++++++++++++++++++++++++

\begin{proposition}[Inégalité de Minkowski]     \label{PropInegMinkKUpRHg}
    Si \( 1\leq p<\infty\) et si \( f,g\in L^p(\Omega,\tribA,\mu)\) alors
    \begin{enumerate}
        \item
            \( \| f+g \|_p\leq \| f \|_p+\| g_p \|\)
        \item
            Il y a égalité si et seulement si les vecteurs \( f(x)\) et \( g(x)\) sont presque partout colinéaires : il existe \( \alpha,\beta\) tels que \( \alpha f+\beta g=0\) presque partout.
    \end{enumerate}
\end{proposition}

\begin{theorem}
    Pour \( 1\leq p<\infty\), l'espace \( L^p(\Omega,\tribA,\mu)\) est complet.
\end{theorem}
\index{complétude}

\begin{proof}
    Cette preuve provient de \cite{SuquetFourierProba}, disponible aussi sous forme d'application du \wikipedia{fr}{théorème_de_Riesz-Fischer}{théorème de Riesz-Fischer} sur Wikipedia. 
    
    Soit \( (f_n)_{n\in\eN}\) une suite de Cauchy dans \( L^p\). Pour tout \( i\), il existe \( N_i\in\eN\) tel que $\| f_p-f_q \|_p\leq 2^{-i}$ pour tout \( p,q\geq N_i\). Nous considérons la sous suite \( g_i=f_{N_i}\), de telle sorte qu'en particulier
    \begin{equation}    \label{EqJLoDID}
        \|g_i-g_{i-1}\|_p\leq 2^{-i}.
    \end{equation}
    Pour chaque \( j\) nous considérons la somme télescopique
    \begin{equation}
        g_j=g_0+\sum_{i=1}^j(g_i-g_{i-1})
    \end{equation}
    et l'inégalité
    \begin{equation}
        | g_j |\leq | g_0 |+\sum_{i=1}^j| g_i-g_{i-1} |.
    \end{equation}
    Nous allons noter
    \begin{equation}        \label{EqSomPaFPQOWC}
        h_j=| g_0 |+\sum_{i=1}^j| g_i-g_{i-1} |.
    \end{equation}
    La suite de fonctions \( (h_j)\) ainsi définie est une suite croissante de fonctions positive qui converge donc (ponctuellement) vers une fonction \( h\) qui peut éventuellement valoir l'infini en certains points. Par continuité de la fonction \( x\mapsto x^p\) nous avons
    \begin{equation}
        \lim_{j\to \infty} h_j^p=h^p,
    \end{equation}
    puis par le théorème de la convergence monotone (théorème \ref{ThoConvMonFtBoVh}) nous avons
    \begin{equation}
        \lim_{j\to \infty} \int_{\Omega}h_j^pd\mu=\int_{\Omega}h^pd\mu.
    \end{equation}
    Utilisant à présent la continuité de la fonction \( x\mapsto x^{1/p}\) nous trouvons
    \begin{equation}
        \lim_{j\to \infty} \left( \int h_j^p \right)^{1/p}=\left( \int | h |^p \right)^{1/p}.
    \end{equation}
    Nous avons donc déjà montré que
    \begin{equation}
        \lim_{j\to \infty} \| h_j \|_p=\left( \int | h |^p \right)^{1/p}
    \end{equation}
    où, encore une fois, rien ne garantit à ce stade que l'intégrale à droite soit un nombre fini. En utilisant l'inégalité de Minkowski (proposition \ref{PropInegMinkKUpRHg}) et l'inégalité \eqref{EqJLoDID} nous trouvons
    \begin{equation}
        \|h_j\|_p\leq \|g_0\|_p+\sum_{i=1}^j\|g_i-g_{i-1}\|_p\leq \|g_0\|_p+1.
    \end{equation}
    En passant à la limite,
    \begin{equation}
        \left( \int| h |^p \right)^{1/p}=\lim_{j\to \infty}\|h_j\|_p \leq \|g_0\|_p+1<\infty.
    \end{equation}
    Par conséquent \( \int| h |^p\) est finie et
    \begin{equation}    \label{EqgLpdUPOBP}
        h\in L^p(\Omega,\tribA,\mu).
    \end{equation}
    En particulier, l'intégrale \( \int h\) est finie (parce que \( p\geq 1\)) et donc que \( h(x)<\infty\) pour presque tout \( x\in\Omega\).

    Nous savons que \( h(x)\) est la limite des sommes partielles \eqref{EqSomPaFPQOWC}, en particulier la série
    \begin{equation}
        \sum_{j=1}^{\infty}| g_i-g_{i-1} |
    \end{equation}
    converge ponctuellement. En vertu du corollaire \ref{CorCvAbsNormwEZdRc}, la série de terme général \( g_i-g_{i-1}\) converge ponctuellement. La suite \( g_i\) converge donc vers une fonction que nous notons \( g\). Par ailleurs la suite \( g_i\) est dominée par \( h\in L^p\), le théorème de la convergence dominée (théorème \ref{ThoConvDomLebVdhsTf}) implique que
    \begin{equation}
        \lim_{j\to \infty} \|g_j-g\|_p=0.
    \end{equation}
    Nous allons maintenant prouver que \( \lim_{n\to \infty\|f_n-g\|_p} =0\). Soit \( \epsilon>0\). Pour tout \( n\) et \( i\) nous avons
    \begin{equation}
        \|f_n-g\|_p=\|f_n-f_{N_i}+f_{N_i}-g\|_p\leq\|f_n-f_{N_i}\|_p+\|f_{N_i}-g\|_p.
    \end{equation}
    Pour rappel, \( f_{N_i}=g_i\). Si \(i\) et \( n\) sont suffisamment grands nous pouvons obtenir que chacun des deux termes est plus petit que \( \epsilon/2\).

    Il nous reste à prouver que \( g\in L^p(\Omega,\tribA,\mu)\). Nous avons déjà vu (équation \eqref{EqgLpdUPOBP}) que \( h\in L^p\), mais \( | g_i |\leq h^p\), par conséquent  \( g\in L^p\).

    Nous avons donc montré que la suite de Cauchy \( (f_n)\) converge vers une fonction de \( L^p\), ce qui signifie que \( L^p\) est complet.
\end{proof}


%+++++++++++++++++++++++++++++++++++++++++++++++++++++++++++++++++++++++++++++++++++++++++++++++++++++++++++++++++++++++++++ 
\section{Inégalité de Hölder}
%+++++++++++++++++++++++++++++++++++++++++++++++++++++++++++++++++++++++++++++++++++++++++++++++++++++++++++++++++++++++++++

\begin{proposition}[Inégalité de Hölder]       \label{ProptYqspT}
    Soit \( \Omega\) un espace mesuré et \( 1\leq p\), \( q\leq\infty\) satisfaisant \( \frac{1}{ p }+\frac{1}{ q }=1\). Soient \( f\in L^p(\Omega)\), \( g\in L^q(\Omega)\). Alors le produit \( fg\) est dans \( L^1(\Omega)\) et nous avons
    \begin{equation}
        \| fg \|_1\leq \| f \|_p\| g \|_q.
    \end{equation}
\end{proposition}
%TODO : une preuve.

\begin{remark}      \label{RemNormuptNird}
    Dans le cas d'un espace de probabilité, la fonction constante \( g=1\) appartient à \( L^p(\Omega)\). En prenant \( p=q=2\) nous obtenons
    \begin{equation}
        \| f \|_1\leq\| f \|_2.
    \end{equation}
\end{remark}

