% This is part of Mes notes de mathématique
% Copyright (c) 2011-2012
%   Laurent Claessens
% See the file fdl-1.3.txt for copying conditions.

%+++++++++++++++++++++++++++++++++++++++++++++++++++++++++++++++++++++++++++++++++++++++++++++++++++++++++++++++++++++++++++
\section{Fourier}
%+++++++++++++++++++++++++++++++++++++++++++++++++++++++++++++++++++++++++++++++++++++++++++++++++++++++++++++++++++++++++++

Ici nous utilisons la convention de la transformée de Fourier de \wikipedia{fr}{Transformée_de_Fourier}{wikipedia}, c'est à dire
\begin{subequations}
    \begin{align}
        \hat f(\xi)&=\int_{\eR} e^{-i\xi x}f(x)dx\\
        f(x)&=2\pi\int_{\eR} e^{i\xi x}\hat f(\xi)d\xi.
    \end{align}
\end{subequations}

L'\defe{espace de Schwartz}{Schwartz!espace}\index{espace!de Schwartz} \( \swS(\eR^n,\eC)\)\nomenclature[Y]{\( \swS(\eR^n,\eC)\)}{fonctions Schwartz} est l'ensemble des fonctions dont toutes les dérivées décroissent plus vite que l'inverse de tout polynôme, c'est à dire
\begin{equation}
    \swS(\eR^n,\eC)=\{ f\in C^{\infty}(\eR^n,\eC)\tq \forall \alpha,\beta\in \eN^n,\sup_{x\in \eR^n}\big| (x)^{\alpha}D^{\beta}f(x) \big|<\infty \}
\end{equation}
où nous utilisons les notations \( x^{\alpha}=(x_1)^{\alpha_1}\ldots (x_n)^{\alpha_n}\) et \( D^{\beta}=\frac{ \partial^n  }{ \partial \beta_1\ldots\partial \beta_n }\).

\begin{proposition}[\cite{MesIntProbb}]
    La transformée de Fourier est une bijection de \( \swS(\eR^n,\eC)\).    
\end{proposition}

\begin{proposition}
    La transformée de Fourier est un morphisme vis-à-vis de la convolution\index{produit!convolution!et Fourier} sur \( L^1(\eR^n)\) :
    \begin{equation}
        \widehat{f*g}=\hat f\hat g.
    \end{equation}
    <++>
\end{proposition}
<++>
