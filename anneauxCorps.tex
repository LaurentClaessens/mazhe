%+++++++++++++++++++++++++++++++++++++++++++++++++++++++++++++++++++++++++++++++++++++++++++++++++++++++++++++++++++++++++++
\section{Généralités}
%+++++++++++++++++++++++++++++++++++++++++++++++++++++++++++++++++++++++++++++++++++++++++++++++++++++++++++++++++++++++++++

\begin{definition}
    Un \defe{anneau}{anneau} est un triple \( (A,+,\cdot)\) avec les conditions
    \begin{enumerate}
        \item
            \( (A,+)\) est un groupe abélien. Nous notons \( 0\) le neutre.
        \item
            La multiplication est associative et nous notons \( 1\) le neutre
        \item
            La multiplication est distributive par rapport à l'addition.
    \end{enumerate}
\end{definition}

\begin{remark}
    Un anneau est ce qu'on appelle «\emph{ring}» en anglais.
\end{remark}


Soit \( X\) un ensemble et $A$ un anneau. Nous considérons \( \Fun(X,A)\)\nomenclature[A]{\( \Fun(X,Y)\)}{les applications de \( X\) vers \( Y\)} l'ensemble des applications \( X\to A\). Cet ensemble devient un anneau avec les définitions
\begin{subequations}
    \begin{align}
        (f+g)(x)=f(x)+g(x)\\
        (fg)(x)=f(x)g(x).
    \end{align}
\end{subequations}
Cela est la \defe{structure canonique}{structure d'anneau canonique} d'anneau sur \( \Fun(X,A)\).

Le \defe{centralisateur}{centralisateur} de \( x\in A\) dans \( A\) est l'ensemble
\begin{equation}
    \{ y\in A\tq xy=yx \},
\end{equation}
le \defe{centre}{centre} de \( A\) est
\begin{equation}
    \{ y\in A\tq xy=yx\forall x\in A \}.
\end{equation}
Un élément \( a\neq 0\) est un \defe{diviseur de zéro}{diviseur!de zéro} à droite si il existe \( b\neq 0\) tel que $ba=0$. L'élément \( a\) est un diviseur de zéro à droite si il existe \( b\) tel que \( ab=0\). Un anneau est \defe{intègre}{intègre!anneau}\index{anneau!intègre} si il est non nul et ne possède pas de diviseurs de zéro.

\begin{example}
    L'ensemble \( \eZ\) avec les opérations usuelles est un anneau intègre.
\end{example}

Un élément \( a\in A\) est \defe{régulier à droite}{régulier à droite} \( ba=0\) implique \( b=0\). Il est régulier ) gauche si \( ab=0\) implique \( b=0\).

L'ensemble \( U(A)\)\nomenclature[A]{\( U(A)\)}{ensemble des inversibles} des éléments inversibles de \( A\) est un groupe pour la multiplication. Nous notons \( A^*=A\setminus\{ 0 \}\).

\begin{lemma}
    Si \( a\) et \( b\) commutent, nous avons la formule
    \begin{equation}        \label{Eqarpurmkbk}<++>
        a^{r+1}-b^{r+1}=(a-b)(\sum_ka^{r-k}b^k).
    \end{equation}
\end{lemma}

\begin{proposition}
    Si \( a\) est un élément nilpotent de l'anneau \( A\), alors \( 1-a\) est inversible. Si \( a\) est nilpotent non nul, alors il est diviseur de zéro.
\end{proposition}

\begin{proof}
    Soit \( n\) le minimum tel que \( a^n=0\). En vertu de la formule \eqref{Eqarpurmkbk} nous avons
    \begin{equation}
        1=1-a^n=(1-a)(1+a+\ldots+a^{n-1})=(1+a+\ldots+a^{n-1})(1-a).
    \end{equation}
    La somme \( 1+a+\ldots+a^{n-1}\) est donc un inverse de \( (1-a)\).
\end{proof}

\begin{definition}
    Si \( A\) et \( B\) sont des anneaux, un \defe{morphisme}{morphisme!d'anneaux} est une application \( f\colon A\to B\) telle que pour tout \( x,y\in A\) nous ayons
    \begin{enumerate}
        \item
            \( f(x+y)=f(x)+f(y)\)
        \item
            \( f(xy)=f(x)f(y)\)
        \item
            \( f(1)=1\)
    \end{enumerate}
\end{definition}

Si \( f\) est un morphisme, nous avons \( f(0)=0\) et \( f(x)^{-1}=f(x^{-1})\).
