% This is part of Exercices et corrigés de CdI-1
% Copyright (c) 2011
%   Laurent Claessens
% See the file fdl-1.3.txt for copying conditions.

\begin{corrige}{0030}

\begin{enumerate}
\item
Le changement de variable est assez visible : $u=x-1$ et $v=y-3$. Dans les nouvelles variable, la limite à trouver est
\begin{equation}
	\lim_{(u,v)\to(0,0)}\frac{ -\sin(u^4+u^2+v^2) }{ u^2+v^2 }.
\end{equation}
Cet exercice est maintenant similaire à l'exercice \ref{exo0028}\ref{Item0028e}. Afin de faire apparaître $\sin(x)/x$, nous multiplions et divisons l'expression par $u^4+u^2+v^2$, et nous arrivons sur
\begin{equation}
	\lim_{(u,v)\to(0,0)}\frac{ u^4+u^2+v^2 }{ u^2+v^2 }=\lim\frac{ u^4 }{ u^2+v^2 }+\underbrace{\lim\frac{ u^2+v^2 }{ u^2+v^2 }}_{=1}.
\end{equation}
La première limite se calcule comme d'habitude, par passage aux coordonnées polaires, et elle vaut $0$. Au final, nous nous retrouvons avec $1\cdot(1+0)=1$.

\item
Ici, le dénominateur vaut $4$ en $(x,y)=(1,1)$, donc la fonction est continue dans un voisinage de ce point. Pas de problèmes, la limite est zéro.

\item
Un passage en polaire et une simplification par $r^2$ amène l'expression
\begin{equation}
	\frac{1}{ \cos^2(\theta)+1+r\cos^3(\theta) }.
\end{equation}
Remarquons que quelle que soit la valeur de $r$, avec $\theta=\pi/2$, nous trouvons que cela vaut $1$, tandis qu'avec $\theta=0$, cette expression est plus petite que $1/2$. Donc, dans n'importe quelle boule centrée en $(0,0)$, la fonction prend au moins une fois la valeur $1$ et une fois une valeur plus petite que $1/2$. Cela prouve que la limite n'existe pas.

\item
Regardons ce qu'il se passe dans la tangente. En passant aux coordonnées polaires, nous trouvons que
\begin{equation}
	\lim_{(x,y)\to(0,0)}\frac{ 2(x^2+y^2) }{ x^2+y^2+x^3 }=\lim_{r\to 0}\frac{ 2 }{ 1+r\cos^3(\theta) }=2.
\end{equation}
Étant donné que celle limite existe (et vaut $2$), la limite demandée vaut $\tan(\pi)=0$.

\end{enumerate}

\end{corrige}
