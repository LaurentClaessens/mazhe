% This is part of Exercices de mathématique pour SVT
% Copyright (C) 2010
%   Laurent Claessens et Carlotta Donadello
% See the file fdl-1.3.txt for copying conditions.

\begin{exercice}\label{exoTD3-0008}

	Modèle à ressources limitées (partage). Soit la suite $(u_n)_{n\in\eN}$ définie par
	\begin{equation}
		\begin{cases}
			u_{n+1}=u_nf(u_n)	&	\text{$\forall n\in\eN_0$}\\
			u_0=x,
		\end{cases}
	\end{equation}
	où $x\geq 0$ est un nombre réel et $f\colon \mathopen[ 0 , \infty [\to \mathopen[ 0 , a \mathclose]$ est la fonction définie par
	\begin{equation}
		f(y)=\begin{cases}
			a	&	\text{si $y<2$}\\
			0	&	 \text{si $y\geq 2$.}
		\end{cases}
	\end{equation}
	\begin{enumerate}
		\item
			Déterminer la limite de la suite $(u_n)_{n\in\eN}$ dans les cas suivants :
			\begin{enumerate}
				\item
					$a=\frac{ 1 }{2}$ et $x\geq 0$ quelconque,
				\item
					$a=1$ et $x=1$,
				\item
					$a=1$ et $x=3$,
				\item
					$a=2$ et $x\geq 0$ quelconque.
			\end{enumerate}
		\item
			On considère $x=1$ avec pour unité le million d'individus et le mois comme unité de temps. Au bout de combien de temps la population sera-t-elle intérieure à $100.000$ individus dans le cas où $a=\frac{ 9 }{ 10 }$ ?

			Même question dans le cas $a=\frac{ 11 }{ 10 }$.
	\end{enumerate}
	

\corrref{TD3-0008}
\end{exercice}
