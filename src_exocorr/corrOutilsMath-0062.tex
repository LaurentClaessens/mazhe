% This is part of Exercices et corrigés de CdI-1
% Copyright (c) 2011
%   Laurent Claessens
% See the file fdl-1.3.txt for copying conditions.

\begin{corrige}{OutilsMath-0062}

    Le gradient de $V$ est le champ de vecteurs
    \begin{equation}
        F(x,y,z)=\frac{-1}{ (x^2+y^2+z^2)^{3/2} }\begin{pmatrix}
            x    \\ 
            y    \\ 
            z    
        \end{pmatrix}.
    \end{equation}
    Il n'est toutefois pas utile de le savoir pour calculer la circulation de $F$ étant donné que nous connaissons le potentiel.

    Quel que soit le chemin $\sigma$ reliant le point $a$ au point $b$, nous aurons $\int_{\sigma}F=V(b)-V(a)$. Ici nous avons donc
    \begin{equation}
        \int_{\sigma}F=V(3,4,5)-V(1,2,3)=\frac{1}{ \sqrt{50} }-\frac{1}{ \sqrt{14} }.
    \end{equation}

\end{corrige}
