\begin{exercice}\label{exoCourbesSurfaces0005}

	Étudier (comme dans l'exercice \ref{exoCourbesSurfaces0002}) les courbes définies en coordonnées polaires $(r, \theta )$ par:
	\begin{multicols}{2}
		\begin{enumerate}
			\item\minsyndical
				\label{CSCi}
				$\displaystyle  r(\theta) =1+\cos \theta$ (la \defe{cardioïde}{cardioïde})
			\item\minsyndical
				\label{CSCii}
				$\displaystyle  r(\theta) = \sin (2\theta)$
			\item\coolexo
				\label{CSCiii}
				$\displaystyle  r(\theta) = \frac{\sin (\theta)}{\theta}$. 
			\item\coolexo
				\label{CSCiv}
				$\displaystyle r(\theta) = \frac{\theta-1}{\theta + 1}$
			\item\coolexo
				\label{CSCv}
				$\displaystyle r(\theta) = \cos (\theta)-\cos (2\theta)$
			\item\minsyndical
				\label{CSCvi}
				$\displaystyle r(\theta) = \frac{\cos (\theta)}{1 + \sin (\theta)}$
		\end{enumerate}
	\end{multicols}
	

\corrref{CourbesSurfaces0005}
\end{exercice}
