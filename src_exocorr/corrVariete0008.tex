% This is part of Exercices et corrigés de CdI-1
% Copyright (c) 2011
%   Laurent Claessens
% See the file fdl-1.3.txt for copying conditions.

\begin{corrige}{Variete0008}

	\begin{enumerate}

		\item
			Le chemin proposé est $\gamma(t)=(t,t/2)$, et l'intégrale est
			\begin{equation}
				\int_0^2t^2dx\begin{pmatrix}
					1	\\ 
					1/2	
				\end{pmatrix}+t^2dy\begin{pmatrix}
					1	\\ 
					1/2	
				\end{pmatrix}=
				\int_0^2\left( t^2+\frac{ t^2 }{ 4 } \right)=4.
			\end{equation}
		\item
			Le nouveau chemin est
			\begin{equation}
				\gamma(t)=(t,\frac{ t^2 }{ 4 }),
			\end{equation}
			et le chemin dérivé est $\gamma'(t)=(1,t/2)$. L'intégrale est
			\begin{equation}
				\int_0^2\omega_{\gamma(t)}\begin{pmatrix}
					1	\\ 
					t/2	
				\end{pmatrix}=
				\int_0^2(\frac{ t^3 }{ 2 }+\frac{ t^3 }{ 2 })=4
			\end{equation}
			
		\item
			Afin de ne pas s'ennuyer avec des racines carrés, il est plus facile de paramétrer le chemin en regardant $x$ comme fonction de $y$, c'est à dire
			\begin{equation}
				\gamma(t)=(2t^2,t),
			\end{equation}
			avec $t$ parcourant l'intervalle $\mathopen[ 0 , 1 \mathclose]$. Nous trouvons
			\begin{equation}
				\omega_{\gamma(t)}=4t^3dx+4t^4dy.
			\end{equation}
			L'intégrale est alors
			\begin{equation}
				\int_0^1\omega_{\gamma(t)}\begin{pmatrix}
					4t	\\ 
					1	
				\end{pmatrix}=
				\int_0^116t^4+4t^4=4.
			\end{equation}
		
		\item
			Le premier chemin est vertical et est $\gamma_1(t)=(0,t)$, tandis que le second est horizontal à la hauteur $1$ et est $\gamma_2(t)=(t,1)$. Pour le premier, $t\in\mathopen[ 0 , 1 \mathclose]$, tandis que pour le second, $t\in\mathopen[ 0 , 2 \mathclose]$. Nous avons $\omega_{\gamma_1(t)}=0$, donc le premier chemin ne compte pas. Il reste à intégrer
			\begin{equation}
				\omega_{\gamma_2(t)}=2tdx+t^2dy,
			\end{equation}
			donc
			\begin{equation}
				\int_0^2(2tdx+t^2dy)\begin{pmatrix}
					1	\\ 
					0	
				\end{pmatrix}=4.
			\end{equation}

	\end{enumerate}
	Toutes les intégrales valent $4$. La raison est que la forme $\omega$ est exacte,et que l'intégrale d'une forme différentielle exacte entre deux points ne dépend pas du chemin suivit. Afin de prouver que $\omega$ est exacte, il faut trouver une fonction $f(x,y)$ pour laquelle $\omega=df$. Il faut donc que
	\begin{equation}
		\begin{aligned}[]
			\frac{ \partial f }{ \partial x }=2xy&&\text{et}&&
			\frac{ \partial f }{ \partial y }=x^2.
		\end{aligned}
	\end{equation}
	La première équation montre, par simple intégration par rapport à $x$, que $f(x,y)=x^2y+c(y)$. La fonction $c(y)$ est la constante d'intégration; elle est constante par rapport à $x$, mais peut dépendre de $y$. La seconde équation fixe cette constante en fonction de $y$. Nous avons que $c(y)$ doit être une constante. Donc
	\begin{equation}
		\omega=d(x^2y+c)
	\end{equation}
	pour n'importe quel $c\in\eR$. Un simple calcul montre que $f(2,1)-f(0,0)=4$.

\end{corrige}
