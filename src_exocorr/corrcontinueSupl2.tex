% This is part of Exercices et corrigés de CdI-1
% Copyright (c) 2011
%   Laurent Claessens
% See the file fdl-1.3.txt for copying conditions.

\begin{corrige}{continueSupl2}

Pour les définitions précises, voir cours. Voici des exemples.
\begin{enumerate}
\item N'importe quelle fonction qui fait un saut en la valeur $a$, ou qui y tend vers l'infini. Par exemple $f(x)=\frac{ 1 }{ x-a }$.
\item La fonction $f(x)=\frac{ x^2-1 }{ x-1 }$ n'existe pas en $x=1$. La fonction $g(x)=x+1$, toutefois, est égale à $f$ partout et est continue en $x=1$.
\item La fonction \eqref{EqSupl1Dirich} fait évidement l'affaire. Moins tordu, on peut prendre n'importe quelle fonction non continue en $a$.

\end{enumerate}


\end{corrige}
