\begin{corrige}{006}
A general form of a function $\dpt{ \xi }{ S^1 }{ \Cyl }$ is
\[ 
  \xi(x,y)=(x,y,h(x,y))
\]
only defined on $x^2+y^2=1$. Here, $h$ is a smooth function. We naturally associate to $\xi$ the field
\[ 
  X_{(x,y)}=h(x,y)e^{i\big( \frac{ \pi }{ 2 }+\theta \big)}
\]
where $\theta$ is the angular polar coordinate of $(x,y)$. We have to prove that this field is smooth. For this, we have to prove that 
\begin{equation} \label{eq:xytohxy}
  (x,y)\to(x,y,h(x,y)e^{i\big( \frac{ \pi }{ 2 }+\theta \big)})
\end{equation}
is smooth at each point of $x^2+y^2=1$. Since $h$ is smooth, by definition it can be extended. So we can extend equation \eqref{eq:xytohxy} keeping the same form around any $(x_0,y_0)\in S^1$. The result is smooth in the sense of functions from an open set of $\eR^2$ to an open set in $\eR^3$. 


Let us now prove that a vector field defines a function. The general form of a vector field on $S^1$ is 
\begin{equation}
X_{(x,y)}=h(x,y)e^{i\big(\frac{ \pi }{ 2 }+\theta\big)}
\end{equation}
where $h$ is smooth on $S^1$ and can thus be extended into a smooth function on a neighbourhood of each point of $S^1$. The map $\dpt{ \xi }{ S^1 }{ \Cyl }$ we were looking for is thus
\[ 
  \xi(x,y)=(x,y,h(x,y)).
\]

\paragraph{A second way is possible}

The reader should remark that the construction given up to here is essentially the construction of exercise \ref{exo003}. We can direclty exploit the diffeomorphism $\dpt{ f }{ TS^1 }{ \Cyl }$. So in a first time, we pick a $\dpt{ \xi }{ S^1 }{ \Cyl }$ and we want to buils a vector field $X^{\xi}$, i.e. a map $\dpt{ X^{\xi} }{ S^1 }{ TS^1 }$. The natural candidate is
\[ 
  X^{\xi}=f^{-1}\circ\xi.
\]
On the one hand, smoothness comes from smoothness of $f^{-1}$ and $\xi$ and on the other hand, $X^{\xi}(x)\in T_xS^1$ because of projection property of $\xi$.

In a second time, we pick a vector field $\dpt{ X }{ S^1 }{ TS^1 }$ and we want to build a map $\dpt{ \xi^{X} }{ S^1 }{ \Cyl }$. The candidate is 
\[
\xi^X=f\circ X.
\]
 You have to check that $\pr_1\circ\xi^X=\id$; it comes from construction of $f$.

Now in order to say that the data of the function is ``the same'' that the data of the vector field, we have to prove that the two constructions are inverse each other. In other words, you are now able to do two things: if I give you a vector field, you can give me a function and if I give you a function, you can give me a vector field. Let me give you the function $\xi$. So you give me a vector field $X^{\xi}$. Now I give you $X^{\xi}$; will you give back the original function $\xi$ ? If not, the whole construction has no interest.
\[ 
  \xi\to X^{\xi}\to\xi^{X^{\xi}}\stackrel{?}{=}\xi.
\]
From definitions, $\xi^{X_{\xi}}=f\circ X^{\xi}=f\circ f^{-1}\circ\xi=\xi$. This is a good point. The other is 
\[ 
  X\to\xi^X\to X^{\xi^X}\stackrel{?}{=}X;
\]
from definitions, $X^{\xi^X}=f^{-1}\circ\xi^X=f^{-1}\circ f\circ X=X$.


\end{corrige}
