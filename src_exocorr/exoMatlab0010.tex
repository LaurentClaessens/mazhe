\begin{exercice}\label{exoMatlab0010}

	Thorgal, XIII, Kid Paddle et Gaston se rendent dans un Kebab. Thorgal prend $8$ assiettes gyros spéciales, $10$ frites et $6$ boissons. Il paie $43$ euros. XIII mange $2$ frites et deux boissons, mais ne prend pas d'assiette gyros spéciale, et il paie $9$ euros. Kid Paddle se contente d'une assiette gyros spéciale et d'une boisson et paie $4.5$ euros.

	Gaston voudrait prendre une assiette gyros spéciale avec frites. Combien devra-t-il payer ?

	\par
	\emph{Indice :} si $A$, $F$ et $B$ désignent les prix des assiettes, des frites et des boissons, il faut résoudre le système
	\begin{subequations}
		\begin{numcases}{}
\nonumber
			8A+6B+10F=43\\
\nonumber
			2B+2F=9\\
\nonumber
			A+B=4.5
		\end{numcases}
	\end{subequations}
	et en déduire la valeur de $A+F$.

\corrref{Matlab0010}
\end{exercice}
