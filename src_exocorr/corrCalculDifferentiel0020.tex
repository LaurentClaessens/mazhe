\begin{corrige}{CalculDifferentiel0020}

	Nous devons montrer que $g$ est continue, et puis que ses dérivées partielles existent et sont également continues.
	
	Commençons par prouver que $g$ est continue. Nous devons prouver que pour tout $h\in\eR$,
	\begin{equation}		\label{EqeCDzvgtlly}
		\lim_{(x,y)\to(h,h)}\frac{ f(x)-f(y) }{ x-y }=f'(h).
	\end{equation}
	Étant donné que $f$ est de classe $C^2$, nous pouvons effectuer un développement en suivant la formule \eqref{EqDevfautouraeps}.
	\begin{subequations}
		\begin{align}
			f(x)&=f(h)+f'(h)(x-h)+\alpha(x)(x-h)		\label{EqeCDzzszfx}\\
			f(y)&=f(h)+f'(h)(y-h)+\alpha(y)(y-h)
		\end{align}
	\end{subequations}
	où la fonction $\alpha$ a la propriété que $\lim_{x\to h} \alpha(x)=0$. En substituant ces expressions dans la limite à calculer \eqref{EqeCDzvgtlly}, nous trouvons
	\begin{equation}
		\begin{aligned}[]
			\lim_{(x,y)\to(h,h)}\frac{ f(x)-f(y) }{ x-y }=f'(h)+\lim_{(x,y)\to(h,h)}\big( \alpha(x)-\alpha(y) \big)=f'(h),
		\end{aligned}
	\end{equation}
	ce qu'il fallait. La fonction $g$ est donc continue. Il s'agit maintenant de prouver la continuité des dérivées partielles aux points $(h,h)$ (la fonction $f$ étant $C^2$, le quotient qui définit $g$ est certainement $C^1$ aux points autres que $(h,h)$).

	En vertu de la proposition \ref{PropDerContCun}, pour prouver que $g$ est $C^1$, il suffit de prouver que les dérivées partielles existent et sont continues. Nous devons donc vérifier que $\lim_{(x,y)\to(h,h)}\partial_xf(x,y)=\partial_xf(h,h)$ (et idem  pour $y$). D'abord, si $(x,y)$ n'est pas de la forme $(h,h)$, nous avons
	\begin{equation}		\label{Eqzzdzpxgxyun}
		\partial_xg(x,y)=\frac{ f'(x)(x-y)-\big( f(x)-f(y) \big) }{ (x-y)^2 }.
	\end{equation}
	Ensuite, 
	\begin{equation}
		\partial_xf(h,h)=\lim_{t\to 0} \frac{ g(h+t,h)-g(h,h) }{ t }=\lim_{t\to 0} \frac{ \frac{ f(h+t)-f(h) }{ (h+t)-h }-f'(h) }{ t }.
	\end{equation}
	Pour effectuer cette limite, il ne suffit pas d'écrire $f(h+t)$ en substituant $x$ par $h+t$ dans \eqref{EqeCDzzszfx}. En effet, simplifications, il reste
	\begin{equation}
		\lim_{t\to 0} \frac{ \frac{ f'(h)t+\alpha(h+t)t }{ t }-f'(h) }{ t }=\lim_{t\to 0} \frac{ \alpha(h+t) }{ t }.
	\end{equation}
	Cette dernière limite est indéterminée; il faut donc développer jusqu'à un ordre supérieur :
	\begin{equation}
		f(h+t)=f(h)+tf'(h)+\frac{ t^2 }{ 2 }f''(h)+\beta(t)t^2
	\end{equation}
	En effectuant le calcul, nous trouvons que
	\begin{equation}
		\partial_xf(h,h)=\frac{ 1 }{2}f''(h).
	\end{equation}
	Nous calculons par ailleurs $\lim_{(x,y)\to(h,h)}\partial_xg(x,y)$ en substituant le développement (voir \eqref{Eqfydevfx})
	\begin{equation}
		f(y)=f(x)+f'(x)(y-x)+f''(x)\frac{ (y-x)^2 }{2}+\beta(y-x)(y-x)^2
	\end{equation}
	dans l'équation \eqref{Eqzzdzpxgxyun} :
	\begin{equation}
		\begin{aligned}[]
			\lim_{(x,y)\to(h,h)}&\partial_xg(x,y)\\&=\lim_{(x,y)\to(h,h)}\frac{ f'(x)(x-y)-\Big( -f'(x)(y-x)-f''(x)\frac{ (y-x)^2 }{2}-\alpha(y-x)(y-x)^2 \Big) }{ (x-y)^2 }\\
			&=\lim_{(x,y)\to(h,h)}\frac{ f''(x) }{2}-\alpha(y-x)\\
			&=\frac{ f''(h) }{2}.
		\end{aligned}
	\end{equation}
	Étant donné que $\lim_{(x,y)\to(h,h)}\partial_xg(x,y)=\partial_xg(h,h)$, nous avons la continuité de la dérivée première dans la direction de $x$. La dérivée dans la direction de $y$ se traite de la même façon, et nous obtenons la continuité.

	La fonction $g$ est donc $C^1$ par la proposition \ref{PropDerContCun} parce que ses dérivées partielles sont continues.

\end{corrige}
