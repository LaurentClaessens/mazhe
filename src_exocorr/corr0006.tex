% This is part of Exercices et corrigés de CdI-1
% Copyright (c) 2011
%   Laurent Claessens
% See the file fdl-1.3.txt for copying conditions.

\begin{corrige}{0006}

\begin{enumerate}
\item $x_k=1$ pour tout $k$,
\item $x_k=(-1)^k$
\item $x_k=k$,
\item $x_k=(-1)^k$,
\item Impossible parce que pour $\epsilon=0.01$, il n'y a pas d'entiers $n$ tels que $| n-\pi |\leq\epsilon$,
\item $x_k=1-\frac{1}{ k }$,
\item $x_k=(-1)^k$,
\item $x_k=\frac{1}{ k }$,
\item impossible par le théorème de la page 43,
\item impossible, en effet, par le point précédent, il faudrait une suite non bornée. Donc pour tout $M$, il existe un $K$ tel que $x_K>M$. Mais, étant donné que la suite est croissante, pour tout $k>K$, $x_k\geq x_K>M$. Cela prouve que la suite a l'infini comme limite.
\item Impossible parce que décroissant et divergent implique non bornée. Donc $\forall M<0$, il existe un $K$ tel que $k>K$ implique $x_k<M$. Toute sous suite \og converge\fg{}  donc également vers $-\infty$.
\end{enumerate}


\end{corrige}
