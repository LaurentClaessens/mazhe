% This is part of Analyse Starter CTU
% Copyright (c) 2014
%   Laurent Claessens,Carlotta Donadello
% See the file fdl-1.3.txt for copying conditions.

\begin{exercice}\label{exoanalyseCTU-0005}
    
    Pour s'exercer à manipuler les fonctions trigonométrique inverses
    \begin{enumerate}
        \item
            Montrer que pour tout \( y\in \eR\) nous avons
            \begin{equation}
                \cos\big( \arctan(y) \big)=\frac{1}{ \sqrt{1+y^2} }
            \end{equation}
            et
            \begin{equation}
                \sin\big( \arctan(y) \big)=\frac{ y }{ \sqrt{1+y^2} }.
            \end{equation}
        \item
            Montrer que pour tout \( y>0\),
            \begin{equation}
                \arctan(y)+\arctan(\frac{1}{ y })=\frac{ \pi }{2}.
            \end{equation}
        \item
            En déduire que pour tout \( y<0\),
            \begin{equation}
                \arctan(y)+\arctan(\frac{1}{ y })=-\frac{ \pi }{ 2 }.
            \end{equation}
    \end{enumerate}

\corrref{analyseCTU-0005}
\end{exercice}
