% This is part of Analyse Starter CTU
% Copyright (c) 2015
%   Laurent Claessens,Carlotta Donadello
% See the file fdl-1.3.txt for copying conditions.

\begin{corrige}{analyseCTU-0103}

  \begin{enumerate}
  \item Soit $f:\,\eR\to\eR$ une fonction paire.
      \begin{equation*}
        \begin{array}{|c|c|c|}
         \hline
         \text{Function} & \text{Sym\'etrie} & \text{Preuve ou exemple}\\
          \hline
            f_1 (x) = f(-x) & \text{paire} & f_1 (-x) = f(-(-x)) = f(-x) = f_1(x) \\
            \hline
            f_2 (x) = f(x)-1 & \text{paire} & f_2 (-x) = f(-x)-1 =f(x)-1 = f_2(x) \\
            \hline
            f_3 (x)= 
            \begin{cases}
              f(x) &\text{si } x>0, \\
              f(-x)& \text{si } x\leq 0,
            \end{cases} &\text{paire} & \text{En effet } f_3(x) = f(|x|),\\
            && \:\text{donc } f_3(-x) = f(|-x|)= f(|x|) = f_3 (x) \\
            \hline
            f_5(x) = \sqrt{(f(x))^2} &\text{paire}& \text{En effet } f_5(x) = |f(x)|,\\
            && \:\text{donc } f_5(-x) = |f(-x)| = |f(x)| = f_5(x)\\
            \hline
            f_6(x)= |f(x)| + f(x) &\text{paire}& f_6 \: \text{est la somme de deux fonction paires} \\
            \hline
            f_7(x) = xf(x)&\text{impaire}& f_7 \: \text{est le produit d'une fonction paire et une impaire} \\
            &&f_7(-x) = -xf(-x) = -xf(x) = -f_7(x)\\
            \hline
            f_8(x) = g(f(x)), \:\text{avec $g$ impaire} &\text{paire}& f_8(-x) = g(f(-x))= g(f(x))=f_8(x) \\
\hline
        \end{array}
      \end{equation*}
\item  Soit $f:\,\eR\to\eR$ une fonction impaire.
      \begin{equation*}
        \begin{array}{|c|c|c|}
         \hline
         \text{Function} & \text{Sym\'etrie} & \text{Preuve ou exemple}\\
          \hline
            f_1 (x) = f(-x) & \text{impaire} & f_1 (-x) = f(-(-x)) = -f(-x) = -f_1(x) \\
            \hline
            f_2 (x) = f(x)-1 & \text{ ni paire, } & f_2 (-x) = f(-x)-1 =-f(x)-1  \\
&\text{ni impaire}&f_2 \:\text{est paire si et seulement si } f (x) = 0, \:\text{pour tout }x\\
            \hline
            f_3 (x)= 
            \begin{cases}
              f(x) &\text{si } x>0, \\
              f(-x)& \text{si } x\leq 0,
            \end{cases} &\text{paire} & \text{En effet } f_3(x) = f(|x|),\\
            && \:\text{donc } f_3(-x) = f(|-x|)= f(|x|) = f_3 (x) \\
            \hline
            f_5(x) = \sqrt{(f(x))^2} &\text{paire}& \text{En effet } f_5(x) = |f(x)|,\\
            && \:\text{donc } f_5(-x) = |f(-x)| = |-f(x)| = f_5(x)\\
            \hline
            f_6(x)= |f(x)| + f(x) &\text{ ni paire, }& f_6(-x) = |f(x)| - f(x) \neq f_6(x)\:\text{et } -f_6(x)  \\
&\text{ni impaire} & f_6 \:\text{est paire si et seulement si } f (x) = 0, \:\text{pour tout }x\\
            \hline
            f_7(x) = xf(x)&\text{paire}& f_7 \: \text{est le produit de deux fonctions impaires} \\
            &&f_7(-x) = -xf(-x) = xf(x) = f_7(x)\\
            \hline
            f_8(x) = g(f(x)), \:\text{avec $g$ impaire} &\text{impaire}& f_8(-x) = g(-f(x))= -g(f(x))=-f_8(x) \\
\hline
        \end{array}
      \end{equation*}
  \end{enumerate} 
\end{corrige}
