% This is part of (almost) Everything I know in mathematics
% Copyright (c) 2015
%   Laurent Claessens
% See the file fdl-1.3.txt for copying conditions.

%+++++++++++++++++++++++++++++++++++++++++++++++++++++++++++++++++++++++++++++++++++++++++++++++++++++++++++++++++++++++++++ 
\section{The conformal condition}
%+++++++++++++++++++++++++++++++++++++++++++++++++++++++++++++++++++++++++++++++++++++++++++++++++++++++++++++++++++++++++++

%--------------------------------------------------------------------------------------------------------------------------- 
\subsection{Preliminary discussion}
%---------------------------------------------------------------------------------------------------------------------------

If one looks at \cite{ooIYOHooMRMfXl,ooDPRUooOFPyPH}, one sees that a conformal transformation is a transformation of a (pseudo)riemannian manifold that leaves the metric unchanged up to a positive scalar function : \( g'_{\mu\nu}(x')=\Omega(x)g_{\mu\nu}(x)\). Thus we are looking for the maps \( \phi\colon M\to M\) that realise that condition.

Since our objective is to do differential geometry, we cannot follow the computations in \cite{ooIYOHooMRMfXl,ooDPRUooOFPyPH} because there are too much «infinitesimal» in that and we don't understand anything\footnote{No pun intended : these books are very okay, but as far as we are interested in bundles, there are some works to do.}.

In order to determine the Poincaré group that leaves the metric invariant, we were fortunate because of theorem \ref{ThoDsFErq} that ensured linearity of the map \( \phi\). Our search for the Poincaré group\footnote{By the way given by theorem \ref{THOooQJSRooMrqQct}.} was thus simplified by two circumstances :
\begin{itemize}
    \item \( \phi\) and \( d\phi\) are the same.
    \item the condition \( g_{\mu\nu}=g'_{\mu\nu}\) does not involves a specific point, so that we had not to ask ourself questions about the «base point» of the vectors.
\end{itemize}
Here we are in a more complicated situation. 

Here is our first try. Let \( (M,g)\) be a (pseudo)riemannian manifold. A conformal map will be \( \phi\colon M\to M\) such that
\begin{equation}        \label{EQooCCMMooXVbTAd}
    g_{\phi(x)}\big( d\phi_xv,d\phi_xw \big)=\Omega(x)g_x(v,w)
\end{equation}
for a function \( \Omega\in C^{\infty}(M)\). The condition \eqref{EQooCCMMooXVbTAd} has to hold for every \( x\in M\) and \( v,w\in T_xM\).

Well. This is not the correct definition. Let us be clear : the set of maps satisfying \eqref{EQooCCMMooXVbTAd} is for sure interesting, but this is not the what we call the conformal group.

The reason is that we want to encode a deforming material which respects the angles. A vector at point \( A\) is an arrow joining point \( A\) to a point \( B\). The image of the vector \( \vect{ AB }\) has to be \( \vect{ \phi(A)\phi(B) }\) when the material is deformed. Thus vectors have to be transported by \( \phi\) instead of \( d\phi\). We could speak about affine spaces as described around definition \ref{DEFooQELZooEXvxgw}, but instead we will describe our subject with a vector bundle.

%--------------------------------------------------------------------------------------------------------------------------- 
\subsection{The setting}
%---------------------------------------------------------------------------------------------------------------------------

Let \( V\) be a finite dimensional vector space and \( E=V\times V\) be the trivial vector bundle with fibre \( V\). We denote \( V_x=\{ (x,v)\tq v\in V \}\) and for each \( x\in V\) we have a non-degenerate bilinear form \( g_x\colon V\to V\). We define
\begin{equation}
    g\big( (x,v),(x,w) \big)=g_x(v,w)
\end{equation}
and we will often directly write \( g_x(v,w)\) or \( v\cdot w\) when \( v,w\) belong to \(V_x\) instead of \( V\).

A map \( \phi\colon V\to V\) also acts on the vector bundle as
\begin{equation}
    \phi(x,v)=\big( \phi(x),\phi(x+v)-\phi(x) \big).
\end{equation}
This way to act translates the fact that for a vector, we displace the ending point as well as the starting point with \( \phi\). This is not the same as displacing the vector by \( d\phi_x\).

\begin{definition}      \label{DEFooVKNBooFBWQQM}
    A \defe{conformal map}{conformal map} is a \(  C^{\infty}\) map \( \phi\colon V\to V\) for which there exists a function \( \Omega\in C^{\infty}(V)\) satisfying
    \begin{equation}        \label{EQooOZDUooCDaIrh}
        v\cdot w=\Omega(x) \phi(v)\cdot \phi(w)
    \end{equation}
    for every \( x\in V\) and every \( v,w\in V_x\).
\end{definition}
With no abuse of notations, the condition \eqref{EQooOZDUooCDaIrh} reads, for \( v,w\in V\) :
\begin{equation}
    g_x(v,w)=\Omega(x)g_{\phi(x)}\big(  \phi(x+v)-\phi(x),\phi(x+w)-\phi(x)  \big).
\end{equation}

Moreover we consider the case in which the metric is flat an \( g_x=\eta\) for every \( x\).

%--------------------------------------------------------------------------------------------------------------------------- 
\subsection{Generators}
%---------------------------------------------------------------------------------------------------------------------------

We are going to determine the generators of the conformal group\footnote{Did you checked that maps satisfying definition \ref{DEFooVKNBooFBWQQM} form a group ?}, that is the conformal maps that are next to the identity in the following sense : we consider a family \( \phi_t\) of conformal maps written under the form
\begin{equation}
    \phi_t(x)=x+t\sigma_t(x)
\end{equation}
with \( \| \sigma_t(x) \|\leq 1  \) for every \( x\).

The aim in this way to work is to capture the following idea. If \( \phi\) is a conformal map such that points the neighbourhood \(  \mO=B(0,1/2)  \) of \( x_0\) remain in \( B(x_0,1)\), then in \( \mO\) we have \( \phi(x)=x+f(x)\) with \( \| f(x) \|\leq 3/2\).

The condition to be satisfied for every \( t\) is :
\begin{equation}        \label{EQooNQOAooALfgJc}
    \eta(v,w)=\Omega_t(x)\eta\Big(   x+v+t\sigma_t(x+v)-x-t\sigma_t(x),x+w+t\sigma_t(x+w)-x-t\sigma_t(x)    \Big).
\end{equation}
We develop the expression of the right hand side :
\begin{equation}
    \begin{aligned}[]
        \eta(v,w)&+\eta\big( v,t\sigma_t(x+w)-t\sigma_t(x) \big)+\eta\big( w,t\sigma_t(x+v)-t\sigma_t(x) \big)\\
        &+\eta\big( t\sigma_t(x+v),t\sigma_t(x+w) \big)\\
        &-\eta\big( t\sigma_t(x+v),t\sigma_t(x) \big)\\
        &-\eta\big( t\sigma_t(x),t\sigma_t(x+w) \big)\\
        &+\eta\big( t\sigma_t(x),t\sigma_t(x) \big).
    \end{aligned}
\end{equation}
The whole has to be proportional to \( \eta(v,w)\). The last four terms are regrouped as
\begin{equation}
    t^2\big( \sigma_t(x+v),\sigma_t(x) \big)\cdot \big( \sigma_t(x+w)-\sigma(x) \big)=t^2\alpha_x(v,w)
\end{equation}
where \( \| \alpha_x(v,w) \|\leq 4\). For every \( t\), the equation
\begin{equation}
        \eta(v,w)=\Omega_t(x)\eta(v,w)+\eta\big( v,t\sigma_t(x+w)-t\sigma_t(x) \big)+\eta\big( w,t\sigma_t(x+v)-t\sigma_t(x) \big)+t^2\alpha_x(v,w)
\end{equation}
holds. The second term in the right hand side has to be proportional to \( \eta(v,w)\) with a coefficient that we name \( \omega_t(x)\) :
\begin{equation}
    t\eta\big( v,\sigma_t(x+w)-\sigma_t(x) \big)+t\eta\big( w+\sigma_t(x+v)-\sigma_t(x) \big)+t^2\alpha_x(v,w)=\omega_t(x)\eta(v,w).
\end{equation}
We divide by \( t\) and take the limit \( t\to 0\). The fact that the limit in the left hand side exists ensure us that \( \lim_{t\to 0} \frac{ \omega_t(x) }{ t }\) exists. We name it \( A(x)\). it is not mandatory to have \( \lim_{t\to 0} \sigma_t(x)=0\) because \( \sigma_t\) is bounded and \( \lim_{t\to 0} \phi_t(x)=x\) even with \(\sigma_t(x)\) remaining large. Denoting \( \sigma(x)=\lim_{t\to 0} \sigma_t(x)\) we get
\begin{equation}        \label{EQooLBWNooRbaXzL}
    \eta(v,\sigma(x+w)-\sigma(x))+\eta(w,\sigma(x+v)-\sigma(x))=A(x)\eta(v,w).
\end{equation}

\begin{probleme}
    Well. This is the way I see to get rid of the exceeding terms \( \alpha_x\). I guess that there is something better. The problem with the «regular» approach is that writing \( \phi(x)=x+\epsilon(x)\) with small \( \epsilon\) gives an extra term (which correspond to our \( \alpha\)) of the form
    \begin{equation}
        d\epsilon_x(v)\cdot d\epsilon_x(w).
    \end{equation}
    I can understand why \( \epsilon\) being small a quadratic term in \( \epsilon\) can be eliminated by a limiting process (as we've done here), but why on Earth \( \epsilon\) being small the differential \( d\epsilon\) has to be small ?

    My idea is that for each \( t\), we have written \eqref{EQooNQOAooALfgJc} which is true for every \( x\) in a small neighbourhood of \( x_0\). The neighbourhood depends on \( t\). Our limit process will give a \( \sigma\) that is valid in no neighbourhood of \( x_0\), but which is a generator (definition of a generator). Then the game will be to retrieve a correct formula for the group.
\end{probleme}

Back to \eqref{EQooLBWNooRbaXzL}. We write it for \( tv\) instead of \( v\), we divide by \( t\) and take the limit. The first term remain unchanged, the second becomes 
\begin{equation}
    \eta\big( w,d\sigma_x(v) \big)
\end{equation}
and the right hand side remain unchanged. Same game with \( w\) and we are left with
\begin{equation}        \label{EQooLMIAooBgdgMj}
    v\cdot d\sigma_x(w)+w\cdot d\sigma_x(w)=A(x)\eta(v,w).
\end{equation}

\begin{remark}
    Equation \eqref{EQooLMIAooBgdgMj} is the one that reads
    \begin{equation}
        \partial_{\mu}\epsilon_{\nu}+\partial_{\nu}\epsilon_{\mu}=\omega(x)g_{\mu\nu}
    \end{equation}
    in books.
\end{remark}
