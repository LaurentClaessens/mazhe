% This is part of Exercices et corrections de MAT1151
% Copyright (C) 2010
%   Laurent Claessens
% See the file LICENCE.txt for copying conditions.

 Ce chapitre contient les exercices et rappels théorique non encore triés pour les prochaines séances. Avec l'avancement du quadrimestre, ce chapitre va progressivement se vider au profit des chapitres de théorie et d'exercices.

\thispagestyle{empty}
{
\center {\bf Exercices de Calcul Numérique (MAT1151). \\
Année académique 2004-2005.\\
Séance 7: Équations différentielles}
}

\vspace{4mm}

\noindent {\bf 0}  Résoudre le problème de 
Cauchy suivant: $y'=xy^2\;;\;y(0)=1$.

\vspace{4mm}


 \noindent {\bf 1}  Décrire la résolution numérique par méthode d'Euler du problème précédent
pour $x\in[0,1]$ discrétisé par 4 points $\{x_0=0,x_1,x_2,x_3=1\}$ équidistants.

\vspace{4mm}

\noindent{\bf 2} Même question par la méthode du développement de Taylor.
\vspace{4mm}

\noindent{\bf 3}  Même question par une méthode de Runge-Kutta d'ordre 2.
\vspace{4mm}

\noindent{\bf 4}  
Décrire précisément dans le cas du problème de Cauchy $y'(x)=f(x,y)\;;\;y(x_0)=y_0\quad f\in C^\infty(\eR)$
la méthode de Runge-Kutta d'ordre 3.

