\ctg{G\'{e}om\'{e}trie analytique. Exercices.}
\ctd{Exercices.}
\ifpdf\makeatletter\@twosidefalse\makeatother\fi
\part[Cours]{G\'{e}om\'{e}trie analytique\\ Exercices\protect\thispagestyle{ctu}}
\ifpdf\makeatletter\@twosidetrue\makeatother\fi
\cleardoublepage
\endinput
