% This is part of (almost) Everything I know in mathematics
% Copyright (c) 2013-2014
%   Laurent Claessens
% See the file fdl-1.3.txt for copying conditions.

%+++++++++++++++++++++++++++++++++++++++++++++++++++++++++++++++++++++++++++++++++++++++++++++++++++++++++++++++++++++++++++
					\section{Functional, representation and automorphism}
%+++++++++++++++++++++++++++++++++++++++++++++++++++++++++++++++++++++++++++++++++++++++++++++++++++++++++++++++++++++++++++

Let $A$ be an algebra and consider a linear functional $\varphi\colon A\to \eC$. That induces a GNS representation $\pi\colon A\to \End(V)$ with
\begin{equation}
  V=\frac{ A }{ \{ b\tq\varphi(ab)=0\,\forall a \}. }
\end{equation}
If $\varphi$ is nondegenerate, the latter ideal reduces to $\{ 0 \}$. Let us assume a sort of Riesz theorem: for every linear functional $\psi\colon A\to \eC$, there exists a $b\in A$ such that $\psi(a)=\varphi(ab)$, $\forall a$. If one fixes $b$, one can see $\varphi(ba)$ as a functional for $a$, and thus define $\sigma(b)$ by
$\varphi(ba)=\varphi\big(a\sigma(b)\big)$.

One can check that this $\sigma$ is an automorphism of $A$.

%+++++++++++++++++++++++++++++++++++++++++++++++++++++++++++++++++++++++++++++++++++++++++++++++++++++++++++++++++++++++++++
\section{Commutant}
%+++++++++++++++++++++++++++++++++++++++++++++++++++++++++++++++++++++++++++++++++++++++++++++++++++++++++++++++++++++++++++

Let $M\subset\oB(\hH)$ be a collection of bounded operators on $\hH$. The \defe{commutant}{commutant} of $M$ in $\oB(\hH)$ is\nomenclature[O]{$M'$}{The commutant of $M$}
\begin{equation}
M'=\{ S\in\oB(\hH)\tq \text{ $TS=ST$ for every $T\in M$ } \}.
\end{equation}
From now let $M$ be a self-adjoint algebra of operators on an Hilbert space $\hH$ which contains $\mtu$. We have three lemmas.

\begin{lemma}		\label{LemUnVN}
If $v\in\hH$ and if $P$ is the projection onto $\overline{ Mv }\subseteq \hH$, then $P$ commutes with all operators in $M$. 
\end{lemma}

\begin{lemma}		\label{LemDeuxVN}
If $S\in M''$, $v\in \hH$ and $\epsilon >0$, there exists a $T\in M$ such that $\| Sv-Tv \|\leq\epsilon$. Moreover, in the finite dimensional case, there exists a $T\in M$ such that $Sv=Tv$.
\end{lemma}

\begin{proof}
Consider $P$ as in lemma \ref{LemUnVN}, then $P\in M'$, so that $XP=PX$. Now, the fact that $M$ is unital gives $v\in \overline{ Mv }$ from which we deduce $Pv=v$. Thus we have
\[ 
  Xv=XPv=PXv\in\overline{ Mv }
\]
because $Xv=P(Xv)$. The lemma now result from the fact that an element of the closure of $Mv$ is as close as we want from $Mv$.
\end{proof}

\begin{lemma}		\label{LemTroisVN}
If $T\in M''$, $v_1, \ldots v_n\in\hH$, and $\epsilon>0$, there exists a $T\in M$ such that $\| Sv_i-Tv_i \|\leq\epsilon$ for all $i$. Moreover if the finite dimensional case, there exists a $T$ such that $Sv_i=Tv_i$.
\end{lemma}

\begin{proof}
Apply lemma \ref{LemDeuxVN} to the set
\[ 
  N=\{ T\oplus T\oplus\ldots\oplus T\in\oB(\hH\oplus\ldots\oplus\hH)\tq T\in M \}.
\]
\end{proof}

\begin{corollary}
In the finite dimensional case, we have $M''=M$.
\end{corollary}

\begin{proof}
Take a basis of $\hH$ in the lemma \ref{LemTroisVN}.
\end{proof}

\begin{theorem}[Double commutant]		\label{ThoDoubleCommutant}
	Let $M\subset\oB(\hH)$ be a set of bounded operators on the Hilbert space $\hH$. The double commutant $M''$ is the strong closure of $M$, i.e. the set of strong limits of nets.
\end{theorem}

\begin{proof}
A first evidence is that $M\subseteq M''$. Secondly, every commutant is strongly closed, so that $M''$ in particular is strongly closed. So it remains to be proved that for each $S\in M''$ is the limit of some net in $M$ in the strong topology. For that, consider the directed set of finite subsets of $\hH$ and consider the directed set
\[ 
  A=\{ \text{finite subsets of $\hH$} \}\times ]0,\infty[
\]
on which we say $(F,\epsilon)\geq (F',\epsilon')$ if and only if $F'\subseteq F$ and $\epsilon <\epsilon'$.

For each $a=(F,\epsilon)\in A$, we define $T_{(F,\epsilon)}$ as in lemma \ref{LemTroisVN}, so we have, for all $v\in F$,
\[ 
  \| T_{(F,\epsilon)}v-Sv \|\leq\epsilon,
\]
which proves that $T_{(F,\epsilon)}$ converges to $S$ in the strong topology.
\end{proof}

Notice that the operator $S$ is in general \emph{not} a limit of a sequence.

\begin{definition}
A \defe{von Neumann algebra}{von Neumann algebra@von Neumann algebra} is an unital $*$-subalgebra of $\oB(\hH)$ which is equal to its double commutant.
\end{definition}

Let $G$ be a group and $\pi$, an unitary representation of $G$ and $M$ be the commutant of $\pi(G)$. Since $\pi(g)^*=\pi(g)^{-1}=\pi(g^{-1})$, we have that $\pi(G)$ is self-adjoint and $M$ is a von~Neumann algebra.

We say that a von~Neumann algebra is a \defe{factor}{factor} if its center is trivial, i.e. reduces to $\eC\,\id$.

\begin{proposition}	\label{PropprojrepresVN}
Let $\pi$ be a unitary representation of $G$ on $\hH$ and $\hH_1$, an invariant subspace of $\pi(G)$. We have
\begin{itemize}
\item the orthogonal projection on $\hH_1$ belongs to $M$,
\item if $P\in M$ is a projection, then $P\hH$ is a subrepresentation of $\pi$, i.e. the closed subspace $\pi(G)P\hH$.
\end{itemize}
\end{proposition}
So, if $\pi$ is irreducible then $M$ is made of multiples of identity.

\begin{lemma}[\cite{Wassermann}]		\label{LeminvarMprime}
Let $\hS$ be a selfadjoint part of $\oB(\hH)$. A closed subspace $\hH_1$ of $\hH$ is $\hS$-invariant if and only if the orthogonal projection on $\hH_1$ belongs to $\hS'$.
\end{lemma}

\begin{proof}
First, suppose that $\hH_1$ is $\hS$-invariant. Thus for every $v\in\hH_1$ and $w\in \hH_1^{\perp}$, we have
\[ 
  0=\langle Sv,w, \rangle =\langle v, S^*w\rangle 
\]
where $S^*\in\hS$ by assumption. We conclude that $S^*w\perp\hH_1$ for every $S\in\hS$. Since $\hS^*=\hS$, we have that $\hH_1^{\perp}$ is $\hS$-invariant. Now if $P$ is the orthogonal projector on $\hH_1$, we decompose $x\in\hH$ as
\begin{equation}		\label{EqDecomxPxSP}
  x=Px\oplus(\id-P)x\in\hH_1\oplus\hH_1^{\perp}.
\end{equation}
since $\hH_1^{\perp}$ is $\hS$-invariant, or every $S\in\hS$, we have $S(\id-P)x\in\hH_1^{\perp}$, so that $PS(\id-P)x=0$. Thus, using the decomposition \eqref{EqDecomxPxSP}, $PSx=PSPx\oplus 0=PSPx$. But $Px\in \hH_1$, then $SPx\in\hH_1$ and $P\big( SPx \big)=SPx$. We have proven that $PS=SP$, it is $P\in\hS'$.

For the second part, assume that $P\in\hS'$, then for every $v\in\hH_1$, we have $v=Pv$ and
\[ 
  Sv=SPv=PSv\in\hH_1,
\]
which proves that $\hH_1$ is $\hS$-invariant.
\end{proof}


\begin{proposition}
If $\hH_1$ and $\hH_2$ are equivalent subrepresentations of $\hH$, then the intertwining operator $W\colon \hH_1\to \hH_2$ determines on $\hH$ a partial isometry such that $P_1=W^*W$ and $P_2=WW^*$.
\end{proposition}

The following decomposition is the \defe{polar decomposition}{polar!decomposition!in von~Neumann algebra}\index{decomposition!polar in von~Neumann algebra}.
\begin{proposition}		\label{PropPolarvNA}
There exists a partial isometry $V\colon \overline{ \Image\big(| T |\big) }\to \overline{ \Image(T^*) }$ such that $T=V| T |$.
\end{proposition}

In other words, any element of a von~Neumann algebra is the product of a positive operator by a projection. A version of this decomposition for operators in Hilbert spaces is given in lemme \ref{LemPolarHilbert}.

\begin{proof}
The operator $V$ must satisfy $V\big( | T |v \big)=Tv$ for every $v\in\hH$. That operator is a partial isometry because
\[ 
  \big\| | T |v \big\|^2=\langle | T |v, | T |v\rangle =\langle | T |^2v, v\rangle =\langle T^*Tv, v\rangle =\| Tv \|^2.
\]
It remains to be proved that $V\in M$.
\end{proof}



\begin{lemma}		\label{LemVNCommunit}
Every von~Neumann algebra is the commutant of an unitary representation.
\end{lemma}

\begin{proof}
Let $M$ be a von~Neumann algebra and consider that group $G=U(M')$, the group of unitary operators on $M'$. Let $T\in M'$, it reads as the sum of self-adjoint operators by
\[ 
  T=\frac{ 1 }{2}(T+T^*)-\frac{ i }{2}(iT-iT^*).
\]
Thus we can restrict ourself to self-adjoint operators. Let $S\in M'$, it can be written as a combination of unitary operators:
\[ 
  S=\frac{ 1 }{2}\left( S+i\sqrt{1-S^2} \right)+\frac{ 1 }{2}\left( S-i\sqrt{1-S^2} \right).
\]
Notice that the square root makes sense because $1-S^2$ is positive. So $M'$ is spanned by $U(M')$ and then $M=M''=\big( U(M') \big)'$.
\end{proof}

%++++++++++++++++++++++++++++++++++++++++++++++++++++++++++++++++++++++++++++++++++++++++++++++++++++++++++++++++++++++++++++
\section{Examples of von Neumann algebras}

The very first example of von~Neumann algebra is $\oB(\hH)$ itself which is the commutant of the identity.

%----------------------------------------------------------------------------------------------------------------------------
\subsection{Algebra \texorpdfstring{$L^{\infty}(X)$}{LX}}


 For the second example, consider an $\sigma$-finite measure space $(X,\mu)$, and then define $\hH=L^2(X,\mu)$ and $M=L^{\infty}(X)$, the set of measurable bounded functions on $X$. The algebra $M$ acts on $\hH$ by pointwise multiplication and is therefore a $*$-subalgebra of $\oB(\hH)$. In order to prove that $ L^{\infty}(X)$ is a von~Neumann algebra, we prove that $ L^{\infty}(X)'= L^{\infty}(X)$.

Assume for simplicity that $\mu(X)<\infty$ for simplicity. Now consider $T\in L^{\infty}(X)'$ and $f_)(x)=1\in L^{\infty}(X)$ because the measure is finite. For each $g\in L^{\infty}(X)$, we have
\[ 
  Tg=T(gf_0)=gTf_0=gh
\]
where $h=Tf_0$. So an element of $ L^{\infty}(X)'$ reveals to be a multiplication by a function. We have
\[ 
  \| gh \|_2=\| Tg \|_2\leq \| T \|\cdot \| g \|_2
\]
That proves that $\| h \|_{\infty}\leq \| T \|$ because if $\| h \|_{\infty}=\| T \|+\delta$, then we have a set (of non vanishing measure) on which $h$ is bigger than $\| T \|$. We conclude that $h\in L^{\infty}(X)$.

%----------------------------------------------------------------------------------------------------------------------------
\subsection{Countable direct sum of Hilbert spaces}

Consider $\hH_{\infty}=\hH\oplus\hH\oplus\ldots=\{ f\colon \eN\to \hH\tq \sum \| f(n) \|^2<\infty \}$. One can show that it is an Hilbert space. Let
\[ 
  M=\{ T_{\infty}\tq T\in\oB(\hH) \}
\]
where $(T_{\infty}f)(n)=T\big( f(n) \big)$. One claims that this is a von~Neumann algebra.

Let us see that of a direct sum of two copies of $\hH$. In that case an element of $M$ reads 
$\begin{pmatrix}
T&0\\
0&T
\end{pmatrix}$ with $T\in\oB(\hH)$. An element of $M'$ must be of the form 
$\begin{pmatrix}
A&B\\
C&D
\end{pmatrix}$ with $[A,T]=[B,T]=[C,T]=[D,T]=0$ for every $T\in\oB(\hH)$. Thus we have
\begin{equation}
M'=\left\{ \begin{pmatrix}
A&B\\
C&D
\end{pmatrix}\tq A,B,C,D\in\eC\mtu \right\}.
\end{equation}
In turn, one can see that the commutant of the right hand side is $M$ itself.

Let us now go back with the case of $\hH_{\infty}$. The map $T\mapsto T_{\infty}$ provides an isomorphism $M\simeq\oB(\hH)$ as $*$-algebras. But it is not sufficient to conclude that $M$ is a von~Neumann algebra because the topologies do not correspond. A net $T_{\alpha}$ strongly converges in $M$ when $T_{\alpha}f$ converges for every $f\in M$. But $f$ is an infinite list of vectors in $\hH$, so that the strong topology in $M$ is something like an infinite collection of strong topology on $\oB(\hH)$. Every strongly open set in $\oB(\hH)$ is strongly open in $M$, while the reciprocal is not true.

%----------------------------------------------------------------------------------------------------------------------------
\subsection{Direct limit}

Let us take the example of direct limit of vector spaces given on page \pageref{PgExDirectLimVS}. We are considering the matrix algebras $M_k=\eM_{n_k}(\eC)$, and $A_n=M_1\otimes_{\eC}\ldots\otimes_{\eC} M_n$, together with the maps
\begin{equation}
\begin{aligned}
 \sigma_n\colon A_n&\to A_{n+1} \\ 
   T_1\otimes\ldots\otimes T_n&\mapsto T_1\otimes \ldots\otimes T_n\otimes \mtu. 
\end{aligned}
\end{equation}
We pose $A=\lim_{\rightarrow}A_n$.

\begin{lemma}
If $\pi_1$ and $\pi_2$ are $*$-homomorphisms from $A$ to $\oB(\hH)$, then
\begin{equation}		\label{Eqpiundeuxnirmea}
  \| \pi_1(a) \|=\| \pi_2(a) \|
\end{equation}
for every $a\in A$. In other words, there is an unique way to close $A$ in the norm topology.
\end{lemma}

\begin{probleme}
If one does not ask $\pi_i$ to be faithful, I can take $\pi_1(A)=0$ as counter-example. Thus I think that I have to add the faithful assumption.
\end{probleme}

\begin{proof}
One has $\| \pi_i(a) \|^2=\| \pi_i(a^*a) \|$, so that we only have to prove \eqref{Eqpiundeuxnirmea} in the case of self-adjoint elements of $A$. Using what is said in the proof of proposition \ref{prop:unicitenormcsa}, we have (with obvious notations) $\| \pi_i(b) \|=r_{\oB(\hH)}\big( \pi_i(b) \big)$. Since $b$ belongs to one of the $A_k=\eM_{n_k}(\eC)$ and $\dim A_k <\infty$.

\end{proof}

Now we ask the question of an actual way to represent $A$ on an Hilbert space. First, as matrix algebra, $M_k=\oB(\hH_k)$ for a certain finite dimensional Hilbert space $\hH_k$. We form the Hilbert space
\[ 
  \hH_1\otimes\ldots\otimes\hH_k
\]
with the inner product 
\begin{equation}
\langle v_1\otimes\ldots\otimes v_k, w_1\otimes\ldots\otimes w_k\rangle =\langle v_1, w_1\rangle \ldots\langle v_k, w_k\rangle.
\end{equation}
Now pick unit vectors $v_k\in\hH_k$ and define the maps
\begin{equation}
\begin{aligned}
 \hH_1\otimes\ldots\otimes\hH_{k-1}  &\to \hH_1\otimes\ldots\otimes\hH_k \\ 
   w_1\otimes\ldots\otimes w_{k-1}&\mapsto w_1\otimes\ldots\otimes w_{k-1}\otimes v_\otimes v_kk 
\end{aligned}
\end{equation}
that can be shown to be isometries. Finally, we consider the Hilbert space
\begin{equation}
	H =\bigotimes_1^{\infty}(\hH_k,v_k) = \text{completion of }\lim_{\rightarrow}(\hH_1\otimes\ldots\otimes\hH_k).
\end{equation}
Notice the dependence in the vectors $v_k$. Now we define the map $\pi\colon A\to \oB(H)$ by
\begin{equation}
\pi(T_1\otimes\ldots\otimes T_k)(w_1\otimes w_2\otimes\ldots)=T_1w_1\otimes\ldots\otimes T_kw_k\otimes w_k\otimes w_{k+1}\otimes\ldots
\end{equation}
More intrinsically, $\pi\colon \bigotimes M_k\to \oB\big( \bigotimes(\hH_k,v_k) \big)$,
\begin{equation}
\pi\big( \otimes T_k \big)(\otimes w_k)=\otimes T_kw_k.
\end{equation}
One can show that the strong closure of $\pi(A)$ is $\oB(H)$. The strong topology on $\oB(\hH)$ is generated by the open sets
\[ 
  \mU(S,v,\epsilon)=\{ T\in\oB(H)\tq \| Tv-Sv \|\leq\epsilon \}
\]
with $S\in\oB(\hH)$, $v\in H$ and $\epsilon>0$. The direct limit $\hH=\lim_{\rightarrow}(\hH_1\otimes\ldots\otimes\hH_k)$, is given by a vector space $\hH$ and maps $\varphi_k$ such that
\[ 
  \xymatrix{%
   \hH_1\otimes\ldots\otimes\hH_k \ar[rr]^{\sigma}\ar[dr]_{\varphi}		&	&	\hH_1\otimes\ldots\otimes\hH_{k+1}\ar[ld]^{\varphi_{k+1}}\\
   &	\hH
}
\]
commutes where $\sigma(w_1\otimes\ldots\otimes w_k)=w_1\otimes\ldots\otimes  w_k\otimes v_{k+1}$. The space $\hH$ is the free vector space generated by symbols of the form $w_1\otimes\ldots\otimes w_k\otimes v_{k+1}\otimes\ldots$ and we pose $\varphi_i(w_1\otimes\ldots\otimes w_i)=w_1\otimes\ldots\otimes w_i\otimes v_{i+1}\otimes\ldots$ and the Hilbert space $H$ is the completion of $\hH$.

The definition of $A$ proceeds in the same way: $M_i$ are matrix algebras and we pose $A_n=M_1 \otimes\ldots\otimes  M_n$ with the map $\sigma(T_1\otimes\ldots\otimes  T_n)=T_1\otimes\ldots\otimes  T_n\otimes\mtu$, and
\[ 
  \xymatrix{%
   A_n \ar[rr]^{\sigma}\ar[dr]_{\varphi_n}		&	&	A_{k+1}\ar[ld]^{\varphi_{k+1}}\\
   &							A
}
\]
with $\varphi_i(T_1\otimes\ldots\otimes  T_i)=T_1\otimes\ldots\otimes  T_i\otimes\mtu \otimes\ldots$, the space $A$ being the free vector space generated by the symbols $T_1\otimes\ldots\otimes  T_j\otimes\mtu\otimes\ldots$ 

Now an element of $H$ reads
\[ 
  \sum_{i=1}^{\infty}w_1^i\otimes\ldots\otimes  w_{k_i}^i\otimes v_{k_i+1}^i\otimes\ldots
\]
with $w_l^i\in\hH_l$, and we act on it by an element of $A$ by
\begin{equation}
\begin{split}
\pi(T_1\otimes\ldots\otimes T_k\otimes\cun\otimes\ldots)\big( \sum_{i=1}^{\infty} w_1^i\otimes&\ldots\otimes w_{k_i}^i\otimes v_{k_i+1}^i\otimes\ldots \big)\\
		&=	\sum_{i=1}^{\infty}T_1w_1^i\otimes\ldots\otimes T_{k_i}w_{k_i}^i\otimes v_{k_i+1}^i\otimes\ldots
\end{split}
\end{equation}
where some of the $T_j$ are subject to actually be $\mtu$. We are now going to prove that the strong closure of $\pi(A)$ is $\oB(H)$. Let $E$ be the set of finites sets of elements of the form $w_1\otimes\ldots\otimes  w_k\otimes v_{k+1}\otimes\ldots$ and the directed set $I=E\times ]0,\infty[$.

\begin{probleme}
I'm not sure of the next affirmation.
\end{probleme}

Let $T\in\oB(H)$. For each $(F,\epsilon)\in E$, there is a $T_{(F,\epsilon)}\in\pi(A)$
\begin{equation}
\| T_{(F,\epsilon)}X_k-TX_k \|<\epsilon
\end{equation}
for all $X_k\in F$. In that case, the limit of the net $(F,\epsilon)\to T_{(F,\epsilon)}$ is $T$, which shows that the strong closure of $\pi(A)$ is $\oB(H)$.

So that example does not provide new example of von~Neumann algebra.
\begin{proposition}
Let $V_k=\eC^{n_k}$ endowed with the standard representation of $\eM_{n_k}(\eC)$. Then the representation $\pi\colon \bigotimes_k \eM_{n_k}\to \oB\big( \bigotimes_k(V_k,v_k) \big)$ fulfils 
\[ 
  \big( \pi(M) \big)''=\oB\big( \bigotimes_k(V_k,v_k) \big)
\]
where $M=\bigotimes_k\eM_{n_k}$.
\end{proposition}

\begin{proof}
No proof.
\end{proof}

Notice that $\bigotimes_{1}^{\infty}\eM_{n_k}$ can differ from $\bigotimes_{1}^{\infty}\eM_{n'_k}$ when the $n_k$ and $n'_k$ do not agree. In fact, we have
\[ 
\bigotimes_{1}^{\infty}\eM_{n_k} =\bigotimes_{1}^{\infty}\eM_{n'_k}
\]
if and only if $\prod_k n_k=\prod_k n'_k$ in the sense of generalised products: each prime factor arise the same number of time in both side. For example, $\eM_2\otimes\eM_2\otimes\ldots=\eM_4\otimes\eM_4\otimes\ldots\neq \eM_3\otimes\eM_3\otimes\ldots$.

Take for example $V_k=\eC^2$ for every $k$ and then $v_k=v$, $v'_k=v'$. In this case, the representations $\pi$ and $\pi'$ are equivalent if and only if $v=\lambda v'$. This provides an Hilbert sphere of inequivalent irreducible representations.

%+++++++++++++++++++++++++++++++++++++++++++++++++++++++++++++++++++++++++++++++++++++++++++++++++++++++++++++++++++++++++++
\section{Continuous dimensions}

Let $\cA$ be a $*$-algebra with a faithful state $\varphi$ defining the inner product $\langle a, b\rangle =\varphi(a^*b)$. We define a representation $\rho$ of $A$ on itself by $\rho(a)b=ab$.

Let us take the situation and the notations of proposition \ref{Propstaretattraces}, and for $n=2$, consider the choice 
\[ 
  s_{\lambda}=
\begin{pmatrix}
\lambda/(1+\lambda)\\
		&		1/(1+\lambda)
\end{pmatrix}
\]
with $0\leq\lambda\leq 1$.

Let $M=\bigotimes_1^{\infty}\eM_2(\eC)$ and consider the positive continuous form $\varphi_{\lambda}$ on $M$ defined by
\begin{equation}
\varphi_{\lambda}(a_1\otimes a_2\otimes\ldots\otimes a_k\otimes\mtu\otimes\ldots)=\varphi_{\lambda}(a_1)\varphi_{\lambda}(a_2)\ldots=\tr(a_1s_{\lambda})\tr(a_2s_{\lambda})
\end{equation}
Notice that the product is finite because from a certain point, $a_i=\mtu$. It is on the other hand not difficult to see that the $\varphi_{\lambda}(a)=0$ only if $a_i=0$ for every $i$. The purpose now is to follow the GNS construction of proposition \ref{PropGNSanother}. The remark we just made says that there is no ideal to quotient with in order to have the Hilbert space of representation.

The representation of $\cA$ we get is the simple $\rho(a)b=ab$ of $\cA$ on its completion. Let $M_{\lambda}$ be the double commutant in this representation. 

\begin{theorem}[Powers]
The von~Neumann algebra $M_{\lambda}$ are all distinct for different values of $0\leq\lambda\leq 1$.
\end{theorem}
\begin{proof}
No proof.
\end{proof}

Let us see an example of that result. When $\lambda=0$, we get the standard representation of matrices on $\eC^n$, so that $M_0=\oB(\hH)$. When $\lambda=1$, we can show that 
\[ 
  \tr(T)=\langle V, TV\rangle 
\]
where $V=v\otimes v\otimes\ldots\otimes v\otimes\ldots$ is a trace on $M_1$ while there does not exist any trace on $\oB(\hH)$. We conclude that $M_1\neq M_0$.

%++++++++++++++++++++++++++++++++++++++++++++++++++++++++++++++++++++++++++++++++++++++++++++++++++++++++++++++++++++++++++++
\section{Cantor}

Let $M=\bigoplus_{k=1}^{\infty}M_k$ with $M_k=\eM_{n_k}(\eC)$ that can be seen as $\Span\{ T_1\otimes\ldots \}$ with $T_k=\mtu$ for sufficiently large $k$. Now take the case $n_k=2$ for all $k$. The algebra $M$ naturally acts on the space of locally constant functions on the Cantor set.

More generally, for arbitrary $n_k$, one can think of $M$ as an algebra of endomorphisms of the space of locally constant functions on $\prod\eZ_{n_k}$.
\begin{probleme}
Still to be developed.
\end{probleme}

%+++++++++++++++++++++++++++++++++++++++++++++++++++++++++++++++++++++++++++++++++++++++++++++++++++++++++++++++++++++++++++
					\section{More general state}
%+++++++++++++++++++++++++++++++++++++++++++++++++++++++++++++++++++++++++++++++++++++++++++++++++++++++++++++++++++++++++++

Let $\varphi_i$ be state on $M_i$. We define 
\begin{equation}
\begin{aligned}
 \varphi\colon M&\to \eC \\ 
   T_1\otimes T_2\otimes\ldots&\mapsto \varphi_1(T_1)\varphi_2(T_2)\ldots 
\end{aligned}
\end{equation}
where the product is finite because $T_k=\mtu$ for sufficiently large $k$. One can prove that $\varphi$ fulfills
\begin{enumerate}
\item $\varphi(\mtu)=1$,
\item $\varphi(T^*T)\geq 0$,
\item\label{enuitemvarpsdex} $\varphi(T^*S^*ST)=\| S \|^2\varphi(T^*T)$,
\end{enumerate}
so that $\varphi$ is in particular a state on $M$. The norm $\| S \|$ is the following. We know that $S\in\bigotimes_1^NM_k\subseteq M$ but, by construction, $\bigotimes_1^NM_k=\End(\eC^{n_1}\otimes\ldots\otimes \eC^{n_N})$. The norm of $S$ is taken as the operator norm in the sense of that endomorphism space.

The property \ref{enuitemvarpsdex} shows that the multiplication by $S$ is a bounded operator. We can build the GNS representation and define $M_{\varphi}$ to be $M''$ is that representation.

\begin{definition}
A \defe{factor of type $II_1$}{factor!of type $II_1$} is an infinite dimensional factor $M$ which accepts a non vanishing linear functional $\tr\colon M\to \eC$ such that
\begin{itemize}
\item $\tr(ST)=\tr(TS)$,
\item $\tr(T^*T)\geq 0$,
\item the function $\tr$ is continuous for the ultraweak topology.
\end{itemize}
\end{definition}


%+++++++++++++++++++++++++++++++++++++++++++++++++++++++++++++++++++++++++++++++++++++++++++++++++++++++++++++++++++++++++++
					\section{Group measure space construction}
%+++++++++++++++++++++++++++++++++++++++++++++++++++++++++++++++++++++++++++++++++++++++++++++++++++++++++++++++++++++++++++

Let $(X,\mu)$ be a measured space that we assume to be $\sigma$-finite, and $G$, a discrete countable group acting on $X$ in such a way that for every $g\in G$,
\begin{enumerate}
\item if $E\subset X$ is measurable then $gE$ is measurable,
\item if $\mu(E)=0$, then $\mu(gE)=0$.
\end{enumerate}
We do not impose the action to preserve the measure. As an example we take $G\subset\SL(2,\eR)=\left\{ \frac{ ax+b }{ cx+d } \right\}$ acting on $\eR\cup\{ \infty\}=\eR P^1$.

%---------------------------------------------------------------------------------------------------------------------------
					\subsection{First attempt}
%---------------------------------------------------------------------------------------------------------------------------

Take $L^2(X,\mu)$ as Hilbert space and, to $f\in  L^{\infty}(X,\mu)$, we associate the pointwise multiplication operator $M_f\colon  L^2(X,\mu)\to  L^2 (X,\mu)$. We also introduce the new measure $g\mu$ by
\begin{equation}
		(g\mu)(E)=\mu(g^{-1}E),
\end{equation}
and the action of $G$ on the functions by
\begin{equation}
		(gf)(x)=f(g^{-1}x).
\end{equation}
So we have
\begin{equation}
		\int_X (gf)(x)d(g\mu)(x)=\int_X f(x)d\mu(x).
\end{equation}

Notice that if $f\in L^{\infty}(X,\mu)$, we have $gf\in L^{\infty}(X,\mu)$, but when $f\in  L^2$, there are no guarantee that $gf\in  L^2$.

Let $\mu_1$ and $\mu_2$ be two measures on the set $X$. One says that $\mu_2$ is \defe{absolutely continuous}{absolutely continuous} with respect to $\mu_1$ if every $\mu_1$ null set is $\mu_2$ null.

\begin{theorem}[\href{http://en.wikipedia.org/wiki/Radon-Nikodym_theorem}{Radon-Nikod\'ym}]		\label{ThoRadonNikodym}\index{Radon-Nikod\`ym theorem}
Let $\mu_1$ and $\mu_2$ be two $\sigma$-finite measures on $X$. The measure $\mu_2$ is absolutely continuous with respect to $\mu_1$ if and only if there exists a measurable positive function $f$ such that $\mu_2=f\mu_1$.
\end{theorem}

The useful statement in our case is:
\begin{proposition}
There exists an unique function 
\[ 
	\frac{ d f\mu }{ d \mu }\colon X\to ]0,\infty[
\]
such that
\begin{equation}	\label{EqDefRadonNiko}
		\int (gf)(x)\left( \frac{ d g\mu }{ d \mu } \right)(x)d\mu(x)=\int (gf)(x)d(g\mu)(x)=\int f(x)d\mu(x).
\end{equation}
\end{proposition}
Applying $g^{-1}$ to the function $(gf)(x)\left( \frac{ d g\mu }{ d \mu } \right)(x)$ and applying the theorem, we see that for every function $f$, we have
\[ 
  \int f(x)\left( \frac{ d g\mu }{ d \mu } \right)(gx)\left( \frac{ d g^{-1}\mu }{ d \mu } \right)(x)=\int f(f)d\mu(x),
\]
so that
\begin{equation}
  \left( \frac{ d g\mu }{ d \mu } \right)(gx)\left( \frac{ d g^{-1}\mu }{ d \mu } \right)(x)=1.
\end{equation}

Using the function provided by that theorem, we define
\begin{equation}
\mU_gf=(gf)\cdot\left( \frac{ d g\mu }{ d \mu } \right)^{\frac{ 1 }{2}},
\end{equation}
so that $\| \mU_gf \|_{ L^2(X,\mu)}=\| f \|_{ L^2(X,\mu)}$. Moreover we have the following two important relations
\begin{enumerate}
\item $\mU_g\mU_h=\mU_{gh}$,
\item $\mU_g M_f \mU_g^{-1}=M_{gf}$
\end{enumerate}
where $M_f$ stands for the operator of pointwise product with $f$. The second relation implies $\mU_g M_f=M_{gf}\mU_g$, thus we have
\begin{equation}
M:=\{ M_f,\mU_g \}''=\{ \sum_{i=1}^n M_{f_i}\mU_{g_i} \}
\end{equation}
(because the double commutant is the strong closure) which is a $*$-algebra of operators. Notice that $\mU_g\mU_g^*=1$ because $\mU_g$ is an isometry.

We say that an action $G\times M\to M$ is \defe{ergodic}{ergodic} when $gf=f$ for every $g\in G$ implies that $f$ is constant almost everywhere.

\begin{lemma}		\label{LemergoBLCmu}
If the action $G\times X\to X$ is ergodic, then $M=\oB\big(  L^2(X,\mu) \big)$.
\end{lemma}

\begin{proof}
Let us first study the commutant $\{ M_f,\mU_g \}'$ which is of course contained in $\{ M_f \}'$. But we know that the commutant of $ L^{\infty}(X,\mu)$ is $ L^{\infty}(X,\mu)$, so that
\[ 
  \{ M_f \}'=\{ M_f\tq f\in L^{\infty}(X,\mu) \}.
\]
Is there an element in $ L^{\infty}(X,\mu)$ which commutes with all the elements of the form $\mU_g$, or in other words, is there a $f\in  L^{\infty}(X,\mu)$ such that $M_f=\mU_g M_f\mU_g^*=M_{gf}$ ? The only element $f$ such that $M_f=M_{gf}$ for every $g\in G$ is $f=1$, since the action is ergodic.
\end{proof}

That lemma shows that we didn't construct any interesting von~Neumann algebras in the ergodic case.

%---------------------------------------------------------------------------------------------------------------------------
					\subsection{Second attempt}
%---------------------------------------------------------------------------------------------------------------------------

Let $H= L^2(X\times G,\mu)$. Notice that $G\times X$ is nothing else than a countable number of copies of $X$, on which each of them we consider the measure $\mu$. The multiplication operator is now replaced by
\begin{equation}
\begin{aligned}
 M_f\colon H&\to H \\ 
   (M_f\varphi)(x,h)&=f(x)\varphi(x,h),
\end{aligned}
\end{equation}
and the unitary operator $\mU_g$ is replaced by
\begin{equation}
\begin{aligned}
 \mU_g\colon H&\to H \\ 
   (\mU_g\varphi)(x,h)&=\varphi(g^{-1}x,g^{-1} h)\left( \frac{ d g\mu }{ d \mu } \right)^{\frac{ 1 }{2}}(x)
\end{aligned}
\end{equation}
The introduction of the Radon-Nikod\'ym function serves to preserve the norm. The so defined operators have the following properties:
\begin{enumerate}
\item $\mU_g M_f\mU_g^*=M_{gf}$,
\item $\mU_g\mU_h=\mU_{gh}$,
\item $\mU_g^{-1}=\mU_{g^{-1}}=\mU_g^*$.
\end{enumerate}
Now we define
\begin{equation}
   M(G,X)=\{ \mU_g,M_f \}''=\{ \sum_{i=1}^nM_{f_i}\mU_{g_i} \},
\end{equation}
 and we will show (later) that
\begin{theorem}
If the action is ergodic, then $M(G,X)$ is a factor.
\end{theorem}
\begin{proof}
No proof up to now.
\end{proof}
We emphasize the progress: lemma \ref{LemergoBLCmu} says that the commutant is trivial while now the center only is trivial.

As an example, take $X=S^1\subset \eC$ and $G=\eZ$, the action being $g\cdot z= e^{2\pi i\theta g}z$ for some irrational $\theta$. One can prove, using Fourier transform, that this equation is ergodic.

Let us now define
\begin{equation}
\begin{aligned}
  \varphi\colon M(G,X)&\to \eC \\ 
   T&\mapsto \langle f_0, Tf_0\rangle  
\end{aligned}
\end{equation}
where $f_0\in  L^2(X\times G,\mu)$ is defined as follows
\[ 
  f_0(g,x)=
\begin{cases}
\mu(X)^{-2}		&\text{if $g=e$}\\
0			&\text{otherwise.}
\end{cases}
\]
We show that the so defined $\varphi$ is a trace over $M(G,X)$, i.e. it satisfies $\varphi(TS)=\varphi(ST)$ for every $S$, $T\in M(G,X)$. We know that $M(G,X)$ is generated by expressions of the form $L_f\mU_g$. When $g\neq e$, the functions $f_0$ and $\mU_gf_0$ have disjoint support, so that $\varphi(M_f\mU_g)=0$. If $g=e$, the computation is easy and we finally find
\begin{equation}
  \varphi(M_f\mU_g)=
\begin{cases}
\frac{1}{ \mu(X) }\int_X f(x)d\mu(x)		&\text{if $g=e$}\\
0						&\text{if $g\neq e$}.
\end{cases}
\end{equation}
It is easy to check that this expression is a trace on $\{ \sum_iM_{f_i}\mU_{g_i} \}$. Since $\oB(H)$ has no trace, we know that $M(G,X)$ is a non trivial von~Neumann algebra. Stated in a different way, what we just proved is that, provided that $\mu$ is a $G$-invariant \defe{probability measure}{probability!measure} (i.e. $\mu(X)=1$), the formula
\begin{equation} 		\label{Eqvpmupickid}
  \varphi_{\mu}\big( \sum_{h\in G} M_{f_h}\mU_h \big)=\int_X f_e(x)d\mu(x).
\end{equation}
extends to a trace state on $M(G,X)$.

\begin{proposition}
If $\mu$ is not invariant (but still $\mu(X)=1$), then $\varphi_{\mu}$ is still a state, but no more a trace.
\end{proposition}

\begin{proof}
By definition, $\varphi_{\mu}(T)=\langle f_0, Tf_0\rangle $ where $f_0(e)=1$ and $f_0(g)=0$ otherwise. We have
\begin{align*}
\varphi_{\mu}\Big( \big( \sum M_{f_h}\mU_h \big)^*\big( \sum M_{f_h}\mU_h \big) \Big)&=\varphi_{\mu}\Big( \sum_{h_1,h_2}\mU_{h_1}^*M_{f_{h_1}}^*M_{f_{h_2}} \mU_{h_2} \Big)\\
		&=\varphi_{\mu}\Big( \sum_{h_1,h_2}  M_{f_{h_1}}^*M_{f_{h_2}}\mU_{h_1^{-1}}\mU_{h_2} \Big)
\end{align*}
because $\mU_g^*=\mU_g^{-1}=\mU_{g^{-1}}$ commutes with the $M_f$. Now, according to the expression \eqref{Eqvpmupickid}, the function $\varphi_{\mu}$ pick up the identity component and integrates. So we have
\begin{align*}
\varphi_{\mu}\Big( \big( \sum M_{f_h}\mU_h \big)^*\big( \sum M_{f_h}\mU_h \big) \Big)&=\sum_{h}\int_X h^{-1}(f^*_hf_h)d\mu\\
							&=\sum+h\int_X| f_h |^2(h^{-1}x)d\mu(x)\geq0.
\end{align*}

\end{proof}

%---------------------------------------------------------------------------------------------------------------------------
					\subsection{First generalisation}
%---------------------------------------------------------------------------------------------------------------------------

Let us replace the space $ L^{\infty}(X,\mu)$ by any von~Neumann algebra $N$ of operators on the Hilbert space $\hH$. As Hilbert $H$ space we take the completion of $\hH\times l^2(G)$ and we assume to have an action of $G$ over $N$. We know that the metric of a $C^*$-algebra is determined by its algebra structure, so that the action must be isometric. We assume the action to be \emph{via} strongly continuous automorphisms. We have $H\otimes l^2(G)=l^2(G,H)$. For $t\in N$, we define 
\[ 
  (M_T\varphi)(g)=g^{-1}(T)\varphi(g)
\]
where $\varphi\in l^2(GmH)$. We also define
\[ 
  (\mU_h\varphi)(g)=\varphi(h^{-1}g).
\]
Notice that the group does not act on $H$ while in the previous constructions, it did act on $X$. We still have the relation
\begin{equation}
\mU_hM_T\mU_h^*=M_{hT}.
\end{equation}
Now, we form the von~Neumann algebra
\begin{equation}
M(G,N)=\{ \mU_h,M_T \}''.
\end{equation}

%---------------------------------------------------------------------------------------------------------------------------
					\subsection{Second generalisation}
%---------------------------------------------------------------------------------------------------------------------------

Now we replace the discrete group $G$ by a second countable locally compact topological group. Typical examples are $(\eR,+)$ or the group ``$ax+b$'' generated by matrices of the form $\begin{pmatrix}
a&b\\0&1
\end{pmatrix}$ with $a>0$ and $b\in \eR$, acting on the real line by affine transformations. Each such group has an unique (up to constant multiple) Borel measure $m$ such that each compact set has finite measure and which is in the same time the Haar measure: for every compactly supported functions on $G$,
\[ 
  \int_G f(hg)dm(g)=\int_G f(g)dm(g).
\]
The measure on ``$ax+b$'' is $dm=\frac{1}{ a^2 }da\,db$.

We consider on $G\times N$ the product topology and we assume the action $G\times N\to N$ to be strongly continuous. Now, we proceed as before: we take the Hilbert space $H$ the completion of
\[ 
  \hH\otimes  L^2(G,m)=L^2(G,\hH),
\]
and $M_T$ is defined by the same formulas as before. The von~Neumann algebra that we obtain is denoted by $M(G,N)$.

%---------------------------------------------------------------------------------------------------------------------------
					\subsection{One particular case}
%---------------------------------------------------------------------------------------------------------------------------
\label{sssOnePartCaseMG}

Take a discrete group $G$, so that $M(G)$ is generated by the operators $\mU_g\in\oB\big( l^2(G) \big)$ who are defined by
\begin{equation}
		(\mU_h\varphi)(g)=\varphi(h^{-1}g).
\end{equation}
Consider the function
\begin{equation}
		f_e(g)=
\begin{cases}
1		&\text{if $g=e$}\\
0		&\text{otherwise.}
\end{cases}
\end{equation}
We have 
\[ 
  (\mU_hf_e)(g):=f_h(g)=
\begin{cases}
1		&\text{if $g=h$}\\
0		&\text{otherwise,}
\end{cases}
\]
so that $\{ f_h \}$ is an orthonormal basis of $l^2(G)$ and $\mU_{h_1}f_{h_2}=f_{h_1h_2}$. We conclude that $\mU_h$ is a permutation of the basis vectors. 

\begin{proposition}		\label{ProplDeuxGFGP}
The module $l^2(G)$ over $M(G)$ is projective and finitely generated.
\end{proposition}

\begin{proof}
The fact that $l^2(G)$ as module over $M(G)$ is finitely generated comes from the fact that $f_e$ by itself generates the basis $\{ f_h \}$ as we just said. 

In order to prove that the module is projective, we will prove the condition \ref{ItemTroisCarecterisationProjectif} of proposition \ref{PropEquivProjModule}. Let $\modM$ be a $M(G)$-module and $\rho\colon \modM\to l^2(G)$ be a surjective module map. Consider any $\xi\in\modM$ such that $\rho(\xi)=f_e$, define $s(f_e)=\xi$ and extend by linearity and action of $M(G)$. The so defined map $s$ obviously fulfils $\rho\circ s=\id|_{l^2(G)}$.
\end{proof}

\begin{probleme}
That statement and the proof are correct uhm ? I use them on page \pageref{PglDeuxGFGPutiliseIci}.
\end{probleme}

For each $S\in M(G)'$, we define $f_S=S(f_e)\in l^2(G)$. Now if $S\in M(G)\cap M(G)'$, we have
\begin{enumerate}
\item $S\mU_h=\mU_hS$,
\item $S\mU_hf_e=\mU_hS f_e$.
\end{enumerate}
The operators $W_k$ defined by
\[ 
  (W_kf)(g)=f(gk)
\]
commute with $\mU_h$. So they commute with $M(G)$ and with $S$. Therefore we have
\[ 
  W_kSf_h=SW_kf_h=Sf_{hk^{-1}},
\]
and taking $h=k$, we find
\begin{equation}
  f_S(hg)=f_S(gh)
\end{equation}
for every $g$, $h\in G$. That means that $f_S$ is constant on the conjugacy classes because $g$ and $hgh^{-1}$ are two elements of the form $g_1g_2$ and $g_2g_1$ with $g_1=h$ and $g_2=gh^{-1}$.

If $G$ has no finite conjugacy class (except $\{ e \}$), then $f_S$ has to be a multiple of $f_e$ because its norm would contains infinitely many constant non zero terms. But when $S_1$ and $S_2$ belongs to $M'(G)$ with $f_{S_1}=f_{S_2}$, then $S_1=S_2$ because $S_1f_h=S_1\mU_hf_e=\mU_1S_1f_e=\mU_hf_{S_1}$, while the same computation with $S_2$ gives $S_2 f_h=\mU_hf_{S_2}$. So $S_1$ and $S_2$ agree on a basis.

For this reason, if we assume that $G$ has infinite conjugacy class, we have $M(G)'\cap M(G)=\eC\mtu$. The most famous example of such a group is $G=F_k$, the group of formal words of $1\ldots k$ with $k>1$.

Now fix an element of finite order $h\in G$, and the natural homomorphism $\eZ\to G$ given by $n\mapsto h^n$. The group $G$ has a natural equivalence relation $g_1\sim g_2$ if and only if there exists a $n\in\eZ$ such that $g_1=g_2^n$. We denote by $G/\eZ$ the quotient of $G$ by this relation. One class of this space is
\[ 
  \eZ_{g}=\{ g^n\tq n\in\eZ \},
\]
and we have
\begin{equation}
l^2(G)=\bigoplus_{g\in G/\eZ}l^2(\eZ_g).
\end{equation}
The operator $\mU_g$ acts on $l^3(\eZ_g)$ by translation: $\mU_gg^n=g^{n+1}$. Now we fix a $g\in G$ and we make the identifications
\begin{equation}
l^2(\eZ_g)\simeq l^2(\eZ)\simeq L^2(S^1),
\end{equation}
the latter identification being given by the Fourier series of a function on the circle (seen as a periodic function on $\eR$). In that framework, $\mU_g$ acts by translation of $1$ on $l^2(\eZ)$:
\[ 
  \mU_g(x_i)_{i\in\eN}=(x_i+1)_{i\in\eN}.
\]
A function $f\in C(S^1)$ is bounded (since $S^1$ is compact), to that if $s\in L^2(S^1)$, the product function $(fs)$ still belongs to $L^1(S^1)$ and then corresponds to an element of $l^2(S^1)$. Thus we have a map
\[ 
  C(S^1)\to\oB\big( l^2(\eZ_g) \big).
\]
If one acts separately on each of the ``fixed'' $g$, we obtain an action
\[ 
  C(S^1)\to\oB\big( \bigoplus_{g\in G/\eZ}l^2(\eZ_g) \big).
\]
We define the operator $\mU\in \bigoplus_{g\in G/\eZ}l^2(\eZ_g)  $ as acting on $l^2(\eZ_g)$ with $\mU_g$. That allows us to consider
\begin{equation}
\begin{aligned}
 C(S^1)&\to M(G) \\ 
   z^n&\mapsto \mU^n 
\end{aligned}
\end{equation}
that can be composed with the trace $\varphi$ to give
\[ 
  f\mapsto\int_{S^1} f(z)dm(z)
\]
for each $f\in C(S^1)$.


%+++++++++++++++++++++++++++++++++++++++++++++++++++++++++++++++++++++++++++++++++++++++++++++++++++++++++++++++++++++++++++
					\section{More about projections}
%+++++++++++++++++++++++++++++++++++++++++++++++++++++++++++++++++++++++++++++++++++++++++++++++++++++++++++++++++++++++++++

Let $M$ be a von~Neumann algebra and $p$, $q$, two projections in $\oB(\hH)$. We define\nomenclature[O]{$p\vee q$}{The smallest projection bigger than $p$ and $q$}\nomenclature[O]{$p\wedge q$}{The biggest projection smaller than $p$ and $q$}
\begin{align}
p\vee q		&=\text{projection onto $\overline{ \Image(p)+\Image(q) }$}\\
p\wedge q	&=\text{projection onto $\Image(p)\cap\Image(q)$}
\end{align}
A projection is determined by the closed space of its range, so one can order the projection by ordering the closed subspace of $\hH$.

\begin{lemma}		\label{LemDimSupDeuxProjs}
We have
\begin{equation}
	\dim(R_1\vee R_2)=\dim R_1+\dim R_2-\dim(R_1\wedge R_2)
\end{equation}
for any two projections $R_1$ and $R_2$ in $M$.
\end{lemma}

\begin{proof}
Left as an exercise. 
\end{proof}

For two projections $p$ and $q$, we write that $p\preceq q$\nomenclature[C]{$p\preceq q$}{A partial ordering notion over the projections of a von~Neumann algebra} if there exists a partial isometry $V$ such that
\begin{enumerate}
\item $V^*V=p$,
\item $VV^*\leq q$.
\end{enumerate}
One proves that this determines a partial ordering on the projections.

\begin{proposition}
If $p$ and $q$ belong to $M$, then the projections $p\vee q$ and $p\wedge q$ belong to $M$ too.
\end{proposition}

\begin{proof}
There exists an unitary representation $\pi\colon G\to \End(\hH)$ whose $M$ is the commutant: $M=\pi(G)'$ by lemma \ref{LemVNCommunit}. Since $p$ is a projection, proposition \ref{PropprojrepresVN} says that $p\hH$ is an invariant subspace of $\pi(G)$. The same being true for $q$, we have that the spaces $\overline{ \Image(p)+\Image(q) }$ and $\Image(p)\cap\Image(q)$ are invariant too. Now the proposition \ref{PropprojrepresVN} assures that the corresponding projections (namely $p\vee q$ and $p\wedge q$) are part of $M$.
\end{proof}

Before to define the infinite case $p_1\vee p_2\vee\ldots$, we need a lemma.

\begin{lemma}
If $\{ T_{\alpha} \}$ is a net of selfadjoint bounded operators such that
\begin{enumerate}
\item $\sup_{\alpha}\| T_{\alpha} \|<\infty$,
\item if $\alpha >\beta$, then $\langle v, T_{\alpha} v\rangle >\langle t, T_{\beta} v\rangle $ for every $v\in\hH$.
\end{enumerate}
Then $\{ T_{\alpha} \}$ strongly converges to an operator $T$, see condition \eqref{EqDEflimforte}.
\end{lemma}

\begin{proof}
We want the limit $T$ to fulfil $\langle v, Tv\rangle =\lim_{\alpha\to\infty}\langle v, T_{\alpha} v\rangle$. From weak-compactness of the unit ball, that formula defines the quadratic form $Q(v)=\lim_{\alpha\to\infty}\langle v, T_{\alpha} v\rangle $ with $\| v \|=1$. The form $Q$ is then defined on the whole space $\hH$ by homogeneity: $Q(\lambda v)=\lambda^2 Q(v)$. The polarization identity\index{polarization!identity}
\[ 
  4\real\langle v, T_{\alpha} w\rangle =\langle (v+w), T_{\alpha}(v+w)\rangle -\langle (v-w), T_{\alpha}(v-w)\rangle 
\]
defines $Q(v,w)$ and then defines the value of $\langle v, Tw\rangle $ for every $v$ and $w$. This is the candidate to be the strong limit of $T_{\alpha}$. The question is now to know if this is an actual strong limit.

Since the net is increasing, we have $\langle v, (T-T_{\alpha})v\rangle \geq0$, so that $T-T_{\alpha}$ is positive which implies that $(T-T_{\alpha})^{1/2}$ is well defined. We have
\[ 
  \| (T-T_{\alpha})^{1/2}v \|=\langle v, (T-T_{\alpha})v\rangle \to 0,
\]
so that the strong limit of $(T-T_{\alpha})^{1/2}$ is zero. 

It is not true in general that, in the strong topology, $a_{\alpha}\to a$ and $b_{\alpha}\to b$ imply $a_{\alpha} b_{\alpha}\to ab$. But it is true when $a_{\alpha}$ and $b_{\alpha}$ are contained in a ball. Since $\sup \| T-T_{\alpha} \|^{1/2}<\infty$, we can thus make
\[ 
  \slim(T-T_{\alpha})=\slim\big( (T-T_{\alpha})^{1/2} \big)^2=0.\quad \slim_{a\to 0}f(a)
\]
\end{proof}
In short, that lemma claims that $M$ contains the limits of all bounded increasing nets of selfadjoint operators. In particular, for a net of projections $P_{\alpha}\in M$ with $\alpha\in X$,
\begin{align*}
	\bigvee_{\alpha}P_{\alpha}&\in M,&
	\Wedge_{\alpha}P_{\alpha}&\in M.
\end{align*}

\begin{proposition}
Let $\{ P_{\alpha} \}$ and $\{ Q_{\alpha} \}$ to be two nets of projectors on the same directed set, and suppose that we have operators $T_{\alpha}\colon \Image(P_{\alpha})\to \Image(Q_{\alpha})$ such that
\begin{enumerate}
\item $\sup_{\alpha}\| T_{\alpha} \|<\infty$,
\item $\left.T_{\alpha}\right|_{\Image(P_{\beta})}=T_{\beta}$,
\end{enumerate}
then there exists an operator $T\colon \Image\big(\bigvee_{\alpha} P_{\alpha}\big)\to \Image\big(\bigvee_{\alpha}Q_{\alpha}\big)$  with
\[ 
  T|_{\Image(P_{\alpha})}=T_{\alpha}.
\]
If moreover $P_{\alpha}$, $Q_{\alpha}$ and $T_{\alpha}$ belong to $M$, then $T\in M$.
\end{proposition}

\begin{proof}
No proof.
\end{proof}

\begin{probleme}
I think that $p_1\vee p_2\vee \ldots$ is now well defined when $p_i$ are projections. Indeed, the projections $P_k=p_1\vee \ldots p_k$ form a net and, by the proposition, the limit belongs to $M$. More generally, if $A$ is a set of projections, for $\bigvee_{p\in A}p$, one considers the set of parts of $A$ (which is partially ordered) and then the net $p_{\alpha}=\bigvee_{p\in\alpha}p$ where $\alpha$ is a part of $A$.

Now, one has to understand why the result is still a projection.
\end{probleme}


\begin{lemma}		\label{LemPTQnnzero}
Let $M$ be a factor. If $P$ and $Q$ are nonzero projections in $M$, then there is a $T\in M$ such that $PTQ\neq 0$.
\end{lemma}

\begin{proof}
Let $U(M)$ be the unitary group of $M$. We saw that if $T$ commutes with $U(M)$, then it commutes with all $M$. Assume that $PTQ=0$ for every $T\in M$, then in particular it holds for $T=U\in U(M)$ and we have $UPU^*Q=0$.

Consider the operator $R=\bigvee_{U\in U(M)}UPU^*$ which is a projection in $M$. Since the set $\{ UPU^*\tq U\in U(M) \}$ is invariant under the adjoint action of $U(M)$, we have $URU^*=R$, or $UR=RU$ for every $U\in U(M)$. That proves that $R\in M'$. So $R\in M\cap M'$ and is thus a multiple of identity by the fact that $M$ is a factor: $R=\id$ (the multiple has to be $1$ because $R$ is a projection).

We said, on the other hand, that $UPU^*Q=0$, which means that the range of $UPU^*$ is orthogonal to the one of $Q$ for every $U\in U(M)$, so that the range of $R$ has to be orthogonal too. That contradicts the fact that $R$ is a multiple of identity.
\end{proof}

\begin{lemma}		\label{LemVVPVVQfactreu}
Let $M$ be a factor and $P$, $Q$ two nonzero projections in $M$, then there exists a nonzero partial isometry $V\in M$ such that $V^*V\leq P$ and $VV^*\leq Q$.
\end{lemma}

\begin{proof}
Let $T$ be such that $PTQ\neq 0$ (by lemma \ref{LemPTQnnzero}), and $V$ be the partial isometry part of $QTP$. The image of $V^*V$ is at most the one of $QTP$ which is smaller (or equal) to the image of $Q$, so $V^*V\leq Q$. For the same reason, the image of $VV^*$ is contained in $P^*T^*Q^*\hH=PY^*Q\hH\subseteq\Image(P)$. Then $VV^*\leq P$.
\end{proof}

\begin{proposition}
Let $M$ be a factor and $P$, $Q$ two nonzero projections in $M$, then there exists a nonzero partial isometry $V\in M$ such that 
\begin{enumerate}
\item $V^*V\leq P$,
\item $VV^*\leq Q$
\end{enumerate}
and either $V^*V=P$, or $VV^*=Q$ or both.
\end{proposition}

\begin{proof}
Let $X$ be the set of partial isometries $V$ such that $V^*V\leq P$ and $VV^*\leq Q$. We write $P_V=V^*V$ and $Q_V=VV^*$. We define a partial ordering on $X$ by $V_1\leq V_2$ if $P_{V_1}\leq P_{V_2}$, $Q_{V_1}\leq Q_{V_2}$ and $V_2P_{V_1}=V_1$. In that case, $V_2$ is some kind of extension of $V_1$.

If one considers an increasing sequence $V_1\leq V_2\leq\ldots$, the Zorn's lemma assures the existence of a $V$ bigger than all the elements of the sequence. We claim that this $V$ is the solution of the proposition. Indeed, suppose $V^*V\neq P$ and $VV^*\neq Q$. Then the operators $P'=P-V^*V$ and $Q'=Q-VV^*$ are nonzero projections. Now, lemma \ref{LemVVPVVQfactreu} provides a partial isometry which contradicts maximality of $V$.
\end{proof}

The \defe{dimension}{dimension!of a von~Neumann algebra} of the von~Neumann algebra $M$ is the set of equivalence class of projections in $M$. This is a linearly ordered set.

Let $M=M(G)$ where $G$ is a group with only infinite conjugacy classes (but the one of identity). We saw that there is a trace $\tau\colon M\to \eC$, and that $\forall t\in[0,1]$, there exists a projection whose trace is $t$.

\begin{proposition}
If $T\geq 0$ is an element of $M$ such that $\tau(T)=0$, then $T=0$.
\end{proposition}

\begin{proof}
We know the injection $M\to l^2(G)$ defined by $T\mapsto f_T=Tf_e$. Since the right translation commutes with every $T_1$ (because $T_1(gf_e)=g(T_1f_e)$), we have $T_1f_g=0$ for every $g$ if $T_1f_e=0$. On the other hand,
\[ 
  \tau(T_1^*T_1):=\langle f_e, T_1^*T_1f_e\rangle =\| Tf_e \|^2
\]
which is positive. So if $\tau(T_1^*T_1)=0$, then $T_1f_e=0$, and so $T_1f_g=0$ which proves that $T_1=0$. This concludes the proof that $T\geq 0$ and $\tau(T)=0$ imply $T=0$ because every positive element reads as a product $T_1^*T_1$. 
\end{proof}

Now let define $\tau'\colon \dim M\to [0,1]$ by
\begin{equation}
	\tau'\big( [P] \big)=\tau(P).
\end{equation}
It is well defined because if $P\sim Q$, then we have a $u$ such that $P=uu^*$ and $Q=u^*u$, so that $\tau(P)=\tau(uu^*)=\tau(u^*u)=\tau(Q)$ by cyclic invariance of the trace. The map $\tau'$ preserves the order because $P-Q$ is positive when $P\geq Q$.

\begin{proposition}
The map $\tau'$ is a bijection
\end{proposition}

\begin{proof}
What we have to prove is that $P\sim Q$ if $\tau(P)=\tau(Q)$. Let $V^*V=P$ and $VV^*\leq Q$ and compute
\[ 
  \tau(Q-VV^*)=\tau(Q)-\tau(VV^*)=\tau(Q)-\tau(V^*V)=\tau(Q)-\tau(P)=0.
\]
The fact that $\tau(Q-VV^*)=0$ implies $Q-VV^*=0$, which in turn proves that $V$ implements the equivalence $P\sim Q$.
\end{proof}

%---------------------------------------------------------------------------------------------------------------------------
					\subsection{Comparison of projections}
%---------------------------------------------------------------------------------------------------------------------------

In this section we follow \cite{Wassermann}. Let $M$ be a von~Neumann algebra and $p$, $q$, two projectors. \defe{Equivalence of projectors}{equivalence of projectors}\label{PgEaivVNMurray}  is given by $p\sim q$\nomenclature[C]{$p\sim q$}{Equivalence of projectors} if there exists a partial isometry $u\in M$ such that $u^*u=p$ and $uu^*=q$. If $p$ is equivalent to a subprojection of $q$, we write $p\prec q$.

\begin{proposition}
If $p=\oplus_i p_i$ and $q=\oplus_iq_i$ are orthogonal sums of projectors with $p_i\sim q_i$, then $p\sim q_i$. If moreover $p_i\prec q_i$ for every $i$, then $p\prec q$.
\end{proposition}
\begin{proof}
No proof.
\end{proof}

Let $T\in\oB(\hH)$. The closed spaces $\overline{ \Image(T) }$ and $\overline{ \Image(T^*) }$ are the \defe{left support}{left!support} and \defe{right support}{right!support} of $T$.

\begin{theorem}
Let $T$ be an invertible operator, then it reads under the form
\[ 
  T=| T |U
\]
where $U$ is unitary. This decomposition is the unique one of the form $T=SU$ with $S$ positive and $U$ unitary.
\end{theorem}

\begin{proof}
Let $T=PU$, then $TT^*=PUU^*P=P^2$, then $P=\sqrt{TT^*}$ and then $U=P^{-1}T$. That proves unicity of the decomposition. For existence, we pose $U=(TT^*)^{-1/2}T$ and $P=(TT^*)^{1/2}$. Then we check
\begin{equation}
	UU^*=(TT^*)^{-1/2}TT^*(TT^*)^{-1/2}=\mtu.
\end{equation}
 Now, the operator $U$ is the product of two invertible operators, so that it is invertible. Thus the fact that $UU^*=\mtu$ forces $U^*U=\mtu$.
\end{proof}

When $T$ is not invertible, we have a weaker result of decomposition.
\begin{proposition}		\label{PropdecmbTbV}
Every $T\in\oB(M)$ can be written under the form
\[ 
  	T=| T |V
\]
 where $V\colon \overline{ \Image(T^*) }\to \overline{ \Image(| T |) }$ is a partial isometry whose final projection is the support of $| T |=\sqrt{TT^*}$.
\end{proposition}

\begin{proposition}
The left and right support of any operator in $\oB(\hH)$ are equivalent projections.
\end{proposition}

\begin{proof}
Let $T=| T |V$ be the decomposition of $T$ by proposition \ref{PropdecmbTbV}. By definition, $e=V^*V$ is a projection and we have $(1-e)V^*V(1-e)=0$ as can be checked by developing the expression and using $e^2=e$. Since $e=e^*$, the latter equation rewrites
\[ 
  \big( V(1-e) \big)^*\big( V(1-e) \big)=0,
\]
so that $V(1-e)=0$ and $Ve=V$ (we used lemma \ref{LemTTzepoT}). Now look at $f=VV^*$, we have
\[ 
  fV=VV^*V=Ve=V,
\]
so $f$ is a projection such that $fV=V$. We deduce that $V\colon e\hH\to f\hH$ is an isometry. The projectors $e$ and $f$ are the \defe{initial}{initial projection} and \defe{final}{final projection} \defe{projections}{projection!initial and final} of $V$.
\end{proof}
\begin{probleme}
This proof is not finished.
\end{probleme}

Using lemma \ref{LemkerTkersqrtT}, we have in particular
\begin{equation}
\overline{ \Image(T^*) }=(\ker T)^{\perp}=\big( \ker(T^*T)^{1/2} \big)^{\perp}=\Image(T^*T)^{1/2}.
\end{equation}

%+++++++++++++++++++++++++++++++++++++++++++++++++++++++++++++++++++++++++++++++++++++++++++++++++++++++++++++++++++++++++++
					\section{Type I factor and factorization}
%+++++++++++++++++++++++++++++++++++++++++++++++++++++++++++++++++++++++++++++++++++++++++++++++++++++++++++++++++++++++++++

\begin{lemma}		\label{LemPMPPMPprime}
Let $M$ be a von~Neumann algebra and $P\in M$, a projection. Then
\begin{equation}
(PM'P)'=PMP,
\end{equation}
and 
\begin{equation}		\label{EqLemPMPPMPprimedeux}
(PMP)'=PM'P
\end{equation}
where we see $PMP$ as an algebra of operators on $P\hH\subseteq\hH$. 
\end{lemma}

Let us give an intuitive argument with matrices before to give a proof. For the first claim, consider $M$ as the set of two by two matrices 
$
\begin{pmatrix}
a&b\\
c&d
\end{pmatrix}
$, and $P=
\begin{pmatrix}
1&0\\
0&1
\end{pmatrix}$. In that case, $PMP=\begin{pmatrix}
a&0\\
0&0
\end{pmatrix}$ while $M'=\{ \begin{pmatrix}
\alpha&0\\
0&\beta
\end{pmatrix}\}$ with $[\alpha,a]=0$. Thus we have $PM'P=\begin{pmatrix}
\alpha&0\\
)&0
\end{pmatrix}$.

For the second claim, suppose $Q\in(PNP)'$. We are looking for a $R\in M'$ such that $Q=PRP$. Notice that $Q$ itself does not belong to $M'$ in general. Take for example
\begin{align*}
M&=\begin{pmatrix}
\star&0&\star\\
0&\star&0\\
\star&0&\star\\
\end{pmatrix},		
&
P&=\begin{pmatrix}
1\\
&1\\
&&1
\end{pmatrix},
&
Q&=\begin{pmatrix}
1\\
&0\\
&&0
\end{pmatrix}.
\end{align*}
we have $M'=\begin{pmatrix}
a\\
&b\\
&&a
\end{pmatrix}$, so that $Q$ does not lie in $M'$, but
\[ 
  Q=P\begin{pmatrix}
1\\
&0\\
&&1
\end{pmatrix}P
\]
anyway. We see on that example that the matrix $R$ to be chosen is bigger than $Q$.
	
\begin{proof}[Proof of lemma \ref{LemPMPPMPprime}]
Let $Q\in (PMP)'$ and $R$ be the projection onto the closure of $MQ\hH$. The latter space being invariant under $M$, the projection $R$ belong to $M'$ by lemma \ref{LeminvarMprime}. We are now going to prove that $Q=PRP$. Since $P=P^2$ commutes with $R$, the operator $PRP$ is the projection onto the intersection of the target spaces of $R$ and $P$, because $PRP=PR=RP$. Therefore $Q$ is smaller than $P$, so that $MQ\hH=MPQ\hH$. That proves that $PRP$ is the projection on
\[ 
  PMQ\hH=PMPQ\hH=QPMP\hH=QP\hH=Q\hH
\]
where we used the fact that $Q\in(PMP)'$. That concludes the proof that $Q=PRP$.
\end{proof}

\begin{lemma}		\label{LemMapipsomPMmP}
If $M$ is a factor and $P$, a nonzero projection in $M$, then the map
\begin{equation}
\begin{aligned}
 M'&\to PM'P \\ 
   S&\mapsto PSP 
\end{aligned}
\end{equation}
is a $*$-algebra isomorphism.
\end{lemma}
 
\begin{proof}
The fact that the map is multiplicative and surjective is clear. The only point to prove is injectivity. So we will prove that $S=0$ under the assumption $SP=0$. We have $0=MSP=SMP$, so that $S$ vanishes on $MP\hH$.

The latter subspace is obviously invariant under $M$, but also under $M'$ because for every $S_1\in M'$, we have $S_1MP\hH=MPS_1\hH\subseteq MP\hH$. We conclude that the projection onto $MP\hH$ lies in $M'\cap M$. Hence that projection must be a multiple of the identity or zero. Since that space contains at least $P\hH$, the zero possibility is ruled out. Thus the projection onto $MP\hH$ is the identity and $MP\hH=\hH$. Now, $SMP=0$ implies $S=0$ which concludes the proof of the lemma.
\end{proof}

A projection $P\in M$ is a \defe{minimal projection}{minimal!projection}\index{projection!minimal} $P$ if if $0<Q\leq P$ implies $Q=P$. In other words, the projection $P$ is minimal if $PMP=\eC P$ because the projection onto $PMP$ is of course smaller or equal to $P$. A minimal projection is always finite, indeed, a projection $Q$ such that $Q\sim P$ and $Q\leq P$ (which exists when $P$ is infinite) contradicts minimality of $P$. 

\begin{definition}
Let $M$ be a factor. It is
\begin{itemize}
\item of \defe{type $I$}{type $I$ factor}\index{factor!of type $I$} if $M$ contains a non vanishing minimal projection,
\item of \defe{type $II$}{type $II$ factor}\index{factor!of  type $II$} if $M$ contains a non vanishing finite projection (and is not of type I)
\item of \defe{type $III$}{type $III$ factor}\index{factor!of type $III$} if no projection in $M$ is finite (but the vanishing one).
\end{itemize}
\end{definition}

A type $II$ factor is of \defe{type $II_1$}{factor!of type $II_1$} if it is finite and of \defe{type $II_{\infty}$}{factor!of type $II_{\infty}$} if it is not finite.

Let us now examine what are the type I factors.\label{PgtypeIonavu} Suppose that $M$ is a factor of type I and $P\in M$ is a minimal projection. Then $\eC P=PMP\subseteq\oB(P\hH)$, and formula \eqref{EqLemPMPPMPprimedeux} in lemma \ref{LemPMPPMPprime} makes $(PMP)'=PM'P=\oB(P\hH)$. Now the map of lemma \ref{LemMapipsomPMmP} is an isomorphism, then
\[ 
  M'\simeq\oB(P\hH)
\]
as $*$-algebra. Thus $M'$ is a factor. This factor is moreover of type I because $\oB(P\hH)$ has minimal projections, namely projections on one dimensional spaces. 

If we repeat the same argument with $M'$ instead of $M$, we obtain that 
\begin{equation}		\label{EqMPHtypeIBh}
		M\simeq\oB(P'\hH)
\end{equation}
where $P'$ is a minimal projection in $M'$.

What we proved is
\begin{proposition}
If $M$ is a factor of type $I$, there exists a separable Hilbert space $\hH$ such that
\begin{equation}
	M=\oB(\hH).
\end{equation}
\end{proposition}

When $M=\oB(\hH)$ is of type $I$ says that $M$ is of \defe{type $I_n$}{factor!of Type $I_n$} if $\dim\hH=n$ where $1\leq n \leq\infty$.

\begin{proposition}
If $M$ is of type $I_n$ with $1\leq n <\infty$, then $M\simeq \eM_n(\eC)$ and the unique trace is the ususal matrix trace up to renormalization:
\begin{equation}
	\tr(T)=\frac{1}{ n }\sum_{i=1}^nT_{ii}.
\end{equation}
\end{proposition}
\begin{proof}
No proof.
\end{proof}

In particular we have that the composition of the two minimal projection $PP'=P'P$ is a rank one projection in $\oB(\hH)$. Now consider $Q=PP'$ and $v$ be the unital vector in the target space of $Q$, i.e. $Qv=v$ and $\| v \|=1$.

\begin{proposition}
Let $M$ be a factor of type I and $P$, a minimal projection. In the same way, let $P'$ be a minimal projection in $M'$. Let $Q=PP'$ and $v\in Q\hH$ such that $\| v \|=1$. Then
\begin{equation}
\begin{aligned}
 M'Q&\to P\hH \\ 
   SQ&\mapsto Sv 
\end{aligned}
\end{equation}
is an isomorphism.
\end{proposition}

\begin{proof}
Since $Q$ projects on the space spanned by $v$, the fact that $S_1v=S_2v$ implies $S_1=S_2$ and the injectivity is proved. For surjectivity, we know that, $M$ being a factor of type I and $P$ a minimal projection, we have $\eC P=PMP\subseteq\oB(\hH)$ where the last equality has to be understood in the sense of that $T\in PMP$ restricts to an operator on $P\hH$, and that this restrictions completely defines $T$. But as operators on $P\hH$, we have $(PMP)'=PM'P$ (lemma \ref{LemPMPPMPprime}), while lemma \ref{LemMapipsomPMmP} assures that the latter algebra is $M'$. Thus $M'=\oB(P\hH)$ and $M'v=P\pH$.
\end{proof}

\begin{proposition}
In the same way, we have that
\begin{equation}
\begin{aligned}
  MQ&\to P'\hH \\ 
   TQ&\mapsto Tv 
\end{aligned}
\end{equation}
is an isomorphism
\end{proposition}

\begin{proof}
No proof.
\end{proof}

\begin{proposition}
Let $M$ be a factor $S\in M$ and $T\in M'$, then if $ST=0$, then $S=0$ or $T=0$.
\end{proposition}

\begin{proof}
Let $P$ be the projection on the right support of $S$, i.e. onto $\Image(S^*)=(\ker S)^{\perp}$. This operator is equal to the limit\quext{To be proven.}
\[ 
  P=\lim_{n\to\infty}(S^*S)^{1/n}.
\]
So $P$ is a strong limit, and then belongs to $M$. Since, by assumption, $ST=0$, we have $S^*ST=0$, and then $PT=0$. Since $M$ is a factor, it implies that $P=0$ or $T=0$.
\end{proof}

\begin{proposition}
Let $M\subseteq\oB(\hH)$ be a factor. Then the map
\begin{equation}
\begin{aligned}
 M\otimes_{\eC}M'&\to \oB(\hH) \\ 
   S\otimes T&\mapsto ST 
\end{aligned}
\end{equation}
is an injective $*$-homomorphism. Its image is strongly dense in $\oB(\hH)$.
\end{proposition}

\begin{proof}
First, as $M$ is a factor, we have
\[ 
  \{ ST\tq S\in M\text{ and }T\in M' \}'=\eC\id,
\]
so that $\{ ST \}''=\oB(\hH)$, which proves that $\{ ST \}$ is strongly dense in $\oB(\hH)$.

We have elements $S_i\in M$ and $T_i\in M'$ ($i=1,\ldots,n$) such that 
\begin{equation}		\label{EqDecSTiMMprime}
	\sum_{i=1}^nS_iT_i=0,
\end{equation}
and we are going to prove that the set $\{ S_i \}$ is not linearly independent, so that all the $S_i$, or all the $T_i$ vanish. An element of $M\otimes M'$ can be decomposed under the form \eqref{EqDecSTiMMprime} in several ways.

Let $Q$ be the projection on the closure of the space
\begin{equation}	\label{EqEspaceTTHHnv}
	\big\{    
		(TT_1v,\ldots TT_nv)\in\hH\oplus\ldots\oplus\hH\tq T\in M', v\in\hH
	\big\}.
\end{equation}
This space is invariant under the algebra
\begin{equation}
	M'^{(n)}=\Big\{ 
\begin{pmatrix}
T&&0\\
&\ddots\\
0&&T
\end{pmatrix}
\tq T\in M' \Big\},
\end{equation}
so that $Q\in\Big( M'^{(n)} \Big)'$. Since $\{ T_1Tv\tq v\in\hH, T\in M \}=\{ T_1TSv\tq v\in \hH,T\in M,S\in M' \}$ the space \eqref{EqEspaceTTHHnv} is also invariant under
\begin{equation}		\label{EqSSSinvMBig}
	\Big\{ 
\begin{pmatrix}
S&&0\\
&\ddots\\
0&&S
\end{pmatrix}
\tq S\in M \Big\},
\end{equation}
On the other hand, the operators which commute with all $M'^{(n)}$ are element of $\eM_n(M)$, then
\[ 
  Q\in\Big( M'^{(n)} \Big)'=\eM_n(n),
\]
and for the same reason, the operators which commute with all \eqref{EqSSSinvMBig} are elements of $\eM_n(M')$, so that
\[ 
  Q\in\eM_n(M\cap M')=\eM_n(\eC).
\]
consider now the operator
\[ 
  R=
\begin{pmatrix}
S_1&\ldots & S_n\\
0&\ldots & 0 \\
\vdots  &&\vdots\\
0&\ldots & 0 \\
\end{pmatrix}.
\]
We havea
\[ 
\begin{pmatrix}
S_1&\ldots & S_n\\
0&\ldots & 0 \\
\vdots  &&\vdots\\
0&\ldots & 0 \\
\end{pmatrix}\cdot
  \begin{pmatrix}
TT_1 v\\
TT_2 v\\
\vdots\\
TT_nv
\end{pmatrix}
=
\begin{pmatrix}
S_1TT_1v+\ldots+S_nTT_nv\\
0\\
\vdots\\
0
\end{pmatrix}.
\]
The first component is
\[ 
  T\big( \sum_i S_iT_iv \big)
\]
which vanishes by assumption, so that $RQ=0$. Let us write $Q=(q_{ij}\id)$ as element of $\eM_n(\eC)$. The relation $RQ=0$ says that
\begin{equation}
		\sum_{i=1}^n S_iq_{ij}=0
\end{equation}
for every $j$. Since $Q$ is a non vanishing projector, at least one of these relations is nontrivial. That nontrivial relation shos that the $S_i$'s are vanishing. If not, we have $Q=0$, which means that $T_i=0$.
\end{proof}

%---------------------------------------------------------------------------------------------------------------------------
					\subsection{Tensor product of von~Neumann algebras}
%---------------------------------------------------------------------------------------------------------------------------

Let $M_1\subseteq\oB(\hH_1)$ and $M_2\subseteq\oB(\hH_2)$ be two von~Neumann algebras. We define\nomenclature[C]{$M_1\bar\otimes M_2$}{tensor product of von~Neumann algebras} the \defe{tensor product}{tensor product!of von~Neumann algebras}
\begin{equation}
	M_1\bar\otimes M_2=\{ S_1\otimes S_2\tq S_i\in M_i \}''\subseteq\oB(\hH_1\otimes\hH_2)
\end{equation}
