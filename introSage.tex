%+++++++++++++++++++++++++++++++++++++++++++++++++++++++++++++++++++++++++++++++++++++++++++++++++++++++++++++++++++++++++++
\section{Propagande : utilisez un ordinateur !}
%+++++++++++++++++++++++++++++++++++++++++++++++++++++++++++++++++++++++++++++++++++++++++++++++++++++++++++++++++++++++++++

Si vous faites des exercices supplémentaires et que vous voulez des corrections, n'oubliez pas que vous avez un ordinateur à disposition. De nos jours, les ordinateurs sont capables de calculer à peu près tout ce qui se trouve dans vos cours de math\footnote{Avec une notable exception pour les limites à deux variables.}.

%D'ailleurs, je te rappelle que nous sommes est déjà largement dans le vingt et unième siècle et que tu te destines à une carrière professionnelle dans laquelle tu auras des calculs à faire; si tu n'es pas encore capable d'utiliser un ordinateur pour faire ces calculs, il est temps de combler cette lacune.

%---------------------------------------------------------------------------------------------------------------------------
\subsection{Lancez-vous dans Sage}
%---------------------------------------------------------------------------------------------------------------------------

Le logiciel que je vous propose est \href{http://www.sagemath.org}{Sage}. C'est depuis 2012 un logiciel disponible pour l'épreuve de modélisation de l'agrégation en mathématique. Pour l'utiliser, il n'est même pas nécessaire de l'installer sur votre ordinateur~: il tourne en ligne, directement dans votre navigateur.

\begin{enumerate}

	\item
		Aller sur \href{http://www.sagenb.org}{http://www.sagenb.org}
	\item
		Créer un compte
	\item
		Créer des feuilles de calcul et amusez-vous !!

\end{enumerate}

Il y a beaucoup de \href{http://lmgtfy.com/?q=sage+documentation}{documentation} sur le \href{http://www.sagemath.org}{site officiel}\footnote{\href{http://www.sagemath.org}{http://www.sagemath.org}}.

Si vous comptez utiliser régulièrement ce logiciel, je vous recommande \emph{chaudement} de \href{http://mirror.switch.ch/mirror/sagemath/index.html}{l'installer} sur votre ordinateur.

%---------------------------------------------------------------------------------------------------------------------------
\subsection{Exemples de ce que Sage peut faire pour vous}
%---------------------------------------------------------------------------------------------------------------------------

Voici une liste absolument pas exhaustive de ce que Sage peut faire pour vous, avec des exemples. 
\begin{enumerate}

	\item
		Calculer des limites de fonctions, voir l'exercice \ref{exoINGE11140028},
	\item
		D'autres limites et tracer des fonctions, voir l'exercice \ref{exoINGE11140031}.
	\item
		Calculer des dérivées, voir exercice \ref{exo0013}.
	\item
		Calculer des dérivées partielles de fonctions à plusieurs variables, voir exercice \ref{exoFoncDeuxVar0002}.
	\item
		Calculer des primitives, voir certains exercices \ref{exo0017}
	\item
		Résoudre des systèmes d'équations linéaires. %Lire \href{http://www.sagemath.org/doc/constructions/linear_algebra.html#solving-systems-of-linear-equations}{la documentation} est ce qui fait la différence entre l'être humain et le non scientifique.
        Voir les exercices  \ref{exoINGE1121La0016} et \ref{exoINGE1121La0010}.
	\item
		Tout savoir d'une forme quadratique, voir exercice \ref{exoINGE1121La0018}.
	\item
		Calculer la matrice Hessienne de fonctions à deux variables, déterminer les points critiques, déterminer le genre de la matrice Hessienne aux points critiques et écrire extrema de la fonctions (sous réserve d'être capable de résoudre certaines équations), voir les exercices \ref{exoFoncDeuxVar0029} et \ref{exoFoncDeuxVar0028}.
	\item
		Lorsqu'il y a une infinité de solutions, Sage vous l'indique avec des paramètres, voir l'exercice \ref{exoDerrivePartielle-0007}. Pour les fonctions trigonométriques, 
        \begin{verbatim}
sage: solve(sin(x)/cos(x)==1,x,to_poly_solve=True)                                                         
[x == 1/4*pi + pi*z1]
sage: solve(sin(x)**2==cos(x)**2,x,to_poly_solve=True)
[sin(x) == cos(x), x == -1/4*pi + 2*pi*z86, x == 3/4*pi + 2*pi*z84]
        \end{verbatim}

	\item
		Calculer des dérivées symboliquement, voir exercice \ref{exoDerive-0002}.
	\item
		Calculer des approximations numériques comme dans l'exercice \ref{exoOutilsMath-0028}.
    %\item
    %    Calculer dans un corps de polynômes modulo comme \( \eF_p[X]/P\) où \( P\) est un polynôme à coefficients dans \( \eF_p\). Voir l'exemple \ref{ExemWUdrcs}.
	\item
        Tracer des courbes en trois dimensions, voir exemple \ref{ExempleTroisDxxyyOM}. Notez que pour cela vous devez installer aussi le logiciel Jmol. Pour Ubuntu, c'est dans le paquet \info{icedtea6-plugin}.
\end{enumerate}

%Sage peut toutefois vous induire en erreur si vous n'y prenez pas garde parce qu'il sait des choses en mathématique que vous ne savez pas. Par conséquent il peut parfois vous donner des réponses (mathématiquement exactes) auxquelles vous ne vous attendez pas. Voir page \ref{ooOPWYooDDSZWx}.
