%%%%%%%%%%%%%%%%%%%%%%%%%%
%
   \section{Liens rapides}
%
%%%%%%%%%%%%%%%%%%%%%%%%


\begin{enumerate}
%\item Le chapitre \ref{ChapThoComsGroupes}, "Examples of groups and representations" donne les décompositions d'Iwasawa et les structures de racines de tout ce dont on a besoin dans la suite.

\item 
La gueule des $J_1$ et $J_2$ des équations \eqref{EqGeueuleJun} et \eqref{EqgueueleJdeux}, ainsi que les éléments $V$,$W$,$X$,$Y$ qui s'ajoutent quand on monte en dimension : équations \eqref{EqGeueuleVWXY}.

\item
Les relations de commutation dans $\sA\oplus\sN$ : \eqref{EqTableSOIwa} et \eqref{TableSeconde}.

%\item
% Le chapitre « Black holes in anti de Sitter spaces » dit grosso-moddo tout ce qui est connu sur BTZ vu comme espace homogène. Dans la version "arxiv" de l'article avec Stephane\cite{lcTNAdS}, il y a plus de trucs sur le lien entre $AdS_3$ et les autres $AdS_n$. La construction comme quotient par un groupe discret y est faire ay début. L'arnaque est que $AdS_3=SL(2,R)$, mais que évidement les orbites fermées du $AN$ de $SL(2,R)$ ne sont pas celles du $AN$ de $SO(2,2)$. Or toute la littérature Detournay-Spindel-Bieliavsky-Rooman est faite sur $SL(2,R)$, donc les résultats doivent être traduit via l'isomorphisme $SL(2,R)=SO(2,2)/SO(1,2)$.

%\item
% Le point \ref{SubSecGEneBHGrop} « Generalization and group setting » brosse le tableau.

\item
 La décomposition $\sG=\sH\oplus\sQ$ d'espace réductif, équations \eqref{eq:gene_H} et \eqref{EqGeneRedQ}.

\item
	Action de $\sigma$ et $\theta$ sur les espaces de racines aux lemmes \ref{LemSigmaThetaRootSpaces} et \ref{LemSigmaChangeDeux}.

%\item
%La singularité est donnée par la définition \ref{Singular}.

%\item
% La proposition \ref{PropNormZeroEQnil} et le corollaire \ref{CorNilLightQ} donnent les géodésiques en termes de la décomposition en espace réductif

%\item
% Si t'es aussi con que moi, la remarque \ref{RemGedNonInvarChoix} t'évitera de ne rien comprendre pendant une semaine ;)

\item
Le point \ref{SubSecTwoCharSing}, « Two other characterizations of the singularity » donne les fameuses deux caractérisations équivalentes de la singularité.

\item
Dans le point \ref{SecExistenceHor} « Existence of a non trivial horizon », on trouve un point qui est dans le trou noir et un point qui n'y est pas. C'est très explicite. C'est exactement ce genre de résultats qu'il faudrait prouver sans écrire de matrices.

%\item
%La proposition \ref{PropSingQTiV} montre comment la singularité change d'une dimension à l'autre. Il suffit d'inclure canoniquement la dimension inférieure dans la dimension un de plus et agir par un groupe bien connu. Le résultat n'est prouvé que pour passer de $3$ à $4$, mais la généralisation ne fait que peu de doutes.

%\item
%Le théorème \ref{ThoEqHorQCoore} montre comment l'horizon change d'une dimension à l'autre. Même remarque qu'avant. C'est ce résultat que je cherche à redémontrer maintenant sans utiliser de calcul matriciel. C'est ce truc là que j'ai démontré il y a un mois. Il y a encore pas mal de fautes de typo dans la preuve. De plus, la démonstration de l'équation de l'horizon $\hH\equiv u^2-x^2-z^2=0$ est fausse (mais l'équation est presque certainement juste).

%\item
%Et enfin, la section \ref{SubSecVanNormChar} « Using the vanishing norm characterisation » est ce que je suis en train de faire. J'y prouve déjà qu'un élément de la forme $[an]$ est bien dans la zone $\| J_1^* \|=0$, de façon assez générale pour croire que ce n'est pas particulier à $AdS$. Je n'ai pas encore relu du tout.


\end{enumerate}

%En ce qui concerne les dimensions supérieures, l'Oracle a dit que
%\[ 
% \begin{split}
%AdS_3\leadsto P&=SL(2,\eR)\times SL(2,\eR)\\
%AdS_4\leadsto P&=Sp(2,\eR)
%\end{split} 
%\]
%et $\Spin(1,3)=SL_2(\eC)$.

