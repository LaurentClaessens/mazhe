% This is part of Exercices et corrigés de CdI-1
% Copyright (c) 2011
%   Laurent Claessens
% See the file fdl-1.3.txt for copying conditions.




\section{Intégration}

\exerNico 
Soient $n,m \in \N \cup \{0\}$.
Calculer
$$
\int_0^1 x^n (1-x)^m \,dx
\quad \text{ et } \quad
\int_{-1}^1 (1+x)^n (1-x)^m \,dx
$$



\exerNico 
Soient $a,b >0$. 
Calculer
$$
\int_0^{\pi /2} \displaystyle \frac{d \varphi}{a^2 \sin^2 \varphi + b^2 \cos^2 \varphi}
$$


\exerNico  
Calculer la longueur de l'arc de la parabole $y = x^2,\;x \in [0,b]$.

\exerNico  
La {\bf parabole de Neil} $\nu$ est la courbe définie par
$\nu (t) = (t^2,t^3), \, t \in \Rn$.
\begin{enumerate}
\item Esquisser la parabole de Neil.

\item Quelle est la signification du paramètre $t$?

\item Calculer la longueur de l'arc 
$\left\{ \nu (t) \mid t \in [0,\tau] \right\}$.
\end{enumerate}

\exerNico  
Une {\bf hélice} $\gamma$ de pas $2 \pi h$ est une courbe dans $\Rn^3$ définie par
$$
\gamma (t) \,=\, \left( r \cos t , r \sin t , h t \right)  .
$$

\begin{enumerate}
\item Esquisser $\gamma$ et expliquer le mot ``pas''.

\item Calculer la longueur de l'arc sur la hélice si on fait un tour.
\end{enumerate}

\exerNico Calculez la longueur des arcs de courbe suivants:
\begin{enumerate}
\item $y= \ln(1-x^2)  \hspace{3.5cm} 0\leq x\leq \f{1}{2}$
\item  $y= x^{3/2}  \hspace{4.57cm} 0\leq x\leq 5$
\item $y = 1-\ln(\cos x) \hspace{3cm} 0\leq x \leq \f{\pi}{4}$
\item l'arc de cubique déterminé par $y=x^3+x^2+x+1$ avec $0\leq x \leq 1$.
\end{enumerate}

