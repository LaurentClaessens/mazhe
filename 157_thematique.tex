% This is part of le Frido
% Copyright (c) 2016
%   Laurent Claessens
% See the file fdl-1.3.txt for copying conditions.

Ceci est un type d'index thématique.

\begin{InternalLinks}{Fonctions Lipschitz}
    \item
        Définition : \ref{DEFooQHVEooDbYKmz}.
    \item
        La notion de Lipschitz est utilisée pour définir la stabilité d'un problème, définition \ref{DEFooYIFAooSJbMkC}.
\end{InternalLinks}

\begin{InternalLinks}{Polynôme de Taylor}
    \item
        Énoncé : théorème \ref{ThoTaylor}.
        \item
            Le polynôme de Taylor généralise à l'utilisation de toutes les dérivées disponibles le résultat de développement limité donné par la proposition \ref{PropUTenzfQ}.
        \item
            Il est utilisé pour justifier la méthode de Newton autour de l'équation \eqref{EQooOPUBooYaznay}.
\end{InternalLinks}

\begin{InternalLinks}{Inégalité de Jensen}
    \item
        Une version discrète pour \( f\big( \sum_i\lambda_ix_i \big)\), la proposition \ref{PropXIBooLxTkhU}.
    \item
        Une version intégrale pour \( f\big( \int \alpha d\mu \big)\), la proposition \ref{PropXISooBxdaLk}.
    \item
        Une version pour l'espérance conditionnelle, la proposition \ref{PropABtKbBo}.
\end{InternalLinks}


