		Les exercices font partie de la matière du cours. Certaines conventions d'écriture et de langages, ainsi que certaines techniques qui seraient trop lourdes à expliquer en détail dans le texte ne sont introduites qu'au moment de leur utilisation dans les exercices.


Nous avons effectué une certaine «classification» des exercices en y ajoutant des petits symboles.
\begin{enumerate}
	\item Le symbole \minsyndical\ marque les exercices à faire à tout prix. Il faut les faire tous.
	\item Le symbole \boringexo\ signifie que l'exercice ne va en principe pas apporter de nouvelles techniques. Ces exercices sont à faire après avoir fait les exercices de type \minsyndical, si on a encore des doutes.
	\item Le symbole \coolexo\ indique que l'exercice sera plus difficile et qu'il vaut mieux l'éviter avant d'avoir bien compris les exercices de type \minsyndical.
	\item Le symbole \mortelexo\ est appliqué aux exercices qui sont des compléments de la matière, mais qui ne sont pas à faire de façon obligatoire.
\end{enumerate}
