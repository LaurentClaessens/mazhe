% This is part of Exercices et corrigés de CdI-1
% Copyright (c) 2011
%   Laurent Claessens
% See the file fdl-1.3.txt for copying conditions.

\begin{exercice}\label{exoIntMult0010}

La fonction $x \mapsto e^{-x^2}$ ne possède pas de primitives parmi 
les fonctions élémentaires et pourtant on peut évaluer
\begin{equation}
	\int_{-\infty}^{+\infty} e^{-x^2} dx := \lim_{R \to +\infty} \int_{-R}^{+R} e^{-x^2} dx 
\end{equation}
de la fa\c con suivante:
\begin{equation}		\label{EqCalculInteeemoisxcar}
	\begin{aligned}[]
		l^2 &= \lim_{R \to +\infty} \left( (\int_{-R}^{+R} e^{-x^2} dx)( \int_{-R}^{+R} e^{-y^2} dy) \right) \\
		&= \lim_{R \to +\infty} \left( \iint_{K}e^{-(x^2+y^2)} dx dy \right) \\
		&= \lim_{R \to +\infty} \left( \iint_{C_R}e^{-(x^2+y^2)} dx dy \right) 
	\end{aligned}
\end{equation}
où $K$ est le carré de demi côté $R$ centré à l'origine et de côtés parallèles aux axes et $C_R$ est le cercle de rayon $R$ centré à l'origine.
\begin{enumerate}
\item
Calculer la dernière intégrale et déduisez en la valeur de $l$.
\item
Justifier les étapes du calcul.
\end{enumerate}

\corrref{IntMult0010}
\end{exercice}
