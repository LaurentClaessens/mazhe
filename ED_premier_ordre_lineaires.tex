% This is part of Analyse Starter CTU
% Copyright (c) 2014
%   Laurent Claessens,Carlotta Donadello
% See the file fdl-1.3.txt for copying conditions.

%+++++++++++++++++++++++++++++++++++++++++++++++++++++++++++++++++++++++++++++++++++++++++++++++++++++++++++++++++++++++++++ 
\section{Équations différentielles linéaires du premier ordre}
%+++++++++++++++++++++++++++++++++++++++++++++++++++++++++++++++++++++++++++++++++++++++++++++++++++++++++++++++++++++++++++

\begin{definition}[Équation différentielle linéaire du premier ordre]
Soit $I\subset\eR$ un intervalle .

Une  \defe{équation différentielle linéaire du premier ordre}{équation différentielle!linéaire du premier ordre} est une équation différentielle de la forme 
\begin{equation}\label{eq_lin_ordre_un}
  a(x)y' + b(x) y = c(x), \quad\text{pour } x\in I, 
\end{equation}
o\`u $a$, $b$, $c$ sont des fonction de $\eR$ dans $\eR$ et $a\neq 0$ pour tout $x\in I$ . 

On dit que $a$, $b$, $c$ sont les coefficients de l'équation \eqref{eq_lin_ordre_un}.
\end{definition}
\begin{remark}\label{remarque_lineaire}
  Une fonction $f: \eR\to\eR$ est dite \emph{linéaire} si pour tout $x_1$, $x_2$ dans $\eR$ et pour tout couple de constantes $\lambda$ et $\mu$ on a 
  \begin{equation}\label{eq_linearite}
    f(\lambda x_1 + \mu x_2) = \lambda f(x_1) +\mu f (x_2).
  \end{equation}
Ces équations différentielles sont dites linéaires parce que la partie de l'équation qui contient $y$ (le membre de gauche) satisfait la propriété \eqref{eq_linearite} par rapport à $y$. En effet par les propriétés de la dérivée nous avons que 
\[
 a(x)(\lambda y_1 + \mu y_2 )' + b(x) (\lambda y_1 + \mu y_2 ) =\lambda ( a(x)y'_1 + b(x) y_1 ) + \mu( a(x)y'_2 + b(x) y_2 ).
\]
\end{remark}
\begin{definition}
  L'équation \eqref{eq_lin_ordre_un} est dite \defe{homogène}{équation différentielle!linéaire du premier ordre, homogène} quand $c$ est la fonction nulle. 
Si \eqref{eq_lin_ordre_un} n'est pas homogène on dit que l'équation 
\begin{equation}\label{eq_lin_ordre_un_hom}
  a(x)y' + b(x) y =0,
\end{equation}
est son \defe{équation homogène associée}{équation homogène associée}.
\end{definition}
Toute équation linéaire du premier ordre homogène est une équation du premier ordre à variables séparables, comme on en a vu dans la section précédente et en particulier dans l'exemple \ref{exemple_eq_hom}. Nous n'allons pas répéter les détails du procédé pour trouver sa solution générale, qui aura la forme suivante 
\begin{Aretenir}
  \begin{equation}\label{solgeneqlinordre1}
    \mathcal{Y}_h=\left\{Ke^{-\int\frac{b(x)}{a(x)}\, dx}\,:\, K\in\eR\right\}.
  \end{equation}
\end{Aretenir}
\begin{proposition}
  \begin{enumerate}
  \item Soit $y_p$ une solution particulière de l'équation \eqref{eq_lin_ordre_un} et $y_h$ une solution particulière de l'équation homogène associé \eqref{eq_lin_ordre_un_hom}. Alors la fonction somme $z= y_p+y_h$ est encore une solution particulière de l'équation \eqref{eq_lin_ordre_un}.
  \item Soient $y_1$ et $y_2$ deux solutions particulières de \eqref{eq_lin_ordre_un}. Alors la fonction différence $w = y_1-y_2$ est un solution particulière de \eqref{eq_lin_ordre_un_hom}.
  \end{enumerate}
\end{proposition}
\begin{proof}
  \begin{enumerate}
  \item 
    \begin{equation}
      a(x)\left(y_p+y_h\right)' + b(x)\left(y_p+y_h\right)-c(x)  =\left( a(x)y'_p+ b(x)y_p-c(x)\right) + \left( a(x)y'_h+ b(x)y_h\right) = 0.
    \end{equation}
  \item 
    \begin{equation}
      a(x)\left(y_1-y_2\right)' + b(x)\left(y_1-y_2\right) =\left( a(x)y'_1+ b(x)y_1-c(x)\right) -\left( a(x)y'_2+ b(x)y_2-c(x)\right) = 0.
    \end{equation}
  \end{enumerate}
\end{proof}
Cette proposition permet de démontrer le théorème suivant, qui est le plus important de cette section.
\begin{theorem}
  Soit $y_p$ une solution particulière de l'équation \eqref{eq_lin_ordre_un} et $\mathcal{Y}_h$ la solution générale de l'équation \eqref{eq_lin_ordre_un_hom}, alors la solution générale de l'équation \eqref{eq_lin_ordre_un} est l'ensemble 
  \begin{equation}
    \mathcal{Y} = \mathcal{Y}_h +y_p = \left\{z= y_h + y_p\,:\, y=h \in\mathcal{Y}_h \right\}.
  \end{equation}
\end{theorem}
\begin{Aretenir}
  La résolution d'une équation différentielle linéaire du premier ordre comporte trois étapes :
  \begin{enumerate}
  \item résolution de l'équation homogène associée ;
  \item recherche d'une solution particulière de l'équation non homogène ;
  \item somme de la solution générale de l'équation homogène et de la solution particulière trouvée au point précédent.
  \end{enumerate}
\end{Aretenir}
La partie qui nous manque encore est de savoir comment trouver une solution particulière de l'équation non homogène \eqref{eq_lin_ordre_un}. Si la fonction $c$ dans \eqref{eq_lin_ordre_un} est une constante ou un polynôme simple, ou une exponentielle alors on peut essayer de deviner. Cette méthode cependant n'est pas la plus sûre pour des débutants.  

\begin{example}
  On considère l'équation 
  \begin{equation}
    y'-5y = 10, \qquad x\in\eR.
  \end{equation}
Comme tous les coefficients de l'équation sont constants on peut essayer de trouver une solution constante. 

Toutes les fonctions constantes on dérivée nulle, par conséquent, si une solution constante existe elle doit satisfaire $-5y = 10$, ce qui veut dire que la solution constante est $y(x)\equiv -2$. 
\end{example}

\begin{example}
  On considère l'équation 
  \begin{equation}
    xy'+y = x+1, \qquad x\in\eR^{+,\star}.
  \end{equation}
Comme le membre de droite de l'équation est un polynôme de degré un on cherche une solution de la forme $y(x) = Ax + B$ avec $A$ et $B$ dans $\eR$.

Par substitution on obtient $Ax + (Ax +B) = x+1$, c'est à dire que une solution particulière de l'équation est $y(x) = x/2+1$. 
\end{example}

\begin{example}
   L'équation 
  \begin{equation}
    xy'-y = x+1, \qquad x\in\eR^{+,\star}.
  \end{equation}
ressemble beaucoup à celle de l'exemple précédent, cependant il n'existe pas un polynôme de degré un qui en soit solution. 

Dans un cas comme celui-ci, il faut rapidement abandonner la divination et replier sur la méthode, plus technique mais plus sûre, dite  \emph{variation de la constante} 
\end{example}


\subsection{Méthode de variation de la constante}

\begin{itemize}
\item Soit $\mathcal{Y}_h$ la solution générale de l'équation homogène associé à \eqref{eq_lin_ordre_un}. Il s'agit d'une famille à un paramètre de fonctions. La première étape de cette méthode consiste à construire un candidat solution particulière $y_p$ en remplaçant le paramètre dans  $\mathcal{Y}_h$ par une fonction $C: \eR \to\eR$ à déterminer. 

  \begin{example}
    L'équation homogène associée à $y'-y = \cos(x)$ est $y' - y = 0$, dont la solution générale est $\mathcal{Y}_h = \{Ce^x \,:\, C\in\eR\}$. Le candidat solution sera alors $y_p = C(x)e^x$, avec $C$ fonction à déterminer.
  \end{example}

 \item  La deuxième étape de cette méthode consiste à injecter $y_p$ dans l'équation. Cela permet de trouver une équation différentielle  à variables séparables pour $C$, en principe plus facile à résoudre que l'équation de départ.
 
  \begin{example}
    On continue avec l'exemple précédent. On a $y_p' = C'(x) e^x + C(x) e^x$, d'où 
    \[
    (C'(x) e^x + C(x) e^x) - C(x) e^x = \cos(x),
    \]
    c'est à dire 
    \[
    C'(x)  = \cos(x)e^{-x}.
    \]
  \end{example}
\item  La troisième étape de la méthode consiste à trouver une solution particulière de l'équation différentielle pour $C$ et, par conséquent déterminer une $y_p$.

  \begin{example}
    La solution générale de 
    \[
    C'(x)  = \cos(x)e^{-x}.
    \]
    est $\mathcal{C} = \left\{e^{-1}\frac{(\sin(x)-\cos(x))}{2} +K \,:\, K\in\eR\right\}$. Il nous suffit une solution particulière, nous pouvons donc choisir $K=0$ et alors la solution particulière de \eqref{eq_lin_ordre_un} sera $y_p (x)= \frac{\sin(x)-\cos(x)}{2} $.
  \end{example}
\end{itemize}
\begin{remark}
  Le plus souvent en intégrant l'équation pour $C$ on en trouvera la solution générale. Dans ce cas on peut remplacer $C$ par cette solution générale et obtenir d'un seul coup la solution générale de l'équation \eqref{eq_lin_ordre_un} , c'est à dire sans faire la somme entre la solution générale de l'homogène associée et la solution particulière.  

  \begin{example}
    Dans l'exemple qu'on vient de voir la solution générale de \eqref{eq_lin_ordre_un} est 
    \begin{equation}
      \mathcal{Y} = \mathcal{Y}_h + y_p = \left\{Ce^x + \frac{(\sin(x)-\cos(x))}{2} \,:\, C\in\eR\right\}. 
    \end{equation}
On obtient le m\^eme résultat est écrivant $\mathcal{Y} = \left\{e^{-x}\left(e^{-1}\frac{(\sin(x)-\cos(x))}{2} +K \right) \,:\, K\in\eR\right\}$. Notez qu'on a changé le nom du paramètre de $C$ à $K$ seulement pour souligner qu'on obtient de m\^eme résultat par deux chemins différents, sinon les deux expression sont équivalentes !
  \end{example}
\end{remark}
