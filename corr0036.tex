% This is part of Exercices et corrigés de CdI-1
% Copyright (c) 2011
%   Laurent Claessens
% See the file fdl-1.3.txt for copying conditions.

\begin{corrige}{0036}


Comme pour l'exercice précédent, toutes ces fonctions sont clairement continues, dérivables et à dérivée continue sur $\eR_0$ (attention cependant à la dernière fonction !). La question intéressante est donc de savoir ce qu'il se passe en $0$.

\begin{enumerate}

\item 
Notons $f$ cette fonction. On observe que
\begin{equation*}
  \limite x 0 f(x) =
  \begin{arrowcases}
    \limite[x > 0] x 0 \frac{2x + a}{1 + e^{\frac1x}} = 0\\
    \limite[x < 0] x 0 \frac{2x + a}{1 + e^{\frac1x}} = a
  \end{arrowcases}
\end{equation*}
en utilisant les règles de calculs. La différence entre les deux résultats provient du fait que $\lim_{x\to 0} e^{1/x}$ vaut soit $\infty$ soit $0$ suivant que $x\to 0$ en venant par les positifs ou par les négatifs. Donc $f$ est continue en $0$ si et seulement si le paramètre $a = 0$.

Regardons la dérivabilité de $f$ en $0$ lorsque $a = 0$. On obtient~:
\begin{equation*}
  \limite x 0 \frac{f(x)}{x} =
  \begin{arrowcases}
    \limite[x > 0] x 0 \frac{2}{1 + e^{\frac1x}} = 0\\
    \limite[x < 0] x 0 \frac{2}{1 + e^{\frac1x}} = 2\\
  \end{arrowcases}
\end{equation*}
et donc la fonction n'est pas dérivable en $0$.

\item
Notons $f$ cette fonction. On sait déjà que
\begin{equation*}
  \limite[x\neq 0] x 0 f(x) =  \limite[x\neq 0] x 0 \frac{\sin x}{x} = 1 = f(0)
\end{equation*}
ce qui prouve que $f$ est continue en $0$.

En fait $f$ est dérivable en $0$ puisque, en utilisant deux fois la règle de l'Hopital,
\begin{equation}
	\begin{aligned}[]
		\lim_{\epsilon\to 0}\frac{ f(\epsilon)-f(0) }{ \epsilon }	&=	\lim_{\epsilon\to 0}\frac{ \sin(\epsilon)-\epsilon }{ \epsilon^2 }\\
			&=	\lim_{\epsilon\to 0}\frac{ \cos(\epsilon)-1 }{ \epsilon }\\
			&=	\lim_{\epsilon\to 0}\frac{ -\sin(\epsilon) }{ 1 }\\
			&=	0,
	\end{aligned}
\end{equation}
c'est-à-dire $f'(0) = 0$.

Par ailleurs si $x\neq 0$, les formules usuelles de dérivation donnent
\begin{equation*}
f^\prime(x) = \frac{x \cos(x) - \sin(x)}{x^2}
\end{equation*}
et la continuité de la dérivée revient alors à étudier la continuité de
\begin{equation*}
  f^\prime(x) =
  \begin{cases}
    \frac{x \cos(x) - \sin(x)}{x^2} & \text{si $x \neq 0$}\\
    0 & \text{si $x = 0$}
  \end{cases}
\end{equation*}
via les méthodes usuelles. Calculons donc grâce à l'Hospital :
\begin{equation*}
  \limite[x\neq 0] x 0 \frac{x \cos(x) - \sin(x)}{x^2} =
  \limite[x\neq 0] x 0 \frac{\cos(x) - x \sin(x) - \cos(x)}{2x} = 
  \limite[x\neq 0] x 0 \frac{-\sin(x)}{2} = 0
\end{equation*}
ce qui prouve que $f^\prime$ est bien continue en $0$.

En fait, la fonction $\frac{\sin(x)}x$ est même infiniment dérivable en $0$, et \emph{analytique}.

\item
Notons $f$ cette fonction. Montrons que $f$ est dérivable en $0$. En effet, en appliquant encore la règle de l'Hospital (notez l'astuce de calcul pour éviter de tourner en rond) :
\begin{equation*}
  \limite[x\neq 0] x 0 \frac{f(x)}x =
  \begin{arrowcases}
    \limite[x> 0] x 0 \frac{e^{\frac{-1}x}}x = \limite[x> 0] x 0
    \frac{\frac 1x}{e^{\frac{1}x}} = \limite[x> 0] x 0
    \frac{1}{e^{\frac{1}x}} = 0\\
    \limite[x < 0] x 0 0 = 0
  \end{arrowcases}
\end{equation*}
ce qui prouve bien la dérivabilité en $0$.

Par ailleurs, on trouve évidemment avec les règles de calculs~:
\begin{equation*}
  f^\prime(x) =
  \begin{cases}
    \frac{1}{x^2}e^{\frac{-1}x} & \text{si $x >0$}\\
    0 & \text{si $x \leq 0$}
  \end{cases}
\end{equation*}
et, toujours grâce à l'Hospital (avec la même astuce de calcul), on voit que cette fonction est également continue en $0$.

En fait, cette fonction est indéfiniment dérivable, et toutes ses dérivées en $0$ sont nulles malgré que la fonction elle-même soit non-nulle pour $x > 0$ ; en d'autres termes c'est une fonction qui est très plate autour de $0$, mais pas constante.

\item
Au lieu de calculer $\lim f(x)$, nous calculons $\lim e^{\ln f(x) }$ (qui est égal, bien entendu) :
\begin{align*}
  \limite[x \neq 0] x 0 \left(\frac{\sin(2x)}{x}\right)^{(x+1)} &=
  \limite[x \neq 0] x 0 \exp{{(x+1)}
    \ln\left(\frac{\sin(2x)}{x}\right)}\\
  &= \exp{\limite[x \neq 0] x 0 {(x+1)}
    \ln\left(\frac{\sin(2x)}{x}\right)}\\
  &= \exp\big({1\cdot \ln(2)}\big) = 2
\end{align*}
où on a utilisé la continuité des fonctions $\exp{}$ et $\ln{}$, les règles de calculs\footnote{Es-tu capable de justifier le fait que $\lim_{x\to 0}\frac{ \sin(2x) }{ x }=2$ ?} et la règle de l'Hospital. Comme $2 \neq f(0) = 1$, la fonction n'est pas continue en $0$.


\end{enumerate}

\end{corrige}
