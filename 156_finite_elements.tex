% This is part of (almost) Everything I know in mathematics
% Copyright (c) 2016
%   Laurent Claessens
% See the file fdl-1.3.txt for copying conditions.

%+++++++++++++++++++++++++++++++++++++++++++++++++++++++++++++++++++++++++++++++++++++++++++++++++++++++++++++++++++++++++++ 
\section{Variational formulation (not too rigorous)}
%+++++++++++++++++++++++++++++++++++++++++++++++++++++++++++++++++++++++++++++++++++++++++++++++++++++++++++++++++++++++++++

As mentioned in the title, we are not going to deal with existence of the derivative and the integrals that we will write down.

Let the partial derivative equation
\begin{subequations}        \label{EQooZAISooSylvFH}
        \begin{numcases}{}
            -\Delta u=f\\
            u|_{\partial \Omega}=0
        \end{numcases}
    \end{subequations}
where \( \Delta u=\sum_{j=1}^n\frac{ \partial^2 v  }{ \partial x_j }\) on the open bounded part \( \Omega\) of \( \eR^n\). 

We are searching the solutions in a vector space
\begin{equation}
    V=\{ v\colon \Omega\to \eR\tq v|_{\partial \Omega}=0 \}.
\end{equation}
Our aim is to found a bilinear form \( a\colon V\times V\to \eR\) and a linear map \( L\colon V\to \eR\) such that the solutions of the original problem \eqref{EQooZAISooSylvFH} are solutions of the problem
\begin{subequations}
    \begin{numcases}{}
        u\in V\\
        a(u,v)=L(v)\,\forall v\in V
    \end{numcases}
\end{subequations}
The choice of \( V\), \( a\) and \( L\) is a \defe{variational formulation}{variational!formulation} of the differential equation\footnote{I said «not too rigorous» in the title, so please don't ask yourself now what space $V$ can be.}.

%--------------------------------------------------------------------------------------------------------------------------- 
\subsection{Better}
%---------------------------------------------------------------------------------------------------------------------------

<++>

We are going to do better.

In that purpose we recall the by part integral of formula \eqref{EQooQSMNooKHwbqp} that we use with \( u\) and  \( \frac{ \partial v }{ \partial x_j }\) :
\begin{equation}
    \int_{\Omega}u\frac{ \partial^2 }{ \partial x_j^2 }=-\int_{\Omega}\frac{ \partial u }{ \partial x_j }\frac{ \partial v }{ \partial x_j }+\int_{\partial \Omega}u\frac{ \partial v }{ \partial x_j }n_j.
\end{equation}
Since the function \( u\) belongs to \( V\) the boundary term vanishes and making the sum over \( j\) we get :
\begin{equation}
    \int_{\Omega}u\Delta v=-\int_{\Omega}\nabla u\cdot \nabla v.
\end{equation}
